
%%%%%%%%%%%%%%%%%%%%%
\section{Scepticisme et dogmatisme}
%%%%%%%%%%%%%%%%%%%%%
%{\it }
Ces analyses nous permettent, semble-t-il, de prendre
parti dans la controverse toujours ouverte du « scepticisme » 
de Hume. Laporte penchaït pour la thèse du
scepticisme absolu que rejetait jadis Compayré, que
conteste aujourd’hui M. Leroy qui verrait plutôt dans la
philosophie de Hume un « probabilisme ».

Hume assigne des limites à notre savoir, tous les historiens 
l’accordent, mais où ces limites se placent-elles
exactement ?

Hume distingue deux types de connaissance : les relations 
d’idées et les faits. Les relations d’idées (une proposition 
telle que trois fois cinq est égal à la moitié de trente
exprime une relation entre ces nombres) présentent cette
particularité que l’on peut les découvrir « par la seule
opération de la pensée sans dépendre de rien de ce qui
existe dans l’univers ». Les vérités nécessaires que nous
pouvons poser dans ce domaine sont des vérités purement
formelles, rigoureuses mais vides. Les conclusions auxquelles 
nous aboutissons ne sont rigoureusement nécessaires 
que parce qu’elles n’expriment rien d’autre que
l'accord de l'esprit avec ses propres définitions. Tel est
le propre de la pensée logique, mathématique, et aussi
juridique (une proposition telle que celle-ci : « Où il n’y
a pas de propriété il ne peut pas y avoir d’injustice »,
est une proposition nécessaire si nous avons préalablement
défini l'injustice comme la violation de la propriété).

Les propositions relatives aux faits sont bien différentes.
Si les données immédiates, les « impressions » originaires
s'imposent comme telles, on n’en peut dire autant des
conclusions que nous posons en raisonnant sur les faits.
Le principe de causalité — nerf du raisonnement expérimental — 
n’a rien à voir, nous l’avons montré, avec la
relation formelle de principe à conséquence. Il peut donc
sembler que Hume conteste sa valeur et se montre sceptique 
à l'égard de toutes nos inductions expérimentales.
Hume inaugurerait ainsi une tradition (promise en Angleterre 
à une grande fortune) de logiciens sceptiques selon
laquelle on peut concevoir une {\it pensée} vraie (la pensée
logique, purement formelle) mais non une {\it connaissance}
vraie (toute affirmation sur les faits devant se limiter à
%29
%{\it }
l’expérience immédiate). Cependant la plupart des historiens 
voient en Hume un partisan résolu de la méthode
expérimentale. Selon Norman Kemp Smith, Hume serait
sceptique seulement pour tout ce qui est artificiel, trop
subtil, trop compliqué. Pour Greig, il s’agit d’un « scepticisme
à limites bien définies. Il s’étend uniquement à
la théorie de la connaissance, à la métaphysique, il ne
touche pas à la morale, la politique, la méthode expérimentale, 
la vie commune ». Et on peut rappeler que déjà
Hegel lorsqu’il opposait au scepticisme antique (qui est
au fond un platonisme, qui ne doute des faits concrets
que pour nous amener à la contemplation des idées) le
scepticisme moderne (qui, à l’opposé, doute des idées
métaphysiques mais fait confiance à la raison expérimentale), 
proposait Hume comme exemple typique de ce
scepticisme moderne, qui serait donc seulement ce que
nous appelons aujourd’hui un positivisme. Auguste Comte
n’a-t-il pas signalé Hume comme un de ses précurseurs ?

Et Hume lui-même s’est fait l’apôtre de la science
expérimentale. Il proclame son admiration pour Newton
dont l’œuvre géniale « passera, triomphante, jusqu’à la
postérité la plus reculée ». Loin de suspecter, semble-t-il,
la méthode expérimentale il veut la généraliser en l’appliquant 
à ce que nous appelons aujourd’hui les « sciences
humaines ». Le {\it Traité de la nature humaine} est expressément
présenté comme un {\it Essai pour introduire la méthode expérimentale 
de raisonnement dans les sujets moraux}.

Seulement le projet de Hume en s’accomplissant aboutit
paradoxalement à mettre en question les principes mêmes
qui l’ont inspiré. L’étude de la « nature humaine » et plus
particulièrement des mécanismes de la croyance se présente, 
si l’on veut, comme une psychologie positive mais
elle débouche sur tout autre chose qu’une confirmation
et qu’une extension de la méthode expérimentale. En
fait, l'étude psychologique des origines de notre croyance
en la causalité ruine la valeur de ce principe de causalité,
fondement de la raison expérimentale elle-même.

%30
%{\it }
L'analyse de la causalité n’est nullement équivalente
chez Hume, nous l’avons montré, à une théorie du fondement de 
l'induction. Le principe de causalité se réduit au
fait psychologique de l’attente d’un phénomène, attente
produite par une longue habitude. Le principe de causalité
ne peut donc pas fonder en raison nos habitudes inductives, puisqu'il 
n’est pas lui-même autre chose que l’expression. de ces habitudes. 
Expliquer, par exemple, notre
attente de l’ébullition de l’eau quand nous la chauffons,
par l’habitude ce n’est pas justifier notre attente. Rien
ne prouve que l’avenir ressemblera au passé, que le soleil
se lèvera demain, que l’eau chauffée demain se mettra à
bouillir. Il n'y a pas d’impossibilité {\it logique} que les arbres
se mettent à « fleurir en décembre et à dépérir en mai ».
Le contraire d’un fait est toujours possible car il n’implique pas contradiction. 
Le prince indien qui refusait de
croire les premières relations sur le gel ne raisonnait pas
autrement que les maîtres de la connaissance expérimentale. 
N'ayant « jamais vu d’eau en Moscovie pendant
l’hiver », il induisait, comme tout le monde, d’après ses
habitudes. Il ne pouvait croire en l’existence de la glace
puisqu'il avait « toujours vu l’eau fluide sous son propre
climat ». Tout se passe en effet comme si « n’importe
quoi pouvait produire n’importe quoi », et Hume dans son
{\it Histoire d'Angleterre} s’applique à montrer que des événements 
dérisoires ont souvent pour effet des conséquences
capitales.

La découverte de la signification purement psychologique 
de la causalité n’est pas présentée par Hume comme
une victoire de la raison expérimentale, mais comme
une conclusion qui ruine la valeur de notre savoir : « Quelle
déconvenue éprouvons-nous nécessairement quand nous
apprenons que cette connexion, ce lien, cette énergie se
trouvent seulement en nous; que ce n’est rien qu’une détermination 
de l'esprit acquise par accoutumance. » Une
telle découverte « coupe court à jamais à l'espoir d’obtenir
satisfaction », elle futilise aussi bien les inductions de
%31
%{\it }
la science expérimentale que celles de la vie quotidienne.
« L’entendement se détruit complètement lui-même et
ne laisse plus le moindre degré d’évidence à aucune proposition 
de la philosophie ou de la vie courante. »

Le scepticisme philosophique de Hume se situe, on le
voit, sur un tout autre plan que le doute provisoire du
savant. Il arrive que le savant s'interroge longuement
sur la cause d’un phénomène qu’il ne sait pas rattacher
à ce qu’il appelle les lois de la nature. Mais le savant ne
doute pas un seul instant de la validité de ces lois, ou
tout au moins de l’existence de lois naturelles même s’il
les ignore. Quand une cause « ne produit pas son effet
habituel » le savant pense que « d’autres causes cachées
sont intervenues ». Le doute du savant n'intervient que
sur fond de dogmatisme, de confiance dans l’ordre de la
nature. Ce qui étonne Hume ce n’est pas tel phénomène
isolé, c’est l’ensemble des phénomènes, leur ordre habituel
mais incompréhensible dans la nature tout entière. Le
doute du savant porte, comme le doute du vulgaire, sur
{\it l’insolite}. Le doute du philosophe porte tout au contraire
sur le cours {\it habituel} du monde parce que le philosophe
a su reconnaître dans l’habitude un simple fait psychologique, qu’il a su se délivrer de cette magie de Phabitude qui chez la plupart des hommes endort l’étonnement.

Le scepticisme philosophique de Hume est donc un
scepticisme absolu qui n’épargne même pas les raisonnements de la vie quotidienne, et c’est cela qui explique
le ton tragique et l’accent de désespoir des pages qui terminent la quatrième partie du {\it Traité de la nature humaine}
(ton de désespoir qui surprend évidemment les historiens
partisans de la thèse d’un scepticisme modéré chez Hume).
Hume se juge alors « exclu de tout commerce humain,
complètement abandonné et sans consolation » : « Quand
je tourne mes regards en moi-même, je ne trouve que
doute et ignorance », « je suis effrayé et confondu de cette
solitude désespérée où je me trouve placé dans ma philosophie ».

%32
%{\it }
Cependant ce scepticisme demeure « philosophique »,
c’est-à-dire purement spéculatif. Il est sans prises et sans
influence sur notre pratique quotidienne. A l’opposé des
maîtres stoïciens dont il fut le fervent lecteur, Hume
juge qu’il n’est ni souhaitable, ni possible de vivre comme
on pense. Le scepticisme c’est l’attitude purement intellectuelle, 
purement réflexive, de quelques moments de
spéculation dans le cabinet de travail du philosophe.
Mais dès que Hume, étourdi par ses doutes, quitte son
cabinet, se détend, rejoint ses amis au club et « joue au
tric-trac », la « nature» le « guérit » de sa « mélancolie
philosophique ». Après quelques heures de distraction,
les spéculations philosophiques: de Hume lui paraissent
«si froides, si forcées, si ridicules que je ne pourrais trouver
le cœur d’y pénétrer tant soit peu ». Le philosophe a opéré
une critique radicale de la connaissance ({\it Knowledge}).
Mais l’homme demeure rivé à ses croyances ({\it belief}).
Le mécanisme psychologique, qui nous porte à croire à la
causalité en transportant à son concomitant habituel
l'impression vivace que nous donne le fait actuellement
donné, continue à jouer dans l’imagination du philosophe
lui-même. Ce mécanisme « conservera son influence aussi
longtemps que la nature humaine demeurera la même ».
Nous n’avons pas à « craindre que notre philosophie sape
jamais les raisonnements de la vie courante » parce que,
précisément, la croyance est un acte « de la partie sentante
plutôt que de la partie pensante » de notre être. La croyance
est au fond un fait biologique. C’est « une espèce d’instinct
naturel » que le raisonnement philosophique ne peut pas
empêcher. « La nature, par une nécessité absolue et incompréhensible 
nous a déterminés à juger aussi bien qu’à
respirer et à sentir. » Ne constatons-nous pas que les gens
les plus incultes, que les animaux eux-mêmes savent en
quelque sorte « induire », tirer les leçons de l’expérience ?
Scepticisme et dogmatisme ne se distribuent pas chez
Hume selon les régions de la connaissance (comme s’il
y avait par exemple un scepticisme métaphysique et un
%33
%{\it }
dogmatisme scientifique), mais plutôt selon les niveaux
de la réflexion. Le scepticisme réflexif (qui est absolu)
coexiste en quelque sorte avec un dogmatisme instinctif,
Nous sommes ici tout près de Pascal qui notait que « la
nature confond les pyrrhoniens » tandis que « la raison
confond les dogmatiques ». Pour Hume aussi « la nature
maintiendra toujours ses droits et prévaudra à la fin
sur tous les raisonnements abstraits ». Tout se passe en
quelque sorte comme si la nature avait préféré confier à
l'instinct une fonction aussi importante que celle qui
consiste à prévoir les « effets » à partir des « causes ». Le
scepticisme de Hume aboutit donc par un nouveau renversement 
à une sorte de confiance irrationnelle envers la
nature, cette divinité du {\footnotesize XVIII}$^\text{e}$ siècle, 
véritable « Providence laïcisée ». Il y aurait comme une « harmonie 
préétablie entre le cours de la nature et la succession de nos
idées » bien que cette harmonie reste incompréhensible et
que les forces qui gouvernent la nature nous soient « totalement 
inconnues ». Le dernier mot de Hume serait-il
affirmation d’un finalisme mystérieux ? « Ceux, dit-il,
qui se réjouissent à contempler les causes finales ont ici
ample occasion de manifester leur émerveillement. »
Hume rejoindrait-il le « sentimentalisme » des philosophes
écossais, ses compatriotes auxquels on l’oppose d’ordimaire ? 
Brunschvicg n’hésitait pas à l’affirmer : « Hume a
respiré dans le {\footnotesize XVIII}$^\text{e}$ siècle l’atmosphère de l’optimisme.
L’argumentation sceptique des considérants ne lui interdit
pas les conclusions par lesquelles invoquant la bonté de
la nature en faveur de nos croyances spontanées, il rejoindra
ou précédera le gros de l’armée écossaise ! »

Pourtant cet optimisme finaliste ne se substitue pas au
scepticisme. Paradoxalement il s’y ajoute fournissant une
sorte de philosophie de rechange à l’usage de la pratique
et des moments où il convient de vivre plus que de penser.
Mais le scepticisme réapparaît sitôt que le philosophe se
livre de nouveau au démon de la réflexion solitaire et
désintéressée. Hume ne se fait pour ainsi dire aucune
%34
%{\it }
illusion sur la valeur philosophique de ce dogmatisme
instinctif. Il conclut curieusement qu’au philosophe sceptique 
« négligence et inattention peuvent seules apporter quelque 
remède et c’est pour cette raison que je me
repose entièrement sur elles ».

%%%%%%%%%%%%%%%%%%%%%%%%%%%%%%%%%%%%%%%%%%%%%%%%%%%%%%%%%%%%%%%%%%%%%%%%%%%
