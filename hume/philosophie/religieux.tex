
%%%%%%%%%%%%%%%%%%%%%
\section{Le problème religieux}
%%%%%%%%%%%%%%%%%%%%%
%{\it }

Nous retrouvons, en étudiant les réflexions de Hume
sur Ja religion, la distinction fondamentale — mise en
lumière dans l’étude de la causalité — entre la question
du fondement et celle de l’origine. Cest un problème de
savoir si la notion de connexion nécessaire est ou non
fondée en raison. C’est un autre problème de savoir pourquoi,
en fait, nous croyons à la connexion nécessaire ;
ce deuxième problème substitue à la recherche philosophique
du fondement l’enquête psychologique sur l’origine.
Les deux questions se posent également à propos de l’affirmation
religieuse. Au tout début de sa dissertation sur
l’{\it Histoire naturelle de la religion}, Hume déclare que toute
enquête sur la religion doit s'attacher à résoudre deux
problèmes, celui du fondement de la religion {\it (its foundation in reason )}
et celui de ses sources psychologiques {\it (its origin in human nature)}.
Le premier problème porte sur
les preuves philosophiques de l’existence de Dieu (notamment,
pour Hume, la preuve de Dieu à partir des merveilles de la nature),
le second sur les origines psychologiques de la croyance religieuse.
Le premier problème sera
traité dans les {\it Dialogues sur la religion naturelle}, le second
fait l'objet de l’{\it Histoire naturelle de la religion}. C’est peu
de dire que les deux problèmes sont distincts, que les
motifs pour lesquels, selon les philosophes, nous devrions
croire sont différents des mobiles pour lesquels la plupart
des hommes croient effectivement. En fait, les motifs
philosophiques théoriques et les mobiles psychologiques
réels sont radicalement opposés. Le philosophe, pour
prouver Dieu, invoque l’ordre de la nature, l’enchaînement
%40
%{\it }
des phénomènes, admirable dans sa régularité, l’heureuse
harmonie des organes dans les êtres vivants. Tout cela
laisse indifférents les hommes vulgaires : « Plus la nature
est régulière et uniforme, moins ils ont envie de la creuser. »
L’accoutumance dissimule l'énigme du. monde, étouffe
toute inquiétude philosophique. C'est au contraire le
désordre, les accidents, les catastrophes imprévues qui
font que les hommes interrogent les puissances célestes.
Les joueurs, les mariniers, qui à tout instant dépendent du
hasard, sont les hommes les plus dévots. Les maladies,
le malheur, la mort nous font.« Er ce fléchir les
genoux » que les événements heureu : Bref, «ce qui, ee
un bon esprit, fait une des plus fortes objections contre
l'existence de l’Étre suprême est pour l’homme du commun
ment pour Lui ». 

La croyance en Dieu comme toutes les autres croyances,
prend sa source dans la « nature humaine », dans nos
passions de crainte et d'espérance. Ge « psychologisme »
fait de l'{\it Histoire naturelle de la religion} un ouvrage très
original à son époque. Non seulement pour Hume 
polythéisme est le point de départ chez tous les peuples
de la croyance religieuse (et non pas la dégradation d’un
prétendu monothéisme primitif); mais encore ce polythéisme
primitif ne relève aucunement de la psychologie
intellectualiste qu’on lui appliquait communément ; les
dieux, bienveillants ou maléfiques, des peuples primitifs,
ne sont pas tant des principes explicatifs des choses que
la projection, dans le ciel, de nos angoisses et de nos espoirs.

A ce polythéisme d’origine psychologique s'oppose
cependant le déisme rationnel du philosophe. Tandis que
le peuple « ne monte jamais aux cieux par la contenplation »,
l’idée d’une « Cause intelligente » de l'Univers
frappe le philosophe « comme une évidence qui porte
conviction ». Hume l’affirme à plusieurs reprises dans
l'{\it Histoire naturelle de la religion}. Faut-il voir dans ces
affirmations une précaution verbale, une habileté de Hume
qui espère ainsi faire passer l'audace de ses analyses psychologiques
%41
%{\it }
et historiques de la religion populaire ? Ou
bien Hume est-il un déiste sincère ?

Pour tenter de répondre à cette question, il faut examiner
de près les {\it Dialogues sur la religion naturelle}. L'ouvrage
avait paru très destructeur aux rares amis auxquels le
philosophe l’avait confié. Blair pensait qu’il eût mieux
valu le détruire, Adam Smith refusa, comme on sait, la
responsabilité de sa publication posthume. L'ouvrage se
présente comme le récit par le jeune Pamphile — élève
du philosophe Cléanthe — d’une suite de discussions entre
trois personnages : Cléanthe, déiste résolu, partisan de la
preuve de Dieu par la finalité du monde ; Philon le sceptique,
Déméa, mystique antirationaliste. Il est manifeste,
bien que les historiens de Hume ne l’aient pas toujours
remarqué, que ce livre est inspiré du {\it De Natura Deorum}
de Cicéron, longue discussion de Cotta, de Velleius et de
Balbus sur les conceptions épicurienne et stoïcienne de la
divinité. Hume emprunte à Cicéron non seulement la
forme du dialogue mais beaucoup d’idées ; Cicéron avant
Hume déclare : « Sur la nature des Dieux on varie, personne ne
nie leur existence. » Le héros déiste s’appelle
Cléanthe comme le maître stoïcien. Comme lui il rejette
le hasard épicurien et insiste sur l’évidence de la providence.

Quel est dans ces dialogues le porte-parole de Hume ?
Pamphile, à la fin de l’ouvrage, dit que Cléanthe a raison.
Mais Pamphile n’est pas Hume et c’est le jeune disciple
de Cléanthe. Il conclut tout naturellement en faveur de
son maître ! Hume lui-même s’est longuement expliqué
sur le sens de son ouvrage dans une longue lettre de 1751
à son ami Gilbert Elliot of Minto. Il lui confie qu'avant
l’âge de vingt ans il avait écrit un gros manuscrit sur le
problème religieux, brûlé par la suite et dont la substance
a dû passer dans les {\it Dialogues}. I1 « commençait par une
anxieuse recherche des arguments susceptibles de confirmer
l’opinion commune. Puis des doutes surgissaient,
étaient dissipés. revenaient encore ». Or la forme dialoguée
%42
%{\it }
est celle qui convient le mieux à une pensée hésitante.
Hume écrit à son ami qu’il a « voulu éviter cette vulgaire
erreur qui consiste à ne mettre que des absurdités dans
la bouche des adversaires ». Il ajoute, il est vrai, que
Cléanthe est le héros du livre, et il prie. Gilbert Elliot de
lui communiquer, après lecture du. manuscrit des {\it Dialogues},
« toutes les pensées qui vous viendront pour fortifier la position
de Cléanthe ». Mais,c’est là une remarque
{\it ad hominem} car précisément les opinions philosophiques
d’Elliot sont telles que si on jouait les {\it Dialogues}, ce dernier
« pourrait tenir fort bien le rôle de Cléanthe ». Et Hume
ajoute, ce qui est capital : « Pour mon compte, je me
réserverai le rôle de Philon que; vous le reconnaissez,
je pourrai tenir avec assez de naturel. » Et Hume aura
beau écrire plus tard, à Balfour, qu’il « s’est efforcé de
réfuter le sceptique Philon » cela n’empêche pas qu il
donne dans les {\it Dialogues} au philosophe sceptique la
part du lion. Comme le remarque J. Y. T. Greig : « Les
discours de Cléanthe couvrent trente-cinq pages, ceux de
Philon, qui doit être réfuté et réduit au silence, couvrent
cent quatorze pages. »

La preuve de Dieu par les causes finales, cependant, ne
manque pas d’impressionner Philon-Hume. C’est la preuve
favorite des déistes du {\footnotesize XVIII}$^\text{e}$ siècle. Leur optimisme
naturaliste s’y reflète complaisamment. Kant lui-même,
étranger à tout naturalisme, trouvera encore cet argument
« respectable ». Hume l’avait découvert très tôt dans le
{\it De Natura Deorum}. Le philosophe stoïcien auquel Cicéron
donne la parole conteste aux épicuriens que « ce monde
si riche et si beau » puisse résulter d’un concours fortuit
d’atomes : « Si l’on croit cela, pourquoi ne penserait-on
pas également que, en jetant d’une manière quelconque
en quantité innombrable les vingt et une lettres, il pourrait
résulter de ces lettres, jetées sur le sol, les {\it Annales}
d’Ennius telles qu’on pourrait les lire ensuite ? Je ne sais
si le hasard pourrait seulement venir à bout même d’une
seule ligne. » Le Cléanthe de Hume reprend inlassablement
%43
%{\it }
l’argument. La structure d’une fleur, l’anatomie du
moindre animal offrent maintes preuves plus fortes d’un
dessein originel qu’une demeure construite par un architecte ou qu’un livre de Tite-Live ou de Tacite. Tout
comme un livre est produit par l'intelligence humaine,
le « soigneux ajustement des causes finales » est un langage
dans lequel nous pouvons aisément déchiffrer l'intention
d’un Créateur souverainement intelligent.

Mais Philon, un instant déconcerté, soumet la preuve à
une critique serrée. Le raisonnement analogique qui nous
fait expliquer la production du monde matériel par une
Intelligence suprême, analogue à l'intelligence humaine
quoique infiniment supérieure, manque de rigueur. La
nature dans son ensemble est. malgré tout trop différente
des produits de l’industrie humaine pour que ce raisonnement soit décisif, « Le monde, dans son ensemble,
ressemble manifestement plus à un animal où à un végétal
qu’il ne fait à une montre et à un métier à tricoter. » Et
que gagne-t-on à expliquer l’ordre de la matière par
Pœuvre d’un esprit ? Sans doute organisation des parties
d’une maison témoigne-t-elle de Pintelligence de l’architecte. Mais la façon dont les idées s’organisent dans l’esprit
de l’architecte lui-même demeure une énigme. L’intelligence de l’homme suppose l’organisation ; elle n’en est
pas la clef. L’ordre n’est pas expliqué par un dessein,
mais un dessein suppose lui-même un ordre. On dira que
Dieu est précisément l'explication ultime de cette énigme.
Maïs inventer un Dieu revient à redoubler l’énigme plutôt
qu’à la résoudre. Si l’ordre du monde s’explique par l’intelligence divine, il faudrait expliquer comment les idées
se disposent en ordre dans cette intèlligence suprême.
Et si nous alléguons que « les idées qui composent la raison
de l’Être Suprême se disposent en ordre d’elles-mêmes et
par leur propre nature », nous pouvons aussi bien faire
l’économie de ce Dieu et dire que « les parties du monde
matériel se disposent en ordre d’elles-mêmes et par leur
propre nature ». Il n’est pas plus clair d'expliquer la
%44
%{\it }
matière par l'esprit que l’esprit par la matière :  « la
matière peut contenir la source de l’ordre originellement
en elle-même aussi bien que l'esprit. » 

Et puis à quoi bon s’émerveiller sur l'extraordinaire
adaptation des organes chez les animaux et chez les
plantes ? « Je voudrais bien savoir comment un animal
pourrait subsister à moins que ses parties ne fussent
adaptées. » Ici Hume annonce Auguste Comte et Darwin.
Le principe des « conditions d’existence » se substitue au
principe de finalité. La sélection des plus aptes est en
quelque sorte automatique et ne résulte pas d’un dessein :
« Bien des mondes auront pu être gâchés durant une éternité avant que ce système ne fût mis au jour. » Cependant
Philon reconnaît que sa réfutation est subtile et laborieuse.
Devant la simplicité de l’argument téléologique, il avoue
que «toutes les objections paraissent arguties et sophismes
purs ». 

Accordons à Cléanthe que l’univers témoigne d une
finalité. Seulement (et c’est ici que Philon reprend l’avantage) cette finalité n’a rien à voir avec ce qu'on appelle
une providence. Les soigneux artifices de la nature semblent
n’avoir pour but que de rendre « plus amère la vie de tout
être vivant ». Les espèces animales et végétales ont été
dotées d’armes offensives et défensives parfaitement
adaptées à la lutte de tous contre tous. La création est le
théâtre d’un perpétuel carnage : « Tout animal est entouré
d’ennemis qui recherchent incessamment sa misère et sa
destruction. » L’homme lui-même est sans cesse menacé
par les maladies, les inondations, les tremblements de
terre. Les pages très éloquentes que Hume, dans la
Dixième partie des {\it Dialogues}, consacre au problème du
mal nous font penser à cette remarque de Léon Brunscvicg. Si le Dieu créateur « est un Dieu artiste, n’est-ce
pas au sens néronien du mot » ? Toutes les justifications
philosophiques du mal sont dérisoires ; on dit que la
douleur est un avertissement; mais une véritable providence eût donné aux êtres vivants des signaux moins
%45
%{\it }
catastrophiques ; on prétend que le mal résulte inévitablement, à titre d’accident, des lois générales de la création ; mais une providence bienveillante n’eût pas dédaigné
d’agir par des volontés particulières. Et pourquoi les
forces des êtres vivants sont-elles si étroitement limitées ?
Enfin à supposer que les vents, les pluies, la distribution
de la chaleur et du froid résultent d’une volonté bienveillante, pourquoi ces « ressorts » sont-ils si souvent « déréglés », aboutissant à la destruction de ces vies qu’ils seraient
censés protéger ?

Si Dieu est bon il n’est pas tout-puissant, s’il est tout-puissant on ne peut parler de sa bonté. Le vieil argument
d’Épicure demeure irréfutable. Sans doute, si nous avions
par ailleurs des preuves certaines de l’existence d’un
créateur souverainement bon, on pourrait s’exercer à
montrer que l’existence du mal dans l’univers n’est pas
{\it incompatible} avec la Providence. Mais on ne saurait, de
toute évidence {\it inférer} la bonté de Dieu du spectacle de
l’Univers ! Ce spectacle, à le considérer sans préjugé,
« n’éveille pas d’autre idée que celle d’une nature aveugle,
imprégnée par un grand principe vivifiant et laissant
tomber de son giron, sans discernement ni soin maternel,
ses enfants estropiés et avortés ».

De toute façon la critique du principe de causalité,
thème fondamental de la philosophie de Hume, n’a-t-elle
pas d’avance dévalué toutes les preuves de l'existence de
Dieu ? Nous avons vu que l’inférence causale n’est qu’une
habitude née de notre expérience passée. Or Dieu n’est
pas une donnée de l’expérience. C’est une cause inventée,
non une Cause observée. Hume rappelle, dans ses {\it Dialogues},
la découverte fondamentale des ouvrages antérieurs :
«Il n’y a pas d’être dont la non-existence implique contradiction ? Par conséquent il n’y a pas d’être dont l’existence
soit démontrable. »

Si nous nous en tenions là, il faudrait convenir que le
scepticisme de Hume est tout proche de l’athéisme. Mais
les discussions entre Philon et Cléanthe ne doivent pas
%46
%{\it }
nous faire oublier les étranges relations entre Philon et
Déméa. Le sceptique et le « mystique » sont toujours
d’accord pour critiquer le rationalisme et l'optimisme de
Cléanthe. Sans doute Hume semble-t-il vouloir montrer,
non sans humour, que le mystique est  « un athée qui
s’ignore ». Parler comme Déméa de « la nature adorablement mystérieuse de Dieu », ou dire comme Philon que
Dieu est inconnaissable, ne revient-il pas au même
Mais dans les dernières pages du livre nous voyons que
c’est Philon qui, tout au contraire, rejoint le point de vue
de Déméa. Le scepticisme, en humiliant le dogmatisme
orgueilleux de la raison naturelle, ne nous prépare-t-il 
pas à accueillir la révélation ? Le secpticisme philosophique
n'est-il pas une propédeutique à la foi ? « Une personne
pénétrée d’un juste sentiment des imperfections de la
raison naturelle, dit Philon, volera à la vérité révélée
avec la plus grande avidité... Être un sceptique philosophique
c’est, chez un homme lettré, le premier pas et le
plus essentiel menant à être un vrai chrétien, un croyant »
N'est-ce pas en ce sens que Hume entendait le mot de
Bacon qu’il cite volontiers : « Un peu de philosophie
éloigne de la religion, beaucoup de philosophie y ramène ? »
On ne peut s’empêcher ici de penser à Pascal : « Le pyrrhonisme
sert à la religion. » Songeons, par exemple, que
tandis que Kant écrivait en 1780 sa {\it Critique de la raison
pure}, Hamann lui envoya un travail sur les « Dialogues » de
Hume dans l’espoir que la critique du Dieu des philosophes
disposerait Kant à accueillir le Dieu de la révélation chrétienne.

On objectera peut-être que l’appel à la révélation n’est
chez Hume, que persiflage et pirouette. Hume est sévère
à l'égard du christianisme. Pensant à sa propre enfance,
il rappelle les critiques que Machiavel — précurseur de
Nietzsche sur ce point — adressait à l’éducation chrétienne qui,
en insistant sur les mortifications, la pénitence,
l’obéissance, décourage l’énergie et débilite la {\it virtu}. Il
déteste les prêtres — au moins autant que les Anglais —
%47
%{\it }
(ce qui, pour un Écossais bon teint, n’est pas peu dire).
Dans son {\it Histoire d'Angleterre}, il condamne l’autoritarisme
clérical et le fanatisme. Et ce qui est plus important, dans
Son {\it Essai sur les miracles}, il raille la foi en la résurrection
du Christ, sous un déguisement facile à percer (« Supposez
que tous les historiens qui traitent de l’Angleterre s’accordent sur ce
que le 1$^\text{er}$ janvier 1600 la reine Elisabeth
mourut... qu'après un mois d’inhumation elle reparut,
réoccupa le trône et gouverna l’Angleterre pendant trois
ans »). Hume rejette les miracles au nom de la causalité
physique, tout en expliquant la croyance au miracle par
les passions de la nature humaine, par la causalité psychologique.
Mais précisément, quand on sait ce que Hume
pense du principe de causalité (simple habitude mentale
sans valeur rationnelle), on peut considérer que sa critique
du miracle, qui reste purement psychologique (humainement il n’est
guère possible de croire au miracle), n’est
Pas son dernier mot. Une résurrection, disait Pascal,
est ni plus ni moins énigmatique qu’une naissance ;
seule l’habitude nous. rend l’une croyable, l’autre incroyable,
« populaire façon d’en juger ». Et Hume lui-même, à la fin de
l’{\it Essai sur les miracles}, réserve la possibilité de ce miracle
intérieur qu’est la foi du croyant,
« miracle continu dans sa propre personne qui lui donne
une détermination à croire ce qui est le plus contraire
à la coutume ». L’essai sur l'{\it Immortalité de l’âme} se termine
de la même façon. Et Philon, à la fin des {\it Dialogues}, s’écrie :
« Plaise au Ciel d’alléger notre profonde ignorance en
offrant à l’humanité quelque révélation particulière. »
Persiflage pur et simple, dira-t-on, mais qui atteint la
philosophie de Hume lui-même Car un empiriste conséquent
ne peut soulever d’objection contre ce fait, contre cette
expérience intime vécue d’une révélation particulière !
Assurément, Hume ne prétend nulle part avoir reçu
pareille révélation. Du moins y a-t-il là une issue possible,
une porte que son impitoyable critique laisse entrouverte.
%{\it }
%%%%%%%%%%%%%%%%%%%%%%%%%%%%%%%%%%%%%%%%%%%%%%%%%%%%%%%%%%%%%%%%%%%%%%%%%%%
