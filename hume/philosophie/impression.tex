
%%%%%%%%%%%%%%%%%%%%%
\section{De l’ « idée » à l’ « impression »}
%%%%%%%%%%%%%%%%%%%%%
%
%{\it }
La philosophie anglaise du {\footnotesize XVIII}$^\text{e}$ siècle, amoureuse du concret, du donné, méfiante à l’égard des abstractions et
des hypothèses, se veut tout à la fois ingénue et rigoureuse ; rigoureuse parce qu’ingénue, parce qu’attentive à
rester fidèle à l'expérience vécue, à l’immédiat.

Ceci est plus sensible encore chez Hume que chez
Berkeley. L’ingénuité berkeleyenne n’est pas exempte
d'artifice et de présupposés. Toute l’œuvre de Berkeley
est inspirée par des préoccupations apologétiques. Hume,
en revanche, est un pur philosophe qui entend se laisser
conduire jusqu’au bout de sa réflexion quelles qu’en
soient l’issue et les conséquences. Discutant sur le problème
de la liberté il note fermement qu’ « il n’est pas certain
qu’une opinion soit fausse de ce qu’elle est de dangereuse
conséquence ».

Le radicalisme philosophique de Hume se manifeste
d’abord en ce qu’il entend remonter à des données originaires. Le point de départ de la réflexion philosophique
se trouvera donc dans ces données de la conscience, au
sens le plus large que Locke et Berkeley nommaient les
{\it idées} et que Hume appelle les {\it perceptions}. Car tandis que
Locke et Berkeley entendent par idées tout ce qui constitue
le contenu de la conscience, Hume opère une distinction
importante. La toute première phrase du {\it Traité de la nature
humaine} dit en effet que « toutes les perceptions de
l'esprit humain se ramènent à deux genres distincts que
%{\it }
%18
j’appellerai impressions et idées ». Seules les impressions
sont originaires, les idées ne sont que « des copies de nos
impressions », des reflets atténués de nos sensations dans
le miroir de nos pensées. Sommairement présentée, cette
thèse semble être celle de l’empirisme le plus plat. La
doctrine de Hume s’identifierait à un sensualisme grossier.
et se contenterait de reprendre la formule célèbre de
Locke : « Il n’y a rien dans l’entendement qui n’ait d’abord
été dans les sens. {\it Nihil est in intellectu quod non prius fuerit in sensu.} »

Mais si nous l’examinons avec quelque attention, nous
voyons que la distinction humienne des impressions et
des idées se révèle beaucoup plus intéressante et beaucoup
plus profonde.

Tout d’abord il ne faudrait pas imaginer que pour
Hume l'esprit soit purement passif, une « table rase »,
une pâte molle où s’inscriraient mécaniquement les stimuli
externes. Assurément un aveugle-né ne saurait, faute
d’impressions, se faire la moindre idée des couleurs. Mais
supposons « un homme familiarisé avec les couleurs de
tout genre sauf avec une nuance particulière de bleu que
le hasard ne lui a jamais fait rencontrer. Qu’on place
devant cet homme toutes les diverses nuances de cette
couleur, à l’exception de cette nuance particulière, dans
une gradation descendante de la plus foncée à la plus
claire ». Cet homme, assure Hume, « percevra un vide »,
pourra « suppléer à ce défaut par sa seule imagination »
et « se donner l’idée de cette nuance particulière que
cependant ses sens ne lui ont jamais fournie ». Exception
unique mais très significative. Ceci témoigne d’un élan
de l'imagination, d’un dynamisme de l'esprit humain,
d’une activité psychologique subjective qui, dans l’empirisme original du philosophe écossais, est fondamentale.

De plus l'impression ne s’oppose pas à l’idée comme
une sensation d’origine externe s’opposerait à un phénomène psychologique intérieur. À vrai dire Hume ne
s’interroge pas sur la source des impressions. Pour lui les
%19
impressions sont des données originaires au-delà
desquelles on ne saurait remonter. En ce sens Hume n’est
nullement l’adversaire de l’innéisme, il le remarque
expressément : « Si l’on entend par inné ce qui est primitif,
ce qui n’est copié d'aucune impression antérieure, alors
nous pouvons affirmer que toutes nos impressions sont
innées et que nos idées ne le sont pas. »

Il faut ajouter qu’à côté des « impressions de sensation »
nous avons des  « impressions de réflexion » qui comprennent
les émotions, les passions, tous les phénomènes de
désir et de volonté. Sans doute « la plupart » des impressions de réflexion proviennent indirectement des impressions de sensation par l'intermédiaire des idées. Par
exemple l’idée de plaisir, l’idée de douleur provoquent dans
l’âme de nouvelles impressions de désir ou d’aversion.
Mais il reste que la distinction impression-idée est une
distinction purement psychologique. Elle s’opère sur des
critères internes. L’impression se définit par ce que Hume
appelle sa {\it vividness}, sa {\it liveliness}, c’est-à-dire
moins peut-être par son intensité que par sa présence vivante, par son
caractère actuel. L’impression ne désigne donc pas spécifiquement
la sensation d’origine externe mais, comme dit
Hume, c’est « un mot tout nouveau pour désigner nos
perceptions les plus vivantes quand nous entendons,
voyons, touchons, aimons, haïssons et voulons ».

%{\it }
Les philosophes ne raisonnent que trop souvent sur des
« idées indistinctes et obscures ». Ils croient penser et ils
ne font que parler. Hume les invite alors à rechercher les
« impressions » authentiques. Il ne s’agit pas de tout
ramener au sensoriel. Il s'agit plutôt d'une critique du
langage, d’une invitation à retrouver la pensée vivante,
actuelle : « Quand nous soupçonnons qu’un terme philosophique
est employé sans aucun sens ni aucune idée
correspondante, nous n’avons qu’à rechercher {\it de quelle
impression dérive cette idée supposée}. » Aller de l’idée à
l'impression c’est demander aux philosophes « ce qu’ils
entendent par substance et inhérence » ; c’est leur demander :
%20
« Qu’existe-t-il effectivement ? » Il s’agit de « porter
les idées à la lumière » pour écarter les discussions stériles
en démasquant les pseudo-idées. L'appel à l’impression
signifie chez Hume, dit très bien Laporte, « non le préjugé
du sensualisme mais la haine du verbalisme ». Les impressions
de Hume ce sont les idées claires de Descartes, celles
qui sont « présentes et manifestés à un esprit attentif »,
ce sont les « données immédiates » que Bergson voudra
retrouver. Et il est très remarquable que Husserl invoquera
Hume, au même titre que Descartes, comme précurseurs
de sa méthode phénoménologique dônt le but sera précisément d’écarter les hypothèses et les théories pour
revenir « aux choses elles-mêmes », pour tenter d’élucider,
de dévoiler ({\it enthüllen}) des significations. Ce que nous
demande Husserl, commente Gaston Berger, « c’est de
revenir aux données originaires, à ce qui est supposé par
toutes les opérations symboliques et toutes les représentations mais que le jeu même de ces opérations symboliques
risque toujours de nous faire perdre de vue. Il faut aller
des concepts vides par lesquels une idée est seulement
visée à l’intuition directe et concrète de l’idée, tout comme
Hume nous apprend à revenir des idées aux impressions ».
Retrouver les impressions, c’est tout simplement retrouver l’immédiat.
Comme le dit Hume dans ses {\it Dialogues
sur la religion naturelle} : « Est-il nécessaire d’établir ce
que chacun éprouve au-dedans de soi ? Il est seulement
nécessaire de nous le faire éprouver, s’il se peut, d’une
façon plus intime et plus sensible. » Telle est la tâche difficile de la philosophie, car cette ingénuité qui nous est
demandée est, en fait, une conquête ardue. L’immédiat
n’est pas donné immédiatement ; il faut le reconquérir
en deçà des préjugés et des théories, en deçà des formules
vides qui s’insinuent et s'imposent, créant des habitudes
de langage qui se transforment en habitudes de pensée ou
plutôt de non-pensée. Le contact avec l'immédiat ne saurait
être un point de départ ; il est le fruit et la récompense
d’une critique sévère.
%21
%%%%%%%%%%%%%%%%%%%%%%%%%%%%%%%%%%%%%%%%%%%%%%%%%%%%%%%%%%%%%%%%%%%%%%%%
