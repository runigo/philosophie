
%%%%%%%%%%%%%%%%%%%%%
\section{Esthétique, morale, politique}
%%%%%%%%%%%%%%%%%%%%%
%34
%{\it }
Nous abordons ici le domaine des jugements de valeur.
Hume leur applique la même méthode de réflexion critique
qu'aux jugements de réalité portant sur les {\it matters of fact}.
Ici encore toute affirmation à prétention ontologique se
trouve réduite à une simple « croyance », à une réaction
subjective de la « nature humaine ».

Par exemple ce que nous appelons la beauté n’est pas
une qualité inhérente aux choses. Elle ne réside pas dans
des relations objectives. Autrement Euclide n’eût pas
manqué d’en faire une des propriétés du cercle ! En fait,
la beauté n’est que dans l’âme qui la contemple. Elle
est une réaction émotionnelle de Îa nature humaine (tout
comme la connexion nécessaire n’était pas dans les choses
mais dans l’esprit).

Mais subjectiviser les valeurs esthétiques n'est-ce pas
les futiliser, nous vouer au scepticisme ? L’un verra de la
laideur dans ce qui semble beauté à un autre et si toutes les
appréciations sont subjectives, « rechercher la vraie beauté
ou la vraie laideur est une enquête aussi vaine que de
prétendre établir le vrai doux ou le vrai amer ».

Seulement le scepticisme auquel nous aboutissons par là
est purement théorique. En fait les hommes s’accordent
assez bien sur les valeurs esthétiques. Nous reconnaissons
tous, dit Hume à plusieurs reprises, qu’ « il n’y a pas
autant de génie chez Ogilby que chez Milton » (Ogilby,
1600-1676, est un plat traducteur de Virgile). Une admiration durable 
s’attache aux grandes œuvres. Tout platonisme de l’idée de beauté est rejeté.
%35
%{\it }
Mais la validité pratique de nos jugements de goût repose sur l’uniformité
de la nature humaine. Le subjectivisme de Hume ne fait
donc pas obstacle à ce que Laporte appelait le « dogmatisme
du sentiment ». Cependant l’uniformité des sentiments
humains n’est pas absolue. On parle d’un bon goût et
d un mauvais goût. Il faudrait donc dit Hume dans sa
{\it Dissertation sur la règle du goût} « trouver une règle par
laquelle il serait possible de réconcilier les sentiments
divers des hommes ou du moins d’obtenir une décision
qui confirme un sentiment ou en condamne un autre »
Finalement, en matière de goût esthétique, on ne se ficra
qu’à des juges compétents dont le sens du beau est particulièrement 
délicat, affiné par une longue expérience
Mais c’est là substituer au {\it consensus omnium} le {\it consensus
optimi cujusque}. Sans doute le jugement de ces {\it happy few}
peut être empiriquement appuyé par la confiance que
l’ensemble des hommes leur accorde. Mais Hume lorsqu’il
réfléchit en philosophe trouve cette justification fragile
En fait, parler d’un bon et d’un mauvais goût, c’est corriger
le subjectivisme par un finalisme discret qui reste une
hypothèse métaphysique. Hume oscille entre un optimisme
purement empirique et un scepticisme réfléchi. Le problème 
du juge compétent dans le cadre du subjectivisme
de la {\it Dissertation sur la règle du goût} demeure une « question
embarrassante qui semble nous replonger dans la même
incertitude dont le but de cette dissertation était de nous
délivrer ».

Nous retrouvons la même ambiguïté dans la façon dont
Hume prétend résoudre le problème moral. Le bien et le
mal ne sont pas des relations objectives. Nous ne pouvons
pas fonder nos devoirs sur quelque principe qui existerait
en soi. Par exemple la volonté divine telle qu’elle peut
être connue par la philosophie ne peut pas justifier la
condamnation du suicide. On dit communément que la
Providence ne permet pas aux hommes de disposer de
leur vie. Mais c’est là bien mal argumenter. En effet, tout
ce qui se passe dans l’univers, sans exception, « peut être
%36
%{\it }
appelé l’action du Tout-Puissant », car tout procède de
lui par le truchement des pouvoirs dont il a doté ses créatures. 
« Une maison qui s'écroule d’elle-même n’est ni
plus ni moins détruite par la Providence que si elle est
démolie par les mains de l’homme. » Rien n’échappe aux
lois de la nature. Les conséquences de nos jugements,
de nos passions, les tempêtes, les inondations, les épidémies 
sont également des effets de la Providence. L’homme
qui se tue ne trouble donc pas l’ordre de l'univers, mais
l’exprime. Si la disposition de la vie humaine était réservée
à l’action particulière du Créateur,  « il serait aussi criminel
de sauver une vie que d’en détruire une ». En se plaçant
au point de vue ontologique, on ne peut donc pas plus
condamner le suicide que le meurtre. Cela Hume ne le
dit pas mais c’est une conséquence logique et l’{\it Essai
sur le suicide} fait penser aux raisonnements du marquis
de Sade qui justifie son immoralisme par un « chantage ontologique » du même style.

La « raison » laïcisée, chère à certains philosophes du
{\footnotesize XVIII}$^\text{e}$ siècle, ne peut pas davantage fonder la morale.
Car la pure raison, si elle éclaire la relation de moyen à
fin, ne fournit elle-même aucune fin : «Si une passion ne
se fonde pas sur une fausse supposition et si elle ne choisit
pas des moyens impropres à atteindre ses fins, l’entendement 
ne peut ni la justifier ni la condamner. Il n’est
pas contraire à la raison que je préfère la destruction du
monde entier à une égratignure de mon petit doigt. »
Les moralistes rationalistes (à la manière de Clarke)
passent subrepticement des jugements de réalité (qu’il
s’agisse d’affirmations métaphysiques ou d’autres propositions) 
aux jugements de valeur. Lisons leurs ouvrages :
Tout à coup, dit Hume, au lieu de la copule {\it est}, qui était
habituelle dans leurs propositions, « nous ne rencontrons
que des propositions où la liaison est établie par {\it doit} ou
ne doit pas ». Et ce tour de passe-passe n est aucunement justifié.

Où donc se trouve la source du jugement moral ?
%37
%{\it }
Uniquement dans nos sentiments, dans nos réactions subjectives. 
« La morale est plus proprement sentie que jugée. »
Une action, un sentiment, un caractère seront déclarés
«vicieux » ou « vertueux ». Pourquoi ? Parce que la vue
ou le récit de tel acte provoque en nous « un plaisir ou un
malaise d’un genre particulier ». Nous éprouvons un sentiment 
d’approbation ou de désapprobation. Le jugement
éthique, comme le jugement esthétique, repose sur des
tendances spécifiques de la nature humaine. La reconnaissance 
de la radicale subjectivité du jugement éthique, de
son relativisme, n’est nullement, comme M. Leroy le
fait remarquer, une « liquidation de la morale ». La réduction 
psychologique de l’idée de cause ne nous empêchait
pas de croire à la causalité. La réduction psychologique de
Vidée de bien ou de mal ne peut guère nous troubler ;
d’abord parce que « nous ne pouvons pas plus changer nos
sentiments personnels que le mouvement des cieux »,
ensuite parce que l’uniformité de la nature humaine
(thème très cher à Hume, nous le savons) confère à nos
sentiments éthiques une universalité remarquable. Remarquons 
ici que Hume n’ignore nullement la diversité des
mœurs et des coutumes, la multiplicité des civilisations
dans l’espace, leur évolution dans le temps. Ce n’est
pas là une découverte des sociologues ultérieurs. Montaigne — que 
Hume avait lu — avait largement insisté
sur ce thème et Hume le reprend lui-même dans le {\it Dialogue
avec Palamède}. Sur bien des articles, la morale des anciens
Grecs et des Européens du {\footnotesize XVIII}$^\text{e}$ siècle paraît s’opposer.
Mais il y a des sentiments permanents (nul peuple n’a
jamais prôné la lâcheté ou la déloyauté). Et les divergences
éthiques représentent le plus souvent des applications
diverses, dans des situations différentes, de principes
uniformes de la nature humaine. La pratique de l’exposition 
des nouveau-nés, qui révolte la conscience moderne,
n’excluait pas des sentiments de compassion : la mort
passait alors pour moins redoutable que la misère.

Cependant Hume n’ignore pas qu’un fondement 
%38
%{\it }
purement empirique de la morale est plutôt une explication
psychologique qu’une justification philosophique. Si je
parle à un malhonnèête homme, reconnaît-il, du « contentement 
intérieur qui résulte des actions louables et
humaines, du plaisir délicat de l’amour et de l’amitié
désintéressés, il pourrait me répondre que ce sont là des
plaisirs pour qui est capable de les goûter, mais que, pour
sa part, il se trouve d’un tour d’esprit et de dispositions
tout à fait autres. Ma philosophie n’apporte aucun remède
à un tel cas ». Les héros du marquis de Sade ne parleront
pas autrement de la morale et du sentiment. Et si Hume
s'efforce toujours, à partir de considérants sceptiques,
d'aboutir à des conclusions rassurantes, on peut se demander 
dans quelle mesure, en voulant persuader le lecteur,
il parvient à se convaincre lui-même. Ayant écrit : « Quand
on déclare une action vicieuse on veut simplement dire,
qu'étant donné la constitution {\it particulière} de sa nature,
on a un sentiment de blâme en assistant à la dite action »,
Hume, dans une lettre du 16 mars 1740, soumet son texte
à Hutcheson, et lui demande s’il n’est pas dangereux de
s’exprimer en ces termes. Il est curieux de noter que, dans
le livre III du {\it Traité de la nature humaine} qui paraît chez
l'éditeur Thomas Longman à la fin de 1740, le mot « particulière » 
disparaît de ce texte. Hume a littéralement ici
effacé le problème, ne pouvant le résoudre.

On retrouverait les mêmes oscillations dans la pensée
politique de Hume. Sa réflexion critique le porterait vers
des solutions révolutionnaires ou tout au moins libérales.
Pas plus pour lui que pour Pascal les « grandeurs d’établissement » 
ne sont respectables. Il est bien « rare de
rencontrer une race de rois ou une forme de république qui
ne se fonde pas primitivement sur l’usurpation et la
violence ». La science historique disqualifie le traditionalisme 
tout comme la psychologie démystifie la causalité.
Et pourtant, en pratique, « il n’y a pas de maxime plus
conforme à la prudence que de se soumettre paisiblement
au gouvernement que nous trouvons établi dans le pays
%39
%{\it }
où il nous arrive de vivre ». La critique spéculative débouche
sur le conservatisme pratique. Ici encore la pensée est une
chose, la vie en est une autre. Pensons comme les whigs
mais votons pour les tories !
%%%%%%%%%%%%%%%%%%%%%%%%%%%%%%%%%%%%%%%%%%%%%%%%%%%%%%%%%%%%%%%%%%%%%%%%%%%
