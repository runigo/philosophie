
%%%%%%%%%%%%%%%%%%%%%
\section{Critique de quelques notions énigmatiques : causalité, probabilité, substance}
%%%%%%%%%%%%%%%%%%%%%
%{\it }
L’analyse de la relation de causalité par Hume est
justement célèbre. Cette relation où l’on voit communément le principe fondamental de la raison humaine est
pour un philosophe empiriste assez extraordinaire. Tandis
qu’une relation d’idées comme la ressemblance porte sur
plusieurs faits qui tous sont donnés dans l’expérience
— et que nous pouvons alors comparer sans difficulté —,
la relation causale porte sur deux termes dont l’un est
absent : « Pas de fumée sans feu. » Je constate la présence
de la fumée et j’en infère au nom du principe de causalité
qu’un feu a été allumé ; ou bien je constate qu’on allume
un feu et je prévois qu’il y aura de la fumée. Au nom de
la relation causale, je « dépasse le témoignage de mes
sens », je prévois ce qui va se produire, j’infère quelque
chose dont je n’ai pas l’expérience actuelle. De quel droit
puis-je tirer d’un fait donné une conséquence qui le déborde,
de quel droit mon esprit fait-il ce « pas en avant » ? On
répond en invoquant la « connexion nécessaire » par laquelle
tel fait donné produira infailliblement tel autre fait. On
parle de « pouvoir », d’ « efficace », de « qualité productrice ». Il y aurait entre le fait présent et ce qu’on en
infère un lien nécessaire tel que le fait étant donné, l’autre
fait ne peut pas ne pas se produire. Mais d’où vient cette
idée de connexion nécessaire qui est, en effet, le nerf du
principe de causalité ? Il faut ici, selon la méthode de
Hume, découvrir l’impression originaire dont cette idée
dérive. Et pour trouver cette impression il faut se livrer
à ce que Hume appelle une enquête, il faut chercher cette
impression « à toutes les sources d’où elle peut découler »
comme on organise, dit Hume, une battue dans une forêt
pour trouver un objet précieux qui a été perdu.
%22
%{\it }
Demandons-nous d’abord si l’idée de cause peut dériver
d’une impression de sensation.

Tout le monde peut constater que la fumée apparaît
quand le feu est allumé, que l’ébullition suit l’échauffement
de l’eau, que lorsqu'une boule de billard en heurte une
autre, cette seconde boule se meut à son tour. Mais cette
constatation banale est celle d’une simple conjonction,
d’une succession entre deux phénomènes. Nous voyons
bien que l’échauffement de l’eau a précédé l’ébullition,
que le mouvement de la première boule de billard a précédé
le mouvement de la seconde. Mais nous ne voyons pas
entre les deux événements de connéxion nécessaire. Nous
constatons un « et puis », nous ne constatons pas un
« parce que ». Nous avons beau examiner, retourner en
tous sens le phénomène que nous appelons « cause » nous
ne pouvons découvrir en lui l « efficace » qui produirait
l'effet. Rien dans le mouvement de la première bille de
billard ne nous donne la moindre indication sur le mouvement qui va lui succéder. Nul objet dit Hume, « ne nous
montre jamais par les qualités qui paraissent aux sens,
soit les causes qui le produisent, soit les effets qui en
naïssent ». Aucun effet n’est assignable a {\it priori}. Il est
vrai que nous pouvons répéter l'expérience, que nous
avons pu constater des centaines de fois, que l’échauffement de l’eau provoque son ébullition. La conjonction
entre les deux événements apparaît dans notre expérience
passée comme une conjonction constante. Mais le pourquoi
de cette liaison ne nous est toujours pas donné. La répétition
à l'infini d’une énigme n’est pas une solution de cette
énigme : « La simple répétition d’une impression passée,
même à l'infini, n’engendrera jamais une nouvelle idée
originale comme celle de connexion nécessaire ; le nombre
d’impressions n’a, dans ce cas, pas plus d'effet que si nous
nous en tenions à une seule. » L'idée de connexion nécessaire ne peut donc provenir d’une impression de sensation.

Tournons-nous à présent vers les impressions de réflexion,
c’est-à-dire vers le sentiment que nous avons des opérations de notre esprit.
%23
%{\it }
N’avons-nous pas la conscience
immédiate d’être la cause des mouvements de notre
corps propre ? (C’est ce que Maine de Biran affirmera
au début du {\footnotesize XIX}$^\text{e}$ siècle.) N’avons-nous pas également le
sentiment de notre pouvoir efficace sur nos volitions, sur
le cours de nos idées ? (C’est ce que soutiendra W. James
à la fin du {\footnotesize XIX}$^\text{e}$ siècle.) Mais Hume va répondre négativement 
à ces deux questions. Nous avons bien l’impression,
par exemple, d’une succession, d’une conjonction constante
entre notre intention d’accomplir un mouvement, et ce
mouvement lui-même. Nous constatons que nous voulons
et puis que le mouvement s’effectue. Mais ici encore les
deux événements nous sont donnés dans l’expérience
sans que nous saisissions en nous un véritable « pouvoir
efficace ». Nous constatons que la volonté paraît avoir un
effet sur la langue ou sur les doigts, non sur le cœur et sur
le foie : « Cette question ne nous embarrasserait pas si
nous avions conscience d’un pouvoir dans le premier
cas. » L’homme qui vient d’être frappé de paralysie est
tout surpris de ne pouvoir remuer une jambe. Il ressent
son vouloir comme nous ressentons le nôtre. Mais son
vouloir n’est suivi d’aucun effet. Notre vouloir est bien
suivi d’un effet, mais nous ne comprenons pas plus notre
pouvoir que le paralytique ne comprend son impuissance.
D'ailleurs « l'anatomie » nous enseigne que le mouvement
ne se produit que par le jeu des os et des muscles, par
l’action des « esprits animaux » que nous n’avons aucunement 
conscience de déclencher. Si nous réfléchissons
honnêtement, nous saisissons que le lien entre notre désir
et les mouvements de notre corps reste aussi mystérieux
que si nous avions «le pouvoir d’écarter les montagnes ou
de contrôler les astres dans leurs orbites ».

La maîtrise de l'esprit sur lui-même n’est pas plus
claire que sa maîtrise sur le corps. Nous constatons sans
le comprendre que nos pensées nous obéissent mieux que
nos sentiments, que « nous sommes plus maîtres de nos
pensées le matin que le soir, à jeun qu’après un repas
%24
%{\it }
copieux ». « Où est alors le pouvoir dont nous prétendons
être conscients ? » Dans tous les cas nous avons seulement
des impressions de contiguité et de succession entre les
événements. Et cependant l’idée de causalité est tout autre
chose : « Un objet peut être contigu et antérieur à un autre
sans qu’on le considère comme sa cause. » Mais cette idée de
causalité, de connexion nécessaire n’est nullement éclaircie.

Tout au long de cette enquête Hume a suivi de très
près la critique de Malebranche. Celui-ci concluait, comme
on sait, que l’efficace n’est nullement dans le phénomène
abusivement appelé cause et qui n’est qu’une occasion.
Malebranche refuse aux créatures, même les plus élevées,
tout pouvoir causal (« un ange ne saurait remuer un
fétu de paille »).et place la causalité en Dieu seul. Cette
hypothèse métaphysique est pour Hume « trop étrange
pour jamais apporter avec elle la conviction à aucun
homme suffisamment informé de la faiblesse de la raison
humaine ». Les grands systèmes métaphysiques nous
proposent un voyage « au pays des fées »; ce sont des
fictions qui débordent le domaine de l’expérience, des
impressions réellement données : « Notre ligne est trop
courte pour sonder l’immensité de pareils abîmes. »
L'analyse de Hume va s’orienter dans un domaine tout
différent. Elle va se placer à un point de vue nouveau qui
est le point de vue psychologique. Hume entend bien ne
pas sortir du domaine des faits et le seul fait incontestable
est ici un fait psychologique. Bien que l « enquête » philosophique 
ne nous ait livré aucune impression originale
de causalité, c’est un fait que nous croyons tous à la cause,
à la connexion nécessaire, au pouvoir efficace. C’est donc
notre croyance qu’il faut expliquer. Nous n’avons l’expérience 
d’aucune force efficace présente dans un phénomène,
mais pourquoi croyons-nous que ce phénomène sera nécessairement 
suivi d’un autre ? Il ne s’agit pas de fonder
rationnellement le principe de causalité, il s’agit de rendre
compte de notre croyance. Pour résoudre le problème,
Hume le réduit à sa dimension psychologique. Il ne s’agit
%25
%{\it }
pas tant de chercher le fondement légitime du principe
de causalité que d’élucider son origine psychologique.

Or, la constante conjonction des objets, nous l’avons
dit, n’a aucune influence sur les objets eux-mêmes. Le
millième cas de conjonction est tout aussi énigmatique
que le premier. Seulement cette conjonction constante a
de l'influence sur notre esprit. L’habitude de voir deux
objets liés ensemble produit en nous une tendance très
forte à attendre le deuxième objet si le premier se présente
à nous une fois de plus : « Après une répétition de cas
semblables, l’esprit est porté par habitude lorsqu’apparaît
un des deux événements à attendre son concomitant
ordinaire et à croire qu’il existera. » C’est la transition,
le glissement aisé de l’imagination d’un objet à son concomitant 
habituel qui fournit l’unique impression d’où
dérive l’idée de connexion nécessaire. Dès lors celle-ci
n’a aucune portée ontologique, elle n’a qu’un sens 
psychologique : « Elle n’est qu’une impression intérieure de
l'esprit, une détermination à porter nos pensées d’un objet
à l’autre. » La causalité n’est donc pas un principe qui
régit les choses, elle n’est qu’un « principe de la nature
humaine ». Hume le déclare expressément : « La nécessité
est quelque chose qui existe dans l’esprit, non dans les
objets. » Hume a eu le sentiment très vif de faire œuvre
révolutionnaire par son analyse de la causalité. Désormais,
dit-il, « la connexion nécessaire dépend de l’inférence au
lieu que ce soit l’inférence qui dépende de la connexion
nécessaire ». Et il a bien saisi que ce renversement des
perspectives pouvait paraître paradoxal et scandaleux :
« Quoi ! l'efficacité des causes se trouve dans la détermination 
de l’esprit ! Comme si les causes n’opéraient pas en
toute indépendance de l'esprit et ne continueraient pas
d’opérer même s’il n’y avait aucun esprit pour les contempler 
et raisonner à leur sujet. La pensée peut bien dépendre
des causes pour son opération mais non les causes de la
pensée ! C’est renverser l’ordre naturel ! »

Ce renversement paradoxal n’en est pas moins la clef
%26
%{\it }
de toutes les réductions psychologiques opérées par Hume
à propos de toutes les idées qu’il soumet à son enquête.
La notion de probabilité subit le même déplacement
psychologique, qu’il s’agisse d’une probabilité confuse
lorsque notre inférence manque de conviction, parce que
l'expérience passée qui la détermine est trop éloignée ou
n’a pas été assez souvent répétée, ou qu’il s’agisse d’une
probabilité « philosophique » c’est-à-dire mathématique,
susceptible de se calculer : Par exemple en jetant un dé
à six faces, nous disons qu’il y a une chance sur six que
telle face (celle qui porte le n° 3 par exemple) sorte. Il
n’y a certes « rien de semblable à la Chance dans le monde »
et la probabilité est — comme la causalité — une impression ou plus exactement une impulsion de notre esprit.
Seulement l'élan de l’imagination qui, dans l’expérience
de la causalité est dirigé par l'habitude dans une seule
direction, se divise ici en autant de tendances qu’il y a
de faces sur le dé. Si quatre faces portent le même chiffre,
les deux autres étant différentes, nous disons que le chiffre
des premières a quatre chances sur six de sortir parce que
« impulsion moindre détruit l’impulsion plus forte dans
la mesure de sa propre force ».

Hume applique la même méthode psychologique dans sa
critique de la notion de substance. Il retrouve ainsi à propos
de la substance matérielle le point de vue de Berkeley. Le
sens commun croit à l’existence de «substances », à la permanence 
des choses bien que nous apercevions constamment
de nouveaux objets. Quand je sors de mon bureau, je ne vois
plus la table, la bibliothèque, le tapis, mais je ne doute
pas de les retrouver tout à l’heure. Le sens commun admet
que « les perceptions discontinues sont reliées par une
existence réelle dont nous n’avons pas conscience ». Cette
bibliothèque vitrée en noyer est une chose qui existe,
même si je ne la vois pas ! Une telle affirmation, cependant,
n’est pour le philosophe rigoureux qu’une hypothèse
gratuite. Mon expérience me livre des phénomènes discontinus, rien de plus, et on ne saurait fonder en raison
%27
%{\it }
l'affirmation des « substances ». Hume substitue à cette
affirmation ontologique une simple explication psychologique 
de notre croyance aux substances. En fait notre
imagination fond les unes dans les autres les apparences
semblables et successives. L’habitude de retrouver constamment 
les mêmes phénomènes à leur place engendre une
attente irrésistible. L’affirmation de l'existence de la
bibliothèque n’est que l'expression de la certitude que
j'ai de la retrouver en pénétrant dans mon cabinet de
travail. Mon imagination glisse si aisément le long de la
succession des apparences qu’elle métamorphose la simple
succession en identité véritable.

Mais Hume va beaucoup plus loin que Berkeley, car il
étend à la notion de substance spirituelle la critique de la
notion de substance matérielle. Le sens commun affirme
l'existence de substances spirituelles, il m’attribue par
exemple un moi un et identique, ce moi n’étant nullement
une impression mais ce à quoi nos diverses impressions ou
idées sont censées se rapporter. Mais je n’ai pas réellement
l'expérience d’un tel sujet permanent. J’assiste seulement à
l’incessant défilé de mes états de conscience. Souffrances et
joies, passions diverses, se succèdent sans jamais pouvoir
exister toutes ensemble en même temps. Le support de tous
ces phénomènes, cette « substance » qu’on appelle l'âme ou
le moi n’est à la lettre qu’une invention de notre imagination.
C’est ici encore l’imagination qui prolonge l’élan que la
mémoire lui communique et « comme une galère mise en
mouvement par les rames court sur son erre sans nouvelle
impulsion ». Habile à masquer la discontinuité de tous les phénomènes, elle nous persuade, en glissant aisément d’un état
à un autre, de l'identité et de la substantialité de notre être.

%%%%%%%%%%%%%%%%%%%%%%%%%%%%%%%%%%%%%%%%%%%%%%%%%%%%%%%%%%%%%%%%%%%%%%%%%%%
