
%%%%%%%%%%%%%%%%%%%%%
\section{Pratique de la philosophie}
%%%%%%%%%%%%%%%%%%%%%
%{\bf }  {\footnotesize V}$^\text{e}$ siècle  {\it }
{\bf S{\footnotesize PIRITUALISME}}

\begin{itemize}[leftmargin=1cm, label=\ding{32}, itemsep=1pt]
\item {\footnotesize ÉTYMOLOGIE} : latin {\it spiritualis}, où
{\it spiritalis}, « propre à la respiration »,
« spirituel ».
\item {\footnotesize SENS ORDINAIRE} : doctrine consistant à soutenir que : 1.
l'esprit ne se réduit pas à la vie ; 2.
la vie ne se réduit pas à la matière
(opposé à matérialisme).
\item {\footnotesize MÉTAPHYSIQUE} : théorie philosophique
selon laquelle il existe deux substances radicalement distinctes, l’esprit et la nature, la première étant
caractérisée par la pensée et la
liberté, la seconde par l'étendue et
le mouvement.
\item {\footnotesize MORALE} : théorie
selon laquelle la vie humaine
comporte ses propres fins (la justice, la liberté...) qui ne sauraient se
réduire aux seuls intérêts du vivant,
et qui peuvent même être amenées
à les contredire.
\end{itemize}


 

Dans son sens métaphysique, le terme
spiritualisme recouvre une longue tradition qui remonte à Anaxagore* ({\footnotesize V}$^\text{e}$ siècle av. J.-C.) et qui se perpétue dans toute la
philosophie  cartésienne  (Descartes,
Malebranche, Spinoza, Leibniz,...). Pour
Anaxagore, l'esprit est le rous, c'est-à-dire
« l'âme » ou « le souffle » fluide et mouvant, qui s'oppose à la matière, solide et
inerte. La philosophie idéaliste dans son
ensemble admet elle aussi cette opposition catégorique entre l'esprit (principe
d'unification et de connaissance) et la
matière (étalée et inerte) ainsi que la
suprématie du premier.

Le dualisme cartésien radicalise encore
cette opposition mais reconnaît l’autonomie de la matière : celle-ci obéit à ses
propres lois ({\it cf.} Mécanisme) et n'est
donc plus totalement subordonnée à
l'esprit. En outre, chez Descartes, la
conscience cesse d’être un « souffle » ou
une « flamme » pour devenir un principe
de connaissance et de représentation de
l'univers tout entier. La question qui sera
posée par la suite concerne les relations
(difficilement intelligibles) entre deux
substances tenues pour radicalement
étrangères l’une à l’autre.

Au {\footnotesize XX}$^\text{e}$ siècle, un débat très vif continue
d'opposer les tenants du matérialisme
({\it cf.} Esprit) et les tenants d’un spiritualisme métaphysique renouvelé. Selon
Bergson (qui représente cette seconde
position) l'esprit n’est pas un effet du
corps, et la vie en général est irréductible à la matière. Loin d’être la clé du
fonctionnement de l'esprit, le cerveau
n'en est que l'instrument et le support :
il n'est que «l’ensemble des dispositifs
qui permettent à l'esprit de répondre à
l'action des choses par des réactions
motrices [..] dont la justesse assure la
parfaite insertion de l'esprit dans la réalité » (Bergson, « L'âme et le corps », {\it L'Énergie spirituelle}).

\begin{itemize}[leftmargin=1cm, label=\ding{32}, itemsep=1pt]
\item {\footnotesize TERME VOISIN} : dualisme ; idéalisme.
\item {\footnotesize TERME OPPOSÉ} : matérialisme.
\item {\footnotesize CORRÉLATS} : Âme ; esprit ;
mécanisme ; vitalisme.
\end{itemize}

%%%%%%%%%%%%%%%%%%%%%%%%%%%%%%%%%%%%%%%%%%%%%%%%%%%%%%%%%%%%%%%%%%%%%%%%
