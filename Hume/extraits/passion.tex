
%%%%%%%%%%%%%%%%%%%%% 8
\section{Passion et raison}
%%%%%%%%%%%%%%%%%%%%%
%{\footnotesize XVIII}$^\text{e}$ siècle
%{\it }
%65
%{\it }
Les passions ne peuvent être contraires à la raison que
dans la mesure où elles {\it s’accompagnent} d’un jugement ou
d’une opinion. Selon ce principe qui est si évident et si
naturel, c’est seulement en deux sens qu’une affection
peut être appelée déraisonnable. Premièrement, quand
une passion, telle que l’espoir ou la crainte, le chagrin ou
la joie, le désespoir ou la confiance, se fonde sur la supposition
de l’existence d’objets qui, effectivement, n’existent
pas. Deuxièmement, quand, pour éveiller une passion,
nous choisissons des moyens pertinents pour obtenir la
fin projetée et que nous nous trompons dans notre jugement
sur les causes et les effets. Si une passion ne se fonde
pas sur une fausse supposition et si elle ne choisit pas des
moyens impropres à atteindre la fin, l’entendement ne
peut ni la justifier ni la condamner. Il n’est pas contraire
à la raison de préférer la destruction du monde entier à
une égratignure de mon doigt. Il n’est pas contraire à la
raison que je choisisse de me ruiner complétement pour
prévenir le moindre malaise d’un Indien ou d’une personne
complétement inconnue de moi. Il est aussi peu contraire
à la raison de préférer à mon plus grand bien propre un
bien reconnu moindre et d’aimer plus ardemment celui-ci
que celui-la. Un bien banal peut, en raison de certaines
circonstances, produire un désir supérieur à celui qui naît
du plaisir le plus grand et le plus estimable ; et il n’y a là
rien de plus extraordinaire que de voir, en mécanique,
un poids d’une livre en soulever un autre de cent livres
grâce à l’avantage de sa situation. Bref, une passion doit
s’accompagner de quelque faux jugement pour être déraisonnable ;
même alors ce n’est pas, à proprement parler,
la passion qui est déraisonnable, c’est le jugement.

Les conséquences sont évidentes. Puisqu’une passion ne
peut jamais, en aucun sens, être appelée déraisonnable,
sinon quand elle se fonde sur une supposition erronée ou
%66
%{\it }
quand elle choisit des moyens impropres à atteindre la
fin projetée, il est impossible que la raison et la passion
puissent jamais s’opposer l’une à l’autre et se disputer le
commandement de la volonté et des actes. Au moment
même ou nous percevons l’erreur d’une supposition ou
l'insuffisance de certains moyens, nos passions cédent à
notre raison sans aucune opposition. Je peux désirer un
fruit parce que je lui accorde une saveur exquise : mais si
vous me convainquez de mon erreur, je cesse de le désirer.
Je peux vouloir accomplir certaines actions comme moyens
d’obtenir un bien désiré ; mais, comme ma volonté de ces
actions est seulement secondaire et qu’elle se fonde sur la
supposition que ces actions sont causes de l'effet projeté,
dés que je découvre l’erreur de cette supposition ces
actions me deviennent indifférentes. ({\it Traité de la nature
humaine}, liv. II, 3$^\text{e}$ partie, section III, trad. L{\footnotesize EROY}.)

%%%%%%%%%%%%%%%%%%%%%%%%%%%%%%%%%%%%%%%%%%%%%%%%%%%%%%%%%%%%%%%%%%%%%%%%
