
%%%%%%%%%%%%%%%%%%%%% 4
\section{Relations d'idées et connaissance des faits}
%%%%%%%%%%%%%%%%%%%%%
%{\footnotesize XVIII}$^\text{e}$ siècle
%{\it }

Tous les objets de la raison humaine ou de nos recherches
peuvent se diviser en deux genres, à savoir les relations
d’idées et les faits. Du premier genre sont les sciences de
la géométrie, de l’algébre et de l’arithmétique et, en bref,
toute affirmation qui est intuitivement ou démonstrativement
certaine. Le carré de l'hypoténuse est égal au
carré des deux côtés, cette proposition exprime une relation
entre ces figures. Trois fois cinq est égal à la moitié
de trente exprime une relation entre ces nombres. Les
propositions de ce genre, on peut les découvrir par la
seule opération de la pensée, sans dépendre de rien de ce
qui existe dans l’univers. Même s’il n’y avait jamais eu
de cercle ou de triangle dans la nature, les vérités démontrées
par Euclide conserveraient pour toujours leur certitude
et leur évidence.

Les faits, qui sont les seconds objets de la raison humaine,
on ne les établit pas de la même manière ; et l’évidence de
leur vérité, aussi grande qu’elle soit, n’est pas d’une
nature semblable à la précédente. Le contraire d’un fait
quelconque est toujours possible, car il n’implique pas
contradiction et l'esprit le conçoit aussi facilement et
aussi distinctement que s’il concordait pleinement avec
la réalité. Le soleil ne se lèvera pas demain, cette proposition
n’est pas moins intelligible et elle n’implique pas
%59
%{\it }
plus contradiction que l'affirmation : il se lèvera. Nous
tenterions donc en vain d’en démontrer la fausseté. Si
elle était démonstrativement fausse, elle impliquerait
contradiction et l'esprit ne pourrait jamais la concevoir
distinctement. (Enquête sur l’entendement humain, section
IV, trad. L{\footnotesize EROY}.)

%%%%%%%%%%%%%%%%%%%%%%%%%%%%%%%%%%%%%%%%%%%%%%%%%%%%%%%%%%%%%%%%%%%%%%%%
