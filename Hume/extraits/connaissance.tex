
%%%%%%%%%%%%%%%%%%%%% 4
\section{Relations d'idées et connaissance des faits}
%%%%%%%%%%%%%%%%%%%%%
%{\footnotesize XVIII}$^\text{e}$ siècle
%{\it }

Tous les objets de la raison humaine ou de nos recherches
peuvent se diviser en deux genres, a savoir les relations
d’idées et les faits. Du premier genre sont les sciences de
la géométrie, de l’algébre et de l’arithmétique et, en bref,
toute affirmation qui est intuitivement ou démonstrati-
vement certaine. Le carré de Vhypoténuse est égal au
carré des deux cétés, cette proposition exprime une rela-
tion entre ces figures. Trois fois cing est égal a la moitié
de trente exprime une relation entre ces nombres. Les
propositions de ce genre, on peut les découvrir par la
seule opération de la pensée, sans dépendre de rien de ce
qui existe dans l’univers. Méme s’il n’y avait jamais eu
de cercle ou de triangle dans la nature, les vérités démon-
trées par Euclide conserveraient pour toujours leur certi-
tude et leur évidence.

Les faits, qui sont les seconds objets de la raison humaine,
on ne les établit pas de la méme maniére ; et l’évidence de
leur vérité, aussi grande qu’elle soit, n’est pas d’une
nature semblable a la précédente. Le contraire d’un fait
quelconque est toujours possible, car il n’implique pas
contradiction et Vesprit le congoit aussi facilement et
aussi distinctement que s’il concordait pleinement avec
la réalité. Le soleil ne se lévera pas demain, cette propo-
sition n’est pas moins intelligible et elle n’implique pas
%59
%{\it }
plus contradiction que affirmation : il se lévera. Nous
tenterions donc en vain d’en démontrer la fausseté. Si
elle était démonstrativement fausse, elle impliquerait
contradiction et lesprit ne pourrait jamais la concevoir
distinctement. (Enguéte sur l’entendement humain, sec-
tion IV, trad. L{\footnotesize EROY}.)

%%%%%%%%%%%%%%%%%%%%%%%%%%%%%%%%%%%%%%%%%%%%%%%%%%%%%%%%%%%%%%%%%%%%%%%%
