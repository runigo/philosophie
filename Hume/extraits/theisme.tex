
%%%%%%%%%%%%%%%%%%%%% 12
\section{Motifs raisonnables et mobiles passionnels du théisme}
%%%%%%%%%%%%%%%%%%%%%
%{\footnotesize XVIII}$^\text{e}$ siècle
%{\it }

La doctrine d’un Dieu suprème, créateur de la nature,
est trés ancienne. Elle s’est répandue dans des pays vastes
et fort peuplés et y a été adoptée par des gens de tout
rang et de toute condition. Cependant quiconque penserait
que cette doctrine a dû son succés à la force décisive des
raisons effectivement invincibles sur lesquelles elle est
%76
%{\it }
indubitablement fondée, montrerait par là qu’il connaît
bien mal l'ignorance et la stupidité du peuple et les
préjugés incurables qui l’attachent a ses superstitions.
Même de nos jours, et en Europe, demandez a un homme
du commun pourquoi il croit en un Créateur tout-puissant :
Jamais il n’alléguera la beauté des causes finales qu’il
ignore tout a fait. Il n’étendra pas la main pour vous
inviter à admirer la souplesse et la variété des jointures
des doigts, tous flexibles du même côté, le contrepoids
que les autres doigts recoivent du pouce, la douceur des
parties charnues à l’intérieur de la main, et toutes les
conditions qui rendent ce membre parfaitement adapté
à sa fonction. Car à tout cela il est depuis longtemps
accoutumé ; tout cela il le voit sans y faire attention,
avec une totale indifférence. Mais l'homme du commun
vous parlera de la mort étonnante et soudaine de telle
personne, de la chute et des contusions de telle autre, de
l'excessive sécheresse de telle saison, du froid et des pluies
de telle autre. Voila le genre d’événements qu’il impute a
l'opération directe de la Providence. De tels accidents qui,
pour des philosophes de bon raisonnement, constituent
les difficultés majeures contre la thése d’une Intelligence
suprême, sont pour l’homme du peuple les seuls arguments
en sa faveur. ({\it Histoire naturelle de la religion}, chap. VI.)

%%%%%%%%%%%%%%%%%%%%%%%%%%%%%%%%%%%%%%%%%%%%%%%%%%%%%%%%%%%%%%%%%%%%%%%%
