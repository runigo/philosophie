
%%%%%%%%%%%%%%%%%%%%% 15
\section{Le mal}
%%%%%%%%%%%%%%%%%%%%%
%{\footnotesize XVIII}$^\text{e}$ siècle
%{\it } 
%80
%{\it }
Du concours de ces {\it quatre} circonstances dépend donc la
totalité ou la plus grande partie du mal de nature : si
toutes les créatures vivantes étaient incapables de peine,
ou si le monde était administré par des volitions particulières,
le mal n’eût jamais trouvé accès dans l’univers ;
et si les animaux étaient doués d’une large provision de
forces et de facultés, au-dela de ce qu’exige la stricte
nécessité, ou si les divers ressorts et principes de l’univers
étaient assez exactement construits pour conserver toujours
le tempérament juste et le juste milieu, il y aurait
eu nécessairement trés peu de mal en comparaison de ce
que nous ressentons effectivement. Que prononcerons-nous donc
en cette occasion ? Dirons-nous que ces circonstances
ne sont pas nécessaires et qu’elles auraient
aisément pu être changées dans l’agencement de l’univers ?
Cette décision semble trop présomptueuse pour des
créatures aussi aveugles et ignorantes que nous. Soyons
plus modestes en nos conclusions. Convenons que, si la
bonté de la Divinité — j’entends une bonté telle que
celle de l’homme — pouvait être établie sur des raisons
a priori passables, ces phénoménes, si facheux qu’ils
fussent, ne suffiraient pas à renverser le dit principe,
mais pourraient aisément, de quelque manière inconnue,
se concilier avec lui. Mais néanmoins affirmons que, comme
cette bonté n’est pas préalablement établie, mais doit
être inférée d’aprés les phénomènes, il ne peut y avoir
aucun motif en faveur d’une telle inférence, quand il y a
tant de maux dans l’univers, et qu’il eût été si aisé d’y
remédier, pour autant que l’entendement humain peut
être admis a juger en un tel sujet. Je suis assez sceptique
pour convenir que les mauvaises apparences, nonobstant
tous mes raisonnements, peuvent être compatibles avec
des attributs tels que vous les supposez : mais assurément
elles ne sauraient jamais {\it prouver} ces attributs. Une telle
%81
%{\it } 
conclusion ne saurait résulter du scepticisme : il faut
qu’elle provienne des phénomènes, et de notre confiance
dans les raisonnements que nous en déduisons.

Voyez, autour de vous, cet univers. Quelle immense
profusion d’êtres animés et organisés, sentants et agissants !
Vous admirez cette variété et cette fécondité
prodigieuses. Mais examinez d’un peu plus près ces existences
vivantes, les seules qu’il vaille la peine de considérer.
Combien elles sont hostiles et destructrices les unes
pour les autres ! Combien insuffisantes, toutes tant qu’elles
sont, pour leur propre bonheur! Combien méprisables ou
odieuses au spectateur ! Le tout n’éveille pas d’autre idée
que celle d’une nature aveugle, imprégnée par un grand
principe vivifiant, et laissant tomber de son giron, sans
discernement ni soin maternel, ses enfants estropiés et
avortés !

Ici le système manichéen s’offre comme une hypothèse
propre a résoudre la difficulté ; et, sans doute, a certains
égards, il est trés spécieux, et présente plus de probabilité
que l’hypothése ordinaire, en ce qu’il donne une explication
plausible de l’étrange mélange de bien et de mal
qui parait dans la vie. Mais si nous considérons d’autre
part l'uniformité et l'accord parfaits des parties de l’univers,
nous n’y découvrirons aucune marque du combat
d’un être malveillant contre un être bienveillant. Il y a
sans doute une opposition de peines et de plaisirs dans les
affections des créatures sentantes ; mais toutes les opérations
de la nature ne s’accomplissent-elles pas par une
opposition de principes, celle du chaud et du froid, de
l'humide et du sec, du léger et du lourd ? La vraie conclusion,
c’est que la source originelle de toutes choses est
entièrement indifférente à tous ces principes, et ne préfére
pas plus le bien au mal que la chaleur au froid, la sécheresse
a l'humidité, ou le léger au lourd.

Il y a {\it quatre} hypothèses possibles touchant les premiéres
causes de univers : {\it qu}’elles sont douées d’une parfaite
bonté, {\it qu}’elles possédent une parfaite malice, {\it qu}’elles
%82
%{\it }
sont opposées et possèdent à la fois de la bonté et de la
malice, {\it qu}’elles ne possédent ni bonté ni malice. Des
phénoménes mélangés ne sauraient jamais prouver les
deux premiers principes, qui sont exempts de mélange.
L’uniformité et la fermeté des lois générales semblent
s’opposer au troisième. Le quatrième semble donc de
beaucoup le plus probable.

.. Halte! halte! s’écria Déméa ; où votre imagination
vous entraine-t-elle ? J’ai fait alliance avec vous en vue
de prouver la nature incompréhensible de l’\^Etre divin et
de réfuter les principes de Cléanthe, qui voudrait mesurer
toutes choses au moyen d’une régle et d’un étalon humains.
Mais je vous vois à présent vous jeter dans tous les lieux
communs des plus grands libertins et incroyants, et trahir
cette sainte cause que vous épousiez en apparence. êtes-vous
donc secrétement un plus dangereux ennemi que
Cléanthe lui-méme ? — Et vous, tardez-vous tant a vous
en apercevoir ? répondit Cléanthe. Croyez-moi, Déméa,
votre ami Philon, depuis le commencement, n’a fait que
s’amuser à nos dépens a tous deux ; et l’on doit confesser
que le raisonnement peu judicieux de notre théologie
vulgaire n’a donné qu’une trop juste prise a sa raillerie. La
totale infirmité de la raison humaine, l’incompréhensibilité
absolue de la nature divine, la grande et universelle misère
et la méchanceté plus grande encore des hommes : assurément
ce sont là d’étranges lieux communs pour être si tendrement
chéris d’ecclésiastiques et de docteurs orthodoxes.
({\it Dialogues sur la religion naturelle}, 11$^\text{e}$ partie, trad. Davin.)

%%%%%%%%%%%%%%%%%%%%%%%%%%%%%%%%%%%%%%%%%%%%%%%%%%%%%%%%%%%%%%%%%%%%%%%%
