
%%%%%%%%%%%%%%%%%%%%% 5
\section{D'où vient la croyance en l'existence ininterrompue des objets ?}
%%%%%%%%%%%%%%%%%%%%%
%{\footnotesize XVIII}$^\text{e}$ siècle%
%{\it }

Une interruption dans l’apparition aux sens n’implique
pas nécessairement une interruption dans l'existence.
Admettre l’existence continue des objets ou perceptions
sensibles n’implique aucune contradiction. Nous pouvons
laisser aisément libre cours à notre tendance à admettre
cette existence. Quand l'exacte ressemblance de nos
perceptions nous leur fait attribuer l’identité, nous pouvons
écarter l’interruption apparente par la fiction d’un
être continu qui peut remplir ces intervalles et conserver
à nos perceptions une parfaite et entière identité.

Mais, comme ici nous n’imaginons pas seulement cette
existence continue, mais que nous y croyons, la question
est de savoir d’où naît une pareille croyance? Cette
question nous mène a la quatrième partie de ce système.
J’ai déja prouvé que la croyance, en général, ne consiste
en rien d’autre que la vivacité d’une idée ; et qu’une idée
peut acquérir cette vivacité par sa relation à quelque
impression présente. Les impressions sont naturellement
les perceptions les plus vives de l’esprit ; cette qualité est
en partie transférée par la relation à toute idée conjointe.
La relation produit un passage coulant de l’impression à
l'idée et même engendre une tendance à réaliser ce passage.
L’esprit glisse si aisément d’une perception à l’autre
qu’il perçoit à peine le changement et qu’il conserve pour
la seconde une part considérable de la vivacité de la
première. Il est mû par l’impression vive et cette vivacité
%60
%{\it }
est transférée a l’idée reliée sans grande atténuation lors
de la transition en raison du passage coulant et de la
tendance de l’imagination.

Mais admettez que cette tendance naisse d’un autre
principe que de cette relation ; évidemment elle doit avoir
toujours le même effet, elle transfére la vivacité de l'impression
a l’idée. Or c’est exactement le cas présent. Notre
mémoire nous présente un nombre énorme d’exemples
de perceptions parfaitement semblables les unes aux
autres qui reviennent à différents intervalles de temps
après de considérables interruptions. Cette ressemblance
nous donne une tendance à considérer comme identiques
ces perceptions interrompues; et aussi une tendance à
les relier par une existence continue pour justifier cette
identité et éviter la contradiction dans laquelle nous
enveloppe nécessairement, semble-t-il, l'apparition discontinue
de ces perceptions. Ici nous avons donc une
tendance a feindre l’existence continue de tous les objets
sensibles; et comme cette tendance naît de certaines
impressions vives de la mémoire, elle confére de la vivacité
à la fiction ; ou, en d’autres termes, elle nous fait croire à
l'existence continue des corps. Si parfois nous attribuons
l'existence continue à des objets qui nous sont parfaitement
nouveaux et de la constance et de la cohérence
desquels nous n’avons aucune expérience, c’est parce
que la manière dont ils se présentent à nos sens ressemble
à celle des objets constants et cohérents ; cette ressemblance
est source de raisonnement et d’analogie et nous
porte à attribuer des qualités identiques à des objets
semblables. ({\it Traité de la nature humaine}, liv. I, 4$^\text{e}$ partie,
section II, trad. L{\footnotesize EROY}.)

%%%%%%%%%%%%%%%%%%%%%%%%%%%%%%%%%%%%%%%%%%%%%%%%%%%%%%%%%%%%%%%%%%%%%%%%
