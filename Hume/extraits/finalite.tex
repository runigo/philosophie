
%%%%%%%%%%%%%%%%%%%%% 14
\section{La finalité n'est pas une providence (discours de Philon)}
%%%%%%%%%%%%%%%%%%%%%
%{\footnotesize XVIII}$^\text{e}$ siècle
%{\it }

Pourquoi y a-t-il si peu que ce soit de misère dans le
monde ? Ce n’est pas par hasard assurément. C’est donc
par quelque cause. Est-ce par l’intention de la Divinité ?
Mais elle est parfaitement bienveillante. Est-ce contraire
a son intention ? Mais elle est toute-puissante. Rien ne
peut ébranler la solidité de ce raisonnement, si court,
si clair, si décisif, A moins d’affirmer que ces sujets passent
toute humaine capacité et que nos mesures ordinaires
de la verité et de la fausseté n’y sont pas applicables :
théorie que j’ai soutenue d’un bout a Vautre, mais que
%79
%{\it }
vous avez, depuis le commencement, rejetée avec dédain
et indignation.

Mais je consentirai à me retirer encore de ce retranchement,
car je nie que vous puissiez jamais m’y forcer ;
j’accorderai que l’existence de la peine ou de la misère
chez l'homme soit {\it compatible} avec celle d’un pouvoir et
d'une bonté infinis chez la Divinité, même selon le sens
que vous donnez à ces attributs : en quoi toutes ces concessions
vous avancent-elles ? Une compatibilité simplement
possible ne suffit pas. Vous devez {\it prouver} l’existence de
ces attributs purs, sans mélange et sans borne, d’aprés
les présents phénoménes, mélangés et confus, et d’aprés
ceux-la seulement. Entreprise pleine d’espoir! Si purs
et si peu mélangés que fussent les phénoménes, encore, étant
finis, seraient-ils insuffisants dans ce but. Combien davantage
alors qu’ils sont en outre si disparates et si discordants !

Ici, Cléanthe, je me sens à l’aise dans mon argumentation.
Ici je triomphe. Auparavant, quand nous argumentions
touchant les attributs naturels d’intelligence
et de dessein, j’avais besoin de toute ma subtilité sceptique
et métaphysique pour échapper a vos prises. En bien des
spectacles de l’univers et de ses parties, de ces derniéres
surtout, la beauté et la convenance des causes finales
nous frappent avec une force si irrésistible, que toutes les
objections paraissent — ce que je crois qu’elles sont en
effet — arguties et sophismes purs ; et nous ne pouvons
alors imaginer comment il nous fut jamais possible d’y
accorder quelque poids. Mais il n’est pas de spectacle de
la vie humaine ou de la condition de l’humanité, d’où
nous puissions, sans la plus grande violence, inférer les
attributs moraux, ou apprendre à connaitre cette infinie
bienveillance, jointe à un infini pouvoir et a une sagesse
infinie, qu’il nous faut découvrir uniquement par les
yeux de la foi. C’est votre tour maintenant de tirer la
rame fatigante, et de soutenir vos subtilités philosophiques
contre les clairs préceptes de la raison et de l’expérience.
({\it Dialogues sur la religion naturelle}, 10$^\text{e}$ partie.)

%%%%%%%%%%%%%%%%%%%%%%%%%%%%%%%%%%%%%%%%%%%%%%%%%%%%%%%%%%%%%%%%%%%%%%%%
