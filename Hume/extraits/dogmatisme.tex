
%%%%%%%%%%%%%%%%%%%%% 7
\section{Dogmatisme spontané et scepticisme réflexif}
%%%%%%%%%%%%%%%%%%%%%
%{\footnotesize XVIII}$^\text{e}$ siècle
%{\it }

La vue intense de ces multiples contradictions et imperfections
de la raison humaine m’a tant excité, elle a tant
échauffé mes pensées que je suis prêt a rejeter toute
croyance et tout raisonnement et que je ne peux plus
considérer une opinion même comme plus probable ou
%63
%{\it }
comme plus vraisemblable qu’une autre. Où suis-je ?
Et que suis-je ? De quelles causes tiré-je mon existence
et à quelle condition retournerai-je ? Quel est l’être dont
je dois briguer la faveur, et celui dont je dois craindre la
colère ? Quels êtres m’entourent? Sur qui ai-je une
influence, et qui en exerce une sur moi? Toutes ces
questions me confondent et je commence à me trouver
dans la condition la plus déplorable qu’on puisse imaginer,
enveloppé de l’obscurité la plus profonde et absolument
privé de l’usage de tout membre et de toute
faculté.

Trés heureusement il se produit que, puisque la raison
est incapable de chasser ces nuages, la Nature elle-même
suffit à y parvenir; elle me guérit de cette mélancolie
philosophique et de ce délire soit par relâchement de la
tendance de l’esprit, soit par quelque divertissement et
par une vive impression sensible qui effacent toutes ces
chimères. Je dîne, je joue au tric-trac, je parle et me réjouis
avec mes amis ; et si, aprés trois ou quatre heures d’amusement,
je voulais revenir à mes spéculations, celles-ci
me paraitraient si froides, si forcées et si ridicules que
je ne pourrais trouver le c{\oe}ur d’y pénétrer tant soit
peu.

Alors donc je me trouve absolument et nécessairement
déterminé à vivre, à parler et à agir comme les autres
hommes dans les affaires courantes de la vie. Mais, en
dépit de mon inclination naturelle et du cours de mes
esprits animaux et de mes passions qui me ramènent à
l'indolente croyance aux maximes générales du monde,
je sens toujours que subsiste ma précédente disposition,
si bien que je suis prêt à jeter au feu tous mes livres et
tous mes papiers et à me résoudre à ne plus jamais renoncer
aux plaisirs de la vie pour l’amour du raisonnement et de
la philosophie. Car tels sont mes sentiments dans l’humeur
chagrine qui me gouverne à présent. Je puis céder, mieux,
il faut que je cède au courant de la nature en me soumettant
à mes sens et à mon entendement ; et, par cette
%64
%{\it }
aveugle soumission, je montre très parfaitement ma
disposition sceptique et mes principes. Mais s’ensuit-il
que je doive lutter contre le courant de la Nature qui me
porte à l'indolence et au plaisir; que je doive me retirer
en quelque mesure du commerce et de la société des
hommes, qui est si agréable ; et que je doive me torturer
la pensée avec des subtilités et des sophismes au moment
même où je ne peux me prouver le caractère raisonnable
d'une application aussi pénible, ni avoir une suffisante
perspective d’arriver par son moyen à la vérité et à la
certitude ? Quelle est l'obligation où je suis de gaspiller
ainsi mon temps ? A quoi cela peut-il servir, qu'il s’agisse
du service de l’humanité ou de mon intérêt privé ? Non :
si je dois être un sot, comme le sont certainement tous ceux
qui raisonnent et croient à quoi que ce soit, mes sottises
seront du moins naturelles et agréables. Si je lutte contre
mon inclination, j’aurai une bonne raison pour lui résister :
et je ne serai plus entrainé à errer à travers des solitudes
désolées et de rudes passages, comme j’en ai rencontré
jusqu’ici.

Tels sont mes sentiments de mélancolie et d’indolence :
et certes je dois avouer que la philosophie n’a rien à leur
opposer : elle attend la victoire plus du retour d’une disposition
sérieuse et bien inspirée que de la force de la
raison et de la conviction. Dans tous les événements de la
vie, nous devons toujours conserver notre scepticisme.
Si nous croyons que le feu chauffe et que l’eau rafraichit,
c’est seulement parce que cela nous coûte beaucoup trop
de peine de penser autrement. Mieux, si nous sommes
philosophes, ce doit être seulement sur des principse
sceptiques et par l’inclination que nous ressentons à nous
employer de cette manière. Quand la raison est vive et
qu’elle se mêle à quelque penchant, il convient de lui
donner son assentiment. Quand il n’en est rien, elle
ne peut jamais avoir de titre à agir sur nous. ({\it Traité
de la nature humaine}, liv. I, 4$^\text{e}$ partie, section VII,
trad. L{\footnotesize EROY}.)

%%%%%%%%%%%%%%%%%%%%%%%%%%%%%%%%%%%%%%%%%%%%%%%%%%%%%%%%%%%%%%%%%%%%%%%%
