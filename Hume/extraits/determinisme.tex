
%%%%%%%%%%%%%%%%%%%%% 11
\section{Le déterminisme des actes humains s'accorde mieux avec les exigences morales que la théorie de libre-arbitre}
%%%%%%%%%%%%%%%%%%%%%
%{\footnotesize XVIII}$^\text{e}$ siècle
%{\it }

Il n’y a pas de méthode de raisonnement plus commune,
et pourtant il n’y en a pas de plus blamâble, que de tenter
de réfuter une hypothèse, dans les discussions philosophiques,
par le danger de ses conséquences pour la religion
%73
%{\it }
et la morale. Quand une opinion conduit a des absurdités,
elle est certainement fausse ; mais il n’est pas certain
qu’une opinion soit fausse parce qu’elle est de dangereuse
conséquence. Nous devons donc éviter absolument de
pareils lieux communs, car ils ne servent en rien la découverte
de la vérité ; ils servent seulement à rendre odieuse
la personne d’un adversaire. Cette remarque, je la fais
de manière générale, sans prétendre en tirer un avantage.
Je me soumets franchement à un examen de ce genre et
j’oserai affirmer que les deux doctrines de la nécessité et
de la liberté, telles qu’elles sont exposées plus haut, non
seulement s’accordent avec la morale, mais sont absolument
essentielles pour la soutenir.

La nécessité peut se définir de deux manières, qui répondent
aux deux définitions de la {\it cause} dont elle constitue
une partie essentielle. Elle consiste, soit dans la constante
conjonction d’objets semblables, soit dans l’inférence de
l'entendement d’un objet à un autre. Or, la nécessité,
dans ces deux sens (qui, certes, reviennent au fond au
même), tout le monde a reconnu, bien que tacitement,
dans les écoles, en chaire et dans la vie courante, qu’elle
appartenait à la volonté humaine; jamais personne n’a
prétendu nier que nous puissions tirer des inférences au
sujet des actes humains, et que ces inférences se fondent
sur l'expérience de l’union d’actes semblables avec des
motifs, des inclinations et des circonstances semblables.
On peut différer sur un seul point : soit, peut-être, qu’on
refuse de donner le nom de nécessité à cette propriété des
actes humains ; mais, du moment qu’on en comprend le
sens, le mot, j’espére, ne peut causer aucun trouble ; soit
qu’on soutienne qu’il est possible de découvrir quelque
chose de plus dans les opérations de la matière. Mais ce
point, il faut le reconnaitre, ne peut être d’aucune conséquence
pour la morale ou la religion, de quelque importance
qu’il puisse être pour la philosophie naturelle ou la
métaphysique. Nous pouvons nous tromper ici en affirmant
qu’il n’y a pas d’idée d’une autre nécessité ou
%74
%{\it }
connexion dans les actions des corps; mais, assurément,
nous n’attribuons rien aux actes de l'esprit que ce que
chacun leur accorde et doit leur accorder de bon gré. Nous
ne changeons aucune circonstance du système orthodoxe
recu a l'égard de la volonté ; nous le changeons seulement
dans ses circonstances relatives aux objets et aux causes
matérielles. Il ne peut donc rien y avoir de plus innocent,
au moins, que cette doctrine.

Comme toutes les lois se fondent sur des récompenses
et des punitions, on admet comme principe fondamental
que ces motifs ont une action régulière et uniforme sur
l'esprit et, tout à la fois, produisent les bonnes actions et
préviennent les mauvaises. Nous pouvons donner a cette
action le nom qu’il nous plait ; mais, comme elle est habituellement
conjointe a l’action, il faut la considérer comme
la cause et la regarder comme un exemple de la nécessité
que nous voulions établir ici.

Le seul objet propre de la haine ou de la vengeance est
une personne, une créature douée de pensée et de conscience ;
quand des actes criminels ou injustes éveillent
cette passion, c’est seulement par leur relation à une
personne, par leur connexion avec elle. Les actes sont,
par leur nature même, temporaires et périssables ; et
quand ils ne procédent pas de quelque cause dans le
caractére et les dispositions de l'homme qui les accomplit,
ils ne peuvent contribuer à accroître ni son honneur,
s’ils sont bons, ni son infamie, s’ils sont mauvais. Les
actes eux-mêmes peuvent être blâmables ; ils peuvent être
contraires à toutes les règles de la morale et de la religion ;
mais la personne n’en est pas responsable et, comme les
actes ne procédent de rien, en elle, qui soit durable et
constant, et qu’ils ne laissent derrière eux rien de cette
nature, il est impossible que la personne puisse devenir,
à cause d’eux, l’objet d’une punition ou d’une vengeance.
Donc, conformément au principe qui nie la nécessité et,
par conséquent, les causes, un homme est aussi pur et
sans tache après qu’il a commis le crime le plus horrible
%75
%{\it }
qu’au premier moment de sa naissance ; et son caractère
n’est en rien affecté par ses actions, puisque celles-ci n’en
dérivent pas et que la malignité des unes ne peut jamais
servir de preuve de la dépravation de l’autre.

On ne blâme pas les hommes pour des actions qu’ils
accomplissent à leur insu et par accident, quelles qu’en
puissent être les conséquences. Pourquoi ? sinon parce
que les principes de ces actions sont seulement temporaires
et qu'ils s’achèvent sur eux seuls. On blâme moins
les hommes pour des actions qu’ils accomplissent à la
hâte et sans préméditation que pour des actions qui
procédent d’une délibération. Pour quelle raison ? N’est-ce
pas que la précipitation du caractère, bien qu’elle soit une
cause et un principe constant dans l’esprit, agit seulement
par intervalles et ne corrompt pas tout le caractère ?
Et encore, le repentir efface tous les crimes s’il s’accompagne
d’une réforme de la vie et des moeurs. Comment peut-on
l'expliquer ? Sinon en affirmant que les actions rendent
une personne criminelle en tant qu’elles constituent des
preuves de l’existence de principes criminels dans l’esprit ;
quand un changement de ces principes fait qu’elles cessent
d’être de justes preuves, elles cessent pareillement d’être
criminelles. Mais, sauf dans la doctrine de la nécessité,
elles ne seraient jamais de justes preuves et, par suite,
elles.ne seraient jamais criminelles. ({\it Enquète sur l’entendement},
section VIII, trad. L{\footnotesize EROY}.)

%%%%%%%%%%%%%%%%%%%%%%%%%%%%%%%%%%%%%%%%%%%%%%%%%%%%%%%%%%%%%%%%%%%%%%%%
