
%%%%%%%%%%%%%%%%%%%%% 
\section{Le suicide et la providence}
%%%%%%%%%%%%%%%%%%%%%
%{\footnotesize XVIII}$^\text{e}$ siècle
%{\it } 
40) LE SUICIDE ET LA PROVIDENCE

Pour prouver que le suicide n’est pas une transgression
de notre devoir envers Dieu, les considérations suivantes
peuvent peut-être suffire. Dans le but de gouverner le
monde matériel, le Créateur tout-puissant a établi des
lois générales et immuables par lesquelles tous les corps,
de la planète la plus grande à la plus petite particule de
matière, sont maintenus dans leur sphère et leur fonction
propres. Pour gouverner le monde animal, il a doué toutes
les créatures vivantes de pouvoirs physiques et mentaux :
de sensations, de passions, d’appétits, de mémoire et de
jugement, qui déterminent ou règlent leur comportement
dans le cours de la vie auquel elles sont destinées. Ces
deux principes distincts, qui gouvernent le monde matériel
et le monde vivant, empiètent continuellement l'un sur
l'autre, entravent ou favorisent leurs opérations réciproques. 
Les pouvoirs de l'homme et des autres animaux
sont limités et orientés par la nature et les qualités des
corps environnants ; et les mouvements et les actions de
ces corps sont sans cesse modifiés par les opérations de
tous les animaux. L’homme est arrêté par des rivières
dans son passage, et les rivières, lorsqu’elles sont dirigées
de facon appropriée, communiquent leur force a des
%70
%{\it } 
machines qui sont au service de l'homme. Mais quoique
les domaines des pouvoirs de la matière et des pouvoirs des
êtres vivants ne soient pas gardés entièrement séparés,
il n’en résulte pas de désaccord ou de désordre dans la
création ; au contraire, du mélange, de l'union, du contraste
des divers pouvoirs de la matière inerte et des créatures
vivantes, résulte cette sympathie, cette harmonie, cet
accord qui fournit la preuve la plus sûre de la Suprême
Sagesse. La providence divine ne se montre pas immédiatement
dans chaque opération mais gouverne toute chose
par ces lois générales et immuables qui ont été établies
depuis le commencement du temps. Tous les événements,
en un sens, peuvent être appelés l’action du Tout-Puissant ;
ils procédent tous de ces pouvoirs dont il a doué ses créatures. 
Une maison qui tombe sous son propre poids n’est
ni plus ni moins détruite par la providence que telle autre
qui est démolie par des mains humaines ; car les facultés
de homme ne sont pas moins l’{\oe}uvre de la providence
que les lois du mouvement et de la gravitation. Le jeu
des passions, la décision du jugement, l’obéissance des
membres, tout est opération de Dieu ; et par les principes
qui régissent les êtres vivants aussi bien que par ceux qui
réglent les corps inertes, c’est Lui qui a institué le gouvernement
de l’Univers... Il n’y a pas d’événement — si
important soit-il à nos yeux — que Dieu ait soustrait
aux lois générales qui gouvernent l’Univers ou qu'il ait
réservé à son action particulière. Le bouleversement des
États, des Empires dépend ainsi du plus petit caprice
de la passion de quelques individus ; et la vie des hommes
est raccourcie ou prolongée par le plus petit accident de
l'atmosphére ou du régime alimentaire, par le soleil ou
par la tempète... Comme, d’une part, les éléments et les
autres parties inanimées de la création accomplissent leurs
opérations sans se soucier de l'intérét particulier et de la
situation des hommes, les hommes, de leur cété, sont
confiés a leur propre jugement, à leur décision personnelle,
dans les divers conflits de l'existence et ont le droit d'utiliser
%71
%{\it } 
chacune des facultés dont ils sont doués pour satisfaire
aux exigences de leur confort, de leur bonheur,
de leur conservation. Quelle est alors la signification du
principe selon lequel un homme qui, fatigué de la vie,
accablé de douleur et de misére, surmonte courageusement
toutes les craintes naturelles de la mort et s'évade de cette
scène cruelle — aurait encouru l'indignation de son
Créateur en empiétant sur le rôle de la divine providence
et en troublant l'ordre de l'univers ? Prétendons-nous que
le Tout-Puissant s'est réservé à lui-même — comme un
domaine particulier — le droit de disposer des vies humaines
et n'a pas soumis cet événement au même titre que les
autres aux lois générales par lesquelles l'univers est
gouverné ? Cela est complétement faux, car la vie des
hommes dépend des mêmes lois que la vie des autres
animaux ; toutes sont assujetties aux lois générales de la
matière et du mouvement...

Puisque la vie des hommes dépend, pour l'éternité, des
lois générales de la matière et du mouvement, celui qui
dispose de sa propre vie est-il criminel... ? Cela paraît
absurde. Toute action, tout mouvement de l'homme,
introduit quelque chose de nouveau dans l’ordre de quelques
éléments de la matière et détourne de leur cours ordinaire
les lois générales du mouvement. En mettant ensemble
ces conclusions, nous voyons que la vie humaine dépend
de ces lois générales et qu’il n’y a pas empiétement sur
le rôle de la Providence quand on les détourne ou qu'on
les altère ; n’est-ce pas la conséquence de la libre disposition
par chacun de sa propre vie ?

... Pour détruire l'évidence de cette conclusion, nous
devrions faire valoir une raison qui nous permettrait
d'excepter le cas particulier du suicide. Est-ce parce que
la vie humaine serait d'une si grande importance qu'il
serait présomptueux d'en confier la disposition au jugement
humain ? Mais la vie d'un homme n'a pas plus
d’importance pour l’univers que la vie d’une huître... Et
si la disposition de la vie humaine était si strictement
%72
%{\it } 
attribuée au domaine réservé du Tout-Puissant que les
hommes outrepasseraient leurs droits en en disposant
librement, alors il serait également criminel d’agir pour
sauver une vie ou pour la détruire. Si je détourne une
pierre qui va me tomber sur la tête, je trouble le cours de
la nature et j’envahis le domaine réservé du Tout-Puissant
en prolongeant ma vie au-delà de la durée qui... lui a été
assignée.

Une mouche, un insecte sont capables de détruire cet
être puissant dont la vie serait d’une telle importance !
Est-il donc absurde de supposer que l’humaine sagesse
peut légitimement disposer de ce qui dépend de causes
a ce point insignifiantes ? Il n’y aurait pas de crime pour
moi a détourner le Nil ou le Danube si j’étais capable de
réaliser de tels projets. Quel serait donc le crime de détourner
quelques onces de sang de leur cours naturel ?

N’enseignez-vous pas que lorsqu’un malheur m’arrive,
même a cause de la méchanceté de mes ennemis, c’est
mon devoir de me résigner a la Providence ? et que les
actions des hommes sont les opérations du Tout-Puissant
au même titre que les actions des êtres inanimés ? Donc
si je me tue d’un coup de ma propre épée, c’est en fait des
mains de la Divinité que je recois ma mort tout comme si
la mort m’était venue d’un lion, d’un précipice, ou d’une
fiévre. {\it (Essai sur le suicide.)} 

%%%%%%%%%%%%%%%%%%%%%%%%%%%%%%%%%%%%%%%%%%%%%%%%%%%%%%%%%%%%%%%%%%%%%%%%
