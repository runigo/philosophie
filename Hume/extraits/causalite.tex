
%%%%%%%%%%%%%%%%%%%%%
\section{Origine de l'idée de causalité}
%%%%%%%%%%%%%%%%%%%%%
%{\footnotesize XVIII}$^\text{e}$ siècle
%{\it }

Supposez qu’un homme, pourtant doué des plus puissantes
facultés de raison et de réflexion, soit soudain
transporté dans ce monde ; il observerait immédiatement,
certes, une continuelle succession d’objets, un événement
en suivant un autre ; mais il serait incapable de découvrir
autre chose. Il serait d’abord incapable, par aucun raisonnement,
d’atteindre l’idée de cause et d’effet, car les
pouvoirs particuliers qui accomplissent toutes les opérations
naturelles n’apparaissent jamais aux sens; et il
n’est pas {\it raisonnable} de conclure, uniquement parce
qu’un événement en précède un autre dans un seul cas,
que l'un est la cause et l’autre effet. Leur conjonction
peut être arbitraire et accidentelle. Il n’y a pas de raison
d'inférer l’existence de l'un de l’apparition de l’autre.
En un mot, un tel homme, sans plus d’expérience, ne
ferait jamais de conjecture ni de raisonnement sur aucune
%54
%{\it }
question de fait; il ne serait certain de rien d’autre que
de ce qui est immédiatement présent à sa mémoire et
à ses sens.

Supposez encore que cet homme ait acquis plus d’expérience
et qu’il ait vécu assez longtemps dans le monde
pour qu’il ait remarqué la conjonction constante d’objets
ou d’événements familiers; que résulte-t-il de cette
expérience ? Il infére immédiatement l’existence d’un
des objets de l’apparition de l’autre. Il n’a pourtant acquis,
par toute son expérience, aucune idée, aucune connaissance
du pouvoir caché par lequel l’un des objets produit
l'autre; et ce n’est par aucun progrès de raisonnement
qu’il est engagé a tirer cette conclusion. Mais il se trouve
toujours déterminé à la tirer ; et, même si on le convainquait
que son entendement n’a aucune part dans l’opération,
il continuerait pourtant le même cours de pensée. Il y a un
autre principe qui le détermine a former une telle conclusion.

Ce principe, c’est l'accoutumance, l’habitude. Car,
toutes les fois que la répétition d’une opération ou d’un
acte particulier produit une tendance a renouveler le
même acte ou la même opération sans l’impulsion d’aucun
raisonnement ou progrès de l’entendement, nous disons
toujours que cette tendance est effet de l’accoutumance.
En employant ce mot, nous ne prétendons pas que nous
avons donné la raison dernière d’une telle tendance.
Nous désignons seulement un principe de la nature
humaine, universellement reconnu et bien connu par ses
effets. Peut-être pouvons-nous ne pas pousser plus loin
nos recherches, ni prétendre donner la cause de cette
cause ; mais il faut que nous nous en contentions comme
du principe dernier que nous puissions assigner pour nos
conclusions tirées de l’expérience. C’est une satisfaction
suffisante que de pouvoir aller aussi loin ; et il n’y a pas
à nous irriter de l’étroitesse de nos facultés parce qu’elles
ne veulent pas nous conduire plus loin. Assurément, nous
avons ici, au moins, une proposition trés intelligible,
sinon une vérité, quand nous affirmons que, après la
%55
%{\it }
constante conjonction de deux objets — chaleur et flamme,
par exemple, ou poids et solidité —, nous sommes déterminés
par la seule accoutumance a attendre l’un quand
paraît l’autre. Cette hypothèse, semble-t-il, est la seule
qui explique la difficulté ; pourquoi tirons-nous de mille
cas une conclusion que nous étions incapables de tirer
d’un seul cas, qui ne différe à aucun égard des précédents ?
La raison est incapable de varier de pareille manière. Les
conclusions qu’elle tire de la considération d’un cercle
sont les mêmes que celles qu’elle formerait à l’examen de
tous les cercles de univers. Mais si on n’a vu qu’un seul
corps se mouvoir sous l’impulsion d’un autre, personne
n’inférerait que tout autre corps se mouvra sous une
impulsion analogue. Toutes les conclusions tirées de
l'expérience sont donc des effets de l'accoutumance et
non des effets du raisonnement.

Alors, l'accoutumance est le grand guide de la vie
humaine. C’est ce seul principe qui fait que notre expérience
nous sert, c’est lui seul qui nous fait attendre, dans
le futur, une suite d’événements semblables a ceux qui ont
paru dans le passé. Sans l’action de l’accoutumance, nous
ignorerions complètement toute question de fait en dehors
de ce qui est immédiatement présent à la mémoire et aux
sens. Nous ne saurions jamais comment ajuster des moyens
en vue de fins, ni comment employer nos pouvoirs naturels
pour produire un effet. Ce serait à la fois la fin de toute
action aussi bien que de presque toute spéculation. ({\it Enquête
sur l’entendement humain}, trad. André L{\footnotesize EROY}, section V.)

%%%%%%%%%%%%%%%%%%%%%%%%%%%%%%%%%%%%%%%%%%%%%%%%%%%%%%%%%%%%%%%%%%%%%%%%
