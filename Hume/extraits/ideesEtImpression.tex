
%%%%%%%%%%%%%%%%%%%%%
\section{Idées et impressions}
%%%%%%%%%%%%%%%%%%%%%
%{\footnotesize XVIII}$^\text{e}$ siècle
%{\it }
Chacun accordera volontiers qu’il y a une différence
considérable entre les perceptions de l'esprit quand on
sent la douleur d’une chaleur excessive ou le plaisir d’une
chaleur modérée et quand, par la suite, on rappelle à la
mémoire cette sensation ou quand on l’anticipe par l’imagination.
Ces facultés peuvent imiter ou copier les perceptions
des sens, mais elles ne peuvent jamais atteindre
la force et la vivacité de la sensation originelle. Le plus
que nous en disions, même quand elles opérent avec la
plus grande vigueur, c’est qu’elles représentent leur objet
d’une maniére si vivante que nous pouvons presque dire
que nous les touchons ou les voyons ; mais, sauf si l’esprit
est troublé par la maladie ou la folie, elles ne peuvent
jamais arriver à un degré de vivacité tel qu’il rende ces
perceptions complétement indiscernables. Toutes les
couleurs de la poésie, malgré leurs splendeurs, ne peuvent
jamais peindre les objets naturels d’une telle manière
qu’on prenne la description pour le paysage réel. La
pensée la plus vive est encore inférieure à la sensation la
plus terne.

Nous pouvons observer qu’une distinction analogue se
retrouve dans toutes les autres perceptions de l'esprit.
Un homme, dans un accès de colère, est animé de manière
trés différente de celui qui pense seulement a cette émotion.
Si vous me dites qu’une personne est amoureuse, je comprends
aisément ce que vous voulez dire et je me forme
%52
%{\it }
une juste conception de la condition de cette personne ;
mais je ne peux jamais prendre a tort cette conception
pour les agitations et les désordres réels de la passion.
Quand nous réfléchissons a nos affections et sentiments
passés, notre pensée est un miroir fidèle et elle copie ses
objets avec vérité; mais les couleurs qu'elle emploie
sont pales et ternes en comparaison de celles qui habillent
nos perceptions originelles. Il n’est pas besoin d’un discernement
attentif ou d’un esprit métaphysique pour
marquer la différence qu’il y a entre les unes et les autres.

Voici donc que nous pouvons diviser toutes les perceptions
de l’esprit en deux classes ou espèces, qui se distinguent
par leurs différents degrés de force et de vivacité.
Les moins fortes et les moins vives sont communément
nommées {\it pensées} ou {\it idées}. L’autre espèce n’a pas de nom
dans notre langue et dans la plupart des autres langues ;
je suppose qu’il en est ainsi parce qu’il n’est pas nécessaire,
pour des desseins autres que philosophiques, de
les ranger sous une appellation ou un nom général. Usons
donc de liberté et appelons-les {\it impressions}, en employant
ce mot dans un sens qui différe quelque peu du sens
habituel. Par le terme {\it impression}, j’entends donc toutes
nos plus vives perceptions quand nous entendons, voyons,
touchons, aimons, haissons, désirons ou voulons. Et les
impressions se distinguent des idées, qui sont les moins
vives perceptions, dont nous avons conscience quand
nous réfléchissons à l’une des sensations ou à l’un des
mouvements que je viens de citer.

... Voici done une proposition qui, non seulement,
semble en elle-même simple et intelligible, mais qui, si
on en fait un usage convenable, peut rendre toute discussion
également intelligible et peut bannir tout ce
jargon qui s’est emparé si longtemps des raisonnements
métaphysiques et les a discrédités. Toutes les idées, spécialement
les idées abstraites, sont par nature indistinctes
et obscures ; l’esprit n’a sur elles qu’une faible prise ;
il est porté a les confondre avec d’autres idées semblables ;
%53
%{\it }
quand nous avons souvent employé un terme, même sans
lui donner un sens distinct, nous sommes portés à imaginer
qu’une idée déterminée y est annexée. Au contraire,
toutes les impressions, c’est-a-dire toutes les sensations,
externes ou internes, sont fortes et vives ; leurs limites
sont plus exactement déterminées ; il n’est pas facile de
tomber dans l’erreur ou de se méprendre à leur sujet.
Quand donc nous soupçonnons qu’un terme philosophique
est employé sans aucun sens ni aucune idée correspondante
(comme cela se fait trop fréquemment), nous n’avons
qu’a rechercher {\it de quelle impression dérive cette idée supposée.}
Si l’on ne peut en désigner une, cela servira à confirmer
notre soupçon. En portant les idées sous une lumière
aussi claire, nous pouvons raisonnablement espérer
écarter toute discussion qui pourrait surgir au sujet de
leur nature et de leur réalité. ({\it Enquête sur l’entendement
humain}, trad. André L{\footnotesize EROY}, section II.)

%%%%%%%%%%%%%%%%%%%%%%%%%%%%%%%%%%%%%%%%%%%%%%%%%%%%%%%%%%%%%%%%%%%%%%%%
