
%%%%%%%%%%%%%%%%%%%%% 13
\section{La preuve de Dieu par la finalité (discours de Cléanthe)}
%%%%%%%%%%%%%%%%%%%%%
%{\footnotesize XVIII}$^\text{e}$ siècle
%{\it }
Supposez donc que vous entriez dans votre bibliothéque,
ainsi peuplée de volumes naturels, contenant la
raison la plus raffinée et la plus délicate beauté : vous
serait-il aucunement possible d’ouvrir l’un d’eux, et de
douter que sa cause originelle ne présente l'analogie la
plus forte avec l’esprit et l'intelligence ? Quand il raisonne
et discourt, quand il discute, argumente, fait valoir ses
vues et ses théories, quand il s’adresse tantôt au pur
%77
%{\it }
intellect, tantôt aux affections, quand il rassemble, dispose
et orne toute considération appropriée au sujet,
pourriez-vous persister à prétendre que tout cela, au fond,
n’a pas de signification réelle, et que la première formation
de ce volume dans les reins de son parent originel ne
provient pas d’une pensée ni d’un dessein ? Votre obstination,
je le sais, n’atteint pas ce degré de fermeté : même
votre badinage et votre enjouement sceptiques seraient
confondus devant une si éclatante absurdité.

Mais s’il y a quelque différence, Philon, entre ce cas
supposé et le cas réel de l’univers, elle est tout à l’avantage
de ce dernier. L’anatomie d’un animal offre mainte preuve
plus forte d’un dessein que la lecture de Tite-Live ou de
Tacite; et quelque objection que vous éleviez dans le
premier cas, en me renvoyant à un spectacle aussi insolite
et aussi extraordinaire que la première formation des
mondes, la même objection s’applique à la supposition
de notre bibliothéque végétante. Choisissez donc votre
parti, Philon, sans ambiguité ni échappatoire : ou bien
affirmez qu’un livre raisonnable n’est nullement la preuve
d'une cause raisonnable, ou bien admettez une cause
semblable pour toutes les {\oe}uvres de la nature.

Laissez-moi remarquer en outre, continua Cléanthe,
que cet argument religieux, loin d’être affaibli par ce
scepticisme que vous affectez si fort, en acquiert plutôt
de la force, et devient plus solide et plus indiscutable.
Exclure tout argument ou tout raisonnement, de quelque
espèce qu’il soit, c’est ou affectation ou folie. La profession
déclarée de tout sceptique raisonnable, c’est seulement de
rejeter les arguments abstrus, éloignés et raffinés, d’adhérer
au sens commun et aux clairs instincts de la nature, et
de donner son assentiment chaque fois que des raisons,
quelles qu’elles soient, le frappent avec une force si entière,
qu’il ne saurait sans la plus grande violence s’en empécher.
Or, les arguments en faveur de la religion naturelle sont
manifestement de ce genre ; et il n’y a que la plus perverse,
la plus obstinée métaphysique qui les puisse rejeter.

%78
%{\it }
Considérez l’{\oe}il, disséquez-le ; contemplez-en la structure
et l'agencement; et dites-moi, d’aprés votre propre
sentiment, si l'idée d’un auteur de cet agencement ne
pénètre pas immédiatement en vous, avec une force
pareille à celle de la sensation. La conclusion la plus immédiate
est assurément en faveur d’un dessein ; et il faut du
temps, de la réflexion et de l’étude pour rassembler ces
objections, frivoles encore qu’abstruses, qui peuvent
soutenir l'incroyance. Qui peut considérer les éléments
mâle et femelle de chaque espèce, la correspondance de
leurs parties et de leurs instincts, leurs passions et le
cours entier de leur vie avant et après la génération,
sans être obligé de s’apercevoir que la propagation de
l'espèce est un but poursuivi par la nature ? Des millions
et des millions de tels exemples se présentent en chaque
partie de l'univers ; et nul langage ne peut transmettre
une signification plus intelligente, plus irrésistible, que
le soigneux ajustement des causes finales. A quel degré
d’aveugle dogmatisme faut-il donc qu’on soit parvenu,
pour rejeter des arguments si naturels et si convaincants ?
({\it Dialogues sur la religion naturelle}, 3$^\text{e}$ partie, trad. Maxime
D{\footnotesize AVID}.)
%{\footnotesize }
%%%%%%%%%%%%%%%%%%%%%%%%%%%%%%%%%%%%%%%%%%%%%%%%%%%%%%%%%%%%%%%%%%%%%%%%
