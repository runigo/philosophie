
%%%%%%%%%%%%%%%%%%%%% 17
\section{Discours terminal de Philon : scepticisme et religion révélée}
%%%%%%%%%%%%%%%%%%%%%
%{\footnotesize XVIII}$^\text{e}$ siècle
%{\it }
%84
%
Si l'ensemble de la théologie naturelle, comme quelques-uns
semblent le soutenir, se résout en une seule proposition,
simple, quoique un peu ambiguë, ou du moins indéfinie,
savoir : {\it que la ou les causes de l'ordre dans l'univers
présentent probablement quelque lointaine analogie avec
l'intelligence humaine} ; si cette proposition n’est pas
susceptible d’extension, de variation ni d’explication
plus particuliére ; si elle ne fournit aucune inférence qui
affecte la vie humaine ou qui puisse être la source d’une
action ou d’une abstention quelconque ; et si l’analogie,
imparfaite comme elle l’est, ne peut être étendue plus
loin qu’a l’intelligence humaine, et ne saurait, avec la
moindre apparence de probabilité, être transportée aux
qualités de l’esprit ; si tel est en effet le cas, que peut faire
l’homme le plus curieux, le plus contemplatif et le plus
religieux, que de donner un franc et philosophique assentiment
à la proposition, aussi souvent qu’elle se présente,
et de croire que les arguments sur lesquels elle s’établit
l'emportent sur les objections qui s’y opposent ? Quelque
étonnement, en vérité, naîtra naturellement de la grandeur
de l’objet ; quelque mélancolie, de son obscurité ; quelque
mépris de la raison humaine, de ce qu’elle ne puisse donner
de solution plus satisfaisante en ce qui regarde une si
extraordinaire et si magnifique question. Mais, croyez-moi,
Cléanthe, le sentiment le plus naturel qu’un esprit
bien disposé doive éprouver en cette occasion, c’est une
attente et un désir ardents, qu’il plaise au ciel de dissiper,
d’alléger du moins, cette profonde ignorance, en offrant
a l'humanité quelque révélation particulière, et en nous
découvrant quelque chose de la nature, des attributs et
des opérations du divin objet de notre foi. Une personne
pénétrée d’un juste sentiment des imperfections de la
raison naturelle volera à la vérité révélée avec la plus
%85
%{\it }
grande avidité ; tandis que le hautain dogmatique, persuadé
qu’il peut élever un système complet de théologie
par le seul secours de la philosophie, dédaignera toute
autre aide et rejettera cette institutrice superflue. \^Etre
un sceptique philosophique, c’est, chez un homme lettré,
le premier pas, et le plus essentiel, menant à être un vrai
chrétien, un croyant : proposition que je recommanderais
volontiers à l'attention de Pamphile ; et j’espére que
Cléanthe me pardonnera d’intervenir jusqu’a ce point
dans l’éducation et l’instruction de son pupille. ({\it Dialogues
sur la religion naturelle}, 12$^\text{e}$ partie, trad. D{\footnotesize AVID}.)
%{\footnotesize }
%%%%%%%%%%%%%%%%%%%%%%%%%%%%%%%%%%%%%%%%%%%%%%%%%%%%%%%%%%%%%%%%%%%%%%%%
