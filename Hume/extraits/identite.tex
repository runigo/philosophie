
%%%%%%%%%%%%%%%%%%%%% 6
\section{L'identité personnelle}
%%%%%%%%%%%%%%%%%%%%%
%{\footnotesize XVIII}$^\text{e}$ siècle
%{\it }
Il y a certains philosophes qui imaginent que nous
avons à tout moment la conscience intime de ce que nous
appelons notre moi ; que nous sentons son existence et sa
%61
%{\it }
continuité d’existence ; et que nous sommes certains,
plus que par l’évidence d’une démonstration, de son identité
et de sa simplicité parfaites. La plus forte sensation
et la plus violente passion, disent-ils, au lieu de nous
distraire de cette vue, ne font que l’établir plus intensément ;
elles nous font considérer leur influence sur le moi
par leur douleur ou leur plaisir. Essayer d’en fournir une
preuve plus compléte serait en affaiblir l’évidence ; car
aucune preuve ne peut se tirer d’aucun fait dont nous
ayons une conscience aussi intime ; et il n’y a rien dont
nous puissions être certains si nous doutons de ce fait.

Malheureusement toutes ces affirmations positives sont
contraires à l’expérience elle-méme, qu’on invoque en
leur faveur; et nous n’avons aucune idée du moi à la
manière qu’on vient d’expliquer ici. En effet, de quelle
impression pourrait dériver cette idée ? A cette question,
il est impossible de répondre sans contradiction ni absurdité
manifestes ; pourtant c’est une question à laquelle
il faut nécessairement répondre, si nous voulons que l’idée
du moi passe pour claire et intelligible. Il doit y avoir une
impression qui engendre toute idée réelle. Mais le moi,
ou la personne, n’est pas une impression, c’est ce à quoi
nos diverses impressions et idées sont censées se rapporter.
Si une impression engendre l’idée du moi, cette impression
doit demeurer invariablement identique pendant tout le
cours de notre existence : car le moi est censé exister de
cette maniére. Or il n’y a pas d’impression constante et
invariable. La douleur et le plaisir, les passions et les
sensations se succédent les unes aux autres et jamais elles
n’existent toutes en méme temps. Ce ne peut donc étre
d’aucune de ces impressions, ni d’aucune autre qu’est
dérivée l’idée du moi ; par conséquent une telle idée n’existe
pas.

Mais en outre, quel doit être le sort de toutes nos perceptions
particuliéres dans cette hypothése ? Elles sont
toutes différentes, discernables et séparables les unes des
autres ; on peut les considérer séparément et elles peuvent
%62
%{\it }
exister séparément : elles n’ont besoin de rien pour soutenir
leur existence. De quelle manière appartiennent-elles donc
au moi et comment sont-elles en connexion avec lui ?
Pour ma part, quand je pénétre le plus intimement dans
ce que j’appelle moi, je bute toujours sur une perception
particulière ou sur une autre, de chaud ou de froid, de
lumiére ou d’ombre, d’amour ou de haine, de douleur ou
de plaisir. Je ne peux jamais me saisir, moi, en aucun
moment sans une perception et je ne peux rien observer
que la perception. Quand mes perceptions sont écartées
pour un temps, comme par un sommeil tranquille, aussi
longtemps je n’ai plus conscience de moi et on peut dire
vraiment que je n’existe pas. Si toutes mes perceptions
étaient supprimées par la mort et que je ne puisse ni
penser, ni sentir, ni voir, ni aimer, ni haïr aprés la dissolution
de mon corps, je serais entièrement annihilé et
je ne conçois pas ce qu’il faudrait de plus pour faire de
moi un parfait néant. Si quelqu’un pense, aprés une réflexion
sérieuse et impartiale, qu’il a, de lui-méme, une
connaissance différente, il me faut l’avouer, je ne peux
raisonner plus longtemps avec lui. Tout ce que je peux
lui accorder, c’est qu’il peut être dans le vrai aussi bien
que moi et que nous différons essentiellement sur ce point.
Peut-être peut-il percevoir quelque chose de simple et de
continu qu’il appelle lui : et pourtant je suis sûr qu'il n’y
a pas en moi de pareil principe. ({\it Traité de la nature
humaine}, liv. I, 4$^\text{e}$ partie, section VI, trad. L{\footnotesize EROY}.)

%%%%%%%%%%%%%%%%%%%%%%%%%%%%%%%%%%%%%%%%%%%%%%%%%%%%%%%%%%%%%%%%%%%%%%%%
