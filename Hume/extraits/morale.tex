
%%%%%%%%%%%%%%%%%%%%% 9
\section{Le fondement de la morale}
%%%%%%%%%%%%%%%%%%%%%
%{\footnotesize XVIII}$^\text{e}$ siècle
%{\it } 
Prenez une action reconnue comme vicieuse : un meurtre
prémédité, par exemple. Examinez-la sous tous les aspects
et voyez si vous pouvez découvrir ce point de fait, cette
existence réelle que vous appelez vice. De quelque manière
que vous la preniez, vous trouvez seulement certaines
passions, certains motifs, certaines volitions et certaines
pensées. Il n’y a pas d’autre fait dans ce cas. Le vice vous
échappe entièrement tant que vous considérez l'objet.
Vous ne pouvez le trouver jusqu’au moment où vous
tournez votre réflexion sur votre propre c{\oe}ur et découvrez
un sentiment de désapprobation qui naît en vous contre
cette action. Voila un fait : mais il est objet de conscience
et non de raison. Il se trouve en vous et non dans l'objet.
Si bien que, lorsque vous affirmez qu’une action ou un
caractère sont vicieux, vous voulez simplement dire que
sous l’effet de votre constitution naturelle, vous éprouvez,
à les considérer, un sentiment de blâme. On peut donc
%67
%{\it }
comparer le vice et la vertu aux sons, aux couleurs, à la
chaleur et au froid qui, selon la philosophie moderne, sont
non pas des qualités des objets, mais des perceptions de
l'esprit : cette découverte en morale, comme l'autre
découverte en physique, doit être regardée comme un
progrés considérable des sciences spéculatives ; pourtant,
comme l’autre découverte aussi, elle a peu ou pas d’influence
en pratique. Rien ne peut être plus réel, rien ne
peut nous intéresser davantage que nos propres sentiments
de plaisir et de douleur ; si ces sentiments sont favorables
à la vertu et défavorables au vice, rien de plus ne peut
être réclamé pour règler notre conduite et nos m{\oe}urs.

Je ne peux m’empécher d’ajouter à ces raisonnements une
remarque que, sans doute, on peut trouver de quelque
importance. Dans tous les systèmes de morale que j’ai
rencontrés jusqu’ici, j’ai toujours remarqué que l’auteur
procéde quelque temps selon la manière ordinaire de
raisonner, qu’il établit l'existence de Dieu ou qu’il fait
des remarques sur la condition humaine ; puis tout a coup
j'ai la surprise de trouver qu’au lieu des copules {\it est} ou
{\it n'est pas} habituelles dans les propositions, je ne rencontre
que des propositions où la liaison est établie par {\it doit} ou
{\it ne doit pas}. Ce changement est imperceptible ; mais il est
pourtant de la plus haute importance. En effet, comme ce
{\it doit} ou ce {\it ne doit pas} expriment une nouvelle relation et
une nouvelle affirmation, il est nécessaire que celles-ci
soient expliquées : et qu’en même temps on rende raison
de ce qui parait tout a fait inconcevable, comment cette
nouvelle relation peut se déduire d’autres relations qui
en sont entièrement différentes. Mais, comme les auteurs
n’usent pas couramment de cette précaution, je prendrai
la liberté de la recommander aux lecteurs, et je suis persuadé
que cette légère attention détruira tous les systèmes
courants de morale et nous montrera que la distinction
du vice et de la vertu ne se fonde pas uniquement sur les
relations des objets et qu’elle n’est pas percue par la
raison.
%68
%{\it } 

... La morale est donc plus proprement sentie que jugée ;
pourtant cette conscience ou ce sentiment est communément
si doux et si tempéré que nous sommes portés à le
confondre avec une idée, selon notre habitude courante
de prendre pour identiques les choses qui ont entre elles
une grande ressemblance.

La question suivante est de savoir de quelle nature sont
ces impressions et de quelle manière elles agissent sur
nous. Ici nous ne pouvons demeurer longtemps a hésiter,
mais nous devons affirmer que l'impression, qui naît de
la vertu, est agréable et que celle qui procéde du vice est
déplaisante. A tout moment, l'expérience nous en convainc.
Il n’y a pas de spectacle plus beau ni plus séduisant qu’une
noble et généreuse action : il n’y en a pas qui éveille en
nous plus de répulsion qu’une action cruelle et traitresse.
Nulle jouissance n’égale la satisfaction que nous recevons
de la compagnie de personnes que nous aimons et estimons ;
et la plus grande de toutes les punitions est d’être
obligés de passer notre existence avec des gens que nous
haïssons ou méprisons. Une piéce de théâtre ou un roman
peuvent même nous apporter des exemples du plaisir
que la vertu nous procure et de la douleur qu’engendre le
vice.

Or, puisque les impressions distinctives, qui nous font
connaître le bien moral ou le mal moral, ne sont rien que
des douleurs ou des plaisirs {\it particuliers}, il s’ensuit que,
dans toutes les enquètes au sujet de ces distinctions morales,
il suffira de montrer les principes, qui nous font ressentir
une satisfaction ou un malaise à la vue d’un caractère,
pour nous satisfaire sur le point de savoir pourquoi ce
caractére est louable ou blâmable. Une action, un sentiment,
ou un caractère, est vertueux ou vicieux ; pourquoi ?
Parce que sa vue cause un plaisir ou un malaise d’un genre
particulier. Si donc nous donnons une raison du plaisir ou
du malaise, nous expliquons suffisamment le vice ou la
vertu. Avoir le sens de la vertu, ce n’est rien de plus que
de {\it ressentir} une satisfaction d’un genre particulier à la
%69
%{\it }
contemplation d’un caractère. C’est ce {\it sentiment} lui-même
qui constitue notre éloge ou notre admiration. Nous
n’allons pas plus loin ; nous ne recherchons pas la cause
de cette satisfaction. Nous n’inférons pas qu'un caractère
est vertueux de ce qu’il plait ; mais, en sentant qu'il plait
de cette manière particulière, nous sentons effectivement
qu’il est vertueux. C’est le même cas que dans nos jugements
sur les beautés de tout genre, sur les goûts et les
sensations. Notre approbation est comprise dans le plaisir
immédiat que ceux-ci nous apportent. ({\it Traité de la nature
humaine}, liv. III, 1$^\text{re}$ partie, section I, trad. L{\footnotesize EROY}.)

%%%%%%%%%%%%%%%%%%%%%%%%%%%%%%%%%%%%%%%%%%%%%%%%%%%%%%%%%%%%%%%%%%%%%%%%
