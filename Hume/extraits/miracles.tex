
%%%%%%%%%%%%%%%%%%%%% 18
\section{Essai sur les miracles}
%%%%%%%%%%%%%%%%%%%%%
%{\footnotesize XVIII}$^\text{e}$ siècle
%{\it }

Supposez, par exemple, que le fait, que le témoignage
essaie d’établir, participe de l’extraordinaire et du
merveilleux ; dans ce cas, l’évidence, qui résulte du
témoignage, admet une diminution plus ou moins grande en
proportion de ce que le fait est plus ou moins inhabituel.
La raison, qui nous fait accorder du crédit aux témoins
et aux historiens, ne se tire pas d’une {\it connexion} percue
{\it a priori} entre le témoignage et la réalité; elle vient de
ce que nous sommes accoutumés à trouver de la conformité
entre eux. Mais quand le fait attesté est tel qu’il est
rarement tombé sous notre observation, il se produit
alors une lutte entre deux expériences contraires ; l'une
d'elles détruit l’autre dans la mesure de sa force et
l’expérience supérieure peut seulement opérer sur l’esprit par
son surcroit de force. C’est exactement ce même principe
de l’expérience qui nous donne un certain degré d’assurance
en l’attestation des témoins et qui nous donne aussi,
dans ce cas, un autre degré d’assurance contre le fait
qu’essaient d’établir les témoins ; cette contradiction
engendre nécessairement un balancement et une destruction
mutuelle de croyance et d’autorité.

{\it Je ne croirais pas une telle histoire, même si Caton me
%86
la racontait}, c’était une maxime proverbiale à Rome, même
du vivant de ce patriote philosophe \footnote{Plutarque, {\it Vie de Caton} (H) ; {\it Caton le jeune}, chap. XIX.}. L’incrédibilité
d’un fait, accordait-on, pouvait invalider une autorité
aussi grande.

Le prince indien qui refusa de croire les premières
relations sur les effets du gel raisonnait correctement ;
et il lui fallait naturellement un témoignage trés fort
pour accorder son assentiment à des faits produits par un
effet de la nature qui ne lui était pas familier et qui avait
si peu d’analogie avec les événements dont il avait eu une
expérience constante et uniforme. Ces faits, cependant,
n’étaient pas contraires à son expérience, ils n’y étaient
pas conformes \footnote{Aucun Indien, évidemment, ne pouvait avoir l’expérience
que l’eau ne géle pas sous les climats froids. Car c’est placer la
nature dans une situation qui lui est entièrement inconnue ;
et il lui est impossible de dire a priori ce qui en résultera. Il
faut faire une nouvelle expérience dont la conséquence est
toujours incertaine. On peut parfois conjecturer par analogie
ce qui suivra ; mais ce n’est encore que conjecture. Et il faut
avouer que, dans le cas présent du gel, l’événement se produit
contrairement aux règles de l’analogie et qu’il est tel qu’un
Indien raisonnable ne pourrait l’attendre. L’action du froid sur
l'eau n’est pas graduelle avec les degrés de froid ; mais, chaque
fois qu’elle parvient au point de congélation, l’eau passe en un
moment d’une extréme liquidité à une parfaite dureté. On
peut donc appeler extraordinaire un tel événement et il faut un
témoignage assez fort pour le faire accepter des habitants d’un
climat chaud; mais cet événement n’est pourtant pas miraculeux,
ni contraire a l’expérience uniforme du cours de la nature
dans des cas ou toutes les circonstances sont les mêmes. Les
habitants de Sumatra ont toujours vu l’eau fluide sous leur
propre climat et ils doivent tenir pour un prodige le gel de leurs
cours d’eau ; mais ils n’ont jamais vu d’eau en Moscovie pendant
l'hiver ; par suite ils ne peuvent être raisonnablement positifs
sur ce qui en résulterait (H).}.

%87
Mais, pour accroître la probabilité contraire à l’attestation
des témoins, supposons que le fait qu’ils affirment,
au lieu d’être seulement merveilleux, soit réellement
miraculeux ; et supposons aussi que le témoignage, considéré
à part et en lui-même, se monte à une preuve entière ;
dans ce cas, il y a preuve contre preuve, et il faut que la
plus forte prévale, mais pourtant non sans que sa force
soit diminuée en proportion de celle de la preuve opposée.

Un miracle est une violation des lois de la nature;
comme une expérience ferme et inaltérable a établi ces
lois, la preuve qui s’oppose à un miracle par suite de la
nature même du fait est aussi entière qu’aucun argument
imaginable tiré de l’expérience. Pourquoi est-il plus que
probable que tous les hommes doivent mourir ? que du
plomb ne peut, de lui-même, rester suspendu en l'air?
que le feu consume le bois et que l’eau l’éteint ? sinon
parce qu’on a trouvé que ces événements étaient conformes
aux lois de la nature ; et il faut donc une violation de ces
lois ou, en d’autres termes, il faut un miracle pour les
empécher de se produire. On n’estime pas qu’un fait est un
miracle s’il ne se produit jamais dans le cours commun de
la nature. Ce n’est pas un miracle qu’un homme, apparemment 
en bonne santé, meure subitement ; car un tel genre
de mort, bien que plus inhabituel qu’un autre, on a pourtant
fréquemment observé qu’il se produisait. Mais c’est
un miracle qu’un mort puisse revenir à la vie ; car le fait
n’a jamais été observé à aucune époque ni en aucun pays.
Il faut donc qu’il y ait une expérience uniforme contre
tout événement miraculeux, sinon l’événement ne mériterait
pas cette appellation.

... La maxime, qui nous conduit communément dans
nos raisonnements, est que les objets dont nous n’avons
aucune expérience ressemblent à ceux dont nous avons
l'expérience ; que ce que nous avons trouvé le plus habituel
est toujours le plus probable ; et que là où il y a opposition
d’arguments, nous devons donner la préférence a ceux
qui sont fondés sur le plus grand nombre d’observations
passées. Mais bien que, en procédant d’après cette règle,
%88{\it }
nous rejetions volontiers un fait inhabituel et incroyable
à un degré ordinaire, pourtant, quand il pousse plus loin,
l'esprit n’observe pas toujours la même règle ; mais quand
on affirme un événement absurde et miraculeux à l’extrême,
il admet plutôt plus volontiers un tel fait en raison de
cette circonstance même qui devrait en détruire toute
l'autorité. La passion de la {\it surprise} et de {\it l’étonnement},
qui naît des miracles, est une émotion agréable ; aussi
nous donne-t-elle une tendance sensible à croire les événements
dont elle procéde. Cela va si loin que même
ceux qui ne peuvent pas jouir immédiatement de ce
plaisir, et ne peuvent croire a ces miraculeux événements
dont on les informe, aiment encore à participer à cette
satisfaction de seconde main et par rebondissement, et
qu'ils mettent leur orgueil et leur plaisir à exciter l'admiration
d’autrui.

... Supposez que tous les historiens qui traitent de
l’Angleterre s’accordent sur ce que, le 1$^\text{er}$ janvier 1600,
la reine Elisabeth mourut ; qu’avant et aprés sa mort
ses médecins et toute la cour la virent, comme c’est l’habitude
pour les personnes de son rang; que le Parlement
reconnut et proclama son successeur; et qu’après un
mois d’inhumation elle reparut, réoccupa le trône et
gouverna l’Angleterre pendant trois ans ; je dois l’avouer,
je serai surpris du concours de tant de circonstances
bizarres, mais je n’aurais pas la moindre inclination à
croire à un événement aussi miraculeux. Je ne douterais
pas de sa mort prétendue et de ces autres circonstances
publiques qui la suivirent ; j’affirmerais seulement que
cette mort fut prétendue, qu’elle ne fut pas réelle, qu’elle
ne pouvait pas l’être. Vous m’objecteriez en vain la difficulté
et presque l’impossibilité de tromper le monde
dans une affaire d’une telle importance ; la sagesse et le
solide jugement de cette fameuse reine ; le peu d’avantage
quelle pouvait retirer d’un artifice aussi pauvre, si même
elle en retirait un ; tout cela pourrait m’étonner ; mais je
répliquerais encore que la friponnerie et la sottise humaines
%89
sont des phénomènes si courants que je croirais que les
événements les plus extraordinaires naissent de leur
concours plutôt que d’admettre une violation aussi remarquable
des lois de la nature...

... Ce que nous avons dit des miracles peut s’appliquer
sans modification aux prophéties ; et, certes, toutes les
prophéties sont de réels miracles, et c’est seulement
comme telles qu’on peut les admettre comme preuves
d'une révélation. S’il n’était pas au-dessus du pouvoir
de la nature humaine de prédire les événements futurs,
il serait absurde d’user d’une prophétie comme d’un argument
en faveur d’une mission divine ou d’une autorité
venue du ciel. Si bien que, somme toute, nous pouvons
conclure que la religion chrétienne ne s’est pas seulement
accompagnée de miracles à ses débuts, mais que même à
ce jour aucun homme raisonnable ne peut y croire sans un
miracle. La pure raison ne suffit pas à nous convaincre de
sa véracité ; quiconque est mû par la foi à y donner son
assentiment est conscient d’un miracle continu dans sa
propre personne, qui bouleverse tous les principes de son
entendement et lui donne une détermination à croire ce
qui est le plus contraire à la coutume et à l’expérience.

%%%%%%%%%%%%%%%%%%%%%%%%%%%%%%%%%%%%%%%%%%%%%%%%%%%%%%%%%%%%%%%%%%%%%%%%
