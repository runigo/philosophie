
%%%%%%%%%%%%%%%%%%%%% 16
\section{ Il ny a pas d'être dont la non-existence implique contradiction}
%%%%%%%%%%%%%%%%%%%%%
%{\footnotesize XVIII}$^\text{e}$ siècle
%{\it }

L’argument que je voudrais soutenir, répondit Déméa,
est l’argument courant. Tout ce qui existe doit avoir
une cause ou raison de son existence, étant absolument
%83
%{\it }
impossible que rien se produise soi-même, ou soit la cause
de sa propre existence. Donc, en remontant des effets aux
causes, ou bien nous devons poursuivre le cours d’une
succession infinie, sans jamais atteindre aucune cause
ultime, ou bien nous devons finalement avoir recours a
quelque cause ultime qui soit {\it nécessairement} existante. Or,
que la première supposition soit absurde, on le peut
prouver ainsi. Dans la chaîne ou succession infinie des
causes et des effets, chaque effet en particulier est déterminé
à exister par le pouvoir et l'efficace de la cause qui le
précéda immédiatement ; mais la chaîne ou succession
éternelle tout entière, prise dans son ensemble, n’est
déterminée ni causée par rien : et pourtant il est évident
qu’elle requiert une cause ou raison, autant que
tout objet particulier qui commence d’exister dans le
temps...

... Je ne laisserai pas à Philon, dit Cléanthe — quoique
je sache qu’élever des objections fasse ses principales
délices — le soin de relever la faiblesse de ce raisonnement
métaphysique. Il me semble si manifestement mal fondé,
et en même temps de si peu de conséquence pour la cause
de la piété et de la religion véritables, que je me risquerai
à en montrer la fausseté.

Je commencerai par observer qu’il y a une évidente
absurdité a prétendre démontrer une chose de fait,
ou la prouver par des arguments {\it a priori}, quels qu’ils
soient. Rien n’est démontrable, à moins que le contraire
n’implique contradiction. Rien de ce qui est distinctement
concevable n’implique contradiction. Tout ce que
nous concevons comme existant, nous le pouvons aussi
concevoir comme non-existant. Il n’y a donc pas d’être
dont la non-existence implique contradiction. Par conséquent,
il n’y a pas d’être dont l’existence soit démontrable.
Je propose cet argument comme entièrement
décisif, et consens volontiers à faire reposer dessus la
controverse entière. ({\it Dialogues sur la religion naturelle},
9$^\text{e}$ partie.)

%%%%%%%%%%%%%%%%%%%%%%%%%%%%%%%%%%%%%%%%%%%%%%%%%%%%%%%%%%%%%%%%%%%%%%%%
