
%%%%%%%%%%%%%%%%%%%%% 3
\section{Le caractère paradoxal de l'explication psychologique
de l'idée de cause}
%%%%%%%%%%%%%%%%%%%%%
%{\footnotesize XVIII}$^\text{e}$ siècle
%{\it }

L’idée de nécessité nait d’une impression. Aucune
impression apportée par les sens ne peut engendrer cette
idée. Celle-ci doit donc dériver de quelque impression
%56
%{\it }
interne ou impression de réflexion. Il n’y a pas d’autre
impression interne en relation avec le fait, qui nous occupe
actuellement, que la tendance, produite par la coutume,
à passer d’un objet à l’idée d’un autre objet qui l’accompagne
habituellement. Telle est donc l’essence de la nécessité.
Somme toute, la nécessité est quelque chose qui existe
dans l’esprit ; mais non dans les objets ; il nous est impossible
d’en former une idée, même la plus lointaine, si nous
la considérons comme une qualité des corps. Ou bien nous
n’avons pas d’idée de la nécessité, ou bien la nécessité
n'est que la détermination de la pensée a passer des causes
aux effets et des effets aux causes d'aprés l’expérience de
leur union.

... Jai conscience que de tous les paradoxes que j’ai
eu, ou que j’aurai par la suite, l’occasion d’avancer au
cours de ce traité, le paradoxe présent est le plus violent
et que c’est seulement à force de preuves solides et de
raisonnements que je peux espérer le faire admettre et
triompher des préjugés invétérés de l’humanité. Avant
de nous ranger à cette doctrine, combien de fois devons-nous
nous répéter {\it que} la simple vue de deux objets, ou
de deux actions, même unis, ne peut jamais nous donner
l'idée d’un pouvoir ou d’une connexion entre eux : {\it que}
cette idée nait de la répétition de leur union : {\it que} la répétition
ne découvre ni ne produit rien dans les objets, mais
qu’elle agit seulement sur l’esprit par la transition coutumière
qu’elle produit : {\it que} cette transition coutumière
est donc identique au pouvoir et à la nécessité qui, par
suite, sont des qualités des perceptions et non pas des
objets et qui sont senties intérieurement par l’âme et
non pas perçues a l’extérieur dans les corps ? L’étonnement
accompagne communément tout ce qui est extraordinaire ;
et cet étonnement se change immédiatement en une
estime ou un mépris, du plus haut degré, selon que nous
approuvons ou désapprouvons le sujet. Je le crains
beaucoup : bien que le précédent raisonnement m’apparaisse
le plus court et le plus décisif qu’on puisse imaginer,
%57
%{\it }
pourtant, avec la généralité des lecteurs, l’inclination de
l'esprit prévaudra et leur donnera un préjugé contre la
doctrine présente.

Cette inclination contraire s’explique aisément. C’est
une observation courante que l'esprit a beaucoup de
penchant à se répandre sur les objets extérieurs et à unir
à ces objets les impressions intérieures qu’ils provoquent
et qui apparaissent toujours au moment où ces objets se
découvrent aux sens. Ainsi, comme certains sons et certaines
odeurs accompagnent toujours, trouvons-nous, certains
objets visibles, nous imaginons naturellement une
conjonction, même locale, entre les objets et les qualités,
bien que les qualités ne soient pas de nature à admettre
une telle conjonction et n’existent en réalité nulle part.
Mais j’en parlerai plus complétement par la suite. En
attendant, il suffit de remarquer que ce même penchant
est la raison qui nous fait admettre que la nécessité et le
pouvoir se trouvent dans les objets que nous considérons
et non dans notre esprit qui les considère ; néanmoins il
nous est impossible de former l'idée la plus lointaine de
cette qualité, quand nous ne la prenons pas comme la
détermination de l’esprit à passer de l'idée d’un objet à
celle d’un autre objet qui l’accompagne habituellement.

Mais, bien que ce soit là la seule explication raisonnable
que nous puissions donner de la nécessité, la notion contraire
est si bien rivée dans l’esprit par les principes mentionnés
plus haut que je ne doute pas que beaucoup de lecteurs
traiteront mon opinion d’extravagante et de ridicule.
Quoi ! l’efficacité des causes se trouve dans la détermination
de l'esprit ! Comme si les causes n’opéraient pas en toute
indépendance de l’esprit et ne continueraient pas d’opérer,
même s’il n’y avait aucun esprit pour les contempler et
raisonner a leur sujet. La pensée peut bien dépendre des
causes pour son opération, mais non les causes de la pensée.
C’est renverser l’ordre naturel et poser comme second ce
qui, en réalité, est premier. A toute opération correspond
un pouvoir qui lui est proportionné ; et il faut placer ce
%58
%{\it }
pouvoir dans le corps qui opére. Si nous retirons le pouvoir
d’une cause, il nous faut l’attribuer a une autre ; mais le
retirer de toutes les causes et l’attribuer a un être qui n’a
aucune espèce de rapport à la cause ni à l’effet, sinon par
la perception qu’il en a, c’est une grossiére absurdité,
contraire aux principes les plus certains de la raison
humaine. ({\it Traité de la nature humaine}, liv. I, 3° partie,
section XIV, trad. André L{\footnotesize EROY}.)

%%%%%%%%%%%%%%%%%%%%%%%%%%%%%%%%%%%%%%%%%%%%%%%%%%%%%%%%%%%%%%%%%%%%%%%%
