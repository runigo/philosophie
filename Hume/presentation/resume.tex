\begin{center}
\Large
Résumé
\normalsize
\end{center}
\vspace{3cm}
\begin{itemize}[leftmargin=1cm, label=\ding{32}, itemsep=21pt]
\item {\bf Objet} : Initiation à la philosophie de Hume.
\item {\bf Contenu} : Précis de philosophie.
\item {\bf Public concerné} : Philosophe en herbe.
\end{itemize}

\vspace{3cm}

Philosophe écossais du {\footnotesize XVIII}$^{\text{e}}$ siècle, David Hume est empiriste. Son œuvre nous offre une critique de la métaphysique à travers un scepticisme "académique".

\vspace{3cm}

Ce document est la reproduction d'un précis des classes supérieures écrit par André Vergez. L'ordre des chapitres a été modifié : certains chapitres du précis ont été déplacés en annexe de ce document.

\vspace{3cm}

Le premier chapitre de ce document synthétise l'œuvre philosophique de Hume. Le second chapitre est constitué par des  extraits de ses ouvrages. Enfin, se trouvent en annexe de ce document, une biographie, la liste des ouvrages de David Hume et la bibliographie d'André Vergez.

