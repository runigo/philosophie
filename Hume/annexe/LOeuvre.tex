
%%%%%%%%%%%%%%%%%%%%%
\chapter{L'{\oe}uvre}
%%%%%%%%%%%%%%%%%%%%% \textsc{}

%%%%%%%%%%%%%%%%%%%%%%%%%
\section{Ouvrages publiés du vivant de Hume}
%%%%%%%%%%%%%%%%%%%%%%%%%
% 
%{\it }
{\it A treatise of Human Nature} (1739-1740, 3 vol.).

{\it Essays moral and political} (3 vol., 1741-1742, 1748).

{\it Philosophical essays concerning Human Understanding}
(1748 ; à partir de 1758 le mot {\it Inquiry} remplace {\it Philosophical essays}).

{\it An Inquiry concerning the principles of Morals} (1751).

{\it Political discourses} (1752).

{\it The History of Great Britain} (1754-1757).

{\it Four Dissertations : 1. The natural history of Religion ;
II. Of the passions ; III. Of tragedy ; IV. Of the standard
of Taste (1757).} En 1755, Hume avait remis à l’éditeur
les trois premiéres de ces dissertations et une quatriéme
consacrée aux mathématiques et à la physique. Hume
la retire sur le conseil d’un ami mathématicien. Il la
remplace par deux dissertations : {\it On suicide} et {\it On the
immortality of soul.} Tandis que ces deux essais sont
déja en vente, Hume les retire et les remplace par une
seule dissertation : {\it Of the Standard of Taste. }

{\it The history of England} (1759 a 1767).

{\it Exposé succinct de la contestation qui s’est élevée entre
M. Hume et M. Rousseau} (1766).

%%%%%%%%%%%%%%%%%%%%%%%%%
\section{Ouvrages posthume}
%%%%%%%%%%%%%%%%%%%%%%%%%

{\it The life of David Hume written by himself} (1777).
{\it Two essays (On Suicide et The Immortality of the Soul)}
(1777).

%50 HUME
%{\it }
 

{\it Dialogues concerning Natural Religion} (1779).

J. H. \textsc{Burton}, {\it Life and Correspondance of David Hume,}
Edinburgh, 1846, 2 vol.

J. Y. T. \textsc{Greig}, {\it The letters of David Hume}, Oxford, 2 vol.,
1932.

R. \textsc{Klibansky} et E. C. \textsc{Mossner}, {\it }New letters of David
Hume, Oxford, 1954.

Rappelons qu’une édition anglaise classique comprend
toute l’{\oe}uvre philosophique de Hume : {\it The philosophical
works of David Hume}, éd. T. H. Green and T. H. Grose,
London, 1874-1875, 4 vol.

%%%%%%%%%%%%%%%%%%%%%%%%%
\section{Traductions françaises}
%%%%%%%%%%%%%%%%%%%%%%%%%{\it }

{\it Œvres philosophiques choisies (Enquéte sur l'entendement,
Traité de la nature humaine, Dialogues de la religion
naturelle)}  traduites par Maxime \textsc{David} avec préface
de \textsc{Lévy-Bruhl}, Paris, Alcan, 1912. La traduction
Maxime \textsc{David} des {\it Dialogues sur la religion naturelle}
a été rééditée en 1964 chez J.-J. Pauvert dans la collection 
« Libertés » avec une présentation et des notes
de Clément \textsc{Rosset}.

{\it Traité de la nature humaine}, préfacé et traduit par André
\textsc{Leroy}, Editions Aubier (1$^{\text re}$ éd., 1946), 2 vol.

{\it Enquête sur l'entendement humain} (trad. \textsc{Leroy}, Aubier,
1947).

{\it Enquête sur les principes de la morale. Les quatre philosophes} 
(trad. \textsc{Leroy}, Aubier, 1947).

Il existe une traduction frangaise de 1788 des quatre
dissertations {\it (L’histoire naturelle de la. religion, Les passions, 
La tragédie, La règle du goût)}. Une nouvelle traduction 
des {\oe}uvres de Hume est en cours d’édition.

 

 
%%%%%%%%%%%%%%%%%%%%%%%%%%%%%%%%%%%%%%%%%%%%%%%%%%%%%%%%%%%%%%%%%%%%%%%%
