
%%%%%%%%%%%%%%%%%%%%%
\chapter{La vie}
%%%%%%%%%%%%%%%%%%%%%
%{\it }{\oe}

David Hume est né le 26 août 1711 à Édimbourg
oÙ son pére exergait la profession d’avocat. Ce
dernier étant mort en 1714, Mrs Hume se retira
avec ses trois enfants John, Katherine et David
dans la propriété familiale de Ninewells, le domaine
des « Neuf Sources » situé dans la pittoresque campagne
(avec ses falaises, ses ruisseaux et ses bois)
du Berwickshire. L’oncle de David, pasteur du
village voisin de Chirnside, dirigea sa toute premiÈre
éducation. L’enseignement religieux que recut le
jeune David semble avoir été particulièrement
austère et maladroit (le Révérend George Hume
se plaisait, dans ses sermons, à humilier publiquement
les jeunes filles dont la grossesse révélait les
péchés charnels). L’antipathie précoce de Hume
pour le christianisme vient en partie de là.

Cependant le petit David échappa assez rapidement
à cette atmosphère déprimante. Élève dès
l'âge de onze ans du collége d’Édimbourg (renommé
à juste titre et qui deviendra plus tard université),
il se trouve dans une ambiance intellectuelle beaucoup 
plus stimulante. Il y écoute les cours de « philosophie
naturelle », c’est-a-dire de physique, de
Robert Stewart (disciple de Newton aprés avoir été
cartésien) et se souviendra certainement de ses
%6
%{\it }{\oe}
leçons lorsqu’il rêvera d’appliquer la méthode expérimentale
à la morale et à la métaphysique. Mais
la formation de Hume au collége fut essentiellement
littéraire. C’est de cette époque que date son goût
de Virgile et de Cicéron. Ce sont les textes (notamment
le {\it De Natura Deorum}) où Cicéron résume les
débats philosophiques des Stoiciens et des Epicuriens
qui découvrent à Hume le monde des discussions
métaphysiques. Revenu à Ninewells dés sa quinzième
année, le jeune David se livre avec passion à la
lecture des anciens et des modernes. Il dévore
Montaigne, Bacon, Malebranche, Bayle, mais aussi
Milton, Pope, Swift, Shaftesbury. À l’âge de vingt
ans, il a déja rempli un gros cahier de réflexions sur
le problème religieux, sur la psychologie, sur l’histoire.
Cette activité intellectuelle bouillonnante, les
leçons de scepticisme qu’il tire de lectures aussi
diverses, un conflit avec sa famille qui voudrait,
malgré lui, l'orienter vers des études juridiques
provoquent une crise de dépression passagère dont
nous trouvons le témoignage dans un curieux brouillon
de lettre à un médecin célèbre (qui n’est pas
comme on ]’a cru George Cheyne, mais le D$^\text{r}$ Arbuthnot).
David qui n’a hérité de son pére que d’une
toute petite rente doit de toute urgence prendre
un état. Aprés un bref essai dans le commerce (au
service d’un marchand de Bristol), David Hume
décide de ne plus résister à sa vocation : II sera
philosophe et homme de lettres, et il entend conquérir
la gloire. Pour pouvoir subsister, il se rend en France
(où la vie est à l’époque beaucoup moins chère),
s’installe à Reims en 1734 à l'hôtel du {\it Perroquet
vert}, puis à La Flèche en Anjou, tout près du collège
%7
%{\it }{\oe}
de Jésuites ou Descartes fut élève. Il y rédige, à
peine âgé de vingt-trois ans, son chef-d’{\oe}uvre :
le {\it Traité de la nature humaine}.

Revenu A Londres en 1737, il a la chance de trouver
un éditeur, et la prudence (ou si l’on veut la faiblesse)
de supprimer les chapitres sur la religion
(il espère la protection de l’évèque Butler). Les deux
premiers livres du traité ainsi publiés « tombérent
mort-nés de la presse », raconta plus tard Hume dans
sa courte {\it Autobiographie}. Ce n’est pas tout a fait
vrai. En fait l’{\oe}uvre intéressa quelques critiques,
mais n’atteignit pas le grand public (qui seul donne
la notoriété). A cette époque, Hume entre en relations
avec Hutcheson, professeur à Glasgow, qui
lui présente son jeune étudiant Adam Smith (qui
restera toujours l’ami de Hume) et lui trouve un
éditeur pour les deux livres suivants du {\it Traité de
la nature humaine} dont le succés n’est pas plus
grand. Hume cependant ne doute pas de sa valeur.
Son échec vient de la présentation trop lourde et
trop savante de sa philosophie et non pas du fond
({\it more from the manner than the matter} dira l'{\it Autobiographie}).
Hume décide alors d’écrire des essais
courts et brillants et publie en 1741 a Édimbourg
ses {\it Essais moraux et politiques} (humilié par ses
insuccés il présente d’ailleurs cet ouvrage comme son
premier livre !). Cette fois les lecteurs sont nombreux,
et Hume croit pouvoir présenter sa candidature a
la chaire de philosophie morale de l’Université de
Glasgow. L’opposition des chrétiens empéche sa
nomination. En 1746, Hume devient le secrétaire
particulier du général Saint-Clair, un Ecossais qui
est son parent éloigné, et l’accompagne dans une
%8
%{\it }{\oe}
mission diplomatique à Vienne et A Turin. Pendant
son voyage paraissent ses {\it Essais philosophiques sur
l'entendement humain} (plus tard {\it Enquête sur l’entendement
humain}), qui reprennent dans un style nouveau
les deux premiers livres du {\it Traité de la nature
humaine}. Cette fois les chapitres sur le miracle et
sur la providence particulière paraissent avec le
reste (1748). A son retour c’est le troisiéme livre du
Traité que Hume reprend avec l’{\it Enquête sur les
principes de la morale} (1751). Dés lors la notoriété
de Hume s’affirme. Il entre en relations épistolaires
avec Montesquieu à propos de l’{\it Esprit des lois}.
Et s'il échoue une nouvelle fois (1751) dans sa candidature
a l’Université de Glasgow, il devient conservateur
de la bibliothéque de la « Faculté des Avocats »
à Edimbourg. Il trouve là tous les documents
nécessaires pour écrire de 1754 A 1761 sa monumentale
{\it Histoire d’ Angleterre}, de Jules César à Jacques II.
Le premier volume qui traite des régnes de Jacques
I$^\text{er}$ et et de Charles I$^\text{er}$ déclenche un petit scandale
dans les milieux religieux, et Hume n’améliore pas
son cas auprés des chrétiens en publiant ses {\it Quatre
dissertations} (dont son {\it Essai sur l'histoire naturelle
de la religion} et son {\it Essai sur le suicide}). Il est vrai
qu'il s’empresse de retirer l’{\it Essai sur le suicide}
ainsi qu’un {\it Essai sur l’immortalité de l’âme} pour les
remplacer par un {\it Essai sur la règle du goût}. Hume
subit aussi avec patience les tracasseries des avocats
d’Édimbourg qui lui reprochent d’avoir acheté
pour la bibliothéque les {\it Contes} de La Fontaine, et
l'{\it Histoire amoureuse des Gaules} de Bussy-Rabutin !

C’est en France que Hume devait connaître la
gloire. A l’appel de lord Hertford, ambassadeur
%9
%{\it }{\oe}
d’Angleterre à Paris, il va exercer de 1763 à 1766
les fonctions de secrétaire d’ambassade. Il sera
même quelque temps — lorsque lord Hertford
est rappelé et en attendant son successeur — « chargé
d’affaires », c’est-à-dire en fait ambassadeur, grâce
à la puissante protection de la comtesse de Boufflers.

A Edimbourg Hume inquiéte, à Londres il n’est
qu’un Ecossais, qu’un intellectuel provincial. A
Paris, le petit monde des philosophes qui a lu ses
Essais, qui connaît l’opposition de Hume à la
« superstition » et au « fanatisme » le tient pour un
pphilosophe de premier plan. Il deviendra trés vite
l'ami de d’Alembert, de Diderot, d’Helvétius, du
baron d’Holbach. Certes, officiellement, Hume est
déiste, à un diner du baron d’Holbach il confesse
même n’avoir jamais rencontré d’athée (« Regardez
autour de vous, répond le baron : Il y en a quinze
autour de cette table ! »). En fait ses positions anti-religieuses,
son Essai sur les miracles le rendent
sympathique aux encyclopédistes qui le tiendront
désormais pour un « frére ». A Paris, c’est un véritable
triomphe. Les plus grandes dames s’arrachent ce
quinquagénaire bedonnant. C’est la duchesse de
La Valliére qui tient à le voir dès son arrivée a Paris,
avant même qu’il ait pu changer de costume !
C’est Mme Du Deffand, Mme Geoffrin, Mlle de Lespinasse
qui le fêtent dans leurs « salons ». même
l'apparence physique de Hume, assez ingrate (il
est obése, son visage empaté est peu expressif)
qui lui avait valu naguére, tandis qu’il voyageait
en Italie avec le général Saint-Clair, les railleries
du jeune James Caulfield, futur lord Charlemont,
et quelques déboires amoureux, est maintenant
%10
%{\it }{\oe}
trouvée sympathique. Son visage un peu lourd, son
fort accent écossais donnent au grand philosophe
un air débonnaire, une simplicité de bon aloi
(Mme Du Deffand l'appelle « mon cher paysan »,
Mme Geoffrin « mon gros drôle, mon gros coquin »).
Hume vole de succès en succès, écrit à son ami le
D$^\text{r}$ Robertson : « Je ne mange que de l’ambroisie,
je ne bois que du nectar, je ne respire que de l’encens,
je ne marche que sur des fleurs. » a

Marie-Charlotte Hippolyte de Campet de Saujeon,
comtesse de Boufflers et maîtresse en titre du prince
de Conti, avait dès 1761 écrit à Hume pour lui dire
toute son admiration pour sa philosophie « sublime ».
Elle avait tenté de le rencontrer au cours d’un voyage
à Londres, sans succés car Hume n’avait pas voulu
quitter Édimbourg. A Paris elle l'invite tout de
suite à ses lundis, puis à ses vendredis plus intimes.
Il semble que cette jolie femme de trente-cing ans
ait été quelque peu amoureuse du gros philosophe
vieillissant. Hume la considéra toujours comme une
amie trés chère (le philosophe, cinq jours avant sa
mort, lui écrit encore le 20 août 1776 pour lui
annoncer qu’il se sent perdu et lui dire une dernière
fois son « affection » et son « respect »), mais il ne
parait pas que leurs relations aient jamais été plus
intimes. Nous connaissons mal la vie intime de Hume,
mais il semble que le philosophe, resté célibataire
(à Edimbourg, sa s{\oe}ur Katherine tenait son ménage)
se soit, à cause d’anciens déboires, ou par souci de
préserver son indépendance et son travail, toujours
méfié des passions amoureuses.

En 1766, le nouvel ambassadeur d’Angleterre, le
duc de Richmond, arrive à Paris et Hume repart
% 11
%{\it }{\oe}
en Angleterre. C’est ici que se situe un des épisodes
les plus mal éclaircis de la vie de Hume. Dès ses
lettres de 1761 la comtesse de Boufflers avait intéressé
Hume a Jean-Jacques Rousseau. Pendant le
séjour à Paris, la marquise de Verdelin demande au
philosophe écossais de chercher pour Jean-Jacques
un refuge en Angleterre. Celui-ci, proscrit de Genéve,
sa ville natale, interdit de séjour en France, persécuté
par les habitants de Môtiers-Travers, en butte à la
haine des encyclopédistes qui le tiennent pour un
dévot, est dans une des périodes les plus critiques de
son existence. Hume, ému par les malheurs de
Jean-Jacques, et tout d’abord enthousiasmé par sa
simplicité et sa franchise (Rousseau est un Socrate
moderne, dira-t-il), part avec lui le 4 janvier 1766.
Ils arrivent à Londres le 13 ou Jean-Jacques est
fété et reçoit de Hume mille témoignages d’amitié.
Mais Rousseau qui déteste le monde et cherche la
solitude n’entend rester à Londres que jusqu’à
l'arrivée de sa servante-maîtresse Thérèse Le Vasseur.
En fait c’est seulement le 19 mars 1766 qu’il part
pour Wooton, maison de campagne dans les bois du
Derbyshire qu’un ami de Hume, Davenport, avait
mise à sa disposition.

Moins de trois mois ont suffi pour altérer la belle
amitié de Hume et de Rousseau. Désormais Rousseau
considére Hume comme un traitre et comme un
malhonnête homme. Que s’est-il passé ? Il est certain
que Rousseau a toujours eu un caractère soupçonneux,
une tendance paranoiaque au délire d’interprétation
que les réelles persécutions dont il fut
victime n’ont fait qu’aggraver. Avant même d’atteindre Calais,
le premier soir du voyage, un curieux
% 12
%{\it }{\oe}
incident se produit. Hume, Rousseau et Luze, un
ami suisse qui va à Londres pour ses affaires, descendent
dans un hotel de Senlis et couchent dans une
chambre à trois lits. Jean-Jacques qui cherche en
vain le sommeil entend soudain Hume dire a plusieurs
reprises « à pleine voix » et avec « une véhémence extrême » :
« Je tiens Jean-Jacques Rousseau. »
Hume n’est-il pas l’ami intime des ennemis les plus
acharnés de Rousseau, les encyclopédistes athées,
Diderot, d’Alembert et d’Holbach ? Ne s’est-il
pas mis d’accord avec cette clique pour faire de
Rousseau en quelque sorte son prisonnier ? Dès le
départ ce rêve ou cette hallucination (car Luze
n’a pas été éveillé par ces voix véhémentes) met
Rousseau sur ses gardes.

Cela suffit-il pour que nous n’accordions aucun
crédit aux accusations formulées ultérieurement par
Jean-Jacques ? Certaines assurément sont délirantes.
A Londres, Rousseau est exaspéré par les
compliments de Hume qui a toujours à son chevet
un tome de la {\it Nouvelle Héloïse}. Pure hypocrisie,
juge Rousseau, car Hume ne peut aimer ce roman !
En réalité Hume, qui fait effectivement des réserves
sur les idées de Jean-Jacques, apprécie son roman,
le tient pour le chef-d’{\oe}uvre du philosophe français
(lettre à Blair du 25 mars 1766). Mais il ya plus
grave. Hume n’ignore pas qu’à Paris son ami Walpole
a fait à Rousseau une plaisanterie trés méchante
(écrivant et faisant publier une lettre d’invitation a
Rousseau, au nom du roi de Prusse qui lui promet a
son choix faveurs ou persécutions, puisqu’il parait
aimer les persécutions !). Or tandis que Rousseau se
préoccupe de faire venir à Londres des papiers pour
% 13
%{\it }{\oe}
la rédaction de ses {\it Confessions}, papiers restés en
France, Hume lui propose Walpole comme commissionnaire !
Hume fait faire une enquête à la banque
Rougemont sur les vraies ressources de Jean-Jacques
qui crie toujours misère. D’autre part Hume veut
toujours se charger d’expédier et de retirer de la
poste le courrier de Jean-Jacques, et celui-ci se
plaindra que son courrier lui parvienne toujours
décacheté et grossièrement refermé. Hume répondra
plus tard que s’il tenait à se charger du courrier de
Rousseau, c’est parce que ce dernier, alléguant sa
pauvreté, ne voulait payer le port d’aucune lettre
(a l’époque, c’est le destinataire qui payait) et que
lui, David Hume, ne tenait pas a laisser trop longtemps
le courrier à la poste, mais désirait le soustraire
promptement {\it from the curiosily and indiscretion
of the clerks of Post-office} !

Quoi qu’il en soit de toutes ces accusations, il
reste que Hume porte la responsabilité d’avoir mis
la querelle sous les yeux du public en laissant éditer
a ses amis français l’{\it Exposé succinct} de son différend
avec Rousseau, livrant ainsi Jean-Jacques a de
nouvelles railleries. Ce fut peut-être en cette affaire
sa seule faute, une entreprise, écrira-t-il à Adam
Smith le 17 octobre 1767, que « l’un et l’autre nous
avons été enclins a blâmer parfois, à regretter
toujours ». Si Hume ne fut pas le traître qu’a imaginé
Rousseau, peut-être manqua-t-il ici de patience et
de générosité. Une des faiblesses de Hume est d’avoir
été toujours trés préoccupé de sa réputation. Et
il craignait que Rousseau, dans ses {\it Confessions}, ne
raconte l'histoire à sa maniére (en réalité le récit de
Rousseau s’arrétera avant les événements de 1766).
% 14
%{\it }{\oe}

Hume, grace à son fidèle protecteur, lord Hertford,
devient sous-secrétaire d’État en 1767. Par un
étrange retour des choses c’est lui qui, entre autres
affaires, est chargé de régler les conflits et de décider
de l’avancement des pasteurs de cette Église
d’Écosse qui naguére avait tenté d’entraver sa
carriére !

En 1769 Hume rentre à Édimbourg désormais
riche et considéré. Il choisit de finir ses jours dans
sa ville natale, intense foyer de culture, qui est bien
plutôt que Londres la capitale intellectuelle de
l’Angleterre de ce temps. A Édimbourg, cette
« Athènes du Nord », Hume retrouve en effet des
amis éminents, Adam Smith, le juriste lord Kames,
Ferguson, et aussi des adversaires loyaux et courtois
comme le théologien George Campbell qui avait
rédigé une critique de l’{\it Essai sur les miracles}, critique
fort appréciée par Hume lui-même. Hume
s’emploie à corriger ses {\oe}uvres pour de prochaines
éditions, à répondre à son courrier, et il rejoint
fréquemment ses amis écossais au {\it Poker Club} et a
la {\it Select Society}, club philosophique et littéraire
qu’il avait fondé lui-même en 1754.

Très rapidement sa santé décline. Il souffre d’une
tumeur de l'intestin, et dès le début de 1776 il se
sait perdu. En avril il rédige son testament. Il a
dans ses papiers un ouvrage inédit commencé
dès 1751 et dont il a déja soumis à cette époque les
premiers chapitres à son ami Gilbert Elliot of Minto :
{\it Les dialogues sur la religion naturelle}. Adam Smith
n'est pas trés favorable à la publication de cet
ouvrage. C’est donc le neveu de Hume qui sera
chargé de cette édition posthume. {\it Les Dialogues}
% 15
%{\it }{\oe}
paraîtront en 1779, plus de deux ans aprés la mort
de Hume.

Hume est mort avec la plus grande sérénité.
Dans sa bréve autobiographie, rédigée le 18 avril 1776
il déclare : « Il est difficile d’être plus détaché de la
vie que je ne le suis à présent. » Le 13 août, il dit qu'il
se console d’abandonner des amis, car « {\it hélas, on
ne laisse que des mourants}, comme Ninon de Lenclos
le dit sur son lit de mort. La mort m'apparaît si
peu terrible maintenant qu’elle approche, que je
dédaigne de citer des héros et des philosophes comme
exemples de courage. Le témoignage d’une femme de
plaisir qui néanmoins était également philosophe
est suffisant ». Hume s’éteignit sans angoisse dans
l'aprés-midi du dimanche 25 août 1776. C’était la
veille de son soixante-cinquième anniversaire.
%%%%%%%%%%%%%%%%%%%%%%%%%%%%%%%%%%%%%%%%%%%%%%%%%%%%%%%%%%%%%%%%%%%%%%%%
