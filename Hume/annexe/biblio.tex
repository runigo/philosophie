
%%%%%%%%%%%%%%%%%%%%%
\chapter{Bibliographie}
%%%%%%%%%%%%%%%%%%%%%

%%%%%%%%%%%%%%%%%%%%%%%%%
\section{En anglais}
%%%%%%%%%%%%%%%%%%%%%%%%%{\it }

Hendel, {\it Studies on the philosophy of D. Hume}, Princeton,
1925.

A. E. Taylor, {\it David Hume and the miraculous}, Gambridge, 1927.

J. Laird, {\it Hume’s Philosophy of Human Nature}, London,
1932.

{\it Hume and present day problems}, Aristotelian Society,
suppl., vol. XVIII, London, 1939 (4 symposia sur
l'identité du moi, sur les concepts {\it a priori}, sur l’éthique,
sur la religion naturelle avec des articles de Taylor,
de Lairp, de Jessop).

Norman Kemp Smith, {\it Philosophy of David Hume}, London,
1941.

D. G. C. Mac Nabb, {\it David Hume, His theory of Knowledge
and Morality}, London, 1954.

E. G. Mossner, {\it The life of David Hume}, London, 1954.

%%%%%%%%%%%%%%%%%%%%%%%%%
\section{En français}
%%%%%%%%%%%%%%%%%%%%%%%%%

G. Compayré, {\it }La philosophie de D. Hume, Paris, 1873.

G. Lechartier, {\it David Hume sociologue et moraliste},
Paris, 1900.

L. Levy-Bruhl, Orientation de la pensée de D. Hume,
{\it Revue de métaphysique et de morale}, 1909.

A. Leroy, {\it Critique et religion chez D. Hume}, Paris, 1931.

%92
Laporte, Le septicisme du Hume, {\it Revue philosophique,}
1933-1934.

G. Brercer, Husserl et Hume, {\it Revue internationale de
philosophie}, 1939.

{\it Mélange David Hume}, divers articles, {\it Revue internationale
de philosophie}, Bruxelles, 1952.

DELEUZE, Empirisme et subjectivité, Paris, Presses universitaires 
de France, 1953.

A. Leroy, {\it David Hume}, Paris, Presses Universitaires de
France, 1953.

O. Brunet, Philosophie et Esthétique chez D. Hume,
Paris, Librairie A.-G. Nizet, 1965.

%%%%%%%%%%%%%%%%%%%%%%%%%%%%%%%%%%%%%%%%%%%%%%%%%%%%%%%%%%%%%%%%%%%%%%%%
