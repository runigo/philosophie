
%%%%%%%%%%%%%%%%%%%%%
\section{Encyclopédie de la philosophie}
%%%%%%%%%%%%%%%%%%%%%
%
\subsection{Nécessité}
Est dit nécessaire « ce qui ne peut être autrement qu'il n'est »,
telle est du moins la définition aristotélicienne de la notion (grec {\it anankaion}).
Elle s'applique
notamment pour caractériser la nature de
la relation qui relie entre elles les propositions (prémisses et protases) du syllogisme scientifique ou démonstratif. Est
nécessaire ce qui appartient à un sujet
partout et toujours : ainsi la propriété
d’avoir la somme de ses angles intérieurs
égale à deux droits est une propriété
nécessaire pour tout triangle en tant que
tel. Mais, outre le sens logique et ontologique, il y a un sens psychologique, de
contrainte inévitable. Ce qui a conduit certains à émettre l'hypothèse que le concept
est peut-être né d’une projection anthropomorphique de l’idée originelle de coercition,
ce qui aurait pour conséquence que la
notion de nécessité déontologique (c’est-à-dire d’obligation) précéderait historiquement toutes les autres formes de nécessité.
La notion de nécessité physique et causale comme subordination aux lois de la
nature et celle de nécessité logique
comme propriété de ce qui est « forcément » vrai en vertu des lois logiques
seraient apparues par la suite, selon un
ordre inversé par rapport à celui qui est
parfois considéré logiquement rationnel.

\subsection{Nécessité et modalité}

La notion leibnizienne de nécessité
comme vérité dans tous les mondes possibles à été reprise dans l’analyse sémantique contemporaine des modalités.
D'après ces analyses, il apparaît que la
notion modale fondamentale est la notion
de possibilité plutôt que celle de nécessité. À côté de cette notion leibnizienne
de nécessité absolue, la logique. contemporaine a caractérisé différentes notions
de nécessité relative, en-isolant des sous-classes pertinentes à l’intérieur de la
classe de tous les mondes possibles,
comme celle des mondes où sont valides
les lois de la nature ou les lois de code
pénal. Toute logique de la nécessité,
quelle qu’elle soit, a comme condition
minimum d'éviter ce que l’on appelle
« l’effondrement des modalités », à savoir
la démonstration de l’équivalence entre
une proposition nécessaire et une proposition dépourvue d’opérateurs modaux
(cette équivalence est présente dans toute
philosophie fataliste ou strictement déterministe : il suffit de penser à l’assimilation
opérée par Hegel entre ce qui est rationnel et ce qui est réel). Dans une acception
élargie, la nécessité logique est parfois
assimilée à l’analycité. Les énoncés analytiques résultent nécessaires dans la
mesure où les règles linguistiques
« créent » la signification des termes que
l’on y trouve, que ces termes soient des
constantes logiques (« et », « ou »,
« tous », etc.) ou des mots du langage
ordinaire comme « célibataire » ou « non-
marié ». Du point de vue pragmatiste
d'auteurs comme W.V.O. Quine, la nécessité d’un énoncé quelconque consiste dans
son immunité à l’intérieur du système des
connaissances prouvées, c’est-à-dire dans
le fait que le renoncement à l’énoncé en
question est trop coûteux pour le système
dans son ensemble. Tout en excluant qu’il
existe une distinction historiquement définitive entre analytique et synthétique et
entre nécessaire et contingent, Quine.
pense toutefois que la nécessité logique
peut être justement attribuée. à des
énoncés dont la vérité dépend uniquement des constantes logiques qui sont
présentes en lui. Il s’agit d’un pas en avant
par rapport à la conception du Tractatus
de L. Wittgenstein, qui assimilait les
nécessités logiques aux tautologies du calcul propositionnel, excluant de cette
manière les vérités logiques qui dépendent de la présence de quantificateurs. De
toute façon, cette conviction est partagée
par presque toute  l'épistémologie
contemporaine, surtout par l’épistémologie empiriste : que la nécessité des lois
scientifiques et des inférences garanties
par elles ne dépend pas de l’existence de
connexions nécessaires dans la nature,
mais doit être indirectement référée à la
nécessité logique ou au concept d’implication logiquement nécessaire.

%\vspace{0.35cm}
%$\to$ existence ; mondes possibles ; possibilité ; quantificateurs ; Quine

%%%%%%%%%%%%%%%%%%%%%%%%%%%%%%%%%%%%%%%%%%%%%%%%%%%%%%%%%%%%%%%%%%%%%%%%
