
%%%%%%%%%%%%%%%%%%%%%
\section{Pratique de la philosophie}
%%%%%%%%%%%%%%%%%%%%%

{\bf NÉCESSITÉ}

\begin{itemize}[leftmargin=1cm, label=\ding{32}, itemsep=1pt]
\item {\footnotesize ÉTYMOLOGIE} : latin {\it necessitas}, « l'inéluctable », « besoin impérieux », « obligation impérieuse », « nécesssité logique ».
\item {\footnotesize SENS ORDINAIRES ET PHYSIQUE} : état de choses qui ne peut pas ne pas exister.
\item {\footnotesize MÉTAPHYSIQUE} : puissance (parfois divinisée) qui gouverne le cours
de la réalité.
\item {\footnotesize LOGIQUE} : {\bf 1}. Caractère de ce qui ne peut être faux, de
ce qui est universellement vrai. {\bf 2}. Relation inévitable entre deux propositions.
\item {\footnotesize SENS DÉRIVÉ} : le besoin ; ce dont un être ne peut pas
se passer.
\end{itemize}

La nécessité est une modalité logique
qui s'oppose à la simple possibilité mais
aussi à l'impossibilité, ainsi qu'à la
contingence. Elle peut qualifier des
pensées, des idées ou des principes :
elle signifie alors que leur vérité ne peut
pas être refusée par l'esprit (par
exemple parce qu'elle est évidente ou
logiquement démontrée ; parce que le
contraire est impossible). Au sens physique, la nécessité renvoie au déterminisme ; au sens métaphysique, elle renvoie au fatalisme. L'acception morale
du terme est trompeuse : une obligation
(ce qui nous apparaît nécessaire) se distingue justement d'une contrainte en ce
qu'il est toujours possible de ne pas la
respecter : elle renvoie donc davantage
à une exigence qu'à une nécessité ({\it cf.}
textes de Spinoza en annexe).

\begin{itemize}[leftmargin=1cm, label=\ding{32}, itemsep=1pt]
\item {\footnotesize TERMES VOISINS} : besoin ; destin ; fatalité ; universalité.
\item {\footnotesize TERMES OPPOSÉS} : contingence.
\item {\footnotesize CORRÉLATS} : Catégorie ; déterminisme ; liberté ; modalité.
\end{itemize}

%%%%%%%%%%%%%%%%%%%%%%%%%%%%%%%%%%%%%%%%%%%%%%%%%%%%%%%%%%%%%%%%%%%%%%%%%%%
