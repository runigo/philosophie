
%%%%%%%%%%%%%%%%%%%%%
\section{Encyclopédie de la philosophie}
%%%%%%%%%%%%%%%%%%%%%
%1
{\bf Déterminisme} : terme qui qualifie en
science et en philosophie toute doctrine
affirmant la connexion nécessaire de tous
les phénomènes selon le principe de causalité. Le déterminisme est un trait dominant de la pensée moderne du {\footnotesize XVII}$^\text{\,e}$ au
{\footnotesize XIX}$^\text{\,e}$ siècle. Il faut en chercher les prémisses
dans la révolution scientifique initiée à
partir des travaux de Galilée, dont l’un
des effets principaux a été de supprimer
le recours aux causes finales dans l’explication des phénomènes de la nature, évolution qui est à la base du mécanisme de
Descartes, Hobbes, Gassendi et Spinoza.
Il revient principalement à Gassendi de
faire le lien entre le mécanisme et le
déterminisme modernes d’une part, et les
anciennes conceptions atomistes des épicuriens d’autre part (issues de Leucippe
et de Démocrite). Chez les Anciens, le
déterminisme présente deux facettes :
l'une proprement physique et l’autre
morale. Démocrite conçoit la nature
comme entièrement réglée par le mouvement des atomes dans l’espace vide. Ce
déterminisme matérialiste rigide, auquel
même l’homme et son âme sont assujettis,
est partiellement rectifié par Épicure puis
par Lucrèce, dont le but est de découvrir
un fondement physique à la possibilité du
libre vouloir et du hasard. C’est pourquoi
ils assouplissent leur déterminisme en
introduisant dans leurs doctrines le
concept de {\it clinamen} (déviation), selon
lequel les atomes sont parfois en mesure
d’infléchir spontanément leur direction
propre et de rompre la chaîne des causes
nécessaires. À l’époque moderne, le
déterminisme se heurte à son tour au problème de la compatibilité de la conception mécaniste de la nature propre à la
science avec les thèses fondamentales de
la religion et de la morale chrétiennes
(existence et liberté de l’âme, action providentielle de Dieu). De Descartes à
Leibniz et à Kant, on tend à séparer, sur
le plan métaphysique ou bien sur le mode
« critique », la réalité du monde naturel
(reconductible aux principes de la divisibilité de la matière et du mouvement, et
donc à des lois nécessaires), de la réalité
du penser et du vouloir, compris comme
des activités autonomes et spontanées,
capables de s’autodéterminer librement
en dehors des conditionnements matériels. En opposition avec cette tendance,
les « épicuriens » et les libertins modernes
adoptent des solutions de type radicalement matérialiste : aux raisons scientifiques se mêlent chez eux des motivations
idéologiques : hostilité morale à l’encontre du christianisme et lutte politique
contre l'Eglise. Cette attitude culmine à
%3
l’époque des Lumières notamment avec
Diderot, Claude Adrien Helvétius, Julien
Offray de La Mettrie, Paul Henri d'Holbach. D'un point de vue strictement scientifique, le déterminisme trouve son
expression la plus accomplie au début du
{\footnotesize XIX}$^\text{\,e}$ siècle. avec l’œuvre de Pierre Simon
Laplace qui soutient que si, à un instant
donné, toutes les forces agissant sur la
nature et la position de tous les corps
étaient connues, il serait possible en principe de prévoir tous les états futurs de
l'univers; pour une intelligence ainsi
faite, tout serait clair et certain. l’avenir
aussi bien que le passé ({\it Essai philosophique sur les probabilités}, 1814). Différents courants du positivisme développent
des conceptions déterministes, en particulier les matérialistes allemands Karl Vogt
et Jakob Moleschott, qui soutiennent
(non sans quelques motifs d’ordre politique et idéologique, outre que scientifique) que les phénomènes que l’on dit
spirituels ont en fait un fondement mécanique et matériel. Par la suite, les thèses
déterministes reviennent au premier plan
avec le débat sur l’évolutionnisme darwinien et sur ses applications psychologiques et sociologiques. Dans ses formes
les plus radicales, le déterminisme conduit
à soutenir d’un côté la réduction de tous
les phénomènes de la pensée à leur origine biologique (Chauncey Wright, Ernst
Haeckel), et de l’autre la réduction de
l'individu à la dynamique des lois sociales
(William Baldwin Spencer, William Graham Sumner). C’est contre ces formes de
déterminisme, en grande partie idéologiques, que réagissent les différents courants spiritualistes du début du {\footnotesize XX}$^\text{\,e}$ siècle ; les
réactions ainsi suscitées ne sont elles-mêmes pas moins idéologiques. Les critiques que, du point de vue strictement
logique, Charles Sanders Peirce adresse
dès 1892 aux thèses déterministes sont
plus pertinentes : il anticipe divers aspects
de la révision formelle du concept de
causalité nécessaire opérée par le néo-positivisme en ce siècle. Sur le plan rigoureusement scientifique, la crise du
déterminisme coïncide avec l’abandon,
par les physiciens, du modèle mécaniste
universel. Avec la théorie quantique en
particulier, les conditions initiales et
nécessaires de l'hypothèse déterministe
ne peuvent plus se réaliser, puisqu'il n’est
plus possible de déterminer de façon univoque, pour chaque instant donné, l’état
%4
du système physique, autrement dit les
positions et les moments absolus de tous
les points matériels composant le système
lui-même. Ceci en vertu du  {\it principe d’incertitude}  formulé par Werner Heisenberg
(1927), qui soutient que toute mesure
physique provoque une perturbation du
système que l’on entend mesurer. En physique atomique notamment, dit Heisenberg, « on ne peut en aucune façon faire
abstraction des modifications que les instruments d’observation produisent sur
l’objet observé ». Ainsi s'effondre ce que
Peirce avait indiqué comme le présupposé
théorique et inconsciemment « métaphysique » de tout le déterminisme classique,
à savoir que la nature constituerait un Système en soi, absolu, achevé et réel.

%\vspace{0.35cm}
%$\to$\ \ cause, hasard

%%%%%%%%%%%%%%%%%%%%%%%%%%%%%%%%%%%%%%%%%%%%%%%%%%%%%%%%%%%%%%%%%%%%%%%%%%
