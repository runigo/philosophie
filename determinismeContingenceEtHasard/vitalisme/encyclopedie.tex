
%%%%%%%%%%%%%%%%%%%%%
\section{Encyclopédie de la philosophie}
%%%%%%%%%%%%%%%%%%%%%
%1
{\bf Vitalisme} : terme qui se réfère aux doctrines qui soutiennent la particulière spécificité des phénomènes de la vie et donc
leur  irréductibilité aux phénomènes
mécaniques et à la dynamique purement
physico-chimique du monde inorganique.
On peut rapprocher le vitalisme de la biologie d’Aristote, reprise par les scolastiques au Moyen Âge. Il s’agit d’un
vitalisme  « hylémorphe », c’est-à-dire
qu'il considère les fonctions vitales
comme un principe intrinsèque à la
nature et à la forme du vivant. Dans un
sens plus restreint et plus courant, On qualifie
%
de vitalistes ces théories qui, pendant
la seconde moitié du {\footnotesize XVIII}$^\text{\,e}$ siècle, s’opposèrent au mécanisme en supposant, pour
expliquer le phénomène de la vie, un principe formatif, ou « force », qui agirait
comme cause finale. Ce  vitalisme
moderne, dans un essai de conciliation
%2
des résultats de la science avec le finalisme de la tradition, n’identifia plus la vie
à l'âme ou principe animateur spirituel.
Par exemple, les médecins de l'école de
Montpellier, et en particulier Paul Joseph
Barthez (1734-1806) qui collabora aussi à
l'{\it Encyclopédie}, imaginèrent le principe
vital comme une force inconsciente qui
agit comme principe organisateur à un
niveau moléculaire. Buffon et Lamarck
partagèrent certains aspects du vitalisme,
qui rencontra de graves difficultés au
début du {\footnotesize XIX}$^\text{\,e}$ siècle face au progrès de la biologie et en particulier face à la réalisation
de la synthèse de l’urée sur la base de
seuls éléments inorganiques (1828), à la
formulation du principe de la conservation de l’énergie (1842) et enfin face à la
théorie darwinienne (1859) qui expliquait
la formation des organismes vivants par
l’accumulation d'événements fortuits provenant du milieu physique. Contre ces
nouvelles données et théories scientifiques et vitalistes (ou néo-vitalistes), ils
tentèrent de démontrer, sur des bases
expérimentales, l'existence dans l’organisme vivant d’une « finalité primaire »
irréductible aux données et à l'influence
du milieu. Cette finalité primaire prit différents noms chez les scientifiques et les
penseurs de la fin du {\footnotesize XIX}$^\text{\,e}$ siècle : « force vitale » pour le biologiste Claude Bernard
(1813-1878), « force dominante » pour le
biologiste J. Reinke (1849-1931), « psychoïde » et « entéléchie » pour le biologiste et philosophe Hans Driesch (1867-1941), « élan vital » chez Bergson. L’entéléchie est « l’agent non mécanique », hors
de l’espace et du temps, qui peut susciter
la vie mais que l’on ne peut pas décrire
en termes naturels positifs.

Jakob J. von Uexküll (1896-1944) soutient la thèse de la biologie par rapport
aux sciences physico-mathématiques. Il
ajoute, à la matière et à la force, un troisième élément décisif pour l’observation
scientifique : la « forme biologique ». Elle
n’est pas à proprement parler une fin mais
une fonction à laquelle l'organisme vivant
est déjà adapté au moment même où il
perçoit le milieu environnant. Il n’existe
donc pas de milieu naturel unique, mais il
y en a autant que de formes biologiques.
Toute forme est donc parfaite dans les
limites de son milieu. Le darwinisme,
considérant l’homme comme le sommet
de l’évolution et cette dernière comme un
processus d’accumulation, néglige la
%3
nécessaire pluralité des milieux et le fait
par lequel toute forme vivante devait
autrefois accumuler les processus d’adaptation pour subsister, comme cela arrive
de fait dans un milieu autonome. Le vitalisme de Uexküll ne renvoie plus à une
force extérieure aux lois de la nature mais
à la cohérence structurelle de la vie. Plus
tard, du fait des développements décisifs
de la biologie moléculaire et de la génétique, la discussion entre les vitalistes et
les mécanistes a perdu beaucoup de son
intérêt et de son actualité.
%%%%%%%%%%%%%%%%%%%%%%%%%%%%%%%%%%%%%%%%%%%%%%%%%%%%%%%%%%%%%%%%%%%%%%%%%%%%%
