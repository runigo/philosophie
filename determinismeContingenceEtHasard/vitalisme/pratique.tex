
%%%%%%%%%%%%%%%%%%%%%
\section{Pratique de la philosophie}
%%%%%%%%%%%%%%%%%%%%%

{\bf VITALISME}

\begin{itemize}[leftmargin=1cm, label=\ding{32}, itemsep=1pt]
\item {\footnotesize ÉTYMOLOGIE}: latin {\it vita}, « vie ».
\item {\footnotesize SENS ORDINAIRES}: {\bf 1.}  Doctrine de
l'école de Montpellier, selon
laquelle un « principe vital » régit les
phénomènes de la vie. {\bf 2.} Toute
théorie qui s'oppose à une réduction de la vie à ses caractères physico-chimiques et recourt à une
force vitale distincte de la matière.
\end{itemize}

Le médecin allemand Georg E. Stahl,
expose, entre 1704 et 1708, une doctrine
qualifiée d’animisme, et reproche à la
médecine mécaniste de ne pas prendre
en compte la vie elle-même. Théophile
de Bordeu et Paul-Joseph Barthez étudient la médecine à Montpellier où sont
discutées les thèses de G. E. Stahl ; Bordeu ne pense pas non plus que la vie
puisse s'expliquer par le mécanisme et, à
partir de 1751, attribue dans sa physiologie un rôle régulateur à une « âme » qui
est en fait la sensibilité des fibres nerveuses. Cette âme est partout où se manifeste un mouvement spontané, et chaque
organe est pourvu d'une sensibilité et
d'une action propres ; pour cette raison le
corps vivant est comparé à un essaim
d’abeilles, l’action du cerveau sur les nerfs
assurant toutefois l'unité de l'être vivant.
La philosophie de Bergson, par
exemple, peut être considérée comme
un vitalisme, car elle considère que les
propriétés de la matière et le déterminisme que la raison peut y trouver
n'expliquent pas l'essentiel de la vie, soit
l'évolution créatrice.

\begin{itemize}[leftmargin=1cm, label=\ding{32}, itemsep=1pt]
\item {\footnotesize TERMES VOISINS} : animisme ; Organicisme.
\item {\footnotesize TERMES OPPOSÉS} : déterminisme ; mécanisme.
\item {\footnotesize CORRÉLATS} : évolution ; évolutionnisme ;
finalisme ; mécanisme ; organisme ;
spiritualisme ; vie ; vivant.
\end{itemize}

%{\footnotesize XIX}$^\text{\,e}$ siècle.
%%%%%%%%%%%%%%%%%%%%%%%%%%%%%%%%%%%%%%%%%%%%%%%%%%%%%%%%%%%%%%%%%%%%%%%%%%%
