
%%%%%%%%%%%%%%%%%%%%%
\section{Encyclopédie de la philosophie}
%%%%%%%%%%%%%%%%%%%%%
%
%1
{\bf Hasard} : notion qui a fait l’objet de définitions différentes, mais que l’on peut ramener fondamentalement à deux : le hasard
comme événement n’ayant aucune cause
objective et contredisant par là toute
conception rigoureusement déterministe
de la réalité (déterminisme) ; et le hasard
comme événement dont on ne connaît pas
les causes. L’existence objective du
hasard, entendu dans le premier sens, fut
affirmée pour la première fois par Épicure dans le cadre d’une conception atomiste de la réalité : 
il s’agit de la capacité
des atomes de dévier spontanément et
fortuitement de leur ligne de chute verticale, que Lucrèce appela en latin {\it clinamen}. La notion de hasard dans le second
sens, c’est-à-dire la négation de sa réalité
objective et sa réduction à une lacune de
la connaissance humaine, fut au contraire
partagée par les principaux courants de
pensée de l'Antiquité, qui, concevant
l’univers comme organisé en chacun de
ses points par une raison cosmique,
voyaient le hasard comme une cause
purement incompréhensible pour l'esprit
humain.

Les philosophes chrétiens excluaient
aussi le hasard d’un monde dont l’ordre
est prescrit par la Providence divine. Ils
l’expliquaient par le caractère limité de la
raison humaine. Dans la même perspective se trouvent certains philosophes
modernes qui ont analysé de façon spécifique la notion de hasard. Spinoza, considérant toute chose comme produite
nécessairement par la nature divine, niait
l'existence de faits contingents ou occasionnels, et expliquait ceux qui semblaient
tels comme étant dus aux limites de notre
imagination, incapable d’embrasser les
choses dans la totalité de leurs rapports.

%2
Leibniz et Bossuet préféraient au
contraire distinguer la contingence, qui
caractérise la connaissance humaine, et la
nécessité, qui est propre au savoir divin, si
bien que ce qui apparaît à l’homme
comme le fruit d’un pur hasard advient
toujours pour l'esprit divin en fonction
d’une harmonie préétablie. Pour Hume
enfin, la relation de cause à effet n’étant
rien d’autre qu’une habitude produite par
l'observation renouvelée de successions
déterminées entre les phénomènes, on est
en présence d’un hasard lorsque les expériences positives de cette succession sont
en nombre égal ou inférieur à celui des
expériences négatives.
Une tentative originale d’explication du
hasard en termes de causalité fut faite au
{\footnotesize XIX}$^\text{\,e}$ siècle par Antoine Augustin Cournot, qui
dans son {\it Essai sur les fondements de nos
connaissances} (1851) définit un événement accidentel ou fortuit comme la coïncidence de deux ou de plusieurs séries
indépendantes de causes : le caractère fortuit de certains événements vient de ce
que leurs causes déterminantes sont indépendantes, alors que leurs effets se mêlent
brusquement et de façon inopinée. Dans
son essai de 1878 intitulé {\it L'Ordre dans la
nature}, Charles Saunders Peirce aborda à
son tour la notion de hasard dans son rapport avec celle de nécessité. Par une
démarche rigoureusement logique, il
démontra que les concepts de nécessité et
de hasard sont, considérés en soi, dénués
de sens et interchangeables : un monde
dont on suppose que les caractères se
combineraient totalement au hasard révélerait effectivement à la longue qu'il possède un ordre implacable, bien supérieur
à celui qui caractérise le monde dans
lequel nous vivons. Le fait est que hasard
et nécessité doivent toujours, pour acquérir un sens, être rapportés aux intérêts
perceptifs et vitaux des êtres. La polémique de Peirce s’insérait dans les discussions relatives à l’évolutionnisme de
Darwin, qui soutenait le caractère purement fortuit des variations morphologiques des espèces vivantes. Ce hasard
éliminait, selon Darwin, tout finalisme
ainsi que tout plan providentiel et créationniste dans la nature; mais selon
d’autres savants (par exemple A. Gray).
il confirmait la nécessité de l’intervention
divine et la présence dans le monde d’un
projet intelligent. On peut trouver une
version moderne de cette discussion dans
%3
l'ouvrage de Jacques Monod, {\it Le Hasard
et la Nécessité} (1971). D’après Monod, la
« mutation » initiale qui donne naissance
à l’évolution des êtres vivants est un fait
purement fortuit ; mais comme les êtres
vivants, contrairement à tout autre être
dans l’univers, sont dotés d’{\it invariance} et
de {\it téléonomie} (c’est-à-dire de la capacité
de transmettre leur loi structurelle et de
modifier leurs performances en fonction
de l’environnement), la mutation, une fois
inscrite dans le code génétique, est reproduite à des milliards d'exemplaires et
entre dans le champ de la sélection, autrement dit de la nécessité.


%%%%%%%%%%%%%%%%%%%%%%%%%%%%%%%%%%%%%%%%%%%%%%%%%%%%%%%%%%%%%%%%%%%%%%%%
