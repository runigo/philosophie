
%%%%%%%%%%%%%%%%%%%%%
\section{Pratique de la philosophie}
%%%%%%%%%%%%%%%%%%%%%

{\bf HASARD}

\begin{itemize}[leftmargin=1cm, label=\ding{32}, itemsep=1pt]
\item {\footnotesize ÉTYMOLOGIE} : arabe {\it az-zabr}, « dé », « jeu de dés ».
\item {\footnotesize SENS ORDINAIRES} : {\bf 1.} Cause imaginaire d’événements ou de phénomènes inexpliqués (synonyme de sort, fortune). {\bf 2.} Concours de circonstances imprévisible et surprenant (synonyme de coïncidence, accident).
\item {\footnotesize PHILOSOPHIE} : {\bf 1.} Cause accidentelle d'événements ou de
phénomènes qui n'ont pas été provoqués délibérément (ex. : une rencontre « due au hasard »). {\bf 2.}  Phénomènes ou événements produits par la rencontre imprévisible de
séries causales indépendantes ({\it cf.} Cournot en annexe).
\item {\footnotesize ÉPISTÉMOLOGIE}: {\bf 1.} Indéterminisme de la matière, au
niveau de la microphysique, c'est-à-dire impossibilité de prévoir ou de
déterminer le comportement des particules (selon la fameuse « relation d'incertitude » de Werner K. Heisenberg). {\bf 2.}  Comportement des événements pour lesquels s'applique la loi des grands nombres, parce que, globalement, ils se comportent selon les règles de la
probabilité (on peut prévoir le comportement moyen de tels ensembles).
\end{itemize}

Le mot courant hasard regroupe des
notions aussi diverses que l’inexpliqué
(ce dont on ignore la cause), l’inexplicable (ce qui est sans raison, au moins
apparente), l’indéterminé (par exemple,
le parcours d’une étoile filante), et, plus
généralement, le contingent (ce qui
aurait pu ne pas se produire) et le fortuit
(imprévisible). La philosophie s'efforce
de dissiper au moins certaines confusions majeures. Il est tout d’abord préférable d'opposer approches subjective
et objective d'événements également
inexpliqués. Les effets inattendus de certaines actions (la chute inopinée d’un
pot de fleurs mal fixé) ne sont pourtant
pas indéterminés, comme l'explique
Aristote. Si le pot de fleurs tue un passant, c’est un « hasard » (fâcheux !), en ce
sens que l’action apparaît finalisée (délibérée) sans cependant l'avoir été. Le
hasard est ici une illusion de finalité :
ou encore, selon la formule de Bergson, « un mécanisme se comportant
comme s'il avait eu une intention ». Mais
il n'y a rien d’irrationnel dans ce type de
hasard, puisque la chute (du pot de fleurs) est explicable et donc rationnelle.
%
Objectivement, au contraire, le hasard
renvoie au caractère véritablement fortuit, et généralement imprévisible, de
rencontres entre chaînes causales indépendantes : il procède alors, selon Cournot, d’un « concours de faits rationnellement indépendants les uns des
autres ». La complexité du réel est telle
que la réduction de ce type de hasard
ne peut être scientifiquement envisagée : autant dire que toute représentation naïvement déterministe de l'univers,
tant naturel qu'humain, nous est désormais interdite ({\it cf.} Déterminisme). Les
physiciens et les biologistes contemporains semblent se rallier à l’idée de
« déterminisme approché », qui s'efforce
de prendre en compte tous les éléments
d'incertitude attachés aux relations
complexes qu'entretiennent des structures interactives (par exemple les écosystèmes). À la suite de la physique, la
biologie a accompli d’incontestables
progrès en utilisant des analyses probabilistes et en dégageant des lois statistiques pour des phénomènes dont seul
le comportement global peut faire l’objet
d'analyses vraiment fiables.

\begin{itemize}[leftmargin=1cm, label=\ding{32}, itemsep=1pt]
\item {\footnotesize TERMES VOISINS} : accident ; contingence ; fortune ; imprévisibilité ; indétermination ; sort.
\item {\footnotesize TERMES OPPOSÉS} : finalité ; prévisibilité ; providence.
\item {\footnotesize CORRÉLATS} : absurde ; cause; déterminisme ; événement ; irrationnel ; loi ; raison.
\end{itemize}

%%%%%%%%%%%%%%%%%%%%%%%%%%%%%%%%%%%%%%%%%%%%%%%%%%%%%%%%%%%%%%%%%%%%%%%%%%%
