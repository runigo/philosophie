
%%%%%%%%%%%%%%%%%%%%%
\section{Encyclopédie de la philosophie}
%%%%%%%%%%%%%%%%%%%%%
%1
{\bf philosophie de la contingence} : terme philosophique désignant, d’une façon
générale, toute théorie qui s'oppose au
déterminisme, dans la mesure où elle ne
reconnaît pas un ordre causal invariable
et nécessaire aux phénomènes. Plus particulièrement, la philosophie de la contingence renvoie à la philosophie d'Emile
Boutroux. Critiquant la classification des
sciences de Comte, Boutroux affirme que
le passage d’une science à l’autre (des
plus simples aux plus complexes : de la
physique à la chimie, de la biologie à la
sociologie, etc.) n’est pas seulement déterminé par une différence dans les
méthodes d'investigation, mais qu'il correspond à une différence objective existant entre des ordres distincts de
phénomènes. Le passage d’une science à
l’autre correspond aussi, pour Boutroux,
à une série de mondes absolument différents (le monde physique, chimique, organique, etc.). Chacun d’eux apparaît avec
%2
un caractère nouveau, original, imprévisible (contingent) par rapport au précédent. Chaque saut d’un ordre à l’autre
comporte un démenti du principe de causalité et révèle la présence, dans la nature,
d’un principe de liberté. Boutroux insiste,
en particulier, sur l’irréductibilité des phénomènes biologiques aux lois chimiques
et physiques, et sur l’irréductibilité de la
conscience humaine à l’ordre biologique.


%\vspace{0.35cm}
%$\to$ vitalisme

%%%%%%%%%%%%%%%%%%%%%%%%%%%%%%%%%%%%%%%%%%%%%%%%%%%%%%%%%%%%%%%%%%%%%%%%%%%
