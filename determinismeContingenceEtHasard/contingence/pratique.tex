
%%%%%%%%%%%%%%%%%%%%%
\section{Pratique de la philosophie}
%%%%%%%%%%%%%%%%%%%%%

{\bf CONTINGENCE}

\begin{itemize}[leftmargin=1cm, label=\ding{32}, itemsep=1pt]
\item {\footnotesize ÉTYMOLOGIE} : latin {\it contingere}, « arriver par hasard ».
\item {\footnotesize SENS ORDINAIRE} : possibilité qu'une chose
arrive ou non.
\item {\footnotesize LOGIQUE} : une des
quatre modalités de la logique classique, avec la nécessité, la possibilité et l'impossibilité ; est contingente une proposition qui n'est ni
vraie ni fausse, en tant qu'elle porte
sur quelque chose qui peut être
aussi bien que ne pas être.
\item {\footnotesize MÉTAPHYSIQUE} : ce qui n'a pas en soi sa
raison d'être ; en théologie, la
« preuve par la contingence » part
de la contingence du monde pour
montrer qu'il faut remonter jusqu’à
Dieu pour en trouver la cause ; pour
l'existentialisme athée, la contingence du monde est radicale, au
contraire, et place l'homme devant
l'absurde.
\end{itemize}

\vspace{0.35cm}
C’est Aristote qui, le premier, cherche à
penser la contingence. Il oppose la
science théorique, qui porte sur le nécessaire, à l’action pratique, qui porte sur
le contingent. Dans la mesure où l’action
vise une fin qui n'existe pas encore, elle
se rapporte à un futur, qui est lui-même
contingent. L'exemple qu'Aristote en
donne est celui d'une bataille navale.
« Nécessairement, il y aura demain une
bataille navale où il n'y en aura pas ;
mais il n'est pas nécessaire qu'il y ait
demain une bataille navale, pas plus
qu'il n'est nécessaire qu'il n'y en ait
pas. » La contingence est, en effet, l'objet
d’une délibération, d’un choix réfléchi et
renvoie à la liberté qui en est inséparable. Au contraire, la métaphysique
classique, avec Leibniz notamment,
définira la contingence négativement,
comme une simple limite à la connaissance. Si une chose nous apparaît
comme contingente, c'est parce que
nous en ignorons la cause, Pour Leibniz,
il existe un principe de raison suffisante
selon lequel chaque chose existe nécessairement, principe par lequel Dieu
choisit parmi tous les mondes possibles
le meilleur. Mais, alors, se pose le problème de la compatibilité d’un tel principe déterministe avec celui de la liberté
humaine. En posant la contingence radicale du monde, son absence de justification, Sartre et l’existentialisme athée
redonnent au contraire tout son sens à
l'idée de liberté,

\begin{itemize}[leftmargin=1cm, label=\ding{32}, itemsep=1pt]
\item {\footnotesize TERME VOISIN} : hasard.
\item {\footnotesize TERME OPPOSÉ} : nécessité.
\item {\footnotesize CORRÉLATS} : absurde ; existence ; existentialisme ; futur ; liberté.
\end{itemize}

%%%%%%%%%%%%%%%%%%%%%%%%%%%%%%%%%%%%%%%%%%%%%%%%%%%%%%%%%%%%%%%%%%%%%%%%%%%
