
%%%%%%%%%%%%%%%%%%%%%
\chapter{Autonomie}
%%%%%%%%%%%%%%%%%%%%%

({\it nf}\,)

\begin{itemize}[leftmargin=1cm, label=\ding{32}, itemsep=1pt]
\item {\footnotesize ÉTYMOLOGIE} : grec {\it autonomos} (de
{\it autos}, «lui-même», et {\it nomos},
« loi »), « qui se gouverne par ses
propres lois ».
\item {\footnotesize SENS ORDINAIRE} : capacité d’un individu ou d'un
groupe de déterminer lui-même le
mode d'organisation ou les règles
auxquelles il se soumet ; en ce sens,
on parlera, par exemple, de l'autonomie d’un élève, c'est-à-dire de sa
capacité d'organiser son travail sans
aide ni contraintes extérieures.
\item {\footnotesize SENS PHILOSOPHIQUE ET MORAL} : {\bf 1.} Chez Kant, caractère de la
volonté en tant qu'elle se soumet
librement à la loi morale édictée par
la raison pure pratique, par respect
de cette loi, et à l'exclusion de tout
autre mobile.
{\bf 2.} Liberté morale du
sujet qui agit conformément à ce
que lui dicte sa raison, et non par
simple obéissance à ses passions.
\end{itemize}


L’autonomie, chez Kant, peut se définir
comme liberté, au sens négatif, c'est-à-dire comme indépendance à l'égard de
toute contrainte extérieure, mais aussi et
surtout au sens positif comme législation
propre de la raison pure pratique.
L'autonomie de la volonté est, selon
Kant, « le principe suprême de la moralité » ({\it Fondement pour la métaphysique
des mœurs}). En effet, une action ne peut
être véritablement morale, si elle obéit à
des mobiles sensibles, extérieurs à la raison législatrice. Par exemple, si j'agis par
amour de l'humanité, je n’agis pas par
devoir, mais par sentiment. Or, une
action dont la maxime repose sur un
sentiment ne peut prétendre à l’universalité et servir de loi à tout être raisonnable. En revanche, et quel que soit
mon sentiment pour l'humanité, « traiter
l'humanité en ma personne et en la personne de tout autre toujours en même
temps comme une fin, et pas simplement comme un moyen » est une
maxime exigible universellement, un
devoir pour chacun; la volonté qui
détermine son action à partir d’elle est
une volonté autonome, en tant qu'elle se
soumet librement à la loi de la raison
pure pratique.

\begin{itemize}[leftmargin=1cm, label=\ding{32}, itemsep=1pt]
\item {\footnotesize TERME VOISIN} : liberté.
\item {\footnotesize TERME OPPOSÉ} : hétéronomie.
\item {\footnotesize CORRÉLATS} : devoir ; dignité ; liberté ; morale ; personne ; raison ; volonté.
\end{itemize}

%%%%%%%%%%%%%%%%%%%%%%%%%%%%%%%%%%%%%%%%%%%%%%%%%%%%%%%
