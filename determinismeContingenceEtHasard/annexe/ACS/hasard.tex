
%%%%%%%%%%%%%%%%%%%%% 269
\section{Hasard}
%%%%%%%%%%%%%%%%%%%%%
%{\it }
Ce n’est ni l’indétermination ni l’absence de cause. Quoi de plus
déterminé qu’un dé qui roule sur une table ? Le {\it six} sort ? C’est là
un effet, qui résulte de causes très nombreuses (le geste de la main, l’attraction
terrestre, la résistance de l’air, la forme du dé, sa masse, son angle de contact
avec la nappe, ses frottements contre elle, ses rebonds, son inertie...). Si lon
juge pourtant légitimement que le {\it six} est sorti {\it par hasard}, c’est que ces causes
sont trop nombreuses et trop indépendantes de notre volonté pour qu’on
puisse, lorsqu’on jette le dé, choisir ou prévoir le résultat qu’on obtiendra. Ainsi
le hasard est une détermination imprévisible et involontaire, qui résulte de la
% 270
rencontre de plusieurs séries causales indépendantes les unes des autres, comme
disait Cournot, rencontre qui échappe pour cela à tout contrôle comme à toute
intention. Ce n’est pas le contraire du déterminisme : c’est le contraire de la
liberté, de la finalité ou de la providence.

Un autre exemple ? On peut reprendre celui de Spinoza, dans l’Appendice
de la première partie de l’{\it Éthique}. Une tuile tombe d’un toit. Il y a à cela des
causes (le poids de la tuile, la pente du toit, le vent qui soufflait, un clou rongé
par la rouille, qui finit par céder), dont chacune s’explique à son tour par une
ou plusieurs autres, et ainsi à l’infini. Vous étiez, à ce moment précis, sur le
trottoir, juste à la verticale du toit. Cela s'explique aussi, ou peut s'expliquer,
par un certain nombre de causes : vous alliez à un rendez-vous, vous aviez
choisi l'itinéraire le plus simple, vous pensiez que la marche à pied vous ferait
du bien... Ni la chute de la tuile ni votre présence sur le trottoir ne sont donc
sans causes. Mais les deux séries causales (celle qui fait tomber la tuile, celle qui
vous amène où vous êtes), outre leur complexité propre, qui suffirait à les
rendre hasardeuses, sont indépendantes l’une de l’autre : ce n’est pas parce que
la tuile tombe que vous êtes là, ni parce que vous êtes là qu’elle tombe. Si elle
vous brise le crâne, vous serez donc bien mort {\it par hasard} : non parce qu'il y
aurait à une exception au principe de causalité, mais parce que celui-ci s’est
exercé de façon irréductiblement multiple, imprévisible et aveugle. Ou bien il
faut imaginer un Dieu qui aurait voulu ou prévu la rencontre de la tuile et de
votre crâne. La providence est un anti-hasard, et le hasard une anti-religion.

Le hasard se calcule, mais dans sa masse plutôt que dans son détail. C’est ce
qui permet aux assureurs de mesurer les risques que nous prenons, et qu’ils
prennent : un accident de voiture, aussi imprévisible qu’il puisse être, fait partie
d’une série (le nombre d’accidents dans une période donnée) qui peut se prévoir
à peu près. C’est vrai aussi pour les jeux de hasard. S’il est impossible, sauf
trucage, de prévoir le résultat d’un seul coup de dé, il est facile de calculer la
répartition statistique de coups très nombreux : chacune des six possibilités se
vérifiera, si vous jouez assez longtemps, environ une fois sur six, et s’approchera
d'autant plus de cette moyenne que la série sera plus longue. C’est pourquoi la
chance ne dure pas toujours, ni la malchance, du moins dans les phénomènes
qui ne dépendent que du hasard. Simplement la vie ne dure pas assez longtemps,
et est soumise à des causes trop lourdes et trop constantes, pour que le
hasard vienne toujours corriger l’injustice.

Toute vie n’en est pas moins hasardeuse, dans son détail comme dans son
principe. La naissance de chacun d’entre nous, quelques années avant notre
conception, était d’une probabilité extrêmement faible ; comme c’est vrai aussi
des naissances de nos parents, de nos grands-parents, etc., qui conditionnent la
nôtre, il en résulte que notre existence, il y a quelques siècles, était d’une probabilité
% 271
quasi nulle, comme celle, si l’on prend assez de recul, de tout événement
contingent. C’est en quoi tout réel, aussi banal qu’il soit, a quelque chose
de rétrospectivement surprenant, qui tient au fait qu’il était non seulement
imprévisible à l'avance mais hautement improbable : c’est l'exception du possible.
L'univers fait une espèce de loterie, dont le présent serait le gros lot.
D’aucuns s’étonnent que ce soit justement ce numéro-là qui soit sorti, alors
que c'était tellement improbable... Mais qu'aucun numéro ne sorte, une fois la
loterie lancée, l’était bien davantage.

%%%%%%%%%%%%%%%%%%%%%%%%%%%%%%%%%%%%%%%%%%%%%%%%%%%%%%%%%%%%%%%%%%%%%%%%%%%%%%%%%%%%%
