
%%%%%%%%%%%%%%%%%%%%% 163
\section{Déterminisme}
%%%%%%%%%%%%%%%%%%%%%
%{\it }
Doctrine selon laquelle tout est déterminé, c’est-à-dire
soumis à des conditions nécessaires et suffisantes, qui
sont elles-mêmes déterminées. Le déterminisme, en ce sens, n’est qu’une généralisation
du principe de causalité. C’est une chaîne de causes, ou plusieurs, ou
l’ensemble de ces chaînes, à quoi rien n’échappe, ni lui-même : on peut agir sur
lui, le changer, le maîtriser, mais point en sortir. C’est le labyrinthe des causes,
ou plutôt des effets. Kant a bien vu qu’il excluait à la fois la contingence et la
fatalité ({\it C. R. Pure}, Analytique des principes, PUF, p. 208, Pléiade, p. 960). La
multiplicité des causes explique tout, mais n’impose rien.

Le déterminisme n’est qu’un autre nom pour le hasard (comme pluralité
des séries causales), en tant qu’il est connaissable. On ne le confondra pas avec
le prédéterminisme, qui suppose qu’il existe une chaîne {\it unique et continue} de
causes, de telle sorte que l’avenir serait tout entier inscrit dans le présent,
% 164
comme le présent résulterait nécessairement du passé. C’est donner au temps
une efficace qu’il n’a pas. Ni avec l’idée d’une prévision possible : un phénomène
peut être intégralement déterminé tout en restant parfaitement imprévisible
(c’est le principe des jeux de hasard et des systèmes chaotiques). Le temps
qu'il fera dans six mois n’est écrit nulle part : il n’est pas {\it déjà déterminé} ; mais
il le sera dans six mois. Ainsi le déterminisme n’est pas un fatalisme : il n’exclut
ni le hasard ni l'efficacité de l’action. Il permet au contraire de les penser. De là
la météorologie et le parapluie.

%%%%%%%%%%%%%%%%%%%%%%%%%%%%%%%%%%%%%%%%%%%%%%%%%%%%%%%%%%%%%%%%%%%%%%%%%%%%%%%%%%%%%
