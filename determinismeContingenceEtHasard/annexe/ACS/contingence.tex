
%%%%%%%%%%%%%%%%%%%%% 36
\section{Contingence}
%%%%%%%%%%%%%%%%%%%%%
%{\it }
On la définit ordinairement comme le contraire de la nécessité :
est contingent, explique Leibniz, tout ce dont le
contraire est possible, autrement dit tout ce qui {\it pourrait} ou {\it aurait pu} ne pas
être. Ces conditionnels sont à prendre en considération. Car quelle {\it condition} supposent-ils ?
Que le réel ne soit pas ce qu’il est. C’est en quoi tout, dans le temps,
est contingent (le néant était possible aussi, ou un autre réel), aussi sûrement que
tout, au présent, est nécessaire (ce qui est ne peut pas ne pas être pendant qu’il
est). Si le temps et le présent sont une seule et même chose, comme je le crois, il
faut en conclure que {\it contingence} et {\it nécessité} ne s'opposent que pour l’imagination :
quand on compare ce qui est, fut ou sera (le réel), à autre chose, qui
pourrait ou aurait pu être (le possible, en tant qu’il {\it n'est pas} réel). Au présent, ou
{\it sub specie aeternitatis}, seul le réel est possible : tout le contingent est nécessaire,
tout le nécessaire est contingent. C’est où Spinoza et Lucrèce se rejoignent.

%%%%%%%%%%%%%%%%%%%%%%%%%%%%%%%%%%%%%%%%%%%%%%%%%%%%%%%%%%%%%%%%%%%%%%%%%%%%%%%%%%%%%
