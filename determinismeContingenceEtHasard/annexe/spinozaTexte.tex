
%%%%%%%%%%%%%%%%%%%%%
\chapter{Spinoza, texte}
%%%%%%%%%%%%%%%%%%%%%

%%%%%%%%%%%%%%%%%%%%%%%%%
\section{La liberté d'expression}
%%%%%%%%%%%%%%%%%%%%%%%%%
%


{\it Dans son Traité théologico-politique, Spinoza pose avec fermeté les fondements
d’un État démocratique et laïque. Cette laïcité ne doit pourtant pas être comprise
comme la négation, le refus de reconnaissance des croyances religieuses. À partir
de l'acceptation, qui doit être partagée par tous, des valeurs fondamentales de
charité et de justice, l'État laissera libre cours (tout en se réservant un droit de
contrôle) aux opinions particulières, aux dogmes et aux pratiques propres aux
différentes confessions. C’est là un véritable traité de la tolérance, où l'adhésion à
l’État s'obtient à travers la pluralité reconnue des convictions.}

« [...] si personne ne peut abandonner sa liberté de juger et de penser ce qu'il veut,
si par contre chacun est maître de ses propres réflexions, par un droit supérieur de
Nature, il suit que dans un État, jamais on ne pourra tenter, si ce n’est avec une issue
totalement désastreuse, de faire que les hommes, d'opinions si diverses et
contraires, ne s'expriment cependant que selon le décret du pouvoir souverain [...].
En effet comme le libre jugement des hommes est tout à fait divers et que chacun
pense à lui seul tout savoir, et qu'il est impossible que tous pensent également la
même chose, et parlent d'une seule voix, ils ne pourraient vivre en paix si chacun
n'avait pas renoncé au droit d'agir selon le seul décret de sa pensée. C'est donc seulement au droit d'agir selon son propre décret que l'individu a renoncé, non au
droit de raisonner et de juger ; par suite personne ne peut, sans danger pour le droit
du pouvoir souverain, agir à l'encontre du décret de celui-ci, mais il peut totalement penser et juger, et par conséquent aussi s'exprimer, à condition cependant
qu'il se contente de parler et d'enseigner, et de défendre son opinion par la seule
Raison, sans introduire par la ruse, la colère et la haine, quelque mesure contraire à
l'État qui ne ressortirait que de l'autorité de son propre vouloir. »

\vspace{0.25cm}
Spinoza, {\it Traité théologico-politique} (1670), trad. M. Pardo,
Paris, Hatier, coll. « Les classiques Hatier de la philosophie », 1999, pp. 24 et 25-26.

\subsection{La liberté d'expression}

Pour Spinoza le terrorisme idéologique de l'État, sa prétention à réglementer ce
qu'il faut ou ce qu'il ne faut pas penser, est à la fois injuste, dangereux et inutile.
Injuste : il remet en cause le droit des gens à n’adopter des opinions qu'après un
libre examen. Dangereux : l'État devient forcément, par cet abus de pouvoir,
l’objet d’une haine populaire qui peut à tout moment se déchaîner contre lui.
Inutile : on peut bien contrôler les déclarations des individus, mais en aucun cas
les pensées qu'ils forment en leur for intérieur.

\subsection{Le devoir d’obéissance}

Mais si l’État reconnaît le droit de chacun de dire et de penser ce qu'il veut (dans
certaines limites que l’auteur s’attachera dans la suite du texte à énoncer), cela ne
signifie pas pour autant qu'il reconnaît le droit de chacun à agir comme il l'entend.
La diversité des opinions est en effet telle qu'elle serait suivie, si l'on ne disposait
d’une instance supérieure de décision, d’une égale incohérence et anarchie dans
l’action. L'État, en même temps qu'il doit garantir à chacun le droit de dire ce qu'il
pense des lois, doit s'assurer qu'il y obéit bien : c’est la juste mesure de sa puissance.

\section{Liberté et nécessité}

{\it Je sens que je veux une chose, et il arrive que je ne parvienne même pas à expliquer
pourquoi je la veux. Je fais certains choix dans ma vie, et j'arrive difficilement à en
rendre compte ensuite. Dans certaines situations, j'hésite longtemps avant de
prendre une décision. C’est dans cette absence de déterminations objectives de ma
volonté que la philosophie à souvent placé l'expérience irréfutable de ma liberté.
En ce sens, la liberté humaine s’opposait à la nécessité des lois de la nature : alors
que chaque mouvement dans la nature est strictement réglé par des lois immuables,
l’homme, lui, a le pouvoir de se décider à partir de rien, de commettre des actes
totalement gratuits. Grandeur de la liberté humaine ; Spinoza va s'attacher
précisément à montrer combien elle se révèle illusoire à l'analyse.}

« J'appelle libre, quant à moi, une chose qui est et agit par la seule nécessité de sa
nature ; contrainte, celle qui est déterminée par une autre à exister et à agir d'une
certaine façon déterminée. [...] Pour rendre cela clair et intelligible, concevons une
chose très simple : une pierre, par exemple, reçoit d'une cause extérieure qui la
pousse, une certaine quantité de mouvement et, l'impulsion de la cause extérieure
venant à cesser, elle continuera à se mouvoir nécessairement. Cette persistance de
la pierre dans le mouvement est une contrainte, non parce qu'elle est nécessaire,
mais parce qu'elle doit être définie par l'impulsion d’une cause extérieure. [...]
Concevez maintenant, si vous voulez bien, que la pierre, tandis qu’elle continue de
se mouvoir, pense et sache qu'elle fait effort, autant qu'elle peut, pour se mouvoir.
Cette pierre assurément, puisqu'elle a conscience de son effort seulement et qu’elle
n'est en aucune façon indifférente, croira qu'elle est très libre et qu’elle ne persévère dans son mouvement que parce qu’elle le veut. Telle est cette liberté humaine
que tous se vantent de posséder et qui consiste en cela seul que les hommes ont
conscience de leurs appétits et ignorent les causes qui les déterminent. Un enfant
croit librement appéter le lait, un jeune garçon irrité vouloir se venger et, s’il est poltron, vouloir fuir. Un ivrogne croit dire par un libre décret de son âme ce qu'ensuite,
revenu à la sobriété, il aurait voulu taire. De même un délirant, un bavard, et bien
d'autres de même farine, croient agir par un libre décret de leur âme et non se laisser contraindre. »

\vspace{0.25cm}
Spinoza, {\it Lettre 58}, trad. C. Appuhn, Paris, GF-Flammarion, 1966, pp. 303-304.

\subsection{Liberté et contrainte}

Pour Spinoza, le contraire de la liberté n'est pas la nécessité, mais la contrainte.
La contrainte renvoie aux déterminations extérieures qui m'obligent à une action,
et la liberté à celles qui dépendent uniquement de ma nature propre. Il ne faut
donc pas dire qu'un acte libre est un acte sans raisons, mais un acte dont les
raisons se tirent de ma seule personne, sans avoir à recourir à d’autres causes.
On n'opposera plus liberté et nécessité, mais nécessité externe (contrainte) et
nécessité interne (liberté).

\subsection{L’ignorance des causes}

La prétendue liberté de volonté est donc strictement imaginaire. L'absence de
motifs objectifs pour une action ne renvoie pas en effet à la présence positive de
la liberté en nous, mais à l'ignorance des causes qui nous déterminent : on a la
conscience de son désir sans le savoir de ce désir. Ce n’est pourtant pas parce
qu'on ne connaît pas les raisons de ses actes qu’on agit réellement sans raisons,
mais c’est bien pour cela qu'on a l'illusion d'agir librement.

\subsection{La critique de la philosophie cartésienne}

Spinoza peut apparaître comme un disciple de Descartes puisque le seul ouvrage
qu’il publie sous son nom expose les {\it Principes de la philosophie de Descartes}
(1663), qu'il démontre « selon la méthode géométrique ». En fait, les concepts
cartésiens : la règle de l’évidence, la rigueur mathématique, ne constituent que la
forme sous laquelle Spinoza expose, dès le {\it Traité de la réforme de l’entendement}
(1661), une philosophie différente de celle de Descartes. Contre celui-ci, Spinoza
affirme qu'il n’est pas nécessaire de passer par le doute systématique pour trouver
la vérité, car le vrai est « signe de lui-même et du faux ». L'erreur n’est pas le
résultat d’une intervention positive de la volonté. Elle n’est qu’« une privation de
connaissance qu'enveloppent les idées inadéquates, c'est-à-dire mutilées et
confuses ». Il faut donc purifier l’entendement de ses idées confuses en procédant
à la critique de l'imagination. Pour ce faire, doit être constituée une science de la
nature humaine qui dépasse le dualisme cartésien de l’âme et du corps. En effet
ce dualisme rend incompréhensible l’action de l’une sur l’autre, et confus le fondement de la morale.
%%%%%%%%%%%%%%%%%%%%%%%%%%%%%%%%%%%%%%%%%%%%%%%%%%%%%%%
