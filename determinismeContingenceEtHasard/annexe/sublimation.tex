
%%%%%%%%%%%%%%%%%%%%%
\chapter{Sublimation}
%%%%%%%%%%%%%%%%%%%%%

\begin{itemize}[leftmargin=1cm, label=\ding{32}, itemsep=1pt]
\item {\footnotesize ÉTYMOLOGIE} : latin {\it sublimatio},
« action d'élever ».
\item {\footnotesize SENS ORDINAIRE} : action de transformer, d'élever,
de purifier, ou encore de dériver
une tendance vers un but spirituel
ou  altruiste.
\item {\footnotesize PSYCHANALYSE} : 
transformation de pulsions ou de
sentiments inacceptables en désirs
orientés vers des buts socialement
valorisés (notamment esthétiques
ou religieux).
\end{itemize}

La notion de sublimation renvoie à la
fois au sublime, terme qui désigne, en
esthétique notamment, un sentiment
particulièrement élevé et délicat ; et à la
sublimation au sens chimique : passage
d'un corps de l'état solide à l'état
gazeux. La sublimation est un processus
quasi chimique (ou magique) de spiritualisation de sentiments qui
deviennent ainsi moralement acceptables. Selon Freud, elle est la capacité
qu'ont certains hommes de dévier leurs
pulsions sexuelles vers des buts sans
rapport avec la sexualité, tels que l’activité artistique ou l’investigation intellectuelle. Le cas de Léonard de Vinci, qui
aurait surmonté son homosexualité par
le biais de ses travaux artistiques et
scientifiques, est souvent cité (Freud, Un
Souvenir d'enfance de Léonard de
Vinci). Plus banalement, la fonction
d'une cure psychanalytique pourrait être
de renforcer en chacun de nous cette
aptitude à la sublimation : mais on peut
se demander si tous les hommes bénéficient de cette heureuse disposition, si
toutes les activités socialement valorisées peuvent constituer des dérivatifs
satisfaisants, ou si seules les activités les
plus « nobles » peuvent véritablement
remplir cette fonction.


\begin{itemize}[leftmargin=1cm, label=\ding{32}, itemsep=1pt]
\item {\footnotesize CORRÉLATS} : Catharsis ; psychanalyse ; pulsions ; refoulement ; sublime.
\end{itemize}

%%%%%%%%%%%%%%%%%%%%%%%%%%%%%%%%%%%%%%%%%%%%%%%%%%%%%%%
