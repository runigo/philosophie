
%%%%%%%%%%%%%%%%%%%%%
\chapter{Spinoza, pratique de la philosophie}
%%%%%%%%%%%%%%%%%%%%%

%%%%%%%%%%%%%%%%%%%%%%%%%
\section{Pratique de la philosophie}
%%%%%%%%%%%%%%%%%%%%%%%%%

\subsection{Repères biographiques}

SPINOZA BARUCH\ \ \ (1632-1677)

\vspace{0.25cm}
Né dans une famille de juifs portugais émigrés en Hollande au {\footnotesize XVI}$^\text{e}$
siècle, il suit les cours de l’école talmudique d'Amsterdam et continue
son instruction librement, tout en travaillant dans le commerce de son
père. Il fréquente les milieux des
chrétiens libéraux et des libres penseurs. En 1656, il est excommunié
par la Synagogue et doit gagner sa
vie en polissant des verres de
lunettes d'approche. Très attaqué
depuis 1670 par les milieux religieux
à cause de la parution de son {\it Traité
théologico-politique}, il vit en reclus,
entouré de quelques amis, après la
chute violente de la République en
1672. En 1678, ses œuvres posthumes sont interdites « en tant que
profanes, athées et  blasphématoires »,

\subsection{Une philosophie du salut ou de la béatitude : l’Éthique}

Apprendre à penser doit nous permettre
de trouver le Souverain Bien, un « bien
véritable qui puisse se communiquer » et
donner le suprême contentement ou
« béatitude » : ce bien, c’est la vie selon
la raison, qui nous « sauve » du trouble
des passions. Cette finalité morale de la
recherche spinoziste est exposée dès le
début du Traité de la réforme de l'entendement et donnée comme le fil directeur
du livre central de Spinoza : l'{\it Éthique}
(1661-1675) — qu'il renoncera à publier
pour des raisons de sécurité.

{\bf 1.} La méthode géométrique. Spinoza
met à l'épreuve les bases de sa méthode
jetées dans le {\it Traité de la réforme de
l’entendement} : partir de « l’idée de l'être
le plus parfait » pour en faire découler
toutes les idées et toute la pratique
humaine, L'{\it Éthique} expose la morale
réflexive la plus haute qui découle des
principes métaphysiques que Spinoza
pose par définition, comme le ferait un
mathématicien. Ainsi chaque partie du
livre part de définitions, d’axiomes, de
propositions suivies de leurs démonstrations et qu'il développe dans des scolies
et des appendices. Les cinq parties
traitent successivement de: « Dieu » ;
« Nature et origine de l'esprit » ; « Nature
et origine des sentiments » : « La servitude humaine ou les forces des sentiments » ; « La puissance de l’entendement ou la liberté humaine ».

{\bf 2.} Les principes métaphysiques. Spinoza établit que, par définition, il
n'existe d'autre substance dans la
nature que Dieu, dont tout est mode ou
attribut. De Dieu, « c'est-à-dire la
nature », découle par la seule nécessité
naturelle tout ce qui existe (les modes
de l'être). Tous les événements sont soumis à un strict déterminisme. Il n'y a pas
de causes finales. Le libre arbitre est
une illusion. Il y a une parfaite adéquation (ou corrélation) entre les choses
(dont l'attribut est l'« étendue ») et les
idées (dont l’attribut est « la pensée »).

{\bf 3.} Les trois genres de connaissance et la
maîtrise des passions. L'homme du
commun se satisfait d’un prétendu
savoir acquis par « ouï-dire » et « expérience vague », perceptive, non critiquée : c’est « le premier genre de
connaissance » qui se manifeste par un
comportement passionnel, en particulier
superstitieux et fanatique, car « le désir
est l'essence de l’homme ». Celui qui
commence à s'instruire, apprend à définir et à déduire selon la méthode mathématique, rationnelle. Il peut alors s’affranchir des passions en en formant des
idées claires et distinctes. Il vit selon « le
deuxième genre de connaissance ».
Celui qui atteint « le troisième genre de
connaissance », ou connaissance intuitive, voit chaque chose, chaque événement, comme découlant de la nature
divine, c'est-à-dire de la nécessité naturelle, Il connaît toute chose à la fois dans
sa singularité et son lien avec la totalité.
Il n’est plus hanté par la crainte. Il
éprouve au contraire la joie la plus
haute.

{\bf 4.} La fausse morale et le vrai bien. Spinoza analyse, dans la quatrième partie
de l’{\it Éthique}, tous les faux sentiments,
les faux comportements comme la
crainte, la honte, la tristesse, qui n’engendrent dans la vie sociale qu'une
fausse concorde. Il leur oppose les vrais
sentiments fondés sur les idées positives
de l'entendement : la joie, l'amour pour
les idées vraies, pour son prochain en
tant qu'il est dirigé par la raison. Ainsi le
vrai bien repose sur l'extension de la
puissance de connaître. L'homme
découvre qu'il n'y a rien de plus utile
qu'un autre homme qui vit selon la raison, dans une cité raisonnable, et que
partager la vraie connaissance permet de
jouir de la vie en écartant les idées tristes
de la haine, de la vengeance et de la
mort.

Après avoir écarté l'illusion de la liberté,
Spinoza, dans la cinquième partie de
l'Éthique, lui redonne son vrai statut. Le
philosophe accède à la vraie liberté et sa
béatitude est la vertu elle-même. Il
expérimente alors « l'amour intellectuel
de Dieu » c’est-à-dire sa propre éternité.

\subsection{Une philosophie de la politique}

Si le Souverain Bien doit pouvoir se
communiquer, le philosophe doit chercher les conditions politiques de sa réalisation. Spinoza prend parti dans les
débats des milieux républicains sur la
tolérance. Le {\it Traité théologico-politique}, publié anonymement en 1670 et
aussitôt attaqué par les autorités religieuses (juives et calvinistes), est un
ouvrage polémique, véritable apologie
pour la tolérance et l'indépendance des
pouvoirs religieux et politiques. La
liberté de philosopher doit être totale
pour que règne la paix dans un État.
Celui-ci doit être le garant d’une foi universelle, fondée avant tout sur les règles
morales de la justice et de la charité.

Le {\it Traité politique} (1675-1677), inachevé, est, lui, un exposé synthétique
sur les fondements de l’État. Spinoza
définit la fin de la société civile par la
paix et la sécurité. Son interrogation permanente est: « Comment contenir la
multitude ? », question suscitée par la
violence de la « révolution » orangiste
qui mit fin à la République en 1672.
Après avoir défini le droit par la puissance, il pose le problème général de la
conservation des régimes politiques.
Spinoza examine la façon dont ce problème peut être résolu pour chacun des
trois régimes typiques : monarchie, aristocratie, démocratie. La question reste
en suspens en ce qui concerne la démocratie, l'ouvrage étant resté inachevé.
Cependant la démocratie est définie
comme le régime « le meilleur », car « le
plus naturel » et « le plus rationnel.
Mais la question demeure : dans quelle
mesure la multitude peut-elle gouverner
ses propres passions ?

\begin{itemize}[leftmargin=1cm, label=\ding{32}, itemsep=1pt]
\item {\footnotesize PRINCIPAUX  ÉCRIS} : {\it Éthique} (1661-1675);  {\it Traité de la réforme de l'entendement} (1661);   {\it Traité théologico-politique} (1670) ;   {\it Traité politique} (1677);  {\it Correspondance} (1661-1676).
\end{itemize}

%%%%%%%%%%%%%%%%%%%%%%%%%%%%%%%%%%%%%%%%%%%%%%%%%%%%%%%%%%%%%%%%%%%%%%%%%%%%%%%%%%%%%
