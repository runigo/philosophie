
%%%%%%%%%%%%%%%%%%%%%
\chapter{Cournot, pratique de la philosophie}
%%%%%%%%%%%%%%%%%%%%%

%%%%%%%%%%%%%%%%%%%%%%%%%
\section{Pratique de la philosophie}
%%%%%%%%%%%%%%%%%%%%%%%%%

\subsection{Repères biographiques}

COURNOT ANTOINE AUGUSTIN\ \ \ (1801-1877)

\vspace{0.25cm}
Après avoir été professeur de mathématiques, puis inspecteur général,
Cournot devient recteur de l’Académie de Dijon. Ses premières œuvres
portent sur les mathématiques. Par la
suite, il est amené à étendre ses
thèses sur les probabilités à l’ensemble des connaissances humaines.

\subsection {Un rationalisme prudent}

Les premiers travaux de Cournot portent
sur les principes mathématiques de
l'économie et sur la question des probabilités. Plus généralement, il est
amené à s'interroger sur la rationalité à
l'œuvre dans les sciences, en partant du
principe que science et philosophie
doivent s’allier et s'épauler mais non pas
se confondre (cas du scientisme). Sans
la science, en effet, la philosophie
« s'égarerait dans des espaces imaginaires », tandis que la science sans la
philosophie s’épuiserait dans une
rigueur stérile. Au rationalisme triomphant du {\footnotesize XVII}$^\text{e}$ siècle, Cournot oppose
donc la démarche prudente d’une raison vouée à la recherche et résignée à
la modestie. Aussi est-il amené à considérer qu'un monde ordonné n'exclut ni
le hasard ni l’insignifiance.

\subsection {La théorie du hasard}

Laplace pensait qu'une intelligence qui
« connaîtrait toutes les forces dont la
nature est animée » ainsi que les dispositions de tous ses éléments, serait en
mesure de prévoir avec certitude le
devenir entier de l'univers ({\it Essai philosophique sur les probabilités}, 1814).
Cette idée, aussi séduisante soit-elle, est
cependant fausse au yeux de Cournot.
Car la rationalité du monde laisse une
large place à des faits sans raison, qu’on
appellera « aléatoires » ou « fortuits » (ils
auraient pu ne pas se produire). Un tel
constat n'implique absolument pas,
cependant, une démission de la raison :
le hasard, en effet, peut être rigoureusement traité. Car aucun événement —
et Cournot rejoint sur ce point le déterminisme classique — n'est un commencement absolu. Mais les séries causales
(enchaînements linéaires et prévisibles
de causes et d'effets) sont le plus souvent indépendantes les unes des autres,
et ce sont leurs rencontres qui sont aléatoires. Les événements fortuits sont donc
sans raison (sans nécessité) et non pas
sans causes. Le hasard n'est donc pas
dû à notre ignorance : notion positive, il
désigne un « concours de causes indépendantes » qui peut être abordé scientifiquement. Ainsi, selon Cournot, le réel
n'est ni décousu, ni absurde: il est
ordonné globalement et non pas gouverné dans ses moindres détails par une
rationalité implacable. La raison, à
l'œuvre dans l’histoire comme dans les
sciences, doit cesser d’être despotique
pour devenir régulatrice et critique.

\begin{itemize}[leftmargin=1cm, label=\ding{32}, itemsep=1pt]
\item {\footnotesize PRINCIPAUX ÉCRITS} : {\it Exposition de
la théorie des chances et des probabilités} (1843) ; {\it Essai sur les fondements de nos connaissances et
sur les caractères de la critique philosophique} (1851) : {\it Les Institutions
d'instruction publique en France}
(1864): {\it Matérialisme,  vitalisme,
rationalisme} (1875).
\end{itemize}

%%%%%%%%%%%%%%%%%%%%%%%%%%%%%%%%%%%%%%%%%%%%%%%%%%%%%%%
