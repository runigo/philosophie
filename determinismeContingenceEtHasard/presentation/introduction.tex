
\thispagestyle{empty}

\begin{center}
\Large
%Introduction
Préambule
\normalsize
\end{center}
\vspace{3cm}

Ce document est une compilation d'articles provenant de trois ouvrages : un dictionnaire encyclopédique de poche, {\it La pratique de la philosophie} destiné aux lycéens, et une encyclopédie de la philosophie destinée aux néophytes. 

\vspace{1.3cm}

Chaque chapitre contient les articles correspondant à une notion particulière. Ces notions ont été choisies en raison de leurs liens avec la question du hasard. Ces choix ont été guidés : {\bf 1.} Par les renvois vers d'autres articles présent dans les ouvrages. {\bf 2.} Mes propres choix, liés à mes connaissances. {\bf 3.} La volonté d'obtenir une quantité raisonnable d'information.

\vspace{1.3cm}

Dans {\it La pratique de la philosophie}, l'article concernant la nécessité renvoit à des textes de Spinoza dont le choix reste subjectif à l'ouvrage. J'ai néanmoins reproduit en annexe ces textes ainsi que l'article concernant Spinoza. (les autres annexes sont d'autres renvois de cet ouvrage)

Les articles compilés dans ce document comportent donc les choix "discutables" réalisés dans les trois ouvrages utilisés. Il s'agit donc d'un document de "travail" destiné à apporter quelques points de vues philosophiques de manière relativement élémentaire.

\vspace{1.3cm}

Les trois premiers chapitres abordent les thèmes du hasard, de la nécessité puis du vitalisme. Les chapitres suivants élargissent le champ de vision philosophique en abordant les thèmes du déterminisme, de la contingence et de la providence.

\vspace{2.3cm}

\hfill Stephan Runigo

%%%%%%%%%%%%%%%%%%%%%%%%%%%%%%%%%%%%%%%%%%%%%%%%
