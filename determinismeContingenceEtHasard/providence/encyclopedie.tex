
%%%%%%%%%%%%%%%%%%%%%
\section{Encyclopédie de la philosophie}
%%%%%%%%%%%%%%%%%%%%%
%
%
{\bf Providence} : terme désignant l’action par laquelle Dieu gouverne l’histoire et les
événements naturels. Fondamentalement
absente de la religion grecque (car le sort
du monde dépend en dernière analyse du
destin compris comme puissance supérieure aux dieux eux-mêmes), la notion de
providence (en gr. {\it pronoia}) fait son apparition dans la philosophie grecque avec
Socrate, qui l’identifie avec l’activité finaliste de Dieu, et avec Platon, qui la fait
coïncider avec l’organisation du cosmos
opérée par le Démiurge. Aristote nie
l'existence de la providence : le premier
moteur non mû ne peut être conditionné
par la connaissance des réalités finies,
c’est pourquoi il attire à soi tous les étants
de manière nécessaire. L'existence d’une
providence ordonnant le cosmos est affirmée par le stoïcisme, qui la conçoit
comme rationalité et finalité immanente à
la matière, et donc l’identifie avec la solidarité qui relie nécessairement tous les
êtres (« {\it fatum} » et « destin »). La foi en
un Dieu providentiel est l'élément constitutif de l'Ancien Testament : non seulement Dieu connaît, dispose et dirige tout
événement terrestre (dans la formulation
dogmatique chrétienne postérieure, c’est
là la « providence commune »), mais il
entre dans les aventures humaines au
point d'instaurer un pacte de fidélité réciproque, l’alliance, avec le peuple d'Israël
(ce cas d'élection particulière fait partie
de la « providence spéciale »). Dans la
doctrine chrétienne, la distinction entre
« providence naturelle » (l’action divine
qui s’accomplit à travers des agents naturels) et « providence surnaturelle » (l’action de Dieu lorsqu'elle s’exerce dans des
formes extraordinaires) est tout aussi courante. Dans l’histoire de la pensée, la
principale objection à l’affirmation d’une
providence transcendante est venue de
l'existence du mal dans le monde. La
réponse  apologétique élaborée par
Boèce, et reprise par Gottfried W. Leibniz, se fonde sur la considération selon
laquelle le mal n’est pas effectivement le
mal, dans la mesure où la puissance de
Dieu sait tirer le bien même de ce qui se
présente comme mal.

%\vspace{0.35cm}
%$\to$ alliance


%%%%%%%%%%%%%%%%%%%%%%%%%%%%%%%%%%%%%%%%%%%%%%%%%%%%%%%%%%%%%%%%%%%%%%%%%
