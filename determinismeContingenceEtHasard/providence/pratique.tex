
%%%%%%%%%%%%%%%%%%%%%
\section{Pratique de la philosophie}
%%%%%%%%%%%%%%%%%%%%% ({\it nf\;}) 

{\bf PROVIDENCE}

\begin{itemize}[leftmargin=1cm, label=\ding{32}, itemsep=1pt]
\item {\footnotesize ÉTYMOLOGIE} : latin {\it pro}, « avant », {\it videre}, « voir ».
\item {\footnotesize MÉTAPHYSIQUE ET THÉOLOGIE} : attribut divin, grâce auquel Dieu guide le cours des événements en fonction de la fin (du but) qu'il leur assigne.
\end{itemize}

On peut distinguer, avec Malebranche,
une Providence générale : l’ordre général et harmonieux du monde, fixé une
fois pour toutes : il s’agit des lois de la
nature, conçues comme l'expression de
la perfection du plan divin ; et une Providence particulière : l'intervention personnelle de Dieu dans le cours des événements, par des miracles, pour
remédier à certains désordres.

\begin{itemize}[leftmargin=1cm, label=\ding{32}, itemsep=1pt]
\item {\footnotesize TERMES VOISINS} : dessein de Dieu ; destin.
\item {\footnotesize TERMES OPPOSÉS} : contingence ; hasard.
\item {\footnotesize CORRÉLATS} : christianisme ; Dieu ; finalisme ; loi (de la nature) ; miracle.
\end{itemize}

%%%%%%%%%%%%%%%%%%%%%%%%%%%%%%%%%%%%%%%%%%%%%%%%%%%%%%%%%%%%%%%%%%%%%%%%%%%
