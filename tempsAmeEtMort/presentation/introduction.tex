
\thispagestyle{empty}

\begin{center}
\Large
%Introduction
Préambule
\normalsize
\end{center}
\vspace{3cm}

%\label{hachette}
%\label{garzanti}
%\label{hansen}
%\label{comte}
%\cite{dictionnaireHachette}
%\cite{philosophiePratique}
%\cite{encyclopediePhilosophie}
%\cite{dictionnairePhilosophique}. 

Ce document est une compilation d'articles provenant de quatre ouvrages : un dictionnaire encyclopédique de poche, {\it La pratique de la philosophie} destiné aux lycéens, une encyclopédie de la philosophie destinée aux néophytes, et le dictionnaire philosophique d'André Comte-Sponville.

\vspace{1.3cm}

%Chaque chapitre contient les articles des trois premiers ouvrages correspondant à une notion particulière. Ces notions ont été choisies en raison de leurs liens avec la question de l'âme.
les chapitres contiennent les articles des trois premiers ouvrages, les articles du dictionnaire de philosophie d'André Comte-Sponville sont reproduits en annexe.

Chacun des chapitre traite une notion particulière choisie en raison de ses liens avec la question de l'âme,
l'article de l'encyclopédie de la philosophie concernant la notion d'{\it Aufhebung} est reproduit en annexe.


% correspondant à une notion particulière. Ces notions ont été choisies en raison de leurs liens avec la question de l'âme.
%Les choix ont été guidés : 1. Par les renvois vers d'autres articles présent dans les ouvrages. 2. Mes propres choix, liés à ma subjectivité. 3. La volonté d'obtenir une quantité raisonnable d'information.

\vspace{1.3cm}

%Les articles compilés dans ce document comportent donc les choix "discutables" réalisés dans les quatre ouvrages utilisés. Il s'agit donc d'un document de travail destiné à apporter quelques éléments de réflexion et une synthèse relativement élémentaire des points de vues philosophiques.

%Les trois premiers chapitres abordent les thèmes .
% Les chapitres suivants élargissent le champ de vision philosophique en abordant les thèmes .

\vspace{2.3cm}

\hfill {\it Les documents de Zécriture}

\hfill \texttt{www.zecriture.fr}

\vspace{1.3cm}

\hfill \textsc{Numérisation :} Stephan Runigo

\hfill \textsc{Illustration :} Christiane Audhuy

%%%%%%%%%%%%%%%%%%%%%%%%%%%%%%%%%%%%%%%%%%%%%%%%
