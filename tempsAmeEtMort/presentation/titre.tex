\begin{titlepage}
%
%~\\[1cm]

%\newcommand{\HRule}{\rule{\linewidth}{0.5mm}}

\begin{center}
\includegraphics[scale=0.7]{./presentation/mort2}
\end{center}

%\textsc{\Large }\\[0.5cm]

% Title \\[0.4cm]
\HRule

\begin{center}
{\huge \bfseries  L'âme, le temps \\
et la mort\\[0.4cm] }
\end{center}

\HRule \\[1.5cm]


% Author and supervisor
\begin{minipage}{0.4\textwidth}
\begin{flushleft} \large
%{\it Tempora mutantur et nos in illis}
Approche philosophique
\end{flushleft}
\end{minipage}
\begin{minipage}{0.4\textwidth}
\begin{flushright} \large
%\emph{Auteur:}\\
%Stephan \textsc{Runigo}
\end{flushright}
\end{minipage}

\vfill


% Author and supervisor
\begin{minipage}{0.4\textwidth}
\begin{flushleft} \large
%\emph{Auteur:}\\
%Stephan \textsc{Runigo}
\end{flushleft}
\end{minipage}
\begin{minipage}{0.4\textwidth}
\begin{flushright} \large
%\emph{Auteur:}\\
%Stephan \textsc{Runigo}
Extraits de dictionnaires et d'encyclopédies
\end{flushright}
\end{minipage}

\vfill
{\sf \footnotesize
\begin{itemize}[leftmargin=1cm, label=\ding{32}, itemsep=1pt]
\item {\bf Objet : } Étudier les concepts d'âme, de mort et de temps.
\item {\bf Contenu : } Définitions et philosophie encyclopédique.
\item {\bf Public concerné : } Interressé par la question de l'âme.
\end{itemize}
}

\vfill
% Bottom of the page
{\large \today}

\end{titlepage}
