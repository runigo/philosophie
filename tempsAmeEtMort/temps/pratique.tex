
\section{Pratique de la philosophie}

Qu'il s'agisse de la machine à remonter le temps de H. G. Wells ou du thème classique de la fuite du temps, les hommes semblent s'être toujours attachés à vouloir conjurer ou annuler les effets d'un temps dont l'irréversibilité même suscite l'angoisse d'une fin prochaine.

\subsection{Du temps cyclique au temps linéaire}

C'est peut-être par rapport à cette peur du temps qui passe, à l'imminence de la mort, que le mythe ou la religion cherchent à nier l'irréversibilité du temps en le représentant sous la forme d'un cycle — ou de ce temps hors du temps qu'est l'éternité. Ainsi la roue des générations, dans le mythe d'Er l'Arménien de Platon, réintègre l'existence humaine dans un mouvement cyclique où le passé se répète et où chaque chose, une fois advenue, retourne à ce qu'elle était. Mais cette roue justement est aussi la figure de la nécessité qui scelle à jamais les vies en autant de destins. La circularité du temps, si elle annule le poids du passé, ferme l'homme à l'avenir comme champs de ses actions possibles et lieu de réalisation de sa liberté. De même, l'éternité, espérance d'un au-delà, d'une vie après la mort, n'est-elle pas aussi ce qui empêche de pouvoir vivre ici et maintenant, en faisant du présent la promesse perpétuelle d'un avenir qui n'est jamais là ?

On ne peut donc reconnaître la spécificité du temps sans en accepter l'irréversibilité, la ligne continue qui, du passé, s'avance dans le futur. Certes, la linéarité du temps fait du passé le domaine de l'irrémédiable, de l'avenir la perspective de notre mort, mais dans cette tension ou va-et-vient perpétuel entre les deux, elle ouvre le présent à la liberté de l'homme. Mais qu'est-ce que le présent ? Un moment vécu dans la continuité d'un autre, le sentiment intérieur d'une durée qui se prolonge, ou une somme d'instants égaux que les pendules mesurent et que la science quantifie ?

\subsection{Temps subjectif et temps objectif}

Si vous connaissiez le temps aussi bien que je le connais moi-même vous ne parleriez pas de le gaspiller comme une chose. Le temps est une personne" : en répondant ainsi à Alice (Lewis Carroll, {\it Alice au pays des merveilles}) qui s'étonnait de voir une montre indiquant seulement le jour du mois et non pas l'heure, le Chapelier met l'accent sur l'un des paradoxes majeurs du temps. Le temps est à la fois une puissance extérieure, une réalité objective sur laquelle nous n'avons pas de prise et qu'indique seulement les aiguilles d'une montre — sans que jamais nous puissions l'appréhender directement — et, en même temps, nous vivons avec lui comme avec une personne à laquelle nous sommes liés objectivement, et dont il faut s'attirer les faveurs.

Comme le montre Bergson, rien de commun donc entre le temps connu par la science, temps mesurable, quantifiable, milieu homogène dans lequel les choses évoluent, et le temps vécu qu'il nomme, par opposition, "sentiment intérieur de la durée". Universel et objectif, le temps de la science n'existe paradoxalement pour personne. Au niveau de la conscience intime, le temps s'allonge ou s'accélère, pèse ou s'oublie en fonction des aléas de la vie, au gré de l'humeur du moment. Ainsi, la force de l'habitude pourra-t-elle donner l'impression qu'il ne s'est rien passé, l'attente ou l'impatience faire de chaque minute qui coule autant d'heures angoissantes, la mort d'un parent cher amener la conscience d'une époque révolue. Le temps vécu est subjectif, il est qualitatif, fait de moment hétérogènes, de vitesses différentes. Mais c'est le même temps que la science présente comme une succession d'intervalles invariables : les instants.

Qu'est-ce alors que le temps, et s'agit-il en fait du temps ou de temps différents qui s'articulent les uns aux autres ?

\subsection{Les différents modes de temporalité}

Si la science semble nous présenter le temps comme une référence unique et absolue, c'est que peut-être l'utilisation des montres ou chronomètres est pour nous une évidence, comme s'il s'agissait par là de saisir le temps réel des choses. Pourtant ce temps lui même a une histoire, celle des instruments de mesure qui servent à l'apréhender. Loin d'exister dans les phénomènes naturels comme une de leur propriétés objective, le temps comme succession d'intervalles réguliers est le résultat de l'évolution des rapports de l'homme au monde extérieur. Ainsi, dans l'antiquité, prévoyait-on les éclipses avec précision, mais le temps de la vie quotidienne ne faisait l'objet que d'une approximation, la durée de l'heure variant en fonction des saisons. À l'époque médiévale, le rythme des journées de travail, la vie religieuse impriment au temps une régularité nouvelle : l'heure a une détermination fixe. Mais ce n'est qu'au milieu du XVII${^{\text e}}$ que les instruments de mesure du temps seront diffusés au niveau de la vie quotidienne.

Loin d'être la référence unique, universelle et absolue à partir de laquelle les phénomènes dans leurs ensemble pourrait être normés, le temps est donc un système de relations, quelquechose de relatif qui est fonction de l'histoire des hommes et de la structure même de leur expérience. Telle est la force de la réflexion kantienne, d'avoir pensé le temps non plus comme une propriété réelle des choses, un absolu, mais comme une forme {\it a priori} de la sensibilité, c'est-à-dire comme structure du rapport du sujet à lui-même et au monde.

Le temps alors, n'est ni ce que la science mesure, ni ce qu'un individu particulier ressent subjectivement. Ces temps sont différents, irréductibles les uns aux autres, parce que le temps lui même n'est rien, si ce n'est le rapport ou le système de relation entre des temporalités multiples et hétérogènes. il y a des temps — temps vécu, temps de la science, mais aussi temps économique ou naturel — qui ont chacun des rythmes différents, des facteurs spécifiques, et qui, dans la représentation, la conscience ou la connaissance que les hommes en ont, expriment autant de strates de leur histoire; et celle-ci se déroule non pas le long d'une ligne unique, mais dans le jeu même de temporalités divergentes.


\begin{itemize}[leftmargin=1cm, label=\ding{32}, itemsep=1pt]
\item {\bf Textes clés} : Platon, {\it Timée}; saintAugustin, {\it Confessions}, livre XI; E. Kant, {\it Critique de la raison pure}; M. Heidegger, {\it L'Être et le temps} (première partie); V. Jankélévitch, {\it L'irréversible et la nostalgie}.
\item {\bf Terme voisin} : durée, temporalité.
\item {\bf Corrélats} : espace; éternité; existence; futur; mort; passé; présent; temporalité.
\end{itemize}

%%%%%%%%%%%%%%%%%%%%%%%%%%%%%%%%%%%%%%%%%%%%%%%%%%%%%%%%%%%%%%%%%%%%%%%%%
