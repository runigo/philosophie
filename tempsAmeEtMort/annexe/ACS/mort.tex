
%%%%%%%%%%%%%%%%%%%%% 393
\section{La Mort}
%%%%%%%%%%%%%%%%%%%%%
%{\it }
Le néant ultime. Ce n’est donc rien ? Pas tout à fait pourtant, puisque
ce rien nous attend, ou puisque nous l’attendons. Disons que la
mort n’est rien, mais que nous mourrons : cette vérité au moins n’est pas rien.
Épicure et Lucrèce, sur cette question, me paraissent plus judicieux que Spinoza.
« Un homme libre ne pense à aucune chose moins qu’à la mort, dit une
fameuse proposition de l’{\it Éthique}, et sa sagesse est une méditation non de la mort
mais de la vie » (IV, 67). À la seconde affirmation, j'adhère absolument. Mais à
la première, non, ni ne vois comment les deux peuvent être compatibles. Comment
méditer la vie sans penser la mort, qui l’achève ? C’est au contraire parce
que nous pensons que la mort n’est rien, dirait Épicure (rien pour les vivants,
puisqu'ils sont vivants, rien pour les morts, puisqu'ils ne sont plus), que nous
pouvons profiter de la vie sereinement. À quoi bon autrement philosopher ? Et
comment le faire en laissant la mort de côté ? Celui qui a peur de la mort, il a
peur, exactement, {\it de rien}. Comment n’aurait-il pas peur de tout ? Alors qu’il n’y
a rien à craindre dans la vie, expliquait encore Épicure, pour celui qui a compris
que le mal le plus redouté, la mort, n’est rien pour nous ({\it Lettre à Ménécée}, 125).
Encore faut-il la penser strictement {\bf --} comme néant {\bf --} pour cesser de l’imaginer
(comme enfer ou comme manque) et de la craindre. Cela suffira-t-il ? Ce n’est
pas sûr. Et même ce n’est pas, lorsque la mort sera toute proche, le plus probable.
Mais pourquoi la pensée devrait-elle suffire ? Comment le pourrait-elle ? Et
qu'importe qu’elle ne suffise pas, si cette idée vraie, ou qui nous paraît telle, nous
aide, ici et maintenant, à vivre mieux ? Une philosophie, même insuffisante, vaut
mieux que pas de philosophie du tout.

Apprendre à mourir ? Ce n’est qu’une partie, non la plus importante ni la
plus difficile, du général apprentissage de vivre. Au reste, et comme l’a dit plaisamment
Montaigne, quand bien même nous ne saurions mourir, nous aurions
tort de nous en inquiéter : « Nature nous en informera sur-le-champ, pleinement
et suffisamment » (Essais, III, 12). S’il faut penser la mort, ce n’est pas
%394
pour apprendre à mourir {\bf --} nous y parviendrons de toute façon {\bf --} mais pour
apprendre à vivre. Penser la mort, donc, pour l’apprivoiser, pour l’accepter,
puis pour penser à autre chose. « Je veux qu’on agisse, écrit merveilleusement
Montaigne, et qu’on allonge les offices de la vie tant qu’on peut ; et que la mort
me trouve plantant mes choux, mais nonchalant d’elle, et encore plus de mon
jardin imparfait » ({\it Essais}, I, 20).

%%%%%%%%%%%%%%%%%%%%%
\subsection{Mourir}
%%%%%%%%%%%%%%%%%%%%%
C'est le passage ultime, où rien ne passe. C’est pourquoi on ne
meurt pas : on agonise (mais les mourants sont vivants, hélas),
puis on est mort (mais les morts ne sont plus). Mourir est un acte sans sujet, et
sans acte : un rond dans l’eau du destin, une imagination, une fantasmagorie,
cette fois bien douloureuse, de l'amour-propre. Le corps lâche son âme comme
un pet, voilà ce qu’il faut dire, et le pet seul, à l’avance, se rebiffe. Qu’as-tu,
mon corps, à te soucier de tes vents ?
%%%%%%%%%%%%%%%%%%%%%%%%%%%%%%%%%%%%%%%%%%%%%%%%%%%%%%%%%%%%%%%%%%%%%%%%%%%%%%%%%%%%%
