
\section{Pratique de la philosophie}

\begin{itemize}[leftmargin=1cm, label=\ding{32}, itemsep=1pt]
\item {\bf Étymologie} : grec {\it anemos}, "air", "soufle"; latin {\it anima}, "soufle", "vie", "âme" (principe vital), et {\it animus} "esprit", "âme" (siège de la pensée).
\item {\bf Biologie} : principe de vie, de croissance et de mouvement; principe organisateur du vivant.
\item {\bf Psychologie} : principe ou organe de la pensée.
\item {\bf Religion} : principe spirituel, immatériel et éternel, de l'homme et dans certaines religions, de tous les vivants.
\item {\bf Sens dérivé} : l'esprit qui anime quelque chose, au point de lui donner le pouvoir d'exprimer la pensée, les sentiments, etc., ou de le faire ressembler à un être vivant (particulièrement en esthétique).
\item {\bf Terme voisin} : esprit.
\item {\bf Termes opposés} : corps, inertie.
\end{itemize}

\subsection{L'âme}

À travers la notion d'âme, se rejoignent des préoccupations en apparence très hétéroclites, puisqu'elle est tour à tour convoquée dans la tradition biologique pour expliquer l'organisation et le fonctionnement du vivant, en métaphysique et en psychologie pour rendre compte de la pensée et de l'affectivité, et que c'est sur elle que se fonde, en religion, la croyance en l'immortalité. L'âme est d'abord le souffle qui anime un corps vivant.

L'âme apparaît en effet d'abord comme le principe d'organisation du vivant. Matérielle (par exemple, dès l'Antiquité, chez Démocrite et Empédocle, chez les épicuriens et les stoïciens) ou immatérielle (chez les pythagoriciens et les platoniciens, puis dans la tradition classique dominante), la notion d'âme est requise pour rendre compte de la complexité de la vie et articuler les diverses fonctions vitales. À ce titre, Aristote, dans le {\it Péri psuchés (De  l'âme)}, ouvrage de référence pour l'ensemble de l'histoire de la notion, en étudie les diverses manifestations dans la totalité des corps animés, en fonction d'une complexité croissante et hiérarchisée de l'univers des êtres vivants. L'âme est alors conçue comme la forme immatérielle des corps vivants, indissociable de ceux-ci, et elle exerce diverses fonctions, elles-mêmes hiérarchisées : fonction nutritive, présente chez tous les vivants et renvoyant à l'âme simple des végétaux, assurant la croissance et la reproduction ; fonction sensitive, apparaissant avec les animaux inférieurs ; fonction motrice (ou « appétitive ») se rajoutant aux deux précédentes chez les animaux supérieurs; enfin fonction « intellective » (délibérative, voire spéculative) chez l'homme. C'est cette dernière fonction, apparaissant au sommet de la hiérarchie du vivant, qui, lorsqu'elle est privilégiée, conduit au sens exclusivement spirituel, ou métaphysique, voire religieux : l'âme, principe de pensée, privilège et essence de l'homme, ouvrant sur la liberté et la moralité. L'âme est alors le plus souvent conçue comme totalement immatérielle, séparable du corps et peut donc, dans de nombreuses doctrines, être considérée comme immortelle et éternelle ({\it cf.} entre autres, le {\it Phédon} de Platon, qui tire argument de la parenté entre l'âme connaissante et les Idées éternelles
pour présenter l'immortalité de l'âme comme « un beau risque à courir »;
la tradition judéo-chrétienne ; 
la {\it Deuxième Méditation} de Descartes, qui ne démontre pas l'immortalité de l'âme, mais assure sa possibilité en établissant l'indépendance substantielle de l'âme par rapport au corps). Dans une perspective religieuse, l'âme est de surcroît présentée comme un don de Dieu assurant le privilège de l'homme face au reste de la Création. Dans le sillage du matérialisme moderne, qui nie l'existence de l'âme, la biologie cherche à faire l'économie de la notion, jugée trop métaphysique, ce qui conduit certains philosophes contemporains à interroger sa pertinence générale ({\it cf.} le « fantôme dans la machine » de Gilbert Ryle; {\it cf.} Esprit).

\subsection{Belle âme}

Hegel désigne par « belle âme » l'état de la «bonne conscience » morale, lorsqu'elle se satisfait de la seule pureté de ses intentions. Forte de ce repli dans la simple conviction subjective : {\bf 1.} Elle se trouve dans l'incapacité d'agir réellement dans le monde tel qu'il est effectivement ; Or, une « conscience morale » incapable d'agir est-elle une conscience ? N'est-elle pas plutôt une forme d'ignorance ? Est-elle morale ? N'est-elle pas plutôt une chimère ouvrant la voie à l'hypocrisie ? {\bf 2.} Elle se croit en état de réformer le monde tel qu'il est — ce qui la condamne, tel Don Quichotte, à se battre contre des moulins à vent. La « belle âme » a compris que la pureté de l'intention est un élément essentiel de la morale, mais elle ne s'est pas encore donné les moyens d'agir moralement, ni par conséquent de faire exister de la moralité dans le monde ({\it cf.} Idéalisme, Morale et Éthique).

\subsection{Union de l'âme et du corps}

Le problème de l'union de l'âme et du corps se pose dans le cadre classique des dualismes, qui les considèrent comme deux réalités de nature radicalement différente, voire — le plus souvent — comme deux substances distinctes. Comment expliquer, dans ces conditions, que non seulement elles cohabitent en un même être, tel que l'homme, mais qu'elles puissent même
agir l'une sur l'autre, comme c'est le cas par exemple dans les passions ?

Ainsi Descartes, par exemple, parle-t-il de l'union consubstantielle de l'âme et du corps comme d'une réalité indiscutable, mais pour une part incompréhensible par un esprit humain. Ce problème ne se pose évidemment pas dans le cadre des monismes, ni a fortiori des matérialismes, qui dénoncent le dualisme comme une illusion, refusant de considérer l'âme et le corps comme deux réalités différentes, mais affirmant au contraire qu'ils sont deux aspects d'une même réalité.
%%%%%%%%%%%%%%%%%%%%%%%%%%%%%%%%%%%%%%%%%%%%%%%%%%%%%%%%%%%%%%%%%%%%%%%%%%
