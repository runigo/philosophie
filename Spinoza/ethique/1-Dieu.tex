%{\footnotesize XIX$^\text{e}$} siècle — {\bf }{\it }{\it «  »}
%https://www.les-philosophes.fr/spinoza/livres-achat/spinoza-ethique.html
%1/10

\section{Livre I : De Dieu}
%%%%%%%%%%%%%%%%%%%%%%%%%%%%%

Dans l’Éthique, Spinoza, fasciné par la rigueur et la clarté du modèle
mathématique, se donne pour visée d’appliquer la {\bf méthode géométrique} à la
philosophie.

On sait que cette méthode procède par définitions, axiomes, puis s’élève de
propositions en propositions rigoureusement {\bf déduites} les unes des autres
jusqu’à atteindre la vérité à démontrer.

\vspace{0.5cm}
Les définitions ont pour objectif d’indiquer précisément le sens des termes
dont on va se servir, afin qu’il n’y ait nulle ambiguïté ou équivoque possible.

Les axiomes posent des vérités considérées comme évidentes, et qui vont servir
de base aux démonstrations. Ce ne sont plus de simples définitions, au sens où
ils ne se contentent pas de déterminer le sens d’un mot dont on va se servir :
ce sont de vrais jugements prédicatifs qui disent quelque chose de quelque
chose et prétendent au statut de vérité évidente.

\vspace{0.5cm}
Prenons quelques exemples de définitions et d’axiomes proposés par Spinoza.

Parmi les définitions, on trouve au livre I, celle, célèbre, de la {\bf cause de soi}
({\it causa sui}) : {\it « Par cause de soi, j’entends ce dont l’essence enveloppe
l’existence, autrement dit ce dont la nature ne peut se concevoir qu’existante »}.

Ou encore celle de la {\bf substance} : {\it « Par substance, j’entends ce qui est en soi
et se conçoit par soi : c’est-à-dire ce dont le concept n’a pas besoin du
concept d’autre chose, d’où il faille le former »}.

Ces deux définitions sont fondamentales, car elles définissent Dieu pour
Spinoza. Dieu est cause de soi, et Dieu est substance. Plus précisément,
Dieu est seule cause de soi, et seule substance.

Parmi les axiomes, on trouve par exemple : {\it « Étant donné une cause déterminée,
il en suit nécessairement un effet, et au contraire, s’il n’y a aucune cause
déterminée, il est impossible qu’un effet s’ensuive »}.

On voit que Spinoza ne se contente pas ici de définir le concept de cause, mais
affirme une vérité reconnue comme évidente sur la réalité du rapport entre un
effet et une cause.

\vspace{0.5cm}
A partir de ce mécanisme argumentatif {\it « more geometrico »}, Spinoza
s’élève donc de propositions en propositions, qu’il déduit les unes des autres.

Par exemple, de la proposition 7 : {\it « A la nature d’une substance appartient
d’exister »}, et de la proposition 8 : {\it « Toute substance est nécessairement
infinie »}, Spinoza déduit la proposition 11 : {\it « Dieu autrement dit une substance
consistant en une infinité d’attributs dont chacun exprime une essence
éternelle et infinie existe nécessairement »}.

\vspace{0.5cm}
Spinoza utilise des termes issus de l’ancienne scolastique : substance,
attribut, essence, etc. Mais il leur donne un sens nouveau, et l’utilisation
de cette méthode géométrique en philosophie est novatrice.

Surtout, les résultats auxquels il va parvenir dans l’Éthique sont éminemment
modernes.

Dieu en effet est certes substance infinie, dont tout le reste n’est
qu’attribut, Dieu certes est perfection, dont tout le reste procède.
Mais Spinoza au livre IV, définit au détour d’une proposition ce qu’il entend
par Dieu : la Nature.

Il parle de la {\it « puissance même de Dieu, autrement dit de la Nature »} :{\it Deus
sive Natura}.

Or si Dieu n’est autre chose que la Nature, il n’y a en fait pas de Dieu. Dieu
n’est qu’un nom donné à la Nature, et il n’y a pas de Dieu transcendant
(extérieur et supérieur) à celle-ci.

C’est donc là un des premiers philosophes qui, au {\footnotesize XVII$^\text{e}$} siècle, développe, de
manière dissimulée pour échapper à la censure, un {\bf athéisme} au cœur de sa
philosophie.

\vspace{0.5cm}
Dieu, autrement dit la Nature, est donc défini comme substance ayant une
infinité d’attributs. Expliquons rapidement le sens traditionnel de ces termes
« substance » et « attribut ». Prenons un homme. Il est riche, grand, célèbre.
Ces déterminations sont des attributs, au sens où elles peuvent être modifiées
sans que le sujet ne disparaisse. Cet homme riche peut devenir pauvre, il
restera tout de même cet individu. Ce sont donc des déterminations non
essentielles d’une personne (ou d’une chose), que l’on appelle des attributs.

%2/10
\vspace{0.5cm}
Les attributs ne sont pas des étants, n’ont pas d’existence consistante en eux
même. En revanche, le sujet porteur de cet attribut (l’homme lui-même) est une
substance. Et ce qu’on ne peut pas lui enlever sans qu’il disparaisse fait
partie de sa substance.

\vspace{0.5cm}
La substance, c’est ce qui se tient en dessous des accidents ({\it sub} : en
dessous, {\it tenere} : se tenir).

Depuis Aristote, ces termes ont ce sens là ; l’homme (substance) a divers
attributs (ou accidents) : beau, grand, naïf, etc. Or Spinoza renverse
radicalement cette perspective. Chez lui, l’homme n’est plus une substance
subsistant en soi.

\vspace{0.5cm}
Il n’est plus en lui-même qu’un {\bf attribut}, un accident, de la seule substance
qui existe véritablement : Dieu, autrement dit la Nature.

Il n’est naturellement pas le seul à être frappé de cette dégradation, de ce
changement ontologique. L’ensemble des étants (tout ce qui est) n’est
qu’attribut de Dieu.

Chez les stoïciens, l’homme n’était certes compris que comme partie du Tout,
qui est le cosmos. Mais l’homme restait substance, parmi les substances.
Ici ce n’est plus le cas.

\vspace{0.5cm}
On comprend dès lors pourquoi Spinoza dit que {\it « tout ce qui est est en Dieu
et rien ne peut sans Dieu ni être ni se concevoir »}.

Dieu est cause de tout, mais pas au sens traditionnel où tout serait un effet
de son intellect et de sa volonté infinie. C’est humaniser Dieu que de lui
accorder ces facultés (intellect et volonté). A la rigueur, on peut le lui
accorder métaphoriquement, mais alors il faut garder à l’esprit que l’un et
l’autre de ces deux attributs représentent réellement en Dieu "toute autre
chose que ce que les hommes, d’ordinaire, entendent vulgairement par là".
La volonté et l’intellect réels de Dieu divergent de notre conception
traditionnelle de la volonté et de l’entendement tout autant que le Chien
zodiacal (la constellation du Chien) diffère de l’animal.

En fait, Dieu est {\bf cause de tout} non pas au sens où il décide de tout, mais au
sens où tout découle nécessairement de sa nature.

\vspace{0.5cm}
Spinoza se place dans une perspective résolument {\bf déterministe}, au sens où pour
lui tout événement, loin de se produire par hasard, découle d’une cause,
elle-même dérivant d’une cause antérieure, qui n’est à son tour que l’effet
d’une cause précédente, etc. En remontant la chaîne des causes, on parvient à
la 1ère des causes qui est la cause de soi (causa sui) : Dieu.

Donc tout être humain, tout étant, tout événement, est un effet nécessaire de
la nature de Dieu, dans son déploiement historique. Rien n’est contingent, rien
ne se produit par hasard : {\it « Dans la nature des choses, il n’y a rien de
contingent, mais tout y est déterminé par la nécessité de la nature divine,
à exister et à opérer d’une manière précise »}.

Néanmoins Spinoza appelle « contingentes » les choses singulières, non pas au
sens où elles se produiraient sans cause, mais au sens où leur cause ne se
situe pas en elle-même, mais dans d’autres causes ou événements ultérieurs.
Seul Dieu n’est pas contingent dans la mesure où en tant que cause de soi,
il porte sa cause en lui-même.

\vspace{0.5cm}
Spinoza peut distinguer alors deux choses : la « nature naturante » : Dieu
lui-même, cette cause de soi d’où tout provient, et la « nature naturée »,
à savoir l’ensemble des attributs créés par ce dynamisme et qui restent en Dieu.

Voici la définition précise de ces deux termes :

Nature naturante : {\it « ce qui est en soi et se conçoit par soi, autrement dit
tels attributs de la substance, qui expriment une essence éternelle et infinie »}.

Nature naturée : {\it « tout ce qui suit de la nécessité de la nature de Dieu, ou
de ses attributs, ou des manières de ses attributs, en tant qu’on les considère
comme des choses qui sont en Dieu »}.

\vspace{0.5cm}
Les étants puisqu’ils découlent de la nature parfaite de Dieu, sont en
eux-mêmes parfaits, ainsi que les événements qui se produisent : {\it « les choses
ont été produites par Dieu avec la suprême perfection : puisque c’est de la
plus parfaite nature qui soit qu’elles ont suivi nécessairement »}.

%3/10
\vspace{0.5cm}
On remarque que certains éléments se rapprochent du stoïcisme (le déterminisme,
la perfection du monde, la nature comme Dieu), et que l’on aurait pu les voir
écrits dans les Pensées de Marc-Aurèle.

Mais ce sont les seuls. La théorie de Spinoza se distingue radicalement du
stoïcisme sur l’ensemble des autres points. Elle a une originalité et une
consistance propre à elle.

\vspace{0.5cm}
Dans le célèbre appendice du livre I, Spinoza insiste sur l’importance qu’il y
a à lutter contre les {\bf préjugés}. Ces préjugés sont les superstitions véhiculées
par la religion, en premier lieu le {\bf finalisme}, c’est-à-dire l’idée que l’homme
est à la fin de la Création (ou encore que Dieu a tout créé pour l’homme, en
vue de son bonheur).

Cela repose sur un autre préjugé, à savoir que l’homme se considère comme
libre, parce qu’il a conscience de vouloir ou de désirer des choses. Mais
il n’est pas conscient du fait qu’il y a des causes qui le déterminent
nécessairement à vouloir ou à désirer ces choses. Ce n’est pas réellement une
volonté libre dont il dispose, il n’est donc {\bf pas libre} : {\it « les hommes se croient
libres, pour la raison qu’ils ont conscience de leurs volitions et de leur
appétit, et que les causes qui les disposent à appéter et à vouloir, ils les
ignorent, et n’y pensent pas même en rêve »}.

L’idée selon laquelle les Dieux ont tout créé pour les hommes, pour que ces
derniers les honorent, et que plus on les honore, plus on en tirera de grands
bénéfices (dont celui de la vie éternelle), relève de la superstition. On
comprend pourquoi Spinoza a été excommunié par ses coreligionnaires juifs
d’Amsterdam, et ce même si l’Éthique n’est jamais parue de son vivant.

\vspace{0.5cm}
En réalité pour Spinoza, c’est le finalisme (à l’œuvre dans le
judéo-christianisme) qui est impie : {\it « cette doctrine supprime la perfection
de Dieu : car si Dieu agit à cause d’une fin, c’est nécessairement qu’il aspire
à quelque chose qui lui manque »}.

Or puisque les hommes pensent que les choses sont faites pour eux, ils ont
forgé les notions (ou fictions) suivantes : {\it « le bien, le mal, l’ordre, la
confusion, le chaud, le froid, la beauté et la laideur »} et puisqu’ils pensent
être libres {\it « la louange et le blâme, le péché et le mérite »}.

En réalité, rien de tout cela n’existe. Cela n’a de sens que par rapport à
nous, mais ne renvoie à aucune réalité en soi. Il n’y a pas de bonnes ou de
mauvaises actions, il n’y a que des actions qui se déroulent nécessairement,
découlant de la substance divine infinie, dont certaines nous arrangent et
d’autres nous nuisent.

De même, les hommes appellent « ordre » ce qui leur est facile d’imaginer,
{\it « comme si l’ordre était quelque chose dans la nature indépendamment de notre
imagination »}.

Un dernier exemple : {\it « si le mouvement que reçoivent les nerfs à partir des
objets qui se représentent par les yeux contribue à la santé, les objets qui
le causent sont dits beaux, et ceux qui excitent un mouvement contraire laids »}.

Pour résumer : {\it « chacun a jugé des choses d’après la disposition de son cerveau
et a pris pour les choses les affections de son imagination »}.

\vspace{0.5cm}
De ce fait, le monde est {\bf parfait}, et ce n’est pas parce que certaines choses
ou événements nuisent aux hommes qu’il en perd sa perfection.

