%{\footnotesize XIX$^\text{e}$} siècle — {\it }{\it «  »}{\bf }
%https://www.les-philosophes.fr/spinoza/livres-achat/spinoza-ethique.html

\section{Livre IV : de la servitude humaine ou : des forces des affects}
%%%%%%%%%%%%%%%%%%%%%%%%%%%%%%%%%%%%%%%%%%%%%%%%%%%%%%%%%%%%%%%%%%%%%%%%%

La servitude désigne l’impuissance de l’homme à maîtriser ou contrarier les
affects. Spinoza consacre ce livre à l’étude de ce phénomène, ainsi qu’à
l’étude de ce qu’ont les affects de bien et de mal.

Spinoza commence par examiner l’origine de la formation en nous des concepts
de perfection, de bien et de mal.

Le concept de {\bf perfection} naît du fait de vouloir réaliser quelque chose et
d’atteindre ce but. C’est toujours par rapport à un but que l’on détermine la
perfection de quelque chose : {\it « par exemple si quelqu’un a vu quelque chose
d’inachevé et a su que le but de l’Auteur de cette œuvre était de construire
une maison, il dira que la maison est imparfaite, et parfaite au contraire
sitôt qu’il verra l’œuvre parvenue à la fin que son Auteur avait décidé de
lui donner »}.

En revanche si on ignore le but de l’auteur de l’œuvre, on ne saura si elle
est parfaite ou non.

\vspace{0.5cm}
Puis les hommes forgent des concepts universels, par exemple ceux de tours,
de maisons, etc. et {\it « chacun appela parfait ce qu’il voyait convenir avec
l’idée universelle de la chose »}, {\it « idées qu’ils tiennent pour les modèles des
choses »}. Ils pensent que la nature elle-même les a en vue et se les propose
pour modèles. Si la chose ne convient pas au concept modèle, {\it « ils croient
alors que la nature elle-même a fait défaut ou a péché »}.

Or Spinoza a montré dans l’appendice du livre I que la nature n’agit pas à
cause d’une fin, que le {\bf finalisme} est un préjugé, et que le déterminisme à
l’œuvre dans la nature consacre bien la relation de cause à effet, mais pas
la notion de cause finale.

%8/10
\vspace{0.5cm}
Il est faux donc de croire que la nature soit parfaite ou imparfaite
puisqu’elle ne vise pas un but ; les notions de perfection et d’imperfection
ne sont que des fictions introduites par les hommes.

De même bien et mal {\it « ne désignent pas non plus rien de positif dans les choses,
mais rien d’autre que des manières de penser ou notions que nous formons de ce
que nous comparons les choses entre elles »}.

\vspace{0.5cm}
Pourtant, il faut conserver ces termes, pour pouvoir former une idée de la
nature humaine.

Ce pourquoi Spinoza commence le livre IV en définissant ainsi les concepts
moraux : {\it « par bien et mal, j’entendrai ce que nous savons avec certitude
être un moyen d’approcher ou de s’éloigner du modèle de la nature humaine
que nous nous proposons »}.

De ce fait, la notion de bien ne recouvre d’autre signification que celle
d’{\bf utilité} : {\it « par bien, j’entendrai ce que nous savons avec certitude nous
être utile »}, ou encore de ce qui nous est bénéfique : {\it « nous appelons bien
ou mal ce qui sert ou bien nuit à la conservation de notre être »}.

Enfin, Spinoza rapproche les notions de bien et de mal de celles de joie et
de tristesse, telles qu’il les a défini au livre précédent, comme passage à
une perfection plus ou moins grande : {\it « la connaissance du bien et du mal
n’est rien d’autre que l’affect de joie ou de tristesse, en tant que nous en
sommes conscients »}.

\vspace{0.5cm}
Notons que c’est dans la proposition 4 de ce livre que l’on trouve la célèbre
assimilation de Dieu à la nature : [...] {\it « la puissance même de Dieu, autrement
dit de la Nature »} ({\it Deus sive natura}).

\vspace{0.5cm}
Ce qui fait la servitude humaine, c’est que nous ne pouvons, en vertu du
déterminisme, maîtriser un affect, ou y échapper, de par notre volonté.
En fait, {\it « un affect ne peut être contrarié ou supprimé que par un affect
contraire et plus fort que l’affect à contrarier »}.

Spinoza consacre plusieurs propositions d’ordre psychologiques à préciser
la manière dont un affect s’empare de nous. Par exemple : {\it « un affect par
rapport à une chose que nous imaginons comme nécessaire est toutes choses
égales d’ailleurs, plus intense qu’à l’égard d’une chose possible ou
contingente, autrement dit non nécessaire »}.

\vspace{0.5cm}
Spinoza annonce à partir de la proposition 18, qu’il va maintenant montrer
ce que la raison prescrit, à savoir {\it « quels sont les affects qui conviennent
avec les règles de la raison humaine et quels sont au contraire ceux qui leur
sont contraires »}.

Or {\it « la raison ne demande rien contre la nature, c’est donc elle-même qui
demande que chacun s’aime lui-même, recherche ce qui lui est utile, et
aspire à tout ce qui augmente notre perfection, et que chacun s’efforce
autant qu’il est en lui de conserver son être »}.

Spinoza appelle {\bf vertu} le fait d’ {\it « agir d’après les propres lois de sa nature »},
et donc {\it « le fondement de la vertu est l’effort même pour conserver son être
propre »}.

Ce qui nous est le plus utile, c’est ce qui est de même nature que nous. Donc
{\it « à l’homme rien de plus utile que l’homme »}. De ce fait, l’homme que gouverne
la raison, c’est-à-dire celui qui recherche son propre bonheur, n’aspire à
rien d’autre qu’au bonheur général de tous les hommes.

Spinoza remet donc en cause le principe selon lequel agir vertueusement est
agir de manière désintéressée. L’égoïsme, bien compris, est le fondement de
l’altruisme, pourrait-on dire, puisque autrui nous est foncièrement utile.

\vspace{0.5cm}
Si tous les hommes vivaient sous la conduite de la raison, chacun agirait
selon le souverain droit de nature, c’est-à-dire ferait tout ce qui suit de
la nécessité de sa nature ; mais à cause des affects, qui surpassent la
puissance ou vertu de l’homme, {\it « ils se trouvent entraînés diversement,
contraires les uns aux autres »}.

\vspace{0.5cm}
%9/10
Une solution est que les hommes renoncent à leur droit naturel et s’assurent
mutuellement de ne pas se nuire, en entrant en société.

Puisqu’en effet un affect ne peut être contrarié que par un affect plus fort,
on résistera à un affect (désir de voler quelqu’un) par un autre affect (la
crainte d’un dommage plus grand, à savoir celui d’une punition par la société).

\vspace{0.5cm}
Rappelons que ce livre IV est consacré à l’exposé des raisons de la servitude
humaine, l’homme étant assujetti à la puissance des affects. Or dès ce livre
IV, Spinoza présente les moyens par lesquels l’homme peut conquérir sa liberté,
en luttant contre la puissance des affects.

Il suffit d’utiliser cette puissance même des affects en tournant celle des uns
contre celle des autres, afin qu’ils se {\bf neutralisent}.

\vspace{0.5cm}
Spinoza présente un second moyen de se libérer : {\it « à toutes les actions
auxquelles nous détermine un affect qui est une passion, nous pouvons être
déterminés sans lui par la raison »}.

Notre raison peut créer en nous certains désirs, et ceux-ci, en tant qu’ils
procèdent de la raison nous détermineront à faire ce qui est le mieux pour
nous, autrement dit {\it « ce qui se conçoit adéquatement par la seule essence de
l’homme »}. En effet, {\it « un Désir qui naît de la raison ne peut être excessif »}.

Nous pouvons donc nous libérer des affects, en en créant d’autres {\bf plus adaptés}
ou plus utiles, par {\bf la raison}.

\vspace{0.5cm}
La peur ne fait pas partie des affects raisonnables : {\it « qui est mené par la
crainte, et fait le bien pour éviter le mal, n’est pas mené par la raison »}.

Agir par la raison est agir par joie ou désir. Ce sont les superstitieux qui
cherchent à contenir le mal par la peur de la mort. Ainsi, le sage est celui
qui a dépassé cette peur de la mort, et qui n’y accorde même aucune pensée :
{\it « l’homme libre ne pense rien moins qu’à la mort, et sa sagesse est une
méditation non de la mort, mais de la vie »}.

Enfin, il s’agit de supporter d’une âme égale ce qui nous arrive, y compris
les événements tristes, que nous n’avons pas pu éviter du fait de notre
puissance limitée : {\it « la puissance de l’homme est extrêmement limitée et
infiniment surpassée par la puissance des causes extérieures »}. On peut
rapprocher ce dernier point (subir, se résigner, sans passion négative –
tristesse, peur, envie...) du stoïcisme.


