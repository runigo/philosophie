%{\footnotesize XIX$^\text{e}$} siècle — {\it }{\it «  »}{\bf }
%https://www.les-philosophes.fr/spinoza/livres-achat/spinoza-ethique.html

\section{Livre II : de l’Esprit}
%%%%%%%%%%%%%%%%%%%%%%%%%%%%%%%%%%%%%%%%%%

Ici encore, Spinoza commence par poser des définitions et des axiomes.

On relèvera la définition spinoziste du {\bf corps} : {\it « par corps, j’entends une
manière qui exprime, de manière précise et déterminée, l’essence de Dieu en
tant qu’on la considère comme chose étendue »}.

Elle s’inspire de la définition cartésienne du corps comme chose étendue que
l’on trouve dans les Méditations Métaphysiques, qui s’oppose à l’esprit
considéré comme chose pensante. Néanmoins chez Spinoza, ces deux attributs
caractérisent Dieu, ou la nature : {\it « la Pensée est un attribut de Dieu,
autrement dit Dieu est chose pensante »} et {\it « l’étendue est un attribut de Dieu,
autrement dit Dieu est chose étendue »}.

%4/10
\vspace{0.5cm}
Si l’homme pense, il n’est pas une substance mais {\it « l’essence de l’homme est
constituée par des modifications précises des attributs de Dieu »}, {\it « quelque
chose qui est en Dieu, et qui sans Dieu ne peut ni être ni se concevoir,
autrement dit une affection qui exprime la nature de Dieu de manière précise
et déterminée »}.

On voit donc que Spinoza se distingue fondamentalement de Descartes sur ce
point, pour qui l’homme est formé de deux substances, la substance pensante
et la substance étendue.

\vspace{0.5cm}
Pour Spinoza, l’esprit humain n’est pas une substance, mais {\it « une partie de
l’intellect infini de Dieu »}. Ce qui permet cette belle affirmation : {\it « Dieu
constitue la nature de l’Esprit humain »}.

Le corps quant à lui est {\it « l’objet de l’idée constituant l’esprit humain »} ou
encore {\it « une manière de l’Etendue précise et existant en acte, et rien d’autre »}.

\vspace{0.5cm}
Cette conception permet de donner une solution au problème de {\bf l’union du corps
et de l’esprit} fondamentalement différente de celle de Descartes.

Tout d’abord, il y a bien union : {\it « Que l’Esprit est uni au Corps, nous l’avons
montré de ce que le Corps est objet de l’Esprit »}.

Ensuite il n’y a pas deux choses distinctes, mais une seule et même chose,
envisagée sous {\bf deux points de vue différents} : {\it « l’idée du Corps et le Corps,
c’est-à-dire l’Esprit et le Corps, est un seul et même individu, que l’on
conçoit tantôt sous l’attribut de la Pensée, tantôt sous celui de l’Etendue »}.

\vspace{0.5cm}
Cette solution résolument novatrice du problème de l’esprit et corps amène
Spinoza à reconsidérer la nature de deux facultés : la connaissance et la
volonté. Ces deux facultés sollicitent en effet, dans leur acception
traditionnelle, le corps et l’esprit.

Les philosophes ont en effet essayé de déterminer l’importance du corps dans
le processus de la connaissance. Comment connaît-on une chose ? Au moins dans
un premier temps grâce à la perception du corps.

Pour Spinoza, nous percevons par les sens les choses extérieures, et nous en
formons des concepts universels. Mais les sens nous les présentent {\it « de manière
mutilée et confuse, et sans ordre pour l’intellect »}. Ce pourquoi ce premier
genre de connaissance est imparfait : {\it « j’ai l’habitude d’appeler de telles
perceptions connaissance par expérience vague »}.

Cette connaissance du {1$^\text{er}$} genre regroupe ce qu’on l’on entend par opinion ou
imagination.

Elle est l’unique source de fausseté.

\vspace{0.5cm}
La connaissance du {2$^\text{nd}$} genre désigne la raison, c’est-à-dire le fait
d’atteindre les notions communes et les idées adéquates des propriétés
des choses.

La connaissance du {3$^\text{ème}$} genre est une science intuitive, qui {\it « procède de
l’idée adéquate de l’essence formelle de certains attributs de Dieu vers
la connaissance adéquate de l’essence des choses »}.

Ces deux derniers genres de connaissance sont nécessairement vrais.

\vspace{0.5cm}
La vérité d’une idée apparaît dans son évidence : {\it « qui a une idée vraie, en
même temps sait qu’il a une idée vraie, et ne peut pas douter de la vérité
de la chose »}.

Puisque notre esprit est {\it « une partie de l’intellect infini de Dieu »}, {\it « il est
tout aussi nécessaire que les idées claires et distinctes de l’esprit soient
vraies, que cela est nécessaire des idées de Dieu »}.

{\bf La raison} perçoit les choses telles qu’elles sont en soi. Elle saisit donc
leur caractère nécessaire ; de même {\it « il est de la nature de la raison de
percevoir les choses sous une certaine espèce d’éternité »} ({\it sub specie
aeternitatis}).

\vspace{0.5cm}
Spinoza doit également, suite à sa nouvelle conception de l’union de l’esprit
et du corps, revisiter la notion traditionnellement admise de {\bf volonté}. En
effet la volonté est classiquement une faculté par laquelle l’esprit agit sur
le corps. Mais qu’est la volonté, si corps et esprit sont une même chose,
envisagée sous deux angles différents ?

%5/10
\vspace{0.5cm}
Spinoza commence par rappeler qu’en fonction de sa conception déterministe,
{\it « dans l’Esprit nulle volonté n’est absolue, autrement dit libre, mais l’Esprit
est déterminé à vouloir ceci ou cela par une cause, qui elle aussi est
déterminée par une autre, et celle-ci à son tour par une autre, et ainsi à
l’infini »}.

Cela vaut de la même manière pour les prétendues facultés de comprendre, de
désirer, d’aimer... qui sont donc {\it « soit purement fictives, soit rien que des
étants métaphysiques, autrement dit des universaux que nous avons coutume de
former à partir des particuliers »}.

\vspace{0.5cm}
En fait, ce qui est réel, c’est non pas la volonté mais telle ou telle {\bf volition}
particulière qui accompagne telle ou telle de nos idées : {\it « l’intellect et la
volonté ont avec telle idée ou volition le même rapport que la pierrité avec
telle pierre, ou que l’homme avec Pierre et Paul »}.

Il faut remarquer que Spinoza n’entend pas par volonté le désir (duquel il
traitera dans le livre suivant), mais {\it « la faculté d’affirmer et de nier la
vérité ou la fausseté de quelque chose »}.

\vspace{0.5cm}
De plus les volitions elles-mêmes disparaissent, puisque l’idée enveloppe en
elle-même sa propre négation ou affirmation (ce en quoi elles ne sont pas des
simples images, ou peintures, à laquelle l’esprit décide ou non de donner son
assentiment). Elles portent en elles-mêmes leur propre assentiment.

Prenons un exemple : ce n’est pas la volonté qui refuse l’idée de se jeter dans
le vide, c’est cette idée elle-même qui porte son propre principe de persuasion
(ou de refus). De même que la vérité se manifeste d’elle-même, de par sa propre
lumière ou évidence, comme on l’a vu.

Ce pourquoi Spinoza peut affirmer {\it « dans l’esprit il n’y a aucune volition,
autrement dit aucune affirmation et négation, à part celle qu’enveloppe l’idée,
en tant qu’elle est idée »}.

Il n’y a donc, de ce fait, aucune différence entre volonté et intellect.
Comprendre une idée, c’est y adhérer ou la refuser, selon l’affirmation ou la
négation qu’elle porte. Ou encore, c’est l’idée elle-même qui s’impose à nous,
ou au contraire, suscite en nous une opposition.

\vspace{0.5cm}
Pourtant, certains n’adhèrent-ils sincèrement pas à des idées fausses ? Spinoza
le nie : ils ne sont pas certains de cette idée, au mieux, ils ne doutent pas.
Or la certitude est quelque chose de positif et non une simple privation de
doute. Seule une idée vraie peut provoquer la certitude.

De la même manière, Spinoza balaie d’autres objections. Par exemple, le fait
que depuis Descartes, on distingue la volonté de l’intellect car la première
est infinie, tandis que le second est fini. Ou encore qu’alors on ne pourra
choisir entre deux idées ayant la même force d’assentiment, tel {\bf l’âne de
Buridan} qui se laisse mourir parce qu’il est placé à égale distance de foin
et d’eau.

Spinoza commence à envisager les conséquences pratiques des deux premiers
livres. La doctrine spinoziste permet de rendre {\it « l’âme tranquille de toutes
les manières »}, du fait qu’ {\it « elle enseigne que nous agissons par le seul
commandement de Dieu, et que nous participons de la nature divine »} et enfin
que {\it « notre suprême félicité consiste dans la seule connaissance de Dieu »}.

