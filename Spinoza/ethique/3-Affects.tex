%{\footnotesize XIX$^\text{e}$} siècle — {\it }{\it «  »}{\bf }
%https://www.les-philosophes.fr/spinoza/livres-achat/spinoza-ethique.html

\section{Livre III : des affects}
%%%%%%%%%%%%%%%%%%%%%%%%%%%%%%%%%%%%

Les affects désignent les passions et désirs humains. Ce serait une erreur de
considérer qu’ils sont {\bf contre-nature}. Au contraire puisque tout procède
nécessairement de la substance divine, tout est naturel : {\it « pour la plupart,
ceux qui ont écrit des affects semblent traiter, non de choses naturelles qui
suivent les lois communes de la nature, mais de choses qui sont hors de la
nature. On dirait même qu’ils conçoivent l’homme dans la nature comme un
empire dans un empire. Car ils croient que l’homme perturbe l’ordre de la
nature plutôt qu’il ne le suit »}.

C’est là une profonde erreur, au contraire {\it « rien ne se fait dans la nature
que l’on puisse attribuer à un vice de celle-ci ; car la nature est toujours
la même, et a partout une seule et même vertu et puissance d’agir »}.

%6/10
\vspace{0.5cm}
De ce fait, les affects apparaissent ou disparaissent selon la même nécessité
que les choses ou événements. Une science des affects est donc possible : {\it « je
considèrerai les actions et appétits humains comme s’il était question de
lignes, de plans ou de corps »}.

Des définitions qui ouvrent le livre III, nous retiendrons celle des affects :
{\it « par affect, j’entends les affections du corps qui augmentent ou diminuent la
puissance d’agir de ce corps et en même temps les idées de ces affections »}.

\vspace{0.5cm}
La première proposition défend l’idée que notre esprit agit en tant qu’il a des
idées adéquates et pâtit en tant qu’il a des idées inadéquates ; une idée
adéquate étant, selon la définition qu’il en a donnée dans le livre II, une
{\it « idée qui en soi, sans rapport à l’objet, a les propriétés ou les dénominations
intrinsèques de l’idée vraie »}.

Peut-on dire simplement qu’une idée adéquate est une idée vraie ? Probablement,
bien que la formulation de Spinoza laisse plutôt penser qu’une idée adéquate
est une idée bien formée, ayant la forme ou les propriétés d’une idée vraie.

Le corollaire de cette première proposition est remarquable : {\it « l’Esprit est
sujet à d’autant plus de passions qu’il a plus d’idées inadéquates, et au
contraire, agit d’autant plus qu’il a plus d’idées adéquates »}. Peut-on en
déduire qu’un esprit qui atteindrait les dernières vérités serait pure action ?

\vspace{0.5cm}
On a vu dans le livre II la critique spinoziste de la faculté de volonté, et
l’affirmation de l’idée selon laquelle le corps et l’esprit ne sont qu’une
seule et même chose, envisagée selon deux points de vue différents.

C’est pourquoi Spinoza rappelle que {\it « le corps ne peut déterminer l’esprit à
penser, ni l’esprit déterminer le corps au mouvement, ni au repos, ni à quelque
chose d’autre »}.

Ce qui détermine tel ou tel mouvement de notre corps, c’est plutôt un mouvement
antérieur d’un autre corps, ou encore Dieu comme substance étendue ; de même
une pensée ne peut avoir pour cause qu’une autre pensée antérieure, ou Dieu
comme substance pensante.

\vspace{0.5cm}
De ce fait, {\it « l’ordre ou l’enchaînement des choses est un, qu’on conçoive la
nature sous l’un ou l’autre de ces attributs »} autrement dit {\it « l’ordre des
actions et passions de notre corps va par nature de pair avec l’ordre des
actions et passions de notre esprit »}.

Il est donc faux de croire comme on le fait communément que le corps se meut
sous l’impulsion de l’esprit. D’ailleurs, le corps peut se mouvoir sans esprit,
ainsi que le montre l’exemple des somnambules. Enfin, nul ne sait ce que peut
le corps : {\it « l’expérience n’a appris à personne jusqu’à présent ce que le Corps
peut faire par les seules lois de la nature »}.

On croit que le corps est inerte sans esprit ? Spinoza fait remarquer par
l’exemple du sommeil que si le corps est inerte, l’esprit est inerte.

\vspace{0.5cm}
Spinoza développe à nouveau une critique de la volonté, ou du {\bf libre arbitre},
inconciliable avec le déterminisme fondamental de sa pensée : {\it « Ainsi croit le
bébé aspirer librement au lait, et l’enfant en colère vouloir la vengeance, et
le peureux la fuite. L’homme ivre croit que c’est par un libre décret de
l’Esprit qu’il dit ce que, redevenu sobre, il voudrait avoir tu ; alors
pourtant qu’il n’a pas pu contenir l’impulsion qu’il a eu à parler »}.

Pour synthétiser : {\it « les hommes se croient libres, pour la seule raison qu’ils
sont conscients de leurs actions et ignorants des causes par quoi elles sont
déterminées »}.

\vspace{0.5cm}
La célèbre proposition 6 affirme le principe du {\bf conatus} : {\it « chaque chose, autant
qu’il est en elle, s’efforce de persévérer dans son être »}.

Toute chose en effet s’oppose à ce qui pourrait venir supprimer son existence.
Or cet effort, cette tendance fondamentale en tout être, n’est rien d’autre que
la volonté (lorsqu’on le considère du point de vue de l’esprit), ou le désir,
(lorsqu’on le considère à la fois du point de vue de l’esprit et du corps).

%7/10
\vspace{0.5cm}
Spinoza utilise indistinctement les termes « appétit » ou « désir » mais fait
cette précision terminologique : le désir est un appétit dont l’homme est
{\bf conscient} : {\it « le Désir est l’appétit avec la conscience de l’appétit »}.

Cet effort, autrement dit, le désir, est si fondamental qu’il {\it « n’est rien
d’autre que l’essence même de l’homme »}.

Si nous désirons ou voulons une chose, c’est n’est pas parce que nous la
jugeons bonne ; {\it « au contraire, si nous jugeons qu’une chose est bonne, c’est
précisément parce que nous nous y efforçons, nous la voulons, ou aspirons à
elle, ou la désirons »}.

\vspace{0.5cm}
L’esprit peut pâtir de grands changements et passer à une perfection tantôt
moindre, tantôt plus grande. C’est ici que l’on trouve la célèbre définition
de la {\bf joie} spinoziste : {\it « par Joie, j’entendrai donc une passion par laquelle
l’Esprit passe à une plus grande perfection »} ; la tristesse désigne l’affect
contraire.

Les trois affects primaires sont pour Spinoza l’allégresse, la douleur, et le
désir (les deux premiers représentent la joie et la tristesse, mais rapportés
à la fois à l’esprit et au corps).

De la définition spinoziste de la joie se déduit celle de l’amour : {\it « l’Amour
n’est rien d’autre qu’une Joie qu’accompagne l’idée d’une cause extérieure, et
la Haine, rien d’autre qu’une Tristesse qu’accompagne l’idée d’une cause
extérieure »}.

\vspace{0.5cm}
Spinoza définit un grand nombre d’affects, qui naissent de la composition des
trois affects primitifs : l’espérance, la crainte, le désespoir, l’émulation,
etc. et propose un certain nombre de lois psychologiques. Par exemple : {\it « si
nous imaginons que quelqu’un aime ou désire ou a en haine quelque chose que
nous aimons, désirons, ou avons en haine, par là même, nous aimerons, etc.
la chose avec plus de constance »}.

L’auteur de l’Éthique propose pour conclure une définition générale des affects
dans laquelle il répète que {\it « le Désir est l’essence même de l’homme, en tant
qu’on la conçoit comme déterminée, par suite d’une quelconque affection
d’elle-même à faire quelque chose »}, désir étant entendu au sens le plus large,
incluant appétits et volontés.


