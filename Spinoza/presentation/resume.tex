\begin{center}
\Large
Résumé
\normalsize
\end{center}
\vspace{3cm}
\begin{itemize}[leftmargin=1cm, label=\ding{32}, itemsep=21pt]
\item {\bf Objet : } présentation de la philosophie de Spinoza.
\item {\bf Contenu : } fiche de lecture de l'Éthique.
\item {\bf Public concerné : } philosophe en herbe.
\end{itemize}

\vspace{3cm}

%https://www.les-philosophes.fr/spinoza/livres-achat/spinoza-ethique.html

Tiré de l'excellent site \texttt{www.les-philosophes.fr}, ce document est un résumé de l'ouvrage de Spinoza.


\vspace{0.5cm}
{ \it Introduction du résumé}

\vspace{0.5cm}
L’Éthique de Spinoza n’est publiée qu’à sa mort, en 1677, pour éviter la censure. Ce livre est d’ailleurs interdit dès l’année suivante. Il y développe ses idées à la façon des mathématiciens (en faisant s’enchaîner des propositions rigoureusement déduites les unes des autres). Dieu, la liberté, les passions, sont examinés tour à tour, pour élaborer une nouvelle définition du sage.

\vspace{3cm}

