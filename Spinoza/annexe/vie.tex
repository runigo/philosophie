%{\footnotesize XIX$^\text{e}$} siècle — {\bf }{\it }{\it «  »}
%https://www.les-philosophes.fr/spinoza/livres-achat/spinoza-ethique.html
%%%%%%%%%%%%%%%%%%%%%
\chapter{Biographie}
%%%%%%%%%%%%%%%%%%%%%
Spinoza est un philosophe néerlandais du {\footnotesize XVII$^\text{ème}$} siècle (1632-1677). Il prend ses distances vis-à-vis du judaïsme et est ainsi excommunié. Il gagne sa vie en taillant et polissant des verres pour les lunettes et les microscopes. Face à la censure, et aux risques encourus, il renonce à publier son œuvre principale, l’{\it Éthique}, de son vivant. Celle-ci ne sera publiée qu’à sa mort, avec deux autres ouvrages : le {\it Traité de la Réforme de l’Entendement} et le {\it Traité politique}.
%%%%%%%%%%%%%%%%%%%%%%%%%%%%%%%%%%%%%%%%%%%%%%%%%%%%%%%%%%%%%%%%%%%%%%%%%%%%%%%%%%%%%
