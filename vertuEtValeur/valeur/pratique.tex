
%%%%%%%%%%%%%%%%%%%%%
\section{Pratique de la philosophie}
%%%%%%%%%%%%%%%%%%%%%

\subsection{Valeur}

%{\bf }{\it }{\bf --}{\footnotesize X}$^\text{e}$

\begin{itemize}[leftmargin=1cm, label=\ding{32}, itemsep=1pt]
\item {\bf \textsc{Étymologie} :} latin {\it valor}, de {\it valere},
« être bien portant », puis « valoir ».
\item {\bf \textsc{Sens ordinaire} :} qualité
des choses, des personnages; des
conduites, que leur conformité à
une norme où leur proximité par
rapport à un idéal rendent
particulièrement dignes d'estime.
\item {\bf \textsc{Économie} :} prix de quelque chose, dont
le calcul intègre différents facteurs.
\end{itemize}
Leurs interrogations sur la société ont
conduit Aristote, et beaucoup plus tard
Marx, à analyser les mécanismes de la
division du travail, et à montrer que
l'échange des produits suppose la possibilité
de les comparer : pour être équitable,
un échange doit porter sur des
objets de même valeur (« équi-valents »).
L'équivalent-monnaie permet de fixer le
prix (la valeur) des choses. Au {\footnotesize XIX}$^\text{e}$
siècle, Marx a montré que cette valeur
ne dépend pas seulement de l'importance
du besoin satisfait par la marchandise :
il faut aussi, entre autres, distinguer
la « valeur d'usage », qui se
mesure à l'utilité du produit pour le
consommateur, et la « valeur d'échange »,
dont la mesure prend en compte à la
fois le critère de l'utilité ét le travail
nécessaire à la production. La question
des valeurs est par ailleurs au centre des
interrogations sur les fondements de la
morale et de la science: réalités
idéales et transcendantes pour Platon,
normes indiscutables de la conduite
pour Descartes, des valeurs comme le
bien et le vrai sont au contraire, pour
Nietzsche, strictement relatives aux
%460
intérêts de ceux qui ont pu imposer
leurs choix comme universels.

\begin{itemize}[leftmargin=1cm, label=\ding{32}, itemsep=1pt]
\item {\bf \textsc{Terme voisin} :} prix; qualité.
\item {\bf \textsc{Corrélats} :} axiologie ; échange ;
éthique ; jugement ; morale ;
norme/normatif ; règle ; respect ;
volonté.
\end{itemize}

%%%%%%%%%%%%%%%%%%%%%%%%%%%%%%%%%%%%%%%%%%%%%%%%%%%%%%%%%%%%%%%%%%%%%%%%%%%
