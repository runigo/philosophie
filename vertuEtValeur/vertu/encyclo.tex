
%%%%%%%%%%%%%%%%%%%%%
\section{Encyclopédie de la philosophie}
%%%%%%%%%%%%%%%%%%%%%
%{\bf }{\bf --}{\it }

\subsection{Vertu}
Terme qui en grec ({\it arete}) comme en
latin ({\it virtus}) avait à l’origine le sens
d’excellence d’une qualité telle qu’elle se
révèle dans ses effets, comme la vertu
d’un couteau est de couper. Elle n’était
donc pas limitée à l’action des hommes.
Dans la conception classique de la vie, la
vertu humaine était surtout la force de
l'esprit qui n'était pas séparée de la
% 1650
vigueur physique. Ce sens réapparaîtra
pendant la Renaissance, on peut ainsi
penser à Machiavel pour qui la « vertu »
(par opposition à la « fortune ») est la
capacité de dominer les circonstances en
fonction de ses buts, indépendamment du
jugement moral et religieux que l’on peut
porter sur eux et des moyens employés. À
l'époque moderne, Nietzsche reprendra
cette conception de l'Antiquité et de la
Renaissance pour désigner la « volonté de
puissance » par opposition à l’ascétisme
de vertus chrétiennes trompeuses.

\subsubsection{Classifications des vertus,
Moralité et sagesse}


L'élaboration philosophique de la
notion de vertu s'accompagne en Grèce
du développement de l’éthique, dont elle
constitue immédiatement le concept central
(elle le restera au moins jusqu’à
Kant). Selon Platon, les vertus propres à
l'esprit humain dépendent toutes de la
domination de la partie rationnelle sur les
parties irrationnelles (et ils ont leur exact
équivalent dans le bon État politique en
ce qu’il est constitué des différentes catégories
de citoyens). Dans la République,
il illustre les quatre vertus que l’ensemble
de la pensée chrétienne appellera, à partir
de saint Ambroise, les vertus cardinales
(ou principales) : la tempérance par rapport
aux désirs, la fermeté ou le courage,
la sagesse (ou la « prudence ») et la justice.
Cette dernière réunit en elle toutes
les autres vertus en assurant l'harmonie
dans la vie de l’homme (et dans celle de
l'État) mais la prudence (que Platon ne
distingue pas de la « sagesse ») consiste en
la domination de la rationalité dans la vie
de l’homme, elle est donc la condition
même de la vie vertueuse.

La sagesse a un rôle tout aussi central
chez Aristote, mais elle est cette fois distincte
de la prudence et a, par rapport à
elle, une position inférieure. Pour Aristote,
la prudence est une vertu « dianoétique »
(qui appartient à la part rationnelle
de l'esprit) mais concerne la conduite tout
entière de la vie. Elle fixe donc les critères
des vertus « éthiques », ou bien relative à
l’action. En tant que vertu dianoétique, la
prudence est une connaissance philosophique
mais, à la différence des autres
vertus dianoétiques, qui s'intéresse purement
à la connaissance (« esprit », « science »,
« sagesse ») où aux techniques
(« arts »), elle a pour objet l’activité pratique.
%
Cela ne signifie pas que, selon Aristote,
pour pratiquer les vertus éthiques, il
soit nécessaire d’être philosophe, dans le
cas contraire, pour les pratiquer, il faudrait
attendre l’âge mûr alors que les vertus
éthiques (ou les vices qui leur sont
contraires) sont acquises dès l’enfance.
Au contraire, une vie éthiquement vertueuse
est la condition pour accéder aux
vertus dianoétiques. L'éducation éthique
est fournie par l'exemple que donnent les
hommes qui se révèlent sages (ou qui sont
considérés comme tels par la majeure partie
de leurs concitoyens) dans la pratique,
dans la conduite de leur vie. Le critère de
la vertu éthique, que le philosophe élabore
en illustrant ce en quoi consiste la
sagesse, est celui de la position moyenne
vis-à-vis des vices qui lui sont respectivement
opposés mais qui sont tous
contraires à la vertu. La plus grande partie
de l’Éthique à Nicomaque est consacrée
à l'illustration des différentes vertus
éthiques, avec des analyses précises qui
ont eu une grande influence sur l’ensemble
de la réflexion psychologique et
morale postérieure (on peut aussi évoquer
les analyses connexes contenues
dans la Rhétorique). En ce qui concerne
l'élaboration du concept même de « vertu »,
contenue dans le livre II, Aristote
insiste sur son caractère de « disposition ».
Être vertueux (ou vicieux) n’est
pas un acte isolé en tant que tel, mais l’habitude
à un certain type d’acte, même si à
son tour elle dérive de la pratique effective
d’actes conformes. C’est de cela que
dérive le caractère central du problème
de l’éducation morale des jeunes à l’âge
où leurs dispositions et leurs attitudes se
forment et se fixent pour le reste de leur
vie. Aristote soutient que l’on peut enseigner
la vertu (une question célèbre qui
avait été débattue par les sophistes puis
par Socrate dans les dialogues platoniciens),
même si elle l’est plus par
l’exemple que par des préceptes abstraits.
Elle n’est pas « par nature », justement
parce qu’elle dépend de la répétition d’actions
déterminées qui sont à l’origine
totalement sous notre contrôle. Aristote
veut éviter l’ascétisme platonicien et
reconnaît que, pour le bonheur, les circonstances
qui ne dépendent pas de la
volonté humaine et donc de sa vertu doivent
pourtant être suffisamment satisfaisantes.
C’est à cette condition que la
réalisation des vertus éthiques assure le
%1651
bonheur. L'Éthique à Nicomaque se
conclut sur l'affirmation que la « sagesse »
assure un bonheur encore plus grand, plus
proche de la béatitude divine, pure théorie
consacrée à la contemplation de ce qui
est immuable et éternel, totalement extérieure
à ce monde du devenir dans lequel
se déroule en revanche la vie pratique des
hommes.

La sagesse, ou prudence, conserve son
caractère central et devient même
lunique vertu dans la morale épicurienne
comme dans la morale stoïcienne. Les
épicuriens la comprennent comme le calcul
rationnel des plaisirs en fonction de
leur intensité, de leur durée, etc., alors
que les stoïciens l’opposent aux passions
dans une perspective ascétique. L'idéal du
sage stoïcien est la libération complète
des passions (« apathie »), il ne faut donc
pas les accepter, il ne faut même pas les
laisser nous troubler. La vertu est sa
propre récompense et elle assure automatiquement
le bonheur, même dans des
conditions de vie que les hommes non
sages jugeraient malheureuses. Pour ceux
qui n’atteignent pas l’idéal du sage, il faut
formuler des règles de conduite qui évitent
le dérèglement des passions en s’inspirant
du critère de ce qui est le plus
« adapté » (ou « opportun ») dans le cadre
de la vie privée et sociale. La conception
historique de la vertu connaîtra un retour
important entre les {\footnotesize XVI}$^\text{e}$ et
{\footnotesize XVII}$^\text{e}$ s. avec
Descartes (pour qui la plus grande vertu
est justement une « passion » particulière :
la magnanimité) et Spinoza pour
qui la béatitude n’est pas la récompense
de la vertu.

\subsubsection{Vertus morales et vertus théologales}


Le christianisme introduit l’idée de vertus
surnaturellés, couramment appelées
vertus « théologales » (foi, espérance,
charité) par opposition à celles, purement
humaines, que l’éthique antique prend en
considération. Mais le rapport entre ces
deux niveaux est très différent en fonction
des rapports que les divers courants théologiques
établissent entre la nature
humaine et le surnaturel. Alors que saint
Thomas récupère l'éthique  aristotélicienne,
en se limitant à soutenir la supériorité
des vertus théologales sur les
vertus « cardinales », saint Augustin avait
en revanche déclaré « fausses » les vertus
humaines comme pratiquées par les
païens. Ce sont les « vices splendides »

vertu

mais ils restent des vices parce qu'ils sont
motivés par l’orgueil humain (désir de
gloire), même quand ils conduisent au
sacrifice héroïque de soi pour l'avantage
de sa patrie terrestre. L’unique vraie
vertu est la charité, l'amour de Dieu, qui
le diffuse chez les élus indépendamment
de leur action, dans le cas contraire, la
grâce serait superflue. Cette conception
fera sa réapparition au {\footnotesize XVI}$^\text{e}$ s. avec la
Réforme protestante et au {\footnotesize XVII}$^\text{e}$ s. avec le
jansénisme.

\subsubsection{Vertu comme sacrifice
et vertu comme spontanéité}


Dans la pensée moderne, la conception
de la vertu oscille entre deux tendances
opposées. La vertu est soit comprise
comme sacrifice de soi, soit comme spontanéité,
élan naturel. La première conception
est reprise dans une intention
libertine par Pierre Bayle et Bernard
Mandeville, ils cherchent ainsi à montrer
que la réalisation uniforme de la vraie
vertu, en admettant que cela soit possible,
conduirait à la ruine de la société dont le
bien-être est en revanche le fruit du vice
des hommes même si, hypocritement, on
ne veut pas le reconnaître. Les prétendues
« vertus publiques » sont l’autre visage
des « vices privés ». D'autre part, par
sacrifice de soi (au profit de sa patrie), on
comprend aussi la «vertu» dont parle
Montesquieu dans le cadre de la sociologie
politique : la vertu est la caractéristique
des citoyens des républiques en
plein épanouissement (par opposition à
l' « orgueil » qui domine dans les régimes
monarchiques et à la « peur » qui domine
dans les régimes despotiques). La conception
de la vertu comme élan naturel est en
revanche propre à l’éthique anglaise qui,
contrairement à Hobbes, se fonde sur le
« sens moral » de la bienveillance (chez
Shaftesbury ou Hutcheson) ou de la sympathie
(comme pour Hume ou Smith).
Cette conception optimiste de la nature
humaine marquera une grande partie du
déisme du {\footnotesize XVIII}$^\text{e}$ s. Elle trouvera ensuite
une expression dans la théorie de la belle
âme.

La morale de Kant représente la réaffirmation
la plus radicale de la vertu
comme sacrifice, en opposition avec les
inclinations sensibles et les intérêts individuels.
Pour Kant, la « théorie de la vertu »
se distingue essentiellement de la « théorie
du droit » (c’est ce que reprennent les
%1652
deux parties de la Métaphysique des
mœurs), en ce que, dans le cas du droit,
on ne vise que la conformité extérieure
d’une action à une norme indépendamment
de l'intention dans laquelle cette
action a été commise. En revanche, la
vertu tient totalement, et seulement, dans
l'intention de respecter la loi morale indépendamment
de toute autre motivation et
des considérations des résultats éventuels
de l’action. La vertu est par conséquent la
soumission de la part de la volonté à ce
qui est commandé par l’« impératif catégorique ».
Il exprime la forme que la loi
morale a dans l’homme et en général chez
tout être rationnel fini : un impératif (et
donc la vertu) présuppose en effet qu’il
n’y ait pas une adéquation spontanée de
la volonté à la loi, mais pour cela, il est
nécessaire de combattre les inclinations
qui conduisent en direction contraire.
L’adéquation spontanée est la « sainteté »
mais elle n’est propre qu'à Dieu. Selon
Kant, la loi morale est la même pour Dieu
et pour les hommes, mais Dieu n’a pas de
devoirs, parce que en lui rien ne s'oppose
à la loi morale, il n’a donc pas de vertu
à proprement parler. Le rigorisme de la
conception kantienne de la vertu, et la
thèse selon laquelle la vertu coïncide avec
l’« autonomie » de la volonté, rappellent
la morale stoïcienne, mais Kant refuse le
stoïcisme en ce que ce dernier a prétendu
que de cette façon l’exigence du bonheur
de l’homme pouvait être satisfaite. Justement
parce que c’est un être fini et qu’il
vit dans un monde sensible déterminé par
la nécessité mécanique, l’homme a, selon
Kant, besoin de bonheur mais dans le
même temps, rien n’assure qu’il puisse
l’atteindre dans ce monde, d’autre part, la
raison morale réclame, par le critère de
justice, que l’on donne un bonheur proportionné
à la vertu de chaque personne
(et un malheur proportionné au vice). À
cet ensemble d’exigences, Kant propose
une solution à travers la foi rationnelle en
limmortalité de l’âme et en l’existence
d’un Dieu juge équitable, selon la doctrine
des « postulats » de la raison pure
pratique. Hegel critique la totalité de la
conception morale de Kant. Il voit dans
sa doctrine des vertus une expression
typique du subjectivisme moderne, en
opposition avec le « caractère éthique »
qui apparut pendant l’Antiquité et que
Hegel considère comme supérieure à la
pure moralité (comprise comme le niveau
des « vertus privées ») même en ce qui
concerne le présent. Après lui, la notion
de vertu a pratiquement disparu de la
problématique de l’éthique depuis que
d’autres notions ont pris une position centrale,
qu’il s'agisse du bien moral
(« bon ») ou du « devoir ». L’éthique
contemporaine manifeste encore un vif
intérêt pour quelques notions qui étaient
autrefois considérées comme des vertus
spécifiques, c’est en particulier le cas pour
la notion de justice.

%%%%%%%%%%%%%%%%%%%%%%%%
