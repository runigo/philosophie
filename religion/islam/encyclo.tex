
%%%%%%%%%%%%%%%%%%%%%
\section{Encyclopédie de la philosophie}
%%%%%%%%%%%%%%%%%%%%%
%{\bf }{\bf --}{\it }

\section{Coran}
Livre sacré des musulmans. Le
terme désigne plus précisément la « récitation »
psalmodiée des sourates de la
Révélation, et ensuite, par extension, le
recueil des révélations transmises par le
prophète Mahomet. Les musulmans ne
voient pas dans le Coran l’œuvre de
Mahomet, mais la Parole même de Dieu,
Mahomet n'étant que l’intermédiaire physique
de la Révélation. D’après le Coran
lui-même, l’archétype céleste du Livre est
descendu en arabe, par le truchement de
l'ange Gabriel, dans l’esprit du Prophète.
Pour la doctrine islamique orthodoxe, le
Coran est donc le plus grand miracle de
Mahomet et atteste de sa qualité de prophète.
%335
L’orthodoxie doctrinale soutient
même que le Coran, en tant que Verbe
éternel de Dieu, est coéternel à Dieu, et
incréé. Les études orientalistes acceptent
unanimement le Coran comme l’œuvre
authentique de Mahomet, qui le « révéla »
par séquences au cours des années
610-632. Le texte actuel est cependant
postérieur à la mort du Prophète. D’après
la tradition la plus répandue, Abu Bakr,
le premier calife, ordonna de rassembler
tout ce qui, de la Parole de Dieu transmise
par Mahomet, avait été écrit ou
confié à la mémoire des individus. Zayd
ibn Thabit, secrétaire de Mahomet, disposa
les cent quatorze chapitres ou « sourates »
par ordre de grandeur (les plus
longues sont placées au début, et les plus
courtes à la fin). Le Coran comprend des
sourates de La Mecque et de Médine,
c’est-à-dire celles qui remontent à la première
ou à la deuxième période de la prédication
de Mahomet : celles qui sont
antérieures à l’année 622, année de l’émigration
(hégire) à Médine, témoignent
d’un souffle religieux et poétique plus
marqué ; les sourates postérieures sont
d’un ton plus discursif et d’un contenu
plus prescriptif. Malgré l’absence d’ordre
systématique, on peut isoler les thèmes
principaux suivants : discours et avertissements
aux croyants, préceptes relatifs au
culte et règles juridiques, récits bibliques
comportant des développements et des
variantes, discours eschatologiques, invectives
et attaques contre les non-croyants,
allusions de caractère historique et autobiographique.
La tradition islamique s’est
toujours opposée à la traduction du
Coran en d’autres langues, et ce n’est que
récemment, pour les pays musulmans de
langue non arabe, que le modernisme islamique
a autorisé les traductions, de préférence
accompagnées du texte arabe
auquel elles servent de commentaire.

\subsection{Chiites}
Chiites nom générique désignant les partisans
d’Ali; le shî\textquoteleft a est le parti d’Ali.
Pour les chiites (env. 10 \% des musulmans
en l’an 2000, principalement concentrés en
Iran), le califat doit se transmettre héréditairement
et leur foi est ouverte à une
dimension eschatologique : en effet, si
pour les sunnites, Mahomet est le dernier
prophète, celui qui, comme « sceau des
prophètes », a achevé et clos la Révélation,
pour les chiites, il en va autrement.
Ils sont ouverts à la venue d’autres « révélateurs »
de la révélation commencée par
Mahomet. Ali et les imams sont les dépositaires
du sens secret de la Révélation.
Dans ce contexte, on comprend pourquoi
la religion chiite s’est dotée d’un clergé
censé guider les fidèles parfois contre
l'autorité d’un calife incompétent. Les
chiites vénèrent en outre les personnages
pieux analogues aux saints du christianisme,
et les honorent dans des mausolées
ou des tombeaux qui sont autant de lieux
du culte. L’Irak est ainsi le principal lieu
de pèlerinage des musulmans chiites
puisque c’est à Najaf que repose Ali et à
Samarra que le XII$^\text{e}$ imam aurait disparu.
Du point de vue artistique, l’interdit de la
représentation y est moins rigoureux que
dans l'islam sunnite. L'identité chiite est
plus persane qu’arabe, Ali ayant épousé
la fille du dernier roi Sassanide.

%—> imam + imamites e islam e ismaé-
%liens e Mahdi e mu'‘tazilites e sunnites

  
%%%%%%%%%%%%%%%%%%%%%%%%
