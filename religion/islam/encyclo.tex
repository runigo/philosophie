
%%%%%%%%%%%%%%%%%%%%%
\section{Encyclopédie de la philosophie}
%%%%%%%%%%%%%%%%%%%%%
%{\bf }{\bf --}{\it }

\subsection{Coran}
Livre sacré des musulmans. Le
terme désigne plus précisément la « récitation »
psalmodiée des sourates de la
Révélation, et ensuite, par extension, le
recueil des révélations transmises par le
prophète Mahomet. Les musulmans ne
voient pas dans le Coran l’œuvre de
Mahomet, mais la Parole même de Dieu,
Mahomet n'étant que l’intermédiaire physique
de la Révélation. D’après le Coran
lui-même, l’archétype céleste du Livre est
descendu en arabe, par le truchement de
l'ange Gabriel, dans l’esprit du Prophète.
Pour la doctrine islamique orthodoxe, le
Coran est donc le plus grand miracle de
Mahomet et atteste de sa qualité de prophète.
%335
L’orthodoxie doctrinale soutient
même que le Coran, en tant que Verbe
éternel de Dieu, est coéternel à Dieu, et
incréé. Les études orientalistes acceptent
unanimement le Coran comme l’œuvre
authentique de Mahomet, qui le « révéla »
par séquences au cours des années
610-632. Le texte actuel est cependant
postérieur à la mort du Prophète. D’après
la tradition la plus répandue, Abu Bakr,
le premier calife, ordonna de rassembler
tout ce qui, de la Parole de Dieu transmise
par Mahomet, avait été écrit ou
confié à la mémoire des individus. Zayd
ibn Thabit, secrétaire de Mahomet, disposa
les cent quatorze chapitres ou « sourates »
par ordre de grandeur (les plus
longues sont placées au début, et les plus
courtes à la fin). Le Coran comprend des
sourates de La Mecque et de Médine,
c’est-à-dire celles qui remontent à la première
ou à la deuxième période de la prédication
de Mahomet : celles qui sont
antérieures à l’année 622, année de l’émigration
(hégire) à Médine, témoignent
d’un souffle religieux et poétique plus
marqué ; les sourates postérieures sont
d’un ton plus discursif et d’un contenu
plus prescriptif. Malgré l’absence d’ordre
systématique, on peut isoler les thèmes
principaux suivants : discours et avertissements
aux croyants, préceptes relatifs au
culte et règles juridiques, récits bibliques
comportant des développements et des
variantes, discours eschatologiques, invectives
et attaques contre les non-croyants,
allusions de caractère historique et autobiographique.
La tradition islamique s’est
toujours opposée à la traduction du
Coran en d’autres langues, et ce n’est que
récemment, pour les pays musulmans de
langue non arabe, que le modernisme islamique
a autorisé les traductions, de préférence
accompagnées du texte arabe
auquel elles servent de commentaire.


\subsection{{\it Sharia} (ou {\it Charria})}
Terme arabe (qui
signifie « chemin », « voie ») désignant les
lois et règlements qui régissent l’existence
des musulmans. La {\it Sharia} est en fait le
prolongement et la concrétisation de la
Loi divine ({\it shar}) ; elle joue le rôle d’une
jurisprudence consistant notamment en
une discussion sur la Loi divine. Le terme
de Sharia apparaît une fois dans le Coran,
en XLV, 18 : « Nous t’avons placé sur une
voie [procédant] de l'Ordre. Suis-la donc
et ne suis point les doctrines pernicieuses
de ceux qui ne savent pas ! » (trad. Régis
Blachère). La {\it Sharia} est fondée sur
quatre sources, qui sont dans l’ordre le
Coran, la {\it Sunna}, l'accord unanime de la
Communauté ({\it idjma}) et le raisonnement
analogique ({\it kiyas}). La {\it Sunna} consiste en
une norme ou une pratique traditionnelle
ayant obtenu l’approbation du Prophète
ainsi que des pieux musulmans d’autrefois,
définissant en fait l’orthodoxie. L’accord
désigne quant à lui le consensus
auquel parviennent les grands théologiens
et juristes reconnus à une époque et qui
font ensuite autorité. Dans la pratique,
bien que venant en troisième, cette source
est la plus importante de celles qui définissent
le droit religieux. Quant au raisonnement
juridique analogique, il renvoie à
la méthode adoptée par les juristes musulmans
permettant de produire la définition
d’une règle qui n’aurait pas été expressément
formulée dans le Coran ou les
hadith. Les {\it ulama} (ou {\it ulémas}) sont ceux
qui étudient la science de la Loi.

On compte quatre écoles de jurisprudence
({\it madhhab}) de l'islam sunnite,
fixant les principes officiels du droit :
l’école hanafite, l’école hanbalite, l’école
malikite et l’école shafi’ite. Elles se consacrent
à l'interprétation de la {\it Sharia} et
sont encore reconnues aujourd’hui.

L'école hanafite s’inscrit dans la tradition
des enseignements délivrés par les
écoles juridiques anciennes de Koufa et
de Bassora (Irak). Son « père fondateur »
est l’imam Abu Hanifa (v. 700-v. 767).
C’est sur l’enseignement de cette école
que les dynasties abbassides, seldjoukides
et ottomanes ont fondé leur système juridique
officiel. Une certaine souplesse, par
rapport aux sources reconnues que constituent
%1501
le Coran et les hadith, caractérise
cet enseignement hanafite : si le droit doit
d’abord se tourner vers ces sources premières,
il est possible d’y adjoindre de
nouvelles réflexions lorsque ces sources
ne permettent pas de traiter une situation
donnée. Cette école participe donc d’une
approche qui fait une place, dans le droit,
à la réflexion rationnelle.

Tel n’est pas le cas du madhhab hanbalite,
qui fixe la jurisprudence de l’islam
sunnite. Considérant que le droit a une
origine divine, il proscrit toute réflexion
personnelle ({\it ra’y}) qui viendrait s’y ajouter,
toute initiative de spéculation qui, par
son origine humaine, ne pourrait que
tendre à introduire le péché. C’est donc le
dogmatisme et la lecture littérale des
textes sacrés qui prévalent dans cette
école fondée sur l’enseignement de
l'imam Ahmad b. Hanbal, et qui trouve
sa popularité en Irak et en Syrie jusqu’au
{\footnotesize XVI}$^\text{e}$ s., et demeure la doctrine juridique
officielle de l’actuelle Arabie Saoudite.

L'école malikite (anciennement appelée
école de Médine) se réfère quant à
elle aux enseignements de l’imam Malik
ibn Anas (v. 715-795), qui vécut principalement
à Médine et composa le premier
traité de droit musulman, le {\it Al-Muwatta}.
Ce madhhab privilégie l’opinion personnelle
ainsi que le raisonnement par analogie,
sur les hadith. L’aire de rayonnement
de ce madhhab est notamment l’Afrique
du Nord, de l'Ouest et le Soudan.

L'école shafi’ite est centrée sur l’enseignement
de l’imam Abu Abdallah
Muhammad ibn Idris al Shafï’i (767-820),
qui prit une part considérable à la mise
au point de la législation musulmane. Au
cours de ses nombreux voyages en Orient
arabe, il put entrer en contact avec
l’enseignement des autres madhhab (notamment
le malikite et le hanafite), pour
proposer une synthèse originale et novatrice
de la jurisprudence musulmane, en
combinant volonté divine et réflexion
humaine, traitant aussi bien de la vie
publique que de la vie privée du musulman.
Exposant sa réflexion dans {\it La
Lettre}, il est tenu pour le père fondateur
du droit musulman. Au cas où le Coran
ou les hadith ne permettent pas de traiter
une situation qui se présente, il est permis
d'utiliser le raisonnement et la déduction
analogique ({\it kiyas}).


%—> chiites  imam  imamites  islam sunnites

\subsection{{\it Sunna}}
Terme arabe signifiant « conduite »,
«comportement ». Elle constitue l’une
des quatre sources de la théologie et du
droit musulman, avec le Coran, l'accord
unanime de la Communauté ({\it idjma}) et le
raisonnement analogique ({\it kiyas}). Elle
occupe le second rang (ceci à l’instigation
des Shaff'ites), dans l’ordre des sources,
après le Coran. La {\it Sunna} consiste en un
dit, un fait ou un silence (interprété alors
comme un consentement tacite) du prophète
Mahomet en des circonstances particulières
de sa vie. Cette conduite est
% 1560
considérée comme étant dotée d’une efficacité
normative, parce qu’elle est inspirée
de Dieu, et qu’elle devient source de
droit pour résoudre des problèmes juridiques
déterminés. À la source de la
{\it Sunna}, il y a les dits ({\it hadith}) qui se réfèrent
à un comportement déterminé du
Prophète et qui font l’objet d’une transmission
orale par une chaîne de témoins.
L'étude de ces {\it hadith} fait l’objet d’une
science particulière dont l’une des tâches
consiste à en vérifier l’authenticité.

%—> Sharia

\subsection{Sunnites}
Désigne l’ensemble des musulmans
qui suivent la voie droite de la
{\it sunna} (c’est pourquoi on les présente parfois
comme des musulmans orthodoxes).
Les sunnites sont ceux qui, lors des
conflits occasionnés par la succession du
Prophète, se sont ralliés au califat (de
l'arabe {\it khalifa}, « remplaçant », « lieutenant »,
« vicaire ») ‘umayyade demeuré
en place après l’assassinat d’Ali. Les sunnites
représentent plus de 89 \% des
musulmans. Ils suivraient la « voie modérée »
entre les deux autres grandes
branches qui ont divisé l'islam dès la mort
du Prophète, les kharidjites et les chiites.
Les sunnites ont en effet défendu le principe
de l'élection du calife par une minorité
de notables (en fait, le calife désigne
son successeur et ce choix est ratifié par
l'assemblée de notables, les {\it ulémas}), s'opposant
ainsi aux chiites qui tenaient au
principe de filiation par le sang et aux
khârjites, pour lesquels tout religieux
pieux à la vertu irréprochable peut diriger
la communauté. Toutefois cette position
médiane n’entraîne pas nécessairement la
modération. En témoigne l’émergence au
{\footnotesize XVIII}$^\text{e}$ s., dans le cadre d’une réaction
anti-ottomane des populations bédouines de
l’Arabie saoudite, du wahhâbisme. Terme
provenant du nom d’un chef religieux
nommé Ibn ‘Abd al-Wahâb qui s’est allié
avec un chef de guerre, Mohamed Ibn
Séoud, et qui sont à l’origine du royaume
d’Arabie saoudite. Le wahhâbisme est un
traditionalisme paradoxal puisqu'il est
enraciné dans un régime politique (la
monarchie) dont le concept est absent du
Coran comme de la {\it Sunna}, cependant il
est strictement opposé aux diverses ramifications
de l'islam et les wahhabites
%1560
relient très étroitement le temporel et le
religieux par une pratique rigoriste de la
{\it Sharia}.
%—> chiites  Sharia  Sunna

\subsection{Chiites}
Chiites nom générique désignant les partisans
d’Ali; le shî\textquoteleft a est le parti d’Ali.
Pour les chiites (env. 10 \% des musulmans
en l’an 2000, principalement concentrés en
Iran), le califat doit se transmettre héréditairement
et leur foi est ouverte à une
dimension eschatologique : en effet, si
pour les sunnites, Mahomet est le dernier
prophète, celui qui, comme « sceau des
prophètes », a achevé et clos la Révélation,
pour les chiites, il en va autrement.
Ils sont ouverts à la venue d’autres « révélateurs »
de la révélation commencée par
Mahomet. Ali et les imams sont les dépositaires
du sens secret de la Révélation.
Dans ce contexte, on comprend pourquoi
la religion chiite s’est dotée d’un clergé
censé guider les fidèles parfois contre
l'autorité d’un calife incompétent. Les
chiites vénèrent en outre les personnages
pieux analogues aux saints du christianisme,
et les honorent dans des mausolées
ou des tombeaux qui sont autant de lieux
du culte. L’Irak est ainsi le principal lieu
de pèlerinage des musulmans chiites
puisque c’est à Najaf que repose Ali et à
Samarra que le XII$^\text{e}$ imam aurait disparu.
Du point de vue artistique, l’interdit de la
représentation y est moins rigoureux que
dans l'islam sunnite. L'identité chiite est
plus persane qu’arabe, Ali ayant épousé
la fille du dernier roi Sassanide.

%—> imam + imamites e islam e ismaé-
%liens e Mahdi e mu'‘tazilites e sunnites

  
\subsection{Soufisme}
 (de l’arabe « {\it sûf} », laine, premiers
 ascètes musulmans ainsi désignés à
cause du vêtement de laine rugueuse
qu'ils portaient)

Doctrine ésotérique de
l'islam et mouvement mystique et ascétique
ayant influencé des dissidences
chiites (ismaïlisme, Druzes). Elle connaît
son développement maximum à Bagdad
entre 750 et 950. La présence invisible de
Dieu dans le cœur du croyant est poursuivie
à travers l’expérience ascétique et
l'union extatique (dans l’amour physique
notamment) permettant d’atteindre à
l'amour et à la connaissance du Créateur.
La symbolique de l’arbre de la connaissance
représente les progrès de la méditation
et de la sagesse ; et la barrière qui
sépare l’homme de Dieu est symbolisée
par la montagne cosmique ({\it Qâf}). Le soufisme
se caractérise par un idéal austère
de pauvreté et de prière combattant le
vice ; mais ce mouvement individualiste
fut condamné par l'islam traditionnel,
entraînant la mise à mort de al-Hallädj en
922.

%%%%%%%%%%%%%%%%%%%%%%%%
