
%%%%%%%%%%%%%%%%%%%%%
\section{Encyclopédie de la philosophie}
%%%%%%%%%%%%%%%%%%%%%
%{\bf }{\bf --}{\it }
\subsection{Culte}
(en latin {\it cultus}, de {\it colere}, « vénérer »)

Terme qui indique un rapport entre
l’homme et des entités ou des forces non
humaines considérées comme supérieures.
Dans la langue courante, on
désigne par le terme {\it rite} des modes de
comportement par lesquels le culte s’effectue.
Dans ce sens général, le culte a un
caractère principalement ou exclusivement
religieux ; il peut s'adresser à des
divinités, à des esprits, ou à tout élément
de la nature (animaux, plantes, corps
célestes, cours d’eau, feu, etc.) ; il peut
aussi s'adresser aux ancêtres, aux
« âmes » des défunts; il peut encore
s'adresser à un sujet collectif (une
communauté ou un groupe de personnes
liées par un ensemble de traits sociaux,
d’activités, ou par une initiation, etc.) ou
à un sujet individuel qui, en raison de privilèges,
de compétences ou de dons particuliers,
représente une collectivité
(souverain, prêtre, chaman, etc.). En histoire
ou en science des religions, en sociologie
et en anthropologie culturelle, la
notion de culte est sujette à controverses,
précisément en rapport (et souvent en
opposition) avec la notion de rite. Des
représentants de l’école historique et
culturelle, ou des chercheurs liés à cette
école, comme par exemple R. Will ({\it Le
Culte}, 1925-1935) et Sigmund Mowinckel
({\it Religion et culte}, 1953), reconnaissent
dans le culte la réalisation suprême de
l'expérience religieuse et opposent, au
moins en principe, le rite au culte, comme
la religion à la magie. La position de phénoménologues
de la religion, tel Gerardus
van der Leeuw ({\it La Religion dans son
%352
essence et ses manifestations}, 1933), est
plus nuancée. Pour van der Leeuw, le
culte n’est pas nécessairement religieux, il
consiste à se mettre en rapport d’{\it officium}
(service) avec le sacré en tant que « mystère ».
Il conteste l’idée selon laquelle le
culte devrait relever uniquement du
domaine religieux (il écarte donc implicitement
l’opposition entre culte-religion et
rite-magie), parce que son interprétation
des rapports avec le mystérieux prévoit
un équilibre, précisément cultuel, entre
force humaine et force radicalement
étrangère ; alors que pour Will, qui part
lui aussi des idées exprimées par Rudolf
Otto, lequel se plaçait dans la perspective
essentiellement chrétienne d’une théologie
systématique, le culte, en tant que rapport
avec le mystère, est expérience et
acceptation de la dépendance à l’égard du
divin, la religion s’opposant alors à la
magie (au rite) tout comme elle se distingue
d’une tentative consistant à contrôler
et à dominer les forces extra-humaines
plutôt qu’à s’y soumettre. L'identification
presque absolue du culte au rite est
accomplie par l’école de Leo Frobenius,
en particulier par Adolf Jensen ({\it Mythe et
culture chez les peuples primitifs}, 1950)
pour qui le culte s’épuise totalement dans
le rite lorsque l’action rituelle se trouve
être renouvellement d’un mythe fondateur,
donc ouverture à l’être-saisi ({\it Ergriffenheit})
par les forces primordiales qui
fondent l’existence et la culture humaine.
À partir d'Emile Durkheim ({\it Les Formes
élémentaires de la vie religieuse}, 1912) et
de l’école française de sociologie, des
chercheurs ont appréhendé différemment
la notion de culte, en y reconnaissant une
expression des traditions et des conditions
sociales collectives : le sujet du culte est
la communauté, et non pas l’individu, qui
voit dans le culte la formulation de ses
aspirations et de ses intentions « religieuses »
intimes. Mais le principal disciple de
Durkheim, Marcel Mauss, écrira dans ses
travaux de vieillesse qu’il n’a jamais rencontré,
au cours de décennies de
recherche, de cultes purement religieux ni
de rites purement magiques, mais bien
toujours des faits magiques et religieux.
La distinction entre culte et rite se présente
donc de façon nouvelle (et elle a
influencé de nombreux chercheurs,
comme Will, qui n’appartenaient pas au
courant sociologique) : pour Durkheim
déjà, le culte se différencie du rite parce
%
qu’il est non seulement collectif mais
aussi systématique et stable, alors que le
rite est individuel, occasionnel, et lié à des
contingences déterminées et non pas
périodiques de la vie (naissance, mariage,
activités diverses, mort). De ce point de
vue, le culte apparaît comme « un système
de rites ».

%—> Durkheim ; Otto (Rudolf)

%%%%%%%%%%%%%%%%%%%%%%%%
