
%%%%%%%%%%%%%%%%%%%%%
\section{Encyclopédie de la philosophie}
%%%%%%%%%%%%%%%%%%%%%
%{\bf }{\bf --}{\it }
\subsection{Culte}
(en latin {\it cultus}, de {\it colere}, « vénérer »)

Terme qui indique un rapport entre
l’homme et des entités ou des forces non
humaines considérées comme supérieures.
Dans la langue courante, on
désigne par le terme {\it rite} des modes de
comportement par lesquels le culte s’effectue.
Dans ce sens général, le culte a un
caractère principalement ou exclusivement
religieux ; il peut s'adresser à des
divinités, à des esprits, ou à tout élément
de la nature (animaux, plantes, corps
célestes, cours d’eau, feu, etc.) ; il peut
aussi s'adresser aux ancêtres, aux
« âmes » des défunts; il peut encore
s'adresser à un sujet collectif (une
communauté ou un groupe de personnes
liées par un ensemble de traits sociaux,
d’activités, ou par une initiation, etc.) ou
à un sujet individuel qui, en raison de privilèges,
de compétences ou de dons particuliers,
représente une collectivité
(souverain, prêtre, chaman, etc.). En histoire
ou en science des religions, en sociologie
et en anthropologie culturelle, la
notion de culte est sujette à controverses,
précisément en rapport (et souvent en
opposition) avec la notion de rite. Des
représentants de l’école historique et
culturelle, ou des chercheurs liés à cette
école, comme par exemple R. Will ({\it Le
Culte}, 1925-1935) et Sigmund Mowinckel
({\it Religion et culte}, 1953), reconnaissent
dans le culte la réalisation suprême de
l'expérience religieuse et opposent, au
moins en principe, le rite au culte, comme
la religion à la magie. La position de phénoménologues
de la religion, tel Gerardus
van der Leeuw ({\it La Religion dans son
%352
essence et ses manifestations}, 1933), est
plus nuancée. Pour van der Leeuw, le
culte n’est pas nécessairement religieux, il
consiste à se mettre en rapport d’{\it officium}
(service) avec le sacré en tant que « mystère ».
Il conteste l’idée selon laquelle le
culte devrait relever uniquement du
domaine religieux (il écarte donc implicitement
l’opposition entre culte-religion et
rite-magie), parce que son interprétation
des rapports avec le mystérieux prévoit
un équilibre, précisément cultuel, entre
force humaine et force radicalement
étrangère ; alors que pour Will, qui part
lui aussi des idées exprimées par Rudolf
Otto, lequel se plaçait dans la perspective
essentiellement chrétienne d’une théologie
systématique, le culte, en tant que rapport
avec le mystère, est expérience et
acceptation de la dépendance à l’égard du
divin, la religion s’opposant alors à la
magie (au rite) tout comme elle se distingue
d’une tentative consistant à contrôler
et à dominer les forces extra-humaines
plutôt qu’à s’y soumettre. L'identification
presque absolue du culte au rite est
accomplie par l’école de Leo Frobenius,
en particulier par Adolf Jensen ({\it Mythe et
culture chez les peuples primitifs}, 1950)
pour qui le culte s’épuise totalement dans
le rite lorsque l’action rituelle se trouve
être renouvellement d’un mythe fondateur,
donc ouverture à l’être-saisi ({\it Ergriffenheit})
par les forces primordiales qui
fondent l’existence et la culture humaine.
À partir d'Emile Durkheim ({\it Les Formes
élémentaires de la vie religieuse}, 1912) et
de l’école française de sociologie, des
chercheurs ont appréhendé différemment
la notion de culte, en y reconnaissant une
expression des traditions et des conditions
sociales collectives : le sujet du culte est
la communauté, et non pas l’individu, qui
voit dans le culte la formulation de ses
aspirations et de ses intentions « religieuses »
intimes. Mais le principal disciple de
Durkheim, Marcel Mauss, écrira dans ses
travaux de vieillesse qu’il n’a jamais rencontré,
au cours de décennies de
recherche, de cultes purement religieux ni
de rites purement magiques, mais bien
toujours des faits magiques et religieux.
La distinction entre culte et rite se présente
donc de façon nouvelle (et elle a
influencé de nombreux chercheurs,
comme Will, qui n’appartenaient pas au
courant sociologique) : pour Durkheim
déjà, le culte se différencie du rite parce
%
qu’il est non seulement collectif mais
aussi systématique et stable, alors que le
rite est individuel, occasionnel, et lié à des
contingences déterminées et non pas
périodiques de la vie (naissance, mariage,
activités diverses, mort). De ce point de
vue, le culte apparaît comme « un système
de rites ».

%—> Durkheim ; Otto (Rudolf)

%%%%%%%%%%%%%%%%%%%%%%%%{\footnotesize X}$^\text{e}$
%%%%%%%%%%%%%%%%%%%%%%%%{\bf }{\bf --}{\it }
\subsection{Déisme}
Mouvement philosophique d’origine
anglaise qui s'affirme aux {\footnotesize XVII}$^\text{e}$ et
{\footnotesize XVIII}$^\text{e}$ s., et se propage par la suite en France
et en Allemagne. Au {\footnotesize XVI}$^\text{e}$ s., le mot « déisme »
s’oppose à celui d’« athéisme » pour
désigner tout simplement l'attitude de
quiconque croit en l’existence de Dieu.
Mais le terme se trouve déjà, en un sens
%370
particulier, chez Pascal qui, l’opposant à
« christianisme », juge inacceptable pour
le chrétien la façon qu’a le déisme de
considérer Dieu. La thèse principale du
mouvement déiste est que l’on ne doit
concevoir la nature de Dieu que suivant
les attributs que lui confère la raison naturelle,
indépendamment de toute révélation
et en refusant tout ce qui, dans les
religions relevant d’une confession historique,
ne s’accorde pas avec la simple raison.
Le déisme se fonde sur l’opposition
entre la religion naturelle ou rationnelle
(universelle) d’un côté, et les religions
positives ou historiques (particulières) de
l’autre. Toutes les religions positives doivent
être passées au crible de la religion
naturelle afin qu’émergent, de cette
confrontation, les erreurs et les absurdités
dont aucune d’entre elles n’est exempte.
On a coutume de considérer l’essai de
John Locke intitulé {\it Le Christianisme raisonnable
tel que révélé par les Ecritures}
(1695) comme un signe avant-coureur
immédiat du déisme. Dans ce texte, le
philosophe oppose la doctrine simple et
raisonnable que l’on peut tirer des Evangiles
au foisonnement d’absurdités doctrinaires
échafaudées par les différents
conciles (en particulier le concile de
Nicée), et qui a abouti à la doctrine chrétienne
officielle. Déjà dans son {\it Dictionnaire
historique et critique} (1696), Pierre
Bayle, mû par son scepticisme foncier et
par la critique des sources bibliques
(Ancien et Nouveau Testament), avait
ouvert la voie à une attitude tolérante
hostile à toute inféodation confessionnelle,
ainsi qu’à une lecture sans préjugés
de l’Écriture sainte. Dans {\it Le Christianisme
sans mystères} (1696), John Toland
refuse dans les Écritures tout ce qui ne
s’accorde pas avec la raison et avec le
principe de l’uniformité de la nature, au
premier chef, les miracles. Matthew Tindal,
dans {\it Le Christianisme aussi vieux que
la Création ou l'Évangile considéré
comme: une reproduction de la religion
naturelle} (1730), soutient que les vérités
rationnelles contenues dans le christianisme
n'avaient pas besoin d’être révélées ;
en outre, ce qu’il y a de contraire à
la raison dans la religion judéo-chrétienne
est de loin plus fruste et plus empreint de
superstition que dans les autres religions
positives. La critique des textes est
conduite par Anthony Collins ({\it Essai sur
l'usage de la raison}, 1707 ; {\it Discours sur la
% 371
liberté de pensée}, 1713; {\it Discours sur les
fondements de la religion chrétienne}, 1724)
sur le terrain d’une analyse philologique
rigoureuse. La lecture et la comparaison
de l'Ancien et du Nouveau Testament
lamènent à penser que ces textes doivent
être considérés comme des expressions
allégoriques : pris à la lettre, ils ne
seraient qu’un amoncellement d’inepties
et d’incohérences. D’autre part, leur lecture
correcte est ardue lécriture
biblique, caractérisée par l’absence de
voyelles et de signes de ponctuation, se
prête à des interprétations différentes,
toutes discutables, y compris celle accréditée
par l’Église. Thomas Woolston, ami
et collaborateur de Collins, pousse la critique
des textes jusqu’au sarcasme,
démontrant que, s’ils étaient pris à la
lettre, Jésus apparaîtrait comme un vulgaire
imposteur ({\it Discours sur les miracles
de Jésus-Christ}, 1727-1729). La discussion
sur la crédibilité de la Bible prit l'allure
d’un débat judiciaire portant sur la validité
des témoignages concernant les
miracles, et particulièrement la résurrection
(cf. T. Sherlock, {\it Examen des témoignages
sur la résurrection de Jésus}). Elle
prit aussi la forme d’un examen de l’héritage
païen recueilli, au prix de mutations
qui ne sont que superficielles, par la religion.
chrétienne (Conyers Middleton,
{\it Lettre de Rome, montrant l’exacte conformité
entre catholicisme et paganisme},
1729). Dans ses {\it Dialogues sur la religion
naturelle} (1779), et dans {\it L'Histoire naturelle
de la religion} (1757), David Hume
soutient qu’il est impossible de démontrer
l'existence de Dieu au moyen d’arguments
rationnels {\it a priori} (car l’existence
n’est pas déductible de son idée) ; il considère
toutefois la religion comme un fait
universel qui trouve son origine dans la
crainte inhérente à la condition humaine.
En France, le mouvement déiste prit des
formes radicales : condamnation de toutes
les religions positives, refus de la révélation
et des miracles, dénonciation de l’imposture,
c’est-à-dire de l’utilisation de la
superstition à des fins politiques. Des
écrits, tout d’abord anonymes et clandestins
({\it Examen de la religion}, 1745 ; {\it Examen
du Nouveau Testament}, 1769),
connurent une grande diffusion, préparant
ainsi le terrain à l'offensive des
Lumières. Parmi les œuvres des grands
représentants de ce mouvement, il faut
rappeler avant tout celles de Voltaire : les
%
{\it Lettres philosophiques} (1734), le {\it Traité
sur la tolérance} (1763) et le {\it Dictionnaire
philosophique} (1764) constituent un
recueil des principaux arguments des
déistes anglais contre les religions révélées,
les miracles, la superstition et le
fanatisme religieux. L'Allemagne connut
un déisme plus tempéré, tendant à considérer
la religion d’un point de vue historique.
Hermann Samuel Reimarus ({\it Traités des
plus importantes vérités de la religion
naturelle}, 1754, et {\it Fragments d’un anonyme},
publication posthume que Gotthold
Ephraim Lessing fit paraître entre
1774 et 1778) se situe dans la tradition du
déisme le plus radical. Il ne voit en Jésus
et Jean le Baptiste que des hommes qui
aspiraient à devenir chefs politiques, mais
ont échoué : dès lors, la résurrection de
Jésus n’est rien d’autre qu’un conte
inventé par ses disciples. Parmi les déistes
modérés, mentionnons Christian Wolff
({\it Théologie naturelle}, 1736-1737) et Moses
Mendelssohn ({\it Les Heures matinales ou
Leçons sur l'existence de Dieu}, 1785). Lessing
(avec {\it Le Christianisme de la raison},
1753 ; {\it Sur la genèse de la religion révélée},
1753-1755 ; {\it Sur la preuve de la force et de
l'esprit}, 1777; {\it L'Éducation du genre
humain}, 1780 ; {\it Ernst et Falk. Dialogues
maçonniques}, 1780) occupe une place à
part ; on trouve même chez lui, à côté des
thèmes déistes classiques, la réhabilitation
des religions positives, vues comme des
phases du développement et de l’éducation
de l'humanité, de l’enfance jusqu’au
Stade de la maturité caractérisé par une
religion purement rationnelle. {\it La Religion
dans les limites de la simple raison}
(1793) d’'Emmanuel Kant et l’{\it Essai d’une
critique de toute révélation} (1792) de
Johann Gottlieb Fichte peuvent être
considérés comme des prolongements du
déisme.

%%%%%%%%%%%%%%%%%%%%%%%%
