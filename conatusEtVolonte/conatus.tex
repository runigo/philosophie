\section{Conatus spinoziste et volonté schopenhauerienne}

La lecture de Spinoza par Schopenhauer s’inscrit dans un contexte de crise philosophique, bien résumé par
Schopenhauer lui-même. Pour les philosophes qui viennent après Kant, il s’agit « de combler le grand vide
que les résultats négatifs de la philosophie kantienne ont causé ».
Comment ce vide a-t-il été comblé jusqu’à présent? A la suite de la critique kantienne de toute philosophie
spéculative, écrit Schopenhauer, « presque tous les gens qui philosophent en Allemagne se sont rejetés sur
Spinoza: toute la série d’essais manqués sous le nom de philosophie postkantienne n’est que du spinozisme
ajusté sans goût ». Par « essais manqués », Schopenhauer entend les philosophies de Fichte, Schelling,
Hegel, qu’il nomme les « néospinozistes ». À l’en croire, Schopenhauer serait le seul continuateur de Kant
en Allemagne, et ferait cavalier seul contre les « néospinozistes ».
La question que nous posons est la suivante: si toute philosophie allemande post-kantienne consiste en un
« ajustage » du spinozisme, Schopenhauer échappe-t-il lui-même, comme il le prétend, à cette règle? Nous
tenterons de répondre à cette question en distinguant les trois points de vue qui commandent la lecture
schopenhauerienne de Spinoza: la théorie de la connaissance, la métaphysique, la morale.

\subsection{La théorie de la connaissance}

En ce qui concerne la théorie de la connaissance, Spinoza fournit à Schopenhauer l’occasion de proclamer
son rattachement à Kant et à certains acquis du criticisme. S’attaquer à Spinoza, c’est également s’attaquer
aux trois « néo-spinozistes », philosophes de profession, « soucieux d’oblitérer la philosophie de Kant, pour
pouvoir naviguer à nouveau sur le canal envasé du vieux dogmatisme ». Certes, Schopenhauer présente son
propre système comme un dogmatisme, au sens où il prétend identifier la chose en soi, contrairement à Kant.
Il y a bien là une extension de la connaissance au-delà du phénomène. Mais il s’agit d’un « dogmatisme
immanent », au sens où les énoncés sur le monde se fondent sur l’expérience du
monde tel qu’il est donné. Il se distingue en cela du dogmatisme « transcendant » combattu par Kant, qui
recourt à un principe extérieur au monde, soustrait à l’intuition, pour expliquer celui-ci . C’est ce rôle
décisif de l’intuition dans la connaissance que Schopenhauer fait valoir contre Spinoza et les « néo-
spinozistes », également nommés les « trois sophistes allemands ».

\subsubsection{La méthode}

Schopenhauer s’attaque ainsi à la méthode de Spinoza. Sa lecture est à l’évidence gouvernée par le chapitre
de la Critique consacré à la « Discipline de la raison pure dans son usage dogmatique ». C’est de façon
illégitime, écrit Schopenhauer, que Spinoza habille la philosophie des vêtements de la géométrie, en utilisant
une longue suite de propositions, démonstrations, scolies et corollaires. Manque en effet au philosophe [die]
Konstruktion der Begriffe, cette construction dans l’intuition qui seule assure aux concepts validité et
exhaustivité. Et Schopenhauer de railler: « L’habit ne fait pas le moine ». Ce n’est
pas en prenant l’habit du géomètre que le philosophe peut prétendre atteindre à la même rigueur et à la même
évidence, puisqu’il existe une différence irréductible entre philosophie et mathématique, la seconde seule
étant capable de valider ses définitions en les construisant dans l’intuition. Le philosophe, sauf à tomber dans
l’arbitraire, ne doit donc pas partir de définitions abstraites mais de ce qui est immédiatement donné à son
intuition.
Schopenhauer se défend donc d’être « spinoziste »: « Mes énoncés ne reposent pas sur des chaînes
logiques, mais immédiatement sur le monde offert à l’intuition ». Il suit la voie inverse de celle
de Spinoza: « Ma philosophie est obtenue et présentée selon la méthode analytique, non synthétique. »
Schopenhauer part du réel offert à l’intuition tandis que Spinoza construit le réel à partir de simples concepts.
La seconde attaque contre Spinoza, également inspirée par Kant, concerne le statut de l’étendue ou espace.

\subsubsection{L’étendue}

Schopenhauer loue Spinoza pour la définition qu’il donne du temps dans les Cogitata metaphysica,
selon laquelle le temps n’est pas une affection des choses mêmes, mais un modus cogitandi ou un
être de raison. Cette définition s’accorde en effet avec la conception kantienne du temps comme forme de la
représentation. Il en va tout autrement de l’étendue, que Spinoza attribue aux choses mêmes « au moyen
d’une pure affirmation ». Certes l’étendue n’est pas, comme le croyait Descartes, le
« contraire de la représentation », et en un sens Spinoza a raison d’affirmer que l’étendue et la pensée sont
une seule et même chose, envisagée sous deux points de vue différents. Mais ce que Spinoza n’a pas vu,
c’est que l’étendue est interne à notre représentation: « L’étendue en effet n’est d’aucune manière le contraire
de la représentation, mais est entièrement inhérente à celle-ci. » Ayant manqué ce point essentiel, Spinoza
est incapable de prouver qu’à notre représentation de choses étendues correspondent bien des choses
étendues hors de notre représentation. En effet, une fois l’étendue dogmatiquement posée hors de ma représentation,
il devient impossible de
prouver qu’elle existe bien indépendamment de celle-ci. Seul Kant résout le problème en le dépassant:
l’étendue ou espace étant la simple forme de notre représentation des choses extérieures, cela n’a plus de
sens de demander si cette étendue existe hors de notre représentation. Au plan de la représentation, le réel se
réduit aux phénomènes liés selon la catégorie de causalité, au sein de l’espace et du temps en nous.
Dans le même ordre d’idées, un second grief est formulé à l’encontre de Spinoza. Celui-ci supprime la
relation causale entre le corporel et le spirituel. Schopenhauer cite Ethique, II, 5 : « Les idées [...] admettent
pour cause efficiente, non les objets mêmes qu’elles représentent – autrement dit les choses perçues -, mais
Dieu lui-même, en tant qu’il est chose pensante. » Pour ce motif encore, dit Schopenhauer, Spinoza se met
dans l’incapacité de résoudre la question du statut du représenté. Ma représentation d’un corps a-t-elle un
corrélat corporel hors de moi? Rien ne permet de l’affirmer puisque ma représentation a pour cause Dieu
comme chose pensante, et non un corps.

\subsubsection{La distinction entre phénomène et chose en soi}

Schopenhauer reproche à Spinoza de prétendre connaître les choses en soi, en posant sans preuve que
celles-ci sont en elles-mêmes étendues, conformément à la manière dont nous nous les représentons. Certes il repère chez Spinoza plusieurs passages qui semblent anticiper sur la position kantienne. On
peut trouver dans l’Éthique une distinction entre le phénomène et la chose en soi, « avec cette idée que seul
le premier, le phénomène, nous est accessible ». Schopenhauer cite entre autres textes Ethique,
II, XVI, cor. 2: « Les idées que nous avons des corps extérieurs indiquent plutôt la constitution de notre corps
que la nature des corps extérieurs. » Il commente ce passage comme suit: « Nous ne connaissons ni les
choses ni nous-mêmes tels qu’ils sont en eux-mêmes, mais seulement tels qu’ils apparaissent » . Mais
Spinoza, comme on sait, prétend dépasser cette limitation dans le second genre de connaissance, et assigne
l’étendue aux choses mêmes, ce que Schopenhauer sanctionne quelques lignes plus loin en rappelant le statut
idéal de l’étendue.
Si l’étendue était une propriété des choses en soi, écrit encore Schopenhauer, alors assurément nous
aurions une connaissance des choses en soi. Mais Spinoza assigne l’étendue aux choses sans pouvoir le
démontrer, de sorte que la base de son système s’effondre. Faute d’avoir découvert le véritable statut de
l’étendue comme forme de la représentation, Spinoza n’a pas saisi correctement la ligne de partage entre le
phénomène (la chose saisie selon les formes subjectives de l’espace et du temps) et la chose en soi (la chose
telle qu’elle est indépendamment des formes de notre représentation): « La ligne de partage entre le réel et
l’idéal, l’objectif et le subjectif, la chose en soi et le phénomène, n’est pas correctement cernée ».
Dans le cadre de cette lecture « néokantienne » de Spinoza, on peut également évoquer la critique de
l’emploi spinoziste de la preuve ontologique et de la confusion opérée par Spinoza entre cause et raison.
Cette critique développée au § 8 de La quadruple racine du principe de raison suffisante peut pour une
bonne part se réclamer de Kant, qui, d’une part, réfute la preuve ontologique cartésienne, jugement
synthétique a priori qu’aucune intuition ne vérifie, qui, d’autre part, comme le souligne Schopenhauer au §
 de cet ouvrage, distingue expressément entre le principe formel de la connaissance (ratio) et son principe
matériel (causa).

\subsubsection{Points communs}

Schopenhauer pense partager deux thèses avec Spinoza. Premièrement, s’agissant de la disposition
d’esprit requise pour philosopher, Schopenhauer trouve en Spinoza un modèle. Spinoza a toujours su
préserver, contrairement aux « trois sophistes » (Fichte, Schelling, Hegel), son indépendance d’esprit, et a
vécu « comme doit vivre un philosophe digne de ce nom ». Pour servir la vérité, en effet, on doit « tenir sur
ses propres jambes » et « ne servir aucun maître ». C’est parce qu’il était pleinement conscient de cette vérité
que Spinoza, selon Schopenhauer, a décliné la proposition qui lui était faite d’occuper un poste de professeur
à l’université. Dans De la volonté dans la nature, Schopenhauer affirme que la
philosophie perd sa dignité dès qu’elle devient une profession, et que la philosophie universitaire est
’« antagoniste de la philosophie sérieuse ».
Deuxièmement, Schopenhauer s’accorde avec Spinoza sur la question de l’origine des concepts.
Schopenhauer s’oppose aussi bien à Leibniz qu’à Kant en reprenant à son compte la théorie spinoziste des
notions communes. Il réfute Leibniz et Wolff par Spinoza: le sensible n’est pas de l’intelligible confus; au
contraire, l’intelligible est du sensible confus. En effet Spinoza a montré que toute notion générale résulte
d’une confusion de sensations. L’imagination ne conserve que les caractères communs à diverses sensations
distinctes les unes des autres et donne ainsi naissance aux concepts généraux de la raison. L’intelligible
résulte donc de la confusion du sensible. Schopenhauer écrit: « Je dois avouer, à l’honneur de Spinoza, qu’à
l’encontre de ces philosophes [Leibniz, Wolff], et avec un sens plus droit, il déclare que toutes les notions
générales naissent de la confusion inhérente aux connaissances intuitives ».
Hormis le concept de causalité, qui est la « condition de l’intuition empirique », tous les
autres concepts dérivent de l’expérience selon Schopenhauer, qui sur ce point au moins déclare s’opposer à
Kant: « Au contraire de Kant, je dis que [...] les concepts ne sont jamais que des abstractions tirées de
l’intuition ».
Même si la théorie schopenhauerienne de la connaissance se sépare à plusieurs égards de celle de Kant,
elle est néanmoins nettement influencée par le criticisme. Spinoza sert la plupart du temps de repoussoir
dogmatique à un écrivain obnubilé par le souci de se démarquer de ses concurrents « néo-spinozistes ». En
revanche, au plan métaphysique, Schopenhauer est tributaire de la doctrine et de la terminologie
spinozistes.

\subsection{La métaphysique : volonté et conatus}

Si l’on compare la Volonté schopenhauerienne au concept spinoziste de volonté, il faut souligner, avec
Otrun Schulz, une nette opposition entre nos deux auteurs. À la thèse cartésienne de la séparation entre la
volonté et l’entendement, et de la plus grande extension de la volonté, s’oppose, comme on sait, la thèse
spinoziste de l’identité. Schopenhauer critique les deux thèses, pourtant opposées, de Spinoza et de
Descartes, leur reprochant d’identifier la volonté à un mode de penser et de la ramener à un jugement, c’est-
à-dire au pouvoir d’affirmer et de nier. Spinoza inverse l’ordre des choses selon Schopenhauer: il tient
l’étendue pour réelle alors qu’elle n’est qu’idéale; il tient la volonté pour idéale en la réduisant à un modus
cogitandi, alors qu’elle est das alleinige wahrhafte Reale: l’unique chose vraiment réelle.
Faut-il donc en conclure à la profonde originalité de l’ontologie schopenhauerienne ?

\subsubsection{Volonté, natura naturans , effort}

Comme on sait, Schopenhauer entend par Volonté (der Wille) une force originelle déployée dans toutes les
forces de la nature, force soustraite au principe de raison et aveugle. Par son caractère irrationnel, elle se
distingue assurément de la substance spinoziste. Contrairement à Spinoza, Schopenhauer soustrait
radicalement à la rationalité la substance du monde (la Volonté). Hormis cette différence certes décisive, la
Volonté schopenhauerienne reprend plusieurs caractéristiques de la substance spinoziste, envisagée comme
natura naturans. Schopenhauer met en équivalence les deux expressions, celle de Volonté comme « force
agissante » et celle de « nature naturante »: die wirkende Kraft, die natura naturans.
Il déclare que la force agissante, la Volonté ou natura naturans, « est immédiatement présente,
entière et indivise en chacune de ses œuvres innombrables: d’où il résulte qu’à ce titre et en soi elle ne
connaît ni le temps ni l’espace ». A l’instar de la nature naturante spinoziste, la
Volonté est une et indivise dans les étants multiples, et échappe, prise en elle-même, au temps et à l’espace.
Schopenhauer lui assigne en outre la liberté, au sens spinoziste du terme: la volonté ne dépend d’aucune
autre chose que d’elle-même pour être et agir: « Elle ne peut dépendre d’aucune autre chose, pas plus dans
son existence et dans son essence que dans sa conduite et dans son activité ».
Reprenant les termes mêmes de Spinoza, Schopenhauer déclare que la connexion des idées (Ideennexus) et
la connexion causale des choses (Kausalnexus der Körper) expriment toutes deux (aüssern) le même être, la
Volonté. C’est cette activité fondamentale s’exprimant dans les phénomènes, cet
effort présent dans tous les étants, assigné par Spinoza au conatus, que Schopenhauer avoue traduire en
vouloir-vivre, Willen zum Leben. On peut donc souligner sur ce point l’étroite
parenté entre nos deux auteurs: « Chez moi comme chez Spinoza, écrit Schopenhauer, le monde existe par
lui-même et grâce à son énergie intrinsèque ». L’influence de Spinoza sur Schopenhauer
apparaît très nettement si l’on examine la manière dont cette énergie inhérente au monde se rapporte aux
étants de ce monde.
La Volonté, nous dit Schopenhauer, se manifeste dans les différents étants comme un effort. Elle
encourage à faire effort: zur Anstrengung aufmuntert. Elle se présente en chaque
étant comme un effort pour se maintenir dans l’être: Seine Grundbestrebung ist die Selbsterhaltung eines
jeden Wesen. Chaque étant se définit ainsi par un effort pour se conserver, sich selbst zu
erhalten, pour chercher ou poursuivre, ein Suchen oder Verfolgen, éviter ou fuir, ein Meiden oder Fliehen
. Schopenhauer reprend manifestement la définition spinoziste du conatus, effort par lequel chaque
chose tend à persévérer dans son être.
Cependant Schopenhauer réserve toute puissance et, partant, toute réalité à la Volonté seule, alors que pour
Spinoza au contraire le conatus constitue l’essence et la puissance de chaque mode. Pour Schopenhauer,
seule la Volonté est réelle, et le multiple n’en est que la manifestation illusoire, telle qu’elle s’offre à notre
représentation. Ce qui s’offre à nous sous forme de multiplicité, dans l’espace et le temps, n’est qu’un
phénomène inconsistant (Schein). Seule existe réellement la chose en soi, la Volonté une: « Toute multiplicitéest une apparence: tous les individus de ce monde, coexistants et successifs, si infini qu’en soit le nombre, ne
sont pourtant qu’un seul et même être, qui, présent en chacun d’eux, et partout identique, seul vraiment
existant, se manifeste en tous. » Curieusement, Schopenhauer attribue cette thèse, selon laquelle « toute
multiplicité est apparente », entre autres auteurs, à Spinoza: « Spinoza est devenu synonyme de cette
doctrine. » On trouve la même erreur dans Le monde comme volonté et comme représentation: « Spinoza
ne voyait en elles [les choses] que les accidents de la substance unique existant seule éternellement ».
Tout dément cette interprétation dans l’œuvre de Spinoza, qui refuse précisément de confondre les modes
avec des êtres de raison ou des « auxiliaires de l’imagination ». Les modes ne sont pas la projection
fantastique d’un substance unique, qui serait seule réelle. Chaque mode a, au contraire, une puissance et une
individualité irréductibles. Chez Spinoza, à la différence de Schopenhauer, on a toujours affaire à un conatus
singulier. Ainsi la puissance de l’homme n’est-elle pas la simple projection de la puissance divine, seule
effective, puisque l’essence ou puissance de Dieu « s’explique » elle-même par l’essence de l’homme.
Nous verrons tout à l’heure le parti que tire Schopenhauer de cette lecture erronée
de Spinoza dans le domaine moral.
Il n’en demeure pas moins que Schopenhauer reprend du conatus un trait essentiel. Spinoza le définit
comme un effort pour persévérer dans l’existence, pour conserver ou renouveler les parties qui appartiennent
au mode. De même, Schopenhauer assigne à la Volonté une fonction de conservation:
« Elle dirige l’économie de l’organisme, et, comme vis medicatrix naturœ, ramène à l’ordre les irrégularités
qui ont pu s’y glisser ». Dans cette perspective, la thèse schopenhauerienne de la
subordination des fonctions de connaissance à un effort pour persévérer dans l’existence nous paraît
présenter, jusqu’à un certain point, une parenté avec les analyses développées par Spinoza.

\subsubsection{Volonté et connaissance}

Pour appuyer sa thèse selon laquelle les forces intellectuelles sont les « servantes de la Volonté »,
Schopenhauer cite dans les Parerga le célèbre passage de l’Éthique: « Nous ne tendons pas vers une chose
pat appétit ou désir, parce que nous jugeons qu’elle est bonne; c’est l’inverse ». En
d’autres termes, c’est le désir qui conditionne l’exercice du jugement, et non pas l’inverse: l’activité
intellectuelle est au service des exigences de la Volonté désirante. Schopenhauer développe l’idée que toute
notre activité mentale est au service des exigences de la Volonté qui cherche à affirmer l’existence de notre
corps. Ses analyses du rapport entre notre activité représentative et la Volonté sont fortement influencées
par le Livre III de l’Éthique.
Spinoza présente le conatus comme un effort pour persévérer dans l’être, pour écarter la tristesse, pour
imaginer ce qui détruit la cause de la tristesse. L’esprit, écrit Spinoza, « répugne à
imaginer ce qui diminue ou contrarie sa puissance et celle de son corps ». De même,
Schopenhauer écrit que « la Volonté interdit à l’entendement certaines représentations », précisément celles
qui dépriment ou affaiblissent l’individu. L’esprit, écrit Spinoza, s’efforce d’imaginer une
chose qui exclut l’existence présente du mauvais souvenir, c’est-à-dire du souvenir triste et
affaiblissant. De même, Schopenhauer explique l’oubli de certains souvenirs pénibles de la manière suivante:
« Elle [la volonté] réfrène l’intellect et le force à détourner ailleurs son attention ».
Les résistances qui provoquent l’oubli ne partent pas de l’intellect lui-même, poursuit-il,
mais de la volonté ou désir qui « constitue l’essence même de l’homme » .
S’efforçant d’exclure les idées tristes, l’esprit, pour Spinoza, s’efforce d’imaginer ce qui est cause de joie
et augmente notre puissance d’agir. C’est par cet effort que Schopenhauer explique qu’
« une mémoire, même faible, retient parfaitement ce qui a de la valeur pour la passion actuellement
dominante ». Sur la fonction sélective de la mémoire, qui annonce la théorie freudienne
du refoulement, Schopenhauer parle le même langage que Spinoza: l’activité représentative est gouvernée
par l’appétit, en vertu duquel l’individu recherche ce qui est utile à sa conservation. Toutes les analyses
schopenhaueriennes du rapport entre Volonté et intellect sont gouvernées par ce principe.
À un niveau supérieur, la Volonté incite l’intellect à lutter contre les passions tristes par la connaissance de
la nécessité: nous pourrions nous épargner neuf fois sur dix la contrariété, écrit Schopenhauer, « si nous
comprenions les choses exactement et par leurs causes, si nous en connaissions la nécessité et la vraie
nature ». Ainsi la Volonté « pousse l’intellect [...] à reconnaître les principes et les
conséquences », lui assignant le « rôle d’un consolateur ». L’accent spinoziste de
ces passages est indéniable: pour cesser d’éprouver la tristesse consécutive à un événement et se consoler,
l’homme doit comprendre que l’enchaînement des causes et des effets rendait cet événement nécessaire,
donc inévitable.
Pour Schopenhauer comme pour Spinoza en effet, même les actions humaines sont soumises au régime de
la nécessité. Dans son essai Sur la liberté du vouloir, Schopenhauer s’autorise de la critique spinoziste du
libre arbitre . Il cite entre autres textes la Lettre  à Schuller, où Spinoza explique
l’illusion de la liberté par l’ignorance des causes qui nous déterminent à vouloir ceci plutôt que cela. De la
même manière, Schopenhauer montre que l’homme se croit doté d’un pouvoir de choix parce qu’il ne voit
pas la chaîne causale qui le détermine. Il a ainsi l’illusion d’une absence de causes déterminantes, den
Augenschein der Ursachlosigkeit. Schopenhauer partage avec Spinoza la thèse d’un
déterminisme absolu. Il s’accorde ainsi avec lui sur la définition négative et sur la définition positive de la
liberté humaine: la liberté n’est pas le libre arbitre qui n’est qu’une illusion; elle consiste dans le fait de
développer son essence sans contrainte, en étant « déterminé par soi seul à agir ».

\subsubsection{La question du salut}

Lorsqu’il présente sa conception du salut suprême, Schopenhauer se réclame encore de Spinoza. Pour
Schopenhauer, l’homme se libère de la volonté active, du vouloir-vivre douloureux par le développement
d’une connaissance immédiate et désintéressée des « idées ». L’idée représente selon lui l’essence de chaque
genre d’êtres, pierre, arbre, animal, être humain, et correspond à un degré d’objectivation de la Volonté. Ainsi
la pierre objective la Volonté par sa résistance, l’animal par ses instincts, l’homme par une rationalité
conditionnée par le désir. La contemplation désintéressée de l’idée ou essence éternelle des genres d’êtres
offre le salut, parce qu’elle correspond au moment « où la connaissance se libère du service de la Volonté »,
où l’individu incarné se mue en « sujet connaissant pur, affranchi de la volonté, de la douleur et du temps ».
Cette transformation de l’individu soumis aux conditions de l’espace et du temps, soumis à l’espoir et à la
crainte, en sujet connaissant pur, considérant l’essence éternelle des choses, Schopenhauer la repère
précisément dans le Livre V de l’Éthique: « C’était aussi ce que, petit à petit, Spinoza découvrait, lorsqu’il
écrivait: Mens œterna est, quatenus res sub œternitatis specie concipit » . C’est bien un apaisement
comparable à celui conçu par Spinoza que vise Schopenhauer dans la contemplation des choses sub
œternitatis specie.
Aussi, par exemple, la contemplation de mon essence éternelle, celle qui qualifie le genre humain, offre-t-
elle un dépassement de mon individualité incarnée, en proie aux passions, et notamment à la crainte de la
mort. La destruction de mon existence dans la mort n’atteint pas mon essence véritable.
La mort est incapable de m’anéantir si je prends conscience de ma nature ou essence « primitive et
éternelle ». Schopenhauer s’autorise encore de Spinoza sur ce point capital, commentant le célèbre passage:
Sentimus experimurque nos œternos esse: pour se croire impérissable en effet,
l’homme doit considérer que son essence, contrairement à son existence incarnée, est sans commencement.
Cette essence correspond à la manière dont la Volonté s’objective dans l’être humain en
général, abstraction faite de tout ce qui caractérise mon individualité. Pour Spinoza en revanche, ma véritable
essence est toujours une essence singulière. Une fois de plus, l’individu, pour Schopenhauer, est inessentiel,
ma substance véritable n’ayant rien à voir avec mes caractéristiques singulières, tandis que pour Spinoza au
contraire la substance n’est rien, indépendamment des modes singuliers qui l’expriment.
La séparation introduite par Schopenhauer entre la substance (la Volonté) et l’individu conduit à un second
point de divergence entre nos deux auteurs. Pour Schopenhauer, « il est de l’intérêt de la Volonté que la
pensée s’exerce le plus possible ». En effet, plus sa pensée se
développe, plus l’individu est capable de prévoir, de s’adapter au monde, de se conserver et d’affirmer la
Volonté. Or l’« intérêt de la Volonté » n’est justement pas le même que celui de l’individu. L’intérêt de la
Volonté est la conservation de l’espèce: elle veut que les individus se conservent pour pouvoir continuer à
vouloir aveuglément dans l’espèce, au mépris de la souffrance individuelle. L’individu souffre tant qu’il se
tient au service de la Volonté, et doit s’en libérer pour atteindre le salut. Pour Spinoza en revanche, le salut de
l’individu est inséparable de l’affirmation de son conatus.
Nous avons affaire à deux conceptions opposées du salut individuel, liées à deux conceptions différentes
de la satisfaction: pour Schopenhauer, la satisfaction de la Volonté n’est que la suspension provisoire de la
souffrance, du manque, et conduit bientôt à l’ennui, tandis que pour Spinoza, « ce qui satisfait un désir » est
« ce qui conduit à la joie ». Pour Schopenhauer, l’affirmation de la Volonté est source
de souffrance pour l’individu, pour Spinoza, l’affirmation du conatus est source de joie.
Il s’agit bien dans les deux doctrines de se libérer par la connaissance, mais on ne se libère pas de la même
chose. Pour Spinoza, l’homme se libère de la passivité et de la tristesse en développant sa puissance de
connaître, en affirmant son conatus. Pour Schopenhauer au contraire, l’homme se libère du vouloir-vivre
douloureux. Aussi l’apaisement obtenu n’est-il pas le même ici et là. L’apaisement résultant du troisième
genre de connaissance est pour Spinoza dans le prolongement du conatus, puisqu’il correspond à son plus
haut degré d’effectuation. Il s’agit d’un sentiment positif. Pour Schopenhauer au contraire, l’apaisement est
négatif, puisqu’il correspond à la cessation de la Volonté.C’est précisément contre la thèse de l’affirmation
du conatus, assimilé par Schopenhauer à l’égoïsme et au
mal, que se développe la critique schopenhauerienne de l’« immoralisme » de Spinoza.

\subsection{La morale}

En dénonçant l’immoralisme de Spinoza, Schopenhauer règle une fois de plus ses comptes avec les « trois
sophistes » mentionnés ci-dessus. Le renouvellement du spinozisme et du panthéisme en nos jours, écrit-il,
« a entraîné ce profond abaissement de la morale ». En effet, « on tend à placer dans la jouissance et le bien-
être la fin dernière de l’existence humaine ». La critique de Spinoza permet d’une part à
Schopenhauer de dénoncer l’immoralisme des doctrines panthéistes - à commencer par celle de Hegel ;
d’autre part, elle répond à son souci constant de marquer l’originalité, sans doute suspecte à
ses propres yeux, de sa doctrine par rapport à celle de Spinoza. La critique schopenhauerienne de la morale
de Spinoza, dont l’enjeu n’est rien de moins que de renverser le spinozisme, repose cependant sur deux
contresens, deux erreurs d’interprétation.

\subsubsection{Optimisme et immoralisme}

Schopenhauer déduit l’immoralisme de Spinoza de son optimisme supposé. Le panthéisme, déclare
Schopenhauer, « est optimiste par essence ». En effet, si Dieu constitue l’essence intime
du monde, cela signifie que « le monde avec tout son contenu est parfait et tel qu’il doit être ». En conséquence, l’homme n’a plus qu’à jouir de l’être, qu’à rechercher son propre avantage.
Suum utile quœrere: tel est, selon Schopenhauer, le « principe égoïste » de Spinoza, qui
consacre son immoralisme. Pour Schopenhauer en effet le critère de l’action morale est l’« absence de tout
motif égoïste ».
Le mal, pour Schopenhauer, c’est précisément le conatus, l’effort par lequel chaque individu cherche à
maintenir son existence, à augmenter sa joie. Cet effort pour se conserver, que Schopenhauer retraduit en
vouloir-vivre, définit l’égoïsme: « Mais l’égoïsme est-il autre chose que la volonté d’entretenir son existence,
le vouloir-vivre? » La problématique morale de Schopenhauer, qui est de savoir quel type de motif est
susceptible de contrecarrer l’intérêt égoïste, suppose que l’égoïsme soit reconnu comme un mal, ce que
précisément Spinoza n’admet pas. Schopenhauer donne un statut positif au mal, que Spinoza tient pour une
simple notion subjective. Substituant aux notions de bien et de mal celles de bon (ce qui cause la joie) et de
mauvais (ce qui cause la tristesse), Spinoza échappe à l’alternative de l’optimisme et du pessimisme.
Schopenhauer reconnaît d’ailleurs sur ce point l’originalité de son adversaire, qui n’accorde aucune valeur
objective aux notions de bien et de mal: « Spinoza déclare purement conventionnelle, nulle en soi, toute
distinction entre le juste et l’injuste, et plus généralement entre le bien et le mal ».
Dès lors, on ne comprend pas comment Schopenhauer peut prétendre renverser le spinozisme. Ce
renversement consiste selon Schopenhauer à remplacer l’optimisme, qui affirme la positivité du bien, par le
pessimisme, qui affirme la positivité du mal. Or, si le bien et le mal n’ont chez Spinoza
aucun statut ontologique, comment Schopenhauer peut-il qualifier sa doctrine d’« optimisme »? Comment
peut-il affirmer que le Dieu de l’Éthique, auquel Spinoza « a seulement enlevé la personnalité »,
« s’applaudit de sa création et trouve que tout a tourné pour le mieux »? Pour Spinoza, si le mal
n’existe pas, ce n’est pas parce que le Bien est et fait être, ce n’est pas parce que tout est pour le mieux dans
le meilleur des mondes (Spinoza n’est pas Leibniz). C’est au contraire, comme Schopenhauer l’a pourtant
bien vu, parce que « toute distinction entre le bien et le mal est nulle en soi ».
Il nous semble que Schopenhauer déforme à dessein la doctrine de Spinoza en la qualifiant d’optimisme,
pour pouvoir se présenter comme étant le seul à l’avoir renversée, et marquer par là même son originalité.

\subsubsection{La compassion}

L’ontologie de Spinoza aurait dû, d’après Schopenhauer, le conduire à une morale de la compassion.
Comme on sait, Schopenhauer réserve toute puissance et, partant, toute réalité à la
Volonté. Seule la Volonté est réelle, et le multiple n’en est que la manifestation illusoire: « Toute multiplicité
est une apparence: tous les individus de ce monde, coexistants et successifs, si infini qu’en soit le nombre, ne
sont pourtant qu’un seul et même être, qui, présent en chacun d’eux, et partout identique, seul vraiment
existant, se manifeste en tous ».
On a vu que Schopenhauer attribuait à tort cette thèse, selon laquelle toute multiplicité est apparente, à
Spinoza: « Spinoza ne voyait en elles [les choses] que les accidents de la substance unique existant seule
éternellement ». Cette identité substantielle de tous les êtres aurait dû, selon Schopenhauer,
conduire Spinoza à développer le thème d’une compassion universelle: tout être souffrant ne se distinguant
pas, fondamentalement, de moi-même doit m’inspirer la compassion et m’inciter à alléger sa souffrance.
L’absence de compassion de Spinoza envers les animaux, que Schopenhauer repère en Ethique, IV, 37,
scol., lui apparaît ainsi comme une contradiction, une fausse conséquence du monisme spinoziste.
Mais il est clair que Schopenhauer se méprend entièrement sur la nature du rapport entre la substance et
les modes. Pour preuve, la parenté qu’il croit pouvoir établir, dans Le fondement de la morale, entre Spinoza
et le bouddhisme. Schopenhauer croit pouvoir retrouver chez Spinoza l’opposition entre l’Identité suprême
de l’Un (tad êkam), seule réalité authentique, et le multiple, que le courant bouddhiste de l’hindouisme
présente comme une illusion (le voile de Maya).
Or les modes, on l’a vu, ne sont pas la projection fantastique d’une substance unique qui serait seule réelle.
L’idée de mode ne sert jamais à retirer toute puissance à la créature, étant entendu que les modes participent
à la puissance de Dieu, sont des parties de la puissance divine, et des parties singulières. C’est cette
consistance et cette singularité du mode qu’occulte entièrement Schopenhauer. Spinoza résiste à la thèse de
l’indifférenciation essentielle des êtres, que Schopenhauer lui prête pour mieux pouvoir dénoncer une
contradiction dans sa doctrine. Comme le fait justement observer Max Grunwald, « la vanité de
Schopenhauer se met en campagne de la manière la plus violente contre celui qui menace de lui faire perdre
quelque chose de son originalité, et dans un emportement aveugle il déniche des contradictions, là où en
réalité il ne s’en trouve aucune ». Bien plus, il nous semble que la contradiction réside plutôt dans la
position de Schopenhauer lui-même, qui présente l’action accomplie par compassion comme désintéressée.
La définition schopenhauerienne de la compassion comme identification à autrui reprend une idée
spinoziste, à savoir la représentation d’autrui souffrant comme semblable à moi.
Cependant, Spinoza considère que la pitié, comme passion triste, est mauvaise et inutile, puisque l’on peut
obtenir le bien qui en résulte, à savoir la suppression de la tristesse d’autrui, par la
seule raison: « le bien qui en résulte, c’est par la seule raison que nous désirons le faire ».
Tandis que Spinoza envisage la valeur de l’action du point de vue de son résultat, Schopenhauer considère
que seule la pureté de l’intention lui confère une valeur. La raison, comme calcul intéressé, ne saurait, quel
que soit le résultat de l’action qu’elle commande, conférer à celle-ci une valeur morale. Seule la
compassion, qui repose sur l’intuition de l’identité fondamentale entre moi et autrui, permet d’expliquer une
action pourvue d’une valeur morale: je n’agis plus égoïstement, de manière intéressée, mais uniquement en
vue du bien d’autrui. La raison servante du désir, toujours ordonnée à l’intérêt individuel - quand bien même
celui-ci composerait avec celui d’autrui -, perd avec Schopenhauer ses droits au plan moral.
Seulement, objectera-t-on, peut-on dire que la compassion m’incite à supprimer la souffrance d’autrui au
mépris de mon propre intérêt ? N’implique-t-elle pas la suppression de ma propre souffrance, une fois que je
me suis identifié à autrui ? En définitive, on peut douter du caractère désintéressé de l’acte accompli par
compassion, puisqu’il revient à la suppression de ma propre souffrance. En soulageant autrui, je sers mon
intérêt, puisque c’est moi-même, fondamentalement, que je soulage.
Schopenhauer est-il le seul continuateur de Kant en Allemagne? Échappe-t-il au trait caractéristique de la
philosophie allemande postkantienne, qui est de procéder à un « ajustage » du spinozisme?
S’agissant de la théorie de la connaissance, il est clair que Schopenhauer se démarque nettement de
Spinoza en reprenant certains éléments de la critique kantienne (l’irréductibilité de la philosophie aux
mathématiques, le statut idéal de l’étendue ou espace comme forme de la représentation, la distinction entre
le phénomène et la chose en soi). Mais Schopenhauer se démarque également de Kant en réduisant les
phénomènes à des apparences, et en prétendant accéder à la chose en soi par l’intuition.
Au plan moral, Schopenhauer s’oppose à l’eudémonisme de Spinoza, et s’accorde avec Kant pour
caractériser l’action morale par le désintéressement, mais il s’oppose aussi bien à Kant qu’à Spinoza en
destituant la raison au profit du sentiment de compassion, seul motif censé pouvoir contrecarrer l’égoïsme.
Au plan métaphysique en revanche, Schopenhauer soutient à l’évidence des thèses beaucoup plus proches
de celles de Spinoza que de celles de Kant: l’unicité de l’être, le désir comme élément premier en l’homme,
la subordination de l’activité représentative à l’effort (Anstrengung) pour persévérer dans l’existence, la
quête du salut dans le dépassement des déterminations spatio-temporelles de l’existence.
Bien qu’elle se présente comme la continuation du kantisme, la philosophie de Schopenhauer, comme le
dit M. Grunwald, « se trouve être une pousse sur le tronc du spinozisme », et « nous pouvons intégrer sa
philosophie [celle de Schopenhauer] à la chaîne des systèmes qui part de Spinoza ». Aussi sommes-nous
conduits à conclure, avec Clément Rosset, que « le plus grand tort de Schopenhauer est très probablement de
s’être cru kantien ».

\subsection{Notes}

Les références des textes non traduits de Schopenhauer sont données d’après l’édition allemande des
Sämtliche Werke (désormais SW), établie par W. E von Lohneysen, Stuttgart, Suhrkamp Taschenbuch, 1986,
5 tomes. Nous traduisons les passages cités et indiquons le tome et la page. Les références à Spinoza sont
données d’après l’édition des Œuvres complètes, Gallimard, « La Pléiade ».

Le monde comme volonté et comme représentation, trad. française de A. Burdeau
revue et corrigée par R. Roos, PUE.

Voir par exemple De la quadruple racine du principe de raison suffisante, trad. de J. Gibelin, Vrin.

Parerga und Paralipomena, SW, IV : « Puisque je suis moi-même kantien » ;  « Ma
philosophie n’est que l’aboutissement (das Zu-Ende-Denken) de la sienne [de celle de Kant]. » Dans De la
volonté dans la nature, trad. de Édouard Sans, PUF « Quadrige », Schopenhauer se
présente comme un « continuateur » de Kant, au sens où « l’on ne voit plus trace » dans sa doctrine de
théologie et de psychologie rationnelles.
 
dans la philosophie allemande post-kantienne un passage obligé. Kant laisse en effet à ses successeurs la
tâche de penser l’unité systématique, d’assumer une unité qui n’est jamais achevée dans sa doctrine critique :
« Penser après Kant, c’est donc revenir à Spinoza », Totalité et subjectivité. Spinoza dans l’idéalisme
allemand, Vrin.

L’expérience fondamentale est celle de mon corps comme réalité voulante.
La chose en soi ne peut nous être révélée de l’extérieur, où l’on ne perçoit que l’apparence, mais de
l’intérieur, grâce à l’expérience intime de l’intuition. Le sentiment intérieur que nous avons de nous-mêmes
nous révèle, à travers le désir et le corps, la réalité même du monde, c’est-à-dire la Volonté saisie de
l’intérieur.

Il convient de préciser que Schopenhauer n’entend pas par intuition exactement la même chose que Kant.
L’intuition correspond chez Kant au versant matériel de la représentation. Or, pour qu’il y ait intuition, selon
Schopenhauer, la matière ou la sensation ne suffit pas. C’est seulement lorsque les modifications de notre
corps sont rapportées à leurs causes par l’intellect qu’on a l’intuition de ces dernières comme objets. Faute de
cet acte de l’entendement pur, il n’y aurait qu’une conscience sourde et comme végétative des modifications
de l’objet immédiat. L’intuition n’est donc pas seulement sensible, mais aussi intellectuelle, puisqu’elle
suppose la loi de causalité : « C’est cette loi qui, d’une manière primitive et absolue, rend possible toute
intuition, par suite toute expérience ».

Schopenhauer cite comme modèle de philosophe dogmatique, « négligeant la connaissance intuitive »,
Spinoza, « dont la méthode est de démontrer par concepts ». Il nomme en ce sens les hégéliens des
« spinozistes », et juge ainsi leur manière de philosopher : « Cette algèbre de simples concepts qu’aucune
intuition ne vérifie est condamnée à l’erreur ».

Le monde, où Schopenhauer explicite la distinction entre la
synthèse et l’analyse : « Je pars de l’expérience et de la conscience de soi naturelle, donnée à chacun, pour
arriver à la volonté, mon seul élément métaphysique : je suis ainsi une marche ascendante et analytique. Les
panthéistes au contraire prennent, à l’inverse de moi, la voie descendante et synthétique ; ils partent de leur
Dieu, que, parfois sous le nom de substantia ou d’absolu, ils obtiennent de nous par leurs instances ou nous
imposent, et c’est cet être entièrement inconnu qui doit expliquer par la suite tout ce qui est connu. »

« L’étendue n’existe que dans la représentation, et ne lui est
donc pas opposée mais subordonnée. »

On sait que Schopenhauer réduit à la causalité les douze catégories kantiennes, dont onze seraient
« comme de fausses façades sur une fenêtre ».

Ces points sont abordés dans la communication de Bernard Rousset.

Schopenhauer s’explique ainsi : « Là où elle [la philosophie] a le moins sa place, c’est dans les
universités, où la faculté de théologie occupe le premier rang, c’est-à-dire que les choses y sont réglées avant
même que la philosophie paraisse [...]. La philosophie universitaire actuelle [...] est l’ancilla larvée, et
destinée aussi bien que la scolastique à servir la théologie ».

Par exemple, Schopenhauer prétend atteindre la chose en soi par l’intuition, alors que celle-ci, selon Kant,
nous laisse au plan du phénomène. Il considère le monde phénoménal comme une illusion (Schein) et se croit
en cela fidèle à Kant, alors que celui-ci au contraire distingue entre Schein et Erscheinung. Kant considère le
phénomène non comme une illusion mais comme la chose même telle qu’elle se manifeste à nous, comme
une Erscheinung de la chose.

On peut parler d’une métaphysique de Schopenhauer qui, s’il renonce à la théologie et la psychologie
rationnelles, déploie en revanche une cosmologie. Contrairement à Kant, Schopenhauer dépasse le plan
phénoménal, celui de la représentation, pour celui de l’être en soi, la Volonté, « mon seul élément
métaphysique ».
Otrun Schulz, Wille und Intellekt bei Schopenhauer und Spinoza, Peter Lang, Francfort-sur-le-Main ; recensé par Florence Albrecht, Archives de philosophie.

Voir De la volonté dans la nature, où Schopenhauer reprend à son compte la définition spinoziste
de la liberté : « C’est avec raison que Spinoza dit : Ea res libera dicitur, quœ ex sola suœ natura necessitate
existit, et a se sola ad agendum determinatur ».

Schopenhauer emprunte à Spinoza le concept d’expression pour traduire le rapport de la Volonté aux
phénomènes : « Tout ce qui est est l’expression de la Volonté ». La raison de cet emprunt
est la suivante. Contre le courant idéaliste engagé par Fichte, Schopenhauer entend conserver la chose en soi
qui marque les limites du connaître humain en le faisant dépendre d’une chose qui affecte l’âme. En ce sens,
il demeure kantien contre les « trois sophistes », qui affranchissent le sujet connaissant de sa dépendance à
l’égard d’une chose extramentale. Mais, contre Kant, Schopenhauer dénonce l’application de la catégorie de
causalité à la chose en soi. La définition de la chose en soi comme « cause non sensible du phénomène »
est illégitime, puisque l’usage légitime des catégories est restreint au plan phénoménal.
Pour penser le lien entre la Volonté et les phénomènes, Schopenhauer utilise le concept
spinoziste d’expression, qui permet de concevoir non pas un rapport entre deux choses distinctes, mais une
même chose envisagée sous deux aspects différents. Ce que Schopenhauer retient de l’idée spinoziste
d’expression, c’est la possibilité qu’elle offre de penser une identité essentielle entre la substance (la Volonté)
et les modes (les phénomènes). La relation d’expression entre la chose et ses phénomènes dispense
Schopenhauet d’établir entre eux, comme le faisait Kant, un rapport de cause à effet. Mais il convient de
souligner que Schopenhauer fait subir une torsion à l’idée spinoziste d’expression. Chez Spinoza
l’expression est indissociable de la causalité puisque c’est la causa sui qui s’exprime dans les modes en les
causant. Chez Schopenhauet en revanche, l’expression est arrachée à la causalité, la Volonté s’exprimant
dans les phénomènes sans les causer.

Schopenhauer reconnaît sur ce point sa dette à l’égard de Spinoza, sans trop y insister toutefois. A ceux
qui prétendent qu’il s’est inspiré de Schelling, il répond que si l’on tient à lui trouver des prédécesseurs, il
faut plutôt se reporter, entre autres auteurs, à Spinoza. Et il cite précisément les passages du Livre III de
l’Éthique se rapportant à la théorie du conatus et du désir comme essence de l’homme.

La Volonté, comme effort (Anstrengung) pour conserver, « met en mouvement l’association des idées, [...]
pousse l’intellect, son serviteur, à coordonner ses pensées ».

Dans les Parerga  Schopenhauer réaffirme que le désir est primordial dans l’homme et il
cite l’Éthique pour appuyer sa thèse : « L’appétit n’est rien d’autre que l’essence même de l’homme, et de la
nature de cette essence suivent nécessairement les choses qui servent à sa conservation » (Ethique, III, souligné par Schopenhauer).

Pourquoi pas dix fois sur dix ? Schopenhauer se distingue assurément de Spinoza par son pessimisme.

Peut-être s’agit-il d’ailleurs moins d’un renversement que d’une simple inversion du sens des thèses
maîtresses de Spinoza. C’est l’avis d’Elisabeth de Fontenay : « Le compte dont il [Schopenhauer] doit
s’acquitter, le compte donc qu’il a à régler avec le grand Juif immanentiste est considérable : conatus comme
formulation positive et matérialiste dont le vouloir-vivre, malgré l’appareil kantien, ne fait, peut-être,
qu’inverser les signes » ; « La pitié dangereuse », dans Présences de Schopenhauer, loc. cit., p. .

Nous traduisons : « [...] Si ce monde est un Dieu, alors il est à lui-même sa
propre fin et doit s’honorer et se réjouir de son existence. Alors "saute, Marquis !" toujours joyeux, jamais
triste. »

De la volonté dans la nature, op. cit., p. .

Notons en effet qu’à chaque fois qu’il s’attache à manifester l’originalité de sa doctrine, Schopenhauer la
compare à celle de Spinoza : voir notamment Le monde, p. -, et Parerga, SW, IV, p.  sqq.

Soulignons que l’antisémitisme de Schopenhauer conditionne, notamment sur la question des animaux, sa
lecture de Spinoza. Schopenhauer prête au judaïsme un profond mépris pour les animaux, renvoyant aux
chapitres I et IX de la Genèse. Ainsi, commentant le texte suivant de Spinoza : « Ce
qui se trouve dans la Nature, les hommes mis à part, la norme de notre utilité ne demande pas que nous le
conservions, mais elle nous conseille de le conserver pour divers usages, de le détruire, ou de l’adapter par
tous les moyens à notre usage », Schopenhauer quitte son rôle dephilosophe « critique » et invoque la « puanteur
juive » (foetor judaicus) de Spinoza. Dans
les remarques qui "émaillent son exemplaire de l’édition Paulus des œuvres de Spinoza, on trouve également
plusieurs insultes antisémites à l’adresse du Juif athée : cf. Arthur Schopenhauer. Der handschriftliche
Nachlass, vol. V, édité pat Arthur Hübscher, , Francfort-sur-le-Main, notamment.
Schopenhauet se reconnaîtrait-il dans ces mélancoliques dénoncés par Spinoza, qui « méprisent les hommes
et admirent les bêtes » ?
 On peut saisir ici sur le vif l’une des erreurs auxquelles conduit l’interprétation de Spinoza comme
moniste. Sut cette question, on se reportera à l’article de Pierre Macherey, « Spinoza est-il moniste ? », dans
Spinoza : puissance et ontologie, Actes du colloque organisé par le Collège international de philosophie en
mai 1993, Éditions Kimé.

Max Grunwald, Spinoza in Deutschland, Berlin, , p. -. C’est nous qui traduisons.

Dans le Traité théologico-politique, chap. XVI, Spinoza écrit que les lois de la raison servent l’« intérêt
véritable des hommes ». Ce chapitre est présenté par Schopenhauer comme le « véritable résumé de
l’immoralité de Spinoza ». Dans Le fondement de la morale, Schopenhauer
imagine que Spinoza ait à expliquer le phénomène suivant : Caïus renonce à tuer son rival amoureux alors
qu’il est pourtant à l’abri de représailles. L’explication que Spinoza mettrait dans la bouche de Caïus serait,
selon Schopenhauer, la suivante : « Rien de plus utile à l’homme que l’homme lui-même : c’est pourquoi je
n’ai pas voulu tuer un homme ». La réponse de Spinoza s’appuierait donc sur un calcul de l’utilité :
je n’épargne pas autrui pat pitié pour lui, mais parce que ma raison me fait voir que mon intérêt est de
composer avec sa puissance. Schopenhauer oppose alors à l’utilité, mobile de l’action selon Spinoza, la
compassion désintéressée, comme étant le motif le « plus pur », fournissant le « fondement de la morale ».

Loc. cit., respectivement p. , .

Clément Rosset, Schopenhauer, philosophe de l’absurde, PUF.

Auteur
Christophe Bouriau
