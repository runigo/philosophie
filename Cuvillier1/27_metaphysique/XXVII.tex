\chapter{La connaissance métaphysique}
%chapitre XXVII
%LA CONNAISSANCE MÉTAPHYSIQUE
%329. L'idée d’une connaissance métaphysique. — 330. La « philosophie première »
%d’après Aristote. — 331. Position actuelle du problème. — 332. Le
%fondement de la Connaissance. — 333. Le fondement de l'Action. — 334. Caractère
%de la connaissance métaphysique.

\section{L'idée d’une connaissance métaphysique}% 329.
Nous avons observé à plusieurs reprises (voir notamment \S 53-56) qu’on peut
distinguer au moins deux étages de la connaissance humaine : 1° la
{\it connaissance vulgaire}, celle du « sens commun », essentiellement
{\it empirique} et, en grande partie, {\it sensible} ; — 2° {\it la connaissance scientifique},
qui est déjà d’un niveau plus élevé : elle ne se contente plus de
« l’expérience vague » ; elle n’est plus empirique, mais {\it expérimentale},
c’est-à-dire qu’elle fait appel à une expérience méthodiquement
conduite, organisée et systématisée ; elle est, en grande partie, une
reconstruction {\it intelligible}, donc {\it conceptuelle}, du réel ; elle ne se borne
plus à constater, mais s’efforce d'{\it expliquer}, et elle le fait le plus souvent
à l’aide de relations générales, de {\it lois} qu’elle établit entre les faits.

On est en droit de se demander si, {\it au-dessus} de ces deux types de
connaissance, il n’en existe pas une troisième, capable de nous faire
pénétrer plus profond dans l’explication des choses et de nous fournir
un contact plus intime avec le réel. Inutile, en effet, d’insister sur les
insuffisances de la connaissance vulgaire. Mais la connaissance scientifique
elle-même a ses limites. On vient de le voir (\S 317) : la vérité
scientifique est toujours une vérité {\it construite}, et cette construction,
sans être arbitraire, repose cependant sur certains postulats, sur certains
principes, qui ne sauraient prétendre à une valeur absolue. Certes,
dans les Sciences expérimentales, ces constructions sont toujours
%580
astreintes à venir s’éprouver au contact du {\it réel} et elles peuvent
atteindre à une certaine {\it objectivité} (\S 318). Mais il ne s'agit pas ici d’un
réel absolu : il s’agit d’une réalité connue {\it à travers} certains procédés
d'observation et de mesure, et il apparaît nettement aujourd’hui que le
déterminisme de ces procédés intérfère toujours plus ou moins avec
le déterminisme propre au phénomène étudié (\S 271). Quant à l’explication
scientifique, sans exclure totalement la notion de {\it cause} (\S 267),
elle use plus souvent de la notion de {\it loi} (\S 265-266) ; or la loi n’est qu'un
{\it rapport} entre les faits ou plutôt entre certains éléments des faits. Bref,
la connaissance scientifique est une connaissance {\it relative}.

\section{La « Philosophie première » d’après Aristote}% 330.
On comprend dès lors qu’il y ait place, au delà de cette connaissance relative,
pour une connaissance {\it métaphysique} qui s’efforcerait d'atteindre
l'{\it absolu}. — Déjà Aristote avait distingué, des « philosophies » —
c’est-à-dire des sciences — particulières, une discipline qu'il avait
nommée \textbf{\textit {philosophie première}} (le terme de {\it métaphysique} ne fut inventé
que plus tard) et à laquelle il assignait un triple objet.

1° Chacune des autres « philosophies », remarquait-il, étudie un
objet particulier, une {\it espèce} de l’être. Ainsi la mathématique est la
science de la quantité ; la « physique » est la science de la « nature »,
c'est-à-dire de « ce genre de l’être qui porte en soi le principe de son
mouvement ». Seule la philosophie première étudie l'\textbf{\textit {être absolu}} ou
\textbf{\textit {l'être en tant qu’être.}}

2° Par suite, la philosophie première pénètre plus profondément dans
la connaissance des causes que les autres branches du savoir. Elle porte
sur « les causes premières et les principes », c’est-à-dire sur les causes
qui se suffisent à elles-mêmes, qui ne sont pas les effets d’autres causes,
et sur les propositions d’où découlent toutes les autres.

3° Mais, pour Aristote, la causalité ne s’épuise pas dans la pure
efficience, dans la causalité mécanique. La cause d’un être et, plus que
toute autre, la cause première, le premier moteur immobile d’où
dépend tout mouvement, tout changement, est une cause finale vers
laquelle tend tout ce qui est, — de sorte que la philosophie première
est la recherche des {\it fins dernières} des êtres et des choses: beaucoup plus
que de leurs origines au sens moderne du mot.

\vspace{0.44cm}
\begin{minipage}[c]{.45\linewidth}
\begin{center}
\includegraphics[scale=0.3]{./27_metaphysique/092}
\end{center}
\end{minipage}
\hfill
\begin{minipage}[c]{.45\linewidth}
\begin{center}
La métaphysique telle qu'on la représentait au
{\footnotesize XVIII}$^\text{e}$ siècle.
\end{center}
\vspace{0.31cm}

\hspace{0.91cm}{\it L'}Iconologie {\it de Cochin et Gravelot
(1796), d'où est extraite cette figure,
l'accompagne de ce curieux commentaire :
« Le bandeau placé au-dessous
de ses yeux, sans lui dérober la lumière d'en haut,
l'empêche seulement
de regarder vers le globe de la Terre ;
elle se couvre d'une partie de sa draperie
pour ne s'occuper que de la contemplation
des objets célestes. » On
remarquera aussi qu'elle foule aux
pieds une horloge, ce qui signifie sans
doute que son objet se situe en dehors
du temps.}
\end{minipage}

\section{Position actuelle du problème}% 331.
Bien qu'il ait été
ainsi posé en termes fort abstraits, le triple problème posé par Aristote
est loin d’être aujourd’hui périmé. — 1° D’abord, par delà les réalités
si variées dont la Science, à vrai dire, nous fait surtout connaître les
%581
modalités, il reste à la Métaphysique à s'interroger sur leurs essences
mêmes. Ainsi, savoir quelles sont les propriétés de tel ou tel corps
chimique, l’hydrogène ou le phosphore par exemple, est
un problème scientifique, et la Métaphysique n’a pas de
problème à se poser touchant ces corps en particulier. Mais
elle peut se poser un problème concernant l’essence de
la {\it Matière} en général. De même, il n’y a pas de problème
métaphysique qui se pose au sujet des fonctions
spéciales de la Vie comme la nutrition, la respiration,
le fonctionnement du système nerveux, etc. Mais il
y a un problème métaphysique concernant la nature,
l'essence même de {\it la Vie} en général. De même encore, la
Psychologie comme science étudiera chacune des fonctions
psychiques en particulier ; elle pourra même nous
donner une idée d’ensemble de la vie de l’esprit. Mais il
appartiendra à la Métaphysique de s'interroger sur la
nature intime de l'Esprit, sur ses rapports avec le
corps, etc. Ce sont précisément ces problèmes que nous
étudierons dans le chapitre suivant. — 2° Il en est de
même pour le problème des causes. Les causes que détermine
la Science (quand elle use de ce concept de {\it cause})
sont toujours des causes
{\it secondes}, c’est-à-dire qu’elles sont elles-mêmes les effets d’autres
%582
causes, et celles-ci à leur tour l’effet d’autres causes antérieures, et
ainsi de suite à l'infini. La Métaphysique est en droit de se demander
s’il est impossible de remonter jusqu’à une cause vraiment {\it première},
qui se suffise à elle-même et qui ne soit l’effet d'aucune autre cause.
3° Enfin il est un dernier problème qui semble appartenir à la Métaphysique
plus qu’à la Science : c’est celui de la {\it finalité}. On a vu
(\S 264, etc.) que, sauf peut-être dans certains cas spéciaux, comme
celui des Sciences biologiques (\S 276), la Science n’use guère de la
notion de {\it fin}. Elle est, en tous cas, incompétente pour résoudre le
problème des fins {\it dernières} de l’homme, c’est-à-dire de sa {\it destinée}
(tome II, chap. XXVI)

\section{Le fondement de la Connaissance}% 332.
Mais le problème
métaphysique peut aussi être posé sous une forme plus moderne, et,
cette fois encore, c’est la Science elle-même qui nous y conduit. La
Science, en effet, est {\it œuvre de l'esprit}. Mais l'esprit n’est pas ici un
simple « instrument ». Ce serait supposer que la Science a des fins
purement pratiques. Or on a vu qu’il n’en est rien : la Science est une
connaissance désintéressée (\S 238). Quelles que soient ses limites, que
nous venons de rappeler, elle a pour but de satisfaire un besoin proprement
intellectuel : le besoin de comprendre, et c’est en ce sens
qu’elle est une authentique « expérience spirituelle » (\S246). Les
Sciences physiques sont l’esprit s'imposant à la matière, au monde
extérieur, et les rendant compénétrables à l’intelligence humaine. Les
Sciences de l’homme sont la pensée ne se contentant plus de se « vivre »
elle-même, mais cherchant à se connaître vraiment, à se comprendre.
Or comment cela est-il possible? Comment l’esprit peut-il parvenir
à dominer la matière et à se comprendre lui-même? La pensée ne
semble-t-elle pas nous apparaître ainsi comme une {\it réalité première},
conformément au {\it Cogito}, au « Je pense, donc je suis » de Descartes qui,
après avoir révoqué en doute toutes les existences qu’il avait admises
jusque-là, s'aperçoit qu’il y a au moins une de ces existences qu’il ne
peut nier et qui est impliquée dans son doute même : l’existence de
sa propre pensée ? Le problème de l’Être nous apparaît ainsi comme
le fondement même du problème de la Connaissance. Pour que celle-ci
soit possible, pour que les choses soient compénétrables à la pensée,
ne faut-il pas que l'Être en son essence intime soit connaturel à la
pensée, qu’il soit de nature spirituelle ? — Ici encore, nous arrivons
à cette conclusion que la Science ne se suffit pas. Au delà de la Science,
nous apercevons l'{\it esprit} au-dessus de son ouvrage, l’esprit qui fait la
science elle-même. Nous nous plaçons donc dans une toute autre
%583
perspective, qui est celle de la Métaphysique, puisqu'il s’agit de la
nature même de l’Être en soi. Comme Hegel l’a dit de Descartes, en
posant la pensée comme réalité première, on « reprend tout par le
commencement » et la Science elle-même ne nous paraît désormais
définissable qu’en fonction de la pensée.

\section{Le fondement de l’Action}% 333.
Ce n’est pas seulement le
problème de la Connaissance qui nous conduit à la Métaphysique.
C’est aussi, on le verra au tome II, celui de l’Action. Certes, nous
disposons, pour diriger nos actes, de la connaissance morale qui, sous
sa forme spontanée, nous est fournie par la {\it conscience} (tome II,
chap. XIII). Celle-ci suffit le plus souvent à nous donner le sens des
{\it valeurs}, à nous permettre de discerner le bien du mal, à nous faire
connaître nos devoirs, etc. Elle peut d’ailleurs atteindre un niveau plus
élevé avec la {\it réflexion} morale et principalement la réflexion philosophique.
La Morale philosophique est précisément la théorie raisonnée
de ces {\it valeurs} qui nous sont d’abord données dans la conscience et
qui dirigent notre conduite.

Toutefois il faut, ici encore, se demander si le problème ne doit pas
être posé à un niveau supérieur. — {\it A.} Les valeurs, en effet, tout en
étant transcendantes à l’ordre des faits, de la réalité empirique, ne
sont pas gratuites : elles doivent avoir un {\it fondement}, et comment
pourraient-elles en avoir un si elles étaient sans point d’appui dans
l'ordre de l’Être ? C’est pourquoi la Morale, si elle peut et doit, en
première démarche, se constituer indépendamment de la Métaphysique,
requiert cependant, lorsqu'elle veut remonter jusqu’à ses principes
premiers, un fondement métaphysique (tome II, chap. XXV).
— {\it B.} Si d’autre part nous remarquons qu’il n’y a de valeurs que pour
une conscience et que, par suite, les valeurs sont essentiellement
{\it choses d’esprit} (\S 18, et t. II, chap. XIII), il nous apparaît que la notion
de la primauté de l’esprit s’impose ici comme à propos de la connaissance
pure et que la métaphysique qui convient ici, ne peut être
qu’une {\it philosophie de l'esprit}.

\section{Caractère de la connaissance métaphysique}% 334.
Ainsi définie, la connaissance métaphysique ne diffère pas seulement de la
connaissance scientifique (et aussi, on le verra au tome II, de la
réflexion morale) par sa {\it généralité} plus élevée, par l’{\it esprit de synthèse}
qu’elle requiert, plus impérieusement encore que celle-ci. Elle en
diffère aussi par le fait que, ainsi que l’ont fait observer plusieurs
philosophes contemporains, les problèmes métaphysiques sont ceux
%584
où « le questionneur comme tel est lui-même pris dans la question »
(Heidegger)
{\scriptsize (L'idée avait déjà été indiquée dès 1910 par Lawrence-Pearsall
Jacks dans {\it The Alchemy of Thought})}
ou encore, comme l'écrit Jean Wahl, où « le problème
empiète sur nous-mêmes », de telle sorte qu’ « en même temps que les
questions que le philosophe pose, portent sur un horizon très général,
il est pris lui-même dans cet horizon. »

Une telle connaissance ne saurait donc être le fruit d’une pure
construction conceptuelle, et l’on s’explique sans peine que, lorsqu'il
s’agit de tels problèmes, Bergson (\S 326) ait récusé l'intelligence
discursive et préconisé à sa place une {\it intuition} qui nous permettrait
de « coïncider avec » l'élan même de la vie et de saisir la spontanéité
créatrice immanente à l’être. Certains auteurs contemporains sont
d’ailleurs allés plus loin encore. Cette aptitude métaphysique qui est
refusée à l'intelligence, ils l’ont attribuée au sentiment : « Le sentiment,
écrit J. Wahl, n'est-il pas tendance vers ce qui le comble, vers l’autre
que lui ? Transcendance et sentiment sont unis par nature, bien plus
que transcendance et raison le furent jamais. » Et le même auteur soutient
que la Poésie s'apparente à la Métaphysique parce qu’elle aussi
nous fait saisir l’universel, quoique sous une forme subjective et non
conceptuelle.

De fait, on peut dire que, depuis toujours, la pensée humaine, au
lieu d'emprunter, pour atteindre l’absolu, les voies austères et parfois
décevantes de la connaissance intellectuelle qui laisse encore le {\it sujet}
connaissant en dehors de l’{\it objet} connu, a prétendu surmonter cette
dualité par la « possession intime et complète de l’objet ». Si l’on en
croit Lucien Lévy-Bruhl, c’est ce besoin d’une {\it communion} totale,
d’une {\it fusion} avec l’objet à connaître qui fait « le ressort principal des
doctrines dites anti-intellectualistes. Ces doctrines reparaissent périodiquement,
et à chaque réapparition elles retrouvent faveur. Car elles
promettent ce que ni la science positive pure ni les autres doctrines
philosophiques ne peuvent se flatter d’atteindre : le contact intime et
immédiat avec l'être, par l'intuition, par la compénétration, par la
communion réciproque du sujet et de l’objet, par la pleine participation,
en un mot, que Plotin a décrite sous le nom d’extase. » Besoin
qui, selon Lévy-Bruhl, demeure, chez l’homme, « plus impérieux et
plus intense que le besoin de se conformer aux exigences logiques »,
car « il est plus profond, il vient de plus loin », il vient des origines
mêmes de l’humanité.

%585
Mais nous avons vu ci-dessus (\S 326, fin) quelles difficultés soulève
la notion d’{\it intuition}, en tant que voie d'accès à la Métaphysique.
Quant au {\it sentiment}, ne faut-il pas dire plutôt, avec le psychologue
Henri Delacroix : « Le sentiment intercepte la réalité et se donne
l'apparence d’une réalité. Mais il n’est pas créateur, il n’est pas une
intelligence ou un équivalent de l’intelligence... Il n’y a rien dans le
cœur qui ne passe par l'esprit. » Peut-on d’ailleurs parler encore de
{\it connaissance} lorsque le sujet {\it s’absorbe} dans l’objet ?

En réalité, ce qu’exige, nous semble-t-il, la connaissance métaphysique,
c’est, au point de départ, un contact intime avec l'être, disons
mieux : une {\it participation} à l'être. C’est ce que nous fournit précisément
l'intuition du {\it cogito} : par cette intuition — intellectuelle en
même temps qu’existentielle — nous saisissons immédiatement notre
participation à la pensée, à la réalité spirituelle, laquelle d’ailleurs
(\S 358-359) est à la fois {\it être} et {\it valeur}. Mais, à partir de là, la Métaphysique,
désormais en prise sur l’être, peut s’édifier par une dialectique
qui ne sera plus pure construction de concepts, mais développement
de cette intuition fondamentale. « La Métaphysique, écrit Étienne
Gilson, est science à partir du point où, s’étant saisie du principe,
elle commence d’en déduire les conséquences ; mais le sort de la
doctrine se joue sur l’intellection du principe... L’aptitude du principe
à éclairer le réel sous tous ses aspects en confirmera sans doute la
vérité à mesure que se construira la doctrine ; mais c’est l’évidence
propre du principe, vu par l’entendement dans l’acte même de le
concevoir, qui en fait essentiellement la certitude. Quand on a saisi le
sens du principe, le déroulement de la doctrine se fait à sa lumière :
il n’en fait pas la vérité, il ne fait que la manifester. » — C’est ce
cheminement que nous essaierons de suivre dans les deux prochains
chapitres, spécialement dans le chapitre XXIX.

{\footnotesize 
\section{Sujets de travaux}% SUJETS DE TRAVAUX

{\bf Exercices.} — {\it Distinguer et classer les différents sens du mot métaphysique
et de ses dérivés dans les citations suivantes} : « On nomme métaphysique ce
qui surpasse la nature et qui est au-delà de la causalité et du langage »
(Erennios, néo-platonicien, IIIe siècle) ; « Elle est dicte métaphysique en
tant qu’elle considère {\it ens} l’être et les choses qui ensuivent à lui » (Christine
de Pisan) ; « Les choses métaphysiques, lesquelles ne dépendent point des
sens » (Descartes) ; « La façon dont il [Desargues] commence son raisonnement
en l’appliquant aux lignes droites et aux courbes est d'autant plus
belle qu’elle est plus générale et semble être prise de ce que j'ai coutume de
nommer la Métaphysique de la Géométrie » (Id.) ; « Il [Leibniz] saisissait
dans tout les principes les plus élevés et les plus généraux, ce qui est le
caractère de la métaphysique » (Fontenelle) ; « La métaphysique a cela de
%586
bon qu’elle ne demande pas des études préliminaires bien gênantes : c’est là
qu’on peut tout savoir sans avoir rien appris » (Voltaire) ; « Le vivant et
l’animé, au lieu d’être un degré métaphysique des êtres, est une propriété
physique de la matière » (Buffon) ; « Je veux mourir s’il y a dans toutes
ces têtes-là [celles des peintres] le premier mot de la métaphysique de leur
art » (Diderot) ; « On peut regarder la métaphysique comme un grand pays
dont une petite partie est riche et bien connue, mais confine de tous côtés
à de vastes déserts » (d'Alembert) ; « Je ne sais quelle métaphysique du
cœur s’est emparée de nos théâtres » (Id.) ; « Une connaissance métaphysique,
c’est-à-dire une connaissance qui dépasse l’expérience, ne doit contenir
que des jugements {\it a priori} » (Kant) ; « Comme la théologie, la métaphysique
tente d’expliquer la nature intime des êtres, l’origine et la destination de
toutes choses ; mais, au lieu d’y employer les agents surnaturels proprement
dits, elle les remplace par des {\it entités} ou abstractions personnifiées » (Comte) ;
« S'il existe un moyen de posséder une réalité absolument au lieu de la
connaître relativement, de se placer en elle au lieu d'adopter des points de
vue sur elle, la métaphysique est cela même. La métaphysique est la science
qui prétend se passer de symboles » (Bergson) ; « Un philosophe n’est pas
philosophe s’il n’est métaphysicien ; et c’est l’intuition de l'être qui fait le
métaphysicien » (Maritain) ; « Déraciner l'interprétation d’après laquelle
le besoin métaphysique serait comme une curiosité transcendante : c’est
plutôt un appétit, l'appétit de l’être. Il vise à la possession de l'être par la
pensée » (G. Marcel) ; « L'homme tel que la métaphysique nous le présente,
cest-à-dire comme un {\it étant}... » (Heidegger) ; « La métaphysique est liée
à l’être et au tout, mais non pas comme idées, bien plutôt à l'être et au tout
comme sentiments » (J. Wahl).

{\bf Exposés oraux.} — 1. La « philosophie première » d’après Aristote (voir
Ravaisson, {\it Essai sur la métaphysique d' Aristote} ; Cl. Piat, {\it Aristote}, liv. I,
chap. I). — 2. La métaphysique d’après Descartes (cf. préface des {\it Principes}
et {\it Méditations}). — 3. La conception existentielle de la métaphysique (voir
les textes de G. Marcel, Heidegger et J. Wahl).

{\bf Discussion.} — Le rôle du sentiment dans la connaissance métaphysique.

{\bf Lectures.} — {\it a.} L. Liard, {\it La Science positive et la Métaphysique}, 1879.
— {\it b.} A. Fouillée, {\it L'Avenir de la Métaphysique} fondée sur l'expérience,
1889. — {\it c.} H. Bergson, {\it Introduction à la Métaphysique}, dans la R. M. M.,
janv. 1903, p.1 (reproduit dans La Pensée et le Mouvant, p. 201). —
{\it d.} Ch. Dunan, {\it Légitimité de la Métaphysique}, ibid., sept. 1906, p. 651. —
{\it e.} H. de Keyserling, {\it La réalité métaphysique}, ibid., juill. 1911, p. 467. —
{\it f.} L. Lavelle, {\it La Présence totale}, Aubier, 1934. — {\it g.} G. Marcel, {\it Journal
métaphysique}, n.r.f., 1935 (spéc. p. 279-284) ; et {\it h. Le mystère de l'être}, Aubier,
2 vol., 1951. — {\it i.} M. Heidegger, {\it Qu'est-ce que la Métaphysique ?}, trad. fr.,
Gallimard, 1938 (texte difficile). — {\it j.} J. Wahl, {\it Existence humaine et
transcendance}, Baconnière, 1944 (notamment p. 78 : {\it Poésie et Métaphysique}) ;
{\it k. Note sur la Métaphysique}, dans la R. M. M., oct. 1947, p. 228 ; et {\it Traité
de Métaphysique}, 1957 ; {\it l.} G. Gusdorf, {\it Traité de Métaphysique}, A. Colin,
1956 ; {\it m.} Fr. D'Hautefeuille, {\it Les problèmes éternels de la Métaphysique},
La Colombe, 1961. — {\it n.} E. Gilson, {\it Introduction à la philosophie chrétienne},
Vrin, 14962. — {\it o.} G. Gusdorf, {\it Mythe et métaphysique}, Flammarion, 1963.
— {\it p.} J. Wahl, {\it L'expérience métaphysique}, Flammarion, 1965. — {\it q.}
L. Lavelle, {\it Science, esthétique, métaphysique}, A. Michel, 1967, 3e partie.
}
