
\chapter{La matière, la vie, l'esprit}

%chapitre XXVIII
%LA MATIÈRE, LA VIE, L'ESPRIT
%SOMMAIRE
%335. Matière, Vie, Esprit. — 336. Le Matérialisme. — 337. Le Matérialisme
%mécaniste. — 338. Le Matérialisme dialectique. — 339. Discussion du Matérialisme.
%— 340. Qu'est-ce que la Matière ? — 341. Le Vitalisme, — 342. Le
%Vitalisme antique. — 343. La médecine vitaliste. — 344. Le Vitalisme romantique.
%— 345. Le Vitalisme bergsonien. — 346. Le néo-vitalisme contemporain.
%— 347. Discussion du Vitalisme. — 348. Le Spiritualisme : de l’âme à
%l'esprit. — 349. Le Spiritualisme cartésien. — 350. Le Spiritualisme contemporain,
%— 351. Les rapports de l'esprit et du corps.

\section{Matière, Vie, Esprit}% 335. 
Le monde dans lequel nous
sommes plongés nous paraît se ramener à trois ordres de réalité que
nous désignons par les trois termes : matière, vie, esprit. L’Idéalisme,
interprété comme doctrine ontologique, prétendait ramener ces trois
réalités à une seule : la dernière. Mais cette réduction nous a paru
difficilement acceptable et ruineuse pour l’objectivité de la connaissance.
Comment dès lors définir ces trois ordres de réalité ? Comment
concevoir leurs rapports ? Une réduction en un autre sens serait-elle
possible? Enfin, si tout ne peut se réduire à l’esprit, ne faut-il pas
cependant affirmer l’irréductibilité et l'indépendance de l'esprit
lui-même ?

\section{Le Matérialisme}% 336.
Le Matérialisme est précisément la
doctrine qui prétend réduire tous les ordres de réalité à la matière.
Il ne doit pas être confondu avec le {\it Réalisme}. On a vu qu’il y a un
réalisme de l’intelligible : c’est le cas du réalisme platonicien et même,
en un sens (\S 111 B), du réalisme cartésien. Le Réalisme n'implique
d’ailleurs pas nécessairement la négation de la pensée comme fonction
propre et originale de l'esprit. Le {\it Matérialisme} est au contraire un
\textbf{\textit {monisme}} qui ne reconnaît qu’un seul ordre de réalités ou, comme on
dit de {\it substances} (voir \S 361), les substances matérielles. En ce sens,
il est l’antithèse exacte de l’{\it Idéalisme} qui est, lui aussi, {\it un monisme},
%588
puisqu'il ramène tout à l’existence unique de la pensée. Mais il
s'oppose aussi au {\it Spiritualisme} qui affirme l'indépendance de l'esprit.
— Le Matérialisme a pris plusieurs formes au cours de l’histoire. Il
peut reposer, notamment, sur une {\it non-distinction} du matériel et du
spirituel ou bien sur une réduction {\it consciente} de celui-ci à celui-là.
C'est ce dernier seul qui constitue le Matérialisme proprement dit
et qui répond à la définition d’Aug. Comte : « {\it Le Matérialisme est la
doctrine qui prétend expliquer le supérieur} [c’est-à-dire le plus
complexe] {\it par l’inférieur} [c’est-à-dire le plus simple]. » Le premier,
au contraire, où le spirituel n’est {\it pas encore distingué} du matériel est
l'{\it Hylozoïsme} des premiers philosophes grecs et des Stoïciens : nous en
traiterons à propos du Vitalisme (\S 342). Le Matérialisme proprement
dit peut prendre à son tour une forme {\it statique} : c’est le Matérialisme
mécaniste, ou une forme {\it dynamique} : c’est le Matérialisme dialectique
de la doctrine marxiste.

\section{Le Matérialisme mécaniste}% 337.
{\it A.} Dans l'antiquité, le
premier est représenté par les \textbf{\textit {Atomistes}} : Leucipe, Démocrite,
Épicure, qui réduisent tout ce qui existe aux {\it atomes}, conçus comme
de petites particules indivisibles, et à leurs mouvements dans le {\it vide}.
Selon Démocrite, l’âme se compose d’atomes, comme les corps;
mais, principe de vie et de mouvement, elle est faite d’atomes plus
légers et plus subtils. Plus tard, Épicure reprend cette doctrine,
avec des visées non plus scientifiques, mais morales ; et Lucrèce
la développe en latin dans son poème {\it De rerum natura}.

\vspace{0.24cm}
{\footnotesize Dans une lettre à un disciple, Épicure explique que l’âme est nécessairement
corporelle : car il n'y a d'incorporel que le vide ; or « le vide ne peut ni
agir ni pâtir » ; une âme incorporelle n'aurait donc ni activité ni sensations.
Elle est comparable à « un souffle mêlé d’une certaine quantité de chaleur
et répandu dans tout l'organisme ». La perception est produite par des
« simulacres », images extrêmement petites « détachées des objets et reproduisant
leurs formes et leurs couleurs » et qui, entrant « dans nos yeux et
notre pensée » avec un mouvement très rapide, « produisent par leur accumulation
l'apparence d’un objet unique et permanent ». — En reprenant ces
théories, Lucrèce distingue cependant entre l'âme ({\it anima}) et l'esprit
({\it animus}). La première, étant, comme l'avait dit Épicure, répandue dans
tout le corps, ne peut être le siège des pensées et des sentiments, indépendants
du corps. Ceux-ci doivent donc être rapportés à un esprit localisé dans le
cœur, où nous les sentons.}
\vspace{0.31cm}


Dans l'intention d’Épicure et surtout de Lucrèce, ces théories ont
pour but d’apaiser l'angoisse humaine et la crainte de la mort, provoquées
par les croyances populaires à l’au-delà. C’est en effet une
caractéristique du Matérialisme proprement dit de s'opposer, le plus
%589
souvent, aux croyances religieuses. Au {\footnotesize XVII}$^\text{e}$ siècle cependant,
Gassendi (1592-1655) reprendra à son compte l’atomisme d’Épicure
et sa théorie matérialiste de l’âme, mais en y superposant la notion
d’une âme spirituelle, capable de raison et de volonté libre, et celle
d’un Dieu créateur. 

{\it B.} Le \textbf{\textit {Mécanisme cartésien}} devait, au {\footnotesize XVIII}$^\text{e}$ siècle surtout,
conduire certains philosophes jusqu’au Matérialisme. Définissant la
matière par l'étendue (\S 111 B), Descartes avait expliqué par « la
figure et le mouvement » toutes les propriétés physiques des corps
et il n’avait pas craint d’étendre cette interprétation {\it mécaniste} aux
organismes vivants : notre corps est une {\it machine} gouvernée par une
âme ; mais les animaux, qui n’ont pas d’âme, sont de pures machines.
Certes, Descartes, spiritualiste, avait réservé les droits de l’âme
humaine. Mais, au {\footnotesize XVIII}$^\text{e}$ siècle, certains auteurs s’autoriseront de
sa doctrine mécaniste pour la généraliser à tout l’être de l’homme.

\vspace{0.24cm}
{\footnotesize C'est ainsi que le médecin La Mettrie, dans son {\it Homme-machine} (1748),
ne reproche à Descartes que d’avoir, dans sa théorie des animaux-machines,
refusé la sensibilité aux bêtes et d’avoir admis chez l'homme deux substances
là où une seule suffisait. D'Holbach écrit de même dans son {\it Système
de la nature} (1770) : « Descartes est le premier qui ait établi que ce qui
pense doit être distingué de la matière. D'où il conclut que notre âme ou
ce qui pense en nous est un esprit. N’eût-il pas été plus naturel de conclure
que, puisque l’homme qui est matière et qui n’a d’idées que de la matière,
jouit de la faculté de penser, {\it la matière peut penser ?} » L'homme, en effet,
est « un être purement physique » et tout, en lui, provient de sa sensibilité
physique. C'était aussi le principe du livre d'Helvétius, {\it De l'esprit} (1758).
Les mêmes thèmes se retrouvent également chez Diderot.}
\vspace{0.31cm}


{\it C.} Au {\footnotesize XIX}$^\text{e}$ siècle, le Matérialisme va prendre une allure un peu
différente du fait de sa liaison avec la \textbf{\textit {Physiologie.}} Mais, celle-ci en
étant encore, au début du siècle, à ses premiers pas, cette influence
ne le modifiera pas profondément. Dès 1796, c’est Cabanis qui, dans
son {\it Histoire physiologique des sensations}, lance la formule fameuse :

\vspace{0.24cm}
{\footnotesize « Pour se faire une idée juste des opérations de la pensée, il faut considérer
le cerveau comme un organe particulier destiné à la produire, de
même que l’estomac et les intestins à faire la digestion, le foie à filtrer la
bile... Le cerveau digère en quelque sorte les impressions ; il fait organiquement
la sécrétion de la pensée. »}
\vspace{0.31cm}


En Allemagne, à partir de 1850, toute une école de médecins et
de naturalistes prétend, en présentant la pensée comme une {\it fonction
du cerveau}, parler au nom de la science. Les théories de l’évolution,
interprétées dans le sens d’une évolution linéaire à partir d’une hypothétique
« monère » primitive, viennent ici appuyer le matérialisme.

%590
\vspace{0.24cm}
{\footnotesize Karl Vogt (1817-1895) utilise le darwinisme ; il aggrave la formule de
Cabanis en disant : « Le cerveau sécrète la pensée comme le foie sécrète la
bile et les reins l'urine. » Louis Büchner, dans son ouvrage {\it Force et Matière}
(1855) qui eut alors un énorme retentissement, soutient un matérialisme
mitigé qui présente la pensée comme un produit de la matière sans qu'elle
soit elle-même matière ; il distingue en effet, de façon assez obscure, la
{\it matière} de la {\it force}. Le hollandais Moleschott (1822-1893), Henri Czolbe
(1819-1873), Du Bois-Reymond, célèbre par son {\it ignorabimus} (nous ignorerons
toujours ce que c’est que la force, ce que c'est que la matière, et
comment elles peuvent penser : 1872), défendent des thèses analogues.
Ernest HæckeL ({\it Les Énigmes de l'univers}, 1899) soutient un « monisme »
inspiré du darwinisme au sens le plus dogmatique.}
\vspace{0.31cm}


On peut enfin rattacher au Matérialisme la théorie \textbf{\textit {épiphénoméniste}}
de la conscience, qui fait de celle-ci un simple « reflet » de l’activité
cérébrale sans action sur notre comportement (voir \S 31 B). Elle a été
surtout soutenue, elle aussi, par des physiologistes ou des médecins.
{\scriptsize (On peut considérer la {\it Cybernétique} comme une forme nouvelle de
l'épiphénoménisme : nous en avons dit quelques mots ci-dessus \S 31 B.)}

\vspace{0.24cm}
{\footnotesize Ce fut le point de vue des Anglais Thomas Huxcey (1825-1895) et Henry
Maudsley (1835-1918) et, en France, de Félix Le Dantec (1869-1917)
qui l’a présenté sous sa forme la plus absolue, tout en admettant que « les
éléments des substances brutes ont leur conscience élémentaire ». Ce fut
également, en Amérique, celui du naturaliste Jacques LœB (1859-1924) et
des psychologues {\it behavioristes} (J.-B. Watson, A.-P. Weiss, R.-B. Perry)
qui définissaient l'esprit « l’ensemble des réactions de l'organisme sur son
milieu » et proclamaient la « stérilité » de la conscience.}
\vspace{0.31cm}


\section{Le Matérialisme dialectique}% 338.
Il ne faut pas confondre
le {\it Matérialisme dialectique} avec le Matérialisme mécaniste. On a vu
au \S 299 B que K. Marx a été amené à concevoir sa doctrine par
réaction contre celle de Hegel. Chez Hegel, écrira-t-il plus tard,
« la dialectique marche sur la tête : il suffit de la remettre sur ses
pieds pour lui donner une physionomie raisonnable », En effet, « pour
Hegel, le mouvement de la pensée qu’il personnifie sous le nom de
l’Idée, est le démiurge de la réalité, laquelle n’est que la forme phénoménale
de l’Idée. Pour moi, au contraire, dit Marx, le mouvement
de la pensée n’est que la réflexion du mouvement réel, transporté et
transposé dans le cerveau de l’homme ». A son origine, le Matérialisme
dialectique est donc surtout un {\it Réalisme} qui s’oppose à l’Idéalisme
hégélien. Mais il est aussi un {\it humanisme} anti-religieux qui s'inspire
de celui du néo-hégélien Feuerbach (1804-1872).

\vspace{0.24cm}
{\footnotesize Comme celui-ci, Marx fait grief à la religion d’« aliéner» l'homme, c’est-à-dire
d’être une idéologie illusoire qui situe l’homme en dehors de lui-même
et le détourne de ses activités réelles. Mais, au matérialisme de Feuerbach
%591
comme à celui du {\footnotesize XVIII}$^\text{e}$ siècle, Marx reproche : 1° d’être purement {\it statique} :
l’objet, le réel y est posé comme un donné, et non comme la base d’une
activité de l’homme, d’une « praxis » ; or, dit Marx, « les philosophes n'ont
fait qu'interpréter le monde ; il s’agit maintenant de le changer » ; 2° de
{\it méconnaître l'histoire} : il n’y a pas d’essence absolue de l’homme ; « l’histoire
tout entière n’est qu’une transformation continue de la nature humaine » ;
3° d'admettre un {\it déterminisme mécanique} qui fait de la pensée une simple
fonction du cerveau et de l’homme un produit des circonstances, notamment
de l’éducation : il oublie ainsi 4 que les circonstances sont transformées
par l’homme et que l’éducateur lui-même doit être éduqué» ; en outre, les
conditions sociales de la pensée ne sont pas moins importantes que les
conditions physiologiques.}
\vspace{0.31cm}

A cette époque (1844-45), Marx écrit que, « dans l’état de société,
le subjectivisme et l’objectivisme, le spiritualisme et le matérialisme
perdent leur opposition et, par suite, leur existence » et que « la vieille
opposition du spiritualisme et du matérialisme a été partout mise
de côté » par Feuerbach. Par la suite, à la fois sous l'influence des
nécessités polémiques, comme l’a dit Fr. Engels, et des théories
darwiniennes où Marx crut trouver une confirmation de ses idées, la
doctrine marxiste s’étrécit en un Matérialisme strict où toute la vie
spirituelle de l’homme, discréditée sous le nom d’{\it idéologie}, apparaît
trop souvent comme un simple « reflet », plus ou moins illusoire, des
conditions économiques, spécialement de l’état des {\it forces productives
matérielles} à un moment donné de l’histoire. Toutefois, ainsi qu’il a
déjà été remarqué à propos du Matérialisme historique (\S 299) qui
n’est que l’application au devenir humain du Matérialisme dialectique,
ce dernier n’est pas, comme on le dit parfois, un {\it économisme} exclusif :
il fait place aux autres facteurs du développement humain, même
aux facteurs {\it idéologiques} auxquels il attribue une {\it action en retour}
sur la base économique. En outre, la pensée profonde de Marx semble
bien avoir été que l’activité économique elle-même n’est pas d’ordre
purement matériel, puisque le travail, par exemple, implique une
représentation {\it idéale} du résultat à obtenir.

\section{Discussion du Matérialisme}% 339.
Deux critiques fondamentales
peuvent être adressées au Matérialisme.

{\it A.} La première
est que la réduction qu’il prétend opérer de la vie spirituelle à ses
conditions matérielles cst inintelligible.

\vspace{0.24cm}
{\footnotesize N'insistons pas sur la forme que Cabanis, K. Vogt, etc., lui ont donnée
en affirmant que la pensée n’est qu’une {\textit{\textsf{sécrétion}}} du cerveau. Büchner
lui-même critique cette formule ; et, de fait, comment comparer avec une
sécrétion, qui est quelque chose de sensible, de pondérable, de localisable
dans l’espace, la pensée, qui ne tombe pas sous les sens, qui n’est ni pondérable
%592
ni localisable ? Au reste, le cerveau n’est pas une glande, et c’est un
non-sens physiologique de lui attribuer une « sécrétion ». Il ne vaut guère
mieux de dire, avec les {\it épiphénoménistes}, que la pensée est un {\textit{\textsf{reflet}}} de ce
qui se passe dans le cerveau. Le terme de « reflet » n’est qu’une métaphore
qui n’a aucun sens précis et qui n’est pas plus heureuse que les termes de
{\it lueur}, de {\it phosphorescence}, etc., qu'ont employés parfois les mêmes auteurs
pour désigner la pensée. — A peine serait-il moins impropre de parler de
la pensée comme d’une \textsf{\textit {fonction}} du cerveau. D'abord, on a vu (cf. \S 28-30) que
ni les physiologistes ni les psychologues ne conçoivent plus ainsi le rôle du
cerveau. La pensée est liée à l’ensemble de notre {\it action}, de nos {\it comportements},
et le cerveau a pour fonction d’être l'organe {\it régulateur} de ces comportements,
non pas du tout de fabriquer, en quelque sorte, de la pensée : « Un
cerveau séparé de l’être vivant est incapable de pensée et d'action. Le cerveau
est un des éléments d’un circuit extrêmement complexe que nous appelons
{\it l’action}. En réalité, l'homme pense avec tout son corps : il pense avec ses
mains, ses pieds, ses oreilles, aussi bien qu’avec son cerveau. Il est absolument
ridicule de dire que sa pensée dépend d’une partie de lui-même : c'est
comme si on disait que notre habileté manuelle dépend de nos ongles.
L'activité psychologique est une activité d'ensemble, et non pas une
activité locale. {\it Le cerveau est tout simplement un ensemble de commutateurs} »
(Pierre Janet). — D'autre part, une {\it fonction} est une \textsf{\textit {abstraction}}, qui désigne
simplement un ensemble d'opérations des organes. La pensée est une
{\it réalité concrète} et, qui plus est, c’est une réalité {\it qui se connaît elle-même},
c’est une réalité {\it pour-soi}. Là est la difficulté majeure à laquelle se heurtera
toujours, semble-t-il, le Matérialisme. On aura beau insister sur les {\it conditions}
physiologiques de la pensée, et notamment de la pensée consciente,
qu’il n’est d’ailleurs pas question de nier (voir \S 26-31) : on ne nous fera
jamais comprendre comment un phénomène matériel, quel qu'il soit,
pourrait engendrer ce quelque chose d’essentiellement nouveau qu'est la
{\it prise de conscience} par soi-même de l'être pensant et même la {\it conscience}
simple. « La matière de demain, pas plus que celle d’aujourd’hui, écrivait
le psychologue A. Binet, ne peut engendrer que des effets matériels. » Et,
faisant allusion aux recherches d’un histologiste qui avait longuement étudié
les tissus cérébraux, Binet ajoutait : « C’est que l'étude, si patiente, si
minutieuse qu'on la suppose, de cet écheveau nerveux, ne pourrait jamais
nous faire connaître ce que c’est qu’un état de conscience, si nous ne le
savions déjà par ailleurs ; car ce n’est jamais dans le champ du microscope
qu’on voit passer un souvenir, une émotion ou un acte de volonté. »}
\vspace{0.31cm}

On peut même dire qu’il y a dans le Matérialisme une sorte de
\textbf{\textit {contradiction}} interne. Toute affirmation d’une réalité étrangère à
l'esprit implique déjà la pensée (c’est, on l’a vu, ce qu’il y a d’irréfutable
dans l’Idéalisme) ; et, s’il prenait fantaisie à la pensée de se nier elle-même,
ainsi qu’il arrive précisément, en un sens, dans le Matérialisme,
elle {\it s’affirmerait encore dans sa négation même} : car tout jugement
est œuvre de l’esprit.

\vspace{0.24cm}
{\footnotesize Ces objections valent contre le Matérialisme dialectique aussi bien que
contre le Matérialisme mécaniste. La théorie marxiste n'apporte en effet
ici rien d’essentiellement nouveau. Lorsqu'un Engels affirme : « Notre
conscience et notre pensée... ne sont que les produits d’un organe matériel,
%593
corporel, le cerveau. La matière n'est pas un produit de l'esprit : c'est
l'esprit qui est le produit supérieur de la matière », il ne fait que répéter
une formule un peu simple héritée du scientisme physiologique du début du
{\footnotesize XIX}$^\text{e}$ siècle. Lorsqu'un Lénine affirme : « L'univers n’est que matière
en mouvements, il en revient au Matérialisme mécaniste. Lorsque certains
théoriciens marxistes déclarent que la superstructure idéologique est « le
reflet » de la base matérielle, ils empruntent une métaphore bien creuse
à l'épiphénoménisme, que cependant ils repoussent.}
\vspace{0.31cm}

{\it B.} La seconde difficulté concerne la notion de \textbf{\textit {matière}} elle-même.
Cette notion est loin d’être aussi claire que certains matérialistes
semblent le supposer. Qu’on se rappelle toutes les difficultés
auxquelles s’est heurté le Réalisme vulgaire, celui précisément qui
identifie « le réel » avec la réalité sensible (\S110). Bien des matérialistes,
comme Du Bois-Reymond (1818-1896), ont avoué leur ignorance touchant
l’élément ultime de cette réalité. Beaucoup aussi, afin de pouvoir
expliquer, comme disait Comte, le supérieur par l’inférieur,
{\it mettent déjà dans celui-ci les propriétés de celui-là} : c’est ainsi que
D’Holbach, Diderot, etc., affirment que « la matière peut penser »
et que Le Dantec admet jusque dans les éléments chimiques une
conscience élémentaire, que Büchner déclare la {\it force} irréductible à la
matière, etc. Mais qu’est-ce que la « force » ? comment la matière peut-elle
penser ? que peut bien être la conscience d’une cellule, d’un atome,
d’un électron ? Avouons-le : ces formules sont aussi {\it dénuées de sens} que
celle des Atomistes de l’antiquité quand ils affirmaient que la pensée
est faite d’atomes subtils ! On glisse ici dans un véritable roman de
la matière, et l’on se paye de mots.

\vspace{0.24cm}
{\footnotesize Sans tomber dans un aussi grossier verbalisme, le Matérialisme dialectique,
ici encore, prête le flanc à une objection analogue. Tout, dans la vie
sociale, nous dit-on, se ramène, en dernière analyse, à l’action des « forces
productives ». Mais on a pu montrer récemment que, dans ses premiers
écrits, Karl Marx entend par là à peu près toutes les activités de l’homme,
y comprises même les « forces spirituelles », et plus tard, il continuera à
affirmer que l'acte producteur par excellence, le {\it travail}, contient un élément
idéal (\S 299 B 3°). Comment donc peut-on affirmer que la « base » est uniquement
matérielle ?}
\vspace{0.31cm}

\section{Qu'est-ce que la Matière ?}% 340.
Essayons donc de préciser la
notion de {\it matière}, et pour cela examinons rapidement son évolution
chez les philosophes et les savants.

{\it A. Dans l'antiquité}, la matière fut d’abord conçue comme animée :
c’est l’{\it Hylozoïsme} auquel nous avons déjà fait allusion et sur lequel
nous reviendrons (\S 342).

Les Atomistes sont peut-être les seuls qui aient eu, dans l’antiquité,
%594
une conception claire de la matière. C’est déjà une conception \textbf{\textit {géométrique}}
où seules les qualités primaires (\S 114) : l’étendue, la
pesanteur, la distribution dans l’espace, le mouvement, sont reconnues
réelles ; Démocrite précise que la cause du mouvement ne
réside pas dans une {\it force} extérieure aux atomes. Mais c’est aussi une
conception \textbf{\textit {discontinuiste}} puisque la matière se résout en particules
indivisibles.

\vspace{0.24cm}
{\footnotesize Platon ne nomme même pas la matière : « espèce obscure et difficile
à concevoir », dit-il lui-même de ce qu’il nomme « le réceptacle de tout
devenir ». — Cette conception se précise chez Aristote : il y a bien une
matière ({\it hylè}) dans les corps ; mais, pure « puissance », virtualité indéterminée,
{\it elle n'existe pas sans la forme}, sans la détermination des qualités
telles que le chaud, le froid, le lourd, le léger, etc. Elle est indestructible ;
car tout ce qui périt, se résout en elle.}
\vspace{0.31cm}

{\it B. Le Mécanisme cartésien}. Il faut arriver à Descartes pour
retrouver une conception claire de la matière, celle qui a déjà été
indiquée au \S 111 B : la matière est la {\it res extensa}, la « chose étendue » ;
toutes ses propriétés se ramènent à la
\textbf{\textit {figure}} et au \textbf{\textit {mouvement.}} Mais,
comme l’étendue est divisible à l'infini, il n’y a ni atomes ni vide :
cette conception \textbf{\textit {géométrique}} de la matière est en même temps une
conception \textbf{\textit {continuiste.}} Scientifiquement, elle se traduit par le principe
de la {\it constance de la quantité de mouvement} (produit m.{\bf v} de
la masse par la vitesse), base de toute la Physique cartésienne.

{\it C. Le Dynamisme leibnizien}. On a vu au \S 113 comment Leïbniz
a critiqué la conception cartésienne et comment il s’est efforcé de
réserver, dans sa doctrine idéaliste, une certaine place à la matière,
du moins comme principe de résistance ou de limitation de la perception
des monades. D’autre part, il avait découvert que ce n’est pas,
comme l’avait cru Descartes, la « quantité de mouvement », mais la
« quantité des forces vives » (définie par le produit $m.v^2$) qui demeure
constante : par là, Leibniz se trouvait beaucoup plus près du principe
moderne de la conservation de l’\textbf{\textit {énergie.}} Aussi se fait-il une conception
{\it dynamiste} de la matière elle-même. Celle-ci ne peut s’expliquer uniquement,
selon lui, par des propriétés mécaniques : comme toute
substance, elle est « force active », elle possède une « propriété agissante »,
et, « non plus que la substance spirituelle, ne cesse jamais d’agir ». Dès
1671, Leibniz réduisait l’opposition de l'esprit et de la matière en
disant que tout corps est une {\it mens momentanea}, une âme qui vit dans
l'instant, qui ne sait pas se souvenir, tandis que, dans l'esprit, l’effort
interne et l’action extérieure se conservent. Dans la {\it Monadologie},
il affirmera « qu’il y a un monde de créatures, de vivants, d’animaux,
%595
d’entéléchies
{\scriptsize ({\it Entéléchie}, du grec entélôs, parfaitement,
et {\it echein}, se trouver. Aristote avait
désigné par ce mot la pleine réalisation des virtualités d’un être. Leibniz l’applique aux
monades comme se suffisant à elles-mêmes en tant que sources de leurs états internes
(\S 113))},
d’âmes dans la moindre partie de la matière », à tel
point que celle-ci « peut être conçue comme un jardin plein de
plantes ct comme un étang plein de poissons ».

{\it D. La matière dans la science moderne}. Ces deux conceptions, cartésienne
et leibnizienne, ont trouvé un écho dans la science.

\vspace{0.24cm}
{\footnotesize Le mécanisme cartésien a inspiré toutes les théories dites \textsf{\textit {cinétiques}} qui
ont eu cours jusqu’au début du {\footnotesize XX}$^\text{e}$ siècle.
Dès 1690, Chr. Huyghens, en
posait le principe, en harmonie avec la Physique cartésienne, en disant que
les propriétés de la lumière ne peuvent s'expliquer que par « le mouvement
de quelque matière, au moins, ajoutait-il, dans la vraie philosophie, dans
laquelle on conçoit la cause de tous les effets naturels par des raisons de
mécanique ». Les théories {\it ondulatoires} ou {\it vibratoires} (\S 74) étaient toutes
des applications de ce principe. En 1900, au Congrès international de
Physique, un savant proclamait : « L'esprit de Descartes plane sur la
Physique moderne. Plus nous pénétrons dans la connaissance des phénomènes
naturels, plus se développe et se précise l’audacieuse conception
cartésienne : il n’y a dans le monde physique que de la matière et du mouvement. »
Un grand physicien anglais, lord Kelvin, déclarait que, pour lui,
comprendre un phénomène physique, c'est « pouvoir en faire un modèle
mécanique correspondant ».

La conception dynamiste de Leibniz est passée, elle aussi, dans le domaine
scientifique. Dès 1759, le P. Boscovich interprétait la notion de l'atome
comme celle d’un pur centre de forces. Mais ce furent surtout les découvertes
de l'{\it équivalent mécanique de la chaleur} (Mayer et Joule) et de la {\it dégradation}
de l'énergie (S. Carnot ct Clausius) qui permirent à Helmoltz (1847) de
dégager cette notion d'{\it énergie} et de créer l'\textsf{\textit {énergétique}}. La distinction de
l'{\it énergie cinétique} (correspondant à la notion leibnizienne de la « force vive »)
et de l'{\it énergie potentielle} permit de préciser la doctrine, bientôt exploitée
sur le terrain philosophique par le chimiste allemand Ostwald et le physicien
français Duhem. Selon Ostwald (1853-1932), toutes les qualités de la
matière se ramènent à {\it l'énergie} : le poids est une sorte d'énergie de position,
et la conscience elle-même est de nature énergétique : « l'énergie » apparaissait
ainsi comme la substance même du monde, comme une véritable
« chose en soi ». Ostwald s’imaginait avoir ruiné ainsi le Matérialisme dont,
en 1895, il proclamait la faillite définitive, sans s’apercevoir qu'il prêchait
lui-même une sorte de « Matérialisme métamorphosé » (H. Driesch). Pierre
Duhem (1861-1916) alla presque aussi loin en soutenant que c'est l'abaissement
de {\it tension} de l'énergie et sa dispersion, mesurés par le principe de
Carnot (entropie), qui caractérisent l’action matérielle.

Mais, vers la fin du {\footnotesize XIX}$^\text{e}$ siècle, on assistait déjà à un {\it recul de l'énergétisme}.
Dès 1912, Poincaré déclarait que « les anciennes hypothèses mécanistes
et atomistes [avaient] pris assez de consistance pour cesser de nous
apparaître comme des hypothèses ». Certes, l'{\it atome} des chimistes modernes
est bien différent de l'{\it atome} des anciens
{\scriptsize (Pour les anciens (Démocrite, Épicure), l'atome était l'élément dernier et indivisible
de la matière. Pour les savants modernes, l'atome est tout un monde qu'on a souvent
comparé à un {\it système solaire en miniature}, le noyau atomique formant le gentre de ce
système. « Chaque progrès de la physique, écrivait Poincaré, nous révèle une nouvelle
complication de l’atome » : c’est de plus en plus vrai aujourd’hui !)}
; mais, dès alors, il avait cessé d’être
%596
« une fiction commode » pour devenir « une réalité ». La théorie cinétique
ds gaz, appuyée de nombreuses théories analogues (théorie des solutions, etc.),
venait étayer la conception {\it discontinuiste} et {\it mécaniste} de la matière. Certes,
par la suite, les nouvelles théories physiques, spécialement celle de la Relativité
(\S 269), ont bouleversé bien des points de vue classiques. Mais, comme
l’observe P. Mouy, « s’il est vrai que la physique de notre temps a remplacé
la théorie de l’éther par celle de \textsf{\textit {champs}}
{\scriptsize (A la notion classique de l’espace neutre et infini, la théorie de la Relativité substitue
celle d'un {\it champ} doué de propriétés déterminées et fini, quoique illimité, l'espace
de Riemann (cf. ci-dessus \S 98 B et 269))}
abstraits, elle est au fond, et peut-être
sans le soupçonner, cartésienne », cartésienne à la façon de Malebranche
dont « l'étendue intelligible » (\S 111) était précisément un système de rapports
abstraits. — D'autre part, il apparaissait de plus en plus qu'il n'y
avait pas lieu d’{\it opposer} l'énergie à la matière pondérable. Celle-ci semblait
caractérisée autrefois par la notion de {\it masse}. Or « la nouvelle mécanique
affirme l’inertie de l'énergie, c'est-à-dire la variation de la masse d’un corps
proportionnellement à l’énergie interne de celui-ci, unissant en une seule les
deux notions de masse et d'énergie » (Langevin). — Enfin il se révélait que
matière et énergie sont de {\it structure} identique. On a vu (\S 268) que, lorsqu'il
s’est agi d'expliquer la structure du rayonnement lumineux, les théories
scientifiques, parties de la notion {\it discontinuiste} d’une {\it émission} corpusculaire,
durent en venir à une conception {\it continuiste}, celle des théories {\it ondulaloires},
mais que, depuis 1900, avec la physique {\it quantique}, s'est imposé un
retour partiel à la notion d'une structure « granulaire » de l'énergie lumineuse.
C’est de cette conception granulaire qu'étaient partis, à propos de
la matière, les Atomistes de l'antiquité, et l'on vient de voir que cette
conception s'était de plus en plus imposée, de nos jours, aux Sciences
physiques. Mais la Mécanique ondulatoire (\S 268 E) introduit au sein même
de l’atome des {\it ondes} associées aux corpuscules de matière (par exemple, aux
électrons). Ainsi, « pour la matière comme pour la lumière, l’aspect atomique
et discontinu des entités élémentaires se double d’un aspect continu
et ondulatoire » et, par suite, « matière et lumière apparaissent comme
beaucoup plus semblables dans leur structure qu'on ne le pensait autrefois » (L. De Broglie).}
\vspace{0.31cm}

Il résulte de tout cela que les Physiciens d’aujourd’hui se trouvent
assez embarrassés quand il s’agit de définir la matière. La chose
d’ailleurs n’est pas nouvelle. Déjà, dans le Discours préliminaire de
l'{\it Encyclopédie}, D'Alembert remarquait que plus les savants « approfondissent
l’idée qu’ils se forment de la matière et des propriétés qui
la représentent, plus cette idée s’obscurcit et paraît vouloir leur
échapper ». En 1909, H. Poincaré déclarait : « L’une des découvertes
les plus étonnantes que les physiciens aient annoncées dans ces
dernières années, c’est que la matière n’existe pas », et il se demandait
si, la matière se définissant par la {\it masse} et cette dernière notion se
trouvant compromise, il ne fallait pas conclure à {\it la fin de la matière}.

%597
Depuis lors, le problème n’a fait que se compliquer. L. de Broglie
écrivait en 1946 : « Si j'étais un humoriste, je dirais qu'ayant passé
notre vie à étudier les atomes, nous ne savons plus du tout ce que
c'est. »

{\it E. Conclusion}. On voit donc que la notion même de {\it matière} est
sujette à discussion. La matière, a-t-on dit (M. Lodetti), n’est pas
une substance, mais « une relation aux œuvres humaines : il n’y a
de matière que pour le travail. » Elle n’est même pas, comme l’a
voulu Bergson (cf. \S 345), une sorte de dégradation de l’être : « La
matière ne provient pas d’un esprit ou d’un être originel qui tombe de
lui-même et se dégrade, elle est {\it une image} issue de la conduite des
hommes qui, considérant l’être par rapport à des projets, l’utilisent
en vue de leurs réalisations » (Id.). En ce sens, on peut dire que la
matière est « imaginaire ».

\section{Le Vitalisme}% 341.
Le Vitalisme, au sens philosophique du
terme, peut être considéré comme une doctrine {\it intermédiaire} entre
le Matérialisme et le Spiritualisme et qui s'apparente parfois de
façon assez étrange à l’un et à l’autre. Il consiste à faire de \textbf{\textit {la Vie}}
une entité qui forme une {\it totalité} indivisible et qui se montre rebelle
aux analyses de l'intelligence. Tantôt cette réalité se confond avec
la pensée, tantôt elle s’en distingue. Mais, même en ce dernier cas,
il y a \textbf{\textit {continuité}} entre la vie et la pensée.

\section{Le Vitalisme antique}% 342.
Dans l'Antiquité, le Vitalisme
va parfois jusqu’à l’\textbf{\textit {hylozoïsme}} (grec : {\it hylè}, matière; {\it zôon}, être
vivant), c’est-à-dire jusqu’à une conception de l’univers qui en fait
un organisme, un être vivant.

\vspace{0.24cm}
{\footnotesize {\it A.} C'est ainsi que les premiers philosophes grecs distinguent encore
fort mal le {\it matériel}, le {\it vital} et le {\it spirituel}. Thalès qui considère l’eau,
l'élément marin, comme l’élément primordial et la substance même du
monde, lui attribue une âme, c’est-à-dire un principe de mouvement et,
sans doute, de vie : « Toutes les choses, dit-il, sont pleines de dieux. » Pour
Héraclite, c'est le feu qui est le principe premier, mais c’est un feu
« éternellement vivant », d'où naissent les âmes. Ames et corps représentent
divers degrés de tension, comme les cordes d’une lyre. Déjà Héraclite dépeint
l’univers comme quelque chose d’essentiellement {\it mouvant} et il s'oppose
à la conception des Pythagoriciens « d’après laquelle on pourrait expliquer
l’univers au moyen d'unités mathématiques indivisibles et fixes que
l'on combinerait ensemble » (R. Berthelot). Beaucoup de ces traits se
retrouveront jusque dans les doctrines modernes.}
\vspace{0.31cm}

{\it B.} Platon considère l’âme comme un \textbf{\textit {principe de mouvement}} :
elle est « ce qui se meut soi-même ». Mais elle est, en même temps,
parente des Idées et l’on verra (\S 348) que par là Platon annonce le
%598
spiritualisme proprement dit. — Aristote professe un \textbf{\textit {finalisme,}}
beaucoup plus proche du vitalisme. La « nature » ({\it physis}) est pour lui
ce qui a son principe moteur en soi-même, alors que les produits de
l'art ({\it technè}) viennent d’un principe extérieur. La nature est donc
\textbf{\textit {forme}} et, en un sens, \textbf{\textit {âme}} :
on a vu en effet (\S 340 A) que la matière,
selon Aristote, ne serait rien sans la forme ; et, d’autre part, c’est
l’âme qui est {\it la forme du corps}. Il y a trois espèces d’âmes, correspondant
aux trois formes de la vie dans l’univers : l’âme {\it végétative} ou
{\it nutritive}, la seule que possèdent les plantes ; l'âme {\it sensitive}, qui est aussi
{\it motrice}, privilège des animaux ; et enfin l’âme {\it intellectuelle}, caractérisée
par la raison et qui est propre à l’homme. Celui-ci est, en effet,
{\it le but} de la nature. Car « la nature ne fait rien en vain » : l’univers
entier est soulevé, par une loi de finalité, de la {\it matière} amorphe vers
la {\it forme} définie, de l’{\it inférieur} vers le {\it supérieur}, du {\it mécanisme} vers
l'{\it intelligence} et la {\it pensée contemplative}. Aristote en vient parfois à
reprendre certaines formules hylozoïstes des premiers philosophes,
comme par exemple : « Tout est plein d'âme. »

{\it C.} L'hylozoïsme reparaît nettement chez les Stoïciens. On les
qualifie souvent de {\it matérialistes} ; et ils le sont en effet en ce sens qu’ils
soutiennent que tout est matériel. Mais il faut bien voir ce qu'ils
entendent par {\it matière} : « Pour les Stoïciens, le principe de l’univers
comme de l’âme, c’est une énergie à des degrés divers de \textbf{\textit {tension,}} ...et
cette tension ne se distingue pas essentiellement de l’âme elle-même :
c’est l’Ame, c’est l'Esprit qui est plus ou moins tendu dans le monde. »
Celui-ci est un organisme, qui, comme tous les vivants, possède une
âme. Cette \textbf{\textit {Ame du monde}} est « un feu artiste ». La philosophie
stoïcienne est, en somme, « une théorie d’après laquelle la matière est
vivante : c’est une spontanéité vitale plus ou moins concentrée ou
relâchée, irréductible au mécanisme et au raisonnement discursif »
(René Berthelot).

\vspace{0.24cm}
{\footnotesize {\it D.} Les {\it Néo-Platoniciens} accentueront cet hylozoïsme. C’est ainsi que,
pour Plotin, toute force active, dans la nature, a une \textsf{\textit {âme}} : « Nous disons
qu'une chose ne vit pas, parce que le mouvement qu’elle reçoit de l'univers
n’est pas accessible à nos sens »; mais, en réalité, « tout être a une part
d'âme qui lui vient de l'univers ». La terre elle-même a une âme, grâce à
laquelle « elle donne aux plantes le pouvoir d'engendrer ». Ce « vitalisme
intempérant », comme dit É. Bréhier, est pour Plotin un moyen de {\it faire
rentrer chaque être dans le grand courant de la vie universelle}. Les Stoïciens
avaient déjà dit que, dans le monde, « tout conspire ». Plotin va plus loin :
« Tout se passe, dit-il, dans l'univers comme en un animal où l’on peut, grâce
à l'unité de son principe, connaître une partie d’après une autre »; et ainsi
« tout est plein de signes », partout il y a des correspondances, par où Plotin
légitime {\it l'astrologie}. Enfin Plotin déclare que, dans la mesure où la vie de
l'âme s'élève jusqu’à l'\textsf{\textit {unité}} de son principe, « il se produit une pénétration
%599
de plus en plus intime des termes les uns dans les autres », tandis qu'au
contraire, dans la mesure où l’âme se laisse subordonner au corps, « il se
fait comme une dispersion, comme une désagrégation de cette unité de la
vie spirituelle », notion que l’on retrouvera jusque chez Bergson (R. Berthelot).}
\vspace{0.31cm}

\section{La médecine vitaliste}% 343.
« Si la thèse vitaliste a été
exposée pour la première fois avec une précision vraiment scientifique
chez les médecins du milieu du {\footnotesize XVIII}$^\text{e}$ siècle, elle se rattache à une
tradition beaucoup plus ancienne, la tradition d’une médecine qui
croit impossibles ou insuffisantes les explications d’ordre mécanique
ou chimique et qui croit nécessaire de faire intervenir un principe
irréductible à ces principes mécaniques ou chimiques », et, à travers
cette dernière, comme on vient de le voir, « nous remontons jusqu’à
la métaphysique néo-platonicienne » (R. Berthelot).

\vspace{0.24cm}
{\footnotesize Cette médecine se développe, à l’époque de la Renaissance, chez des
médecins, à la fois alchimistes et astrologues, comme ce Paracelse (1493-1541),
qui eut une si grande influence en Allemagne et dont l’ascendant
s’exerça encore sur la pensée romantique. Paracelse qui prétendait avoir
découvert la {\it pierre philosophale} et l'{\it élixir de vie}, s'appuyait sur une
conception mystique de l’univers, où tout se tient, où tout est correspondance et
signe à déchiffrer, où le développement de l'être vivant s'explique par une
{\it Archée}, principe de vie participant à la fois à la matière et à la pensée,
notion qui sera reprise un peu plus tard par son disciple J.-B. Van Helmont
(1577-1644)
{\scriptsize (Van Helmont est ce médecin qui soutenait qu'on peut obtenir une génération
spontanée de souris vivantes en bouchant avec une chemise sale un vase dans lequel
on avait mis des grains de blé : au bout de 21 jours, le blé se transmuait en souris !)}.

Leïbniz lui-même fait allusion, au début de ses {\it Nouveaux Essais}, à ce
qu’il peut y avoir de « raisonnable » dans l'opinion de ceux (tel son ami
Van Helmont, le fils du précédent) qui, dit-il, « donnent de la vie et de la
perception à toutes choses ». On a vu ci-dessus que lui-même n'était pas
tellement éloigné de cette manière de voir et que, sous l'influence sans
doute des découvertes accomplies à l’aide du microscope qui avait permis
d'apercevoir un monde d’« animalcules », comme on disait alors, là où on
ne s’attendait pas à en trouver, il en venait à employer des formules presque
hylozoïstes (\S 340 C, fin).

Au {\footnotesize XVIII}$^\text{e}$ siècle, le vitalisme passe sur le terrain scientifique avec Stahl
sous la forme de {\it l’animisme}. Tandis que, suivant Leibniz, « l'âme n’a pas
d'action directe sur le corps et ne fait qu'accompagner de sa volonté et de
sa conscience des mouvements qui sont des conséquences de mouvements
antérieurs », selon Stahl au contraire « l'âme est vraiment et en tous sens
la cause du mouvement dans le corps qu’elle anime... Les opérations vitales,
internes, pour échapper au raisonnement, n’en sont pas moins des opérations
de la raison. Sans conscience ? Non ; mais sans cette conscience expresse et
distincte à laquelle seule s'appliquent et la réflexion et la mémoire »
(Ravaisson). — On a vu au \S283 comment l’animisme s’est rétréci en {\it vitalisme}
proprement dit, distinguant de l’âme le principe vital, avec Barthez et
%600
l'école de Montpellier, puis en organicisme avec Bichat, reportant le principe
vital à chaque système d'organes.}
\vspace{0.31cm}


\section{Le Vitalisme romantique}% 344.
Au {\footnotesize XIX}$^\text{e}$ siècle, la philosophie
romantique va « opérer la confusion de la vie au sens biologique
avec {\it la vie spirituelle} » (A. Lalande). Chez les romantiques allemands
notamment, s’établit une conception mystique de la « vie
profonde », selon laquelle l’homme peut communier avec le tout
en approfondissant sa vie propre, non par l'intelligence, mais par la
foi ou l'intuition. Les poètes Novalis, Hölderlin en viennent à faire
de la poésie la vraie philosophie qui unit les contradictoires et « nous
réconcilie avec tout » : « Être, vivre, dit Hölderlin, c’est assez ; tout
ce qui se contente de vivre, est égal à soi-même dans le monde divin. »

{\it A.} « Cette mystique qui mêle des formules kantiennes aux formules
chrétiennes et aux formules vitalistes, se combine chez Schelling
avec l’hylozoïsme des physiologues de la Grèce primitive. » La philosophie
de la nature de Schelling fait de la nature une {\it activité vivante},
un organisme qui puise en lui-même les sources de son perpétuel
rajeunissement. Même les affinités chimiques, les attractions magnétiques
relèvent de ce principe de vie immanent aux corps matériels
et qui se manifeste aussi dans l’{\it instinct}. Pour saisir ce principe de
vie, il faut faire appel, non à l'intelligence, mais à une \textbf{\textit {intuition}} analogue
à celle de l'{\it artiste} ou du {\it mystique}, qui réalise l’unité du sujet et
de l’objet.

{\it B.} De Schelling
{\scriptsize (Ravaisson avait connu Schelling à Münich (voir R. Ph., juill.-sept. 1952, p. 454).
Il le cite souvent ainsi que Stahl dont il fait grand éloge. Il s’inspire aussi de Van Helmont
de Barthez, ainsi que d’Aristote, de Plotin et de Leibniz)},
ce vitalisme est passé, en France, chez F. Ravaisson
(1813-1900). L'idée qui domine dans la philosophie de Ravaisson
est celle de {\it la vie}. La \textbf{\textit {nature}} comme l’être, consiste dans « le principe
de la vie ». C’est avec la vie que commence l’individualité et, en ce
sens, « il n’y a d’êtres, à parler exactement, que les vivants ». Comment,
en effet, nous apparaît la vie, « sinon comme une sorte de mouvement
par lequel le vivant se crée incessamment lui-même ? Et la vie
n'est-elle pas partout dans le monde? Qui sait même si elle n’est pas
tout? » Toute la suite des êtres n’est que la progression « d’un seul et
même principe » et, jusque dans le cristal, comme l’a dit Herder, on
peut discerner « un instinct aveugle, mais infaillible ». Le {\it mouvement}
lui-même est « une sorte de vie » et, à ce propos, Ravaisson rappelle
l’opinion d’Aristote pour qui, « comme pour les premiers philosophes,
nommément Thalès, c’est à la vie que tout remonte ». Cette vie est
toujours, à quelque degré au moins, {\it conscience}, car « le mouvement
%601
est toujours l'effet immédiat d’une volonté, et cette volonté a toujours
quelque conscience d’elle-même ». La vie ne se fragmente pas,
et le temps qu’elle implique est « une durée définie continue » : c’est
une totalité. Dans sa célèbre thèse sur {\it l’Habitude}, Ravaisson
cherche précisément à montrer que « c’est la même force » qui, « se
multipliant sans se diviser, s’abaissant sans descendre », va de la
volonté pleinement consciente à l’activité automatisée : « Dans le
sein de l’âme elle-même, ainsi qu’en ce monde inférieur [de l’automatisme]
qu’elle anime et qui n’est pas elle, se découvre encore, comme
la limite où le progrès de l'habitude fait redescendre l’action, la
spontanéité irréfléchie du désir, l’impersonnalité de la {\it nature}. »
Cette \textbf{\textit {spontanéité de la vie}} est un « dynamisme irreprésentable et
inexplicable » : si nous ne pouvons comprendre comment se forment
et se réparent les organismes, c’est qu’« échappant, comme l’a vu Stahl,
à toutes conditions d'imagination, [ils] ne peuvent en conséquence
être des objets de calcul et de raisonnement ». Ici encore l'exemple
de l’habitude nous instruit : il nous montre une « intelligence obscure »
succédant à la réflexion, une « intelligence immédiate où l’objet et le
sujet sont confondus » et qui est « une intuition {\it réelle} où se confondent
le réel et l’idéal, l'être et la pensée ». Cette intuition, Ravaisson qui
s’était beaucoup occupé d’art, la rapproche, tout comme Schelling, de
l'{\it intuition esthétique}. L'univers lui-même où Geoffroy Saint-Hilaire
avait cru trouver une « unité de plan » (\S277), n’est-il pas « comme une
pièce de musique où le motif essentiel paraît et disparaît pour reparaître
et pour émerger enfin » ?

\section{Le Vitalisme bergsonien}% 345.
R. Berthelot a montré
comment les différents thèmes qui caractérisent le Vitalisme, principalement
ceux qui avaient été développés par Ravaisson
{\scriptsize (Voir dans {\it La Pensée et le Mouvant} l'étude de Bergson sur {\it La vie et l'œuvre de
Ravaisson} (1904))}, sont venus
se fondre dans la philosophie de Bergson, On a déjà vu quel privilège
Bergson attribue à l'\textbf{\textit {intuition}} sur l'\textbf{\textit {intelligence,}} parce que la
première, étant issue de l'instinct, lui-même modelé « sur la forme
même de la vie », nous conduit « à l’intérieur de la vie » et, à la limite,
serait capable de nous en livrer « les secrets les plus intimes », tandis
que l'intelligence, analytique, mécanique, est affligée d’« une incompréhension
naturelle de la vie ». On a vu aussi comment la {\it sympathie
esthétique} constitue, chez Bergson tout comme chez Ravaisson ou
chez Schelling, le meilleur exemple de cette intuition qui nous
replace « à l’intérieur même de l’objet »(\S326). On verra enfin comment
%602
la \textbf{\textit {liberté,}} selon Bergson, coïncide essentiellement avec la vie puisqu’elle
s'identifie à ce « progrès dynamique » de l’acte volontaire où les motifs
sont « de véritables êtres vivants »: elle est {\it création} (t. II, ch. XXV).

Mais qu'est-ce que la {\it vie} ? Ici Bergson prétend « dépasser à La fois
le mécanisme et le finalisme » qui, au fond, sont dupes tous deux de
la même illusion de l'intelligence fabricatrice.

\vspace{0.24cm}
{\footnotesize Le \textsf{\textit {mécanisme}} découpe artificiellement la continuité dynamique de la vie
en éléments inertes et assimile l’organisme à une machine : « les cellules
seront les pièces de la machine, l'organisme en sera l'assemblage ». C’est
dans cette erreur qu’est tombé l’évolutionnisme de Spencer, critiqué dans
{\it L' Évolution créatrice}, et qui, visant « à reconstituer l’évolution avec des
fragments de l’évolué », supprime en réalité toute évolution et tout devenir.
Mais le finalisme classique ne vaut guère mieux : Leibniz supprime le temps
en en faisant « une perception confuse, relative au point de vue humain »;
Kant distingue finalité externe et finalité interne (\S 276), mais « la finalité
est externe ou elle n’est rien du tout ». Et surtout, le finalisme classique
implique la notion d’un {\it but}, c’est-à-dire d’« un modèle préexistant » ; il
suppose un {\it plan} préétabli. Or « il y a plus et mieux ici qu'un plan qui
se réalise. Un plan est un terme assigné à un travail : il clôt l’avenir dont il
dessine la forme. Devant l’évolution de la vie, au contraire, les portes de
l'avenir restent grandes ouvertes ». En réalité, « la vie travaille {\it comme si} elle
avait des idées générales, celles de genre et d'espèce, comme si elle suivait
des plans de structure en nombre limité ». Mais il n’y a pas de plans. {\it La
finalité n'est pas au terme, elle est au point de départ} : « Elle tient à une
identité d’impulsion, et non pas à une aspiration commune. »}
\vspace{0.31cm}

La vie est « un effort pour greffer sur la nécessité des forces physiques
la plus grande somme possible d’indétermination ». L'évolution des
organismes révèle « un principe interne de direction », un effort, mais
un effort différent de « l'effort conscient de l'individu » et « autrement
profond », en un mot un \textbf{\textit {élan vital.}} Cet « élan vital » est de {\it nature
spirituelle} : « la vie est d’ordre psychologique » ; c’est \textbf{\textit {« la conscience
lancée à travers la matière ».}}

\vspace{0.24cm}
{\footnotesize La conscience est donc « le principe moteur de l’évolution », laquelle,
loin d’être un enchaînement mécanique de types préformés, est au contraire
« créatrice », c’est-à-dire qu'elle est {\it invention}, jaillissement incessant de
formes nouvelles. « Tout se passe comme si un large courant de conscience
avait pénétré dans la matière », mais avait été contraint à la fois de {\it se
ralentir} et de {\it se diviser} en une multitude de séries divergentes par suite des
obstacles que celle-ci lui opposait. La « Conscience en général » est bien
« coextensive à la vie universelles : mais, tandis qu’elle s’assoupissait chez
le végétal, elle s’éveillait de plus en plus chez les autres vivants, se scindant
encore ici en deux directions selon qu’elle « fixait son attention sur son
propre mouvement » — c’est l'{\it intuition} (mais, chez l’animal, « la conscience
s’est trouvée à ce point comprimée par son enveloppe qu’elle a dû rétrécir
l'intuition en instinct ») — ou bien qu'elle la fixait « sur la matière qu’elle
traversait » — c’est l'{\it intelligence}.}
\vspace{0.31cm}

%603
Mais qu'est donc, à son tour, {\it la matière} ? Selon Bergson, comme
selon tous les vitalistes, la réalité est, dans son fond, un courant de
vie, un renouvellement incessant ; c’est « une continuité mouvante »
dans laquelle nos besoins seuls découpent des objets distincts, des
{\it choses} : en réalité, « il n’y a pas de {\it choses}, il n'y a que des {\it actions} ».
L'espace comme le temps homogènes ne font qu’exprimer « le double
travail de solidification et de division » que nous faisons subir à cette
continuité pour nous y assurer les points d'appui de notre action.
Autrement dit, il y a ici « deux processus de direction opposée » et
qui se distinguent par « une différence de tension interne ». Dans
l'un, la tension est au maximum : l'{\it intuition}, en nous concentrant en
nous-mêmes, nous y fait retrouver « l’effort générateur de la vie » ; elle
aboutit ainsi à « la spiritualité ». L'autre qui est au contraire détente,
qui se détache du passé pour se laisser absorber par l'action extérieure
et présente, aboutit à l'extension, caractéristique de « la
matérialité » : c’est la direction de l'{\it intelligence}. {\it Il résulte de là que
« la spiritualité » est à chercher aux antipodes de l'intellectualité} : en
effet, tandis que « l'intuition est l'esprit même et, en un sens, la vie
même », l'intelligence, au contraire, « s’y découpe par un processus
imitateur de celui qui a engendré la matière », et {\it c’est donc « la même
inversion du même mouvement qui crée à la fois l'intellectualité de
l'esprit et la matérialité des choses »}. Ainsi, « la vie est un mouvement,
la matière est le mouvement inverse ». De ce point de vue, cette
dernière apparaît comme quelque chose de négatif : la matière est
« un relâchement de l’inextensif en extensif » et l'{\it extension} n'est
qu’« une tension qui s’interrompt ».

\vspace{0.24cm}
{\footnotesize Si l’on compare la vie à un jet de vapeur qui s'échappe d'un récipient
où la vapeur est à haute tension, la matière pourrait être symbolisée par
les gouttelettes d’eau qui retombent : leur chute n'est que « la perte de
quelque chose, une interruption, un déficit ». Ou plutôt — car cette première
image est trop déterministe — pensons à un geste comme celui d'un bras
qu'on lève : la matérialité sera le bras qu'on laisse retomber, « geste créateur
qui se défait » et dans lequel subsiste pourtant « quelque chose du vouloir
qui l’anima ». Ou encore, si je plonge ma main dans de la limaille de fer,
la vie sera l’élan de ma main qui y pénètre ; la matière, ce sera la limaille
« qui se comprime et résiste à mesure que j'avance ».}
\vspace{0.31cm}

Ainsi, en réalité, tout est un. {\it Il y a dans la matière elle-même une
« participation à la spiritualité »}. Et, comme « la matière et la vie qui
remplissent le monde, sont aussi bien en nous », il ne nous est pas
impossible de « replacer notre vouloir lui-même dans l'impulsion qu'il
prolonge », de retrouver au fond de nous-mêmes « le principe qui n’a
qu'à se détendre pour s'étendre », « le pur vouloir, le courant qui
%604
traverse la matière en lui communiquant la vie ». Nous sommes
immergés dans un « océan de vie » où notre être, « ou du moins
l'intelligence qui le guide », s’est formé « par une espèce de solidification
locale ». Les individualités ne sont que «les ruisselets entre
lesquels se partage le grand fleuve de la vie ». En {\it remontant} la pente
qui descend de la vie vers la matière, « l'intelligence, se résorbant
dans son principe, revivra à rebours sa propre genèse ». La philosophie
n’est rien d’autre que cet « effort pour {\it se fondre à nouveau dans
le tout} » et pour « nous porter {\it jusqu'au principe même de la vie en
général} ».

\section{Le néo-vitalisme contemporain}% 346.
On a vu au \S283
qu’en opposition avec toutes ces théories vitalistes, {\it la Biologie
contemporaine a accompli tous ses progrès, effectué toutes ses découvertes
dans le sens d’une conception, sinon mécaniste, du moins} physico-chimique
{\it de la vie}. Toutefois ces résultats, ainsi que l’a remarqué
avec perspicacité Abel Rey, peuvent « être la base d’une double
offensive ». On peut certes — et c’est même leur signification la plus
obvie — les interpréter dans le sens d’une réduction de la vie organique
à un ensemble de phénomènes physico-chimiques très complexes.
« Mais, avec autant de logique, on peut considérer que, si la vie obéit
aux lois de la matière et paraît en continuité avec celle-ci, c’est que
la matière brute, inorganique, n’est qu'une superficielle illusion :
au fond, elle enferme déjà les éléments actifs qui rendront possible,
dans certaines conditions, la vie et lui donneront naissance. La vie
ne fait qu’exalter et rendre saisissable à l'observation la plus grossière
des forces, des activités qui sont déjà dans la matière brute et qu'y
discernerait une observation plus fine. » C’est ce qui explique qu’on
ait assisté de nos jours à une renaissance, chez certains savants, des
théories vitalistes et même animistes puisque le principe de la vie
y est conçu, comme chez Bergson, de façon psychologique et plus ou
moins identifié à la conscience. Il n’est pas jusqu’à la médecine animiste
qui n'ait connu un regain de faveur avec la médecine dite
« psycho-somatique ».

\vspace{0.24cm}
{\footnotesize Tandis que H. Driesch ressuscitait les vieilles {\it entéléchies} aristotéliciennes,
les biologistes {\it holistes} superposaient aux mécanismes physico-chimiques une
conception « totalitaire » de l'organisme \S283). D’autres parlaient d'une
« conscience cellulaire » (Pierre-Jean). Fr. Houssay définissait la vie
comme « une réhabilitation d'énergie », et L. Guénot même admettait un
« finalisme mitigé » caractérisé, chez l'être vivant, par une faculté d'{\it invention}
qui d'ailleurs reste « un mystère ». — C'est une notion analogue que
nous trouvons dans les théories plus récentes du paléontologiste Albert Vandel,
qui s'inspire de la doctrine bergsonienne. Selon lui, « la vie a deux faces ».
%605
Elle offre un aspect mécanique, car « la vie n’est pas toujours création »;
il y a une « évolution régressive ». Mais, sous l’autre aspect, elle nous présente
le vivant comme un être qui « se fait », ainsi qu'avait dit Bergson,
alors que la matière est une réalité qui « se défait ». Elle s'oriente alors
« vers la spontanéité, l'invention et la création » ; elle apparaît comme une
puissance d'« organisation répondant à un but ». Dans l’ensemble, l'évolution
manifeste « la montée de l’esprit » : du psychisme protoplasmique, on
passe au psychisme nerveux; de l'obscure intelligence de l'espèce, à l’intelligence
individuelle, grâce à laquelle l’homme se dépassera lui-même.
La conscience {\it émerge} de la vie, comme la vie elle-même semble {\it émerger}
de la matière. Et dès lors, puisque l'adaptation biologique ne peut se
comprendre que par analogie avec le fait {\it psychique} de l'invention, pourquoi
rejeter toute « expression anthropomorphiques ? pourquoi même ne pas
penser qu’au lieu que ce soit la vie qui se réduise à la matière, ce soit au
contraire la matière inanimée qui est une {\it dégradation} de la matière vivante ?
A. Vandel reconnaît toutefois qu'il eût été souhaitable d'employer deux
termes distincts pour l'{\it invention organique} et pour l'{\it invention humaine}.
Combien étrange en effet est cette « puissance d'invention » qui, du propre
aveu de l’auteur, « s’attarde volontiers dans des voies sans issue » et semble
se plaire « à multiplier les formes dégradées »!}
\vspace{0.31cm}


\section{Discussion du Vitalisme}% 347.
Une première remarque
s'impose. Malgré les différences de détail, les doctrines que nous
venons d’exposer coïncident par un certain nombre de thèmes ou de
caractères communs : on peut dire qu’il y a une tradition vitaliste
qui remonte, on l’a vu (\S 342), aux toutes premières doctrines de la
philosophie grecque
{\scriptsize (Il faudrait sans doute faire une part aussi aux influences {\it orientales}, perceptibles
chez les Stoïciens et surtout chez Plotin. Sur les affinités de la pensée de Bergson avec
celle de l'Orient, voir le curieux livre de Lydie Aboirux :
{\it La philosophie religieuse de Bergson})} et dont les échos se retrouvent dans les doctrines
modernes, à tel point qu’un Ravaisson, par exemple, se recommande
encore de Thalès.

{\it A.} Le thème fondamental est celui du \textbf{\textit {finalisme.}} Il se retrouve,
avec des variantes, chez Aristote comme chez les Stoïciens, chez
Ravaisson et Bergson comme chez certains biologistes contemporains.
Mais on a vu (\S276) que la signification du finalisme dans la science est
surtout d'imposer une limite au déterminisme analytique et mécaniste,
de sauvegarder la notion fondamentale de la solidarité du vivant, et
qu’en dehors de ce rôle surtout négatif, il se révèle le plus souvent
{\it stérile} sur le plan scientifique, parfois {\it illusoire}. On a vu également que
cette notion de finalité a été interprétée dans les sens les plus divers, au
point de se réduire chez Bergson à celle d’un pur {\it dynamisme}, de ce
que nous avons appelé « une finalité sans fin ».

{\it B.} Dans l’ensemble cependant, et malgré la tentative de Goblot
%606
pour faire accepter une finalité « non-intentionnelle » (\S 276), l’expérience
montre qu'il est bien difficile de séparer la notion de {\it fin} de
celles d’une tendance plus ou moins consciente, plus ou moins analogue
à la volonté humaine
{\scriptsize (Le texte d'A. Vanne cité aux {\it Exercices}, n° 3, est typique à cet égard)}. Le \textbf{\textit {glissement de l'organique au psychologique,}}
par l'intermédiaire de cette ambiguïté, conduit alors à affirmer que
« l’acte vital est de même nature que l'acte de conscience » (Bergson),
voire à {\it personnifier} la vie et la nature elle-même
{\scriptsize (Voir le texte de Bergson cité aux {\it Exercices} n° 4)}, imaginées de façon
tout anthropomorphique. A. Lalande signale les « graves et fréquentes
équivoques » qui résultent de l’{\it emploi métaphorique du mot
« vie »}, transféré ainsi du domaine biologique au domaine spirituel
(la {\it vie} de l'esprit, la {\it vie} morale, la {\it vie} religieuse, la {\it vie} des vérités,
etc.), équivoques qui peuvent même s'étendre au domaine politique
et social
{\scriptsize (Voir l'{\it Exercice} n° 5)}.

{\it C.} On en vient ainsi à prendre de telles \textbf{\textit {métaphores}}
{\scriptsize (Le P. Laberthonnière parle un peu sévèrement, à propos de Bergson, de « la
richesse de son langage en métaphores, en comparaisons, en images qui surgissent,
abondantes et surabondantes, à toutes les pages, on pourrait dire à toutes les lignes,
et qui, en éblouissant le lecteur, lui fait trop souvent croire qu'il pense quand il ne fait
qu'imaginer »)} pour des
idées explicatives. Les termes de {\it tension}, de {\it détente}, de {\it dispersion},
d’{\it élan} qu’on rencontre chez tous les vitalistes, depuis Héraclite
jusqu’à Bergson en passant par les Stoïciens, ne sont guère que cela.
Celui d'{\it invention} dont usent, comme on a vu, même certains savants
contemporains, est une transposition si manifeste, en sens inverse,
du, domaine psychologique au domaine biologique qu’A. Vandel
reconnaît que « l'emploi d’un même mot pour désigner des phénomènes
certainement très différents risque de créer des confusions ».

{\it D.} L'abus de ces métaphores donne souvent au Vitalisme un
caractère plus \textbf{\textit {littéraire}} que philosophique. Nous avons montré les
affinités du Vitalisme moderne avec le Romantisme (voir R. Berthelot).
D’où cette tendance à privilégier l'intuition \textbf{\textit {esthétique}}
par rapport à la réflexion philosophique, à préférer la poésie, l’art,
la musique comme expressions de « la vie » à toute traduction conceptuelle.
Ce caractère du Vitalisme n’a d'ailleurs pas peu contribué à
son succès
{\scriptsize (Sur le succès du bergsonisme, voir Julien Benda, {\it Une philosophie pathétique})},

{\it E.} Du romantisme on passe facilement à l’\textbf{\textit {occultisme.}} De tous
temps d’ailleurs, le Vitalisme a supposé dans la nature des {\it forces
cachées} qu’on ne peut découvrir que par des procédés spéciaux.

%607
Les Stoïciens faisaient grand cas de la divination, Plotin de l’astrologie ;
un Paracelse se livre à de véritables divagations à propos des prétendues
{\it correspondances} entre le « macrocosme » (l’univers) et le « microcosme »
(l'organisme humain) ; Schelling ambitionnait de retrouver « la
clé de la vieille magie » et sa philosophie de la nature confine, selon Bréhier,
à la {\it théosophie} et au {\it spiritisme} ; Bergson lui-même a fait confiance aux
« expériences spirites » et à la {\it métapsychie}.

{\it F.} Cette tendance du Vitalisme se renforce du fait qu’il est aussi,
comme nous l’avons dit, un \textbf{\textit {totalisme.}} Depuis les Stoïciens jusqu’à
Ravaisson et Bergson, jusqu’à certains biologistes d’aujourd’hui,
un de ses thèmes favoris est que « tout est un », ce qui conduit à
confondre tous les paliers de l'être et à considérer l’univers, à la
manière de l’hylozoïsme primitif, comme un organisme où « tout se
tient » et où, par suite, un élément permet de deviner les autres.

{\it G.} Rien d'étonnant dès lors si le Vitalisme qui, on le sait (voir
\S 143 et 235), a fait obstacle au progrès scientifique, répudie la pensée
analytique et le point de vue mathématique de la quantité pour ne
reconnaître valable qu’une \textbf{\textit {intuition}} essentiellement qualitative,
par laquelle la conscience individuelle s’identifie à « la conscience
universelle » en s’absorbant en elle.

\vspace{0.24cm}
{\footnotesize Nous avons noté que, déjà chez Héraclite, le Vitalisme s'oppose au
{\it mathématisme} pythagoricien. De même, la physique qualitative d'Aristote
(\S 110) s'oppose à l’essai de physique quantitative de Platon. Paracelse
soutient que c’est le démon qui a rendu l’âme purement « logastrique »,
c'est-à-dire esclave du raisonnement, en lui arrachant sa partie céleste.
Schelling écrivait en 1806 : « La nature sait, non par science, mais par
son essences même ou de manière magique : le temps viendra où les sciences
cesseront de plus en plus et où la connaissance immédiate apparaîtra.» On
a vu que Ravaisson considère la vie comme {\it irreprésentable et inexplicable},
comme échappant au calcul et au raisonnement. Bergson nous dit, à propos
de l'instinct, qu’il ne faut pas chercher à rendre {\it intelligible} ce qui n’est
pas de la nature de l'intelligence. Même un L. Guénot reconnaît que la
faculté d'invention qu’il suppose dans la nature vivante, est « un mystère »
{\scriptsize (La philosophie, — sinon la science, — ne doit peut-être pas répudier tout « mystère ».
Mais encore faut-il, comme on le verra plus loin (\S 351 fin), qu’elle sache le placer
où il se trouve)}.}
\vspace{0.31cm}


{\it H.} Le vitalisme est donc lié à l’\textbf{\textit {anti-intellectualisme}}
{\scriptsize (Voir ci-dessus page 572, n. 1)} ; et c’est
par là qu’il prétend, le plus souvent, être un spiritualisme. Mais ne
serait-il pas un \textbf{\textit {spiritualisme à rebours,}} puisqu’au lieu de chercher
la spiritualité du côté de l'intelligence et de la pensée claire, il
représente l'intelligence comme une « détente » et, pour tout dire,
une {\it déchéance} de la spiritualité, tandis qu’il croit au contraire rencontrer
celle-ci du côté de l'{\it instinct} et d’on ne sait quelle {\it pensée obscure},
%608
sans conscience (Leibniz, Stahl, Ravaisson, Bergson)? On se souvient
des réserves que nous avons faites sur la notion bergsonienne de
l'{\it intuition} (\S 326). On verra au tome II que les mêmes réserves s’imposent
à propos de la conception que nous présente Bergson du {\it moi
profond} (chap. VII) et de la {\it liberté} (chap. XXIV), plus proche, telle
qu’il l’entend, de l’impulsion vitale et instinctive que de l’authentique
liberté morale, à propos enfin de ce qu’il nomme la « morale ouverte »
(chap. XVI). — {\it Il n’est nullement évident en effet que le « moi spirituel »
soit dans le prolongement direct du « moi vital »} (\S 116). Comme l’a dit
Brunschvicg, « c’est un préjugé de prétendre qu’en remontant vers
l’élémentaire et le primitif, nous nous rapprochons d’un fond permanent »
de spiritualité. « Bien plutôt, un effort méthodique est requis
afin d’arracher à la nuit de l’inconscience le résidu de l’élémentaire et
du primitif, afin d’en faire décidément justice. » Loin d’en « faire
justice », le Vitalisme, en prétendant établir, par l’entremise du {\it vital},
une certaine continuité entre le {\it matériel} et le {\it spirituel}, ne demeure-t-il
pas dupe de l’\textbf{\textit {antique confusion}} qui est à la base de l’hylozoïsme ?
— « Je ne sais, écrit Brunschvicg, s’il est aisé de tirer au clair la
distinction du vitalisme et du matérialisme. »

\vspace{0.24cm}
{\footnotesize De fait, R. Berthelot a relevé, chez Bergson, l'emploi de métaphores
étrangement matérielles. Préoccupé de distinguer la spontanéité vitale de
la finalité intellectuelle, Bergson la caractérise comme « une poussée, une
force s’exerçant par derrière ». Ici c'est un « jet de vapeur », un geste du
bras, la main pénétrant dans la limaille. Là c’est la tension progressive d’un
« élastique infiniment petit ». Ce langage « ne manifeste-t-il pas à sa manière
une parenté secrète entre le dynamisme soi-disant {\it matérialiste} des Stoïciens
et le dynamisme soi-disant {\it spiritualiste} de Bergson ? » demande R. Berthelot.
On trouverait des exemples aussi curieux chez Ravaisson qui compare
l’interpénétration des consciences individuelles lorsque « les créatures se
réuniront en Dieu », à « ces courants ou ondes électriques qui se traversent
sans s'empêcher », chez Schelling surtout qui s'inspire de l’action de
l'oxygène, de l'électricité et du magnétisme. Ces métaphores n'ont rien
à envier au « feu artistes » des Stoïciens.}
\vspace{0.31cm}

« Ainsi se vérifie l’idée qu’un {\it énergétisme spirituel} peut être aussi
bien interprété dans un sens matérialiste, comme l'ont fait les Stoïciens,
que dans un sens spiritualiste. » D'autre part, tous les faits par
lesquels les vitalistes ont prétendu caractériser les phénomènes
biologiques et dont nous avons donné quelques exemples au \S 283,
ont été ou sont près d’être expliqués par la physico-chimie. C’est donc
ailleurs que doit être placée, comme dit R. Berthelot, « la ligne de
démarcation » entre le matériel et le spirituel : « Ce qui est vraiment
propre à l’esprit, ce ne sont pas les propriétés par lesquelles Bergson
[et les vitalistes] {\it ont} essayé de le caractériser, puisque ces propriétés
%609
lui sont communes dans certains cas avec la matière ; ce sont d’autres
propriétés, {\it celles-là mêmes sans doute par lesquelles l'école cartésienne
et les grands rationalistes ont essayé depuis des siècles de caractériser
l'esprit}. »

\section{Le Spiritualisme : de l’âme à l'esprit}% 348.
Il nous faut
donc maintenant préciser la notion même de l'esprit. Ce que nous
venons de dire du Vitalisme, suffit à montrer que cette notion ne
s’est que peu à peu épurée des éléments étrangers qui s’y trouvent
d’abord mêlés et que, même de nos jours, {\it l’homme a toujours quelque
difficulté à concevoir la spiritualité pure}.

\vspace{0.24cm}
{\footnotesize 
{\it A.} La \textsf{\textit {notion d'âme}} s’est dégagée péniblement du matérialisme primitif :
Héraclite, par exemple, soutient que l’âme se nourrit, par la respiration,
de l'air ambiant sans lequel il n’y aurait ni vie ni raison. {\it Principe vital}
autant que principe pensant, elle a des fonctions organiques et l'on a vu
(\S 342) qu'Aristote fait encore correspondre les diverses espèces d’âmes
aux différentes formes de la vie chez les plantes, les animaux et l’homme.
C'est Platon qui avait, le premier, introduit un élément de spiritualité
en caractérisant l’âme, « prisonnière » dans le corps, comme étant, en même
temps que principe de mouvement, parente des Fées et, à ce titre, participant
au caractère « intelligible, divin et indestructible » de celles-ci : son
dialogue du {\it Phédon} est consacré à établir l'immortalité de l'âme. — Aristote
revient à une idée plus proche du vitalisme : l'âme, dit-il, est « quelque
chose du corps » ; mais ceci ne veut point dire qu’elle soit elle-même corporelle ;
elle est la « forme » du corps, « l’acte premier (entéléchie) d’un corps
organisé ayant la vie en puissance », et l’âme de l’homme contient un
principe, l'intelligence (grec {\it noûs}), qui est « une autre espèce d'âme » et
qui lui vient « du dehors ». — Les Stoïciens reviennent, comme on l’a vu
(\S 342), à l'hylozoïsme et définissent l’âme « un souffle ({\it pneuma}) uni à
notre nature et qui, pénétrant tout le corps, en fait l'unité ».

{\it B.} Cette \textsf{\textit {notion du « pneuma »}}, c'est-à-dire du souffle vital, a joué un très
grand rôle dans la médecine et même la philosophie antiques. La plupart
des médecins grecs font du {\it pneuma}, en même temps que la force qui anime
le corps, l'âme elle-même, de même que, pour les Stoïciens, tout en étant
un corps, le {\it pneuma} possède tous les attributs de l'esprit. Nulle part cependant,
ni chez les médecins ni chez les philosophes, ce terme de {\it pneuma} ne
désigne une réalité immatérielle. C’est seulement chez le néo-platonicien
Philon que cette acception apparaît, vraisemblablement sous l'influence
de la Bible, appuyée d’ailleurs par celle du platonisme. La théologie
hébraïque distinguait en effet entre la {\it nephesch}, âme organique siégeant
dans les entrailles, et le {\it ruach}, principe de la pensée. C'est ainsi la tradition
judéo-chrétienne qui va dégager la notion de l'âme spirituelle.

{\it C.} \textsf{\textit {Le Christianisme}} apporte une notion nouvelle de la vie spirituelle. La
dualité de la nature humaine s'affirme dans l'opposition établie par
saint Paul entre la chair et l'esprit. Ce n’est pas que cette notion de la
spiritualité pure s'établisse sans difficulté : des apologistes chrétiens comme
Tertullien (né vers 155) et Lactance (né vers 250), peut-être même
saint Hilaire de Poitiers (né vers 315), ont pensé que l’âme est de nature
matérielle. C'est saint Augustin qui, faisant la synthèse du christianisme
%610
et du platonisme, fixe la doctrine spiritualiste : l’âme est esprit ({\it spiritus}),
car elle est pure pensée et l’on ne peut attribuer la pensée à la matière ; elle
conserve le passé par la mémoire et peut penser l’abstrait ; elle prend
conscience de soi jusque dans l'erreur : « {\it Si fallor, sum}. Si je me trompe, c’est
que j’existe. » Mais la Scolastique médiévale, s'inspirant d’Aristote plus
que de Platon, reviendra à l'idée de l’âme {\it forme du corps} et continuera à lui
attribuer des fonctions organiques en même temps que spirituelles. Sa
liaison avec le corps sera conçue sous la forme d’une \textsf{\textit {union substantielle}} dans
le « composé humain ».}
\vspace{0.31cm}

\section{Le Spiritualisme cartésien}% 349.
C'est Descartes qui va
définitivement libérer la notion d'{\it âme} de ses attaches vitalistes.
Au mot latin {\it anima} par lequel on la désignait et qui évoque l’idée du
souffle vital, il substitue le mot {\it mens}, esprit. « Les premiers auteurs
des noms, dit-il dans sa controverse avec Gassendi, n’ont pas distingué
en nous ce principe par lequel nous sommes nourris, nous croissons et
faisons sans la pensée toutes les autres fonctions qui nous sont
communes avec les bêtes, d’avec celui par lequel nous pensons : ils
ont appelé l’un et l’autre du seul nom d’{\it âme}. » Selon Descartes,
le corps n’est qu’une machine (\S 337 B) : point besoin par conséquent
d’une âme {\it végétative} ni d’une âme {\it sensitive} ou {\it motrice}.
\textbf{\textit {L'âme est
pure pensée}} : « Aussi, dit-il, l’ai-je le plus souvent appelée du nom
d'{\it esprit} : car {\it je ne considère pas l'esprit comme une partie de l'âme,
mais comme cette âme tout entière qui pense}. »

\vspace{0.24cm}
{\footnotesize 
Une « chose qui pense », c'est « une chose qui doute, qui entend [comprend],
qui conçoit, qui affirme, qui nie, qui veut, qui ne veut pas, qui
imagine aussi et qui sent ». Sans doute, dans ces deux dernières fonctions,
l'esprit est dépendant du corps. Mais il existe aussi des formes de pensée
où l'esprit agit seul : « Dans l’intellection, l'esprit ne se sert que de soi-même,
au lieu que dans l'imagination, il contemple quelque forme corporelle. »
Je puis par exemple {\it imaginer} un polygone de mille côtés, mais
seulement de façon confuse, tandis que je peux le {\it concevoir} de façon
parfaitement claire et distincte.}
\vspace{0.31cm}

Descartes nous offre ainsi un {\it nouveau type de spiritualité}, celui
de la pensée pure, de la pensée « claire et distincte », celle qui consiste
à se libérer du {\it sensible} pour s’élever à l’{\it intelligible}, celle qui fera de la
science elle-même et spécialement de la Mathématique une « expérience
spirituelle » (\S 246). — Il importe toutefois d’observer, ainsi
que nous l’avons fait à propos du {\it cogito} (\S 112), que cet esprit qui est
en nous, qui est nous, nous le saisissons immédiatement, dans l’intuition
du « je pense », grâce à notre expérience intérieure et comme un
existant concret : « {\it Ego} cogito ; {\it moi}, je pense », disent la traduction
latine du {\it Discours} et les {\it Méditations}, et c’est pourquoi l’intuition de la
pensée se confond pour Descartes avec celle de l’âme. Le {\it cogito} cartésien,
%611
tout en ouvrant la voie à l’idéalisme moderne, est orienté vers
le {\it spiritualisme} plutôt que vers l’idéalisme.

Ce caractère du Cartésianisme va d’ailleurs s’accentuer chez
Malebranche. Certes celui-ci, au moins autant que Descartes, professe
que {\it « l'esprit pur », c’est « l’entendement »} et que « c’est une
erreur de prendre l’imagination pour l'esprit ». Ses {\it Entretiens sur la
Métaphysique} nous mettent en garde contre l'erreur qui consiste à
confondre nos « sentiments » avec les « idées », le {\it sentir} avec le
{\it connaître} : « Ce n’est que par les idées pures et exemptes de fantômes
que l'esprit s’unit à la vérité » et s'élève au « pays des intelligences ».
Le platonisme de saint Augustin vient d’ailleurs renforcer ici l’inspiration
cartésienne (\S 323). Mais Malebranche, soutient d'autre part
que nous n’avons pas, à proprement parler, d'\textbf{\textit {idée}} de notre âme,
nous en avons seulement \textbf{\textit {« sentiment intérieur ».}} Nous « sentons » nos
modifications, nous ne les « connaissons » pas, et « nous ne savons de
notre âme que ce que nous sentons se passer en nous », Ainsi, l’intuition
du {\it cogito} porte sur l'{\it existence} de l'âme, non sur son essence. Par là,
Malebranche ouvrait la voie à une Psychologie fondée sur l'{\it expérience
intérieure}.

\section{Le Spiritualisme contemporain}% 350.
C’est précisément
en ce sens que s’est orienté le Spiritualisme français au {\footnotesize XIX}$^\text{e}$ siècle
et de nos jours — {\it A.} Maine de Biran (fig. 93) fait ici figure de précurseur.
Il privilégie le point de vue du {\it sujet} en remarquant qu’un être
capable de {\it se connaître lui-même} n’est nullement comparable aux
choses qu’on ne peut connaître que du dehors. Or précisément cette
expérience intérieure du sujet lui fait découvrir un « fait primitif »
qui n’est autre que \textbf{\textit {l'effort moteur volontaire}} où le {\it moi} ou plutôt
le {\it je} se saisit lui-même, non certes comme une « substance » ainsi que
l’avait cru Descartes, mais comme une « force », une « cause active »,
— et cela dans sa relation avec la {\it résistance} que l’inertie corporelle
oppose à l’action de la volonté. {\it Cette force par laquelle se manifeste
l'activité du moi, n’est pas une force vitale, c'est une force} \textbf{\textit {spirituelle}} ou,
comme dit Biran, « hyperorganique » : elle est donnée dans une
expérience spécifiquement psychologique. Elle agit aussi d’ailleurs dans
l'{\it effort d'attention}. Dès 1795, Biran notait dans son {\it Journal} : « Je
me sens différemment modifié lorsque, laissant errer mon imagination,
je vois se succéder une foule d’images sans suite, et lorsque, méditant,
comparant, calculant, je range mes idées dans un certain ordre, je
cherche avec attention leurs rapports. Voilà deux états bien différents. »
Plus tard, il croira même expérimenter un état où, dépassant
« l'effort ordinaire de l'attention », l'esprit se libère de l'obstacle
%612
que le corps oppose à « l'intuition interne » et ne fait plus « que recevoir
la lumière qui lui est appropriée » (\S 351 F).

{\it B.} Par la suite, le Spiritualisme va se lier à l’\textbf{\textit {Idéalisme,}} par
exemple chez Lachlier (1832-1918) qui s'inspire de Ravaisson
en le corrigeant dans le sens intellectualiste. Lachelier rappelle
le mot de Bossuet : \textbf{\textit {« Ce qui est proprement spirituel, c'est
ce qui est intellectuel. »}} Dépassant la psychologie empirique, il s’élève
à « l'analyse réflexive » et de là à une métaphysique qui reconstitue
par synthèse le « progrès dialectique de la pensée ».

\vspace{0.24cm}
{\footnotesize Ce progrès nous mène de la conscience sensible, nécessaire à notre
représentation du monde « puisque ce monde ne peut exister sans elle»,
à la {\it conscience intellectuelle} qui lie les phénomènes selon des rapports
nécessaires, et enfin à une troisième conscience qui est {\it conscience pure} :
c'est une « connaissance réfléchie des deux autres»  et qui, « supérieure à
toute nature et affranchie de toute essence », est {\it pure affirmation de soi}
et pure {\it liberté}. La pensée apparaît ainsi comme « l'être idéal qui contient
ou pose {\it a priori} les conditions de toute existence », et « le dernier point
d'appui de toute vérité et de toute existence, c'est {\it la spontanéité de l'esprit} ».}
\vspace{0.31cm}

{\it C.} Émile Bourroux fonde son Spiritualisme sur sa philosophie
de la \textbf{\textit {contingence}} (t.II,ch.XX V). {\it L'esprit n'est pas plus nécessairement
lié à la vie que celle-ci à la matière}. Certes on pourrait être
tenté de voir dans la conscience
%Fig. 93. — Maine de Biran.
qu’on attribue aux êtres inférieurs un terme de passage rétablissant
la continuité avec les phénomènes physiologiques. Mais la conscience
%613
{\it humaine} présente « plus qu’une différence de degré » avec celle-ci, qui
n’est qu’« un agrégat de sensations conscientes, sans lien entre elles ».
Chez l’homme, la conscience est « {\it l'acte par lequel une multiplicité
et une diversité d'états sont rattachés à un moi et à un seul, l’appropriation
des phénomènes à un sujet permanent} ». Ainsi, « plus que tous
les autres êtres, la personne humaine a une existence propre, est à
elle-même son monde ».

{\it D.} Le Spiritualisme de R. Le Senne (1883-1954) a été qualifié de
\textbf{\textit {spiritualisme existentiel.}} Il combine un {\it idéalisme absolu} qui prend son
point de départ dans l’œuvre de Hamelin (\S 115 C), avec des inspirations
issues du Bergsonisme et de la Phénoménologie.

\vspace{0.24cm}
{\footnotesize 
Privilégiant l'{\it expérience totale} « qui est l'existence même », mais se refusant
cependant à sacrifier, avec Bergson, « la détermination intellectuelle »,
il assigne pour but à la Philosophie de « revenir par l'intelligence à l’expérience
commune et invariables ». Or l'expérience n’est possible que par un
{\it esprit}. Mais qu'est-ce qu’un esprit ? C'est un centre, un « foyer d'attention»
qui fait de l’expérience un tout : « Quand j’affirme que je suis un esprit,
je veux dire que je me distingue des choses par la conscience que corrélativement
j'ai d'elles et de moi. » Il y a donc, au sein de cette « unité
dynamique de liaison », une {\it ambiguïté radicale} : son essence est d’être une
relation entre extériorité et intériorité, entre le sujet et l’objet, entre l’être
et le devoir-être, entre l'Esprit infini et la multitude des esprits finis. Le {\it moi}
qui est l’un de ces esprits finis trouve en lui, non seulement des déterminations,
mais aussi des contradictions. Mais la contradiction est féconde : le
doute qui est sa forme intellectuelle, permet à la connaissance de se développer ;
la douleur qui est sa forme sensible, est à l’origine de la moralité :
{\it la conscience naît de l'obstacle}. La « conscience troublée » est la source de la
vie de l'esprit : c'est ainsi que le {\it moi empirique} déborde ses propres limites,
s'élève jusqu’au {\it moi de valeur} et, en apercevant la valeur (qu’il {\it découvre},
mais ne crée pas), retrouve l’Absolu.}
\vspace{0.31cm}

{\it E.} Plus nettement {\it existentielle} encore est la Philosophie de l'esprit
de L. Lavelle (1883-1951). Avant J.-P. Sartre, Lavelle a posé la
thèse selon laquelle « chaque homme choisit en quelque sorte son
passé », de sorte que la liberté est « constitutive du moi ». Mais le
Spiritualisme de Lavelle est aussi un \textbf{\textit {essentialisme}} qui s’apparente
à la philosophie de Malebranche.

\vspace{0.24cm}
{\footnotesize D'après lui, en effet, « {\it l'expérience fondamentale n'est pas celle de la contradiction,
mais celle de la} \textsf{\textit {participation}} ; il suffit, pour la dégager, de se dépouiller
de tout attachement à l'égard des modes particuliers. et de descendre jusqu’à
l'essence affirmative de son être même ». Si l'esprit se saisit lui-même comme
une {\it activité}, comme la seule activité digne de ce nom (car toute activité
matérielle est causée ou subie plutôt que causante et agissante), c’est la
{\it participation} qui produit l'apparition de la {\it conscience} : « Pour être, notre
pensée doit saisir, en le faisant sien, un des aspects de l'être total, ce qui
lui permet de se distinguer de l’être et pourtant d’en faire partie. » Le moi
%614
se {\it choisit} et se {\it crée} ainsi, dans l'expérience intérieure, en participant à cet
être total qui est essentiellement \textsf{\textit {Acte}} et dont il reçoit son initiative. La
pensée est « participation à la présence infinie de l’Esprits et la liberté
elle-même est participation et imitation de la puissance créatrice. {\it L'existence
n'est que le moyen par lequel l'essence se conquiert}, et l'essence est
parente de la valeur, par l'intermédiaire de laquelle s'effectue cette participation
à l'être. La vocation de chacun est ici-bas de réaliser les virtualités
de \textsf{\textit {valeur}} qu’il porte en lui.}
%
\vspace{0.31cm}

On voit que, dans l’ensemble, le Spiritualisme contemporain, du
moins en France, n’a pas été infidèle à l'inspiration cartésienne :
il est demeuré {\it intellectualiste}. Mais il se pose alors pour lui un problème
qui déjà s’était posé, de façon aiguë, pour Descartes : le problème des
rapports de l'esprit et du corps.

\section{Les rapports de l'esprit et du corps}% 351.
Tant que l’âme
est conçue plus ou moins comme un principe vital, ses attaches avec
le corps ne sont pas rompues. On a vu que les Scolastiques avaient, avec
Aristote, défini l’âme comme « la forme du corps » et que Leïbniz
{\scriptsize (Chez Leïbniz, le corps représentant le point de vue des autres monades les
rapports de l'esprit et du corps se ramènent à l'harmonie préétablie entre les
monades (\S 118 4))}
avait caractérisé la matière comme une {\it mens momentanea}, un esprit
enfermé dans l'instant. On peut dire qu’à ce stade, le problème des
rapports entre l'esprit et le corps n’est pas posé, puisque les deux
termes sont, plus ou moins, comme dans les doctrines vitalistes,
ramenés à un seul, sans qu’on sache bien toujours, comme on l’a vu,
au profit duquel des deux. À partir du moment au contraire, où
l’âme est définie comme \textbf{\textit {pure pensée,}} comme \textbf{\textit {esprit,}} le problème
se pose, mais devient difficile à résoudre.

{\it A.} \textbf{\textit {Le dualisme cartésien.}} Pour Descartes, l’âme est « une
substance dont toute la nature n’est que de penser ». D’autre part,
« la nature de la matière ou du corps pris en général » consiste uniquement
« en ce qu’il est une substance étendue en longueur, largeur et
profondeur » (cf. \S 340 B). Comment ces \textbf{\textit {deux substances}} qui se
conçoivent chacune séparément de façon claire et distincte, qui sont
même opposées l’une à l’autre puisque l’une est indivisible, l’autre
divisible à l'infini, peuvent-elles s’unir dans l’être humain? C’est un
fait cependant que la pensée {\it imaginative} d’une part, les « sentiments
de douleur, de faim, de soif, etc. » d’autre part me montrent, non
seulement qu’en tant qu’âme, je suis uni à un corps, mais que je n'y
suis pas logé « ainsi qu’un pilote en son navire » (pour reprendre une
expression d’Aristote), mais « que je lui suis conjoint très étroitement
%615
et tellement confondu et mêlé, que je compose comme un seul tout
avec lui
{\scriptsize (Dans les Passions de l'âme, Descartes répète que « l'âme est véritablement jointe
à tout le corps ». Il ajoute cependant qu'elle « exerce ses fonction plus particulièrement »
dans une petite glande du cerveau, glande pinéale (épiphyse), grâce à laquelle elle
dirige Les « esprits animaux » dans les différentes parties du corps et produit ainsi nos
mouvements)}. Et, protestant contre certaines interprétations qui lui
faisaient dire que l’homme, comme formé d’une âme et d’un corps,
est un « être par accident », Descartes précise que « l’homme est un
véritable {\it être par soi}, et que l’âme est réellement et substantiellement
unie au corps, non par sa situation et sa disposition, mais par une
véritable union ». C’est donc à la notion scolastique de l’\textbf{\textit {union substantielle}}
que Descartes en revient. Certes, il concède que l’âme et
le corps ne sont pas nécessairement liés puisqu'il est de leur nature
de pouvoir exister l’un sans l’autre, mais l’union des deux substances
n’est, à ses yeux, contingente que parce qu’elle a été voulue par le
Créateur : elle constitue désormais {\it la nature même de l’homme}. — On
verra plus loin les objections que cette doctrine a soulevées et comment
Descartes y a répondu.

{\it B.} \textbf{\textit {Le monisme spinoziste.}} Spinoza reproche à Descartes d’avoir
conçu l’âme « si radicalement distincte du corps » qu’il n’a pu expliquer
leur union. Pour lui, la Pensée et l’Étendue, tout en étant chacune
« conçue par soi », ne sont pas des substances, mais des \textbf{\textit {attributs}}
de la Substance unique et infinie, qui est Dieu (\S 355). Chez l’homme,
« l’âme et le corps sont une seule et même chose qui est conçue tantôt
sous l’attribut de la Pensée, tantôt sous celui de l’Étendue ». L'âme
de l’homme n’est rien d’autre que l’idée de son corps, et par là « nous
connaissons, non seulement que l’âme est unie au corps, mais aussi
ce qu’il faut entendre par union de l’âme et du corps ». Le dualisme
cartésien se charge ainsi en un \textbf{\textit {monisme.}} Mais Spinoza ne se fait-il pas
illusion en croyant avoir résolu par là la difficulté à laquelle se heurtait
Descartes ? est-il plus facile de concevoir comment la Substance
unique nous apparaît sous les deux attributs de la Pensée et de
l’Étendue que de comprendre l’action mutuelle des deux substances ?
Le monisme n’a qu’un avantage : c’est d'établir une sorte de {\it parallélisme}
entre ce qui se passe dans l'esprit et ce qui se passe dans le
corps. En effet, dit Spinoza, « l’ordre et l’enchaînement des choses
est un, que l’on conçoive la Nature sous tel attribut ou sous tel autre ;
et, par conséquent, l’ordre des actions et passions de notre corps
concorde par nature avec l’ordre des actions et passions de l’Ame »,

{\it C.} \textbf{\textit {L'occasionalisme de Malebranche.}} Malebranche critique la
notion scolastique d'{\it union substantielle} qui, dit-il, n’est pas un
%616
« principe de l'explication ». Au reste, on sait (\S265A) que, selon lui,
aucun être créé ne possède de « pouvoir causant ». Or un rapport de
causalité serait encore plus difficile à concevoir entre l’âme et le corps
que partout ailleurs. En effet, « {\it l'esprit et le corps n'ont entre eux
aucun rapport essentiel} ». Comment comprendre que le corps, qui
n’est qu’étendue (\S 340 B), puisse agir sur un esprit, et réciproquement
que ma volonté puisse même me faire lever le bras, puisqu'il me
faudrait pour cela connaître le trajet des « esprits animaux » dans
tous mes nerfs et muscles, alors que l’homme le plus ignorant accomplit
ce mouvement sans difficulté? Et pourtant, « l'expérience me
convainc que mon esprit dépend de mon corps »
{\scriptsize (d'ailleurs là, selon Malebranche, une véritable « contradiction », un scandale
métaphysique, lequel ne s'explique que dans la perspective de la doctrine chrétienne ;
l’âme est « en épreuve dans notre corps » à la suite du péché originel)}, Comment expliquer
cela ? C’est que Dieu a établi certaines {\it lois générales de l'union de
l'âme et du corps}, il a voulu « que j’eusse certains sentiments, certaines
émotions, quand il y aurait dans mon cerveau certaines traces, certains
ébranlements d’esprits. Il a voulu, en un mot, et il veut sans
cesse que les modalités de l’esprit et du corps fussent réciproques ».
Ainsi, nos états d’âme, nos états corporels redeviennent causes les
uns des autres, mais seulement \textbf{\textit {causes occasionnelles}} : Dieu seul est
proprement {\it cause}, mais il « communique sa puissance aux créatures
et les unit entre elles » en établissant leurs modalités « causes occasionnelles
des effets qu’il produit lui-même, causes occasionnelles qui
déterminent l’efficace de ses volontés, en conséquence des lois générales
qu'il s’est prescrites ».

{\it D.} \textbf{\textit {Le parallélisme psycho-physiologique.}} Cette théorie de Malebranche,
plus encore que celle de Spinoza, ouvrait les voies à une
conception très positive, d’après laquelle les deux séries de nos états
d'âme et de nos états corporels, des phénomènes psychologiques
et des phénomènes physiologiques, sont liées entre elles par des
rapports généraux, des {\it lois}, sans qu’il soit nécessaire de faire intervenir,
à proprement parler, une action causale de deux substances
hétérogènes. De fait, au cours du {\footnotesize XIX}$^\text{e}$ siècle, beaucoup de philosophes
et de psychologues crurent trouver dans le {\it parallélisme psychophysiologique}
une hypothèse de travail utile à leurs recherches.
Bergson lui-même a reconnu que, sous cette forme, le parallélisme
a pu rendre des services. Mais nous n’avons pas à revenir sur cet
aspect de la question qui a déjà été examiné ci-dessus (cf. \S 31 C).
C’est ici du parallélisme {\it doctrinal} qu’il s’agit. De ce point de
vue et sans revenir sur les critiques qui peuvent être adressées au
%617
parallélisme méthodologique et qui valent {\it a fortiori} ici (caractère
purement hypothétique, et d’ailleurs peu vraisemblable, d'un parallélisme
{\it terme à terme} ; conception inexacte des rapports de l’organisme
avec la pensée consciente qui fait de celle-ci une sorte de doublure
du fonctionnement de celui-là ; etc.), il suffira de remarquer qu’une
telle interprétation ne résout rien. En admettant même que le parallélisme
fût rigoureux entre le corporel et le spirituel, une telle correspondance
serait à expliquer, elle poserait un problème, loin de résoudre
quoi que ce soit. De fait, le parallélisme a été interprété philosophiquement :
1° soit dans un sens {\it matérialiste}, lorsqu’on lui fait dire, avec
Bergson : « Un état cérébral étant posé, un état psychologique
déterminé s'ensuit »
{\scriptsize (Cette interprétation est d'ailleurs bien arbitraire : sous le nom de {\it parallélisme},
Bergson critique, en réalité, l'{\it épiphénoménisme}, Comme le fait remarquer D. Parodi,
« une relation unilatérale et partielle n'est pas un parallélisme »)} ; il aboutit alors à l’{\it épiphénoménisme}, déjà
critiqué \S 31 B et 339 ; — 2° soit, au contraire, dans un sens plus ou
moins {\it leibnizien}, lorsqu'il privilégie la série des faits spirituels (par
exemple Ruyer) ; mais c’est faire retour à l’{\it idéalisme} (\S 113 et 119) ;
— 3° soit enfin dans le sens d’un monisme neutre, inspiré du spinozisme
qui, on l’a vu (ci-dessus B), laisse encore la question ouverte.

{\it E.} \textbf{\textit {L'esprit incarné.}} Mais peut-être la question, telle que nous
l'avons considérée jusqu'ici, était-elle mal posée? Nous avons admis
que ce que nous révèle l’expérience intérieure, c’est une sorte d'esprit
pur qui, comme dit Descartes, « pour être, n’a besoin d'aucun lieu ni
ne dépend d’aucune chose matérielle ». Mais en est-il bien ainsi ?

{\it a.} On a vu (\S 350) que, selon Maine De Biran, le moi se saisit
d’abord dans l'effort moteur volontaire. Le « fait primitif » n’est pas
le {\it cogito} cartésien, c’est un {\it volo} : ce n’est pas le « je pense », c’est un
« je veux » ou plus précisément « je meus mon corps ». Ainsi, ce qui
nous est donné, c’est bien une {\it dualité primitive}, mais cette dualité
est intérieure à la conscience. {\it La certitude de notre existence enveloppe
celle des deux termes de cette dualité}, l'expérience immédiate inclut
celle du corps comme {\it résistance} à la volonté motrice. Sans doute,
je ne {\it comprends} pas l’action de ma volonté sur mes muscles, mais
l'évidence du « sentiment intérieur » est supérieure à celle de la
raison. C’est ce qu'avait bien vu Malebranche, encore qu’il ait eu tort,
selon Biran, de refuser au moi la conscience de sa propre activité causale.

{\it b.} En un autre sens, Bergson a montré aussi que le rapport que
nous donne l'expérience intérieure entre notre corps et notre esprit
n’est pas celui qu’a indiqué Descartes. Le {\it parallélisme}, dont Bergson
impute l’origine au cartésianisme, repose sur l’idée de « l’équivalence »
%618
entre l’état psychique et l’état cérébral. Or l'expérience nous enseigne
tout autre chose. Certes, « la conscience est incontestablement liée au
cerveau chez l’homme ». Mais l’étude de faits tels que les aphasies
montre « qu’entre la conscience et l'organisme il y a une relation
qu’aucun raisonnement n’eût pu construire {\it a priori}, une correspondance
qui n’est ni le parallélisme ni l’épiphénoménisme ni rien qui y
ressemble ». Notre pensée consciente est orientée vers \textbf{\textit {l’action :}}
n’avons-nous pas vu que notre {\it perception} consciente ne retient du
réel que ce qui est utile à l’action présente (\S 118), que la {\it mémoire}
ne rappelle à la conscience que les souvenirs capables d'éclairer
l’action commencée et conserve les autres dans l’inconscient (voir le
chap. IX)? Le {\it rôle du corps} est donc de « jouer », de « mimer » la vie
de l'esprit, d’en « extraire tout ce qu’elle a de jouable et de matérialisable »,
« d’en souligner les articulations motrices comme fait le
chef d’orchestre pour une partition musicale ». Le cerveau est l’organe
de ce \textbf{\textit {choix}} entre les divers mécanismes préparés dans notre système
nerveux, notre moëlle notamment, pour répondre aux sollicitations
de l’action présente. Celui qui pourrait voir à l’intérieur n’y retrouverait
de la pensée que « ce qui est exprimable en gestes, attitudes et
mouvements du corps ». Le cerveau « n’est pas, à proprement parler,
organe de pensée ni de sentiment ni de conscience ; mais il fait que
conscience, sentiment et pensée restent tendus sur la vie réelle et
par conséquent capables d'action efficace » : il est « l’organe de
\textbf{\textit {l'attention à la vie}} ». Ainsi, la vie de l'esprit n’est pas un « effet »
de la vie du corps, elle ne lui est pas « équivalente » ni même nécessairement
liée : « Tout se passe au contraire comme si le corps était utilisé
par l'esprit. »

{\it c.} À cette conception « instrumentale », Gabriel Marcel oppose
une conception purement \textbf{\textit {existentielle.}} Selon lui, faire du corps un
instrument, c’est en faire un pur {\it objet} : car il n’est alors qu'un
appareil à fins multiples » considéré uniquement du dehors. Et, par
surcroît, c’est « convertir l’âme en corps » : car, pour qu’elle puisse
utiliser le corps comme outil, il faut qu’elle-même possède les virtualités
que cet outil actualisera. Dès lors, l’âme et le corps deviennent
« des termes qu’on croit strictement définis et qu’on suppose reliés
entre eux d’une façon déterminée » : de là le dualisme, le parallélisme
et les différents systèmes qui ont tenté d'expliquer leurs rapports. De
fait, je puis percevoir mon corps comme une {\it chose}, tel que les autres le
perçoivent et que moi-même je perçois le leur, tel que l’étudient
l'anatomie et la physiologie. Représenté sur le mode de l'{\it avoir}, je
dis alors que j'ai un corps. Mais ce n’est pas ainsi que \textbf{\textit {mon}} corps
m'est donné dans l'expérience : {\it senti} (et non plus représenté) sur le
%619
mode de l'{\it être}, je peux dire {\it en ce sens} (car, bien entendu, il ne s’agit
plus du corps-objet, ce qui nous ferait tomber dans « un matérialisme
grossier et insane ») que « je \textbf{\textit {suis}} mon corps ». « J’existe incarné :
mon corps est immédiatement présent à l’âme, comme le monde au
corps et l’âme au corps. » Il y a ainsi une « priorité absolue » du corps :
car, « lorsque j'affirme qu’une chose existe, c’est toujours que je considère
cette chose comme raccordée à mon corps, comme susceptible
d’être mise en contact avec lui, si indirectement que ce puisse être ».
La médiation du corps est « nécessaire pour faire attention à quoi
que ce soit ». Cette médiation n’étant pas instrumentale, on l’appellera,
faute d’un meilleur terme, « médiation {\it sympathique} ». C’est à ce corps
{\it sujet}, ce corps en première personne, à la fois {\it senti} et {\it sentant} que je
suis « lié fondamentalement et non accidentellement » : c’est en ce
sens que je suis un être \textbf{\textit {incarné.}} Si je me prends ainsi, dans cette
« unité indécomposable » impliquée dans le {\it sentir}, mon existence
devient \textbf{\textit {« un indubitable existentiel ».}} De ce point de vue, « il n’y a
pas de problème des rapports de l’âme et du corps. Je ne peux pas
me mettre en face de mon corps et me demander ce qu’il est par rapport
à moi. Mon corps {\it pensé} cesse d’être mien » : c’est un corps en
troisième personne. Il ne faut pas convertir le {\it mystère} en {\it problème}. Or
il y a là « un type de vérité essentiellement mystérieux », une expérience
de caractère « à la fois mystérieux et intime » où le lien avec
mon corps demeure « opaque », où il n’y a « rien dont je puisse avoir
véritablement une idée, rien de conceptualisable ».

{\it d.} En un sens plus intellectualiste, M. Merleau-Ponty interprète
lui aussi le rapport du corps à l'esprit tout autrement qu’un rapport
instrumental. À vrai dire, du point de vue \textbf{\textit {phénoménologique}} (cf.
\S 118 B), le problème ne se pose plus : car alors, « seul subsiste à titre
originel le rapport du sujet épistémologique et de son objet ». Il n’est
plus question d’interactions causales entre deux {\it substances}, mais de
« différents degrés d'intégration » qui vont du physique au psychique
en passant par le vital : l’esprit n’est pas « une nouvelle sorte d’être,
mais une nouvelle forme d’unité ». Par suite, au lieu de dire que l’âme
{\it agit} sur le corps, il faut dire « que le fonctionnement corporel est
intégré à un niveau supérieur à celui de la vie et que le corps est
devenu vraiment corps humain ». Au lieu de dire que le corps {\it agit} sur
l’âme, disons « que le comportement s’est désorganisé pour laisser
place à des structures moins intégrées ». Ainsi, « les rapports de l’âme
et du corps, obscurs tant qu’on traite par abstraction le corps comme
un fragment de matière, s’éclaircissent quand on voit en lui le porteur
d’une dialectique ». Le corps n’est pas un instrument, c’est plutôt
une manifestation significative de l’âme.

%620
{\it F.} \textbf{\textit {Conclusion.}} Ce que nous retiendrons de ces interprétations,
c’est qu’il y a bien en effet un « indubitable », à savoir que \textbf{\textit {l’homme
réel, l'existant concret se saisit comme « une unité indissoluble »
à la fois corporelle et spirituelle.}} Mais était-il nécessaire pour cela
d’« incarner » à ce point notre esprit dans notre être corporel
{\scriptsize (Les penseurs même les plus spiritualistes, sans parler bien entendu d'une certaine
Psychanalyse, nous répètent aujourd'hui avec une telle insistance que l'homme est
« un être incarné », voire « un être charnel », qu'il en vient envie, comme disait Voltaire...
de marcher à quatre pattes. « Il est dangereux, dit Pascal, de trop faire voir à l’homme
combien il est égal aux bêtes. »)} que
nous puissions dire : « je suis mon corps », ou bien d'aller jusqu'à
élever le corps à la dignité de « porteur d’une dialectique » ? C’est là,
à notre avis, méconnaître {\it deux sortes d'états d'âme} dont nous avons
l'expérience directe : 1° ceux, tels que la souffrance physique, où notre
corps se présente à nous, malgré tout, comme une {\it chose} (voir tome I,
chap. VIII), une chose à laquelle nous nous sentons liés, il est vrai,
mais qui alors est {\it contre} nous, qui s'impose à nous avec une nécessité
brutale, comme un objet {\it extérieur à notre volonté} ; 2° ceux, plus rares
sans doute, tels que la méditation, l'attention intérieure, où nous
expérimentons qu’il ne nous est possible de nous élever à certaines
{\it significations}, de nous « intégrer à un niveau supérieur » qu'en nous
détachant, en quelque mesure, de notre corps, en nous élevant au-dessus
de cette liaison {\it vitale} que nous avons avec lui : l’attention qu'on
a voulu ramener à un état physiologique, n’en est-elle pas jusqu'à un
certain point indépendante (ci-dessus \S 50) ? la connaissance nest-elle
pas parfois une élévation de l'esprit du plan du {\it sensible} à celui du pur
{\it intelligible} (cf. \S 246) ? L'évolution de pensée d’un Maine de Biran
est ici bien caractéristique (\S350) : parti de l'expérience de
l'effort musculaire, Biran en vient, dans ses derniers écrits, à distinguer,
au-dessus de la « vie animale », au-dessus même de la « vie
humaine », une véritable « vie de l'esprit » où « le lien vital de l'âme
avec le corps se trouve affaibli », où l’homme devient capable d’« entendre
la voix de cette Raison éternelle qui parle au-dedans de nous ».
L'expérience intérieure conduit ici Biran à s'exprimer comme Malebranche.

Descartes avait donc bien posé la question : c'est parce qu'il
y a en nous une {\it faculté de l’intelligible} que nous sommes des {\it esprits}.
Et c’est précisément pourquoi on peut parler ici de « mystère »
{\scriptsize (C'est justement pour avoir refusé de reconnaitre le {\it mystère} là où il est réellement
que tous les vitalistes (\S 347 E) et nombre de spiritualistes sont allés le chercher dans
un {\it merveilleux} de bas étage : occultisme, spiritisme, télépathie, etc. Biran lui-même voit
une indication dans le magnétisme, alors à la mode ! C'est du Spiritualisme frelaté)}. Si
l'esprit était aussi étroitement « incarné » qu'on veut bien le dire,
%621
si d'autre part le corps lui-même était déjà porteur de significations,
il n'y aurait pas de « mystère » et le problème, comme le dit M. Merleau-Ponty,
ne se poserait même plus. C’est justement parce que la
vie spirituelle est proprement celle de l'\textbf{\textit {esprit}} au sens cartésien et
que c’est à l’esprit qu’il appartient en propre, comme nous l’a montré
la psychologie de la perception (\S 80-82), de donner des significations,
que l’être humain est ce « paradoxe », cette « chimère », ce
« monstre » dont parle Pascal. Qu’en effet le lien entre les deux éléments
dont est faite notre nature, soit bien, comme on dit aujourd’hui,
\textbf{\textit {existentiel,}} ce serait une naïveté de croire que ce sont les
philosophes contemporains qui l’ont découvert. Saint Augustin
n’écrivait-il pas déjà : « La manière dont les esprits s’unissent aux
corps, est tout à fait étonnante ; elle dépasse l’intelligence de l’homme;
{\it et c'est pourtant cela même qui est l'homme} » ? Mais le témoignage le
plus caractéristique est certainement ici celui de Descartes, lorsqu'il
s’est efforcé de répondre aux objections qui lui étaient faites touchant
la difficulté qu’il y a, dans sa doctrine, à expliquer comment l’âme,
« substance non étendue et immatérielle », peut mouvoir le corps ou
comment celui-ci est capable de troubler l’âme, purement spirituelle,
par « quelques vapeurs ». Nous pouvons, répond Descartes
{\scriptsize ({\it Lettre à la princesse Elisabeth} du 28 juin 1643. Cf. {\it Entretien avec Burman} : « Cela
est très difficile à expliquer ({\it explicatu difficillimum}), mais l'expérience suffit. »)}, concevoir
distinctement, d’une part, l'{\it âme} « par l’entendement pur »,
d’autre part, le {\it corps}, c’est-à-dire l'étendue, soit par l’entendement
seul, soit (et mieux) « par l’entendement aidé de l’imagination ». Mais
l'{\it union} qui est entre les deux, est mieux connue {\it par « ceux qui ne
Philosophent jamais »} et qui « considèrent l’un et l’autre comme une
seule chose » : car « concevoir l’union qui est entre deux choses, c’est
les concevoir comme une seule ». L'homme est « une seule personne
qui a ensemble un corps et une pensée ». Cette union est quelque chose
« que chacun éprouve toujours en soi-même sans philosopher », Il est
donc inutile « d'occuper son entendement » à méditer longuement
cette question. « {\it C’est en usant seulement de la vie et des conversations
ordinaires et en s’abstenant de méditer et d'étudier aux choses qui
exercent l'imagination, qu’on apprend à concevoir l'union de l'âme et
du corps}.» Qu'est-ce à dire, si ce n’est que le rapport entre le spirituel
et le corporel doit être plutôt \textbf{\textit {vécu}} que conceptuellement pensé ?

Mais, s’il en est ainsi, ne serait-ce pas parce que la présence de la
pensée pure, si précaire qu’elle soit en nous, est le signe d’une {\it transcendance} ?
« Le moi, écrivait Maine de Biran dans sa {\it Note sur l’idée
d'existence} (1824), tient à un principe plus haut que lui, savoir : à une
%622
raison suprême, à un {\it Logos}
{\scriptsize (Logos, mot grec qui signifie à la fois Raison et Parole, Chez Platon et les néo-platoniciens, c'est l'Intellect. Chez les Chrétiens, c'est le Verbe)}.» L’existentiel ne nous met pas ici en
présence d’un {\it irrationnel}, mais bien au contraire d’une {\it participation}
(\S 360).

\section{Sujets de travaux}% SUJETS DE TRAVAUX

{\bf Exercices.} — 1. {\it Distinguer de façon précise le sens des termes} : matérialisme,
monisme, monadisme, réalisme. — 2. {\it Faire de même pour les termes} : 
animisme, idéalisme, panpsychisme, pneumatisme (Kant, {\it R. pure}, trad.
Tremesaygues p. 376), spiritisme, spiritualisme. — 3. {\it Commenter ce texte
d'A. Vandel} : « L'adaptation est une manifestation téléologique ou elle n'est
rien. Voici donc que la notion d'adaptation qui tient à l’organique par ses
manifestations morphologiques ou physiologiques, introduit en même temps
du psychologique, en tant qu'activité orientée vers un but. » — 4. {\it Commenter
ce texte de bergson} : « La résistance de la matière brute est l'obstacle
qu'il fallut tourner d'abord. La vie semble y avoir réussi à {\it force d'humilité,
en se faisant très petite et très insinuante}, biaisant avec les forces physiques
et chimiques, {\it consentant} même à faire avec elles une partie du chemin... Il
fallait que la vie {\it entrât dans les habitudes} de la matière brute... Mais la
matière organisée a une limite d'expansion bien vite atteinte... Il fallut
sans doute des siècles d'effort et des prodiges de {\it subtilité} pour que la vie
tournât ce nouvel obstacle. Elle {\it obtint} d'un nombre croissant d'éléments
prêts à se dédoubler qu'ils restassent unis. » — 5. {\it Étudier les rapports du
vitalisme avec les doctrines totalitaires en vous inspirant de ce passage
d'A. Lalande} : « Ces procédures [des gouvernements totalitares] ont mis
en avant la prétention déclarée de placer la Vie au-dessus de la vérité ou
plus exactement de détruire l’idée même de vérité au profit des croyances
jugées nécessaires au triomphe vital d’un superorganisme totalitaire, État
ou collectivité différenciée et disciplinée... Au nom d'une vue biologique du
monde portée à l'absolu, donc d’une soi-disant vérité de fait, on vise à
renverser ce sans quoi il n'y aurait pas de fait. » — 6. {\it Expliquer cette phrase
de Gaston Bachelard} : « La pensée pure doit commencer par un refus de la vie. »
— 7. {\it Étudier cette pensée de Maine de Biran} : « S'il y a une évidence mathématique
que personne ne conteste, quoiqu'elle soit différente de l'espèce de
clarté propre aux représentations sensibles ou même qu'elle lui soit opposée
dans ces fondements, pourquoi n’y aurait-il pas aussi une évidence psychologique
également opposée à la clarté des représentations du dehors ? » —
8. {\it Dégager les idées essentielles de ce texte de Malebranche} : « L'âme n'est
point répandue dans toutes les parties du corps afin de lui donner la vie
et le mouvement ; et le corps ne devient point capable de sentiment par
l'union qu'il a avec l'esprit... Chaque substance demeure ce qu'elle est...
Toute l'{\it alliance} de l'esprit et du corps qui nous est connue, consiste dans une 
correspondance naturelle et mutuelle des pensées de l'âme avec les traces du
cerveau et des émotions de l’âme avec les mouvements des esprits animaux. »

{\bf Exposés oraux.} — 1. {\it Le matérialisme dialectique} d'après H. Lefebvre.
— 2. {\it Le spiritualisme de Maine de Biran} (voir V. Delbos, {\it Figures et doctrines
de philosophes}, VII, et {\it La Philosophie française}, chap. XIII; H. Gouhier,
introd. à son éd. des {\it Œuvres choisies}, Aubier).

%623
{\bf Discussions.} — 1. {\it La matière peut-elle penser ?} — 2. {\it L'intellectualité
va-t-elle dans le sens de la matière ou dans le sens de l'esprit ?}

{\bf Lectures.} — {\it a.} Descartes, {\it Méditations métaphysiques.} — {\it b.} Spinoza,
{\it Éthique}, éd. Classiques Larousse, p. 52 et fragments 2 et 8. — {\it c.} Malebranche,
{\it Entretiens sur la Métaphysique}, éd. Vrin, t. I, entr. I et IV.
— {\it d.} F. Ravaisson, {\it De l'Habitude} (1838), éd. Baruzi, Alcan, 1933. —
{\it e.} Lachelier, {\it Psychologie et Métaphysique} (1885), Alcan. — {\it f.} Le Dantec,
{\it L'Athéisme}, Flammarion, 1906 (sur le matérialisme, 3$^\text{e}$ partie). — {\it g.} Bergson,
{\it L'Évolution créatrice}, Alcan, 1907, chap. III ; {\it h.} {\it L'Énergie spirituelle},
Alcan, 1919, I, II et VII ; et {\it i.} {\it La Pensée et le Mouvant}, Alcan, p. 92, et IX.
— {\it j.} René Berthelot, {\it Un Romantisme utilitaire}, Alcan, 1913, tome II. —
{\it k.} L. Brunschwicg, {\it De la Connaissance de soi}, Alcan, 1931. — {\it l.} L. Lavelle
{\it Le Moi et son destin}, Aubier, 1936. — {\it m.} L. De Broglie, {\it Matière et
Lumière}, A. Michel, 1937. — {\it n.} R. Ruyer, {\it La Conscience et le corps}, Alcan,
1937. — {\it o.} R. Le Senne, {\it Introd. à la Philosophie}, 2$^\text{e}$ et 3$^\text{e}$ parties. —
{\it p.} H. Lefebvre, {\it Le Matérialisme dialectique}, P. U. F., 1939. — {\it q.} XI$^\text{e}$
Semaine de Synthèse, {\it }Qu'est-ce que la Matière ? P. U. F., 1945. — {\it }r. M. Merleau-Ponty,
{\it La Structure du comportement}, P. U. F., 1942, chap. IV. —
{\it s.} L. Brunschwicg, {\it La Philosophie de l'esprit}, P. U. F., 1949. — {\it t.} A.Burloup,
{\it De la Psychologie à la Philosophie}, Hachette, 1950, chap. IV et V,
— {\it u.} G. Marcel, {\it Le Mystère de l'être}, Aubier, 1951, {\it t}. I, 5$^\text{e}$ leçon. —
{\it v.} Fr. Grégroire, {\it La nature du psychique}, P. U. F., 1957. — {\it w.} Milka Lodetti
{\it La matière imaginaire}, éd. Peyronnet, 1959. — {\it x.} Joseph Roy, {\it La métaphysique
de la vie}, Maisonneuve, 1964.

