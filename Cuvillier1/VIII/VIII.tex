\chapter{La réalité du monde sensible}
%
%
%chapitre VIII
%LA RÉALITÉ DU MONDE SENSIBLE
%{\bf —}{\bf —} SOMMAIRE
%109. Position du problème. {\bf —} 110. Le Réalisme vulgaire. {\bf —} 11. Le Réalisme
%philosophique. {\bf —} 112. Le « cogito » cartésien. {\bf —} 113. L’Idéalisme rationa-
%liste : À. Leibniz ; B. Kant. {\bf —} 114. L'Idéalisme empiriste. {\bf —} 115. L'Idéalisme
%dialectique : 4. Fichte ; B. Hegel ; C. Hamelin. {\bf —} 116. L'Idéalisme critique et
%réflexif. {\bf —} 117. L'Idéalisme anglo-saxon. {\bf —} 118. Les doctrines contempo-
%raines : À. le Bergsonisme ; B. la Phénoménologie ; C. l’Existentialisme. {\bf —}
%119. Conclusion.

\section{Position du problème}% 109
Question étrange pour le sens commun que celle de la réalité du monde sensible !

Comme nous l’avons dit (§ 101 A), en effet, celui-ci s’imagine volontiers
que cette réalité est {\it donnée} dans la perception, voire dans la
sensation même, considérée comme une simple copie du monde
extérieur. Il suffirait donc que nous percevions les objets sensibles
pour qu'{\it immédiatement} ceux-ci nous apparaissent comme {\it s'imposant}
à nos sens, donc comme existant en dehors et indépendamment de
nous. La réalité du monde sensible serait ainsi une « évidence » que
le sens commun ne songe même pas à mettre en doute.

Mais l’analyse de la {\it perception} (chap. V) et surtout celle de la
notion d’{\it objet} (chap. VII) nous ont déjà mis en garde contre ce
{\it réalisme vulgaire}. En réalité, comme nous le suggérions à la fin du
chapitre précédent (§ 108), c’est tout le problème des rapports entre
la pensée et le réel qui se trouve ainsi posé. Deux solutions opposées
peuvent être ici adoptées :
% 154

{\bf —} Ou bien l’être en soi, le réel, est d’une nature totalement étrangère
à la pensée ; il y a une « matière »\footnote{{\it Matière} est pris ici au sens où ce
terme s'oppose à {\it forme} : cf. § 324.} de la connaissance irréductible
à l’acte par lequel nous la connaissons : c’est le \textbf{\textit {Réalisme}}.

{\bf —} Ou bien l’être est homogène à la pensée ; « la matière de la connaissance
est totalement réductible à l’acte de connaître » : c’est l’\textbf{\textit {Idéalisme}}\footnote{Voir page 559, note. {\bf —} Nous empruntons cette définition de l’Idéalisme à
A. \textsc{Lalande}.}.

\section{Le Réalisme vulgaire}% 110
C'est ici la \textbf{\textit {réalité sensible}} qui
est érigée en absolu : le monde extérieur existe {\it tel que nous le percevons}.
Le sens commun ne doute guère que la couleur, la chaleur, etc.,
n'existent, comme telles, dans les choses elles-mêmes.

La Physique d'Aristote\footnote{On verra cependant (§ 340) que sa Métaphysique refuse à la matière toute existence
absolue.} est encore très proche de ce réalisme vulgaire :
c'est une physique essentiellement {\it qualitative}, un \textsf{\textit {réalisme des qualités sensibles}}.
Le monde est un ensemble de {\it formes} ou de {\it qualités}, telles que le chaud et le
froid, le sec et l’humide, qui sont les véritables éléments des choses. La
{\it Scolastique} médiévale ne fera guère que reprendre, en l’exagérant encore,
ce point de vue, lorsque, sous les noms de {\it formes substantielles}, de {\it qualités
occultes}, elle érigera à l'état d’entités les propriétés sensibles des choses
(les corps lourds tombent {\it ob insitam gravitatem}, à cause de la gravité qui
est en eux ; la lumière éclaire parce qu’elle a une {\it vis luminosa}, etc.).

L’argument principal du réalisme est l'{\it apparence d'immédiateté}
que prend pour nous la réalité sensible dans la perception extérieure,
où la pensée semble appréhender directement une réalité différente
d’elle. La meilleure preuve de l’existence absolue des objets extérieurs
n'est-elle pas, pour le sens commun, que nous les voyons, que nous les
touchons, etc. ? « Cette terre est réelle : je le sens bien, fait dire
\textsc{Malebranche} à l’un des interlocuteurs de ses {\it Entretiens métaphysiques}.
Quand je frappe du pied, elle me résiste. Voilà qui est solide,
cela ! » Comment, au reste, ajoute-t-on parfois, serait-il possible
d’expliquer toutes ces représentations que nous avons des choses si
elles n'étaient pas causées par quelque réalité étrangère à notre pensée ?

Mais ces arguments sont faciles à réfuter, et le réalisme vulgaire
se heurte à quantité d’objections.

1° Ce sont d’abord les {\it illusions des sens} et surtout celles du {\it rêve} :
quand nous rêvons, ne croyons-nous pas voir, toucher, entendre ?
et pourtant ces états subjectifs ne correspondent à aucune réalité
hors de nous.

2° L’{\it immédiateté} des objets sensibles est d’ailleurs {\it illusoire}. L'analyse
% 155
psychologique nous a montré :

a. que la {\it sensation} n’est nullement la copie fidèle des choses ;

b. que la {\it perception} sensible est, pour une grande part, une {\it construction}
opérée par l'esprit ;

c. que la notion d’{\it objet} elle-même, loin d’être une donnée immédiate,
résulte d’une dissociation du syncrétisme perceptif primitif (voir
ci-dessus, chap. V et VII).

3° Les {\it Sciences physiques} nous montrent que beaucoup de réalités
physiques ne sont pas directement perceptibles à nos sens et que
celles qui le sont, se ramènent, les unes, comme le son, la chaleur, à
des mouvements corpusculaires de la matière, les autres, comme la
lumière et les couleurs, à des rayonnements d’énergie de différentes
fréquences. Il n’est donc plus possible de soutenir que la réalité extérieure
existe hors de notre esprit {\it telle que nous la percevons}.

4° Enfin la {\it Théorie de la Connaissance} (ci-dessous, § 323) dénonce
l'illusion de l’{\it objet absolu}. Comment d’ailleurs une réalité dont la
nature serait foncièrement hétérogène à celle de la pensée, serait-elle
perméable à celle-ci ? Comment la connaissance serait-elle possible ?

\section{Le Réalisme philosophique}% 111
Pour échapper à ces
difficultés, les philosophes ont été amenés, tout en maintenant l’existence
d’un {\it objet} extérieur à la pensée, à le concevoir tout autrement
que ne le fait le réalisme vulgaire. {\it A.} Déjà, dans l’antiquité, Platon
s’était refusé à reconnaître pour la réalité absolue ce monde sensible
toujours changeant, qui « s'écoule » sans cesse, comme avait dit
Héraclite. Il l’avait comparé aux silhouettes que des prisonniers
voient passer sur le fond d’une caverne où ils sont enchaînés depuis
leur plus jeune âge et qu’ils prennent pour des êtres réels, alors
qu’elles ne sont que les ombres projetées par des figures d'hommes,
d’animaux ou d'objets qu’on fait défiler derrière eux. Notre âme
elle aussi est « prisonnière » du corps et c’est pourquoi nous sommes
dupes des fantômes que nous présentent nos sens. Mais les vraies
réalités sont d’un tout autre ordre : ce sont les \textbf{\textit {Idées}}, c’est-à-dire les
{\it essences intelligibles}, éternelles, immuables, qui sont comme les archétypes
des choses sensibles et qui existent indépendamment d’elles et
en dehors d’elles (§ 323). {\bf —} Ce n’est pourtant pas là un idéalisme au
sens moderne du terme : c’est bien un réalisme puisque les Idées,
tout en étant des essences à la contemplation desquelles notre pensée
s'élève par la {\it dialectique}, possèdent une réalité {\it ontologique}, une
réalité {\it en soi} en dehors et au-dessus de notre pensée. On peut dire
cependant qu'il y a, dans le platonisme, un pas fait vers l’idéalisme :
car la réalité absolue, étant ici \textbf{\textit {d'ordre intelligible}}, n'est plus constituée
% 156
par le monde sensible, par des « choses » hétérogènes à la pensée.

{\it B.} Ce caractère est encore plus net dans le Cartésianisme. On
sait comment Descartes, après avoir révoqué en doute l’existence
du monde extérieur, en vient à affirmer sa propre existence en tant
qu’être pensant, parce qu’elle est saisie immédiatement dans l’acte
du « je pense ». Mais {\it l'existence de la réalité sensible, du monde des
corps a besoin d’une démonstration} et celle-ci devra passer par le détour
de l’existence de Dieu. Seule, en effet, l’existence d’un Être parfait,
qui, donc ne peut être trompeur, m’assure que la « très grande inclination »
que j'ai à croire à l'existence des choses corporelles, n’est pas
illusoire. Toutefois la véracité divine ne garantit pas la vérité des
donnée {\it sensibles}; car nos sens nous ont été donnés pour nous
apprendre, non pas « la nature des choses, mais seulement ce en quoi
elles nous sont utiles ou nuisibles » (voir § 79). Elle ne garantit que la
vérité des idées « claires et distinctes » : autrement dit, {\it ce qui est réel
dans le monde extérieur, ce n’est pas ce que nos sens y perçoivent, c’est ce
que notre entendement y conçoit}. Or la seule idée claire et distincte que
nous concevions sur les choses corporelles, c’est qu’elles sont \textbf{\textit {étendues}}
(c’est ce que démontre dans la seconde {\it Méditation} le célèbre
exemple du morceau de cire) et, par suite, ce que je puis en affirmer,
c’est que « toutes les choses, généralement parlant, qui sont comprises
dans l’objet de la géométrie spéculative, s’y trouvent véritablement ».
D'où un \textbf{\textit {Réalisme géométrique}} qui ne reconnaît comme propriétés
réelles des corps que « l’extension, la figure et le mouvement ».

Malebranche (1638-1715) ira encore plus loin que Descartes
et ne craindra pas de dire que rien ne nous garantirait de façon certaine
l'existence des corps si la foi ne nous enseignait que Dieu a
créé un monde. Revenant à une conception platonicienne, il remarque
que, si la terre me résiste quand je la frappe du pied, {\it mes idées me
résistent} bien plus énergiquement encore. Ce sont donc ces idées qui
sont réelles et notamment celle de \textbf{\textit {l'étendue intelligible}}. Cette idée,
« nécessaire, éternelle, immuable », dont il m’est impossible de modifier
à volonté les propriétés (qu’on essaye donc de concevoir dans un
cercle deux diamètres inégaux !), dépasse infiniment notre esprit et
par conséquent « il n’est pas possible qu’elle n’en soit qu’une modification ».
C’est donc une réalité, qui existe dans l’entendement divin
où elle est l’archétype des corps (cf. ci-dessus, § 99).

On le voit : le Réalisme semble conduit, par la logique même de
son évolution, à identifier de plus en plus le {\it réel} avec l’{\it intelligible}.
Un Berkeley aura alors beau jeu à objecter (§ 114) que la notion
d’{\it étendue} n’est elle-même qu’une abstraction. Que restera-t-il alors
du Réalisme?

% 157
{\it C.} De nos jours, les progrès de la Psychologie et de l’Épistémologie
d’une part, ceux des Sciences physiques d’autre part ont rendu de
plus en plus problématique l’existence d’une réalité qui, tout en étant
distincte de la pensée, se refléterait en celle-ci selon une image adéquate.
Certains philosophes cependant, surtout anglo-saxons, ont
tenté de sauvegarder la proposition fondamentale du Réalisme, à
savoir que \textbf{\textit {l'être existe en dehors de la connaissance que nous en
avons}}.

C'est ainsi qu'H. Spencer a proposé sous le nom de \textsf{\textit {Réalisme transfiguré}}
une doctrine qui, sans donner dans l'illusion du « réalisme grossier », sans
croire « qu'aucun mode de l'existence objective ni les connexions qui unissent
ces modes, soient en réalité ce qu'ils paraissent être », maintient cependant
que « l'existence objective est indépendante de l'existence subjective ».
Le sujet et l’objet, bien qu'ils constituent chacun un « {\it nexus} inconnu de
nous » existent réellement, tout comme un cube et un miroir courbe où
il se reflète ont une existence réelle bien que l’image du cube varie selon leurs
positions respectives.

Le \textsf{\textit {Néo-réalisme}} anglais et américain s’est orienté, tantôt dans le sens
empiriste et presque matérialiste avec le manifeste américain de 1910 dont
plusieurs signataires allaient jusqu’au {\it behaviorisme} pur, tantôt dans le sens
platonicien avec la philosophie de Samuel Alexander (1859-1938) et l'école
d'Oxford. Le principe commun à ces orientations si différentes paraît être
que \textsf{\textit {la connaissance comme telle ne modifie pas l'objet connu}}. Un philosophe
comme Alexander admet d’ailleurs dans le réel, par opposition au « réductionnisme »
des théories monistes, divers paliers ou niveaux de réalité,
chacun se manifestant par un {\it principe d'émergence}, c'est-à-dire un « moment
critique où jaillit une synthèse qualitative, hétérogène et complexe par
rapport à ses éléments » (par exemple, la vie par rapport aux éléments
physico-chimiques, la conscience par rapport à la vie organique).

Le philosophe allemand Wunpr (1832-1920) avait distingué de même du
« réalisme naïf » un \textsf{\textit {Réalisme critique}} qui sait faire la différence entre la connaissance
même et ses objets, tandis que le premier croit naïvement à la valeur
{\it réelle} des représentations. Le passage de l’un à l’autre s'effectuerait à travers
les trois stades de la {\it perception} qui distingue déjà le permanent des qualités
changeantes, de la {\it connaissance intellectuelle} qui, pour atteindre le constant,
s'éloigne de plus en plus de l'intuition immédiate (par exemple, quand elle
ramène toutes les propriétés de la matière à ses propriétés géométriques) et
de la {\it connaissance rationnelle} qui établit ses enchaînements de concepts
au delà même de l'expérience.

Le Réalisme se trouve donc contraint à des replis successifs de
sorte qu’il finit par se réduire à un pur principe {\it épistémologique} (plutôt
qu’ontologique) garantissant l’{\it objectivité} de la connaissance. Chez les
philosophes anglo-saxons, la limite entre Réalisme et Idéalisme
devient même parfois fort incertaine.

% 158
\section{Le « cogito » cartésien}% 112
Descartes avait ouvert une
autre voie. En posant la {\it pensée} comme la {\it première existence certaine},
\textbf{\textit {la seule immédiatement donnée}}, en tant qu’elle est impliquée
dans le doute lui-même, il avait, comme l’a dit Hegel, « tout repris
par le commencement ». Certes, Descartes finissait par conclure à
l'existence réelle du monde extérieur (§ 111 B). Mais désormais, cette
existence était {\it suspendue elle-même à l'existence de la pensée}, plus
« aisée à connaître » que toute autre, puisqu'elle n’a besoin de rien
d'autre pout être connue, tandis que, comme on l’a vu ci-dessus,
l'existence des corps est un problème. Descartes frayait ainsi la voie
aux doctrines {\it idéalistes}, et Kant a pu dire qu’il y a chez lui un « idéalisme
problématique » ou, tout au moins, provisoire.

Deux remarques sont cependant ici nécessaires. 1° Comme on le verra
plus tard (§ 328 {\it fin}), le {\it cogito} ne nous fait pas seulement saisir une pure idée,
mais bien un existant concret : Descartes explique ailleurs que c’est « la
connaissance des propositions particulières » qui nous mène à celle des
vérités générales.
Sa doctrine est donc {\it spiritualiste}, non {\it idéaliste}. {\bf —} 2° Il
est d’ailleurs discutable que le {\it cogito} nous donne l'intuition de l'{\it essence} de
la pensée. Malebranche pourra soutenir que nous n'avons pas d'{\it idée}
claire de l'âme, mais seulement un « sentiment intérieur » (§ 349).

\section{L’Idéalisme rationaliste}% 113
Descartes s'était contenté de
dire que la pensée est la seule réalité {\it immédiate}. Certains de ses
successeurs iront jusqu’à affirmer qu’elle est, en dernière analyse,
{\it la seule réalité} : ce sera l’Idéalisme pur. Mais celui-ci s’est présenté,
tantôt sous la forme rationaliste, tantôt lié à l’empirisme.

{\it A.} Leibniz (annexe) reproche à Descartes d’avoir substantialisé
l'étendue en en faisant l’essence de la matière : l'étendue, toujours
divisible, ne peut présenter le caractère d’unité nécessaire à une
substance. Par suite, {\it les substances ne peuvent être que spirituelles} :
ce sont des \textbf{\textit {monades}} ou unités spirituelles, analogues à des {\it âmes},
encore qu’il vaille mieux réserver le nom d’{\it âmes} aux monades supérieures
douées de perception claire et de mémoire. Toute monade
possède à quelque degré {\it perception} et {\it appétition}. La perception doit
être distinguée de l’{\it aperception} ou conscience distincte : il existe
en effet des « petites perceptions » qui sont des perceptions confuses
et inconscientes, et les monades s’échelonnent en une hiérarchie qui
va depuis les monades inférieures où la conscience est comme « étourdie »,
jusqu’aux {\it esprits} ou {\it âmes raisonnables} qui possèdent la pensée
claire et la connaissance des vérités nécessaires et éternelles. L’{\it appétition}
est le principe interne qui fait passer la monade d’une perception
%159
à une autre : les modifications des monades ne peuvent
leur venir en effet de causes
externes, car elles « n’ont point de
fenêtres par lesquelles quelque
chose y puisse entrer ou sortir ».

Ce système des monades soulève
au moins deux difficultés. 1° Que
deviennent, dans cet univers purement spirituel, le monde extérieur
et, en particulier, {\it la matière} ? Leibniz
ne les nie pas absolument ou, du,
moins, il leur donne un équivalent.
Chaque monade créée, en même
temps que force active, est comme
lestée d’un principe de passivité :
elle perçoit tout l'univers, mais {\it de
son point de vue particulier} et à sa
manière ; il y a en elle des perception confuses : « C’est cette perception
confuse des rapports logiques et vrais
des choses qui leur donne pour nous
l'apparence d'objets situés dans
l’espace et le temps » (Boutroux).
Les corps n'ont pas d'unité : ce sont
nos perceptions qui leur en donnent
une ; en ce sens, ce sont « des être
d'imagination, des phénomènes.
Mais ce sont des phénomènes à la fois
{\it bien fondés}, puisqu'ils représentent
les monades et leurs rapports et {\it bien
liés} puisqu'ils sont « liés justement
comme les vérités intelligibles le
demandent ». {\bf —} 2° La seconde difficulté concerne la {\it communication des
substances} puisque les monades
n'ont pas de rapports entre elles,
comment se fait-il que leurs perceptions soient d'accord ? Leibniz répond
à cette difficulté par la théorie de
l'\textsf{\textit {harmonie préétablie}}. Dieu, l'Esprit
infini dont toutes les monades créées
sont « des productions », les a mises
d'accord une fois pour toutes, de
sorte que « les perceptions des choses
externes arrivent à chacune à point
nommé en vertu de ses propres lois », à peu près comme un habile horloger
qui aurait réglé des horloges différentes de manière qu’elles marquent
toujours la même heure. {\bf —} Mais une telle solution qui n’est guère que la
% 160
formulation sur un autre mode du problème lui-même, n'est-elle pas bien
artificielle ?

{\it B.} Kant (1724-1804) devait donner une forme moins étroite à
l’Idéalisme. On verra (§ 324) qu’il maintient l’existence de « choses
en soi » ou {\it noumènes} : c’est, dans sa doctrine, une survivance du
Réalisme. Mais ces noumènes nous sont inaccessibles, puisque, dès
qu’il connaît, l’entendement applique à la « matière » de la connaissance
certaines « formes », certains « principes transcendantaux »,
qui font que cette connaissance est nécessairement {\it relative} : elle
porte sur des phénomènes, non sur les « choses en soi ». Tel est le principe
de l’\textbf{\textit {Idéalisme transcendantal}}. C’est bien un idéalisme puisque
la seule réalité {\it connaissable}, le phénomène, tout en possédant une
« réalité objective », est, pour une part au moins, un produit de l’entendement.
On peut même dire, en ce sens, que « l’entendement est l’origine
de l’ordre universel de la nature », autrement dit : de la nature
elle-même « en tant qu’objet d’expérience possible ». Toutefois, Kant
ne va pas jusqu’à mettre en doute l’existence des objets hors de nous.
Il critique la mise en doute par Descartes de cette existence, de même
que l’idéalisme dogmatique de Berkeley (§ 114), et, préludant à
certaines analyses de la Phénoménologie contemporaine, il prétend
montrer, dans la {\it Critique de la Raison pure}, que « l'expérience extérieure
est proprement immédiate » et que « c’est seulement par le
moyen de celle-ci qu’est possible sinon la conscience de notre propre
existence, du moins la détermination de cette existence dans le
temps, c’est-à-dire l’expérience interne ».

On remarquera, de l’Idéalisme leibnizien à l’Idéalisme kantien, un
changement de perspective capital. Le premier, d'inspiration purement
mathématique, aboutissait à une conception de la connaissance
comme {\it tirée exclusivement du sujet connaissant}, puisqu'elle n’est qu’une
explicitation en pensée claire des virtualités confuses incluses dans la
monade : il ne pouvait établir l’impératif d’{\it objectivité} de la connaissance
que grâce à l'hypothèse bien gratuite de « l'harmonie préétablie ».
Le second fait beaucoup plus grande la part de l’expérience et
de l’objectivité : mais ne retombe-t-il pas dans les difficultés du
Réalisme (§ 110 {\it fin}) lorsque, maintenant le dualisme du sujet et
de l’objet sous la forme du dualisme de la {\it forme} et de la {\it matière} de la
connaissance, il prétend expliquer comment l’esprit s'impose aux
« choses » ? Nous montrerons cf-dessous (§ 324 {\it fin}) que Kant n’a pas
répondu au problème par lui-même posé : comment la connaissance
est-elle possible? Son principal mérite est d’avoir fondé l’idéalisme
% 161
sur la notion de l’activité de l'esprit et d’avoir vu que, jusque dans la
perception sensible, l'esprit est à l’œuvre.

\section{L’Idéalisme empiriste}% 114
À la suite de la critique cartésienne,
le philosophe anglais Locke (1690) avait établi une distinction
entre les qualités sensibles. Les unes, les qualités {\it premières} ou {\it primaires},
sont inséparables de l’idée de matière : ce sont la solidité,
l'étendue, la forme, le nombre, le mouvement ou le repos ; elles sont
donc « dans les corps »
et existent réellement.
Les autres, les qualités {\it secondes}, telles
que la couleur, la
saveur, l'odeur, peuvent être supprimées,
au moins par abstraction, sans que soit
supprimée la notion
même de corps : elles
ne sont donc effectivement rien d’autre
que « le pouvoir de
produire diverses sensations en nous par
le moyen des qualités premières », par
exemple par le mouvement de particules
très petites.

L'évêque  anglican
G. Berkeley (1685-1753)
qui veut combattre
{\it le matérialisme}, va s’efforcer de montrer que les qualités premières
tombent sous le coup des mêmes critiques que les qualités
secondes et il poussera ainsi l’idéalisme jusqu’à l’\textbf{\textit {immatérialisme}},
jusqu’à la négation de la matière. Mais à la différence des doctrines
étudiées dans le paragraphe précédent, l’idéalisme de Berkeley est un
idéalisme empiriste, fondé sur une critique de l’abstraction.

Nominaliste (voir ci-dessous § 182 C), Berkeley refuse toute réalité aux
idées abstraites. Or la notion de « substance matérielle » n’est qu'une abstraction :
nous ne pouvons savoir ce qu'est la matière si nous la dépouillons
% 162
des {\it qualités} par lesquelles elle se révèle, dit-on à nos sens. Or, ces qualités
n'ont aucune existence en dehors de notre esprit : on l’admet généralement
pour les qualités secondes, mais les qualités premières ne peuvent être
perçues sans les secondes. Lorsqu'on parle, par exemple, de l'{\it étendue}, de
quelle étendue s’agit-il : celle que nous montre la vue ou celle que nous donne
le toucher ? De même, les sens nous représentent-ils jamais un mouvement
dépouillé de toutes les autres qualités visibles ou tangibles? Les qualités
premières sont d’ailleurs aussi variables, aussi peu consistantes que les couleurs,
les saveurs, les odeurs : un animal très petit percevra comme une
montagne énorme ce que nous distinguons à peine, et pour nous-mêmes
l'étendue visible d’un objet ne paraît-elle pas dix ou cent fois plus grande
selon que nous nous éloignons ou nous approchons ? De même, la solidité
d'un objet n’est rien d'autre que sa résistance à la pression plus ou moins
grande que nous exerçons sur lui.

Il résulte de là qu’il n’y a pas d’objets {\it extérieurs} à la pensée : les
« choses » n’ont aucune existence distincte de la perception même que
nous en avons. {\it Leur existence consiste à être perçues} : \textbf{\textit {esse est percipi}}
(annexe), de même que l’existence des esprits consiste à percevoir : esse
est percipere. Si l’on appelle idées, non pas les idées abstraites, mais
l’objet immédiatement donné à la pensée, les choses sensibles ne sont
que des idées.

Dans les Dialogues d'Hylas et de Philonoüs (1713) où le personnage
d'Hylas (du grec : hylè, matière) représente l'homme du sens commun qui
croit à la matière, et Philonoüs l’idéaliste, porte-parole de Berkeley lui-même,
celui-ci prétend que cette doctrine donne plus que toute autre satisfaction
au sens commun et qu’elle ne conduit nullement au scepticisme : elle
maintient en effet que « les choses sensibles sont celles-là mêmes qui sont
perçues immédiatement par les sens » ; que les couleurs existent par exemple,
en tant qu’« idées », telles que nous les percevons ; que le son « réel » est
bien ce que nous entendons, et non ce mouvement vibratoire dont nous
parlent les physiciens. Au fond, le sens commun ne croit pas à une matière
distincte de ce que nous percevons par les sens.

Une difficulté subsiste cependant, analogue à celle qu’avait rencontrée
Leibniz : comment expliquer que les choses sensibles soient,
ainsi que me le montre l’expérience, indépendantes de mon intelligence ?
que je ne puisse percevoir n’importe quoi à volonté ? C'est,
répond \textsc{Berkeley}, qu’elles existent, non seulement dans mon
intelligence en particulier, mais dans « une Intelligence omniprésente
et éternelle qui les présente à notre vue d’une certaine manière et
suivant certaines règles qu’elle a elle-même posées et que nous
nommons lois de la nature ». La solution, comme chez Leibniz, se
trouve donc en Dieu.

Il y a cependant une profonde différence entre l’Idéalisme leibnizien,
lié à toute une conception scientifique et même mathématicienne,
% 163
et l'Idéalisme de Berkeley, inspiré surtout d’un sentiment mystique,
mais qui « reste étranger au besoin d'expliquer et de comprendre
vraiment la nature » et réduit la pensée à une suite d'« états de
conscience » discontinus et passifs, si bien qu’on a pu dire qu’« à cet
idéalisme qui ne semble voir de réalité que dans la pensée, c’est
peut-être le sentiment de ce qu’est la pensée même qui manque le
plus » (\textsc{Parodi}). Ce sera précisément, comme on l’a vu, l’œuvre de
Kant de montrer, au contraire, que la pensée est partout active.

\section{L'Idéalisme dialectique}% 115
Cette notion de l’activité
de la pensée va d’ailleurs prendre chez les successeurs de Kant une
forme plus précise : Kant s’était contenté de dresser un tableau des
« catégories » (chap. XVII) ; ses successeurs vont les déduire en montrant
comment elles dérivent des oppositions qui se font jour au sein de
la pensée elle-même du fait de sa propre activité. L’idéalisme va devenir
un idéalisme dialectique.

{\it A.} Fichte (1762-1814) part de l’idée que notre essence est un vouloir,
un acte. Par suite, comme il l'explique dans la Doctrine de la science (1794),
il ne peut rien y avoir dans le moi qui ne résulte de sa propre activité. Le
principe premier de la dialectique sera donc {\it l'acte par lequel le Moi se pose
lui-même}
%
\footnote{Dans la Doctrine, Fichte prétend déduire ce principe lui-même du principe d'identité :
A = A, dont la certitude, dit-il, est immédiate. Ce dernier ne pose pas, en effet,
la nécessité absolue de A ; il pose seulement que, si A est posé, il est : ce qui est nécessaire,
c'est donc la liaison entre les deux termes. Or la liaison n’est certaine que parce qu'il y a
derrière elle un Moi identique qui en pose la nécessité. L'identité du Moi est donc plus
fondamentale que A = A. {\bf —} Mais il y a là un artifice logique qui, aux yeux de Fichte,
est si peu indispensable que, dans le {\it Nouvel exposé de La Doctrine de la science} (1797),
il présente l'acte par lequel le Moi se pose lui-même comme une donnée primitive de
l'intuition intellectuelle.}.
%
Il ne s’agit pas ici du moi empirique, de la conscience individuelle,
mais bien du \textsf{\textit {Moi absolu}}, infini, qui, étant simplement « ce dont l'essence
consiste en ce qu'il se pose lui-même comme étant », est antérieur à la distinction
du moi empirique et du non-moi et enferme à la fois le sujet et l’objet.
Avec lui seul, on ne pourrait donc rien construire et surtout, du point de
vue pratique, une puissance qui va d'elle-même et directement à l'infini,
n'aurait aucune efficacité causale. Il faut donc introduire un deuxième
principe : le {\it Moi s'oppose un non-moi}. Le non-moi n’est pas une réalité indépendante :
il résulte du « choc » ({\it Anstoss}) que le Moi subit nécessairement
dans son développement pour devenir conscience et action. Mais cette
opposition antithétique du Moi et du non-moi requiert à son tour une
synthèse, qui sera possible grâce au troisième principe : {\it je pose dans le
Moi, en opposition au moi déterminé, un non-moi déterminé}. On revient ainsi
à la conscience réelle qui suppose une limitation réciproque, une action
mutuelle du moi et du non-moi. L’idéalisme fichtéen est donc un \textsf{\textit {Idéalisme
subjectif}}, dans lequel tout dérive de l’activité du Moi. Le monde est l’apparence
produite par le Moi absolu, contraint, pour prendre conscience de
lui-même, de se déterminer, de se limiter, de se donner un objet.

% 164

{\it B.} L'Idéalisme de Hegel (1770-1834) est un \textbf{\textit {Idéalisme objectif}}.
Le principe du développement dialectique n’est plus dans le moi :
il est dans l’objet, dans le concept ou l’Idée posée comme une réalité
en soi et qui n’arrive à la conscience de soi que par son propre développement
interne. Le sujet n’est que le témoin passif de ce mouvement :
« La pensée subjective se borne à assister à ce développement
de l’Idée comme à celui de l’activité propre de sa Raison. » Le principe
interne de l’univers est une pensée objective : c’est l’\textbf{\textit {Idée}} en tant
qu’Être pur. Son développement est une \textbf{\textit {dialectique}}, c’est-à-dire
une {\it logique}, non pas du tout une {\it histoire}. Ontologie et Logique se
confondent : « Tout ce qui est réel est rationnel, et tout ce qui est
rationnel est réel. » Rien ne se produit du côté de l’être qui n’ait sa place
marquée dans le développement logique de la pensée et, d’autre part,
tout ce qui est une forme ou un moment de ce développement se
produit inévitablement dans l’ordre des choses. L'élément moteur
de la dialectique est la {\it contradiction}, c’est-à-dire l'opposition des
contradictoires au sein même de l’être concret, opposition qui se
résout en une {\it synthèse} qui les unifie à la fois en les abolissant et en
les dépassant (double sens de l'allemand {\it aufheben}, abroger et soulever\footnote{Ce qui a fait dire à J. Lequier (1814-1862) : « Un chef d'école qui a porté le courage
de l'absurde jusqu'à l’héroïsme, a rencontré dans une bizarrerie de la langue allemande
toute une révélation ; il a distingué, il a mis à part, il a admiré un mot à double sens
qui signifie tout à la fois {\it poser} et {\it enlever}. Ce mot est devenu le fondement sur lequel il a
construit un système.}).
L’Être pur et le pur Néant sont identiques : car l'Être pur,
sans détermination, est le plus pauvre des concepts. Cette antithèse
de l’{\it Être} et du {\it Néant} se résout dans la synthèse du {\it Devenir}. Le Devenir
a pour résultat l'{\it existence}, c’est-à-dire l'être qualifié. L’antithèse de
la {\it qualité} est la {\it quantité}, et la synthèse s’opère dans la {\it mesure}, etc. {\bf —}
Mais cette \textbf{\textit {Logique}}, cette déduction abstraite des catégories où
l'esprit n’est encore considéré qu’{\it en soi}, dans l’Idée à la fois logique
et métaphysique, n’est que la première démarche de l’hégélianisme.
Il faut encore que l’esprit sorte de son isolement, qu’il se disperse au
dehors : la \textbf{\textit {Nature}} sera ce moment de la vie dialectique de l’Idée où
elle s’extériorise avant de venir enfin s’intérioriser dans l’{\it Esprit}.
N’insistons pas sur la {\it Philosophie de la Nature} qui a mené Hegel à
des constructions aventureuses et parfois ridicules\footnote{Voir Émile \textsc{Meyerson}, {\it De l'Explication dans les sciences}, tome II, p. 23-31.}. {\bf —} Plus importante
est la {\it Philosophie de l'Esprit} : l'\textbf{\textit {Esprit}}, c’est «l’Idée revenue en
elle-même de sa dispersion extérieure, le concept réfléchi en soi,
l'être universel qui a supprimé ses manifestations extérieures et qui,
sortant de la Nature, se recueille dans sa propre idéalité». Il est
% 165
d’abord « esprit subjectif » : {\it âme} encore engagée dans la nature, puis
{\it conscience}, {\it raison} et enfin {\it esprit} proprement dit, objet de la Psychologie.
Il se manifeste ensuite comme «esprit objectif » dans le {\it droit},
la {\it moralité}, la {\it société} et l’{\it État}. Enfin, « l'esprit absolu », c’est l’Idée
retrouvant l'unité de son essence et de son existence dans l’{\it art}, la
{\it religion} et la {\it philosophie}.

C. Dans ses {\it Éléments principaux de la représentation} (1907), le
philosophe français Octave \textsc{Hamelin} (1856-1907) définit l’Idéalisme
la doctrine selon laquelle « l’objet de la pensée consiste en idées », {\bf —}
la pensée ne pouvant atteindre un objet extérieur à elle, {\bf —} mais aussi
pour laquelle « les idées » n’existent pas en elles-mêmes, ne doivent
donc pas être traitées comme des {\it choses}. L’Idéalisme de \textsc{Hamelin} est,
comme les précédents, un idéalisme dialectique, mais qui s’inspire
du {\it néo-criticisme} de Charles Renouvier (ci-dessous, chap. XVII) au
moins autant que de la dialectique hégélienne.

Il présente avec celle-ci deux différences essentielles. 1° Le rapport de
la thèse et de l’antithèse n’est pas celui de deux {\it contradictoires} qui s’excluent
comme chez Hegel, et c’est pourquoi Hamelin préfère l’expression d'\textsf{\textit {idéalisme
synthétique}} à celle d’idéalisme dialectique : « La synthèse qui concilie
les opposés, ne les nie pas ; et elle n’a pas à les nier, parce qu’ils ne sont pas
contradictoires : ils sont seulement des {\it contraires}\footnote{Pour la différence, voir notre Nouveau Vocabulaire.} ; et pour bien caractériser
les contraires, il faut dire que ce sont des {\it corrélatifs}. À la contradiction
hégélienne, dit Hamelin, nous substituons la {\it corrélation}.» {\bf —} 2° La déduction
dialectique des catégories qui, chez Hamelin comme chez Renouvier,
part de la catégorie fondamentale de la {\it Relation}, aboutit ici, non comme
chez Hegel à un « universel concret » (l'Esprit absolu, qui est universel en
tant qu’il est capable d’un nombre indéfini d'applications, est concret en
ce sens qu'il est une totalité unique et indivisible où l’individu est absorbé
et noyé)\footnote{Fichth lui aussi, surtout à partir de 1804, en vient à donner à l'effort spirituel qui,
dans sa première doctrine, émanait du moi individuel un sens cosmique où la nécessité
éternelle se substitue à la liberté : « L'individu, dont la spontanéité souveraine était
proclamée, est maintenant encadré dans un devenir qui le dépasse, comme s'il était
décidément impossible à un Allemand d’attribuer à l’activité morale une spontanéité
véritable et d'en faire plus que l'expression d’une force cosmique universelle» (Bréhier).
Il y a là en effet une profonde différence d'inspiration entre la dialectique hégélienne et
même fichtéenne et la synthèse hamelinienne.}, mais bien, à la catégorie de {\it Personnalité} et à l'affirmation qu’en
définitive, l'être, c’est l'Esprit, mais l'Esprit {\it en tant que conscience et que
volonté} : « S'il y a quelque force apparente dans la définition empiriste de
l'existence, savoir : qu’{\it exister, c’est être perçu}, cette force vient au fond de
ce que ce qui a été voulu est comme tel un donné et de ce que, d'autre
part, la manière dont un sujet s'expose et se prête aux représentations pour
les percevoir ou au moins pour les éprouver, dépend elle-même directement
ou indirectement de quelque volition et enveloppe de la contingence.
Exister, c’est être voulu. »
% 166
La « synthèse totale » s’achève ainsi, chez Hamelin, dans l’affirmation
de l'Esprit comme \textbf{\textit {activité libre}}.

\section{L’Idéalisme critique et réflexif}% 116
Cette notion de la
liberté de l’esprit est encore plus nette chez Léon \textsc{Brunschvicg}.
Celui-ci (voir chapitre XIX et chapitre XXVI), avait été conduit
par l’épistémologie à un Idéalisme à la fois critique et réflexif qui,
au-dessus de la science, voit {\it l’esprit} lui-même dont elle est l’œuvre.
Idéalisme \textbf{\textit {critique}}, selon la tradition kantienne, mais qui va plus
loin que celui de Kant puisqu’ici toute notion de « chose en soi » est,
comme chez Hamelin, éliminée : « Une chose qui serait au delà de
la connaissance serait, par définition, l’inaccessible, l’indéterminable ;
elle équivaudrait pour nous au néant.» La connaissance n’est pas
«un accident qui s’ajoute à l'être». Mais Idéalisme \textbf{\textit {réflexif}} aussi,
pour lequel l'esprit se révèle à lui-même dans sa propre histoire, dans
« le progrès de la conscience » qu’il prend de sa propre activité scientifique,
esthétique, philosophique, religieuse. Activité essentiellement
{\it libre} d’ailleurs, en ce sens d’abord qu’elle est libérée de toute compromission
avec les sources impures de fausse spiritualité : le « moi
spirituel » n’est pas le « moi vital » et la vraie vie de l’esprit n’est pas la
« vie intérieure » : elle ne se confond ni avec l’expérience de l'effort
comme chez Maine de Biran ni avec « la vie » telle que l’entend Ravaisson
ni avec «l'élan vital» ou «l'instinct » privilégiés par Bergson.
{\it L'esprit, c’est l’intellect}, la « conscience intellectuelle », laquelle, sans
doute, réside dans ses « miroirs vivants actifs », mais dépasse toujours
l'individu. Activité libre encore en cet autre sens qu’elle est libérée
de toute emprise « dialectique ». Il ne s’agit plus ici de ramener l'esprit
à une idée éternelle ni de construire une table immuable des catégories.
L’esprit n’est pas {\it idée} : c’est au contraire l’idée qui est « une
action de l’esprit » et lui-même « se définit par la capacité de former
des idées ». Tant s’en faut d’ailleurs qu’il les forme une fois pour
toutes ! Il ne saurait être question de « deviner {\it a priori} les cadres de
la science». L'activité de l'esprit fait éclater tous ces systèmes
de catégories en lesquels on avait prétendu l’enfermer, ainsi que le
montre l’histoire de la pensée mathématique et de la pensée physique.
« La contingence du progrès humain est radicale : le centre de la
réflexion s’y déplace perpétuellement, entraînant dans ce déplacement,
avec la vision que l'intelligence a de l’univers, la perspective
de sa propre histoire. »

\section{L’Idéalisme anglo-saxon}% 117
Très différent est l’Idéalisme
anglo-saxon qui, le plus souvent, tourne le dos à l’intellectualisme
%167
et, par réaction contre l’atomisme associationniste, est hanté par
l'idée de {\it totalité} : il est d’ailleurs fréquemment influencé par l’hégélianisme.
Nous avons déjà remarqué qu’il n’est pas toujours facile à
distinguer du Réalisme (§ 111 {\it fin}) et nous en verrons un nouvel
exemple dans la philosophie de \textsc{Bradley} (1846-1924). Selon un philosophe
anglais, la proposition qui le définirait le mieux serait celle-ci :
« La connaissance n’est pas une relation externe. »

{\it A.} Cette formule s'applique certainement à la position de F. H. \textsc{Bradley}
qui, dans {\it Apparence et réalité} (1893), s'est appliqué à chercher une « mesure»
({\it standard}) permettant de distinguer l’« apparence », c’est-à-dire le phénomène,
et les divers degrés de la réalité. Aux yeux de Bradley, la science est
peu de chose dans l’ensemble de la spiritualité humaine. Les concepts
scientifiques, tels que ceux d’{\it espace}, de {\it temps}, d'{\it énergie}, de {\it causalité}, etc.,
sont des «idées de travail» ; mais ils n’expriment que des rapports externes
et sont, par suite, incapables de nous donner l'essence intime du réel. La
détermination du réel par un concept, du {\it that} par un {\it what}, est fausse et
incomplète. Elle ne peut devenir vraie que rapportée à une « expérience
intégrale», indépendante de toute relation externe d’un sujet à un objet
et qui est plutôt {\it présence} à la sensibilité ({\it feeling}) d’une \textsf{\textit {totalité cohérente
de rapports internes}}. Cette totalité, c’est le {\it réel} lui-même. En ce sens, « être
réel », c'est être donné dans cette expérience sensible immédiate. Bradley
déclare lui-même ignorer si sa doctrine mérite d’être qualifiée de {\it réaliste}
ou d’{\it idéaliste}. La réalité, identique à la vérité, c’est, dit-il, «le monde tout
entier vu à travers la philosophie», c’est l’Absolu. Mais nous ne pouvons
qu'approximer cette {\it expérience totale} : l'extension de notre expérience
s’oppose d’ailleurs à sa cohérence. C’est pourquoi l’idée tend au delà de
l'idée, la personnalité vers quelque chose qui est supérieur à la personne,
la moralité vers quelque chose qui dépasse la morale. L'idéalisme de
Bradley s'achève en une théodicée où l’individuel, l'erreur, le mal se justifient
comme fragments d’un Absolu qui est lui-même individu et système et
dans lequel ils s’absorbent en perdant leur caractère fragmentaire. Mais
n'est-ce pas en revenir à « l’universel concret » de Hegel (§ 115 {\it C}) ?

{\it B.} Pareillement ambiguë est la philosophie de Josiah \textsc{Royce} (1855-1916)
qui fut pourtant le représentant principal de l’idéalisme aux États-Unis.
Royce distingue en effet trois interprétations possibles du « réel » : 1°
l’interprétation {\it réaliste}, pour laquelle le réel, c’est le présent, ce qui est « sous
la main » : mais le réalisme méconnaît que toute relation entre la pensée
et le réel transcendant fait rentrer ce réel dans le cercle de l'expérience
possible ; le « fait brut» n’est un fait qu’en fonction de l'acte par lequel
la pensée le reconnaît ; c’est la pensée qui lui donne sa signification ({\it meaning}),
et toute signification est immanente : elle implique une finalité,
un acte de {\it volonté} par lequel l'esprit la pose ; {\bf —} 2° l'interprétation {\it mystique},
pour laquelle le réel, c’est l'immédiat, c’est l’intériorité : mais le mysticisme
perd de vue le contraste entre l’Être absolu et notre imperfection et, comme
le réalisme, réduit ainsi l’être à un pur néant ; {\bf —} 3° l'interprétation {\it criticiste}
qui identifie l'être avec le valable, avec le bien-fondé : mais le criticisme
est insuffisant, car il ne réussit pas à éliminer la dualité entre l’actuel et le
possible, entre le concrètement vérifié et l’idéalement vérifiable. L'idéalisme
critique doit donc être prolongé en un \textsf{\textit {Idéalisme absolu}} qui privilégie le
concret et qui, sans trmber dans l’anthropomorphisme, reconnaît cependant
% 168
la même « vie d'expériences animant par ses pulsations tous les aspects de
la réalité, tous les domaines de l'être. « Le monde est réel {\it en} et {\it pour} une
Pensée qui embrasse tout et connaît tout, une Pensée universelle {\it pour
laquelle} toute relation, toute vérité existe et qui évalue avec rigueur les
démarches boiteuses et discursives de la nôtre. » L’idéalisme absolu requiert
ainsi l’existence d’un Être tout-connaissant ({\it All Knower}). Mais, entre cette
Pensée qui est expérience absolue et présence totale, et la nôtre, il y a
cependant, précise Royce, une relation assez directe pour que notre individualité
n’aille pas se perdre dans la sienne. Le Moi absolu s'exprime en une
multiplicité d'individus solidaires, mais dont chacun conserve sa liberté,
précisément en tant qu'il est unique. Communauté, individu et absolu se
concilient en une unité supérieure.

\section{Les doctrines contemporaines}% 118
Un a vu que la distance
qui séparait l’Idéalisme du Réalisme a plutôt tendu à diminuer, voire
parfois à s’effacer. Le fait est encore plus frappant dans les doctrines
contemporaines. Nous allons examiner trois doctrines qui sont peut-être,
au fond, des Réalismes, mais qui, on va le voir, présentent incontestablement
des aspects les apparentant à l’Idéalisme.

{\it A.} \textbf{\textit {Le Bergsonisme.}} Que, par certains côtés, le bergsonisme soit
bien un Réalisme, il nous paraît difficile de le nier\footnote{C'est un point qui a été mis en lumière par L. Husson dans son livre
{\it L'intellectualieme de Bergson}.} Dès {\it Matière
et Mémoire}, \textsc{Bergson} nous présente nos « images » comme ne faisant
qu'un avec le réel, et la «perception pure» comme un acte par
lequel « nous nous plaçons d'emblée dans les choses ». Cette perception
{\it pure}, dégagée de l’apport de notre mémoire, nous fournirait, si nous
pouvions y atteindre, « des visions instantanées qui feraient partie
des choses plutôt que de nous ». Grâce à elle, «les qualités sensibles
de la matière elles-mêmes seraient connues en {\it soi}, du dedans et non
plus du dehors » et l’on pourrait dire alors que « la matière est absolument
comme elle paraît être ». L’{\it Introduction à la Métaphysique}
affirme que la « réalité extérieure » est « immédiatement donnée à
notre esprit » et que « le sens commun a raison sur ce point contre
l’idéalisme et le réalisme des philosophes ». Dans {\it La Pensée et le
Mouvant}, Bergson répète : « Ce n’est pas en nous, c’est en eux que nous
percevons les objets ; c’est du moins en eux que nous les percevrions
si notre perception était {\it pure}.» Vers la fin de {\it L’Évolution créatrice},
il était allé jusqu’à écrire que l’intuition sensible elle-même, si elle
pouvait se mettre en continuité avec l'intuition supra-intellectuelle
(8 326), n’atteindrait plus simplement, comme chez Kant, «le fantôme
d’une insaisissable chose en soi. C’est... dans l’absolu qu’elle nous
introduirait» et, dès 1903, l’{\it Introduction à la métaphysique} avait
% 169
rangé au contraire l’Idéalisme parmi les doctrines « qui contestent
notre esprit le pouvoir d’atteindre l'absolu ».

On peut observer cependant que le genre de réalité ainsi reconnu
au monde sensible est du même type que celui que nous avons rencontré
chez Berkeley, qui lui aussi en venait à légitimer le sens
commun (§ 326). Ce réalisme n’est pas celui de la science (§ 319) et,
comme l’a remarqué D. \textsc{Parodi}, le mode d’existence attribué aux
choses, ainsi ramenées à un flux capricieux d'{\it images}, « est quelque chose
de si vague, de si sourd et de si indistinct, qu’il se réduit pour ainsi
dire à rien ». Au reste, Bergson lui-même n’avait-il pas pris soin de
nous avertir qu’en nous présentant, au début de {\it Matière et Mémoire},
le réel immédiat comme un monde d'{\it images}, « au sens le plus vague
où l’on puisse prendre ce mot », il se rapprochait de l’Idéalisme ?
« Que toute réalité ait une parenté, une analogie, un rapport enfin
avec la conscience, c’est ce que nous concédions à l’idéalisme par cela
même que nous appelions les choses des {\it images}. » Et il avait caractérisé
sa doctrine comme nous indiquant « la position à prendre entre
l’Idéalisme et le Réalisme » par le fait qu’elle dénonçait {\it le postulat
commun à l’un et à l’autre, à savoir qu’ils tiennent tous deux « les opérations
élémentaires de l'esprit, perception et mémoire, pour des opérations
de connaissance pure ».} Dès lors, partant du cours contingent et capricieux
des perceptions dans la conscience, l’Idéalisme ne réussit pas
à rendre compte de l’ordre déterminé qui lie les phénomènes naturels
dans l’espace homogène, c’est-à-dire à rejoindre le réel. Partant au
contraire de ce dernier, le Réalisme échoue à expliquer comment
celui-ci se répète sous forme d’images dans notre esprit.

Si au contraire on renonce au postulat indiqué ci-dessus, si j’admets
«que ma perception consciente a une destination toute pratique »
qui est « de préparer des actions », je comprends « que tout le reste
m’échappe et que tout le reste cependant soit de même nature que
ce que je perçois » ; et, de même, si je reconnais que l’espace homogène
lui aussi « concerne notre action et notre action seulement », je fais
tomber «l’insurmontable barrière que le Réalisme élevait entre les
choses étendues et la perception que nous en avons ». Ainsi, « nous
replaçons la perception dans les choses, et nous voyons Réalisme et
Idéalisme tout près de coïncider ». Dans {\it La Pensée et le Mouvant}, le
problème est posé de façon identique : \textsc{Bergson} y affirme la possibilité
de « mettre fin à l'antique conflit du Réalisme et de l’Idéalisme
en déplaçant la ligne de démarcation entre le sujet et l’objet, entre
l'esprit et la matière ». Il s’agit donc de départager de façon nouvelle
l'{\it objectif} et le {\it subjectif}. -- Mais que dire de cette solution qui s’édifie,
en some, sur les ruines de la notion d’{\it objectivité}, celle-ci étant ramenée,
% 170
par suite du privilège accordé à l’idée d'{\it action}, à la simple cutilité
pratique » de la science, destinée avant tout à «nous fournir le meilleur
moyen d’agir sur les choses »?

{\it B.} \textbf{\textit {La Phénoménologie.}} Bergson fait allusion, dans le dernier passage
cité ci-dessus, aux « théories de la connaissance qui, dit-il, ont vu
le jour dans ces derniers temps, à l'étranger surtout » et qui témoignent
qu’« on revient à l’immédiatement donné » ou du moins qu’on y tend.
La Phénoménologie de Edmund \textsc{Husserl} (1859-1938) peut être
considérée comme la principale de ces théories.

C’est bien, en effet, de l'{\it immédiat} que Husserl prétend partir. Mais cet
immédiat n’est nullement pour lui le monde sensible. Il est clair, dit-il,
« que l'expérience sensible universelle, dans l’évidence de laquelle le monde
nous est perpétuellement donné, ne saurait être considérée, sans plus,
comme apodictique, c’est-à-dire comme excluant de façon absolue la possibilité 
de douter de l'existence du monde ». Aucun objet ne nous est donné
que « comme objet d’une conscience réelle ou possible », comme objet de
l'{\it ego cogitans}. Par suite « non seulement la nature corporelle, mais l’ensemble
du monde concret qui m’environne n’est plus pour moi, désormais, un monde
existant, mais seulement {\it phénomène d'existence} ». {\bf —} Le seul immédiat
véritable, ce sont les \textbf{\textit {essences}}, c’est-à-dire les réalités intelligibles en tant
que données à la pensée. Mais ces essences (voir § 327) sont bien des {\it objets}
en présence desquels la pensée prend une attitude contemplative. C’est
précisément ce que signifie la notion d’{\it intentionnalité} : la conscience n’est
jamais, comme dans le {\it cogito} cartésien (§ 112), conscience du sujet pur;
elle est toujours conscience {\it de} quelque chose qui la transcende. Husserl a
d’ailleurs bien précisé qu’il s’agissait pour la Phénoménologie d’en « revenir
aux choses elles-mêmes ». Les essences sont des réalités intelligibles qui,
sans être « hypostasiées » sur le plan ontologique comme dans le platonisme,
ne sont nullement l’œuvre de la pensée : il y a une « existence eïdétique »
(existence des essences) et c’est une erreur de confondre l'essence avec la
conscience de l’essence qui, elle, est un produit. En ce sens, la Phénoménologie
est incontestablement un {\it réalisme}.

Mais Husserl lui-même dès 1913, dans le livre I de ses {\it Ideen}, proteste qu'il
n'existe pas de « réalité absolue», cette notion étant aussi contradictoire
que celle d’un « carré rond». En effet, nulle réalité n'existe sans une « donation
de sens» ({\it Sinngebung}) et c’est la conscience qui est ici la « donatrice ».
Sans doute, il ne saurait être question d’en revenir à un idéalisme subjectif,
parent de celui de Berkeley : la « réduction phénoménologique » (§ 327) s’y
oppose. Mais il n’en est pas moins vrai que toute réalité se constitue dans
la conscience : « Tout ce que nous nommons {\it objet}, ce dont nous parlons,
ce que nous avons sous les yeux à titre de réalité, ce que nous tenons pour
possible ou invraisemblable..., tout cela est déjà par là même un objet de
conscience ; tout ce qui peut être et s'appeler monde et réalité doit être
représenté dans le cadre de la conscience réelle et possible.s Dans les
{\it Méditations cartésiennes} (1929), Husserl qualifie sa doctrine d’\textsf{\textit {Idéalisme
transcendantal}}, « bien que dans un sens fondamentalement nouveaus. Il ne s’agit,
précise-t-il, ni d’un idéalisme psychologique, ni d’un idéalisme kantien
qui laisserait ouverte la possibilité d'un monde de choses en soi. C’est un
idéalisme qui « ne s'oppose pas par un jeu d'arguments à quelque {\it réalisme} »
%171
et qui n’est rien de plus qu’une explicitation des \textsf{\textit {sens}} de tout ce qui peut
être donné à l’{\it ego} comme sujet de connaissances possibles, et notamment
« du sens de la transcendance que l'expérience me donne réellement : celle
de la nature, de la culture, du monde en général ».

On comprend l’hésitation des commentateurs en face de cette
doctrine\footnote{Husserl lui-même qui avait adopté le terme d'{\it idéalisme} avant les {\it Ideen} (1913)
et qui l'emploie à nouveau dans les {\it Méditations cartésiennes}, ne l'emploie pas dans les
{\it Ideen} elles-mêmes.}. Tandis que les uns l’accusaient de mêler une sorte de
réalisme platonicien à un idéalisme subjectif, d’autres admettaient
qne, réaliste au début, \textsc{Husserl} était devenu idéaliste par la suite.
Peut-être faut-il dire, avec M. \textsc{Merleau-Ponty}, que, « loin d’être,
comme on l’a cru, la formule d’une philosophie idéaliste, la réduction
phénoménologique est celle d’une philosophie existentielle ».

{\it C.} \textbf{\textit {L’Existentialisme.}} L’Existentialisme est assurément un réalisme.
Selon la remarque d’un philosophe américain, J.-P. \textsc{Sartre},
en posant comme principe commun à toutes les variétés de l’existentialisme
l’aflirmation que {\it l'existence précède l'essence}, a donné, par
antithèse, la meilleure définition de l’Idéalisme, celui-ci étant, au
contraire, l’affirmation de l’antériorité logique de l’essence sur l’existence
(H. G. \textsc{Townsend}).

Dans l’existentialisme sartrien, « les objets, les choses, le monde, tout le
réel, tout ce qui peut avoir forme d’être ou rapport avec l'être et même les
prétendus « faits mentaux » et le Moi de la psychologie sont ici transcendants
à la conscience : tout le {\it transcendantal} s’est réfugié dans une affirmation
réaliste : celle de l'{\it en-soi} » (G. \textsc{Varet}). L'{\it en-soi} devient ainsi le
prototype de l'être. Certes, les « choses » paraissent d’abord se réduire « à la
totalité liée de leurs apparences ». Mais on aperçoit bien vite que ces apparences
elles-mêmes réclament « un être qui ne soit plus lui-même apparences,
un être {\it transphénoménal}, à savoir le sujet qui les perçoit. Et ainsi, se trouve
dépassé d'emblée l’Idéalisme, selon lequel « l'être est mesuré par la connaissance» :
on saisit au contraire « un être qui échappe à la connaissance et qui
la fonde ». {\it Penser} n’est qu’un mode de l'{\it exister}.

Mais, d’autre part, l’Existentialisme est fils de la Phénoménologie,
dont on a dit que «sa plus importante acquisition est d’avoir joint
l'extrême subjectivisme à l'extrême objectivisme » (\textsc{Merleau-Ponty}).
De fait, dès 1936, J.-P. Sartre écrivait que « la dualité sujet-objet
[devait] disparaître des préoccupations philosophiques », et {\it L'Être
et le Néant} (1943) s’intitule {\it essai d’ontologie phénoménologique} :
«L'être d’un existant, c’est précisément ce qu’il apparaît.» Le
« transphénoménal » lui-même, tel qu’il est ici entendu, «c’est au
vrai le phénoménologique comme tel, puisque c’est, dans l'intuition
phénoménale, le {\it sens d’être} du phénomène, sens que seule sa phénoménalisation
%172
 effective permet de déchiffrer » (Varet). {\bf —} L'auteur de
{\it L’ Être et le Néant} prétend, comme Bergson, se situer « par delà le réalisme et
l’idéalisme ». Mais, en réalité, n’oscille-t-il pas «entre leur
affirmation simultanée et leur exclusion mutuelle » (Varet)?

\section{Conclusion}% 119
De toutes ces doctrines, si diverses en apparence,
on peut cependant dégager, nous semble-t-il, une idée commune.
C'est que le monde sensible, {\it tel que le sens commun se le représente},
ne possède pas la réalité qu’il lui attribue. Berkeley et, en un sens,
Bergson ont beau soutenir que les objets extérieurs sont bien ceux
qui sont saisis immédiatement par les sens. Ils sont pourtant très
loin, l’un et l’autre, du sens commun. Celui-ci croit à des objets purs,
à des choses matérielles, placés en face d’une conscience, elle-même
sujet pur, dont le rôle se bornerait à les contempler. Or cette dualité
est une erreur. La conscience commence, chez l’enfant, par un {\it syncrétisme}
où sujet et objet ne sont pas encore nettement distincts
(ci-dessus, § 83 et t. II, ch. VIII). Même pour l’adulte, il n’est pas plus
de sujet pur que d’objet pur : comme l’a montré la phénoménologie,
la conscience est intentionnelle, elle est toujours conscience {\it de} quelque
chose (ci-dessus, § 118 B). La psychologie nous montre d’autre part
que l’objet est toujours plus ou moins mis en {\it forme} et construit par
l'esprit (voir les § 81-82 et 84). {\bf —} On peut donc conclure qu’il y a bien
une réalité du monde sensible, non pas en ce sens que celui-ci existerait
tel que nous le percevons, mais en ce sens que, quelle que soit la part
de l'esprit qui perçoit et qui, en percevant, sélectionne et structure le
« donné » brut {\bf —} en grande partie, comme l’avaient déjà vu Descartes
et Malebranche et comme l’a montré aussi Bergson, sous l'influence
d’exigences pragmatiques, {\bf —} il y a aussi, dans la perception, {\it ce qui
résiste à l'esprit}, une {\it matière} à laquelle il se heurte et surtout sur laquelle
s’exerce notre {\it action} (§ 340 {\it E}). L'esprit construit le réel (chap. VII,
§ 103-104), mais il ne le construit ni arbitrairement, ni de toutes
pièces.

\section{Sujets de travaux}% 


{\it Exercices.} {\bf —} 1. {\it Classer les diverses acceptions dans lesquelles peuvent
être pris, en philosophie, en esthétique, en morale et dans la langue courante,
les mots} idéalisme {\it et} réalisme. {\bf —} 2. {\it Commenter ces définitions de l’idéalisme
Philosophique en essayant de discerner à quel philosophe chacune d'elles peut
s'appliquer de préférence} : « L’idéalisme antique se caractérise par la distinction
du sensible et de l’intelligible, par la découverte de l’{\it Eidos}» (\textsc{Darbon}),
« Pour l’idéaliste, il n’y a rien de plus dans la réalité que ce qui apparaît à {\it ma}
conscience ou à la conscience en général» (\textsc{Bergson}), « Axiome où se résume
la philosophie idéaliste : ce qui existe des choses, ce sont les idées que
%173
l'esprit en possède» (\textsc{Lyon}), « L'{\it idéalisme dogmatique} regarde l'espace avec
tout ce dont il est la condition comme quelque chose d'impossible et, par
conséquent, rejette également l'existence des choses matérielles qui y
sont contenues» (\textsc{Kant}), « En ontologie, l’idéalisme consiste à dire que les
choses ne sont rien de plus que nos propres pensées » (\textsc{Goblot}), « Doctrine
suivant laquelle, la philosophie se réduisant à la théorie de la connaissance,
nous ne pouvons atteindre que le subjectif et le phénoménal»s (É. Hazévv),
« [Selon l’idéalisme], l'affirmation de l'être a pour base la détermination de
l'être comme connu, par opposition au réalisme qui a pour base l'intuition
de l'être en tant qu'être » (\textsc{Brunschvicg}), « C’est la thèse fondamentale
de l’{\it idéalisme critique} d’avoir établi l'autonomie du sujet, la liberté absolue
de l'esprit et d’avoir posé l’Être, non plus comme une réalité indépendante,
ayant une existence en soi et à soi, mais comme purement relatif à l'esprit»
(X. \textsc{Léon}), « L'idéalisme consiste à croire que le monde, tel du moins que
je puis le connaître et en parler, se compose exclusivement de représentation,
et même de {\it mes} représentations, actuelles ou possibles, matérielles ou
formelles [temps, espace, causalité, etc.]» (\textsc{Lachelier}). {\bf —} 3. {\it Commenter
ce passage de G. Lyon à propos des rapports entre l'idéalisme et la science} :
« Qui supposera que l’investigation des causes naturelles soit jamais entravée
par la doctrine qui a précisément pour essence de consacrer sans restrictions
les droits illimités de notre savoir ? Nier l'existence de la matière,
n'est-ce pas écarter de l’horizon scientifique tout {\it block-stone} opposé à la
curiosité rationnelle ? N'est-ce pas interdire tout arrêt mis à la déduction
ou à l'analyse? N'est-ce pas déclarer que rien, en son ultime fond, n’est
inaccessible et inconnaissable ? N'est-ce pas, avec Leibniz, proclamer le
principe de l’universelle intelligibilité?» (G. \textsc{Lyon}).

Exposés oraux. {\bf —} 1. {\it Les éléments idéalistes de la philosophie de Malebranche}
(d’après G. \textsc{Lyon}, chap. IV). {\bf —} 2. {\it L'idéalisme de Berkeley d'après
les} Dialogues (voir l'introd. de \textsc{Parodi} et \textsc{Lyon}, chap. VIII). {\bf —} 3. {\it La
Phénoménologie} (d’après G. \textsc{Berger}, {\it Les thèmes principaux de la phénoménologie
de Husserl}, dans {\it Études de Métaphysique et de Morale}, 1944, I,
p. 22, et l’avant-propos de M. \textsc{Merleau-Ponty}).

{\bf Discussions.} {\bf —} 1. {\it Le platonisme est-il un idéalisme ou un réalisme ?} {\bf —}
2. {\it L'idéalisme entrave-t-il ou favorise-t-il la recherche scientifique ?}

{\bf Lectures.} {\bf —} Outre les lectures déjà indiquées pour le chapitre VII :
{\it a.} \textsc{Malebranche}, {\it Entretiens sur la Métaphysique} (1688), éd. Vrin, tome I,
entr. I, III et VI. {\bf —} {\it b.} \textsc{Leibniz}, {\it La Monadologie} (1714), éd. Boutroux,
Delagrave. {\bf —} {\it c.} \textsc{Berkeley}, {\it Dialogues entre Hylas et Philonoüs} (1713),
trad. Beaulavon-Parodi, Alcan. {\bf —} {\it d.} Georges \textsc{Lyon}, {\it L'Idéalisme en
Angleterre au {\footnotesize XVIII}$^\text{e}$ siècle}, Alcan, 1888, chap. I, IV et VIII. {\bf —} {\it e.} \textsc{Bergson},
{\it Matière et Mémoire}, Alcan, 1898, spéc. chap. IV. {\bf —} {\it f.} \textsc{Parodi}, {\it La Philosophie
contemporaine en France}, Alcan, 1925, spéc. chap. VIII et XI et
suppl. {\bf —} {\it g.} R. \textsc{Le Senne}, {\it Introd. à la Philosophie}, Alcan, 1939, 1$^\text{re}$ partie.
{\bf —} {\it h.} M. \textsc{Merleau-Ponty}, avant-propos à sa {\it Phénoménologie de la perception},
Gallimard, 1945. {\bf —} {\it i.} Gilbert \textsc{Varet}, {\it L'ontologie de Sartre}, P. U. F.,
1948. {\bf —} {\it j.} A. \textsc{Lalande}, {\it La Raison et les normes}, Hachette, 1949, chap. III
et VII {\bf —} {\it k.} R. \textsc{Blanché}, {\it Les attitudes idéalistes}, P. U.F., 1949.
{\bf —} {\it l.} H. G. \textsc{Townsend}, {\it L'idéalisme en Amérique}, dans M. \textsc{Farber}, {\it L’activité
philos. contemp. aux É.-U.} trad. fr., P. U. F., 1950, I, ch. III. {\bf —} {\it m.} Joseph
\textsc{Moreau}, {\it L'horizon des esprits}, P. U. F., 1960. {\bf —} n. G. \textsc{Mauchaussat},
{\it L'Idéalisme de Lachelier}, P. U. F., 1961.

