\chapter{La raison}% XVII.

%209. Qu'est-ce que la Raison? — 210. Les principes rationnels. — 211. Les
%catégories. — 212. Rationalisme et Empirisme classiques. — 213. Discussion.
%— 214. Raison et vie. — 215. Raison et société, — 216. La Raison vivante.
%— 217. La Raison, faculté de l'Ordre. — 218. Fonction pratique de la Raison.
%— 219. Les limites de la connaissance rationnelle,

\section{Qu'est-ce que la Raison ?}% 209.
Il ne faut pas confondre le
Raisonnement et la Raison
{\scriptsize (La confusion est fréquente, surtout de nos jours. Certains auteurs, même classiques,
prennent d'ailleurs le mot {\it raison} comme synonyme de {\it raisonnement}. Ainsi Pascal,
notamment, {\it Pensées}, fr. 252 : « La raison agit avec lenteur, et avec tant de vues sur
tant de principes. qu'à toute heure elle s’assoupit ou s’égare, manque d’avoir tous ses
principes présents. » C’est la définition même de la pensée discursive (voir \S 206-207).)}.
Le Raisonnement lui-même est astreint à
certaines {\it normes}, il est soumis à une certaine {\it régulation} : il doit
notamment être « logique », c’est-à-dire cohérent. {\it Ces normes, cette
régulation, ce sont celles de la Raison}. Ce sont elles notamment qui
rendent possible cette intuition récapitulative qui, au terme du raisonnement, nous permet de saisir « d’une seule vue de l'esprit »,
ainsi que disait Descartes, une suite discontinue d’arguments
logiquement enchaînés et de l’apercevoir alors comme continue et
évidente : il s’agit alors d'une évidence proprement {\it rationnelle}.
D'autre part, on a vu (\S 159 {\it fin}) que, quelles qu’aient été ses origines
%341
pratiques, l'intelligence peut parvenir à s’émanciper de ces
origines et se hausser à la prise de conscience des rapports comme
tels, c’est-à-dire substituer de l’{\it intelligible} au sensible. On verra que
c’est, en grande partie, l’œuvre de la Science que de retrouver ainsi
de l’{\it intelligible}, non seulement dans les formes et les rapports mathématiques, mais aussi dans ces phénomènes de la nature qui, comme
le disait John Stuart Mill, « ne nous offrent au premier coup d’œil
qu’un chaos suivi d’un autre chaos », voire dans ces phénomènes
humains qui semblent à première vue le jouet de nos fantaisies et
de nos caprices. Or, c’est là encore {\it faire œuvre de Raison}. Qu’est-ce
donc que la Raison? Les philosophes classiques ont fait consister
la Raison en un système de {\it principes}, dits « principes rationnels »
ou « principes directeurs de la connaissance », et de {\it catégories}, les
« catégories de l’entendement », qui constitueraient, Pour ainsi dire,
l’armature de la pensée logique. Mais on verra par la suite que ce
n’est là que la Raison « constituée », comme a dit A. Lalande, la
Raison cristallisée, à une époque de son histoire, en certains cadres
logiques, mais que, par derrière celle-ci, il existe une Raison vivante,
« constituante », dont principes et catégories ne sont que l'expression.

\section{Les principes rationnels}% 210.
Étudions d’abord les principes
rationnels. — {\it A}. Le plus.fondamental est le \bi{principe d'identité} qu’on a
énoncé : « Ce qui est, est », « A est A ». Il n’est que l’expression codifiée,
donc quelque peu artificielle, de ce besoin de {\it cohérence logique} qui est
peut-être l’exigence essentielle de la Raison. Ainsi, lorsqu’en Mathématiques nous avons défini un concept (le concept de cercle par
exemple), il va de soi que, dans toute la suite du raisonnement et
des théorèmes, nous devons maintenir ce concept {\it identique} à sa définition première. De même, nous ne pouvons admettre logiquement
qu’un théorème puisse être en contradiction avec un autre théorème
(du moins, dans un système donné de postulats : cf. le \S 253 {\it A}) :
la condition première de la vérité est que la pensée demeure {\it d'accord
avec elle-même}. Même dans les Sciences expérimentales, expliquer,
c’est souvent {\it identifier}, ramener au « même », c’est-à-dire à ce qui
est déjà connu. C’est ainsi qu'Aug. Comte considérait comme le
type de l'explication scientifique la réduction des phénomènes de
pesanteur, de la gravitation des astres, etc., à l’attraction newtonienne, parce que tous ces phénomènes apparaissent désormais,
disait-il, « comme n’étant {\it qu’un seul et même fait} envisagé sous divers
points de vue ».

On verra toutefois(cf. le chap. XXI) que les choses ne sont pas aussi
simples. L'identité ne se trouve atteinte, le plus souvent, que par un
%342
mouvement « dialectique » de la pensée qui consiste à surmonter des
contradictions. Dès l'{\it observation} des faits, l'esprit se trouve en présence d'erreurs à rectifier, d’oppositions apparentes à dépasser.
Ce caractère se retrouve éminemment dans les grandes {\it théories}, à tel
point qu’on a pu y faire consister l’esprit scientifique lui-même.
Peut-être d’ailleurs se retrouve-t-il au fond de la pensée en général :
l’évolution de la pensée de l’enfant consiste, elle aussi, sinon à surmonter des contradictions, du moins à dissiper des confusions
(ci-dessus, \S 165), et la pensée de l’adulte lui-même recèle des contradictions internes. En ce sens, on pourrait dire que la connaissance
est, tout entière, « un effort pour résoudre des contradictions »
(H. Wallon). — D’autre part, dans le domaine expérimental, l’exigence d'{\it identité} se heurte, comme l’a montré É. Meyerson (1859-1934),
à la structure d’un réel qui ne se laisse pas, immédiatement et sans
effort, réduire aux normes de la Raison (voir ci-dessous \S 319).

{\it B.} On considérait autrefois comme pouvant se déduire du principe
d’identité le \bi{principe de non-contradiction} et le \bi{principe de l'alternative} ou \bi{du tiers exclu}. Ces principes expriment deux relations
fondamentales entre un attribut A et l’attribut contradictoire non-A,
{\it tous deux étant bien définis par rapport à un même sujet}. — Aristote
énonçait le premier : « Une même chose ne peut pas, en même temps
et sous le même rapport, être et ne pas être dans un même sujet » ;
autrement dit, « A n’est pas non-A ». Par exemple, si un nombre
entier est pair, il n’est pas impair, et réciproquement, Le principe du
tiers exclu pose une alternative : {\it étant supposé que A et non-A sont
contradictoires}, un sujet est nécessairement soit A, soit non-A, et il
n'y a pas de troisième solution possible, Par exemple, un nombre
entier est nécessairement ou bien pair ou bien impair, et le principe du
tiers exclu me contraint à affirmer l’un ou l’autre. C’est sur ce principe qu'est fondé le raisonnement par l’absurde. On verra cependant
qu'il a soulevé quelques difficultés en Mathématiques.

{\it C.} Le \bi{principe de causalité} peut s’énoncer : « Tout fait a une cause
et, dans les mêmes conditions, la même cause est toujours suivie du
même effet. » Il signifie donc « qu'aucun changement n’est absolument inintelligible, car au-dessous de tout changement on peut
découvrir quelque chose qui n’a pas changé, une cause qui se prolonge
dans un effet, une permanence ou tout au moins une équivalence »
(D. Roustan). Le principe de causalité apparaît ainsi comme une
sorte de compromis entre l'exigence d’identité de l'esprit et l’irrationalité, au moins apparente, d’un univers en perpétuel devenir : il
n’est « que le principe d'identité appliqué à l'existence des objets
dans le temps » (E. Meyerson). Par exemple, c’est la même cause, la
%343
poussée d’Archimède, qui explique qu’un corps flotte ou qu’il s’immerge ; c’est la même cause, la pression atmosphérique, qui explique
les variations de la colonne barométrique. La notion de {\it permanence}
est encore plus nette lorsqu'un Lavoisier montre qu’à travers les
changements des propriétés physiques et chimiques on retrouve
dans le composé le poids des composants, ou lorsque le physicien
découvre, sous les transformations qualitatives de l’énergie, l’équivalence de leur valeur quantitative. — Le \bi{principe du déterminisme}
selon lequel tout phénomène est déterminé par ses conditions d’existence et donc rigoureusement prévisible si l’on connaît celles-ci
(\S 210 C et 270), n’est qu’une autre forme de la même idée. En effet,
ce qui demeure constant, c’est beaucoup moins, aux yeux du savant
moderne, une réalité substantielle, une « chose », qu’un {\it rapport}.
Lorsqu'un savant ramène à un même {\it type} une diversité d’êtres, il
découvre, dans ces êtres peut-être très différents en apparence, une
{\it identité de structure}. En ce sens, le type est déjà une sorte de loi qui
veut que certains caractères essentiels soient toujours unis (cf. \S 277
et 300 {\it A}). La loi proprement dite plie des phénomènes divers dans leurs
particularités de détail à l’{\it identité d’un même rapport} : une loi est un
rapport {\it constant} entre des éléments qui varient, et ainsi l’explication
scientifique consiste toujours à découvrir sous le changement apparent
une identité fondamentale.

{\it D.} Les philosophes classiques avaient ajouté à ces principes celui de
\bi{finalité} selon lequel les différents éléments d’un tout sont comme des
moyens ordonnés en vue d’une fin. On verra (cf. \S 276) que la finalité est de peu d’utilité dans la science. Au fond, sous sa forme primitive,
elle est encore un effort pour plier la nature aux lois de l’esprit, mais
cette fois de la manière la plus naïve, celle qui consiste à assimiler la
nature à l’homme en lui prêtant des {\it tendances} ou des {\it intentions} analogues
à celles de la nature humaine.

\section{Les catégories}% 211.
On rattache aussi à la Raison, au sens
large, les « catégories de l’entendement », c’est-à-dire les concepts les
plus généraux qui sont comme les cadres logiques de la pensée.
Aristote distinguait dix catégories {\scriptsize (C'étaient : l'{\it essence} par ex., homme, cheval ; la {\it quantité} : long de 3 coudées ; la
{\it qualité} : blanc, grammatical ; la {\it relation} : double, demi ; le {\it lieu} : au Lycée, au marché;
le {\it temps} : hier, l’an passé ; la {\it situation} : couché, assis ; la {\it manière d'être} : chaussé, armé;
l'{\it action} : couper, brûler ; la {\it passion} : être coupé, ètre brûlé.)}, Kant en compta douze, groupées
sous les quatre chefs de la {\it quantité}, de la {\it qualité}, de la {\it relation} et de la
{\it modalité} et dont voici le tableau :

%344
\begin{center}
\begin{tabular}{ | c | c | c | c | }
  \hline
 Quantité & Qualité & Relation & Modalité \\
  \hline
 Unité & Réalité & Substance et accident & Possibilité \\
 Pluralité & Négation & Cause et effet & Existence \\
 Totalité & Limitation & Réciprocité & Nécessité \\
  \hline
\end{tabular}
\end{center}

\vspace{0.24cm}
{\footnotesize La catégorie d'unité correspond aux {\it jugements universels} (ex. : l’homme
est mortel), celle de pluralité aux {\it jugements particuliers} (certains hommes
sont sages) et celle de totalité aux {\it jugements singuliers} (Socrate fut un sage).
Celle de réalité aux {\it jugements affirmatifs} (l'âme est spirituelle), celle de négation aux {\it jugements négalifs} (l'âme n’est pas matérielle) et celle de limitation
aux {\it jugements indéfinis} ou à attribut négatif (l’âme est immortelle). La
relation de la substance à l’accident (inhérence) correspond aux {\it jugements
catégoriques} (cet homme est blond), la catégorie de causalité aux {\it jugements
hypothétiques} (la chaleur dilate les corps ou : si un corps s’échauffe, il se
dilate) et celle de réciprocité aux {\it jugements disjonctifs} (un corps est solide
ou liquide ou gazeux). Celle de possibilité aux {\it jugements problématiques}
(il se peut qu'il fasse beau), celle d'existence aux {\it jugements assertoriques}
(il fait beau) et celle de nécessité aux {\it jugements apodictiques} (un corps est
nécessairement étendu).}
\vspace{0.31cm}


Kant avait toutefois distingué la Raison proprement dite ({\it Vernunft}), définie comme la « faculté des principes », de l’Entendement
({\it Verstand}). Ses successeurs {\it néo-criticistes} français, Charles Renouvier
(1815-1903) et Octave Hamelin (1856-1907) vont au contraire faire
de la Raison le système même des catégories et rejetteront comme
artificielle la distinction de la raison et de l’entendement.

\vspace{0.24cm}
{\footnotesize
Tout en affirmant que la connaissance est essentiellement relative
(voir \S 324), Kant avait maintenu la notion d’une « chose en soi », d’un
absolu, d’ailleurs inaccessible à l'esprit humain. Le {\it néo-criticisme} proclame
au contraire que, « puisque dans la représentation {\it tout est relatif} et que rien
n’est connu qu’à la faveur d’une relation quelconque, la loi la plus générale
entre toutes est la {\it relation} même ». Or, selon Renouvier, les catégories
sont précisément « les lois premières et irréductibles de la connaissance ».
Par suite, la \si{Relation} est « la première des catégories », que les autres
({\it nombre, position, succession, qualité, devenir, causalité, finalité, personnalité})
« ne font que diversifier ». Mais, pour Renouvier, cette liste des catégories,
au lieu d’être construite {\it a priori} comme chez Kant, n’a « qu’une valeur
empirique » ; c’est un simple « tableau de l’esprit humain ». — Hamelin au
contraire, tout en adoptant le même point de départ et une table des
catégories presque identique, prétendra construire « synthétiquement » ces
« éléments principaux de la représentation » par une dialectique ascendante
inspirée de la méthode hégélienne (ci-dessus \S 115). Les catégories s’{\it engendrent} ainsi selon un processus ternaire (thèse, antithèse, synthèse) et se
%345
groupent en triades qui partent de la plus abstraite pour aboutir à la plus
concrète : {\it relation-nombre-temps, temps-espace-mouvement, mouvement-qualité-altération,
altération-spécification-causalité, causalité-finalité-personnalité}.}
\vspace{0.31cm}

Renouvier lui-même avait reconnu que de telles systématisations
sont quelque peu arbitraires. Il s’en faut d’ailleurs que les catégories
fondamentales de la pensée humaine soient immuables. Une notion
comme celle de la {\it substance} que Kant avait maintenue dans sa table
des catégories, a perdu aujourd’hui beaucoup de son importance.
Elle figure, dit Renouvier, « les idées d'identité et de permanence
qu’on applique aux sujets empiriques et variables de phénomènes ».
Mais, pour cela, les philosophes classiques crurent devoir supposer
un {\it substratum} assez mystérieux, caché en quelque sorte sous {\scriptsize (Il suffit de songer à l’étymologie du mot {\it }substance : c'est ce qui se tient (latin :
{\it stare}) sous (latin : {\it sub}) les apparences changeantes.)} les
phénomènes et qui resterait identique alors que ces phénomènes
changent. Or, on l’a vu plus haut, {\it ce qui demeure identique} sous le
changement, c’est le plus souvent, pour la science, un rapport, et non
une « chose », non une réalité mystérieuse, et c’est pourquoi ni Renouvier ni Hamelin ne l’ont admise au nombre des catégories. Mais on
verra (cf. \S 265) que la notion de {\it cause} elle aussi a considérablement
évolué et qu’elle semble parfois, au moins dans la science, s’effacer
devant celle de {\it loi}. Nous avons déjà dit (\S 210 D) que le rôle de celle
de {\it finalité} est plus discutable encore. Ceci nous conduira à nous
demander si cette conception de la Raison comme constituée par une
armature rigide de principes et de catégories est bien celle qui
convient.

\section{Rationalisme et Empirisme classiques}% 212.
C’est cependant
cette conception qui a presque exclusivement inspiré les théories
classiques.

{\it A.} Le \bi{Rationalisme} classique, en particulier, a à peu près identifié
ce système de principes et de catégories avec la Raison elle-même.
D’après lui, la Raison est universelle et éternelle : elle est la même
chez tous les hommes, et elle ne change pas, n’évolue pas. Pour
Descartes (fig. 2), elle est la « lumière naturelle » qui se confond
avec l’exercice même : de la pensée attentive, soucieuse de « clarté »
et de « distinction » : c’est en ce sens qu’on peut parler d'idées innées,
telles que l’idée de l'infini, l’idée de l’étendue et la plupart des notions
mathématiques, etc. ; mais ces idées innées n’existent pas en nous
toutes faites dès la naissance : ce sont des « semences de vérités qui
sont naturellement en nos âmes » et, au fond, elles ne diffèrent pas
%346
de la faculté même qu’a l’âme de penser. En ce sens, l’esprit humain a,
dit Descartes, « je ne sais quoi de divin »; car ces semences de vérité
ont été « mises en nous » par Dieu. — Cette interprétation
métaphysique s’accentue chez Malebranche (figure 95). Les
{\it idées} deviennent pour lui des
essences éternelles et immuables que nous « voyons » en Dieu.
En effet la Raison, étant universelle, ne peut appartenir en
propre à la créature qui est toujours un être particulier :
« La Raison qui éclaire l’homme, est le Verbe ou la Sagesse de
Dieu même » et toute connaissance véritable est \bi{participation}
à cette Raison divine, \bi{illumination} de l'esprit par Dieu. —
Leibniz (figure 22) insiste plutôt sur le caractère {\it virtuel}
de l’innéité, déjà reconnu par Descartes. Les idées et les vérités
nous sont innées, dit-il, comme une statue serait préformée dans
un bloc de marbre dont les veines en dessineraient déjà la
figure. L’esprit n’en a pas moins une activité propre indépendante
de l’expérience : {\it rien}, disaient
les empiristes, {\it n’est dans l’intellect qui n'ait été d’abord dans
la sensibilité}; Leibniz ajoute : \bi{sauf l’intellect lui-même}. Il
déclare même que « toutes les pensées et actions de notre âme
viennent de son propre fonds » et semble ainsi admettre que
toute la connaissance n’est que le développement des virtualités
propres à l’esprit, un passage de la pensée obscure à la pensée claire.
— Kant (figure 33) fait une place beaucoup plus large à l’expérience. L’esprit est bien constitué, selon lui, par des \bi{formes
%347
a priori} : les deux {\it formes a priori de la sensibilité} qui sont l’espace
et le temps, les douze {\it catégories de l’entendement} dont nous avons
donné ci-dessus le tableau, et les trois {\it Idées de la Raison}, celles
du {\it moi}, du {\it monde} et de {\it Dieu} ( chap. XXV). Mais ces {\it formes} ne sont
que des cadres vides que l’esprit impose au donné brut de l'intuition
et elles ne prennent valeur de connaissance que par la « matière », le
contenu qu'y introduit l'expérience, si bien que, lorsque cette
« matière » fait défaut, comme dans le cas de la métaphysique ontologique qui prétend dépasser toute expérience, il n’est plus de connaissance valable {\scriptsize (Malgré son rationalisme métaphysique, Malebranche avait déjà, beaucoup
mieux que Leibniz, trop exclusivement mathématicien, reconnu ce rôle indispensable
de l’expérienec (voir ci-dessous \S 219).)}.

{\it B.} L'\bi{Empirisme} au contraire n’a reconnu aucun principe {\it a priori}.
L'esprit n’est originellement, selon lui, qu’une « table rase » (allusion
aux tablettes de cire sur lesquelles écrivaient les Anciens) qui reçoit
simplement les empreintes qui lui viennent de l'extérieur. Toute
connaissance est donc {\it a posteriori}, et la Raison n’est que l’ensemble
des habitudes qui se sont ainsi imprimées peu à peu en lui. Déjà
John Locke (1632-1704), combattant la doctrine cartésienne des
idées innées, avait prétendu montrer que toutes ces idées nous viennent
de l'expérience sensible. Il avait cependant reconnu encore à l’esprit
quelque activité en ajoutant à la \bi{sensation}, comme source de nos
idées, la \bi{réflexion}, définie comme « la connaissance que l’âme prend de
ses propres opérations ». — Plus tard, Condillac (1715-1780) ramènera ces deux sources de nos idées à une seule : {\it la sensation}. — Mais
c’est surtout David Hume (fig. 34) qui a présenté de l’Empirisme
une interprétation intéressante, laquelle sera le point de départ des
réflexions de Kant. Il a principalement analysé la catégorie de \bi{causalité}. Loin d'être une idée innée, loin d’impliquer une connexion logiquement nécessaire entre la cause et l’effet, celle-ci n’est, à ses yeux,
qu'une \bi{habitude} ({\it habit, custom}) ou une \bi{croyance} ({\it belief}), purement
subjective, qui résulte, dans notre esprit, des successions régulières,
des « conjonctions constantes » que nous présente l’expérience (voir
ci-dessous \S 265) et qui nous fait attendre l'effet quand nous percevons
la cause. — Au {\scriptsize XIX}$^\text{e}$ siècle, J. S. Mill (1800-1873) et Herbert Spencer
(1820-1903) modifieront à peine cette doctrine, en y ajoutant, le
premier, la notion d’ « {\it associations inséparables} » que créent dans notre
esprit les séquences répétées des phénomènes et qui nous donnent
l'illusion d’une nécessité logique, le second, la notion de l’{\it évolution}
et, avec elle, celle du long passé de l’espèce.
%348
\section{Discussion}% 213.
Dans l’ensemble, on observera que les deux
doctrines ont plutôt {\it tendu à se rapprocher}. Le Rationalisme, avec
Kant, renonce à l’idée d’un
savoir inné et admet que l’expérience est indispensable pour
constituer une connaissance
effective, tandis que l’Empirisme, avec Spencer, reconnaît
que, grâce à l’expérience ancestrale devenue héréditaire, l’esprit de l’individu a cessé d’être
une « table rase ». C’est qu’en effet, comme on le verra bientôt
(\S 216), les deux doctrines pèchent par une conception trop
étroite de la Raison. — D'abord ni l’une ni l’autre, l’Empirisme
pas plus que le Rationalisme, ne font de place à l’idée d’une
véritable {\it évolution de la Raison}. Or l’examen des principes et des
catégories nous a déjà montré
que cette idée semble bien s’imposer. — D'autre part, chaque
doctrine se heurte, dans son
domaine, à des difficultés insurmontables. Pour le Rationalisme
classique, le principe d'identité
par exemple (que l’on peut considérer comme la norme fondamentale de la pensée rationnelle)
tient à l'{\it essence même de l'esprit
humain}. Or, sur une telle affirmation, un psychologue tel que
Pierre Janet ne nous laisse aucune illusion : « Le principe
d'identité est, disait-on autrefois, une loi absolue de l'esprit
à laquelle la pensée ne peut pas
échapper. Quelle erreur ! Dans les bavardages, dans les rêves, dans les
délires, les contradictions et les absurdités sont perpétuelles. » Et l’on
sait que bien d’autres psychologues contemporains se sont appliqués
%349
à mettre en lumière l’existence en nous de tout un fonds d’{\it irrationnel}
(voir les \S 41 et 165). Tout au moins, si l’on veut y voir une loi de
la pensée, faudrait-il interpréter le principe d'identité de façon beaucoup plus complexe, dialectique, comme il a été dit au \S 210. — Mais,
d’autre part, ce n’est pas une moindre erreur de croire, avec l’Empirisme, que ce principe nous est imposé par l'expérience. « Le principe
d'identité ne peut venir de l'expérience, dit A. Spir {\scriptsize (African Srir, philosophe d'origine russe, 1838-1890.)}, par la raison
bien simple que l’expérience {\it ne s'accorde pas} avec lui. » Les phénomènes sont sans cesse changeants et mouvants : l’identique, ce serait
l'absolu. Si l’on se résigne à voir dans le principe en question « une
simple loi formelle de la pensée », alors « aucun objet d’expérience
ne pourrait être désigné par un prédicat qui fût discernable de son
concept » {\scriptsize (Le problème s'était déjà posé dans l'antiquité, dans l’école des Éléates (voir \S 190)
et chez ces « terribles gens » dont parle quelque part Platon (ce sont peut-être les Mégarisques ou plutôt les Cyniques et dont la doctrine rendait impossible toute application
d'un attribut à un sujet, donc tout jugement.)} : il faudrait toujours dire : A {\it est} A, jamais : A {\it est} B. Autrement dit, l’Empirisme met l’ordre dans les choses elles-mêmes, dans
le donné, et, à cet égard, nous avons vu que certaines doctrines
contemporaines, telles que le Gestaltisme, relèvent de l’empirisme pur
(\S 82 B). Mais ne serait-ce pas plutôt l’esprit — et ce serait là la vérité
profonde qu’il y aurait à retenir du Rationalisme — qui ordonne les
choses, le donné de l’expérience selon ses exigences propres?

\section{Raison et vie}% 214.
Les doctrines contemporaines se sont
appliquées à rendre compte de ces exigences et, pour cela, à libérer la
Raison du « splendide isolement » où l’avaient confinée les doctrines
classiques, à la mettre en relation avec les conditions réelles et
concrètes de l’existence humaine. Or nous savons que ces conditions
sont de deux sortes : {\it biologiques} d’une part, {\it sociologiques} de l’autre.
Commençons par les premières. — Pourquoi, en dépit des difficultés
que nous venons d’exposer et bien que l’expérience ne le vérifie pas
à strictement parler, ne s'accorde pas avec lui, le principe d’identité
est-il devenu malgré tout la norme suprême de la pensée {\it rationnelle} ?
Si l’on en croit certains philosophes, ce serait qu’il répond à une
\bi{exigence vitale}. En tant qu'être vivant, l’homme doit nécessairement
{\it s'adapter} au monde ambiant. Or toute adaptation serait impossible
dans un monde dont le changement serait la loi absolue, où l’on ne
retrouverait jamais rien de constant. Il lui réussit donc de traiter
comme pratiquement identiques des objets ou des situations qui ne
sont qu’approximativement semblables : n’est-ce pas déjà ce que
%350
réalise l'habitude (tome II, chap V) ? « Le principe d'identité des
logiciens, dit Th. Ruyssen, n’est autre chose que l’expression formelle de la loi d'habitude. » — C’est à un point de vue quelque peu
analogue que s’est placé le \bi{pragmatisme} de W. James qui, comme on
le verra (cf. le \S 325), a défini la vérité par la réussite. D’après lui,
le « sentiment de rationalité » ne serait rien d’autre qu’un état « de
tranquillité, de paix, de repos » pour l'esprit : il y a avantage, pour
l’homme, à se sentir d’accord avec lui-même et le principe d'identité
se justifierait par son utilité : « Après l'intérêt qu’il y a pour un homme
à respirer librement, le plus grand de tous ses intérêts, c’est celui
qu’il y a pour lui à ne pas se contredire. » — Il y aurait à faire, sur
ces interprétations, bien des réserves. Il existe entre les exigences de
l'{\it adaptation biologique} (\S 156) et celles de la raison le même « décalage » qu’entre l’habitude et le concept (\S 184), entre la simple expérience mentale et l’expérience logique qui implique la prise de conscience des normes rationnelles de la pensée (\S 204 {\it B}). A plus forte
raison, les exigences de la rationalité sont-elles parfois aux antipodes
de la recherche de la « tranquillité » mentale (\S 177). Il n’en est pas
moins intéressant d’observer que ces exigences sont déjà comme
{\it préfigurées dans le biologique}, tout en se situant elles-mêmes sur un
plan très supérieur.

\section{Raison et société}% 215.
D'autre part, « un des principes fondamentaux de la sociologie est que l’esprit humain, avec son contenu
d’idées et de sentiments, {\it a une histoire}, qu’il est conditionné par le
temps et le milieu » (G. Davy). Aussi certains penseurs, s'inspirant
de ce point de vue \bi{sociologique}, se sont-ils efforcés de montrer que
cette histoire s’étend à la Raison elle-même. Nous avons déjà fait
allusion à la thèse de L. Lévy-BruhL selon laquelle il existerait,
dans les sociétés archaïques, une « mentalité primitive » qui serait,
non pas {\it antilogique} ni même {\it alogique}, mais {\it prélogique}, en ce sens
qu’elle obéirait moins au principe d'identité qu’à une {\it loi de participation} que l’on peut formuler ainsi : « Dans les représentations
collectives de la mentalité primitive, les objets, les êtres, les phénomènes peuvent être, d’une façon incompréhensible pour nous, à la
fois eux-mêmes et autre chose qu’eux-mêmes. » Ainsi, le primitif
croit « participer » à l’essence de son totem ; pour lui, le portrait participe au modèle, le nom à la chose ou à l’être nommé, etc. On a vu
comment, par la suite, Lévy-Bruhl a atténué cette opposition
entre pensée logique et pensée primitive et a même renoncé à la qualification de {\it prélogique}. Il avait d’ailleurs toujours admis « la coexistence congénitale, dans la nature humaine, de deux mentalités qui ne
%351
disparaissent jamais ni l’une ni l’autre ». Toutefois, même si, comme on
tend à le penser aujourd’hui, {\it primitivité} et {\it rationalité} constituent
deux structures toujours plus ou moins « présentes dans tout esprit
humain », il n’en reste pas moins : 1° que la rationalité tend à prédominer dans notre mentalité moderne ; 2° que la primitivité elle-même
repose, au fond, sur une {\it interprétation mystique de l'identité} puisqu’elle
admet « une identité fondamentale » entre tous les êtres (t. II ch. VII).
— Au reste, Durkheim avait proposé une tout autre interprétation des
\bi{origines sociales de la raison}. On a déjà vu (\S 199) comment, selon
lui, l’impersonnalité et la stabilité des concepts sont la marque de
leur origine sociale. À plus forte raison en est-il ainsi, à ses yeux, des
{\it }catégories {\scriptsize (Le principe de causalité, par exemple, se trouverait impliqué. sous une forme
encore grossière, dans les rites mimétiques (tel le rite des « faiseurs de pluie ») des primitifs sous la forme de la croyance implicite que « le semblable produit le semblable ».)} qui commandent notre pensée logique : ce sont de véritables « impératifs de la pensée » analogues aux « impératifs de la
volonté » (voir tome II, chapitre VI), des « normes extérieures et
supérieures au cours de nos représentations qu’elles dominent et
règlent impérativement ». C’est dire qu’elles ne peuvent être issues de
l'expérience individuelle. Le sociologisme maintient ainsi le principe
fondamental de l’{\it apriorisme} rationaliste selon lequel « la connaissance est formée de deux sortes d'éléments irréductibles l’un à l’autre
et comme de deux couches superposées » : les impressions subjectives
et changeantes de l'individu et un élément transcendant à ces impressions qui, pour Durkheim, vient de la société. C’est en s’inspirant de
cette conception que le psychologue Blondel est allé jusqu’à risquer
cette formule que nous avons déjà citée : « Volonté et raison sont les
deux splendides présents que la société dépose dans notre berceau. »

Nous ferons, sur cette thèse {\it sociologiste}, les mêmes réserves que
nous avons faites ou ferons sur les thèses analogues concernant la
croyance (\S 177), le concept (\S 186) ou la volonté. Ce que la société
« dépose dans notre berceau », ce ne sont pas seulement, hélas ! des
principes rationnels, mais aussi bien des préjugés, des « pré-liaisons »
comme disait Lévy-Bruhl, voire des superstitions. Il ne faut pas
oublier que la pensée collective est toujours la pensée {\it propre à un
groupe}, tandis que la raison est {\it universelle} ou du moins tend à l’universalité. — Ceci dit, il est bien vrai que, comme l’affirmait Lévy-Bruhl, « à des types sociaux différents correspondent des mentalités
différentes ». Si la société n’est pas la {\it source} des valeurs logiques,
pas plus que celle des valeurs morales (t. II, chapitre XIII), elle leur
confère certaines déterminations ({\it ibid}, ch. XIV) en rapport avec sa
%352
structure, et c’est pourquoi il y a une {\it histoire de la raison} : « Il y a eu,
écrit le sociologue Marcel Mauss, et il y a encore bien des lunes mortes
ou pâles ou obscures au firmament de la raison. Le {\it petit} et le {\it grand},
l’{\it animé} et l’{\it inanimé}, le {\it droit} et le {\it gauche}, [le {\it masculin} et le {\it féminin}]
ont été des catégories. La notion de {\it substance} a eu parmi ses prototypes, en Inde et en Grèce, la notion de nourriture ! » Même la pensée
scientifique n’a pas été l’œuvre de l'individu : il lui faut, dit G. Bachelard, « une réalité sociale, l’assentiment d’une cité physicienne et
mathématicienne », et de fait, les « catégories » mathématiques
d’abord, les « catégories » physiques ensuite, avec le langage et le
symbolisme qu’elles nécessitent, ne se sont développées que dans une
atmosphère sociale : la société des mathématiciens, la société des physiciens.

\section{La Raison vivante}% 216.
Ces considérations nous mènent à
une conception de la Raison tout autre que celle des doctrines classiques. Désiré Roustan (1873-1941) voit dans la Raison l’effort
même par lequel l’esprit, au lieu de s’adapter passivement {\it au} monde
extérieur comme le voulait l’Empirisme, s’adapte {\it le} monde extérieur
en créant lui-même l’outillage conceptuel nécessaire pour se l’assimiler.
Loin d’être, pour ainsi dire, une faculté de tout repos, « la moindre
démarche de la raison implique un risque couru » : n’est-ce pas déjà
un risque que de considérer comme {\it identiques}, ainsi qu’il a été dit
ci-dessus, des objets qui ne sont que partiellement semblables ? et le
risque n'est-il pas plus grand encore « lorsque, faisant usage du principe de causalité, nous voulons conclure du présent à l’avenir » ?
Loin d’être un système de cadres rigides posés une fois pour toutes,
la Raison, « dans son application aux problèmes sans nombre que lui
posent la nature et la vie, manifeste une souplesse singulière, une surprenante faculté d'adaptation aux circonstances, une ingéniosité
inépuisable. Aucun effort pour inventorier a priori ses richesses ne
peut nous révéler les ressources qu’elle découvrira en elle-même
quand nous la mettrons aux prises avec l’expérience ». D’autre part,
« l’histoire de toute grande idée scientifique nous montre toujours
qu’elle n’est ni conçue purement a priori, ni simplement dictée par
l'expérience ». — C'est une conception analogue qu’a développée
Léon Brunschvicg dans ses belles études sur {\it la Philosophie mathématique} et {\it la Causalité physique}. Rationalisme et Empirisme classiques
ont été conçus, remarque-t-il, à l’aide d’une logique antérieure à
l'apparition de la Science, et c’est pourquoi {\it ils ont séparé Raison et
Expérience « comme on sépare le moule qui a reçu la pâte et le gâteau
qui sort du moule »}. Tandis que le premier « rêvait d’un savoir rationnel
%353
qui se dispenserait d'interroger l'expérience », — Platon dit que la
connaissance n’est que « réminiscence » et Leibniz, on l’a vu, est bien
près d'admettre qu’elle n’est que le développement interne des virtualités propres à l’esprit, — l’Empirisme imaginait une expérience
purement passive « qui dispenserait d'exercer l’activité propre à la
pensée ». Or « le savoir humain, celui qui est l’objet de l'expérience
humaine, doit sa vérité à la connexion qui s’établit entre la rationalité et l’objectivité, On perd de vue le cours réel et l’existence même de
ce savoir lorsqu'on se préoccupe de pousser hors de soi rationalité et
objectivité, pour aboutir à isoler et à opposer la double entité d’une
raison absolue et d’un {\it objet absolu} ». Le philosophe a au contraire
pour mission « de suivre, non seulement dans leur progrès indéfini,
mais aussi dans leur intime solidarité, le double devenir de la rationalité et de l’objectivité » (Brunschvicg): On verra plus loin (chapitres XX et XXI), d’une part, que la pensée mathématique n’est
parvenue à se constituer sur le plan purement rationnel qu’en
partant de {\it bases très empiriques} et qu’elle conserve toujours le contact
avec la connaissance de l’univers, — d’autre part, que la pensée expérimentale, loin d’être un enregistrement passif, implique une continuelle {\it activité de l'esprit}, dès la constatation même des faits, puis dans
la conception de l'hypothèse, dans le raisonnement expérimental
(cf. ci-dessus, \S 202 B) et dans la création des formes mathématiques
destinées à exprimer l’expérience, enfin dans le processus dialectique
par lequel se trouvent rectifiées les erreurs antérieures et dépassées
les contradictions (t. II, \S 319). Bien mieux, c’est au contact de l’expérience scientifique et sous l'impulsion de ses enrichissements actuels
que la Raison se trouve amenée à refondre ces {\it cadres très généraux
de la pensée} que sont nos conceptions traditionnelles de l’{\it espace} et
du {\it temps} (ci-dessus, \S 98 et 153) et le déterminisme lui-même (cf. le
\S 271). En ce sens, on peut dire avec G. Bachelard que « la Science
simplifie le réel et complique la raison » et même peut-être que le
rationalisme appliqué aux divers domaines de la science se résout en
une pluralité de « rationalismes régionaux » dont les structures sont
adaptées aux différents « champs de pensée » du savoir scientifique. —
En résumé, comme l’avait déjà écrit É. Bourroux, « une éducation
constante, une formation de la Raison en vue de l'interprétation de
l'expérience, voilà ce que nous montre l’histoire de l’entendement
humain. La Raison n’est nullement demeurée immobile et identique,
comme on l’a cru. La Raison est une réalité, donc elle vit, donc elle
se nourrit de réalités, et par là même s'adapte et se développe ».
%354
\section{La Raison, faculté de l’Ordre}% 217.
Est-ce à dire que cette
évolution de la Raison s’accomplisse au hasard des circonstances, sous
des influences purement extérieures et sans unité profonde ? Non,
certes ! Car il faut distinguer, comme A. Lalande (1867-1963) l'avait
proposé dès 1909, deux formes de la Raison. La première, la \bi{Raison
constituée}, est « la raison telle qu’elle existe à un moment donné »,
telle qu’elle est, par exemple, « dans notre civilisation et à notre
époque », telle qu’elle est aussi pour le peintre ou pour le savant, pour
le physicien ou pour le biologiste (car, dans tous ces cas, elle présente
quelques variantes). Mais, par delà cette raison momentanément
cristallisée dans un « code public de règles constituées et de vérités
reconnues » et qui est « posée comme un absolu par tous ceux qui
n’ont pas acquis, à l’école des historiens ou des philosophes, l'esprit
critique nécessaire », il y a la \bi{Raison constituante} qui est « la Raison
même dans ce qu’elle a de plus essentiel » et qui est « tendance active,
personnelle, inventive », — de sorte que « la pérennité qu’on ne
trouve pas dans la statique de la raison, peut avoir place dans sa
dynamique ou, comme on dit volontiers aujourd’hui, dans sa dialectique ». Cette Raison se manifeste surtout comme un effort d’\bi{assimilation} : au degré le plus bas, assimilation des choses entre elles ;
puis assimilation des choses à l'esprit ; enfin assimilation des esprits
entre eux (d’où création d’une communauté intellectuelle et morale).
La Raison est, en effet, essentiellement \bi{normative} : elle est créatrice
de {\it normes} et constitutive de {\it valeurs} ; et la plus fondamentale de toutes
ces valeurs, celle vers laquelle semblent s'orienter toutes les « transformations convergentes » de nos principes et de nos catégories, est,
en définitive, « la supériorité du {\it Même} sur l’{\it Autre} », comme disait
Platon, la valeur de l’identité ou, tout au moins, de « la décroissance
des altérités ». — Cette conception d’A. Lalande exprime admirablement un fait que nous aurons mainte fois l’occasion de constater.
C’est qu’à travers ses multiples inventions ou créations, la Raison
vivante manifeste une tendance constante à \bi{organiser}, à \bi{unifier},
à \bi{ramener le divers à l'identique} — on a dit (\S 210) qu’expliquer,
dans la Science, c’est le plus souvent {\it identifier} — ou encore à résoudre
des contradictions. Ainsi, le « principe d’identité » exprime bien,
quoique sous une forme trop simple et trop rigide, l’exigence fondamentale de la Raison. Celle-ci nous apparaît donc, en dernière analyse,
comme étant essentiellement, en tant qu’effort d’identification et
d’unification, une \bi{faculté d'ordre}. Elle n’est rien d’autre que l’esprit
même considéré dans son {\it dynamisme unificateur}, dans son {\it effort immanent d'organisation interne}.
%355
\section{Fonction pratique de la Raison}% 218.
Il importe d’ailleurs
de préciser qu’ainsi définie, la Raison n’étend pas seulement sa juridiction à notre vie intellectuelle : elle a aussi, en tant que faculté
d’ordre, un rôle considérable à jouer dans notre vie morale : elle est
\bi{régulation de l’action} tout autant que régulation de la connaissance.
En effet, comme l'avait fortement marqué Malebranche, l'Ordre,
comme hiérarchie de « perfections », n’est pas moins nécessaire à la
vie morale qu’à la pensée, et l’on verra (t. II, ch. XIV) qu’il est
de l'essence même des valeurs d’être \bi{hiérarchiques}. Par l’un de ses
aspects au moins, la vie morale est harmonie intérieure, \bi{cohérence},
fidélité à soi-même, de sorte que rester d’{\it accord avec soi} peut être
une règle morale comme c’est aussi une règle logique. Enfin la \bi{pensée
claire} qui est un des aspects de la pensée rationnelle, est nécessaire
à l’action aussi bien qu’à la connaissance : nous verrons plus tard
que savoir ce que l’on veut, apercevoir nettement le but est une des
conditions de l’activité volontaire (voir tome II, chap. VI). La
philosophie morale ne fait que prolonger cet effort de la moralité
vécue vers la pensée claire : car, en dépit de certain irrationalisme
contemporain, les valeurs elles aussi ont besoin d’une justification
rationnelle (t. II, \S 128 {\it fin}).

\section{Les limites de la connaissance rationnelle}% 219.
Une question
peut cependant se poser ici, analogue à celle que nous nous sommes
déjà posée à propos des limites de la spéculation intellectuelle : {\it la
Raison est-elle la norme suprême ? N'existe-t-il rien au delà de la connaissance rationnelle ?} — Nous y avons déjà répondu en partie au \S 208.
Mais la réponse nous apparaît maintenant avec plus de clarté encore.
Si la Raison n’est pas nécessairement une armature rigide et fixe,
si par delà la {\it Raison constituée}, momentanément figée dans certains
cadres logiques, certains principes ou certaines catégories, il y a une
{\it Raison constituante}, beaucoup plus souple et qui est essentiellement
faculté d’ordre, nous comprenons facilement comment une invention,
par exemple, peut paraître irrationnelle aux contemporains qui s’en
tiennent aux normes de la Raison constituée sans être pour cela
irrationnelle en elle-même. C’est ainsi qu’un nombre tel que $\sqrt 2$
apparut comme « irrationnel » aux Pythagoriciens parce qu’ils étaient
habitués à considérer une portion de droite comme formée d’un
nombre déterminé de points, d’où il résultait qu’un segment rectiligne devait toujours être commensurable avec un autre. C'est de
la même façon que l'attraction newtonienne fut d’abord regardée
comme irrationnelle, parce qu’étant donné les idées qu’on se faisait
sur le mode d’existence qui appartient à la matière, on ne comprenait
%356
pas comment un corps peut agir à distance, c’est-à-dire là où il n’est
pas. Il n’y avait pourtant là, comme le dit Éd. Le Roy, qu’ « anticipation de savoir sous une règle de rationalité plus large ». — Toutefois,
deux points sont ici à préciser. 1° Répétons que la Raison ne saurait
suffire à constituer {\it avec ses seules ressources} une connaissance. Essentiellement faculté d'ordre, il lui faut, comme disait Kant, une « matière »
à organiser, et cette « matière » ne peut lui venir que de l’expérience.
Rêver d’un savoir purement rationnel qui procéderait de façon exclusivement déductive sans jamais interroger l'expérience est, comme l’a
dit Brunschvicg (\S 216), une illusion. C’est ce que, par opposition
à la physique essentiellement déductive de Descartes où l’expérience
n’intervenait que pour trancher entre les diverses conséquences
possibles d’un même principe, faisait observer Malebranche, lorsqu'il disait que « {\it la nature n'est point abstraite} » et que c’est l’expérience seule qui peut nous instruire des lois que Dieu a imposées à la
nature. Sur ce point, un mathématicien comme Henri Poincaré est
d'accord avec un physiologiste comme Claude Bernard : « L’expérience est l’unique source des connaissances humaines » (à condition,
bien entendu, qu’on entende {\it l'expérience}, non à la manière empiriste,
mais comme dirigée et organisée par la Raison). — 2° Nous n'avons
pas le droit d’affirmer {\it a priori} que tout doive se plier aux normes de
la Raison humaine : comme le dit Pascal, il y a « deux excès : exclure
la raison, n’admettre que la raison ». Il peut y avoir, sinon de l’{\it irrationnel} proprement dit — car on peut toujours soutenir, au moins à
titre de postulat indispensable à la recherche intellectuelle, que tout
irrationnel n’est qu’un irrationnel {\it provisoire} (comme il est arrivé en
effet pour beaucoup de phénomènes qui semblaient autrefois irréductibles à la connaissance rationnelle et que la Science est cependant
parvenue à expliquer) — du moins du {\it transrationnel}, c’est-à-dire
quelque chose qui {\it dépasse} la Raison humaine sans nécessairement la
{\it contredire}. Tel est précisément, selon les théologiens {\scriptsize (Il faut prendre garde au sens très spécial que les théologiens donnent au mot
{\it rationalisme} : « Erreur de ceux qui rejettent toute révélation pour s’en tenir aux seuls
enseignements directs de leur raison personnelle » (abbé Elie Blanc, {\it Dict. de Philosophie}).
En ce sens, le « rationalisme » est une des « erreurs » condamnées par le célèbre {\it Syllabus}
du pape Pie IX (1864).)}, le cas, non pas
de la Métaphysique qui relève de la Raison naturelle de l’homme {\scriptsize (Il ne faut pas confondre, comme on le fait parfois, la Métaphysique et la Religion
ou même la Théologie ; Descartes fait souvent la distinction, tant dans le {\it Discours de
la Méthode} que dans ses lettres (par ex., 27 mai 1630, 27 août 1639, etc.).)},
mais de la Foi religieuse qui exige un apport surnaturel, une « révélation », et à laquelle l'esprit humain ne saurait atteindre par ses
propres moyens. Ici, toutefois, les opinions diffèrent : tandis que les
%357
théologiens catholiques et l’Église elle-même {\scriptsize (Dans les Canons du 1$^\text{er}$ Concile du Vatican (1870), il est dit qu'il ne peut jamais y
avoir de « véritable désaccord » entre la Raison et la Foi, et même : 1° que la « droite
raison » doit pouvoir démontrer « les fondements » de la Foi, tels que l'existence de Dieu ;
2° qu’elle peut même, « éclairée par la Foi », obtenir « quelque intelligence des mystères »,
c'est à dire de l'objet propre de celle-ci (ci-dessous, p.626 n.1).)}, continuant la tradition
de saint Thomas d'Aquin, se sont toujours résolument refusés à
opposer la Raison et la Foi, certains théologiens réformés, tel que le
théologien contemporain Karl Barth, ont affirmé l’irrationalité
foncière de la Foi, Dieu étant, à leurs yeux, le « totalement Autre »,
l'« Inconnu », dont l’homme ne peut rien dire qu’en corrigeant immédiatement son affirmation par la négation opposée (théologie « dialectique ») : ainsi, l’homme découvre le Créateur {\it dans ses œuvres}, « les
cieux, dit l’Écriture, racontent la gloire de Dieu », mais la même
Écriture nous dit aussi que Dieu est un Dieu {\it caché} ; de même, l’expérience religieuse est celle de la {\it grâce}, mais c’est aussi celle du {\it péché}, etc.
Cette conception a été étendue par le philosophe allemand Karl
Jaspers à la philosophie elle-même : celle-ci n’est pas, selon Jaspers,
une {\it connaissance} ; la philosophie ne peut qu’éveiller l'individu en
l'invitant à rompre avec la raison conceptuelle s’il veut essayer de
saisir la seule réalité, à savoir ce qu'il y a d’unique dans chaque
existence individuelle {\scriptsize (Cette position a été très vivement critiquée par les théologiens catholiques : voir
le livre du P. J. Hamer sur {\it Karl Barth}, Desclée-de Brouwer, 1949, chap. XIII, et
celui du P. J. de Tonquédec  sur {\it L'Existence d'après K. Jaspers}, Beauchesne, 1945, chap. final.)}. Plus récemment (1950), le même auteur a
cependant fait une place à la Raison dans sa philosophie, mais en
continuant à l’opposer à l’entendement et en la définissant comme
une aspiration, d’ailleurs toujours insatisfaite, à la vérité et à la
« communication » entre les hommes : « La raison apparaît comme
l’esquisse de l'humanité espérée, autant qu’il dépend de nous de la
produire. » Mais n’est-ce pas encore définir la Raison par un critère
bien extérieur, la communication des consciences pouvant se réaliser
dans l’irrationnel aussi bien que sur le plan de la pensée rationnelle
(tome II, chap. IX)?

\section{Sujets de travaux}

{\footnotesize
{\bf Exercices.} — 1. {\it Commenter cette définition de} Maurice Blondel. : « Raison
selon son étymologie, c’est : compte expressément examiné et vérifié, calcul
explicatif et confirmé, rattachement à des principes et justification des
faits selon des rapports intelligibles ou des lois contingentes, aperception
de vérités claires ou de relations normales » (voir tout le passage dans La
Pensée, Alcan, 1934, t. I, p. 370-377). — 2. {\it Qu'y a-t-il de commun entre les
divers emplois du mot raison dans les expressions} : « livre de raison », « en
%358
moyenne et extrême raison », « en raison inverse de... », « raison d’une progression (arithmétique ou géométrique) », « raison d'État », « mariage de
raison », {\it et dans les citations suivantes} : « La puissance de bien juger et distinguer le vrai d'avec le faux qui est proprement ce qu’on nomme le bon sens
ou la raison, est naturellement égale en tous les hommes » (Descartes), « La
raison n’y peut rien déterminer [dans l'existence de Dieu]» (Pascal),
« Le cœur a ses raisons » (Ip.), « Et le raisonnement en bannit la raison »
(Molière),« Le comte est donc si vain et si peu raisonnable ? » (P.Corneille),
« La raison du plus fort est toujours la meilleure » (La Fontaine), « Il fait
un froid contre toute raison » (Sévigné), « La raison, en tant qu’elle nous
détourne du péché, s’appelle la conscience » (Bossuet), « Je hais les raisons
quand je veux quelque chose » (Th. Corneille), « Où est-elle, cette Raison
suprême ? N'est-elle pas le Dieu que je cherche? » (Fénelon), « La raison
finira par avoir raison » (D'Alembert), « Un homme doué à mesure égale
de jugement et d'imagination est un être de raison » (Diderot), « Le caractère de la raison le plus marqué, c’est le doute, c’est la délibération » (Buffon),
« On met ainsi en évidence non seulement la loi [d’un phénomène], mais la
raison de la loi : la raison objective est trouvée, la raison subjective est satisfaite » (Cournot), « L'honnête homme juge en sa propre cause comme en
celle d'autrui. Cela s'appelle étre raisonnable» (Rauh), « Il ne suffit pas
d'avoir raison contre l'erreur, il s'agit d'en avoir raison » (Bernanos) ? —
3. {\it Comment comprenez-vous ces définitions} de Lagneau : « Nous appelons raison le pouvoir de sortir de soi en affirmant une loi supérieure dont l’homme
trouve en lui l’idée et, en dehors, le reflet seulement... La raison s'achève
dans l’acte par lequel elle se détache d'elle-même en se comprenant... La raison, c'est la pensée en tant qu’elle pose sa propre nature ». — 4. {\it Commenter
cette boutade de} H. Poincaré : « C’est notre esprit qui fournit une catégorie à
la nature. Mais cette catégorie n’est pas un lit de Procuste dans lequel nous
contraignons violemment la nature en la mutilant selon que l’exigent nos
besoins. Nous offrons à la nature un choix de lits parmi lesquels nous
choisissons la couche qui va le mieux à sa taille. »

{\bf Exposés oraux.} — 1. {\it L'évolution du rationalisme}, d'après L. Brunschvicg
et D. Roustan. — 2. {\it La raison normative}, d'après A. Lalande, surtout
chap. IV, VI-VII et IX. — 3. {\it Le transrationalisme}, d'après Cournot.

{\bf Discussions.} — 1. {\it Raison et expérience}. — 2. {\it Discuter ce jugement} : « Cette
raison universelle, ce déterminisme, ces catégories qui expliquent tout, ont
de quoi faire rire l'homme honnête » (Albert Camus).

{\bf Lectures.} — {\it a.} A. Cournot, {\it Matérialisme, vitalisme, rationalisme}, Hachette,
1875, 4$^\text{e}$ section. — {\it b.} Émile Meyerson, {\it Identité et réalité}, Alcan, 1907.—
— {\it c.} E. Goblot, {\it Traité de Logique}, chap. XVIII, \S II et {\it d. Le Système des
sciences}, chap. XX, A. Colin, 1918 et 1922. — {\it e.} L. Brunschvicg, {\it L'Orientation du rationalisme}, dans la R. M. M., juill.-sept. 1920, et {\it f. Héritage
de mots, héritage d'idées}, P. U. F., 1945, chap. I et app. II. — {\it g.} D. Roustan, {\it L'Évolution du rationalisme, dans La Raison et la vie}, P.U.F., 1946, I.
— {\it h.} D. Essertier, {\it Les Formes élémentaires de l'explication}, Alcan, 1927,
chap. V. — {\it i.} G. Davy, {\it La Psychologie des primitifs} d'après Lévy-Bruhl,
dans {\it Sociologues d'hier et d'aujourd'hui}, Alcan, 1931. — {\it j.} A. Lalande,
{\it La Raison et les normes}, Hachette, 1948. — {\it k.} G. Bachelard, {\it Le Rationalisme
appliqué}, P. U. F., 1949. — {\it l.} É. Durkneim, {\it Pragmatisme et Sociologie},
Vrin, 1955. — {\it m.} G.-G. GRANGER, {\it La Raison}, P. U. F., 1956.
 }
%359
