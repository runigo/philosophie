
\chapter{L'espace}
% chapitre VI
%L'ESPACE
%89. La perception de l'étendue : différents problèmes. — 90. Le problème
%psychologique. — 91. « L'espace perceptif ». — 92. La construction de
%l'espace. — 93. Espace, mouvement et réalité. — 94. La perception de
%la troisième dimension. — 95. L'appréciation de la distance. — 96. Erreurs
%et troubles de la perception de l’espace. — 97. Le concept d'espace. — 98.
%Le problème épistémologique. — 99- Le problème métaphysique.

\section{La perception de l'étendue : différents problèmes}% 89.
La perception sensible présente encore un autre caractère, — d'ailleurs
difficilement isolable, on le verra bientôt, de celui que nous étudierons
dans le chapitre suivant : le caractère \textbf{\textit {spatial.}} Les objets que nous
percevons nous paraissent {\it }étendus : ils présentent des plans, des
surfaces. Ils nous paraissent en outre se situer à une certaine {\it }distance,
offrir un certain {\it relief}, constituer des {\it volumes}. En un mot, nous percevons
les choses dans un {\it espace à trois dimensions}, comme douées de
longueur, largeur et profondeur. — Différents problèmes vont se
proposer ici à notre réflexion. Ce sera d’abord un problème psychologique :
ce caractère {\it spatial} est-il inhérent à la sensation, en tant
que donnée immédiate de la sensibilité ? ou bien est-il acquis, se forme-t-il
peu à peu ? Le problème \textbf{\textit {épistémologique}} consistera ici à s'interroger
sur la valeur et la portée de notre {\it idée}, de notre concept de l'espace,
question qui intéresse au premier chef la Géométrie, mais aussi toute
notre connaissance du monde extérieur. Le problème \textbf{\textit {ontologique}} ou
\textbf{\textit {métaphysique}} portera sur l’essence ou la réalité absolue de l’espace.

\section{Le problème psychologique}% 90.
Le problème psychologique
paraît ici plus aisé à poser qu’à propos de la notion d'objet. On va
voir pourtant que sa position même soulève des difficultés analogues.
Parmi les doctrines classiques, deux s'étaient opposées sur ce point,
dont l’une, le \textbf{\textit {nativisme,}} soutenait que la spatialité est une {\it donnée
%116
immédiate} de la connaissance sensible, que les impressions sensorielles
font naître, sans aucune expérience antérieure du sujet, des sensations
d’étendue, de formes, de distance, de profondeur, etc.{\scriptsize (Le
{\it nativisme}, qui est à peu près le point de vue du sens commun, ne doit pas
être confondu avec l'{\it innéisme}. Il soutient en effet que l'étendue (concrète)
est donnée {\it dans} l'expérience sensible elle-même, tandis que l’{\it innéisme},
celui de Kant par exemple (voir \S 97) admet que l’espace abstrait est une
forme {\it a priori}, c'est-à-dire {\it antérieure}, au moins en droit, à toute expérience
possible)}, —
tandis que l’autre, la théorie génétique (improprement appelée parfois
{\it empiriste}), regardait la perception de l’étendue (et, à plus forte
raison, le concept, l’idée abstraite d’espace) comme progressivement
acquise au cours de toute une éducation. Cette seconde théorie partait
de l’idée que la sensation est en elle-même quelque chose de
purement subjectif, donc de qualitatif et qui, par suite, ne peut
avoir primitivement quoi que ce soit de spatial. Le problème, à peu
près insoluble, était alors d'expliquer comment avec du non-spatial
l'esprit peut construire du spatial. On s’en tirait en ramenant la notion
d’espace, soit à celle de {\it coexistence} (Spencer), soit à celle d’{\it hétérogénéité
qualitative} (théorie des signes locaux de Lotze et Wundt{\scriptsize (Une
sensation tactile, disait-on, présente toujours, outre l'élément commun, une
différence qualitative qui varie selon les points du corps touchés. Ce sont ces
{\it signes locaux} qui nous permettent de localiser la sensation. Mais, pour
localiser une sensation, ne faut-il pas déjà avoir la notion d'espace? D'autre
part, eette théorie ne tenait pas compte de la {\it mobilité} des organes : la main
se déplaçant, n'importe lequel de ses points peut être impressionné par
n'importe quel point de l’espace. Enfin elle n'expliquait pas
la perception de la {\it distance})}).
Mais cette réduction était injustifiée. En outre, le point de départ était
faux ; on verra(\S 101 B) que la subjectivité pure n’est pas une donnée,
mais, tout comme l’objectivité, le produit de toute une évolution
psychique. — Le point de vue {\it nativiste} semble donc plus proche de
la vérité. De fait, gestaltistes, phénoménologues, existentialistes
reviennent aujourd’hui à un point de vue fort analogue. Au fond, nous
avons déjà touché à ce problème de l’espace quand nous avons parlé
de la perception des « formes » : « Une forme géométrique n’est pas
seulement une qualité originale : c’est un système de relations entre
des points, lignes, surfaces qui le constituent » (Guillaume). Or on
sait que, selon le Gestaltisme, la perception de ces « formes » est
immédiate. — De même, M. Merleau-Ponty nous dit : « Il est
essentiel à l’espace d’être toujours “ déjà constitué ” et nous ne le
comprendrons jamais en nous retirant dans une perception sans monde.
Il ne faut pas se demander pourquoi l’existence est spatiale. L’expérience
perceptive nous montre au contraire que [les faits qui constituent
la spatialité] sont présupposés dans notre rencontre primordiale avec
l’être et que l'être est synonyme d’être situé. » La profondeur, en
%117
particulier, qui est « la plus existentielle de toutes les dimensions »,
est « vécue hors de toute géométrie ». En ce sens, « il n’y a pas de problème
de la distance ». Nous avions déjà remarqué, à propos de la
notion d’objet, que les théories de « l’expérience immédiate » tendent
à supprimer les problèmes.

Mais ici aussi il y a lieu de faire la part de ce qui est valable dans
cette conception et de ce qui ne peut être admis. Ce qui est vrai, c’est
que la perception de l’espace est inséparable de la perception des
choses et qu’il n’existe pas de « perception sans monde »; c’est aussi
que, dans la mesure où la perception s’objective, elle vient s’intégrer,
comme le dit J. Piaget, à « un système spatio-temporel et causal »
(cf. \S107) ; c’est enfin que ceci ne peut avoir lieu qu’à condition que,
— de même que la conscience est spontanément tournée vers l’objet,
— la perception présente dès l’origine quelque structuration d’où
l'espace ne soit pas absent. — Mais nous retrouvons ici une équivoque
exactement parallèle à celle que nous avons rencontrée dans les
mêmes théories à propos de la notion d’objet. Cette structuration de
« l’espace perceptif » est bien loin encore d’être celle de l’espace {\it représenté}
et, à plus forte raison, celle de l’espace {\it géométrique}, tel que
l'espace euclidien. La philosophie de « l'expérience immédiate » paraît
bien être ici dupe d’une illusion qu’on pourrait appeler l’{\it illusion de
l'immédiat} (ci-dessus, \S 58-59) : « L’adulte, ayant perdu tout souvenir
des étapes antérieures à une telle transformation, s’imagine alors que
chaque perception utilise dès l'origine les systèmes de coordonnées ou
les rapports de verticalité et d’horizontalité, en réalité très complexes,
qui sont achevés seulement entre 8 et 9 ans » (J. Piaget). Selon
Piaget, les faits correctement observés « démontrent au contraire de
la manière la plus nette combien il est illusoire d’attribuer au sujet
humain la connaissance innée ou psychologiquement précoce d’un
espace d’emblée structuré selon un système de coordonnées rectangulaires
à deux ou trois dimensions » et, par suite, « il est évident que
la perception de l’espace comporte une construction progressive et
n’est pas donnée toute faite dès les débuts de l’évolution mentale ».

\section{« L’espace perceptif »}% 91.
Pour mieux voir comment s’effectue
cette {\it construction} de l’espace, demandons-nous en quoi peut
consister « l’espace perceptif » primitif en le distinguant soigneusement
de l’espace représenté. Certes il existe bien pour le tout jeune
enfant un milieu structuré lié à son activité, mais celui-ci demeure
« entièrement centré sur son activité propre » : c’est, selon l’expression
de Piaget, un « {\it espace pratique et égocentrique} » qui est encore
bien loin de la représentation d’un milieu ordonné comprenant le
%118
sujet lui-même à titre d’élément. Le physicien Mach (1838-1916)
avait déjà remarqué autrefois qu’il existe un « espace physiologique »
lié à nos mouvements et où l’on pourrait, à la rigueur, distinguer
trois dimensions « pratiques » : l’{\it avant} et l'{\it arrière}, le {\it haut} et le {\it bas}, la
{\it droite} et la {\it gauche}. Mais ces dimensions n’ont primitivement qu’un
caractère tout \textbf{\textit {subjectif}} et même affectif : l'{\it avant}, c’est la direction
de la tendance, du désir, celle de l’élan animal de l’être qui fond sur
sa proie ; le {\it bas}, c’est la direction dans laquelle nous nous sentons
attirés par la pesanteur ; quant à la distinction de la {\it droite} et de la
{\it gauche}, on verra plus loin qu’elle n’est nullement immédiate et qu’elle
est peut-être d’origine {\it sociale} plutôt que physiologique. Dans le même
ordre d’idées, W. James avait fait observer que certaines de nos
sensations au moins, peut-être toutes, présentent un caractère de
\textbf{\textit {voluminosité}} ({\it voluminousness}) : « Les grondements du tonnerre ont un
autre volume de sonorité que le grincement d’un crayon sur une
ardoise. Quand on entre dans un bain chaud, on éprouve une sensation
autrement {\it épaisse} que si une épingle vous égratigne. Une légère
douleur névralgique au visage paraît moins profonde que la douleur
pesante d’un furoncle ou la souffrance massive d’un lumbago. » Enfin
on s’accorde à reconnaître que les sensations de la vue, peut-être
aussi celles du toucher, ont une certaine \textbf{\textit {extensivité}} ({\it extensity}) : la
perception de la couleur est inséparable (autrement que par abstraction)
de celle d’une certaine {\it étendue} colorée ; même un point brillant
se détache sur un {\it fond} sombre ; quant au toucher, il est bien vain de
se demander, avec Maine de Biran, ce qu’il nous donnerait si son
organe était réduit à « un ongle très aigu » : en fait, c’est toujours une
certaine {\it surface}, si petite soit-elle, de notre peau qui est intéressée.
Mais cette voluminosité et cette extensivité ne sont encore que des
\textbf{\textit {qualités}} qui font corps avec la sensation elle-même : elles sont encore
bien loin de se projeter sous la forme d’un « espace objectif » dans un
milieu intellectuellement structuré et comportant des figures nettement
déterminées.

\section{La construction de l’espace}% 92.
J. Piaget a cru pouvoir
distinguer, chez l’enfant, trois périodes de la constitution du
« champ spatial » correspondant aux six stades de la constitution de
l’objet (\S 103). Chez le nouveau-né, ce qui est donné, c’est simplement
le champ des réflexes et des premières habitudes, c’est aussi « la
perception de la lumière et l’accommodation propre à cette perception ».
Au point de départ, « {\it l’espace n’est donc nullement la perception
d’un contenant, mais bien celle des contenus, c’est-à-dire des corps
eux-mêmes} ». C’est pourquoi \textbf{\textit {« la notion d'espace ne se comprend
%119
qu’en fonction de la construction des objets »}} (Piaget). Dans la première
période, qui correspond aux deux premiers stades de cette
construction, l’espace consiste en « groupes » purement {\it pratiques}, en
ce sens qu’ils demeurent liés à l’action, et {\it hétérogènes}, en ce sens
qu’il n'existe encore aucun lien entre les divers espaces sensoriels, entre
l’{\it espace buccal} qui est, selon Stern {\scriptsize (L. William Stern,
psychologue allemand (1871-1938))}, le premier de ces schèmes constitutifs
de l’espace enfantin (succion), et l’{\it espace tactile} (préhension), ni
entre celui-ci et l’{\it espace visuel} (l'enfant ne sait pas encore saisir un objet
qu’il voit). On ne peut guère supposer non plus, dans cette première
période, une véritable notion de la distance. De ce point de vue, l’espace
entier est pour l’enfant « analogue à l’espace céleste de la perception
immédiate : une masse fluide sans profondeur (bien que l’œil s’accommode
aux diverses distances), parcourue par des images qui s’entrepénètrent
ou se détachent sans lois et se déforment ou se reforment
alternativement ». Par exemple, dans un objet appuyé contre une
paroi, un vêtement pendu à un porte-manteau, le bébé ne voit qu’une
tache qui ressort à peine sur le fond. Une seconde période comprend
les troisième et quatrième stades de l’évolution sensori-motrice. Elle
est caractérisée, d’abord par la {\it coordination de la vision et de la préhension}
(d’où la constitution de « groupes », encore relatifs cependant à
la perspective propre de l’enfant : il apprend à saisir, donc à localiser
les objets {\it par rapport à lui}), puis, au quatrième stade, par la {\it coordination
des actions entre elles} (d’où la constitution du « groupe des
opérations réversibles » : l’enfant devient capable de cacher et de
retrouver, ce qui est un progrès vers l’espace objectif, mais sans qu’il
ait encore une notion nette de la position relative des objets {\it entre eux}).
La troisième période (à partir de l’âge d’un an) est caractérisée au
contraire par l’apparition de cette notion et la constitution de ce
qu’on peut appeler « {\it l’espace expérimental} » : au cinquième stade,
tout ce qui est perçu s'organise dans un champ commun et l’enfant
prend conscience de ses propres déplacements qu’il situe ainsi par
rapport à ceux des objets ; au sixième, l’enfant parvient au « groupe
représentatif » où l’espace est constitué comme un milieu immobile,
dans lequel il se situe lui-même.

\section{Espace, mouvement et réalité}% 93.
On voit donc ce qu’on
peut retenir des théories de l'expérience immédiate : gestaltisme ou
phénoménologie, et ce en quoi elles sont insuffisantes. — {\it A.} Le
Gestaltisme a bien vu {\it l'union étroite de la perception spatiale et du
mouvement} : « Dans une telle doctrine, écrit P. Guillaume, la sensibilité
%120
et la motricité sont étroitement unies. C’est dans la structure
même de l’excitation perceptive et de l’excitation motrice qu’il faut
chercher l’explication de leurs rapports. » C'est bien ce que vient de
nous montrer l’analyse précédente : espace perceptif et espace représenté,
malgré le « décalage » qu’exige le passage de l’une à l’autre, ont
cependant, selon Piaget, « un facteur commun », et ce facteur commun
qui présente ainsi « une importance essentielle pour l’interprétation
de l'intuition spatiale », c’est la \textbf{\textit {motricité,}} source des opérations
supérieures, notamment par l'intermédiaire du {\it dessin} (voir
{\it Exercice} 2), aussi bien que des « perceptions spatiales les plus élémentaires ».
— Mais le Gestaltisme a entendu cette union dans le
sens d’une détermination du mouvement par la perception des formes.
L’intuition de l’espace géométrique apparaît ainsi comme devant
être essentiellement de nature {\it contemplative, imagée}. L’analyse génétique
de Piaget établit au contraire qu’elle est de nature \textbf{\textit {constructive;}}
car c’est {\it l'élément moteur qui dirige l'élément figuratif ou imagé, et
non l'inverse}. « L’image visuelle d’une forme plane ou d’un volume en
perspective, d’une projection ou d’une section, d’un rabattement, etc.,
englobe bien plus de mouvements du sujet qu’on ne le croit généralement
et se trouve être beaucoup plus l’image d’une action possible
relative à ces formes qu’une action visuelle simple, »

{\it B.} D’autre part, il est parfaitement juste de dire avec M. Merleau-Ponty
que « l’espace n’est pas le milieu dans lequel se disposent les
choses, mais le moyen par lequel la position des choses devient possible »,
qu’il doit donc être pensé « comme la puissance universelle
de leurs connexions ». Toute l’analyse précédente nous a montré que
la construction de l’espace est \textbf{\textit {solidaire de la constitution des choses,}}
et c’est aussi ce que démontrent les deux expériences célèbres de
Stratton et de Wertheimer.

\vspace{0.24cm}
{\footnotesize {\it Expérience de Stratton.} — On sait que l’image des objets
se peint, sur la
rétine, renversée comme dans un appareil photographique, ce qui ne nous
empêche pas de les voir en position normale {\scriptsize (On s’est posé
d'ailleurs à ce sujet un problème parfaitement vain, celui de savoir comment
l'esprit (ou le cerveau) « redresse » l’image rétinienne. Mais « l’image
rétinienne n’a pas d'existence psychologique propre et l'image mentale n’est
pas son simple décalque » (H. Wallon). L'erreur est ici de s’imaginer une sorte
d’« œil interne » (esprit ou cerveau ?) qui regarderait l’image rétinienne
inversée comme notre œil regarde les objets, ce qui est pure mythologie :
car l'image rétinienne n’est pas « perçue ». Bien mieux, on peut dire qu’à
proprement parler elle n’existe pas. La rétine est constituée par une
mosaïque discontinue de cellules nerveuses dont chacune transmet au cerveau
l’impression reçue. Si cette pluralité d’impressions est interprétée par nous
comme une image continue et orthoscopique des choses, n'est-ce pas, ainsi que
le suggère encore H. Wallon, « en raison de leur maniement » qui, lui, ignore
les lois optiques ?)}. « Si l’on fait porter à un sujet des lunettes qui
redressent les images rétiniennes, le paysage entier paraît
%121
d’abord {\it irréel} et renversé ; au second jour de l’expérience, la perception
normale commence de se rétablir, à ceci près que le sujet a le sentiment que
son propre corps est renversé. Au cours d’une seconde série d’expériences
qui dure huit jours, les objets apparaissent d’abord renversés, mais moins
irréels que la première fois. Le second jour, le paysage n’est plus renversé,
c’est le corps qui est senti en position anormale. Du troisième au septième
jour, le corps se redresse progressivement et paraît être enfin en position
normale, {\it surtout quand le sujet est actif...} Les objets extérieurs ont de plus
en plus l’aspect de la « réalité ». Dès le cinquième jour, les gestes qui se
laissaient d’abord tromper par le nouveau mode de vision, vont sans erreur
à leur but... A la fin de l’expérience, quand on retire les lunettes, les objets
paraissent non pas renversés, mais « bizarres » et les réactions motrices
sont inversées : le sujet tend la main droite quand il faudrait tendre la
gauche {\scriptsize (Les expériences de G. M. Stratton (1896) ont été reprises
par P.H. Ewert (vers 1925) qui a montré la possibilité du rétablissement de la
vision droite au bout de 14 jours, avec mise en évidence du rôle de la loi de
l'effet selon laquelle toute réponse est renforcée par le succès, affaiblie
au contraire ou éliminée par l'échec.)} » (Merleau-Ponty).

{\it Expérience de Wertheimer.} — De même, « si l’on s'arrange pour qu’un sujet
ne voie la chambre où il se trouve que par l'intermédiaire d’un miroir
qui la reflète en l’inclinant de 45° par rapport à la verticale, le sujet voit
d’abord la chambre “ oblique ”. Un homme qui s’y déplace semble marcher
incliné sur le côté. Un morceau de carton qui tombe le long du chambranle
paraît tomber selon une direction oblique. L'ensemble est “ étrange ”.
Après quelques minutes, un changement brusque intervient : les murs,
l’homme qui se déplace dans la pièce, la direction de chute du carton
deviennent verticaux » (Id.).}
\vspace{0.31cm}

La première de ces expériences nous montre que « renverser un
objet, c’est lui ôter sa signification » (comme il est manifeste quand on
regarde un visage humain à l’envers), et toutes deux, que le sentiment
de {\it réalité} et la perception de l’{\it espace} sont intimement liés : dès que
cette dernière est perturbée, les objets paraissent irréels, étranges,
bizarres. L'expérience de l’espace, c’est notre expérience vécue du
monde où nous existons. Ce n’est pas une raison toutefois pour faire
de notre représentation de l’espace quelque chose d’{\it inanalysable.}
Nous avons vu, au contraire, que notre représentation de l’espace
\textbf{\textit {se construit}} peu à peu, non certes à partir d’une perception totalement
a-spatiale comme l’avait admis l’ancienne théorie génétique
(\S 90), mais à partir d’une spatialité implicite et encore très imparfaitement
structurée (\S 91-92), exactement comme notre représentation
du réel se construit, non certes à partir d’une « perception
sans monde », mais, on le verra (\S 102-103), sans prise de conscience
de l’objet comme tel. Dans cette construction de l’espace, interviennent :
1° une intelligence \textbf{\textit {active,}} et 2° la \textbf{\textit {motricité}} de notre
corps. 1° M. Merleau-Ponty reconnaît que, « si les déformations
perspectives nous étaient expressément données, nous
%122
n’aurions pas à apprendre la perspective ». Et en effet l'observation
la plus simple nous montre que notre représentation de la perspective
est toute farcie de conventions
qui d’ailleurs ont varié au cours des siècles et dans les différentes
civilisations. 
%(fig. 20)
N'est-ce pas avouer que notre représentation
géométrique de l’espace n’est pas toute donnée dans la
perception et qu’il s’y superpose une activité propre de l’intelligence ?
2° Enfin il ne suffit pas, pour exclure toute synthèse
des éléments sensibles et des éléments moteurs, de remarquer
que, dans l’expérience de Wertheimer, la rectification s’effectue
d’elle-même « sans aucune exploration motrice ». Croit-on
que notre expérience motrice antérieure n’y soit pour rien ? De
même, si dans l’expérience de
Stratton le sujet se trouve d’abord désemparé, n’est-ce pas que toutes
les coordinations sensitivo-motrices auxquelles il est habitué, se
trouvent perturbées ? Nous allons précisément constater le rôle
capital de ces éléments moteurs dans la perception de la troisième
dimension.
%Fig 20. — Les conventions de la prespective.
%(D'après Maspzno, L'Archéologie égyptienne, Grund et Maguet, éd.)

\section{La perception de la troisième dimension}% 94.
Par « troisième
dimension », il faut entendre la {\it distance}, la {\it profondeur}, et aussi
le {\it relief} qui est la distance, par rapport à nos sens, des différentes
parties de l’objet perçu. Les mêmes difficultés se sont présentées ici
qu’à propos de la perception spatiale en général. On a souvent tenté
en effet d’expliquer la perception de la troisième dimension comme
si elle procédait des deux autres: on s’imaginait que « l’image rétinienne »
ou l’image perçue, primitivement sans distance ni profondeur,
est ensuite {\it projetée à distance} grâce à différents facteurs dont
il sera question plus loin. Or on va voir qu’il faut totalement renoncer
à cette position de la question : la troisième dimension fait partie
intégrante de notre expérience totale de l’espace. Mais ce n’est pas
%123
une raison pour déclarer qu’il « n’y a pas de problème de la distance »
et pour nier ici toute évolution {\it génétique} et toute {\it activité} propre de
l’esprit : si l’objet est d’emblée perçu à distance, la représentation
correcte de la distance est le fruit de tout un apprentissage. Ici encore
il faut, avec J. Piaget, distinguer entre « le comportement et la
conscience ». L'enfant « accommode ses yeux et ses mains à la distance »
bien avant de savoir se {\it représenter} la différence entre objet proche
et objet éloigné comme une différence objective de profondeur.

{\it A.} L'expérience du toucher \textbf{\textit {moteur}}, l'expérience \textbf{\textit {kinésique}} est
inséparable, nous l'avons déja remarqué (\S 91), de notre expérienee
d’un espace à trois dimensions.

\vspace{0.24cm}
{\footnotesize « Vers les couleurs vives, vers les objets aux contours arrêtés,
l'enfant
tend les bras. Non qu'il ait déjà le sens de la direction et de l'éloignement :
les mouvements de ses bras et de son torse continuent simplement a s'acorder
avec ceux de ses yeux » (Th. Ruyssen). On peut rappeler ici les observations
du psychologue américain J.-M. Baldwin (1861-1934) sur sa fille
âgée de 9 mois. Il avait disposé, à différentes distances de l'enfant, des
papiers de couleurs variées capables d'attirer son attention. En pareil cas,
« l'enfant tient compte avec une régularité absolue des objets qui sont,
comme on dit, {\it à bout de bras} : avec plus d’hésitation déjà, il s'intéresse
aux objets situés à une distance qu'il peut, avant de savoir marcher,
atteindre avec extension du bras et flexion du tronc. Au delà, les objets
lui sont indifférents, bien qu’il les voie. Leur recul est un peu pour lui ce
qu'est pour l’adulte celui des étoiles : un {\it au delà} pratiquement inaccessible
et inappréciable pour les sens » (cf. \S 92). \textsf{\textit {Le « monde en profondeur » est donc
tout simplement celui de l'expérience motrice.}} Il y a pour l'enfant — et ceci
restera encore vrai, en grande partie, pour l'adulte — un « espace proche »
et un « espace lointain » (Piaget), le premier étant « le domaine des objets
à saisir », le second celui où nos gestes ne se sont pas encore risqués et où
« les distances relatives demeurent indiscernables ». — Plus tard même, la
notion que se fait l’enfant des distances, est fonction de ses {\it déplacements} :
« À 4 ou 5 ans, n'ayant jamais voyagé, je ne pus malgré tous mes efforts,
quand on me décrivit comme infinie la mer où se jetait la rivière de ma ville
natale, me la représenter que comme une pièce d’eau carrée de 4 ou 5 mètres
de côté. Et pourtant la riviére était bien large d’une cinquantaine de
mètres (Ruyssen). Anatole France enfant avait « acquis la certitude
que la Chine était derrière l'Arc-de-Trigmphe », Même chez l’adulte, il est
facile de voir que la {\it représentation} de la distance varie avec le genre de vie :
le paysan qui n’est jamais sorti de son village, imagine la France avec
l'étendue d’un canton.

On a souvent remarqué que nous évaluons les \textsf{\textit {distances verticales}} beaucoup
plus inexactement encore que les distances horizontales : c'est que, dans le
premier cas, nous manquons d'expérience motrice (indépendamment d'autres
raisons qu’on verra plus loin). C’est déjà vrai de l'enfant :
« Preyer {\scriptsize (W. Preyer, psychologue allemand (1841-1897), auteur
d’un livre célèbre sur l'Âme de l'enfant, 1881, trad. fr. 1887.)},
par la fenêtre d’un second étage, lance un papier à un enfant de près de
%124
deux ans. Celui-ci ramasse l’objet et le lui tend avec le désir visible de
recommencer ce jeu. » Que de bévues ne commettons-nous pas nous-mêmes sur
l’appréciation de la hauteur d'un phare, d’une tour, d’un clocher et surtout
d’un avion dans le ciel ! Mais on avait déjà observé jadis que « les marins,
habitués à grimper dans les cordages, ont l'œil autrement juste », et aujourd’hui
l’aviation a ajouté à notre expérience de l’espace la dimension verticale.
— Il est à noter cependant que, dans cette expérience, il entre une
certaine \textsf{\textit {activité de l'intelligence}}. Un garçonnet de 7 ans observé par Franz
(vers 1840) et « qui n’avait aucun défaut de la vue, mais dont l'intelligence
était faible », était incapable d’estimer la distance verticale des objets :
« il tendait fréquemment la main vers un clou du plafond ou vers la lune ».}
\vspace{0.31cm}


{\it B.} La perception de la distance \textbf{\textit {par la vue}} a été l’objet de nombreuses
discussions. Certains auteurs ont prétendu que la vue ne
nous apporterait par elle-même {\it aucune expérience de la distance} et nous
fournirait seulement {\it des images sans relief}. Tout au plus, certains
éléments, surtout moteurs, du processus de la vision nous permettraient-ils
un {\it apprentissage progressif} de la vision à distance. Mais on
verra que les arguments de base sont {\it sans valeur} et que les éléments
en question ne pourraient nous donner cette perception à distance
{\it si nous ne la possédions déjà}.

\vspace{0.24cm}
{\footnotesize 1° Les arguments invoqués étaient : $\alpha$. {\it l'argument de Berkeley}
{\scriptsize (George Berkeley, évêque anglican et philosophe irlandais,
1685-1753. Sur sa
doctrine métaphysique, voir chap. VIII, \S 114.)} : la distance
d’un point quelconque à l'œil est une droite qui, quelle que soit sa
longueur, se projette toujours sur la rétine sous la forme d’un point ; —
$\beta$. {\it l'observation des aveugles-nés} opérés qui, tel l’aveugle de Cheselden (\S 84),
déclarent parfois que les objets leur paraissent « toucher » leurs yeux. —
2° La vision à distance se constituerait grâce à : $\alpha$. {\it l’accommodation} qui est,
on le sait, la mise au point de l'œil pour les objets situés à moins de
quelques mètres ; — $\beta$. {\it la convergence des yeux} sur l'objet, réalisée par les muscles
externes de l'œil et qui est évidemment plus grande pour les objets proches
que pour les objets lointains ; — $\gamma$. {\it la disparité des deux impressions rétiniennes}.

{\it Discussion}. — 1° {\it L'argument de Berkeley} ne tient pas compte de la
complexité de la vision, qu’il schématise à l'excès, et l’isole de l’ensemble
de notre expérience perceptive. L'observation des {\it aveugles devenus voyants}
n'est pas parfaitement concluante. Car, d’une part, le toucher ayant été
jusque-là, pour l’aveugle, le type de la perception immédiate, il se peut
que l'expression « toucher les yeux » signifie seulement que les objets
viennent maintenant impressionner ses sens sans qu'il ait besoin de se
déplacer. D'autre part, la perception visuelle du nouveau voyant ne s’effectue
pas dans les conditions normales (on n’opère pas les deux yeux à la fois ;
l’accommodation ne se fait pas, par suite de l’ablation du cristallin et
surtout elle intervient brutalement {\it dans un monde perceptif déjà constitué
à l’aide des autres sens} et auquel elle ne s'intègre, dans bien des cas, jamais
%125
complètement {\scriptsize (Celui qui a été autrefois voyant, recouvre la vue
avec joie. Mais il en est tout autrement de l'aveugle {\it de naissance}. « Ceux qui
ont assisté à ce qui nous semble tenir un peu du miracle, nous dépeignent une
figure crispée, presque douloureuse, réflexe de défense qui se produit quand
la lumière frappe pour la première fois la membrane sensible. » Certains
aveugles-nés opérés mettent parfois plusieurs années à acquérir une vue {\it à peu
près normale}. C’est que chez eux « l'évolution du cerveau s'est faite sans que
celle de la fonction l’accompagnât, de sorte qu'il existe entre les deux un
décalage qui rend des plus difficiles l’adaptation de l'une à l’autre »
(Dr H. Bouquet).)}. — 2° Quant aux éléments {\it moteurs} de la vision, il faut
avouer qu’ils sont le plus souvent {\it inconscients} : comment donc pourraient-ils
nous donner la sensation de la distance? {\it L’accommodation} n'intervient
d’ailleurs que pour les petites distances et « au delà de 4 mètres, il est
impossible de constater une différence de netteté entre les images d’un objet
suivant qu’on accommode pour sa distance même ou pour une distance
beaucoup plus considérable » (Bourdon). La {\it convergence} ne pourrait à la
rigueur nous renseigner que lorsque le point fixé se trouve à environ 1 mètre
de nous ; les différences deviennent déjà peu sensibles à 2 mètres, presque
insensibles vers 4 ou 5 mètres, nulle au delà de 16 mètres. Les propriétés
des {\it images rétiniennes} se heurtent à ce fait que {\it nous ne regardons jamais
vers ces prétendues « images »} (cf. p. 121, note), {\it mais vers l'objet} : beaucoup
de personnes les ignorent et la rétine n’est jamais objet de perception
{\scriptsize (L'ophtalmologie comparée donne les mêmes résultats. Dans sa
{\it Biologie de la vision} (coll. A. Colin, 1945), le Dr Marie-Louise Verrier nous
apprend que « les mouvements de convergence des yeux chez les Rapaces diurnes
sont sinon nuls, du moins ... négligeables ». Or ces oiseaux manifestent une
très grande acuité visuelle et une remarquable appréciation des distances.
L'auteur en conclut que « la perception la plus nette des formes et du relief
semble indépendante de la vision binoculaire »)}.}
\vspace{0.31cm}


{\it C.} On peut donc conclure avec Renée Dejean qu’« il n’y a en
aucune manière une {\it projection} des images par l’esprit, dans le sens
d’une extériorisation d’images d’abord sans distance. Dès qu’il y a
perception visuelle, c’est {\it à distance} que nous regardons, non vers la
rétine ». Est-ce à dire que les éléments que nous venons d’énumérer
soient sans importance? Non certes, et des expériences bien connues,
comme celles de la stéréoscopie, des anaglyphes
{\scriptsize (On sait qu'on appelle ainsi des images stéréoscopiques-confondues,
l'une en rouge, l’autre en vert, qu’on observe respectivement avec un lorgnon
vert et un lorgnon rouge, ce qui donne une intense impression de relief.)}
et du cinéma en
relief, d’autres encore (voir {\it Exercice} 4) montrent bien le rôle de la
vision binoculaire dans la {\it représentation distincte} et {\it l’estimation précise}
de la distance, dans la vision en relief, etc. Mais ces éléments sont
intimement liés à la \textbf{\textit {fixation oculaire de l’objet}}, qui est ici essentielle :
« La vision à distance des images est toujours liée à la fixation oculaire »
et c’est « dans la mesure où les mouvements oculaires se coordonnent,
de façon à permettre la fixation et la vision nette, [que]
la distance de l’image se précise, ce qu’elle ne pourrait faire si l’image
n’était déjà perçue à distance » (Dejean). C’est lorsque nous faisons
attention d’une façon élective à une distance donnée, que les axes
%126
visuels convergent pour cette distance, et l’on verra (\S 106) que
ces mouvements de convergence, comme ceux de l’accommodation
(cf. \S 95 B), s’observent même en l’absence de tout excitant et sous
l'influence de la seule représentation mentale. Il y a donc ici une
\textbf{\textit {« activité prospective de l'esprit »,}} activité expectative et même
« interrogative et provocatrice »; car « non seulement nous nous
attendons à l’action du réel sur nous ; mais, dans la vision par exemple,
anticipant sur la distance de l’excitant, nous adaptons notre système
récepteur pour cette distance et rendons effective l’action du réel
sur nous, de simplement possible qu’elle était » (Dejean). Et ainsi,
l'explication est à chercher dans « les conditions {\it psychologiques} elles-mêmes,
qui déterminent la fixation à telle distance dans le champ
visuel, l’activité prospective de l’esprit vers toutes les directions
du champ visuel et en particulier vers telle distance » (Id.).

\section{L’appréciation de la distance}% 95.
Le rôle de cet élément
mental sera encore plus manifeste si l’on considère, non plus seulement
la perception {\it à distance}, mais la \textbf{\textit {représentation}} de la distance
et son \textbf{\textit {appréciation correcte.}} Les erreurs mêmes que nous commettons
sur ce point montrent bien que cette appréciation est le résultat
d’un apprentissage et d’un acte proprement intellectuel : le {\it jugement.}

{\it A.} Inutile d’insister sur l'appréciation des distances par le \textbf{\textit {toucher}}
ou plutôt le sens \textbf{\textit {kinésique,}} qui n’intervient plus guère chez le voyant,
mais qui joue un rôle très important chez l’aveugle. « Dire qu’un
objet est à une certaine distance de nous, c’est, en termes psychologiques,
dire que pour entrer {\it en contact} avec lui, il nous faudrait exécuter
un certain nombre d’{\it efforts}. Dire d’un objet qu’il est plus éloigné
de nous qu’un certain autre, c’est également, en termes psychologiques,
dire qu’il nous faudrait faire {\it plus d’efforts} pour entrer en
contact avec lui » (Cresson{\scriptsize (André Cresson, philosophe français, 1859-1950)}).

{\it B.} L’appréciation des distances par \textbf{\textit {la vue}} devient, chez l’adulte,
une véritable {\it inférence}, qui s’appuie sur les signes suivants : 1° le
nombre d’{\it objets interposés} entre l’objet perçu et le sujet (plus il y a
de collines entre une montagne et nous, plus nous la jugeons éloignée ;
au contraire, quand de tels points de repère manquent, par exemple
dans une plaine uniforme, en mer ou verticalement quand il s’agit
d'évaluer la hauteur des nuages ou d’un avion, nous sous-estimons la
distance ; de même, par une nuit sans lune où les objets sont à peine
visibles, l'horizon semble plus proche) ; — 2° la {\it superposition} des
objets (la lune ne nous apparaît plus éloignée que les nuages que
%127
lorsqu'ils passent devant elle) ; — 3° les {\it vitesses différentes de déplacement}
des objets quand nous nous déplaçons nous-mêmes (en chemin
de fer, en voiture, les objets proches fuient plus vite que les objets
lointains) ; — 4° la {\it netteté} plus ou moins grande de l’image (nous
surestimons la distance d’une image floue, par exemple dans le brouillard,
tandis que, dans l’air très sec et limpide des pays nord-africains,
une ville éloignée paraît proche) ; — 5° la {\it grandeur apparente} des
objets connus (un objet petit, par rapport à sa taille normale, nous
paraît, {\it toutes choses égales d’ailleurs}, plus éloigné ; si, dans une chambre
totalement obscure, on place, à égale distance du sujet, deux ballons
de baudruche que l’on peut gonfler et éclairer à volonté, le sujet ne
voit pas grossir celui des deux ballons que l’on gonfle : il le voit
s'approcher, de même qu'il voit l’autre s’éloigner si on le laisse dégonfler ;
l'accommodation se modifie d’elle-même comme si les ballons
s’approchaient ou s’éloignaient réellement) ; — 6° les {\it ombres portées},
qui jouent un grand rôle dans l’appréciation du relief (les inégalités
d’une route, invisibles au jour, apparaissent, la nuit, lorsqu'elles sont
éclairées par les phares d’une automobile, à cause des ombres).

{\it C.} Nous apprécions également la distance d’après les sensations de
l’\textbf{\textit {ouïe.}} L'enfant apprend assez vite à localiser les sons et les bruits et
tourne la tête dans leur {\it direction} (vers la dixième semaine) tout en
commettant des erreurs (p. 106, note). Mais l’appréciation de la {\it distance}
est bien plus difficile, et nous-mêmes nous trompons souvent.
Les éléments qui entrent ici en jeu sont : 1° la différence d’{\it intensité}
des impressions reçues par l’une et l’autre oreille ; — 2° la différence
des {\it moments} où chacune les reçoit ; — 3° les différences de {\it phase} que
présentent les ondes sonores de l’une à l’autre. Les aveugles apprécient
la distance d’un obstacle d’après l’altération que produit dans
les bruits ou les sons la proximité de cet obstacle {\scriptsize (Et aussi, semble-t-il, d’après les légères variations produites dans la pression de
J'air sur le visage par l'approche de l'obstacle.)}.

\section{Erreurs et troubles de la perception de l’espace}% 96.
La
perception de l’espace, étant en partie construite, est susceptible,
tout comme la perception des objets et du réel, de certaines erreurs
et aussi de certains troubles pathologiques.

{\it A.} Nous commettons souvent des \textbf{\textit {erreurs}} sur la {\it distance} des objets
en interprétant à faux les critères indiqués au \S 95 B et C. La nuit,
nous apprécions fort mal la distance d’une lumière isolée, par exemple
de la lanterne d’une voiture sur une route, parce que les points de
repère nous manquent. Beaucoup des \textbf{\textit {illusions d'optique}} analysées
au \S 88 impliquaient des interprétations arbitraires des relations de
%128
grandeur, de longueur, de distance. Un exemple célèbre est celui de
l'illusion qui nous fait « voir » le soleil ou la lune beaucoup plus
grands à l'horizon que lorsqu'ils sont haut dans le ciel.

\vspace{0.24cm}
{\footnotesize Malebranche avait cru pouvoir l'expliquer par l'augmentation, dans
cette position, de leur {\it distance apparente}. celle-ci étant due à la fois au plus
grand nombre d'objets interposés (\S 95 Bj et au fait que le ciel nous
apparaît comme une voûte surbaissée. Cette explication a été longtemps
admise, et B. Bourdon écrivait en 1936 qu’aucun argument probant n'a
été apporté contre elle. Au reste, la même illusion d’un agrandissement au
voisinage de l'horizon se constate aussi pour les nuages et les constellations.
Nous croyons que l’explication de Malebranche peut être maintenue, à
condition toutefois de l'intégrer à une conception plus « synoptique » de la
perception spatiale. Grandeur apparente et distance apparente ne sont pas
deux éléments indépendants : elles sont liées, comme le dit M. Merleau-Ponty,
dans une « organisation d'ensemble du champ », la grandeur apparente
n'étant rien d'autre « qu’une manière d'exprimer notre vision de la
profondeur ».}
\vspace{0.31cm}


Mais, parmi les illusions de la perception de l’espace, les plus
remarquables sont les \textbf{\textit {illusions de relief,}} telles celles qu’on observe
dans les figures 18 E et F (page 112), dans chacune desquelles deux
interprétations sont possibles. C’est ici qu’est le plus net le phénomène
déjà signalé à la fin du chapitre V : le changement {\it brusque} d’interprétation
selon que l’on fixe telle ou telle partie du dessin ou que
celui-ci est considéré comme vu de dessus ou de dessous, ce qui met
en lumière le rôle de « {\it l’activité prospective de l'esprit} » signalé plus
haut. Les peintres provoquent des illusions de profondeur en utilisant
des moyens tels que : perspective, grandeur relative, jeux de
lumière et d’ombres (\S 95 B). « Quand nous regardons un tableau où
l'artiste s’est servi de ces moyens, il y a conflit entre leurs effets
et les influences qui tendent à nous faire percevoir le tableau tel
qu’il est réellement, c’est-à-dire plan » (Bourdon). Le même conflit
s’observe quand, a l’aide d’un pseudoscope, on cherche à voir en creux
ce qu’on voit ordinairement en relief. Mais il est très remarquable
que cette inversion se produit beaucoup plus facilement avec des
objets ou des images peu connus. Au contraire, {\it il est impossible de
voir inversé le relief d’un visage humain} (à moins de regarder à l’intérieur
d’un masque). Ici encore, on voit que la perception de l’espace
est liée à un « modèle mental » de l’objet.

{\it B.} Il y a d’autre part une \textbf{\textit {pathologie}} de la perception de l’espace,
liée le plus souvent à la pathologie du sentiment du réel.

\vspace{0.24cm}
{\footnotesize Certaines maladies s’accompagnent de troubles
dans la perception de la
{\it distance} ou dans celle des {\it dimensions}. Un vieillard, dans les prodromes
d’une attaque d’hémiplégie, voyait tous les objets se rapprocher et s’éloigner
%129
tour à tour. Un autre malade apercevait son médecin comme un géant
et, un moment après, comme un nain. Mais c'est aussi chez les {\it psychasthéniques}
et les {\it schizophrènes} que la perception de l’espace est troublée.
Les premiers éprouvent souvent des sentiments de {\it désorientation} et
éprouvent des difficultés à se conduire, même dans leurs appartement.
Les choses leur paraissent, tantôt se perdre dans on ne sait quel lointain,
tantôt se rapetisser étrangement, ce qui est souvent l'expression d’un sentiment
d’éloignement moral plutôt que matériel : « L'illusion de la vue se
trouve sous la dépendance de l'impression d'isolement, de fuite du monde »
(Janet). Quant aux schizophrènes, vivant dans un monde autistique
\S 41), ils vivent aussi dans un espace différent de l'espace commun : l’espace
leur paraît « vide » et dénué de signification ; l’un d’eux déclare ne plus
comprendre comment les aiguilles d’une pendule peuvent passer d'une
position à une autre. On remarquera dans le texte cité au \S 106, {\it fin}, le sentiment
d’{\it immensité} [en même temps que d'{\it irréalité}) éprouvé à l’égard de
l'immeuble de l’école. La même malade voit les choses, les personnes mêmes
qui s’approchent d'elle « grossir, grossir démesurément ». — En somme,
{\it l’espace lui-même est déstructuré}, il se disloque comme le réel lui-même.}
\vspace{0.31cm}


\section{Le concept d’espace}% 97.
Nous n’avons étudié jusqu'ici
que la perception ou la représentation de l’étendue concrète. Mais
que penser de la notion de {\it l’espace abstrait}, du \textbf{\textit {concept}} d’espace,
tel que l’utilisent les géomètres ?

{\it A.} Kant avait admis que ce concept est une forme \textbf{\textit {a priori,}} c'est-à-dire
antérieure à toute expérience, que l'esprit appliquerait au
donné sensible dès que nous percevons. Cette conception, solidaire
du rationalisme innéiste, est inadmissible (chap. XVII): on verra
d’ailleurs qu’il existe, non pas un seul concept possible de l’espace,
celui de l’espace euclidien, mais plusieurs (\S 98). Bien mieux, comme
l’a signalé Piaget, il n’est nullement évident que la construction de
l’espace perceptif et même représentatif commence par les formes
euclidiennes : droite, plan, etc. Il semble au contraire qu’elle débute,
comme on l’a dit ci-dessus, par des rapports {\it topologiques}, c’est-à-dire
de {\it position} respective, plutôt que par des rapports métriques, alors
que la Topologie n’est entrée dans la science que fort tardivement
(cf. ci-dessous p.399-401).

{\it B.} Il faut donc admettre que c’est \textbf{\textit {à partir de l'étendue concrète,}}
celle que nous construisons à l’aide des données sensibles, que se
forme le concept de l’espace. Mais ici un problème se pose : chacun de
nos sens, la {\it vue} et le {\it toucher} principalement, nous donne un espace
perceptif dont les propriétés ne sont pas identiques à celles des autres,
et l’on a même vu (\S 92) qu’à l’origine il n’y a chez l’enfant aucune
coordination entre ces différents espaces. La vue est un sens {\it synthétique}
(\S 78 B) qui nous donne toute l'étendue du champ d’un seul
coup, tandis que le toucher est analytique. La vue nous fournit un
%130
espace {\it lointain}, tandis que le toucher ne nous donne que l’espace
proche. Enfin et surtout, les dimensions visuelles {\it varient}, tandis que
les dimensions tactiles demeurent fixes : l’aveugle de Cheseldan à
qui l’on montrait le portrait de son père en miniature, s’étonnait
« qu’un grand visage pût être représenté dans un aussi petit espace ».

Aussi certains philosophes ont-ils conclu que notre espace familier
était construit avec l’{\it un} ou l’{\it autre} de ces deux sens exclusivement.
Au {\footnotesize XVIII}$^\text{e}$ siècle, Berkeley avait soutenu que c’est de l’espace tactile
seul que nous nous servons, tandis que, peu après, E. Platner, se
fiant à une observation faite sur un aveugle (1785), reportait ce privilège
sur la vue, thèse qui a été reprise récemment par Gelb et
Goldstein, selon lesquels « il n’existe à proprement parler qu’un
espace visuel ». — Ce qu’on peut, semble-t-il, accorder à ces derniers,
c’est que, chez le voyant, les données de la {\it vue} finissent par prédominer
à cause des avantages qu’elles présentent. Mais ce qui a été dit ci-dessus
montre bien que les éléments {\it tactilo-moteurs} jouent également
un grand rôle dans la constitution de la notion d'espace. Les différences
entre les deux « atlas », comme disait Taine (\S 84), ne sont pas
telles qu’ils ne puissent se fondre en un champ spatial unique.

\vspace{0.24cm}
{\footnotesize Là-dessus le témoignage des aveugles
est unanime. Déjà les pensionnaires
de l’Institution des Jeunes Aveugles déclaraient à Taine qu'ils n'avaient
pas besoin, pour se {\it figurer} un objet, pour {\it imaginer} une ligne ou une surface,
« de se représenter les sensations successives de leur main promenée dans
telle ou telle direction ». Ce témoignage est confirmé par celui de P. Villey
qui déclare que l’aveugle finit par acquérir une véritable « vue tactile » et
que l’objet « jaillits synthétiquement dans son imagination tout comme
chez le voyant ; que, d'autre part, l'étendue du champ {\it auditif} supplée à
l'insuffisance d’étendue du champ tactile ; et qu’enfin même la variabilité
des dimensions visuelles a, pour l’aveugle, un équivalent dans le sens des
obstacles (p. 128, note) grâce auquel il se sent, en quelque sorte, moins
« écrasé » par un objet lointain que par un objet proche.}
\vspace{0.31cm}

La construction du concept d’espace s’effectue donc, à partir de
toutes ces données perceptives, comme celle de tous les autres {\it concepts}
(voir ci-dessous chap. XIV, et chap. XX, \S 249-250). On va voir
d’ailleurs que, dans cette construction, se manifeste la {\it liberté créatrice}
de l'esprit.

\section{Le problème épistémologique}% 98.
On a cru longtemps en
effet que l’espace géométrique tel que nous sommes habitués à nous
le représenter, comme un milieu homogène, infini, à trois dimensions,
etc., était le seul concevable. Par suite, la Géométrie classique,
dite \textbf{\textit {euclidienne,}} apparaissait comme la seule possible. Or le progrès
même de la science a conduit à mettre en question la \textbf{\textit {valeur}} de cette
%131
conception, et c’est ainsi que surgit le {\it problème épistémologique}. Nous
aurons à y revenir au tome II à propos des différentes sciences.
Bornons-nous iei à quelques indications.

{\it A.} Au cours du {\footnotesize XIX}$^\text{e}$ siècle, se sont constituées des \textbf{\textit {Géométries}}
dites {\it non-euclidiennes} reposant sur d’autres postulats et sur d’autres
conceptions de l’espace que la Géométrie classique (voir chap. XX,
p.409), et ces nouvelles Géométries se sont révélées parfaitement
cohérentes du point de vue logique et capables même d'applications
empiriques. On peut donc concevoir des {\it hyperespaces}, c'est-à-dire des
espaces à plus de trois dimensions ; des espaces pour lesquels l'Axiome
des parallèles (postulatum d’Euclide) n’est plus valable, etc. Il se
révèle ainsi que l’espace euclidien n’est pas le seul que puisse construire
la Raison et que, si elle avait jusqu'ici privilégié celui-là, c'est
uniquement qu’il correspond mieux que les autres aux conditions
{\it moyennes} de l'expérience humaine ordinaire (ci-dessous \S 253 D)

{\it B.} La \textbf{\textit {Physique}} contemporaine a exigé um renouvellement tout
semblable de la notion d'{\it espace physique}. La théorie de la Relativisé
(ci-dessous, \S 269) aboutit à la notion d’un espace indissolublement lié
à une quatrième dimension : {\it celle du temps}. Ce « continuum quadri-dimensionnel »
est précisément analogue à l’un de ces espaces non-euclidiens
conçus par les mathématiciens, l’{\it espace courbe} de B. Riemann, illimité, mais
non infini, où il est impossible de tracer des
figures semblables à n'importe quelle échelle (fig. 43, p. 409). D'autre
part, cet \textbf{\textit {espace-temps}} n'est pas un milieu neutre, un {\it cadre} où se
situeraient les choses, un {\it contenant} qui aurait pour contenu les phénomènes :
c’est une réalité physique, une propriété de l’univers, et les
« courbures » de l’espace s’identifient à la distribution des masses
matérielles, à peu près comme une membrane de caoutchouc se
trouverait incurvée par la masse d’une bille qu’on y déposerait. La
notion d’espace tend ainsi à se confondre avec la notion de \textbf{\textit {champ,}}
conçue comme « l’ensemble des propriétés physiques qui caractérisent
à chaque instant les divers points de l’espace et qui s’ expriment
par des fonctions des coordonnées d'espace et de temps »; c'est le
champ « qui crée et qui modèle l’espace et le temps en leur donnant
un contenu physique » (L. de Broglie).

{\it C.} L'espace \textbf{\textit {biologique}} n’est pas, lui non plus, un milieu vide
indifférent à son contenu. Déjà le lamarckisme (voir \S 278) nous
avait appris à ne jamais séparer ces deux termes : l'organisme vivant
et son milieu. Mais certaines théories récentes ont resserré encore ce
rapport {\it organisme-milieu}. Celle de K. Goldstein, par exemple,
aboutit à la notion d’un {\it champ} totalitaire formé par l’un et l’autre
et organisé selon certaines lois d’ensemble ; la {\it grandeur} même des
%132
formations organiques qui serait « un des caractères essentiels de
l'être », devrait être prise en considération. Dans un ordre d’idées
moins ambitieux, les travaux de Lecomte du Noüy sur la cicatrisation
ont montré que la régénération des tissus, dans une plaie,
s’oriente selon certaines directions privilégiées, par exemple qu’elle
vient combler d’abord les angles aigus.

{\it D.} L'espace \textbf{\textit {social}} est plus riche encore. É. Durkheim avait soutenu
que « l’organisation sociale a été le modèle de l’organisation spatiale
qui est comme un décalque de la première ». C’est ainsi que, dans certaines
sociétés archaïques, « l’espace est conçu sous la forme d’un
cercle immense, parce que le camp a lui-même une forme circulaire »
et que le cercle spatial est divisé à l’image du cercle tribal : l’espace
comporte autant de régions distinctes qu’il y a de clans dans la tribu
et orientées comme le sont ceux-ci dans le campement. Cette thèse,
qui se relie à l’ensemble de la thèse durkheimienne sur l’essence de
la raison (chap. XVII), est peut-être en elle-même excessive. Mais
il est bien vrai que notre existence sociale contribue à déterminer
notre conception de l’espace.

\vspace{0.24cm}
{\footnotesize La distinction de la {\it droite} et de la {\it gauche}, en particulier,
loin d'être impliquée dans la nature organique de l’homme, semble bien être le produit
de représentations collectives : la \textsf{\textit {prééminence de la main droite}} a été consacrée
par les croyances religieuses et l’on a pu soutenir que la distinction en
question équivaut, dans la plupart des sociétés, à celle du {\it sacré} et du {\it profane}.
La représentation que l’on se fait de l’espace varie selon les sociétés considérées.
Le primitif, dit L. Lévy-Bruhl, se meut dans l’espace comme nous,
« mais autre chose est l’action dans l’espace, autre chose la représentation
de cet espace ». Or, chez lui, l'espace est un espace « chargé de qualités »
et de participations mystiques, avec lequel il se sent lié {\it spirituellement}.
C’est à la même conclusion qu’aboutit M. Leenhardt à propos des Mélanésiens :
leur espace est « qualitatif » : il est lié à là représentation des
séjours mythiques des ancêtres ; et il est discontinu : « Le pays lui-même
n'est pas un espace vide ; il n'existe qu'aux lieux où vivent des groupes
humains en rapport avec le clan initial.» M. Granet nous en dit autant des
Chinois : « Aucun des philosophes chinois n’a trouvé un intérêt à considérer
l'espace comme une simple extension résultant de la juxtaposition d’éléments
homogènes, comme une étendue dont toutes les parties seraient
superposables. Tous préfèrent voir l’espace comme un complexe de domaines,
de climats et d'orients.» M. Halbwachs enfin, généralisant cette idée,
remarque que, jusque dans nos sociétés, \textsf{\textit {« il y a autant de façons de se
représenter l’espace qu'il y a de groupes »}} : il y a naturellement un « espace »
national, celui que délimitent les frontières ; mais il y a aussi un espace
juridique, un espace économique et surtout un espace religieux : toutes
les religions ont leurs « lieux saints ». Ainsi, « le lieu occupé par un
groupe n’est pas comme un tableau noir sur lequel on trace, puis efface
des chiffres et des figures... Le lieu a reçù l'empreinte du groupe, et réciproquement ».}
\vspace{0.31cm}

%133
Plus généralement encore, on ne peut séparer la vie d’un groupe
de la \textbf{\textit {base spatiale}} sur laquelle il vit : c’est, on le verra (cf. le \S 296),
l’idée fondamentale de ce qu’on appelle en Sociologie la « morphologie
[ou encore : écologie] sociale ». Or {\it l'étendue de cette base} peut varier
étrangement. Paul Valéry a dit que notre monde moderne n’est plus
« une figure semblable » des sociétés d’autrefois (voir \S 291) : « L’Europe
me fait songer, ajoutait-il, à un objet qui se trouverait brusquement
transposé dans un espace plus complexe, où tous les caractères
qu’on lui connaissait et qui demeurent en apparence les mêmes, se
trouvent soumis à des {\it liaisons} toutes différentes. » Il s’est produit
un « agrandissement et un accroissement de connexions du champ des
phénomènes politiques » qui rendent les prévisions et les calculs des
hommes d’État « plus vains que jamais ils ne l’ont été ».

\section{Le problème métaphysique}% 99.
Il apparaît ainsi que l’espace
homogène des mathématiciens est une construction assez arbitraire
et loin du réel. Comment dès lors ne pas mettre en question
l'{\it essence} et la {\it réalité} même de cet espace? Nous voici conduits au
problème {\it métaphysique}.

{\it A.} Les Cartésiens avaient maintenu, en général, une conception
\textbf{\textit {réaliste}} de l’espace. Pour Descartes lui-même, c’est l'{\it étendue} qui
constitue à elle seule toute l’essence de la matière (chap. XXVIII).
L’étendue ou l’espace possède donc une réalité propre : « elle ne diffère
de la substance [corporelle] que par cela seul que nous considérons
quelquefois l’étendue sans faire de réflexion sur la chose même
qui est étendue ». — Malebranche va beaucoup plus loin. Dans
ses {\it Entretiens sur la métaphysique}, il déclare : « L’étendue est une
réalité et, dans l’infini, toutes les réalités s’y trouvent. Dieu est donc
étendu, aussi bien que les corps, puisque Dieu possède toutes les réalités
absolues. Mais Dieu n’est pas étendu comme les corps ; car il
n’a pas les limitations et les imperfections de ses créatures. » Autrement
dit, ce qui existe en Dieu, c’est l’{\it idée} de l’étendue, c’est l'{\it étendue}
{\it intelligible}, éternelle, infinie, immuable. Celle-ci est en lui comme
l’archétype de l’étendue sensible, toujours limitée et divisible, dans
l’étoffe de laquelle, pour ainsi dire, sont taillés les corps. — Spinoza
enfin, au lieu de briser ainsi en deux l’étendue en y distinguant l’intelligible
du sensible, l’élève directement jusqu’à Dieu en en faisant
l’un des deux {\it attributs} essentiels de la Substance unique et infinie,
l’autre étant la pensée (chap. XXIX). Mais il distingue lui aussi cette
étendue indivisible, sans parties, de celle que nous présente l’imagination,
de sorte que, si « Dieu est chose étendue », il est cependant
incorporel. — Mais c’est surtout avec Newton que s’est affirmé le
%134
réalisme de {\it l’espace absolu}, conçu comme un cadre vide {\scriptsize (Les
conceptions de Malebranche et celle de Spinoza, impliquant la notion d’un
ensemble indivisible et intelligible, se rapprochaient au contraire de celle
d’un {\it champ} abstrait. Sur le caractère très moderne de cette notion, voir
ci-dessus \S 98 B et ci-dessous, \S 269, 2°.)}, indifférent à
son contenu, « sans relation aux choses externes » et qui, par lui-même
« demeure toujours identique et immobile ».

{\it B.} D’autres philosophes, au contraire, ont soutenu la thèse de la
pure \textbf{\textit {idéalité}} de l’espace, celui-ci n’ayant selon eux aucune réalité
en dehors de notre esprit. C’est ainsi que Leibniz allègue, contre le
réalisme de Descartes, que l’étendue, étant toujours divisible, n’est
qu’un « être par agrégation » et ne peut présenter l’{\it unité} nécessaire à
une substance : « Tout n’y est que collection ou amas de parties. » Il
combat également le point de vue de Newton qui fait de l’espace « un
être réel absolu ». Pour lui, l’espace n’est que « l’ordre des coexistences
possibles », et, pour en avoir l’idée, « il suffit de considérer les rapports
de situation et les règles de leurs changements sans avoir
besoin de se figurer aucune réalité absolue hors des choses dont on
considère la situation ». — Kant écrit de même, dès 1770 : « L'espace
n’est pas quelque chose d’objectif ni de réel, il n’est ni une substance
ni un accident ni une relation, mais quelque chose de subjectif et
d’idéal, issu, selon une loi fixe, de la nature de l’esprit à la manière
d’un schéma destiné à coordonner l’ensemble de tout ce qui vient de
l’extérieur par les sens. » Plus tard, il fera de l’espace une des deux
{\it formes a priori} de la sensibilité, si bien qu’il affirmera à la fois la
« réalité empirique » de l’espace, en ce sens que toute expérience
externe est donnée dans cet espace, et son « idéalité transcendantale »
puisqu'il n’est que la condition {\it a priori} de cette expérience.

{\it C.} Bergson qui, on le verra bientôt (cf. le \S 118), a prétendu
« mettre fin au conflit du réalisme et de l’idéalisme », a opposé
{\it l’espace} comme « simultanéité de termes qui, identiques en qualité,
se distinguent néanmoins les uns des autres », à la {\it durée concrète} qui
est, au contraire, « succession de changements qualitatifs » sans extériorité
(cf. \S 148), le premier étant caractéristique du monde de la
matière, la seconde du monde spirituel. Ceci ne signifie pas que
l’espace soit identique à la matière. Sans doute, il y a bien une « spatialité
des choses », mais cela doit s’entendre de l’étendue telle que la
perçoit, par exemple, l’animal, ce qui lui permet parfois de s’y orienter
de façon si merveilleuse. L’{\it espace homogène}, au contraire, n’est ni
une propriété des choses ni une « condition essentielle de notre faculté
de connaître » : c’est une simple création de l'{\it intelligence} humaine,
%135
faculté dont le rôle, selon Bergson, est purement pratique ; c’est un
« schème de notre \textbf{\textit {action}} sur la matière ». Il exprime la nécessité où
nous sommes, pour agir, de découper et de solidifier en objets cette
« continuité mouvante » qu'est le réel. Il y a ainsi « une régression de
l’extra-spatial se dégradant en spatialité ». La matière elle-même est
précisément « un relâchement de l’inextensif en extensif » (cf. le \S 345),
et ainsi « elle a beau ne point coïncider tout à fait avec le pur espace
homogène : elle s’est constituée par le mouvement qui y conduit et
dès lors elle est sur le chemin de la géométrie ».

{\it D.} Nous ne chercherons pas à trancher ici ce débat, ce qui impliquerait
une prise de position à l'égard du conflit du Réalisme et de
l’Idéalisme, problème qui ne sera abordé qu’au chapitre VIII. Disons
seulement qu’il nous paraît impossible, contrairement à la tentative
de solution bergsonienne : 1° de réduire le concept d’espace à un
concept purement pratique; 2° d’opposer radicalement l’{\it espace}
comme caractéristique du monde de la {\it matière}, au {\it temps} (ou à la
durée) comme caractérisant le monde {\it spirituel}.

1° L'étude psychologique et génétique de la notion d’espace nous
a montré qu’il existe un « espace perceptif », une étendue concrète,
indissolublement liés à notre perception des choses, — et c’est là
l’aspect \textbf{\textit {réaliste}} de cette notion, — mais que cette spatialité vague et
à peine structurée a été reconstruite par l'esprit, transposée sur le
plan de l’{\it intelligible}, d’abord au niveau de la représentation courante
(\S 97), ensuite au niveau des concepts scientifiques (\S 98), —
ce qui en est l’aspect \textbf{\textit {idéaliste,}} « L'espace géométrique n’est pas un
pur décalque de l’espace physique construit en même temps que lui :
l’abstraction de la forme est une véritable reconstruction de celle-ci »
(Piaget), et, quoiqu’elle ait toujours son point de départ dans les
actions propres du sujet et dans l’espace sensori-moteur, un tel
« décalage » nous interdit de considérer le concept d’espace comme un
instrument purement pratique ; il nous y fait apercevoir plutôt \textbf{\textit {un
instrument créé par l'esprit pour la rationalisation du réel,}} et nous
comprenons en même temps comment cet instrument, créé d’abord
pour répondre aux conditions moyennes de l’expérience humains
ordinaire (\S 98 {\it A}), a dû être assoupli et refondu pour pouvoir
s’adapter aux conditions prodigieusement élargies de l’expériencé
scientifique et d’une vie humaine de plus en plus ouverte (\S 98 B-D).

2° L'opposition nous paraît donc résider beaucoup moins entre
les notions de l’{\it espace} et du {\it temps}, que d'ailleurs la science contemporaine
tend à lier dans un même système de coordonnées (\S 98 B),
qu'entre \textbf{\textit {deux niveaux de notre connaissance du monde extérieur,}}
%136
celui de l'{\it intuition} vague et confuse de la spatialité, correspondant à
notre expérience première de ce monde et celui du {\it concept} défini de
l’espace, structuré par l’intelligence devenue consciente de ses exigences
propres, en un mot : celui du {\it sensible} et celui de l’{\it intelligible}.
Or on verra que la même distinction s'impose à propos du {\it temps}
(\S 154 et fig. 27).

\section{Sujets de travaux}% SUJETS DE TRAVAUX

{\bf Exercices.} — 1. {\it Commenter et discuter ce passage de} G. Duhamel : « Le
petit homme [un enfant de deux ans] vit dans un espace qui n’a que deux
dimensions. Cet espace est large, long, mais sans profondeur : ce n’est
guère qu’une surface. Toutes les choses se peignent sur un écran, l'œil
sans agilité ne cherche guère au delà. La main s’élance pour saisir. Eh !
comment, dites-moi se mouvoir dans un espace plat ? Rien de plus simple :
le petit homme emporte partout son espace avec soi. » — 2. {\it Faites de même
avec ce passage de} L. Brunschvicg : « Quelque paradoxal que soit un pareil
énoncé, ce n’est pas en contemplant l’objet que l’on est arrivé à poser
comme règle de vérité l’immutabilité du contour, c’est en agissant pour en
reconstituer artificiellement l’aspect... L’intuition peut se borner à prendre
d'ensemble une idée de l’objet : le dessin exige que l’on procède trait par
trait. » — 3. {\it Observez sur vous-même comment vous parvenez à apprécier
la distance d'un objet lointain, soit sur terre, soit en mer.} — 4. {\it Faites cette
expérience suggérée par} Malebranche {\it dans la} Recherche de la Vérité,
{\it Liv. I, chap. IX} : « Que l’on suspende au bout d’un filet une bague dont
l'ouverture ne nous regarde pas ou bien qu’on enfonce un bâton dans la
terre et qu’on en prenne un autre à la main qui soit courbé par le bout ;
que l’on se retire à trois ou quatre pas de la bague ou du bâton ; que l’on
ferme un œil d’une main et que de l’autre on tâche d’enfiler la bague ou de
toucher de travers et à la hauteur environ de ses yeux le bâton avec celui
que l’on tient à la main; et on sera surpris de ne pouvoir peut-être faire
en cent fois ce que l’on croyait très facile, » — 5. {\it Expliquer cette formule de
M. Merleau-Ponty} : « Ce qui garantit l’homme contre le délire ou l’hallucination,
c’est la structure de son espace. » — 6. {\it Étudier cette suggestion de
Jean Wahl} : « Voir comment l’espace a été conçu de diverses façons par les
Chinois (Granet), par les Grecs, par le gothique, par le baroque. Il est
quelque chose qui doit être carré — ou percé, ajouré — ou gonflé, creusé.
L'espace de Cézanne avec ses distances, ses iridescences sans irisations,
ses passages. »

{\bf Exposés oraux.} — 1. {\it La notion de} l'étendue intelligible {\it chez Malebranche
d'après la} Recherche, {\it X$^\text{e}$ Éclaircissement, 2° et 3° objections, et les} Entretiens,
I, \S VII et X ; II, \S VI ; VIII, \S IV à VIII. — 2. {\it La « prééminence de
la main droite »} d’après R. Hentz, {\it Mélanges de sociologie religieuse}, Alcan,
1928, p. 99. — 3. {\it L'espace chez les primitifs} d'après L. Lévy-Bruhl et
M. Leenhardt, {\it Do Kamo}, Gallimard, 1947, p. 61 et 137.

{\bf Discuter : 1.} {\it ces définitions de l’espace} : « l’ordre des coexistences », « la
synthèse de la figure et de la quantité », « l’aptitude à contenir des corps »;
%137
— {\bf 2.} {\it la thèse suivant laquelle la perception de la distance est tout entière
acquise.}

{\bf Lectures.} — {\it a.} B. Bourdon, {\it La Perception visuelle de l'espace}, Costes,
1902 ; et {\it b.} {\it La perception}, dans le {\it Nouveau Traité de Dumas}, tome V, 1936.
— {\it c.} P. Villey, {\it Le Monde des aveugles}, Flammarion, 1914. — {\it d.} Lévy-Bruhl,
{\it La Mentalité primitive}, Alcan, 1922, p. 86-93. — {\it e.} Renée Déjean,
{\it La Perception visuelle}, I, Alcan, 1926. — {\it f.} J. Piaget, {\it La Construction
du réel...}, Delachaux, 1937, chap. Il; et {\it g. La Représentation de l'espace
chez l'enfant}, P. U. F., 1948. — {\it h.} M. Merleau-Ponty, {\it Phénoménologie
de la perception}, Gallimard, 1945, p. 281 et suiv. — {\it i.} M. Halbwachs,
{\it La Mémoire collective}, P. U. F., 1950, chap. IV. —— {\it j.} G. Bachelard, {\it La
Poétique de l'espace}, P. U. F., 1957. — {\it k.} J. Piaget et collab., {\it L'épistémologie
de l’espace}, P. U. F., 1964.

