\chapter{Le temps}
%chapitre XI  LE TEMPS
%SOMMAIRE
%146. La perception de la durée : différents problèmes. — 147. Les problèmes
%psychologiques. — 148. La durée vécue et ses fluctuations. — 149. La construction
%du temps. — 150. Présent, passé et avenir. — 151. Les troubles de
%la perception du temps. — 152. Le concept du temps. — 153. Le problème
%épistémologique. — 154. Le problème métaphysique. — 155. Signification du temps.

\section{La perception de la durée : différents problèmes}% 146.
Non seulement notre vie psychique s’écoule dans la {\it durée}, mais
l'étude de la mémoire et de l’imagination nous a montré que notre
pensée se projette dans le passé et dans l’avenir et qu’elle est capable
d'introduire dans cet écoulement certains repères. Les problèmes qui
se posent ainsi à propos du {\it temps} sont sensiblement parallèles à
ceux qui se posent à propos de l’{\it espace} (chap. VI), et il n’y a pas lieu
de s’en étonner : on verra que le temps, comme l’espace, est un des
instruments intellectuels à l’aide desquels l’esprit construit le monde.
— Il se posera donc ici : 1° des problèmes {\it psychologiques}, concernant
la perception de la durée et la formation de la notion de temps ;
2° des problèmes {\it épistémologiques} ; 3° des problèmes {\it métaphysiques}.

\section{Les problèmes psychologiques}% 147.
Le problème psychologique
principal peut être formulé comme l’a fait H. Delacroix :
« Le temps se constate-t-il comme une forme irréductible ou comme
une donnée immédiate de l’expérience en général ou de l’expérience
interne ? Se construit-il à partir de certaines données plus simples ? »
Et l’on retrouve ici, comme à propos de l’espace (\S 90), des théories
{\it génétiques} qui se placent à ce second point de vue et s’efforcent de
« construire le temps avec des qualités », et des théories {\it nativistes} qui
admettent le premier, telle celle de Bergson affirmant « que la durée
est le fond même de la qualité psychologique et par conséquent une
%223
donnée immédiate de la conscience ». Mais d’autres problèmes s’ajouteront
à ce problème fondamental. La durée n’est pas uniforme :
tantôt elle s'écoule lentement, tantôt elle passe avec la rapidité d’un
éclair ; il faudra chercher à expliquer ces {\it variations du sentiment de la
durée}. D’autre part, il y a lieu, de même que nous avons distingué la
perception de l’étendue concrète et la notion de l’espace abstrait,
d'étudier, après la durée vécue, la notion du {\it temps abstrait}.

\section{La durée vécue et ses fluctuations}% 148.
« On peut dire,
en un sens, du temps comme de l’espace, qu’il est donné déjà dans
toute perception élémentaire : \textbf{\textit {toute perception dure}}, de même
que toute perception est étendue. Mais cette durée première est aussi
éloignée du temps proprement dit que l’étendue de la sensation l’est
elle-même de l’espace organisé » (Piaget). Qu’est-ce donc que cette
« durée première » ?

{\it A.} On sait (\S 37) que Bergson l’a décrite comme une \textbf{\textit {durée
concrète}}, ne faisant qu’un avec la succession de nos états. La « pure
durée » n’est « qu’une succession de changements qualitatifs qui se
fondent, se pénètrent, sans contours précis, sans aucune tendance à
s’extérioriser les uns par rapport aux autres, sans aucune parenté avec
le nombre » : c’est « l’hétérogénéité pure ». Telle est « la forme que
prend la succession de nos états de conscience quand notre moi se
laisse vivre, quand il s’abstient d'établir une séparation entre l’état
présent et les états antérieurs » (Bergson).

{\it B.} Il y a là une excellente description de cette « profonde solidarité,
dans la conscience, du présent avec ce qui le précède et ce qui le suit »
(Delacroix) qui caractérise le temps. Mais y a-t-il lieu d'opposer,
aussi radicalement que le fait Bergson, cette continuité qualitative
de la durée au temps homogène et mesurable, dont les moments
sont numériquement distincts ? « Cette durée pure, remarque Delacroix,
dont l’émotion lyrique et musicale est l’approximation la
plus juste, ne risque-t-elle pas de devenir une sorte d’extase où justement
la conscience du temps s’efface? » Seule, son {\it hétérogénéité}, en
introduisant « la distinction à travers la pénétration », l'empêche de
s’évanouir dans « l’éternel présent » des mystiques et de « retomber
à l’immobilité par sa fluidité même ». Ne suffira-t-il pas, dès lors,
que la conscience soit « prête à s’analyser » pour que, de cette hétérogénéité
naisse une manière de discontinuité? Au reste, la durée est-elle
si essentiellement continue qu’on veut bien le dire? Cette continuité,
observe J. Nabert, « n’est rien moins que certaine, ...elle est
beaucoup plus l’œuvre de l'intelligence qu’une donnée immédiate de
la conscience », et l’on verra au \S 155 qu’à la description bergsonienne,
%224
G. Bachelard à opposé une conception {\it dialectique} de la durée :
« Prise dans le détail de son cours, nous avons toujours vu une durée
précise et concrète fourmiller de lacunes. » Quoi qu’il en soit, il est
certain que par là la durée est sur le chemin du temps structuré.

Autre objection. Bergson caractérise la durée comme l’expression
de la {\it spiritualité} pure. Mais l’homme est un « être incarné » : le rythme
de sa vie spirituelle ne peut être sans relation avec celui de sa vie
corporelle. Les fluctuations du sentiment de la durée sont donc nécessairement
en rapport avec les multiples rythmes : cardiaque, respiratoire,
etc., dont est faite notre vie organique. On a même parlé de
« l’horloge chimique » que constituent nos réactions chimiques internes
(Lecomte du Noüy) et l’on a pu constater l'influence de la température
corporelle sur les fluctuations du sentiment de la durée.

Mais ces {\it fluctuations} dépendent d’autres facteurs encore. Distinguons
d’abord le sentiment immédiat de la durée {\it au moment} où elle
s’écoule, et l’appréciation {\it rétrospective} du temps écoulé. Dans le premie
cas, on peut énoncer cette loi que {\it plus le temps est rempli, plus il nous
paraît court} et que {\it plus il est vide, plus il nous paraît long}.

\vspace{0.24cm}
{\footnotesize Tout le monde a pu constater qu’une grande activité abrège le temps,
tandis que celui-ci s’allonge dans l’inaction, le désœuvremént et dans l’ennui
qui souvent les accompagne. Le plaisir et la joie font passer le temps très
rapidement : c’est qu'ils sont expansifs ; l’être vit alors en dehors de lui-même.
Que le temps semble lent au contraire dans la tristesse ! c’est que
l’être s’y replie sur soi. Enfin, si notre intérêt intellectuel est vivement
excité par un spectacle, une audition, une occupation quelconque, nous en
sentons même pas le temps passer, tandis que, dans l’attente, le temps
s’étire démesurément.}
\vspace{0.31cm}

Ces faits sont faciles à expliquer : quand le temps est bien rempli,
notre attention se porte, non sur l’écoulement du temps lui-même,
mais sur ce qui nous occupe ou nous intéresse, tandis que, quand le
temps est sans contenu, « nous nous enfermons dans le pur sentiment vide »
(James). Ceci est surtout net dans l’{\it attente} où
l’idée de ce que nous attendons fait en quelque sorte le vide à son
profit dans notre esprit ; et, pour peu que l’attente se prolonge, cette
idée perpétuellement déçue ne nous laisse plus d’autre ressource que
de faire un sort à chaque minute, à chaque seconde qui passe et ainsi
de multiplier le temps en le morcelant à l'infini.

Il en va tout autrement dans l'appréciation {\it rétrospective} du temps
écoulé. C’est au contraire ici le temps rempli qui semble long quand
on se le rappelle après coup, et le temps vide qui paraît bref. C’est
qu’il s’agit alors d’un véritable {\it jugement} d'évaluation. Or nous évaluons
le temps écoulé, à peu près comme l’espace, d’après son contenu :
%225
une période qui évoque à notre esprit des souvenirs nombreux nous
paraît avoir été longue, tandis qu’une période vide se contracte à nos
yeux en une durée très courte. C’est peut-être aussi ce qui explique
les variations de l’appréciation du temps {\it selon l’âge}. « À mesure que
l'on vieillit on trouve le temps plus court » (W. James). Comme
on l’a vu (\S 134 À), le vieillard qui a souvent une bonne mémoire pour
le passé lointain, retient beaucoup moins bien le passé récent qui
souvent ne laisse guère de traces chez lui.

\vspace{0.24cm}
{\footnotesize Peut-être aussi faut-il faire intervenir la considération de la
proportionnalité de la période considérée par rapport à l’ensemble
de la vie et, plus encore, l'orientation générale de la vie psychique :
« La décennie 1920-1930, écrit un éminent critique, a filé moins
vite pour moi que pour mon père : et celle de 1940-1950 a pris à mes
yeux l'allure vertigineuse d'une descente en toboggan, alors que mes
fils, impatients de vivre, ont trouvé ce temps lent et long » (Émile Hennio).}

\section{La construction du temps}% 149.
L’appréciation du temps
écoulé nous met déjà en présence d’un temps \textbf{\textit {construit}} par opposition
à la simple durée {\it vécue}. Le philosophe J.-M. Guyau (1854-1888)
avait déjà parfaitement fait la distinction en montrant ce qu'est ici
le {\it syncrétisme} chez l'enfant : « L'enfant n'ayant pas développé l’art
du souvenir, tout lui est présent. Il ne distingue nettement ni les temps,
ni les lieux, ni les personnes. Les enfants mêlent ce qui a été, ce qui
est ou sera ; ils ne vivent pas comme nous dans le {\it réel}, dans le {\it déterminé} :
ne distinguant et ne {\it percevant} rien très nettement, ils {\it rêvent}
à propos de tout. L'enfant retient et reproduit des images beaucoup
plus qu’il n’invente et ne pense ; et c’est précisément à cause de cela
qu’il n’a pas l’idée nette du temps. » Comment donc va se faire le
passage de cette confusion primitive au temps organisé et structuré ?
J. Piaget qui a étudié la question conjointement avec celles de la
formation de l’idée d'{\it objet} et de l’idée d'{\it espace} (\S 92 et 103), distingue
également ici, chez l'enfant, {\it six étapes} successives. Sans entrer dans
le détail, disons simplement que, dans les premières, le temps comme
l’espace adhère encore à \textbf{\textit {l’action}} même de l’enfant. Ce n’est qu’à
partir du quatrième stade que l’enfant commence à « ordonner les
événements eux-mêmes, et non plus seulement ses propres actions ».
Ce ne sera cependant qu’au sixième que l’objectivation des séries
temporelles s’étendra à la représentation elle-même, « c’est-à-dire que
l’enfant, devenant capable d'évoquer des souvenirs non liés à la perception
directe, parvient par cela même à les situer dans un temps
qui englobe toute l’histoire de son univers ». — Dans l’ensemble,
cette étude génétique montre que « le temps, comme l’espace, se
construit peu à peu et implique l'élaboration d’un système de relations »
%226
(Piaget). En un sens, comme le dit Delacroix, ce temps
représenté est « un luxe », car l'exemple de l’enfant montre qu'il peut
y avoir succession (relativement ordonnée) d’actes sans qu’il y ait
représentation de la succession. « Des enfants arriérés, à l’âge de
douze ou quatorze ans, ont encore beaucoup de peine à se représenter
la perspective de l’histoire ; tous les siècles se confondent parce qu’ils
ne savent pas construire le temps. » Avant d’étudier comment cette
construction s'achève dans le {\it concept} du temps (\S 152), il nous faut
examiner comment le sentiment de la durée se structure déjà dans la
distinction des trois moments du temps.

\section{Présent, passé et avenir}% 150.
{\it A.} Qu'est-ce d’abord que
\textbf{\textit {le présent}} ? Au sens strict du terme, ce serait l'instant ponctuel, saisi
par intuition au moment même où il passe. Mais un tel « point » de
durée est insaisissable : il n’est que la limite idéale entre le passé et
le futur immédiat. Au moment même où nous cherchons à l’appréhender,
un tel présent est déjà devenu du passé :
\begin{center}
{\it Et l'instant où je parle est déjà loin de moi,}
\end{center}
disait Boileau. C’est là le {\it présent spécieux} ou {\it apparent}, comme l’a
nommé W. James. Mais le présent {\it réel} est tout autre chose. Guyau
l'avait déjà fort bien remarqué : « Le présent est un état {\it réel}, qui a
déjà sa {\it quantité de durée}. Si bref qu’il soit, le présent n’est pas un
éclair, un rien, une abstraction analogue au point mathématique : il
a un commencement et une fin ; de plus il touche à quelque chose, avec
quoi il forme continuité. » Les psychologues de laboratoire ont essayé
de mesurer cette « quantité de durée », c’est-à-dire l'intervalle de temps
maximum qui peut être saisi par la conscience comme un tout. Mais
les résultats ont varié de un cinq-centième de seconde (Exner) à
douze secondes (Wundt) ! C’est qu’une telle tentative est bien vaine.
Les limites du présent sont très indécises : il enveloppe à la fois un
écho de ce qui vient de se passer et une annonce ou une attente de ce
qui va arriver. Car, ainsi que l’avait encore bien vu Guyau, le présent
est lié à l’\textbf{\textit {agir.}} C’est cette idée que reprend Bergson lorsqu'il écrit :
« Ce que j'appelle mon présent, c’est mon attitude vis-à-vis de l'avenir
immédiat, c’est mon action imminente. » Le présent apparaît ainsi
avec un caractère d'{\it insertion dans le réel} et un caractère de {\it nouveauté} ;
et, comme le réel lui-même (\S 104 A), il est à la fois ce qui {\it offre prise}
à notre action et ce qui y résiste dans une certaine mesure.

{\it }B. Dans le \textbf{\textit {passé}}, il y a lieu de distinguer le {\it passé récent}, qui est du
présent évanescent : c’est « celui dont le récit a encore un caractère
affectif, qui détermine sinon des actes consommés, tout au moins des
%227
tentatives d'actes avortés, des déceptions et des regrets », — et le
{\it passé lointain}, qui est « le passé » proprement dit. C’est de celui-ci
qu’on peut dire ce que nous avons dit au \S 126, à savoir qu'il est, en
un sens, du réel, mais un réel {\it durci} sur lequel nous ne pouvons plus
rien. La seule action qu’il puisse susciter de notre part, c’est, comme le
dit Janet, celle du {\it récit}. Certes, dans cet acte du récit, nous le
{\it reconstruisons}, comme nous l’a montré l’étude de la mémoire. Mais
nous sentons bien que nous ne pouvons pas le reconstruire {\it à notre
gré} : la mémoire ne peut se permettre les fantaisies de l’imagination.
C’est pourquoi le passé peut, tout comme le présent, susciter parfois
en nous quelque angoisse, par exemple dans le cas du {\it regret} ou du
{\it remords}. Toutefois, en reculant dans le temps, il perd son caractère
affectif. Le passé {\it très lointain} ne provoque plus guère en nous de
réaction : nous nous en désintéressons ; c’est du passé {\it mort.}

{\it C.} Quant à \textbf{\textit {l'avenir}} ou au \textbf{\textit {futur}}, il y a lieu d’y distinguer aussi le
{\it futur immédiat} qui n’est guère que le prolongement de l’action ou du
désir présents, et le {\it futur lointain}. C’est de la représentation de celui-ci
qu'il a été question au \S 139. Ici l'esprit se sent beaucoup plus {\it libre}
que dans la représentation du passé. Ce n’est pas à dire toutefois,
nous l'avons déjà fait observer, que le futur soit le pur {\it imaginaire}.
« Le futur, écrivait Guyau, à l’origine, c’est le {\it devant être}, c’est ce que
je n’ai pas et ce dont j'ai désir ou besoin, c’est ce que je travaille à
posséder ; comme le présent se ramène à l’activité consciente et jouissant
de soi, le futur se ramène à l’activité tendant vers autre chose,
cherchant ce qui lui manque... L'{\it avenir}, ce n’est pas {\it ce qui vient vers
nous}, mais {\it ce vers quoi nous allons}. » On peut même dire, avec P. Burgelin,
qu'il y a une « primauté de l'avenir », en ce sens que la pensée
va au rebours du temps, qu’elle part de l'avenir et que c'est cette
pensée de l’avenir qui donne son sens au présent (et même au passé).
C’est pourquoi il n’est pas paradoxal de soutenir, avec P. Janet, que
« le futur est la notion qui se rapproche le plus des réalités ordinaires »,
car « il implique une action que nous ferons, que nous désirons faire ».
Mais ici une seule conduite réussit : c’est celle de l’{\it attente}. Ajoutons
que, comme il y a un passé très lointain, il y a aussi un futur {\it très
lointain}, trop lointain (à nos yeux) pour qu’il soit autre chose qu’un
pur {\it savoir} et qu’il suscite en nous, même si ce savoir est une certitude,
une {\it conduite} appropriée. Quand nous disons : « Je mourrai un jour »,
notre attitude d'esprit n’est pas du tout la même que celle d’un malade
qui dit : « Je vais mourir. »

Ces considérations suffisent à montrer que ces trois moments du
temps sont bien plus des constructions de l'esprit que des données
immédiates, — des constructions qui sont étroitement liées aux
%228
modalités de notre action et de notre représentation du monde. C'est
pourquoi leurs limites sont fort incertaines chez l’enfant. C’est pourquoi
aussi elles s’effacent chez les malades mentaux.

\section{Les troubles de la perception du temps}% 151.
On a vu
(\S 148) que la perception de la durée est liée à certains {\it rythmes physiologiques}.
Il n’est donc pas étonnant que certains malades affligés
d’anesthésie viscérale n’aient « plus le sentiment de la fuite du temps :
le jour, ils dépendent uniquement de l’horloge et, lorsqu'ils s’éveillent
le matin, ils n’ont pas conscience d’avoir dormi » (Titchener). —
Certaines « extases » pathologiques
({\scriptsize Qu'il ne faut pas confondre avec l'extase des mystiques : voir là-dessus H. DeLacroix,
{\it Les grands mystiques chrétiens}, Alcan, 1908.})
s’accompagnent d’une sorte d’\textbf{\textit {arrêt
du temps}} : Rousseau, à propos de ses rêveries au bord du lac de
Bienne, nous décrit cet état de béatitude où l’âme n’a plus besoin
« de rappeler le passé ni d’enjamber sur l’avenir, où le temps n’est
rien pour elle, où le présent dure toujours sans marquer néanmoins
sa durée ni aucune trace de succession ». — Sous l’influence des narcotiques,
il se produit parfois un \textbf{\textit {allongement démesuré du temps.}}

\vspace{0.24cm}
{\footnotesize Th. Gautier, dans l’auto-observation rapportée au \S 138, déclare qu'à
son calcul, l’état éprouvé sous l’action du haschich « dura environ trois
cents ans, car les sensations s’y succédaient tellement nombreuses et pressées
que l'appréciation réelle du temps était impossible ». Th. de Quincey
dans ses Confessions d'un mangeur d'opium, signale le même phénomène
à la fois pour l’espace et pour le temps : « L'espace s’amplifiait, tendait vers
un infini inexprimable. Pourtant, j'éprouvais une inquiétude moindre
que devant l'accroissement extraordinaire de la durée : parfois je croyais
avoir vécu soixante-dix ans, cent ans en une seule nuit. Que dis-je? il
m'arrivait d’éprouver pendant ces quelques heures des sensations dont le
cours s’étendait sur un millénaire. » Moreau de Tours avait signalé dans
son étude sur le haschich (1845) cet allongement du temps qui fait que le
commencement d’une phrase prononcée semble déjà prodigieusement loin
alors qu’elle n’est même pas terminée, et il avait attribué ce fait à une
désagrégation mentale qui empêche le sujet de « vivre dans le présent » et
d’être attentif aux objets présentant « un intérêt actuel ».}
\vspace{0.31cm}

Un phénomène analogue s’observe chez les {\it schizophrènes}. Une
malade observée par le D$^\text{r}$ Vinchon est convaincue, au bout de
quelques secondes, que toute une minute s’est écoulée. Elle s'ennuie
et déclare : « Ces messieurs font des jours et des nuits à n’en plus
finir. » Chez ces malades, {\it l’activité est nulle}, en dehors des besognes
automatiques : « Ils rêvent et ne vont pas plus loin que jouer la comédie
de leur rêve... Pour comprendre leur ennui, imaginons que notre
vie n’est plus faite que des rêvasseries de nos heures de malaise »
%229
(Vinchon)({\scriptsize Même à l’état normal il peut y avoir désaccord
entre le temps vécu et le temps le l'action : « Tantôt le temps du moi
semble marcher plus vite que le temps du monde, nous avons l'impression
que le temps s'écoule rapidement, la vie nous sourit et nous sommes joyeux,
tantôt au contraire le temps du moi paraît retarder sur celui du monde, le
temps alors s'éternise, nous sommes moroses et l'ennui s'empare de nous »
(Minkowski)}). — Dans les périodes d’{\it agitation}, le temps semble
\textbf{\textit {s’accélérer.}}
« Le temps passe comme un rêve, dit une malade de Janet. Il
marche cinquante fois plus vite qu’à l’ordinaire. »

En réalité, tous ces troubles sont équivalents : ils marquent une
\textbf{\textit {déstructuration}} du temps, analogue à celle de l’espace (\S 96).
Un psychasthénique déclare : « Quand on m'interroge sur la durée, je
réponds au hasard : des minutes ou des heures, comme on voudra. »
De là, chez ces malades, une \textbf{\textit {confusion
des moments du temps.}} Une
jeune fille psychasthénique ne comprend plus le sens des mots {\it hier,
aujourd’hui, demain} (Janet). La schizophrène du D$^\text{r}$ Vinchon « vit
dans le même temps le présent, le passé et l'avenir ». Plusieurs se
déclarent « immortelles ». Certaines racontent leur passé comme une
suite, d’ailleurs éternelle, de morts et de résurrections, délire qui
s'associe chez elles « à l’imprécision de l’orientation dans le temps ».

\vspace{0.24cm}
{\footnotesize Le sens de l’{\it avenir} est troublé comme celui du passé : « L'avenir est devenu
pour moi un trou noir où il n’y a rien. Je ne peux plus rien espérer, puisque
je ne peux plus rien attendre. » « Quand elle parle de ce que nous appelons
l'avenir, dit Janet à propos d’une extatique de la Salpêtrière, elle nous
paraît faire sans cesse des prophéties, mais elle ne distingue pas comme
nous ce genre de réalité que nous appelons l’avenir » : pour elle, l’avenir se
situe sur le même plan que le passé. — Des trois moments du temps, « {\it la
notion du présent est celle qui est le plus souvent troublée, celle qui disparaît
le plus aisément} ». Dès qu’il y a des troubles de l’activité réfléchie, les malades
ne semblent plus s'intéresser au présent : « Le présent n’a plus pour eux la
même importance que pour nous, ils ne peuvent pas plus appréhender le
présent que le réel » (Janet).}
\vspace{0.31cm}

Tous ces troubles montrent, à l’évidence : 1° comment le sentiment
même de la {\it durée} s’altère avec la notion du {\it temps} structuré ; —
2° comment celle-ci est liée, comme celle de l’espace (\S 93), à notre
{\it action} sur le monde extérieur et à la notion du {\it réel}.

\section{Le concept du temps}% 152.
La notion du {\it temps} structuré, le
\textbf{\textit {concept}} du temps diffère du sentiment de la
{\it durée} comme le concept
de l’espace diffère de l’étendue perceptive (\S 97). Celle-ci est {\it concrète} :
elle n’est rien d’autre que l’écoulement même de nos états de conscience
en nous. Le temps est {\it abstrait} : c’est un cadre vide, un ordre,
un système de rapports. — La durée est {\it hétérogène} : continue, elle
est cependant disparate et s’écoule tantôt vite, tantôt lentement. Le
%230
temps est {\it homogène} et uniforme, constitué de parties identiques les
unes aux autres. — La durée est {\it qualitative} et ne peut guère se
mesurer. Le temps est {\it quantitatif} et mesurable, et il est aussi
« essentiellement
relatif : si tous les phénomènes se ralentissaient et s’il en
était de même de la marche de nos horloges, nous ne nous en apercevrions
pas » (Poincaré). — Enfin la durée est purement {\it intérieure} et
{\it subjective}, tandis que le temps a quelque chose d'{\it objectif}.

{\it A.} Ici comme pour l’espace, Kant avait admis que ce concept est
une pure « forme » \textbf{\textit {a priori}},
c’est-à-dire préexistant dans l'esprit à
tout donné concret. Mais on verra que ce qu’il avait pris ainsi pour une
forme {\it universelle} et {\it immuable} n’est qu’un des multiples instruments
{\it construits} par l'esprit pour la rationalisation du réel (\S 153).

{\it B.} Cette construction s'effectue, comme l’a montré l’étude génétique
de la notion de temps chez l’{\it enfant} (\S 149), {\it à partir} du sentiment
confus d’une durée non structurée. Lévy-Bruhl a montré que, chez
les {\it primitifs}, la représentation du temps conserve, à de certains égards
du moins, quelque chose de ce caractère : « Elle se rapproche plutôt
d’un sentiment subjectif de la durée, non sans quelque analogie avec
celui qui a été décrit par Bergson » et, par suite, « le temps n’a pas
pour eux les divisions qu’il a pour nous ». Mais un temps aussi amorphe
ne pouvait suffire à une action ordonnée et à une représentation
rationnelle du monde. « Le temps psychologique, dit Poincaré, est
impuissant à classer deux phénomènes qui ont pour théâtre deux consciences
différentes ou {\it a fortiori} deux phénomènes physiques. »

{\it C.} Selon Bergson, la transformation de la durée en temps homogène
et mesurable résulterait de « l’intrusion de l’idée d’{\it espace} dans
le domaine de la conscience pure... Familiarisés avec cette dernière
idée, obsédés même par elle, nous l’introduisons à notre insu dans
notre représentation de la succession pure; nous juxtaposons nos
états de conscience de manière à les apercevoir simultanément ; bref,
nous projetons le temps dans l’espace », et nous obtenons ainsi le
« concept bâtard » du temps homogène. Bâtard, précise Bergson,
parce que « par un côté, c’est un état de conscience ; par un autre,
c’est une pellicule superficielle de matière où coïncideraient le sentant
et le senti. À chaque moment de notre vie intérieure correspond ainsi
un moment de notre corps et de toute la matière environnante :
...cette matière semble alors participer de notre durée consciente », et
« graduellement, nous étendons cette durée à l’ensemble du monde
matériel ». — Cette interprétation, bien qu’acceptée par de nombreux
auteurs, ne nous paraît nullement s’imposer. Sans doute, le temps,
on l’a vu plus haut, est en relation {\it avec notre action} et celle-ci se
déroule le plus souvent dans l’espace. Mais il n’y a pas là de liaison
%231
nécessaire : l’action réfléchie de la volonté est au moins aussi « spirituelle »
que le laisser-aller de la « mélodie intérieure » de la durée.
D'ailleurs, le mot « juxtaposition » est ici purement métaphorique :
il signifie seulement {\it simultanéité, coexistence} ; or, il n’est pas évident
que ces notions soient équivalentes à celle d’espace (\S 90). Il paraît
arbitraire enfin d’affirmer que le temps attribué aux phénomènes
matériels implique une participation à la durée consciente : ne faut-il
pas y voir plus simplement un ordre objectif de séquences causales,
dans lequel la conscience n’a absolument rien à voir? — En réalité,
le concept de temps n’est pas plus « bâtard » que celui d’espace : il se
construit parallèlement à celui-ci et de la même façon. De même que
l'esprit s’élève de l’étendue perceptive à l’espace abstrait (\S 97), il
construit de même, en partant de la durée concrète, le temps homogène
et mesurable. On peut dire avec René Berthelot que « Bergson
transforme à tort en une opposition entre l’espace et le temps ce qui
constitue en réalité une opposition entre deux formes corrélatives
soit de l’espace, soit du temps ». L'opposition de la durée vécue et du
temps conceptuel n’est pas l’opposition du spirituel et de l’extensif,
c’est-à-dire du matériel (voir \S 345) ; elle « relève de l’opposition
plus générale entre le sensible et l’intelligible » (R. Berthelot). Le diagramme
suivant résume cette discussion.

\vspace{0.4cm}
\begin{center}
\textsc{Genèse du temps conceptuel}
\end{center}

\vspace{0.4cm}
\begin{minipage}[c]{.35\linewidth}
\begin{center} \textsc{selon Bergson}
\vspace{0.7cm}

{\it Esprit} \hfill {\it Matière}
\vspace{0.2cm}

Durée concrète \hfill Espace

\vspace{0.2cm}
\hspace{1.7cm} $ \searrow \hfill \swarrow $ \hspace{0.7cm}
\vspace{0.2cm}

%\vspace{0.7cm}
Temps conceptuel
\end{center}
\end{minipage}
\hfill
\begin{minipage}[c]{.45\linewidth}
\begin{center} \textsc{selon Berthelot}
\vspace{0.7cm}

{\it Sensible} \hfill {\it Intelligible}

\vspace{0.3cm}
Étendue concrète $\rightarrow$ Espace conceptuel

\vspace{0.7cm}
Durée concrète $\rightarrow$ Temps conceptuel
\end{center}
\end{minipage}

\section{Le problème épistémologique}% 153.
Prolongeant les problèmes
psychologiques, le même problème épistémologique se pose
ici qu’à propos du concept d’espace. De même que l’espace euclidien
a été considéré longtemps comme le seul concevable, la seule notion
admissible du temps paraissait être le \textbf{\textit {temps newtonien}},
ce « temps
absolu, vrai et mathématique, comme dit Newton lui-même, considéré
en soi et sans relation aux choses externes », qui « coule uniformément,
%232
soit que les mouvements soient prompts, soit qu’ils soient
lents » et qui coulerait de même « quand il n’y aurait aucun mouvement ».
C’est le temps {\it scientifique}, celui de la Mécanique et de la
Physique classiques ; c’est aussi le temps {\it pratique}, celui de nos horloges
et de nos indicateurs de chemins de fer.

{\it A.} On a déjà vu (\S98 B) que la \textbf{\textit {Physique}}
contemporaine, avec
ses extensions du côté de la Microphysique et de la Théorie de la
relativité, a été amenée à abandonner cette notion trop simple du
temps newtonien, pour celle d’un \textbf{\textit {espace-temps}},
c’est-à-dire d’un milieu quadridimensionnel, d’un \textbf{\textit {champ}}
où les coordonnées de l’espace et du temps sont solidaires :
le temps est un \textbf{\textit {temps local}} et l’on
verra, à propos des Sciences expérimentales (chap. XXI), les conséquences,
en apparence paradoxales, de cette conception.

{\it B.} Mais la Physique n’est pas seule à exiger une refonte du concept
traditionnel du temps. Il existe, comme l’ont montré différentes
recherches, un \textbf{\textit {temps biologique}}
qui est loin d’être le temps uniforme
de la conception classique et qui est lié aux {\it rythmes organiques}.
Les études de Louis Lapicque (1866-1952) sur la chronaxie ont établi
que la vitesse avec laquelle se déroulent les divers processus d’excitation
d’un nerf ou d’un muscle, diffère d’un tissu à l’autre ; celles
de Lecomte du Noüy sur la cicatrisation des plaies (\S 282),
que la vitesse de cicatrisation est incomparablement plus
grande chez les jeunes que chez les gens âgés : une plaie qui se cicatrise
en moins de vingt jours chez un enfant de dix ans, exige cent
jours chez un homme de soixante ans. Ainsi « les jeunes et les vieux,
réunis dans le même espace, vivent dans des univers séparés où la
valeur du temps est profondément différente » (Lecomte du Noüy).

{\it C.} Plus complexe encore est le \textbf{\textit {temps social.}}
Durkheim a soutenu
ici la même thèse qu’à propos de l’espace. Le temps, dit-il, n’est pas
{\it mon temps} : « C’est le temps tel qu’il est objectivement pensé par
tous les hommes d’une même civilisation. » C’est « un cadre abstrait
et impersonnel ». Il vient donc, non de l'individu, mais de la société.
Halbwachs est allé presque jusqu’à nier l'existence d’un temps
subjectif comme celle d’une mémoire individuelle (\S 131). Thèse
excessive, car la durée intérieure est le point de départ de la construction
du temps objectif. Mais les conditions sociales de notre existence
contribuent effectivement dans une large mesure à structurer notre
notion du temps comme celle de l’espace.

\vspace{0.24cm}
{\footnotesize Dans la « mentalité primitive », la représentation du temps conserve, on
l'a déjà dit (\S 152 B), quelque chose de l’\textsf{\textit {indifférenciation}} de la durée pure.
« Les langues mélanésiennes, dit M. Leenhardt, n'expriment pas le temps ;
le temps reste indifférencié. » Pour les indigènes, un fait comme l'arrivée
%233
des Blancs en Nouvelle-Calédonie ne fait pas partie d’un passé historique ;
car toutes nos formulations du temps ne correspondent à rien chez eux.
Comme l'avaient montré Hubert et Mauss, ce temps des primitifs n’est
pas « une quantité pure, homogène dans toutes ses parties et exactement
mesurable ». C’est un \textsf{\textit {temps concret}} : « Les parties du temps ne sont pas indifférentes
aux choses qui peuvent s’y passer. » D'où sa « nature {\it qualitative} » :
il est « composé de parties discontinues, hétérogènes ». Lévy-Bruhl avait
cité l'exemple de ces nègres de Guinée qui divisent le temps en « temps
heureux » et en « temps malheureux » et il l'avait rapproché de la distinction
classique des périodes {\it fastes} et {\it néfastes}. Presque partout en effet, les divisions
du temps \textsf{\textit {correspondent à celles de la vie sociale et spécialement religieuse}},
marquées par la périodicité des {\it fêtes}, des {\it rites}, des {\it cérémonies}. Ce sont ces fêtes
qui fournissent les points de repère nécessaires à la mesure du temps :
« Le calendrier est l’ordre de la périodicité des rites » (Hubert et Mauss).
Certes, on tient compte de certaines périodicités naturelles : cours des astres,
saisons, etc. Mais, à cet ordre naturel, la société superpose un ordre qui lui
est propre : le calendrier naturel « n’a d’existence réelle, dit Leenhardt,
qu’appuyé sur un rituel qui lui donne forme ». — C’est à peu près dans les
mêmes termes que M. Granet parle de la représentation du temps dans la
Chine classique. Pas plus qu’ils n’ont conçu un espace homogène, les philosophes
chinois eux-mêmes n’ont imaginé une durée uniforme, toujours
semblable à elle-même. « Chaque période du temps est marquée par les
attributs propres à une saison de l’année, à une heure du jour. » Mais ici
encore, les saisons n’ont fourni que des emblèmes sensibles : « La représentation
chinoise du temps se confond avec celle d’un ordre liturgique. »

Ce caractère {\it qualitatif} et {\it ritualisé} du temps s’est conservé mieux que
partout ailleurs dans le \textbf{\textit {temps religieux}}. Il y a des « saisons liturgiques »
(dom Cabrol), des temps de pénitence, de deuil, d'attente (Avent), d’allégresse
et de fête, ayant chacun leur couleur affective, qu’exprime celle des
ornements sacerdotaux. La semaine est d'institution proprement religieuse
et, dans la journée même, toutes les religions ont leurs heures liturgiques
{\scriptsize (Citons ici ce beau passage de Gabriel Le Bras dans son {\it Introduction à l'histoire
de la pratique religieuse} : « Le crépuscule, dans son tour du monde, anime l'un après
l’autre toutes les pagodes, tous les minarets et tous les clochers. De l’Extrême-Orient
aux lisières de l'Occident, la chaîne est ininterrompue des hymnes lucernaires. Dissonances
des bonzes, psalmodie gutturale des lamas, vocalises des muezzins, vêpres des
monastères, polyphonies des chapelles d'Oxford, ces chants mélancoliques à la nuit
tombante nous ont ému sous tous les cieux. Ils sont un acte continu de la {\it laus perennis}
{\scriptsize (louange perpétuelle)}»)}.

Mais, plus généralement, chaque groupe social a sa mémoire et, par suite
son temps. Il y a ainsi \textsf{\textit {« multiplicité et hétérogénéité des durées collectives »}}
(Halbwachs). Nous avons déjà remarqué (\S 131 B) que les différentes ères
propres aux diverses civilisations n’ont pas \textbf{\textit {le même point de départ}}. Mais, si
nous y regardons bien, nous retrouvons la même diversité dans notre propre
vie sociale : ni l’année officielle, ni l’année scolaire, ni l’année religieuse ne
commencent à la même date ; l’année paysanne se règle sur le cours des
travaux agricoles, tandis que l’année commerciale ou industrielle suit les
fluctuations saisonnières du mouvement des affaires. — Surtout, ces divers
temps ne se déroulent pas au \textsf{\textit {même rythme}}, On vit plus lentement en milieu
{\it rural} ou même {\it artisanal} qu’en milieu {\it urbain} et {\it industriel}. La {\it technique
moderne} nous a fait passer, comme a dit A. Koyré, « du monde de là peu
près à l’univers de la précision » et, comme a dit un autre, au « temps harcelant ».
Proust l'avait déjà noté : « Depuis qu’il existe des chemins de fer,
%234
la nécessité de ne pas manquer le train nous a appris à tenir compte des
minutes alors que chez les anciens Romains dont la vie était moins pressée,
la notion non pas de minutes, mais même d'heures fixes existait à peine. »
Les Orientaux sourient, remarque G. Friedmann, lorsqu'ils entendent
l’Européen prononcer sans cesse le mot {\it vite} : à celui-ci, il faut « un temps
exactement mesuré » pour tous ses actes. « l’Asiatique veut du temps pour
ne rien faire, jouir de sa respiration ». Enfin les différentes {\it périodes de
l'histoire} n’ont pas non plus le même rythme. Michelet remarquait, dès
1872, que « l'allure du temps a tout à fait changé : il a doublé le pas d’une
manière étrange », et l’on a pu parler, de nos jours, d’une « accélération de
l'histoire » (D. Halévy). Sans doute sous l'influence des inventions, les
événements, les transformations du monde s'effectuent aujourd’hui à une
vitesse inconnue au moyen âge ou même au {\footnotesize XVII}$^\text{e}$ siècle.

Certains sociologues se sont appuyés sur ces faits pour distinguer, ainsi
que l'avait déjà suggéré Halbwachs, une pluralité de temps « socio-culturels »
ainsi que dit le sociologue américain P. A. Sorokin, chacun possédant sa
structure propre, à la fois qualitative et discontinue. On est même allé
jusqu’à fragmenter le temps social en une multitude de concepts différents.
C'est ainsi que G. Gurvitch ne distingue pas moins de huit formes du
temps social : 1° le « temps de longue durée et au ralenti », tel celui des
milieux paysans ; 2° le « temps {\it trompe-l'œil} » qui comporte des crises soudaines,
tel celui des milieux urbains, des foules politiques ; 3° le « temps de
battements irréguliers entre l'apparition et la disparition des rythmes »,
celui des sociétés en transformation, particulièrement au point de vue
technique ; 4° le « temps cyclique », celui des groupes mystiques, qui
domine dans les sociétés archaïques ; 5° le « temps en retard sur lui-même »
celui des groupements clos, des corporations fermées, dont les symboles
sociaux, le droit notamment, sont déjà dépassés au moment où ils se cristallisent ;
6° le « temps d’alternance entre avance et retard », celui des
communautés qui secouent leur tendance à l’immobilisme ; 7° le « temps en
avance sur lui-même », où la discontinuité triomphe : celui des révolutions,
des classes en lutte ; 8° enfin, le « temps explosif », celui des communautés
créatrices, dont les créations sont immédiatement dépassées. Ainsi « chaque
réalité sociale sécrète son temps et ses échelles de temps ». — Mais la subtilité
de ces distinctions ne fait-elle pas évanouir la notion de temps elle-même ?
En tous cas, ce « temps caméléon », comme l’a nommé l'historien F. Braudel
est inutilisable aussi bien pour l'historien que pour le sociologue.}
\vspace{0.31cm}

{\it D.} Il n’est pas jusqu’au \textbf{\textit {temps psychologique}} qui ne comporte
bien des variations, individuelles et autres. Il y a pour chacun de nous,
comme a dit G. Poulet, une « distance intérieure », un champ spirituel
qui, pour ce qui est aboli à nos yeux ou pour ce que nous attendons,
s'étend plus ou moins loin dans le passé ou dans l’avenir. Un Marcel
Proust, par exemple, a vécu dans son passé plus que dans le présent,
tout en le réinventant, de manière à en faire « le temps retrouvé ».
D'autre part, comme l’a observé Jean Wahl, on pourrait distinguer
ici aussi bien des formes du temps : le temps de l'ennui, le temps de
la colère, le temps de l’hésitation, le temps du désir, etc.
%235

\section{Le problème métaphysique}% 154.
Puisqu’il n’y a pas de
temps {\it unique}, l’essence et la réalité même du temps semblent, comme
celles de l’espace (\S 99), devoir être mises en question.

{\it A.} Descartes a soutenu à propos du temps la même conception
\textbf{\textit {réaliste}} que pour l’espace. Il y a, dit-il, « des attributs qui appartiennent
aux choses » et d’autres « qui dépendent de notre pensée ».
Certes, si nous songeons au temps mesuré, nous dirons qu’il « n’est
rien qu’une certaine façon dont nous pensons à la durée » des choses
(voir {\it Exercice 1}). Mais, en dehors de ce temps mesuré, il y a une
« véritable durée des choses » qui est indépendante de ce qui s’y
passe : « Nous ne concevons point que la durée des choses qui sont
mues soit autre que celle des choses qui ne le sont point » et, « si deux
corps sont mus pendant une heure, l’un vite et l’autre lentement,
nous ne comptons pas plus de temps en l’un qu’en l’autre ». —
Comme on l’a vu au début du \S 153, cette conception du temps sera
reprise par Newton qui fera du temps un véritable {\it absolu}.

{\it B.} Leibniz et Kant soutiennent au contraire la pure \textbf{\textit {idéalité}} du
temps comme de l’espace. « Les instants hors des choses, dit Leibniz,
ne sont rien et ne consistent que dans leur ordre successif. » Le temps
n’est rien d’autre que cet ordre, d’ailleurs éternel et « fondé en Dieu » :
l’{\it ordre des successions possibles}. — Pour Kant, « le temps n’est pas
quelque chose d’objectif ni de réel ; il n’est ni une substance ni un
accident ni une relation, mais une condition subjective que la nature
de l'esprit rend nécessaire pour coordonner tous les objets sensibles
selon une loi déterminée ». C’est donc une {\it forme} de l’expérience
interne, mais cette forme est {\it a priori}. Car les « axiomes du temps » tels
que : « le temps n’a qu’une dimension ; des temps différents sont non
pas simultanés, mais successifs », ne peuvent être tirés de l’expérience,
qui « ne saurait donner ni universalité rigoureuse ni certitude apodictique »
comme c’est le cas pour ces principes nécessaires.

{\it C.} La \textbf{\textit {théorie bergsonienne}} de la durée constitue, en un sens, un
retour au {\it réalisme}. Aux yeux de Bergson en effet, la durée ne fait
qu'un avec le réel lui-même : elle est d’ailleurs, dès les {\it Données immédiates}
comme plus tard dans l’{\it Introduction à la Métaphysique}, qualifiée
de « réelle » en même temps que de « concrète ». Le réel, c’est « le
vécu, le concret ». Et {\it La Pensée et le Mouvant} précise : « Comment ne
pas voir que l’essence de la durée est de couler ?... Ce qui est réel,
c’est le flux, c’est la continuité de transition, c’est le changement lui-même.
Ce changement est indivisible, il est substantiel, »

{\it D.} En harmonie avec les conclusions des chapitres VI et VIII, nous
pensons qu’on peut dire, en effet, que la durée est, en un sens {\it réelle.}
Nous remarquerons cependant : 1° qu’il n’y a pas lieu de privilégier
%236
cette notion de {\it durée} qui exprime la temporalité de la conscience,
par rapport aux autres formes du temps qui expriment bien, elles
aussi, des réalités : aux yeux des physiciens, l’{\it espace-temps} auquel il a
été fait illusion \S 98 B et 153, est bien une réalité physique ; le temps
physiologique et le temps social ont bien aussi leur réalité et leurs
caractères propres ; — 2° qu’il n’y a pas lieu surtout d’opposer la durée
comme réalité spirituelle au temps abstrait comme concept artificiel :
la véritable opposition est ici, comme il a été dit au \S 152, entre le
{\it sensible} ou l’{\it intuitif} et l'{\it intelligible}. En présence du devenir universel
l’homme s’est efforcé de construire de ce devenir une représentation
intelligible dont le temps est l'instrument ; mais il s’aperçoit
aujourd’hui que ce n’est pas {\it un} concept, mais une {\it pluralité} de concepts :
temps newtonien, temps physique, temps physiologique, temps
social, temps psychologique qu’il lui faut inventer pour les adapter
aux différentes « régions » du réel.

\section{Signification du temps}% 155.
Un dernier problème métaphysique
sur lequel nous ne donnerons que quelques indications (qu’on
pourra compléter à l'aide de la bibliographie spéciale indiquée aux
{\it Lectures}), est celui de la {\it signification} du temps. Le temps semble
notre grand ennemi : les poètes ont souvent exprimé la mélancolie
de l’homme devant la fuite du temps qui emporte ce qu’il a de plus
cher et finit par l’emporter lui-même. Toute notre vie est une lutte
contre le temps, qu’il s'agisse soit de faire revivre un passé à tout
jamais disparu, soit d’anticiper sur un avenir qu’appelle notre impatience
et qui nous déçoit lorsqu'il se fait présent. Aussi les philosophes
se sont-ils efforcés de donner au temps un sens. Dans l’Antiquité,
Platon, fasciné par la hantise des Idées éternelles, ne voit dans le
temps que « l’image mobile de l'éternité ». Aristote se borne à le
définir comme « l'aspect par où le mouvement [c’est-à-dire le devenir]
comporte nombre ». Plotin reprend le point de vue platonicien : le
temps marque une déchéance de l’être et l’âme doit s’en affranchir
pour s’absorber dans l’éternel par l’extase. Mais saint Augustin, tout
en s'inspirant aussi de Platon, se refuse à cette absorption et pense
que c’est dans le temps que chaque individu doit faire son salut. —
Ces deux points de vue revivent chez les modernes. Le point de vue
pessimiste se retrouve chez Kant pour lequel le temps n’est, en somme,
qu’une faiblesse de l’esprit humain qui l'empêche d’avoir accès à
l'absolu et à l’éternel. Hegel au contraire nous présente une conception
{\it dialectique} du temps, il s’efforce de découvrir « dans et pour la
conscience elle-même, les moments, les étapes, les actes spirituels par
lesquels se constitue le concept du temps », d’un temps qui n’est pas
%237
notion abstraite et vide, mais qui « est lui-même esprit et concept » et
par conséquent « enrichissement, vie, victoire » (A. Koyré). — Bien
proche de cette conception est celle de G. Bachelard qui critique
la description bergsonienne de la durée vécue (\S 148 B). Selon lui,
il y a « une pluralité de durées qui n’ont ni le même rythme, ni la
même solidité d’enchaînement, ni la même puissance de continu ».
Il y a un {\it temps voulu} distinct du {\it temps vécu} et « le fil du temps est
couvert de nœuds ». C’est précisément cette discontinuité du temps,
laquelle en fait une dialectique de l’instant et de l’intervalle, un
rythme des {\it oui} et des {\it non}, qui nous permet de penser et de vouloir
et qui fonde notre liberté : l'instant est créateur et l'intervalle est
reprise et possibilité d’acte. — Pour J. Guitton, « le temps de
l’homme, pour être saisi dans sa vérité, doit être détaché à la fois du
temps éternel et du temps vital ». Le temps est « le lieu de la croissance
spirituelle » qui permet à chacun de nous de devenir ce qu’il est et
d’être ce qu’il a. Seul son accident est de couler ; mais « son essence
est de conserver », de {\it nous} conserver pour nous préparer à l'éternité.
— F. Alquié n’est pas aussi optimiste. À ses yeux, le « désir d’éternité »
n’est qu’une illusion méprisable, si l’on entend par là une éternité
affective. Mais, par delà celle-ci, il y a l’éternité spirituelle qui
est celle des lois et des valeurs. Notre mission est d’opérer la réalisation
de ces valeurs dans le temps et de permettre ainsi à l’éternel
de descendre, à travers l’action, dans le devenir. Quoi qu’il en soit,
l'expérience du temps est toujours douloureuse. — L. Lavelle
enfin se refuse à opposer le temps à l’éternité. De celle-ci même, le
temps n’est pas absent, mais il s’y trouve comme transfiguré et nous
y retrouvons « une liberté qui veut éternellement la vie qu’elle
s’est faite ». — A travers ces diversités d'interprétation, on retrouve
une direction commune. « Les anciens prétendaient nous libérer du
temps par l’exercice de la pensée ; ils nous proposaient une conversion
mentale qui ne laissait subsister du monde que des rapports nécessaires
entre des essences devant lesquelles la conscience personnelle devait
s’effacer. Pour les modernes, le réel est une histoire fondée sur des
actes de liberté ; la personnalité se forme peu à peu dans le temps ; et
l'éternité, au lieu de la détruire, l’accomplit » (Lavelle).

\section{Sujets de travaux}% SUJETS DE TRAVAUX

{\bf Exercices.} — 1. {\it Étudiez les divers emplois et définitions des mots} temps {\it et}
durée {\it dans les phrases suivantes} : « Je laisse au philosophe et aux gens de loisir
À mesurer le temps par mois et par journées : Je compte quant à moi le
temps par le désir » (Desportes), « Le temps que nous distinguons de la
durée prise en général et que nous disons être le nombre du mouvement, n’est
%238
rien qu’une certaine façon dont nous pensons à cette durée » (Descartes),
« Un temps est une partie de durée mesurée » (D. de Tracy), « Ce temps
qui les donna [nos beaux jours], ce temps qui les efface Ne nous les rendra
plus » (Lamartine), « Le temps a toujours été conçu comme une espèce
de changement qui se retrouve dans tous les autres changements »
(Fouillée), « Le temps est invention ou il n’est rien » (Bergson),
« J'éprouvais un sentiment de fatigue à sentir que tout ce temps si long
non seulement avait sans une interruption été vécu, pensé, sécrété par
moi, qu'il était ma vie, mais encore que j'avais à toute minute à le
maintenir attaché à moi » (Proust), « L'art délicat de la durée, le temps, sa
distribution et son régime, sa dépense à des choses bien choisies, était une
des grandes recherches de M. Teste » (Valéry), « Derrière le {\it déjà} et le {\it pas
encore}, nous trouverons le souvenir et l'attente, et derrière le souvenir et
l'attente, le remords et le regret, le désir et la crainte. Nous atteignons ce
qu’on peut appeler le fond existentiel du temps » (Wahl), « L'étude du
concept de temps est inséparable des lois dans lesquelles le temps intervient »
(H. Mineur, astronome). — 2. {\it Étudiez sur vous-même les fluctuations du
sentiment de la durée dans quelques cas typiques (réverie, attente, travail)}. —
3. {\it Essayez d'expliquer ces deux textes : a.} « L'avenir se comporte comme le
temps dans le temps, le passé comme l’espace dans le temps » (Scheling);
{\it b.} « Le passé et l’avenir du temps en tant qu’existant dans la nature sont
l'espace ; car il est le temps nié » (Hegel). — 4. {\it Expliquer cette réflexion de
M. Pradines} : « On ne peut agir dans le présent seul ; tout acte est un façonnement
de l’avenir par un être qui sent qu’il s’y prolonge. »

{\bf Exposés oraux.} — 1. {\it La durée bergsonienne,} depuis les {\it Données immédiates}
jusqu'aux dernières œuvres. — 2. {\it Le sentiment du temps chez les romantiques}
(voir {\it Le Lac, Olympio,} et {\it cf.} G. Poulet, {\it Études sur le temps humain}, Plon,
1951, tome II). — 3. {\it Le temps mélanésien} d'après Leenhardt, {\it Do Kamo},
Gallimard, 1947, chap. VI. — 4. {\it Le temps chez Kant} (voir Le Senne,
{\it Introd. à la Philosophie}, p. 101).

{\bf Discussions.} — 1. {\it }Continuité ou discontinuité de la durée psychologique ?,
— 2. {\it }Temps et existence.

{\bf Lectures.} — {\it a.} Bergson, {\it Données immédiates de la conscience}, Alcan
18388 ; et {\it b. Durée et Simultanéité}, Alcan, 1922. — {\it c.} Guyau, {\it La Genèse de
l’idée de temps}, Alcan, 1890. — {\it d.} H. Hubert et M. Mauss, {\it La Représentation
du temps dans la magie et la religion}, dans {\it Mélanges d'histoire des religions},
Alcan, 1909, p.189. — {\it e.} J. Vinchon, {\it L’ Évaluation du temps chez les schizophrène}s,
dans le {\it Journal de Psychologie}, 1920, p. 415. — {\it f.} L. Lévy-Bruhl,
{\it La Mentalité primitive}, Alcan, 1922, chap, II. \S VI. — {\it g.} H. Piéron, {\it Le
Problème psychologique de la perception du temps}, dans l’{\it Année psychologique},
tome XXIV (1923). — {\it h.} P. Janet, {\it L' Évolution de la mémoire et la
notion de temps}, Chahine, 1928, I et III; et {\it i. De l’ Angoisse à l'extase},
Alcan, 1926, tome I, 2° partie, chap. II. — {\it j.} M. Minkowski, {\it Le Temps vécu},
d’Artrey, 1933. — {\it k.} H. Delacroix, {\it La Conscience du Temps}, dans le {\it Nouveau
Traité} de Dumas, Alcan, 1936, tome V, p. 305. — {\it l.} Lecomte du
Noüy, {\it Le Temps et la vie}, Gallimard, 1936. — {\it m.} D. Halévy, {\it Essai sur
l'accélération de l'histoire}, Self, 1948. — {\it n.} M. Halbwachs, {\it La Mémoire
collective}, P. U. F., 1950, chap. III. — Bibliographie spéciale pour le \S 155 :
{\it o.} G. Bachelard, {\it La Dialectique de la durée}, Boivin, 1936. — {\it p.} Michel
Souriau, {\it Le Temps}, Alcan, 1937. — {\it q.} Jean Guitton, {\it Justification du
%239
temps}, P. U. F., 1941. — {\it r.} F. Alquié, {\it Le Désir d'éternité}, P. U. F., 1944.
— {\it s.} P. Burgelin, {\it L'Homme et le Temps}, Aubier, 1945. — {\it t.} L. Lavelle,
{\it De l'Éternité et du Temps}, Aubier, 1945. — {\it u.} J. Pucelle, {\it Le Temps},
P.U.F., 1955. — {\it v.} P. Fraisse, {\it Psychologie du temps}, P. U. F., 1957.
— {\it w.} G. Gurvitch, {\it Structures sociales et multiplicité des temps}, dans {\it Bull.
Soc. fr. Philos.}, séance du 31 janvier 1959. — {\it x. Le Temps}, n° spécial des
{\it Études philosophiques}, 1962, n° 1. — {\it y.} M. Heidegger, {\it L'être et le temps},
trad. fr., Gallimard, 1964. — {\it z.} F. Gonseth, {\it Le problème du temps}, Dunod,
1964. — {\it aa.} E. A. Lévy-Valensi, {\it Le temps dans la vie psychique},
Flammarion, 1965.

