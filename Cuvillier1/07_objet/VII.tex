\chapter{L'idée d'objet et le réel}
\addtocounter{chapitre}{1}
%
%chapitre VII
%L'IDÉE D'OBJET ET LE RÉEL
%SOMMAIRE
%10 La perception du réel : différents problèmes. {\bf —} 101. Le problème
%psychologique : comment il doit être posé. {\bf —} 102. Le syncrétisme du sujet et de
%l’objet. {\bf —} 103. La constitution du réel. {\bf —} 104. Les critères du réel. {\bf —}
%105. Degrés et formes du réel. {\bf —} 106. Les troubles de la perception du réel. {\bf —}
%107. Le problème épistémologique. {\bf —} 108. Le problème métaphysique.

\section{La perception du réel : différents problèmes}% 100
En étudiant la perception, nous avons {\bf —} par abstraction {\bf —} laissé de
côté un de ses caractères importants : quand nous percevons quelque
chose, ce que nous percevons nous apparaît comme \textbf{\textit {réel}}, c’est-à-dire
comme possédant une existence {\it objective} et {\it permanente}, indépendante
de la perception que nous en avons. Autrement dit, la perception
s'accompagne de l’idée d'\textbf{\textit {objet}} (au sens philosophique du terme),
ou de {\it réalité extérieure}. {\bf —} Plusieurs problèmes peuvent ici se
poser. D’abord, un problème \textbf{\textit {psychologique}} : la perception se
présente-t-elle d'emblée avec ce caractère? la notion d’{\it objet} est-elle
immédiate? Ou bien est-elle acquise ou construite? Dans cette
seconde hypothèse, comment se forme-t-elle ? C’est principalement
ce problème que nous examinerons dans le présent chapitre. Mais
nous aurons aussi à indiquer au moins comment se posent deux autres
problèmes, l’un \textbf{\textit {épistémologique}} concernant les conditions de l’{\it objectivité}
de la connaissance, et tout spécialement de la connaissance
sensible
(\S \thechapitre .8),
%(\S 107),
l’autre \textbf{\textit {ontologique, métaphysique}} concernant l’essence
et peut-être l’existence même de la réalité sensible
(\S \thechapitre .9),
%(\S 108),
{\bf —} problèmes
qui seront examinés plus en détail dans le chapitre suivant.

\section{Le problème psychologique : comment il doit être posé}% 101
Le premier problème peut sembler relativement simple. Il est,
en réalité, difficile à bien poser. Nous distinguerons trois points de
vue possibles,
%139

{\it A}. Du \textbf{\textit {point de vue du sens commun}}, le problème se pose
à peine. Nous avons déjà remarqué (\S 79) que le sens commun s’imagine
que la connaissance sensible est la pure et simple {\it copie} du réel.
Pour lui, le {\it sujet} percevant saisit immédiatement l’{\it objet} perçu dans
sa réalité et son extériorité mêmes. La philosophie de l’« école écossaise »
ne fut guère plus qu’une systématisation de ce point de vue
du sens commun. C’est ainsi que Th. \textsc{Reid} déclare qu'« un seul
argument suffit pour nous démontrer l’existence de l’objet ; c’est
que nous le percevons », et W. \textsc{Hamilton} soutient que, « dans la
perception, nous sommes conscients immédiatement d’un {\it moi} et
d’un {\it non-moi}, connus ensemble et en opposition mutuelle », nous
saisissons « deux existences par une même et indivisible intuition ».
En somme, cette position de la question {\it supprime le problème}. {\bf —}
On verra bientôt que la Phénoménologie revient, de nos jours, par
une voie plus savante, à ce point de vue.

{\it B.} Tout autre fut le \textbf{\textit {point de vue des Philosophes classiques}}.
Ce point de vue résulte d’une prise de conscience du rôle du {\it sujet}
dans la perception, plus ou moins altérée par le point de vue de la
« conscience close » qui fut longtemps celui des philosophes. Dans une
telle perspective, il n’y a pas, comme le voulait Hamilton, de « conscience
du monde extérieur ». La conscience est enfermée en elle-même:
elle n’est pas autre chose que l’intuition (voir \S 59-60) que nous
prenons de nos propres états internes, nullement de quelque chose
d’extérieur à nous. La perception n’échappe pas à cette loi : la sensation,
nous l’avons vu (\S 85), n’est qu’un {\it signe subjectif}, qu’un retentissement,
dans la conscience, de l’action de l’objet extérieur sur nos
organes sensoriels, et ce qu’y ajoute la perception proprement dite,
ce sont des souvenirs (ou tout au moins des images), des jugements, etc.,
c’est-à-dire encore des états internes. C’est ce qui a fait dire à \textsc{Taine}
que la perception est « une hallucination vraie ». Entendons par là
qu’elle est, tout comme une hallucination (\S 110 A), un état {\it purement
subjectif}, mais qui, en pareil cas, est {\it objectivé} par nous à bon droit.
Le problème consisterait alors à expliquer comment nous passons
de cette {\it subjectivité pure} à la notion d'{\it objet}. {\bf —} Mais cette position
de la question n’est pas plus correcte que la précédente. Elle a été
vivement combattue par la Phénoménologie et le Gestaltisme. La
Phénoménologie soutient en effet que la conscience est « intentionnelle »,
toujours {\it tendue vers} un objet : « Toute conscience, dit \textsc{Husserl},
est la conscience {\it de} quelque chose. » J.-P. \textsc{Sartre} écrit dans le même
sens que « la perception, comme la vérité chez Spinoza, est {\it index sui}
{\scriptsize (\textsc{Spinoza} soutient en effet que, par son {\it évidence intrinsèque}, « le vrai est à lui-même
sa propre marque ». Mais ceci ne s'applique chez lui qu'à l’idée « adéquate », c’est-à-dire
claire et distincte. On remarquera que J.-P. \textsc{Sartre} transporte ici cette thèse de l’{\it intelligible} au {\it sensible})}».
% 140

Ainsi, « lorsque je perçois une table, je ne {\it crois} pas à l’existence de
cette table. Je n’ai nul besoin d’y croire, puisqu’elle est là en personne.
Il n’y a pas un acte supplémentaire par lequel, percevant d’ailleurs
cette table, je lui conférerais une existence {\it crue} ou {\it croyable}. Dans
l’acte même de perception, la table se découvre, se dévoile, elle m’est
donnée ». Certains gestaltistes, comme \textsc{Koffka}, ont soutenu de même
que, dans la perception, l’objet est donné non seulement {\it en même
temps} que le sujet, mais même par {\it priorité} puisque la conscience est
essentiellement tournée vers lui.

Il y a une part de vrai dans ces critiques, mais aussi une large
part d’équivoque. Ce que nous en retiendrons, c’est que, si nos sensations
ou nos perceptions sont bien, en un sens, des {\it états internes}, elles
ne sont telles qu’{\it aux yeux du philosophe} qui a pris conscience du rôle
du sujet percevant, mais elles ne sont {\it nullement saisies comme telles}
par la conscience spontanée. Le sens commun ne songe même pas
que nous ne connaissons les objets extérieurs que par les impressions
subjectives qu’ils font sur notre conscience. {\it Psychologiquement}, la
subjectivité pure n’est pas plus une donnée immédiate que l’opposition
ou la relation entre le sujet et l’objet. Sur ce point, nous donnerons
donc gain de cause à la Phénoménologie et au Gestaltisme. {\bf —}
Voici cependant où est l’équivoque. Dire que la notion de {\it sujet} n’est
pas immédiate, ce n’est pas nécessairement reporter le bénéfice de
cette immédiateté sur la notion d’{\it objet}. Il y a là une fausse opposition
et un faux problème qui consistent à appliquer à la pensée implicite
(\S 39), des distinctions qui ne sont valables que pour des niveaux
supérieurs (\S 44). En réalité, la notion de {\it l’objet comme tel}, c’est-à-dire
comme constituant une réalité nettement distincte du sujet qui
le perçoit, exige une prise de conscience qui, pas plus que celle du {\it sujet
comme tel}, ne peut être primitive. On verra, en particulier (\S 106 B),
que les cas pathologiques, notamment celui des {\it psychasthéniques}
étudiés par P. \textsc{Janet}, donnent un démenti indiscutable à l’assertion
de Sartre selon laquelle la « croyance » au réel ne constituerait pas
un acte psychologique distinct.

C. Du \textbf{\textit {point de vue psychologique}}, le seul point de départ correct
est donc la notion d’un {\it syncrétisme} primitif où les notions de {\it sujet} et
d’{\it objet} ne sont pas encore distinguées et à partir duquel elles se différencieront
au cours d’une \textbf{\textit {genèse}} progressive. Le principal tort des
doctrines de « l’expérience immédiate » est, comme nous l’avions déjà
remarqué (voir le \S 80 {\it fin}), de méconnaître ce point de vue {\it génétique}
%141
et, par le fait même, la considération des {\it niveaux} de pensée. Car,
ainsi que l’affirme J. \textsc{Piaget}, « l'observation et l’expérimentation
combinées semblent démontrer que la notion d’objet, loin d’être
innée ou donnée toute faite dans l’expérience, se construit peu à peu »,
et l’on a vu, par l’étude de l’{\it égocentrisme} enfantin et de l’{\it autisme}
(\S 40-41), qu’il faut que la pensée atteigne un certain niveau pour que
la distinction entre {\it sujet} et {\it objet} devienne explicite.

\section{Le syncrétisme du sujet et de l’objet}% 102
Il existe donc,
à l’origine de la pensée, une \textbf{\textit {confusion initiale du sujet et de l’objet}}.
Le genre d’existence que le tout jeune enfant attribue aux choses
est de même nature que sa propre existence à lui : on peut aussi
bien dire qu’il y a alors assimilation des choses au moi que, comme le
dit J. \textsc{Piaget}, « absorption du moi dans les choses par indifférenciation
du subjectif et de l’objectif ». C’est ce qui explique et le {\it réalisme}
de l’enfant (le rêve considéré comme réel : un enfant dit que les
chambres où l’on dort sont « pleines de rêves ») et son {\it artificialisme}
et son {\it finalisme} (\S 40) et, d’une façon générale, sa confusion entre le
{\it réel} et l'{\it imaginaire}. Cette confusion se traduit souvent dans le {\it dessin}
de l'enfant, comme nous l’avons observé à propos du dessin du « bonhomme »
(\S 83 {\it fin}) dans lequel l'enfant objective la représentation
qu’il se fait de son propre corps. Son comportement ne doit pas ici
nous faire illusion : « Le premier contact entre le sujet agissant
et le milieu, c’est-à-dire la prise de possession des choses par l’assimilation
réflexe, n’implique en rien la conscience de l’objet » (\textsc{Piaget}).
Ainsi qu’il a été dit \S 83 {\it D}, le premier univers de l’enfant est un monde
de « tableaux perceptifs » qui « ne sont pas dissociés des activités qui
les utilisent », qui « sortent du néant au moment de l’action pour y
rentrer avec son extinction ». « Univers sans objets », par conséquent,
si l’on entend par {\it objets} des choses conçues comme permanentes,
extérieures au moi et qui continuent à exister même quand on ne les
perçoit ou ne les utilise plus. Univers sans causalité, non plus, si l’on
entend par {\it causalité} des relations établies {\it entre les choses}, puisque
ces relations sont ici « masquées par les rapports entre l’action et ses
résultats désirés ». Ce n’est donc pas un univers, comme celui de
l’adulte, « à la fois stable et extérieur, relativement distinct du monde
intérieur et dans lequel le sujet se situe comme un terme particulier
parmi l’ensemble des autres » (\textsc{Piaget}). En somme, on ne peut pas
dire qu’il y ait encore {\it connaissance du monde extérieur} comme tel,
comme réalité objective distincte du moi.

% 142
\section{La constitution du réel}% 103
Pour parvenir à cette notion
d’un univers vraiment objectif, l'enfant a, selon J. \textsc{Piaget}, {\it six
étapes} à parcourir. Dans les deux premières (réflexes et premières habitudes),
l’enfant est bien capable de reconnaître certains « tableaux »,
mais ne leur attribue aucune permanence substantielle : on n’observe
encore chez lui aucune conduite relative aux objets qu’on fait disparaître.
Dans la troisième, un début de permanence est conçu (prolongement
des mouvements de préhension), mais il n’y a pas encore de
recherche systématique pour retrouver les objets disparus. A la quatrième,
l’enfant commence à rechercher des mains un objet disparu
du champ visuel (qu’on a caché derrière un écran), mais il ne sait pas
encore tenir compte de la succession des déplacements visibles : il
cherche toujours l’objet au même endroit que la première fois, {\it même
s’il a vu} qu’on l’a caché ailleurs, sous une couverture par exemple. À
la cinquième étape (12-18 mois), l’objet est constitué comme substance
permanente, et l'enfant tient compte des déplacements {\it dans
la mesure où ils s’opèrent dans le champ de la perception directe}. À la
sixième étape (qui débute vers 18 mois), il y a représentation des
objets absents et l’enfant tient compte de tous les déplacements possibles,
mêmes invisibles. {\bf —} L'enfant parvient ainsi à se constituer un
{\it monde objectif} en fonction de trois critères. Est objectif pour lui :
1° ce qui donne prise à la {\it prévision} (trois premières étapes) ; 2° ce qui
se prête à des {\it expériences distinctes, mais concordantes} (union de la
recherche visuelle et de la recherche tactile au quatrième stade) ;
3° ce qui est relié {\it d’une manière intelligible} à un {\it système de relations
spatiales et causales} (cinquième et surtout sixième étapes).

Il ne faudrait pas croire cependant qu’une fois cette notion de la
réalité objective ainsi constituée, l’enfant ait vis-à-vis d’elle exactement
la même attitude que l’adulte. De deux à sept ans environ, il
admet encore la {\it coexistence} de deux réalités hétérogènes et, pour lui,
également réelles : le monde du jeu et de l'imaginaire et le monde
de l'observation. C’est seulement de 7 à 11 ans que s’introduit dans
sa pensée un début de hiérarchie entre ces deux mondes, le second
devant être définitivement privilégié comme seul réel après 11 ou
12 ans grâce à la formation d’un nouveau plan de pensée, celui de
la pensée formelle et de l'expérience logique.

\section{Les critères du réel}% 104.
Les trois critères qui permettent
à l’enfant de discerner ce qui est objectif ne sont pas tellement
différents, ainsi que le remarque \textsc{Piaget}, de ceux de l'adulte
et même du savant. Efforçons-nous en effet de prendre conscience
de ce qui, {\it pour nous}, caractérise l’objet réel. Nous allons retrouver
%143
ici, selon une distinction déjà indiquée (\S 25), trois séries de critères.

{\it A.} Ce sont d’abord des critères que l’on peut appeler \textbf{\textit {biologiques}}
en ce sens qu’ils se rattachent à notre {\it action}. Mais ici il est nécessaire
de préciser. On vient de voir en effet que, chez le tout jeune enfant,
l’objet adhère encore à l’action au point qu’il ne se détache pas
vraiment du sujet et que les schèmes qui suffisent à cette action
n’impliquent nullement par suite la conscience de l’objet comme tel.
Il n’en est pas moins vrai que « le réel », c’est bien {\it pour nous} à la fois
ce qui {\it offre prise} à notre action et ce qui y {\it résiste}. Ce qui lui offre
prise : car la croyance à la permanence de l’objet ne se réduit pas,
comme l'avait cru Stuart Mill, à « une possibilité permanente de sensations » :
elle est, ainsi que l’a dit René \textsc{Duret}, croyance à « une
série possible de réponses sensorielles {\it aux différentes phases de l'effort
moteur} ». L'existence de l’objet a ici pour caractère, selon l'expression
allemande, la {\it Vorhandenheit}, la présence « à portée de notre main ».
L'imaginaire, au contraire, {\bf —} et aussi, nous le verrons (chap. XI), le
passé ou l'avenir, {\bf —} est ce qui n’offre aucune prise à notre action.
Mais, d’autre part, le réel est aussi \textbf{\textit {ce qui résiste}}. On sait (\S 76) que
Maine de Biran avait voulu voir dans la sensation de résistance qui
accompagne l’effort moteur volontaire le processus constitutif de
l’objectivation. Mais, ainsi que l’objecte \textsc{Piaget}, « le sujet peut fort
bien incorporer la sensation de l’obstacle au schème de son activité
propre, étant donné que toute action corporelle est limitée et s’accompagne
de la conscience plus ou moins claire de cette limitation ». Le
réel peut cependant résister d’une autre manière à l'effort du sujet :
«il n’y a plus seulement résistance par opposition de forces, comme
dans les contacts entre l’activité musculaire et la masse d’un solide
{\scriptsize (Ce n’est d’ailleurs pas exactement ainsi que l’entendait Maine de Biran : pour lui,
la résistance était surtout constituée par l'inertie du muscle qui s'oppose à l'influx
nerveux moteur)},
mais résistance par complication du champ de l’action et intervention
d'obstacles empêchant le sujet de percevoir l'objectif ». C’est
alors que {\it le corps} lui-même devient « un terme parmi les autres »
(c'est ce qui se produit à la sixième étape de l’objectivation chez
l'enfant) et se trouve ainsi engagé « dans un système d’ensemble
marquant les débuts de l'objectivité véritable » (\textsc{Piaget}).

B. Mais il y a aussi des critères \textbf{\textit {sociaux}} du réel. Et d’abord {\it l'accord
des témoins} : si je crois entendre dans mon appartement un bruit de
souris ou celui d’un cambrioleur qui essaye de crocheter ma porte,
je serai sûr de ne pas me tromper si les autres personnes qui habitent
% 144
avec moi entendent la même chose. Mais ce n’est là encore qu’un
critère inter-individuel. L'expérience montre que, dans certains cas,
l'accord des témoins ne se réalise qu’en fonction de toute une {\it mentalité
collective} (ci-dessous \S 286, 2°). « Les notions d’objet et d’objectivité
varient d’une époque à l’autre. Les choses sont ou non objectives
suivant qu’elles se conforment ou non à la vision, plus exactement à
la {\it prévision} du réel, propre aux civilisations considérées » (\textsc{Blondel}).
Nous avons rappelé ci-dessus la formule de L. \textsc{Lévy-Bruhl} selon
laquelle les primitifs « ne perçoivent rien comme nous ». Cette formule
s’applique à leur conception du réel aussi bien qu’à celle des
objets particuliers. Rapportant le jugement d’un ethnographe d’après
lequel la confusion par excellence est, chez eux, « la confusion du
subjectif et de l'objectif », Lévy-Bruhl observe que ce jugement ne
rend pas compte fidèlement de leur état d'esprit. Le contraste violent
qui existe chez nous entre les deux domaines n’existe pas chez
eux : « Leur perception est orientée autrement. Ce que nous appelons
réalité objective y est uni, mêlé, et souvent subordonné à des éléments
mystiques, insaisissables, que nous qualifions aujourd’hui de
subjectifs. »

Le rêve, par exemple, n’est pas, pour eux, une « représentation subjective
suspecte » : pour eux comme pour l'enfant, quoique pour des raisons différentes,
le rêve est aussi {\it réel} que les perceptions de l’état de veille. Un
Indonésien a rêvé qu’un Européen, éloigné à ce moment de près de 150 milles
de son jardin, lui a volé des potirons. Il vient lui réclamer le prix de ses
potirons, tout en convenant qu’il ne les a pas pris effectivement. {\bf —} De façon
plus générale, le primitif croit à l'existence d’un \textsf{\textit {monde invisible}} qui, pour lui,
est {\it au moins aussi réel} que le monde perceptible aux sens. Par exemple,
le médecin-sorcier guérit un malade en extrayant de son corps un objet,
cause de la maladie, qui n’est visible que pour lui seul. Chez de nombreuses
peuplades, les êtres invisibles ne sont perceptibles qu’aux sorciers, féticheurs,
shamans : ils n’en sont pas moins réels.

Les Romains ont pu croire à la réalité de « pluies de sang »(\S 286,
{\it fin}) ; les hommes du moyen âge, à la possession démoniaque et à la
sorcellerie
{\scriptsize (Même au {\tiny XVII}$^\text{e}$ siècle, ces croyances étaient encore courantes jusque dans les
familles cultivées. Le D$^\text{r}$ P. \textsc{Delbet}, dans {\it Le Caractère de Pascal} (1947), raconte que,
vers l’âge d’un an, Blaise Pascal étant tombé gravement malade, ses parents firent
venir, non un médecin, mais... une sorcière. Selon le récit de Marguerite Périer, sa nièce,
«l’état de l'enfant s’aggrava au point qu’on le crut prêt à mourir. C'était l'effet d’un
sort jeté par une sorcière. Le sort était à la mort et, pour sauver l'enfant, il fallait que
quelqu'un mourût pour lui, et transporter le sort ». Heureusement on pouvait le transporter
sur une bête. Étienne Pascal offrit un cheval ; la sorcière déclara qu'un chat
suffirait. Enfin, elle prépara un cataplasme avec trois sortes d'herbes cueillies par un
enfant de moins de sept ans. Le petit Blaise entra alors en convalescence)}
; et beaucoup de nos contemporains croient à la réalité
% 145
de la télépathie, de la transmission de pensée et de la radiesthésie
{\scriptsize (Voir là-dessus Marcel \textsc{Boll}, {\it L'occultisme devant la science}, P. U. F., 1966)}.
La technique moderne nous fait apparaître aujourd’hui comme
{\it réelles} des quantités de choses (aviation, radio, télévision, etc.) qui
eussent peut-être passé autrefois pour des miracles ou des inventions
du démon : un homme du moyen âge entendant la voix humaine
reproduite par un phonographe ou un appareil de radio aurait sans
doute cru à de la sorcellerie. L'avenir verra sans doute d’autres inventions
qui n’entrent pas dans notre « réalité » d’aujourd’hui.

{\it C.} Il y a enfin {\bf —} et peut-être surtout {\bf —} des critères \textbf{\textit {intellectuels}}
du réel. 1° C’est d’abord la \textbf{\textit {concordance des expériences}}. On a vu
que ce critère joue déjà chez l’enfant qui, dès l’âge d’un an, sait unir
les données de la vue à celles du toucher, et réciproquement. Que
faisons-nous d’autre nous-mêmes lorsque, soupçonnant le relief d’une
figure d’être une illusion d’optique, nous touchons l’objet pour nous
assurer que le relief est bien réel? Si je crois entendre le bruit de la
mer ou la sirène d’un transatlantique, alors que tous les objets qui
me tombent sous les yeux me disent que je suis à Paris ou à Clermont-Ferrand,
j'en conclus que je me suis trompé. Le premier critère du
réel est donc ici l’accord des différents sens. {\bf —} 2° Ce n’est pas tout :
le sensible lui-même doit venir s'intégrer dans un {\it système mental}. Il
y a ici, comme le dit \textsc{Piaget}, « un élément rationnel ou déductif »
en union étroite avec l’élément empirique : « Ne constitue un objet
réel que le phénomène relié d’une manière intelligible à l’ensemble
d’un système spatio-temporel et causal. » Le second critère est donc
une certaine \textbf{\textit {cohérence logique}}. Si, étant à Paris, je crois apercevoir
de loin un de mes amis dans la rue et si d'autre part je suis certain
que cet ami a pris hier soir l'avion pour Dakar, je ne puis m'empêcher
de penser que je suis dupe d’une illusion. Ne peut être considéré
comme réel que ce qui est d’abord logiquement {\it possible}. {\bf —} 3° Généralisons
encore. La représentation et le sentiment du réel exigent,
d’après ce qui précède, une confrontation de la perception actuelle
avec {\it tout l’ensemble} de notre psychisme, donc l’intervention de cette
{\it activité de synthèse} où nous avons reconnu une des formes les plus
hautes de l'activité consciente (\S 46) : « Saisir une perception avec le
sentiment que c’est bien le réel, écrit \textsc{Janet}, c’est-à-dire coordonner
autour de cette perception toutes nos tendances, toutes nos activités,
c’est l'œuvre parfaite de l’attention », et c’est pourquoi cette \textbf{\textit {fonction
du réel}}. qui « exige une complexité spéciale de l’opération psychologique »,
disparaît, comme on va le voir (\S 106), chez certains malades
ou affaiblis mentaux. {\bf —} 4° En ce sens, on peut dire encore avec
%146
P. \textsc{Janet} que « le réel, c’est ce que l’on croit après réflexion », c’est
« la conséquence d’une croyance réfléchie ». Autrement dit, dans l’appréciation
de la réalité des choses comme dans l'identification des
objets (\S 86), il y a un jugement, au moins implicite, le \textbf{\textit {jugement
d’extériorité}}, dont le logicien \textsc{Goblot} a formulé ainsi le schéma : « Ce
phénomène {\it présent} et {\it mien} est {\it présent}, mais n’est pas {\it mien} » (puisqu’il
est objectivé). Ce jugement d’extériorité est, en même temps,
selon l’expression de \textsc{Janet}, un jugement de « présentification » : il
y a véritablement une « conduite du présent», qui disparaît dans certains
affaiblissements mentaux (p. 151). On verra à propos de la
mémoire (chap. XX) que \textsc{Descartes} avait déjà observé que cette « présentification »,
qui fait saisir l’objet comme \textbf{\textit {nouveau}}, exige un acte
propre de l'intelligence. Reconnaissons toutefois que cet acte n’est
pleinement conscient que dans certains cas, par exemple quand nous
nous demandons si ce que nous percevons est bien réel ou si nous
ne serions pas victimes d’une illusion.

\section{Degrés et formes du réel}% 105
Nous avons beaucoup simplifié
le problème en admettant qu’il y a pour l’homme {\it un} réel unique.
A vrai dire, il faut distinguer ici diverses perspectives. {\it A.} Il y a
d’abord, remarque P. \textsc{Janet}, « pour l’esprit réfléchi, des \textbf{\textit {degrés}} de réalité,
du réel véritable et complet et du {\it demi-réel} ».

« Nous ne mettons pas sur le même plan la réalité d’un de nos amis et la
réalité du dîner que nous avons eu avec lui. L’ami est une réalité qui persiste,
qui est encore la même aujourd’hui, qui sera la même demain, tandis
que le dîner a eu une réalité assez forte pendant que nous le mangions et a
une réalité bien moins forte, quand il n’est plus qu’un souvenir. Parmi les
réalités complètes, nous mettons les corps et les esprits ; parmi les demi-réalités,
nous mettons les événements et bien des choses du même genre. »
Un événement {\it passé} peut se raconter, mais il ne provoque plus d’\textsf{\textit {action}}
comme le présent, si ce n’est précisément l’acte du {\it récit} (\S 131 {\it A}). Quant au
{\it futur}, il ne peut encore provoquer qu’une seule conduite : celle de l'{\it attente}
(\S 150). Il y a enfin du \textsf{\textit {presque réel}} où l’on peut ranger, selon Janet, « la
notion d’{\it acte}, la notion de {\it force} et surtout la notion du {\it présent} ».

{\it B.} Mais le réel ne comporte pas seulement des degrés, il comporte
aussi des \textbf{\textit {formes}} diverses. Laissons de côté ces « sous-univers », dont
parle W. \textsc{James}, où notre esprit se réfugie quand l’univers « réel » ne
le satisfait pas : univers de l’{\it idéal}, du {\it rêve}, de la {\it fiction}. On vient de
voir que ce n’est pas, même à nos yeux, du « réel » véritable. Remarquons
plutôt que, par delà la {\it réalité sensible} qui est celle de l’expérience
quotidienne, il y a : 1° l’univers du {\it savant}, le réel tel que nous
le montre la science et d’où sont bannies les « qualités secondes » :
couleurs, sons, odeurs, saveurs, du moins telles que nous les percevons,
%147
et qui se réduit, sinon tout à fait à « de la figure et du mouvement »
comme le voulait le mécanisme cartésien, du moins à quelques
« réalités » non perceptibles aux sens, telles que l’atome et ses composants,
et à des champs de forces régis par quelques équations (cf. ci-dessous
\S 236, 317-319 et 339 {\it D}) ; {\bf —} 2° l'univers de l’{\it artiste} ou du
 {\it poète}, pour lesquels cet univers scientifique, constitué surtout d’abstractions,
est aux antipodes de la réalité vraie : leur univers, à eux,
également éloigné d’ailleurs de celui de la vie pratique, est un monde
d’harmonies, d’élégances, de beautés où rien de ce qui est vulgaire
ou discordant n’a place et dont on peut dire avec Baudelaire :

\begin{center}
{\it Là, tout n’est qu’ordre et beauté,

Luxe, calme et volupté.}
\end{center}

3° Aux yeux du {\it métaphysicien}, tout cela n’est encore qu’{\it apparences}.
Par delà ces apparences, la réalité {\it absolue} est constituée soit, comme
chez Platon, par un monde d’essences intelligibles, existant en soi,
les {\it Idées}, dont le monde sensible n’est qu’une grossière copie (voir
\S 111), soit, comme chez Leibniz, par un univers d’esprits, les {\it monades},
dont la matière n’est, au fond, qu’une sorte de dégradation (voir
\S 113 A), etc. {\bf —} 4° Le {\it croyant} enfin considère tout ce monde d’ici-bas
comme une réalité transitoire et a confiance qu’un jour viendra
où nos yeux s’ouvriront sur des réalités d’un autre ordre.

Tout ceci confirme l’idée indiquée ci-dessus, selon laquelle le réel
est toujours un {\it système} plus ou moins {\it construit}. On verra bientôt
(\S 107-108) les problèmes que cette constatation peut soulever.

\section{Les troubles de la perception du réel}% 106
La construction du réel étant une opération complexe, il n’est pas étonnant
qu’elle subisse parfois des altérations.

{\it A.} \textbf{\textit {L'hallucination}} a été définie une « perception sans objet »
(D$^\text{r}$ Morel). Autrement dit, le sujet perçoit comme une réalité objective
quelque chose qui n’existe pas. {\bf —} C’est d’ailleurs ce qui nous
arrive à tous dans le {\it rêve} (voir t. II, ch. VII). Dans les états intermédiaires
entre la veille et le sommeil, quand nous nous endormons ou
nous éveillons lentement, se produisent les \textbf{\textit {hallucinations hypnagogiques}}
ou « visions du demi-sommeil ».

Ce sont des images qui apparaissent subitement et que les sujets comparent
à des photographies ou à des films en couleur. Ainsi, un étudiant en
médecine voit en s’endormant la préparation à laquelle il a travaillé dans la
journée ; un pêcheur à la ligne voit son bouchon flotter, puis s’enfoncer
dans l’eau. En réalité, ces représentations semblent bien être le plus souvent
assez floues, et l'esprit y assiste en spectateur {\it sans encore les objectiver
vraiment}, sans les poser comme des {\it réalités} actuellement existantes.
%148

Comment expliquer ces phénomènes? Certains auteurs ont fait
intervenir les phosphènes ou lueurs entoptiques (\S 73 B). Il se peut
que ceux-ci jouent un rôle. Mais il s’agit, en vérité, d’une {\it modification
générale de l'orientation de la conscience} : « Ce qui caractérise la vision
hypnagogique, a dit un auteur qui a spécialement étudié ces phénomènes,
c’est une modification d’ensemble de l’état du sujet.» L’attention
volontaire, en particulier, y est « devenue incapable de s’appliquer
à des événements extérieurs plus intéressants » (B. Leroy).

Dans les états \textbf{\textit {pathologiques}}, les hallucinations sont {\it beaucoup
plus nettement objectivées} et se présentent sous des formes multiples.

Elles sont tantôt {\it visuelles} (hallucinations terrifiantes de l’intoxication
alcoolique ; hallucinations « lilliputiennes » où le malade voit des personnages
minuscules se promener sur les meubles ; hallucinations des délires
progressifs où le sujet a l'impression qu’on « lui envoie des visions »), tantôt {\it
auditives} (bruits vagues, sifflets, cloches, et surtout « voix » proférant ou
chuchotant des ordres, des menaces ou des injures). tantôt {\it tactiles} (frôlements,
insectes courant sur la peau), tantôt {\it gustatives} ou {\it olfactives}, tantôt
même {\it cénesthésiques} ou {\it kinésiques} (parfois le sujet sent son corps, libéré
de la pesanteur, flotter dans l’air
{\scriptsize (Peut-être un poème de \textsc{Baudelaire}, {\it Élévalion}, est-il l'expression poétique d’un
trouble de ce genre)}).

Ces hallucinations morbides semblent avoir le plus souvent un
point de départ sensoriel ou organique : phosphènes ou objet réel
ou intoxication interne (selon E. \textsc{Wolff}, le sujet confond données
extérieures et données endogènes) ; D. \textsc{Lagache} a montré le
rôle du {\it rythme respiratoire} dans les hallucinations auditives. {\bf —}
\textsc{Taine} avait admis autrefois que l’hallucination n’est rien qu’une
{\it image} intense qui, comme toutes les images, a une tendance naturelle
à s’objectiver, mais qui, en ce cas, n’est pas {\it réduite}, comme il
arrive normalement, par une sensation antagoniste. Cette explication
n’est plus admise aujourd’hui, tant parce qu’elle repose sur une fausse
conception de l’{\it image} (\S 148) que parce qu’elle implique une conception
{\it atomiste} de notre vie psychologique. Ce n’est pas une raison toutefois
pour nier le rôle de l’{\it image}, qui demeure ici essentiel, {\bf —} D’autres
auteurs (D$^\text{r}$ Mourgue) ont insisté sur le rôle des {\it mouvements} : par
exemple, dans les hallucinations visuelles, des mouvements oculaires.
On a constaté que l’idée d’obscurité entraîne une dilatation pupillaire,
tout comme l’obscurité elle-même ; que l’image d’un objet rapproché
provoque des réflexes d’accommodation et de convergence,
tout comme un objet réel (Piéron), etc. Or on a vu quel est le rôle
des mouvements dans la perception (\S 83 C et 108 A). {\bf —} L'explication
% 149
véritable semble toutefois plus complexe. Comme l’hallucination
hypnagogique, mais plus encore, l’hallucination morbide implique
\textbf{\textit {toute une attitude de la conscience}}. « Une simple hallucination, dit
Pierre \textsc{Janet}, est tout un délire. » Le malade en effet n’a presque
jamais ses hallucinations, du moins visuelles, en présence du médecin.
On peut conclure, avec J.-P. \textsc{Sartre}, qu’« une activité systématisée
dans le domaine du réel semble exclure les hallucinations ». Celles-ci
ne consistent donc pas en « une altération de la structure primaire de
l’image », mais bien plutôt en « un bouleversement radical de l’attitude
de la conscience à l’égard de l’irréel ». Il se produit une désintégration
de la conscience « incompatible avec l’existence d'une synthèse
personnelle et d’une pensée orientée », une « chute de potentiel »
où toute structure s’abolit et où les deux mondes de l'{\it objectif} et
du {\it subjectif} se sont écroulés. Tant il est vrai que leur distinction
implique bien une structure supérieure de la pensée et que la « fonction
du réel », comme l’a nommée \textsc{Janet} (\S 46), est une des plus
hautes de l’esprit.

{\it B.} C’est également ce que va nous montrer l’étude des cas, cependant
très différents, où le sujet \textbf{\textit {perd le sentiment du réel}}. 1° Ainsi
les \textbf{\textit {psychasthéniques}} étudiés par P. \textsc{Janet} « continuent à avoir la
sensation et la perception du monde extérieur, mais ont perdu le
sentiment de réalité qui ordinairement en est inséparable».

Ce déficit se traduit, chez ces malades, par des sentiments d’\textsf{\textit {irréel}},
d’\textsf{\textit {imaginaire}}
et, par suite, de \textsf{\textit {dépaysement}} : « Je ne puis, dit une malade, me mettre
dans l’idée que vous et les gens qui m'entourent, vous vivez réellement,
que vous êtes de vraies personnes », « Il me semble que tout est faux, même
les objets que je vois », « Je ne vis plus sur terre, dit une autre, puisque je
ne vois plus rien qui existe réellement », {\bf —} des sentiments d’\textsf{\textit {étrangeté}} :
« Les choses me font l’effet de devenir drôles, il y a quelque chose qui n’est
pas comme de coutume », {\bf —} des sentiments d’\textsf{\textit {obscurité}} : les objets apparaissent
comme « enveloppés d’un nuage », les personnes comme « des
ombres ». Mais les métaphores les plus fréquentes sont celles du \textsf{\textit {rêve}} : « Je vis
dans le rêve, … j'entends parler comme si j'étais dans un rêve, quand je
vois mes camarades d'hôpital, je me dis que ce sont les figures d’un rêve »
ou encore celles du \textsf{\textit {néant}} et de la \textsf{\textit {mort}} :
« Il est inutile de rien faire, dit
une femme de 58 ans, puisque tout est mort, on m'a mise dans un tombeau
où il n'y a rien, tout est vide, il n'existe plus personne, c'est comme si
j'étais morte moi aussi. » {\bf —} Un cas très curieux est celui de l'\textsf{\textit {illusion du
déjà-vu}} : une malade trouve que sa vie actuelle reproduit « mot pour mot »
sa vie passée : « Je retrouve l'acte que je fais, la pensée que j'ai, dit une
autre, tout comme si je les revivais de nouveau » ; un malade souffrant de ce
trouble est conduit un jour devant la gare Montparnasse, à Paris, où une
locomotive, renversant les butoirs, est tombée par les baies du premier étage
sur le trottoir : il hausse les épaules en disant qu’on répète toujours les
mêmes mauvaises plaisanteries, qui coûtent fort cher, pour amuser les
Parisiens. On avait interprété cette {\it illusion du déjà-vu} comme un trouble de
% 150
la mémoire. P. \textsc{Janet} a fait remarquer avec raison que c'est « une négation
du caractère {\it présent} du phénomène plutôt qu’une affirmation de son caractère
{\it passé} », donc un trouble de la perception à laquelle manque le caractère
de {\it nouveauté} et de {\it réalité}.

Dans tous ces cas, il ne s’agit nullement de troubles de la sensibilité :
l'examen de la vision, de l’audition, etc., les révèle parfaitement
normales. Parfois même « la vision paraît surexcitée ; elle
devient nette, détaillée, précise ; le malade remarque la forme et la
couleur de chaque feuille d’arbre ; chaque dos de livre lui apparaît
avec sa physionomie ». Ce qui fait défaut, c’est « l’attention aux choses
présentes » ; c’est la faculté de « croire » au réel; c’est en somme « la
synthèse des sensations nouvelles avec les anciennes images », laquelle
constitue la reconnaissance et « présentifie » la perception (Janet).
Toutefois il s’agit plutôt ici d’un {\it affaiblissement} de cette activité
de synthèse que d’une destruction totale.

2° Il n’en va pas de même dans la \textbf{\textit {schizophrénie}}, où le sentiment
du réel s’efface à peu près complètement.

Dans le {\it Journal d'une schizophrène} publié par M.-A. \textsc{Sechehaye}, on voit
une enfant qui, passant devant une école et entendant les élèves chanter,
éprouve tout à coup un « sentiment bizarre » d’{\it irréalité}. « Il me semblait
que je ne reconnaissais plus l’école, elle était devenue grande comme une
caserne, et tous les enfants qui chantaient me paraissaient être des prisonniers
obligés de chanter. C'est comme si l’école et le chant des enfants
étaient séparés du reste du monde. » Ces sentiments d’irréalité, qui débutèrent
dès l’âge de cinq ans, s’accentuèrent plus tard. « L'immeuble de
l'école, dit la malade, devenait immense, lisse, irréel, et une angoisse inexprimable
m'étreignait... Les élèves et les maîtresses semblaient des marionnettes
qui évoluaient sans raison, sans but. Je ne reconnaissais plus rien,
plus personne. C’est comme si la réalité s'était diluée, évadée de tous ces
objets et de ces gens. »

On le voit : la schizophrénie est, selon l’expression du Dr J. Vinchon,
« une rupture de contact avec le réel », et cette rupture est ici
à peu près totale. L’{\it autisme} (\S 41) devient prédominant, et le sujet
régresse à une {\it confusion de l'objectif et du subjectif}, qui s'exprime dans
ses œuvres (voir fig. 25). Le D$^\text{r}$ \textsc{Ferdière}, qui a étudié des poupées
confectionnées par ces malades, y a reconnu un mécanisme de « projection
du moi dans les choses » analogue à celui de l’égocentrisme
enfantin : ces poupées « sont des confessions, et c’est ce qui explique que
leurs auteurs les cachent jalousement » (P.-M. \textsc{Schuhl}).

\section{Le problème épistémologique}% 107
Ces analyses psychologiques
nous permettent dès maintenant de comprendre comment se
posera pour nous le problème {\it épistémologique}, celui qui touche à
%151
l’\textbf{\textit {objectivité}} et à la \textbf{\textit {valeur}} de notre connaissance du {\it réel}. Tant qu'avec
le sens commun on considère la pensée comme une sorte de décalque
ou de réplique interne de l’objet (ce que J.-P. \textsc{Sartre} a appelé
l'{\it illusion d’immanence}), le problème ne fait guère difficulté. Mais, à
partir du moment où la psychologie de la perception nous a permis
de prendre conscience que ce réel, que nous croirions volontiers donné,
est en vérité {\it construit}, l’objectivité ne peut plus se définir comme la
simple {\it conformité de la pensée à une sorte d'objet en soi}. Nous verrons
que le {\it fait scientifique}, loin d’être un pur donné, est encore bien plus
{\it construit} que le fait de la connaissance courante (chap. XXI, \S 260); que,
par suite, l’objectivité scientifique est plutôt un {\it effort d’objectivisme}
qu’une copie passive du réel (voir \S 318); et nous aurons à nous
demander si la {\it vérité} peut se définir simplement comme une vérité-objet,
comme une « adéquation de l'esprit et de la chose » ainsi que
le voulait la scolastique médiévale (voir \S 323). Ici la Psychologie
nous conduit à la {\it Théorie de la Connaissance} elle-même, c’est-à-dire
au problème du {\it fondement} de la connaissance (\S 16 {\it A}), et elle nous
met déjà en garde contre une solution trop simple de ce problème.

\section{Le problème métaphysique}% 108
Mais la Psychologie nous
met en présence d’un problème plus fondamental encore, d’un problème
{\it ontologique}, c’est-à-dire qui concerne l’{\it essence} même du réel, et
que nous avons déjà effleuré à propos de l’espace (\S 99). Nous avons
l’habitude d’opposer le {\it réel} à la {\it pensée}, la {\it chose} à l’{\it idée}. Or la psychologie
de la perception et, en particulier, l'étude génétique de la notion
d’objet nous ont montré que le réel est, si l’on peut ainsi parler, pour
une grande part, pétri de pensée, puisque la distinction entre {\it sujet}
et {\it objet}, entre {\it imaginaire} et {\it réel} est le résultat de toute une évolution
psychologique, et que la notion du réel implique des opérations
mentales complexes où entrent souvenir, jugement, etc. Le réel
serait-il donc, comme l’ont soutenu les philosophes \textbf{\textit {idéalistes}}, réductible
à la pensée? ou faut-il maintenir la notion d’une réalité faite
de « choses » radicalement hétérogènes à la pensée, ce qui est la solution
\textbf{\textit {réaliste}} ? C’est ce problème que nous allons examiner dans le
chapitre suivant.

\section{Sujets de travaux}


Exercices. {\bf —} 1. {\it Distinguer et classer les différents sens du mot} objet {\it dans
les phrases suivantes} : « Ils ne sont point certains si les choses qu'ils voient
sont objets ou illusions » (G. de \textsc{Balzac}), « O trop aimable objet qui m'avez
trop charmé» (\textsc{Corneille}, {\it Polyeucte}, à propos de Pauline), « Elle voit
(quel objet pour les yeux d’une amante !) Hippolyte étendu... » (\textsc{Racine}),
« Lorsque je dors, mes idées se forment en moi sans l'intermédiaire des
%152
objets qu’elles représentent » (\textsc{Descartes}), 4 L'éternité se présente à nos
yeux comme le digne objet du cœur de l’homme » (\textsc{Bossuet}), « l'ous les
objets paraissent sombres et en confusion le matin aux premières lueurs de
l’aurore » (\textsc{Fénelon}), « L'objet du mariage est d’avoir des enfants » (\textsc{Bufon}),
« Tout ce qui existe pour la connaissance, donc le mondé entier, n’est
qu’objet ({\it Objekt}) vis-à-vis du sujet» (\textsc{Kant}), «La représentation qui ne
peut être donnée que par le moyen d’un objet ({\it Gegenstand}) unique, est
une intuition» (Ip.). «C'est sur le toucher seul que se fonde l'acte qui
nous fait reconnaître l'identité permanente du même objet » (\textsc{Biran}), « Les
choses futures peuvent être l’objet d’une obligation » (\textsc{Code Civil}), « Toute
hallucination est bel et bien une sensation, et l’objet serait là qu'il n'y
aurait ni une meilleure ni une autre sensation ; mais l’objet n’est pas là »
(W. \textsc{James}), « Penser l’objet, c’est penser quelque chose pour quoi je ne
compte pas » (G. \textsc{Marcel}), « L'esprit, c’est l’objet nié, c’est-à-dire toujours
dépassé » (\textsc{Le Senne}). {\bf —} 2. {\it Même exercice pour les mots} réel {\it et} réalité :
« Il faut de l’agréable et du réel [dans l’éloquence] ; mais il faut que cet
agréable soit lui-même pris du vrai» (\textsc{Pascal}), « Le mouvement est quelque
chose de relatif ; mais la force est quelque chose de réel et d’absolu »
(\textsc{Leibniz}), « Le précis, l'absolu, l’abstrait.. ne peuvent se trouver dans
le réel parce que tout y est relatif» (\textsc{Buffon}), « Le réel est étroit, le
possible est immense» (\textsc{Lamartine}), « C’est une vision que la réalité »
(\textsc{Musset}), « Et la réalité fait envoler le rêve » (Th. \textsc{Gautier}), « Ce que nous
appelons la réalité est un certain rapport entre nos sensations et nos souvenirs
qui nous entourent simultanément » (M. \textsc{Proust}), « L’imaginaire
est sans profondeur... Le réel se prête à une exploration infinie, il est inépuisable »
(\textsc{Merlau-Ponty}). {\bf —} 3. {\it Analyser cette observation de Marcel
Proust} : « Françoise [sa bonne] n’avait pu soupçonner la mer d’irréel qui me
baignait encore tout entier et à travers laquelle j'avais eu l’énergie de faire
passer mon étrange question : 4 Il est bien dix heures, Françoise? donnez-moi
mon café au lait. » A force de volonté, je m'étais réintégré dans le réel. »
{\bf —} 4. {\it Essayez d'observer sur vous-même les « hallucinations hypnagogiques »}.
{\bf —} 5. {\it Comment expliquez-vous la} prédominance du détail {\it qui se produit
souvent dans les représentations des psychasthéniques ou. des schizophrènes ?}

Exposés oraux. {\bf —} 1. {\it Le syncrétisme enfantin}, d'après \textsc{Piaget}, p. 107,
233 et passim., et \textsc{Wallon}, p. 177 et suiv. {\bf —} 2. {\it Exposé plus complet des
stades de la constitution du réel chez l'enfant}, d'après \textsc{Piaget}. {\bf —} 3. {\it La
schizophrénie} (d’après E. \textsc{Minkowski}, {\it La schizophrénie}, Payot, 1927, et
A.-M. \textsc{Sechehaye}, {\it Journal d'une schizophrène}, P. U. F., 1950).

Discussion. {\bf —} {\it Existe-t-il un critère certain du réel ?}

Lectures. {\bf —} {\it a.} \textsc{Taine}, {\it De l'intelligence}, tome I, liv. II, chap. I, et tome II,
liv. I, chap. I et II. {\bf —} {\it b.} Bernard \textsc{Leroy}, {\it Les visions du demi-sommeil},
Alcan, 1926. {\bf —} {\it c.} R. \textsc{Duret}, {\it Les facteurs pratiques de la croyance dans la
perception}, Alcan, 1929. {\bf —} {\it d.} J. \textsc{Piaget}, {\it La construction du réel chez l'enfant},
Delachaux et Niestlé, 1937, chap. I ct concl. ; et {\it e.} {\it La représentation du
monde chez l'enfant}, Alcan, 1938. {\bf —} {\it f.} J.-P. \textsc{Sartre}, {\it L'imaginaire}, Gallimard,
1940, 4$^\text{e}$ partie. {\bf —} {\it g.} H. \textsc{Wallon}, {\it L'évolution psychologique de
l'enfant}, coll. A. Colin, 1941{\bf —}h, Edgar \textsc{Wolff}, {\it La Sensation et l'image},
Carcassonne, 1943.
