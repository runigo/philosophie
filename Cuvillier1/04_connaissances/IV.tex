\chapter{Les différents types de connaissances}
%chapitre IV
%SOMMAIRE
%52. Les différents types de connaissance. — 53. Les espèces de la connaissance.
%— 54. La connaissance vulgaire. — 55. La connaissance scientifique. —
%56. La connaissance philosophique. — 57. Les formes de la connaissance :
%pensée intuitive et pensée discursive. — 58. La notion d’intuition. — 59. L'intuition
%psychologique. — 60. L'intuition sensible, — 61. L'intuition intellectuelle.
%— 62. L'intuition métaphysique. — 63. La pensée discursive et le
%raisonnement. — 64. Rôles respectifs de l'intuition et du raisonnement. —
%65. Analyse et Synthèse, — 66. L'Analyse : ses différentes formes. — 67. La
%Synthèse : ses différentes formes. — 68. Rôles respectifs de l'analyse et de la
%synthèse. — 69. Les abus de l’analyse et de la synthèse : l'esprit de système.

\section{Les différents types de connaissance}% 52.
Il existe différents
types de connaissance, qui sont susceptibles de varier : 1° selon
les \textbf{\textit {objets}} étudiés ; 2° selon les \textbf{\textit {espèces}} de la connaissance ; 3° selon
les \textbf{\textit {formes}} de la connaissance. Nous négligerons, pour le moment, le
premier point de vue. Non certes qu’il soit invraisemblable que le
types de connaissance approprié varie selon l’{\it objet} à connaître, selon
qu’il s’agit, par exemple, de connaître l’état d’âme d’autrui, ou bien
les propriétés de tel corps chimique, ou encore les événements de
telle période historique. Mais nous retrouverons ces différences en
%63
étudiant les fonctions de la connaissance ; et d’ailleurs ne serait-ce
pas préjuger d’un problème important que d'admettre à l’avance que
le type de connaissance doive nécessairement différer selon la nature
de l’objet à connaître ? Nous étudierons donc surtout ici les {\it espèces}
et les {\it formes} de la connaissance.

\section{Les espèces de la connaissance}% 53.
Supposons: un même
objet : les propriétés de la matière par exemple. Nous les connaissons
tous par l’{\it usage} et l'{\it expérience commune} : ce sera une première espèce
de connaissance. Mais la Physique et la Chimie ont aussi pour objet
de nous faire connaître ces propriétés de la matière : c’est la connaissance
{\it scientifique}, qui constituera la deuxième espèce de connaissance.
Enfin la {\it Métaphysique} pourra se poser, sur le même objet : la
matière, des problèmes, tel celui de sa nature intime (chap. XXVII),
voire celui de son existence (chap. VIII), que ne se pose pas la
Science : ce sera une troisième espèce de connaissance. — Au delà de:
ces trois espèces de connaissance, de source humaine, on peut en
concevoir une quatrième : la {\it Foi} religieuse qui, selon les théologiens,
est bien une connaissance, quoiqu’elle soit fondée, non sur la vérité
intrinsèque, saisie par la Raison, de son objet (les {\it dogmes}), mais sur
l’autorité de la Révélation divine. Nous n’avons cependant pas à en
traiter ici, n’ayant à considérer que les types de connaissance proprement
humains
{\scriptsize (Sur la différence entre Métaphysique et Religion, voir chap. XVII, fin)}.

\section{La connaissance vulgaire}% 54.
La première espèce de connaissance
sera donc la connaissance courante ou quotidienne, celle qu’il
est d’usage d'appeler la {\it connaissance vulgaire}, non qu’elle soit caractéristique
d’une certaine classe d'hommes, mais au contraire parce que
c’est la connaissance de tout le monde, en somme celle du « sens
commun ». C’est de cette connaissance que Spinoza disait qu’elle
est acquise par {\it ouï-dire} ou par {\it expérience vague} : « Je sais par ouï-dire
seulement quel a été mon jour de naissance ; que j'ai eu tels parents,
et autres choses semblables dont je n’ai jamais douté. Je sais par
expérience vague que je mourrai ; si je l’affirme en effet, c’est que j'ai
vu bien d’autres êtres semblables à moi rencontrer la mort, bien que
tous n’aient pas vécu le même espace de temps et ne soient pas morts
de la même maladie. » Par \textbf{\textit {ouï-dire}}, il faut donc entendre tout ce qui
%64
nous parvient à titre de simple « information », soit par la conversation
ou le rapport de nos semblables, soit aussi, pourrions-nous ajouter
aujourd’hui, par les procédés modernes de diffusion tels que la
presse, la radio, etc. C’est ainsi que nous apprenons que tel de nos
amis vient de se marier, que tel homme d'État est mort hier, qu’un
cyclone a ravagé les côtes de la Floride, etc. On remarquera le caractère
{\it particulier}, {\it contingent} et {\it incoordonné} d’une telle connaissance :
les choses que nous apprenons ainsi ont eu lieu à tel endroit, à telle
date ; elles auraient pu aussi bien ne pas arriver ou arriver autrement ;
enfin ce sont, comme disent les journaux, des « faits divers », qui
n’ont pas de lien entre eux. Quant à l'\textbf{\textit {expérience vague,}} c’est l’expérience
{\it non méthodique}, celle qui, comme dit Spinoza, « s’étant fortuitement
offerte et n’ayant été contredite par aucune autre, est demeurée
comme inébranlée en nous ». C’est l’expérience de la vie quotidienne,
celle qui apprend déjà au tout jeune enfant que le feu brûle, que l’eau
mouille, etc. Cette connaissance {\it empirique} (mais non {\it expérimentale})
atteint déjà, contrairement à la connaissance par ouï-dire, à une certaine
{\it généralité}, d’ailleurs nécessaire pour la conduite de notre {\it vie
pratique}, mais parfois trompeuse : « le bois flotte », disons-nous en
nous fondant sur l’expérience des bois de nos climats, mais il est des
bois très denses des régions tropicales qui ne flottent pas. — Il y a
lieu de remarquer que cette « connaissance vulgaire » est étroitement
soumise aux {\it influences du milieu social}. On verra en effet (chap. V,
\S 85) que notre représentation du monde s’encadre dans un système
de représentations collectives propre à telle ou telle civilisation que
véhicule avec lui le langage.

\vspace{0.24cm}
{\footnotesize 
C'est pourquoi ce qu’on appelle « le bon sens »
{\scriptsize (Nous prenons ici ce terme au sens courant, non dans celui que lui donne Descartes
au début du Discours de la Méthode quand il identifie le bon sens à la Raison ou à la
faculté de bien juger)}
 ou plutôt le « sens
commun » qui est le résultat de cette expérience limitée, est si variable avec
les époques, avec les milieux sociaux : certains Indiens d'Amérique, lorsqu’on
leur dit qu'il y a des hommes qui se donnent volontairement la mort,
accueillent cette affirmation avec un sourire d’incrédulité polie ; jusqu’à
Copernic, le « bon sens » se refusait à admettre que la Terre tourne sur elle-même
et se meut dans l’espace ; jusqu’à l'aviation, il était contraire au
« bon sens » qu’un « plus lourd que l'air », élevé dans l'atmosphère, ne
tombât pas. D’Alembert s’amusa un jour à formuler quelques « lois physiques »,
vraisemblables aux yeux du sens commun, et qui n’en sont pas
moins totalement fausses, par exemple : « le baromètre doit monter pour
annoncer la pluies (car, lorsqu'il doit pleuvoir, l’air est plus humide, donc
plus lourd) ; « c’est en hiver qu’il doit surtout tomber de la grêle » (car l’air
est plus froid), etc. Ce prétendu « bon sens » est d’ailleurs bien souvent
%65
dénué de tout esprit critique : ses préjugés lui paraissent des « évidences » et,
comme le remarquait Fénelon
{\scriptsize (Mais, hélas ! à la louange de ce qu'il appelle le « sens commun »!)},
c’est lui « qui prévient tout examen
qui rend l’examen même de certaines questions ridicule, qui fait que malgré
soi on rit au lieu d'examiner ».}
\vspace{0.31cm}

Ce serait en effet une erreur de croire que cette connaissance vulgaire,
sous prétexte qu’elle est spontanée et sans méthode, soit une
sorte de vue directe et naïve des choses telles qu’elles sont. Il y entre, en
réalité, beaucoup d'\textbf{\textit {interprétation.}} On verra que la perception sensible
des {\it objets extérieurs}, du monde matériel (chap. V, \S 84), est
déjà toute une {\it construction} de l’esprit. Mais la connaissance vulgaire
peut aussi porter sur le Monde spirituel et notamment sur les sentiments
de nos semblables : or nous montrerons (t. II, chap. IX) que
cette {\it connaissance d’autrui}, loin d’être, comme on le laisse entendre
parfois, le fruit d’une expérience immédiate, implique elle aussi pas
mal d'interprétation. Enfin la connaissance vulgaire peut aussi porter
sur la {\it connaissance de soi} : chacun de nous, même sans avoir fait de
psychologie ni beaucoup pratiqué « l'examen de conscience », se
« connaît » lui-même dans une certaine mesure. Mais ici encore, que
de distance de la simple et naïve expérience intime du « courant de
conscience » (chapitre II, \S 37) à la pleine prise de conscience du
{\it moi} par lui-même qui constituera la personnalité vraie (tome II,
chap. VIII)!

\section{La connaissance scientifique}% 55.
La connaissance du
« deuxième genre », telle que la décrit Spinoza, peut être assimilée
à la connaissance scientifique. Soit par exemple, dit-il, ce problème :
étant donnés trois nombres, en trouver un quatrième qui soit au troisième
comme le second est au premier. « Des marchands diront ici
qu’ils savent ce qu’il faut faire pour trouver ce quatrième nombre
parce qu’ils n’ont pas encore oublié le procédé que, sans démonstration,
ils ont appris de leurs maîtres. D’autres, de l’expérience des cas
simples, tirent un principe universel. » Ce sera là de la connaissance
vulgaire. Mais les Mathématiciens, eux, « s'appuyant sur la démonstration
d’Euclide, savent quels nombres sont proportionnels entre eux :
ils le concluent de la nature de la proportion et de cette propriété lui
appartenant que le produit du premier terme et du quatrième égale
le produit du second et du troisième ». Autrement dit, la connaissance
scientifique est une connaissance \textbf{\textit {systématisée,}} qui porte sur des
relations \textbf{\textit {générales}} et dont les résultats sont des conclusions appuyées
%66
sur des \textbf{\textit {preuves}} : {\it démonstration} dans les sciences rationnelles comme
les Mathématiques, {\it vérification expérimentale} dans les sciences de
la nature comme la Physique. D’où, au lieu de la contingence dont se
contente la connaissance vulgaire, une certaine {\it nécessité} : « Rien,
dit Aristote, n’étonnerait autant un géomètre que de voir la diagonale
du carré commensurable au côté. » Dans les sciences expérimentales,
les faits, au lieu de rester isolés et fortuits comme dans la connaissance
vulgaire, deviennent des {\it phénomènes} considérés dans leurs
éléments susceptibles de répétition et rattachés eux-mêmes à des {\it lois
générales}, voire à de grandes {\it théories} qui coordonnent l’ensemble
(ci-dessous chap. XXI). Cette systématisation fait que la connaissance
scientifique est \textbf{\textit {explicative}} : tandis que la connaissance vulgaire était
exclusivement tournée, soit vers la pratique, soit vers la satisfaction
d’une curiosité purement sensible, la science répond à une curiosité
{\it intellectuelle} (\S 238); elle est un essai d'interprétation {\it intelligible} de
l'univers. Enfin il va de soi que ce résultat ne peut être obtenu qu’à
l’aide de tout un ensemble de \textbf{\textit {méthodes}} précises, minutieuses (méthodes
de {\it mesure} notamment) qu’ignore entièrement la connaissance vulgaire.

\section{La connaissance philosophique}% 56.
On a déjà vu (ci-dessus \S 14-19)
quelques-uns des caractères essentiels de la connaissance
philosophique. Nous nous bornerons à rappeler ici que ce mode
de connaissance diffère de la connaissance scientifique, à la fois par
la nature même des problèmes sur lesquels elle porte, et par les
méthodes, les procédés de pensée qu’il requiert
{\scriptsize (Notons toutefois que, dans la mesure où l’on peut parler de connaissance morale
par exemple, il serait facile d’y distinguer les trois espèces de connaissance étudiées
ci-dessus : il y a une connaissance vulgaire de la moralité qui est impliquée dans la pratique
même de la moralité (ci-dessus \S 7) ; il y a une connaissance scientifique des faits moraux
qui est la « science des mœurs » (tome IT, \S127) ; il y a enfin une philosophie morale,
la Morale proprement dite, qui est de nature « réflexive » comme toute philosophie
(519) et qui peut se prolonger en une Métaphysique de la valeur (tome Il, chapitre XXV))}.
— {\it A.} Les problèmes
philosophiques sont : 1° soit des problèmes \textbf{\textit {axiologiques}} c’est-à-dire
qui ont pour objet des {\it valeurs}. En tant que Logique, Épistémologie,
Théorie de la connaissance (ou Gnoséologie), la philosophie étudie la
{\it valeur de la connaissance}, c’est-à-dire le problème de la {\it vérité} : ce
problème sera étudié au chapitre XXVI; en tant que Morale, elle
examine la {\it valeur de l’action}, c’est-à-dire le problème du {\it bien moral} :
ce problème sera examiné au tome II, chapitre XV ; — 2° soit des
problèmes \textbf{\textit {métaphysiques,}} c’est-à-dire qui ont pour objet la nature
intime des choses et l’Être en général : cette idée d’une connaissance
%67
métaphysique sera étudiée spécialement au chapitre XXVII. —
B. De tels problèmes qui, répétons-le, diffèrent essentiellement des
problèmes scientifiques, exigent de toute évidence de tout autres
méthodes que celles qu’emploie la Science : intuition (\S 62) ou tout
autre procédé de pensée. Ceci nous amène à étudier les différentes
formes de connaissance dont dispose l’esprit humain.

\section{Les formes de la connaissance : pensée intuitive et pensée
discursive}% 57.
La question des {\it formes} de la connaissance est bien
distincte de celle de ses {\it espèces}. La pensée se présente en effet à nous
sous deux formes différentes. Tantôt elle paraît saisir son objet par
une sorte de « vue {\it directe}, d’appréhension {\it immédiate} », c’est-à-dire
sans intermédiaires : c’est la pensée \textbf{\textit {intuitive}} (du latin {\it intueri}, voir,
contempler). Tantôt elle procède par démarches successives en passant
par différents {\it intermédiaires} : c’est la connaissance {\it médiate} ou pensée
\textbf{\textit {discursive,}} dont la principale forme est le {\it raisonnement}, mais à laquelle
on peut rattacher aussi ces deux procédés très généraux de l’esprit que
sont l'{\it analyse} et la {\it synthèse} (\S 69). Or ce qui a été dit ci-dessus nous
montre déjà que ces différentes {\it formes} peuvent intervenir, à des degrés
divers sans doute et avec quelques variantes, dans toutes les {\it espèces}
de la connaissance. C’est ainsi que la connaissance vulgaire qui fait
surtout appel aux données de l'{\it intuition sensible} (\S 60), comporte
aussi, comme on l’a vu, une part considérable d'{\it interprétation}, de
{\it construction}, qui relève de la pensée discursive. On vient de voir à
l'instant que certaines doctrines philosophiques demandent à « l’intuition »,
— une intuition, à vrai dire, d’un tout autre ordre que
l'intuition sensible, — ce qu’elles ne croient pouvoir obtenir de la
pensée analytique et raisonnante. Mais, comme nous aurons principalement
à étudier dans ce qui suit la connaissance scientifique,
il importe surtout de remarquer que celle-ci ne met pas en jeu d’{\it autres
facultés de l'esprit} que celles sur lesquelles repose la connaissance la
plus courante. La Science ne prétend pas, comme certaine philosophie,
avoir accès à des procédés de pensée privilégiés, tels qu’une intuition
supra-sensible ou supra-intellectuelle. Elle utilise les mêmes procédés
que la connaissance vulgaire, mais en les précisant, en les rendant
plus méthodiques, plus rigoureux et, en définitive, plus objectifs et plus
sûrs. C’est ce qu’affirmait Descartes au début de ses {\it Règles pour la
direction de l'esprit}, où il écrivait que « toutes les sciences ne sont
%68
rien d’autre que la sagesse humaine qui reste une et toujours la même,
quelle que soit la différence des sujets auxquels on l’applique »,
sagesse qui n’est elle-même que le bon emploi de cette « lumière naturelle »
qu’est en nous la Raison. — C’est pourquoi il y a lieu d'étudier
de plus près ces différentes formes de la connaissance.

\section{La notion d’intuition}% 58.
Certaines philosophies, dites {\it intuitionistes},
ont attribué, nous l'avons dit, à l'intuition une très grande
importance et une très grande valeur, la considérant, dans la Science
même, comme seule féconde, et comme seule capable, au delà de la
Science, d'atteindre l’objet de la Métaphysique, l'absolu, l'être « en
soi ». En réalité, il n’est pas de concept plus confus que cette notion
d’{\it intuition}, et on désigne sous ce nom des modes de pensée très différents
{\scriptsize (« Je crois, écrivait autrefois le philosophe Alfred Fouillée, que le mot intuition,
métaphore empruntée au sens de la vue, devrait être banni d’une philosophie rigoureuse
ou ne devrait être employé qu'avec une définition précise. » (Voc. de Lalande, sub V°.))}.
Cette équivoque tient elle-même à l'ambiguïté du terme
\textbf{\textit {immédiat,}} qui peut désigner deux notions, non seulement distinctes,
mais même opposées. 1° Dans l’usage journalier du terme, l'{\it immédiat},
c'est ce qui est \textbf{\textit {psychologiquement}} premier pour l'esprit déjà éduqué,
ce qui nous est donné tout fait par le sens commun ou la conscience
morale. Mais on verra bientôt (\S 59-60) que ce « tout fait » peut être,
en réalité, le produit de toute une élaboration, de toute une construction,
et que sa simplicité est entièrement illusoire : c’est ce qui arrive,
on le sait, pour notre représentation courante des objets extérieurs.
— 2° Dans un second sens, le seul correct, l'{\it immédiat}, c’est ce qui est
premier \textbf{\textit {en droit,}} ce qui est réellement simple, indécomposable ; c’est
le donné pur par opposition au construit. Serait seule proprement
{\it intuitive} la pensée qui saisirait de l'immédiat en ce second sens.

\section{L'intuition psychologique}% 59.
Or il est facile de voir que la
seule réalité qui nous soit, en ce sens, immédiatement donnée, en
dehors de toute interprétation ou construction, c’est {\it notre propre
pensée}, ce sont {\it nos propres états de conscience}, tels qu’ils s’écoulent en
nous, comme par exemple une sensation de douleur, un sentiment de
tristesse, etc. (\S 22). La seule intuition véritable serait donc l'intuition
{\it psychologique}, celle qui nous fait saisir, comme disait Bergson,
« notre moi qui dure ». Il y a là une « vision directe de l’esprit par
l'esprit, vision qui se distingue à peine de l’objet vu, connaissance qui
est contact et même coïncidence ». — Encore est-il nécessaire de bien
distinguer ici ce qui est vraiment {\it donné}, donc {\it intuitif}, de tout ce que
%69
notre esprit y surajoute le plus souvent presque aussitôt. C’est ainsi
que {\it nous interprétons} notre douleur, notre tristesse, en leur supposant
telle ou telle {\it cause} : il est évident que ceci n’est plus donné par l’intuition.
Le danger est particulièrement grave pour les sentiments complexes,
les habitudes, etc., qui ont été formés en nous par l’éducation : « Il
arrive que l’on prenne pour simple et immédiat ce qui est, en réalité,
complexe, ce qui résulte d’une progressive élaboration. Nos habitudes
acquièrent peu à peu une simplicité, une sûreté et une perfection qui
leur donnent toutes les apparences de l’instinct et effacent jusqu’à la
moindre trace des tâtonnements dont elles sont le fruit » (G. Davy).
Il peut même arriver que je me suggestionne parfois quelque peu
moi-même et que je m’imagine éprouver spontanément tel ou tel sentiment,
une tristesse par exemple, qui n’a pas de racines profondes en
moi. Enfin, c’est par un morcelage quelque peu arbitraire que je distingue,
dans le flux continuellement mouvant de mes états intérieurs,
telle sensation ou tel sentiment. La seule donnée pure de l’intuition,
ce serait donc, ainsi que l’a montré Bergson, cet écoulement
continu des états de conscience en moi. Mais il est évident qu’une telle
intuition n’est pas une connaissance : elle consiste à {\it sentir}, à {\it éprouver},
à {\it vivre} ce qui se passe en nous, mais non pas à le {\it connaître} ; car
toute connaissance suppose, d’une part, un sujet connaissant qui se
distingue de l’{\it objet} à connaître et, d'autre part, des éléments quelque
peu {\it stables} et {\it définissables}.

\section{L’intuition sensible}% 60.
N’existe-t-il pas cependant une forme
de l’intuition, celle que nous fournissent nos sens, qui nous fait saisir
une réalité extérieure à nous ? Lorsque je perçois une chose par la vue,
l’ouïe, le toucher, etc., l’intuition ne me met-elle pas en contact direct
avec un \textbf{\textit {objet}} étranger à ma conscience ? C’est en effet ce que croit le
sens commun et c’est aussi la thèse que prétend ressusciter certaine
Phénoménologie contemporaine ; mais il suffit de réfléchir un instant
pour comprendre qu’il y a là une illusion et que le cas de « l'intuition
sensible » se ramène en réalité à celui de l'intuition psychologique.
Quelles sont, par exemple, les « données immédiates » que me fournit
la vue ? C’est une certaine forme plus ou moins éclairée ou ombrée,
une certaine couleur, peut-être, comme l’a admis la Psychologie de
la Forme (\S 81), une certaine « structure » (dans une figure géométrique
par exemple). Rien de plus. Or il est bien clair que ces propriétés
que j’attribue à un objet extérieur, ne me sont immédiatement
données (en admettant même que cet objet existe réellement) qu’à
titre de {\it sensations}, c’est-à-dire de {\it modifications de ma conscience}, à tel
point que je pourrai en éprouver de toutes semblables en rêve. Il
%70
apparaît donc que la connaissance que nous avons des objets extérieurs,
est en réalité une connaissance {\it médiate}, en ce sens que nous ne
connaissons les « choses » que par l'intermédiaire des impressions
qu’elles font sur notre conscience. Certes, celles-ci ne sont pas d'emblée
saisies comme « subjectives », mais elles ne nous donnent pas non
plus de façon explicite la notion d’une réalité extérieure (cf. chap VII
et VIII) ni celle de toutes les propriétés constituantes d’un « objet ».
Il suffit d’ailleurs de songer à ce qu’évoque immédiatement (mais
au {\it premier sens} de ce terme) la perception visuelle d’un objet pour
se rendre compte qu'aux données des sens, notre esprit {\it surajoute}
quantité de propriétés que nous ne connaissons que par notre expérience
antérieure. L’étude de l’attention et de la perception (\S 47-51
et chapitre V) nous révéla déjà le rôle des « schèmes préperceptifs ».
De même, nous croyons « voir » qu’un fruit est mûr ou ne l’est
pas ; mais il est bien clair que nous {\it interprétons} ici certaines données
de la vue (la couleur notamment), autrement dit : nous {\it jugeons}. Et il
n’en est pas autrement pour les autres sens (ci-dessous \S 87 B). Rien
de tout cela n’est immédiat, au sens propre du terme. La distinction
même du {\it sujet} percevant et de l’{\it objet} perçu, qui implique une relation
entre deux termes distincts, est beaucoup moins immédiate qu’on ne
le croit, et la psychologie de l’enfant a établi qu’elle est, en réalité,
acquise. La {\it perception} est donc, pour la majeure partie, une construction
de l'esprit.

Il n’en est pas moins vrai que l'intuition sensible est la base de
notre connaissance du monde extérieur. Depuis l’avènement de la
science expérimentale, les savants ont renoncé à faire la science par
pure construction ou déduction conceptuelle, avec l’aide de la seule
raison. L’expérience est devenue indispensable, et l'intuition sensible
est le point de départ nécessaire de l'expérience. Ainsi que l’écrivait
Aristote, « la sensation n’est point la connaissance ; mais qui
n’aurait pas la sensation, ne pourrait rien connaître ».

\section{L’intuition intellectuelle}% 61.
Il existe cependant une autre
forme d’intuition, toute différente, dont on peut dire qu’elle est vraiment
connaissance, peut-être même la seule vraie forme de connaissance :
c’est l’{\it intuition intellectuelle}, celle qui nous fait saisir des rapports.
Mais elle peut prendre deux formes différentes.

{\it A.} La première est \textbf{\textit {l'intuition d'invention :}} c’est un fait que, dans
la science notamment, la découverte s'effectue souvent grâce à un
trait de lumière soudain qui donne tout de suite au savant le sentiment
d’avoir trouvé : c’est l’{\it euréka} d’Archimède. Mais on verra
bientôt (chap. X) que cette intuition n’est rien de plus qu’une {\it anticipation}
%71
encore confuse de la connaissance claire, un {\it pressentiment des
rapports} (c’est d’ailleurs le sens courant du mot {\it intuition} : avoir
l'intuition que... c’est pressentir). C’est une forme encore enveloppée,
donc provisoire de la connaissance et qui, en dépit du sentiment de
certitude et de clarté qui l'accompagne presque toujours, peut souvent
être trompeuse.

{\it B.} Tout autre est \textbf{\textit {l'intuition d’évidence}} proprement dite, celle
qui nous fait saisir, sans aucun doute possible, la clarté d’une idée ou
encore la vérité et même la nécessité logique d’un axiome mathématique
(\S 252 A), ou encore celle qui nous fait synthétiquement saisir la
liaison logique des différentes articulations d’un {\it raisonnement}. Dans
ce dernier cas, c’est en somme, comme l’a montré Descartes ({\it Règles
pour la direction de l'esprit}, XI) une intuition {\it récapitulative} qui nous
fait apercevoir d’un seul coup d’œil ({\it uno intuitu}, disait-il) le rapport
ou l’ensemble des rapports précédemment établis par la pensée analytique
et le raisonnement. C’est pourquoi il recommandait de parcourir
« d’un mouvement continu de l'esprit et sans interruption de la
pensée » la suite des arguments d’une démonstration jusqu’à ce qu’on
les aperçoive « tous à la fois » comme les anneaux d’une seule et unique
chaîne. Mais, si l’on y réfléchit bien, on s’apercevra que même les
principes premiers, tels que l’axiome : « Deux quantités respectivement
égales à une même troisième sont égales entre elles », ne sont
saisis intuitivement et avec évidence qu’à la suite d’un mouvement
rapide de l’esprit qui parcourt l’enchaînement des différents rapports
dont il est composé (on « voit » que, si A = C et si B = C, nécessairement
A = B). — Ainsi, tandis que l’intuition d'invention {\it précède}
l’aperception distincte des rapports, l'intuition d’évidence la suit.

\section{L’intuition métaphysique}% 62.
Enfin, certains philosophes
ont admis une {\it intuition métaphysique} qui nous permettrait d’appréhender
directement : {\it a.} soit une \textbf{\textit {existence,}} comme chez Descartes,
selon lequel, dans tout acte de pensée, la pensée se saisit elle-même
comme âme, comme substance spirituelle existant en soi; car tel est
le sens du « je pense, donc je suis» ({\it cogito, ergo sum}) qui, en dépit de
sa forme, est bien, aux yeux de Descartes, une intuition, et non un
raisonnement ; nous montrerons (chapitre XXVII, \S 334 fin) le rôle
de cette intuition d’existence comme point de départ de la Métaphysique ;
— {\it b.} soit des \textbf{\textit {essences pures,}} des structures universelles
et extra-temporelles, indépendantes des faits empiriques, comme
dans la {\it Wesensschau} (intuition des essences) du fondateur de la {\it Phénoménologie},
le philosophe allemand E. Husserl, ou dans l'intuition
%72
émotionnelle des valeurs de Max Scheler (t. II, chap. XIV, \S 131) ;
— {\it c.} soit enfin une \textbf{\textit {réalité}} qui est à la fois essence et existence, comme
chez Bergson, pour qui l'intuition qui est « sympathie » avec l’objet
à connaître, nous permettrait de saisir cet objet du dedans, en coïncidant
avec ce qu’il a d’original et d’unique (\S 326).

\section{La pensée discursive et le raisonnement}% 63.
Par opposition
à l'intuition, le raisonnement est une forme de pensée \textbf{\textit {discursive,}}
c’est-à-dire qui exige la {\it médiation} de certains éléments. Quels sont
ces éléments? Du point de vue {\it logique}, on peut distinguer bien des
formes du raisonnement : car ce qui importe alors, c’est la {\it valeur} du
raisonnement, or ces différentes formes sont loin d’avoir toutes même
valeur. Mais, à travers ces différences, l’analyse psychologique
(chap. XVI) découvre quelque chose de commun, qui réside précisément
dans la nature de cet élément {\it médiateur}. Cet élément n’est autre
que le \textbf{\textit {concept,}} c'est-à-dire l’idée abstraite et générale (ci-dessous,
\S 206). Le raisonnement est une {\it reconstruction par concepts} de ce qui
nous est donné d’abord, soit sous forme sensible, soit sous forme d’intuition
anticipatrice (\S 61 A). Poser un concept, c’est en effet poser une
sorte de \textbf{\textit {loi générale}} de tout un groupe d’être ou d’objets mentaux, et
c’est cette généralité qui va nous expliquer la fécondité du raisonnement.
Le raisonnement est ainsi une rationalisation, une élévation au
niveau de l’intelligible de ce qui n’était que {\it senti}, {\it éprouvé} ou {\it pressenti}.

Le raisonnement se présente sous différentes {\it formes} : {\it déduction},
{\it induction}, {\it raisonnement par analogie}, etc, qui seront étudiées au
chapitre XVI. Nous voudrions insister surtout ici sur son {\it rôle}, par
rapport à celui de l'intuition.

\section{Rôles respectifs de l’intuition et du raisonnement}% 64.
Intuition et raisonnement collaborent sans cesse dans les opérations
de la pensée, mais y jouent des rôles différents.

{\it A.} Le premier rôle de l’intuition est de \textbf{\textit {fournir des données}} à la
pensée. On ne saurait trop insister sur l’importance des données
empiriques fournies par l'intuition sensible aux sciences expérimentales
(\S 60). De même, les données de l'intuition psychologique
(\S 59) sont la base de notre connaissance de nous-même et des autres
et peuvent être le point de départ d’une métaphysique (\S 62).

{\it B.} Un second rôle de l'intuition est d’\textbf{\textit {anticiper}} sur la pensée discursive :
c’est l'intuition d'{\it invention}. Il s’agit alors d’une intuition
{\it intellectuelle} qui n’est que l’aperception rapide des rapports. Dans
la science, cette intuition, très importante et très féconde puisqu'elle
%73
est la source de presque toutes les découvertes, ne suffit cependant
jamais : elle a toujours besoin d’une preuve.

{\it C.} Ce sera là, précisément, le rôle du raisonnement. \textbf{\textit {Par l'intuition,
on trouve ; mais c'est par le raisonnement seul qu’on prouve.}}
Prouver, en effet, ce n’est pas autre chose qu’intégrer la proposition
à prouver dans un système intellectuel, dans un système qui ne soit
plus un simple {\it donné}, mais qui soit compénétrable à l’esprit parce
que, partiellement au moins, construit par lui, qui soit, en un mot,
{\it de l’intelligible}. C'est dans la déduction que cette reconstruction est
la plus parfaite parce qu’effectuée avec de purs concepts, comme en
mathématiques : d’où la {\it certitude} plus grande du raisonnement
déductif. Dans l'induction, c’est-à-dire dans le raisonnement expérimental,
cette certitude semble moins grande parce que la reconstruction
conceptuelle y est mêlée d’éléments empiriques (observations,
expériences, mesures) et que le résultat n’est valable qu’autant qu’il
est {\it vérifié} par l’expérience. Dans tous les cas cependant, c’est bien au
raisonnement, en tant que reconstruction par concepts, qu’appartient
la preuve.

{\it D.} Un dernier rôle enfin échoit à l’intuition. C’est, comme on l’a vu
(\S 61 B), de \textbf{\textit {résumer,}} de \textbf{\textit {récapituler}} les démarches successives du
raisonnement, afin de les apercevoir comme en un tableau, en une
vue unique de l’esprit. Mais il va de soi que cette intuition, d’ordre
intellectuel encore, qui parachève l’œuvre du raisonnement, suppose
préalablement la pensée discursive. Elle est même, en un sens, immanente
à celle-ci : car celle-ci suppose l’aperception des liens logiques
qui en unissent les différentes articulations.

\section{Analyse et Synthèse}% 65.
Outre le raisonnement proprement
dit, la pensée discursive met en jeu deux procédés très généraux
de la pensée qui jouent un rôle important dans la connaissance : ce
sont l'{\it analyse} et la {\it synthèse}. Nous avons montré (\S 46 B) qu’outre
l’activité purement conservatrice qui laisse l'esprit esclave du passé
et du « tout fait » (\S 39), il existe en lui deux fonctions supérieures
qui sont : 1° une activité de dissociation par laquelle il est capable de
séparer les éléments d’un tout mental ; 2° une activité de {\it synthèse} par
aquelle il construit des ensembles {\it nouveaux}. On a constaté que ces
deux fonctions forment l'essentiel du {\it jugement} (voir les \S 170 et 173),
qui est lui-même l'opération fondamentale de la pensée réfléchie.
L'{\it analyse} et la {\it synthèse} ne sont que la mise en œuvre, sous forme de
procédés logiques, méthodiques, de ces deux fonctions de l’esprit
(ce qui confirme l’idée cartésienne indiquée au \S 57). — D’une façon
%74
générale, on peut observer que notre pensée, chaque fois qu’elle se
trouve en présence d’un objet à connaître, procède selon un \textbf{\textit {rythme
en trois temps}} : « Toute connaissance, a-t-on dit, est une analyse entre
deux synthèses. » À vrai dire, la première phase ne mérite guère ce
nom de synthèse. C’est plutôt, comme a dit Renan, un \textbf{\textit {syncrétisme,}}
c’est-à-dire une vue globale, mais {\it confuse}, dans laquelle les éléments
ne sont pas encore distingués. Supposons par exemple que nous cherchions
à comprendre le fonctionnement d’une machine un peu
compliquée. Nous commençons par l’examiner dans son ensemble en
essayant de voir à quel usage elle répond, quel travail elle accomplit.
Puis, dans une seconde phase qui est précisément celle de l’\textbf{\textit {analyse,}}
nous en distinguons les différents mécanismes, par exemple, dans une
automobile, le moteur, le carburateur et l’alimentation en essence,
l’appareillage électrique, le refroidissement, le changement de
vitesses, etc. Au besoin, si nous le pouvons, nous démontons la
machine, pour mieux distinguer ses éléments, pour mieux comprendre
la structure et la fonction de chacun d’eux dans l’ensemble. Enfin,
dans une troisième phase, la \textbf{\textit {synthèse,}} nous replaçons (par la pensée ou
réellement) chacun de ces éléments dans l’ensemble, et nous obtenons
une nouvelle vue globale, mais cette fois {\it distincte} parce qu’elle est
appuyée sur une analyse préalable.

\section{L’Analyse : ses différentes formes}% 66.
Nous définirons donc
l'analyse la \textbf{\textit {décomposition d’un tout en ses éléments}} ou, si l’on veut,
la \textbf{\textit {réduction du complexe au simple.}} Il importe de remarquer en effet
n’est pas plus simple que le tout. Elle est seulement {\it plus petite}, soit
dans l’espace (si je brise un morceau de craie, chacun de ses fragments
est plus petit, mais c’est encore du carbonate de chaux), soit dans
le temps (si je distingue dans l’histoire différentes périodes, chacune
d’entre elles comprend encore des faits économiques, politiques, diplomatiques,
culturels, etc., tout comme l’ensemble), soit dans l’extension
{\footnotesize (Les logiciens appellent extension d'un terme l’ensemble des êtres, objets ou faits
qu'il désigne, tandis qu'ils nomment compréhension l'ensemble des caractères, propriétés,
etc., attribuables à ce terme)}
logique : dans ce dernier cas, l’opération logique appelée {\it division}
aboutit même à des résultats {\it plus complexes}, car les espèces
d’un genre sont plus riches en compréhension que le genre ({\it Mammifère},
{\it Oiseau}, etc., sont plus complexes que {\it Vertébré}). À plus forte raison,
ne faut-il pas confondre {\it éléments} et {\it détails} : les détails sont ce qui
{\it singularise} quelque chose, ce qui donne par exemple aux faits, dans
%75
une période historique, leur « couleur locale » ; ils permettent donc
de mieux {\it décrire}, mais non pas de mieux {\it comprendre}. Ce qui fait,
au contraire, l'intérêt intellectuel de l'analyse, c’est qu’en ramenant
le complexe au simple, elle constitue un progrès dans l'{\it explication}
des choses. L’élément est d’ailleurs {\it général} : il se retrouve le même
dans plusieurs cas de même espèce et c’est pourquoi l’analyse, en nous
faisant discerner les éléments, nous permet aussi de saisir {\it leurs
rapports}. Elle prépare ainsi la synthèse.

L'analyse peut être, soit purement {\it idéale}, soit {\it réelle}. L'analyse
\textbf{\textit {idéale}} est celle qui ne porte que sur les {\it idées} des choses, non sur les
choses elles-mêmes ; elle consiste à décomposer les choses seulement
{\it en pensée}. Quand les auteurs du {\footnotesize XVIII}$^\text{e}$ siècle imaginent des états
fictifs pour expliquer des faits plus complexes (par exemple, l’« état
de nature » pour rendre compte de l’état de société), ils font une analyse
purement {\it idéale} ou idéologique qui ne porte pas sur l’objet même à
connaître. De même, lorsqu'un romancier « analyse » les sentiments
d’un de ses personnages, il se borne à décomposer ceux-ci {\it par la
pensée}. L'analyse \textbf{\textit {réelle}} est, au contraire, celle qui isole les éléments
dans l’objet lui-même. Telle est l’analyse chimique lorsque, par
exemple, dans le carbonate de chaux, elle isole du carbone, du calcium
et de l’oxygène. Mais il ne faudrait pas croire que l’analyse {\it réelle}
soit nécessairement {\it matérielle}. Je fais bien une analyse réelle en
Physique lorsque je distingue dans un courant électrique l’intensité,
la tension, la résistance du conducteur et que je les fais varier expérimentalement
indépendamment les unes des autres. D’autre part, l’analyse
peut porter sur un phénomène d’ordre {\it spirituel}, et l’on verra
(ci-dessous, chap. XXIV) qu’il existe en Psychologie des méthodes
d’analyse {\it réelle}, et non pas seulement idéale comme celle du romancier
dont nous parlions plus haut.

Enfin, selon que l’objet sur lequel elle porte est d’ordre abstrait
ou d'ordre empirique, l’analyse sera {\it rationnelle} ou bien {\it expérimentale}.
Elle sera \textbf{\textit {rationnelle}} (et, par suite, {\it }idéale) en Mathématiques où il
s'agira, par exemple, de remonter d’une proposition complexe aux
propositions plus simples sur lesquelles elle s’appuie (démonstration
régressive, analyse algébrique : \S 254 D). Elle sera \textbf{\textit {expérimentale}} dans
les sciences de faits, comme la Physique ou la Physiologie, où le plus
souvent elle se confondra avec l’{\it expérimentation} (\S 262 et 275 C) et
l'{\it induction} (\S 264).

\section{La Synthèse : ses différentes formes}% 67.
La synthèse est
l'opération réciproque de l'analyse : elle consiste à \textbf{\textit {reconstituer le
tout à l’aide des éléments}} distingués par l'analyse et redescend donc
%76
du simple au complexe. Comme l’analyse, elle peut être : 1° {\it idéale}
(construction de théories) ou {\it réelle} (synthèse chimique) ; 2° {\it rationnelle}
ou {\it expérimentale}. Elle est \textbf{\textit {rationnelle}} en Mathématiques où elle
se confond avec la {\it déduction constructive} (ci-dessous, \S 254 A) qui, selon
le précepte formulé par Descartes dans sa troisième règle, part des
notions les plus simples « pour monter peu à peu comme par degrés »
jusqu'aux plus complexes : il s’agit alors d’un véritable « édifice
logique » qui se construit petit à petit et dont la Géométrie nous
offre le meilleur exemple. Dans les sciences de faits, la synthèse est
\textbf{\textit {expérimentale,}} soit lorsqu'elle construit ces vastes systèmes qu’on
nomme {\it théories} ou {\it grandes hypothèses} (\S 268) et qui, s’ils ne sont pas
vérifiables directement, le sont cependant par leurs conséquences,
soit aussi dans le passage des lois {\it générales} aux cas {\it particuliers}. On
l’a dit ci-dessus (\S 66) : c’est le général qui est le simple ; chaque
cas particulier est constitué, au contraire, par un enchevêtrement
toujours complexe d’influences diverses. Il n’est pas, dans la nature,
de phénomène réel qui obéisse à une seule loi : même un corps qui
tombe obéit à la fois à la loi de la pesanteur et à celle de la résistance de
l'air. C’est pourquoi, dès qu’il s’agit d’expliquer un fait particulier
(par exemple un accident de chemin de fer, d'avion, etc.), il s’introduit
toujours une certaine marge de {\it contingence} : car ce fait est la
résultante de plusieurs lois ou causes dont il faudrait faire la synthèse,
mais qui ne sont peut-être pas toutes intégralement connues. On
verra (\S 239) que ce problème se pose de façon particulièrement aiguë
dans le passage de la {\it théorie} à la pratique.

\section{Rôles respectifs de l’analyse et de la synthèse}% 68.
Il résulte de là que l’analyse et la synthèse n'auront pas exactement
le même rôle. —— {\it A.} En principe, on peut dire avec la {\it Logique de
Port-Royal} que l'analyse est une « méthode d'invention » et la synthèse
une « méthode de doctrine » (c’est-à-dire d’enseignement). On
comprend en effet que, par sa marche {\it ascendante} qui « remonte »
du donné aux éléments simples, l’analyse est surtout un procédé
de \textbf{\textit {recherche}} ou de \textbf{\textit {découverte,}} tandis que la synthèse, suivant une
marche descendante qui « progresse » des éléments simples au complexe,
peut servir à \textbf{\textit {enseigner}} à autrui comment on passe, selon un ordre
logique (celui des théorèmes de la Géométrie par exemple), de ceux-là
à celui-ci. — {\it B.} Toutefois, il peut y avoir intérêt pratiquement à initier
celui qu’on veut instruire, non pas à la vérité toute faite et mise en
ordre logique, mais à {\it l’art même de la recherche}, à lui faire {\it trouver par
lui-même} ce qu'il a à apprendre : en ce cas, c’est la méthode analytique
qui conviendra. Inversement, la synthèse est souvent {\it féconde} et l’on
%77
verra (\S 268 {\it fin}) que, dans la science, les cas ne sont pas rares où elle
permet de {\it découvrir} du nouveau. — {\it C.} Mais il faut remarquer surtout
que l’analyse et la synthèse sont bien moins deux méthodes distinctes
que deux procédés \textbf{\textit {complémentaires}} l’un de l’autre. On l’a vu ci-dessus :
il n’y a pas de véritable synthèse sans analyse préalable, et
réciproquement la synthèse constitue presque toujours la {\it contre-épreuve}
indispensable de l’analyse, comme on le voit bien par l'exemple
de la Chimie.

\section{Les abus de l'analyse et de la synthèse : l’esprit de
système}% 69.
Si on les isole l’une de l’autre, l’analyse et la synthèse
risquent d’aboutir à des excès qui, en définitive, ont le même résultat :
les {\it simplifications abusives} et, comme le disait Claude Bernard,
« l’absence du sentiment de complexité des phénomènes naturels ».
L’analyse donne l’habitude de rechercher partout l’élément simple :
« C’est pourquoi, dit Cl. Bernard, nous voyons quelquefois des mathématiciens,
très grands esprits par ailleurs, tomber dans des erreurs
de ce genre : ils simplifient trop et raisonnent sur les phénomènes
tels qu’ils les font dans leur esprit, non tels qu’ils sont dans la nature.»
On a même beaucoup critiqué, de nos jours, la {\it pensée analytique} qui,
dit-on, aboutit à des résultats artificiels. Certes, il est des analyses
maladroites (ci-dessous, \S 247 {\it fin}, \S 310 {\it A}, etc.). Mais l’analyse, et
l’abstraction qui en est le résultat, sont la condition de toute connaissance
claire et distincte, de tout « discours » où l’on veut savoir ce que
l’on dit. — Au reste, la synthèse peut aussi avoir ses excès : ce sont
les constructions aventureuses, édifiées sur des bases insuffisantes ;
c’est le travers de l’esprit philosophique. La synthèse dégénère alors
en \textbf{\textit {système.}} Analysons cette idée de {\it système} et voyons d’abord les
{\it avantages} des systèmes.

Le terme de {\it système} n'implique pas nécessairement en effet une
idée péjorative. D’une façon générale, un système est un {\it ensemble
de choses ou d'idées considérées comme soumises à un principe unique}
ou du moins à un petit nombre de principes. Il y a des systèmes
{\it matériels} : le « système planétaire », le « système neryeux », etc.
Mais le plus souvent le mot s'applique à des ensembles {\it intellectuels}.
En ce sens, on peut dire, avec Condillac, qu'« un système n'est
autre chose que la disposition des différentes parties d’un art ou d’une
science dans un ordre où elles se soutiennent toutes mutuellement
et où les dernières s'expliquent par les premières. Celles qui rendent
raison des autres s'appellent {\it principes} ; et le système est d'autant
plus parfait que les principes sont en plus petit nombre : il est même
à souhaiter qu’on les réduise à un seul ». Il y a des systèmes {\it pratiques}
%78
 : systèmes de mesure (système métrique, système C. G. S., M. T. S.
ou M. K. S. A., etc.), de classification (système de Linné), de gouvernement
(système représentatif, le « système continental » de Napoléon),
financiers (le système de Law), d'éducation (le système de Pestalozzi),
etc. ; — et des systèmes {\it théoriques} (le mot est alors à peu près
synonyme de {\it doctrine} ou de {\it théorie}) : le système de Copernic, le
système de Leibniz, etc. L'avantage des systèmes est évident : 1° ils
constituent une {\it économie de pensée}, puisqu'ils permettent de rattacher
une multiplicité d'éléments à un ou à quelques principes ;
2° ils soulagent la {\it mémoire} : en médecine par exemple, dit Laënnec,
« les faits nombreux et disparates ne se classent dans la mémoire qu’à
l’aide d’un lien systématique » ; 3° enfin et surtout, en {\it coordonnant} les
éléments et en les {\it subordonnant} à un petit nombre de principes, ils y
introduisent un {\it ordre logique}, donc de l’unité et de la clarté ; ils satisfont
à ce besoin essentiellement rationnel de l'esprit (voir le
chap. XVII) de ramener le {\it divers} au {\it même}. — On conçoit cependant
que cette unité puisse être parfois {\it artificielle} : le réel ne se plie pas
toujours à de telles simplifications. Le danger est alors que l’on
s'efforce de plier à tout prix à l’unité du système des éléments qui
refusent d’y entrer parce que le système était trop étroit. \textbf{\textit {L'esprit
de système}} confine à l’esprit {\it dogmatique} : il rend aveugle pour la
complexité du réel, la richesse et la diversité des choses. Dans la pratique,
« l’homme à systèmes » sera celui qui apporte des solutions
toutes faites, alors qu’il s’agit d’{\it inventer} et de se rénover. L'esprit
de système aboutit alors à une cristallisation de la pensée.

\section{Sujets de travaux}% SUJETS DE TRAVAUX


{\bf Exercices.} — 1. {\it Étudier comment peuvent s'appliquer à la connaissance
d'autrui les distinctions indiquées au début du \S 53 pour la connaissance des
objets physiques}. — 2. {\it Distinguer et classer les différents sens du mot système
et de ses composés dans les phrases suivantes} : « La nation moscovite faisait
une nation à part, qui n’entrait pas dans le système de l’Europe » (Fontenelle),
« Il importe qu’un bon système économique ne soit pas un système
de finances ou d'argent » (Rousseau), « Descartes est le premier qui ait
traité le système du monde avec quelque étendue » (D'Alembert), « C'est
une fureur systématique, telle qu'on en voit en Italie » (Stael), « La vraie
philosophie se propose de systématiser, autant que possible, toute l’exis-
tence humaine, individuelle et surtout collective» (Comte), « La science
est la connaissance réfléchie, systématique, méthodique » (Rémusat),
« Loin de se roidir, comme le systématique, contre l'expérience, le savant... »
(Cl. Bernard), « Le savoir, quoi qu’on fasse, est un système » (Hamelin),
« La philosophie a changé de systèmes, donc elle se trouvait à l’étroit dans
les systèmes » (Boutroux), « Il y a du système et du travail dans cette
attitude parfaitement triste et dans cet absolu de dégoût » (Valéry, à
%79
propos de Pascal), « Les systèmes tendent à asservir l'esprit humain »
(Bergson), « L'unité doctrinale qui régna jusqu'à la Renaissance interdisait
le système personnel parce qu’elle imposait une structure mentale
commune » (Bréhier).

{\bf Exposés oraux.} — 1. {\it L’ignorance et l'irréflexion : leur action sur la pensée}
(voir Gérard Varer, {\it L'ignorance et l'irréflexion}, Alcan, 1899). — 2. {\it La
notion du « bon sens » au sens vulgaire et au sens cartésien} (voir {\it Disc. de la
Méthode}, début, et {\it Regulae}, I ; Brunschvicg, {\it L’Exp. humaine et la causalité
physique}, p. 572). — 3. {\it Les idées essentielles de Condillac dans le} Traité des
systèmes.

{\bf Discussion.} — 1. {\it La connaissance vulgaire nous met-elle sur le chemin de
la science ou s'oppose-t-elle à celle-ci ?} — 2. {\it Discuter la pensée de Fouillée
citée page 69 note}.

{\bf Lectures.} — {\it a.} {\it Logique de Port-Royal} (1662), 3$^\text{e}$ partie, chap. XX et
4$^\text{e}$ partie. — {\it b.} Spinoza, {\it Éthique} (1677), éd. Classiques Larousse, p. 6 et 66.
— {\it c.} A. Cournot, {\it Essai sur les fondements de nos connaissances} (1851),
chap. I, VII, XVI et XVII. — {\it d.} A. Binet, {\it L'Étude expérimentale de l’intelligence},
Alcan, 1903. — {\it e.} Th. Ribot, {\it La logique des sentiments}, Alcan, 1905.
— {\it f.} E. Goblot, {\it Traité de Logique}, A. Colin, 1918, chap. XI à XVII. —
{\it g.} A. Cresson, {\it Les Réactions intellectuelles élémentaires}, Alcan, 1922. —
{\it h.} Éd. Le Roy, {\it La Pensée intuitive}, Boivin, 1930, tome II. — {\it i.} M. Dorolle,
{\it Le Raisonnement par analogie}, P. U. F., 1949. — {\it j.} Divers, {\it La Synthèse,
idée force dans l'évolution de la pensée}, Albin Michel, 1951. — {\it k.} Jean Lechat,
{\it Analyse et Synthèse}, P. U. F., 1962. — {\it l.} P. Oléron, {\it Les activités intellectuelles},
P. U. F., 1964.

