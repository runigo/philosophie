\chapter{Dieu}
%chapitre XXIX DIEU
%SOMMAIRE
%352. Il existe un problème philosophique de Dieu. — 353. Les diverses
%conceptions de Dieu : 1° le Dieu des philosophes anciens. — 354. Les diverses
%conceptions de Dieu : 2° le Théisme moderne. — 355. Les diverses conceptions
%de Dieu : 3° le Panthéisme. — 356. L'Athéisme. — 357. Preuves classiques
%de l'existence de Dieu. — 358. Primauté de l'esprit. — 359. Être et
%Valeur. — 360. La participation. — 361. Dieu substance et cause. — 362. Dieu
%personne. — 363. Dieu fin suprême.

\section{Il existe un problème philosophique de Dieu}% 352.
« Le
problème de l'existence et des attributs de Dieu, écrivent Janet et
Séailles, est né manifestement des réflexions suggérées à la pensée
humaine par les croyances populaires qui sont le fond de toute religion. »
L'idée de Dieu nous est en effet apportée par la tradition.
Mais, s’il est vrai, comme l’a dit Lachelier, que le rôle de la philosophie
est de chercher à « tout comprendre, même la religion », il lui
appartient de repenser les idées qu’elle reçoit ainsi et de se poser
à ce sujet un problème.

\vspace{0.24cm}
{\footnotesize L'existence de ce problème ne pourrait être niée que de trois points
de vue : 1° du point de vue d’un \textsf{\textit {agnosticisme}} radical (voir p. 567, note 3)
qui considérerait comme inaccessible à la connaissance humaine tout ce qui
concerne l'Être absolu, les causes premières et les fins dernières de l’univers ;
mais on a vu (\S328) qu'un tel agnosticisme n’est pas justifié et que,
par un certain biais tout au moins, la pensée peut se mettre en prise sur
l'Être ; — 2° du point de vue d’un \textsf{\textit {fidéisme}} pur (lequel peut d’ailleurs se
doubler d’un agnosticisme philosophique) qui attribuerait à la {\it foi} religieuse
et à une {\it révélation} surnaturelle le privilège exclusif de nous faire connaître
Dieu ; mais alors la foi apparaît comme bien gratuite : comment choisir
entre les différentes croyances qui se présentent à nous dans les diverses
religions ? Malebranche fait dire à l’un des interlocuteurs de ses {\it Conversations
chrétiennes} : « Ne voyez-vous pas que la certitude de la foi vient
de l'autorité d’un Dieu qui parle et qui ne peut jamais tromper? Si donc
vous n'êtes pas convaincu par la raison qu'il y a un Dieu, comment serez-vous
convaineu qu'il a parlé ? » Il est donc nécessaire de donner à la foi
%625
un préambule rationnel
{\scriptsize (Il faut savoir d'ailleurs (car on se fait souvent des idées fausses à ce sujet) que
l'Église a toujours condamné le fidéisme pur. En 1840, l'autorité ecclésiastique imposa
à l'abbé Bautain, professeur à l'Université de Strasbourg, qui avait soutenu des thèses
fidéistés, de souscrire aux propositions suivantes : « Le raisonnement peut prouver avec
certitude l'existence de Dieu... L'usage de la raison précède la foi et y conduit l'homme,
la révélation et la grâce aidant. » Des propositions presque identiques furent imposées en
1855 à l'abbé Bonnetty, directeur des {\it Annales de Philosophie chrétienne}.)}
et « le problème de Dieu se pose tout autant du
point de vue de ceux qui ont les convictions religieuses les plus solides »
(Van Steenberghen) ; — 3° enfin du point de vue d'un \textsf{\textit {intuitionisme}} religieux
qui admettrait une présence immédiate et perceptible de Dieu à la
conscience humaine (le « Dieu sensible au cœur » de Pascal) ; mais les théologiens
font généralement les plus graves réserves sur ce point de vue. Pour
l’admettre, écrit l’un d’eux, il faudrait établir « qu'il existe pour nous,
au delà de la connaissance conceptuelle, on ne sait quelle intuition intellectuelle
qui nous permettrait d'atteindre directement ou concrètement l’Infini
personnel ». Mais, mis à part le cas de l’expérience mystique, « l'analyse de
la conscience humaine rend extrêmement improbable l'existence en nous
de la faculté mystérieuse dont on rêve... Les échecs répétés de toutes
les tentatives qui ont été faites dans ce sens au cours de l’histoire, depuis
Platon jusqu’à Bergson, laissent bien peu d’espoir aux incorrigibles utopistes
qui rêvent pour nous d’un mode de connaissance angélique ». Car
« Dieu n’est pas un objet d'{\it expérience} humaine, l’Infini n’est pas le terme
{\it immanent} de nos désirs, mais le principe et le terme {\it transcendants} de l’ordre
universels (Van Steenberghen).}
\vspace{0.31cm}

\section{Les diverses conceptions de Dieu : 1° le Dieu des philosophes
anciens}% 353.
Les philosophes ont évidemment interprété les
croyances traditionnelles dans le cadre de leurs systèmes. Il ne faut
donc pas s’étonner de voir l’idée de Dieu diversement définie en
fonction, d’une part de ces croyances, d'autre part de ces systèmes. —
On trouve parfois chez les philosophes anciens un singulier mélange
des croyances populaires polythéistes avec leurs propres conceptions.
On peut cependant caractériser dans l’ensemble le théisme
{\footnotesize (Le terme {\it théisme} désigne d'une façon générale toute doctrine, même purement
philosophique, qui affirme l'existence d'un Dieu personnel. Le mot {\it déisme} désigne une
restriction du théisme qui, tout en admettant l'existence de Dieu, rejette les religions
révélées pour se contenter d’une religion purement « naturelle », et souvent même l'idée
de la Providence.)}
des philosophes
anciens par un certain nombre de traits qui l’opposent assez
nettement au théisme des philosophes modernes. {\it A}. Chez les Anciens,
Dieu est principe d’\textbf{\textit {ordre}} et d’\textbf{\textit {intelligibilité}} beaucoup plus que
principe d'{\it existence} : il a {\it organisé} le monde, mais {\it }ne l’a pas créé. Tels
sont, par exemple, le {\it Démiurge} (du grec : {\it démiourgos}, artisan) de
Platon qui est un dieu-architecte, ou le Dieu d’Aristote, « premier
moteur » immobile, intelligence qui se contemple elle-même, mais
auquel le monde est co-éternel. {\it B.} Le Dieu des philosophes anciens
%626
n’est \textbf{\textit {pas infini.}} Au contraire, l'infini étant pour eux l’indéterminé
(grec : {\it apeiron}), ils considèrent plutôt l’infinité comme une imperfection.
{\it C.} Par suite, ils n’admettent pas que Dieu soit {\it tout-puissant} :
Dicu est \textbf{\textit {mesure, limite, détermination}} (grec : {\it peras}), mais en toute
chose il y a de l’indéterminé et « il est impossible, dit Platon, que le
mal [c’est-à-dire l’indéterminé] soit tout à fait détruit ». {\it D.} Enfin
ils n’ont que raremeñt l’idée d’un Dieu \textbf{\textit {personnel,}} doué d’attributs
{\it moraux} et veillant au bien du monde : le Dieu d’Aristote ignore le
monde et les dieux d’Épicure vivent heureux dans les « intermondes »
sans avoir à se soucier de ce qui se passe au delà. Ce n’est guère que
chez les Stoïciens, en particulier chez Marc-Aurèle, qu’on voit apparaître
l’idée d’un {\it Dieu-providence} envers lequel l’homme peut avoir
des sentiments de confiance et d’amour.

\section{Les diverses conceptions de Dieu : 2° le Théisme
moderne}% 354.
Le théisme moderne a été, au contraire, non seulement
chez les théologiens, mais même chez les philosophes, fortement
influencé par la {\it }conception judéo-chrétienne de Dieu. C’est en effet
de la Bible que vient la notion de l’\textbf{\textit {}}infinitude divine. C’est aussi
dans la Bible, tandis qu’on ne la trouve dans aucune philosophie ni
mythologie antiques, que se rencontre pour la première fois l’affirmation
d’un Dieu \textbf{\textit {créateur}}, c’est-à-dire principe d’{\it existence}, source
d’être ({\it II. Macch.}, VII, 28 : « Je t’en conjure, mon enfant : regarde
le ciel et la terre, vois tout ce qu’ils contiennent, et sache que Dieu
les a créés de rien »). Aussi le théisme moderne reconnaît-il généralement
à Dieu : {\it A.} des \textbf{\textit {attributs métaphysiques}} : Dieu est l’Être
infini, dont la grandeur est incommensurable avec celle de tout être
créé
{\footnotesize (Le philosophe Malebranche, dans une méditation sur l’Adoration en esprit et en
vérité, exprime cette idée en termes quasi mathématiques : « Seigneur, du fini à l'infini,
la distance est infinie. La créature, quelque noble et excellente qu'elle puisse être, comparée
à votre infinie Majesté, s'anéantit entièrement. L’exposant le plus juste du rapport
qu'elle a avec vous, c’est le zéro. Un grain de sable a un rapport très réel avec l'Univers.
Il ne faudrait pas cent chiffres pour exprimer ce rapport. Mais mille millions de chiffres
qui diviseraient l'unité, feraient encore une fraction trop grande pour exprimer le rapport
de l'Univers avec vous. » — Chez le théologien réformé contemporain Karl Barru,
cette idée de l'infinie distanca entre l’homme et Dieu prend la forme d’une opposition,
d'un véritable abime entre l’homme et son Créateur.)};
il est « cause première », ou, plus exactement, cause absolue,
et créateur du monde {\it ex nihilo} (c’est-à-dire qu’il l’a appelé à l’existence
à partir de rien). Il est {\it immuable}, immense (c’est-à-dire soustrait
à toute dimension et en dehors de l’espace), {\it éternel} (c’est-à-dire
en dehors du temps), {\it tout-puissant}, omruscient. Il est, en un mot,
\textbf{\textit {l'Être parfait}} au sens métaphysique du terme, c’est-à-dire l'Être
achevé (perfectus), sans limitation, qui possède en lui la {\it plénitude
%627
de l'être} ; — {\it B.} des \textbf{\textit {attributs moraux}} : Dieu est {\it }la Valeur, la {\it }perfection
morale, comme il est l’Être par excellence : il est donc \textbf{\textit {personnel}}
et possède de façon éminente, à l’infini, les qualités que nous jugeone
en nous être des {\it qualités morales} : sagesse, justice, bonté, etc. ; il est
{\it Providence}, c’est-à-dire qu’il dirige le monde en vue du bien (seul le
{\it déisme} naturaliste rejette parfois cette idée) ; enfin, à ces attributs
moraux, le Christianisme ajoute là notion d’un {\it Dieu Père}, et qui est,
avant tout, {\it amour}.

\section{Les diverses conceptions de Dieu : 3° le Panthéisme}% 355.
Aux conceptions précédentes qui distinguent Dieu et le monde,
s’opposent les conceptions {\it panthéistes}, celles qui, au contraire, posent
l’{\it unité de Dieu et du monde}. Il n’est cependant pas correct de définir
le Panthéisme, ainsi qu’on le fait parfois, comme la doctrine selon
laquelle Dieu est {\it immanent} au monde. Une telle formule ne définit
que le panthéisme vulgaire ou littéraire qui identifie Dieu et « la
nature ». Le Panthéisme philosophique ne consiste nullement à diviniser
le monde dans son existence empirique (la « nature naturée »
selon l'expression scolastique), mais au contraire {\it à considérer le
monde comme la traduction empirique ou l’émanation d'un principe
rationnel (la « nature naturante »)}.

{\it A.} Dans l’antiquité, nous signalerons seulement le panthéisme des
Stoïciens pour lesquels Dieu est « l’âme du monde ». Le monde est
en effet, pour eux, un être animé et, par suite, il y a en lui, comme dans
l’être humain, une « partie dirigeante », une âme, un principe rationnel,
qui contient en soi la raison de tous les développements ultérieurs
de l’univers (voir \S 342 C).

{\it B.} Dans la philosophie moderne, la forme la plus absolue du Panthéisme
est celle que lui a donnée Spinoza dans son {\it Éthique} (1677).
Selon Spinoza,
%(fig. 94)
Dieu est la \textbf{\textit {Substance}} unique et infinie, elle-même
constituée par une infinité d’attributs. Il est « cause de soi »,
c’est-à-dire qu’il est à lui-même sa propre raison d’être. De tous ses
\textbf{\textit {attributs}} « dont chacun exprime une essence éternelle et infinie »,
nous ne connaissons que la Pensée et l’Étendue (on retrouve ici le
dualisme cartésien, mais ramené à l’unité de la Substance, donc
transformé en {\it monisme}). Ces attributs se développent eux-mêmes en
une infinité de \textbf{\textit {modes}}, les uns {\it infinis}, tels que, dans la Pensée, l’entendement
infini, et dans l’Étendue, le mouvement ou le repos, les autres
{\it finis} qui sont les êtres particuliers, la « nature naturée » : les {\it âmes}
sont les modes de la Pensée, les {\it corps} les modes de l’Étendue. Ainsi,
« tout ce qui est, est en Dieu » : Dieu est cause {\it immanente}, et non
{\it transitive}, de toutes choses, et l’on pourrait résumer en cette formule
%628
le système de Spinoza : {\it Dieu seul existe}. Tout découle {\it nécessairement}
de la nature de Dieu : il n’y a dans le monde ni contingence ni liberté
autre que celle qui s’identifie avec la nécessité immanente à cette
nature, et c’est en nous élevant, par la raison, à l'intelligence de
cette nécessité que nous pouvons
nous unir à Dieu par « l'amour
intellectuel » et atteindre la béatitude.

Une autre forme du Panthéisme a été exposée par le philosophe
allemand Hegel. D’après lui, comme on l’a vu au \S 115 B,
l’Absolu est \textbf{\textit {l’Idée}}, vérité à l’état abstrait, pure pensée, dont le
développement dialectique engendre tout ce qui est. À l’Idée,
s'oppose la Nature, existence extérieure, étrangère à la pure pensée.
Ni l’Idée ni \textbf{\textit {la Nature}} ne sont Dieu. Mais la synthèse s’effectue
dans \textbf{\textit {l'Esprit}}, c’est-à-dire dans la pensée devenue consciente d’elle-même.
Encore n’atteint-elle à la pleine conscience de soi que dans
l'{\it Esprit absolu}, dans l'art, la religion et la philosophie. {\it C’est
cet Esprit absolu qui est Dieu}, de sorte qu’on peut dire que, pour
Hegel, Dieu n’est pas, mais se réalise chaque jour, grâce à ces
disciplines, dans l'humanité. « Dieu doit être conçu, écrit-il lui-même,
comme l'esprit dans sa communauté ». Comme le dit É. Bréhier,
dans une telle conception, « il s’agit moins d’atteindre Dieu que de
%Fig. 94. — Benoît DE Spinoza.1632-1677.
consacrer l’homme : non seulement Dieu n’est pas indépendant de
la communauté spirituelle, mais il n’existe comme tel, comme se
connaissant soi-même que dans cette communauté... Le sommet de
l'Esprit dans la doctrine hégélienne, c’est la culture humaine ; la
religion même est considérée comme fait de culture ; elle est connaissance
de Dieu par soi, et Dieu ne se connaît que dans et par cette
%629
culture. Le résultat le plus patent de sa philosophie, c’est de conférer
le sceau divin à toutes les réalités de la nature et de l’histoire : la cité
terrestre se transforme en une Cité de Dieu ».

\section{L’Athéisme}% 356.
En regard de ces positions philosophiques,
que représente exactement l’Athéisme ? « {\it Athéisme}, écrit Pascal,
{\it marque de force d'esprit, mais jusqu’à un certain degré seulement}. »
On pourrait interpréter cette pensée en disant que l’athée fait preuve
d’une certaine « force d’esprit », non seulement — c’est probablement
le sens que Pascal lui donnait — en prenant le contre-pied de croyances
généralement admises, mais en poussant aussi loin que possible
l'explication de l’univers et la justification de nos valeurs {\it sur le plan
purement naturel et humain}. S'il s’agit d'expliquer un phénomène particulier,
on peut le faire, on {\it doit} le faire par les seules lois de la nature,
et non pas remonter immédiatement à la cause première
{\footnotesize (Le P. Teilhard De Chardin cite l'opinion du géologue H. de Dorlodot, ancien
professeur de théologie, selon laquelle il convient « d'attribuer à l’action naturelle des
causes secondes tout ce que la raison et les données positives des sciences d'observation
permettent de leur accorder et de ne recourir à une intervention spéciale de Dieu,... qu'en
cas d’absolue nécessité ».)}.
De même,
s’il s’agit de trouver un fondement à nos valeurs morales, on doit le
chercher d’abord dans les principes de la nature humaine (\S 194).
— L’Athéisme serait la doctrine qui \textbf{\textit {se refuse à dépasser}} cette
étape. A la fin de sa {\it Logique générale}, Renouvier le définit précisément
comme la renonciation à remonter à un Absolu, à une cause première,
à un « sujet d’une synthèse unique et totale des choses », comme
l’aveu d’une « ignorance invincible de la totalité du monde ». En ce
sens qui, dit-il, « n’exclut point le véritable théisme au sens moral »
(lequel admettrait un Dieu parfait, mais non infini ni même nécessairement
unique), l’Athéisme serait « la vraie méthode, la seule fondée
en raison, la seule positive »
{\footnotesize (Hamelin, commentant ce passage, déclare qu’en effet « les méthodes scientifiques
exigent que les principes eux-mêmes soient examinés et fondés rationnellement. Même
en Dieu, la raison ne saurait s'expliquer par un acte arbitraire de la volonté ».)}.
Malheureusement, ajoute Renouvier,
« l’athée déclaré sacrifie presque toujours au matérialisme ; et le
panthéiste, de son côté, se voit appliquer ce nom d’athée contre
lequel il proteste. En ce sens, l’Athéisme est une erreur profonde,
mortelle à l’humanité ».

Il est en effet un Athéisme plus radical que celui qu’accepte
Renouvier : il coïnciderait à peu près avec une sorte de {\it nihilisme}
moral. Ce n’est plus seulement la négation du fondement ultime
%630
des valeurs, c’est la \textbf{\textit {négation de la valeur elle-même}}
{\footnotesize (C'est en un sens diamétralement opposé qu'on a pu parler de « l’athéisme » de
Lacneau (voir p.636, note 2) : ce dernier consiste en effet à affirmer que Dieu est {\it valeur},
et non pas {\it être}.)}.
Est-il sûr
qu’en ce sens, comme l’a soutenu Éd. Le Roy, « il n’y a point
d’athées : car il n’y a personne sans doute qui se contente absolument
de ce qu'il a et de ce qu’il est, qui s’y arrête, qui s’y enferme, personne
qui n’admette au moins pratiquement comme principe moteur de sa
vie un idéal et un au-delà de l’ordre spirituel dont les sollicitations le
travaillent » ? La question, (voir tome II, chap. XXVI) n’est pas si
simple. Car on peut aspirer au delà de ce qui est, on peut poser des
valeurs en admettant qu’elles sont \textbf{\textit {de pure création humaine.}}
C’est précisément, de nos jours, la position de l’existentialisme athée
et c'était aussi celle de Nietzshue quand il constatait la destruction
de toutes les valeurs par « le nihilisme européen » et proclamait dans
son {\it Zarathoustra} : « Dieu est mort ! maintenant nous voulons que le
surhomme vive », ce qui signifiait que pour lui l’humain est incompatible
avec le divin.

Mais c’est là justement que nous pouvons dire avec Pascal : «... {\it jusqu’à
un certain degré seulement} ». L’athée ne manque-t-il pas de « force
d'esprit » en désespérant de pousser jusqu’à la justification totale
ces valeurs qu’il reconnaît peut-être dans la pratique, mais qu’il
renonce à {\it fonder} ? C’est ce que nous aurons à examiner à propos
du problème des rapports entre l’Être et la Valeur.

\section{Preuves classiques de l’existence de Dieu}% 357.
Les philosophes
et les théologiens ont élaboré un certain nombre de {\it preuves} de
l’existence de Dieu. On distingue généralement des preuves métaphysiques,
des preuves « physiques » et des preuves morales.

{\it A. Preuves métaphysiques.} Les preuves métaphysiques sont celles
qui concluent directement de l’idée de Dieu à son existence.

\vspace{0.24cm}
{\footnotesize 
\textbf{1° \textit {\textsf{Preuve ontologique.}}} La plus célèbre et, on le verra plus loin, la plus
profonde est la preuve ontologique, inventée au {\footnotesize XI}$^\text{e}$ siècle par saint Anselme,
reprise plus tard par Descartes, puis par Malebranche, Leibniz,
Spinoza, Hegel. Saint Anselme la formule ainsi : Dieu est l’Être tel qu’il
n’en est pas de plus grand ; or, s’il n'existait que dans la pensée, on pourrait
en concevoir un plus grand, celui qui existerait aussi dans la réalité ; donc
Dieu existe nécessairement dans la réalité. Descartes donne à cette preuve
une forme quasi mathématique : de même que dans l’essence d’un triangle
est nécessairement comprise sa propriété d’avoir la somme de ses angles
égale à deux droits, ou encore dans l’idée d’une montagne, l’idée d’une vallée,
de même dans l'essence de l'Être parfait (ou infini) est nécessairement
comprise l'existence. Spinoza raisonne ainsi : pouvoir exister, c’est puissance ;
%631
donc plus il appartient de réalité à la nature d'une chose, plus
elle a de force pour exister ; or Dieu, étant l'Être infini, est celui qui enveloppe
le plus de réalité ; il existe donc nécessairement. Hegel enfin, posant
« l'identité absolue de Dieu et de la pensée » (\S 115 et 355), en conclut tout
naturellement qu’« à l’égard de Dieu la pensée et l'existence, l’être et la
notion sont identiques ». On le voit : cette preuve est une preuve {\it a priori},
où l’on prétend déduire {\it analytiquement} l'existence de Dieu de son essence
même. Dieu est l’Étre parfait, c’est-à-dire qui possède la plénitude de
l'être : il serait donc contradictoire qu'il n’existât pas.

Kant a dirigé contre cette preuve toute une critique qu’on peut résumer
ainsi : on ne peut déduire analytiquement (\S 202 {\it A} 1°) d’une notion
que ce qui y est déjà implicitement contenu ; or, dans l’{\it idée} de l’Être
infiniment réel, ne peut être contenue que l’{\it idée} de l’existence, et non
l'existence en soi, en dehors de la pensée. Autrement dit, l'existence n’est
pas une « perfection », un attribut dont on puisse enrichir l'essence, la
compréhension logique
{\footnotesize (Rappelons que la compréhension logique d’un terme est l’ensemble des caractères
qui lui sont attribuables)}
d’un sujet ; elle ne fait que le poser dans l'être. —
On peut donner raison à Kant si l’on prend la preuve ontologique sous la
forme où elle est ordinairement présentée, comme un raisonnement
{\footnotesize (La preuve paraît encore plus artificielle lorsqu'on lui donne, comme on le fait
parfois, la forme syllogistique : « Dieu est l’Être parfait ; or l'existence est une perfection ;
donc Dieu existe. »)},
comme un passage discursif de l'essence à l'existence. Une essence est en
effet une pure idée, elle n’a qu’une existence représentative. Elle ne peut
donc nous fournir, comme le dit Kant, qu’une pure {\it possibilité logique}
(\S 328 B) ; et, en ce sens, il est parfaitement vrai que l’essence d’un Être
infini, étant celle d’un Étre qui ne souffre aucune borne, aucune limitation
ou négation, n’enveloppant aucun non-être, semble se poser d'elle-même dans
l’être sans que rien puisse s'opposer à son existence. Mais c’est justement ce
passage qui, du moins sous cette forme, est illégitime : car, nous l’avons dit,
le {\it possible} n’est pas l'{\it être}, l’{\it intelligible} n’est pas le {\it réel}. Pour que la preuve
fût concluante ou, tout au moins, pût fournir un point de départ solide à
l'argumentation, il faudrait donc que l'essence dont on part, fût déjà elle-même
{\it participation à l'être}, que la pensée {\it n’eût pas besoin de sortir d’elle-même
pour reconnaître l'être}. On verra plus loin que, dépouillé de sa forme
discursive, l’argument ontologique a précisément cette signification profonde (\S 358).

Telle qu’elle a été présentée, la preuve ontologique soulève cependant
encore une autre difficulté. Admettons qu'elle prouve bien l'existence
absolue d’un Être infini ou parfait. Encore résterait-il à se demander si cet
Être est bien celui que les hommes, dans leur ensemble, ont appelé Dieu,
c’est-à-dire cette perfection {\it morale}, et non plus seulement métaphysique,
cette {\it Valeur absolue}, douée de personnalité et de vie, qui est pour eux
objet d'amour et d’adoration. Ne se pourrait-il pas, au contraire, que l’Être
souverainement réel fût l’Être infini, mais impersonnel, étranger à tous les
sentiments humains, qu'est le Dieu des panthéistes? C’est en ce sens, en
effet, que Spinoza comme Hegel utilisent l'argument ontologique. Il faudra
donc, pour que celui-ci conclue au Dieu de la conception traditionnelle,
qu'il parte, non de l’idée de l’être abstrait et indéterminé, mais d’un être
réel {\it déjà porteur de valeur}, qui englobe en son essence quelque chose de cette
Valeur que, sous sa forme éminente et transcendante, nous appelons Dieu.
%632

\textbf{2° \textit {\textsf{Preuve par l'idée d'infini ou de parfait.}}}
Une seconde preuve, propre à
Descartes, se fonde sur l'existence en notre esprit de l’idée de l'Être
infini ou parfait. Descartes considère le contenu représentatif ou, comme
il dit, la « réalité objective
{\footnotesize (Le mot {\it }objectif n'avait pas alors le sens qu'il a aujourd’hui, mais presque le sens
opposé. Exister « objectivement », c'était exister « pur représentation, dans l'entendement ».
C'était le sens scolastique. Cf. {\it Nouveau Vocabulaire philosophique}, p. 128.)}
» de cette idée : elle représente l’infini ; cette
« réalité » est donc elle-même infinie. Il recherche alors d'où elle peut lui
venir. Comme il doit y avoir autant de réalité dans la cause que dans l'effet,
cette origine ne peut être en lui-même, qui est imparfait puisqu'il doute.
Il reste donc que cette idée soit une idée « innée », c’est-à-dire qui a été mise
en lui par Dieu « comme la marque de l’ouvrier empreinte sur son ouvrage».

Cette preuve soulève encore, sous cette forme, de multiples objections.
{\it a.} Comment l'infini se laisserait-il enfermer dans une idée ? Descartes, avec
tous les théologiens d’ailleurs, le reconnaît : si nous pouvons avoir quelque
{\it connaissance} de Dieu, nous ne pouvons cependant pas le {\it comprendre}.
En admettant donc que nous ayons l’idée de l'infini, nous n’en possédons
cependant pas une idée {\it adéquate}. Dès lors, le point de départ de l’argument
fait défaut. {\it b.} On peut même contester légitimement que nous
ayons à proprement parler une {\it idée} de l'infini. Ce que nous prenons ici
pour une idée, c’est la tendance de la pensée à se dépasser sans cesse elle-même
(\S 360). {\it c.} Enfin, comme pour la preuve précédente, le caractère discursif
que Descartes donne à celle-ci, l’affaiblit. Tout son raisonnement
s'appuie sur le principe scolastique : « {\it causa æquat effectum}, la cause doit
être égale à l’effet », dont le moins qu’on puisse dire est qu’il est bien abstrait
et entaché d'incertitude. Au reste, dans certains textes, Descartes lui-même
semble interpréter cette preuve tout autrement, dans le sens d’une {\it présence}
de Dieu en nous par son idée : celle-ci fait partie, dit-il, de la « ressemblance »
entre Dieu et l’homme dont parle l'Écriture et « je conçois cette
ressemblance par la même faculté par laquelle je me conçois moi-même »,
donc (\S 328) intuitivement. C’est en effet en ce sens que son disciple Malebranche
interprétera la preuve par l’idée d’infini : il en fait une preuve
de « simple vue » dans laquelle « l'esprit aperçoit l'infini quoiqu'il ne le
comprenne pas ». On verra plus loin (\S 360) ce qu’on peut retenir de cette
interprétation.}
\vspace{0.31cm}

{\it B. Preuves physiques.} — On a appelé preuves « physiques » celles
qui partent d’une existence empirique pour s'élever de là, soit à la
cause première, soit à la fin dernière de ces existants empiriques.

\vspace{0.24cm}
{\footnotesize
\textbf{1° \textit {\textsf{Preuve cosmologique ou par la contingence du monde.}}}
 La première voie
est celle de la causalité efficiente : elle consiste à remonter de l'existence
du monde, considéré comme contingent, à celle de Dieu, défini comme {\it cause
première et nécessaire} de tout ce qui existe. On peut fcrmuler ainsi cette
preuve : le monde, considéré aussi bien dans chacun des êtres qui le composent
que dans leur assemblage, est {\it contingent}, c'est-à-dire qu'il existe,
mais pourrait aussi bien ne pas exister puisque, n'étant pas parfait, il n’a
pas en lui-même sa raison d’être ; il faut donc qu’il existe par la vertu d’un
autre être, distinct de lui et qui soit, comme disaient les Scolastiques,
« cause de soi », c’est-à-dire qui ait en lui-même sa raison d’être, qui soit
l'Être {\it nécessaire}.
%633
Kant a dit que cette preuve recélait toute « une nichée de sophismes ».
On peut en effet élever contre elle de nombreuses objections. {\it a.} Il n’est pas
évident que le monde soit contingent et c’est même, au fond, ce qui est
en question. « Il est permis de craindre, écrit Éd. Le Roy, qu’au fond
d’un pareil jugement il n’y ait en somme que notre ignorance du déterminisme
réel. À chaque instant, la science ne nous révèle-t-elle pas nécessaire
ce que jusque-là nous pensions contingent ? » Et, si chacun des composants
de l'univers, pris isolément, était contingent, qui nous dit que ce
ne serait pas l'ensemble, « la loi des manifestations phénoménales » qui
serait nécessaire ? {\it b.} Au fond, si nous jugeons le monde contingent, c’est
qu’il nous apparaît comme {\it imparfait}. On sous-entend donc, remarque encore
Éd. Le Roy, « une liaison {\it a priori} entre l'existence nécessaire et la perfection
de l'essence », c’est-à-dire qu’ainsi que l'avait déjà observé Kant, la
preuve cosmologique suppose la preuve ontologique. {\it c.} L'idée de {\it cause
première} est presque contradictoire si on la conçoit comme celle d’une cause
qui serait à l’origine des causes secondes. Kant voyait là une des « antinomies »
de la raison pure : le principe de causalité, stipulant que {\it tout} a une
cause, semble nous prescrire de chercher une condition à tout ce qui est
et donc de remonter à l'infini de cause en cause ; la preuve cosmologique
arrête cette régression indéfinie (« il faut s'arrêter », avait dit Aristote)
et pose au contraire une nécessité inconditionnée.

\textbf{2° \textit {\textsf{Preuve téléologique ou par les causes finales.}}}
La seconde voie est celle de
la {\it finalité}. Elle prend pour point de départ l’ordre harmonieux, la finalité
qui, dit-on, règne dans la nature. Celle-ci nous apparaît comme un système
de moyens et de fins. Or tout système de ce genre suppose une cause intelligente.
La nature est donc l’œuvre d’une telle cause.

Cette preuve, quoique l’une des plus anciennes et souvent utilisée parce
qu’elle se prête à des développements littéraires, est cependant l’une des
plus faibles. Descartes, condamnant toute recherche de la finalité dans
la nature, la rejette : l'homme, dit-il, ne doit pas avoir la prétention de
pénétrer les desseins de Dieu. Pascal déclare qu'elle ne peut convaincre
que ceux qui croient déjà et qui aperçoivent incontinent « que tout ce qui
est, n’est autre que l'ouvrage du Dieu qu'ils adorent ». Éd. Le Roy y
voit « une preuve d'orateur et de poète lyrique plutôt que de logicien ».
Elle se heurte en effet aux difficultés suivantes. {\it a.} L'idée de {\it finalité} est une
notion obscure, dont la Science fait de moins en moins usage (\S276)et qui
soulève plus de problèmes qu’elle n’en résout. Dans la mesure où elle fait
intervenir des volontés particulières de Dieu, Malebranche la déclare
indigne de l’Être infini qui agit toujours par les voies les plus simples et par
des lois générales. Dans la mesure où elle est conçue de façon anthropocentrique,
elle fait naïvement de l’homme le centre de l'univers ; or ni
du point de vue scientifique ni du point de vue métaphysique ou théologique
(Dieu n’a fait le monde que pour sa propre gloire, dit Malebranche),
il n'apparaît que le monde soit fait pour l'homme. {\it b.} L'argument semble
même {\it en contradiction} tant avec la preuve cosmologique qu'avec l'argument
non moins célèbre des aspirations de l’âme humaine, qui s'appuient, le
premier sur l’imperfection de l'univers, le second sur le malaise de la conscience
insatisfaite des choses d’ici-bas. {\it c.} De toute façon, il paraît impossible
de soutenir que la finalité de l'univers, si elle existe, soit {\it parfaite}. « S'il
est facile de s’enthousiasmer au spectacle des harmonies cosmiques, il ne
l'est pas moins de se scandaliser à celui des misères qui s’y mêlent » (Le Roy),
et les métaphysiciens ont beaucoup peiné pour résoudre le \textbf{\textit{\textsf{problème du mal}}},
c'est-à-dire pour expliquer la présence du mal dans la nature. On ne saurait
%634
donc conclure avec certitude de cette finalité ee semble si imparfaite, à
une cause infiniment intelligente et bonne. {\it d.} L’argument téléologique
nous permet-il même de conclure à une {\it cause} de l'univers ? Par lui-même, il
ne mène qu’à l’idée d’un Dieu {\it organisateur}, d’un Dieu {\it architecte} qui, comme
le Démiurge de Platon, aurait mis un certain {\it ordre} dans l’univers, mais non
nécessairement à celle d’un Dieu créateur.}
\vspace{0.31cm}

{\it C. Preuves morales.} — Les preuves morales sont celles qui s’appuient,
soit sur les {\it notions} morales fondamentales, soit sur les {\it aspirations}
de l’âme humaine.

\vspace{0.24cm}
{\footnotesize
\textbf{1° \textit {\textsf{Les notions morales}}}
ne se justifient pleinement, dit-on, que si elles se
fondent sur la notion d’un Dieu juste et bon. Le {\it Devoir} notamment implique
une transcendance qui ne s'explique bien que par un commandement
venant de Dieu. La {\it Sanction} exige un Dieu « rémunérateur et vengeurs qui
répare dans l'au-delà les imperfections et les injustices des sanctions
humaines. On verra au t. II que Kant présente l’existence de Dieu comme
un postulat de la moralité : c’est précisément pour que se réalise l’union
du {\it bonheur} et de la {\it vertu}, du bien sensible et du bien moral, qui n’est pas
réalisée ici-bas. On sait cependant que Kant présente ce « postulat », non
comme une vérité spéculative théoriquement établie, mais comme l’objet
d’une « croyance » pratique. Et, en effet, tous ces arguments sont plutôt
des {\it raisons morales de croire} à l’existence de Dieu que des preuves proprement
dites. En fait, d’ailleurs, l’homme peut se sentir obligé moralement
sans reconnaître que cette obligation vient de Dieu.

\textbf{2° \textit {\textsf{Les aspirations de l'âme humaine,}}}
 ajoute-t-on, {\it ne sauraient étre déçues}. Or
elles ne seraient que des illusions si le bien infini, la justice parfaite, la
vérité intégrale que recherchent notre cœur, notre conscience et notre intelligence,
n'étaient que des idoles, s'ils ne possédaient pas une existence
substantielle en Dieu. — Mais, ainsi présenté, l'argument préjuge justement
de {\it ce qui est en question}. Est-il évident, après tout, que les aspirations
humaines doivent être satisfaites ? Certaines doctrines contemporaines nous
montrent l’homme « jeté » dans un monde absurde où ses aspirations ne
trouvent pas d’écho. Il ne suffit donc pas d’alléguer ici une exigence morale,
aziologique. Il faut doubler celle-ci d'un fondement ontologique pour qu’on
puisse être assuré qu’elle est autre chose qu’un risque hérique à courir,
l'enjeu d’un « pari », ou même un leurre.}
\vspace{0.31cm}

Tous ces arguments, ainsi isolés, paraissent donc assez artificiels.
Mais on va voir que, replacés dans une dialectique d’ensemble, ils
prennent, comme l’a fait observer Éd. Le Roys, une signification
nouvelle.

\section{Primauté de l’esprit}% 358.
Nous retiendrons de la « preuve
ontologique » cette idée qu’il y a au moins un \textbf{\textit {être dont nous saisissons
directement l'existence en nous.}} Cet être, c’est celui de la
pensée, de la \textbf{\textit {réalité spirituelle,}} présente en nous en tout acte de
pensée et qui se révèle à nous dans l'intuition du {\it cogito} (\S 112 et
328). Cette intuition nous permet d’affirmer la primauté \textbf{\textit {ontologique}}
de l'esprit. C’est ce qu’exprime Descartes lorsqu'il dit que
%635
« l’âme est plus aisée à connaître que le corps », ce qui doit s’entendre,
non pas du tout d’une connaissance empirique
{\footnotesize (C'est-à-dire en ce sens que la Psychologie serait plus facile que la Physiologie ou
logiquement antérieure à elle, — ce qui serait un pur contresens ! Voir \S 112.)},
ni même en ce sens
que l’essence de l’esprit nous serait plus \textbf{\textit {intelligible}} que celle de la
matière, mais bien en ce sens que \textbf{\textit {l'existence}} de la réalité spirituelle
est logiquement première, qu’elle est plus facile à établir, puisqu’elle
est saisie directement, que celle de la réalité corporelle. Je puis supposer
que celle-ci n’existe pas, je puis la révoquer en doute : l’{\it existence
de l'esprit} n’en reste pas moins certaine. — Mais il y a, d’autre part,
une primauté de l'esprit dans l’ordre \textbf{\textit {axiologique,}} dans l’ordre des
valeurs. Point n’est besoin d’y insister après tout ce qui a été dit en
Morale. La dignité de la personne qui nous est apparue mainte fois
comme la {\it valeur} fondamentale, n’est qu’un autre nom de la valeur
éminente de la pensée, de la dignité de l’esprit, dont la personne est
l’incarnation humaine.

\section{Être et Valeur}% 359.
Ainsi, primauté de l'esprit {\it dans
l'ordre de l’Être}, primauté de l'esprit {\it dans l’ordre de la Valeur}, voilà
cé que nous sommes dès maintenant en droit d’affirmer. Dans l’intuition
du {\it cogito}, nous saisissons un être qui est en même temps valeur
{\footnotesize (C'est l'exagération de cette idée que nous trouvons dans les théories du philosophe
J. Lagneau (1893) sur Dieu : « A tous les degrés de la pensée, écrivait-il, la valeur
est vraiment la réalité que la pensée affirme. » D'où Lagneau conclut : « La réalité
absolue que nous cherchons, la seule qui convient à Dieu, n’est ni l'existence ni l'être,
car dans l'existence et dans l’être, ce qui est réellement ce n’est pas la contingence ni la
nécessité, c’est la valeur. » En ce sens, « Dieu ne peut exister ni comme les objets sensibles
(contingence) ni comme les objets d'intelligence (nécessité) ». Un objet sensible,
en effet, n'existe que par ses relations avec les autres objets, et la nécessité absolue est
contradictoire puisque, ne dépendant de rien, elle subsisterait indépendamment de la
pensée : l'affirmation de la nécessité suppose la liberté, c'est-à-dire l'esprit. C’est pourquoi
Lagneau a sans cesse répété que Dieu n'{\it existe} pas, mais qu'il est pure {\it valeur},
ce qu'il faut entendre en ce sens que {\it Dieu n'est pas un être parmi les autres êtres}. —
C’est ce qu'on a nommé improprement «l'athéisme » de Lagneau. Mais le 1er Concile
du Vatican lui aussi a affirmé « que Dieu est infiniment au-dessus de toute existence ;
qu’à proprement parler, il n'est pas être, mais principe ineffable de l'être ; que sa façon
d’être, à lui, c'est de trôner au-dessus de l’être : {\it super omnia, quae praeler ipsum sunt
et concipi possunt, ineffabiliter excelsus} » (Le Roy).)},
et valeur fondamentale. Nous pouvons donc répondre à la
question des rapports entre l’Être et la Valeur : non (et c’est ce que
nous retiendrons des «preuves morales»), la \textbf{\textit {Valeur et l'Étre}} ne
\textbf{\textit {sont pas foncièrement distincts}}, nous devons au contraire affirmer :
leur unité profonde. Or cette affirmation, si nous l’élevons sur le plan
de l’absolu, n’est autre que l’affirmation même de Dieu : « Dieu est
Esprit », et, comme tel, il est à la fois l’Être absolu et la Valeur
absolue. « L’affirmation de Dieu, écrit Éd. Le Roy, c’est l’affirmation
%636
de la réalité morale comme réalité autonome, indépendante,
irréductible, et même comme réalité première. » — Ajoutons que
ce n’est pas seulement celle de la réalité {\it morale} au sens
strict du terme. C’est aussi celle de la valeur {\it intellectuelle}, de la
Vérité. « La Raison qui éclaire l’homme est la sagesse de Dieu
même », écrivait Malebranche développant ainsi une idée qu'avait déjà exprimée
Descartes lorsqu'il disait que la raison est une « lumière naturelle »
mise en nous par Dieu et qu’il parlait de « certaines
semences de vérité qui sont naturellement en nos âmes »
{\footnotesize (L'idée est d’ailleurs d'origine augustinienne : saint Augustin parle d’une {\it ratio
insita et quodam modo inseminata} qu’on trouve, encore endormie, jusque chez les tout
jeunes enfants et qui n’est autre que cette lumière du Verbe divin qui, selon saint Jean,
« éclaire tout homme venant en ce monde ».)}.
Si le monde, y compris le monde matériel, émane d’une Raison
créatrice (\S 361), nous pouvons comprendre comment notre
propre raison, si elle participe à cette Raison absolue, peut
retrouver quelque chose d’elle-même dans ce monde, comment
la connaissance est possible, en un mot comment il y a une
{\it intelligibilité de l'univers}, à tel point que les Mathématiques
elless mêmes semblent nous révéler parfois « les secrets de la
nature »(\S 256).Et ainsi se trouverait enfin éclairé le problème
que nous avions vainement cherché à résoudre (\S 119 {\it fin},
et 351) des rapports entre la pensée et l’être, entre l'esprit
et les choses.
%Fig. 95. — Nicozas MALEBRANCHE.
%637
\section{La participation}% 360.
Mais, pour nous élever sur ce plan
de l’absolu, toute une dialectique, non pas vain jeu de concepts, mais
désormais en prise sur l’être, nous sera nécessaire. Je constate d’abord
que cette pensée que je saisis en moi par le {\it cogito}, n’est pas seulement
{\it ma} pensée. Elle est {\it participation} à une Pensée qui me dépasse. En
effet : 1° la pensée est \textbf{\textit {irréductible}} : comment unc réalité corporelle
pourrait-elle engendrer même un simple état de conscience? Là est
la difficulté contre laquelle viendra toujours, avons-nous dit (\S 339).
achopper le matérialisme. Ma pensée ne saurait donc s’expliquer ni
par {\it mon être empirique} ni même par {\it l'être empirique du monde matériel}
(et ainsi se retrouve rejointe, en un certain sens, la « preuve cosmologique ») ;
— 2° en tant qu’elle est capable de poser des valeurs,
ma pensée participe d’une réalité qui lui est {\it supérieure}, puisqu’elle
se juge elle-même et qu’elle juge l’univers où elle est située. La Raison
normative, créatrice de valeurs, est présence en nous d’une
\textbf{\textit {transcendance.}}

Mais il faut aller plus loin. Cette transcendance est {\it absolue} ; car cette
réalité spirituelle à laquelle nous participons, ne saurait être une pensée
finie. Malebranche, disait que Dieu nous a refusé la connaissance
intelligible de notre âme parce que, si nous la possédions, l'esprit est
chose tellement éminente que nous en concevrions trop d’orgueil.
Descartes avait écrit avant lui que « l'esprit humain possède je ne
sais quoi de divin ». En effet, en tant que {\it Raison}, notre pensée
enferme, sinon à proprement parler l'idée de l'infini, du moins une
\textbf{\textit {puissance d'’infinitude.}} La pensée tend sans cesse à « aller au delà » :
elle franchit les bornes de l’expérience ; elle « passe à la limite »,
comme disent les mathématiciens, c’est-à-dire qu’elle se libère des
déterminations du sensible pour s'élever, sur le plan de lintelligible,
à l’absolu et à l’universel. La preuve cartésienne « par l’idée
d’infini» peut donc s’interpréter en ce sens qu’il y a dans l'esprit
humain, selon l’expression de L. Brunschvicg, «l'intuition d’un
objet immédiat qui déborde sa capacité propre ». On sait que, dans
cette faculté que possède l’esprit humain de se hausser, par la pensée
abstraite, métaphysique ou même mathématique (\S 249), jusqu’à
lintelligible pur, Malebranche voyait « l’application de l’esprit à
Dieu, la plus pure et la plus parfaite dont on soit naturellement
{\footnotesize C'est-à-dire en deçà du plan proprement religieux, surnaturel.}
capable ». C’est en ce sens qu’on pourrait dire avec Descartes dans
les {\it Regulæ} : « {\it Sum, ergo Deus est}. Je suis, donc Dieu est. » Il y a, dans
la pensée humaine, une présence de la Pensée absolue.
%638
\section{Dieu substance et cause}% 361.
Cette Pensée absolue, cet
Esprit infini est \textbf{\textit {substance}}, non pas sans doute {\it seule} substance comme
l'entend Spinoza, mais seule substance {\it au sens plein du mot}. Descartes
remarque en effet dans ses {\it Principes} que le mot {\it substance} peut
avoir deux sens : ou bien l’on entend par là (par opposition à {\it attribut}
ou {\it qualité}) ce qui est \textbf{\textit {en soi}}, donc ce qui n’a besoin que de soi-même
{\it pour être conçu}, et en ce sens les êtres créés peuvent être
des substances ; ou bien on entend, plus précisément, « une chose
qui existe en telle façon qu’elle n’a besoin que de soi-même pour
exister », autrement dit ce qui est \textbf{\textit {par soi}}, et Descartes ajoute qu’«il
n’y a que Dieu qui soit tel ». Ceci résulte en effet de ce qui a été dit
ci-dessus sur la primauté ontologique de l'Esprit. — Cet Esprit infini,
puisque nous ne pouvons pas concevoir qu’il tienne son existence
d’un autre être, est aussi {\it cause première} ou, plus exactement (car il
ne peut être placé dans le temps, à l’origine des causes secondes et
sur le même plan qu’elles, ce qui d’ailleurs soulèverait, comme il a
été dit \S 357 B 19, des difficultés insolubles), \textbf{\textit {seule vraie cause,}} ainsi
que l’a soutenu Malebranche. Cette thèse de Malebranche a été
souvent mal comprise. Mais elle nous paraît incontestable en ce sens
que le mot {\it cause}, comme Descartes le dit du mot {\it substance}, « n’est
pas univoque au regard de Dieu et des créatures ». Si l’on entend ce
terme au sens plein, c’est-à-dire(\S 265)comme \textbf{\textit {principe d'existence}} et
\textbf{\textit {source d’être,}} en un mot comme activité \textbf{\textit {créatrice,}} il est évident
qu’alors il convient à Dieu seul, puisque la Science nous montre que
«rien ne se crée » et qu’il n’existe entre les êtres du monde que des
échanges et des transformations d'énergie.

\section{Dieu personne}% 362.
Est-ce à dire qu’il faille aller jusqu’au
Panthéisme (\S 355) et regarder tous les êtres du monde comme des
émanations, des modes ou des déterminations particulières de l’unique
Substance divine ? Il ne le semble pas. En effet : 1° le Panthéisme
repousse l’idée de \textbf{\textit {création}} ; or il paraît moins difficile d'admettre que
l’Étre éternel et infini a donné l’existence a des êtres finis et situés
dans le temps que de comprendre la mystérieuse raison qui ferait que
la Substance unique, « au lieu d’enfermer en soi toute sa puissance
productrice, la manifesterait et la développerait par des modes »
(V. Delbos), autrement dit que la « nature naturante » en dehors
de laquelle, selon le Panthéisme, rien n’existe, se déploierait et se
déterminerait en une « nature naturée », faite d’êtres finis et particuliers ;
— 2° englobant l’homme et l’univers dans l’{\it unique réalité},
le Panthéisme sacrifie la \textbf{\textit {liberté}} humaine et {\it justifie comme rationnel
tout ce qui est}, puisque tout ce qui est, résulte nécessairement de la
%639
nature de l’Être universel ; on verra (t. II, chapitre XIX) que Spinoza
et Hegel en viennent à identifier le {\it Droit} avec la {\it puissance} de
l'être, c’est-à-dire avec la force ; le Panthéisme {\it compromet ainsi les
valeurs spirituelles} que personnifie l’idée de Dieu ; — 3° enfin et
surtout, le Panthéisme refuse à Dieu la qualité de {\it personne} parce qu’il
y voit une limitation et une négation ; mais il n’en serait ainsi que si
nous entendions la personnalité en un sens étroitement anthropomorphique :
il va de soi, au contraire, qu’en disant que Dieu est un
être {\it personnel}, on ne veut pas dire qu'il l’est comme l’être humain,
mais qu’il contient, et au delà, de façon éminente, {\it tout ce que représente
pour nous de valeur morale et de richesse spirituelle} la notion de
personne ; et, si l’on identifie Dieu et la Valeur, il y aurait une singulière
inconséquence à ne pas l’admettre.

\section{Dieu fin suprême}% 363.
C’est du même point de vue que la
notion de \textbf{\textit {finalité}} qui, si on l’applique au détail des phénomènes,
risque d’aboutir à un anthropomorphisme et à un anthropocentrisme
puérils (6276 B 10), se justifie au contraire {\it sur le plan métaphysique.}
L’essence de l'Esprit implique en effet {\it tendance vers une fin}, inconsciente
d’abord, puis consciente et rationnellement voulue. Si donc
l'Esprit est la réalité première et absolue, {\it la finalité fait partie de
l'essence même de l’Être}, et le monde doit être conçu comme mû par
l'Esprit et tendant vers lui. Ainsi, se trouverait légitimée la notion
traditionnelle de Dieu « fin suprême ».

\section{Sujets de travaux}% SUJETS DE TRAVAUX

Exercices. — 1. {\it Étudier, d'après ce texte, les différentes fonctions que
Leibniz attribue à l’idée de Dieu dans la} Monadologie : « 38. La dernière
raison des choses doit être dans une substance nécessaire, dans laquelle le
détail des changements ne soit qu’éminemment, comme dans sa source ; et
c’est ce que nous appelons Dieu. 43. En Dieu est non seulement la source
des existences, mais encore celle des essences en tant que réelles ou de ce
qu’il y a de réel dans la possibilité. 84. C’est ce qui fait que les esprits
sont capables d'entrer dans une manière de société avec Dieu et qu’il est à
leur égard, non seulement ce qu’un inventeur est à sa machine, mais ce
qu’un père est à ses enfants. 90. Enfin, sous ce gouvernement parfait il
n’y aurait point de bonne action sans récompense, point de mauvaise sans
châtiment. » — 2. {\it Que signifient exactement} pour vous {\it les expressions} : croire
ou ne pas croire en Dieu ? — 3. {\it Que pensez-vous de cette réflexion de Sully-Prudhomme} :
« J'en arrive à me définir Dieu simplement : ce qui me
manque pour comprendre ce que je ne comprends pas » ? — 4. {\it Commenter
cette assertion de Th. Ruyssen} : « Une existence ne peut être objet de preuve ;
elle ne peut que s’éprouver. » — 5. {\it Comment comprenez-vous celle affirmation
de Descartes} : « J'ai en quelque façon premièrement en moi la notion de
l'infini que du fini » ({\it 3$^\text{e}$} Méd.) {\it et celle-ci de Malebranche} : « La notion de
%640
l'Être infiniment parfait est profondément gravée dans notre esprit : nous
ne sommes jamais sans penser à l’Êtres ({\it Ent. VIII}) ? — 6. {\it Quelle réflexion
vous inspire le rapprochement de ces deux textes} : {\it a}) « Lorsque je me déclare
athée, j'entends seulement dire que je ne suis nullement satisfait par
l'hypothèse dans laquelle les lois de la nature tireraient leur origine d’un
Dieu dont on pourrait parler comme on parle d’un homme » (Le Dantec) ;
{\it b}) « L'homme humanise toutes les causes, même la Divinité. Il lui attribue
des desseins humains, une conduite humaine, et quelquefois jusqu’à ses
propres passions » (Malebranche) ? — 7. {\it Apprécier ce jugement d'E. Saisset
sur le panthéisme} : « Il trouve en face de lui deux réalités que nul esprit
raisonnable ne saurait nier et il entreprend de les réduire à l'unité absolue
d’une seule existence. Le voilà condamné, s’il veut un Dieu réel et vivant
à y absorber les créatures et à tomber dans le mysticisme : ou, s’il lui faut
un univers réel et effectif, à faire de Dieu une pure abstraction. »

Exposés oraux. — 1. {\it Les preuves de l'existence de Dieu et le rôle de l'idée
de Dieu dans le} Discours de la Méthode {\it et dans les Méditations de Descartes}.
— 2. {\it Le problème de Dieu d'après} Lagneau, p. 228 et suiv.

Discussions. — 1. {\it L'idée de « cause première ».} — 2. {\it Valeur morale du
panthéisme.}

Lectures. — {\it a.} Descartes, {\it Méditations métaphysiques} (1641), 2e, 3e et
5e médit. — {\it b.} Bossuet, {\it De la Connaissance de Dieu et de soi-même.} —
{\it c.} Malebranche, {\it Entretiens sur la Métapnysique} (1688), éd. Vrin, 2 vol.,
spéc. entr. II. — {\it d.} Spinoza, {\it Éthique,} liv. I, éd. Classiques Larousse. —
{\it e.} Kant, {\it Critique de la Raison pure}, dialectique transcendantale, liv. II,
chap. III (trad. Tremesaygues, p. 475-559). — {\it f.} Victor Delbos, {\it Le Spinozisme},
Soc. fr. d’impr., 1916. — {\it g.} Éd. Le Roy, {\it Le problème de Dieu}, Art.
du Livre, 1930 ; et {\it h. Introd. à l'étude du probl. religieux}, Aubier, 1944. —
{\it i.} L. Brunschvicg, {\it La Raison et la Religion}, Alcan, 1939, spéc. chap. V. —
{\it j.} F. van Steenberghen, {\it Le probl. philosophique de l'existence de Dieu}, dans
la {\it Rev. philos. de Louvain}, tome XLV (1947), nos 5 à 8, et {\it k. Sciences positives
et existence de Dieu}, même revue, tome LVII (1959), n° 55. — {\it l.} Jules
Lagneau, {\it Célèbres Leçons}, P. U. F., 1950, p. 219-310. — {\it m.} M.-F. Sciacca,
{\it Le Problème de Dieu et de la religion dans la philosophie contemporaine},
Aubier, 1950. — {\it n.} L. Lavelle, {\it Morale et Religion}, Aubier, 2960 (spéc.
Are partie sur {\it Théisme et Panthéisme}, etc:). — {\it o.} J. Delanglade, {\it Le Problème
de Dieu}, Aubier, 1960. — {\it p.} Claude Bruaire, {\it L'affirmation de Dieu},
Seuil, 1964. — {\it q.} C. Tresmontant, {\it Comment se pose aujourd’hui le problème
de l'existence de Dieu}, Seuil, 1965. — {\it r. Études philosophiques}, n° spécial sur
{\it l’Athéisme}, juil.-sept. 1966.
