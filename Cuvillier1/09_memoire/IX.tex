\chapter{La Mémoire}
%chapitre IX
%LA MÉMOIRE
%120. Qu'est-ce que la Mémoire? — 121. Image et souvenir. — 122. Les
%images mentales. — 123. Les fonctions de la mémoire. — 124. La fixation
%des souvenirs et la mémoration. — 125. Le rappel des souvenirs et la remémoration.
%— 126. La reconnaissance des souvenirs. — 127. La localisation
%des souvenirs. — 128. Le problème de la mémoire. — 129. Mémoire et habitude.
%— 130. Mémoire et souvenir pur. — 131. Mémoire et vie sociale, —
%132. La mémoire, fonction intellectuelle. — 133. L'oubli. — 134. Les troubles
%de la mémoire.

\section{Qu'est-ce que la Mémoire ?}% 120.
Il nous faut maintenant
étudier la connaissance en fonction du {\it temps}, c’est-à-dire la connaissance
du {\it passé} et celle de l’{\it avenir}. Le premier de ces deux problèmes
est le problèmes de la mémoire. Le mot {\it mémoire} est un des termes
du langage psychologique qui ont été pris dans les sens les plus divers
et parfois les plus impropres. N’a-t-on pas été jusqu’à parler d’une
« mémoire » de la matière, voulant désigner par là le fait que, même
dans le domaine physique, il y a parfois {\it persistance du passé} (comme
par exemple dans l’{\it hystérésis} magnétique) ? et certains Gestaltistes
n’ont-ils pas invoqué, pour expliquer la mémoire, une sorte de disposition
de la matière à persévérer dans certaines structures? Avec un peu
moins d’impropriété, on a voulu voir dans la mémoire une « fonction
générale de la matière organisée » (Hering, 1874) : un organisme
vivant conserve la « trace » des influences subies et des actions accomplies ;
Ribot, on le verra plus loin, a affirmé que « la mémoire est, par
essence, un fait biologique » ; plus récemment, R. Ruyer a admis (en
un tout autre sens) une « mémoire organique » modelant les organismes
vivants comme une « forme ». De tels emplois du terme nous paraissent
%175
dangereux : la prétendue « mémoire » de la matière n’est qu’une métaphore,
et la mémoire organique se manifeste, non par le souvenir,
mais par l’habitude. A notre sens, l'emploi du mot {\it mémoire} doit être
réservé au domaine {\it psychologique.} Ici encore d’ailleurs bien des
distinctions sont nécessaires.

1° {\it Mémoire immédiate}. — La vie psychique présente une \textbf{\textit {continuité}}
(\S 37) telle que, {\it dans le présent de la conscience, il subsiste toujours
un écho de son passé}. En ce sens, comme le dit Bergson, « toute
conscience est mémoire, conservation et accumulation du passé
dans le présent ». Mais il y a là une extension du sens usuel du terme.
La mémoire vraie n’est pas la conscience ; elle est une {\it fonction particulière}
de la conscience. Non pas simple {\it prolongement} du passé dans
le présent, mais au moins {\it actualisation, rappel} du passé.

2° {\it Mémoire élémentaire}. — Le passé psychique peut revivre, s’actualiser
à nouveau sous trois formes : a. sous la forme {\it motrice} (passé agi) :
je répète des mouvements, des gestes que j'ai appris autrefois ; c’est
l'{\it habitude} (tome II, ch. V) ; — b. sous la forme {\it figurée} ou {\it imagée} (passé
{\it représenté}) : je revois en pensée un paysage, un tableau, un visage
connus ; c’est ce que nous appellerons les {\it images}
(\S 122)
; — c. sous
la forme {\it affective} (passé {\it vécu}) : je revis ou ressens de nouveau une émotion,
un sentiment éprouvés jadis ; c’est la mémoire affective (t. II,
ch. IV). Mais cette {\it résurrection du passé} n’est pas encore la vraie
mémoire : le passé y revit comme quelque chose d’actuel, et non pas
nécessairement {\it avec la marque du passé.}

3° Mémoire {\it vraie.} — Pour qu’il y ait véritablement mémoire, il
faut que le passé soit non seulement {\it rappelé} ou {\it vécu} à nouveau, il
faut qu’il soit {\it reconnu} : c’est, on va le voir, ce qui caractérise le {\it souvenir}
proprement dit par opposition à la simple {\it image}. Nous définirons
donc la mémoire la \textbf{\textit {prise de conscience du passé}} comme tel.

\section{Image et souvenir}% 121.
On peut dire, en gros, qu’il y a la
même différence entre {\it image} et {\it souvenir} qu’entre sensation et perception.
L'{\it image} a été définie autrefois une sensation reviviscente, et
la même controverse s’est élevée à son sujet qu’à propos de la sensation
pure (\S 80). Il nous est possible cependant, ainsi qu’il a été dit
ci-dessus, de {\it revoir par la pensée} un paysage déjà vu, de {\it réentendre
intérieurement} un air de musique déjà entendu, etc. Mais nous connaissons
déjà au moins un état psychique dont la matière est constituée
par de telles {\it images} et où le sujet, n’ayant nullement conscience
qu’elles viennent du passé, les prend pour des réalités actuelles : cet
état est celui du {\it rêve}. Or, bien loin d’identifier le rêve avec la mémoire
comme l’a fait Bergson, nous pensons qu’en rêve {\it il n’y a pas de mémoire}
%176
(t. II, ch. VII). \textbf{\textit {Car la mémoire ne consiste pas à être dupe du passé
et à le prendre pour du présent}} — et c’est précisément ce qui nous
arrive dans le rêve — \textbf{\textit {mais à le connaître comme passé}}. Un fait
analogue se produit dans certains cas pathologiques. Tel celui, que
rappelle le Dr Vinchon, d’un élève du peintre Reynolds qui, après
avoir regardé son modèle une demi-heure, était capable ensuite
de l’imaginer aussi nettement que s’il eût été là en réalité, mais
qui finit ainsi par « mêler son monde imaginaire avec le monde réel »
au point qu’il dut passer plusieurs années dans un asile. Sorti de
l’asile, il « ne conservait aucun souvenir de sa longue maladie ».
Tout semblable est le cas de cette malade de Pierre Janet, Irène,
qui, fortement émue de la mort de sa mère, ne se {\it souvenait} absolument
plus ni de cette mort ni de l’enterrement auquel elle avait assisté et
ne voulait même pas croire que sa mère fût morte. En pareil cas,
il y a, selon l’expression de Janet, {\it restitutio ad integrum}, activité
conservatrice, mais « il n’y a pas de mémoire du tout ».

Le \textbf{\textit {souvenir}} qui est l’acte propre de la mémoire, est donc tout
autre chose qu’une simple résurgence d’images. « {\it Le souvenir n’est
pas l’image}, dit J. Delay, {\it mais un jugement sur l’image dans le
temps}. » On verra en effet qu’il est le produit de {\it tout un ensemble de
fonctions} parmi lesquelles la {\it reconnaissance}, l'{\it attribution au passé} est
essentielle.

\section{Les images mentales}% 122.
Avant d’étudier ces fonctions qui
constituent la mémoire, insistons un peu sur la nature de l’{\it image}.

\vspace{0.24cm}
{\footnotesize 
{\it A.} Il faut d’abord distinguer l’image mentale de phénomènes analogues :
1° la \textsf{\textit {sensation rémanente}} qui est la simple {\it prolongation} de la sensation
après cessation de l’excitant. Le phénomène s'observe surtout pour les
sensations {\it visuelles} : l'impression lumineuse persiste sur la rétine d’un
deux-centième à un dixième de seconde ; c’est pourquoi, si l’on fait tourner
rapidement une lampe électrique de poche, nous voyons un cercle lumineux,
de même qu’au cinéma les clichés successifs qui se projettent sur l'écran,
nous apparaissent comme un film continu et mouvant. De même, quand
nous écoutons s’éteindre le tintement d’une cloche, nous croyons l'{\it entendre}
encore alors que les vibrations ont vraisemblablement cessé ; — 2° la \textsf{\textit {sensation
consécutive}} ou \textsf{\textit {complémentaire}}, spéciale aux phénomènes de la vue, et
qui est comme le {\it négatif} de la sensation première. Si par exemple par une
journée ensoleillée, nous regardons une fenêtre de l’intérieur, ses battants
et ses traverses nous apparaissent en blanc sur fond sombre quand ensuite
nous regardons le mur. Si l’on fixe quelque temps un cercle, un rectangle
de couleur vive et qu’on reporte ensuite les yeux sur un écran blanc, on
aperçoit la même figure teintée de la couleur complémentaire.

Dans tous ces cas, il y a bien {\it sensation}, et non {\it image}, car le phénomène
est {\it lié à une modification persistante de l'organe sensoriel}, que ce soit
l’impression première ou, dans le second cas, une fatigue des éléments rétiniens.}
\vspace{0.31cm}
%177

{\it B.} L'\textbf{\textit {image mentale}} est tout autre chose : c’est, nous dit-on, comme
une réplique affaiblie, dans la conscience, de la représentation première,
mais en {\it l'absence} de l’excitant extérieur et parfois {\it longtemps
après} que celui-ci a cessé d’agir.

\vspace{0.24cm}
{\footnotesize 
Taine nous décrit, par exemple, un paysage de Paris en 1867, le soir au
bord de la Seine, où la rivière luit encore dans la brume naissante, où
s’estompent les profils des monuments, l’abside de Notre-Dame, le dôme du
Panthéon, où les premiers lampadaires qui s’allument piquent de points
lumineux l'ombre qui s'étend et où se fait entendre au pied des quais le
bruissement du fleuve. C’est hier, ajoute-t-il, que j'ai eu ce spectacle, et
pourtant aujourd’hui il y a encore dans ma conscience comme un écho
affaibli des formes, des couleurs et des sons que j'ai réellement perçus.}
\vspace{0.31cm}

Malheureusement, en fonction de l’idée qu’on se faisait alors de la
vie de l’esprit et qui était,on le sait déjà (\S 37 {\it fin}), l’{\it atomisme psychologique},
Taine avait prétendu dissoudre toute cette vie en un ensemble
d'images, douée chacune d’une « force automatique » propre que
viendraient seulement équilibrer celles des sensations ou des autres
images concomitantes(\S 106 A). En ce sens, disait-il, « l’esprit agissant
est un {\it polypier d'images} mutuellement dépendantes », et l’unité de
l'esprit n’est « qu’une harmonie et un effet » de ces forces élémentaires.
A partir de là, se constitua toute une théorie de « l'imagerie mentale »,
où l’on prétendait expliquer toutes les fonctions mentales par des
combinaisons d'images, par exemple la {\it perception} par une simple
addition d'images à la sensation actuelle (\S71). — En outre, l’image
étant considérée comme une simple réplique de la sensation et celle-ci
comme une sorte de {\it copie} de l’objet perçu (\S 79), on en vint à se
représenter l’image elle-même comme une sorte de \textbf{\textit {cliché}} mental
{\scriptsize (Le mot {\it image} lui-même, un des plus mal choisis de tout le langage psychologique,
en faisant penser à une reproduction matérielle, incitait à cette conception erronée.)}
qui subsisterait en nous, à peu près inaltéré, sous forme de « traces » (on
inventa plus tard le terme d’{\it engrammes}
{\scriptsize (Création de l'Allemand R. Semon (1904) qui entendait par là les « empreintes »
produites dans « la substance organique », par les divers stimuli extérieurs et qui
définissait la mémoire, appelée alors la {\it Mnémé}, « la somme des engrammes que
l'organisme a hérités ou acquis pendant sa vie individuelle ».)}
) gravées dans telle région
déterminée du cerveau pour chaque sens. D’où cette conception des
« réservoirs d'images, peints en bleu ou en rouge sur des schémas du
cerveau », comme dit H. Préron, qu’on trouve encore dans certains
manuels d'anatomie.

{\it C.} On s’explique sans peine que, par réaction contre cette conception
à la fois {\it atomiste} et {\it statique}, certains auteurs soient allés jusqu’à
{\it nier l'existence même de l’image}. Cette existence, écrit le D' Moutier,
est « toute hypothétique, toute conventionnelle ». Les {\it behavioristes}
%178
eux aussi ont rejeté l’existence des images et prétendu les ramener
à de simples processus sensori-moteurs.

Cette négation radicale est cependant excessive. Elle est, dit
J.-P. Sartre, « contredite par les données de l’introspection : je
peux, quand je le veux, penser en image un cheval, un arbre, une
maison ». H. Piéron maintient de même que « l’image est une donnée
de sens commun, dont on peut seulement discuter la nature exacte
et le mécanisme ». On peut même affirmer qu’il existe des images,
non seulement pour la vue, mais pour l’ouïe, le toucher, l’odorat,
{\it pour tous les sens}, avec cette seule réserve que, {\it selon les individus,
c’est telle ou telle sorte d'images qui prédomine} : celui-ci est un visuel,
celui-là un auditif, etc. On sait quel est le rôle des images olfactives
dans la poésie de Baudelaire.

1° Ce qui est faux, c’est d’abord l’{\it atomisme mental}. Il n’existe
pas plus d’{\it image} isolée que de {\it sensation} isolée (\S80). La Psychologie
de la Forme, quels qu’aient été ses excès, nous a appris que ce que
nous percevons, ce sont des {\it ensembles}. Ce que l’image reproduit, plus
ou moins fidèlement d’ailleurs, c’est un tel {\it ensemble} perceptif. Peut-être
serait-il plus juste de parler avec Sartre de « conscience [au
sens de : structure] imageante » plutôt que d’image proprement dite.
L'image n’est pas la reproduction {\it de la sensation}, mais de {\it l’objet}
{\scriptsize (J. Lachelier avait déjà écrit ({\it Vocabulaire} de Lalande) : « Ce qui me paraît abus
de langage chez Taine, c’est d’avoir parlé de l’{\it image d'une sensation}. Y a-t-il même en
nous reproduction, sous quelque nom que ce soit, de sensations isolées? Nous ne
cessons au contraire de nous représenter intérieurement, souvent avec une extrême
vivacité, des objets visibles. »)}
 :
« C’est une certaine façon qu’a l’objet de paraître à la conscience ou,
si l’on préfère, une certaine façon qu’a la conscience de se donner un
objet. » L'image n’est ni une « illustration » ni un « support de la
pensée » ; elle est essentiellement pensée et, comme telle, « comprend
un savoir, des intentions » (Sartre). Ajoutons aussi, avec J. Piaget :
{\it des mouvements}. Car, si, comme on l’a vu, la motricité est déjà à
l’œuvre dans l’activité perceptive, elle doit se retrouver dans l’image,
riche en effet en éléments moteurs (\S 93). Il ne faut donc pas seulement
parler, comme on le faisait jadis, du « pouvoir moteur des
images »
{\scriptsize (Expériencs classiques : demander à quelqu’un ce qu’est un escalier en vis ou
en limaçon, — faites-vous expliquer, en feignant l’incrédulité, ce qu'est un
« homme-orchestre », et vous verrez le résultat obtenu.)}
 : le mouvement fait partie de leur structure même.

2° L’ancienne conception de l’image péchait encore par son caractère
{\it statique}. Pas plus que d’images isolées, il n’existe d’{\it images-clichés}
conservées dans le cerveau. Celui-ci, nous le savons (\S 30 {\it fin})
est un {\it organe d’action}, et « c’est une idée puérile que de s’imaginer
%179
qu’il constitue un magasin où se déposent de petits clichés, images
photographiques des événements qui ont affecté les sens ». Certes,
il existe bien des aires ou des zones de l’écorce cérébrale où viennent
se diffuser et se projeter les processus nerveux propres à telle ou telle
espèce de sensations (aires visuelle, auditive, tactile, etc. : \S 34 B).
Mais l'évocation de l’image n’est pas autre chose qu’un « {\it dynamisme
associatif} » consistant dans « la mise en jeu, d’origine centrale cette
fois, des mêmes éléments récepteurs corticaux » correspondant à
l'impression sensorielle, et « l’image n’existe pas en dehors de ce processus
d’évocation » (Piéron). — Ainsi conçue, l’image échappe aux
critiques qu’on lui a adressées : « Une conception dynamique de
l’image, voire des centres d’images, loin d’être périmée, s'accorde
avec les données actuelles de la pathologie cérébrale » (J. Delay).

\section{Les fonctions de la mémoire}% 123.
L'image est comme
l'embryon du souvenir. Il nous faut donc étudier maintenant les
fonctions qui de l’{\it image} font un {\it souvenir}. — {\it A.} Certes, il peut arriver
que le souvenir « s’implante » de lui-même dans l'esprit. Mais que de fois
ne sommes-nous pas obligés de l’y fixer volontairement, parfois laborieusement !
Le souvenir implique donc une première fonction de
l'esprit : la fonction de \textbf{\textit {fixation.}} — {\it B.} De même, le souvenir peut,
dans certains cas, surgir spontanément à l'esprit. Mais que de fois
aussi il nous échappe et semble nous fuir! Il nous faut alors le chercher,
faire un {\it effort} d’évocation. Nous aurons, par suite, à considérer une
seconde fonction de la mémoire : le \textbf{\textit {rappel}} ou \textbf{\textit {évocation}} des souvenirs.
— {\it C.} Enfin, comme on vient de le voir, il n’y a pas souvenir proprement
dit si l’image qui surgit à notre esprit n’est pas {\it reconnue}, c’est-à-dire
plus ou moins consciemment {\it attribuée} au passé. D'où une
troisième fonction : la fonction de \textbf{\textit {reconnaissance.}} — D. Souvent
même le souvenir est {\it daté}, c’est-à-dire {\it localisé} dans le passé. On verra
que cette \textbf{\textit {localisation}} du souvenir n’est qu’un prolongement de la
reconnaissance. — L'étude des cas pathologiques (\S 145) nous montrera
que ces distinctions ne sont pas artificielles, la maladie pouvant
effectivement parfois {\it dissocier} ces fonctions les unes des autres.

\section{La fixation des souvenirs et la mémoration}% 124.
Nous
distinguerons la fixation spontanée et la fixation volontaire.

{\it A.} Comme on vient de le dire, il y a des cas où le souvenir semble
se fixer de lui-même. Mais nous savons tous par expérience que tout
ne se fixe pas ainsi. La conscience est {\it sélective} (\S 37) : elle n’est pas
également sensible à tout ce qui se présente à elle. L’{\it attention spontanée}
(\S 49 A) n’est qu’un prolongement de cette propriété fondamentale.
Elle obéit à une \textbf{\textit {loi d'intérêt,}} qui se retrouve dans la fixation
%180
des souvenirs : nous retenons surtout ce qui nous {\it intéresse}, au
sens le plus large de ce terme, c’est-à-dire ce qui correspond à nos
tendances, à nos préoccupations habituelles, à nos désirs du moment1
{\scriptsize (Examinant la question de la mémoire chez la femme, G. Heymans déclare que les
résultats des observations ne sont pas concordants. Mais, dit-il, « bien des femmes ne
peuvent retenir des idées très simples qu’elles ont entendues cent fois, simplement
parce que celles-ci ne les intéressent pas » (en particulier, les idées abstraites).)}.
Un des facteurs les plus puissants de cet intérêt est l’\textbf{\textit {affectivité}} :
nous retenons ce qui nous « touche », comme dit Malebranche, ce qui
nous émeut, nous attendrit ou nous indigne. « La mémoire, a-t-on dit,
est toujours aux ordres du cœur. » — Ainsi, même dans la fixation
spontanée, il y a déjà une {\it réaction propre de l'esprit}, une structuration
commandée par des facteurs internes : « Il est enfantin, dit R. Ruyer,
de se représenter la formation d’un souvenir comme une impression
physique de l’extérieur sur l’intérieur. »

B. A plus forte raison, l'\textbf{\textit {activité de l'esprit}} intervient-elle dans la
fixation {\it volontaire}. 1° Elle peut se manifester d’abord par la simple
\textbf{\textit {répétition,}} par exemple lorsque nous apprenons un texte par cœur.
Mais, ici comme dans l’{\it habitude} dont la mémoire se rapproche d’ailleurs
beaucoup en pareil cas, la répétition ne suffit pas : les répétitions
doivent être {\it espacées}, de façon à laisser s’accomplir un phénomène
de {\it maturation} sur lequel nous reviendrons au t. II, chap. V.
Certaines conditions physiologiques sont même ici à prendre en considération
{\scriptsize (Ribot remarquait que « pour fixer les souvenirs, il faut du temps parce que la
nutrition cérébrale ne fait pas son œuvre en un instant ». H. Piéron rappelle les
expériences de Speck d'où il résulte que, lorsqu'on respire un air à très faible
tension d’oxygène (8 p. 100), la fixation des souvenirs devient à peu près impossible.)}.
— 2° Mais les conditions proprement psychologiques sont
plus importantes encore : elles constituent ce qu’on peut appeler
l’acte de \textbf{\textit {mémoration.}} Dès qu’il ne s’agit plus d’un apprentissage
purement mécanique, l'\textbf{\textit {attention volontaire}} qui, nous le savons
(\S 49 B et C), est souvent indépendante de l'intérêt immédiat, devient
indispensable. Non seulement elle fixe l’esprit sur l’objet à retenir,
mais elle procède à une double opération d'{\it organisation} et, si l’on
peut ainsi parler, de {\it décantage} du souvenir. {\it Organisation} en ce sens
que nous cherchons à relier le souvenir à ce que nous savons déjà :
« La conservation d’un souvenir, dit W. James, est fonction du nombre
de ses associations. » Mais il s’agit bien moins d’associations au sens
classique du terme (chap. XII) que d’un « acte de synthèse mentale »
(Delay), de l’{\it intégration du souvenir à tout un système mental} ; tantôt
nous nous contentons de ressemblances purement verbales, d’analogies
quelque peu superficielles
{\scriptsize (On connait les acrobaties verbales dont s’est rendue coupable l’ancienne {\it }mnémotechnie
pour aider à retenir les noms des préfectures et sous-préfectures des départe-
ments (ex. : {\it }Caen la nuit vient, l’affreux cha{\it cal va d'os} en os, etc. !). On a composé un
poème de cinq quatrains permettant de retrouver d’après le nombre de lettres des mots
les trente premières décimales du nombre PI.)}
; tantôt nous établissons de véritables
%181
relations logiques, comme par exemple lorsque nous cherchons
à retenir les différents éléments d’une doctrine philosophique. Mais
nous procédons aussi, plus ou moins consciemment, à une véritable
{\it décantation} du souvenir, en ce sens que nous le laissons déposer, pour
ainsi dire, les détails oiseux, les éléments insignifiants et aussi tout ce
qu’il nous déplairait de nous rappeler, pour n’en retenir que ce qu’il
nous est agréable de laisser durer
{\scriptsize (Exemple emprunté à un roman d'André Beaunier, {\it L'Amour et le secret} : une femme
qui vient de perdre son mari, fait son éloge : « Elle vantait en lui des qualités qu’il n'avait
pas eues ni recherchées [sa fidélité, sa douceur]... On faisait là-haut la toilette du mort...
Pendant ce temps, Jenny faisait la toilette du souvenir. Elle en retirait çe qui ne serait
ni agréable ni commode à conserver. Elle apprêtait le souvenir de Jacques à durer sans
accident. »)}
. — Au reste, comme l’a fait remarquer
P. Janet, nous ne fixons pas seulement un souvenir {\it pour nous},
mais, le plus souvent, pour le raconter à {\it autrui}. L’acte de {\it mémoration}
est donc, au fond, un acte {\it social} : telle la sentinelle d’un camp de
primitifs qui fixe dans son esprit les mouvements de l’ennemi en vue
de les rapporter à son chef ; tel l’étudiant qui enregistre une multitude
de choses pour les réciter, le jour de l’examen, à son interrogateur ;
tel le malade qui se répète à lui-même les symptômes qu’il
décrira à son médecin, etc. La mémoire est essentiellement, selon
Janet, une \textbf{\textit {conduite du récit.}}

\section{Le rappel des souvenirs et la remémoration}% 125.
Aristote
avait distingué deux formes du rappel des souvenirs, la {\it mémoire
simple} ({\it mnèmè}) qui consiste dans la conservation du passé et dans son
retour {\it spontané }à l'esprit, et la {\it remémoration} ({\it anamnèsis}) qui est la
faculté de rappeler {\it volontairement} les souvenirs.

{\it A.} On ramène généralement le \textbf{\textit {rappel spontané}} aux lois de l’{\it association
des idées}, que nous étudierons au chapitre XII. On verra qu’une
perception actuelle, qu’un détail à peine aperçu par la conscience
peut suffire à évoquer une expérience passée et parfois à reconstituer
toute une scène. Il y a lieu de faire, ici encore, une place importante
aux {\it états affectifs} qui, nous le verrons (t. II, chapitre VII), fournissent
souvent le thème architectonique du rêve et qui, de même,
donnent le ton, pour ainsi dire, aux souvenirs qui nous reviennent à
l'esprit : lorsque nous sommes tristes, nous n’avons que des souvenirs
tristes comme, lorsque nous sommes joyeux, des souvenirs
joyeux. Il s’agit bien moins ici d’une sélection de souvenirs isolés que
d’une unité de ton, d’une consonance subjective générale.

B. Quant au \textbf{\textit {rappel volontaire}} ou \textbf{\textit {remémoration,}} il exige un effort
%182
caractéristique. W. James a décrit l’état vraiment original où se
trouve notre conscience lorsque nous cherchons, par éxemple, un
nom oublié : « Il y a en nous un vide, mais un vide extraordinairement
actif. Il enveloppe comme un fantôme du mot cherché, fantôme qui
nous fait signe de venir de son côté, qui par moments nous donne, à en
brûler, le sentiment que “ nous le tenons ” et qui s’échappe en nous
laissant retomber sans rien tenir du tout. » Nous avons alors l’impression
{\it d’un ensemble psychique qui cherche à se compléter}. Quel est en ce
cas le rôle de la {\it volonté ?} Il n’est pas d'empêcher le libre jeu de la
spontanéité mentale : tout le monde a pu observer que, lorsque nous
concentrons trop notre effort de remémoration, le souvenir semble
s'éloigner de plus en plus ; il faut alors laisser du repos à l'esprit,
c’est-à-dire laisser jouer la pensée spontanée, et c’est bien souvent
quand nous n’y pensons plus que le souvenir réapparaît. Le rôle de la
volonté est d’écarter les obstacles, les inhibitions qui proviennent de
l'orientation actuelle de notre esprit, et de {\it }recréer une atmosphère,
un ensemble doué d’une certaine tonalité affective, au sein duquel
le souvenir surgira
{\scriptsize (Ce passage du {\it Mystère Frontenac} de François Mauriac est intéressant à cet égard
(Xavier s'efforce d'évoquer l'image de son frère mort) : « Xavier détestait les vers.
Mais maintenant quelques-uns lui revenaient qui avaient gardé l'inflexion de la voix
chérie. Il fallait qu'il les retrouvât pour retrouver l'intonation sourde et monotone de
son frère. Ainsi, ce soir-là, près de la fenêtre ouverte du côté de la rivière invisible, de
même qu'il eût cherché une note, un accord, Xavier récitait sur des tons différents :
{\it Nature au front serein, comme vous oubliez !} »)},
C’est pour cela sans doute qu’on a pu dire que la
remémoration est {\it participation à une conscience autre que la conscience
actuelle}, qu’elle « nous fait communier à l’intimité d’une {\it autre} conscience,
prisonnière de sa durée “ passée ” comme nous le sommes de
la durée présente » (F. Ellenberger).

Toutefois la remémoration a souvent un caractère {\it beaucoup plus
intellectuel}. Il s’en faut en effet que nous puissions toujours communier
ainsi avec le passé. Le plus souvent, il s’agit bien moins d’aller
chercher le souvenir dans un passé où il nous attendrait, tout prêt
à reprendre consistance, que de le \textbf{\textit {recréer}}, tout au moins de le \textbf{\textit {reconstruire}}
pièce à pièce, péniblement : « {\it Le souvenir}, dit M. Halbwachs,
{\it est dans une très large mesure une reconstruction du passé à l’aide de
données empruntées au présent et préparée d’ailleurs par d’autres
reconstructions faites à des époques antérieures et d’où l’image est sortie
déjà bien altérée}. » C’est ainsi que Rousseau avoue qu’en écrivant ses
{\it Confessions}, la mémoire lui manquait souvent ou ne lui fournissait que
des souvenirs incomplets et qu’il « en remplissait les lacunes » par des
détails qu’il imaginait en accord avec ce que vraisemblablement les
%183
choses avaient dû être. De là toutes les \textbf{\textit {déformations}} qu'involontairement
nous faisons subir à nos souvenirs.

\vspace{0.24cm}
{\footnotesize 
Voici un exemple dû à STENDHAL, qui raconte son passage du mont Saint-Bernard
en 1800 : « Enfin j’aperçus à gauche une maison basse. C'est
l’Hospice ! Il me semble que nous entrâmes, ou bien les récits de l’intérieur
de l’Hospice qu’on me fit produisirent-ils une image qui, depuis trente-six
ans, a {\it pris la place de la réalité}. Voilà un danger de mensonge que j'ai aperçu
depuis trois mois que je pense à ce journal. Par exemple, je me figure fort
bien la descente, mais je ne veux pas dissimuler que, cinq ou six ans après,
je vis une gravure que je trouvai fort ressemblante; et mon souvenir {\it n’est
plus} que la gravure... » — Autre exemple d’un écrivain contemporain :
« Mes carnets et ma mémoire ne sont pas d'accord. Il s’en faut ! Si je m’en
réfère à mes notes, c’est un moi geignard, impatient, insatisfait, sans paix
ni joie, que j'y retrouve... Mais, si j'interroge ma mémoire, elle ne me tend
que beaux jours, que chances providentielles, que rencontre de gens de
qualité, qu'amour. C’est un moi radieux, enchanté de sa vie, sinon de soi,
qu’elle ressuscite » (Paul Géraldy).}
\vspace{0.31cm}

\section{La reconnaissance des souvenirs}% 126.
Mais, pour qu’il y
ait vraiment mémoire, il ne suffit pas, nous l’avons dit, que le souvenir
soit rappelé : il faut encore qu’il soit \textbf{\textit {reconnu}}. Il y a lieu de distinguer
ici deux problèmes que l’on confond souvent : la \textbf{\textit {reconnaissance du
présent}} qui concerne plutôt la {\it perception}, et la \textbf{\textit {reconnaissance du
passé comme passé}} qui concerne proprement le souvenir.

{\it A.} Sur le premier cas, nous avons déjà dit le nécessaire (\S 83 {\it C} et
93). Cette reconnaissance « dans l’instantané » est « une reconnaissance
dont le corps tout seul est capable, sans qu'aucun souvenir
explicite intervienne» (Bergson). Ici les gestes, les mouvements
suffisent : reconnaître un objet usuel, c’est {\it savoir s’en servir}. Bien
entendu, à cette attitude motrice s’ajoute un « sentiment de familiarité »
qui n’en est guère que l’écho dans la conscience : « Nous jouons
d’ordinaire notre reconnaissance avant de la penser. »

{\it B.} Il en va tout autrement de la reconnaissance du souvenir. Selon
une théorie de l’école écossaise (Reid), reprise par Ad. Garnier
(1852), « quand nous nous souvenons de ce que nous avons fait, nous
le voyons, nous le vivons, nous sommes dans le passé : nous le percevons
sans intermédiaire. Vieille conception de l'{\it intuition du passé}
qui dominait en 1860: et qui va réapparaître aujourd’hui » (Janet).
Selon Bergson en effet, en dehors de la reconnaissance motrice
dont il vient d’être question et qui n’est guère qu’une facilité de répétition,
il existe, dans la mémoire-souvenir (\S 129), une reconnaissance
qui est {\it expérience immédiate du passé en lant que passé} : le souvenir
« a pour essence de porter une date » (Bergson). Plus récemment,
M. Merleau-Ponty a émis l’opinion que « l’on ne peut comprendre
%184
la mémoire que comme une possession directe du passé sans contenus
interposés ». Ici encore, les théories de « l'expérience immédiate »,
nous ramenant aux conceptions de la philosophie écossaise, prétendent
résoudre le problème en le supprimant. Mais cette conclusion
est {\it contraire aux faits} dont chacun de nous a l’expérience fréquente.
Que de fois n’arrive-t-il pas qu’une vague vision, par exemple,
surgisse à notre esprit et que nous nous demandions s’il s’agit d’un
souvenir réel, ou bien de quelque chose que nous avons lu dans un
livre, ou encore d’une image de rêve ou d’une fantaisie de notre imagination !
Il n’y aura proprement souvenir que lorsqu’{\it après recherche}
nous nous serons convaincus que ce souvenir {\it appartient effectivement
à notre passé}. Descartes l’avait fort bien remarqué :

\vspace{0.24cm}
{\footnotesize 
« Il ne suffit pas, pour que nous nous ressouvenions de quelque chose, que
cette chose ait été observée auparavant par notre esprit et ait laissé dans le
cerveau des traces qui occasionnent sa réapparition à notre pensée ; il faut
en outre que, lorsqu'elle se présente pour la seconde fois, nous reconnaissions
que cela se fait parce qu’elle a été perçue antérieurement par nous ;
c'est ainsi qu'il se présente souvent à l'esprit des poètes des vers qu’ils ne
se rappellent pas avoir jamais lus chez d’autres et qui cependant ne se
seraient pas ainsi présentés à eux s'ils ne les avaient lus ailleurs. D'où il
apparaît que ce ne sont pas n'importe quelles traces laissées dans le cerveau
par les pensées précédentes qui suffisent pour qu’il y ait mémoire : ce sont
celles-là seulement qui permettront à l'esprit de reconnaître qu’elles n’ont
pas toujours été en nous, mais qu’elles y sont venues un jour comme une
nouveauté ({\it de novo}). Or, pour que cela soit possible, j'estime nécessaire que
l'esprit ait dû user, lorsqu'elles s’y imprimaient pour la première fois, de
l’intellection pure, afin de pouvoir remarquer que la chose alors observée
était nouvelle ; il ne peut exister en effet aucune trace corporelle capable
de marquer cette nouveauté » ({\it Lettre à Arnauld} du 29 juillet 1648).}
\vspace{0.31cm}

Ce texte est fort intéressant, non seulement parce que Descartes
y marque la nécessité, pour qu’il y ait souvenir, de l’acte de reconnaissance,
mais surtout parce qu’il lie cet acte à ce que Janet a appelé
l’acte de « présentification» de la perception qui nous fait saisir celle-ci
comme « nouvelle » (\S 104 fin), et parce qu'il en affirme le \textbf{\textit {caractère
proprement intellectuel.}} Constitution du récit et constitution du
{\it présent} sont solidaires. Le souvenir, en effet, est bien différent d’une
représentation {\it imaginaire} : c’est du {\it réel}, du réel passé sans doute,
mais que nous ne pouvons pas modifier à notre guise et qui ne diffère
du réel présent que parce que nous sentons {\it qu’il s'intègre à une autre
conscience que notre conscience actuelle}. C’est là aussi ce qui distingue
le souvenir complet de la simple {\it réminiscence}
{\scriptsize (Ce terme de {\it }réminiscence est équivoque. Nous le prenons ici au sens usuel, où il
désigne un souvenir incomplet, non reconnu. Mais, dans le langage philosophique, il
désigne souvent soit la remémoration, soit le mythe platonicien d’après lequel il
subsisterait en notre âme un {\it ressouvenir} d’un état ancien où nous possédions, selon Platon,
une vue directe des réalités intelligibles (voir \S 190 {\it A}).)}, telle celle des poètes
dont parle Descartes, où cette intégration manque.

%185
Il s’en faut d’ailleurs que la reconnaissance du passé prenne toujours
cette forme pleinement consciente. Parfois elle ne dépasse pas
le stade du \textbf{\textit {sentiment du déjà-vu}} qui accompagne aussi la reconnaissance
du présent, et il se peut que, comme dans ce dernier cas,
ce sentiment soit lié à certains éléments {\it moteurs}, dont, nous le savons
(\S 122 C 10), l’image n’est pas dépourvue. Mais il semble que ce
« sentiment confus » de {\it déjà-vu}, cette « auréole de passé », comme dit
W. James, qui accompagne le souvenir naissant, traduise surtout à
la conscience l'intuition vague et encore obscure de son {\it contexte} mental,
c’est-à-dire de toutes les attaches, des « associations » si l’on veut,
grâce auxquelles il est ancré au fond de nous-même. C’est ainsi, grâce
à ce contexte latent, que nous sentons qu’il nous {\it appartient}.

Mais la \textbf{\textit {reconnaissance proprement dite}} est tout autre chose que
ce sentiment vague. Elle consiste dans une {\it prise de conscience} de ces
rapports entre le souvenir et notre moi passé, qui n'étaient encore
qu’obscurément pressentis dans le sentiment du déjà-vu. Elle devient
alors, comme l’a bien vu Descartes, un acte {\it intellectuel}, un véritable
{\it jugement}, qui consiste à la fois à repousser le souvenir comme actuellement
nouveau, donc à le rejeter de notre moi présent, et à l’attribuer
comme tel à notre moi passé. Ce \textbf{\textit {jugement d’antériorité}} devient
nettement conscient quand il y a eu hésitation de notre part et
qu’enfin nous parvenons à la certitude que « nous avons réellement
vu, entendu ou vécu cela ». Son caractère paradoxal qui consiste en
somme à attribuer explicitement au passé un état redevenu présent,
peut s’exprimer, comme l’a fait Goblot, par cette formule : « Ce phénomène
{\it présent} et {\it mien} est bien {\it mien}, mais n’est {\it pas présent}. »

\section{La localisation des souvenirs}% 127.
Bien souvent, nous
ne nous contentons pas d'{\it attribuer} nos souvenirs au passé, nous cherchons
à les \textbf{\textit {situer}} dans ce passé, à les \textbf{\textit {dater}} : c’est la localisation du
souvenir. Taine et Ribot avaient admis que cette localisation s’effectue
par une sorte de glissement du souvenir sur la ligne du temps
entre certains {\it points de repère} par rapport auxquels il finissait par se
situer. Mais cette explication relève encore trop d’une psychologie
atomiste. Les points de repère ne sont pas {\it extérieurs} au souvenir à
localiser : ils {\it font partie du même ensemble} de notre moi d’antan et
c’est en nous efforçant d'amener celui-ci à la pleine conscience que
nous les y découvrons. La localisation consiste donc à {\it resserrer le
%186
\textbf{\textit {système de rapports}} dans lequel la simple reconnaissance avait réussi
à cerner le souvenir}, et l’on verra que ce système de rapports n’est
autre que le \textbf{\textit {temps}} (chap. XI).

\vspace{0.24cm}
{\footnotesize 
L’exemple cité par Taine est ici bien caractéristique. Il rencontre dans
la rue un homme qu’il se rappelle avoir déjà vu il y a quelque temps.
Celui-ci lui a dit alors « qu’il attendait pour partir les premières pousses
des feuilles » : c'était donc au printemps. Mais quand exactement ? Taine
s'efforce de penser plus clairement son souvenir qui, dit-il, « s’entoure de
détails nouveaux » : il revoit notamment des personnes qui, ce jour-là,
portaient des branches de buis dans la rue. C'était donc le dimanche des
Rameaux !}
\vspace{0.31cm}

Il convient d’insister d’ailleurs sur l'importance de ces \textbf{\textit {points de
repère}} qui jalonnent pour nous le passé, On verra bientôt que ce sont
presque toujours des points de \textbf{\textit {repère sociaux.}}

\section{Le problème de la mémoire}% 128.
Si maintenant nous abordons
le problème général de la mémoire, nous prendrons garde d’abord
de le poser correctement. Les anciens manuels de psychologie le
ramenaient à celui de la {\it conservation du souvenir}. Mais est-il certain
que le souvenir {\it se conserve} à proprement parler ? « La conservation,
écrivait W. James, n’est qu’un nom de la {\it possibilité} de penser à
nouveau et de la {\it tendance} à penser à nouveau une expérience avec ses
anciens concomitants. » Quoi qu’il en soit, le problème ayant été
ainsi posé, les anciennes théories s'étaient efforcées d'expliquer cette
prétendue conservation, soit physiologiquement (\S 129), soit psychologiquement (\S 130).

\section{Mémoire et habitude}% 129.
La première explication a été
surtout illustrée par Ribot dans ses {\it Maladies de la mémoire} (1881).
Disons tout de suite qu’elle consiste au fond à ramener la mémoire à
l'habitude. Ribot considère l’acte de la {\it reconnaissance} comme accessoire :
« tout l'essentiel de la mémoire » consiste, pour lui, dans
la {\it conservation} et la {\it reproduction} (rappel) des souvenirs. Ainsi « la
mémoire est par essence un fait biologique, par accident un fait psychologique ».
Beaucoup de faits physiologiques nous montrent en
effet comment « un état nouveau s'implante dans l’organisme, se
conserve et se reproduit ». Le tissu nerveux, en particulier, possède
au plus haut degré cette double propriété. On peut donc admettre
que la mémoire s’explique, non pas sans doute, comme l'avaient
supposé les Cartésiens (voir le texte \S 126 B), par des « traces », des
« Vestiges », des « empreintes » qui se fixeraient mécaniquement dans
%187
le cerveau — « les modifications résultant de l'impression première,
dit Ribot, ne sont pas conservées dans une matière inerte ; elles ne
ressemblent pas au cachet imprimé sur la cire », — mais par des
« dispositions {\it fonctionnelles} » acquises par les éléments nerveux, auxquelles
s’ajoutent des « associations {\it dynamiques} » entre ces éléments,
devenues stables par la répétition : la simple mémoire d’une pomme,
par exemple, suppose l’action synergique d’une multitude d’éléments
d’ordre visuel (forme, couleur du fruit), tactile, gustatif, etc. — Ribot
fondait cette interprétation sur une double série de faits : 1° la relation
étroite qui existe entre la \textbf{\textit {nutrition}} des tissus vivants et la conservation
des souvenirs : l’enfant chez qui la nutrition est la plus
active, assimile facilement ; le vieillard chez qui elle décline, ne retient
pas le nouveau ; la fatigue est fatale à la mémoire ; 2° les \textbf{\textit {maladies de
la mémoire}} : celles-ci obéissent, selon Ribot, à une \textbf{\textit {loi de régression}}
(souvent appelée aujourd’hui {\it loi de Ribot}) selon laquelle la perte de
la mémoire descend toujours « de l’instable au stable » : elle va par
exemple du {\it plus récent}, moins bien organisé, à l’ancien, mieux consolidé
par de nombreuses répétitions; pour les mots, elle va des {\it noms
propres}, moins souvent répétés, aux {\it noms communs}, puis aux {\it adjectifs}
et aux {\it verbes} pour s'étendre enfin aux {\it interjections}.

{\it Discussion}. Il est de mode aujourd’hui de traiter avec mépris cette
théorie physiologique de Ribot
{\scriptsize (Il y a lieu de s'élever à ce propos, au nom de la probité intellectuelle, contre les
déformations que certains ouvrages récents font subir à l'exposé de la théorie de Ribot.
L'un d'eux s'exprime ainsi : « L'explication la plus grossière consiste à dire que les
souvenirs s'emmagasinent dans le cerveau. C'est en somme celle de Ribot. » Et l’auteur
{\it objecte} à cette prétendue thèse : « Ce qu’on localise, ce n’est pas le souvenir lui-même,
mais les mécanismes cérébraux au moyen desquels le souvenir se développe en images
ou s’actualise en mouvements. » Or c'est là précisément la thèse soutenue par Ribot !
— On lit dans un autre : « {\it La théorie physiologique de Ribot}. Le souvenir serait fixé en
un point du cerveau comme une empreinte matérielle. » Confronter ces interprétations
avec Ribot, p. 27-29, 50, etc., et avec les citations ci-dessus.)}
. On lui oppose même parfois des
objections qui portent absolument à côté.

\vspace{0.24cm}
{\footnotesize 
On lui reproche par exemple d’avoir admis que chaque souvenir correspond
à une «empreinte matérielle» dans une cellule cérébrale
{\scriptsize (Certains sont allés jusqu'à se demander si notre cerveau est assez riche en cellules
pour enregistrer tous nos souvenirs. Ribot répondait déjà qu'un élément mémoriel
pouvant entrer dans plusieurs combinaisons, le nombre de nos cellules corticales était
largement suffisant. Or on évaluait alors ce nombre à 600 000 ; on sait aujourd'hui
qu'il avoisine 14 milliards.)}
. Or non
seulement Ribot n’a nullement privilégié les « cellules », — il a parlé plus
prudemment des « éléments nerveux »
{\scriptsize (On sait que l'unité nerveuse est la fibre, non la cellule qui a, par rapport à la fibre,
un rôle purement trophique.)}
, — mais il convient de porter à
son actif une conception essentiellement {\it }dynamique et {\it }fonctionnelle, comme
on l’a vu, des processus cérébraux nécessaires à l'évocation de l’image et
%188
qui prévaut aujourd'hui (\S 122 C 2°). — On lui fait grief aussi d'avoir
admis les {\it localisations cérébrales}. Mais nous avons montré (\S 29) qu'interprétée
précisément en un sens dynamique, cette théorie est loin d'être
périmée. Au Congrès international de Psychologie de Montréal de 1954,
le Dr Penfield a établi que la stimulation électrique de certaines régions
du cerveau peut faire revivre chez le sujet avec une acuité et un luxe de
détails surprenants des scènes ou des faits du passé.}
\vspace{0.31cm}

Les objections qu’on peut valablement opposer à la théorie de
Ribot sont d’un tout autre ordre. Partant d’une conception \textbf{\textit {épiphénoméniste}}
de la conscience, Ribot a été amené à considérer l’acte
par lequel la conscience {\it reconnaît} le souvenir comme sans importance.
Sa théorie se trouve ainsi {\it faussée dès le point de départ}, dès la définition
qu’il nous donne de la mémoire. Si en effet celle-ci est essentiellement,
ainsi que nous l’avons dit, {\it la prise de conscience du passé
comme passé}, elle est tout autre chose que la « mémoire organique ».
Que nos muscles, nos nerfs, notre cerveau conservent certaines {\it dispositions}
à reproduire les actes qu’ils ont déjà accomplis, c’est infiniment
probable, Mais ce n’est pas là de la mémoire : nos organes ne se
{\it souviennent} pas à proprement parler ; ils {\it répètent}, ce qui est tout différent.
Au fond, Ribot a \textbf{\textit {confondu la mémoire avec l'habitude}}. Or non
seulement l'habitude n’est pas la mémoire, mais, dans bien des cas,
{\it elle dispense de la mémoire} : il suffit que notre corps {\it agisse} sans que
notre esprit se {\it représente} le passé. On va voir que c’est là la distinction
fondamentale introduite par Bergson.

\section{Mémoire et souvenir pur}% 130.
H. Bergson, dans {\it Matière
et Mémoire} (1898), prétend non seulement réfuter la théorie physiologique,
mais éclairer le problème métaphysique de « la relation du
corps à l’esprit ». Il commence par distinguer {\it deux formes bien différentes
de la mémoire}. Si, par exemple, j'apprends par cœur une poésie,
je puis la répéter en scandant les vers et, au bout d’un certain nombre
de répétitions, je dis que {\it je sais} ma leçon. Mais je puis aussi, une fois
la leçon apprise, me {\it représenter} les différentes phases de l’apprentissage,
chacune avec son individualité propre, avec les circonstances
qui l’accompagnaient, comme « un événement déterminé de mon
histoire » occupant une certaine place dans le temps.

Sous la première forme, le « souvenir » de la leçon a {\it tous} les caractères
de l'habitude : comme elle, il s’acquiert par répétition, exige la
décomposition, puis la recomposition de l’acte, s’emmagasine dans
un mécanisme qu’une impulsion initiale suffit à déclencher. Cette
\textbf{\textit {mémoire-habitude}} peut s'expliquer comme l’a voulu Ribot : c’est
« une expérience qui se dépose dans le corps ». — Mais il en va tout
%189
autrement de la seconde, qui est la \textbf{\textit {mémoire vraie.}} Elle n'a {\it aucun}
des caractères de l’habitude : elle consiste à {\it imaginer}, non à répéter ;
l’image, immédiatement datée, s’imprime du premier coup dans l’esprit
sans avoir besoin d’une répétition qui ne ferait « qu’en altérer la
nature originelle ». Il ne saurait donc être question de localiser une
telle mémoire dans le corps. « Image lui-même »
{\scriptsize (Bergson se place ici à un voint de vue idéaliste, d'où le réel lui-même apparait
comme un ensemble d'images (ci-dessus \S 118 B), ce mot étant pris d’ailleurs en un
sens large où il désigne « toute présentation ou représentation sensible » (Lalande).)}
, le corps, et notamment
le cerveau, ne peut emmagasiner la totalité de nos images, puisqu'il
n’est lui-même qu’une image parmi d’autres. D’autre part,
Bergson critique les arguments que la théorie physiologique tirait des
maladies de la mémoire. Sans entrer dans le détail de cette critique,
aussi périmée aujourd’hui que la théorie qu’elle prétend réfuter
{\scriptsize (Bergson objecte notamment que, si l’on admet avec Ribot que l’image a le même
« siège » dans le cerveau que la perception première, celle-ci devrait disparaitre avec
l'image elle-même. Or c’est, dit-il, ce que l'expérience dément : les sujets atteints de
surdité verbale ou psychique, c'est-à-dire qui ne reconnaissent plus les mots entendus ou
les bruits, ne sont nullement sourds. — Il suffirait de distinguer, comme on le fait
aujourd'hui (\S 29 B), entre la zone auditive (réceptrice) et la zone audito-gnosique
(psychique) pour répondre à l’objection.)}
,
notons seulement l’objection majeure, que Bergson tire de la loi de
Ribot elle-même : d’après cette loi, la perte de la mémoire suit une
marche déterminée ; or les lésions du cerveau peuvent « entamer les
cellules » dans un ordre quelconque ; il n’y a donc pas correspondance
entre les lésions du cerveau et les troubles de la mémoires
{\scriptsize (A vrai dire, cette objection n'est guère concluante, elle non plus. Les amnésies
progressives de ce genre s’observent surtout dans les cas de sénilité, où l'analyse
histologique a permis d'observer une dégénérescence générale des éléments nerveux.)}
.

Mais, si l’on renonce à une explication physiologique, comment
rendre compte de la conservation des souvenirs ? « {\it La mémoire}, écrira
plus tard Bergson, {\it n’a pas besoin d'explication. Ou plutôt, il n'y a
pas de faculté spéciale dont le rôle soit de retenir du passé pour le verser
dans le présent. Le passé se conserve de lui-même, automatiquement}. »
On a vu en effet (\S 120) que, selon Bergson, « toute conscience est
mémoire ». Et ainsi, le souvenir ne {\it succède} pas à la perception : il en
est contemporain, et il continue à {\it durer}, fantôme invisible, au fond
de nous-mêmes : « {\it Je crois}, dit Bergson, {\it que notre vie passée est là,
conservée jusqu'en ses moindres détails, que nous n'oublions rien, et
que tout ce que nous avons perçu, pensé, voulu depuis le premier éveil
de notre conscience persiste indéfiniment}. » Ce serait en effet une erreur
de croire que la conscience soit « la propriété essentielle des états
psychiques » : elle n’est que « la marque caractéristique du {\it présent},
c’est-à-dire de l’agissant » (cf. \S 31 D). Ce qui n’agit pas peut donc cesser
%190
d’être conscient sans cesser pour cela d'exister. Tel est précisément le
cas de nos \textbf{\textit {souvenirs purs}} dont, sans le savoir, nous portons avec nous
toute la masse dans notre inconscient. Il y a cependant un cas où ce
monde obscur des fantômes du passé se révèle à nous : c’est celui où
la conscience « se désintéresse » de l’action présente : le cas du {\it rêve}.
C’est donc dans le rêve qu’il faut chercher la {\it mémoire pure}. « Un être
humain qui {\it réverait} son existence au lieu de la vivre tiendrait sans
doute sous son regard, à tout moment, la multitude infinie des
détails de son histoire passée », et Bergson affirme même que plus
nous dormons profondément, plus nous avons « une vision étendue
et détaillée de notre passé ».

Comment donc ces « images de rêve », ces {\it souvenirs purs} vont-ils
devenir conscients, s’incarner dans des {\it souvenirs-images ?} « Le souvenir
actualisé en image diffère profondément du souvenir pur. L'image
est un état présent et ne participe du passé que par le souvenir dont
elle est sortie. » Si elle redevient présente, c’est donc, soit, comme on
l'a vu à propos du rêve, par son rapport avec une {\it sensation} actuelle,
soit plus généralement en entrant en relation avec l’\textbf{\textit {action.}} L’ensemble
de nos souvenirs est comparable à un cône dont la base,
assise dans le passé, demeure immobile tandis que son sommet, situé
dans le plan de l’action, avance sans cesse avec celui-ci. Ce sommet
figure le {\it corps}. Car qu'est-ce que le corps, si ce n’est « un centre
d’action », un « trait d’union entre les choses qui agissent sur moi et
les choses sur lesquelles j’agis » ? — C’est ainsi que, dans la vie normale,
{\it les deux mémoires s’entrepénètrent}. La mémoire pure sert de
base à la mémoire-habitude qui n’en est guère que « la pointe mobile
insérée dans le plan mouvant de l’expérience », et inversement le
cerveau et les appareils sensori-moteurs « fournissent aux souvenirs
inconscients, le moyen de devenir présents ».

{\it Discussion}. Bergson a eu le mérite de distinguer la {\it mémoire} de
l'{\it habitude}, le passé {\it représenté} du passé {\it agi}, ce que n'avait pas fait
Ribot. Mais, comme l’ont remarqué plusieurs auteurs (Janet,
Lacombe), destinée à servir une conception métaphysique, celle
d'après laquelle le corps est un {\it instrument} au service de l'esprit (voir,
\S 351 E), sa théorie relève davantage de la spéculation philosophique
que de la Psychologie, — 1° D’abord Bergson {\it exagère la différence}
entre les deux formes de la mémoire, alors que, comme le suggère
Janet, il s’agit peut-être de deux \textbf{\textit {niveaux}} distincts (chap. III). Des
deux actes de réciter une poésie et de raconter en quelles circonstances
on l’a connue pour la première fois, « l’un est un acte très simple de
récitation, l’autre est un acte mémoriel plus élevé de description et
de narration, et il ne faut pas être surpris de trouver des différences
%191
psychologiques entre ces deux actions ». Ces différences sont-elles
d’ailleurs aussi absolues qu’on le prétend ?

\vspace{0.24cm}
{\footnotesize 
Bien souvent, le souvenir représentatif se fixe lui aussi {\it par répétition} :
un dessinateur qui veut reproduire de mémoire un visage, n'est-il jamais
obligé de le regarder longuement ou à plusieurs reprises, de tenter des
esquisses successives, etc. ? Ces souvenirs-images sont-ils en outre tellement
{\it originaux} ? « La plupart des images que nous utilisons sont des images
banales, obtenues par fusion de souvenirs singuliers : ainsi notre maison,
les rues que nous traversons chaque matin... » (Lacombe).}
\vspace{0.31cm}

2° Le souvenir pur est-il autre chose qu’une \textbf{\textit {hypothèse}} purement
gratuite ? Dès qu’il se révèle à la conscience, il implique, ainsi que le
remarque Janet, « des {\it mouvements} tout aussi bien que le souvenir
moteur ». Une seule expérience permettrait une vérification approchée :
celle du {\it rêve}. Si la théorie bergsonienne était vraie, nous devrions
revivre en rêve des scènes entières de notre passé. Or une enquête
poursuivie à ce sujet par M. Halbwachs auprès de plusieurs personnes
(parmi lesquelles Bergson lui-même) a donné des {\it résultats
négatifs} : les images du passé n’apparaissent en rêve que par fragments
détachés, et « jamais un événement accompagné de toutes
ses particularités et sans mélange d’éléments étrangers, jamais une
scène complète d’autrefois ne reparaît aux yeux de la conscience
pendant le sommeil
{\scriptsize (On a invoqué aussi les « images-éclairs » dans lesquelles, à l’état de veille, une
scène du passé semble se présenter tout à coup à notre esprit (par exemple : les fameuses
réminiscences de Marcel Proust, ou certaines images d'enfance). Mais le plus souvent,
il s’agit de visions très fugitives et très vagues, de paysages par exemple, qui sont comme
« des allusions à des chapitres entiers de notre expérience » et dont l'exactitude
matérielle est « très faible » (Ignace Meyerson).)}
». Comme le dit J. Nogué, « nous ne revivons
pas notre passé dans le rêve, mais avec les données de ce passé nous
nous rebâtissons un {\it présent} nouveau, ce qui est tout différent ».

3° Bien plus aventurée encore est l’hypothèse selon laquelle tout
ce que nous avons vécu, se conserverait dans l’inconscient
{\scriptsize (L'idée se trouve déjà dans Th. de Quincey, {\it Les Confessions d'un Anglais mangeur
d'opium} (1821) : « Il est une chose dont je suis certain : en fait, l'oubli n’existe pas. »)}.
Sans
doute, certains faits montrent que des souvenirs qu’on croyait oubliés
à tout jamais sont capables de se réveiller dans certaines circonstances.
Tel est le cas célèbre de la {\it vision panoramique} des asphyxiés
{\scriptsize (Les gens qui ont failli périr par asphyxie (noyade, étouffement) racontent que,
tout près de la mort, ils ont revu toute leur vie « comme en un tableau ». L'exemple est
aussi cité par de Quincey.)}.
Mais il
est probable qu’en pareil cas, comme dans le rêve (cf.t. II, ch. VII),
il y a « illusion rétrospective du sujet revenu à l’existence normale »
(Lacombe) et non pas résurrection intégrale du passé. Et surtout,
conclure de là que {\it tout} notre passé se conserve, c’est commettre « une
%192
extrapolation, une exagération {\it ad infinitum} » (Janet) de quelques
faits assez limités : Janet pense au contraire que « nous ne conservons
dans la mémoire qu’un nombre infime des événements de notre vie ».

4° Par là, la théorie bergsonienne, loin d’être nouvelle, s'apparente
aux vieilles théories qui font de la mémoire une \textbf{\textit {conservation}} (et
non une reconstruction). Avec elles, Bergson admet que le souvenir
se conserve en nous {\it tel quel}. Il n’est donc pas exagéré de dire avec
J.-P. Sartre : « C’est l’image de Taine qui est passée tout entière,
sans contrôle, comme une acquisition incontestable de la science,
dans la métaphysique bergsonienne. » Résultat paradoxal : cette doctrine
qui à tant insisté sur le vivant et le mouvant, en vient ainsi à
méconnaître la « vie » du souvenir et à le réduire à une image subsistant,
{\it immuable}, non plus dans le cerveau, mais dans les profondeurs
de l’inconscient. « Mais, en réalité, cette image immuable n’existe pas
pour l’observation psychologique. Ce que celle-ci nous montre, c’est
une image se transformant à chaque répétition » (Lacombe). On a
vu en effet, par les déformations du souvenir (\S 125 B), combien
l'image primitive peut se modifier.

5° En même temps, se trouve méconnue l’\textbf{\textit {activité propre de l'esprit}}
dans les différentes fonctions de la mémoire.

\vspace{0.24cm}
{\footnotesize 
Nous avons déjà remarqué qu'il est loin d’être toujours vrai que le souvenir
se {\it fixe} « de lui-même ». — Quant au {\it rappel}, c’est, selon Bergson,
le cerveau, en tant qu’il dirige notre action, qui en est l'instrument. Combien
mystérieuse d’ailleurs est cette fonction du cerveau qui le rend capable de
faire d'un « souvenir pur » un état conscient ! Bergson va jusqu’à dire
qu’ainsi les souvenirs purs « ne se matérialisent que par hasard », profitant
pour ainsi dire d’une attitude du corps : « N'est-ce point ignorer que le rappel
des moments singuliers de notre histoire est souvent le produit d’une
recherche volontaire, d’un effort positif, et non d’un accident ? » (Lacombe).
Bergson à essayé ailleurs de donner une idée plus précise de l'effort de rappel
en le ramenant au développement d’un « schéma dynamique » dont les éléments
s’interpénètrent, en une représentation {\it imagée} dont les éléments
sont distincts. Mais nous avons vu qu’il y a dans le rappel volontaire
(\S 138 B) tout autre chose qu’un déploiement d'images. — C’est encore à
une « projection excentrique de souvenirs-images » vers la perception que
Bergson fait appel pour expliquer la {\it reconnaissance} du présent, dans la
mesure où cette reconnaissance « attentive » se distingue de la reconnaissance
motrice. Quant au passé, nous avons vu (\S 126 B) que, d’après lui,
l'image en est, par elle-même, datée. N’avons-nous pas constaté au contraire
qu’il y a en ce cas tout un travail qui aboutit à situer l'image dans un
{\it système de rapports}, qui lui confère son allure de passé et parfois une date ?}
\vspace{0.31cm}

L'erreur fondamentale de Bercson a été de méconnaître ce caractère
\textbf{\textit {intellectuel}} du souvenir et de situer celui-ci à un niveau de la
vie de l’esprit, celui du {\it rêve}, où il n’y a pas de mémoire du tout (\S121).
%193
« L'opération de la mémoire, écrit Halbwachs, suppose une activité
rationnelle de l’esprit dont celui-ci est bien incapable pendant le
sommeil. » Bergson a ainsi confondu la mémoire avec l’{\it activité conservatrice},
simple {\it restitution du passé} « {\it ad integrum} », comme dit Janet,
une des formes les plus inférieures de l’activité de l’esprit.

\section{Mémoire et vie sociale}% 131.
Ce qui caractérise les théories
précédentes, c’est qu’elles ont prétendu expliquer la mémoire à l’aide
de facteurs purement {\it individuels}. Or :

A. Nous avons remarqué, avec P. Janet, que le plus souvent, si
nous fixons un souvenir (\S 124), c’est {\it pour le raconter à autrui}. Partant
de là, Janet a conclu que « la mémoire est un acte social »
qui a pour but de lutter contre l'{\it absence}, et cet acte est essentiellement
celui du récit. Qu’on se rappelle l’exemple, cité par Janet,
de la sentinelle d’un camp de primitifs : pourquoi fixe-t-elle tant
de choses dans son esprit ? Pour pouvoir ensuite les {\it raconter} à
son chef et faire en sorte que celui-ci agisse comme s’il avait été
présent.

B. Mais il ne faut pas considérer seulement les rapports {\it inter-individuels}.
Il faut faire intervenir le point de vue proprement {\it sociologique}
(\S 34), c’est-à-dire celui des \textbf{\textit {groupes sociaux}} dont nous faisons
partie. Or : 1° ces groupes nous imposent, en certaines circonstances,
l'{\it obligation} de nous souvenir : « Le commerçant infidèle à ses engagements,
l’amant oublieux n’encourent pas seulement les reproches
de ceux qu’ils ont abusés, mais encore le blâme de tout leur groupe
social, qui a posé comme vertus la probité et la constance » (J. Nogué).
— 2° La connaissance de notre passé est faite à la fois de nos souvenirs
personnels et de ce que Ch. Blondel a appelé « des savoirs » ; et ces
{\it savoirs}, en se mêlant aux premiers, finissent par faire figure eux mêmes
de souvenirs. On en a vu au \S 125 un exemple chez Stendhal.
Or ces savoirs sont faits de tout ce qui intéresse notre famille, nos
relations, notre pays, notre {\it groupe}, de tout ce que nous y avons
entendu répéter, et aussi de tout ce que nous savons {\it logiquement} avoir
dû s’y produire. Ainsi, je n’ai peut-être aucune image directe du jour où
je suis allé au lycée pour la première fois, mais je {\it sais} qu il a dû y
avoir une première fois, et je me la représente en fonction de ma vie
ultérieure au lycée. Nous n’avons aucun souvenir de notre naissance,
et pourtant sa date est celle que nous nous « rappelons » le mieux,
parce que la vie sociale nous oblige à chaque instant à la {\it savoir}. —
3° Nous avons déjà noté (\S 127) que nous {\it datons} nos souvenirs à
l’aide de {\it points de repère} sociaux « qui n’ont de sens que par rapport
aux groupes dont nous faisons partie », d'événements qui jalonnent
%194
le cours de la vie familiale, nationale, religieuse (exemple : le dimanche
des Rameaux, dans l’exemple de Taine, cité \S 127)

\vspace{0.24cm}
{\footnotesize 
Certains de ces événements majeurs forment le point de départ de nos
calendriers : la fondation de Rome pour les anciens Latins, la naissance du
Christ pour les Chrétiens, l’hégire pour les Musulmans, etc. Même les événements
de notre vie intime se situent ainsi par rapport aux actes ou aux
événements de la vie collective. Nous disons : « C'était {\it avant} ou {\it après} mon
baccalauréat, {\it avant} ou {\it après} mon mariage, {\it avant} ou {\it après} la guerre », etc.
Stendhal remarque qu’à partir de son arrivée à Paris en 1799, sa vie étant
mêlée « avec les événements de la gazette ». toutes les dates sont sûres.
Je me rappelle, dit Ch. Blondel, que je suis allé au Concours général de
physique un lundi de l'été 1894, et cela parce que c'était le lendemain de
l’assassinat du président Carnot.}
\vspace{0.31cm}

4° Enfin le souvenir, à la différence de l’image de rêve, possède une
{\it consistance}, une {\it continuité}, une {\it objectivité} aussi qui l’oppose à la fois
« à la pleine extériorité de la perception et au capricieux arbitraire
de l’imagination pure ». Le sociologue M. Halbwachs a montré qu'il
tient ces caractères des « cadres sociaux » à l’intérieur desquels il s’organise.
« Pour se souvenir, il faut se sentir en rapports avec une société
d'hommes qui peut garantir la fidélité de notre mémoire... Un homme
qui se souvient seul de ce dont les autres ne se souviennent pas, ressemble
à quelqu’un qui voit ce que les autres ne voient pas : c’est, à
certains égards, un halluciné. » Ce sont ces \textbf{\textit {cadres sociaux}} qui nous
permettent de {\it reconstruire} le souvenir ; car, selon Halbwachs, le
passé ne se conserve pas dans l'individu, mais bien dans la \textbf{\textit {mémoire
collective.}} Chaque groupe, religieux, politique, familial, professionnel
même, a sa « mémoire » propre, ses traditions, ses souvenirs qui se
perpétuent dans le {\it langage}, — car « le langage, dit Lavelle, est la
mémoire de l’humanité », — qui se revivifient dans les fêtes, les cérémonies,
les « commémorations ». C’est en fonction de ces différentes
« mémoires » que chacun de nous reconstruit ses souvenirs propres :
nos souvenirs sont des souvenirs d'étudiant, de soldat, d’époux, de
père de famille, de commerçant, de médecin, etc.

{\it Discussion}. Il est incontestable que la mémoire normale est, en
grande partie, {\it sociale}. Toutefois il ne faudrait pas exagérer ce point
de vue. P. Janet a été jusqu’à écrire : « Un homme seul n’a pas de
mémoire et n’en a pas besoin », — ce qui paraît excessif; car un
Robinson isolé dans son île aurait encore besoin de pouvoir se remémorer
l'endroit où il a trouvé telle plante, celui où il a déposé tel
instrument, etc. La mémoire répond à des nécessités {\it vitales} en même
temps que {\it sociales}. — Halbwachs va jusqu’à mettre en doute « la
possibilité d’une mémoire strictement individuelle ». Ici il faut s’entendre :
%195
que notre mémoire ne se {\it structure}, ne s'{\it organise} que dans des
cadres temporels et rationnels et ne devienne ainsi mémoire proprement
dite que grâce aux facteurs sociaux que nous venons d’examiner,
soit ! Le Pr Deray écrit lui-même que le rêve ou le délire nous
révèlent « ce que serait le jeu naturel de la mémoire libérée des cadres
sociaux ». Il n’en reste pas moins que nos souvenirs ont, comme l’a
reconnu Halbwachs, une « résonance » qui varie d’individu à individu
et que ces cadres ne peuvent expliquer. Il y a des souvenirs que nous
ne conservons que {\it pour nous} : ce sont ceux qui nous sont les plus chers.
Au reste, si grande que soit la part de la reconstruction dans l’évocation
du souvenir, il n’est pas construit {\it ex nihilo}. Le point de départ
de cette reconstruction n’est-il pas nécessairement un résidu sensible,
une {\it image} antérieurement vécue par l'individu ?

\vspace{0.24cm}
{\footnotesize 
Deux cas sont ici possibles. Ou bien cette {\it image} nous est donnée la première
à l’état presque pur : tel est, on le montrera à propos du rêve, le
cas de certaines visions de notre première enfance ; mais ces {\it images d'enfance}
peuvent aussi surgir comme des images-éclairs, à l’état de veille
{\scriptsize (Halbwachs explique ces cas en disant qu'ils se trouvent « au point de croisement
de deux ou plusieurs séries de pensées par lesquelles ils se rattachent à autant de
groupes différents ». Voici un exemple qui nous a été fourni par une de nos élèves et
qui nous paraît difficilement explicable de cette façon : « Quand j'étais encore très petite
(car ma tante me portait encore dans ses bras), j'ai vu, sans doute, une inondation,
dont je n’ai retenu que la vision de l’{\it eau verdâtre qui glissait entre les pavés}. Depuis, j'ai
demandé à maman où nous avons été ainsi. Et je n’ai vu, paraît-il, que la grande
inondation de 1910. J'avais alors deux ans ! »)}
. C'est
alors l’image qui est, en quelque sorte, en {\it quête d'un cadre} (généralement
fourni par le témoignage de la famille) où elle pourra s'organiser et prendre
forme de souvenir consistant. — Ou bien c’est {\it le cadre} qui est donné le
premier. Je sais, par exemple, que j'ai fait des études secondaires ; mais
dans quelle classe, pourrai-je me demander plus tard, les ai-je terminées ?
Le « cadre social » m'offre ici trois possibilités : philosophie, mathématiques
ou sciences expérimentales. Il ne se remplira d’un {\it contenu} déterminé que
grâce à certaines {\it images} que je puis encore évoquer d’un condisciple, d’un
professeur, de tel incident, etc. De même, le cadre « social » m'indique qu'il
y a trois ans, j'ai dû passer les mois d'août et septembre à la campagne
ou à la mer ou à la montagne, parce que c’est la période des vacances.
Mais seule l’image lumineuse de telle plage où je me revois jouant au ballon
avec tels amis au voisinage de tels rochers, m'assurera que c'était à la mer,
à P***, en Bretagne.}
\vspace{0.31cm}

\section{La mémoire, fonction intellectuelle}% 132.
Résumons ce
qui précède. Le souvenir n’est ni l’{\it habitude} ni l’{\it image}, mais il les
suppose l’une et l’autre. En effet, s’il peut être {\it reconstruit} — et l’on
a vu qu’il est beaucoup plus reconstruit que {\it conservé} tel quel
{\scriptsize (Ce point de vue semble aujourd’hui accepté par la plupart des psychologues. « Le
problème de la {\it conservation du passé}, écrit par exemple M. Pradines, a été et semble
être demeuré le problème des problèmes de la mémoire... Mais peut-être que le passé
n'est pas conservé et que la chose ainsi dénommée n'est qu’une reconstruction que nous
en faisons dans le présent... L'immense majorité de nos souvenirs de nous-même n'est
que ca ca pure, à laquelle des lambeaux d'images effilochéces restent confusément
attachées. »)}
 —
%196
c’est sans doute grâce à ces {\it cadres sociaux} qui le structurent ; mais
cette reconstruction serait impossible sans un {\it résidu} de l’action (habitude)
ou de la perception (image) première. Mais, d'autre part, par
la {\it prise de conscience} qu’il implique en tant que {\it connaissance du passé},
il {\it dépasse} à la fois l'habitude, l’image et les cadres où il s’organise.

Nous pouvons donc considérer la mémoire comme une « architecture
fonctionnelle » (Delay) qui met en jeu les \textbf{\textit {trois composantes}}
(\S 25) du psychisme humain. — {\it A.} La notion du « souvenir pur » est
une hypothèse poétique, mais sans fondement sérieux. Le « résidu »
dont nous parlions ci-dessus, ne peut être autre qu’\textbf{\textit {organique.}} Pour
l'habitude, personne ne le conteste. Mais on a vu (\S 129) que, pour
l’image elle-même, les auteurs contemporains sont loin de rejeter la
notion de dispositifs dynamiques « sous-tendant », en quelque sorte,
l’image dans le fonctionnement cérébral. « La psycho-physiologie,
remarquait Halbwachs, a son domaine comme la psychologie sociologique
a le sien.»

{\it B.} D’autre part, il est incontestable que la structuration de la
mémoire lui vient en majeure partie de ces \textbf{\textit {cadres sociaux}} sur lesquels
avait insisté, à si juste titre, M. Halbwachs. Ce fut peut-être
l'erreur la plus profonde de Bergson d’avoir confondu le souvenir
avec cette « pensée de rêve », cet {\it autisme} (\S 41), dont tout au contraire
la mémoire nous libère en nous mettant en rapport avec cette communauté
de traditions et de souvenirs qu’est tout groupe social.

{\it C.} Enfin nous avons eu l’occasion de constater que la plupart des
opérations de la mémoire s’accompagnaient d’actes \textbf{\textit {proprement
intellectuels}} : acte de mémoration, acte de remémoration et surtout
acte de reconnaissance qui enveloppe un véritable {\it jugement}. En ce
sens, la mémoire est une fonction qui se situe à un niveau psychologique
élevé et qui n’existe que chez l’homme : l’animal, dit Janet,
n’a pas de mémoire à proprement parler; il a seulement des habitudes
et fait des « rédintégrations », comme le font certains malades. Chez
l’homme même, « la mémoire du passé n’a rien de primitif ni de
spontané, c’est au contraire une faculté tardive dans l’évolution mentale
de l'individu » (Pradines). Si nous n’avons généralement pas
de souvenirs de nos trois premières années, c’est précisément que
l'enfant « ne sait pas faire l’acte de mémoire » (Janet). Il y a donc
bien là « un acte intellectuel, une construction de l'intelligence » qui
« crée » véritablement le passé en créant le \textbf{\textit {temps,}} non pas la « durée »,
%197
mais le temps structuré et organisé (chap. XI) : « Nous ne savons
pas, dit Janet, si le temps existe ni comment il existe, mais nous
savons que l'intelligence humaine, en créant la mémoire, a essayé de
le représenter et de lutter contre lui. »

\section{L’oubli}% 133.
Qu'est-ce que l'oubli? La même ambiguïté règne
ici dans le langage courant que pour le terme opposé {\it mémoire}. Pas
plus que la mémoire immédiate (\S120), qui est simple {\it continuité} de
la conscience, n’est la vraie mémoire, la {\it disparition} momentanée
d’une représentation {\it du plan de la conscience} n’est oubli. Il faut distinguer,
d’autre part, l’oubli qui est un \textbf{\textit {phénomène normal,}} des
cas {\it pathologiques} (\S 134) :
dans ceux-ci, c’est la {\it fonction}
elle-même qui est atteinte ; ce
sont des pertes {\it de la mémoire} ;
l’oubli est la perte {\it du souvenir}.
C’est pourquoi l’oubli
n’est pas toujours le {\it contraire}
de la mémoire, mais parfois
son {\it complément}. De ce point
de vue, il convient d’en distinguer
deux formes : négative
et fonctionnelle.

{\it A.} L’oubli \textbf{\textit {négatif}} peut
tenir : 1° à une insuffisance
de {\it fixation} : c'est ainsi, on
l’a vu, que s’explique l’oubli
de nos toutes premières années
d’enfance (\S 132 C) ; nous
oublions de même ce qui ne
nous a pas intéressés, ce à
quoi nous n'avons pas fait

Fig. 24. — L'oubli par effacement. 
Il s'agissait de fixer des séries de 50 chiffres.
En abscisse, le nombre de jours. En ordonnées, 
la force du souvenir évaluée selon la méthode 
d'Ebbinghaus. L'acquisition (phase A) a
exigé une répétition par jour pendant dix jours. 
L'effacement (phase E) du souvenir n'apparaît 
que 7 jours après la fin. Au bout de 90 jours,
la persistance est encore de 40 p. 100. 

(D'après Préron, Psychologie expérimentale,
collection Armand Colin.)


attention ; — 2° à une impossibilité temporaire ou définitive de {\it rappel} :
c’est l’oubli par {\it refoulement} (nous n’aimons pas à évoquer nos
fautes, nos erreurs, nos impairs, tous les incidents où nous n’avons
pas eu le beau rôle) ou bien l'oubli par {\it effacement} ; — 3° à une absence
de {\it reconnaissance} : ce sont les simples réminiscences telles que celles
auxquelles fait allusion Descartes (\S 126 B).

\vspace{0.24cm}
{\footnotesize 
On a étudié expérimentalement l'{\it oubli par effacement}. La méthode
d'Ebbinghaus permet de mesurer l'intensité du souvenir latent par le
nombre de répétitions {\it économisées} : si un texte (généralement une suite de
chiffres ou de syllabes quelconques, afin d’éliminer l'influence du sens) a
%198
demandé, la première fois, dix répétitions pour être su et s'il n’en faut plus
de sept, la persistance est de 70 p. 100. On constate, après la {\it phase
d‘acquisition} A, une phase de maturation ; ensuite commence la phase
d’évanouissement E d’abord rapide, puis avec ralentissement progressif.
La courbe ci-contre (fig. 24) qui résume des expériences d'Henri Piéron,
donne la loi du phénomène. En gros, on peut dire que l'oubli croît
proportionnellement à la racine carrée du temps écoulé.}
\vspace{0.31cm}

L’oubli négatif n’a par lui-même aucune fonction : c’est un pur
{\it manque} du souvenir, Encore y a-t-il parfois, dans l’existence humaine,
des choses qu’il vaut mieux oublier (voir Exercice 7).

{\it B.} Il en est tout autrement de l’oubli \textbf{\textit {fonctionnel.}} On a vu que la
fixation et le rappel des souvenirs impliquent un {\it choix}, tantôt spontané,
tantôt volontaire ou à demi volontaire (\S 124). Une « bonne
mémoire » ne consiste pas à tout retenir ni surtout à tout restaurer :
« La mémoire n’est pas une hotte ; il ne s’agit pas de la bourrer ou de
la remplir, mais de faire le triage de ce qu’on y met », et en ce sens,
dans l’art d'apprendre, entre l’{\it art d'oublier} (L. Dugas).

\section{Les troubles de la mémoire}% 134.
La mémoire étant, non
une « faculté » simple, mais un {\it complexus de fonctions}, ses troubles
peuvent atteindre tantôt l’une tantôt l’autre de ces fonctions.

\vspace{0.24cm}
{\footnotesize 
{\it A.} Il y a d’abord des \textsf{\textit {amnésies de fixation}}. Mais il y a lieu de distinguer ici
la fixation de l’image (ou de l'habitude) qui est « un acte biologique élémentaire
commun à l’homme et à l'animal » (Delay), et la {\it mémoration} du
souvenir qui est, comme nous l'avons dit, un acte proprement intellectuel.
Nous aurons donc deux séries de cas : les uns où c’est la base {\it organique}
de la fixation qui fait défaut, les autres où ce sont surtout les conditions de
la mémoration. Au premier cas se rattachent les amnésies {\it congénitales},
celles des dégénérés, idiots ou imbéciles, qui « oublient à mesure », celles
des {\it vieillards} qui ont souvent une bonne mémoire pour le passé ancien,
mais qui n’enregistrent plus rien de nouveau. Mais, dans d’autres cas : amnésies
consécutives à un traumatisme {choc sur la tête, commotion, émotion
violente), syndrome dit « de Korsakoff» qu’on observe notamment dans
l’intoxication alcoolique, il semble que la {\it mémoire immédiate} reste intacte ;
car les images en apparence détruites reparaissent en rêve et, quand la
mémoire se restaure, elle revient dans l’ordre inverse de la loi de Ribot,
c'est-à-dire de l’ancien au récent. Il s’agit alors d’« amnésies de mémoration
ou d'intégration », résultat d’un « déficit de la synthèse mentale » (Delay).
L'exemple classique est celui du malade de Korsakoff (1889), qui racontait
avec beaucoup de relief ses anciens voyages, mais qui, oubliant qu’il
venait de le faire, recommençait son récit dix fois par heure, Dans tous ces
cas, l’amnésie suit une marche \textsf{\textit {antérograde}} : elle s'étend « vers l’avant »,
c'est-à-dire vers l’avenir.

{\it B.} Les \textsf{\textit {amnésies d’évocation}} affectent au contraire la forme
\textsf{\textit {rétrograde}} : elles
s'étendent {\it vers le passé} à partir de la cause qui les a provoquées. Ici les
souvenirs ont été fixés et même parfois « mémorés », mais le rappel en est

%199
impossible dans les conditions normales. On a souvent cité l'exemple de
ce commissionnaire dont parle Taine et qui, ayant égaré un colis alors qu'il
était ivre, ne put retrouver l'adresse où il l’avait porté, que lorsqu'il se
trouva de nouveau en état d'ivresse. En pareil cas, l’amnésie peut être :
1° {\it lacunaire}, c’est-à-dire porter sur une certaine {\it période} de la vie du sujet ;
elle s'ajoute généralement alors à l’amnésie de fixation ; c’est le cas ordinaire
des « absences» des épileptiques ; — 2° {\it électives} ou {\it systématisées},
c’est-à-dire porter sur un ordre d'idées, sur un {\it thème} donné, par exemple
sur tout ce qui se rapporte à un événement, à une personne, etc. La Psychanalyse
a essayé d’expliquer ces amnésies électives comme le résultat d’un
\textsf{\textit {refoulement}}. Nous croyons que cette explication convient en effet dans certains
cas
{\scriptsize (Voici un de ces cas, bien typique, qui est cité par J. Delay. Une jeune femme cultivée
a oublié le nom de l’auteur du {\it Paradis terrestre} et l’attribue à Dante. La psychanalyse
révèle qu’elle a une aversion pour un certain type d'homme blond, qui est celui
d’un de ses cousins qu’elle avait aimé jadis. Ce cousin s'appelait... Milton !)}
. Mais il nous paraît excessif de la généraliser. Un commotionné
de guerre, étudié par G. Dumas, a oublié tout le latin qu'il a appris depuis
l’âge de 12 ans, mais se souvient encore de l'italien, appris à 14 ans et du
peu d’allemand acquis depuis l’âge de 16 ans : où est ici le refoulement
{\scriptsize (On supposera peut-être une antipathie contre le latin ? Les psychanalystes ne
sont jamais à court d'explications de ce genre. Mais le même malade avait aussi une
amnésie élective pour une période historique {\it qu’il avait particulièrement étudiée} : les
débuts de la Révolution française.)}
— 3° {\it progressive} : l’amnésie se généralise alors selon la loi de Ribot, avec
cette réserve cependant que celle-ci « ne vaut qu’à {\it charge affective} et à
{\it répétition} égales des souvenirs » (Delay) ; il est évident notamment que,
pour le vieillard, les souvenirs de jeunesse possèdent une charge affective
qui s'oppose à leur disparition ; un souvenir récent, fréquemment répété,
peut s'évoquer mieux qu’un souvenir ancien. — Outre les causes physiques
(traumatisme, lésions ou dégénérescence cérébrales) ou affectives,
peuvent intervenir ici des causes \textsf{\textit {sociales}} : c’est ce qu'Halbwachs appelle
« l'oubli par détachement d’un groupe » : « Oublier une période de notre vie,
c’est perdre contact avec ceux qui nous entouraient alors », et Halbwachs
rappelle le cas de cette fillette de dix ans environ qu’on découvrit dans les
bois en 1731, qui semblait venir d’outre-Atlantique, mais dont on ne put
jamais savoir l’origine exacte ; car, transportée dans un monde tout nouveau
pour elle, elle avait perdu tout cadre de référence, donc tout souvenir.

C. Les \textsf{\textit {amnésies de reconnaissance}} dépendent tantôt de conditions organiques,
tantôt de conditions plus complexes. Dans le premier cas, ce sont les
{\it agnosies} visuelles, auditives, tactiles, etc. (le sujet ne reconnaît plus les objets,
même usuels) ou bien les {\it apraxies} motrices (le sujet ne sait même plus faire
les gestes). Ces troubles sont liés à des lésions cérébrales siégeant dans les
aires {\it gnosiques} ou {\it praxiques} de l'écorce (\S 29 B). Ainsi « les destructions
de l’aire visuo-gnosique placent le malade dans l'incapacité d'identifier les
objets et les symboles » (Lhermitte). Les conditions sont plus complexes
quand le malade ne reconnaît plus les personnes, même parfois celles de sa
famille, tel ce commotionné qui « ne reconnaissait plus le visage des infirmiers
qui passaient près de lui, même quand il les avait vus dix et vingt fois »
(Dumas). C’est alors un affaiblissement intellectuel général.

D. Enfin on a décrit sous le nom d’\textsf{\textit {hypermnésies}} des troubles qui sont des
surexcitations {\it pathologiques}, non pas de la mémoire, mais de la réviviscence
%200
du passé. On les constate souvent chez les débiles mentaux, tel celui qui « se
rappelait le jour de chaque enterrement fait depuis 35 ans » et qui « pouvait
répéter avec une invariable exactitude le nom et l’âge des décédés » (Ribot)
ou cet autre qui « s'informait auprès de tous ceux qu’il rencontrait de leur
date de naissance, de celle de leurs parents, du nom de jeune fille de leur
femme etc.. et pouvait le répéter plusieurs années après, à tel point qu'on
venait le consulter quand les archives locales étaient en défaut » (R. Williams).
Il s’agit en pareil cas de rédintégrations du passé, et le Pr Delay
y voit avec raison une manifestation de l'\textsf{\textit {autisme.}}}
\vspace{0.31cm}

L’étude des troubles de la mémoire confirme donc les distinctions
que nous avions établies entre la mémoire proprement dite, en tant
que \textbf{\textit {connaissance du passé,}} et : 1° la rédintégration, qui n’est qu’une
manifestation de « l’activité conservatrice » ; 2° la « mémoire » organique,
qui n’est qu'habitude ; 3° la simple réapparition des {\it images}.
Elle nous montre que, si la mémoire a toujours des {\it bases organiques}
plus ou moins faciles à délimiter, elle dépasse cependant de beaucoup
le niveau biologique en tant qu’elle est à la fois {\it sociale} et {\it intellectuelle}.

\vspace{0.24cm}
{\footnotesize 
\section{Sujets de travaux}% SUJETS DE TRAVAUX


Exercices. — 1. {\it Comment faut-il entendre cette pensée de Pascal} : « La
mémoire est nécessaire pour toutes les opérations de la raison [raisonnement] » ?
— 2. {\it Commenter cette affirmation du psychologue van Biervliet} :
« Quant aux corps solides, nous voyons qu'ils retiennent. Cette déformation
extérieure, visible des corps solides, qui va s’accentuant à mesure que le
mouvement modificateur se répète, ne s'explique que par la faculté de
{\it retenir}. » — 3. {\it Étudier les images suggérées par ces deux textes} : {\it a.} « Au crépuscule,
le ciel semble un grand voile jaune dans lequel montent les découpures
des montagnes et des hautes pagodes. C’est l'heure où, en bas, dans
le dédale des petites rues grisâtres, les lampes sacrées commencent à briller,
au fond des maisons toujours ouvertes... tandis qu’au dehors tout s’obscurcit
et que les mille dentelures des vieux toits se dessinent en festons
noirs sur ce ciel d’or clair » (Loti) ; {\it b.} « Toutes les odeurs des bois, l'âcreté
du terreau mouillé sur quoi fermentent les feuilles mortes, les effluves
légères des résines, l’arôme farineux d’un champignon écrasé, tous les murmures,
tous les froissements, toutes les envolées dans les branches, tous
les cris du crépuscule, la crécelle rouillée des coqs faisans, les rappels croisés
des perdrix... et déjà, dans la nuit commençante, ce grincement qui
approche et qui passe à frôler votre tête, avec le vol de la première chevêche
en chasse » (Genevoix). — 4. {\it Après avoir regardé l'étalage d’un magasin
où se trouvent des objets variés, essayez de vous rappeler quels sont ceux que
vous y avez vus. Contrôlez le nombre d'objets retenus et le nombre d'objets
oubliés}. — 5. {\it Étudier les déformations du souvenir d’après la} Vie de Henri
Brulard {\it de Stendhal} (voir Halbwachs, p. 17, p. 63-64). — 6. {\it Expliquer ce
texte de Ravaisson} : « C’est la matérialité, sous la dépendance de laquelle
sont en partie nos sens. qui met en nous l'oubli ; le pur esprit au contraire
qui est tout action, étant par cela même tout unité, toute durée, tout
souvenir, toujours présent à tout et à lui-même, tenant sans se manquer
%201
jamais sous son regard tout ce qu’il est, tout ce qu’il fut, peut-être même,
si l’on ose aller jusqu'où va Leibniz (\S 212 A), tout ce qu'il sera, le pur
esprit voit toutes choses sous forme d’éternité.» {\it Comparer avec la
théorie bergsonienne et discuter}. — 7. {\it Commenter cette pensée de Fr. Mauriac} :
« Que les morts seraient embarrassants s’ils revenaient ! Ils reviennent
quelquefois, ayant gardé de nous une image que nous souhaiterions ardemment
de détruire, pleins de souvenirs que c’est notre passion d'oublier. »
— 8. {\it Et celle-ci de P. Valéry} : « L'homme a inventé le pouvoir des choses
absentes, par quoi il s’est rendu puissant et misérable. »

Exposés oraux. — 1. Les « {\it trois mémoires} » d'après J. Delay, p.16
et p. 7 [l’auteur distingue une mémoire « sensorio-motrice », une mémoire
« sociale » et une mémoire « autistique » ; discuter cette classification]. —2. {\it La
mémoire d'après Marcel Proust} (principales références dans Ch. Blondel,
{\it La Psychographie de M. Proust}, Vrin, 1932, chap. I-III). — 3. {\it Mémoire et
personnalité}. — 4. {\it Des qualités d'une bonne mémoire} [distinguer rétention,
promptitude d’évocation, précision, etc.].

Discussion. — 1. {\it Est-il vrai que l'image n’a qu’une réalité conventionnelle ?}
— 2. {\it La notion de « souvenir pur »}. — 3. {\it Discuter cette affirmation de Remy
de Gourmont} : « L'oubli du passé est une condition de force, d'aptitude au
présent. »

Lectures. — {\it a.} Ribot, {\it Les Maladies de la mémoire}, Alcan, 1881. —
{\it b.} Bergson, {\it Matière et Mémoire}, Alcan, 1898 ; {\it c. L'Énergie spirituelle},
1919 ; et {\it d. La Pensée et le Mouvant}, 1934. — {\it e.} L. Dugas, {\it La Mémoire et
l'oubli}, Flammarion, 1919. — {\it f.} H. Piéron, {\it L'Évolution de la mémoire}, Id.,
1920. — {\it g.} M. Halbwachs, {\it Les Cadres sociaux de la mémoire}, Alcan, 1925 ;
et {\it h.} {\it La Mémoire collective}, P. U. F., 1950 (posth.). — {\it i.} P. Janet, {\it L'Évolution
de la mémoire et la notion de temps}, Chahine, 1928 ; et {\it j.} {\it L'Intelligence
avant le langage}, Flammarion, 1936, 3$^\text{e}$ partie. — {\it k.} I. Meyerson, {\it Les
Images-éclairs}, dans le {\it J. de Psychologie}, juillet 1929, p. 569. — L. R.-E.
Lacombe, {\it La Psychologie bergsonienne}, Alcan, 1933. — {\it m.} J.-P. Sartre,
{\it L'Imagination}, Alcan, 1936; et {\it n.} {\it L'Imaginaire}, Gallimard, 1940. —
0. J. Delay, {\it Les Dissolutions de la mémoire}, et {\it p. Les Maladies de la
mémoire}, P. U. F., 1942, — {\it q.} M. Pradines, {\it Traité de Psychologie générale},
P. U. F., 1943-1946, t. I, p. 199-214, et t. III, p. 50-118. — {\it r.} F. Ellenberger,
{\it Le Mystère de la mémoire}, Mont-Blanc, 1947. — {\it s.} J.-C. Filloux,
{\it La mémoire}, P. U. F., 1949. — {\it t.} G. Gusdorf, {\it Mémoire et personne}, P. U. F.,
2 vol., 1951. — {\it u.} A. Bridoux, {\it Le Souvenir}, P. U. F., 1955. — ». Ét. Souriau,
{\it Le souvenir de l'enfance}, dans le {\it J. de Psychologie}, janv.-juin 1962,
p. 15-57. — {\it w.} Fraisse et Pracer, {\it Traité de Psychologie expérimentale},
fasc. IV : {\it Apprentissage et Mémoire}, par Le Ny, de Montpellier, Oléron
et Florès, P. U. F., 1963. — {\it x.} J. Barbizet, {\it Études sur la mémoire},
Expans. scient. fr., 1964 — {\it y.} J. Barbizet, {\it Pathologie de mémoire},
P. U. F., 1970.}
\vspace{0.31cm}
