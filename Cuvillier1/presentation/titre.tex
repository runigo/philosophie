\begin{titlepage}
%
~\\[1cm]

\textsc{\Large }\\[0.5cm]

% Title \\[0.4cm]

\begin{center}
{\large \bfseries  ARMAND  \\
\Huge CUVILLIER \\}
\end{center}

\HRule
\begin{center}
{\Huge \bfseries La Mémoire}
\end{center}
\begin{center}
{\Large \bfseries COURS DE PHILOSOPHIE
}
\end{center}

\HRule \\[1.5cm]

\begin{center}
%\includegraphics[scale=0.3]{./presentation/ptoleme}
\end{center}

\begin{center}
%\includegraphics[scale=0.3]{./presentation/diagrammesInteractions}
\end{center}


% Author and supervisor
\begin{minipage}{0.4\textwidth}
\begin{flushleft} \large
%\emph{Auteur:}\\
%Stephan \textsc{Runigo}
\end{flushleft}
\end{minipage}
\begin{minipage}{0.4\textwidth}
\begin{flushright} \large
\emph{Numérisation:}\\
Stephan \textsc{Runigo}
\end{flushright}
\end{minipage}

\vfill
Ce document reproduit le chapitre 9 du cours de philosophie d'Armand Cuvillier, le classique de l'initiation philosophique.

Les notes de fin de page sont intégrées au corps du texte, entre parenthèse et en très petit caractère.

Les renvois aux autres paragraphes de l'ouvrage ont été reproduit (symbole \S). Le lecteur de ce document devrait ne pas en tenir compte.
\vfill

% Bottom of the page
{\large \today}

\end{titlepage}
