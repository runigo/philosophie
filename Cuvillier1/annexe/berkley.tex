
%%%%%%%%%%%%%%%%%%%%%
\chapter{Berkley}
%%%%%%%%%%%%%%%%%%%%%

Fig. 23. — La doctrine de Berkley.

Extraite de la traduction des Dialogues d'Hylas et
de Philonoûs par l'abbé de Gua (1750), cette figure
un peu naïve prétend symboliser l'immatérialisme
de Berkeley. Au centre, le mot grec vo: designe
l'âme. Les rayons qui en partent, figurent l'attention
que l'âme prêle aux « idées », c'est-à-dire aux tableaux
perceptifs, tirés ici des « beautés de la nature», que
suscite en elle l'action de l'Être Suprême, représentée
elle-même par un trait émanant d'un triangle,
«symbole de la Divinité». Partout ailleurs s'étend
l'obscurité, c'est-à-dire qu'en dehors de Dieu, de l'âme
et de ses « idées », il n'y a rien.

%%%%%%%%%%%%%%%%%%%%%%%%%%%%%%%%%%%%%%%%%%%%%%%%%%%%%%%%%%%%%%%%%%%%%%%%%%%%%%%%%%%%%
