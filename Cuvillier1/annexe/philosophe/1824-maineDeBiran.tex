
%%%%%%%%%%%%%%%%%%%%%
\chapter{Maine de Biran}
%%%%%%%%%%%%%%%%%%%%%
%Fig. 93.
1766-1824.

Fils d'un medecin réputé pour sa science
et son dévoument, Maine de Biran avait
hérité de son père « une  santé délicate, un
tempérament impressionnable et
mobile à l'excès, soumis à toute les
influences du dehors. De là une sensibilité
extrême qui fit Le tourment de sa
vie » (P. Tisserand). Il avait les traits
fins, la physionomie douce et un peu
féminine, les yeux bleus, le visage pâle
et légèrement amaigri. Des manières
élégantes et un esprit raffiné lui valurent
des succès mondains. Mais il sut aussi
remplir, avec un zèle et une intelligence
remarquables, des fonctions administratives
et politiques. Toutefois son occupation
favorite était la méditation : son
tempérament l'y prédisposait. Vers la
fin de sa vie, son âme délicate, meurtrie
et désemparée, trouva un refuge dans la religion.

%%%%%%%%%%%%%%%%%%%%%%%%%%%%%%%%%%%%%%%%%%%%%%%%%%%%%%%%%%%%%%%%%%%%%%%%%%%%%%%%%%%%%
