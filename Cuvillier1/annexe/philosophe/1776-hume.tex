
%%%%%%%%%%%%%%%%%%%%%
\chapter{Hume}
%%%%%%%%%%%%%%%%%%%%%

David \textsc{Hume}, 1711-1776.
%(D'après C.-N. Cocuin.)

Ambitieux de gloire littéraire, Hume délaissa
le droit, puis le commerce pour la philosophie. D'Angleterre, il vint en France, où
il écrivit son {\it Traité de la nature humaine}.
IL était très connu à Paris et, quoiqu'il
brillât peu dans la conversation, les
grandes dames n'élaient satisfailes que si
elles pouvaient montrer «le gros David »
à leurs réceptions ou dans leurs loges, au
théâtre : « À l'Opéra, dit lord Charlemont,
on voyait souvent sa large face insignifiante entre deux jolis minois. » Ces apparences « insignifiantes » cachaient cependant un philosophe original et profond,
auquel Kant et Comte ont rendu hommage : il voulut « introduire la méthode
expérimentale dans les sujets moraux»
et il fut peut-être le premier à comprendre
l'importance du « problème critique ».

%%%%%%%%%%%%%%%%%%%%%%%%%%%%%%%%%%%%%%%%%%%%%%%%%%%%%%%%%%%%%%%%%%%%%%%%%%%%%%%%%%%%%
