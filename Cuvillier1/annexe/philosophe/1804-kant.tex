
%%%%%%%%%%%%%%%%%%%%%
\chapter{Kant}
%%%%%%%%%%%%%%%%%%%%%

Emmanuel \textsc{Kant}, 1724-1804.

Fils d'un modeste sellier et d'une mère piétiste
qui marqua sa pensée d'une forte empreinte
religieuse, Kant fut toute sa vie de santé
débile. Il s'était soumis à un régime minutieusement réglementé :
il calculait la quantité et la nature de ses aliments, la durée de
son sommeil et de ses promenades. Un seul
jour, il changea son itinéraire habituel :
celui où le courrier de France lui apporta la
nouvelle de la Révolution. « Son front découvert, taillé pour la pensée, était, dit Herder,
le siège d'une gaîté et d'une joie inaltérables ; débordante d'idées, la parole coulait
de ses lèvres ; plaisanterie, esprit, humour
ne lui faisaient jamais défaut. » Sa {\it Critique
de la Raison pure} (1781) marque une date
capitale dans l'histoire de la pensée
Philosophique.

