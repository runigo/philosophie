
%%%%%%%%%%%%%%%%%%%%%
\chapter{Spinoza}
%%%%%%%%%%%%%%%%%%%%%

 

Benoît De Spinoza, 1632-1677.
%629
%Fig. 94. — Benoît DE Spinoza.

Descendant d'une famille juive portugaise,
Spinoza était entré de bonne heure en rapports
avec des sectes chrétiennes des Pays-Bas
où sa famille avait émigré. À 24 ans,
il fut excommunié par la Synagogue
d'Amsterdam. Ami d'hommes puissants et
riches, il refusa toujours places, pensions
et legs et préféra vivre du modeste travail de
polisseur de verres d'optique. Courageux au
point de prétendre afficher, lors de l'assassinat
de son ami Jean de Witt par la populace,
une protestation contre ce crime des
« derniers des barbares », il se montra aussi,
malgré les souffrances de la tuberculose,
fidèle à son idéal du sage, affable, doux et
indulgent, jusqu'au jour même de sa mort.

%%%%%%%%%%%%%%%%%%%%%%%%%%%%%%%%%%%%%%%%%%%%%%%%%%%%%%%%%%%%%%%%%%%%%%%%%%%%%%%%%%%%%
