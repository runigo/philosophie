
%%%%%%%%%%%%%%%%%%%%%
\chapter{Leibniz}
%%%%%%%%%%%%%%%%%%%%%

 

G.-W. \textsc{Leibniz}, 1646-1716.

« Jamais peut-être, écrit Mme L. \textsc{Prenant},
il n'exista de philosophe plus actif, plus
varié, plus mondain, moins susceptible
de symboliser Le sage classique que
G. W. Leibniz.» Esprit singulièrement
Puissant et surloul très souple, il inventa,
en même temps que Newton, le calcul
infinitésimal et s'intéressa à toutes les
sciences de son temps depuis la théologie
et le droit jusqu'à la physique, la linguistique et même l'histoire. Dès vingt ans
il avait formé le projet d'une symbolique
universelle capable de s'appliquer à
toutes les sciences. Très grand métaphysicien, il n'a cependant jamais
synthétisé sa doctrine dans un ouvrage d'ensemble, peut-être parce que son esprit,
sensible à toutes les nuances de la pensée,
répugnait à cette synthèse. Conseiller
aulique du duc de Hanovre et dévoué
à son prince au point de ne reculer devant
rien pour servir ses intéréts, il voulut
avoir aussi une activité diplomatique : il
chercha, sans succès, à persuader à
Louis XIV de conquérir l'Égypte —
pour le détourner de l'Allemagne — et
à Pierre le Grand d'introduire la civilisation occidentale dans ses États.

%%%%%%%%%%%%%%%%%%%%%%%%%%%%%%%%%%%%%%%%%%%%%%%%%%%%%%%%%%%%%%%%%%%%%%%%%%%%%%%%%%%%%
