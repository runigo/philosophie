
%%%%%%%%%%%%%%%%%%%%%
\chapter{Malebranche}
%%%%%%%%%%%%%%%%%%%%%

 

%Fig. 95.
Nicolas Malebranche, 1638-1715.

(Tableau de SANTERRE.)

Malebranche avait été élevé dans un milieu
très religieux : sa mère était parente par
alliance de Mme Acarie, l’introductrice du
Carmel en France. Né faible et mal conformé,
il souffrit par son corps toute sa
vie et ce tempérament dut le prédisposer
aussi à devenir, selon sa propre expression,
un « médilatif » préférant à tout autre genre
de vie la solitude et les entretiens avec le
« Maitre intérieur ». Entré à l'Oratoire en
1660, il se montrait rebelle aux études,
quand il découvrit le Traité de l'Homme
de Descartes : il le lut avec des battements
de cœur. Prêtre très pieux, en même temps
que métaphysicien, savant, psychologue et
moraliste, Malebranche est à la fois un
des plus profonds philosophes et un des
plus grands écrivains français.
%%%%%%%%%%%%%%%%%%%%%%%%%%%%%%%%%%%%%%%%%%%%%%%%%%%%%%%%%%%%%%%%%%%%%%%%%%%%%%%%%%%%%
