
%%%%%%%%%%%%%%%%%%%%%
\chapter{Berkeley}
%%%%%%%%%%%%%%%%%%%%%

(https://www.les-philosophes.fr/auteur-berkeley.html)
Berkeley est un philosophe empiriste irlandais du 18ème siècle (1685-1753). Il enseigne au Trinity College de Dublin et est ordonné prêtre de l’église anglicane. Il est connu pour son ouvrage les Principes de la connaissance humaine dans lequel il développe la doctrine étonnante de l’immatérialisme, mais aussi pour son Essai sur une nouvelle théorie de la vision, un ouvrage d’optique. Il part en Amérique pour fonder un Collège chargé de former des pasteurs anglicans. Il retourne à Londres et finit ses jours à Oxford.


Fig. 23. — La doctrine de Berkley.

Extraite de la traduction des Dialogues d'Hylas et
de Philonoûs par l'abbé de Gua (1750), cette figure
un peu naïve prétend symboliser l'immatérialisme
de Berkeley. Au centre, le mot grec vo: designe
l'âme. Les rayons qui en partent, figurent l'attention
que l'âme prêle aux « idées », c'est-à-dire aux tableaux
perceptifs, tirés ici des « beautés de la nature», que
suscite en elle l'action de l'Être Suprême, représentée
elle-même par un trait émanant d'un triangle,
«symbole de la Divinité». Partout ailleurs s'étend
l'obscurité, c'est-à-dire qu'en dehors de Dieu, de l'âme
et de ses « idées », il n'y a rien.

%%%%%%%%%%%%%%%%%%%%%%%%%%%%%%%%%%%%%%%%%%%%%%%%%%%%%%%%%%%%%%%%%%%%%%%%%%%%%%%%%%%%%
