
	\begin{itemize}[leftmargin=1cm, label=\ding{32}, itemsep=1pt]

% 77
\ib{Fabulation} — \si{Psycho.} {\bf 1.} « Présentation d’un récit imaginaire, souvent vraisemblable, comme étant
réel, sans intention de tromper »
(Piéron). $->$ {\it Dist.} mythomanie* :
« La fabulation n’est qu’une substitution à un déficit de la mémoire »
(Baruk).

— \si{Hist.} {\bf 2.} Procédé qui consiste
à exposer sa pensée sous forme de
mythes$^2$ : « La fabulation platonicienne » (Schuhl).

\ib{Facteur} — \si{Ps. métr.} Élément commun
à plusieurs opérations mentales de
même nature. {\it Cf.} {\it Analyse}$^4$ et {\it G.}

\ib{Factice} — \si{Hist.} {\bf 1.} {\it Chez Descartes} :
« idées factices », celles qui sont
« faites et inventées » par l'imagination. Cf. {\it Adventices}* — {\bf 2.} Souvent, auj. (par imitation de l'all.
{\it faktisch}) qui existe à l’état de
simple fait. Cf. {\it Facticité}$^2$. $->$ En
ce dernier sens, il vaudrait mieux dire {\it factuel}.

\ib{Facticité} — {\bf 1.} Caractère de ce qui est construit par l'esprit ou 
fabriqué par l'art humain : « L'esprit, dans la culture scientifique, n’a
plus peur de la {\it facticité} » (Bachelard); « L’essentielle facticité
donne à l’œil qui contemple ce bronze... » (id.). —
 {\bf 2.} Souvent, auj. (all. : {\it Faktizität}) :
caractère de ce qui existe comme pur fait, sans fondement$^2$, pour un
être-déjà-dans-le-monde : « La facticité du pour-soi » [{\it i. e.} la double
contingence de son existence et de son engagement dans une situation
donnée] (Sartre).

\ib{Faculté} — \si{Méta.} {\bf 1.} Puissance d'agir
({\it habilitas ad agendum} selon la définition des Scolastiques, qui la distinguaient de la propriété$^1$ ou {\it habilitas ad patiendum}).

— \si{Psycho.} {\bf 2.} {\it Facultés de l'âme} : nom donné autref. aux
fonctions psychiques : « Les faits semblables, nous les rapportons à un même
principe que nous appelons faculté et que nous concevons comme une force
de notre esprit » (Royer-Collard).
% 78

\ib{Faible} — \si{Psycho.} Dans la {\it Gestaltpsychologie}, une forme$^4$
est dite faible quand elle est peu structurée,
quand les éléments sont peu intégrés au tout.

\ib{Fait} — \si{Épist.} Toute donnée de l’expérience$^1$ : « Un fait n'est
rien par lui-même, il ne vaut que par l’idée qui s’y rattache » (Cl. Bernard).
$->$ {\it Dist.} événement et phénomène : {\it un événement} est un fait 
considéré avec ses particularités de temps et de lieu ; — fait a un sens
plus général, mais désigne encore une donnée complexe et concrète; on dist. 
qqfs. le « fait brut », {\it i. e.} tel qu'il serait donné dans l'intuition,
et le « fait scientifique » (syn. : phénomène); — un {\it phénomène}$^1$ est
un fait analysé, considéré dans ses éléments abstraits, indépendamment de
toute particularité de temps et de lieu, et avec la notion d’une
répétition possible. — {\it Vérités de fait} : vérités empiriques$^2$ ou
expérimentales$^2$, donc contingentes ({\it opp.}
{\it vérités de raison} qui sont nécessaires$^1$). — {\it Sciences de faits} :
les sciences expérimentales$^2$ ({\it opp.}
{\it sciences rationnelles}). — En fait : voir Droit$^1$ et
{\it Nécessaire}$^2$.

\ib{Famille} — \si{Soc.} {\bf 1.} La société {\it domestique}*, comprenant
tous ceux qui sont parents entre eux. — {\bf 2.} {\it Spéc.}, de nos jours,
la société {\it conjugale} : groupe formé par le père, la mère et les enfants.

— \si{Biol.} {\bf 3.} Groupe morphologique intermédiaire entre le genre$^2$
et l’ordre$^8$ : « La famille des félins ».

\ib{Fanatisme} — \si{Mor.} Attitude qui consiste à pousser jusqu’au bout le
culte exclusif d’une idée : « Le panthéisme est un fanatisme
théologique » (Le Senne). {\it D'où} : intolérance :
% 78 — FAU
« Il n’y a point de fanatisme sans cérémonie » (Alain).

\ib{Fantaisie} [G. {\it phantasia}, imagination].
— \si{Psycho.} {\it Autref.} imagination « Cette fantaisie est une véritable
partie du corps » (Descartes, {\it Reg.}, XII); « Les images qui sont peintes
en la fantaisie » (Port-Royal).

\ib{Fascination} — \si{Ps. path.} État dans
lequel la conscience$^1$ est complètement absorbée par une perception.

\ib{Fatal} —— \si{Méta.} Qui se produit inévitablement, malgré tout effort contraire de la volonté et de l’intelligence humaines.

\ib{Fatalisme} — \si{Méta.} {\bf 1.} {\it Lato.} Autre.
syn. de {\it déterminisme}$^3$. — {\bf 2.} {\it Str.}
\fsb{S. norma.} Doctrine philosophique selon laquelle tout est fatal*.
$->$ Impropre au sens 1. {\it Dist.} déterminisme : « La
liaison des causes et des effets, bien
loin de causer une fatalité insupportable, fournit plutôt un moyen de
la lever » (Leibniz, {\it Théod.}, 55).

\ib{Fatum} [mot latin]. — \si{Hist.} Dans la
{\it Théod.} (préface), Leïbniz distingue : 1° le {\it fatum mahumetanum},
fatalisme$^2$ absolu, aboutissant à l’argument paresseux*; — 2° le {\it fatum
stoïcum} où « tranquillité à l'égard des événements par la
considération de la nécessité qui
rend nos soucis et nos chagrins inutiles »; — 3° le {\it fatum christianum},
résignation confiante aux décrets de
la Providence.

\ib{Faustien} — \si{Hist.} Qualificatif appliqué par O. Spengler à la
civilisation occidentale et {\it spéc.} germanique, fondée sur le devenir*
et l’aspiration à l'infini. {\it Cf.} {\it apollinien}*.

\ib{Faute} — \si{Log.}, \si{Mor.}, \si{Esth.} ({\it Péj.}) Action
d’enfreindre une norme* : « Une faute logique »; « Une faute morale »
(cf. {\it Péché}*); « Une faute de goût ».
% 79

\ib{Fechner (Loi de)} — \si{Psycho.} (Syn. :
{\it loi psychophysique} ou {\it logarithmique}).
Loi selon laquelle l’intensité de la
sensation* varie comme le logarithme de l’excitant*.

\ib{Feed-back} — \si{Techn.} Dans le lang. de
la {\it cybernétique}* : action en retour
qui permet, dans les machines, des
régulations de fonctionnement analogues à celles qu’on observe dans
les organismes vivants.

\ib{Fétichisme} — {\bf 1.} Culte des {\it fétiches},
{\it i. e.} d’objets matériels qui sont
censés posséder des pouvoirs magiques. {\it Chez Comte} : première forme
de « l’état théologique$^2$ » qui consiste
à « attribuer à tous les corps extérieurs une vie analogue à la nôtre,
mais plus énergique » ({\it p. e.} divinisation des astres). — {\it D'où} ext. :
{\bf 2.} Vénération superstitieuse {\it a)} {\it Chez Marx}, « fétichisme de la
marchandise » : illusion qui confère à la marchandise un caractère
« mystique » en lui attribuant une valeur immanente, alors que cette valeur
n’appartient qu'au travail humain qui la produit; — {\it b)} {\it Chez Renouvier},
« fétichisme en philosophie » : façon de penser qui consiste à se forger
des idoles, telles qu’{\it idée en soi, esprit pur, formes$^1$ substantielles,
archées}*, etc.

\ib{Fiat} [mot latin = que cela soit fait]. — \si{Psycho.} Terme employé
{\it not.} par W. James pour caractériser la décision* volontaire.

\ib{Fictif} — \si{Psycho.} {\bf 1.} (Ctr. : réel). Qui est un produit de la
fiction, {\it i. e.} de l’imagination$^4$. — {\bf 2.} Qui repose sur
l'imagination$^4$ : « L'état théologique$^2$ ou fictif » (Comte).

\ib{Fidéisme} — \fsb{S. norma.} \si{Théol.} {\bf 1.} {\it Str.} Doctrine
selon laquelle la religion est objet
de pure {\it foi}$^5$, sans que l'esprit puisse
% 79
apporter à celle-ci aucun préambule rationnel : « Le pur fidéisme
n’est pas moins contraire à l'orthodoxie religieuse qu'à la raison » (Le
Roy). — \si{Crit.} {\bf 2.} {\it Lato} (Ctr. : {\it rationalisme}$^2$).
Toute doctrine qui met la croyance$^2$ au-dessus de la raison
et fait reposer la vérité sur des exigences sentimentales ou morales.

\ib{Figure} — \si{Math.} {\bf 1.} Étendue$^1$ limitée par des lignes : « La
matière ou l'étendue renferme en elle deux propriétés : la première est celle
de recevoir différentes figures » (Malebranche, R. V., I, 1). — \si{Psycho.}
{\bf 2.} Dans la {\it Gestaltpsychologie}, {\it opp.} à {\it Fond}*, — {\bf 3.}
Façon de parler symbolique : « Une figure de rhétorique ». — {\bf 4.} {\it
Ext.} \si{Théol.} Symbole d’une vérité de foi : « Les prophètes prophétisaient
par figures » (Pascal).

— \si{Log.} \si{form.} {\bf 5.} {\it Figure d’un syllogisme} : forme qu'il
prend selon la place du moyen$^2$ terme dans les deux prémisses
(v. {\it Précis}, Ph, II, p. 34).

\ib{Fin} — \si{Épist.} et \si{Méta.} {\bf 1.} ({\it Opp.} : {\it moyen}$^1$).
Ce pour quoi une chose est ou se fait ; ce vers quoi elle tend consciemment
ou inconsciemment : « Vous ne détournerez nul être de sa fin » (La
Fontaine) ; « Il est, dans la nature, des fins que la raison ne saurait
méconnaître » (Ch. Bonnet), — {\bf 2.} But*, ce vers quoi tend un
acte consciemment et intentionnellement$^1$ : « Ce qui est désiré pour
l'amour de soi-même et à cause de
sa propre bonté s’appelle fin » (Bossuet); « Nous sommes sujets à nous
abuser quand nous voulons déterminer les fins ou conseils de Dieu »
(Leibniz, {\it Disc. méta.}, XIX). — \si{Vulg.} 3 Terme, cessation :
« Toutes les choses de ce monde prennent fin » (Sévigné).
%80

\ib{Final} — \si{Épist.} et \si{Méta.} {\bf 1.} Qui se rapporte à une fin$^1$.
{\it Cause finale} ({\it opp.} : {\it efficiente}*) : la fin$^1$ elle-même,
considérée comme la raison d’être de la chose (cf. {\it Cause}$^4$) : « La
connaissance des causes finales n’est pas nécessaire dans la physique »
(Malebranche, {\it Méditations chrétiennes}, XI) ; « La voie des causes
efficientes est plus profonde, mais la voie des finales est plus aisée »
(Leibniz, {\it Disc. méta.}, 22) ; « La science reste défavorable aux causes
finales que nulle part elle ne discerne avec évidence » (Le Roy).
{\it Argument des causes finales} (syn. : {\it physico-téléologique}*) :
celui d’après lequel les faits naturels, « disposés avec ordre, intelligence,
prévoyance pour les besoins et le bien de chaque être », prouvent
« l'existence d’une cause intelligente et souverainement bonne » (Franck).

— \si{Vulg.} 2 ({\it Opp.} : {\it initial}). Terminal, dernier.

\ib{Finalisme} — \si{Méta.} \fsb{S. norma.} Doctrine qui
admet dans l’univers : {\bf 1.} une finalité$^1$ fondée sur l'argument des
causes finales$^1$ (syn. : {\it providentialisme}*); — {\bf 2.} de la
finalité* en un sens quelconque : « Le finalisme renverse l’ordre naturel :
il explique le présent par l'avenir » (Goblot).

\ib{Finalité} — \si{Épist.} et \si{Méta.} {\bf 1.} Caractère de ce qui tend
vers une fin$^2$ de façon consciente : « La volonté est toujours
indissolublement finalité par les desseins, mécanicité par les moyens
 » (Pradines), — {\bf 2.} Adaptation des moyens$^1$ à une fin$^1$, des
organes aux besoins, soit par une activité intelligente, soit par une
« finalité sans intelligence » (Goblot) : « Nier la finalité organique, c'est
le plus audacieux des paradoxes » (id.) ; « La finalité du plaisir et de la
douleur ».
% 80
— {\bf 3.} Dépendance des parties ou éléments à l'égard d’un tout :
« La finalité n’est pas la conformité à l’idée : elle est l'idée » (Hamelin).
— {\bf 4.} {\it Finalité externe} : celle où la fin est extérieure à l'être
considéré. {\it F. interne} : celle où la fin est l'être lui-même (sens 3).
— {\bf 5.} {\it Principe de finalité} : « La nature ne fait rien en
vain » (Aristote) ou « Tout être à une fin$^1$. »

\ib{Finesse (Esprit de)} — \si{Car.} {\it Chez Pascal} (opp. :
{\it esprit de géométrie}) ; « souplesse de pensée » qui donne l'intuition de
la complexité des choses et le sentiment des rapports qui les unissent.

\ib{Fini} — \si{Méta.} Qui a une limite ou une mesure : « J’ai premièrement en
moi la notion de l'infini que du fini» (Descartes, {\it Méd.}, III).

\ib{Finitude} — \si{Méta.} Caractère fini*, {\it spéc.} de l'être humain :
« La finitude du {\it Dasein}* réside dans l’état d’oubli » (Heidegger).

\ib{Fixation} — \si{Psycho.} {\bf 1.} Fonction de la
mémoire$^4$ par laquelle l’esprit assimile les souvenirs : « La force de
fixation d’un souvenir dépend de l'intensité du fait mental primitif »
(Piéron). — \si{Ps. an.} {\bf 2.} {\it Chez Freud} :
le fait que la libido$^1$ reste attachée à un objet d’un stade ancien.

\ib{Fixisme} — \si{Biol.} (Opp. : {\it transformisme}*). \fsb{S. norma.}
Théorie de la fixité des espèces, selon laquelle les espèces$^3$ vivantes
demeurent immuables et sont nettement distinctes les unes des autres.

\ib{Fluctuation} — \si{Biol.} Variation qui se produit chez un être vivant
sous l’action du milieu. $->$ {\it Dist.} mutation*.

\ib{Foi} — \si{Vulg.} {\bf 1.} Garantie : « Sur la
foi des traités ». — {\bf 2.} Fidélité à un
%81
engagement : « C’est parce que nous sommes civilisés que nous nous
imposons le respect de la foi que nous avons jurée » (Davy). {\it Bonne
foi} : sincérité. {\it Mauvaise foi} : duplicité$^2$ ; {\it spéc.},
{\it chez Sartre} : attitude de la conscience qui se masque à elle-même la
vérité, mensonge à soi-même. — {\bf 3.} Confiance : « Quoiqu’à leur nation
[les voleurs] bien peu de foi soit due,... » (Molière).
— {\bf 4.} Syn. de {\it croyance}$^2$ : « Ajouter foi à... ». {\it Chez
Kant} : « foi morale », croyance rationnelle, quoique non
démontrable, à la liberté, à l’existence de Dieu et à la vie future.

— \si{Théol.} {\bf 5.} \fsb{S. abstr.} Adhésion aux
dogmes d’une Église, à des vérités considérées comme révélées : « La
foi est différente de la preuve : l’une est humaine, l’autre est un don de
Dieu » (Pascal, 248). — {\bf 6.} \fsb{S. concr.} Objet
de la foi$^5$, les dogmes : « Je ne croirai jamais que la vraie philosophie
soit opposée à la foi » (Malebranche, {\it Entr.}, VI, 2).

\ib{Fonction} — \si{Math.} {\bf 1.} Une variable
y est fonction d’une variable $x$ quand
à toute valeur déterminée de $x$ correspondent une ou plusieurs valeurs
déterminées de $y$ : on écrit $y$ = f($x$).
\si{Log.} Le sens mathématique de ce mot a été étendu par la logistique*
aux expressions logiques : « Fonction logique, propositionnelle, conceptuelle. » Cf. {\it Sujet}$^2$.

— \si{Biol.} {\bf 2.} Ensemble d'opérations
par lesquelles se manifeste la vie
d'une cellule, d'un tissu, d’un organe, d’un être vivant (nutrition,
relation, reproduction). — {\it Ext.}
\si{Psycho.}, \si{Soc.} {\bf 3.} Ensemble d’opérations par lesquelles se
manifeste la vie mentale ou sociale : « Vouloir,
juger ne sont que différentes fonctions de notre entendement » (Voltaire); « La fonction fabulatrice* »
% 81
(Bergson) ; « La fonction contractuelle » (Davy).

\ib{Fonctionnel} — \si{Psycho.} et \si{Péd.} « Le
point de vue {\it fonctionnel} est celui du rôle joué par tel ou tel processus
dans la vie de l'individu » (Claparède). — {\it Éducation fonctionnelle} :
« celle qui prend le besoin de l'enfant comme levier de l’activité qu'on
désire éveiller chez lui » (id.).

\ib{Fond} — \si{Psycho.} Dans la {\it Gestalttheorie} : partie non structurée,
amorphe, du champ$^2$, sur laquelle se détache une figure$^2$
(voir {\it Précis}, Ph. I, p. 115).

\ib{Fondement} — {\bf 1.} Principe sur lequel repose {\it en fait} un ordre
de phénomènes. — {\bf 2.} Principe sur lequel repose {\it en droit}* un
système d’assertions ou de règles, {\it i. e.} qui les rend {\it légitimes}
du point de vue logique, moral ou juridique. — {\it Cf.} {\it Base}$^1$.

\ib{Force} — \si{Vulg.} {\bf 1.} (Souvent opp. {\it droit}$^1$). Contrainte :
« Céder à la force ». — {\bf 2.} Puissance d'action : « Les forces morales ».
{\it Idée-force} (Fouillée) : la représentation$^1$ considérée comme poussant
à l’action$^2$. — {\bf 3.} Agent$^1$ physique : « Les forces naturelles ».

— \si{Math.} {\bf 4.} En mécanique : tout ce qui est capable de modifier l’état de
repos ou de mouvement d’un corps (cf. {\it Inertie}$^2$). — {\bf 5.}
{\it Force vive} (syn. : énergie actuelle) : demi-produit de la masse par le
carré de la vitesse (1/2 mv$^2$).

— \si{Méta.} {\bf 6.} L’énergie*, considérée comme le principe indéfinissable
qui produit les phénomènes de l’univers : « Force et matière » (Büchner) ;
« La persistance de la force » (Spencer). — {\bf 7.} {\it Chez Leibniz} :
voir {\it Substance}$^1$.

\ib{Formalisme} — \si{Épist.} {\bf 1.} Tendance à
faire passer le point de vue de la
% 82
forme$^3$ avant celui du fond ou de la matière$^3$.

— \si{Mor.} {\bf 2.} {\it Formalisme kantien} : selon Kant, nos actions sont
morales en tant que déterminées par l’élément formel$^2$ qui fait la volonté
raisonnable ({\it i. e.} par l’idée d’une règle universelle), et non par
leurs fins matérielles$^1$.

— \si{Jur.} {\bf 3.} {\it Formalisme juridique} : système juridique fondé sur
le respect des formes$^5$ : « Le formalisme répond à un impératif de la vie
sociale » (H. Lévy-Bruhl).

\ib{Forme} — ({\it Gén.} {\it opp.} matière). \si{Hist.} {\bf 1.} a) {\it Chez
Aristote} : « forme ou cause formelle », ce vers quoi tend tout changement ;
elle est à la fois l’acte$^2$, l'essence*, la perfection et le principe
d'unité de chaque être : « Un des principes d’Aristote est que la forme est
un être distinct de la matière$^1$ » (Buffon). — b) {\it Chez les
Scolastiques} : « forme substantielle », principe substantiel d’un être
individuel défini par son essence spécifique : « Il n’y a du plus et du moins
qu'entre les accidents$^1$ et non point entre les {\it formes} ou
{\it natures} des individus d’une même espèce » (Descartes, {\it Méth.}, I) :
« Dieu est l’acte$^1$ des actes et la forme des formes » (Bossuet) ;
« L'opinion des formes substantielles a qqc. de solide » (Leibniz,
{\it Disc. méta.}, 10). — {\bf 2.} {\it Chez Kant} [all. : Form] : « formes »
de la connaissance, lois que la pensée impose à la {\it matière}$^2$ de
celle-ci, {\it i. e.} au donné pur de la sensation : les « formes pures a
priori de la sensibilité » sont le temps et l’espace ; les « formes » de
l’entendement sont les catégories*; de même, la « forme » de la loi morale
est son caractère impératif, catégorique* et universel.

— \si{Log.} {\bf 3.} {\it Forme d'un jugement} ou {\it d'un raisonnement} : ensemble des
% 82
relations existant entre les termes auxquels ils s'appliquent, abstraction faite de leur matière$^3$ ou contenu.

— \si{Psycho.} {\bf 4.} [{\it Trad.} all. {\it Gestalt}). Ensemble structuré.
{\it Théorie de la forme} [all. : {\it Gestalttheorie}] : théorie selon
laquelle l'esprit, {\it p. e.} dans la perception, saisit d’abord de telles
formes$^4$, et non des éléments. Cf. {\it Précis}, Ph. I, p. 115.

— \si{Jur.} {\bf 5.} Façon de procéder selon certaines règles, formalités ou
formules : « Ils [les juges] n’ont la liberté de juger que selon les formes
qui leur sont prescrites » (Pascal, {\it Prov.}, 14).

\ib{Formel} — \si{Hist.} Qui concerne la forme* : {\bf 1.} au sens
{\it scolastique}. D'où : essentiel : « Le {\it formel} de la concupiscence,
non plus que le {\it formel} du péché, n'est rien de réel » [= c’est qqc. de
négatif, une perte] (Malebranche, {\it Ecl.}, I). {\it Chez Descartes}, la
réalité « formelle » d’une idée est celle qu’elle a dans la chose elle-même
indépendamment de la représentation que nous en avons (voir {\it Éminent}*) :
« Cette réalité que les philosophes appellent actuelle$^2$ ou formelle »
({\it Méd.}, III); — {\bf 2.} au sens {\it Kantien} : « morale purement
formelle » ; — {\bf 3.} au sens {\it logique} : la {\it validité} formelle
(opp. : {\it vérité matérielle}) d’une proposition est celle qui relève des
« conditions de l’usage de l’entendement en gén., sans distinction
d'objets » (Kant) : elle se ramène à la non-contradiction ou accord de la
pensée avec elle-même. {\it Logique formelle} : celle qui étudie les
conditions de la validité formelle (cf. {\it Forme}$^3$).

— \si{Vulg.} {\bf 4.} Explicite, précis :« Un démenti formel ». — {\bf 5.}
{\it Péj.} Qui ne s'occupe que de la forme$^3$ et néglige le fond : « Une
argumentation toute formelle. »
% 83 — GÉN

\ib{Fortuit} — \si{Épist.} Qui relève du hasard* : « La caractéristique des
phénomènes que nous appelons fortuits, c’est de dépendre des causes trop
complexes pour que nous puissions les connaître toutes » (Borel).

\ib{Frange} [trad. angl. {\it margin}]. — \si{Psycho.}
{\it Chez James} : halo subconscient qui
enveloppe certains faits de conscience. Cf. {\it Marginaux}*.

\ib{Frustration} — \si{Ps. an.} État résultant
de l'impossibilité de satisfaire une tendance.

\ib{Futur} — A venir. Cf. {\it Contingent}$^2$.

	\end{itemize}
