
	\begin{itemize}[leftmargin=1cm, label=\ding{32}, itemsep=1pt]

% 77(
\item {\bf Fabulation} — Psycho. 1. « Présentation d’un récil imaginaire, souvent vraisemblable, comme étant
réel, sans intention de tromper »
(Piéron), $->$ Dist. mythomanie* :
« La fabulation n’est qu’une substitution à un déficit de la mémoire »
(Baruk).

— Hist. 2. Procédé qui consiste
à exposer sa pensée sous forme de
mythes$^2$ : « La fabulation platonicienne » (Schubhl).

\item {\bf Facteur} — Ps. métr. Élément commun
à plusieurs opérations mentales de
même nature. Cf. Analyse$^4$ et G.

\item {\bf Factice} — Hist. 1. Chez Descartes :
« idées factices », celles qui sont
« faites et inventées » par l'imagination. Cf. Adventices*, — 9, Souvent, auj. (par imitation de l'all.
faktisch) qui existe à l’état de
simple fait. Cf. Facticité$^2$. $->$ En
ce dernier sens, il vaudrait mieux
dire factuel.

\item {\bf Facticité} — 1. Caractère de ce qui
est construit par l'esprit ou fabriqué
par l'art humain : « L'esprit, dans
la culture scientifique, n’a plus peur
de la facticité » (Bachelard); « L’essentielle facticité donne à l’œil qui
contemple ce bronze... » (id). —
2. Souvent, auj. (all. : Faktizität) :
caractère de ce qui existe comme
pur fait, sans fondement$^2$, pour un
être-déjà-dans-le-monde : « La facticité du pour-soi » [i. e. la double
contingence de son existence et de
son engagement dans une situation
donnée] (Sartre).

\item {\bf Faculté} — Méta. 1. Puissance d'agir
(habilitas ad agendum selon la définition des Scolastiques, qui la distinguaient de la propriété$^1$ ou habilitas ad patiendum).

— Psycho. 2. Facultés de l'âme :
nom donné autref. aux fonctions
psychiques : « Les faits semblables,
nous les rapportons à un même
principe que nous appelons faculté et
que nous concevons comme une force
de notre esprit » (Royer-Collard),
% 78

\item {\bf Faible} — Psycho. Dans la Gestaltpsychologie, une forme$^4$ est dite
faible quand elle est peu structurée,
quand les éléments sont peu intégrés au tout.

\item {\bf Fait} — Épist. Toute donnée de l’expérience$^1$ : « Un fait n'est rien par
lui-même, il ne vaut que par l’idée
qui s’y rattache » (Cl. Bernard).
$->$ Dist. événement et phénomène : un événement est un fait considéré avec ses particularités de
temps et de lieu ; — fait a un sens
plus général, mais désigne encore
une donnée complexe et concrète;
on dist. qqfs. le « fait brut », i. e. tel
qu'il serait donné dans l'intuition,
et le « fait scientifique » (syn. : phénomène); — un phénomène$^1$ est un
fait analysé, considéré dans ses
éléments abstraits, indépendamment de toute particularité de temps
et de lieu, et avec la notion d’une
répétition possible. — Vérités de
fait : vérités empiriques$^2$ ou expérimentales$^2$, donc contingentes (opp.
vérités de raison qui sont nécessaires$^1$). — Sciences de fais :
les sciences expérimentales$^2$ (opp.
sciences rationnelles ). — En fait :
voir Droit$^1$ et Nécessaire$^2$.

\item {\bf Famille} — Soc. 1. La société domestique*, comprenant tous ceux qui
sont parents entre eux. — 2. Spéc.,
de nos jours, la société conjugale :
groupe formé par le père, la mère
et les enfants.

— Biol. 3. Groupe morphologique intermédiaire entre le genre$^2$
et l’ordre$^8$ : « La famille des
félins ».

\item {\bf Fanatisme} — Mor. Attitude qui consiste à pousser jusqu’au bout le
culte exclusif d’une idée : « Le panthéisme est un fanatisme théologique » (Le Senne). D'où : intolérance :
% 78 — FAU
« Il n’y a point de fanatisme
sans cérémonie » (Alain).

\item {\bf Fantaisie} [G. phantasia, imagination].
— Psycho. Autref. imagination
« Cette fantaisie est une véritable
partie du corps » (Descartes, Reg.,
XII); « Les images qui sont peintes
en la fantaisie » (Port-Royal).

\item {\bf Fascination} — Ps. path. État dans
lequel la conscience$^1$ est complètement absorbée par une perception.

\item {\bf Fatal} —— Méta. Qui se produit inévitablement, malgré tout effort contraire de la volonté et de l’intelligence humaines.

\item {\bf Fatalisme} — Méta. 1. Lato. Autre.
syn. de déterminisme$^3$. — 2. Str.
À Doctrine philosophique selon laquelle tout est fatal*. $->$ Impropre
au sens 1. Dist. déterminisme : « La
liaison des causes et des effets, bien
loin de causer une fatalité insupportable, fournit plutôt un moyen de
la lever » (Leibniz, Théod., 55).

\item {\bf Fatum} [mot latin]. — Hist. Dans la
Théod. (préface), Leïbniz distingue : 1° le fatum mahumeitanum,
fatalisme$^2$ absolu, aboutissant à l’argument paresseux*; — 2° le fatum
stoïcum où « tranquillité à l'égard des événements par la
considération de la nécessité qui
rend nos soucis et nos chagrins inutiles »; — 3° le fatum christianum,
résignation confiante aux décrets de
la Providence.

\item {\bf Faustien} — Hist. Qualificatif appliqué
par O. Spengler à la civilisation
occidentale et spéc. germanique,
fondée sur le devenir* et l’aspiration à l'infini. Cf. apollinien*.

\item {\bf Faute} — Log., Mor., Esth. (Péj.) Action
d’enfreindre une norme* : « Une
faute logique »; « Une faute morale »
(cf. Péché*); « Une faute de goût ».
% 79

\item {\bf Fechner (Loi de)} — Psycho. (Syn. :
loi psychophysique ou logarithmique).
Loi selon laquelle l’intensité de la
sensation* varie comme le logarithme de l’excitant*.

\item {\bf Feed-back} — Techn. Dans le lang. de
la cybernétique* : action en retour
qui permet, dans les machines, des
régulations de fonctionnement analogues à celles qu’on observe dans
les organismes vivants.

\item {\bf Fétichisme} — 1. Culte des fétiches,
i. e. d’objets matériels qui sont
censés posséder des pouvoirs magiques. Chez Comte : première forme
de « l’état théologique$^2$ » qui consiste
à « attribuer à tous les corps extérieurs une vie analogue à la nôtre,
mais plus énergique » (g. divinisation des astres). — D'où ext. : 2. Vénération superstitieuse a) Chez
Marx, « fétichisme de la marchandise » : illusion qui confère à la marchandise un caractère « mystique »
en lui attribuant une valeur immanente, alors que cette valeur n’appartient qu'au travail humain qui la
produit; — b) Chez Renouvier,
« fétichisme en philosophie » : façon
de penser qui consiste à se forger
des idoles, telles qu’idée en soi,
esprit pur, formes$^1$ substantielles,
archées*, etc.

\item {\bf Fiat} [mot latin = que cela soit fait].
— Psycho. Terme employé not. par
W. James pour caractériser la décision* volontaire.

\item {\bf Fictif} — Psycho. 1. (Ctr. : réel). Qui
est un produit de la fiction, i. e. de
l’imagination$^4$. — 2. Qui repose sur
l'imagination$^4$ : « L'état théologique$^2$ ou fictif » (Comte).

\item {\bf Fidéisme} — À, Théoi. 1. Str, Doctrine
selon laquelle la religion est objet
de pure foi$^5$, sans que l'esprit puisse
% 79
apporter à celle-ci aucun préambule rationnel : « Le pur fidéisme
n’est pas moins contraire à lorthodoxie religieuse qu'à la raison » (Le
Roy). — Crit. 2. Laio (Ctr. : rationalisme$^2$). Toute doctrine qui met
la croyance$^2$ au-dessus de la raison
et fait reposer la vérité sur des exigences sentimentales ou morales.

\item {\bf Figure} — Math. 1. Étendue$^1$ limitée
par des lignes : « La matière ou
l'étendue renferme en elle deux
propriétés : la première est celle de
recevoir différentes figures » (Malebranche, R. V., I, 1). — Psycho.
2. Dans la Gestaltpsychologie, opp. à
Fond*, — 3. Façon de parler symbolique : « Une figure de rhétorique ». — 4, Ext. Théol. Symbole
d’une vérité de foi : « Les prophètes
prophétisaient par figures » (Pascal).

— Log. form. 5. Figure d’un syllogisme : forme qu'il prend selon
la place du moyen$^2$ terme dans
les deux prémisses (v. Précis, Ph, II,
p. 34).

\item {\bf Fin} — Épist. et Méta. 1. (Opp. :
moyen$^1$). Ce pour quoi une chose
est ou se fait ; ce vers quoi elle tend
consciemment ou inconsciemment :
« Vous ne détournerez nul être de
sa fin » (La Fontaine); « Il est, dans
la nature, des fins que la raison ne
saurait méconnaître » (Ch. Bonnet),
— 2. But*, ce vers quoi tend un
acte consciemment et intentionnellement$^1$ : « Ce qui est désiré pour
l'amour de soi-même et à cause de
sa propre bonté s’appelle fin » (Bossuet); « Nous sommes sujets à nous
abuser quand nous voulons déterminer les fins ou conseils de Dieu »
(Leibniz, Disc. méta., XIX).
— Vulg. 3 Terme, cessation :
« Toutes les choses de ce monde
prennent fin » (Sévigné).
%80

\item {\bf Final} — Épist. et Méta. 1. Qui se
rapporte à une fin$^1$, Cause finale
(opp. : efficiente*) : la fin$^1$ elle-même,
considérée comme la raison d’être
de la chose (cf. Cause$^4$) : « La connaissance des causes finales n’est
pas nécessaire dans la physique »
(Malebranche, Méditations chrétiennes, XI) ; « La voie des causes
efficientes est plus profonde, mais
la voie des finales est plus aisée »
(Leibniz, Disc. méta., 22) ; « La
science reste défavorable aux causes
finales que nulle part elle ne discerne
avec évidence » (Le Roy). Argument des causes finales (syn. : physico-léléologique*) : celui d’après
lequel les faits naturels, « disposés
avec ordre, intelligence, prévoyance
pour les besoins et le bien de chaque
être », prouvent « l'existence d’une
cause intelligente et souverainement bonne » (Franck).

— Vulg. 2 (Opp. : initial). Terminal, dernier.

\item {\bf Finalisme} — Méta. À. Doctrine qui
admet dans l’univers : À. une finalité$^1$ fondée sur l'argument des
causes finales$^1$ (syn. : providentialisme*); — 2. de la finalité* en un
sens quelconque : « Le finalisme renverse l’ordre naturel : il explique le
présent par l'avenir » (Goblot).

\item {\bf Finalité} — Épist. et Méta. 1. Caractère de ce qui tend vers une fin$^2$ de
façon consciente : « La volonté est
toujours indissolublement finalité
par les desseins, mécanicité par les
moyens » (Pradines), — 2. Adaptation des moyens$^1$ à une fin$^1$, des
organes aux besoins, soit par une
activité intelligente, soit par une
« finalité sans intelligence » (Goblot) :
« Nier la finalité organique, c'est le
plus audacieux des paradoxes » (id.);
« La finalité du plaisir et de la douleur ».
% 80
— 3. Dépendance des parties
ou éléments à l'égard d’un tout :
« La finalité n’est pas la conformité
à l’idée : elle est l'idée » (Hamelin).
— 4. Finalité externe : celle où la
fin est extérieure à l'être considéré.
F. interne : celle où la fin est l'être
lui-même (sens 3). — 5. Principe de
finalité : « La nature ne fait rien en
vain » (Aristote) ou « Tout être à
une fini, »

\item {\bf Finesse (Esprit de)} — Car. Chez
Pascal (opp. : esprit de géométrie);
« souplesse de pensée » qui donne
l'intuition de la complexité des
choses et le sentiment des rapports
qui les unissent.

\item {\bf Fini} — Méta. Qui a une limite ou une
mesure : « J’ai premièrement en
moi la notion de l'infini que du fini»
(Descartes, Méd., III).

\item {\bf Finitude} — Méta. Caractère fini*, spéc.
de l'être humain : « La finitude du
Dasein* réside dans l’état d’oubli »
(Heidegger).

\item {\bf Fixation} — Psycho. 1. Fonction de la
mémoire$^4$ par laquelle l’esprit assimile les souvenirs : « La force de
fixation d’un souvenir dépend de
l'intensité du fait mental primitif »
(Piéron). — Ps. an. 2. Chez Freud:
le fait que la libido$^1$ reste attachée
à un objet d’un stade ancien.

\item {\bf Fixisme} — Biol. (Opp. : transformisme*)). À. Théorie de la fixité des
espèces, selon laquelle les espèces$^3$
vivantes demeurent immuables et
sont nettement distinctes les unes
des autres.

\item {\bf Fluctuation} — Biol. Variation quise produit chez un être vivant sous l’action
du milieu. $->$ Dist. mutation*.

Foi. — Vulg. 1. Garantie : « Sur la
foi des traités ». — 2 Fidélité à un
%81
engagement : « C’est parce que nous
sommes civilisés que nous nous
imposons le respect de la foi que
nous avons jurée » (Davy). Bonne
foi : sincérité. Mauvaise foi : duplicité$^2$ ; spéc., chez Sartre : attitude de
la conscience qui se masque à elle-même la vérité, mensonge à soi-même. — 3. Confiance : « Quoiqu’à leur nation [les voleurs] bien
peu de foi soit due,... » (Molière).
— 4, Syn. de croyance$^2$ : « Ajouter
foi à... ». Chez Kant : « foi morale »,
croyance rationnelle, quoique non
démontrable, à la liberté, à l’existence de Dieu et à la vie future.

— Théol. 5. O. Adhésion aux
dogmes d’une Église, à des vérités
considérées comme révélées : « La
foi est différente de la preuve : l’une
est humaine, l’autre est un don de
Dieu » (Pascal, 248). — 6. @ Objet
de la foi$^5$, les dogmes : « Je ne croirai
jamais que la vraie philosophie soit
opposée à la foi » (Malebranche,
Entr., VI, 2).

\item {\bf Fonction} — Math. 1. Une variable

y est fonction d’une variable x quand
à toute valeur déterminée de x correspondent une ou plusieurs valeurs
déterminées de y : on écrit y = f(x).
Log. Le sens mathématique de ce
mot a été étendu par la logistique*
aux expressions logiques : « Fonction logique, propositionnelle, conceptuelle. » Cf. Sujet*,

— Biol. 2. Ensemble d'opérations
par lesquelles se manifeste la vie
d'une cellule, d'un tissu, d’un organe, d’un être vivant (nutrition,
relation, reproduction). — Ext.
Psycho., Soc. 3. Ensemble d’opérations par lesquelles se manifeste la
vie mentale ou sociale : « Vouloir,
juger ne sont que différentes fonctions de notre entendement » (Voltaire); « La fonction fabulatrice* »
% 81
(Bergson)); « La fonction contractuelle » (Davy).

\item {\bf Fonctionnel} — Psycho. et Péd. « Le
point de vue fonctionnel est celui
du rôle joué par tel ou tel processus
dans la vie de l'individu » (Claparède). — Éducation fonctionnelle :
« celle qui prend le besoin de l'enfant
comme levier de l’activité qu'on
désire éveiller chez lui » (id.).

Fond. — Psycho. Dans la Gestalttheorie : partie non structurée,
amorphe, du champ$^2$, sur laquelle
se détache une figure$^2$ (voir Précis,
Ph. I, p. 115).

\item {\bf Fondement} — 1. Principe sur lequel
repose en fait un ordre de phénomènes. — 2. Principe sur lequel
repose en droit* un système d’assertions ou de règles, i. e. qui les rend
légitimes du point de vue logique,
moral ou juridique. — Cf. Base$^1$.

\item {\bf Force} — Vulg. 1. (Souvent opp.

droit$^1$). Contrainte : « Céder à la
force ». — 2. Puissance d'action :
« Les forces morales ». Îdée-force
(Fouillée) : la représentation$^1$ considérée comme poussant à l’action$^2$,
— 3. Agent$^1$ physique : « Les forces
naturelles ».

— Math. 4 En mécanique : tout
ce qui est capable de modifier l’état
de repos ou de mouvement d’un
corps (cf. Inertie$^2$). — 5. Force vive
((syn. : énergie actuelle) : demi-produit de la masse par le carré de la
vitesse (1/2 mv$^2$).

— Méta. 6. L’énergie*, considérée
comme le principe indéfinissable
qui produit les phénomènes de l’univers : « Force et matière » (Büchner);
« La persistance de la force » (Spencer).
— 7. Chez Leibniz: voir Substance$^1$.

\item {\bf Formalisme} — Épist. 1. Tendance à
faire passer le point de vue de la
% 82
forme$^3$ avant celui du fond ou de
la matière$^3$.

— Mor. 2. Formalisme kantien :
selon Kant, nos actions sont morales
en tant que déterminées par l’élément formel$^2$ qui fait la volonté
raisonnable (i. e. par l’idée d’une
règle universelle), et non par leurs
fins matérielles$^1$.

— Jur. 3. Formalisme juridique :
système juridique fondé sur le respect des formes$^5$ : « Le formalisme
répond à un impératif de la vie
sociale » (H. Lévy-Bruhl).

\item {\bf Forme} — (Gén. opp. matière). Hist.
1. a) Chez Aristote : « forme ou cause
formelle », ce vers quoi tend tout
changement; elle est à la fois l’acte$^2$,
l'essence*, la perfection et le principe d'unité de chaque être : « Un
des principes d’Aristote est que la
forme est un être distinct de la
matière$^1$ » (Buffon). — b) Chez les
Scolastiques : « forme substantielle »,
principe substantiel d’un être individuel défini par son essence spécifique : « Il n’y a du plus et du moins
qu'entre les accidents$^1$ et non point
entre les formes ou natures des individus d’une même espèce » (Descartes, Méth., I) : « Dieu est l’acte$^1$
des actes et la forme des formes »
(Bossuet); « L'opinion des formes
substantielles a qqc. de solide »
(Leibniz, Disc. méta, 10). — 2. Chez
Kant [all. : Form] : « formes » de la
connaissance, lois que la pensée
impose à la matière$^2$ de celle-ci, i. e.
au donné pur de la sensation : les
« formes pures a priori de la sensibilité » sont le temps et l’espace;
les « formes » de l’entendement sont
les catégories*; de même, la « forme »
de la loi morale est son caractère
impératif, catégorique* et universel.

— Log. 3. Forme d'un jugement
ou d'un raisonnement : ensemble des
% 82
relations existant entre les termes
auxquels ils s'appliquent, abstraction faite de leur matière$^3$ ou contenu.

— Psycho. 4. [Trad. all. Gestalt).
Ensemble structuré. Théorie de la
forme [all. : Gestalttheorie] : théorie selon laquelle l'esprit, vg. dans
la perception, saisit d’abord de
telles formes$^4$, et non des éléments.
Cf. Précis, Ph. I, p. 115.

— Jur. 5. Façon de procéder
selon certaines règles, formalités ou
formules : « Ils [les juges] n’ont la
liberté de juger que selon les formes
qui leur sont prescrites » (Pascal,
Prov., 14).

\item {\bf Formel} — Hist. Qui concerne la
forme* : 1. au sens scolastique. D'où :
essentiel : « Le formel de la concupiscence, non plus que le formel du
péché, n'est rien de réel » [= c’est
qqc. de négatif, une perte] ([Malebranche, Ecl., I). Chez Descartes, la
réalité « formelle » d’une idée est celle
qu’elle a dans la chose elle-même
indépendamment de la représentation que nous en avons (voir Éminent*) : « Cette réalité que les philosophes appellent actuelle$^2$ ou formelle » (Méd., III); — 2. au sens
Kantien : « morale purement formelle »; — 8. au sens logique : la
validité formelle (opp. : vérité matérielle) d’une proposition est celle
qui relève des « conditions de l’usage
de l’entendement en gén., sans distinction d'objets » (Kant) : elle se
ramène à la non-contradiction ou
accord de la pensée avec elle-même.
Logique formelle : celle qui étudie
les conditions de la validité formelle (cf. Forme$^3$).

— Vulg. 4. Explicite, précis :« Un
démenti formel ». — 5. Péj. Qui ne
s'occupe que de la forme$^3$ et néglige
le fond : « Une argumentation toute
formelle. »
% 83 — GÉN

\item {\bf Fortuit} — Épist. Qui relève du
hasard* : « La caractéristique des
phénomènes que nous appelons
fortuits, c’est de dépendre des
causes trop complexes pour que
nous puissions les connaître toutes »
(Borel).

\item {\bf Frange} [trad. angl. margin]. — Psycho.
Chez James : halo subconscient qui
enveloppe certains faits de conscience. Cf. Marginaux*.

\item {\bf Frustration} — Ps. an. État résultant
de l'impossibilité de satisfaire une
tendance.

\item {\bf Futur} — A venir. Cf. Contingenf$^2$.

	\end{itemize}
