
	\begin{itemize}[leftmargin=1cm, label=\ding{32}, itemsep=1pt]

\item {\bf E}. — Log. form. Désigne les propositions universelles* négatives : « Aucun homme n’est éternel » est une
« proposition en E ».

\item {\bf Ectéité} [L. ecce, voici]. — Méta. Dans
le lang. scolastique : ce qui fait qu’une
essence s’individualise et est présente au monde.

\item {\bf Échec}. — Ps. an. 1. Chez Adler : la
crainte de l’échec est à la base du
sentiment d’infériorité*. — Méta.
2. Chez Jaspers : échec existentiel :
mort, non-savoir, etc. : « L’échec
peut être mouvement vers l’éternité. »

\item {\bf Écholalie, Échopraxie}. — Psycho.
Phénomènes consistant en ce qu'un
sujet hypnotisé ou dégénéré répète
automatiquement les paroles (écholalie) et les gestes (échopraxie) de la
personne qui le suggestionne.

\item {\bf Éclectisme} [G. eklegein, choisir], —
Méta. 1. Méthode philosophique qui
consiste à emprunter à des systèmes
différents des thèses que l’on fond
en un système unique. On a dit aussi
éclecticisme : « J’ai toujours suivi la
méthode de l’éclecticisme : j'ai pris
dans toutes les sectes ce qui m'a
paru le plus vraisemblable » (Voltaire). $->$ Dist. syncrétisme*.

— Hist. 2. A. Doctrine et école
philosophiques de Victor Cousin.

\item {\bf Ecmnésie} [G. ec, dehors, et mnêmê,
mémoire]. — Ps. path. Trouble de
la mémoire où le sujet revit en action
des scènes de son passé [vg. cas
d’Irène, Précis, Ph. I, p. 70].

\item {\bf École (L’)}. — Hist. La philosophie scolastique* : « J'userai ici librement
des mots de l’École » (Descartes).

\item {\bf Écologie}. — Ébpist. Science des relations entre l'organisme vivant et
son milieu. D'où Ext. (Soc.) « étude
de l’interdépendance des institutions
et du groupement des hommes dans
l’espace » (écologie humaine).

\item {\bf Économétrie}. — Épist, « Étude, par la
méthode mathématique et sur la
base de données statistiques, de
relations [économiques]  particulières limitées à un cadre institutionnel donné ». (A. Marchal).

\item {\bf Économie} [G. oïkonomia, administration de la maison]. — 1. @ Mode
d'organisation de la production et
de l'échange : « L'économie urbaine»;
« L'économie contrôlée ». — Ext. 2,
Ordre qui règne dans l’arrangement
des parties d’un ensemble : « L’économie du corps » (La Bruyère);
« L’économie de la nature » (Voltaire).

— O Mor. 3. (Ctr. : prodigalité).
Qualité qui consiste à modérer ses
dépenses. D’où : — Épist. 4. Réalisation d’une fin par les moyens les
plus simples : « La science est une
économie de pensée » (Mach). Principe
% 62
d'économie : principe selon
lequel la nature obtient « le maximum d'effet avec le minimum d’effort ». — Psycho. 5. Méthode d’économie: méthode inventée par Ebbinghaus pour mesurer la rétention des
souvenirs (v. Précis, Ph. I, p. 224).

\item {\bf Économie politique} [G. oïkonomia et
polis, État, — Terme inventé par
A. de Montchrestien, Traité d'Économie politique, 1615]. — Épist.
Étude positive des « rapports des
hommes vivant en société en tant
que ces rapports tendent à la satisfaction de leurs besoins matériels »
(Ch. Gide). Elle comprend l'étude de
la production, de la distribution et
de la consommation des richesses. —
On dit aussi : l’économique ou la
science économique.

\item {\bf Économie sociale}. — Épist. Étude
normative* des phénomènes économiques : vg. recherche des « institutions du progrès social » (Ch. Gide).

\item {\bf Écorce}. — Phol. (Syn. : cortex). Partie
superficielle des hémisphères cérébraux, composée de substance grise.

\item {\bf Écossaise (École)}. — Hist. École
philosophique représentée par Thomas Reid, Dugald Stewart, Hamilton, etc.

\item {\bf Éducation}. — 1. Lato. Processus par
lequel une fonction se développe et
se perfectionne par l'exercice même :
vg. « éducation des sens ».

— Péd. 2. Str. Ensemble des
procédés par lesquels on développe
méthodiquement les facultés de
l'enfant.

\item {\bf Efférent}. — Voir Centre*.

\item {\bf Effet}. — Épist. 1. Phénomène considéré comme produit par une cause
efficiente*. — Psycho. 2. Loi de
l'effet : celle qui pose que « toutes
% 62
choses égales d’ailleurs, une réponse
est renforcée par le succès, affaiblie,
éliminée, ou remplacée à la suite de
l'échec » (Lagache).

\item {\bf Efficace}. — Méta. Qui produit réellement son eflet : « Cause efficace »
(opp. « occasionnelle* »).

\item {\bf Efficience}. — Épist. 1. (Opp. : finalité*). Causalité efficiente* : « La
science ne peut s'intéresser à la finalité qu'après avoir épuisé tout son
effort dans la découverte de l’efficience » (F. Houssay).

— Vulg. 2. [Angl. : efficiency].
Rendement, effet utile : « Le pragmatisme* est une théorie de l’efficience de la connaissance. »

\item {\bf Efficiente (Cause)}. — Celle qui « produit » l'effet (cf. efficace*)]. Cette
expression s'emploie auj. comme
syn. de cause* tout court (aux
sens 1, 2 et même 3), et par opp.
à cause finale (cf. Cause$^4$). Chez
Aristote, au ctr., la cause efficiente
se subordonne à la cause finale :
c’est « l’activité qui sort du fond
même de l’être et tend à réaliser la
fin » (Goblot).

\item {\bf Effort}. — Psycho. Activité d’un être
conscient qui rencontre une résistance. Sensation d’effort : celle qu’on
éprouve quand on tient un poids
soulevé, quand on perçoit la résistance d’un objet, etc. : « L’effort
est le véritable fait primitif du sens
intime » (Maine de Biran).

\item {\bf Égalitarisme}. — Mor. et Pol. 1. A
Tendance à l’égalité*. — 2. À Doctrine qui préconise l’égalité* (aux
sens 7, 8 ou 9).

\item {\bf Égalité}. — Math. 1. O Lato, Propriété
de deux grandeurs ou quantités
d’être mathématiquement équivalentes$^2$. Le signe de l'égalité est =.
— 2. O Str. En géométrie, deux
% 63
figures sont dites égales (ou congruentes) quand elles sont superposables. — 3. @ Formule qui exprime
l'égalité$^1$ de 2 termes : « Une égalité ».

— Log. 4. @ Égalité logique : proposition exprimant que deux termes
sont logiquement équivalents$^1$ :
« Une définition est une égalité logique ».

— Mor. et Pol. O 5. Égalité naturelle : fait que deux ou plusieurs
hommes ont même vigueur, même
agilité, même santé (égalité physique) ou même intelligence, même
mémoire, même vivacité d'esprit
(égalité intellectuelle). — 6. Égalité
morale : principe selon lequel la
personne humaine a partout même
« dignité » ou même valeur. — 7.
Égalité civile et juridique : principe
selon lequel tous les individus sont
4 égaux devant la loi et jouissent des
mêmes droits civils*. — 8. Égalité
politique : principe d’après lequel
« tous les citoyens sont également
admissibles à toutes dignités, places
et emplois publics selon leur capacité » et jouissent des mêmes droits
civiques* : « L'amour de la démocratie est celui de l'égalité » (Montesquieu, Lois, V, 3). — 9. Égalité
j économique ou sociale : principe qui
F peut se formuler, si on l'entend
comme une égalité stricte : « à
chacun selon ses besoins », et si on
l'entend comme une égalité de proportionnalité : « à chacun selon son
travail », ou : « selon sa capacité ».

\item {\bf Ego}. — Voir Je* et Moi*. — Chez
Husserl : « Ego transcendantal »,
sujet pur (cf. Parenthèses*) de la
connaissance tourné vers le monde.

\item {\bf Égocentrisme}. — Psycho. 1. Attitude
intellectuelle de celui qui pense
en ramenant tout à soi. — 2. Chez
l'enfant, « confusion du moi avec le
% 63
monde extérieur » (Piaget). Voir
Précis, Ph. I, p. 84 et 504. $->$ À
éviter au sens 1.

\item {\bf Égoïsme}. — Mor. (Ctr. : altruisme*).
Attitude pratique de celui qui agit
volontairement en ramenant tout à
soi : « L’égoïsme est naturellement
moins pervers dans le grand nombre
que dans le petit » (Renouvier).
«<> Dist. tendance spontanée de
tout être à conserver et développer
sa vie.

\item {\bf Égotisme}. — Esth. 1. Chez Stendhal :
analyse du moi en vue de raîffiner
ses sentiments et d'en jouir : « L’égotisme, mais sincère, est une façon de
peindre le cœur humain ». — 2. Lato
Culte du moi, en gén.

\item {\bf Éidétique} [G. eidos, essence ou image].
— 1. Méta. Dans le lang. de la phénoménologie* : qui concerne l'essence.
— 2. Psycho. Images éidétiques
sorte d'images$^3$ visuelles présentant
une netteté particulière et presque
hallucinatoires (chez l'enfant).

\item {\bf Éjet} [Angl. eject]. — Crit. Terme introduit par W. K. Clifford pour désigner autrui comme objet de connaissance projeté hors du moi, mais
conçu par analogie avec notre
propre moi.

\item {\bf Ek-stase}. — Méta. Chez Heidegger
et Sartre : situation d’un être placé
« en dehors » de lui-même : « L’en$^5$-soi est ce qu'il est, sans aucune dispersion ek-statique de son être »
(Sartre). Les trois ek-stases de la
temporalité* sont le passé, le futur
et le présent,

\item {\bf Élan vital}. — Méta. Chez Bergson :
« élan originel » d’où la vie est issue
et qui se développe, au cours de
l'évolution, en directions divergentes : « Les grands entraîneurs de
% 64
l'humanité semblent bien s'être
replacés dans la direction de l'élan
vital » (Deux Sources).

\item {\bf Éléates}. — Hist. Philosophes de l’école
d’Élée (Parménide, Zénon) qui
affirmaient l'identité absolue de
l'Être avec lui-même.

\item {\bf Électif}. — Psycho. 1. Tendances électives : tendances altruistes ayant
pour objet un individu déterminé
(amitié, amour$^2$).

— Ps. path. 2. Amnésie élective :
amnésie* d’évocation portant, soit
sur une espèce d'images (vg. visuelles), soit sur les souvenirs relatifs
à une idée (amn. systématisée).

\item {\bf Élément}. — 1. Composant plus simple
d'un tout complexe. $->$ Dist.
partie : la partie est plus petite,
mais aussi complexe que le tout. —
Spée. 2. En chimie : a) autref. les
quaire éléments : feu, air, eau, terre,
qu’on a crus simples jusqu’à Lavoisier; b) auj. les corps simples.

\item {\bf Élicite}. — Hist. Chez les Scolastiques :
« acte élicite », celui qui est absolument volontaire.

\item {\bf Émanation}. — Méta. 1. Sir. (Opp. :
création). Dans certaines doctrines
panthéistes* [vg. Plotin], processus
par lequel les êtres particuliers
découlent [L. manare, couler] de
l'Être unique. — 2. Lato. Production par création : « Dieu les produit
continuellement [les substances
créées] par une manière d'émanation comme nous produisons nos
pensées» (Leibniz, Disc. Méta., XIV).

\item {\bf Emblèmes}. — Soc. Représentations
figurées qui symbolisent les idéaux
collectifs. — $->$ Dist. symbole :
voir Précis, Ph. I, p. 334.

\item {\bf Emboîtement des germes}. — V. Préformation*.

% 64
\item {\bf Embryologie}. — Épist. Science des
formes par lesquelles passe l’organisme vivant de l’état d’embryon à
l’état adulte.

\item {\bf Émergence} [Angl. : emergency]. —
Méta. Apparition d’une réalité qui
sort d’une autre de façon imprévisible et sans en être l'effet nécessaire, et qui apporte ainsi une qualité nouvelle (vg. la pensée par rapport aux faits physiques et biologiques). Voir Précis, Ph. II, p. 453.

\item {\bf Émesthèse}. — Voir Autopsie$^2$.

\item {\bf Éminent}. — Hist. Descartes dist. trois
modes d’existence des idées$^4$ : la
réalité objective$^1$, la réalité formelle$^1$,
la réalité éminente : « Tout ce que
nous concevons comme étant dans
les objets des idées, tout cela est
objectivement ou par représentation
dans les idées mêmes. Les mêmes
choses sont dites être formellement
dans les objets des idées quand elles
sont en eux telles que nous les concevons. Elles sont dites y être éminemment quand elles n’y sont pas
à la vérité telles, mais qu’elles sont
si grandes qu’elles peuvent suppléer
à ce défaut par leur excellence »
(2° Rép., défin.). Vg. le monde existe
« éminemment » en Dieu, i. e. non
tel qu’il est en réalité, mais d’une
manière « plus excellente ».

\item {\bf Émotion}. — Psycho. 1. Latiss. Tout
état affectif : vg. Bain, Les Émotions
et la volonté. — 2. Lato. Toute manifestation complexe et organisée
de la vie affective, « équivalente
dans l’ordre affectif, de la perception dans l'ordre intellectuel » :
vg. peur, colère, « émotion esthétique », « émotion morale », « émotion tendre » (Ribot). — 3. Str.
État affectif violent et passager,
« choc brusque, souvent violent,
% 65
intense, avec augmentation ou arrêt
des mouvements » (id.) : vg. peur,
colère, « coup de foudre » en amour.
— Émotions-chocs (trad. coarse emotions, émotions fortes ou grossières) :
les émotions au sens 3, opposées
(not. par W. James) aux émotions-sentiments (trad. subtler emotions,
émotions fines) moins violentes,
mais plus durables (vg. tristesse,
joie, sentiment esthétique) qu'englobe le sens 2. $->$ Le sens propre
est le sens 3.

\item {\bf Émotivité}. — (Car. Disposition à
éprouver des émotions (surtout au
sens 3) : vg. enthousiasme ou indignation faciles, susceptibilité, angoisse devant une tâche à accomplir,
et surtout bouleversement pour des
causes minimes.

\item {\bf Empathie} [G. en, à l'intérieur, cet
pathein, sentir] (Syn. : intropathie).
— Terme (calqué sur l’all. Einfühlung) qui désigne : soit 1. Psycho. la
« substitution imaginative » (Pradines) d’une personne à une autre
dans la sympathie; — soit 2. Esth.
Ja « sympathie symbolique » (Delacroix) qui nous ferait communier
avec l’œuvre d'art : « L’empathie
bergsonienne » (Bayer).

\item {\bf Empirique} [G. empeiria, expérience].
— Épist. 1. Fondé sur l'expérience!
brute, non méthodique : « Remède
empirique ». — 2. (En parlant des
personnes). Qui se conduit uniquement par expérience$^2$, i. e. d'une
façon routinière : « Nous ne sommes
qu’empiriques dans les trois quarts
de nos actions » (Leibniz, Mon., 28).
— 3. Lato. Fondé sur l'expérience
en général, y compris l’expérience-méthodique : « Les définitions empiriques », « Une formule empirique » (i. e. découverte par expérimentation, non par la théorie ni le
%65
calcul). 5° Impropre au sens 3;
dire : expérienciel.

\item {\bf Empirisme}. — Épist. 1. à Ensemble de
procédés empiriques$^1$ (souvent péj.) :
« Les tâtonnements de l’empirisme ».

— Hist. 2. À (Opp.: rationalisme$^1$),
Doctrine selon laquelle la connaissance humaine tout entière dérive,
directement ou indirectement, de
l'expérience$^1$ et qui n’attribue par
suite à l'esprit aucune activité propre : « L’empirisme n’est que la
négation du savoir » (Hamelin).
$->$ Dist. méthode expérimentale :
l'empirisme$^2$ est une doctrine philosophique; la méthode expérimentale
est l'ensemble des procédés par
lesquels se constituent les sciences
de faits*.

— Psycho. 3. Voir Génétique$^4$.

\item {\bf Empiriste}. — Partisan de l'empirisme?, <> Dist. empirique$^2$.

\item {\bf Endophasie} [G. endon, dedans, et
phasis, parole]. — Psycho. Langage*
intérieur.

\item {\bf Énergétique}. — Phys. 1. Science des
propriétés générales de l’énergie$^1$,
abstraction faite des caractères$^2$
particuliers propres à chacune des
formes sous lesquelles elle apparaît.
— 2. Théorie physique fondée sur
le principe de la conservation* de
l'énergie et sur le principe de moindre
action* : « Le système$^2$ énergétique
a pris naissance à la suite de la
découverte du principe du principe
de la conservation de l'énergie. C’est
Helmholtz qui lui a donné sa forme
définitive » (IH. Poincaré).

\item {\bf Énergétisme}. — Méta. À La théorie
énergétique$^2$ érigée en système métaphysique qui fait de l'énergie la
substance même du monde : « L’énergétisme d’'Ostwald ».
% 66

\item {\bf Énergie}. — Phys. L'énergie d'un système de corps se mesure par le
travail$^1$ mécanique qu’il est capable
de produire. Énergie actuelle ou
cinétique : celle qui se manifeste par
le mouvement (égale à la somme
des forces$^5$ vives du système).
Énergie potentielle : celle qui n'existe
qu’en puissance, i. e. travail que les
forces intérieures du système effectueraient si les corps qui le composent obéissaient à l'action de ces
forces.

\item {\bf Englobant} [Trad. all. Umgreifende]. —
Méta. Chez Jaspers : ce qui enveloppe tout horizon particulier dans
lequel nous vivons : « Nous appelons
transcendance l’englobant dans lequel
nous sommes essentiellement, et
nous appelons existence l’englobant
que nous sommes nous-mêmes. »

\item {\bf Ensemble}. — Vulg. 1. Totalité des
parties d'un système : « Dans une
œuvre d'art, c’est l’ensemble qu'il
faut considérer ». — Math. 2. Collection d’objets, de nombres, etc., en
nombre fini (vg. les lettres de l’alphabet) ou infini (vg. la suite des
nombres entiers) : « C’est Cantor qui
a établi la Théorie des Ensembles
infinis ». — Voir Aleph* et Transfini*.

\item {\bf En soi}. — Méta. 1. De façon absolue* :
« Rien n'est grand ni petit en soi »
(Malebranche). — 2. La substance$^1$
est l'être en soi, i. e. qu’elle n'existe
pas en autre chose (opp. attribut$^2$,
qui a besoin, pour exister, d’un
sujet$^3$ dans lequel il existe). — 3.
La réalité en soi est la réalité considérée comme indépendante de la
connaissance que nous en avons :
vg. chez Kant les « choses en soi »
(cf. Noumène*)). — 4. Une fin en soi
est celle qui a une valeur absolue
(elle est fin* par elle-même et non
comme moyen d’une fin plus élevée).
— 5. Chez Sartre : « Il faut opposer
cette formule : l'être en soi est ce
qu’il est, à celle qui désigne l'être
de la conscience [le pour-soi] : celle-ci
en effet a à étre ce qu’elle est... L'être-en-soi n’a pas de dedans qui s’opposerait à un dehors... L'en-soi n’a pas
de secret : il est massif. » Pratiquement, souvent syn. (chez Sartre) de
réalité matérielle.

\item {\bf Entéléchie} [G. entelôs echein, être à
l'état de perfection]. — Hist. 1.
Chez Aristole : état de l'être en
acte$^2$, pleinement réalisé. — 2. Chez
Leibniz : « Toutes les substances
simples ou monades$^2$ créées », parce
qu’il y a en elles « une suffisance
qui les rend sources de leurs actions
internes » (Mon., 18).

\item {\bf Entendement}. — Psycho. et Crit. 1.
(Opp. : volonté$^2$). Faculté de comprendre et de penser$^1$ : « Toutes nos
façons de penser peuvent être rapportées à deux générales, dont l’une
consiste à apercevoir par l’entendement et l’autre à se déterminer par
la volonté$^2$ : ainsi, sentir, imaginer
et même concevoir des choses purement intelligibles ne sont que des
façons différentes d’'apercevoir »
(Descartes, Prince, I, 32); « La
faculté de recevoir différentes idées
et différentes modifications dans
l'esprit est entièrement passive et
j'appelle cette faculté ou cette capacité entendement » (Malebranche,
R. V., I, 1); « La puissance d’apercevoir est ce que nous appelons
entendement : il y a la perception
des idées, la perception de la signification des signes et enfin la perception de la convenance ou disconvenance qu'il y a entre quelques-unes de nos idées » (Leibniz, N. E.,
II, 21). Voir intellection*. — 2.
% 67
(Opp. : connaissance sensible et
raison$^2$). Chez Kant ((Verstand) :
faculté de coordonner les sensations à l’aide des catégories*, de
juger et de raisonner : « Toute notre
connaissance commence par les sens,
passe de là à l’entendement et
s'achève dans la raison » (AR. pure,
Dial., introd., Il). — 3. (Opp. :
raison intuitive). Pensée discursive*,
faculté de raisonner.

\item {\bf Entendre}. — Au XVII$^\text{e}$ siècle : comprendre$^\text{A}$ : « Notre volonté étant
plus ample que l’entendement, je
l’étends aussi aux choses que je
n'entends pas » (Descartes, Méd.
IV).

\item {\bf Enthymème}. — Log. form. Syllogisme* dont une prémisse est sous-entendue : vg. « L'homme a des
droits, donc il a des devoirs ».

\item {\bf Entité}. — Méta. 1. Autref. réalité
totale de l'être individuel : « L’entité ou l'être de la chose » (Descartes, 2$^\text{e}$ Rép.). — 2. Auj. ce qui
constitue l’essence d’un genre$^1$ (not.
Méd. : « les entités morbides », les
types de maladies). — 3. Souvent
péj. : « Ils [les philosophes] voient
quelque effet nouveau : ils imaginent aussitôt une entité nouvelle
pour le produire » (Malebranche,
R. V., III, 8). Chez Comte: « abstraction personnifiée », « Des forces
abstraites, véritables entités »
(Cours, I).

\item {\bf Énumération}. — Log. Induction par
énumération : l'induction$^2$ formelle.
— Voir Dénombrement*.

\item {\bf Épagogique} [G. epagôgé, induction].
— Log. Inductif (spéc. en parlant
de l’induction$^2$ formelle).

\item {\bf Épichérème}. — Log. form. Syllogisme dont les deux prémisses sont
% 67
accompagnées de leurs preuves (vg.
le Pro Milone de Cicéron).

\item {\bf Épicritique}. — Phol. (Opp. : protopathique*). Se dit de la sensibilité tactile superficielle (contact, froid).

\item {\bf Épigénèse}. — Biol. À [Ctr.: préformation*). Théorie selon laquelle les
organes de l'être vivant se constituent, au cours de son développement, par degrés et grâce à une
formation nouvelle, au lieu d'être
préformés dans l'embryon.

\item {\bf Épiphénoménisme}. — Psycho. A.
Théorie selon laquelle la conscience$^1$
serait un simple épiphénomène, i. e.
un phénomène accessoire, sans efficacité, l’élément constitutif du fait
psychique étant essentiellement le
processus nerveux.

\item {\bf Épistémologie} [G. épistêmé, science,
et logos, étude]. — 1. Sir. Étude de
la connaissance scientifique du point
de vue critique$^1$ (i. e. de sa valeur).
— 2. Lato. Gnoséologie*. $->$ Voir
Logique$^2$.

\item {\bf Épisyllogisme}. — Log. form. Voir
Polysyllogisme*.

\item {\bf Épochè} [mot grec]. — Méta. Suspension du jugement. Spéc., dans le
lang. de la Phénoménologie* : refus
de se prononcer sur les problèmes
d'existence et les réalités substantielles. Voir Parenthèses* et Réduction$^2$.

\item {\bf Équation}. — Math. 1. Égalité$^3$ comprenant des inconnues et qui ne se
vérifie que pour certaines valeurs$^6$
de celles-ci. — Épist. 2. Équation
personnelle : correction qu'on fait
‘subir aux observations astronomiques pour chaque observateur.

\item {\bf Équilibre}. — Math. 1. En Mécanique,
un système de forces$^4$ est dit en
% 68
équilibre quand il est susceptible

de demeurer indéfiniment en repos.
— Par anal. : Psycho. et Log. 2. Un
groupement* logique (vg. une classification) est dit en équilibre quand
« la structure des totalités opératoires [qui le constituent] se conserve lorsqu'elles s’assimilent des
éléments nouveaux » (Piaget). —
Éc. pol. 3. Équilibre économique :
état d'ajustement des différents éléments de la vie économique (production, offre* et demande, prix,
profits et salaires, etc.) permettant
un fonctionnement satisfaisant de
l’ensemble.

\item {\bf Équipollents}. — Égaux, en parlant :
1. Log., de deux concepts ayant
même extension ; — 2, Math., de
deux segments orientés équivalents.

\item {\bf Équité}. — Mor. Justice$^1$ qui a égard
à ce qui convient à chaque cas particulier.

\item {\bf Équivalent}. — Log. 1. Deux notions
sont équivalentes quand elles ont
même extension$^3$. Principe de la
substitution des équivalents : celui
d’après lequel deux notions, quantités, etc., équivalentes peuvent être
substituées l’une à l’autre,

— Math. 2. Deux quantités sont
équivalentes quand elles ont même
mesure (spéc., en géométrie, même
aire ou même volume, sans qu'il y
ait nécessairement égalité$^2$).

— Phys. 3. Équivalent mécanique
de la chaleur : travail qui, intégralement transformé en chaleur, donne
une grande calorie. Principe de
l’équivalence : syn. de : principe de
la conservation$^2$ de l’énergie.

\item {\bf Équivoque}. — Log. Caractère des
termes ou des propositions qui peuvent s'entendre en deux ou plusieurs
sens différents « Les termes qui ne
réveillent que des idées sensibles
sont tous équivoques » (Malebranche, R. V., VI, 2, 2), $->$
L'équivoque des termes s'appelle
ambiguïté; celle des propositions
amphibologie.

\item {\bf Ergétiques (Tendances)} [G. ergon,
travail]. — Psycho. Nom donné par
P. Janet aux tendances de l’ordre
du travail et de l’effort qui consistent à agir efficacement sur le réel
(voir Précis, Ph. I, p. 73).

\item {\bf Éristique} [G. eris, dispute]. — Discussion vaine et subtile.

\item {\bf Erlebnis}. — Épist. © Mot all. qqfs
employé pour désigner l’expérience
interne, un « état vécu » de la conscience.

\item {\bf Érôs} [mot grec]. — Hist. 1. Chez
Platon : l'Amour, en gén. — Auj.
2. (Opp. : agapê). L'amour charnel.

\item {\bf Érotique}. — Qui a rapport à l’érôs$^2$.

\item {\bf Erreur}. — Épist, 1. O Action de se
tromper, d'affirmer ce qui est faux :
« L'erreur n’est pas une pure négation, c.-à-d. le simple défaut ou
manquement d'une perfection »
(Descartes, Méd., IV); « L'erreur
consiste dans une privation de connaissance » (Spinoza, Éth., II, 35).
— 2. @ Résultat de l'erreur$^1$ : « Les
erreurs des sens » (v. Précis, Ph. I,
p. 127); « Une erreur de caleul », Au
sens moral : « De ses jeunes erreurs
maintenant revenu » (Racine).

\item {\bf Eschatologie} [G. eschatos, dernier, et
logos]. — Doctrine des fins dernières de l’homme et de l'univers.

\item {\bf Esclavage}. — Soc. Condition dans laquelle le travailleur manuel se
trouve assimilé à un instrument de
travail et est la propriété de celui
qui l'emploie. $->$ Dist. servage*,

\item {\bf Esclaves (Morale des)}. — Mor. Chez
Nietzsche : la morale judéo-chrétienne qui exalte les « vertus d’esclaves » : humilité, résignation, pauvreté, amour des faibles (cf. Ressentiment*) par opp. à la « morale des
maîtres » qui exalte l’orgueil, la
volonté$^6$ de puissance et le culte de
la force.

\item {\bf Ésotérique} [G. esô, à l'intérieur]. —
Réservé aux disciples, aux initiés :
« La philosophie est essentiellement
ésotérique » (Lagneau). — D'où :
occulte, secret.

\item {\bf Espace}. — Crit. et Méta. Milieu homogène et indéfini dans lequel sont
censés situés les objets sensibles, On
peut le définir par l’extériorité mutuelle des parties : partes extra partes. Cf. Étendue* et Euclidien*.

\item {\bf Espèce} [L. species, apparence]. —
Théol. et Méta. 1. Apparence sensible : « Les espèces du pain et du
vin [dans l’Eucharistie] ». Spéc.,
dans le lang. scolastique : images
matérielles qui se détachent des
corps et viennent impressionner les
sens : « On appelle espèce intentionnelle un signe formel de la chose
présentée aux sens, ou une certaine
qualité$^2$ qui, émise par l'objet et
reçue dans le sens$^4$, a le pouvoir de
représenter l’objet même, bien
qu'elle-même ne soit pas perceptible par le sens; elle est appelée
intentionnelle parce que, par son
moyen, le sens tend vers l’objet »
(Eustache de Saint-Paul); « Lorsque
je vois un bâton, il ne faut pas
s’imaginer qu'il sorte de lui de petites images voltigeantes par l’air,
appelées vulgairement des espèces
intentionnelles$^2$, qui passent jusqu’à
mon œil » (Descartes, 6$^\text{es}$ Rép., 9).
Cf. Idée$^5$.

— Log. form. 2. Quand deux termes
généraux$^2$ sont compris en extension$^3$ l’un dans l’autre, le plus grand
s'appelle genre$^1$, le plus petit espèce :
« rectangle » est une espèce du genre
« parallélogramme ». $->$ Un terme
peut être espèce par rapport à un
autre et genre par rapport à un troisième.

— Biol. 3. Groupe d'êtres vivants
présentant certains caractères bien
définis qui constituent un type
héréditaire, gén. impossible à modifier par le croisement.

— Phys. 4. Espèce chimique
corps chimiquement défini.

\item {\bf Esprit} [L. spiritus, souffle]. — Hist.
1. Autref. fluide, gaz, matière
subtile. — 2. Esprits animaux : sorte
de « vent très subtil », constitué
par « les plus petites parties du
sang », qui, « montant continuellement en grande abondance du
cœur dans le cerveau, va se rendre
de là par les nerfs dans les muscles
et donne le mouvement à tous les
membres » (Descartes, Méth., V).

— Méta. 3. Principe de la vie et de
la pensée, âme$^1$ : « Rendre l'esprit »
— 4. (Ctr. : corps$^3$). Principe de la
pensée, âme$^2$ individuelle ; d'où :
être immatériel, âme des morts.
Pur esprit, être immatériel qui n’est
pas lié à un corps$^3$ : « Le premier de
tous les esprits, c’est Dieu » (Bossuet); « Je ne considère pas l'esprit
comme une partie de l'âme$^2$, mais
comme cette âme tout entière qui
pense » (Descartes, 5$^\text{es}$ Rép., II, 4);
« Un seul esprit vaut tout un monde »
(Leibniz, Disc. méta., 36). — 5. (Ctr. :
matière). Le monde de la pensée$^1$, la
réalité spirituelle en gén. : « Avant
l’iomme, l'esprit dormait dans la
nature » (Lagneau); « L'esprit est
le foyer commun qui éclaire et unit
toutes les consciences » (Lavelle).
Chez Hegel : l'Esprit est l’intériorisation
% 70
de la Nature; l'Esprit subjectif est le siège des faits psychiques
(âme, conscience, esprit$^6$); l'Esprit
objectif se manifeste dans le droit,
les mœurs, la moralité; l'Esprit
absolu, dans l’art, la religion et la
philosophie.

— Psycho. 6. Conscience, ensemble des phénomènes psychiques :
« Élevons plus haut nos esprits »
(Bossuet); « Par esprit (mind) nous
entendons ce qui dans l’homme
pense, se souvient, raisonne, veut »
(Reid). — 7. (Opp. : sentiment$^4$,
cœur$^3$). Connaissance, intelligence$^1$ :
« L'esprit comme le cœur a ses idoles»
(Renouvier)). Chez certains, opp. à
Ame$^3$ ou à Vie : « L'Esprit et la Vie,
l'Esprit et l'Ame sont en guerre par
une nécessité naturelle... L'Esprit
est le Dehors absolu comme l’Ame
est le Dedans naturel » (Klages;
voir Ame$^3$). — 8. Tournure ou orientation d’esprit$^6$ particulière : « Un
esprit élevé »; « L'esprit scientifique »; « L'esprit de système ».
Esprit de finesse*, de géométrie* :
voir ces mots. — 9. Inspiration
fondamentale : « L’esprit de la monarchie est la guerre » (Montesquieu) ; « C’est dans cet esprit que... ».

— Car. 10. Vivacité de la pensée
et faculté d'exprimer ses idées de
façon ingénieuse et piquante :
« Quand on court après l'esprit, on
n'attrape que la sottise » (Montesquieu).

\item {\bf Essais et erreurs}. — Psycho. Procédé
de tâtonnements aveugles d’où s’éliminent progressivement les erreurs
et qui, selon certains auteurs, est à
la base de la formation des habitudes. Voir Précis, Ph. I, p. 457; Sc.
et M., p. 15.

\item {\bf Essence}. — Log. (Ctr. : accident*).
« Ensemble des caractères intimes
qui persistent au milieu du changement des relations et des modifications accidentelles » (Liard) : « En
Dieu, l'essence n’est point distinguée
de l’existence » (Descartes, 4$^\text{es}$ Rép.);
« Cela appartient à l’essence d’une
chose, qui fait, lorsque cela est donné
que la chose est nécessairement
posée et, lorsque cela est ôté, que la
chose est nécessairement ôtée »
(Spinoza, Eth., II, déf. 2); « La
possibilité$^1$ est le principe de l’essence; la perfection$^1$ ou le degré
d'essence est le principe de l'existence » (Leibniz); « La dualité entre
les deux mondes de l'essence et de
l'existence est un problème insoluble si l'existence$^2$ n’est pas un
moyen par lequel l'essence comme
telle se réalise » (Lavelle).

\item {\bf Esthétique} (nom). — Crit. 1. Chez
Kant : « esthétique transcendantale », partie de la Critique de la
raison pure qui concerne les formes$^2$
a priori de la sensibilité.

— Esth. 2. Auj., théorie (positive et scientifique, ou bien normative et philosophique) de l’Art et
des conditions du Beau$^3$.

\item {\bf Esthétique} (adj.). — 3. Qui concerne
le beau. Activité esthétique : celle qui
a pour but de réaliser le beau, l’Art$^3$.
Sentiment esthétique : le sentiment
du beau.

\item {\bf Étalonnage}. — Ps. métr. Opération
qui consiste à déterminer la signification des résultats numériques
fournis par un test*.

\item {\bf État}. — Soc. et Pol. 1. Lato. La société
elle-même, quand elle possède des
organes politiques, juridiques et
administratifs différenciés : « La vie
des États est comme celle des hommes » (Montesquieu, Lois, X, 2). —
2. Str. Ensemble de ces organes;
centres directeurs et conscients de
la société : « L'État deviendra puissant »
% 71
(1b., V, 8); « La force collective qu'est l'État, pour être libératrice de l'individu, a besoin elle-même de contrepoids » (Durkheim).
Raison d'État, v. Raison$^8$. — 3.
Spéc., forme de gouvernement :
« L'état républicain a la vertu pour
principe » (Montesquieu, Lois, V, 1).
$->$ À éviter au sens 3. Dist. gouvernement*, nation*, société*.

\item {\bf États (Loi des 3)}. — Hist. Théorie
énoncée par À. Comte (Cours, I) et
selon laquelle l'intelligence humaine
dans son ensemble et chacune des
branches de nos connaissances passent par trois états successifs : l’état
théologique$^2$, l'état métaphysique$^{10}$
et l'état positif$^4$.

\item {\bf Étatisme}. — Pol. 1. (Ctr. : individualisme$^3$). Tendance à placer le salut
de l'État au-dessus de tout. — 2.
(Ctr. : individualisme$^4$). Tendance à
multiplier les attributions administratives et surtout économiques de
l'État$^2$.

\item {\bf Étendue}. — Vulg. 1. Portion finie et
gén. mesurée de l’espace* : « L’étendue d’un champ ».

— Psycho. 2. L'étendue concrète,
celle de la perception, s'oppose à
l’espace* abstrait (voir Précis, Ph. I,
p. 162).

— Méta. 3. Chez Descartes :
attribut essentiel des corps : « La
même étendue qui constitue la nature du corps constitue aussi la
nature de l’espace » (Princ., II, 11 :
voir Textes choisis, I, p. 77). Chez
Malebranche (à partir de 1678) :
étendue intelligible (dist. de « l’étendue locale ou matérielle »), idée
archétype qui représente les corps
dans l’entendement divin : « L’étendue intelligible représente des espaces infinis, mais elle n’en remplit
aucun » (Entr., II, 6).

\item {\bf Éternelles (Vérités)}. — Voir Vérité$^2$.

\item {\bf Éternité}. — Méta. Caractère de l'être
soustrait au devenir et au temps;
intemporalité : « L'éternité diffère
du temps parce qu’elle toute à la
fois (tota simul), tandis que le temps
est succession, et non pas en ce que
l’éternité serait un temps sans commencement ni fin » (Saint Thomas,
S. Th., I, q. 10, 4); « L’étendue créée
est à l’immensité* divine ce que le
temps est à l'éternité » (Malebranche, Entr., VIII, 4).

\item {\bf Éthique} [G. êthê, les mœurs]. — La
Morale$^2$ : « L’éthique est la science
normative primordiale » (Wundt).

\item {\bf Ethnie} [G. ethnos, peuple]. — Soc.
Terme créé pour éviter le mot Race*
et qui désigne un mélange de races
caractérisé par une même culture.

\item {\bf Ethnographie} [G. ethnos et graphê,
description]. — Épist. Étude descriptive de la civilisation$^1$ matérielle et morale des peuples.

\item {\bf Ethnologie}. — Épist. Syn. d'anthropologie* culturelle.

\item {\bf Éthologie} [G. êthos, caractère, et logos
étude]. — Nom autref. donné (vg.
par J. S. Mill) à la caractérologie*.

\item {\bf Étiologie} [G. aitia, cause, et logos]. —
Épist. Recherche des causes (spéc.
en Méd.).

\item {\bf Être (nom)}. — Méta. 1. O Existence :
« Ce qui connaît qqc. de plus parfait
que soi ne s’est point donné l'être »
(Descartes, Princ., I, 20). — 2. @ Ce
qui est : « O Dieu, le plus Être de
tous les êtres » (Fénelon)); « L’être
est vide de toute détermination
autre que l'identité avec lui-même »
(Sartre); « L'’être ne se découvre à
nous que dans l'expérience de l’existence » (Lavelle; cf, Acte$^3$). Chez
% 72
Comte: «le Grand Être », l'humanité.
— 3. Être de raison : ce qui n'existe
que dans la pensée (vg. une notion
mathématique).

\item {\bf Être (verbe)}. — 1. (Sens existentiel).
Syn. de exister : « Dieu est ». — 5.
(Sens attributif ou prédicatif). Copule* exprimant le rapport entre le
sujet$^2$ et l'attribut$^1$ : « Dieu est
infini ».

\item {\bf Être-là}. — Voir Dasein$^2$.

\item {\bf Euclidien (Espace)}. — Math. (Opp. :
lyperespace*). Espace homogène$^2$,
isotrope*, à trois dimensions*, défini
en outre par les 2 postulats de la
ligne droite et le postulat d'Euclide
(par un point hors d’une droite, on
ne peut mener qu'une parallèle à
cette droite).

\item {\bf Eudémonisme} [G. eudaimôn, heureux]. — Mor. À Doctrine morale
selon laquelle le bonheur est le souverain bien.

\item {\bf Euphorie}. — Psycho. Sentiment
agréable de bien-être et de satisfaction, qqfs. pathologique.

\item {\bf Évasion}. — Ps. path. Besoin qu’éprouvent certains malades de fuir (par
Fimagination, la rêverie, etc.) le
réel auquel ils ne peuvent s'adapter.

\item {\bf Événement}. — Voir Fait*.

\item {\bf Évidence, Évident}. — Épist. Manifestement vrai pour tout homme qui
comprend. Dist. : A) l'évidence :
1. rationnelle, fondée sur la raison
pure : « La vérité ne se trouve
presque jamais qu'avec l'évidence »
(Malebranche, R. V., I, 2); « L'évidence de raison consiste uniquement dans l'identité » (Condillac) ; —
2. empirique, not. sensible : « L’Evidence convient à la constatation des
faits, non à celle de l’universalité
% 72
et de la perpétuité des lois qui les
régissent » (Renouvier);

— B) l'évidence : 3, immédiate :
« Un axiome$^\text{1 et 2}$ est une vérité évidente »; — 2. médiate, qui résulte de
la démonstration$^1$ : « L'évidence de
l'intuition n’est pas requise pour les
seules énonciations, mais aussi pour
n'importe quels raisonnements »
(Descartes, Reg., III).

\item {\bf Évocation}. — Psycho. (Syn. : rappel).
Fonction de la mémoire$^4$ par laquelle
les souvenirs sont rappelés à la conscience. On dist qqfs. l'évocation
spontanée et l’évocation volontaire
ou remémoralion.

\item {\bf Évolution}. — Méta. 1. Laio. Suite de
transformations régie par une loi$^5$,
et gén. conçue comme graduelle et
continue : « La formation des mondes
expliquée par voie de développement lent et graduel ou, selon
l'expression, moderne d'évolution »
(Fouillée). — 2. Sir. (spéc. chez
Spencer). Transformation universelle définie surtout par la différenciation$^3$ et l'intégration$^3$ ([v. ces
mots) progressives : « L’évolution
est une intégration de matière, pendant laquelle celle-ci passe d’une
homogénéité indéfinie, incohérente,
à une hétérogénéité définie, cohérente. » (Spencer). — 3. Évolution
créatrice (Bergson) : celle qui, au
lieu de « reconstituer l’évolution
avec des fragments de l’évolué »
[comme chez Spencer), consiste en
un élan* créateur : « L'évolution
est une création sans cesse renouvelée » (E. C., II).

— Biol. 4. Autref. (sens originel) :
« Le mot Évolution, dérivé d'evolvere, à été couramment employé
dans son sens étymologique au
{\footnotesize XVIII}$^\text{e}$ siècle, pour désigner le développement de l'embryon tel que le
% 73
concevaient les préformationnistes »
(M. Caullery; — voir Préformation*):
« Tant de faits rassemblés en faveur
de l’évolution prouvent assez que
les corps organisés ne sont point
proprement engendrés, mais qu'ils
préexistaient  originairement en
petit » (Ch. Bonnet). Cf. Berkeley,
Siris, 233, où le mot anglais est
employé en ce sens. — 5. Auj., transformation, lente ou brusque (v. Mutation*) d’une espèce vivante en une
autre espèce : « En tant que fait,
l’évolution est la seule explication
rationnelle de la nature » (Caullery).

\item {\bf Évolutionisme} (ou  Évolutionnisme).
— Méta. 1. Lato. À Doctrine philosophique d’après laquelle l’évolution, aux sens 1, 2 ou 3, est la loi
générale des êtres (de la matière,
de la vie, de l'esprit, des sociétés).
$->$ Terme le plus souvent appliqué
au système de Spencer. Mais Bergson
distingue de celui-ci « l’évolutionnisme vrai ».

— Biol. 2. Str. (Évol. des espèces)
Syn. de transformisme*.

\item {\bf Exact}. — Épist. Rigoureusement adéquat$^2$ à son objet. Sciences exactes :
les Mathématiques, parce qu'elles
énoncent des propositions vraies sans
approximation aucune. $->$ Dist.
précis*.

\item {\bf Excitation}. — Phol. 1. (Syn. : excitant où stimulus). Phénomène physique ou chimique agissant sur un
être vivant et déterminant de sa part
une réaction$^2$, Se dit spéc. de l’agent$^1$
déterminant une impression$^2$ dans
un organe sensoriel. — 2. Qqfs
syn. d’impression$^2$.

— Ps. phol. 3. Suractivité physique et mentale apparente (vg. dans
l’émotion$^3$.

\item {\bf Exécutif (Pouvoir)}. — Pol. (Syn,
gouvernement). Celui qui est chargé
de faire exécuter les lois.

\item {\bf Exemplarisme}. — Hist. À. Doctrine
qui pose des archétypes$^1$, i. e. des
modèles exemplaires des choses sensibles : « L’exemplarisme platonicien ». — Cf. Archétype$^1$ et Paradigme$^2$.

\item {\bf Exhaustif}. — Log. Qui épuise son
objet « La division$^2$ doit être
exhaustive. »

\item {\bf Exhaustion}. — Math. Méthode d’analyse$^1$ qui consiste à épuiser en qq.
sorte la difficulté en s’approchant
indéfiniment d’une limite (vg. calcul
de $\pi$ par la méthode d’Archimède
en doublant indéfiniment le nombre
des côtés du polygone inscrit).

\item {\bf Existant}. — Méta. 1. I (Nom neutre).
Tout ce qui existe : « Le rapport avec
le monde qui gouverne toutes les
sciences, leur fait chercher l’existant
(das Seiende) lui-même » (Heidegger). — 2. © (Nom masc.) L'homme
en tant qu'il existe$^3$ : « Hamlet était
un existant » (Wahl).

\item {\bf Existence}. — Méta. 1. O Caractère de
ce qui existe* (en tous les sens du
terme) : « L'existence des créatures
est une vraie existence. » (Fontenelle); « L’existence dont nous
sommes le plus assurés et que nous
connaissons le mieux est incontestablement la nôtre » (Bergson, E. C.);
« Réfléchir sur l’existence, c’est déjà
être hors de l'existence » (Wahl);
« Là où règne la raison, la vie devient
existence, et prend sa portée transcendante » (Jaspers). — 2, @ ©
L’existant$^2$ lui-même : « Il appartient
à chaque existence de discerner et de
mettre en œuvre dans la totalité de
l'être cette possibilité dont précisément elle fera son essence » (Lavelle).
% 74

\item {\bf Existential}. — Méta. Chez Heidegger :
qui concerne « l’être dans son ensemble et en tant que tel », et non
seulement l'existence vécue de
l’homme en particulier : « La problématique existentiale tend à mettre
en évidence la structure ontologique* de l’être du Dasein* ».

\item {\bf Existentialisme}. — Méta. (Syn. : philosophie existentielle). A. Mode de
philosopher qui pose le primat de
l’exister$^3$ sur l'essence et qui se
donne pour objet l’analyse de l’existence humaine dans sa réalité concrète et vécue : « L’existentialisme
nous apprend qu'il y a des vues sur
la réalité qui ne peuvent pas être
complètement réduites aux explications scientifiques » (Wahl). $->$
Le terme s'applique aux doctrines
de Jaspers et de Sartre; mais la
philosophie de Heidegger prétend
être existentiale*, non existentielle;
celle de Le Senne a été qualifiée de
spiritualisme existentiel (J. Paumen);
celle de G. Marcel, d’existentialisme
chrétien (mais l’auteur repousse
cette dénomination).

\item {\bf Existentiel}. — Méta. 1. I Qui implique
une affirmation d’existence (au sens
large du terme) : « Le cogito* cartésien est existentiel ». — 2. © Auj.
Qui concerne l'expérience humaine
concrète de l'existence vécue : « La
philosophie existentielle » (syn.
existentialisme*).

\item {\bf Exister}. — Méta. — A) Au sens classique : terme indéfinissable qui implique simplement que l'être dit
existant est posé par nous : 1. soit
comme une réalité : a) nécessaire :
« La nécessité d’être ou d’exister est
comprise en la notion que nous
avons de lui [Dieu] » (Descartes,
Prine., I, 14); « Il n’y a rien de plus
existant ni de plus vivant que lui
% 74
[Dieu] » (Bossuet); — b) contingente
et donnée par l'expérience : $\alpha$. externe : « Il faut conclure qu'il y a
des choses corporelles qui existent »
(Descartes, Méd., VI), ou : $\beta$. interne : « Je suis, j'existe, cela est
certain » (ib., II); — 2. soit comme
une possibilité logique : « Il existe
des fonctions continues dépourvues
de dérivées » (Poincaré); « Il n'existe
pas de cercle carré ».

— B) 3. Auj., ce terme implique
souvent une idée d’expérience concrète et pleinement consciente de
l'existant humain, opp. soit à l’attitude théorique et abstraite, soit à
la banalité de la vie quotidienne :
« Il n’y a guère que les gens malsains
qui se sentent exister » (Biran,
1794); « Je m’accuse de désirer le
libre essor de toutes mes facultés et
de donner un sens complet au mot
exister » (Barrès, Un Homme libre);
« Exister consiste à changer »
(Bergson); « Il y a une chose qui
s'appelle vivre, il y a une autre chose
qui s'appelle exister : j'ai choisi
d'exister » (G. Marcel). Spéc., dans
le lang. existentialiste : « Il n'y à
rien de plus terrible que d’exister
en tant qu'individu » (Kierkegaard).
Chez Heidegger : le mot est qqfs.
écrit ek-sister, i. e. « surgir à la vérité
de l'être » en s’arrachant à la banalité de la vie quotidienne pour
retrouver l'existence authentique$^2$ (v.
ce mot) : « La proposition : l’homme
ek-siste ne répond pas à la question
de savoir si l'homme est réel ou non,
mais à la question de savoir quelle
est l’essence de l’homme. »

$->$ On remarquera que ce dernier sens s'oppose directement à
l'emploi ancien du terme : « Ce n’est
qu'en s'occupant qu'on existe »
(Voltaire); « Le plus lourd fardeau,
c’est d'exister sans vivre » (V. Hugo).
% 75

\item {\bf Exotérique}. — (Ctr. : ésotérique*).
Public, ouvert à tous.

\item {\bf Expectante (Attention)}. — Psycho.
Celle du sujet qui attend [L. exspectare] un signal, un phénomène quelconque.

\item {\bf Expérience}. — Épist. et Crit. A) Expérience immédiate. ([Adj. correspondant : empirique*). 1. © Faculté
de connaître par l'intuition! sensible (expérience externe) ou bien
par l'intuition$^1$ psychologique (expérience interne) avec un minimum
d'interprétation ou d'élaboration :
« La liberté de notre volonté se
connaît sans preuve par la seule
expérience que nous en avons »
(Descartes, Princ., I, 39). — © Auj.,
ce terme est souvent employé pour
désigner ce qu’on éprouve en soi-même immédiatement (expérience
vécue) : « Pour nous, c’est l’expérience qui est le donné essentiel »
(Gusdorf). — 2. @ (Vulg.) Connaissance par l'expérience$^1$ et l'usage
de la vie : « L'expérience se forme
avec l’âge » (Saint-Evremond).

B) M Expérience élaborée. (Adj.
corresp. : expérimental*). 3. O. Connaissance par expérimentation*
« L'expérience est l'unique source
des connaissances humaines » (Cl.
Bernard). — 4. @ Observation$^3$ provoquée par expérimentation* en
vue de contrôler une hypothèse :
« Faire des expériences »; « L’expérience n’est au fond qu’une observation provoquée » (Cl. Bernard);
« L'expérience$^3$ est toujours acquise
en vertu d'un raisonnement précis
établi sur une idée qu’a fait naître
l'observation$^2$ et que contrôle l’expérience$^4$ » (id.).

C) Sens spéciaux. ©. Psycho.
5. Expérience mentale [Trad. all. :
Gedankenexperiment]. Celle qui
« imagine mentalement la variation
% 75
des faits » et qui, selon Mach, constitue l'essentiel du raisonnement$^1$
(cf. Précis, Ph. I, p. 354). Expérience logique : expérience mentale
au second degré où le sujet adopte
une attitude réflerive* à l'égard de
ses propres opérations de pensée
(Piaget). Cf. Précis, Ph. I, p. 356. —
Mor. 6. Expérience morale : a) Intuition affective des valeurs morales
(Scheler); b) Action d’expérimenter
les valeurs morales à l'épreuve de
la vie (Rauh). — Méta. 7. Expérience métaphysique : a) Communion
avec l'absolu par l'intuition$^4$ :
« L'expérience métaphysique se
reliera à celle des grands mystiques »
(Bergson, P. M. II); b) Expérience
de l’être engagé dans l'existence, à
qui son être même apparaît comme
un mystère$^3$ (G. Marcel). — 8. Expérience religieuse : celle dans laquelle
le croyant a le sentiment que « sa
conscience se continue dans une
conscience surhumaine » (James) :
« Une expérience religieuse ayant ses
caractères propres est une chose
qui se constate » (id.).

\item {\bf Expériencialisme}. — Hist. À. Nom
qu’on donne qgis. aux doctrines
[vg. de G. Marcel] de l'expérience$^1$
interne immédiate.

\item {\bf Expérienciel ou Expérientiel}. — Épist.
Qui concerne l'expérience en gén. à
la fois au sens À et au sens B (le sens
de ce terme englobe à la fois celui
d’empirique$^1$ et celui d'expérimental$^1$).

\item {\bf Expérimental}. — Épist, 1. Str. Qui
repose sur l’expérimentation* :
« Sciences expérimentales » s’opp.
en ce sens à « sciences d’observation ». — 2. Lato. Qui repose, soit
sur l’expérimentation*, soit sur des
observations invoquées (mais non
provoquées) en vue de contrôler une
hypothèse : « Quant au raisonnement
% 76
expérimental$^2$, il sera absolument le même dans les sciences
d'observation et dans les sciences
expérimentales$^1$ » (Cl. Bernard). —
3. Latiss. Syn. d'expérienciel* : vg.
« Psychologie$^1$ expérimentale »
(science positive des faits psychiques) opp. à « psychologie$^2$ rationnelle » (étude métaphysique de
l'âme). $->$ Impropre au sens 3.

\item {\bf Expérimentation}. — Épist. Méthode
scientifique qui consiste à provoquer
des observations, faites dans des conditions spéciales, en vue de contrôler
une hypothèse$^3$ : « La physique
moderne repose sur l'expérimentation qui, plus précise que la
simple observation$^2$, réalise volontairement des conditions données
pour voir quels sont les phénomènes
qui se produisent dans ces conditions » (L. de Broglie).

\item {\bf Explicite}. — Épist. (Ctr. : implicite).
Expressément énoncé, ou : pleinement conscient$^2$.

\item {\bf Expliquer} [L. explicare, déplier]. —
Épist. 1. (Au sens traditionnel).
Rendre intelligible$^2$, le plus souvent
en faisant connaître la cause, la loi
ou la raison : « Un fait particulier est
expliqué quand on a indiqué la loi
dont sa production est un cas »
(J. S. Mill). — 2. Auj., on opp. souvent expliquer à comprendre (voir
Comprendre$^4$) : « (Confondre la
compréhension du sens des phénomènes avec leur explication causale »
(Jaspers).

\item {\bf Extase}. — Hist. À. Chez les Néo-platoniciens, spéc. Plotin : union intime
avec l’Un, où l'âme s'anéantit en
Dieu (v. Textes choisis, t. II, p. 346).
— Psycho. 2. État psychique caractérisé par un sentiment de béatitude
et d'union avec l’Absolu : « Dans
% 76
l'extase, la conscience personnelle,
la conscience du moi et du monde
extérieur disparaît presque totalement » (Delacroix). — Ps. path.
3. Syndrome psychopathique caractérisé par la fixité du regard, l’immobilité, la perte de la sensibilité
et un sentiment intense et ineffable
de béatitude. $->$ L'adj. extatique
se rapporte gén. à ce dernier sens.

\item {\bf Extensif}. — Ébpist. Se dit des grandeurs qui peuvent se représenter par
une étendue$^1$ (vg. grandeurs géométriques, la plupart des grandeurs
physiques), opp. aux grandeurs
intensives qui ne peuvent être ainsi
traduites (vg. grandeurs psychologiques : intensité d'une émotion,
d’une sensation).

\item {\bf Extensivité}. — Épist. Caractère de ce
qui est étendu ou implique un certain
sentiment de l'étendue : « L’extensivité des sensations ».

\item {\bf Extension}. — Vuig. 1. Syn. d’étendue$^1$.
— 2, Action d'étendre une assertion ou une dénomination à des
objets auxquels elle ne s’appliquait
pas.

— Log. form. 3. Ensemble des
êtres, objets ou faits auxquels un
concept s’applique, dont il peut être
l’attribut$^1$ (vg. pour « oiseau »
l’ensemble des oiseaux).

\item {\bf Extérieur, Externe}. — Ps. phol. 1.
Sens$^4$ externes : ceux dont les terminaisons nerveuses se trouvent à
la surface du corps (toucher, vue,
ouïe, odorat, etc.), opp. aux sens
internes dont les terminaisons sont
à l'intérieur du corps (cénesthésie*,
sens musculaire, etc.).

— Psycho. et Méta. 2. Qui est en
dehors de la conscience$^1$. Monde
extérieur : le monde sensible. Perception extérieure : celle qui nous fait
connaître le monde extérieur.
% 77 —

— Épist. 3. En histoire : critique
externe des documents, celle qui
porte sur leur forme et a pour but
de déterminer leur authenticité$^1$ et
leur intégrité (opp. critique interne,
celle qui porte sur leur contennu
même, sur les faits attestés).

\item {\bf Extéroceptifs (Sens)}. — Ps. phol.
Ceux qui captent les impressions
venant de l’extérieur$^2$.

\item {\bf Extramondain}. — Méta. Extérieur au
monde : « Cet être unique est supérieur au monde et, pour ainsi dire,
extramondain » (Leibniz).

\item {\bf Extraversion}. — Ps. an. (Opp.
introversion*) Chez Jung : orientation de l'énergie psychique vers
l'extérieur : « L’extraverti pense,
agit par rapport à l’objet. »

\item {\bf Extrêmes}. — Log. form. 1. (Opp.
moyen$^2$). Dans un syllogisme : grand*
et petit* termes. — 2. Les deux
espèces$^2$ opposées d’un même genre$^1$ :
vg. « force » et « faiblesse ».

\item {\bf Extrinsécisme}. — Théol. Tendance à
présenter la foi comme s'imposant à
l’âme uniquement du dehors et par
voie autoritaire, sans tenir compte de
la possible communication de l’âme
avec Dieu.

\item {\bf Extrinsèque}. — Log. (Ctr. :
intrinsèque) Étranger à la nature même
de l’objet ou du fait considéré.

	\end{itemize}
