
	\begin{itemize}[leftmargin=1cm, label=\ding{32}, itemsep=1pt]

\ib{A} — \si{Log. form.} Désigne les propositions universelles*
affirmatives :
« Tous les hommes sont mortels »
est une « proposition en A ».

\ib{Abaque} — [\,L. {\it abacus}, boulier-compteur\,]
— \si{Math.} {\bf 1.} {\it Autref.}, instrument
à compter. — {\bf 2.} {\it Auj.}, tableau
de courbes permettant la détermination
de certaines grandeurs par
le recoupement des tracés ({\it cf.} {\it Précis},
Ph. II, p. 135; Sc. et M., p. 253).
$->$ Ce mot est du {\it masculin}.

\ib{Abduction} — \si{Log. form.} Syllogisme
dont la mineure n’est que probable.

\ib{Aberration} — \si{Psycho.} {\bf 1.} Se dit de
toute activité détournée de son but
naturel : « Les aberrations de l’instinct ».
— \si{Vulg.} {\bf 2.} Égarement : « Il
y a des moments d’aberration dans
les multitudes » (Lamartine).

\ib{Aboulie} — [\,G. {\it a}  privatif et {\it boulesthai},
vouloir\,] — \si{Ps. path.} Impuissance
anormale de la volonté, soit par
insuffisance de « l'incitation psychique »
(Baruk), soit par incapacité
de la pensée réfléchie de passer à
l'acte : « Le mot {\it aboulie} ne désigne
pas la suppression d’une action quelconque :
il désigne exactement la
suppression de l’action {\it réfléchie} »
(Janet).

\ib{Abréaction} — \si{Ps. an.} Décharge émotionnelle
permettant au sujet de se
libérer d’un choc ancien qui n'avait
pu aboutir à une réaction satisfaisante.
— {\it cf.}  {\it Catharsis}.

\ib{Absolu} — [\,L. {\it ab}, {\it solutum}, séparé de\,]
— \si{Crit.} et \si{Méta.} (Ctr. : {\it relatif}\,*).
{\bf 1.} Qui a en soi-même sa raison
d’être; qui, par suite, n’a besoin, ni
pour être conçu, ni pour exister,
d'aucune autre chose; qui est en*
soi et par* soi : « Dieu est l'Être
absolu ». — {\it Spéc}., qui a en soi sa
raison d’être « en tant que parfait,
achevé, total » (Hamilton). {\it Chez
Schelling} : « l'Absolu » est l’Être
universel en qui matière$^2$ et forme$^2$,
sujet$^4$ et objet$^5$, pensée et nature
sont identiques. — {\bf 2.} Qui est indépendant
de tout point de repère
conventionnel: « Mouvement absolu ».
{\it Espace absolu} indépendant des
objets qui le remplissent. {\it Temps
absolu} : indépendant des phénomènes
qui s’y passent. — {\bf 3.} (Syn.: {\it a priori}\,*).
Indépendant de l'expérience : « Des
vérités absolues, c’est-à-dire indépendantes
de la réalité » (Cl. Bernard).
— \si{Vulg.} {\bf 4.} Qui ne comporte
aucune limite, aucune restriction
ni réserve : « Pouvoir absolu »
« Nécessité absolue ».

 — \si{Math.} {\bf 5.} La {\it valeur absolue} d’un
nombre algébrique est la valeur
arithmétique de ce nombre considéré
indépendamment de son signe.

\ib{Abstraction} — \si{Psycho.} {\bf 1.} \fsb{S. abstr.} Opération
intellectuelle qui consiste à
abstraire. — {\bf 2.} \fsb{S. concr.} Produit de cette
opération : « Un point géométrique est
une abstraction de l'esprit » (Voltaire).
% 12

\ib{Abstraire} — [\,L. {\it abs. trahere}, retirer de\,]
— \si{Psycho.} Isoler, pour le considérer
à part, un élément d'une représentation*
qui n'est pas donné séparément
dans la réalité : « Abstraire,
c’est intellectualiser ou spiritualiser
les données sensibles en les dématérialisant,
en laissant tomber les
particularités individuelles » (A.
Marc).

\ib{Abstrait} — \si{Psycho.} et \si{Log.} {\bf 1.} (Ctr. :
{\it concret}). Qui constitue une abstraction$^2$.
Une idée est « plus ou moins
abstraite » qu'une autre, selon que
sa compré\-hension$^2$ est plus ou
moins restreinte que celle de cette
autre. $->$ Ne pas dire qu’un phénomène
psychique est « abstrait »
sous prétexte qu'il n’est pas perceptible
par les sens (voir {\it Concret}\,*). —
\si{Épist.} {\bf 2.} {\it Sciences abstraites}. Expression
équivoque qui désigne : {\it a)} soit,
{\it dans le lang. courant}, les Mathématiques,
la Physique mathématique,
qqfs. même la Logique et la Métaphysique;
— {\it b)} soit, {\it chez Aug.
Comte}, les sciences qui ont pour
objet « la découverte des lois qui
régissent les diverses classes de phénomènes »,
opp. aux {\it sciences concrètes}
qui appliquent ces lois « à l’histoire
effective des différents êtres existants » :
en ce sens, même la Physiologie
et la Sociologie sont abstraites;
— {\it c)} soit, chez {\it Spencer}, celles qui
traitent « des formes sous lesquelles
les phénomènes nous apparaissent »
(Logique et Mathématiques), opp.
aux {\it sciences abstraites-concrètes} qui
étudient « les phénomènes eux-mêmes
dans leurs éléments » (mécanique,
physique, chimie), et aux
{\it sciences concrètes} qui traitent des
phénomènes « considérés dans leur
ensemble » (astronomie, géologie,
biologie, psychologie, sociologie).

— \si{Esth.} {\bf 3.} {\it Art abstrait} (Ctr : {\it figuratif}) : celui qui vise à produire
l'effet esthétique par la seule combinaison des formes ou des couleurs
sans chercher à reproduire la réalité
sensible.

\ib{Absurde} — \si{Log.} {\bf 1.} Contradictoire$^3$.
$->$ {\it Dist}. faux : le faux peut ne pas
être absurde. — {\bf 2.} {\it Démonstration
par l'absurde} : celle qui démontre
une proposition en prouvant que sa
contradictoire$^1$ est absurde ou ({\it p. e.}
en Math.) contradictoire$^1$ avec l’hypothèse$^2$. — {\bf 3.} {\it Réduction à l'absurde} :
opération qui consiste à tirer d’une
proposition une conséquence absurde,
ce qui montre la fausseté de cette
proposition.

— \si{Méta.} {\bf 4.} Dans le lang. philosophique contemporain, le sens de
ce terme a été étendu par les existentialistes jusqu’à désigner, soit la
pure {\it facticité}\,$^2$ ou l'{\it étrangeté} de l'univers (Kierkegaard, Heidegger, Camus),
soit le {\it non-sens} (Sartre), la
condamnation à l'échec (Jaspers) ou
le {\it mystère}\,$^3$ (G. Marcel) de l'existence
humaine : « Cette épaisseur et cette
étrangeté du monde, c’est l'absurde »
(Camus).

\ib{Académie} — \si{Hist.} École philosophique de Platon.
— {\it Nouvelle Académie} :
école probabiliste* d’Arcésilas, Carnéade, etc. D'où, qqfs., au
{\footnotesize XVII}$^\text{e}$ siècle : « les Académiciens »
$=$ les sceptiques.

\ib{Acceptation} — \si{Ps. an.} Attitude qui
consiste à résoudre par une intégration$^2$ psychique, {\it not.} par la socialisation
de sa personnalité, un conflit
opposant le sujet$^5$ à une situation
donnée.

\ib{Accident} — \si{Méta.} {\bf 1.} (Opp. : {\it essence}*).
Ce qui peut être modifié ou supprimé sans que la chose elle-même
change de nature ou disparaisse :
« Le poids, la couleur et tous accidents sensibles » (Montaigne); « Les
déterminations d’une substance qui
ne sont rien d’autre que de ses manières particulières d'exister,
s’appellent {\it accidents} » (Kant, {\it R. pure},
Analyt., II, 2, 3, 1$^\text{re}$ analogie de l’expérience). {\it cf.}  {\it Forme}$^1$.

— \si{Log. form.} {\bf 2.} {\it Conversion par
accident} : conversion* de l’universelle affirmative en particulière
affirmative. — {\bf 3.} {\it Sophisme de l’accident} : celui qui consiste à prendre
un accident$^1$ pour une qualité essentielle ({\it p. e.} définir la matière$^4$ par
l’état solide).
%ACC — j — ACT

\ib{Accommodation} — \si{Soc.} Processus
social conduisant à la cessation des
conflits entre individus ou entre
groupes.

\ib{Accoutumance} — \si{Biol.} Modification
contractée par un être vivant sous
l'influence d’un agent extérieur et
qui fait que celui-ci ne l’affecte plus
comme au début.

\ib{Acculturation} — \si{Soc.} Terme employé
par les sociologues américains pour
désigner les changements qui s’effectuent dans la civilisation d’un groupe
mis en contact avec un autre, surtout si ce dernier est de civilisation
supérieure.

\ib{Achromatopsie} — [\,G. a privatif; {\it chrôma},
couleur; {\it opsis}, vision\,] — \si{Ps. phol.}
Anomalie de la vision dans laquelle
le sujet ne perçoit pas les couleurs.

\ib{Acmè} — [\,mot grec\,] — Point culminant
(de la vie, d’un désir, etc.).

\ib{Acquis} — \si{Biol.} {\bf 1.} (Ctr. : {\it congénital}\,*,
{\it inné}\,*). {\it Caractères acquis} : ceux qui
apparaissent chez l’être vivant au
cours de son existence ({\it opp}. à ceux
qu'il a en naissant).

— \si{Psycho.} {\bf 2.} (Ctr. : {\it immédiat}\,$^2$,
{\it inné}\,*). Qui est le fruit de l’expérience$^2$ ou d’un travail mental,

\ib{Acroamatique} — [\,G. {\it acroasthai}, entendre\,] — \si{Hist.} S’est dit d’abord des
écrits d’Aristote réservés à ses disciples. D’où {\it ext}. : ésotérique*.

\ib{Acte} — \si{Vulg.} {\bf 1.} Tout exercice d’un
pouvoir matériel ou spirituel : « Un
acte d'attention »; « Un acte moral ».
{\it cf.}  {\it manqué}\,*.

— \si{Méta.} {\bf 2.} (Ctr.: {\it puissance}\,$^2$). {\it Chez
Aristote} : l'être pleinement réalisé
(opp. à l'être en voie de devenir) :
{\it p. e.} la plante est l'acte de la graine.
{\it Acte pur} : Dieu, parce qu’il est soustrait au devenir
(voir {\it Forme}\,$^1$). —
 {\bf 3.} {\it Chez Lavelle} : l'être lui-même
considéré dans l’unité de son action:
« L'acte n’est point une opération
qui s'ajoute à l'être, mais son
essence même. »

\ib{Action} — \si{Vulg.} {\bf 1.} \fsb{S. abstr.} Activité$^1$, exercice d’un pouvoir quelconque
« L'action de la volonté ». — {\bf 2.} \fsb{S. concr.}
Ensemble de gestes coordonnés
en vue d'une fin : « Une bonne
action ».

— \si{Mor.} et \si{Méta.} {\bf 3.} (Ctr. : {\it spéculation, théorie}. Syn. {\it pratique}\,$^3$).
Ensemble de tous nos actes et principalement de nos actes volontaires;
conduite humaine. — {\bf 4.} {\it Chez Maurice Blondel} : « L'action est la
synthèse du vouloir, du connaître et de
l'être. » {\it cf.}  Le Roy ({\it R. M. M.}, 1901) :
« Il faut séparer plusieurs sens du
mot {\it action}. Il y a l’action {\it pratique}\,$^1$,
l’action {\it discursive} et l'action {\it profonde}. La première engendre le sens
commun; la seconde règle la science;
c’est la troisième qui doit servir de
critère en philosophie. »

— \si{Math.} {\bf 5.} En Mécanique : produit de l'énergie par le temps.
{\it Principe de moindre action} : principe
selon lequel l’action$^5$ est toujours
minimum. — {\it cf.}  {\it Quantum}* et
{\it Réaction}\,$^1$.
%14

\ib{Activisme} — \si{Crit.} \fsb{S. norma.} Doctrine qui, 
sans accepter les conclusions du
pragmatisme*, fait de la vérité
« une affaire de vie et d’action plutôt
que de pur intellect » (Eucken).

\ib{Activité} — \si{Vulg.} 1 Tout exercice
d’une force, d’un pouvoir quelconque : « L'activité sociale ».

— \si{Psycho.} {\bf 2.} {\it Str.} (Opp. : {\it affectivité}\,$^2$ et {\it connaissance}\,$^1$) Ensemble
des phénomènes psychiques tendant
à l’action$^2$, tels que tendance, instinct, habitude, désir, volonté. —
 {\bf 3.} {\it Lato}. (Ctr. : {\it passivité}). Aspect
très général de la vie psychique qui
se révèle aussi bien dans les faits
d’affectivité et de connaissance que
dans les précédents : « Je suis actif
quand je juge » (Rousseau).

\ib{Actualisme} — \si{Hist.} \fsb{S. norma.} Doctrine ({\it not}.
de G. Gentile) selon laquelle toute
réalité est immanente à l'acte créateur et libre de l'Esprit; d’où résulte
que l'homme doit se dégager de
l'individualité pour s'intégrer au
« Moi » absolu.

\ib{Actuel} — \si{Vulg.} {\bf 4.} (Opp. : {\it passé} ou
{\it futur}). Présent : « L'époque actuelle » ;
« La pensée religieuse ne s'exerce que
dans l’actuel » (G. Marcel).

— \si{Méta.} {\bf 2.} (Syn. : {\it formel}\,$^1$. Ctr. :
{\it potentiel, virtuel}). Qui est en acte$^2$,
pleinement réalisé : « Tout ce qui est
actuel, peut être conçu comme possible » (Leibniz).

— \si{Phys.} {\bf 3.} {\it cf.}  Énergie*.

— \si{Théol.} {\bf 4.} Grâce actuelle : celle
que Dieu accorde comme secours
momentané (opp. {\it grâce habituelle} ou
{\it sanctifiante} : celle qui réside dans
l’âme de façon permanente).

\ib{Acuité sensorielle} — \si{Ps. phol.} Finesse,
pouvoir de discrimination* des sens :
« L'acuité tactile ».

\ib{Adaptation} — \si{Phol.} {\bf 1.} Ensemble des
mouvements par lesquels un organe
se prête à sa fonction.

— \si{Biol.} {\bf 2.} Ensemble des modifications que subit ou effectue un être
vivant pour se mettre en harmonie
avec ses conditions d'existence.

— \si{Soc.} {\bf 3.} Équilibre de l’accommodation* et de l’assimilation* (Piaget).

\ib{Adéquat} — \si{Ps. phol.} {\bf 1.} Excitant adéquat d'un organe : celui qui agit
normalement sur cet organe ({\it p. e.}
pour la vue, la lumière).

— \si{Crit.} {\bf 2.} Qui correspond parfaitement à son objet. — {\bf 3.} {\it Chez
Spinoza}, « idée adéquate » : celle qui,
considérée en elle-même, a toutes
les propriétés intrinsèques de l’idée
vraie ({\it Éth}., II, déf. 4).

\ib{Adéquation} — Correspondance exacte.
Les Scolastiques* définissaient la
vérité « l'adéquation de l’objet et de
l'entendement ».

\ib{Adjectif} — \si{Méta.} {\it Chez F. H. Bradley} :
caractère du {\it what} (prédicat) qui
vient « s'ajouter » au {\it that} (sujet concret). — Voir {\it Précis}, Ph. II, p. 463.

\ib{Adventices (Idées)} — \si{Hist.} {\it Chez Descartes} : représentations$^1$ qui nous
arrivent par les sens : « Entre mes
idées$^4$, les unes me semblent être
nées avec moi; les autres, être étrangères et venir du dehors; et les
autres, être faites et inventées par
moi-même » ({\it Méd}., III); les premières sont les {\it idées innées} ; les
secondes, les {\it idées adventices} ; les
troisièmes, les {\it idées factices}.

\ib{Affect} — \si{Psycho.} État affectif\,*
élémentaire.

\ib{Affectif} — \si{Psycho.} Les « phénomènes
affectifs » sont les phénomènes de la
sensibilité\,$^3$, considérés simplement
en tant qu'ils affectent notre moi
% 15
d’une certaine manière ({\it p. e.} agréable,
désagréable, plaisir, douleur, sentiments, émotions, etc.) : « Beaucoup
de sensations représentatives ont un
caractère affectif » (Bergson, {\it D. I.}).

\ib{Affection} — \si{Méta.} {\bf 1.} {\it Autref}. (not.
chez Spinoza), Manière d’être, modification d’un être considéré comme
passif : « Les affections de la haine,
de la colère, de l’envie, etc., considérées en soi, résultent de la même
nécessité de la nature que les autres
choses singulières » ({\it Eth}., III).

— \si{Psycho.} {\bf 2.} (\si{Vulg.}) Sentiment
tendre : « Avoir de l’affection pour
qqn ». — {\bf 3.} État affectif. — {\bf 4.} {\it Chez
Maine de Biran} : « affection simple »
(syn. : {\it affectivité pure}), état « purement sensitif » auquel l’homme se
trouve réduit quand il n’a encore ou
qu’il n’a plus aucune conscience de
sa personnalité.

\ib{Affectivité} — \si{Psycho.} {\bf 1.} Ensemble
des phénomènes affectifs*. — {\bf 2.}
Fonction psychique correspondant
aux phénomènes affectifs*. — {\bf 3.}
{\it Affectivité pure} : {\it cf.}  {\it Affection}$^4$.

\ib{Afférent} — Voir Centre*.

\ib{Affirmation} — \si{Log.} (Ctr. : négation).
 {\bf 1.} \fsb{S. abstr.} Acte d’affirmer*. — {\bf 2.} \fsb{S. concr.} Produit de cet acte ;
proposition affirmative : « Une affirmation ».
$->$ Dist. assertion*, et {\it cf.}  {\it Assertorique}* et {\it Catégorique}*.

\ib{Affirmer} — \si{Psycho.} et \si{Log.} Poser
un rapport ou une existence comme
vrais : « Une proposition est rarement
affirmée avant d'avoir été niée »
(Piaget).

\ib{A fortiori} — \si{Log.} A plus forte raison.
{\it Raisonner a fortiori}, c’est raisonner
du plus au moins, de l’universel au
particulier, du général au spécial :
{\it p. e.} « Si la médisance est condamnable,
 la calomnie, qui est une médisance
doublée d’un mensonge, l’est aussi ».

\ib{Agapè} — [\,mot grec\,] — L'amour-charité\,$^1$
(opp. {\it Érôs}\,$^2$).

\ib{Agent} — \si{Phys.} {\bf 1.} Force considérée
comme une forme spéciale de l’énergie : « Les agents physiques », la
lumière, la vapeur, l'électricité, la
chaleur, etc.

— \si{Mor.} {\bf 2.} {\it Agent moral} : l'être
raisonnable en tant qu’il est soumis
à la loi morale.

— \si{Psycho.} {\bf 3.} Voir {\it Intellect}\,$^2$.

— \si{Méta.} {\bf 4.} Tout être en tant
qu’il exerce une action$^1$ : « Bien
que l'agent et le patient soient souvent fort différents, » (Descartes).

\ib{Agnosie} — [\,G. {\it a} priv. et {\it gnôsis}, connaissance\,] — \si{Ps. path.} Amnésie
perceptive consistant dans l’ « incapacité de reconnaître les objets ou les
symboles usuels » (Lalande). Elle
peut être visuelle {cécité psychique),
auditive (surdité psychique) ou tactile. $->$ Dist. apraxie*.

\ib{Agnosticisme} — [\,G. {\it agnôsios}, inconnaisable\,]
— \si{Crit.} \fsb{S. norma.} Doctrine selon
laquelle le fond des choses est
inconnaissable pour l'esprit humain.
$->$ {\it Dist}. scepticisme*, et {\it cf.}  {\it Relativisme}* et {\it Subjectivisme}*.

\ib{Agoraphobie} — [\,G. {\it agora}, place publique,
et {\it phobos}. peur\,] — \si{Ps. path.} Peur
maladive des grands espaces.

\ib{Agraphie} — [\,G. {\it a} priv. et {\it graphein},
écrire\,] — \si{Ps. path.} Apraxie* consis-
tant dans la perte des mouvements
de l'écriture, indépendamment de
toute paralysie.

\ib{Agréable} — Voir {\it Plaisir}\,*.

\ib{Agressivité} — \si{Ps. an.} Tendance à
l’attaque et à la destruction qui est,
selon Freud, une des pulsions* fondamentales de l'homme.
% 16

\ib{Airain (Loi d')} — \si{Éc. pol.} Loi (ainsi
nommée par Lassalle, 1864) selon
laquelle le salaire du travailleur se
réduit fatalement à ce qui lui est
nécessaire pour vivre.

\ib{Aleph} — \si{Math.} Nom de la première
lettre de l'alphabet hébraïque ($\aleph$)
qui, dans la Théorie des ensembles*,
symbolise le nombre transfini*
« Le nombre des opérations à faire
est infini, il est même plus grand
que aleph-zéro » (H. Poincaré). —
Voir {\it Précis}, Ph. II, p. 93; Sc., p. 210;
M., p. 210 et 443.

\ib{Alexie} — Voir {\it Cécité}\,*.

\ib{Algèbre} — \si{Épist.} Science du nombre
considéré sous sa forme la plus générale, indépendamment de ses valeurs
particulières, et où l’on étudie surtout les relations\,$^2$ entre ces valeurs.

\ib{Algiques (Sensations)} — [\,G. {\it algos}, douleur\,] — \si{Psycho.} Les sensations de
douleur\,$^2$, considérées comme spécifiques\,$^2$ ({\it cf.}  {\it Précis}, Ph. I, p. 386).

\ib{Algophilie} — [\,G. {\it algos}, et {\it philia}, amour\,]
— \si{Ps. path.} Recherche ({\it gén}. pathologique) de la douleur\,$^2$.

\ib{Algorithme} — [\,de Al Korismi, mathématicien arabe du {\footnotesize IX}$^\text{e}$ siècle\,] — \si{Épist.}
Système de symboles* permettant
d’effectuer des opérations : {\it p. e.} le
langage algébrique.

\ib{Aliénation} — [\,L. {\it alienus}, étranger\,] —
\si{Ps. path.} {\bf 4.} État de l’aliéné, {\it i. e.} de
l’anormal que ses troubles psychiques rendent « étranger » à la
vie sociale,

— \si{Méta.} {\bf 2.} {\it Chez Hegel} : état de la
conscience qui, en tant qu'opposition du sujet et de l’objet,
se dépouille de son moi et en fait une
chose: l'esprit devient ainsi {\it être
pour-soi}\,*, puis {\it nature} : « La nature,
l'esprit aliéné, n’est dans son propre
{\it être-là}\,*, que l’éternelle aliénation de
sa propre subsistance » (Hegel).

— \si{Soc.} {\bf 3.} {\it Chez les hégéliens} : projection de l’activité propre de l'homme
en une force étrangère à lui, sous
forme soit de représentations religieuses (Feuerbach), soit d'une
puissance économique échappant à
son contrôle, mais qui est le résultat
de son travail (K. Marx).

\ib{Altérité} — Caractère de ce qui est
{\it autre}* aux sens 1 ou 2 : « Quoi!
l'âme ne connaît pas elle-même sa
distinction [\,d'avec Dieu\,] ou, comme
parle cet auteur [\,Ruysbroek\,], son
{\it altérité} ? » {Bossuet).

\ib{Alternative} — \si{Vulg.} {\bf 1.} Situation
dans laquelle on n’a le choix qu'entre
deux partis possibles. $->$ Il est
{\it incorrect} de dire : « Avoir le choix
entre deux alternatives ».

— \si{Log.} {\bf 2.} Ensemble de deux
propositions dont l’une est vraie si
l’autre est fausse, et inversement
(schéma : « de deux choses l’une :
ou A est B, ou C est D »), {\it spéc}. de
deux propositions contradictoires\,$^1$
(schéma : « ou tout A est B, ou
quelque A n’est pas B »}. — {\bf 3.} {\it Principe de l'alternative} : « Deux propositions contradictoires$^1$ ne peuvent
être toutes deux fausses » ({\it cf.}  {\it Contradiction}*).

\ib{Altruisme} — \si{Psycho.} et \si{Mor.} (Ctr. :
{\it égoïsme}). Mot créé par A. Comte
pour désigner les sentiments désintéressés qui s'opposent à l’égoïsme.

\ib{Ambiguîté} — \si{Log.} {\bf 1.} \fsb{S. objec.} Équivoque* (en
parlant des termes).

— \si{Méta.} {\bf 2.} \fsb{S. subje.} Dans le {\it lang. existentialiste} : condition de l'être humain
qui est « manque d’être », mais pour
qui « il y a une manière d’être de
ce manque, qui est l'existence »
(S. de Beauvoir).

\ib{Ambivalence} — {\bf 1.} Dualité de sens
opposés de certains termes, {\it p. e.} en
latin, {\it altus} (à la fois : {\it profond} et
{\it élevé}), {\it sacer} ({\it sacré} et {\it maudit}). Par
{\it anal}., \si{Ps. an.} Dualité de sens de
certains symboles du rêve.

— {\bf 2.} \si{Ps. an.} « Tendance à
éprouver un phénomène psychologique à la fois sous deux aspects
contraires, à affirmer et nier successivement un même fait, à exprimer en
même temps deux sentiments opposés » (Piéron.)

— {\bf 3.} \si{Soc.} Double aspect de
certaines valeurs qui reflètent à la
fois la société existante ({\it spéc}. bourgeoise) et l’accroissement du pouvoir
de l’homme : « Toutes les valeurs
culturelles du monde capitaliste sont
ambivalentes » (H. Lefebvre).

\ib{Ame} — \si{Méta.} {\bf 4.} Principe de la vie et
de la pensée [\,L. {\it anima}\,] : {\it p. e.} selon
Aristote, les végétaux ont une {\it âme
nutritive}, les animaux ont de plus
une {\it âme sensitive} et une {\it âme motrice},
l’homme seul possède une {\it âme pensante}. {\it cf.}  Voltaire : « Nous appelons
âme ce qui anime ». — {\bf 2.} Substance
immatérielle qui, selon les spiritualistes*,
est le principe de la vie psychique : « Je connus de là que j'étais une
substance dont toute l'essence ou la
nature n’est que de penser..., en sorte
que ce moi, c'est-à-dire l’âme [\,latin :
{\it mens}\,] par laquelle je suis ce que
je suis, est entièrement distincte du
corps » (Descartes, {\it Méth}., IV).

— \si{Psycho.} {\bf 3.} (Syn. : {\it conscience}\,$^2$,
esprit$^6$). Ensemble des faits psychiques, indépendamment de toute
idée métaphysique : « Un état
d'âme »; « La psychologie est l'étude
de l'âme ou de l'esprit » (Burloud).
$->$ Bien dist. tous ces sens
tandis qu'au sens 3, {\it âme} et {\it esprit}
s'identifient, certains auteurs, surtout allemands, se sont plu, à la suite
% 17
de Nietzsche, à opposer l’âme (all. :
{\it Seele}) comme principe de vie (sens 1)
à l'{\it esprit} (all. : {\it Geist}), celui-ci étant
dit « parasitaire » (L. Klages),
extérieur au monde et à la conscience. Voir {\it Esprit}\,$^7$.

— \si{Soc.} {\bf 4.} {\it Ame des foules, âme
collective}. Ensemble des faits de
psychologie sociale. {\it Spéc}. \si{Ps. an.} :
« Dans l'humanité collective, il y
avait quelque chose comme une âme
collective, à la place de notre conscience individuelle qui n’émergea
que graduellement au cours de
l'évolution » (Jung).

— \si{Hist.} {\bf 5.} {\it Ame du monde}. Principe  {\it gén.} spirituel (matériel
cependant chez les Stoïciens) qui, selon
certains philosophes, joue par rapport à l'univers, comme principe
de vie et d'unité, le même rôle que
l'âme\,$^1$ par rapport au corps.

\ib{Amnésie} — [\,G. {\it a} priv. et {\it mnêmé}, mémoire\,] — \si{Ps. path.} Disparition
totale ou partielle de la mémoire.
— {\it Amnésies de fixation} (syn. : {\it de
conservation}) : celles où la faculté de
retenir elle-même est abolie ({\it cf.}  {\it Continu}\,$^3$).
— {\it A. d'évocation} (syn. : {\it de
reproduction}) : celles où le sujet conserve ses souvenirs, mais ne peut les
rappeler à volonté. — {\it A. lacunaires} :
amnésies partielles d’évocation portant sur une ou plusieurs périodes
déterminées de la vie du sujet ({\it cf.} 
{\it Périodique}* et {\it Rétrograde}*). — {\it A.
systématisées} : amnésies partielles
d'évocation portant sur tous les souvenirs relatifs à un ordre d'idées
déterminé. — {\it A. de reconnaissance} :
celles qui consistent dans un trouble
de la reconnaissance*, soit des objets
extérieurs (cf. {\it Agnosie}*), soit des
idées (cf. {\it Réminiscence}*). — {\it A. de
localisation} : celles où un souvenir
récent est pris pour un souvenir
ancien ou inversement.
% 18

\ib{Amoral} — \si{Mor.} {\bf 1.} Qui ne comporte
pas d'appréciations morales : « La
science est amorale ». — {\bf 2.} En parlant d’une personne : qui manque
de sens moral.

\ib{Amoralisme} — \si{Mor.} {\bf 1.} \fsb{S. norma.} Doctrine qui
rejette tout point de vue moral. —
 {\bf 2.} \fsb{S. posit.} État de l'être amoral$^2$.

\ib{Amour} — \si{Psycho.} {\bf 1.} {\it Lato}.
Mouvement de la sensibilité qui nous porte
vers un être ou un objet et qui s’accompagne de la pensée de cet être
ou de cet objet : « amour du prochain » ({\it spéc}. en parlant des
sentiments de famille : « amour maternel »); « amour du vrai » ; « amour de
Dieu » : « Il y a deux principales
espèces d'amour, un amour de bienveillance, et un amour qu'on peut
appeler d’union » (Malebranche).
\si{Théol.} {\it Pur amour} : amour exclusif
de Dieu indépendant du désir d’être
heureux et du souci du salut (v.
{\it Quiétisme}*). — {\bf 2.} {\it Str}. L’inclination
sexuelle. {\it cf.}  {\it Érôs}\,$^2$.

\ib{Amour-propre} — \si{Psycho.} {\bf 1.} Autref.
(p. e. {\it chez Pascal, La Rochefoucauld}),
amour de soi, égoïsme. — {\bf 2.} {\it Auj}.,
sentiment de la valeur personnelle.

\ib{Amphibologie} — \si{Log.} Équivoque*
(en parlant des propositions).

\ib{Analgésie} — \si{Ps. phol.} Disparition
de la sensibilité à la douleur.

\ib{Analogie} — \si{Épist.} {\bf 1.} {\it Autref}. (not.
chez les mathématiciens grecs, et
aussi chez Cournot), rapport quantitatif, proportion mathématique. —
 {\bf 2.} {\it Auj}., rapport qualitatif, ressemblance : on « raisonne par analogie »
quand on conclut d’une ressemblance constatée à une ressemblance
non constatée.

\ib{Analogon} — [\,mot grec\,] — \si{Psycho.} Représentant,
substitut d'un objet
% 18
« Dans la conscience d'image, nous
appréhendons un objet comme {\it analogon} d’un autre objet » (Sartre).

\ib{Analyse} — [\,G. {\it analuein}, résoudre\,] —
\si{Math.} {\bf 1.} {\it Autref}., méthode de résolution des problèmes qui consistait
à supposer d’abord le problème résolu : « L'analyse des anciens »
(Descartes) ; {\it cf.}  {\it }Analytique$^1$. — {\bf 2.}
{\it Auj}., l'algèbre ({\it spéc}. calcul des fonctions ou calcul infinitésimal).

— \si{Log.} {\bf 3.} (Ctr. {\it }synthèse\,$^1$).
Décomposition d’un tout en ses éléments$^1$; réduction d'un donné
complexe à ses composants simples. $->$ Dist. division$^1$.

— Ps. métr. {\bf 4.} {\it Analyse factorielle} : forme d'analyse consistant à
isoler, dans un ensemble de variables,
des facteurs* qui permettent d’exprimer la valeur\,$^6$ de celles-ci par une
fonction linéaire de ces facteurs.

\ib{Analytique} — \si{Math.} {\bf 1.} {\it Méthode analytique} (Ctr. : {\it synthétique}$^1$) : celle
qui consiste : {\it a)} dans un problème,
à supposer le problème résolu et à
remonter de là aux principes de la
solution ; {\it b)} dans un théorème, à
supposer la conclusion démontrée et
à remonter de là à une proposition
déjà établie. — {\bf 2.} {\it Géométrie analytique} : voir {\it Géométrie}$^2$.

— \si{Log.} (Ctr. : {\it synthétique}). {\bf 3.} Qui
repose sur l’analyse$^3$. — {\bf 4.} {\it Proposition analytique} : celle où l’attribut
est nécessairement$^1$ compris dans le
sujet : {\it p. e.} « Les corps sont étendus ».

— \si{Hist.} {\bf 4.} {\it Chez Kant} : « analytique transcendantale », partie de
la Logique$^5$ transcendantale$^2$ qui
consiste dans « la décomposition de
notre connaissance a priori dans les
éléments de la connaissance pure de
l’entendement », {\it i. e.} les catégories*.

\ib{Anamnèse} — [\,G. {\it anamnèsis}, rappel\,] —
{\bf 1.} \si{Psycho.} Remémoration*. — {\bf 2.} Ps.
path. Rappel des phénomènes antérieurs à une période donnée de la
maladie.

\ib{Anarchie} — [\,G. {\it an} priv. et {\it arché}, commandement\,] — \si{Soc.} {\bf 1.} \fsb{S. posit.} État
d’une société inorganisée ou désorganisée qui n’a pas ou n’a plus de
gouvernement.

— \si{Pol.} et Éc. \si{Soc.} \fsb{S. norma.} {\bf 2.} Doctrine
selon laquelle la société devrait
rejeter tout appel à la contrainte et
se passer de gouvernement, — {\bf 3.}
Doctrine ({\it p. e.} de Proudhon) selon
laquelle le « gouvernement des
hommes » (politique) doit être remplacé par « l'administration des
choses » (économique). $->$ Aux
sens 2 et 3, dire plutôt : {\it anarchisme}.

\ib{Anarthrie} — [\,G. {\it an} priv. et {\it arthron},
articulation\,] — \si{Méd.} Trouble purement moteur de l'articulation des
mots qu’on distingue qqfs (D$^\text{r}$ Pierre
Marie) de l’aphasie*.

\ib{Anatomie} — [\,G. {\it anatomé}, dissection\,]
— \si{Épist.} Étude de la structure des
organes des êtres vivants. {\it Anatomie
fine} : cf.  {\it Histologie}*.

\ib{Anesthésie} — [\,G. {\it an} priv. et {\it aisthèsis},
sensation\,] — \si{Ps. phol.} Disparition
totale ou partielle de la sensibilité\,$^2$,
{\it spéc}. de la sensibilité\,$^2$ tactile. $->$
Les anesthésies visuelles s’appellent
{\it amaurose} (cécité totale), {\it amblyopie}
(cécité partielle) ou {\it achromatopsie}*;
la disparition de la sensibilité à la
douleur s’appelle {\it analgésie}*.

\ib{Angélisme} — \si{Méta.} ({\it Péj}.) Terme
employé par J. Maritain pour désigner l'attitude philosophique qui
fait de l’homme un « ange », {\it i. e.} un
être désincarné.

\ib{Angoisse} — {\bf 1.} \si{Psycho.} Malaise fait à
la fois d’une crainte sans objet bien
déterminé et d’une sensation physique de constriction (« cœur serré ») :
« La conscience de l’angoisse est la
conscience d'une ambivalence$^2$ instinctive » (J. Favez-Boutonier). {\it cf.} 
{\it Anxiété}\,*

— {\bf 2.} \si{Méta.} {\it Chez les existentialistes} :
état d'inquiétude qui résulte, chez
l'existant$^2$ humain, soit de sa liberté
et du pressentiment de la faute possible (Kierkegaard), soit de son
insécurité sous la menace du Néant
(Heidegger) : « L'angoisse est la
saisie réflexive de la liberté par
elle-même » (Sartre).
% 19

\ib{Anima} — [\,mot latin\,] — \si{Ps. an.} {\it Chez
Jung} : image archétype$^2$ de l’âme
chez l’homme, qui représente sa
féminité inconsciente ({\it p. e.} la Kundry
de Parsifal, la Béatrice de Dante).

\ib{Animal} — {\bf 1.} \si{Biol.} {\it Vie animale} : voir
Relation$^5$. — \si{Hist.} {\bf 2.} {\it Animaux-machines} voir {\it Automatisme}$^4$. —
 {\bf 3.} {\it Esprits animaux} : voir {\it Esprit}\,$^2$.

\ib{Animisme} — \si{Psycho.} \si{Soc.} {\bf 1.} Croyance
selon laquelle la nature est régie par
des {\it âmes}, des {\it esprits} ou par des
volontés analogues à la volonté
humaine.

— \si{Méta.} {\bf 2.} \fsb{S. norma.} Doctrine selon laquelle l’âme$^1$ serait le principe de
la vie organique aussi bien que de
la vie psychique.

\ib{Animus} — [\,mot latin\,] — \si{Ps. an.} {\it Chez
Jung} : image archétype$^2$ de l'âme
chez la femme, qui représente sa
masculinité inconsciente ({\it p. e.} Dionysos, Siegfried).

\ib{Anomalie} — Altération du type
normal.

\ib{Anomie} — Absence de loi ou d’organisation.

\ib{Antécédent} — \si{Épist.} (Ctr. : {\it conséquent}). {\bf 1.} Fait qui précède un autre
fait. — \si{Log.} {\bf 2.} Dans une proposition
hypothétique, partie de la proposition qui exprime la condition.

\ib{Anthropocentrisme} — État d'esprit
dans lequel l'homme se considère
lui-même comme le centre de l’univers et s’imagine que le monde est
fait pour lui.

\ib{Anthropologie} — {\bf 1.} {\it Autref}., façon de
parler humaine : « L’Écriture est
pleine d’anthropologies » (Malebranche).

— \si{Épist.} {\bf 2.} {\it Chez Kant} : étude
philosophique de l'homme. — {\bf 3.}
{\it Auj}., ensemble des sciences naturelles, sociales, etc., traitant de
l’homme comme être animal (anthropologie {\it somatique})
ou social (anthropologie {\it culturelle}; syn. : {\it Ethnologie}).

\ib{Anthropomorphisme} — État d'esprit
dans lequel l'homme se représente
tous les êtres ({\it not}. Dieu) sur le
modèle de sa propre nature.

\ib{Antinomie} — \si{Crit.} {\it Chez Kant} : contradiction
dans laquelle tombe la
raison$^2$ quand elle prétend résoudre
les problèmes de la cosmologie*
rationnelle. Voir {\it Précis}, Ph. II,
p. 417.

\ib{Antithèse} — \si{Log.} et \si{Crit.} {\bf 1.} Opposition*
({\it spéc}. celle des contraires*) entre
deux propositions. — {\bf 2.} {\it Chez Kant
et Hegel} : proposition contraire à la
{\it thèse}\,$^2$ et formant le second moment
de l’antithèse$^1$.

\ib{Antithétique} — {\it Chez Kant} : l’ « antithétique de la raison pure » est la
partie de la Dialectique* transcendantale qui contient la théorie des
antinomies*.

\ib{Antitypie} — \si{Hist.} « L’antitypie est
l'impossibilité d'occuper un même
espace avec un autre corps et la
nécessité que l’un ou l’autre se
meuve ou reste au repos » (Leibniz,
L. à Thomasius, 1669).

\ib{Anxiété} — \si{Psycho.} {\bf 1.} {\it Autref}. ({\it p. e.}
Brissaud, 1890), état psychique
%20
concomitant de l’angoisse*, celle-ci
étant définie comme phénomène
purement physique. — {\bf 2.} {\it Auj}.
l'opposition est abandonnée sous
cette forme. Mais il reste vrai qu’ « on
vit l’angoisse plus qu’on ne la pense,
tandis qu’on pense l’anxiété autant
qu'on la vit » (J. Favez-Boutonier).
{\it Cf.}  Inquiétude$^2$.

\ib{Apagogique (Raisonnement)} — [\,G. {\it apagôgé}, abduction*\,]
— \si{Log.} Démonstration par l'absurde\,$^2$ : « Il est
difficile de se passer toujours des
démonstrations apagogiques » (Leibniz, {\it N.E.}, IV, 8, 2).

\ib{Apathie} — [\,G. {\it a} priv. et {\it pathos}, affection\,]
— \si{Car.} {\bf 1.} Insensibilité se
manifestant par la faiblesse et la
lenteur des réactions : « L’apathique
est méditatif et morose » (Le Gall).

— \si{Hist.} {\bf 2.} {\it Chez les Stoïciens} : état
de l'âme devenue inaccessible au
trouble des passions ({\it cf.}  {\it Ataraxie}*)}
et insensible à la douleur.

\ib{Aperception} — \si{Hist.} {\bf 1.} {\it Chez Leibniz} :
perception vive et claire ({\it opp}. perception obscure, subconsciente ou
« petite perception ». — \si{Psycho.}
 {\bf 2.} Appréhension$^1$.

\ib{Aphasie} — [\,G. {\it a} priv. et {\it phasis}, parole\,]
— \si{Ps. path.} Perte totale ou partielle des fonctions du langage (voir
notre {\it Précis}, Ph. I, p. 336-310), les
organes de la phonation* restant
intacts. $->$ {\it Dist}. aphonie, qui suppose au ctr. un trouble, paralysie ou
lésion, de ces organes. — {\it Aphasie
motrice} {{\it autref.} {\it aphémie}) : abolition
de la mémoire motrice verbale (prononciation des mots}. — {\it A.
sensorielle} (syn. : {\it a. de Wernicke} ou {\it surdité verbale}) :
abolition de la mémoire
auditive verbale (intelligence des
mots entendus). — {\it A. de Broca} :
nom donné par Pierre Marie à
%21
l'aphasie sensorielle (ou « aphasie
vraie ») compliquée d’anarthrie*. —
{\it cf.}  {\it Agraphie}* et {\it Cécité* verbale}.

\ib{Aphorisme} — Maxime générale et
concise résumant une théorie médicale, juridique, morale,
métaphysique, etc. : « On a de Mahomet quelques aphorismes
de médecine » {Voltaire); {\it cf.}  Bacon : {\it Aphorismes sur
l'interprétation de la nature} (dans le
{\it Novum Organum}), — Schopenhauer :
{\it Aphorismes sur la sagesse dans la vie}.

\ib{Apodictique (Proposition)} — [\,G. {\it apodeiktikos}, convaincant\,] — \si{Crit.}
Proposition nécessairement$^1$ vraie,
soit en vertu d’une évidence immédiate, soit en vertu d’une
démonstration déductive. Cf.  {\it Modalité}*.

\ib{Apollinien.} — \si{Hist.} {\it Chez Nietzsche}
{opp. : {\it dionysiaque}), le principe
apollinien est le principe contemplatif, source d'harmonie et de
beauté. Ce qualificatif a été appliqué
aussi par Spengler à la civilisation
antique.

\ib{Apologétique} — \si{Théol.} Partie de la
théologie qui a pour objet la défense
de la foi$^5$ contre les objections.

\ib{Apophantique} — {\bf 1.} (Adj.) Qui énonce
un rapport susceptible d'être dit vrai
ou faux : « Le discours apophantique
pose [\,chez Aristote\,] le rapport de la
substance et de ses accidents » (Serrus). — {\bf 2.} (Nom fém.) Théorie
logique des propositions : « L’apophantique aristotélicienne ».

\ib{Aporie} — [\,G. {\it aporia}, situation sans issue\,]
— \si{Épist.} Difficulté insoluble : « Les
apories du cartésianisme. »

\ib{A posteriori} — \si{Crit.} (Ctr. : {\it a priori}).
Postérieur à l’expérience$^1$ ; acquis
grâce à l'expérience$^1$ (s’il s’agit d'une
notion) ou qui se fonde sur l'expérience$^1$,
% 21
sur les faits (s’il s’agit d’un
raisonnement, d’une méthode).

\ib{Apparence} — Voir Dialectique$^4$.

\ib{Appartenance} — \si{Log.} \si{form.} Relation
logique entre un sujet {\it x} et la
classe A dans l'extension de laquelle
il rentre (Le signe de l’appartenance
est $\epsilon$ on écrit: {\it x} $\epsilon$ A, ce qui s’énonce:
« {\it x} est un A »).

\ib{Appétit }— \si{Psycho.} {\bf 1.} {\it Lato}. Tendance,
activité\,$^2$ : « L'appétit est l'essence
même de l’homme » (Spinoza, {\it Éth}.,
III, 9, scolie). Dans le {\it lang. scolastique}, « appétit concupiscible » :
« celui où domine le désir » : on y
rattache l'amour et la haine, le désir
et l’aversion, la joie et la tristesse ; —
« appétit irascible », « celui où domine
la colère », mais qui « serait peut-être
appelé plus convenablement courageux » (Bossuet) : on y rattache le
courage et la crainte, l'espérance et
le désespoir, enfin la colère. — {\bf 2.} {\it Str}.
Tendance se rapportant à l'organisme ({\it p. e.} faim, soif, appétit sexuel,
besoin d’exercice).

\ib{Appétition} — \si{Hist.} {\it Chez Leibniz} :
« action du principe interne qui fait
le changement ou le passage d’une
perception à une autre » dans les
monades\,$^2$ ({\it Mon}., 15).

\ib{Appréhension} — \si{Psycho.} {\bf 1.} Acte le
plus simple de la connaissance par
lequel l'esprit saisit immédiatement
l’objet connu. — {\bf 2.} Acte par lequel
la mémoire saisit immédiatement et
retient une série de souvenirs. —
 {\bf 3.} (\si{Vulg.}) Crainte vague et légère.

\ib{Approche} — (par imitation de l’angl.
{\it approach}, intérêt pour qqc). —
\si{Épist.} Recherche; façon d’aborder
un objet d'étude.

\ib{Apraxie} — [\,G. a priv. et {\it praxis}, action\,]
— \si{Ps. path.} « Incapacité d'exécuter
% 22
correctement des actes habituels,
sans qu'il y ait paralysie » (Lalande).
$->$ {\it Dist}. agnosie*.

\ib{A priori} — (Ctr. : {\it a posteriori}). \si{Crit.}
{\it 1.} Logiquement antérieur à toute
expérience\,$^1$; qui ne s'explique pas
par l'expérience. {\it Chez Kant} : « formes
a priori », voir {\it Forme}\,$^2$. {\it Raisonnement a priori} : celui qui, au lieu de
se fonder sur les faits, s’appuie uniquement sur les règles logiques de la
raison ({\it p. e.} preuve ontologique*).

— \si{Log.} {\bf 2.} Antérieur à {\it telle} série
d’expériences\,$^1$ : « L'idée expérimentale [\,{\it i. e.} l'hypothèse\,] est une {\it idée
a priori} » (Cl. Bernard). $->$ Peu
correct en ce deuxième sens.

\ib{Apriorisme} — \si{Hist.} \fsb{S. norma.} Doctrine qui
pose des « a priori »$^1$ ({\it p. e.} le rationalisme$^1$ classique). Max Scheler a
opposé à l’apriorisme formel et rationaliste de Kant un « apriorisme
matériel$^1$ » dont le contenu consiste
en états affectifs, sentiments ou
valeurs. M. Dufrenne dist. des « a
priori de l’affectivité » et des « a
priori de l'imagination » ({\it p. e.} archétypes\,$^2$ de Jung).

\ib{Aprosexie} — [\,G. {\it a} priv. et {\it prosexis}, attention\,] — \si{Ps. path.} Incapacité de
fixer son attention; dispersion intellectuelle morbide.

\ib{Aptitude} — \si{Psycho.} Disposition physique ou psychique innée formant le
substrat d’une capacité déterminée.

\ib{Arbitre (Libre)} — Voir Liberté$^6$,

\ib{Archéologie} — \si{Épist.} Étude des monuments anciens.

\ib{Archée} — \si{Hist.} {\it Chez Paracelse, Van
Helmont}, etc. : principe vital tenant
à la fois de la matière et de la pensée
(voir {\it Précis}, Ph. II, p. 484).

\ib{Archétype} — {\bf 1.} \si{Méta.} Type idéal des
choses sensibles : {\it p. e. chez Platon},
les Idées$^1$ ; chez Malebranche, les
idées\,$^3$ divines : « La substance [\,de
Dieu\,] renferme l’archétype ou le
modèle éternel des créatures » ({\it R. V.}
IV, 11, 3).

— {\bf 2.} \si{Ps. an.} {\it Chez Jung}, les {\it archétypes} sont les images et symboles
ancestraux, dont l’ensemble forme
l’{\it inconscient collectif} et qu’on retrouve dans les mythologies,
légendes, contes de fées, traditions religieuses, etc.

\ib{Architectonique} — \si{Hist.} Chez Kant :
partie de la Logique\,$^5$ qui enseigne à
coordonner les éléments de la connaissance.

\ib{Argument} — \si{Log.} Raisonnement destiné à prouver ou à réfuter une
proposition, une théorie\,$^2$.

\ib{Aristocratie} — \si{Soc.} \fsb{S. posit.} Système de
gouvernement dans lequel le pouvoir appartient à une caste* ou à une
classe* (aristocraties sacerdotales,
fondées sur la religion; {\it militaires},
sur la naissance ; {\it bourgeoises}, sur la
fortune).

\ib{Arithmétique} — [\,G. {\it arithmos}, nombre\,]
— \si{Épist.} Science théorique du
nombre considéré en tant que valeur
déterminée, exprimée par des chiffres.

\ib{Arithmologie} — Nom donné par Ampère (1834) à la science générale du
nombre, de la quantité pure, comprenant selon lui l’arithmétique*,
l’algèbre*, le calcul* des fonctions
et le calcul* des probabilités\,$^2$.

\ib{Art} — \si{Vulg.} {\bf 1.} (Par opp. d’une part
au {\it savoir théorique}, d’autre part à la
{\it pratique spontanée}). Pratique méthodique, soumise à un ensemble de
règles » ; « Les arts mécaniques » ;
« La Logique$^2$ ou art de penser ».
{\it Arts libéraux} : au moyen âge, grammaire,
% 23
rhétorique, dialectique\,$^3$ (formant le {\it trivium}), arithmétique, géométrie,
astronomie, musique (formant le {\it quadrivium}). — {\bf 2.} Habileté
qqfs. innée, plus souvent acquise
par la pratique ; adresse plus ou
moins calculée dans l'emploi des
moyens pour arriver à un but
« L'art de plaire » ; « L'art de mentir » ;
« L'art le plus innocent tient de la
perfidie » (Voltaire).

— \si{Esth.} {\bf 3.} Activité esthétique$^3$ :
« Les beaux-arts » ; « Les arts plastiques » ; « La beauté est l’objet
propre et exclusif de l’art » (Ravaisson).

\ib{Artificialisme} — \si{Psycho.} \fsb{S. posit.} État d'esprit
de l'enfant qui croit que les objets
et phénomènes extérieurs sont l'œuvre de l’homme (voir {\it Précis}, Ph. I,
p. 84).

\ib{Ascèse} — [\,G. {\it askêsis}, exercice\,] — \si{Mor.}
Tout mode de vie impliquant un
effort de volonté et de renoncement
aux plaisirs sensibles en vue soit
d'acquérir la perfection morale, soit
de poursuivre une œuvre.

\ib{Ascétisme} — \si{Mor.} {\bf 1.} Ensemble d’austérités et de mortifications ayant
pour but la perfection morale individuelle ({\it gén}. avec une idée
religieuse) : « Il y a dans l’ascétisme de
la négation » (Le Senne). — {\bf 2.} \fsb{S. norma.}
Doctrine morale préconisant ce
mode de vie.

\ib{Aséité} — [\,L. {\it a se}, par soi\,] — \si{Théol.}
Attribut de Dieu consistant en ce
qu'il est l’être existant « par* soi ».

\ib{Assentiment} — \si{Psycho.} (Ctr. : {\it doute}).
Adhésion donnée par l'esprit à un
jugement$^2$. L’assentiment comporte
plusieurs degrés, not. : {\it a}) l'opinion$^1$.
— {\it b}) la certitude$^1$.

\ib{Assertion} — \si{Log.} Affirmation* ou
négation de la {\it lexis}* d'un jugement$^2$.

\ib{Assertorique (Proposition)} — \si{Log.}
et \si{Crit.} (Opp. {\it apodictique} et {\it problématique}). Celle qui énonce une
simple assertion sans la poser comme
logiquement nécessaire$^1$ : {\it p. e.} « Il
pleut ». — Cf.  {\it Modalité}\,*.


\ib{Assimilation} — \si{Biol.} {\bf 1.} Fonction
par laquelle la matière vivante convertit en sa propre substance les
matériaux dont elle se nourrit.

— \si{Soc.} {\bf 2.} Processus par lequel un
groupe social, une nation, une civilisation absorbe dans sa propre
culture, sans nécessairement les y identifier tout à fait,
les éléments étrangers, groupes ou individus, qui y
entrent.

— \si{Psycho.} {\bf 3.} Fonction par laquelle la conscience$^1$, en recevant
les impressions qui lui viennent du
dehors, les fond en un tout où elles
se mêlent les unes aux autres.


\ib{Association} — \si{Biol.} 1 Voir {\it Symbiose}*.

— \si{Phol.} {\bf 2.} Connexion dynamique
entre les diverses parties d’un organisme, entre les éléments des centres
nerveux, etc.

— \si{Psycho.} {\bf 3.} {\it Association des
idées} : phénomène psychique par
lequel un état de conscience amène
spontanément$^3$, machinalement la
réapparition d’autres états de conscience. $->$ {\it Dist}. liaison logique
des idées et jugement.

— \si{Soc.} {\bf 4.} {\it Lato}. Groupement
volontaire de plusieurs personnes
dans un but quelconque. — {\bf 5.} {\it Str}.
(Jur.) : « Convention par laquelle
deux ou plusieurs personnes mettent
en commun d’une façon permanente
leurs connaissances ou leur activité
dans un but autre que de partager
des bénéfices ». (Loi 1$^\text{er}$ juill. 1901).

\ib{Associationnisme} — \si{Psycho.} \fsb{S. norma.} Doctrine selon laquelle : 1° toutes les
% 24
opérations intellectuelles, même les
liaisons logiques d'idées (jugement,
raisonnement) et les principes de la
raison, se raméneraient à l’association$^3$ des idées; 2° celle-ci serait
elle-même un phénomène purement
automatique$^2$.

\ib{Assumer} — {\bf 1.} \si{Log.} Admettre une
proposition simplement comme hypothèse pour voir quelles
conséquences en résultent.

— {\bf 2.} \si{Mor.} Accepter une situation avec les responsabilités qu'elle
comporte : « Nous devons assumer
notre temps déchiré et triomphant,
dans ses incertitudes dernières »
(Gusdorf).

\ib{Astrologie} — \si{Hist.} Art de prédire
l'avenir par l’observation des astres.

\ib{Astronomie} — \si{Épist.} Étude scientifique des astres.

\ib{Ataraxie} [\,G. {\it a} priv. et {\it taraxis}, trouble\,]. — \si{Hist.} Tranquillité de l'âme.
{\it Chez les Stoïciens}, c'était l'idéal du
sage ({\it cf.}  {\it Apathie}*).

\ib{Atavisme} — [\,L. {\it atavus}, ancêtre\,] —
\si{Biol.} {\bf 1.} Forme d’hérédité où l'être
vivant hérite de caractères d’ancêtres éloignés. — {\bf 2.} {\it Ext}. Tendance
des êtres vivants à revenir aux types
ancestraux,

\ib{Athéisme} — \fsb{S. norma.} \si{Méta.} {\bf 1.} Négation de
l'existence de Dieu, — {\bf 2.} {\it Str}. Négation de l'existence de Dieu selon la
conception traditionnelle, soit qu’on
nie sa personnalité (c'est en ce sens
que Spinoza fut accusé d’athéisme},
soit qu’on le dise valeur plutôt
qu'être (Lagneau).

\ib{Athymhormie} — [\,G. a priv.; {\it thymos}, cœur; {\it hormè}, élan\,]
\si{Ps. path.} Diminution générale de la vitalité qui s’observe dans
l'hébéphrénie* et la schizophrénie* et qui se manifeste par l'indifférence
affective, le dégoût de tout effort et le désintérêt à l'égard d'autrui et du
monde extérieur.

\ib{Atome} — [\,G. {\it atomos}, insécable, indivisible\,] — \si{Phys.} 1 {\it Dans l'antiquité} :
particule de matière indivisible qui
est l'élément dernier de tout ce qui
existe; selon Démocrite, les atomes
sont éternels, homogènes entre eux,
ils ont une forme, un poids, etc.,
mais ne possèdent pas les propriétés
sensibles (couleur, etc.) des composés. $->$ Dist. cette notion de la
conception moderne (sens 2)
l'{\it atome} des anciens correspondrait
plutôt à la {\it molécule} des chimistes
modernes. — {\bf 2.} Particule de matière
indivisible, dans les conditions ordinaires, par les forces chimiques
aussi bien que par les forces physiques : « La molécule d'hydrogène
est composée de deux atomes ».
{\it Théorie atomique} : hypothèse de
Dalton (1803) selon laquelle les
corps sont formés d’atomes$^2$, ayant,
pour chaque corps simple, un poids
invariable (poids {\it atomique}), et produisant les différentes combinaisons
chimiques par leur juxtaposition. —
 {\bf 3.} {\it Ext}. Tout élément physique indivisible : « Atome d'électricité,
d'énergie »,

— \si{Méta.} {\bf 4.} Tout être, même spirituel, simple et indivisible : « Les
monades* sont les véritables atomes
de la nature » (Leibniz, {\it Mon.}, 3),

\ib{Atomisme} — \fsb{S. norma.} \si{Méta.} {\bf 1.} Doctrine philosophique (Démocrite, Épicure,
Lucrèce) selon laquelle la matière est
formée d’atomes$^1$, — \si{Phys.} {\bf 2.} Théorie
atomique$^2$. — {\it Ext.} \si{Psycho.} {\bf 3.} {\it Atomisme psychologique} : conception
selon laquelle l'esprit serait un
composé d'éléments psychiques
simplement juxtaposés les uns aux
autres. — \si{Soc.} {\bf 4.} {\it Atomisme social} :
conception selon laquelle un groupe
social n’est qu'un total d'individus.

\ib{Atone} — \si{Car.} Sans énergie : « L'apathique est un rétracté atone »
(Le Gall).

\ib{Attention} — \si{Psycho.} Concentration
de l’activité mentale sur un objet
déterminé. — {\it Attention périphérique} : celle qui porte sur les sensations
({\it a. sensorielle}) ou sur les mouvements ({\it a. motrice}). — {\it A. centrale :}
celle qui porte, soit sur l'état intérieur du sujet (observation de
soi-même ou introspection*}, soit sur les
idées (a. intellectuelle ou réflexion$^2$).
— {\it A. spontanée} : celle qui dérive de
nos goûts, de nos tendances naturelles ou acquises. — {\it A. volontaire} :
celle qui consiste à prendre intérêt,
par un effort mental, à ce qui n’en
présente pas naturellement pour
nous. $->$ On parle qqfs. d’une
{\it attention passive} qui s'imposerait à
nous du dehors ({\it p. e.} un bruit violent),
mais cette expression est impropre.

\ib{Attitude} — \si{Phol.}, {\bf 1.} Réaction relevant de la fonction posturale*.

— \si{Psycho.} {\bf 2.} {\it Attitude mentale}
(notion introduite, par Binet et
l’école de Würzbourg) : état de conscience antérieur au mot et à l’image
et qui consiste en « une préparation
à l'acte qui reste intérieure et nous
est révélée par les sensations subjectives qui l’accompagnent » (voir
{\it Précis}, Ph. I, p. 308-310).

— \si{Soc.} {\bf 3.} {\it Attitude sociale} : comportement ou disposition à agir qui,
dans un groupe social, s'impose plus
ou moins aux individus par suite de
normes* ou de représentations collectives ({\it cf.}  {\it Précis}, Ph. I, p. 534-536).

\ib{Attribut} — \si{Log.} \si{form.} {\bf 1.} (Syn. : {\it prédicat}*). Terme qui, dans
une proposition, désigne ce qu’on affirme
ou nie du sujet$^2$.

— \si{Méta.} {\bf 2.} Tout ce qui peut être
dit attribut (au sens 1) d'une substance$^1$ : « La fluidité, la dureté, la
mollesse... se pouvant séparer de la
matière, il s'ensuit que tous ces
attributs ne lui sont point essentiels » (Malebranche, {\it R. V.}, III, 2,
%25
8, 2). — {\bf 3.} Propriété essentielle et
permanente d’une substance$^1$ ({\it gén}.
opp. aux modes$^1$ qui sont des propriétés accidentelles et changeantes) :
« Nous distinguons qqfs. une substance de qqn de ses attributs [\,{\it p. e.}
l'âme, de la pensée ; la matière, de
l'étendue\,] sans lequel néanmoins il
n’est pas possible que nous en ayons
une connaissance distincte (Descartes, Prince. I, 62; {\it cf.}  Mode$^1$) :
« Par attribut, j'entends ce que
l’entendement saisit de la substance
comme constituant son essence »
(Spinoza, {\it Eth.}, I, déf. 4).

\ib{Atypique} — Non conforme au type.
Se dit {\it not.}, dans la \si{Psycho.} des opinions, des façons de penser non
conformistes.

\ib{Audition colorée} — \si{Psycho.} Synesthésie* qui fait que, pour certaines
personnes, tel son est invariablement lié à certaines couleurs ({\it cf.} 
Rimbaud, {\it Voyelles}).

\ib{Aufheben} — \si{Méta.} {\it Chez Hegel} : mot
allemand qui désigne l’action de
dépasser une contradiction : « Aufheben a dans la langue un double
sens : il signifie {\it garder, conserver} et
en même temps {\it faire cesser, mettre
fin à}. L'idée de conserver contient
déjà en elle cet élément négatif
consistant en ce que, pour le garder,
qqc est enlevé à son être immédiat »
({\it Grande \si{Log.}}, IV). Voir {\it Médiation}$^2$
et {\it cf.}  {\it Précis}, Ph. II, p. 460, n.

\ib{Authenticité, Authentique} — \fsb{S. objec.} {\bf 1.} \si{Épist.}
En histoire, un document est dit
{\it authentique} quand il émane bien de
la source à laquelle il est attribué,
$->$ Ce qui ne signifie pas nécessairement qu'il est véridique. — \fsb{S. subje.} {\bf 2.}
\si{Méta.} {\it Chez Heidegger} : l'existence
{\it authentique} est celle qui assume$^2$
sa situation d'{\it être-pour-la-mort} au
% 26
lieu de se réfugier dans l’inauthenticité du {\it On}*. — {\bf 3.} {\it Plus  {\it gén.}} \si{Mor.}
Sincérité : « L’authenticité de l’obligation suppose la sincérité du cœur »
(Le Senne).

\ib{Autisme} — [\,G. autos, soi-même\,] — Ps.
an. Nom donné par Bleuler à l’état
mental des schizophrènes*, où la
pensée a pour unique but la satisfaction personnelle (comme dans la
rêverie) sans souci d'adaptation à
autrui ni au réel.

\ib{Automate} — \si{Vulg.} {\bf 1.} Machine qui se
meut elle-même. Qqfs. appliqué {\it péj.}
à l’homme : « Le sot est un automate,
il est machine » (La Bruyère).

— \si{Méta.} {\bf 2.} Être qui a en lui le
principe de ses mouvements ou de
ses déterminations : « Chaque corps
organique d’un vivant est une espèce
d’automate naturel » (Leibniz, {\it Mon.},
64); « L'âme est un automate spirituel » (id., {\it Théod.}, 403).

\ib{Automation} — \si{Techn.} Système de
mécanisation du travail poussée à
l'extrême et où le contrôle même
des machines est exercé par des
machines.

\ib{Automatisme} — \si{Phol.} 1 {\it Mouvement
automatique} celui qui, tout en
ayant sa source dans l'être qui se
meut, s'explique d’une façon purement mécanique. Chez les êtres
supérieurs : mouvement qui échappe
à la direction des centres cérébraux
et relève seulement des centres inférieurs ({\it p. e.} réflexe*).

— \si{Psycho.} {\bf 2.} {\it Fonction automatique} : celle qui s’expliquerait de
façon purement mécanique ({\it p. e.} l’association des idées selon
l’associationnisme*). — {\bf 3.} {\it Automatisme psychologique} (Janet) :
l’activité psychique inconsciente. {\it cf.}  {\it conservatrice}*.

—— \si{Hist.} {\bf 4.} {\it Automatisme des bêtes} :
théorie de Descartes selon laquelle
les animaux seraient dénués de toute
vie psychique et agiraient comme
des machines, par le simple jeu des
esprits\,$^2$ animaux.

$->$ Bien {\it dist.} tous ces sens :
chez Descartes, l’{\it automatisme} des
animaux est purement physiologique ; l’épiphénoménisme* est aussi
une théorie qui fait de nous des
{\it automates}, au sens physiologique,
mais des automates {\it conscients} ; par
contre, on peut parfaitement dire
que l'instinct {\it p. e.} est {\it automatique}
sans admettre la théorie de Descartes; de même, la doctrine de
Janet (sens 3) ne doit pas être confondue avec l’épiphénoménisme.

\ib{Automorphisme} — \si{Psycho.} « Disposition d'esprit par laquelle nous
tendons à imposer la forme de notre
âme à l’âme d’autrui » (D$^\text{r}$ Logre).

\ib{Autonomie} — [\,G. {\it autos}, soi-même, et
{\it nomos}, loi\,] — \si{Mor.} (Ctr. {\it hétéronomie}*). État de la volonté
raisonnable qui n’obéit qu'à une règle
émanant d’elle-même.

\ib{Autopsie} — \si{Méd.} {\bf 1.} Examen, par dissection, de toutes les parties d’un
corps mort. — \si{Hist.} {\bf 2.} {\it Chez Ampère}
(syn. : {\it Emesthèse}) : système mental
(opp. au {\it système sensitif}) impliquant
la prise de conscience par le moi de
sa propre causalité.

\ib{Autorité} — \si{Épist.} {\bf 1.} Pouvoir de se
faire croire : « L'autorité d’un document, d'un témoin ». — {\bf 2.} {\it Méthode
d'autorité} : celle qui consiste à établir
une assertion, non sur des preuves,
mais sur le seul témoignage.

— \si{Pol.} {\bf 3.} Droit de commander.
$->$ {\it Dist.} contrainte.

\ib{Autoscopie} — \si{Ps. path.} {\bf 1.} Exagération morbide de
la cénesthésie* consistant en ce que certains sujets,  {\it gén.}
hystériques, sentent fonctionner
leurs organes internes, {\it p. e.} leur
cœur. — {\bf 2.} « Hallucination consistant à se voir soi-même devant soi »
(Lalande) : {\it p. e.} Musset dans la {\it Nuit
de Décembre}.

\ib{Auto-suggestion} — \si{Psycho.} Action
de se suggestionner\,$^1$ soi-même, volontairement ou involontairement,
{\it p. e.} par « la lecture d’un livre ou
d’un journal, une conversation, une
méditation, un spectacle » (G.
Dumas).

\ib{Autotélique} — [\,G. {\it autos}, soi-même; {\it telos},
fin\,] — {\it Chez Baldwin} : qui a sa fin
en lui-même ({\it p. e.} l’art pour l’art).

\ib{Autre (L')} — {\bf 1.} (Masc.) Autrui :
« Autrui, c’est l’{\it autre}, c’est-à-dire
le moi qui {\it n’est pas} moi » (Sartre). —
{\bf 2.} (Neutre) {\it Chez Platon} : le divers,
l’hétérogène, le multiple, opp. au
Même et à l’Un. {\it cf.}  Lalande
« Cette valeur [\,de la Raison\,] consiste
dans la supériorité du Même sur
l'Autre » (voir {\it Textes choisis}, I,
p. 299).

\ib{Axiologie} — Théorie des valeurs : « La
science des valeurs a reçu le nom
d’axiologie et c'est sous ce terme
qu’on la désigne en général. Le mot
{\it axios} indique en grec ce qui est précieux, digne d'être estimé et le
verbe {\it axioô} veut dire {\it j'apprécie}.
L’Axiologie serait donc la science
de l’estimation et de l’appréciation »
(Lavelle).

\ib{Axiomatique} — \si{Épist.} {\bf 1.} Système
d’axiomes* au sens 3 : « L’axiomatique de Hilbert ». — {\bf 2.} Théorie de
ces axiomes$^3$ considérés comme
règles purement formelles : « L’Axiomatique s'impose comme loi de tout
expliciter sans rien présupposer »
(R. Blanché).

\ib{Axiome} — [\,G. {\it axiôma}, jugement; {\it d'où} :
principe\,] — \si{Log.} {\bf 1.} Toute proposition évidente par elle-même : « Les
axiomes généraux de la pensée » ({\it p. e.}
principe d’identité*).

— Spéc. \si{Math.} {\bf 2.} {\it Str.} Proposition indémontrable, évidente par
elle-même et s'appliquant d’une
façon très générale à des quantités
indéterminées, {\it p. e.} : « Deux quantités
égales à une troisième sont égales
entre elles » (voir {\it Précis}, Ph. II,
p. 95; Sc. et M., p. 212). — {\bf 3.} {\it Lato.}
Principe indémontrable se trouvant
à la base d’un raisonnement mathématique (y compris définitions et
postulats) [\,voir {\it Précis}, Ph. II, p. 97;
Sc. et M. p. 214\,]. $->$ {\it Dist.} 2 et 3:
{\it Henri Poincaré} emploie toujours
ce mot dans le sens 3, {\it p. e.} : « Les
axiomes géométriques ne sont ni des
jugements synthétiques* a priori ni
des faits expérimentaux : ce sont
des conventions ».
	\end{itemize}
