
	\begin{itemize}[leftmargin=1cm, label=\ding{32}, itemsep=1pt]

%183
\ib{Table} — \si{Épist.} {\bf 1.} {\it Chez Bacon} : « tables de comparution
», registres d’observations dans lesquelles on « fait comparaître » les faits
naturels pour en découvrir « l'essence » (cf. {\it Précis}, Ph. II, p. 122 ;
Sc. et M., p. 238). —  {\bf 2.} {\it Table rase} [L. {\it tabula rasa},
tablette de cire où il n’y a rien d’écrit] : expression empruntée à Aristote
par les empiristes* pour désigner l’état de l'esprit qui, avant toute
expérience, serait d’après eux absolument vide.

\ib{Tabou} — [mot polynésien] — \si{Soc.} {\bf 1.} {\it Str.} Interdit rituel
qui protège une chose ou un être réputé sacré$^2$. — {\bf 2.} {\it Ext.}
Prohibition morale ou coutumière : « Les tabous sexuels » ; « Certains
métiers sont tabou pour le bourgeois » (Goblot).

\ib{Tact} — \si{Ps. phol.} {\bf 1.} {\it Lato.} Le sens du toucher* en gén. —
{\bf 2.} {\it Str.} Le sens du contact et des pressions, du lisse, du
rugueux, etc.

— \si{Car.} {\bf 3.} Délicatesse morale : « Ce tact attentif de l'esprit qui
fait sentir les nuances des fines convenances » (Buffon).

\ib{Tactile} — Qui se rapporte au tact$^2$.

\ib{Tactisme} — \si{Phol.} Mouvement de déplacement produit automatiquement
chez un animal par un agent physique extérieur (p. e. {\it chimiotactisme}
des infusoires).

\ib{Tautologie} — [G. {\it to auton legein}, dire la même chose] — \si{Épist.}
{\bf 1.} Proposition identique$^4$. — {\bf 2.} {\it Chez les logisticiens} :
proposition purement formelle$^3$ qui n’a de sens que par rapport à des
objets possibles, mais qui demeure vraie quels que soient ces objets (opp. :
{\it protocole}*).

— {\bf 3.} ({\it Péj.}) Faute logique qui
% 184
consiste à présenter une répétition
en termes différents comme un progrès$^1$ de la pensée.

\ib{Taxinomie ou Taxonomie} — \si{Épist.} Ancien nom de la systématiques$^3$.

\ib{Technique (nom)} — {\bf 1.} Ensemble des procédés d’un art, d’un métier
ou d’une science : « La technique du cinéma »; « La technique de
l'enseignement ». — {\bf 2.} La science appliquée, {\it spéc.} dans le
domaine industriel : « Les progrès$^2$ de la technique ». Qqfs. {\it péj.} :
« Comment a-t-il [l'esprit européen] remporté ses succès meurtriers ? Par la
technique ! Et qui créa cet instrument de meurtre ? La science, qui compte,
mesure et pèse » (Klages).

\ib{Technologie} — \si{Épist.} Étude des techniques$^2$ : « La Technologie
est symétrique dans le domaine de l’action à la Logique dans le domaine de la
connaissance » (Espinas).

\ib{Tèlé} — [mot grec — loin] — \si{Soc.} En Sociométrie*, facteur qui mesure
la « distance sociale » et l'attraction entre individus ou groupes.

\ib{Téléologie} — [G. {\it telos}, fin, et {\it logos}, étude] — \si{Méta.}
{\bf 1.} \fsb{S. norma.} Doctrine finaliste* : « Ce serait parce que nous
désirons les choses qu’elles seraient bonnes : la proposition de Spinoza
serait ainsi retournée en faveur de la téléologie » (Hamelin). — {\bf 2.} La
finalité elle-même : « C’est bien dans les concepts qu'il convient de situer
l’origine primordiale de la téléologie » (id.).

\ib{Téléologique} — \si{Méta.} Qui suppose de la finalité* : « Les
explications téléologiques ne sont ni partout désirables ni suffisantes
» (Hamelin) ; « L'adaptation est une manifestation téléologique ou elle n’est
rien »
% 184
(Vandel). {\it Preuve téléologique ou Physico-téléologique} : celle qui vise
à prouver l'existence de Dieu par les causes finales$^1$.

\ib{Télépathie} — [G. {\it têlé}, loin, et {\it pathos}, sentiment] —
\si{Psycho.} Communication directe de la pensée à grande distance.

\ib{Témoignage} — \si{Épist.} {\it Str.} {\bf 1.} Attestation volontaire d’un
fait par un témoin : « Un témoignage juste n’est pas la règle, mais
l'exception » (Claparède). — {\it Lato.} {\bf 2.} Document, en gén. : « Une
idée surannée : celle du temps où l’on ne savait lire que les témoignages
volontaires » (M. Bloch). — {\it Latiss.} {\bf 3.} Preuve, indice : « Cet
appétit [d’intériorité] est le témoignage, chez l’homme d’auj., de l'effort
pour ne pas se laisser aspirer par les choses extérieures » (Bréhier). $->$
Le seul sens propre est le sens 1.

\ib{Tempérament} — \si{Car.} Ensemble des dispositions organiques d’un
individu : « L’esprit dépend si fort du tempérament et de la disposition des
organes du corps que... » (Descartes, Méth., VI) ; « Le caractère moral$^4$
n’est que la physionomie du tempérament physique » (Bichat) ; « Le
tempérament sanguin ».

\ib{Tempérance} — \si{Mor.} La vertu de la sensibilité, qui consiste dans la
modération à l'égard des plaisirs sensibles et aussi à l'égard des émotions
et des passions.

\ib{Temporalité} — \si{Méta.} Dans le lang. existentialiste : caractère du
Dasein$^2$ qui est à la fois solidaire de son passé et pro-jet* vers l'avenir.

\ib{Temps} — \si{Crit.} et \si{Méta.} {\bf 1.} Milieu homogène et infini dans
lequel nous paraissent se dérouler les événements :
% 185
« Le temps n’est rien qu’une certaine façon dont nous pensons à la durée$^1$
» (Descartes, {\it Princ.}, I, 57) ; « Tout le temps de ma vie peut être
divisé en une infinité de parties, chacune desquelles ne dépend en aucune
façon des autres » (id., {\it Méd.}, III) ; « Le temps n’est que la condition
subjective$^1$ sous laquelle les intuitions sont possibles en nous » (Kant,
{\it R. pure}, Esth., § 6) ; « Il y aurait lieu de se demander si le temps,
conçu sous la forme d’un milieu homogène, ne serait pas un concept bâtard, dû
à l’intrusion de l’idée d'espace dans le domaine de la conscience pure
» (Bergson, {\it D. I.} II ; cf. {\it Durée}$^2$). — {\bf 2.} Portion finie
du temps$^1$ : « Un temps est une portion de durée mesurée » (Destutt de
Tracy). {\it Spéc.} \si{Ps. métr.} {\it Temps de réaction} : durée qui
s'écoule entre l'excitation$^1$ et la réponse du sujet$^5$. — {\bf 3.}
Époque : « En ce temps-là... ».

\ib{Tendance} — \si{Psycho.} {\bf 1.} {\it Str.} Forme spontanée de
l’activité$^2$ : « La tendance ne nous est donnée que par l’affection
» (Lachelier). On peut, avec Pradines, dist. la « tendance à... », dynamisme
mental indifiérencié, et la « tendance vers... », activité spécifiée à objet
déterminé. — {\bf 2.} {\it Lato.} Tout mode actif$^3$ de la vie psychique : «
La tendance est une forme dynamique et plurivalente, abstraite et pourtant
réelle, innée ou acquise, qui détermine directement ou indirectement un acte
» (Burloud). Cf. {\it Schème}$^3$ et Thème$^2$. $->$ Le sens 1 est le sens
propre.

\ib{Tension psychologique} — \si{Psycho.} Notion introduite par Janet pour
caractériser les différents degrés du {\it niveau mental}. Sont actes de «
haute tension » ceux où : 1° l’unification des éléments est forte (cf. {\it
Synthèse}$^4$) ; 2° la masse des éléments est considérable ({\it p. e.}
l'attention).

\ib{Tensions} — Terme souvent usité pour désigner les oppositions internes
qui se manifestent ou existent à l’état latent dans une réalité humaine : «
Les tensions immanentes à toute réalité sociale. »

\ib{Terme} — \si{Log.} \si{form.} Concept$^1$ représenté pour son expression
verbale : les deux termes d’une proposition sont le sujet$^2$ et
l’attribut$^1$.

\ib{Terminisme} — Syn. de {\it nominalisme}$^1$ : « Le terminisme de G.
d’Occam. »

\ib{Terminologie} — \si{Ling.} Ensemble des termes spéciaux d’une science ou
d’un auteur.

\ib{Test} — [mot anglais = {\it épreuve}] \si{Psycho.} Ensemble d'épreuves
permettant de déterminer chez un individu ou un groupe d'individus, soit leur
niveau mental (tests d'âge ou de développement), soit la présence et le degré
de tel caractère mental (tests d’aptitudes, {\it p. e.} de mémoire,
d'attention, etc.), soit les caractéristiques de la personnalité (tests de
personnalité). Cf. {\it Précis}, Ph. II, p. 220-229; Sc. et M., p. 330-33
{\bf 8.}

\ib{Thanatormée} — [G. {\it thanatos}, mort, et hormê, tendance] —
\si{Ps. an.} Instinct$^4$ de mort, de destruction.

\ib{Thaumaturge} — [G. {\it thauma}, miracle, et {\it ergon}, œuvre] —
\si{Théol.} Qui fait des miracles : « Certains néo-platoniciens se faisaient
passer pour thaumaturges. »

\ib{Théandrique} — [G. {\it theos}, dieu, et {\it anêr}, {\it andros}, homme]
— \si{Méta.} Qui concerne Dieu et l’homme : « Le rapport théandrique est
toujours à la fois donné et idéal » (Le Senne).

\ib{Théisme} — \si{Méta.} ({\it Opp.} : {\it athéisme}* et
{\it panthéisme}*). À. Doctrine qui admet l'existence d’un Dieu* comme cause
première$^4$ transcendante$^2$ au monde. $->$ Sur la différence entre théisme
ancien et théisme moderne, voir
% 186
{\it Précis}, Ph. II, p. 510-512; Sc., p. 446-448 ; et M., p. 426-42 {\bf 7.}
{\it Cf.} {\it Déisme}* et {\it Démiurge}*.

\ib{Thème} — {\bf 1.} Sujet de réflexion ou de discussion : « Un thème
philosophique ». — {\bf 2.} \si{Psycho.} {\it Chez Burloud} (syn. : {\it
tendance thématique} {\it opp.} à {\it tendance schématique}) : « ensemble
des rapports qui constituent la signification commune d’une catégorie
d'objets. »

\ib{Théocratie} — \si{Pol.} Système politique caractérisé par la domination
de la caste sacerdotale : « Les premiers Incas établirent une théocratie
» (Voltaire).

\ib{Théodicée} — [G. {\it theos}, dieu, et {\it diké}, plaidoyer] — \si{Méta.}
{\bf 1.} {\it Chez Leibniz} : partie de la métaphysique qui traite du {\it
problème du mal}* et « justifie » la bonté de Dieu contre les objections
tirées de ce problème. — {\it Ext.} {\bf 2.} {\it Dans l'école de Cousin}, et
imppt : syn. de {\it théologie} au sens {\bf 1.}

\ib{Théologie} — \si{Méta.} {\bf 1.} (Théologie naturelle ou rationnelle).
Partie de la métaphysique qui traite de l’existence et des attributs de Dieu
en s'appuyant uniquement sur la raison$^5$ — \si{Théol.} {\bf 2.} (Théologie
révélée ou dogmatique). Étude des dogmes de la foi$^5$, fondée sur les textes
sacrés et l'autorité de l’Église : « La théologie qui se rapporte à
l’enseignement sacré, diffère en nature de cette autre théologie$^1$ qui se
pose comme partie de la philosophie » (St. Thomas, {\it S. th.}, I, 1, 1).

— \si{Hist.} {\bf 3.} Un système particulier de théologie au sens 1 ou 2 : «
La théologie de Plotin. »

\ib{Théologique} — {\bf 1.} Qui concerne la théologie*, surtout au sens
{\bf 2.} —  {\bf 2.} \si{Hist.} {\it Chez Comte} : « état théologique », état
initial (cf. {\it États}*) de
% 186
l'esprit humain, caractérisé par la croyance à des « agents surnaturels plus
ou moins nombreux dont l’intervention arbitraire explique les anomalies
apparentes de la nature » ({\it Cours}, I) ; l’état théologique comprend
trois stades successifs : le {\it fétichisme}$^1$, le {\it polythéisme} et le
{\it monothéisme}.

\ib{Théorétique} — \si{Épist.} {\bf 1.} Qui concerne la théorie$^1$. —
\si{Hist.} {\bf 2.} {\it Chez Aristote} : « sciences théorétiques », celles
qui ont pour but la spéculation$^1$ pure ({\it p. e.} mathématiques,
physique, métaphysique, {\it opp.} « sciences poétiques [= créatrices] »,
telles que poésie, rhétorique, dialectique$^2$, et « sciences pratiques [= de
l'action] », telles que éthique, économie et politique.

\ib{Théorie} — [G. {\it théôrein}, contempler] — \si{Épist.} \fsb{S. abstr.}
{\bf 1.} (Ctr. : {\it pratique}$^3$). La connaissance spéculative*, la pensée
désintéressée : « La théorie a été souvent l’origine de recherches pratiques
» (Picard). — \fsb{S. concr.} {\bf 2.} Ensemble de conceptions
systématiquement$^1$ organisées sur un sujet déterminé, {\it not.} dans les
sciences expérimentales (syn. : hypothèse$^3$ générale) : « Nous insisterons
sur l'importance des immenses constructions que bâtissent les savants sous le
nom de théories » (id.) ; « La théorie électronique » ; « Les théories
d’Einstein ».

\ib{Théorique} — Spéculatif$^1$ $->$ {\it Dist.} abstrait : {\it théorique}
s’opp. à {\it pratique} ; mais {\it abstrait} s'opp. à {\it concret}.

\ib{Théosophie} — \si{Hist.} \fsb{S. norma.} Nom générique des doctrines de
certains illuminés ({\it p. e.} Paracelse, Weigel, Bœhme, Swedenborg, Hamann,
Saint-Martin, Baader) qui prétendaient communiquer directement avec Dieu et
recevoir de lui des pouvoirs extraordinaires :
% 187
« Même Kant et Hegel sont moins exempts de mysticisme et de théosophie qu'ils
ne le semblent » (Boutroux).

\ib{Thermique (Sens)} — \si{Ps. phol.} Celui qui nous donne les sensations de
chaud et celles de froid.

\ib{Thermodynamique} — \si{Épist.} Partie de la Physique$^6$ qui étudie les
rapports entre le travail$^1$ mécanique et Ia quantité de chaleur.

\ib{Thèse} — {\bf 1.} (Sens général). Ce que soutient un philosophe, un
écrivain, un orateur. — {\bf 2.} \si{Théol.} (Opp. {\it hypothèse}$^4$). La
doctrine à l'état pur.

— \si{Hist.} {\bf 3.} {\it Chez Kant} (opp. : {\it antithèse}$^2$) : le
premier membre des antinomies*. — {\bf 4.} {\it Chez Fichte} : position du
{\it Moi}* par lui-même sans rapport à rien d’étranger. — {\bf 5.} {\it Chez
Hegel et Hamelin} (opp. : {\it antithèse}$^2$ et {\it synthèse}$^2$) :
premier moment de l'opposition dialectique$^6$ qui se résout dans la
synthèse$^2$.

\ib{Thétique} — \si{Hist.} {\bf 1.} {\it Chez Fichte} (cf. {\it Thèse}$^4$) :
qualifie le jugement par lequel une chose est « seulement posée comme
identique à elle-même ». — {\it Lato.} {\bf 2.} (Syn. : {\it positionnel})
existentiel*, qui pose l'être comme existant, {\it opp.} à la conscience «
non-thétique de soi » (Sartre) qui ne prend pas conscience d’elle-même comme
existante.

\ib{Théurgie} — \si{Hist.} « Connaissance des pratiques nécessaires pour
faire agir l'influence divine où et quand on veut » (Bréhier), {\it p. e.}
chez Jamblique. Cf. {\it Théosophie}*.

\ib{Tiers exclu} — Voir {\it Milieu$^2$ exclu}.

\ib{Tolérance} — \si{Mor.} Attitude qui consiste, « non à renoncer à ses
convictions ou à s'abstenir de les manifester, mais à s’interdire tous moyens
% 187
violents, injurieux ou dolosifs pour les propager » (Goblot).

\ib{Topologie} — \si{Épist.} (Syn. : {\it analysis situs}, géométrie de
position). Forme de géométrie fondée sur la notion d’un espace non
quantitatif et où l’on ne considère que les relations de position des
éléments des figures.

\ib{Totalisation} — Voir {\it Rédintégration}*.

\ib{Totalitaire} — {\bf 1.} Intégral, complet, {\it not.} {\it chez Sartre} :
« Une conception totalitaire de l’homme » ({\it opp.} à la conception
analytique « bourgeoise » de l’homme abstrait). — {\bf 2.} \si{Pol.} Qui
concerne le totalitarisme* : « Les régimes totalitaires. »

\ib{Totalitarisme} — \si{Pol.} Système politique dans lequel toutes les
activités de l'être humain sont soumises à l'État : « Toute activité est
politique en puissance, et c'est comme animal politique que l’homme est saisi
dans sa totalité sur le plan existentiel » (Carl Schmitt).

\ib{Totalité} — \si{Crit.} {\bf 1.} Une des catégories kantiennes : synthèse
de l’unité et de la pluralité, elle commande les jugements singuliers*.

— \si{Vulg.} {\bf 2.} L'ensemble : « La totalité des possibles. »

\ib{Totem} — [mot algonquin] — \si{Soc.} Être mythique ({\it gén.} une espèce
animale, qqfs. végétale) considéré comme l'ancêtre éponyme du clan et auquel
on rend un culte. Selon Durkheim, le {\it totémisme} serait la forme
élémentaire de la religion (thèse discutée).

\ib{Toucher} — \si{Ps. phol.} Le « toucher » comprend : 1° le {\it tact}$^2$
ppt. dit: 2° le sens {\it thermique}*; 3° le sens {\it kinésique}* (lui-même
complexe) ; 4° le sens {\it algique}*. {\it Dist.} d'autre part toucher {\it
passif} et toucher {\it actif}.
% 188

\ib{Tout} (nom) — La totalité$^2$ considérée comme ensemble organique$^4$ et
original : « Un tout n’est pas identique à la somme de ses parties, il est
qqc. d’autre et dont les propriétés diffèrent de celles que présentent les
parties dont il est composé » (Durkheim).

\ib{Tradition} — \si{Soc.} Transmission {\it par la voie sociale} (orale,
écrite ou par les actes) des coutumes, institutions, croyances, souvenirs,
etc. communs à un groupe : € La tradition est, par excellence, le fait social
dans sa réalité positive » (Parodi). $->$ Dist. {\it hérédité}, transmission
par la voie {\it physiologique}.

\ib{Traditionalisme} — \fsb{S. posit.} \si{Vulg.} {\bf 1.} Attachement à la
tradition*, culte du passé. — \fsb{S. norma.} \si{Pol.} {\bf 2.} Doctrine
selon laquelle on doit se fier, en matière politique et religieuse, à la
tradition plus qu’à la raison$^5$ (de Bonald, de Maistre, Bautain).

\ib{Transcendance, Transcendant} — \si{Vulg.} {\bf 1.} Supériorité, supérieur
à la moyenne : « Un génie transcendant » (Rousseau).

— \si{Méta.} (Ctr. : immanent*). {\bf 2.} « Une réalité est transcendante {\it
par rapport} à {\it une autre} quand elle réunit les deux caractères : 1° de
lui être supérieure, d’appartenir à un degré plus élevé dans une hiérarchie;
2° de ne pouvoir être atteinte à partir de la première par un mouvement
continu » (Belot, {\it in} Lalande, {\it Vocabulaire}). —  {\bf 3.}
(Transcendance {\it absolue}). {\it Chez Kant} : qui est au-delà de toute
expérience : « Nous appellerons {\it immanents} les principes dont
l’application se tient dans les bornes de l’expérience possible ; mais {\it
transcendants} ceux qui doivent dépasser ces bornes » ({\it R. pure}, Dial.,
introd., I). {\it Chez Jaspers} : « la Transcendance », Dieu : « Le
Transcendant n'est pas dans la conscience, il la dépasse comme qqc. de tout
autre : c’est l’Absolu, en {\it opp.} avec la finitude, la relativité,
l’inachèvement… La Transcendance est au-delà de toute forme » ({\it Philos.},
I, 50 et III, 39). 57 Ces termes qui, au sens propre, impliquent toujours une
{\it supériorité} dans l’ordre des valeurs
% 188
ou de l’être, se sont altérés dans le lang. philosophique contemporain au
point : 1° de devenir simplement syn. d’{\it altérité}, d'{\it autre} : «
Toute application est transcendance » (Bachelard) ; « Par le fait
d’{\it être-là}*, l’homme ne peut se connaître comme tel qu’au sein de qqc.
qui le dépasse et qu’on peut appeler le monde : le monde est donc un
transcendant auquel est lié notre être ; le temps, inhérent à notre
existence, nous révèle une autre transcendance, celle du passé et du futur
» (Bréhier, exposant Heidegger) ; — 2° de s’{\it immanentiser} : « La
transcendance désigne l'essence du sujet [le {\it Dasein}$^2$], elle est la
structure fondamentale de la subjectivité » (Heïdegger) ; « Le philosophe
aura-t-il la force de transcender la transcendance elle-même et de tomber
vaillamment dans l'immanence ? » (Wahl) ; « La transcendance, après avoir été
rendue par Hegel horizontale, devient avec l’existentialisme une propriété du
sujet. Ainsi l'échec est transformé en triomphe, et le problème en solution
» (Alquié) ; — 3° de se fragmenter en {\it sous-notions} qui les nient : «
Une transcendance extra-mentale n’est pas concevable; au ctr., une
transcendance intramentale peut être conçue. À cette transcendance par
l'intimité de nous-même, on peut donner le nom d’{\it intratranscendance}.
Elle appelle d'autre part une {\it extratranscendance} qui résulte de ce
qu’il y aura toujours en Dieu un excès infini par rapport à ce que nous
sommes » (Le Senne) ; « Il y a une hiérarchie dirigée vers le bas, celle dont
un Lawrence a eu conscience quand il nous présentait au-dessous de nous, dans
les bases de l'être, le Dieu inconnu. Il n’y a pas seulement une
% 189
{\it transascendance}, il y a une {\it transdescendance} » (Wahl).

— \si{Math.} {\bf 4.} {\it Analyse transcendante} : le calcul infinitésimal.

\ib{Transcendantal} — (on écrit aussi {\bf Transcendental}) — \si{Hist.}
{\bf 1.} {\it Chez les Scolastiques} : attribut très général qui dépasse les
catégories d’Aristote : « Les termes appelés {\it Transcendantaux} tels que
Être, Chose, Quelque chose » [et qui, selon lui, sont faits d'images
confuses] (Spinoza, {\it Éth.}, II, 40, scolie 1).

— \si{Crit.} {\it Chez Kant} : {\bf 2.} Qui concerne les conditions {\it a
priori} de la connaissance : « Je nomme transcendantale toute connaissance
qui a affaire non pas tant aux objets que, de façon générale, à nos concepts
{\it a priori} des objets » ({\it R. pure}, introd., VII) ; « Il ne faut pas
nommer transcendantale toute connaissance {\it a priori}, mais celle-là seule
qui nous fait connaître que certaines représentations (intuitions ou
concepts) sont appliquées ou possibles exclusivement {\it a priori} et
comment elles le sont » ({\it ib.}, Log., II) ; « Un principe transcendantal
est celui qui représente la condition générale {\it a priori} sous laquelle
seule des choses peuvent devenir objets de notre connaissance » ({\it Jug.},
introd., V). {\it Aperception transcendantale} « cette conscience pure,
originaire et immuable » dont l'unité précède toutes les données de
l'intuition ({\it R. pure}, Analyt., I, 2, 2). {\it Idéalisme
transcendantal} : « doctrine d’après laquelle nous regardons les
phénomènes$^2$ dans leur ensemble comme de simples représentations et non
comme des choses en soi » ({\it ib.}, {\it Dial.}, II, 1, 4). Cf. {\it
Analytique*, Dialectique$^4$, Esthétique$^1$} et {\it Logique}$^5$. —
{\bf 3.} Qui dépasse les bornes de toute expérience possible : « L'usage
% 189
des concepts purs de l’entendement ne peut jamais être transcendantal, mais
seulement empirique » ({\it R. pure}, {\it Analyt.}, II, 3). {\it Apparence}
ou {\it Illusion} (Schein) {\it transcendantale} : voir {\it Dialectique$^4$.
Idées transcendantales} : voir {\it Idée}$^2$. {\it Objet transcendantal} :
la chose en soi (voir {\it En$^4$ soi}), i. e. « qqc. dont nous ne savons
rien et ne pouvons rien savoir », mais qui sert uniquement à « unifier le
divers de l'intuition sensible » ({\it R. pure}, Analyt., II, 3).

\ib{Transcendantalisme} — \si{Hist.} \fsb{S. norma.} Doctrine
philosophico-religieuse d’Emerson et de penseurs analogues : « Il faut
entendre [par {\it transcendantalisme}] que toute expérience, si minime
qu’elle soit, peut nous conduire à un au-delà qui nous révèle l'Univers... Le
transcendantalisme est l’objet d’une foi, non d’une démonstration » (Bréhier).

\ib{Transduction} — \si{Log.} Pseudo-raisonnement qui consiste en un passage
direct du singulier au singulier par simple juxtaposition, sans subordination
à un concept général.

\ib{Transfert} — \si{Psycho.} {\bf 1.} {\it Loi de transfert} : loi selon
laquelle la tonalité affective d’une représentation se communique aux autres
représentations associées à la première : « La loi de transfert gouverne
toute notre vie sentimentale » (Bouglé). — \si{Ps. an.} {\bf 2.} Report sur
le psychanalyste des sentiments jadis éprouvés par le sujet à l'égard d’un
des personnages de son passé.

\ib{Transfini (Nombre)} — \si{Math.} Nombre cardinal qui, dans la théorie des
ensembles*, sert à dénombrer les collections infinies ({\it p. e.} des
nombres entiers, des nombres pairs). — Voir {\it Aleph}*.
% 190

\ib{Transmigration des âmes} — \si{Hist.} \fsb{S. norma.} Doctrine
({\it not.} des pythagoriciens) selon laquelle les âmes passent dans
différents corps qu’elles animent successivement.

\ib{Transformisme} — \si{Biol.} (Ctr. : {\it fixisme}*). \fsb{S. norma.}
Théorie scientifique qui affirme l’évolution$^5$ des espèces vivantes.

\ib{Transitif} — Voir {\it Immanent}$^1$.

\ib{Transrationalisme} — \si{Hist.} Terme proposé par Cournot pour désigner
la « réaction de l’âme contre des habitudes d’abstraction qui la rebutent ».

\ib{Transrationnel} — \si{Crit.} Qui dépasse
la raison, sans pourtant la contredire.

\ib{Transsexualisme} — \si{Ps. path.} Sentiment obsédant qu’éprouve un sujet
d’appartenir au sexe autre que le sien, avec le désir d'en changer pour vivre
conformément à l’image qu'il se fait de lui-même.

\ib{Travail} — \si{Math.} {\bf 1.} En Mécanique : produit de l'intensité
d’une force$^4$ par la projection sur sa direction du déplacement subi par
son point d'appui.

— \si{Psycho.} {\bf 2.} Type d’action (cf. {\it Ergétique}*) par lequel
l'homme agit, selon certaines normes sociales, sur une matière pour la
transformer : « Le résultat auquel le travail aboutit, préexiste idéalement
dans l’imagination du travailleur » (K. Marx) ; « Les psychologues n’ont pas
donné une place suffisante à l'analyse du travail » (Janet). \si{Éc. soc.}
{\it Contrat de travail} : « contrat par lequel l’une des deux parties
s'engage à faire qac. pour l’autre moyennant un prix convenu entre elles
» ({\it C. C.}, 1710). {\it Liberté du travail} : droit qu’a tout individu de
travailler dans les conditions qu'il lui plaît d'accepter. {\it Droit au
travail} : droit$^5$ (revendiqué en 1848) qui oblige l'État à fournir du
travail aux individus. {\it Droit} au {\it produit intégral du travail} :
droit$^5$
% 190
que le travailleur possède sur la richesse qu’il a produite sans avoir à
subir aucun prélèvement dû à la possession privée des moyens de production.

\ib{Tribu} — \si{Soc.} Groupe de clans* possédant ses institutions propres,
qqfs. son culte, et pourvu d’un gouvernement distinct des gouvernés.

\ib{Tropes} — \si{Hist.} Arguments par lesquels les Sceptiques grecs
prétendaient montrer l'impossibilité d’atteindre des vérités certaines. Cf.
{\it Textes choisis}, II], p. 28 {\bf 2.}

\ib{Tropisme} — \si{Biol.} Phénomène d'orientation sur place d’un être vivant
sous l’action d’un agent physique extérieur {\it p. e.} géotropisme,
phototropisme. {\it Cf.} {\it Précis}, Ph. I, p. 439.

\ib{Truisme} — [de l’angl. {\it true}, vrai] — \si{Log.} Vérité banale et
trop évidente pour mériter d'être énoncée.

\ib{Tsédek} — [mot hébreu = juste] \si{Psycho.} Test de jugement moral
inventé par H. Baruk.

\ib{Tychisme} — [G. {\it tyché}, hazard] — \fsb{S. norma.} \si{Épist.}
Doctrine qui affirme l’existence, dansle monde, d’un hasard$^1$ radical.

\ib{Type} — \si{Méta.} {\bf 1.} Modèle idéal d’où dérive un ensemble
d'objets : {\it p. e.} l’Idée$^1$ platonicienne (cf. {\it Archétype}$^1$).
$->$ {\it Dist.} genre$^1$ ou espèce$^2$ : « Des choses peuvent être de même
espèce sans qu'il y ait des motifs d'admettre qu'elles dérivent du même type
» (Cournot). — {\bf 2.} \si{Biol.}, \si{Soc.}, \si{Psycho.} Loi de structure
(cf. {\it Coordonné}$^2$) : « Le type oiseau » ; « Un type racial »; « Les
types de caractère »; « Les types sociaux » (v. {\it Précis}, Ph. Il, p. 206;
Se. et M., p. 315) ; « Les types moraux » (Rauh). —  {\bf 3.} \si{Vulg.}
Exemplaire individuel partieulièrement représentatif : « Cet homme est le
type de l’ambitieux. »
%191

\begin{center}
U
\end{center}

\ib{Ubiquité} — [L. {\it ubique}, partout] — \si{Méta.} (Syn. {\it
omniprésence}). Caractère de l'être qui est partout présent « L’ubiquité
divine. »

\ib{Ultra-choses} — \si{Psycho.} {\it Chez Wallon} : objets qui « dépassent
les données sensibles », {\it p. e.} les origines, la vie, la mort, le ciel.

\ib{Un (L')} — \si{Hist.} {\it Chez Plotin} : l'hypostase* la plus élevée
d’où procèdent* les deux autres et qui est le Bien absolu. Cf. {\it Textes
choisis}, II, p. 345.

\ib{Unicité} — Caractère de ce qui est {\it unique} : « L’unicité de Dieu. »

\ib{Unité} — \fsb{S. abstr.} {\bf 1.} Caractère de ce qui est {\it un}, de ce
qui forme un tout organique$^4$ : « L'unité du moi » ; « L'unité sociale ».
En ce sens, \si{Crit.}, une des catégories kantiennes, qui commande les
jugements universels$^3$.

— \si{Épist.} \fsb{S. concr.} {\bf 2.} Élément d’un tout : « Une unité »
{\it Spéc.} \si{Math.} Le nombre 1. — {\bf 3.} Base d’un système de mesure :
« Les unités M. K. S. A. »

\ib{Univers} — {\bf 1.} Tout ce qui existe dans le temps et dans l’espace : «
L’univers est l'ensemble de tous les êtres créés » (Bonnet) ; « Il y a cette
différence entre le monde et l'univers que le monde est infini » (Diderot). —
{\bf 2.} Le monde visible et qqfs. {\it spéc.} le monde humain : « Dans ce
petit cachot où il [l’homme] se trouve logé, j'entends l'univers,...
» (Pascal, 72) ; « Voilà les spectacles que Dieu donne à l'univers
» (Bossuet). — {\bf 3.} Un monde particulier, vu d’un point de vue
déterminé : « L'univers de mon imagination » (Staël). {\it Spéc.} \si{Log.}
{\it Univers du discours} : la Classe$^1$ totale, ensemble des idées
auxquelles on se réfère
% 191
dans un jugement ou un raisonnement (on le symbolise par le signe $\bigvee$).
%\/

\ib{Universaux} — \si{Hist.} Chez les {\it Scolastiques} : les idées
générales : « Ce que vous alléguez contre les universaux des dialecticiens$^2$
ne me touche point » (Descartes, 5$^\text{es}$ {\it Rép.}). Qqfs., les idées
{\it les plus} générales (universaux de Porphyre) : « On compte ordinairement
cinq universaux : le genre, l'espèce, la différence, le propre et l'accident
» (Descartes, {\it Princ.}, I, 59).

\ib{Universel} — \si{Vulg.} {\bf 1.} Qui concerne l'univers$^1$ : « Moi qui
n'existe que par la force d’une nature universelle » (La Bruyère) ; « L'ordre
universel ». — {\bf 2.} Qui concerne l’univers$^2$ et {\it spéc.} le monde
humain : « Les lois universelles de la nature » ; « Un fluide universel,
extrêmement subtil [l'éther]… » (Fresnel) ; « Le consentement* universel » ;
« Le suffrage universel ».

— \si{Log.} \si{form.} {\bf 3.} {\it Proposition universelle} : celle où le
sujet$^2$ est pris dans toute son extension : {\it p. e.} « Tous les hommes
sont mortels ». — {\bf 4.} Se dit qqfs. des termes : « pris universellement »
= dans toute son extension$^3$. Imppt., syn. de {\it général}$^2$ : « Les
concepts universels. »

— \si{Méta.} {\bf 5.} {\it Chez Hegel} : « universel concret », être réel «
dont le concept est la synthèse, à la fois universelle$^4$, puisqu'il est
susceptible d’un nombre indéfini d'applications, et concrète, en tant qu'il
est une totalité unique et indivisible » (Lalande) : {\it p. e.} la volonté
collective d’un peuple. {\it Cf.} Hamelin : « Comment l'{\it Universel
concret} est-il concret ? Il l'est en ce sens et en ce sens seul, qu'il nie
toute la série des déterminations dont se compose le fini. Il n’est pas autre
chose que la réalité éminente de l’École*, l'être
% 192
indéterminé de Malebranche, la substance$^2$ de Spinoza et enfin, comme Hegel
nous invite à le dire l’abime suprême du Panthéisme oriental. »

\ib{Univoque} — \si{Log.} Qui s'applique avec le même sens dans deux ou
plusieurs cas différents : « Le nom de substance n'est pas univoque au regard
de Dieu et des créatures » (Descartes, {\it Princ.}, I, 51).

\ib{Urgence} — \si{Méta.} {\it Chez Le Senne} : « oppression du moi par une
situation à la fois constrictive et menaçante » : « L’urgence a pour intimité
la souffrance. »

\ib{Ustensilité} — [{\it Trad.} all. {\it Zeughaftigkeit}] — \si{Méta.} {\it
Chez Heidegger} : caractère purement pragmatique du monde dans lequel vit
l’être quotidien.

\ib{Utilitaire} — \si{Vulg.} {\bf 1.} (Gén. {\it péj.}) Qui concerne ou
considère uniquement la vie pratique$^1$ ou l'intérêt personnel : « Un esprit
bassement utilitaire ».

— \si{Hist.} {\bf 2.} Qui concerne ou professe l’utilitarisme* : « La doctrine
utilitaire » ; « Les philosophes utilitaires ».

\ib{Utilitarisme} — [Angl. {\it Utilitarianism}] — \si{Hist.} \fsb{S. norma.}
Système de morale qui prend pour principe « l’utile ou le principe du plus
grand bonheur » (J. Stuart Mill).

\ib{Utopie} — [G. {\it ou} et {\it topos}, qui n’est en aucun lieu] —
{\bf 1.} Description d’une société idéale (de l’{\it Utopia} de Th. Morus,
1516) : « Les utopies sont comme des enveloppes de brume sous lesquelles
s’avancent des idées neuves et réalisables » (Ruyer). — {\bf 2.}
({\it Péj.}). Conception irréalisable : « L'égalité absolue est une utopie ».
— {\bf 3.} {\it Chez Mannheim} : voir {\it Idéologie}$^4$.

	\end{itemize}
