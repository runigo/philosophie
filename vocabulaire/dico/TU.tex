	\begin{itemize}[leftmargin=1cm, label=\ding{32}, itemsep=1pt]
%183
\ib{Table} — Épist. 1. Chez Bacon : « tables
de comparution », registres d’observations dans lesquelles on « fait
comparaître » les faits naturels pour
en découvrir « l'essence » (cf. Précis,
Ph. II, p. 122 ; Sc. et M., p. 238). —
2. Table rase [L. tabula rasa, tablette de cire où il n’y a rien d’écrit] :
expression empruntée à Aristote
par les empiristes* pour désigner
l’état de l'esprit qui, avant toute
expérience, serait d’après eux absolument vide.

\ib{Tabou} [mot polynésien]. — Soc.
1. Str. Interdit rituel qui protège
une chose ou un être réputé sacré$^2$.
— 2. Ext. Prohibition morale ou coutumière : « Les tabous sexuels » ;
« Certains métiers sont tabou pour
le bourgeois » (Goblot).

\ib{Tact} — Ps. phol. 1. Laito. Le sens du
toucher* en gén. — 2. Str. Le sens
du contact et des pressions, du lisse,
du rugueux, etc.

— Car. 3. Délicatesse morale :
« Ce tact attentif de l'esprit qui fait
sentir les nuances des fines convenances » (Buffon).

\ib{Tactile} — Qui se rapporte au tact$^2$.

\ib{Tactisme} — Phol. Mouvement de
déplacement produit automatiquement chez un animal par un agent
physique extérieur (vg. chimiotactisme des infusoires).

\ib{Tautologie} [G. to auton legein, dire la
même chose]. — Épist. 1. Proposition identique$^4$. — 2. Chez les logisticiens : proposition purement formelle$^3$ qui n’a de sens que par rapport
à des objets possibles, mais qui demeure vraie quels que soient ces
objets (opp. : protocole*).

— 3. (Péj) Faute logique qui
% 184
consiste à présenter une répétition
en termes différents comme un progrès$^1$ de la pensée.

\ib{Taxinomie ou Taxonomie} — Épist.
Ancien nom de la systématiques$^3$.

\ib{Technique (nom)} — 1. Ensemble des
procédés d’un art, d’un métier ou
d’une science : « La technique du
cinéma »; « La technique de l'enseignement ». — 2. La science appliquée, spéc. dans le domaine industriel : « Les progrès$^2$ de la technique ». Qqfs. péj. : « Comment a-t-il
[l'esprit européen] remporté ses
succès meurtriers ? Par la technique !
Et qui créa cet instrument de
meurtre ? La science, qui compte,
mesure et pèse » (Klages).

\ib{Technologie} — Épist. Étude des
techniques$^2$ : « La Technologie est
symétrique dans le domaine de
l’action à la Logique dans le domaine
de la connaissance » (Espinas).

\ib{Tèlé} [mot grec — loin] — Soc. En
Sociométrie*, facteur qui mesure la
« distance sociale » et l'attraction
entre individus ou groupes.

\ib{Téléologie} [G. telos, fin, et logos, étude].
— Méta. 1. A. Doctrine finaliste* :
« Ce serait parce que nous désirons
les choses qu’elles seraient bonnes :
la proposition de Spinoza serait
ainsi retournée en faveur de la téléologie » (Hamelin). — 2, La finalité
elle-même : « C’est bien dans les
concepts qu'il convient de situer
l’origine primordiale de la téléologie » (id.).

\ib{Téléologique} — Méta. Qui suppose de
la finalité* : « Les explications téléologiques ne sont ni partout désirables ni suffisantes » (Hamelin) ;
« L'adaptation est une manifestation téléologique ou elle n’est rien »
% 184
(Vandel). Preuve téléologique ou
Physico-téléologique : celle qui vise
à prouver l'existence de Dieu par
les causes finales$^1$.

\ib{Télépathie} [G. têlé, loin, et pathos, sentiment]. — Psycho. Communication
directe de la pensée à grande distance.

\ib{Témoignage} — Épist. Str. 1. Attestation volontaire d’un fait par un
témoin : « Un témoignage juste
n’est pas la règle, mais l'exception »
(Claparède). — Lato. 2. Document,
en gén. : « Une idée surannée : celle
du temps où l’on ne savait lire que
les témoignages volontaires »
(M. Bloch). — Latiss. 3. Preuve,
indice : « Cet appétit [d’intériorité]
est le témoignage, chez l’homme
d’auj., de l'effort pour ne pas se
laisser aspirer par les choses extérieures » (Bréhier). $->$ Le seul sens
propre est le sens 1.

\ib{Tempérament} — Car. Ensemble des
dispositions organiques d’un individu : « L’esprit dépend si fort du
tempérament et de la disposition
des organes du corps que... » (Descartes, Méth., VI) ; « Le caractère
moral$^4$ n’est que la physionomie du
tempérament physique » (Bichat) ;
« Le tempérament sanguin ».

\ib{Tempérance} — Mor. La vertu de la
sensibilité, qui consiste dans la modération à l'égard des plaisirs sensibles et aussi à l'égard des émotions
et des passions.

\ib{Temporalité} — Méta. Dans le lang.
existentialiste : caractère du Dasein$^2$
qui est à la fois solidaire de son
passé et pro-jet* vers l'avenir,

\ib{Temps} — Crit. et Méta. 1. Milieu
homogène et infini dans lequel nous
paraissent se dérouler les événements :
% 185
« Le temps n’est rien qu’une
certaine façon dont nous pensons à
la durée$^1$ » (Descartes, Princ., I, 57) ;
« Tout le temps de ma vie peut être
divisé en une infinité de parties,
chacune desquelles ne dépend en
aucune façon des autres » (id., Méd.,
III) ; « Le temps n’est que la condition subjective$^1$ sous laquelle les
intuitions sont possibles en nous »
(Kant, R. pure, Esth., § 6) ; « I y
aurait lieu de se demander si le
temps, conçu sous la forme d’un
milieu homogène, ne serait pas un
concept bâtard, dû à l’intrusion de
l’idée d'espace dans le domaine de
la conscience pure » (Bergson, D. I.
II ; cf. Durée$^2$). — 2. Portion finie
du temps$^1$ : « Un temps est une portion de durée mesurée » (Destutt
de Tracy). Spéc. Ps. métr. Temps de
réaction : durée qui s'écoule entre
l'excitation$^1$ et la réponse du sujet$^5$.
— 3. Époque : « En ce temps-là... ».

\ib{Tendance} — Psycho. 1. Str. Forme
spontanée de l’activité$^2$ : « La tendance ne nous est donnée que par
l’affection » (Lachelier). On peut,
avec Pradines, dist. la « tendance
à... », dynamisme mental indifiérencié, et la « tendance vers... »,
activité spécifiée à objet déterminé.
— 2. Lato. Tout mode actif$^3$ de la
vie psychique : « La tendance est
une forme dynamique et plurivalente, abstraite et pourtant réelle,
innée ou acquise, qui détermine
directement ou indirectement un
acte » (Burloud). Cf. Schème$^3$ et
Thème$^2$. $->$ Le sens 1 est le sens
propre.

\ib{Tension psychologique} — Psycho.
Notion introduite par Janet pour
caractériser les différents degrés du
niveau mental. Sont actes de « haute
tension » ceux où : 1° J’unification
des éléments est forte (ef. Synthèse$^4$) ;

2° la masse des éléments est considérable (vg. l'attention).

\ib{Terme} — Log. form. Concept$^1$ représenté pour son expression verbale :
les deux termes d’une proposition
sont le sujet$^2$ et l’attribut$^1$.

\ib{Terminisme} — Syn. de nominalisme$^1$ :
« Le terminisme de G. d’Occam. »

\ib{Terminologie} — Ling. Ensemble des
termes spéciaux d’une science ou
d’un auteur.

\ib{Test} [mot anglais = épreuve]. Psycho.
Ensemble d'épreuves permettant
de déterminer chez un individu
ou un groupe d'individus, soit
leur niveau mental (tests d'âge
ou de développement), soit la présence et le degré de tel caractère
mental (tests d’aptitudes, vg. de
mémoire, d'attention, etc.), soit les
caractéristiques de la personnalité
(tests de personnalité). Cf. Précis,
Ph. II, p. 220-229; Sc. et M., p. 330-338.

\ib{Thanatormée} [G. thanatos, mort, et
hormê, tendance]. — Ps. an. Instinct$^4$
de mort, de destruction.

\ib{Thaumaturge} [G. thauma, miracle, et
ergon, œuvre]. — Théol. Qui fait
des miracles : « Certains néo-platoniciens se faisaient passer pour
thaumaturges. »

\ib{Théandrique} [G. theos, dieu, et anêr,
andros, homme]. — Méta. Qui concerne Dieu et l’homme : « Le rapport
théandrique est toujours à la fois
donné et idéal » (Le Senne).

\ib{Théisme} — Méta. (Opp. : athéisme* et
panthéisme*). À. Doctrine qui admet
l'existence d’un Dieu* comme cause
première$^4$ transcendante$^2$ au monde.
$->$ Sur la différence entre théisme
ancien et théisme moderne, voir
% 186
Précis, Ph. II, p. 510-512; Sc.,
p. 446-448 ; et M., p. 426-427. Cf.
Déisme* et Démiurge*.

\ib{Thème} — 1. Sujet de réflexion ou de
discussion : « Un thème philosophique ». — 2. Psycho. Chez Burloud (syn. : tendance thématique opp.
à tendance schématique) : « ensemble
des rapports qui constituent la signification commune d’une catégorie d'objets. »

\ib{Théocratie} — Pol. Système politique
caractérisé par la domination de la
caste sacerdotale : « Les premiers
Incas établirent une théocratie »
(Voltaire).

\ib{Théodicée} [G. theos, dieu, et diké, plaidoyer]. — Méta. 1. Chez Leibniz :
partie de la métaphysique qui
traite du problème du mal* et « jus- 7
tifie » la bonté de Dieu contre les
objections tirées de ce problème. —
Ext. 2. Dans l'école de Cousin, et
imppt : syn. de théologie au sens 1.

\ib{Théologie} — Méta. 1. (Théologie naturelle ou rationnelle). Partie de la
métaphysique qui traite de l’existence et des attributs de Dieu en
s'appuyant uniquement sur la
raison$^5$ — Théol. 2. (Théologie révélée ou dogmatique). Étude des
dogmes de la foi$^5$, fondée sur les
textes sacrés et l'autorité de l’Église :
« La théologie qui se rapporte à
l’enseignement sacré, diffère en nature de cette autre théologie$^1$ qui
se pose comme partie de la philosophie » (St. Thomas, S. th, I, 1, 1).

— Hist. 3. Un système particulier de théologie au sens 1 ou 2 : « La
théologie de Plotin. »

\ib{Théologique} — 1. Qui concerne la
théologie*, surtout au sens 2. —
2. Hist. Chez Comte : « état théologique », état initial (cf. États*) de
% 186
l'esprit humain, caractérisé par la
croyance à des « agents surnaturels
plus ou moins nombreux dont l’intervention arbitraire explique les
anomalies apparentes de la nature »
(Cours, I) ; l’état théologique comprend trois stades successifs : le
fétichisme$^1$, le polythéisme et le monothéisme.

\ib{Théorétique} — Épist. 1. Qui concerne
la théorie$^1$. — Hist. 2. Chez Aristote :
« sciences théorétiques », celles qui
ont pour but la spéculation$^1$ pure
(vg. mathématiques, physique, métaphysique, opp. « sciences poétiques [= créatrices] », telles que
poésie, rhétorique, dialectique$^2$, et
« sciences pratiques [= de l'action] »,
telles que éthique, économie et politique.

\ib{Théorie} [G. théôrein, contempler]. —
Épist. O. 1. (Ctr. : pratique$^3$). La
connaissance spéculative*, la pensée
désintéressée : « La théorie a été
souvent l’origine de recherches pratiques » (Picard). — @. 2. Ensemble
de conceptions systématiquement$^1$
organisées sur un sujet déterminé,
not. dans les sciences expérimentales
(syn. : hypothèse$^3$ générale) : « Nous
insisterons sur l'importance des
immenses constructions que bâtissent les savants sous le nom de
théories » (id.) ; « La théorie électronique » ; « Les théories d’Einstein ».

\ib{Théorique} — Spéculatif$^1$ $->$ Dist.
abstrait : théorique s’opp. à pratique ; mais abstrait s'opp. à concret.

\ib{Théosophie} — Hist. À. Nom générique
des doctrines de certains illuminés
(vg. Paracelse, Weigel, Bœhme,
Swedenborg, Hamann, Saint-Martin,
Baader) qui prétendaient communiquer directement avec Dieu et
recevoir de lui des pouvoirs extraordinaires :
% 187
« Même Kant et Hegel
sont moins exempts de mysticisme
et de théosophie qu'ils ne le semblent » (Boutroux).

\ib{Thermique (Sens)} — Ps. phol. Celui
qui nous donne les sensations de
chaud et celles de froid.

\ib{Thermodynamique} — Épist. Partie
de la Physique$^6$ qui étudie les rapports
entre le travail$^1$ mécanique et Ia
quantité de chaleur.

\ib{Thèse} — 1. (Sens général). Ce que
soutient un philosophe, un écrivain,
un orateur. — 2. Théol. (Opp.
hypothèse$^4$). La doctrine à l'état pur.

— Hist. 3. Chez Kant (opp. : antithèse$^2$) : le premier membre des antinomies*. — 4. Chez Fichte : position du Moi* par lui-même sans
rapport à rien d’étranger. — 5. Chez
Hegel et Hamelin (opp. : antithèse$^2$
et synthèse$^2$) : premier moment de
l'opposition dialectique$^6$ qui se résout dans la synthèse$^2$.

\ib{Thétique} — Hist. 1. Chez Fichte (cf.
Thèse$^4$) : qualifie le jugement par
lequel une chose est « seulement
posée comme identique à elle-même ». — Lato. 2. (Syn. : positionnel) existentiel*, qui pose
l'être comme existant, opp. à la
conscience « non-thétique de soi »
- (Sartre) qui ne prend pas conscience
d’elle-même comme existante.

\ib{Théurgie} — Hist. « Connaissance des
pratiques nécessaires pour faire agir
l'influence divine où et quand on
veut » (Bréhier), vg. chez Jamblique.
Cf. Théosophie*.

\ib{Tiers exclu} — Voir Milieu$^2$ exclu.

\ib{Tolérance} — Mor. Attitude qui consiste, « non à renoncer à ses convictions ou à s'abstenir de les manifester, mais à s’interdire tous moyens
% 187
violents, injurieux ou dolosifs pour
les propager » (Goblot).

\ib{Topologie} — Épist. (Syn. : analysis
situs, géométrie de position). Forme
de géométrie fondée sur la notion
d’un espace non quantitatif et où
l’on ne considère que les relations
de position des éléments des figures.

\ib{Totalisation} — Voir Rédintégration*.

\ib{Totalitaire} — 1. Intégral, complet,
not. chez Sartre : « Une conception
totalitaire de l’homme » (opp. à la
conception analytique « bourgeoise »
de l’homme abstrait). — 2. Pol.
Qui concerne le totalitarisme* : « Les
régimes totalitaires. »

\ib{Totalitarisme} — Pol. Système poli
tique dans lequel toutes les activités
de l'être humain sont soumises à
l'État : « Toute activité est politique en puissance, et c'est comme
animal politique que l’homme est
saisi dans sa totalité sur le plan
existentiel » (Carl Schmitt).

\ib{Totalité} — Crit. 1. Une des catégories
kantiennes : synthèse de l’unité et
de la pluralité, elle commande les
jugements singuliers*.

— Vulg. 2. L'ensemble : « La totalité des possibles. »

\ib{Totem} [mot algonquin]. — Soc. Être
mythique (gén. une espèce animale,
qqfs. végétale) considéré comme
l'ancêtre éponyme du clan et auquel
on rend un culte. Selon Durkheim,
le totémisme serait la forme élémentaire de la religion (thèse discutée).

\ib{Toucher} — Ps. phol. Le « toucher »
comprend : 1° le tact$^2$ ppt. dit: 2° le
sens thermique*; 3° le sens kinésique* (lui-même complexe) ; 4° le
sens algique*. Dist. d'autre part
toucher passif et toucher actif.
% 188

\ib{Tradition} — Soc. Transmission par
la voie sociale (orale, écrite ou par
les actes) des coutumes, institutions, croyances, souvenirs, etc.
communs à un groupe : € La tradition est, par excellence, le fait social
dans sa réalité positive » (Parodi).
$->$ Dist. hérédité, transmission par
la voie physiologique.

\ib{Traditionalisme} — A. Vulg. 1. Attachement à la tradition*, culte du
passé. — À. Pol. 2. Doctrine selon
laquelle on doit se fier, en matière
politique et religieuse, à la tradition
plus qu’à la raison$^5$ (de Bonald, de
Maistre, Bautain).

\ib{Transcendance, Transcendant} — Vulg. 1.
Supériorité, supérieur à la moyenne :
« Un génie transcendant » (Rousseau).

— Méta. (Ctr. : immanent*). 2.
« Une réalité est transcendante par
rapport à une autre quand elle réunit
les deux caractères : 1° de lui être
supérieure, d’appartenir à un degré
plus élevé dans une hiérarchie; 2° de
ne pouvoir être atteinte à partir de la
première par un mouvement continu »
(Belot, in Lalande, Vocabulaire). —
3. (Transcendance absolue). Chez
Kant : qui est au-delà de toute expérience : « Nous appellerons immanents les principes dont l’application se tient dans les bornes de
l’expérience possible; mais transcendants ceux qui doivent dépasser
ces bornes » (R. pure, Dial., introd.,
I). Chez Jaspers : « la Transcendance », Dieu : « Le Transcendant
n'est pas dans la conscience, il la
dépasse comme qqc. de tout autre :
c’est l’Absolu, en opp. avec la finitude, la relativité, l’inachèvement.…
La Transcendance est au-delà de
toute forme » (Philos., I, 50 et
III, 39). 57 Ces termes qui, au
sens propre, impliquent toujours
une supériorité dans l’ordre des valeurs
% 188
ou de l’être, se sont altérés dans
le lang. philosophique contemporain
au point : 1° de devenir simplement syn. d’altérité, d'autre : « Toute
application est transcendance » (Bachelard) ; « Par le fait d’étre-là*,
l’homme ne peut se connaître comme
tel qu’au sein de qqc. qui le dépasse
et qu’on peut appeler le monde : le
monde est donc un transcendant
auquel est lié notre être; le temps,
inhérent à notre existence, nous
révèle une autre transcendance,
celle du passé et du futur » (Bréhier,
exposant Heidegger) ; — 2° de
s’immanentiser : « La transcendance
désigne l'essence du sujet [le Dasein$^2$], elle est la structure fondamentale de la subjectivité » (Heïdegger) ; « Le philosophe aura-t-il la
force de transcender la transcendance elle-même et de tomber vaillamment dans l'immanence ? »
(Wahl) ; « La transcendance, après
avoir été rendue par Hegel horizontale, devient avec l’existentialisme
une propriété du sujet. Ainsi l'échec
est transformé en triomphe, et le
problème en solution » (Alquié) ; —
3° de se fragmenter en sous-notions
qui les nient : « Une transcendance
extra-mentale n’est pas concevable;
au ctr., une transcendance intramentale peut être conçue. À cette
transcendance par l'intimité de
nous-même, on peut donner le nom
d’intratranscendance. Elle appelle
d'autre part une extratranscendance
qui résulte de ce qu’il y aura toujours en Dieu un excès infini par
rapport à ce que nous sommes » (Le
Senne) ; « Il y a une hiérarchie dirigée vers le bas, celle dont un
Lawrence a eu conscience quand il
nous présentait au-dessous de nous,
dans les bases de l'être, le Dieu
inconnu. Il n’y a pas seulement une
% 189
transascendance, il y a une transdescendance » (Wahl).

— Math. 4. Analyse transcendante : le calcul infinitésimal.

\ib{Transcendantal} (on écrit aussi Transcendental). — Hist. 1. Chez les Scolastiques : attribut très général qui
dépasse les catégories d’Aristote :
« Les termes appelés Transcendantaux tels que Être, Chose, Quelque
chose » [et qui, selon lui, sont faits
d'images confuses] (Spinoza, Éth.,
II, 40, scolie 1).

— Crit. Chez Kant : 2. Qui concerne les conditions a priori de la
connaissance : « Je nomme transcendantale toute connaissance qui a
affaire non pas tant aux objets que,
de façon générale, à nos concepts a
priori des objets » (R. pure, introd.,
VII) ; « Il ne faut pas nommer transcendantale toute connaissance a priori, mais celle-là seule qui nous
fait connaître que certaines représentations (intuitions ou concepts)
sont appliquées ou possibles exclusivement a priori et comment elles
le sont » (ib., Log., II) ; « Un principe transcendantal est celui qui
représente la condition générale
a priori sous laquelle seule des
choses peuvent devenir objets de
notre connaissance » (Jug., introd.,
V). Aperception transcendantale
« cette conscience pure, originaire et
immuable » dont l'unité précède
toutes les données de l'intuition
(R. pure, Analyt., I, 2, 2). Idéalisme transcendantal : « doctrine
d’après laquelle nous regardons les
phénomènes$^2$ dans leur ensemble
comme de simples représentations
et non comme des choses en soi »
(ib., Dial., II, 1, 4). Cf. Analytique*,
Dialectique$^4$, Esthétique$^1$ et Logique$^5$.
— 3, Qui dépasse les bornes de
toute expérience possible : « L'usage
% 189
des concepts purs de l’entendement
ne peut jamais être transcendantal,
mais seulement empirique » (R. pure,
Analyt., II, 3). Apparence ou Illusion (Schein) transcendantale : voir
Dialectique$^4$. Idées transcendantales :
voir Idée$^2$. Objet transcendantal : la
chose en soi (voir En$^4$ soi), i e.
« qqc. dont nous ne savons rien et
ne pouvons rien savoir », mais qui
sert uniquement à « unifier le divers
de l'intuition sensible » (R. pure,
Analyt., II, 3).

\ib{Transcendantalisme} — Hist. A. Doctrine philosophico-religieuse d’Emerson et de penseurs analogues : « Il
faut entendre [par iranscendantalisme] que toute expérience, si minime qu’elle soit, peut nous conduire
à un au-delà qui nous révèle l'Univers... Le transcendantalisme est
l’objet d’une foi, non d’une démonstration » (Bréhier).

\ib{Transduction} — Log. Pseudo-raisonnement qui consiste en un passage
direct du singulier au singulier par
simple juxtaposition, sans subordination à un concept général.

\ib{Transfert} — Psycho. 1. Loi de trans
jert : loi selon laquelle la tonalité
affective d’une représentation se
communique aux autres représentations associées à la première : « La loi
de transfert gouverne toute notre
vie sentimentale » (Bouglé). —
Ps. an. 2. Report sur le psychanalyste des sentiments jadis éprouvés
par le sujet à l'égard d’un des personnages de son passé.

\ib{Transfini (Nombre)} — Math. Nombre
cardinal qui, dans la théorie des
ensembles*, sert à dénombrer les
collections infinies (vg. des nombres
entiers, des nombres pairs). — Voir
Aleph*.
% 190

\ib{Transmigration des âmes} — Hist.
A. Doctrine (not. des pythagoriciens) selon laquelle les âmes passent dans différents corps qu’elles
animent successivement.

\ib{Transformisme} — Biol. (Ctr. : firisme*).
À. Théorie scientifique qui
affirme l’évolution$^5$ des espèces
vivantes.

\ib{Transitif} — Voir Immanent$^1$.

\ib{Transrationalisme} — Hist. Terme
proposé par Cournot pour désigner
la « réaction de l’âme contre des
habitudes d’abstraction qui la rebutent ».

\ib{Transrationnel} — Crit. Qui dépasse
la raison, sans pourtant la contredire.

\ib{Travail} — Math. 1. En Mécanique :
produit de l'intensité d’une force$^4$
par la projection sur sa direction
du déplacement subi par son point
d'appui.

— Psycho. 2. Type d’action (cf.
Ergétique*) par lequel l'homme agit,
selon certaines normes sociales, sur
une matière pour la transformer :
« Le résultat auquel le travail aboutit, préexiste idéalement dans l’imagination du travailleur » (K. Marx) ;
« Les psychologues n’ont pas donné
une place suffisante à l'analyse du
travail » (Janet). Éc. soc. Contrat de
travail : « contrat par lequel l’une
des deux parties s'engage à faire
qac. pour l’autre moyennant un
prix convenu entre elles » (C. C.,
1710). Liberté du travail : droit qu’a
tout individu de travailler dans les
conditions qu'il lui plaît d'accepter.
Droit au travail : droit$^5$ (revendiqué
en 1848) qui oblige l'État à fournir
du travail aux individus. Droit au
produit intégral du travail : droit$^5$
% 190
que le travailleur possède sur la
richesse qu’il a produite sans avoir
à subir aucun prélèvement dû à la
possession privée des moyens de
production.

\ib{Tribu} — Soc. Groupe de clans* possédant ses institutions propres, qqfs.
son culte, et pourvu d’un gouvernement distinct des gouvernés.

\ib{Tropes} — Hist. Arguments par lesquels les Sceptiques grecs prétendaient montrer l'impossibilité d’atteindre des vérités certaines. Cf.
Textes choisis, II], p. 282.

\ib{Tropisme} — Biol. Phénomène d'orientation sur place d’un être vivant
sous l’action d’un agent physique
extérieur vg. géotropisme, phototropisme. Cf. Précis, Ph. I,
p. 439.

\ib{Truisme} [de l’angl. true, vrai]. — Log.
Vérité banale et trop évidente pour
mériter d'être énoncée.

\ib{Tsédek} [mot hébreu = juste]. Psycho.
Test de jugement moral inventé
par H. Baruk.

\ib{Type} — Méta. 1. Modèle idéal d’où
dérive un ensemble d'objets : vg.
l’Idée$^1$ platonicienne (cf. Archétypel). $->$ Dist. genre$^1$ ou espèce$^2$ :
« Des choses peuvent être de même
espèce sans qu'il y ait des motifs
d'admettre qu'elles dérivent du
même type » (Cournot). — 2. Biol.,
Soc., Psycho. Loi de structure (cf.
Coordonné$^2$) : « Le type oiseau » ; « Un
type racial »; « Les types de caractère »; « Les types sociaux » (v. Précis,
Ph. Il, p. 206; Se. et M., p. 315) ;
« Les types moraux » (Rauh). —
3. Vulg. Exemplaire individuel partieulièrement représentatif : « Cet
homme est le type de l’ambitieux. »
%191

\begin{center}
U
\end{center}

\ib{Ubiquité} [L. ubique, partout]. — Méta.
(Syn. omniprésence). Caractère de
l'être qui est partout présent
« L’ubiquité divine, »

\ib{Ultra-choses} — Psycho. Chez Wallon :
objets qui « dépassent les données
sensibles », vg. les origines, la vie, la
mort, le ciel.

\ib{Un (L')} — Hist. Chez Plotin : l'hypostase* la plus élevée d’où procèdent*
les deux autres et qui est le Bien
absolu. Cf. Textes choisis, II, p. 345.

\ib{Unicité} — Caractère de ce qui est
unique : « L’unicité de Dieu. »

\ib{Unité} — O 1. Caractère de ce quiest un,
de ce qui forme un tout organique$^4$ :
« L'unité du moi »; « L'unité sociale ».
En ce sens, Crit., une des catégories
kantiennes, qui commande les jugements universels$^3$.

— Épist, @ 2. Élément d’un tout :
« Une unité » Spéc. Math. Le
nombre 1. — 3. Base d’un système
de mesure : « Les unités M. K. S. A. »

Univers\ib{} — 1. Tout ce qui existe dans
le temps et dans l’espace : « L’univers est l'ensemble de tous les êtres
créés » (Bonnet) ; « Il y a cette différence entre le monde et l'univers
que le monde est infini » (Diderot).
— 2. Le monde visible et qqfs.
spéc. le monde humain : « Dans ce
petit cachot où il [l’homme] se
trouve logé, j'entends l'univers,... »
(Pascal, 72) ; « Voilà les spectacles
que Dieu donne à l'univers » (Bossuet). — 3. Un monde particulier,
vu d’un point de vue déterminé :
« L'univers de mon imagination »
(Staël). Spéc. Log. Univers du discours : la Classe$^1$ totale, ensemble
des idées auxquelles on se réfère
% 191
dans un jugement ou un raisonnement (on le symbolise par le signe $\bigvee$).
%\/

\ib{Universaux} — Hist. Chez les Scolas
tiques : les idées générales : « Ce que
vous alléguez contre les universaux
des dialecticiens$^2$ ne me touche
point » (Descartes, 5$^\text{es}$ Rép.). Qqfs.,
les idées les plus générales (universaux de Porphyre) : « On compte
ordinairement cinq universaux : le
genre, l'espèce, la différence, le
propre et l'accident » (Descartes,
Princ., I, 59).

\ib{Universel} — Vulg. 1. Qui concerne
l'univers$^1$ : « Moi qui n'existe que
par la force d’une nature universelle » (La Bruyère) ; « L'ordre universel ». — 2. Qui concerne l’univers$^2$ et spéc. le monde humain :
« Les lois universelles de la nature » ;
« Un fluide universel, extrêmement
subtil [l'éther]… » (Fresnel) ; « Le
consentement* universel » ; « Le
suffrage universel ».

— Log. form. 3. Proposition universelle : celle où le sujet$^2$ est pris
dans toute son extension : vg.
« Tous les hommes sont mortels ». —
4. Se dit qqfs. des termes : « pris
universellement » = dans toute son
extension$^3$. Imppt., syn. de général$^2$ :
« Les concepts universels. »

— Méta. 5. Chez Hegel : « universel concret », être réel « dont le
concept est la synthèse, à la fois
universelle$^4$, puisqu'il est susceptible d’un nombre indéfini d'applications, et concrète, en tant qu'il
est une totalité unique et indivisible » (Lalande) : vg. la volonté
collective d’un peuple. Cf. Hamelin :
« Comment l'Universel concret est-il
concret ? Il l'est en ce sens et en ce
sens seul, qu'il nie toute la série des
déterminations dont se compose le
fini. Il n’est pas autre chose que la
réalité éminente de l’École*, l'être
% 192
indéterminé de Malebranche, la
substance$^2$ de Spinoza et enfin,
comme Hegel nous invite à le dire
l’abime suprême du Panthéisme
oriental. »

\ib{Univoque} — Log. Qui s'applique avec
le même sens dans deux ou plu
sieurs cas différents : « Le nom de
substance n'est pas univoque au
regard de Dieu et des créatures »
(Descartes, Princ., I, 51).

\ib{Urgence} — Méta. Chez Le Senne :
« oppression du moi par une situation à la fois constrictive et menaçante » : « L’urgence a pour intimité
la souffrance. »

\ib{Ustensilité} [Trad. all. Zeughaftigkeit].
— Méta. Chez Heidegger : caractère
purement pragmatique du monde
dans lequel vit l’être quotidien.

\ib{Utilitaire} — Vulg. 1. (Gén. péj.) Qui
concerne ou considère uniquement
la vie pratique$^1$ ou l'intérêt personnel : « Un esprit bassement utilitaire ».

— Hist. 2. Qui concerne ou professe l’utilitarisme* : « La doctrine
utilitaire »; « Les philosophes utilitaires ».

\ib{Utilitarisme} [Angl. Utilitarianism]. —
Hist. A. Système de morale qui
prend pour principe « l’utile ou le
principe du plus grand bonheur »
(J. Stuart Mill).

\ib{Utopie} [G. ou et topos, qui n’est en
aucun lieu]. — 1. Description d’une
société idéale (de l’Utopia de Th. Morus, 1516) : « Les utopies sont comme
des enveloppes de brume sous lesquelles s’avancent des idées neuves
et réalisables » (Ruyer). — 2. (Péj.).
Conception irréalisable : « L'égalité
absolue est une utopie ». — 3. Chez
Mannheim : voir Idéologie$^4$.

	\end{itemize}
