
	\begin{itemize}[leftmargin=1cm, label=\ding{32}, itemsep=1pt]
	
\ib{Je} — \si{Psycho.} et \si{Méta.} Le sujet : {\bf 1.} en tant qu'il prend
conscience de lui-même : « En même temps que je pense, j'ai plus ou moins
conscience de {\it moi}, de mon existence personnelle. Et c’est le {\it Je}
qui a conscience de ce {\it Moi} » (W. James) ; — ou bien au ctr. : {\bf 2.}
en tant qu'il est antérieur, comme « spontanéité naïve », à cette prise de
conscience : « Un homme se sent exister comme conscience avant toute
philosophie... Appelons {\it je} cette conscience... À ce {\it je} s'oppose le
{\it moi} comme la pensée de lui-même » (Le Senne). — {\it Cf.} {\it Ego}* et
{\it Moi}*.

\ib{Jeu (Activité de)} — \si{Psycho.} Activité en excédent qui, chez les êtres
vivants supérieurs, se dépense pour le seul plaisir de se dépenser. $->$
{\it Dist.} le {\it jeu} ppt. dit (cf. {\it Précis}, Ph. I, p. 74-76).

\ib{Joie} — \si{Psycho.} (Opp. : {\it tristesse}). État affectif
général, à tonalité agréable et qui, à la différence du plaisir, s'étend à
l’âme tout entière (cf. {\it Précis}, Ph. I, p. 397).

\ib{Jugement} — \si{Psycho.} {\bf 1.} \fsb{S. abstr.} Fonction mentale qui
consiste à juger$^1$ : « Le jugement est le pouvoir de {\it subsumer}* sous
des règles, {\it i. e.} de décider si une chose est ou non soumise à des
règles données » (Kant, {\it R. pure}, Analyt., II); « Le jugement est
% 105
l’opération fondamentale de la pensée réfléchie » ; « Si mon jugement ne me
trompe pas,... ». —  {\bf 2.} \fsb{S. concr.} Produit de cette fonction : « Il
n’y a que les seuls jugements dans lesquels je dois prendre garde de ne pas me
tromper » (Descartes, {\it Méd.}, III); « Un jugement est essentiellement une
assertion* ».

— \si{Car.} {\bf 3.} \fsb{S. subje.} Qualité d'esprit qui consiste à
{\it bien} juger : « On est qqfs. un sot avec de l’esprit$^{10}$; on ne l’est
jamais avec du jugement » (La Rochefoucauld).

— \si{Log.} \si{form.} {\bf 4.} \fsb{S. concr.} (Syn. : {\it proposition}).
Affirmation ou négation d’un rapport entre un sujet$^2$ et un attribut$^1$
(cf. {\it Juger}$^3$).

\ib{Juger} — \si{Psycho.} {\bf 1.} Affirmer ou nier une existence où un
rapport : « La vraie perfection de l’entendement est de bien
juger » (Bossuet). — {\bf 2.} Apprécier, porter un jugement de valeur* :
« Ceux qui jugent d'un ouvrage sans règles... » (Pascal, 5) ; « On juge un
homme sur ses actes » (Rauh).

— \si{Log.} {\bf 3.} Porter un jugement au sens 4 : « On appelle {\it juger}
l’action de notre esprit par laquelle, joignant ensemble diverses idées, il
affirme de l’une qu'elle est l’autre ou nie de l’une qu’elle soit
l’autre » (Port-Royal).

\ib{Juste} — {\bf A)} (Ctr. {\it injuste}). \si{Mor.} {\bf 1.} (En parlant des
personnes). \fsb{S. subje.} Qui pratique la justice$^2$ : « Il n’y aura jamais
qu’un petit nombre de justes sur la terre » (Voltaire). {\it Spéc.}, au sens
religieux : « Le juste agit par foi$^5$ dans les moindres choses » (Pascal,
504). — {\bf 2.} (En parlant des choses ou des actes). \fsb{S. objec.}
Conforme à la justice$^1$ : « Consentir qu’une âme juste$^1$ soit
éternellement malheureuse, cela n’est pas juste$^2$ » (Malebranche) ;
% 106
« Tant qu'il y aura des riches et des pauvres de naissance, il ne saurait y
avoir de contrat juste » (Durkheim).

— {\bf B)} (Ctr. : {\it faux}). \si{Log.} {\bf 3.} \fsb{S. objec.} Vrai, légitime :
« Une opération juste » ; « Voir les hommes sous l'idée de nécessité, cela
n'est pas juste » (Alain). — \si{Car.} {\bf 4.} \fsb{S. subje.} Qui possède du
jugement$^3$ : « Ceux qui choisissent bien sont ceux qui ont l'esprit juste :
ceux qui prennent le mauvais parti sont ceux qui ont l'esprit faux » (Nicole).

\ib{Justesse} — Qualité de ce qui est juste au sens 3 ou 4 : « La justesse
d’une expression » ; « La même justesse d'esprit qui nous fait écrire de
bonnes choses, nous fait appréhender qu'elles ne le soient pas assez pour
mériter d’être lues » (La Bruyère).

\ib{Justice} — \si{Mor.} {\bf 1.} Principe moral qui exige le respect du
droit$^2$ : « La formule de la justice est claire respecter les droits
d’autrui » (Cousin). — {\bf 2.} Vertu morale qui consiste à respecter et à
promouvoir le droit : « Ne croyez pas que la justice habite jamais dans les
âmes où l'ambition domine » (Bossuet). — {\bf 3.} \fsb{S. objec.} Qualité de
ce qui est juste$^2$ : « La justice d’une revendication ».

— \si{Pol.} {\bf 4.} Pouvoir judiciaire, {\it i. e.} ensemble des institutions
et des personnes qui ont pour fonction d'appliquer la loi aux cas individuels.

\begin{center}
\huge{K}
\end{center}

\ib{Kinésiques ou Kinesthésiques (Sensations)} — [G. {\it kinein}, mouvoir, et
{\it aisthésis}, sensibilité] — \si{Psycho.} Sensations (cutanées ou internes)
qui nous font percevoir les mouvements de nos membres.
% 106

\ib{Korsakoff (Maladie de)} — \si{Ps. path.} Trouble mental caractérisé par
l'amnésie* de fixation avec de la fabulation* et de la confusion* (souvent lié
à l'alcoolisme).

	\end{itemize}
