
	\begin{itemize}[leftmargin=1cm, label=\ding{32}, itemsep=1pt]
	
\ib{Je} — Psycho. et Méta. Le sujet :
1. en tant qu'il prend conscience
de lui-même : « En même temps que
je pense, j'ai plus ou moins conscience de moi, de mon existence
personnelle. Et c’est le Je qui a
conscience de ce Moi » (W. James) ;
— ou bien au ctr. : 2. en tant qu'il
est antérieur, comme « spontanéité
naïve », à cette prise de conscience :
« Un homme se sent exister comme
conscience avant toute philosophie...
Appelons je cette conscience... À ce
je s'oppose le moi comme la pensée
de lui-même » (Le Senne). — Cf.
Ego* et Moi*.

\ib{Jeu (Activité de)} — Psycho. Activité
en excédent qui, chez les êtres
vivants supérieurs, se dépense pour
le seul plaisir de se dépenser. $->$
Dist. le jeu ppt. dit (cf. Précis, Ph. I,
p. 74-76).

\ib{Joie} — Psycho. (Opp. : tristesse).
État affectif général, à tonalité
agréable et qui, à la différence du
plaisir, s'étend à l’âme tout entière
(cf. Précis, Ph. I, p. 397).

\ib{Jugement} — Psycho. 1. O. Fonction
mentale qui consiste à juger$^1$ : « Le
jugement est le pouvoir de subsumer* sous des règles, i. e. de décider
si une chose est ou non soumise
à des règles données » (Kant, R. pure,
Analyt, II); « Le jugement est
% 105
l’opération fondamentale de la
pensée réfléchie » ; « Si mon jugement ne me trompe pas,... ». —
2. @. Produit de cette fonction :
« Il n’y a que les seuls jugements
dans lesquels je dois prendre garde
de ne pas me tromper » (Descartes,
Méd., III); « Un jugement est essentiellement une assertion* ».

— Car. 3. © Qualité d'esprit qui
consiste à bien juger : « On est qqfs.
un sot avec de l’esprit$^{10}$; on ne l’est
jamais avec du jugement » (La
Rochefoucauld).

— Log. form. 4. @ (Syn. : proposition). Affirmation ou négation d’un
rapport entre un sujet$^2$ et un attribut$^1$
(cf. Juger$^3$).

\ib{Juger} — Psycho. 1. Affirmer ou nier
une existence où un rapport : « La
vraie perfection de l’entendement
est de bien juger » (Bossuet). — 2.
Apprécier, porter un jugement de
valeur* : « Ceux qui jugent d'un
ouvrage sans règles... » (Pascal, 5) ;
« On juge un homme sur ses actes »
(Rauh).

— Log. 3. Porter un jugement
au sens 4 : « On appelle juger l’action
de notre esprit par laquelle, joignant
ensemble diverses idées, il affirme
de l’une qu'elle est l’autre ou nie
de l’une qu’elle soit l’autre » (Port-Royal).

\ib{Juste} — A) (Ctr.
injuste). Mor.
1. (En parlant des personnes). © Qui
pratique la justice$^2$ : « Il n’y aura
jamais qu’un petit nombre de justes
sur la terre » (Voltaire). Spéc., au
sens religieux : « Le juste agit par
foi$^5$ dans les moindres choses »
(Pascal, 504). — 2. (En parlant des
choses ou des actes). IN Conforme à la
justice$^1$ : « Consentir qu’une âme
juste$^1$ soit éternellement malheureuse, cela n’est pas juste$^2$ » (Malebranche)
% 106
: « Tant qu'il y aura des
riches et des pauvres de naissance,
il ne saurait y avoir de contrat
juste » (Durkheim).

— B) (Ctr. : faux). Log. 3. Vrai,
légitime : « Une opération juste » ;
« Voir les hommes sous l'idée de
nécessité, cela n'est pas juste »
(Alain). — Car. 4. © Qui possède du
jugement$^3$ : « Ceux qui choisissent
bien sont ceux qui ont l'esprit juste :
ceux qui prennent le mauvais parti
sont ceux qui ont l'esprit faux »
(Nicole).

\ib{Justesse} — Qualité de ce qui est juste
au sens 3 ou 4 : « La justesse d’une
expression » ; « La même justesse
d'esprit qui nous fait écrire de
bonnes choses, nous fait appréhender
qu'elles ne le soient pas assez pour
mériter d’être lues » (La Bruyère).

\ib{}Justice. — Mor. 1. Principe moral qui
exige le respect du droit$^2$ : « La
formule de la justice est claire
respecter les droits d’autrui » (Cousin). — 2. Vertu morale qui consiste
à respecter et à promouvoir le droit :
« Ne croyez pas que la justice habite
jamais dans les âmes où l'ambition
domine » (Bossuet). — 3. I Qualité
de ce qui est juste$^2$ : « La justice
d’une revendication ».

— Pol. 4. Pouvoir judiciaire,
i. e. ensemble des institutions et des
personnes qui ont pour fonction
d'appliquer la loi aux cas individuels.

\begin{center}
K
\end{center}

\ib{Kinésiques ou Kinesthésiques (Sensations)} [G. kinein, mouvoir, et
aisthésis, sensibilité). — Psycho.
Sensations (cutanées ou internes)
qui nous font percevoir les mouvements de nos membres.
% 106

\ib{Korsakoff (Maladie de)} — Ps. path.
Trouble mental caractérisé par
J'amnésie* de fixation avec de la
fabulation* et de la confusion*
(souvent lié à l'alcoolisme).

	\end{itemize}
