
	\begin{itemize}[leftmargin=1cm, label=\ding{32}, itemsep=1pt]

\ib{Valeur} — \si{Vulg.} {\bf 1.} Vaillance, courage physique (cf. «
valeureux ») : « La valeur n'attend pas le nombre des années » (Corneille).
Étendu ensuite aux autres qualités morales : « I] ya une valeur domestique et
privée qui n’est pas moindre que la valeur militaire » (Rollin).

— \si{Méta.}  {\bf 2.} \fsb{S. abstr.} Caractère qui fait que certaines
choses sont dignes d'estime, {\it ou} : \fsb{S. concr.} ces choses
elles-mêmes : « L’affirmation fondamentale de toute pensée est celle de sa
valeur » (Lagneau) ; « Ce qui est {\it digne d’être recherché} est ce que
tout le monde appelle valeur » (Le Senne) ; « L’Être est supérieur aux
valeurs et les fonde » (Alquié). {\it Jugements de valeur} (opp. {\it jug. de
réalité}*) : ceux qui énoncent une estimation, une appréciation, ce que les
choses valent par rapport à une conscience. {\it Moi de valeur} (cf.
{\it Moi}$^4$) : « Ce moi misérable se dilate jusqu’à se confondre en droit
avec l’immensité du moi illimité... Dès lors, au {\it moi empirique}, au moi
de fait, s’oppose le moi {\it a priori}, le moi de la valeur » (Le Senne). —
Par application de ce sens général : \si{Mor.} « Nous devons nous servir de
l'expérience et de la raison pour distinguer le {\it bien} d'avec le {\it
mal} et connaître leur juste valeur » (Descartes, {\it Passions}, 138) ; «
Les valeurs morales » ; — \si{Log.} et \si{Crit.} « Une fonction$^4$
propositionnelle est, dans chaque cas particuler, susceptible de deux
valeurs, {\it vrai} et {\it faux}, et de celles-là seulement » (Couturat) ; «
Les logiques à trois valeurs » (cf. {\it Précis}, Ph. II, p- 25) ; « La
valeur de la connaissance »; — \si{Esth.} « Une œuvre d’art a une valeur
normale, elle est {\it belle} quand elle s’adapte à ses fonctions
% 193
psychiques et sociales... Elle a une valeur négative ou anormale, elle est
{\it laide} quand elle manque à remplir quelqu’une de ces fonctions » (Lalo).
$->$ Bien des contresens sont commis auj. à propos de cette notion, comme à
propos de celle de {\it transcendance} à laquelle elle est liée (cf.
{\it Précis}, Ph. II, p. 274; M.etS., p- 134). Les trois principaux sont : 1°
de la définir, uniquement par l'affectivité : {\it p. e.} « La valeur d’une
chose est sa désirabilité » (Ehrenfels) ; Scheler définit la valeur comme un
{\it a priori} (voir {\it Apriorisme}*) émotionnel et compte l’{\it agréable}
et le {\it désagréable} parmi les valeurs ; or : « L’agréable ne coïncide
[même] pas nécessairement avec l'utilité organique » (Gusdorf) ; — 2° de la
définir en fonction de l'{\it individu} : « La valeur est ce qui est
recherché en fait par l'individu, ce que nos activités visent spontanément
» (Gusdorf) ; or : « Une valeur devient d'autant moins valeur qu’elle est
plus parfaitement individuelle » (Lalande; cf. {\it Synnomique}*) ; — 3° de
parler de « valeurs biologiques » : « La notion de valeur doit être comprise
dans son sens le plus général... Il faut que soient satisfaits les besoins
organiques de nourriture, de boisson... » (Gusdorf), ce qui rabaisse la
valeur sur le plan animal : « L'animal aussi constitue son univers en
fonction de ses besoins. Mais, chez lui, les valeurs, soudées à l'organisme,
demeurent brutes » (id.) ; or : « La notion de valeur est essentiellement
spirituelle » (Le Senne).

— \si{Éc. pol.}  {\bf 3.} (Valeur d'échange). Celle qui se traduit par le
prix. —  {\bf 4.} (Valeur d'usage). « La valeur des choses est fondée sur
leur utilité ou, ce qui revient au même, sur l'usage
% CUVILLIER, — Vocabulaire philosophique.
% 193
que nous pouvons en faire » (Condillac). —  {\bf 5.} (Valeur-travail). « La
valeur est du travail humain cristallisé » (K. Marx).

— \si{Math.}  {\bf 6.} Expression numérique d’une inconnue ou d’une
variable*. {\it Ext.} \si{Log.} 1° expression individuelle$^3$ d’une variable
logique : « Une fonction$^1$ propositionnelle devient une proposition toutes
les fois qu'on y substitue aux variables des valeurs déterminées
» (Couturat) ; 2° sens donné à un mot ou à une expression dans le lang,
ordinaire : « Connaître la valeur des termes. »

\ib{Vanité} — {\bf 1.} \si{Vulg.} Caractère de ce qui est vain, i.e. sans
importance, sans valeur : « La vanité du monde » (Pascal, 161). — {\bf 2.}
\si{Mor.} Sentiment exagéré de la valeur personnelle, qui diffère de
l’orgueil* en ce qu'il s'attache surtout aux petites choses et en ce qu'il
recherche l'approbation d'autrui : « Curiosité n’est que vanité : on ne veut
savoir que pour en parler » (Pascal, 152).

\ib{Variable} — \si{Math.} et \si{Log.} Terme qui, dans une fonction$^1$
mathématique ou dans une fonction$^1$ propositionnelle ({\it variable
logique}) peut prendre différentes valeurs$^6$. Voir {\it Sujet}$^2$.

\ib{Variations concomitantes (Méthode des)} — \si{Épist.} Une des quatre
méthodes expérimentales de J. S. Mill, qui consiste à faire varier
quantitativement un phénomène pour voir si celui qu’on suppose én être
l’effet (ou la cause) varie corréJativement : « La méthode des {\it
variations concomitantes}, qu’il faudrait plutôt nommer la méthode des
variations liées et, en un mot, la méthode des fonctions$^1$ naturelles...
» (Renouvier).

\ib{Vecteur} — \si{Math.} Segment orienté, ou {\it mieux} : synthèse des
éléments communs à des segments orientés équipollents$^2$. Par {\it ext.} du
sens math., {\it vection} se prend qqfs. au sens d’orientation, tendance (cf.
{\it Objectivation}$^2$).

\ib{Végétative (Vie)} — \si{Biol.} (Opp. : {\it relation}$^5$). Ensemble des
fonctions$^2$ vitales communes aux végétaux et aux animaux.

\ib{Velléité} — \si{Psycho.} Volition ébauchée,
qui n'aboutit pas à l'acte : « Une
% 194
volonté imparfaite ou, comme parle l'École*, une velléité » (Bossuet).

\ib{Véracité} — Qualité du témoin ou du témoignage qui {\it dit} vrai. {\it
Véracité dipine}, attribut qui fait que Dieu ne peut ni se tromper ni nous
tromper : « Chez Descartes, la {\it vérité} des idées claires et distinctes
repose sur la {\it véracité} divine.

\ib{Verbalisme} — ({\it Péj.}) Tendance à se satisfaire de mots creux au lieu
d'idées. Cf. {\it Psittacisme}*.

\ib{Verbe} — {\bf 1.} Parole : « La puissance du verbe. » —  {\bf 2.}
\si{Théol.} Le Logos*. {\it Spéc.}, dans la théol. chrétienne, le Fils de
Dieu, deuxième personne de la Trinité : « Le Verbe divin, en tant que Raison
universelle, renferme dans sa substance les idées primordiales de tous les
êtres » (Malebranche, {\it Entr.}, III, 2).

\ib{Véridique} — Qui a de la véracité* : « Un récit véridique. »

\ib{Vérifier} — \si{Épist.} {\bf 1.} (En parlant du chercheur). Contrôler la
vérité$^1$ d’une assertion ou d’une hypothèse$^3$ en la confrontant avec les
faits : « L’astronome a confiance dans les principes de sa science, mais cela
ne l'empêche pas de les vérifier et de les contrôler par des observations
directes » (CL Bernard). —  {\bf 2.} (En parlant des faits). Confirmer «
L'expérience a vérifié l’hypothèse ».

\ib{Vérité} — \si{Crit.} et \si{Épist.} {\bf 1.} \fsb{S. abstr.} Caractère de
ce qui est vrai*. $->$ On peut {\it dist.} : 1° vérité {\it formelle}$^3$ : «
C’est dans l’accord avec les lois de l’entendement que consiste le formel de
la vérité » (Kant, {\it R. pure}, Dial., introd., 1), et vérité {\it
matérielle}$^1$ : « Humainement parlant, définissons la vérité : ce qui est
% 194
énoncé tel qu'il est » (Voltaire) ; « La vérité, de qq. manière qu’on la
définisse, implique l'accord du sujet$^4$ avec l’objet$^5$ » (Hamelin) ; 2°
vérité {\it mathématique} et vérité {\it expérimentale} (cf. {\it Textes
choisis}, II, p. 116) ; 3° vérité {\it absolue} et vérité {\it relative} (cf.
{\it Relativité}*). —  {\bf 2.} \fsb{S. concr.} Ce qui est vrai : « Il y a un
art pour faire voir la liaison des vérités avec leur principe » (Pascal) ; «
Toutes les vérités immuables ne sont que les rapports qui se trouvent entre
les idées, dont l'existence est nécessaire et éternelle » (Malebranche,
{\it Entr.}, X, début). {\it Vérités éternelles} : les vérités de raison,
regardées comme immuables et universelles : « Ces vérités éternelles que tout
entendement aperçoit toujours les mêmes, sont qqe. de Dieu ou plutôt sont
Dieu même » (Bossuet).

— Méta  {\bf 3.} Qqfs., réalité : « La vérité de la chose [la distinction de
l’âme et du corps] » (Descartes, \si{{\it Méd.}}, préf.) ; « On peut douter
de la vérité des choses sensibles » (id. {\it Princ.}, I, 4). $->$ Tous ces
divers sens sont réunis dans cette phrase de Bossuet : « De vérité en vérité,
vous pouvez aller jusqu'à Dieu, qui est la vérité des vérités, la source de
la vérité, la vérité même, où subsistent les vérités que vous appelez
éternelles, les vérités immuables et invariables, qui ne peuvent pas ne pas
être vérités. »

\ib{Vertu} — {\bf 1.} {\it Autref.} puissance, pouvoir : « La vertu de tout
un aimant n’est pas d’autre nature que celle de chacune de ses parties
» (Descartes, {\it Princ.}, IV, 157). {\it Spéc.}, en parlant des {\it forces
occultes}* des Scolastiques : « Façons de parler. Vertu {\it apéritive} d'une
clef, {\it attractive} d'un croc » (Pascal, 55), et ironiquement « Voilà des
paroles bien puissantes !
% 195
Sans doute ont-elles qq. vertu occulte pour chasser l'usure » (id.
{\it Prov.}, 8).

— \si{Mor.}  {\bf 2.} Force morale, disposition permanente à faire le bien ou
{\it spéc.} à pratiquer certains devoirs : « Ce que peut la vertu d’un homme
ne se doit pas mesurer par ses efforts, mais par son ordinaire » (id., 352) ;
« Ils [certains philosophes] confondent les devoirs avec les vertus ou
donnent des noms de vertus aux simples devoirs, de sorte que, quoiqu'il n’y
ait ppt. qu’une vertu, Famour de l'Ordre$^{11}$, ils en produisent une
infinité » (Malebranche, {\it Traité de Mor.}, I, 2, 4). {\it Vertus
cardinales}* : v. ce mot. {\it Chez Aristote et les Scolastiques} : « vertus
intellectuelles », celles qui ont pour objet la connaissance et la
contemplation; « vertus morales », celles qui ont pour objet la conduite
ordinaire : « Les vertus morales sont plus nécessaires à la vie humaine, mais
les vertus intellectuelles sont plus nobles » (St. Thomas, {\it S. th.},
I$^\text{a}$ II$^\text{æ}$, 66, 3). \si{Théol.} {\it Vertus théologales} :
celles qui ont directement Dieu et les fins dernières pour objet (foi,
espérance, charité).

— \si{Pol.} {\bf 3.} {\it Vertu politique} « amour des lois et de la
patrie,... préférence continuelle de l'intérêt public au sien propre
» (Montesquieu, {\it Lois}, IV, 5).

\ib{Vice} — {\it Lato.} {\bf 1.} Défaut : « Le vice d’un
raisonnement »; « Un vice d’organisation, \si{Jur.} « Un vice de forme ».

— {\it Str.} \si{Mor.}  {\bf 2.} Disposition permanente à l’immoralité ou à
certaines formes d’immoralité : « Le vice, qui nous est naturel, résiste à la
grâce surnaturelle » (Pascal, 498) ; « Tous les vices politiques ne sont pas
des vices moraux » (Montesquieu, {\it Lois}, XIX, 11).

\ib{Vide} — \si{Méta.} {\bf 1.} Absence de toute matière : « Il ne peut y
avoir aucun vide, au sens que les philosophes prennent ce mot » (Descartes,
{\it Princ.}, II, 16). — \si{Phys.}  {\bf 2.} Absence de matière pondérable :
« Le vide barométrique. »

\ib{Vigilance} — \si{Méta.} {\it Chez Heidegger} : attitude de la conscience
éveillée ({\it wach}) au monde : « Nous pouvons définir {\it moi vigilant} le
moi qui réalise continuellement la conscience à l’intérieur de son flux de
vécu sous la forme spécifique du {\it cogito.} »

\ib{Virtuel} — \si{Méta.} (Ctr. : {\it actuel}$^2$). Qui n'existe qu’à l’état
de possible, {\it ou} : qui demeure à l’état implicite « Ce que vous dites de
ces connaissances virtuelles me surprend. — Je suis étonné comment il ne vous
est pas venu dans la pensée que nous avons une infinité de connaissances dont
nous ne nous apercevons pas » (Leibniz, {\it N. E.}, I, 1, 5).

\ib{Visée} — \si{Mor.} Orientation générale de la pensée et de la conduite,
dont l'{\it intention}$^1$ n’est que l'adaptation à une situation
déterminée : « La visée est existentiellement la fin$^1$ de l'intention » (Le
Senne).

\ib{Vision en Dieu} — \fsb{S. norma.} \si{Hist.} {\it Chez Malebranche} :
théorie d’après laquelle l’homme connaît, non seulement les « vérités*
éternelles », mais toutes choses, y compris les choses sensibles, par une vue
directe des {\it idées}* intelligibles qui constituent l'essence même de
Dieu, en tant du moins que ces idées sont « participables » par les créatures.

\ib{Vital} — {\bf 1.} Qui se rapporte à la vie. {\it Élan vital} : v. {\it
Élan*. Sens vital} : autref., syn. de {\it Cénesthésie*. Principe vital} :
force analogue à l'âme$^2$, mais différente d'elle, différente
%196
aussi des phénomènes physico-chimiques, et par laquelle on expliquait les
phénomènes de la vie. —  {\bf 2.} {\it Laud.} Indispensable à la vie, soit de
l’organisme, soit du corps social : « Une nécessité vitale. »

\ib{Vitalisme} — A. \si{Biol.} {\bf 1.} ({\it Opp.} : {\it animisme}$^2$ et
{\it mécanisme}$^3$). Théorie biologique (de Barthez, J. Grasset) selon
laquelle les phénomènes biologiques s'expliquent par un {\it principe
vital}$^1$ : « Dans la médecine, ia croyance aux causes occultes, qu’on
l'appelle vitalisme ou autrement, favorise l'ignorance et entraîne une sorte
de charlatanisme involontaire » (Cl. Bernard). — \si{Méta.} {\bf 2.} Doctrine
philosophique qui fait de la vie une entité et qui tend à confondre vie et
spiritualité : « L'union de la psychologie de l’instinct avec un vitalisme
généralisé est l’une des caractéristiques de la philosophie romantique
» (René Berthelot) ; « Je ne sais s’il est aisé de tirer au clair la
distinction du vitalisme et du matérialisme » (Brunschvicg).

\ib{Volition} — \si{Psycho.} \fsb{S. concr.} Acte de volonté$^1$ : « Vouloir,
c’est agir : la volition est un passage à l'acte » (Ribot).

\ib{Volontarisme} — \fsb{S. norma.} \si{Méta.} {\bf 1.} Système philosophique
({\it p. e.} de Schopenhauer) qui fait de la Volonté$^4$ l'essence même de
l'univers. — \si{Psycho.}  {\bf 2.} (Opp. : {\it intellectualisme}$^1$)
Théorie psychologique selon laquelle l’activité$^2$ (avec l’affectivité qui
s'y rattache étroitement) est plus fondamentale dans la vie psychique que
l'intelligence. {\it Spéc.}, à propos du jugement$^1$ : théorie selon
laquelle l'assentiment relève surtout de l’activité$^2$ et même de la
volonté$^1$ ppt. dite (cf. {\it Précis}, Ph. I, p. 299, et {\it Textes
choisis}, I, p. 158) : « C’est
% 196
dans et par un acte libre que l'être concret se pose et entre en possession
de lui-même : en cela consiste la part de vérité que nous reconnaissons au
volontarisme » (Hamelin).

\ib{Volonté} — \si{Psycho.} {\it Str.} {\bf 1.} \fsb{S. abstr.} Forme
réfléchie et pleinement consciente de l’activité$^2$, qui implique
représentation du but et délibération : « La volonté implique le but, puisque
vouloir, c’est vouloir qqc. ; elle implique aussi les moyens » (Hamelin).
{\it Qqfs.} \fsb{S. concr.} une volition* particulière : « Pour trop faire
ses volontés, l’homme s’empêche lui-même d’être heureux » (Bossuet). {\it
Bonne volonté} : voir {\it Bonne}*. —  {\bf 2.} \fsb{S. abstr.} {\it Lato.}
L'activité$^2$ en général : « Nos tendances ne sont que difiérentes formes
d’une tendance unique que l’on a justement nommée la volonté de vivre : nous
sommes volonté avant d'être sensation » (Lachelier). {\it Spéc.}, {\it chez
les Scolastiques}, tendance d’un être vers sa fin : « La volonté de l’homme
se porte naturellement d'elle-même vers le bien et vers les choses qui
conviennent à sa nature » (St. Thomas, {\it S. th.}, I$^\text{a}$
II$^\text{æ}$, 10, 1). {\it Cf.} Malebranche : « Ce mouvement naturel et
continuel, de l'âme vers le bien en général, vers Dieu, c’est ce que
j'appelle ici {\it volonté}, parce que c’est ce mouvement qui rend l'âme
capable d'aimer différents biens. »

—— \si{Car.} 3 \fsb{S. subje.} Qualité morale consistant dans l'énergie de la
volonté$^1$ : « Nous avons plus de force que de volonté » (La Rochefoucauld).

— \si{Hist.}  {\bf 4.} {\it a) Chez Schopenhauer} : le vouloir-vivre
universel, « poussée aveugle et irrésistible » des êtres, qui constitue
l’unique « chose en soi » : « La Volonté est la substance intime, le noyau de
toute chose particulière comme de l’ensemble : elle se manifeste dans la
force aveugle de la
% 197
nature et elle se retrouve dans la conduite raisonnée de l’homme » ; —
{\it b) Chez Nietzsche} : « volonté de puissance » (trad. : {\it Wille zur
Macht}), surabondance de force qui s'exprime par le besoin de dominer les
autres.

— \si{Pol.}  {\bf 5.} {\it Chez J.-J. Rousseau} : « volonté générale », celle
du peuple lorsque : 1° tous ont été consultés; 2° elle édicte des règles
générales, {\it i. e.} sans acception de personnes; 3° elle porte sur une
question d’intérêt commun.

\ib{Volume social} — \si{Soc.} Caractère morphologique* des groupes sociaux
qui consiste dans le « nombre des unités sociales » (Durkheim) qui en font
partie.

\ib{Vrai} — \si{Crit.} {\bf 1.} Qui mérite l’assentiment* : « Ne recevoir
jamais aucune chose pour vraie que je ne la connusse évidemment être telle
» (Descartes, {\it Méth.}, II) ; « Les choses que nous concevons fort
clairement et fort distinctement sont toutes vraies » ({\it ib.}, IV) ; « Les
rapports que nous pouvons affirmer sont qualifiés de {\it vrais} ou de {\it
faux} selon qu'ils s’accordent ou non avec des lois que nous constatons ou
croyons constater, {\it i. e.} selon que ces lois les impliquent ou les
excluent dans les sujets$^1$ où elles paraissent » (Renouvier) ; « Une
géométrie ne peut pas être plus vraie qu’une autre; elle peut seulement être
plus commode » (Poincaré).

— \si{Méta.}  {\bf 2.} Réel : « Le vrai est ce qui est ; le faux, ce qui
n’est point » (Bossuet) ; « Aucun fait ne saurait se trouver vrai ou
existant, sans qu'il y ait une raison$^5$ suffisante pourquoi … » (Leibniz,
{\it Mon.}, 32).

— {\it Ext.} \si{Vulg.} 3 Authentique, qui est tel qu'il doit être : « La
vraie éloquence se moque de l'éloquence » (Pascal, 4) ; « Le vrai bonheur ». —
% 197
{\bf 4.} Sincère : « Pour les religions, il faut être sincère : vrais
païens, vrais juifs, vrais chrétiens » (id, 590) ; « Le premier mérite d'un
auteur est d’être vrai » (D’Alembert).

\begin{center}
W
\end{center}

\ib{Weltanschauung} — \si{Méta.} Mot allemand [= vision du monde] désignant
une conception de l'univers et de la vie : « Une {\it Weltanschauung} n’est
pas encore une philosophie » (Blanché).

\ib{Weber (Loi de)} — \si{Ps. phys.} Loi selon laquelle le seuil*
différentiel de la sensation dépend de la grandeur de l’excitation première.

\ib{Wergeld} — \si{Soc.} Coutume germanique selon laquelle l’auteur d’un
dommage payait une indemnité à la victime ou à ses ayants droit pour se
soustraire à la vengeance privée.

\ib{Würzbourg (École de)} — \si{Psycho.} École de psychologues allemands qui,
vers 1900-1908, chercha, à l’aide de l'introspection* expérimentale, à mettre
en lumière ce qui se passe dans l'esprit du sujet$^5$ au cours d’une
opération intellectuelle. Voir {\it Précis}, Ph. I, p. 30 {\bf 8.}

\begin{center}
Z
\end{center}

\ib{Zermelo (Axiome de)} — \si{Épist.} Postulat logique (dit encore {\it
axiome de choix}) qui intervient dans la théorie des ensembles$^2$ et d’après
lequel, dans un ensemble infini de classes$^1$, on a toujours le droit
d'extraire un élément de chacune de celles-ci.

\ib{Zététique} — [G. {\it zêtein}, chercher] — \si{Hist.} 1. Qualificatif
autref. appliqué aux Sceptiques, {\it spéc.} aux disciples de Pyrrhon. —
\si{Épist.} {\bf 2.} Qui concerne ou constitue une recherche. {\it Analyse
zététique} : celle qui consiste à supposer le problème résolu pour trouver la
solution.

	\end{itemize}
