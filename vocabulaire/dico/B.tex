
	\begin{itemize}[leftmargin=1cm, label=\ding{32}, itemsep=1pt]

\ib{Babillage} — Voir {\it Lallation}*.

\ib{Baroque} — [de l'espagnol {\it }barroco, irrégulier] — \si{Vulg.} {\bf 1.} {\it Péj.} Bizarre,
inattendu. — \si{Esth.} {\bf 2.} Se dit d’un
style qui recherche l’artificiel ({\it p. e.}
colonne torse), voire le tourmenté,
et la surcharge ornementale.

\ib{Base} — \si{Vulg.} {\bf 1.} Ce sur quoi repose
une construction matérielle ou intellectuelle. $->$ Syn. : {\it fondement}*. On
a cherché à {\it dist.} les deux termes
comme correspondant respectivement à un ordre de {\it fait} et à un ordre
de {\it droit}$^2$. Cette distinction paraît
arbitraire : « la base d’un raisonnement » est à la fois son point de
départ et ce qui le justifie; « la base
de l’ordre social » est à la fois ce qui
%28 — BIO
le rend possible et légitime. {\it cf.}  Massilon : « La foi, cette base de l'esprit
chrétien », et Voltaire : « Son honneur [d'un ministre] est la base de
son crédit. »

— Soc. {\bf 2.} {\it Dans le lang. marxiste} :
infrastructure économique de la
société constituée essentiellement
par les {\it forces productives} (matérielles
et humaines) et secondairement par
les {\it rapports de production} (classes$^3$,
institutions, etc.).

— Psycho. {\bf 3.} {\it Personnalité de base} :
v. {\it Personnalité}$^2$.

\ib{Béatitude} — \si{Théol.} et \si{Méta.} Bonheur
stable et parfait : « La béatitude des
élus »; « La Béatitude n’est pas la
récompense de la vertu : c’est la vertu elle-même » (Spinoza, {\it Eth.}, V, 42).

\ib{Beau} — \si{Esth.} Dist. : {\bf 1.} Le sens propre,
purement esthétique : « Un beau
tableau », « Un beau visage est le
plus beau de tous les spectacles » (La
Bruyère). $->$ Encore le {\it beau} doit-il
être {\it dist.}, en ce sens, du {\it joli}, du
{\it gracieux}, etc. (voir {\it Précis}, Ph. I,
p. 560-563); — {\bf 2.} Le sens esthéticomoral : « Une belle action »; « Là, si
tu veux mourir, trouve une belle
mort » (Corneille).
 
— {\bf 3.} Qqfs. ironiquement : « C’est
un beau sophisme ». {\it Chez Hegel} :
« la belle âme », la conscience qui se
réfugie « dans son intimité la plus
profonde » et qui craint « de souiller
la splendeur de son intériorité par
l’action » ({\it Phénoménologie}, VI, C, c).

\ib{Behaviourisme} — [Angl. {\it behaviour}, conduite, comportement] — \si{Psycho.}
1. Comme {\it méthode} ([Bechterev,
Pavlov, Watson) : syn. de « psychologie* de réaction ». — {\bf 2.} À. Comme
{\it doctrine} (Watson, Perry) : forme de
l'épiphénoménisme* qui identifie
l'esprit avec l’ensemble des réactions
de l'organisme.

\ib{Besoin} — \si{Phol.} {\bf 1.} Privation d’une
chose nécessaire à la vie organique :
« Plante qui a besoin de lumière ».

— \si{Psycho} {\bf 2.} État de conscience
qui accompagne la privation de ce
qui est nécessaire (ou regardé comme
nécessaire) à la vie organique ou à la
vie spirituelle : « Besoin d'affection».
$->$ Besoin désigne surtout l'élément
affectif de cet état; l’élément actif
s'appelle {\it tendance} ou {\it appétit}.

\ib{Bien} — \si{Mor.} {\it Dist.} : {\bf 1.} le {\it bien naturel}
(syn. : {\it bonheur}), ce qui satisfait nos
besoins et nos facultés; — {\bf 2.} le {\it bien
moral} ou la valeur : « Si le bien
n'était que ce qui plaît, on ne verrait pas comment les hommes eussent
été amenés à concevoir le devoir »
(Le Senne).

\ib{Bien (Souverain)} — \si{Mor.} {\bf 1.} Celui
qui est au-dessus de tous les autres
biens et dont ceux-ci ne seraient que
les différents aspects ou les conséquences. — {\bf 2.} {\it Chez Kant} : le bien
complet, union du bonheur et de la
vertu.

\ib{Biocœnose} — [G. b{\it ios}, vie, et {\it koinônésis},
communauté] — \si{Biol.} Association
d'animaux ou de végétaux, définie
statistiquement par les espèces qui y
participent.

\ib{Biogénétique (Loi)} — \si{Biol.} (Syn. :
{\it loi de Serres}). Voir {\it Phylogénie}*.

\ib{Biologie} — [G. {\it bios}, vie, et {\it logos}, étude :
mot introduit, en français, par
Lamarck] — {\bf 1.} {\it Lato}. Ensemble des
sciences concernant la vie. — {\bf 2.} {\it Str.}
Science des phénomènes généraux
de la vie, communs aux animaux et
aux végétaux.

\ib{Biologique} — {\bf 1.} Qui concerne la
biologie. — {\bf 2.} Qui concerne la vie:
qui explique un phénomène par les
nécessités vitales : « Théorie biologique
%29 — CAP
de la conscience » (voir {\it Précis},
Ph. I, p. 54); « Toute morale est
d'essence biologique » (Bergson).

\ib{Biosphère} — \si{Biol.} (Terme créé par
E. Suess). « Les vivants se tiennent
et forment un système, une enveloppe de la planète, la {\it biosphère} »
(Le Roy).

\ib{Bonheur} — \si{Mor.} Le bonheur peut
être conçu : {\bf 4.} tantôt négativement
ou statiquement, comme un repos,
une absence de douleur, de soucis
({\it vg}. Épicure); —— {\bf 2.} tantôt positivement et dynamiquement, comme le
développement de l’ensemble des
virtualités de l'être : « L'homme est
un être vivant : son bonheur est
donc de vivre, et la vie est un mouvement, par conséquent un effort,
un regret, une espérance et une
crainte » (Bersot).

\ib{Bonne forme} — \si{Psycho.} Dans la
{\it Gestaltpsychologie} : celle qui présente
le plus d'unité, de simplicité, etc.

\ib{Bonne volonté} — \si{Hist.} {\it Chez Kant} :
celle qui n’a d'autre règle que le
Devoir$^6$ et qui lui obéit uniquement
par respect pour lui; elle est donc
« bonne » par sa forme$^2$, et non par
son but.

\ib{Bon sens} — \si{Hist.} {\bf 4.} {\it Chez Descartes} :
faculté de discerner le vrai du faux :
« La puissance de bien juger et distinguer le vrai d'avec le faux, qui
est proprement ce qu'on nomme le
bon sens ou la raison, est naturellement égale en tous les hommes »
({\it Méth.}, I).

— \si{Vulg.} {\bf 2.} Tendance naturelle à
juger sainement dans les choses de
la vie pratique : « Il avait du bon
sens; le reste vient ensuite » (La
Fontaine). — {\bf 3.} ({\it Opp.} folie, colère,
etc.). État sain de l'esprit : « Est-il
dans son bon sens ? »

\ib{But} — \si{Mor.} Certains auteurs ({\it p. e.}
Scheler) distinguent le {\it but} ({\it Ziel}) qui
implique une représentation en
image, de la {\it fin} ({\it Zweck}) qui est ce
vers quoi tend l'aspiration (voir nos
{\it Textes choisis}, II, p. 170).

	\end{itemize}
