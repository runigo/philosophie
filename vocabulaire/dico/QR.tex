
	\begin{itemize}[leftmargin=1cm, label=\ding{32}, itemsep=1pt]

\ib{Qualité} — (Opp. : {\it quantité}*]. \si{Crit.} {\bf 1.} Une des
catégories* fondamentales de la pensée : propriété$^1$, manière d’être : «
J’ai pensé que je ne ferais pas peu si je montrais comment il faut distinguer
les propriétés ou qualités de l'esprit des propriétés ou qualités du corps
» (Descartes, 2$^\text{e}$ {\it Rép.}) ; « Une qualité est ce qui fait qu'on
appelle une chose d’un tel nom : on ne peut le nier à
% 153
Aristote » ((Malebranche, {\it R. V.}, VI, 2, 2). — {\bf 2.} Propriété
sensible et non mesurable des choses : « Nous mettons la quantité de la cause
[de la sensation] dans la qualité de l'effet » (Bergson, {\it D. I.}, I) ; «
Dès le premier coup d'œil jeté sur le monde, nous y distinguons des {\it
qualités} » (id, {\it E. C.}, IV). {\it Qualités premières} ou {\it
primaires} : celles sans lesquelles les corps ne peuvent se concevoir
(étendue, impénétrabilité). {\it Qualités secondes} ou {\it secondaires}
celles qu'on peut supprimer par abstraction sans supprimer la notion de corps
(couleur, saveur, etc.). {\it Qualités occultes} propriétés non constatables
qu’on supposait dans la nature pour expliquer les phénomènes : « Les livres
de science... sont tous pleins de raisonnements fondés sur les qualités
élémentaires et sur les qualités secondes, comme les {\it attractrices}, les
{\it rétentrices}, les {\it concoctrices}, les {\it expultrices}, et autres
semblables ; sur d’autres qu'ils appellent {\it occultes}, sur les {\it
vertus spécifiques}… » (Malebranche, R. V., III, 2, 8, 1). — \si{Log.}
\si{form.} {\bf 3.} {\it Qualité d’une proposition} : le fait qu'elle est
affirmative ou négative.

— \si{Mor.} {\bf 4.} Disposition ou valeur morale, vertu$^2$ : « Le vrai
courage est une des qualités qui supposent le plus de grandeur d'âme
» (Vauvenargues). — {\it Ext.} {\bf 5.} Valeur, en gén. : « Le monde aurait
été sauvé plus d’une fois si la qualité des âmes pouvait dispenser de la
qualité des idées » (Brunschvicg).

\ib{Quanta} — Voir {\it Quantum}*.

\ib{Quantification} — \si{Phys.} {\bf 1.} Attribution d’un {\it quantum}* : «
La quantification des mouvements intraatomiques ». — \si{Log.} \si{form.}
{\bf 2.} {\it Quantification du prédicat} : théorie (de Hamilton) selon
laquelle on doit,
% 154
dans toute proposition, énoncer la quantité$^3$ de l’attribut$^1$, alors que,
selon la logique classique, celui-ci est toujours pris particulièrement dans
les affirmatives, universellement dans les négatives.

\ib{Quantique} — \si{Phys.} Qui repose sur l'hypothèse des {\it quanta}* : «
La Physique quantique » ; « La théorie quantique du champ$^1$ ».

\ib{Quantité} — (Opp. : {\it qualité}*), \si{Crit.} {\bf 1.} Une des
catégories* fondamentales de la pensée : abstraction$^2$ de la grandeur,
dépouillée de toutes ses qualités$^2$ et considérée seulement comme
mesurable. Voir {\it Continu}$^2$ et {\it Discontinu}$^2$. — \si{Log.}
\si{form.} {\bf 2.} {\it Quantité d'une proposition} : le fait que le
sujet$^2$ y est pris dans toute son extension$^3$ (prop.
{\it universelles}$^3$) ou bien dans une partie seulement de son extension
(prop. {\it particulières}$^2$). — {\it Ext.} {\bf 3.} {\it Quantité d'un
terme} : le fait qu'il est pris dans la totalité ou dans une partie seulement
de son extension$^3$.

\ib{Quantophrénie} — \si{Épist.} Terme inventé par le sociologue américain
Sorokin pour désigner ironiquement la tendance excessive à introduire la
quantité et la mesure dans les sciences de l'esprit (psychométrie,
sociométrie, statistiques, etc.).

\ib{Quantum} — \si{Phys.} {\it Quantum d'énergie} : unité de variation de
l'énergie$^1$ quand celle-ci est considérée comme variant de façon
discontinue$^1$; il a pour expression : $ w = h \nu $ ($h$ étant la constante
de Planck, et $\nu$ la fréquence du rayonnement) : « L’origine de la théorie
des {\it quanta} [de Planck] a été l'étude de la répartition de l’énergie$^1$
entre les fréquences dans le rayonnement » ({\it L.} de Broglie). {\it
Quantum d’action} : extension de la conception précédente à tous les
mouvements corpusculaires à très petite échelle : « Le quantum d'action$^3$
apparaît auj. comme l’une des réalités les plus fondamentales de la Physique
» (id.).

\ib{Quasi-contrat} — \si{Jur.} Engagement implicite sans convention expresse.
% 154

\ib{Quérulance} — [L. {\it querulus}, plaintif] — \si{Ps. path.} Tendance
pathologique à se plaindre d’injustices dont on se croit victime. Cf. {\it
Revendicatif}*.

\ib{Quiddité} — \si{Hist.} {\it Chez les scolastiques} : ce qui répond à la
question {\it quid} ?, {\it i. e.} l’essence*. {\it Quiddité spécifique}, ce
qui caractérise l’espèce$^2$.

\ib{Quiétisme} — \fsb{S. norma.} {\bf 1.} \si{Théol.} Doctrine de certains
mystiques (Molinos, Mme Guyon) qui faisaient consister la perfection
chrétienne dans le {\it pur amour}$^1$ : « Le quiétisme réclame de l'âme
pieuse un amour si désintéressé qu'elle devient indifférente au salut
» (Roustan). — {\it Ext.} \fsb{S. posit.} {\bf 2.} Attitude morale consistant
dans la contemplation pure et l'inactivité spirituelle. {\it D’où} :
\si{Pol.} : « Le temps n'est pas au quiétisme... Les événements ne sont
jamais neutres » (Lamartine, 1832) ; et \si{Mor.} : « Le quiétisme de
l’ivrogne solitaire » (Sartre).

\ib{Quotient d'intelligence} — \si{Ps. métr.} Mesure de l'intelligence par le
quotient de « l’âge mental » (donné par les tests*) divisé par l’âge réel :
pour le sujet normal, ce quotient est égal à {\bf 1.}

\begin{center}
\huge{R}
\end{center}

\ib{Race} — \si{Biol.} Variation de l’espèce$^3$ fixée par l’hérédité. $->$
Concept d'ordre biologique qui ne doit pas être confondu avec les notions
d’{\it ethnie}*, de {\it peuple}, de {\it nation}*, etc. : « La race,
représentant la continuité d'un type physique, représente un groupe naturel
pouvant n’avoir et n'ayant gén. rien de commun avec le peuple, la
nationalité, la langue, les mœurs » (Boule).
% 155

\ib{Racisme} — \si{Pol.} \fsb{S. norma.} Doctrine qui prend la notion de
race* comme base d’un système politique et qui privilégie une « race » par
rapport aux autres.

\ib{Radicalisme philosophique} — \si{Hist.} \fsb{S. norma.} Doctrine de
Bentham, de J. Mill et de J. Stuart Mill (confiance en la raison$^5$,
libéralisme$^\text{2 et 3}$, utilitarisme*, etc.).

\ib{Raison} — {\bf A)} \si{La Raison}. \fsb{S. subje.} \si{Psycho.} et
\si{Crit.} {\bf 1.} « Puissance de bien juger et de discerner le vrai d'avec
le faux... qu’on nomme le {\it bon* sens} ou la {\it raison} » (Descartes,
{\it Méth.}, I) : « La parfaite raison fuit toute extrémité » (Molière): «
Avoir {\it ou} perdre sa raison ». — {\bf 2.} Faculté directrice de la pensée
et de l’action humaines, que l’on fait qqfs. consister en un ensemble de
principes$^2$ universels et immuables ({\it raison constituée}). Mais A.
Lalande en dist. à juste titre la {\it raison constituante}, tendance active
vers l'identité$^3$, faculté normative* et inventive, dont la première n’est
que l'expression temporaire. — Souvent {\it opp.} à {\it expérience} ou à {\it
sensibilité}* : « La Raison pure et nue, distinguée de l'expérience, n’a à
faire qu'à des vérités indépendantes des sens » (Leibniz, {\it Théod.},
début) ; « La raison parle bas; il faut de l'attention pour l’entendre. Mais
les sens... » ((Malebranche). Spéc., {\it chez Kant} : « Toute notre
connaissance commence par les sens, passe de là à l’entendement ({\it
Verstand}) et s’achève dans la raison ({\it Vernunft})… Nous distinguons ici
la raison de l’entendement en la nommant la {\it faculté des principes}
» ({\it R. pure}, {\it Dial.}, introd., II, A) ; « Connaissance par raison et
connaissance {\it a priori} sont identiques » ({\it R. pr.}, préf.). — {\it
Au sens moral} : « La raison, en tant qu’elle nous détourne du péché
% 155
s'appelle la conscience » (Bossuet) ; « La règle de la conduite humaine, la
raison, est indispensable » (Renouvier) ; « L'acte moral est un acte de
raison » (Lagneau). — {\bf 3.} Faculté de connaître l'absolu$^1$ : « La
raison conçoit l'infini, l'éternel, l'absolu, le nécessaire » (Lacordaire).
{\it Spéc.}, participation à la Raïson divine, Logos* : « Chacun sent en soi
une raison bornée et subalterne qui ne se corrige qu'en rentrant sous le joug
d’une autre raison supérieure, universelle et immuable » (Fénelon) ; « La
Raison qui éclaire l’homme, est le Verbe ou la Sagesse de Dieu même
» (Malebranche). — {\bf 4.} (Raison « raisonnante » ou discursive). Faculté
de raisonner : « L’intellect$^2$ tire son nom de l’intime pénétration de la
vérité; la raison, de la recherche et du discours* » (St Thomas, {\it S.
th.}, II$^\text{a}$ II$^\text{ae}$, 5) ; « La raison compare une chose avec
une autre et en découvre les rapports » (Bossuet) ; « [Il faut] séparer
entièrement l’usage analytique et l’abus dialectique de la raison
» (Brunschvicg) ; « La raison est la plus belle de toutes nos facultés, à
condition qu’on n’en fasse pas la faculté qui raisonne, mais celle qui mesure
» (Lavelle; cf. sens 7). — {\bf 5.} (Opp. : {\it foi}$^5$). Connaissance
naturelle de l’esprit sans l’aide de la révélation : « Deux excès : exclure
la raison, n’admettre que la raison » (Pascal, 253) ; « Les dogmes de la foi
et les principes de la raison doivent être d'accord dans la vérité
» (Malebranche, {\it Entr.}, XIV, 4) ; « La sainte haine qu’il [le P. Canaye]
avait contre la raison » (St-Évremond).

— {\bf B)} \si{Des raisons}. \fsb{S. objec.} Épist,  {\bf 6.} Explication ;
ce qui fait comprendre ou justifie : « Les raisons d’un choix »; « Les
raisons me viennent
% 156
après » (Pascal, 276) ; « Le cœur a ses raisons, que la raison$^4$ ne connaît
point » (id., 277) ; « La raison des choses » (Cournot). {\it Principe de
raison suffisante} : celui qui pose que rien n'arrive sans raison, {\it i.e.}
sans qu'on puisse « rendre raison {\it a priori} pourquoi cela est existant
plutôt que non existant et pourquoi cela est ainsi plutôt que de toute autre
façon » (Leibniz, {\it Théod.}, I, 44). — {\bf 7.} Rapport, proportion, loi
[cf. L. {\it reor}, calculer; d’où : {\it ratio}] : « Livre de raison » [= de
comptes]. {\it Spéc.}, \si{Math.} : « en raison directe (ou inverse) de... »,
« moyenne et extrême raison », « la raison d’une progression ». {\it Ext.} :
« La pensée est la raison commune en laquelle les actes intellectuels
conviennent,... l’extension$^1$, la raison commune des actes corporels
» (Descartes, 3$^\text{es}$ {\it Rép.}, II). — \si{Pol.} {\bf 8.} {\it Raison
d'État} : considérations d'intérêt publie par lesquelles on justifie
certaines mesures discutables du point de vue de la pure justice$^1$. —
{\bf 9.} {\it Raisons séminales} : voir {\it Séminales}*.

$->$ La « raison$^\text{A}$ subjective » et la « raison$^\text{B}$ objective
» (Cournot) présentent cependant une certaine unité, qu’indique assez bien ce
philosophe dans {\it Matérialisme, Vitalisme, Rationalisme,} III, § 1 : « Ces
théories et ces méthodes, ces sciences... sont du domaine de la
\textsc{Raison}, de cette faculté supérieure par laquelle l’homme parvient à
dégager l'idée pure de son enveloppe sensible, à saisir un ordre
intelligible, suivant lequel les faits, tant sensibles qu'intelligibles, {\it
rendent raison} les uns des autres. »

\ib{Raisonnable} — \si{Psycho.} et \si{Mor.} \fsb{S. subje.} {\bf 1.} Doué de
raison$^1$, ou : qui agit selon la raison : « Celui qui n’a vu que des hommes
polis et raisonnables,
% 156
ne connaît pas l’homme » (La Bruyère) ; « Se placer dans une attitude
impartiale, impersonnelle,... cela s'appelle être raisonnable » (Rauh). —
\si{Épist.} \fsb{S. objec.} {\bf 2.} {\it Laud.} En faveur de quoi on peut
donner de bonnes raisons$^5$, {\it ou} : conforme à la raison$^1$ : « C’est
une solution raisonnable »; « II n’est permis de suivre les mouvements du
cœur que lorsqu’ils sont raisonnables » (Malebranche) ; « Qu’y a-t-il de
moins raisonnable que de prendre notre intérêt pour motif de croire une
chose ? » (Port-Royal, III, 20, 1).

\ib{Raisonnement} — \si{Psycho.} et \si{Log.} \fsb{S. abstr.} {\bf 1.}
Opération discursive* de la pensée consistant à enchaîner logiquement des
jugements [avec plus ou moins de rigueur selon le type de raisonnement
considéré) et à en tirer une conclusion$^1$ : « Le raisonnement est
le mouvement par lequel la pensée se réalise dans les jugeinents successifs
» (Lagneau). — \fsb{S. concr.} {\bf 2.} Résultat de cette opération : « Un
raisonnement correct »; « Ces tables* [de Bacon] ne sont pas par elles-mêmes
des raisonnements » (Goblot).

\ib{Rapport} — \si{Épist.} {\bf 1.} Relation$^1$ en gén. : « Le monde est une
hiérarchie de rapports » (Hamelin). — {\bf 2.} Affinité : « Les parties du
monde ont un tel rapport l’une avec l’autre... » (Pascal, 72) ; « Dieu a du
rapport avec l’univers comme créateur et comme conservateur » (Montesquieu,
{\it Lois}, I, 1).

\ib{Raptus} — \si{Ps. path.} Impulsion brusque et irrésistible qui pousse le
sujet à des actes tels que violences, suicide, fuite éperdue, etc.

\ib{Rationalisme} — \fsb{S. norma.} \si{Crit.} {\bf 1.} (Opp. : {\it
empirisme}$^2$). Doctrine qui pose la raison$^2$ comme indépendante de
% 157
l'expérience, et même souvent (il vaudrait mieux dire alors
{\it apriorisme}*) affirme que la raison$^2$ est innée, {\it a priori},
immuable et égale chez tous les hommes : « Le rationalisme le plus intrépide,
le plus intransigeant est celui des vieux philosophes Éléates* » (Roustan). —
{\bf 2.} (Opp. : {\it mysticisme}$^2$ ou {\it traditionalisme}$^2$). Doctrine
qui affirme l’autorité souveraine de la raison$^2$ et rejette l'intervention
du sentiment$^4$ ou de la tradition dans l’ordre théorique : « Il faut, dit
le rationalisme, aller vers la vérité avec la seule intelligence » ((Goblot).
— {\it Spéc.} \si{Théol.} {\bf 3.} Doctrine de ceux qui rejettent toute
révélation et tout surnaturel et ne veulent admettre que la raison au sens 5.

— \fsb{S. posit.} {\bf 4.} Rationalité, pensée rationnelle$^1$ : « Toute
critique scientifique doit ramener les faits au rationalisme » (Cl. Bernard).
— {\bf 5.} \si{Ps. path.} {\it Rationalisme morbide} : nom donné par
Minkovski à l'excitation intellectuelle de certains schizophrènes*, qui les
fait raisonner d’une façon absolue sans tenir compte d'aucun sentiment humain.

\ib{Rationnel} — \si{Épist.} et \si{Crit.} {\bf 1.} Qui se rapporte à la
raison, aux sens 1, 2 ou 3 : « Principes rationnels » ; « J’oppose le
rationnel à l'empirique » (Kant, {\it R. pure}, Methodenlehre, III) ; « Faire
œuvre de science rationnelle, c’est chercher à formuler qq. relation
constante en des propositions qui se nomment des lois$^5$ » (Milhaud, Le
Rationnel). —  {\bf 2.} Qui se fonde sur la raison$^4$, {\it i. e.} sur la
déduction pure {\it opp.} à la méthode expérimentale, {\it Sciences
rationnelles} : les mathématiques. {\it Psychologie rationnelle} : v. {\it
Psychologie}$^2$. — {\bf 3.} (Syn. : {\it raisonnable}). Conforme à la
raison : « Voilà une idée raisonnable ; maintenant on dit bien plus
dignement : voilà une déduction rationnelle » (Musset) ; « Une éducation
rationnelle » (Comte).

— \si{Math.} {\bf 4.} {\it Nombre rationnel} (cf. {\it Raison}$^7$) : celui
qui peut se mettre sous la forme d’un rapport entre deux entiers.

\ib{Réaction} — \si{Math.} et \si{Phys.} {\bf 1.} {\it En Mécanique} :
action$^5$ qu’un corps exerce sur un autre en réponse à une action de celui
ci : « La réaction est toujours égale et opposée à l'action$^5$. » {\it En
Chimie} : phénomène qui résulte de la mise en contact de différents corps. —
\si{Biol.} et \si{Psycho.}  {\bf 2.} (Adj. correspondant : {\it
réactionnel}). Réponse de l'organisme à un stimulus* : cf.
{\it Psychologie}$^1$ et {\it Temps}$^2$. — \si{Pol.} {\bf 3.} (Adj.
corresp. : {\it réactionnaire}). Mouvement en sens contraire : « La réaction
thermidorienne », et {\it spéc.} en sens opposé au progrès$^2$ : « Les partis de réaction. »

\ib{Réaliser} — {\bf 1.} Faire exister à titre de réalité objective$^2$ : «
Réaliser un projet » ; « Plotin s'engageait à réaliser la république de
Platon » (Diderot). — {\bf 2.} Faire exister à titre de réalité mentale,
penser eflectivement : « Il ne réalisait pas son échec ». — {\bf 3.} (Souvent
{\it péj.}) Considérer comme réelle une entité$^2$ abstraite : « Les
philosophes ont été de tous temps sujets à réaliser leurs abstractions$^2$
» (Condillac).

\ib{Réalisme} — \si{Vulg.} {\bf 1.} \fsb{S. posit.} Sens des réalités : « Un
homme d’État doit faire preuve de réalisme. » Qqfs. {\it péj.} : « Un
réalisme étroitement utilitaire. »

— \fsb{S. norma.} \si{Esth.} {\bf 2.} (Opp. : {\it idéalisme}$^2$ Syn. : {\it
naturalisme}$^3$). Doctrine esthétique qui assigne pour but à l’art la
reproduction exacte du réel : « Le réalisme et l’idéalisme$^2$ ont un postulat
% 158
commun : l'art existerait tout fait dans la réalité empirique ou
transcendante » (Delacroix). — \si{Méta.} {\bf 3.} (Opp. :
{\it nominalisme}$^1$ et {\it conceptualisme}$^3$). Doctrine de Platon et de
certains scolastiques, qui attribue aux Idées$^1$ ou aux universaux* une
existence en soi : « Le réalisme de saint Anselme ». — \si{Crit.} et
\si{Méta.} {\bf 4.} (Opp. : {\it idéalisme}$^4$). Nom générique des doctrines
qui admettent que la connaissance saisit des réalités véritables, des «
choses » (res) ayant une existence indépendante de la pensée : « Le réalisme
s’installe d'emblée dans l'être : il admet que la pensée accède à l'être
directement » (Blanché).

— \si{Psycho.} {\bf 5.} (Par ext. du sens 4). Tendance de l’enfant à
attribuer à l’objet les résultats de l’activité du sujet (Piaget). Cf. {\it
Précis}, Ph. I, p. 84.

\ib{Réalité} — Voir {\it Réel}*. \si{Épist.} : {\it Jugements de réalité}
(opp. : {\it jug. de valeur}$^2$). Ceux qui énoncent des faits ou des
rapports entre des faits. $->$ On dit encore : {\it jugements constatifs}* ou
{\it positifs}$^4$.

\ib{Réciprocité} — \si{Crit.} {\bf 1.} Catégorie (nommée par Kant {\it
communauté}$^1$) qui exprime l’action réciproque$^1$ de deux substances. —
{\bf 2.} {\it Réciprocité de perspectives} : notion introduite par Th. Litt
({\it Individuum und Gemeinschaft}, 1913) pour interpréter l’action
réciproque$^1$ de l'individuel et du social.

\ib{Réciproque} — \si{Vulg.} {\bf 1.} De l’un à l’autre, mutuel : « Ce besoin
d’une confiance réciproque » (Voltaire). — \si{Log.} {\bf 2.} La réciproque
d’une proposition hypothétique est une autre proposition hypothétique ayant
pour antécédent$^2$ le conséquent$^2$ de la première et pour conséquent
l’antécédent
% 158
de la première. Schéma : soit « si P est vrai, Q est vrai » ; la réciproque
est : « si Q est vrai, P est vrai ». $->$ Dist. {\it inverse}*.

\ib{Reconnaissante} — \si{Psycho.} Fonction de la mémoire$^4$ qui consiste :
1° à éprouver à l’égard des souvenirs le sentiment du {\it déjà-vu}* ; 2° à
les attribuer consciemment au passé : « La reconnaissance est un jugement, un
acte de l’entendement » (Lagneau). {\it Fausse reconnaissance} : v. {\it
Paramnésie}* et {\it Pseudomnésie}*.

\ib{Récurrence} — \si{Épist.} {\bf 1.} Retour en gén., répétition : « Les
récurrences du rythme » (Bayer). — {\bf 2.} Retour sur soi, réaction de
l'effet sur sa propre cause : « La science sociale contribue à modifier son
objet : c’est ce que nous appellerons la récurrence de l’action et de la
connaissance sociales » (Belot) ; « Une histoire {\it récurrente}, une
histoire qu'on éclaire par la finalité du présent » (Bachelard). — {\bf 3.}
{\it Raisonnement par récurrence} (Poincaré) ; celui qui consiste : 1° à
vérifier une relation mathématique pour une valeur$^6$ déterminée de $n$ ; 2°
à démontrer que, si la relation est vraie pour $n - 1$, elle est vraie pour
$n$ ; 3° à l’étendre à toute la série des nombres entiers.

\ib{Rédintégration (Loi de)} — \si{Psycho.} (Syn. : {\it loi de
totalisation}). Un élément psychique qui reparaît à la conscience$^1$, tend à
restaurer l’état total dont il a fait partie.

\ib{Réduction} — \si{Psycho.} {\bf 1.} {\it Chez Taine} : « réduction des
images », le fait qu’une représentation imaginaire se trouve repoussée dans
l'irréel par les sensations qui la contredisent. — \si{Méta.} {\bf 2.} {\it
Dans le lang. phénoménologique} « réduction eïdétique », celle qui consiste à
éliminer les éléments
% 159
empiriques du donné pour n’en retenir que la pure essence; « réduction
phénoménologique », celle qui consiste à mettre « entre parenthèses* » les
{\it existences} empiriques. — \si{Log.} {\bf 3.} Cf. {\it Absurdes}$^3$.

\ib{Réel} — \si{Crit.} et \si{Méta.} {\bf 1.} Qui existe comme existe une «
chose » ({\it res}), opp. : {\it a)} à l’apparent et à l’illusoire; {\it b)}
au simple possible ; {\it c)} à l’abstrait et à l’intelligible$^1$ : « Le
réel est ce qui est actuellement donné ou ce qui peut être donné dans une
expérience... Le {\it vrai} et le {\it réel} sont choses fort différentes.
L’appréhension du réel dans une intuition est déjà une vérité ; mais, à
partir de cette intuition, l'intelligence progresse dans la vérité en
s’éloignant du réel » (Goblot) ; « La science part de l’expérience,
{\it i. e.} du {\it réel} et tend constamment à l'{\it intelligible} » (id.).

— \si{Épist.} {\bf 2.} Qui concerne les « choses », opp. ce qui concerne :
{\it a) les mots} : « définition réelle », celle qui porte sur la chose
elle-même, opp. à « déf, nominale* », plus ou moins conventionnelle ; —
{\it b) les idées} : « la réelle distinction qui est entre l’âme et le corps
» (Descartes, {\it Méd.}, VI), {\it i. e.} celle qui est « selon l’ordre de
la vérité de la chose », {\it opp.} à la distinction « selon l’ordre de ma
pensée », {\it i. e.} entre l’idée de l’âme et l’idée du corps, posée dans
{\it Méd.} II. {\it Analyse réelle} : celle qui décompose l'objet$^5$ lui
même ({\it p. e.} analyse chimique), {\it opp.} à {\it analyse idéale}$^1$
qui décompose l'idée de l’objet ; — {\it c) les personnes} : (\si{Jur.}) «
droits réels », « domaine réel », droits ou pouvoir sur les choses.

— \si{Psycho.} {\bf 3.} {\it Fonction du réel} (Janet) : sentiment de la
réalité perçue et adaptation des actes et de la pensée à cette réalité et au
moment présent.

\ib{Referendum} — \si{Pol.} Consultation des citoyens sur un projet de loi.

\ib{Réflexe} — \si{Phol.} Réaction$^2$ (inhibition, contraction musculaire ou
sécrétion) involontaire succédant automatiquement$^1$, en vertu de connexions
préétablies dans le système nerveux, à l'excitation d’un nerf sensitif. {\it
Réflexe conditionné} ou {\it conjonctif} : réflexe qui, provoqué d’abord par
une excitation À, s’est associé par répétition à une excitation B et finit
par être provoqué par celle-ci. Cf. {\it Précis}, Ph. I, p. 66.

\ib{Réflexif} — \si{Crit.} Qui implique une prise de conscience des
opérations de la pensée par elle-même : « Le caractère réflexif que présente
le progrès de la science moderne » (Brunschvicg). — Cf. {\it Sensitif}$^3$.

\ib{Réflexion} — \si{Psycho.} {\bf 1.} {\it Chez Locke} : « connaissance que
prend l'esprit de ses opérations et de leurs caractères ». {\it Cf.}
Ravaisson : « La vraie méthode psychologique est celle qui, du fait de telle
sensation ou perception, distingue par une opération particulière ce qui
l’achève en la faisant nôtre : cette opération, c'est la réflexion. » —
{\bf 2.} (\si{Vulg.}) Attention intellectuelle$^2$ : « {\it Réflexion} a
premièrement désigné le mouvement d’un corps qui revient après avoir heurté
contre un autre ; il est devenu le nom qu'on donne à l'attention lorsqu'on la
considère comme allant et revenant d’un objet sur un objet » (Condillac).

\ib{Refoulement} — \si{Ps. an.} Processus qui fait qu’une idée ou une
tendance pénible ou dangereuse se trouve rejetée et maintenue dans
l’inconscient. $->$ D'abord attribué par Freud au {\it moi} conscient, le
refoulement a ensuite été considéré par lui
% 160
comme un processus inconscient émanant du {\it sur-moi}*.

\ib{Règle} — \si{Épist.}, \si{Mor.}, \si{Esth.}, etc. {\bf 1.} Formule
prescriptive qui commande ou indique ce qu'il faut faire, en un ordre
d'action quelconque : « Aristote trace les règles de la tragédie de la même
main dont il a donné celles de la dialectique, de la morale, de la politique
» (Voltaire). — {\bf 2.} Qqfs., mais imppt., syn. de {\it Loi}$^5$ (parce que
celle-ci fut d’abord considérée comme une loi$^3$ imposée par Dieu à la
nature) : « Ces règles [sans lesquelles le Créateur ne pourrait gouverner le
monde] sont un rapport constamment établi » (Montesquieu, {\it Lois}, I, 1) ;
« Tous les corps se meuvent selon certaines lois ou règles,... qui dépendent
de la libre volonté de Dieu » (Sigaud de Lafond, 1769).

\ib{Règne des fins} — \si{Hist.} {\it Chez Kant} : idéal de la Raison
pratique consistant en une « union systématique » des êtres raisonnables se
considérant tous réciproquement comme des {\it fins en$^4$ soi} : « Dans le
règne des fins, tout a un prix ou une dignité » ({\it Fond. méta. des mœurs},
II).

\ib{Régression} — Retour en arrière. D'où : {\bf 1.} \si{Biol.}, \si{Psycho.}
Arrêt de développement ou retour à un type moins évolué. {\it Spéc.},
\si{Ps. an.} Retour de la {\it libido} à un stade infantile. — {\bf 2.}
\si{Ps. path.} {\it Loi de régression} : loi selon laquelle, dans l’amnésie
progressive, la perte des souvenirs va « du plus nouveau au plus ancien, du
complexe au simple, du volontaire à l’automatique, du moins organisé au mieux
organisé » (Ribot). — {\bf 3.} \si{Soc.} et \si{Pol.} Transformation en sens
inverse du progrès$^2$ : « Le totalitarisme* est une régression. »
% 160

\ib{Réifier} — Faire une chose$^3$ de ce qui est mouvant : « Les procédés de
réification conceptuelle dont use la raison discursive » (Le Roy).

\ib{Relatif} — \si{Crit.} et \si{Méta.} 1, (Opp. : {\it absolu}$^1$). Qui ne
se suffit pas à soi-même : « Le relatif ne se conçoit que par contraste avec
une existence en soi et par soi » (Liard)). — {\bf 2.} (Opp. :
{\it absolu}$^2$). Qui dépend d'un paramètre$^2$ plus ou moins
conventionnel : « Pourquoi et dans quelle mesure l’espace est-il relatif ? Si
tous les objets et notre corps lui-même, ainsi que nos instruments de mesure,
étaient transportés dans une autre région de l'espace, sans que leurs
distances mutuelles varient, nous ne nous en apercevrions pas Le temps
mesurable est aussi essentiellement relatif » (Poincaré). — {\bf 3.} (Opp. :
{\it absolu}$^3$. Syn. : {\it a posteriori}*. Qui dépend de l'expérience : «
Cette vérité [expérimentale] n’est jamais que relative au nombre
d’expériences et d'observations qui ont été faites » (Cl. Bernard). —
{\bf 4.} Qui implique une relation$^1$ ou est constitué par des
relations$^1$ : « Cause et effet sont des termes relatifs » (Condillac) ; «
L’absolu est encore le relatif,... parce qu'il est le système des
relations$^1$ » (Hamelin). — \si{Vulg.} {\bf 5.} (Opp. : {\it absolu}$^4$).
Qui a des limites : « Toute puissance est relative » (Montesquieu, {\it
Lois}, IX, 9).

$->$ Dire que « la connaissance
humaine est relative » peut s’entendre en deux sens : cf. {\it Relativité}*.

\ib{Relation} — \si{Crit.} {\bf 1.} Une des catégories fondamentales de la
pensée, et même, selon Renouvier, « la catégorie des catégories » ({\it Log.
générale}, III, 27) : « Qu'est-ce que penser, sinon poser des relations ? »

— \si{Épist.} \fsb{S. abstr.} {\bf 2.} Rapport entre
deux objets, phénomènes ou quantité,
% 161
tel que toute modification de l’un entraîne une modification de l’autre : «
La science recherche des relations$^2$ constantes entre les phénomènes ». —
\fsb{S. concr.} {\bf 3.} Formule exprimant une relation$^2$, {\it spéc.} en
\si{Math.} égalité$^3$ ou inégalité.

— \si{Log.} {\bf 4.} {\it Propositions de relation} : celles qui énoncent une
relation$^2$ autre que celle d’{\it inhérence}$^2$ (v. ce mot).

\ib{Relativisme} — \si{Hist.} \fsb{S. norma.} ({\it Opp.} à la fois à {\it
dogmatisme}$^3$ et à {\it scepticisme}*). Doctrine qui admet la relativité*
de la connaissance ({\it p. e.} criticisme*).

\ib{Relativité} — \si{Crit.} {\it Relativité de la connaissance} : caractère
de la connaissance d’être « relative », en ce sens : {\bf 1.} qu’elle ne peut
porter que sur des relations$^2$ (Hamilton, Comte) ; — ou bien : {\bf 2.}
qu’elle dépend du sujet$^\text{4a}$ connaissant et de la constitution de
l'esprit humain (Kant). $->$ Dans les 2 sens, {\it dist.} avec soin {\it
relatif} de {\it faux} ou {\it inadéquat}. Au sens 1, Hamelin écrit : « La
relativité de la connaissance n’est pas, comme on l’a cru qqfs., un obstacle
au savoir : elle en est le moyen ». Au sens 2, voir {\it Phénomène}$^2$.

— \si{Phys.} {\bf 3.} {\it Principe de relativité} (Einstein) : principe
selon lequel : 1° les lois des phénomènes physiques sont les mêmes pour
différents groupes d’observateurs en mouvement de translation uniforme
(relativité restreinte) ou uniformément accélérée ({\it p. e.} gravitation :
relativité généralisée) les uns par rapport aux autres; 2° par suite, la
durée des phénomènes varie suivant qu’elle est mesurée par des observateurs
en repos ou en mouvement par rapport à eux. $->$ Bien {\it dist.} ce sens des
deux précédents auxquels il est même « opposé à certains égards » (Lalande).

%CUVILLIER. — Vocabulaire philosophique.
% 161
\ib{Religion} — \si{Théol.} {\bf 1.} St. Thomas ({\it S. th.}, II$^\text{a}$
II$^\text{ae}$, 81, 5) définit la religion come une vertu$^2$ morale dont
Dieu est la fin et le culte rendu à Dieu l’objet ou la matière$^1$ — \si{Soc.}
{\bf 2.} « Système de croyances [dogmes] et de pratiques [rites] relatives à
des choses {\it sacrées}* et qui unissent en une même communauté morale,
appelée {\it église}, tous ceux qui y adhèrent » (Durkheim). Cf.
{\it Naturel}$^2$ et {\it Positif}$^6$.

\ib{Remémoration} — \si{Psycho.} (Syn. {\it anamnèse}$^1$). Évocation*
volontaire des souvenirs.

\ib{Réminiscence} — \si{Psycho.} {\bf 1.} Souvenir incomplet qui n’est pas
reconnu comme passé : « [l mêle trop à ses souvenirs ceux des autres et ceux
mêmes de ses lectures : c’est ce qu'on peut appeler des réminiscences
» (Sainte-Beuve). — {\bf 2.} Retour spontané à la conscience d’un souvenir
confus : « La réminiscence est un réveil fortuit de traces anciennes dont
l'esprit n’a pas la conscience nette et distincte » (id.).

— {\bf 3.} \si{Hist.} {\it Chez Platon} : « ressouvenir de ce que notre âme a
vu » [les Idées] dans une existence antérieure ({\it Phèdre}, 249 c).

\ib{Remords} — \si{Mor.} Angoisse de la conscience qui a le sentiment d’avoir
commis une faute.

\ib{Repentir} — \si{Mor.} Tristesse de la conscience qui, tout en ayant le
sentiment d'avoir commis une faute, s'efforce de s’en détacher et de
s'orienter vers une vie meilleure.

\ib{Représentatif} — \si{Psycho.} {\bf 1.} (Syn. {\it cognitif}*). Qualifie
les faits de connaissance en tant qu’ils présentent à l'esprit un objet$^5$.

— \si{Pol.} {\bf 2.} {\it Régime représentatif} : celui où le peuple délègue
à des représentants le pouvoir législatif.

% 162
\ib{Représentation} —- \si{Psycho.} et \si{Crit.} {\bf 1.} Fait de conscience
représentatif$^1$; fait intellectuel$^1$. {\it Spéc.}, {\it chez Renouvier} :
«synthèse du sujet et de l’objet dans une conscience », la « chose » en tant
que présente à l'esprit ; la {\it représentation} comprend deux éléments : le
{\it représenté} qui est « ce qu’on appelle un corps avec ses qualités
» (tout ce qui est perçu, senti, etc.), et le {\it représentatif} « qui
rentre dans la classe courante de l'esprit » (pensée, affection, volonté). —
{\bf 2.} {\it Qqfs.} (en donnant au préfixe {\it re} un sens itératif et par
opp. à {\it présentation}), état secondaire*, reproduction d’un état
antérieur.

\ib{Répugner} — \si{Log.} {\it Autref.}, impliquer$^2$ contradiction : « Il
répugne que qqc. vienne de lui [Dieu] qui tende à la fausseté » (Descartes,
2$^\text{es}$ {\it Rép.}).

\ib{Résidus (Méthode des)} — \si{Épist.} Une des quatre méthodes
expérimentales de J. Stuart Mill (voir {\it Précis}, Ph. II, p. 123; Sc. et
M., p. 239).

\ib{Respect} — \si{Mor.} {\bf 1.} {\it Chez Kant} : sentiment que nous
éprouvons en présence d’une valeur morale : « Le respect s’adresse aux
personnes, jamais aux choses » — {\bf 2.} {\it Respect humain} :
pusillanimité qui nous fait « craindre la censure du monde » (Bourdaloue).

\ib{Responsabilité} — \si{Jur.} \fsb{S. objec.} {\bf 1.} {\it Resp. civile} :
obligation$^3$, déterminée par la loi$^1$, de réparer le dommage causé à
autrui. — {\bf 2.} {\it Resp. pénale} : état de celui qui peut être poursuivi
pour un crime ou un délit.

— \si{Mor.} \fsb{S. subje.} {\bf 3.} {\it Resp. morale} : état de l'agent
moral qui : 1° se reconnaît l’auteur de ses actes; 2° en assume le mérite ou
le démérite : « La responsabilité est la solidarité de la personne humaine
avec ses actes » (Blondel).

% 162
\ib{Ressemblance (Association par)} — \si{Psycho.} Un des modes de
l’association$^3$ des idées.

\ib{Ressentiment} — \si{Mor.} Sentiment de rancune qui, selon Nietzsche, est
à l’origine de la « morale des esclaves* » Scheler a repris une thèse
analogue en définissant le ressentiment « un {\it autoempoisonnement
psychologique},... une disposition d'une certaine permanence qui, par un
refoulement* systématique, libère certaines émotions et sentiments [désir de
vengeance, haine, méchanceté, envie] et tend à provoquer une déformation plus
ou moins permanente du sens des valeurs et de la faculté de juger ».

\ib{Retentissement} — \si{Car.} Phénomène qui caractérise la
{\it secondarité}* et qui consiste en ce que les impressions disparues du
champ de la conscience continuent « à influencer notre manière d'agir et de
penser » (Berger).

\ib{Retour éternel} — \fsb{S. norma.} \si{Méta.} Théorie de certains
philosophes anciens (Héraclite, Pythagore, Stoïciens) selon laquelle
l'univers repasse toujours, au terme de plusieurs milliers d'années, par les
mêmes phases. La théorie a été reprise par Nietzsche.

\ib{Rétrograde} — \si{Ps. path.} {\bf 1.} {\it Amnésie rétrograde} : celle où
l'oubli remonte vers le passé à partir du choc (physique ou moral) qui lui a
{\it gén.} donné naissance. — \si{Soc.} {\bf 2.} (Péj.) Qui va en sens
contraire du Progrès$^2$. Fréquent {\it chez Comte} : « L'école rétrograde
» ({\it Cours}, 46$^\text{e}$ leçon) ; « Depuis trois siècles, son influence
[de la « politique théologique$^2$ »] a été essentiellement rétrograde » ({\it
ibid.}).

\ib{Rêve} — \si{Psycho.} {\it Str.} {\bf 1.} État de la pensée pendant le
sommeil : « Le rêve est la vie mentale tout entière, moins l'effort de
concentration » (Bergson, {\it E. S.}, IV). — {\it Lato.} {\bf 2.} Rêverie*,
pensée qu'on laisse aller sans faire effort pour l'adapter à l’action : « Un
être humain qui {\it réverait} son existence au lieu de la vivre... » (id.,
{\it M. M.}, III).
% 163

— \si{Ps. path.} {\bf 3.} Rêve éveillé : autisme* des schizophrènes. —
{\bf 4.} {\it Délire de rêve} : syn. {\it Onirique}$^2$.

\ib{Revendicative (Psychose)} — \si{Ps. path.} Celle où « la personnalité se
durcit dans des récriminations incessantes et des démarches sans fin, qui
constituent la quérulance* » (Baruk). Cf. {\it Persécution}*.

\ib{Rêverie} — \si{Psycho.} Forme de pensée vague, plus ou moins passive,
qqfs. cependant plus ou moins dirigée : « Les pernicieuses rêveries de
l'oisiveté » (Bossuet) ; « Dans la rêverie, on n’est point actif
» (Rousseau) ; « La rêverie complète la vie : elle est l'épanouissement des
tendances refoulées et des virtualités insatisfaites » (Delacroix).

\ib{Réversible} — {\bf 1.} Qui peut être reporté sur autrui, {\it Spéc.},
\si{Théol.} « Réversibilité des mérites » (cf. {\it Communion}$^2$). —
{\bf 2.} Qui peut se faire, soit dans un sens, soit dans un autre :
{\it p. e.} \si{Phys.} : « Aucune transformation d'énergie n’est
intégralement réversible » (cf. {\it Dégradation}*) ; \si{Log.} «
L'intelligence de la mobilité réversible : c’est là le caractère essentiel
des opérations de la logique vivante » (Piaget).

\ib{Réviviscence} — Réapparition d’un état disparu. {\it Spéc.},
\si{Psycho.} : « La réviviscence des souvenirs. »

\ib{Révolution} — \si{Phys.} {\bf 1.} Déplacement d’un mobile sur une courbe
fermée : « Les carrés des temps des révolutions des planètes sont entre eux
comme les cubes de leurs distances du centre commun de leur révolution
» (Malebranche, Éd. XVI) — {\bf 2.} {\it Révolutions du globe} : théorie
géologique de Cuvier ({\it auj.} abandonnée) selon laquelle l'écorce
terrestre aurait été le siège de cataclysmes successifs détruisant les
espèces auj. fossiles.

— \si{Soc.} et \si{Pol.} {\bf 3.} Changement brusque et gén. violent de
régime politique et social : « L’ordre de la société est sujet à des
révolutions inévitables » (Rousseau), $->$ A. Camus a dist. la {\it
révolution} qui « met le ressentiment* à la place de l'amour » et aboutit à
une « mécanique meurtrière et démesurée », et la révolte qui est refus de
l'injustice et sursaut de la conscience (voir {\it Textes choisis}, II, p.
178).

\ib{Rôle social} — \fsb{S. norma.} \si{Soc.} Ensemble des comportements
présentant une certaine unité qui caractérisent dans la société un individu
qui y occupe une situation$^2$ particulière ({\it p. e.} père, mari, médecin,
professeur, chef de groupe) ou qui cherche à incarner une valeur particulière
({\it p. e.} patriote, honnête homme).

\ib{Rompus (Nombres)} — \si{Math.} Ancien nom des fractions.

\ib{Rythme} — \si{Biol.} {\bf 1.} Caractère périodique des phénomènes
vitaux : « Le rythme est d’abord un fait vital, un phénomène organique
» (Delacroix). {\it Rythmes vitaux} comportements périodiques de certains
êtres vivants selon l’alternance du jour et de la nuit, des saisons, des
marées, etc. (v. {\it Précis}, Ph. I, p. 442).

— \si{Esth.} {\bf 2.} Alternance périodique des mouvements ou des sons dans
une œuvre d’art (danse, musique, poésie) : « Le rythme marque la supériorité
du temps ordonné, intellectualisé, la victoire de l’intelligence
organisatrice qui anime la durée monotone du temps » (Delacroix).

	\end{itemize}
