
	\begin{itemize}[leftmargin=1cm, label=\ding{32}, itemsep=1pt]

\ib{Nappe} — \si{Biol.} « Unité de masse évolutive » (Teilhard de Chardin).

\ib{Narcissisme} — \si{Ps. an.} État mental dans lequel la {\it libido}$^2$
est dirigée uniquement vers le moi.

\ib{Narcoanalyse} — \si{Ps. phol.} Mode d'exploration de l’inconscient (pour
des fins, soit psychiatriques, soit
% 123
policières) à l’aide de narcotiques abolissant le contrôle personnel.

\ib{Natalité} — \si{Soc.} Rapport du nombre des naissances au chiffre de la
population pendant une période donnée.

\ib{Nation} — \si{Soc.} {\bf 1.} Groupe social déterminé à la fois par
certaines conditions naturelles et objectives (unité de langue ; indépendance
économique ; unité de gouvernement, etc.) et par des conditions spirituelles
et subjectives (communauté de souvenirs, volonté d’une fin politique
distincte, etc.) : « La nation moderne est un résultat historique amené par
une série de faits convergeant dans le même sens » (Renan). — {\bf 2.}
(Opp. : {\it gouvernement, État}$^2$). Ensemble des citoyens en tant
que constituant une unité morale : « Le principe de toute souveraineté réside
dans la nation » (Décl. 1789).

\ib{Nationalisme} — \si{Soc.} et \si{Mor.} {\bf 1.} \fsb{S. posit.}
Exaltation du sentiment national, « amour immodéré de la nation » (pape Pie
XI). — {\bf 2.} \fsb{S. norma.} Doctrine politique qui fait de la nation$^1$
un absolu. $->$ Dist. {\it patriotisme} (cf. {\it Précis}, Ph. IT, p.
397 ; Sc. et M. p. 400).

\ib{Nationalité} — \si{Soc.} et \si{Pol.} \fsb{S. abstr.} {\bf 1.} Caractère
que possèdent les membres d’une nation$^1$. — \fsb{S. concr.} {\bf 2.} Syn.
de nation$^1$, avec cette différence que la nationalité peut exister à l’état
diffus* avant que la nation soit organisée politiquement (État$^1$) ou après
qu’elle a cessé de l’être. {\it Principe des nationalités} celui d’après
lequel chaque nation doit être considérée comme une personne morale autonome
et a le droit d’exister et de se développer selon son génie propre.

\ib{Nativisme} — Voir {\it Génétique}$^4$.
% 124

\ib{Naturalisme} — \fsb{S. norma.} \si{Méta.} {\bf 1.} Doctrine qui nie
l'existence du surnaturel*. — \si{Mor.} {\bf 2.} Doctrine selon laquelle la
vie psychique n’est que le prolongement de la vie organique. — \si{Esth.}
{\bf 3.} Syn. de {\it réalisme}$^2$.

\ib{Naturante (Nature)} — \si{Hist.} Expression scolastique reprise par
Spinoza chez qui elle désigne le monde en tant que substance infinie, {\it i.
e.} Dieu (opp. {\it nature naturée}, considérée dans la diversité de ses
modes$^2$ finis).

\ib{Nature} — \si{Méta.} {\bf A)} \si{Nature d'un être} : {\bf 1.} Ce qu'un
être est par lui-même ; essence* de cet être (cf. {\it Forme}$^\text{1b}$) :
« La nature de l’amour-propre est de n’aimer que soi » (Pascal, 100) ;
« Notre esprit est de nature à vivre toujours » (Bossuet). {\it Spéc.},
\si{Théol.} (Opp. : {\it grâce}$^1$). Ce que l’homme ou le monde sont de par
leur essence propre, sans intervention surnaturelle* de Dieu : « La nature
est une image de la grâce » (Pascal, 675). {\it Dist.} ici : {\it a)} la
nature avant le péché; {\it b)} la nature corrompue, après le péché : « Il
leur reste [aux hommes] quelque instinct$^4$ impuissant du bonheur de leur
première nature$^\text{a}$, et ils sont plongés dans les misères de leur
concupiscence, qui est devenue leur seconde nature$^\text{b}$ » (id., 430). —
{\bf 2.} {\it Chez Descartes} : « natures simples », essences* dont nous
avons des idées claires et distinctes et auxquelles se ramènent tous les
êtres (cf. {\it Textes choisis}, I, p. 180).

— {\bf B)} \si{La nature} : {\bf 3.} Ensemble de tout ce qui existe,
considéré comme un tout soumis à des lois : « La {\it nature} des philosophes
païens est une chimère ; ... à ppt. parler, ce qu’on appelle nature n’est
rien autre chose que les lois générales que Dieu a établies pour construire
ou conserver son ouvrage » (Malebranche, {\it Tr. de la Nature et de la
Grâce}).$->$ Qqfs. personnifiée : « Ceux qui parlent de médecine font souvent
de la nature une espèce d'être moral$^4$ qui a des volontés » (Condorcet) ;
« La nature a dû hésiter entre deux modes d'activité psychique [l’instinct$^2$
et l'intelligence$^3$] » (Bergson, {\it E. C.}, 11) ; « La nature a
probablement voulu que la femme concentrât sur l'enfant le meilleur de sa
sensibilité » (id., {\it Deux Sources}, I) ; « La nature se préoccupe de la
société plutôt que de l'individu » ({\it ibid.}, II).

\ib{Nature (État de)} — \si{Pol.} (Opp. {\it civil}$^1$). État de l'humanité
antérieur à la vie sociale. $->$ Rousseau ({\it Inégalité}) le présente comme
une simple fiction : il n'a « peut-être point existé » et « probablement
n’existera jamais ».

\ib{Naturel} — {\bf 1.} Qui se rapporte ou qui est conforme à la nature*
(dans tous les sens de ce mot). {\it Sciences naturelles} : les sciences
biologiques (dist. {\it sciences de la nature} qui comprennent en outre la
physique, la chimie, la géologie, etc.). — {\bf 2.} (Opp. :
{\it positif}$^6$). Qui relève uniquement de la nature$^1$ : « Droit
naturel », cf. {\it Droit}$^2$; « Religion naturelle », celle qui relèverait
uniquement de la raison$^5$ et serait indépendante de toute révélation (cf.
{\it Déisme}*).

— {\bf 3.} \si{Math.} {\it Nombre naturel} : le nombre entier.

\ib{Naturisme} — \si{Soc.} \fsb{S. norma.} {\bf 1.} Théorie selon laquelle la
religion a pour origine la divinisation des forces naturelles. — \fsb{S.
posit.} {\bf 2.} Culte de ces forces.

— \si{Méd.} {\bf 3.} Système d'hygiène préconisant une vie plus proche de
la nature.

\ib{Néant} — \si{Méta.} {\bf 1.} Le non-être ; ce qui n'est pas : « Je suis
comme un milieu
% 125
entre Dieu et le néant, {\it i. e.} placé entre le souverain être et le
non-être » (Descartes, {\it Méd.}, IV ; cf. {\it Défaut}*) ; « L'idée du
néant absolu est une pseudo-idée » (Bergson, {\it E. C.}, IV) ; « Le néant ne
peut être donné : il n’est pour nous que la pensée de l'être
raturé » (Lavelle). —  {\bf 2.} Ce qui est de valeur nulle : « L'âme commence
à considérer comme un néant$^2$ ce qui doit retourner dans le
néant$^1$ » (Pascal) ; « L'homme n’est qu'un pur néant par
lui-même » (Malebranche).

— {\bf 3.} {\it Chez les existentialistes} : limitation de l'être, origine de
la négation : « Le Néant n'est pas l'opposé indéterminé à l'égard de
l'existant, il se dévoile comme composant l'être de cet
existant » (Heidegger) ; « Le néant est donné au sein même de l'être, en son
cœur, comme un ver » (Sartre).

\ib{Néantisation} — \si{Méta.} {\it Chez Sartre} : acte par lequel la
conscience se libère de l’{\it en-soi}$^5$ en le pensant : « Le pour-soi
surgit comme néantisation de l'en-soi. »

\ib{Nécessaire} (Ctr. : {\it contingent}). — \si{Épist.} et \si{Méta.} Qui ne
peut ni être autrement, ni ne pas être. {\it Dist.} : {\bf 1.} Ce qui est
{\it logiquement nécessaire} (cf. {\it Apodictique}*), ce que nous ne pouvons
pas concevoir autrement (nécessité {\it rationnelle} ou {\it de droit}) ;
cette nécessité peut être elle-même {\it a) absolue et inconditionnelle} (cf.
{\it Catégorique}*) : « Vérités nécessaires », propositions dont les
contradictoires$^1$ impliqueraient contradiction ou seraient connues comme
fausses a priori*; « L'Être nécessaire », Dieu, parce que sa non-existence
serait contradictoire : « Je sens que je puis n’avoir point été, ... donc je
ne suis pas un être nécessaire » (Pascal, 469) ; — {\it b) relative et
conditionnelle}
% 125
(cf. {\it Hypothétique}*) : {\it p. e.} dans une déduction*, la conclusion
est dite « nécessaire », {\it i. e.} il y aurait contradiction à la nier si
l’on accepte les principes ; — {\bf 2.} ce qui est {\it physiquement}$^1$
nécessaire, ce qui s'impose à nous a posteriori* (nécessité
{\it expériencielle} ou {\it de fait}) : « Les lois de l'équilibre et du
mouvement, telles que l'observation les fait connaître, sont de vérité
nécessaire » (D’Alembert). — Voir {\it Physique}$^1$.

\ib{Négatif} — {\bf 1.} Qui implique une négation* : « Le {\it négatif}
possède une puissance ; c’est elle que nous avons reconnue en morale en
insistant sur l'importance de l’obstacle*, du sacrifice » (Le Senne). —
\si{Math.} {\bf 2.} {\it Nombre négatif} : nombre affecté du signe $-$.

\ib{Négation} — \si{Log.} (Ctr. : {\it affirmation}). \fsb{S. concr.}
{\bf 1.} Acte de nier* ou produit de cet acte : « Deux négations valent une
affirmation » — \fsb{S. abstr.} {\bf 2.} Signe de la négation$^1$.

— \si{Méta.} {\bf 3.} Refus d'existence : « Il faut que la négation soit comme
une invention libre, qu’elle nous arrache à ce mur de positivité qui nous
enserre » (Sartre).

\ib{Négatité} — \si{Méta.} {\it Chez Sartre} : « type particulier de réalité
» qui implique une négation, une absence, une altération, un regret, etc.

\ib{Négativisme} — \si{Ps. path.} {\bf 1.} {\it Str.} Hyperactivité des
muscles antagonistes qui empêchent le malade d'exécuter les mouvements
commandés ou lui font prendre l’attitude opposée. — {\bf 2.} {\it Lato.}
Attitude d'opposition volontaire systématique, {\it not.} dans
l’{\it hébéphrénie}*.

\ib{Négativité} — \si{Méta.} {\it Chez Hegel} : activité de la négation$^1$
comme moment dialectique$^5$.
% 126

\ib{Néoréalisme} — Hist, \fsb{S. norma.} Forme moderne du réalisme$^4$
(surtout chez les Anglo-Saxons : Russell, Alexander, Holt, etc.), dont
l'affirmation essentielle est que l'{\it objet}$^4$ de la connaissance n’est
pas altéré par sa relation avec le {\it sujet}$^4$ connaissant (cf.
{\it Précis}, Ph. II, p. 453, et {\it Textes choisis}, II, p 317).

\ib{Neutres (États)} — \si{Psycho.} États qui
ne seraient à aucun degré ni agréables ni désagréables.

\ib{Névrose} — \si{Ps. path.} Maladie (telle que : neurasthénie,
psychasthénie, hystérie) comportant des troubles des diverses fonctions
psychiques, « caractérisés par l'arrêt du développement sans détérioration de
la fonction elle-même » (Janet). $->$ {\it Dist.} {\it démence}* (impliquant
au ctr. des détériorations des fonctions anciennes) et {\it psychose}*.

\ib{Nier} — \si{Psycho.} Poser un rapport comme faux, ou dire qu’une
existence n’est pas réelle : « L'homme est toujours disposé à nier ce qui lui
est incompréhensible » (Pascal).

\ib{Nihilisme} — [L. {\it nihil}, rien] — \fsb{S. norma.} Doctrine qui nie :
{\bf 1.} \si{Crit.} l’existence de la vérité : « Le nihilisme, dans la
théorie de la connaissance, nie toute vérité générale fixe » (Eisler) ; — 
{\bf 2.} \si{Mor.} la consistance des valeurs : « ... Un nihilisme de la
valeur d’après lequel la valeur ne serait qu’un mirage, un {\it manqué} » (Le
Senne) [cf. {\it Précis}, Ph. II, p. 359] ; — {\bf 3.} \si{Pol.} (syn. :
{\it anarchisme}*) la nécessité de l'État ou d’une organisation sociale
quelconque ({\it p. e.} chez certains intellectuels russes de la seconde
moitié du {\footnotesize XIX}$^\text{e}$ s.).

\ib{Niveau mental} — Voir {\it Tension}*.

\ib{Nocturne} — Qqfs employé {\it auj.} comme synonyme d’{\it inconscient}*
et d’{\it irra\-tionnel}* : « Le côté nocturne de l’âme humaine ».

\ib{Noématique} — \si{Crit.} Qui concerne le noème : « Le sens noématique. »

\ib{Noème, Noèse} — \si{Crit.} {\it Chez Husserl} : le noème [G. {\it noêma},
objet pensé] est l’objet intentionnel$^3$ avec son {\it sens}, son caractère
de {\it réalité}, ses modes d’{\it apparaître}, etc. : « La perception a son
noème, à savoir le {\it perçu comme tel} », tandis que la noèse [G.
{\it noêsis}, pensée] est l’acte même de la connaissance, tourné vers
l'objet : « La réalité psychique concrète sera nommée {\it noèse} ; et le
sens qui vient l'habiter, noème » (Sartre).

\ib{Noétique} — \si{Crit.} {\bf 1.} {\it Chez Husserl} : qui concerne la
noèse*. Est noétique ce qui n’est ni sensible ni empirique, mais est saisi
par l'intuition$^4$ pure.

— \si{Ps. path.} {\bf 2.} {\it Théories noétiques du langage} : celles qui
rattachent l’aphasie à un trouble intellectuel général (opp. : {\it théories
antinoétiques}, qui n’y voient qu'un trouble sensori-moteur).

\ib{Nolonté} — \si{Psycho.} Résistance de la volonté à une impulsion.

\ib{Nombre} — Une des notions fondamentales de l’entendement : « L'idée de
nombre implique l'intuition simple d’une multiplicité de parties ou d’unités,
absolument semblables les unes aux autres » (Bergson, {\it D. I.}, I).
{\it Selon Kant}, le nombre est le {\it schème}$^5$ de la catégorie de
quantité*. — {\it Loi des grands nombres} : les résultats du calcul de la
probabilité$^2$ mathématique s'accordent avec l'expérience pourvu qu'on opère
sur un nombre de cas suffisamment grand.

\ib{Nominale (Définition)} — \si{Log.} Celle qui consiste à définir un mot
par convention (opp. : définition {\it réelle}$^2$)

\ib{Nominalisme} — \fsb{S. norma.} \si{Méta.}, \si{Psycho.} {\bf 1.} ({\it Opp.} : {\it
réalisme}$^3$ et {\it conceptualisme}). Théorie selon laquelle l’idée générale
% 127
« n’est qu’un nom » (Condillac, Taine) et n'a aucune réalité, ni dans
l'esprit, ni hors de l'esprit.

— \si{Épist.} {\bf 2.} {\it Nominalisme scientifique} : conception de la
science d'après laquelle celle-ci se réduit à un langage commode et à des
règles purement conventionnelles : « Nominaliste de doctrine, mais réaliste
de cœur, il [Le Roy] semble n'échapper au nominalisme que par un acte de foi
désespéré » (Poincaré).

\ib{Nomographie} — \si{Épist.} Ensemble des méthodes qui consistent à
représenter les lois scientifiques par des graphiques ou {\it abaques}$^2$.

\ib{Non-euclidien} — \si{Épist.} Qui n’admet pas les postulats de l’espace
euclidien* : « Les géométries non-euclidiennes » (Poincaré).

\ib{Non-moi} — \si{Méta.} L'objet$^5$ ou le
monde extérieur$^2$ en tant que distinct du sujet$^4$.

\ib{Noocentrisme} — \si{Crit.} \fsb{S. norma.} Doctrine ou attitude
gnoséologique* « qui fait graviter l’être autour du connaître » (Blanché).

\ib{Noologiques (Sciences)} — \si{Épist.} {\it Chez
Ampère} : les sciences de l'esprit.

\ib{Noosphère} — \si{Biol.} et \si{Méta.} « C'est vraiment une nappe*
nouvelle, la « nappe pensante » qui, après avoir germé au tertiaire
finissant, s'étale depuis lors par-dessus le monde des plantes et des
animaux : hors et au-dessus de la Biosphère*, une Noosphère » (Teilhard de
Chardin).

\ib{Normal} — \si{Biol.}, \si{Soc.}, \si{Psycho.} (Ctr. :
{\it pathologique}$^2$). Conforme à un type donné, donc qui se rencontre dans
la généralité des cas : « Nous appellerons normaux les faits qui présentent
les formes les plus générales, et nous donnerons aux autres le
% 127
nom de morbides ou de pathologiques$^2$ » (Durkheim).

\ib{Normatif} — \si{Épist.} (Opp. : {\it constatif}*) Qui concerne ou énonce
des normes*: « Un principe normatif ». {\it Sciences normatives} : nom donné
par Wundt (cf. {\it Textes choisis}, II, p. 136) à l'Esthétique$^2$, à la
Logique$^2$ et à l'Éthique* ou Morale$^2$ : « Ce qui caractérise les sciences
normatives, ce n’est pas l'idée de fin..., c’est plutôt qu’elles ont à
établir des jugements de valeur$^2$ » (Goblot).

\ib{Norme} — [{\it L.} norma, règle] — \si{Épist.} \fsb{S. concr.} {\bf 1.}
Type idéal$^2$ : « Ce que nous sommes n’est pas la norme de ce que nous
devons être » (Vinet) ; « L'idéal$^3$ est un principe fait norme » (Le Senne).
— \fsb{S. abstr.} {\bf 2.} Formule exprimant l'idéal sous forme de jugement
de valeur : « Les normes morales. »

\ib{Notion} — \si{Vulg.} {\bf 1.} Connaissance, discernement : « Quelle
notion Précise peut-on avoir du bien et du mal, du beau et du laid, ... sans
une notion préliminaire de l’homme ? » (Diderot) ; « Avoir des notions de
mathématiques. »

— \si{Psycho.} et \si{Épist.} {\bf 2.} Idée$^3$, concept$^1$ : « De pures
sensations ne sont pas des notions » (Bonnet) ; « En mathématiques, les
notions ont cet avantage qu'ayant été une fois déterminées$^1$, elles ne
varient plus » (Condillac).

\ib{Nouménal} — {\it Chez Kant} : qui se rapporte au noumène* : « Moi
nouménal »: voir {\it Moi}$^5$ ; « Volonté nouménale » : celle du moi
nouménal; qui émane, non du caractère empirique, mais du caractère
intelligible$^1$.

\ib{Noumène} — [G. {\it noumenon}, intelligible$^1$] — \si{Crit.} {\it Chez
Kant} (opp. : {\it phé\-nomène}$^2$) : réalité intelligible$^1$, chose
%
en$^3$ soi : « Si j’admets des choses qui soient purs objets de l’entendement
et puissent pourtant être données à une intuition sans pouvoir l'être à
l'intuition sensible, il faudra nommer ces choses des noumènes » ({\it R.
pure}, Analyt., II, 3). D'où : {\it a)} au sens {\it négatif} : « Le concept
d’un noumène est donc un pur concept limitalif qui a pour but de restreindre
les prétentions de la sensibilité » (ibid.) ; — {\it b)} au sens
{\it positif} : « Entendons-nous par là l’objet d’une intuition non
sensible, ... intellectuelle, que toutefois nous ne possédons pas ? ce serait
là le noumène au sens positif » ({\it ibid.}, 2$^\text{e}$ éd.).

\ib{Numineux} — [L. {\it numen}, divinité] — \si{Méta.} {\it Chez R. Otto} :
élément « ineffable » du sacré consistant dans le sentiment d’une réalité qui
nous dépasse et d’un « mystère redoutable » ({\it mysterium tremendum}).

\begin{center}
\huge{O}
\end{center}

\ib{O} — \si{Log.} \si{form.} Désigne les propositions particulières*
négatives : {\it p. e.} « Certains hommes ne sont pas instruits. »

\ib{Objectal} — \si{Ps. an.} Voir {\it Libido}$^2$.

\ib{Objectif} — \si{Crit.} et \si{Méta.} {\bf 1.} {\it Autref.} {\it not.}
{\it chez Descartes} : qui existe dans l’esprit en tant que représenté (voir
{\it Objet}$^4$, et cf. {\it Eminent}* et {\it Idée}$^6$) : « Une chose est
objectivement ou par représentation dans l’entendement par son idée$^4$
» ({\it Méd.}, III) ; « Afin qu’une idée contienne une telle réalité
objective, elle doit avoir cela de qq. cause dans laquelle il se rencontre
pour le moins autant de réalité formelle$^1$ que cette idée contient
% 128
de réalité objective » ({\it ibid.}). Ce sens a été repris par Renouvier : «
J’appellerai {\it objectif} ce qui s’offre comme objet, {\it i. e.} qui vient
représentativement dans la connaissance » ({\it Traité de Log. gén.}, III).
—  {\bf 2.} {\it Auj.} (depuis Kant, — opp. : {\it subjectif}$^1$). Qui
existe hors de l'esprit et indépendamment de la connaissance qu’en a le
sujet$^4$ pensant (voir {\it Objet}$^5$) : « L’espace n’est pas qqc.
d'objectif ou de réel, mais de subjectif$^1$ et d’idéal$^1$ » (Kant). D'où :
extérieur à la conscience : « La pesanteur est une réalité objective » $->$
Bien {\it dist.} ces 2 sens dont la confusion entraînerait de graves erreurs.

— \si{Épist.} {\bf 3.} (Ctr. {\it subjectif}$^2$). {\it Laud.} Fondé sur une
observation impartiale, indépendante des préférences individuelles de
l’auteur : « L'objectif, c’est ce qui est impersonnel » (Hamelin) ; « Un
compte rendu très objectif ». — {\bf 4.} Fondé sur l'étude de phénomènes
objectifs$^2$ : « La méthode objective en psychologie. »

— \si{Méta.} {\bf 5.} {\it Chez Hegel} : « Esprit
objectif », voir {\it Esprit}$^5$.

\ib{Objectivation} — \si{Psycho.} {\bf 1.} Processus par lequel une
sensation, supposée d’abord purement subjective ({\it p. e.} chez Taine), se
trouverait projetée au-dehors et extériorisée dans l’objet$^5$.

— \si{Épist.} {\bf 2.} Processus par lequel la connaissance s'approche sans
cesse de l’objectivité$^1$ : « Le critère cherché s’appliquera moins à une
qualité statique qu’à une {\it vection}, non tant à l’objectivité qu’à
l’objectivation » (Blanché).

\ib{Objectivisme} — \si{Épist.} \fsb{S. posit.} Tendance à privilégier, dans
la connaissance, le point de vue de l’objet en négligeant l’apport du sujet.
% 129

\ib{Objectivité} — \si{Épist.} Qualité qui consiste à être objectif$^3$,
soit : {\bf 1.} en parlant de la connaissance : « L'objectivité qui se
définit, en {\it opp.} avec l'empiricité, toujours sujette à caution, est la
probité de l’objet » (Le Senne) ; « Le meilleur indice de l’objectivité d’une
connaissance réside, pour le savant, dans la convergence des résultats
obtenus par des méthodes diflérentes » (Blanché) ; —  {\bf 2.} en parlant du
sujet$^4$ connaissant : « L’objectivité de l'historien ».

\ib{Objet} — [L. {\it ob-jectum}, ce qui est placé devant] — \si{Vulg.}
{\bf 1.} Ce qui s'offre à la vue, chose perçue : « Quel objet pour les yeux
d’une amante ! » (Racine) ; « Les images des objets se forment au fond de
l'œil » (Descartes). — {\bf 2.} Cause d’un sentiment : « Objet de crainte ».
{\it Spéc.}, au {\footnotesize XVII}$^\text{e}$ s., l'être aimé : « Ce cher
objet à qui j'ai pu déplaire » (Corneille). —  {\bf 3.} Matière$^1$ ou but :
« L'objet d’une science » ; « Ame qui étais née pour un objet
immortel » (Bossuet) ; « La passion a toujours un objet » (Bonnet) ;
« L'objet du mariage est d’avoir des enfants » (Buffon).

— \si{Crit.} et \si{Méta.} [Ce qui est placé devant l'esprit; d'où :]. {\bf
4.} {\it Autref.} (cf. {\it Objectif}$^1$), ce qui est pensé, représenté dans
l'esprit : « J’appelle {\it objet} ce qui dans la représentation s'offre
comme le terme immédiat du connaître : le représenté » (Renouvier). — {\bf
5.} {\it Auj.} (opp. : {\it Sujet}$^4$; cf. {\it Objectif}$^2$), la réalité
extérieure qui est pensée : « J’ai souvent remarqué en beaucoup d'exemples
qu'il y avait une grande différence entre l’objet et son idée » (Descartes,
{\it Méd.}, III) ; « Sous le titre d'objet, on peut renfermer tout ce que
l'être pensant perçoit, comme actuellement distinct du sentiment de son
existence individuelle » (Biran) ; « Il
% Cuvillier. — Vocabulaire philosophique.
% 129
y a trois conceptions possibles de l’objet : l'objet est la chose en soi
[solution que l’auteur rejette], ou c’est l’accord des idées entre elles, ou
c’est la liaison nécessaire par {\it opp.} à des liaisons
contingentes » (Hamelin).

\ib{Obligation} — {\bf 1.} \si{Mor.} \fsb{S. abstr.} Caractère impératif qui
constitue la forme$^2$ de la loi$^3$ morale : « Le caractère spécifique de
l'obligation est de faire, en qq. mesure, violence au désir » (Durkheim). $->$
{\it Dist.} contrainte ou nécessité : le devoir$^6$ est obligatoire, mais
nous laisse libres. — {\bf 2.} \fsb{S. concr.} \si{Mor.} Prescription qui
constitue la matière$^3$ de la loi$^3$ morale : « Les obligations des riches
envers les pauvres. »

— \si{Jur.} {\bf 3.} \fsb{S. abstr.} et \fsb{S. concr.} Prestations
auxquelles nous sommes tenus de par la loi$^1$ juridique : « Von Brinz a
distingué daus toute obligation deux éléments : le {\it debitum} (Schuld) et
l’{\it obligatio} au sens étroit (Haftung). Le {\it debitum} est l'objet
propre de l'obligation, ce que doit fournir le débiteur. L'{\it obligatio}
est le lien de droit qui force le débiteur à fournir sa prestation » (H.
Lévy-Bruhl) ; « Les obligations contractuelles. »

\ib{Obscur} — Ctr. de {\it clair}* (v. ce mot).

\ib{Observation} — \fsb{S. abstr.} \si{Épist.} {\bf 1.} {\it Lato.}
Constatation attentive; investigation : « L'observation est ce qui montre les
faits » (Cl. Bernard). — {\bf 2.} {\it Str.} (Opp. : {\it expérimentation}*).
Constatation attentive des faits tels qu'ils se présentent naturellement :
« Sciences d'observation », sciences de faits, qui n'ont pas recours à
l'expérimentation ({\it p. e.} astronomie). — \fsb{S. concr.} {\bf 3.}
Résultat de l'observation$^1$ : « Dans un sens concret, on a donné le nom
d'{\it observations} aux faits constatés, et c’est dans ce sens que l’on dit :
% 130
observations médicales, observations astronomiques » (Cl. Bernard).

\ib{Obsession} — \si{Ps. path.} Image ou idée qui s'impose à l'esprit du
sujet$^5$. Selon Janet, elle se dist. de l’{\it idée fixe} en ce qu’elle ne
passe gén. pas à l’action et en ce que le sujet a conscience de son caractère
anormal.

\ib{Obstacle} — Ce qui s'oppose à la pensée ou à l’action : « Les obstacles
font partie de la science » (Biran) ; « Les obstacles épistémologiques
» (Bachelard; cf. {\it Précis}, Ph. II, p. 65; Sc. et M., p. 58) ;
« L’obstacle fêle le je* [= le dédouble en sujet et objet] » (Le Senne).
{\it Cf.} {\it Négatif}$^1$ et {\it Situation}$^2$.

\ib{Occasionalisme} — \si{Hist.} \fsb{S. norma.} Doctrine des causes
occasionnelles* (chez Geulinex, Malebranche, etc.).

\ib{Occasionnelle (Cause)} — \si{Méta.} Celle qui n’est que l’occasion à
propos de laquelle Dieu, seule cause efficace*, produit l'effet : « Dieu
établit leurs modalités [des créatures] causes occasionnelles des effets
qu’il produit lui-même : causes occasionnelles qui déterminent l’efficace de
ses volontés, en conséquence des lois générales qu'il s’est
prescrites » (Malebranche, {\it Entr.}, VII, 10).

\ib{Occulte} — Caché, mystérieux, dont les causes sont inconnues : « Les
forces occultes » (cf. {\it Qualités}$^2$) ; « Sciences occultes », celles
qui prétendent connaître les forces occultes (magie, astrologie, spiritisme,
etc.).

\ib{Œdipe (Complexe d’)} — \si{Ps. an.} {\it Chez Freud} : attachement de
l’enfant au parent de sexe opposé, refoulé par suite du conflit avec le
parent de même sexe qu'il aime et craint à la fois.

\ib{Œkoumène} — [G. {\it oïkouménè}, habitée] — \si{Épist.} En géographie
humaine : la Terre en tant qu'habitée par l’homme : « Ce mot recouvre deux
éléments associés : l’idée d’un espace terrestre avec ses limites, l’idée
d’une occupation par l’homme » (Sorre).

\ib{Olfactif} — \si{Ps. phol.} Qui se rapporte à l’odorat.

\ib{Oligarchie} — \si{Pol.} Système politique où le pouvoir, exercé par un
petit nombre de personnes, est exploité par elles dans leur intérêt.

\ib{On} — [{\it Trad.} all. : {\it man}] — \si{Méta.} {\it Chez Heidegger} :
le {\it Dasein}$^2$ engagé par l’existence-en-commun dans l’ensemble des
circonstances extérieures : « Le Soi de la banalité quotidienne, c’est le
{\it On} se constituant dans et par les interprétations qui ont cours
publiquement ». Cf. {\it Authentique}$^2$.

\ib{Onirique} — [G. {\it onar}, rêve] — \si{Psycho.} {\bf 1.} Qui concerne le
rêve$^1$ ou lui ressemble : « Image onirique »; « Pensée onirique ». —
\si{Ps. path.} {\bf 2.} {\it Délire onirique} : sorte de rêve$^1$ ({\it gén.}
pénible) que le malade vit tout éveillé avec hallucinations généralisées de
tous les sens.

\ib{Ontique} — [G. {\it ôn, ontos,} étant] — \si{Méta.} {\it Chez
Heidegger} : qui concerne l’existence quotidienne du {\it Dasein}$^2$ ;
existentiel (opp. : {\it ontologique}*).

\ib{Ontogénie} — \si{Biol.} (Opp. : {\it phylogénie}*). Evolution biologique
de l’être individuel.

\ib{Ontologie} — \si{Méta.} Autre nom de la métaphysique$^1$ comme étudiant «
l'être en tant qu'être », {\it i. e.} indépendamment de ses déterminations
particulières : « Les êtres ayant qqs. propriétés générales comme l’existence,
% 131
la possibilité, la durée, l’examen de ces propriétés forme cette branche de
la philosophie qu’on nomme l’{\it ontologie} ou science de l'être ou
métaphysique générale » (D’Alembert).

\ib{Ontologique} — \si{Méta.} Qui concerne l'être en général. {\it Spéc.},
{\it chez Heidegger} (opp. à {\it ontique}) : existential*. — {\it Preuve
ontologique} : argument inventé par saint Anselme et selon lequel l’existence
est comprise dans l'idée de Dieu en tant qu'Être parfait$^3$ « en même façon
qu'il est compris en celle d’un triangle que ses trois angles sont égaux à
deux droits » (Descartes, {\it Méth.}, IV).

\ib{Opaque} — Non perméable à la pensée. {\it Spéc., chez Sartre}, «
l’opacité de l’être-en-soi » se définit par le principe d'identité qui pose
que « l’en-soi est ce qu’il est », donc hétérogène à ce qui n’est pas lui.

\ib{Opérationalisme} — \fsb{S. norma.} \si{Épist.} Forme renouvelée du
pragmatisme* selon laquelle les concepts doivent se définir en termes
d'opérations physiques. Cf. {\it Opérationnel}*.

\ib{Opérationnel, Opératoire} — \si{Épist.} Qui sert à effectuer des
opérations logiques : « Un concept opérationnel » ; « Le caractère essentiel
de la pensée logique est d’être opératoire, {\it i. e.} de prolonger l'action
en l’intériorisant » (Piaget). Cf. {\it Schème}$^4$.

\ib{Opinion} — \si{Psycho.} et \si{Crit.} {\bf 1.} \fsb{S. subje.}
Assentiment* partiel ; croyance, au sens 1 : « Quand le pénitent suit une
opinion probable, le confesseur le doit absoudre » (Pascal, {\it Prov.}, 5).
Spéc., {\it chez Platon} (grec {\it doxa}) : type de connaissance inférieur à
la science et à la pensée discursive et qui comprend la croyance ({\it
pistis}) et la pensée par images ({\it eïkasia}) : « Ce qu'est l'être au
devenir, ainsi est la connaissance intellectuelle ({\it noêsis}) à l'opinion
» ({\it République}, VI).

— \si{Psycho.} et \si{Soc.} \fsb{S. abstr.} {\bf 2.} Type de
% 131
pensée sociale qui consiste à prendre position, plus ou moins fermement, sur
les problèmes politiques, moraux, philosophiques, religieux : « L'opinion
fait des hommes ce qu'elle veut » (Lacombe) ; « Les valeurs sont choses
d'opinion » (Durkheim) ; « Il existe deux formes de l'opinion, l'opinion
publique et l'opinion privée. La première est d'ordre sociologique: ... la
seconde, d'ordre psychologique », toutefois même celle-ci « répond à une
question sociale, est elle-même une réponse sociale » (Stœtzel). Cf. {\it
Public}$^2$. — \fsb{S. concr.} {\bf 3.} Objet de l'opinion$^2$ : « L'opinion
est un groupe plus ou moins logique de jugements qui, répondant à des
problèmes actuellement posés, se trouvent reproduits en nombreux exemplaires
dans des personnes du même pays, du même temps, de la même société
» (Tarde) ; « Ainsi se vont les opinions, succédant du pour au contre
» (Pascal, 337) ; « Tout le mécanisme social repose sur des opinions
» (Comte, {\it Cours}, I).

\ib{Opposition} — \si{Log.} \si{form.} Mode de déduction* immédiate
consistant à conclure d’une proposition principe à la proposition {\it
opposée}, {\it i. e.} ayant même sujet et même attribut, mais différant de la
première, soit par la quantité$^2$, soit par la qualité$^3$, soit par les deux à
la fois. {\it Cf.} {\it Contradiction}$^1$, {\it Contrariété}*, {\it
Subcontraires}* et {\it Subalternes}*.

\ib{Optatif} — Mode grammatical qui exprime un souhait (puissé-je faire...) :
« La morale concrète emploiera l'indicatif plutôt que l’optatif » (Gusdorf).

\ib{Option} — \si{Mor.} et \si{Pol.} Choix$^1$ ; prise de position : «
L'option se dévoile : être des hommes ou être des riches » (Perroux).
% 132

\ib{Optimisme} — \si{Vulg.} \fsb{S. posit.} \fsb{S. subje.} {\bf 1.}
Disposition à voir le bon côté des choses, {\it ou} attitude morale de
confiance dans la vie, « croyance réfléchie au bien », {\it opp.} à cet «
optimisme tranquille et satisfait pour qui le mal n’est rien que l'ombre qui
fait ressortir la lumière » (Boutroux).

— \si{Méta.} \fsb{S. norma.} {\bf 2.} Doctrine selon laquelle, dans
l'univers, le bien (en tous les sens du terme) l’emporte sur le mal.
{\it Spéc.}, doctrine de Leibniz selon laquelle ce monde est, « entre tous
les mondes possibles, le meilleur de tous » ({\it Théod.}, 416).

\ib{Ordre} — \si{Épist.} {\bf 1.} (Sens général). Une des idées fondamentales
de l’entendement : elle implique une disposition satisfaisante pour la raison
et qui peut être logique ou spatiale ou temporelle, selon la causalité ou la
finalité, en série numérique, etc : « Le rapport de la raison et de l’ordre
est extrême » (Bossuet) ; « L'idée de la forme se confond avec l'idée de
l’ordre » (Cournot). — {\it D'où} : {\bf 2.} (Dans l’espace). « La géométrie
de position ne retient que l’ordre de distribution » (Poincaré). — {\bf 3.}
(Dans le temps). « L'histoire suit l’ordre chronologique ». — {\bf 4.} (Ordre
logique). « L'ordre consiste en cela seulement que les choses qui sont
proposées les premières doivent être connues sans l’aide des suivantes
» (Descartes, 2$^\text{e}$ {\it Rép.}): « Cet ordre, le plus parfait entre
les hommes [celui de la démonstration géométrique]. » (Pascal) ; « Le cœur a
son ordre ; l'esprit a le sien, qui est par principe et démonstration » (id.,
283).

— (Sens spéciaux). \si{Math.} {\bf 5.} Degré : « Courbe {\it ou} infiniment
petit du second ordre ». — \si{Phys.} {\bf 6.} (Causalité ou légalité). «
L'ordre de la nature est constant et universel » [formule du déterminisme$^2$
d’après
% 132
Goblot]. — {\bf 7.} (Finalité). « Un principe d'ordre qui veille, pour ainsi
dire, au maintien des espèces chimiques et biologiques » (Lachelier). —
\si{Biol.} {\bf 8.} Groupe morphologique intermédiaire entre la classe$^2$ et
la famille$^2$ : « L'ordre des carnassiers ». — \si{Soc.} {\bf 9.} Stabilité
sociale : « L'étude statique$^4$ de l’organisme social doit coïncider avec la
théorie positive de l'ordre » (Comte, {\it Cours}, 48$^\text{e}$ leçon) ;
« L'ordre public. »

— \si{Méta.} {\bf 10.} {\it Chez Pascal} : les « trois ordres », règnes de la
nature matérielle et spirituelle et du surnaturel : « La distance infinie des
corps aux esprits figure la distance infiniment plus infinie des esprits à la
charité. Ce sont trois ordres différant de genre. » ({\it Pensées}, 793). —
{\bf 11.} {\it Chez Malebranche} : hiérarchie des perfections : « L'Ordre
immuable ne consiste que dans les rapports de perfection. L'amour de l'Ordre
est l'unique vertu » (\si{Mor.}, I et 111). — {\bf 12.} {\it Chez Leibniz} :
harmonie de l'univers : « Dieu ne fait rien hors de l’ordre... Les miracles
sont dans l’ordre » ({\it Disc. méta.}, VI et VII). — {\bf 13.} {\it Chez
Bergson} « L'ordre existe, c'est un fait » : « l’ordre physique » est
automatique: « l'ordre vital » est « analogue à l’ordre {\it voulu} », et
l’idée de désordre est illusoire ({\it E. C.}, III).

\ib{Organicisme} — \fsb{S. norma.} \si{Biol.} {\bf 1.} (Opp. :
{\it vitalisme}*). Doctrine selon laquelle la vie totale d’un organisme
résulte de la composition des forces particulières des différents organes. —
\si{Soc.} {\bf 2.} Doctrine selon laquelle, la société étant un grand
organisme, les lois biologiques s'appliquent aux phénomènes sociaux aussi
bien qu’à la vie organique.

\ib{Organique} — \si{Biol.} {\bf 1.} {\it Vie organique} : syn. :
{\it végétative}*. — {\bf 2.} Qui concerne
%133
les organismes vivants ou leurs fonctions. {\it Chimie organique} : celle qui
a pour objet « l’étude chimique des matières contenues dans les êtres
vivants » (M. Berthelot).

— \si{Psycho.} {\bf 3.} {\it Sensations organiques} (syn. :
{\it cénesthésie}) : celles qui sont relatives à la vie organique$^1$.

— {\it Ext.} {\bf 4.} Analogue à ce qui existe dans les organismes vivants,
{\it i. e.} impliquant à la fois pluralité et unité, différenciation et
solidarité : {\it p. e.} \si{Soc.} {\it Solidarité organique} : voir
{\it Solidarité}*.

\ib{Organisé} — {\it Cf.} {\it Diffus}*.

\ib{Orgueil} — \si{Mor.} {\bf 1.} Sentiment exagéré de sa propre valeur qui
pousse l'individu à se préférer aux autres et à les dédaigner : « L’orgueil a
deux effets, dont l’un est de vouloir en tout exceller au-dessus des autres ;
l’autre, de s’attribuer à soi-même sa propre excellence » (Bossuet), $->$
{\it Dist.} {\it vanité}*. —  {\bf 2.} Qqfs {\it laud.} Sentiment légitime de
la dignité personnelle : « (Un guerrier) peut mettre l’orgueil même à
pardonner l'offense » (Voltaire).

\ib{Orientation} — \si{Ps. phol.} {\bf 1.} {\it Sens de l'orientation} :
complexus de sensations (du sens statique$^2$, de l’orientation lointaine,
etc.) qui permettent à certains animaux ({\it p. e.} oiseaux migrateurs) de
s'orienter dans l’espace.

— \si{Péd.} {\bf 2.} {\it Orientation professionnelle} : ensemble des
procédés par lesquels on guide les individus vers les professions pour
lesquelles ils ont le plus d’aptitude.

\ib{Orthogénèse} — \si{Biol.} Évolution qui s’accomplit, par développement de
l'être vivant dans une direction bien déterminée et surtout sous l'influence
de causes internes.

\ib{Ouvert} — \si{Méta.}, \si{Mor.}, etc. (Ctr. {\it clos}). Accessible à qqc. de
nouveau : « L'âme ouverte au bien que le Ciel lui envoie » (Corneille). $->$
Qualificatif mis à la mode par Bergson avec sa distinction de « l’âme close »
et de « l'âme ouverte », de la « morale close » (celle de l'obligation$^1$)
et de la « morale ouverte » (celle de l’aspiration), de la « société close »
et de la « société ouverte » : « La société close est celle dont les membres
se tiennent entre eux, toujours prêts à attaquer ou à se défendre... La
% 133
société ouverte est celle qui embrasserait en principe l'humanité entière
» ({\it Deux Sources}, IV). On dit auj. « conscience close », celle qui se
ferme à la communication*, {\it opp.} à « conscience ouverte », celle qui s’y
offre; — rationalisme ouvert » (Bachelard), {\it opp.} au rationalisme figé
dans les cadres de la raison$^2$ constituée.

	\end{itemize}
