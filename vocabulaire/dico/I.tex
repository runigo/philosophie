
	\begin{itemize}[leftmargin=1cm, label=\ding{32}, itemsep=1pt]

\ib{I} — \si{Log.} \si{form.} Désigne les proposi
tions particulières$^2$ affirmatives
{\it p. e.} « Certains hommes sont instruits. »

\ib{Idéal (adj.)} — {\bf 1.} (Syn. : idéel). Qui
n'existe que dans la pensée : « Les
mathématiques représentent les rapports des choses dans les conditions
d'une simplicité idéale » (Cl. Bernard)). — {\bf 2.} {\it Laud.} Parfait : « Ce bien
idéal que toute âme désire » (Lamartine).

\ib{Idéal (nom)} — {\bf 3.} Ce que l’on conçoit
comme un type parfait dans l’ordre
des valeurs esthétiques, morales, sociales, etc. : « Aller à l'idéal et comprendre le réel » (Jaurès); « L'idéal
ne peut être laloi abstraite del’action,
mais l’action même » (Lagneau). —
{\it {\it Cf.}} Médiation$^4$.

\ib{Idéalisme} — \si{Vulg.} {\bf 1.} À Attitude qui
consiste à subordonner ses actes et
ses pensées à un idéal$^3$ moral, intellectuel ou esthétique : « [Au sens
populaire], l’idéalisme consiste à
agir et penser conformément à
lidéal$^3$ » (Le Senne). — \si{Esth.} {\bf 2.} A
({\it Opp.} : réalisme$^2$ ou naturalisme$^3$).
Conception de l’art comme ayant
pour but, non l’imitation du réel,
mais l'expression d’un idéal$^3$ : « Pour
rester d'accord avec l'observation
psychologique, l’idéalisme n’a qu’à
appliquer sa doctrine l’art est
fabrication et produit spirituel »
(Delacroix).

— \si{Méta.} {\bf 3.} A. Idéalisme platonicien : la doctrine des Idées! :
« Dans ce sens, l’idéalisme est un
réalisme$^4$ de l’intelligible » (Le
Senne). — \si{Crit.} et \si{Méta.} {\bf 4.} À.
(Ctr. : réalisme$^4$). Nom générique
des divers systèmes modernes qui
ramènent l'être à la pensée, les
« choses » à l'esprit : « Pour l’idéaliste, il n’y a rien de plus dans la
% 90
réalité que ce qui apparaît à ma
conscience ou à la conscience en
général » (Bergson). Jdéalisme problématique : nom donné par Kant
à la thèse de Descartes que l'esprit
est la seule réalité immédiatement
certaine. {\it Id.} spiritualisie : {\it p. e.} celui
de Leibniz (v. Monade*). {\it Id.} immatérialiste* : celui de Berkeley. {\it Id.}
transcendantal (voir ce mot) : celui
qui « considère les phénomènes, tous
ensemble, comme de simples représentations » (Kant, {\it R. pure}, {\it Dial.}
II, 1). {\it Id.} subjectif : celui de Fichte,
où tout dérive du Moi* absolu se
posant lui-même. {\it Id.} objectif : nom
donné par Schelling à sa propre doctrine (v. Absolu$^1$). {\it Id.} absolu ou dialectique : celui de Hegel, qui dérive
tout de l’Idée$^1$, Zd. synthétique : celui
de Hamelin. Îd. critique : celui de
Brunschvicg (Sur ces deux derniers,
cf. Précis, Ph. II, p. 461-46 {\bf 2.} Pour
d’autres acceptions, cf. Textes choisis, II, p. 327-333).

$->$ Ces sens présentent cependant une certaine unité : 1° des
sens 1 et 2 au sens 4 : « Ce n’est pas
par hasard que le même mot idéalisme exprime la doctrine selon laquelle la raison se retrouve au cœur
même des choses [sens 4] et en même
temps l'effort pratique vers un idéal
de justice ou de beauté [sens 1 et 2] »
(Parodi) ; — 2° du sens 3 au sens 4 :
« L’idéalisme se compose de deux
phases : ... toute l'antiquité a travaillé pour établir ce premier point
que l’objet de la pensée consiste en
idées [sens 3]... Refuse-t-on d’ajouter que les idées n'existent pas en
elles-mêmes [sens 4), essaie-t-on
de les traiter comme des choses, on
régresse en deçà de la première
phase » (Hamelin).

\ib{Idéalité} — \si{Crit.} {\bf 1.} Caractère purement idéal$^1$ : « L’idéalité de l’espace ».

— {\bf 2.} {\it Spéc.}, chez Bradley : caractère
de la qualité (what) quand elle est
disjointe de l'être (that).

\ib{Idéation} — \si{Psycho.} Fonction mentale
qui consiste à former ou à saisir des
idées$^3$. $->$ {\it Qqfs.} dist. de la conception$^1$ (v. Textes choisis, I, p. 185).

\ib{Idée} — [G. idea, image, forme visible;
d'où : forme, espèce] — \si{Méta.} {\bf 1.}
a) Chez Platon : essence intelligible
et éternelle des choses sensibles :
« La théorie des Idées »; « [En art],
l’Idée est à la fois l'essence immobile
et éternelle et la forme créatrice
dans l’espace et dans le temps » (Delacroix); — b) Chez Hegel : « unité
de l'existence et du concept$^1$ », qui
est l’Étre en* soi, engendrant par
son développement dialectique$^5$ la
Nature, puis l'Esprit$^5$ : « Nous
réserverons l'expression Jdée au
concept objectif ou réel, et nous la
distinguerons du concept lui-même,
et plus encore de la simple représentation » (Grande Logique, II).

— \si{Crit.} {\bf 2.} Chez Kant : « idée de la
raison » ou « idée transcendantale » :
« concept nécessaire de la raison$^2$
auquel nul objet, qui lui corresponde, ne peut être donné dans les
sens » ({\it R. pure}, Dial, I, 2); les
« idées de la raison » sont l'idée de
l’âme ou du moi-substance, celle du
monde comme totalité des phénomènes, et celle de Dieu.

— \si{Crit.} et \si{Psycho.} {\bf 3.} {\it Str.} Concept$^1$, représentation abstraite* et
générale$^2$ : « Je n’appelle pas du nom
d’idées les seules images dépeintes en
la fantaisie*; au contraire je ne les
appelle point ici de ce nom » (Descartes, 2$^\text{es}$ {\it Rép.}) ; « Les idées ont une
existence éternelle et nécessaire »
(Malebranche, {\it Entr.}, I, 5); « Les
idées sont en Dieu de toute éternité » (Leibniz, N.-E., III, 4, 17);
« Par idée, j'entends un concept de
% 91
l'esprit, qu'il forme en tant qu'il est
chose pensante » (Spinoza, Éth., II,
déf. 3). Au sens empirique : « Je
crois que nous avons plus d'idées
que de mots » (Diderot); « L’évolution des idées générales » (Ribot). —
 {\bf 4.} {\it Lato.} {\it Autref.}, toute représentation, y compris même les images :
« Quelques-unes de mes pensées
sont comme les images des choses,
et c’est à celles-là seules que convient proprement le nom d'idées »
(Descartes, \si{{\it Méd.}}, III); « Tout objet
que l'esprit aperçoit immédiatement » (Locke); « Les choses perçues
immédiatement [par les sens] sont
des idées » (Berkeley). {\it Qqfs.}, {\it spéc.}
au sens d'image : « L'idée du vert
nous paraît aussi simple que celle du
bleu » (Leibniz, N. E., II, 2); « Par
idées, j'entends les faibles images
que laissent les impressions$^3$ dans
la pensée » (Hume). $->$ Bien que
cet emploi du mot ait laissé des
traces, {\it p. e.} dans « association$^3$ des
idées », le seul sens propre est le
sens 3; l'idée doit être dist. de
l’image.

— \si{Hist.} {\bf 5.} Idées-images (syn. :
espèces$^1$ impresses) : petites images
matérielles, que, selon Démocrite,
Épicure et certains scolastiques, les
objets matériels émettent et qui viennent impressionner les sens. — {\bf 6.}
Idées représentatives (chez Descartes, Locke, etc.) : idées$^4$ considérées comme « êtres représentatifs », {\it i. e.} comme intermédiaires
entre les objets et l’esprit, et participant de l’un et de l’autre.

— \si{Vulg.} {\bf 7.} Dessein, projet
« Avoir une idée ». — {\bf 8.} Opinion :
« Avoir des idées avancées ».

— \si{Ps. path.} {\bf 9.} Idée fixe : voir
Obsession*.

\ib{Identique} — \si{Méta.} A) Deux choses
sont identiques : {\bf 1.} (idem, nec unum)
quand elles sont parfaitement semblables, tout en restant distinctes :
en ce sens il ne peut exister, selon
Leibniz, deux substances identiques
(cf. Indiscernable*); — ou {\bf 2.} (unum,
nec idem) quand, malgré les difiérences apparentes, elles sont substantiellement une seule et même
chose. — B) Un être changeant
reste : 3 identique à lui-même
(unum et idem) quand son présent
demeure intimement solidaire de
son passé.

— \si{Log.} \si{form.} {\bf 4.} Dont le sujet$^2$ et
l’attribut représentent le même être :
« Les propositions identiques sont
indémontrables » (Leibniz). {\it {\it Cf.}} Tautologie$^1$.

\ib{Identité} — \si{Méta.} {\bf 1.} Caractère de ce
qui est identique : « Nous ne devons
jamais regarder l'identité comme
une identité abstraite, à l'exclusion
de toute différence » (Hegel).

— \si{Psycho.} {\bf 2.} Identité personnelle : propriété de la personnalité,
de rester identique$^3$ à elle-même.

— \si{Log.} 3 Principe d'identité :
« Ce qui est vrai, est vrai » (qqfs.
sous forme ontologique : « Une chose
est ce qu’elle est »), ou encore :
« Une même proposition ne peut être
à la fois vraie et fausse (en même
temps et sous le même rapport) » :
« Si le principe d'identité reste la
pierre angulaire de la pensée, c’est
en tant qu'il déclare la supériorité
du Même sur l’Autre$^2$ » (Lalande).

— \si{Math.} {\bf 4.} Égalité$^3$ qui est vraie
pour toutes les valeurs des termes
qui la constituent (le signe de l’identité est $\equiv$).

\ib{Idéologie} — \si{Hist.} {\bf 1.} Chez Destutt de
Tracy et les « idéologues » : étude des
idées$^4$ et de leur origine. — {\bf 2.} ({\it Péj.})
Spéculation creuse sur des concepts
illusoires.
% 92

— \si{Soc.} {\bf 3.} Chez les partisans du
malérialisme historique : ensemble
des idées$^8$, croyances, doctrines
propres à une société ou à une
classe$^3$ ({\it opp.} à l'infrastructure, seule
fondamentale). — {\bf 4.} Chez K. Mannheim : ensemble d'idées, croyances, etc., plus ou moins sincèrement
professées par un groupe d’individus, mais qui ne s’incarnent pas
dans les faits ({\it opp.} : utopie* qui
transforme quelque peu la réalité
historique).

\ib{Idéo-motrice (Action)} — \si{Ps. phol.}
Action par laquelle toute représentation d’un mouvement tend à produire ce mouvement.

\ib{Idio-syncrasie} — [G. idios, particulier,
et syncrasts, mélange] — \si{{\it Méd.}} {\bf 1.}
Ensemble des dispositions physiologiques propres à un individu. —
\si{Psycho.} {\bf 2.} Tempérament et caractère individuels. — {\bf 3.} Traits de
caractère propres à un individu.

\ib{Idiotie} — \si{Ps. path.} Faiblesse d’esprit,
gén. congénitale, se traduisant par
de l’hébétude et de l’obtusion intellectuelle, et liée à des lésions cérébrales.

\ib{Idoles} —[G. eidôla, fantômes] — \si{Hist.}
Chez Fr. Bacon : erreurs les plus
communes, dues à la nature humaine (idola tribus), à la nature de
chaque individu (idola specus), au
langage (idola fori), ou à la mauvaise
philosophie (idola theatri).

\ib{Idonéisme} — \si{Crit.} À. Terme inventé
par F. Gonseth pour désigner « une
philosophie et une méthode qui
jamais ne considèrent l’œuvre de la
pensée comme un monument achevé,
mais qui veulent cette œuvre toujours en chantier, prête à intégrer
les faits et les vues nouvelles ».
% 92

\ib{Ignoratio elenchi} — [Ignorance du sujet]
— \si{Log.} Paralogisme consistant à
prouver ou discuter autre chose que
ce qui est en question.

\ib{Illimité} — \si{Épist.} {\it Dist.} infini : une
circonférence, un espace sphérique
sont illimités (on peut en faire le
tour sans jamais rencontrer de
limite), mais non infinis.

\ib{Ilumination} — \si{Hist.} Chez saint Augustin (et Malebranche) : acte par
lequel Dieu éclaire l'âme humaine
et qui rend possible la connaissance.

\ib{Iluminisme} — \si{Hist.} À. Doctrine des
illuminés (Saint-Martin, Swedenborg) et des théosophes qui croient
recevoir des inspirations spéciales
de Dieu.

\ib{Illusion} — \si{Psycho.} {\bf 1.} {\it Lato.} Toute
erreur provenant d’une fausse apparence : « Les illusions de la passion ».
— {\bf 2.} {\it Str.} Erreur de la perception,
{\it spéc.} celle qui consiste à prendre un
objet pour un autre (v. Précis,
Ph. I, p. 128).

\ib{Image} — M \si{Phys.} {\bf 1.} Reproduction
d'un objet par l'effet de certains
phénomènes d’optique : « Image
réelle, virtuelle ». — \si{Phol.} {\bf 2.} Reproduction d’un objet qui se forme sur
la rétine par suite de la convergence
du cristallin : « Image rétinienne »,
ou sur les membranes de l'œil, agissant comme surfaces réfléchissantes :
« Images de Purkinje ».

— © \si{Psycho.} {\bf 3.} Phénomène psychique consistant dans la réapparition d’une sensation déjà éprouvée
en l'absence de Fobjet qui lui a
donné naïssance : « Les sensations
de l’ouïe, du goût, de l’odorat, du
toucher ont aussi leurs images »
(Taine); « L'image est un acte, et
non une chose » (Sartre). — \si{Psycho.}
et \si{Méta.} {\bf 4.} Toute représentation
%93
mentale d’origine sensible : « Image
lui-même, notre corps ne peut emmagasiner les images puisqu'il fait
partie des images » (Bergson, Mat.
et Mém, III).

\ib{Imaginaire} — \si{Psycho.} {\bf 1.} Produit de
l’imagination$^3$ : « Le vocable fondamental qui correspond à l’imagination, ce n’est pas image, c’est imaginaire » (Bachelard). — {\bf 2.} Chez
Sartre : « vie imaginaire », conduite
en face de l’irréel, qui est « radicalement différente de notre attitude
en face des choses ». Voir Magie$^3$.

— \si{Math.} 3 Nombre imaginaire
(ou complexe) : nombre de la forme
a + bi, l'unité « imaginaire » à étant
telle que $i^2 = - 1$.

\ib{Imagination} — \si{Psycho.} O {\bf 1.} {\it Autref.}
({\it not.} {\footnotesize XVII}$^\text{e}$ siècle) : faculté de penser
par images$^4$, {\it i. e.} de se représenter
les choses de façon sensible : « L’imagination n’est autre chose qu’une certaine application de la faculté qui
connaît, au corps qui lui est intimement présent » (Descartes, \si{{\it Méd.}},
VI; cf. \si{{\it Méd.}}, II : « Imaginer n'est
autre chose que contempler la
figure ou l’image d’une chose corporelle ») ; « L’imagination doit suivre,
mais de fort près, la sensation,
comme le mouvement du cerveau
doit suivre celui du nerf » (Bossuet).
— {\bf 2.} (Imagination reproductrice).
Fonction psychique par laquelle
l'esprit fait revivre les images$^3$
« C’est une chose étrange qu’une
imagination vive qui représente
toutes choses comme si elles étaient
encore » (Sévigné). — {\bf 3.} (Imag.
créatrice). Fonction psychique par
laquelle l'esprit crée de nouvelles
combinaisons d'images$^3$ : « L’imagination du peintre »; « L’imagination
de l'enfant »; « Souvent son imagination lui fournit plus que sa mémoire »
% 93
(La Rochefoucauld). —
 {\bf 4.} (Invention ou activité créatrice
de l'esprit). Fonction psychique
par laquelle l'esprit crée de nouvelles combinaisons d’idées$^3$
« L’imagination scientifique a pour
matériaux des concepts » (Ribot).

— ©. {\bf 5.} Produit de ces diverses
fonctions; conception arbitraire
« C’est une pure imagination »;
« Cette imagination [que la certitude vient des sens] est aussi fausse
que la première » (Port-Royal).

\ib{Imago} — \si{Ps. an.} Chez Jung : projec
tion d’une image ou d’un souvenir
anciens sur une personne ou un
objet : « La vitalité et l’indépendance de l'imago échappent à la
conscience. »

\ib{Imitation} — \si{Psycho.} et \si{Soc.} Repro
duction : {\bf 1.} (lato) volontaire ou involontaire, consciente ou inconsciente ({\it p. e.} Tarde, Lois de l’Imitation) ; — {\bf 2.} ({\it str.}) consciente et volontaire, des gestes ou actes d’autrui :
« Imiter, c’est comprendre » (Rabaud). $->$ Le seul sens propre est
le sens {\bf 2.}

\ib{Immanence} — Méta, {\bf 1.} Caractère de
ce qui est immanent$^2$. — {\bf 2.} Principe d'immanence : celui qui énonce
« l'impuissance radicale de la pensée
à sortir de soi » (Le Roy). — {\bf 3.} Chez
Sartre : « illusion d’immanence »,
celle qui consiste à se représenter la
conscience comme un lieu peuplé
d'images, celles-ci étant elles-mêmes
des simulacres des objets.

\ib{Immanent} — [{\it L.} in, dans, et manere,
rester] — \si{Méta.} Qui réside dans.
D'où : {\bf 1.} {\it Autref.}, action ou cause
immanente : celle qui demeure à
l’intérieur du sujet agissant ({\it opp.}
cause transitive qui s'exerce par-delà
celui-ci sur autre chose) : « Dieu
% 94
est la cause immanente de toutes
choses, non la cause transitive »
(Spinoza {\it Eth.}, I, 18), ce qui est la
formule du panthéisme*, — {\bf 2.}
Plus gén. Qui est contenu dans la
nature d’un être ({\it opp.} : transcendant$^2$) : « Le consentement* n'est
qu’un acte immanent de la volonté »
(Malebranche) ; « L’insatisfaction immanente à la condition humaine »;
« La mathématique immanente aux
choses » (Bergson, {\it P. M.}, II); « La
Valeur est immanente à toutes les
valeurs empiriques » (Le Senne).
Justice immanente : celle qui résulterait du cours naturel des choses.

\ib{Immanentisme} — A. \si{Méta.} {\bf 1.} Doctrine qui considère l’Absolu ou la
Valeur comme immanents$^2$ aux êtres
particuliers : « Un immanentisme
radical dont Spinoza s’est approché,
mais qui chez lui comporte comme
atténuation l'attribution d’une infinité d’attributs à la Substance »
(Le Senne). — \si{Théol.} {\bf 2.} Tendance
de certaines doctrines dites « modernistes » à considérer Dieu comme
immanent$^2$ à l’homme et à nier le
surnaturel. $->$ {\it Dist.}, en ce sens,
« méthode d’immanence », méthode
d’apologétique* qui part de ce qu'il
y a d’immanent? à l’homme pour y
montrer un vide qui ne peut être
comblé que par un apport surnaturel (cf. Wehrlé, La méthode d’immanence, 1911).

\ib{Immatérialisme} — \si{Méta.} À. Nom
donné par Berkeley à son idéalisme$^4$, qui n'’admet comme existence réelle que celle des esprits$^4$.

\ib{Immédiat} — (Ctr. médiat). \si{Crit.}
1. Donné à la conscience directement, sans intermédiaire ni élaboration apparente; donc : psychologiquement premier. — {\bf 2.} Simple,
indécomposable; donc : effectivement
% 94
primitif, premier en droit.
$->$ {\it Dist.} ces deux sens souvent
confondus : « L’immédiat$^1$ que nous
pouvons atteindre n’est jamais un
véritable immédiat$^2$ » (Darbon);
« L'immédiat$^2$ et le primitif est précisément ce qui ne peut jamais être
donné, ayant servi à faire tout ce
qui est donné... L'’intuition même
la plus primitive et la plus immédiate$^1$ n'atteint que des éléments
élaborés » (Pradines).

— \si{Log.} {\bf 3.} Inférences immédiates :
celles où l’on passe sans intermédiaire d’une proposition principe à
la conclusion$^1$ (conversion* et opposition*).

\ib{Immensité} — \si{Méta.} Propriété de
Dieu d'être soustrait aux déterminations de l’espace : « L’étendue
créée est à l’immensité divine ce
que le temps est à l'éternité » (Malebranche, Enir., VIII, 4) ; « L'immensité de Dieu fait que Dieu est
dans tous les espaces. L'espace
infini n’est pas l’immensité de Dieu »
(Leibniz, 5$^\text{e}$ lettre à Clarke, 45-46).

\ib{Immoralisme} — \si{Mor.} {\bf 1.} A. Doctrine
morale qui admet des règles contraires à celles de la morale courante ({\it not.} la doctrine de Nietzsche).
— {\bf 2.} À Qdfs. (mais abusivement),
syn. d'amoralisme$^2$.

\ib{Immortalité de l’âme} — \si{Méta.} Affirmation que l’âme survit indéfiniment à la mort du corps et gén.
qu'elle conserve son individualité.

\ib{Impénétrabilité} — \si{Méta.} Propriété
de la matière qui fait que deux corps
ne peuvent occuper en même temps
la même étendue?. — {\it {\it Cf.}} Antitypie*.

\ib{Impératif} — Qui exprime un commandement, un ordre. Chez Kant :
« l'impératif catégorique* .», le
Devoir.
% 95

\ib{Impersonnel} — {\bf 1.} Non personnel,
ne comportant pas la personnalité :
« Le Dieu des panthéistes est impersonnel ». — {\bf 2.} Non individuel,
objectif$^3$ : « Plus la science avance,
plus elle prend la forme impersonnelle » (Cl. Bernard).

\ib{Implication} — \si{Log.} \si{form.} Relation
formelle$^3$, consistant en ce qu’une
idée ou une proposition en implique$^2$
une autre, et considérée indépendamment de la vérité matérielle de
ces idées ou propositions (Le signe de
l’implication est $\supset$ : « mammifère $\supset$
vertébré »).

\ib{Implicite} — Ctr. d'explicite*.

Impliquer. — \si{Log.} Envelopper, entraîner comme conséquence* en
vertu : {\bf 1.} d'une nécessité expériencielle*: — {\bf 2.} d’une nécessité rationnelle et purement logique : « L’universelle* implique la particulière ».
{\it Spéc.} dans l'expression « impliquer
contradiction$^1$ » : « Il implique contradiction qu’une pensée goit matière » (Voltaire). {\it Autref.}, elliptiquement : « impliquer », être contradictoire : « Il est certain que sa
nature [de Dieu] est possible, ou
qu'elle n'implique point » (Descartes, 2$^\text{es}$ {\it Rép.}).

\ib{Impresses (Espèces)} — Voir Intentionnel$^2$.

\ib{Impression} — \si{Vulg.} {\bf 1.} État global de
la conscience, à caractère affectif,
produit par une action extérieure
antérieurement à toute réflexion :
« Faire bonne impression ».

— \si{Phol.} {\bf 2.} Phénomène physiologique produit par un excitant* sur
une terminaison nerveuse sensitive :
« Impression rétinienne ».

— \si{Psycho.} {\bf 3.} Chez Hume : « toutes
nos sensations, passions et émotions,
alors qu’elles font leur première
%95
apparition dans l’âme », {\it i. e.} état
primaire ({\it opp.} à idée$^4$ ou état secondaire).

\ib{Impulsion} — \si{Psycho.} Tendance ou
instinct agissant en l'absence du
contrôle de la volonté$^1$.

\ib{Imputabilité} — \si{Mor.} Caractère d’un
acte qui peut être attribué à un
agent moral responsable.

\ib{Incarné} — \si{Méta.} Lié à un corps$^3$ :
« Le monde existe pour moi pour
autant que je suis incarné » (Marcel).

\ib{Incertitude (Relations d')} — \si{Phys.}
Relations posées en 1927 par le
physicien Heisenberg et d’après
lesquelles, dans la mécanique intraatomique, il est impossible, à un
même instant, de déterminer à la
fois la position et la vitesse d’un
corpuscule (voir Précis, Ph. II,
p. 143; Sc. et M., p. 260).

\ib{Inclination} — \si{Psycho.} Syn. de tendance* : « Les inclinations des esprits
sont au monde spirituel ce que le
mouvement est au monde matériel »
(Malebranche, R. V., IV, 1). $->$
S'emploie surtout auj. lorsque l’on
considère les effets affectifs des tendances.

\ib{Inclusion} — \si{Log.} \si{form.} Rapport de
deux termes dont l’un englobe
l’autre en extension$^3$.

\ib{Incomplétude (Sentiments d’)} — \si{Ps. path.} Sentiments d’inachevé, d'incomplet, qu'éprouvent certains malades (cf. Psychasthénie*) à l'égard
de leurs perceptions (Janet).

\ib{Incompréhensible} — \si{Crit.} {\bf 1.} Que
l’on ne peut comprendre au sens 2,
inintelligible : « Ce raisonnement est
incompréhensible ». — {\bf 2.} Que l’on
ne peut comprendre au sens 3 : « La
nature de Dieu est immense, incompréhensible et infinie » (Descartes,
%96
\si{{\it Méd.}}, IV) ; « Tout ce qui est
incompréhensible ne laisse pas d’être:
le nombre infini, un espace infini,
égal au fini » (Pascal, 430). $->$
{\it {\it Cf.}} inconcevable$^2$.

\ib{Inconcevable} — \si{Crit.} I À. {\it Str.} Irréductible à des concepts$^1$. — {\bf 2.} {\it Lato.}
Syn. d'incompréhensible au sens 2
« Lorsque Dieu est dit inconcevable,
cela s'entend d’une pleine et entière
conception qui comprenne et embrasse parlaitement tout ce qui est
en lui » (Descartes, 2$^\text{es}$ {\it Rép.}). —
©. {\bf 3.} {\it Latiss.} (et par hyperbole).
Étonnant : « Une ignorance inconcevable ». $->$ En tous les sens, dist.
inintelligible* (voir Textes choisis,
I, p. 186, n. 1).

\ib{Inconnaissable} — \si{Crit.} Qui, tout en
étant réel, ne peut être connu d’aucune façon. Surtout usité danslelang.
de Spencer : « l’Inconnaissable ». —
{\it {\it Cf.}} Agnosticisme*.

\ib{Inconscient} — \si{Psycho.} A) En parlant
des personnes : {\bf 1.} Qui manque de
réflexion : « Un inconscient » —
 {\bf 2.} Qui n’a pas conscience$^1$ (simple)
de... : « Inconscient de ses vrais
sentiments ». — B) En parlant des
faits psychiques : {\bf 3.} Qui échappe à
la conscience$^1$ réfléchie (dire plutôt
subconscient). — {\bf 4.} Qui échappe à
la conscience! simple (sens propre).
Chez Jung : « inconscient collectif »,
ensemble des images et motifs (cf.
Archétypes$^2$) qui symbolisent, dans
la psychè, les instincts fondamentaux de l’homme et préexistent à la
naissance.

— \si{Méta.} {\bf 5.} (Nom). Chez Ed. von
Hartmann : « l’Inconscient », l'Être
en soi, considéré comme Volonté
inconsciente$^2$ guidée par l'Idée.

\ib{Indéfini} — \si{Log.} {\bf 1.} (Ctr. : défini). En
parlant des termes : qui manque de
définition$^1$. — {\bf 2.} ({\it Opp.} : affirmatif
%96
et négatif). En parlant des propositions : limitatif* (v. ce mot).

— \si{Épist.} et \si{Méta.} ({\it Opp.} : fini
et infini). {\bf 3.} Chez Descartes : l'infini$^2$
mathématique : « Pour les choses où
sous quelque considération seulement
je ne vois point de fin, comme l’étendue, la multitude des nombres, la
divisibilité des parties de la quantité... je les appelle indéfinies, et
non pas infinies parce que de toutes
parts elles ne sont pas sans fin ni
sans limites » (Descartes, 1$^\text{es}$ {\it Rép.} ;
cf. {\it Princ.}, I, 27). — {\bf 4.} {\it Auj.} : ce qui,
tout en étant fini, est susceptible
de s’accroître sans cesse; ce dont les
bornes peuvent toujours être reculées : « Un progrès indéfini » (Condorcet).

\ib{Indétermination} — \si{Log.}
et \si{Méta.}
1. Absence de détermination$^1$ —
 {\bf 2.} Absence de déterminisme$^1$, contingence$^2$.

— \si{Psycho.} {\bf 3.} ©. Incapacité de se
décider, irrésolution.

\ib{Indéterminisme} — A. \si{Méta.} et \si{Psycho.} Doctrine qui pose en principe
l’indétermination$^2$ des phénomènes
({\it spéc.} de la volonté humaine).

\ib{Indifférence} — \si{Psycho.} {\bf 1.} Absence
de préférence ou même d'intérêt
pour qqc. ou {\it qqn.} : « Cela m'est de
la dernière indifférence » (Diderot).
{\it Spéc.} « indifférence en matière de
religion » (Lamennais). Etals d’indifférence : cf. Neutres*. — {\bf 2.} Liberté
d'indifférence : celle qui consisterait
à agir sans être déterminé par
aucun motif : « Cette indifférence
que je sens lorsque je ne suis porté...
par le poids d’aucune raison, est le
plus bas degré de la liberté » (Descartes, \si{{\it Méd.}}, IV).

\ib{Indiscernables (Principe des)} — \si{Hist.}
« Il n’y a jamais dans la nature deux
% 97
êtres qui soient parfaitement l’un
comme l’autre » (Leibniz, Mon., 9).

\ib{Individu} — [{\it L.} individuum, chose indivisible, atome] — \si{Méta.} {\bf 1.} Être concret dont les parties sont solidaires
et ne peuvent être séparées sans que
cet être cesse d'être ce qu’il est.
D'où : \si{Biol.} : « Tout être vivant est
un individu ». {\it Ext.} \si{Phys.} Élément
indivisible « La question s'est
posée de savoir s’il est possible de
considérer les corpuscules [de
l'atome] comme des individus physiques parfaitement définis et localisés dans l’espace » ({\it L.} de Broglie).

— \si{Soc.} {\bf 2.} L'unité dont se compose une société : « [individu
acquiert des droits de plus en plus
étendus » (Durkheim).

— \si{Log.} \si{Soc.} {\bf 3.} Terme singulier*, dont l’extension$^3$ est {\bf 1.}

\ib{Individualisation} — \si{Soc.} Processus
par lequel un phénomène devient
individuel$^2$ : « Par un phénomène
d’individualisation, l'autorité diffuse* dans le groupe s’incarne dans
des sujets individuels$^2$ » (Davy). —
{\it {\it Cf.}} Démocratie*.

\ib{Individualisme} — \si{Soc.} À. {\bf 1.} État de
fait caractérisé par les progrès de
l'initiative et de la réflexion individuelle : « L’individualisme est
caractéristique des sociétés évoluées. » — À. {\bf 2.} Tendance à expliquer les phénomènes sociaux par
l'action des individus$^2$ : « L'’individualisme de la sociologie de Tarde. »

— À. \si{Mor.} 3 Doctrine morale,
selon laquelle l’homme a sa fin en
lui-même, la société n'ayant de valeur qu’en tant qu’elle favorise le
développement de la personnalité
individuelle. — \si{Pol.} et \si{Éc. soc.}
 {\bf 4.} (Ctr. : étatisme). Syn. de libéralisme* aux sens 3 ou {\bf 4.} — 5, Doctrine ({\it p. e.} de Stirner) selon laquelle
% 97
le moi ou l’Unique est la seule réalité
(v. Textes choisis, II, p. 206).

— \si{Vulg.} {\bf 6.} ©. Syn. d’égoïsme*,
sauf que « l’égoïsme naît d’un instinct aveugle; l’individualisme procède d’un jugement erroné » (Tocqueville). $->$ Impropre en ce sens.

\ib{Individuation (Principe d’)} — \si{Hist.}
Chez les Scolastiques : ce qui donne
à un être, déjà défini par sa forme$^1$,
une existence concrète et individuelle.

\ib{Individuel} — {\bf 1.} Qui concerne l'indi
vidu$^2$ : « La liberté individuelle ». —
 {\bf 2.} Qui concerne l’individu$^3$ : « Chaque
fait individuel était compliqué; la
loï des grands nombres* rétablit la
simplicité dans la moyenne » (Poincaré).

\ib{Induction} — \si{Vulg.} {\bf 1.} Inférence con
jecturale : « Il ne peut juger des
choses qu’il ne voit pas que par
induction sur celles qu'il voit »
(Rousseau).

— \si{Épist.} et \si{Log.} O. Opération
qui consiste à passer des faits à la
loi$^5$, et gén. de cas singuliers* ou
spéciaux* à une proposition plus
générale$^2$. {\it Dist.} : {\bf 2.} l'induction formelle (syn. : aristotélicienne ou complète), fondée sur l’énumération
complète des espèces$^2$ d’un genre
ou des individus$^3$ d’une collection* :
« L'’induction n’est un moyen certain de connaître une chose que
quand nous sommes assurés que
l'induction est entière [= complète] » (Port-Royal); — {\bf 3.} l’induction amplifiante (syn. : baconienne),
qui étend à tout un genre$^1$ ce qui a
été constaté dans un certain nombre
de cas singuliers* et qui peut être
elle-même : a) spontanée et empirique*, ou b) méthodique et expérimentale : « L’induction baconienne a pour but de trouver et de
% 98
prouver par l’examen des faits les
lois qui les régissent » (Goblot); —
 {\bf 4.} l'induction mathématique, nom
donné par Poincaré au raisonnement par récurrence$^2$.

— {\bf 5.} @. Produit de ces diverses
opérations : « Leur physique (des
anciens] est pleine de qualités occultes* et d'inductions vagues »
(Mairan).

\ib{Ineffable} — Qui ne peut s'exprimer
adéquatement par le langage. {\it Spéc.},
\si{Théol.} : voir Dieu$^2$. — Sur les divers
emplois de ce terme, voir Précis,
Ph. I, p. 340, et cf. Irrationnel$^3$.

\ib{Inertie} — \si{Math.} et \si{Phys.} {\bf 1.} Propriété
de la matière « par laquelle le corps
résiste en quelque façon au mouvement » (Leibniz). Force d'inertie,
force égale et de sens inverse au
produit de la masse par l'accélération. — {\bf 2.} Principe d'inertie : « Un
point matériel libre ne peut passer
du repos au mouvement ou modifier
son mouvement sans l’action d’une
cause extérieure ou force$^4$ ».

\ib{Inférence} — \si{Log.} Opération logique
par laquelle on tire une conclusion$^1$
d’une ou de plusieurs propositions
admises comme vraies. — List.
implication* et raisonnement* : ce
dernier mot désigne surtout les inférences médiales*.

\ib{Infériorité (Complexe ou Sentiment d')}
— \si{Ps. path.} Trouble décrit par
Adler et Janet et consistant dans le
sentiment d'être inférieur à sa
tâche, à autrui, etc — Cf Compensation*,

\ib{Infini} — \si{Méta.} {\bf 1.} Ce en quoi nous ne
concevons aucune limite : « Il n’y a
rien que je nomme proprement
infini, sinon ce en quoi de toutes
parts je ne rencontre point de limites, auquel sens Dieu seul est
% 98
infini » (Descartes, 1$^\text{es}$ {\it Rép.}). {\it {\it Cf.}}
indéfini$^3$. — {\bf 2.} Ce qui, dans un
ordre donné, n'a pas de limite, ou
ce qui est plus grand que tout ce
qui comporte une limite (infini
actuel$^2$). {\it Spéc.}, \si{Math.} : une quantité
variable est infinie ou infiniment
grande quand elle devient plus
grande que toute quantité donnée,
infiniment pelite quand elle devient
plus petite que toute quantilé
donnée ({\it i. e.} elle a pour limite zéro).
$->$ Ne pas dire : « nombre infini »,
ce qui est contradictoire. — {\bf 3.} {\it Qqfs.},
syn. de indéfini$^4$ (infini potentiel).

— \si{Vulg.} {\bf 4.} Très grand : « Ces
agitations infinies qui partagent le
cœur au moment d’un changement » (Massillon).

\ib{Infinitésimal} — \si{Math.} {\bf 1.} Infiniment
petit. Calcul infinitésimal : ensemble
du calcul dilférentiel* et du calcul
intégral*.

— \si{Vulg.} {\bf 2.} Très petit : « Une
couche infinitésimale d’or ou d’argent » (Cournot).

\ib{Infrastructure} — \si{Soc.} Syn. : base$^2$.

\ib{Inhérence} — \si{Méta.} {\bf 1.} Chez Kant :
« Quand ‘on attribue à ce réel dans
la substance [les accidents*] une
existence particulière, on nomme
alors cette existence inhérence pour
la distinguer de lexistence de la
substance, qu’on nomme subsistance » ({\it R. pure}, {\it Analyt.}, II, 2, 3
1$^\text{re}$ anal. de l'expérience). — \si{Log.}
 {\bf 2.} Rapport à son sujet$^2$ d’une qualité
qui lui est attribuée (à l’exclusion
des autres relations : quantitatives,
ordinales dans le temps ou l’espace,
etc.) : Il convient de distinguer
deux genres de propositions : les
propositions d’inhérence [{\it p. e.} « Pierre
est homme »] et les proposilions de
relation [{\it p. e.} « Pierre est plus âgé
que Paul »] » (Lachelier). — {\bf 3.} ({\it {\it Cf.}} :
%99
immanent$^2$). Rapport à son sujet$^2$
d'un attribut qui lui est essentiel :
« Les hommes ont tous un droit
inhérent et naturel à ce dont ils ont
besoin pour leur subsistance »
(Fénelon); « Le pouvoir d'agir est
inhérent dans toute substance »
(Leibniz).

\ib{Inintelligible} — \si{Crit.} {\bf 1.} I. {\it Str.} Qui
n’est pas de l’ordre de l’intelligible :
« La représentation pure ne contient
pas laffirmation d’un ordre fixe
indépendant de nous, inintelligible
(causalité), lequel suppose un ordre
fixe dépendant de nous, intelligible » (Lagneau)). — {\bf 2.} ©. {\it Lato.}
Incompréhensible$^1$ : Cet auteur
est inintelligible ».

\ib{Inné} — \si{Biol.}, \si{Psycho.}, \si{Crit.} (Ctr.
acquis). Qui existe dans un être dès
sa naissance : « Un réflexe inné ».
Idées innées : celles qui, selon l’apriorisme*, sont inhérentes$^3$ à l'esprit
humain et existent en lui sans qu'il
les reçoive du dehors (cf. Adventice*). L’innéité peut être seulement virtuelle : « J'ai appelé ces
idées innées, dans le sens où nous
disons que certaines maladies sont
innées dans certaines familles : non
que les enfants en souffrent dès le
sein de leur mère, mais en ce sens
qu'ils naissent avec une certaine
disposition ou aptitude à les contracter » (Descartes, Notæ in programma quoddam).

\ib{Innéisme} — \si{Psycho.} et \si{Crit.} À Doc
trine qui admet l'existence d’idées$^4$
ou de prineipes$^2$ innés* : « L’innéisme cartésien ». $->$ {\it Dist.} nativisme*.

\ib{Inquiétude} — \si{Psycho.} {\bf 1.} {\it Autref.}, péj.
Agitation d'esprit, incapacité de se
tenir en repos : « L’inquiétude est le
plus grand mal qui arrive en l'âme,
% 99
excepté le péché » (Saint François
de Sales); « L’inquiétude de notre
volonté est une des principales
causes de l'ignorance où nous
sommes » (Malebranche, R. V., IV,
2, 1). — {\bf 2.} {\it Qqfs.} et surtout auj.
{\it laud.} État de la conscience insatisfaite de ce qui est : « L’inquiétude
est essentielle à la félicité des créatures, laquelle ne consiste jamais
dans une parfaite possession »
(Leibniz); « ... en appelant Ame une
certaine inquiétude de vie » (Bergson,
{\it P. M.}, VI); « L'inquiétude humaine,
qu'il ne faut pas confondre avec
une antiélé pathologique. » (Le
Roy).

\ib{Instabilité mentale} — \si{Car.} Anomalie
du caractère se traduisant par un
manque d'unité et de continuité
dans les pensées et dans les actes : le
sujet est « tour à tour inerte et
explosif » (Ribot).

\ib{Instance} — \si{Log.} {\bf 1.} Objection nouvelle alléguée à la suite d’une réplique de l’adversaire à une objection précédente : « J’ai négligé de
répondre au gros livre d’Instances
[de Gassendi] » (Descartes, 5$^\text{es}$ {\it Rép.}).
— {\bf 2.} Chez Bacon : exemple particulier et typique (cf. angl. for instance). D'où : cas particulier.

\ib{Instant} — \si{Méta.} Portion très courte
ou même ponctuelle de la durée :
« L'instant est le croisement du
temps et de l'éternité » (Lavelle).

\ib{Instinct} — \si{Psycho.} {\it Str.} {\bf 1.} Activité
automatique$^3$ (existant surtout chez
l'animal), caractérisée par un ensemble de réactions bien déterminées, héréditaires, spécifiques$^1$,
souvent complexes et paraissant
adaptées à une fin ({\it p. e.} instinct de
mellification chez l’abeille) : « N’est-ce pas là traiter indignement la
% 100
raison de l’homme et la mettre en
parallèle avec l'instinct des animaux ? » (Pascal, Vide). — {\bf 2.} Chez
Bergson ({\it Opp.} : intelligence$^3$) : mode
de connaissance et d’action qui
joue et sent par une sorte de sympathie$^2$ ce que l'intelligence$^3$ analyse
et se représente mécaniquement :
« C’est sur la forme même de la vie
qu'est moulé l'instinct» (E. C, II). —
 {\bf 3.} Chez Freud : forces psychiques
inconscientes qui constituent le ça* :
« Nous donnons aux forces qui agissent à l'arrière-plan des besoins
impérieux du ça et représentent
dans le psychisme les exigences d’ordre somatique*, le nom d’instincts. »

— {\it Lato.} {\bf 4.} Toute activité spontanée$^3$, même si elle ne se manifeste
pas par des réactions déterminées :
« L'instinct de conservation » ;
« L'instinct esthétique » ; « Malgré la
vue de nos misères,... nous avons
un instinct que nous ne pouvons
réprimer, qui nous élève » (Pascal,
411) ; « Les instincts ne sont pas
toujours de pratique ; il y en a qui
contiennent des vérités de théorie :
tels sont les principes du raisonnement lorsque nous les employons
par un instinct naturel » (Leibniz,
N. E., I, 2, 3).

\ib{Institution} — \si{Soc.} « Ensemble d’actes
ou d'idées tout institué que les
individus trouvent devant eux et
qui s’impose plus ou moins à eux »
(Fauconnet et Mauss).

\ib{Instrumentalisme} — \si{Crit.} A. Forme
de pragmatisme* ({\it not.} de J. Dewey)
qui affirme le caractère instrumental
de la vérité, {\it i. e.} que celle-ci est un
simple instrument (tool) pour l’action
et l'enrichissement de l'expérience
ultérieure.

\ib{Intégral (Calcul)} — \si{Math.} Partie du
caleul infinitésimal* où l’on élimine
%100
les infiniment$^2$ petits introduits dans
le calcul différentiel$^1$ pour revenir
aux quantités finies.

\ib{Intégration} — \si{Math.} {\bf 1.} Opération
fondamentale du calcul intégral*.
— \si{Biol.}, Psycho, \si{Soc.} {\bf 2.} Ensemble des phénomènes par lesquels
se constitue l'unité organique d’un
être vivant, d’un système mental,
d’une société. Intégration psychique :
voir Acceptation*.
— \si{Méta.} {\bf 3.} Chez Spencer : passage du diffus à l’organisé et accroissement de matière d'un système.
{\it {\it Cf.}} Évolution$^2$.

\ib{Intellect} — \si{Psycho.} et \si{Crit.} {\bf 1.} Syn. :
entendement$^1$ (v. Intellection*). —
 {\bf 2.} Chez Aristote et les Scolastiques :
« intellect actif » ou « agent »,
fonction active de l'intelligence$^2$ (selon Aristote, c’est l'élément divin et
immortel de l’âme) qui actualise les
formes! intelligibles contenues dans
le sensible et les rend assimilables
par l'intellect « passif » ou « patient », purement réceptif.

\ib{Intellection} — \si{Psycho.} et \si{Crit.} lixercice de l’intellect$^1$ : « L’entendement
répond à ce qui, chez les Latins,
est appelé intellectus, et l'exercice
de cette faculté s’appelle intellection,
qui est une perception distincte,
jointe à la faculté de réfléchir
qui n’est pas dans les bêtes »
(Leibniz, N. E., I, 21, 5). {\it {\it Cf.}} Conception$^1$.

\ib{Intellectualisme} — \si{Psycho.} {\bf 1.} A.
Doctrine qui ramène tous les faits
psychiques aux faits intellectuels
(au sens 1 ou 2) et méconnaît ainsi
l'originalité et la primauté de la
tendance et de l’affectivité : « La
théorie intellectualiste a trouvé sa
plus complète expression chez Herbart, pour qui tout état affectif
n'existe que par le rapport réciproque
% 101
des représentations » ((Ribot).
— D'où (péj.) : tendance à supposer
de la logique et de la réflexion là où
il n'y en a pas.

— \si{Méta.} et \si{Mor.} {\bf 2.} À. Doctrine
qui attribue à l’entendement$^1$ une
valeur supérieure à celle du sentiment ou de l’activité : « La science
sera intellectualiste ou elle ne sera
pas » (Poincaré). {\it Ext.} (péj.), tendance à privilégier la pensée conceptuelle et discursive (cf. Entendement$^3$) : « Il faut renverser cette
vieille idolâtrie intellectualiste »
(Le Roy, {\it R. M. M.}, 1900, p. 71).
Bergson distingue « l’intellectualisme vrai qui vit ses idées » et « un
faux intellectualisme, qui immobilise les idées mouvantes en concepts
solidifiés pour les manier comme
des jetons » ({\it Bull.}, 1901, p. 64).

\ib{Intellectuel} — \si{Psycho.} {\bf 1.} {\it Lato.} (Syn. :
cognitif*, représentatif$^1$). Qui se rapporte à l'intelligence$^1$ : les « faits
intellectuels » ({\it opp.} : affectifs* et
d'activité$^2$) sont les sensations, images
perceptions, souvenirs, idées, jugements, raisonnements. — {\bf 2.} {\it Str.}
({\it opp.} : sensitif*) Qui se rapporte à
l'intelligence$^2$ : « Ce qui est proprement spirituel, c’est ce qui est intellectuel » (Bossuet). $->$ Bien que le
terme ait été qqfs. pris en ce sens,
{\it not.} au {\footnotesize XVII}$^\text{e}$ siècle ({\it p. e.} : « Préparer
les esprits des lecteurs à considérer
les choses intellectuelles et les distinguer des corporelles » [Descartes,
3$^\text{es}$ {\it Rép.}]), il doit être dist. de psychique : ce serait une erreur intellectualistel de ramener toute la vie
psychique à la vie intellectuelle
(même au sens 1).

— \si{Car.} {\bf 3.} Chez qui prédomine la
vie intellectuelle$^2$ : « C'est ce que
j'appellerai le type intellectuel, entendant par là les esprits qui ont une
particulière aptitude à penser avec des
% 101
idées$^3$, à retenir des rapports abstraits,
à enchaîner des concepts suivant
des relations logiques » (Malapert).

\ib{Intelligence} — O. \si{Psycho.} {\bf 1.} {\it Lato.}
Syn. : connaissance, vie intellectuelle$^1$ en gén. : « Dans l'intelligence,
nous aurons la sensation (ou fonctions sensitives), l’entendement (ou
fonctions intellectuelles analytiques),
la raison (ou fonctions rationnelles :
métaphysiques, morales) » (Lagneau). — {\bf 2.} {\it Str.} Syn. : intellect$^1$,
entendement$^1$ : les « opérations de
l'intelligence » sont alors l’idéation,
le jugement, le raisonnement. —
 {\bf 3.} Chez Bergson ({\it opp.} Instinct$^2$) :
forme de pensée qui procède par
analyse$^3$ et discursivement* et dont
le rôle est surtout pratique$^1$ : « L’intelligence est caractérisée par une
incompréhension naturelle de la
vie » ({\it E. C.}, II); « L'intelligence
n’évolue avec facilité que dans
l’espace et ne se sent à son aise que
dans l'inorganisé » ({\it P. M.}, II).

— ®. {\bf 4.} Compréhension$^1$ : « Obtenir quelque intelligence des vérités
que la loi nous enseigne » (Malebranche, {\it Entr.}, {\footnotesize XIV}, 13).

— \si{Car.} {\bf 3.} (Ctr. : inintelligence).
Souplesse d’esprit qui fait qu'on
s'adapte facilement aux situations
nouvelles : « ... Une intelligence si
extraordinaire qu'on eût dit que
rien n'était nouveau pour lui »
(Genlis).

\ib{Intelligible} — \si{Psycho.} et Meta. {\bf 1.}
({\it Opp.} : sensible). Qui relève de l’entendement$^1$ pur : « Nous voilà dans
un monde tout rempli de beautés
intelligibles » (Malebranche, {\it Entr.},
I, 5). Chez Platon : « monde intelligible », celui des Idées$^{1a}$, Chez Malebranche : « étendue intelligible »,
voir Étendue$^3$. Chez Kant : « caractère intelligible » ({\it opp.} : caractère
% 102
empirique), celui par lequel le moi$^5$
nouménal est cause de ses actes
comme phénomènes$^2$ « sans être luimême soumis aux conditions de la
sensibilité » ({\it R. pure}, {\it Dial.}, II, 2,
9, § 3). — \si{Épist.} {\bf 2.} (Ctr. : inintelligible$^2$). Qui peut être compris.

\ib{Intensité} — \si{Épist.} Caractère des
grandeurs intensives (cf. Extensif*),
{\it i. e.} de celles qui, tout en comportant du plus et du moins, ne peuvent ni se mesurer par un nombre,
ni se représenter par une étendue :
« L’intensité d’un sentiment ».

\ib{}Intention. — \si{Mor.} {\bf 1.} ©. Disposition
d'esprit qui fait qu’on se propose
d'atteindre un but ou d’agir conformément à une règle. $->$ Dist,
velléité*. — {\bf 2.} M Le but visé lui-même. — {\bf 3.} Direction d'intention :
action de rapporter ses actes ou ses
paroles à un but qui leur confère
une valeur morale ; d’où (péj.)
action de légitimer ce qu'il y a de
répréhensible dans un acte par l'intention louable qui l'a fait commettre (cf. Pascal, {\it Prov.}, VII).

\ib{Intentionalité} — \si{Méta.} Notion d'origine scolastique, remise en honneur par Franz Brentano et Husserl,
et qui désigne le caractère propre à
la pensée de tendre vers, de s’appliquer à un objet de pensée : « Le mot
intentionalité ne signifie rien d’autre
que cette particularité foncière qu’a
la conscience d’être la conscience
de qqc. » (Husserl).

\ib{Intentionnel} — Â. Qui concerne l'intention$^1$ : « Une erreur intentionnelle ». — {\bf 2.} Dans le lang. scolastique : « espèces intentionnelles »
(syn. : « impresses »), images qui
émanent des corps et viennent
frapper les sens (v. Espèce$^1$). —
 {\bf 3.} Qui présente une intentionalité* :
« Les vécus intentionnels » (Husserl).

\ib{Intérêt} — \si{Mor.} {\bf 1.} Ce qui est utile à
l'individu (intérêt personnel) ou à
l’ensemble des individus d’un groupe
(int. général) ou au groupe comme tel
(int. public). Morale de l'intérêt : ci.
Utilitarisme*.

— \si{Psycho.} {\bf 2.} ©. Attention spontanée provoquée par les objets qui
correspondent à nos tendances
« Prendre intérêt à qqc. » (s’y intéresser). — {\bf 3.} M. Caractère de ce qui
éveille l'intérêt$^2$ : « Présenter de
l'intérêt » (être intéressant). Loi
d'intérêt : celle qui explique les associations d'idées par l'intérêt$^3$ (cf.
Précis, Ph. I, p. 279).

\ib{Intérieur, Interne (adj.)} — 1, 2 et {\bf 3.}
S'{\it opp.} à Extérieur* (voir ce mot)
dans ses divers sens.

\ib{Intérieur (nom)} — {\it Autref.} {\bf 4.} La vie
intérieure? : « Il faut que l’extérieur
soit joint à l’intérieur pour obtenir
de Dieu » (Pascal, 250); « Ce qui,
demeurant hors de lui, ne peut remplir son intérieur » (Bossuet). —
 {\bf 5.} Intériorité* : « Tout respire l’intérieur et la piété dans l’ordonnance
de ce prélat » (id.).

\ib{Intériorité} — Caractère de ce qui est
intérieur$^2$, {\it i. e.} de ce qui est de
l’ordre de la vie intime de l'esprit$^4$ :
« L'appétit d'intériorité qui caractérise notre époque... » (Bréhier).

\ib{Intermittente (Folie)} — Voir Circulaire*,

\ib{Interoceptive (Sensibilité)} — Ps. \si{Phol.}
Celle qui reçoit les impressions
venant des surfaces internes de
l'organisme.

\ib{Interprétation (Délire d’)} — \si{Ps. path.}
Délire caractérisé par la constitution d’un système mental fondé sur
l'attribution aux faits réels de significations fausses, sans hallucinations ni affaiblissement intellectuel,
% 103
mais avec tendance aux raisonnements déductifs artificiels.

\ib{Interpsychologie} — \si{Épist.} Étude des
phénomènes « par lesquels s'exerce
l'action, volontaire ou non, d’un
esprit sur un autre esprit » (Dumas)
(cf. Précis, Ph. I, p. 527).

\ib{Intersubjectivité} — \si{Psycho.} Communication* des consciences individuelles (dist. de la mentalité de
groupe et de l’objectivité$^1$ ppt. dite) :
« Toute intersubjectivité demeure
précaire, qui ne se fonde objectivement » (Blanché).

\ib{Introjection} — \si{Ps. an.} ({\it Opp.} : projection*) Phénomène par lequel l’enfant
s’incorpore l'objet perçu.

\ib{}Introspection. — \si{Psycho.} Observation
de la conscience$^1$ par elle-même,
employée méthodiquement et pour
une fin théorique. Introspection
expérimentale : méthode (préconisée
par l’école de Würzbourg) consistant à faire décrire par le sujet ce
qui s’est passé dans sa conscience
au cours d'une expérience$^3$ psychologique.

\ib{Introversion} — \si{Ps. an.} ({\it Opp.} : extraversion*). Chez Jung : orientation
de l'énergie psychique vers la vie
intérieure du sujet : « L’introverti
n’attribue à l’objet qu'une valeur
tout au plus secondaire et médiate. »

\ib{Intuitif} — \si{Psycho.} et \si{Crit.} I {\bf 1.} Qui a
les caractères ou qui est le fruit de
l'intuition : « La pensée doit être
dite intuitive dans la mesure où
elle est devenue pensée immédiate,
pensée d'immédiat$^2$ » (Le Roy); « Il
y a lieu d’écarter la pensée conceptuelle pour parvenir à une philosophie plus intuitive » (Bergson,
{\it P. M.} II); « Les aspects intuitifs
% 103
[des mathématiques] sont liés aux
diverses impressions de nos sens »
(Bouligand).

— \si{Car.} ©. {\bf 2.} (En parlant des
personnes). Qui pense par intuition$^3$ plutôt que par raisonnement :
« J'ai distingué deux sortes d’esprits
mathématiques, les uns logiciens et
analystes, les autres intuitifs et
géomètres » (Poincaré).

— \si{Péd.} {\bf 3.} Méthode intuitive :
celle qui fait appel à l'intuition$^2$
sensible.

\ib{Intuition} — [{\it L.} intueri, voir] — \si{Psycho.},
\si{Épist.}, \si{Méta.} A) O. Connaissance
immédiate* d’un objet de pensée
actuellement présent à l'esprit ({\it opp.} :
discours$^1$). {\it Dist.} : {\bf 1.} intuition empirique comprenant elle-même
a) l'intuition sensible (celle des sens) :
« L'intuition sensible est en mathématique
l'instrument le plus ordinaire de
l'invention » (Poincaré); b) l'intuition psychologique (celle de la conscience) : « L’intuition simple du
moi par le moi » (Bergson, {\it P. M.}
VI) ; — {\bf 2.} intuition rationnelle ({\it p. e.}
celle des axiomes$^1$) : « L’intuition
s'étend d'une part à toutes ces
natures$^2$ [simples], de l’autre à la
connaissance des liaisons nécessaires
qui sont entre elles, enfin à toutes
les autres choses dont l’intellect
constate avec précision l'existence
soit en lui soit en l'imagination »
(Descartes, Reg, XII); « L’intuition du nombre pur » (Poincaré) ; —
 {\bf 3.} intuition inventive ou divinatrice
(celle qui nous fait pressentir la
vérité) : « L’intuition est l’instrument de l'invention » (Poincaré) ; —
 {\bf 4.} intuition métaphysique qui nous
permet de saisir directement, soit
l'absolu d’une substance ({\it p. e.} Descartes dans le cogito*), soit certaines
essences* intemporelles : « Toute
intuition qui nous donne son objet
% 104
de façon immédiate et originelle,
est source de connaissance légitime »
(Husserl), soit enfin la réalité et
{\it spéc.} « notre moi qui dure » : « Nous
appelons intuition la sympathie par
laquelle on se transporte à l’intérieur
d’un objet pour coïncider avec
ce qu'il a d’unique et par conséquent
d’inexprimable » (Bergson, {\it P. M.}
VI).

— B) {\bf 5.} Acte particulier d’intuition
(en l’un quelconque des
sens A), ou objet de cet acte : « Le
raisonnement abstrait est une suite
d’intuitions » (Lagneau).

\ib{Intuitionisme} — \si{Hist.} $\Delta$. {\bf 1.} En psycho.
théorie ({\it p. e.} Hamilton) selon laquelle
la conscience$^1$ saisit immédiatement
le monde extérieur comme tel. —
 {\bf 2.} Doctrine selon laquelle la raison
nous fait connaître par intuition$^2$
des vérités supérieures à l’expérience
({\it p. e.} Écossais, Hamilton). —
 {\bf 3.} Doctrine selon laquelle l'intuition$^4$
nous permet d'atteindre l'absolu
({\it p. e.} Bergson).

\ib{Invention} — \si{Psycho.} O. {\bf 1.} Syn.
d'imagination au sens {\bf 4.} -> On
dist. qqfs. l'invention qui crée du
nouveau, de la découverte qui se
borne à mettre au jour ce qui n’était
pas connu jusque-là. — {\bf 2.} Produit
de l'invention$^1$ : « Ce sont de
pures inventions » ; « Les inventions
mécaniques ».

\ib{Inverse} — \si{Log.} L'inverse d'une proposition
hypothétique est une autre
proposition hypothétique ayant pour
antécédent$^2$ la négation de l’antécédent
de la première et pour conséquent
la négation du conséquent
de la première. Schéma : soit « si P
est vrai, Q est vrai »; l'inverse est :
« si non-P est vrai, non-Q est vrai ».
$->$ Dist, la réciproque avec laquelle
on la confond souvent.

% 104
\ib{Investigation} — \si{Épist.} Recherche
et constatation des faits : « L'investigation,
tantôt simple, tantôt
armée et perfectionnée, est destinée
à nous faire découvrir et constater
les phénomènes qui nous entourent »
(Cl. Bernard).

\ib{Investir} — \si{Ps. an.} Fixer l'intérêt$^2$
sur (une personne ou un objet).

\ib{Involution} — ({\it Opp.} : différenciation*).
\si{Biol.}, \si{Soc.}, \si{Psycho.} Régression du
différencié à l’homogène. -> Spencer
dit aussi, en ce sens, dissolution :
cf. Lalande, La Dissolution opposée
à l’évolution.

\ib{Ipséité} — [{\it L.} ipse, soi-même] — \si{Méta.}
Dans le lang. existentialiste : caractère
du Dasein$^2$ comme sujet de
l’existence quotidienne.

\ib{Irascible} — Voir Appétit$^1$.

\ib{Ironie} — ([G. eirôneia, interrogation] —
\si{Hist.} Méthode par laquelle Socrate
interrogeait ses auditeurs pour les
amener à trouver la vérité.

\ib{Irrationalisme} — \si{Méta.} $\Delta$. Doctrine
qui privilégie l’irrationnel$^1$ : « L’immoralisme$^1$,
c’est l’irrationalisme de
l’action » (Parodi); « L’histoire des
sciences est l’histoire des défaites
de l’irrationalisme » (Bachelard).

\ib{Irrationnel} — \si{Méta.} {\bf 1.} Contraire à la
raison, absurde$^1$ : « Cet élan [du
vouloir-vivre] où l’homme cherche
à s’oublier, nous égare dans l’irrationnel,
dans l’absurde » (Jaspers).
— {\bf 2.} Irréductible à la raison : a) soit
en nous : « rrationnelle en un sens,
la foi est par là même affective et
active » (Delacroix); b) soit au
dehors : « Nous nous servirons du
terme irrationnel. Il a l'avantage de
marquer qu'il s’agit d’un fait que
nous estimons certain, mais qui reste
et restera irréductible à des éléments
% 105
purement rationnels » (Meyerson).

— \si{Math.} {\bf 3.} Nombre irrationnel
(autref. ineffable). Ctr. de rationnel$^4$ :
{\it p. e.} le nombre $\pi$.

\ib{Isotrope} — \si{Épist.} Semblable à lui
même en toutes les directions
« L'espace euclidien est isotrope ».

	\end{itemize}
