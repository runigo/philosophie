
	\begin{itemize}[leftmargin=1cm, label=\ding{32}, itemsep=1pt]
\item {\bf G (Facteur)} — Ps. métr. Facteur*
qui, selon Spearman, s'ajoute, dans
le profil psychologique de l'individu,
aux facteurs spéciaux (sensibilité,
mémoire, abstraction, etc.) et constitue un facteur général d’intelligence.

\item {\bf Gazouillis} — Psycho. Langage spontané de l'enfant, fait d’onomatopées
et d’imitations de bruits.

\item {\bf Général} — Vulg. 1. Qui s’applique
à la plupart des cas. $->$ Ne jamais
employer ce mot, comme on le fait
dans le langage courant, au sens
de : vague, indéterminé. Tout au ctr. :
« L'idée générale, c’est l’idée définie »
(Burloud).

— Log. form. 2. Str. (Opp. :
singulier* et collectif*). En parlant
des termes : qui désigne une classe$^1$ :
« chien » est un terme général. —
3. Lato. (Opp. : spécial). Plus général : a) en parlant d’un concept ou
d’un terme : dont l'extension$^2$ est
plus grande : « animal » est « plus
% 83
général » que « chien » ; b) en parlant d'une proposition : qui a pour
sujet un terme plus général : « L’animal a des instincts » est « plus générale » que « Le chien a des instincts ».
— 4. Latiss. En parlant des propositions : syn. de universel$^3$. Spéc. en
parlant des lois$^5$ : universellement
valable, indépendamment de l’espace
et du temps. — 5. Cf. Volonté$^5$.

\item {\bf Généralisation} — Psycho. et Log.
©. Opération intellectuelle : 1. par
laquelle un ensemble de propriétés
ou de caractères est pensé comme le
type de toute une classe$^1$ d'êtres,
objets ou faits (cf. Général$^2$); — 2.
par laquelle on passe de propositions spéciales à de plus générales$^3$.
— ©. 3. Produit de cette opération :
« La vraie science craint les généralisations hâtives » (Poincaré).

\item {\bf Généralité} — Log. et Épist. O. 1. Caractère général : « La généralité du
concept ». — @. 2. Généralisation$^3$ :
« Les généralités ne sont pas philosophiques » (Bergson).

\item {\bf Générique} — Log. 1. (Opp. : spécial
ou spécifique). Appartenant au
genre$^1$ tout entier.

— Psycho. 2. Image générique :
celle qui se forme par fusion des
images individuelles des différents
objets d’un même genre$^1$.

\item {\bf }Genèse [G. genesis, devenir] — Développement graduel d’un être,
d’une idée, d’une institution, d’un
type.

\item {\bf Génétique} — (Adj.) Épist. Qui retrace :
1. la genèse réelle : « Méthode génétique », « Classification génétique »;
— 2. la genèse logique (d’une idée).
Définition génétique (ou par génération) : celle qui définit une notion
en montrant comment elle se construit logiquement (vg. la plupart
% 84
des définitions mathématiques). — 3. Épistémologie génétique (Piaget) : méthode
épistémologique qui consiste « à
étudier les connaissances en fonction de leur construction réelle ou
psychologique et à considérer toute
connaissance comme relative à un
certain niveau du mécanisme de
cette construction ». Logique génétique : voir Logiquef.

— Psycho. 4. Théories génétiques : celles qui admettent qu’un
sentiment, une idée, etc., se sont
formés graduellement (opp. th. nativistes qui les considèrent comme
innés ou immédiats). Not., pour la
perception de l'étendue, la théorie
génétique (qqfs. appelée, mais improprement : empiriste) est celle qui
admet que cette perception est
acquise progressivement (la th. nativiste la considérant au ctr. comme
donnée dans la sensation elle-même).

\item {\bf Génétique (nom)} — Biol. 5. La science
de l’hérédité.

\item {\bf Genre} — Log. form. 1. Voir Espèce$^2$.
Genre prochain : celui qui, dans la
hiérarchie des termes en extension$^3$,
est immédiatement supérieur à l’espèce$^2$ considérée : vg. « vertébré »,
genre prochain de « mammifère ».
Genre suprême : celui qui englobe
tous les autres.

— Biol. 2. Groupe morphologique intermédiaire entre la famille$^3$
et l'espèce$^3$.

\item {\bf Gens (Droit des)} [L. gentes, nations].
— Jur. Droit international public.

\item {\bf Géocentrique} — Épist. Qui place la
terre au centre du monde.

\item {\bf Géographie} — Épist. Étude descriptive et « explicative, i. e. raisonnée,
scientifique » (Sorre), du globe terrestre du point de vue physique,
% 84
ethnographique, politique et économique.

\item {\bf Géologie} — Épist. Étude du globe terrestre considéré dans sa genèse*,
du seul point de vue physique.

\item {\bf Géométrie} — 1. Autref. les mathématiques en général : « Ces longues
chaînes de raisons, dont les géomètres [= les mathématiciens) ont
coutume de se servir » (Descartes,
Méth., II). Chez Pascal, esprit de
géométrie (opp. : de finesse*)
esprit de déduction et de rigueur
logique qui suit un principe jusqu'en ses conséquences les plus
éloignées. — 2. Auj. science mathématique de l’espace. Géom. analylique : science qui exprime les
grandeurs géométriques en formules
algébriques. Géom. descriptive
application de la géométrie à la représentation des figures par leurs
projections sur un plan. Géom. de
position : voir Topologie*.

\item {\bf Gestalt} [mot allemand]. — Voir Forme$^4$.

\item {\bf Gestaltisme} — Psycho. Théorie de la
Gestalt*.

\item {\bf Gnomique} [G. gnôémé]. — Hist. Qui
parle par sentences : « Les poètes
gnomiques » (vg. Théognis).

\item {\bf Gnose} [G. gnôsis, connaissance]. —
Hist. Mode de connaissance prétendu supérieur pratiqué par les
Gnostiques* et certains mystiques :
« On a introduit une fausse gnose à
la place de la véritable » (Bossuet).

\item {\bf Gnoséologie} [G. gnôsis, et logos, étude].
— Crit. Nom qqfs. donné à la critique$^1$ de la connaissance.

\item {\bf Gnostiques} — Hist. Hérétiques du
{\footnotesize II}$^\text{e}$ siècle qui pratiquaient la gnose*
et professaient la doctrine de l’émanation$^1$ et une sorte de théosophie*.
% 85

\item {\bf Gouvernement} — Soc. Autorité qui
détient le pouvoir exécutif* et qui
l'exerce, ou en son propre nom,
ou au nom de qq. autorité supérieure (droit divin), ou, en démocratie, au nom de la nation$^2$.

\item {\bf Grâce} — Théol. 1. Participation de
l’homme à la vie divine avant le
péché. — 2. Secours surnaturel et
gratuit par lequel Dieu nous aide à
faire le bien. Voir Actuel$^4$.

— Esth. 3. Qualité esthétique du
gracieux, i. e. de ce qui présente une
certaine aisance dans le mouvement
ou le rythme (voir Précis, Ph. I,
p- 562) : « La grâce d’un Boucher est
plus facilement descriptible que
celle d’un Watteau » (Bayer).

\item {\bf Grandeur} — Épist. 1. Tout ce qui est
susceptible de plus et de moins :
« Toute recherche mathématique à
pour objet de déterminer des grandeurs inconnues d'après les relations qui existent entre elles et des
grandeurs connues » (Comte). Cf.
Extensif*. $->$ Dist. quantité*.

— Ps. path. 2. Folie ou Délire
des grandeurs (syn. : Mégalomanie) :
délire caractérisé par des sentiments de puissance et d’euphorie*;
le sujet se croit milliardaire, roi,
empereur, Dieu, etc.

\item {\bf Grand terme} — Log. form. Dans en
syllogisme (syn. : majeur$^1$) attribut
de la conclusion$^3$ (parce qu'il a
gén. la plus grande extension*).

\item {\bf Granulaire (Théorie)} — Phys. Celle
selon laquelle la matière ou l’énergie
est composée de grains : « Dans la
science moderne, l'élément simple
et indivisible, c’est le grain de matière ou de lumière, neutron, électron ou photon » (L. de Broglie).

\item {\bf Graphique (Méthode)} — Épist. Celle
qui consiste à représenter les phénomènes
%85
ou leurs relations abstraites
par des figures géométriques (diagrammes, échelles, courbes), soit
pour présenter à l'œil un tableau
schématique plus saisissant, soit
pour mettre en évidence les relations constantes entre les faits.

\item {\bf Gratuit} — Sans justification : « Une
supposition gratuite », « L'on est
bien éloigné de recevoir des principes gratuits » [en parlant des
axiomes$^3$] (Leibniz, N. E., IV, 12, 6).
Ext., « acte gratuit », sans motif :
« Commettre gratuitement le crime,
commettre un crime parfaitement
immotivé » (A. Gide).

\item {\bf Grégaire} [L. grex, gregis, troupeau].
— 1. Qui vit en troupe : « Animaux
grégaires ». — 2. Se dit d’un état
social temporaire et diffus* (opp.
vie sociale stable et organisée)

« État grégaire ». — 3. Psycho. Tendance grégaire : tendance de certains êtres vivants à se rassembler
et à vivre ensemble.

\item {\bf Groupe} — Math. 1. Système de termes
dont chacun se tire du précédent
selon une loi définie (vg. suite des
nombres entiers) : « C’est Galois qui
a inventé la Théorie des Groupes
pour généraliser la résolution des
équations algébriques. » — Soc.
2. Ensemble de personnes formant
un tout* (v. ce mot) en ce sens
qu’elles participent aux mêmes sentiments, représentations et jugements de valeur et présentent les
mêmes types de comportement
« Les groupes sociaux sont l’objet
propre de la sociologie ». $->$ Un
groupe$^2$ n'est pas nécessairement
organisé* : il peut être diffus* ou
virtuel* (vg. un public$^2$).

\item {\bf Groupement} — Log. 1. Système d'opérations logiques fermé et réversible,
%86
vg. une classification, une table à
double entrée (Piaget).

— Soc. 2. Groupe$^2$
volontairement constitué, association$^4$.

\begin{center}
H
\end{center}

\item {\bf Habitude} — Biol. et Psycho. Manière
d’être permanente, contractée par
un être vivant à l'égard d’une influence ou d’un acte et qui fait que
cette influence ou cet acte n’exigent
plus de lui, pour la supporter ou
accomplir, le même effort qu’auparavant.

\item {\bf Hallucination} — Ps. path. Image$^3$
prise pour une perception réelle,
Halluc. négative : absence anormale
de perception d’un objet présent.

\item {\bf Haphi-esthésimètre} [G. haphé, toucher]. — Ps. phys. Instrument servant à mesurer l’acuité* tactile.

\item {\bf Harmonie préétablie} — Hist. Chez
Leibniz (Mon., 56 et 78) : accord
établi par Dieu entre les substances
créées et qui explique la concordance
de leurs perceptions sans influence
sur elles d’une substance corporelle
et sans action réciproque de ces
substances les unes sur les autres.

\item {\bf Hasard} — Vulg. Ce qui n’est pas
prévisible : 1. soit qu’on suppose
dans les choses une indétermination$^2$
radicale ; — 2. soit qu'il s'agisse
d'événements si complexes (cf. Fortuit*) qu’on ne puisse en connaître
toutes les conditions : « Il n’y a pas
incompatibilité entre le rôle de ce
que nous appelons le hasard et l’établissement de lois scientifiques »
(Borel) ; — 3. soit qu’on ignore le
déterminisme$^1$ du phénomène; —
4. soit que, se plaçant au point de
vue de la finalité*, on n’en aperçoive
% 86
pas les raisons d’être : « Ce
qui est hasard à l’égard des hommes
est dessein à l'égard de Dieu » (Bossuet). $->$ Terme très équivoque.

\item {\bf Hébéphrénie} [G. hêbê, adolescence, et
phrên, esprit]. — Ps. path. (Syn. :
démence précoce). Psychose qui apparaît gén. de 10 à 25 ans et qui consiste en « une destruction de la cohésion intime de la personnalité, avec
lésions prédominantes de l’affectivité et de la volonté » (Dumas),
qqsf. avec hallucinations ou délire
métaphysique (vg. le Louis Lambert
de Balzac).

\item {\bf Hédonisme} [G. hêdonê, plaisir]. —
Mor. À. Doctrine morale selon laquelle le plaisir est le souverain
bien (Aristippe de Cyrène, Épicure).

\item {\bf Héraclitéisme} — Hist. A.1. Doctrine du
philosophe grec Héraclite. — Ext. 2.
Toute doctrine qui privilégie le
devenir$^2$ et la mobilité des choses :
«  L’héraclitéisme bergsonien »
(Bayer).

\item {\bf Hérédité} — Biol. Transmission des
caractères génériques, spécifiques et
individuels des êtres vivants à leurs
descendants.

\item {\bf Herméneutique} [G. herméneuein, interpréter]. — Épist. Interprétation des
textes anciens, spéc. des textes
bibliques.

\item {\bf Hétérogène} — (Ctr. homogène*).
Composé d’éléments de nature différente.

\item {\bf Hétéronomie} — Mor. (Ctr. autonomie*).
État de la volonté qui reçoit passivement sa loi d’une autorité extérieure ou d’une impulsion étrangère
à la raison.

\item {\bf Heuristique} [G. heuriskein, trouver].
— A) Adjectif : 1. Épist. Qui sert à
% 87
la découverte : « Hypothèse heuristique ». — 2. Péd. Méthode heuristique (opp. dogmatique$^4$) : celle qui
fait découvrir ce qu’on enseigne, par
l'élève lui-même, — B) Nom : 3.
Épist. En histoire, la recherche des
documents.

\item {\bf Hiérarchie} — Ordre de dépendance
des personnes, des idées ou des phénomènes, établi soit d’un point de
vue normatif* (religieux, juridique,
moral, logique, etc.), soit du point
de vue d’une simple dépendance
naturelle (hiérarchie des fonctions
physiologiques en Biol). — Épist.
Hiérarchie des sciences : classification des sciences fondée sur « l’enchaînement naturel » (Comte) et
l’ordre de complexité des phénomènes qu'elles étudient.

\item {\bf Histoire} — 1. @ [Cf. all. : Geschichte]
Ensemble des états par lesquels
passe un être qui change : « Chacun
de nous a une histoire qui nous a
faits ce que nous sommes » (Berger);
« L'esprit est histoire » (Blanché).
— O. Épist. [Cf. all. : Historie]. 2.
Lato. Toute étude à caractère descriptif : « L'histoire naturelle ». —
3. Str. Étude chronologique des
faits sociaux considérés dans leurs
particularités de temps et de lieu :
« Pour bien écrire l’histoire, il faut
être dans un pays libre » (Voltaire).

\item {\bf Histologie} [G. histos, tissu]. — (Syn. :
anatomie fine). Épist. Partie de
l'anatomie qui a pour objet l'étude
des tissus.

\item {\bf Historicité} — 1. M. Épist. Caractère de
réalité historique : « L’historicité de
Jésus ». — 2. ©. Méta. Dans le lang.
existentialiste : condition de l’existant humain qui, tout en étant
engagé dans le temps et solidaire
de son passé, s’en détache en se
% 
situant par rapport à lui et se
projette librement vers l'avenir.

\item {\bf Holisme} [G. holos, total]. — A. Méta.
1. Terme inventé en 1926 par
J. C. Smuts pour désigner la tendance de l’univers à construire des
unités formant un tout et de complication croissante. — D'où : Biol.
2. Doctrine qui considère l’organisme vivant comme un tout indécomposable.

\item {\bf Homaloïdal} [G. homalos, plan]. —
Math. Sans courbure; où l’on peut,
par suite, tracer des figures semblables à n'importe quelle échelle :
« L'espace euclidien est homaloïdal».

\item {\bf Homogène} — 1. Dont toutes les
parties sont de même nature.

— Math. 2. Un milieu est dit
homogène quand on peut y déplacer
une figure sans déformation (milieu
sans courbure [homaloïdal*] ou
à courbure constante [vg. surface
sphérique]).

\item {\bf Hormè} — Ps. an. Terme grec qqfs.
employé comme syn. de pulsion*
ou d’instinct au sens 3.

\item {\bf Humaniser} — 1. Laud. Rendre humain,
i. e. bon et doux : « Humaniser
les mœurs de la nation » (Voltaire).
— 2. Péj. Réduire aux dimensions
de l’homme : « Les hommes humanisent toutes choses» [et même
Dieu] (Malebranche, Entr., VIII, 9).

\item {\bf Humanisme} — Hist. 1. Mouvement
des « humanistes » de la Renaissance (Erasme, Budé) : « L’humanisme n’est pas seulement le goût
de l’antiquité : il en est le culte »
(Ph. Monnier).

— Crit. 2. À. Forme de pragmatisme* professée par F. C. S, Schiller
(d'Oxford) et selon laquelle toute
% 88
connaissance est subordonnée aux
conditions de l'expérience humaine.

— Mor. 3. À. Doctrine qui ne
reconnaît aucune valeur supérieure
à l'être humain (vg. Feuerbach,
Nietzsche) : « L’humanisme n’a de
sens que s’il est possible de fixer
l'humanité dans une nature ne
varietur » (Le Senne). — 4. « Intérêt
majeur témoigné au problème de
l’homme, de sa nature, de son origine, de sa destinée, de sa situation
dans le monde » (A. Etcheverry).
$->$ Terme équivoque.

\item {\bf Hylè} [mot grec = matière]. — Méta.
Chez Husserl : matière$^2$ de la sensation considérée comme pur donné,
indépendamment de son sens intentionnel*,

\item {\bf Hylètique} — (Adj.) [Opp. : éïdétique$^1$].
1. Qui concerne la hylè* : « Les vécus
hylétiques » — (Nom fém.). 2.
Théorie de la hylè* : « L’hylètique
pure se subordonne à la phénoménologie* de la conscience transcendantale » (Husserl).

\item {\bf Hylèmorphisme} [G. hylê, et morphê,
forme]. — Hist. À. Doctrine (d'Aristote et des Scolastiques) selon laquelle les corps$^1$ sont à la fois matière$^1$ (d’où leurs propriétés quantitatives) et forme$^1$ [d’où leurs propriétés qualitatives).

\item {\bf Hylozoïsme} [G. hylê, et zôon, être
vivant]. — Hist. A. Conception
selon laquelle la matière même et
l'univers sont doués de vie (v. Précis,
Ph. II, p. 482-483).

\item {\bf Hyperbolique (Doute)} — Hist. Chez
Descartes : le doute provisoire : « Ce
doute général et universel que j'ai
moi-même appelé hyperbolique et
métaphysique et duquel j'ai dit
qu'il ne fallait point se servir pour...
la conduite de la vie » (7$^\text{es}$ Rép.).

\item {\bf Hyperespace} — Math. Espace diftérent de l’espace euclidien*.

\item {\bf Hyperesthésie} — Ps. path. Augmentation anormale de la sensibilité$^4$.

\item {\bf Hypermnésie} — Ps, path. Surexcitation morbide de la reviviscence
des images (v. Précis, Ph I,
p. 226).

\item {\bf Hypnagogiques (États)} [G. hypnos,
sommeil, et agôgé, conduite]. —
Psycho. États psychiques intermédiaires entre ceux de la veille et
ceux du sommeil (ceux du réveil
sont qqfs. appelés états hypnopompiques).

\item {\bf Hypnose} — Ps. path. (Syn. : sommeil
hypnotique). Somnambulisme* artificiellement provoqué ou qui se produit en dehors du sommeil normal.

\item {\bf Hypnotisme} — Psycho. Ensemble
des moyens par lesquels on provoque l’hypnose*,

\item {\bf Hypostase} — Méta. Nom grec de la
substance*; d’où : 1. Chez Plotin :
l'Un, l'Intelligence et l'Ame : « L'hypostase naît quand la puissance
émanée de l'Un se recueille en
quelque sorte sur elle-même et se
fixe » (Bréhier); — 2. Chez les auteurs chrétiens : les trois personnes
de la Trinité. Plus gén., chez saint
Thomas : « Les substances individuelles sont appelées hypostases
ou substances premières » (S. Th.,
I, 29).

\item {\bf Hypostasier} — Ériger à l'état de sub
stance : « La réalité métaphysique,
telle que Plotin la conçoit, est la
vie spirituelle hypostasiée » (Bréhier).
Souvent péj., ériger à l’état d’entité$^3$ :
« Il n’est pas nécessaire d’hypostasier la conscience collective » (Durkheim).
% 89

\item {\bf Hypostatique (Union)} — Theol. Celle
de la nature divine et de la nature
humaine dans la personne du Christ.

\item {\bf Hypothèse} — Vulg. 1. Lato. Supposition ; proposition admise sans
égard à sa vérité ou à sa fausseté et
sans intention de la soumettre à
vérification. $->$ Dire plutôt en ce
sens conjecture.

— Épist. 2. En Math. (Opp. : conclusion) : données$^1$ d’un problème,
ou proposition admise comme point
de départ dans un théorème : « On
construit, la conséquence* avec
l'hypothèse » (Goblot). — 3. Dans
les Sc. expérimentales : proposition
admise provisoirement et destinée
à être soumise au contrôle de l’expérience : « Une idée anticipée ou une
hypothèse est le point de départ
nécessaire de tout raisonnement
expérimental » (Cl. Bernard). Grandes hypothèses ou hypothèses générales : syn. de théories$^3$ : « L’hypothèse doit toujours être soumise à la
vérification » (Poincaré).

— Théol. 4. (Opp. : thèse). Application d’une vérité générale à des
circonstances particulières et plus
ou moins contingentes.

\item {\bf Hypothétique} — Épist. Conditionnel*,
Proposition hypothétique (opp. catégorique et disjonctive) celle où
l’assertion est subordonnée à une
condition$^2$ (vg. : « Si deux droites sont
parallèles, elles sont équidistantes »).
Syllogisme hypothétique : celui dont
une prémisse est hypothétique.

\item {\bf Hystérie} — Ps. path. Névrose* qui
se manifeste psychologiquement par
« le rétrécissement du champ$^3$ de la
conscience personnelle et la tendance à la dissociation des fonctions qui par leur synthèse$^3$ constituent la personnalité » (Janet). Voir
Pithiatique* et Suggestibilité*.

	\end{itemize}
