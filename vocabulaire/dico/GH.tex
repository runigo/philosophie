
	\begin{itemize}[leftmargin=1cm, label=\ding{32}, itemsep=1pt]

\ib{G (Facteur)} — \si{Ps. métr.} Facteur* qui, selon Spearman, s'ajoute,
dans le profil psychologique de l'individu, aux facteurs spéciaux
(sensibilité, mémoire, abstraction, etc.) et constitue un facteur général
d’intelligence.

\ib{Gazouillis} — \si{Psycho.} Langage spontané de l'enfant, fait
d’onomatopées et d’imitations de bruits.

\ib{Général} — \si{Vulg.} {\bf 1.} Qui s’applique à la plupart des cas.
$->$ Ne jamais employer ce mot, comme on le fait dans le langage courant,
au sens de : {\it vague, indéterminé}. Tout au ctr. : « L'idée générale,
c’est l’idée définie » (Burloud).

— \si{Log.} \si{form.} {\bf 2.} {\it Str.} (Opp. : {\it singulier}* et
{\it collectif}*). En parlant des termes : qui désigne une {\it classe}$^1$ :
« chien » est un terme général. — {\bf 3.} {\it Lato.} (Opp. :
{\it spécial}). Plus {\it général} : a) en parlant d’un concept ou d’un
terme : dont l'extension$^2$ est plus grande : « animal » est « plus
% 83
général » que « chien » ; b) en parlant d'une proposition : qui a pour
sujet un terme plus général : « L’animal a des instincts » est « plus
générale » que « Le chien a des instincts ». — {\bf 4.} {\it Latiss.} En
parlant des propositions : syn. de {\it universel}$^3$. {\it Spéc.} en
parlant des {\it lois}$^5$ : universellement valable, indépendamment de
l’espace et du temps. — {\bf 5.} Cf. {\it Volonté}$^5$.

\ib{Généralisation} — \si{Psycho.} et \si{Log.} \fsb{S. abstr.}. Opération
intellectuelle : {\bf 1.} par laquelle un ensemble de propriétés ou de
caractères est pensé comme le type de toute une classe$^1$ d'êtres, objets ou
faits (cf. Général$^2$); — {\bf 2.} par laquelle on passe de propositions
spéciales à de plus générales$^3$. — \fsb{S. concr.} {\bf 3.} Produit de
cette opération : « La vraie science craint les généralisations hâtives »
(Poincaré).

\ib{Généralité} — \si{Log.} et \si{Épist.} \fsb{S. abstr.} {\bf 1.} Caractère
général : « La généralité du concept ». — \fsb{S. concr.} {\bf 2.}
Généralisation$^3$ : « Les généralités ne sont pas philosophiques » (Bergson).

\ib{Générique} — \si{Log.} {\bf 1.} (Opp. : {\it spécial}
ou {\it spécifique}). Appartenant au genre$^1$ tout entier.

— \si{Psycho.} {\bf 2.} {\it Image générique} : celle qui se forme par fusion
des images individuelles des différents objets d’un même genre$^1$.

\ib{Générosité} — \si{Mor.} {\bf 1.} {\it Chez Descartes} : sorte de grandeur
d’âme qui est, selon lui, « la clef de toutes les autres vertus » et qui fait
que l’homme « s’estime au plus haut point » en tant qu'il possède la « libre
disposition de ses volontés » et qu'il « sent en soi-même une ferme et
constante résolution d'en bien user » ({\it Passions}, art. 153). —
{\bf 2.} \si{Vulg.} Disposition à donner avec libéralité.

\ib{}Genèse [G. {\it genesis}, devenir] — Développement graduel d’un être,
d’une idée, d’une institution, d’un type.

\ib{Génétique} — ({\it Adj.}) \si{Épist.} Qui retrace : 1. la genèse réelle :
« Méthode génétique », « Classification génétique » ; — {\bf 2.} la genèse
logique (d’une idée). {\it Définition génétique} (ou {\it par génération}) :
celle qui définit une notion en montrant comment elle se construit
logiquement ({\it p. e.} la plupart
% 84
des définitions mathématiques). — {\bf 3.} {\it Épistémologie génétique}
(Piaget) : méthode épistémologique qui consiste « à étudier les connaissances
en fonction de leur construction réelle ou psychologique et à considérer
toute connaissance comme relative à un certain niveau du mécanisme de cette
construction ». {\it Logique génétique} : voir {\it Logique}$^4$.

— \si{Psycho.} {\bf 4.} {\it Théories génétiques} : celles qui admettent
qu’un sentiment, une idée, etc., se sont formés graduellement (opp.
{\it th. nativistes qui les considèrent comme innés ou immédiats}). Not.,
pour la perception de l'étendue, la théorie {\it génétique} (qqfs. appelée,
mais improprement : {\it empiriste}) est celle qui admet que cette perception
est acquise progressivement (la th. {\it nativiste} la considérant au ctr.
comme donnée dans la sensation elle-même).

\ib{Génétique (nom)} — \si{Biol.} {\bf 5.} La science de l’hérédité.

\ib{Genre} — \si{Log.} \si{form.} {\bf 1.} Voir {\it Espèce}$^2$.
{\it Genre prochain} : celui qui, dans la hiérarchie des termes en
extension$^3$, est immédiatement supérieur à l’espèce$^2$ considérée :
{\it p. e.} « vertébré », genre prochain de « mammifère ».
{\it Genre suprême} : celui qui englobe tous les autres.

— \si{Biol.} {\bf 2.} Groupe morphologique intermédiaire entre la famille$^3$
et l'espèce$^3$.

\ib{Gens (Droit des)} [{\it L.} {\it gentes}, nations].
— \si{Jur.} Droit international public.

\ib{Géocentrique} — \si{Épist.} Qui place la terre au centre du monde.

\ib{Géographie} — \si{Épist.} Étude descriptive et « explicative, {\it i. e.}
raisonnée, scientifique » (Sorre), du globe terrestre du point de vue
physique, ethnographique, politique et économique.
% 84

\ib{Géologie} — \si{Épist.} Étude du globe terrestre considéré dans sa
genèse*, du seul point de vue physique.

\ib{Géométrie} — {\bf 1.} {\it Autref.} les mathématiques en général : « Ces
longues chaînes de raisons, dont les géomètres [= les mathématiciens) ont
coutume de se servir » (Descartes, {\it Méth.}, II). Chez Pascal,
{\it esprit de géométrie} (opp. : de {\it finesse}*) esprit de déduction et
de rigueur logique qui suit un principe jusqu'en ses conséquences les plus
éloignées. — {\bf 2.} {\it Auj.} science mathématique de l’espace.
{\it Géom. analylique} : science qui exprime les grandeurs géométriques en
formules algébriques. {\it Géom. descriptive} : application de la géométrie à
la représentation des figures par leurs projections sur un plan.
{\it Géom. de position} : voir {\it Topologie}*.

\ib{Gestalt} [mot allemand]. — Voir Forme$^4$.

\ib{Gestaltisme} — \si{Psycho.} Théorie de la {\it Gestalt}*.

\ib{Gnomique} [G. {\it gnômê}]. — \si{Hist.} Qui parle par sentences : « Les
poètes gnomiques » ({\it p. e.} Théognis).

\ib{Gnose} [G. {\it gnôsis}, connaissance]. — \si{Hist.} Mode de connaissance
prétendu supérieur pratiqué par les Gnostiques* et certains mystiques : « On
a introduit une fausse gnose à la place de la véritable » (Bossuet).

\ib{Gnoséologie} [G. {\it gnôsis}, et {\it logos}, étude].
— \si{Crit.} Nom qqfs. donné à la {\it critique$^1$ de la connaissance}.

\ib{Gnostiques} — \si{Hist.} Hérétiques du
{\footnotesize II}$^\text{e}$ siècle qui pratiquaient la gnose*
et professaient la doctrine de l’émanation$^1$ et une sorte de théosophie*.
% 85

\ib{Gouvernement} — \si{Soc.} Autorité qui détient le pouvoir exécutif* et
qui l'exerce, ou en son propre nom, ou au nom de qq. autorité supérieure
(droit divin), ou, en démocratie, au nom de la nation$^2$.

\ib{Grâce} — \si{Théol.} {\bf 1.} Participation de l’homme à la vie divine
avant le péché. — {\bf 2.} Secours surnaturel et gratuit par lequel Dieu nous
aide à faire le bien. Voir {\it Actuel}$^4$.

— \si{Esth.} {\bf 3.} Qualité esthétique du {\it gracieux}, i. e. de ce qui
présente une certaine aisance dans le mouvement ou le rythme (voir
{\it Précis}, Ph. I, p- 562) : « La grâce d’un Boucher est plus facilement
descriptible que celle d’un Watteau » (Bayer).

\ib{Grandeur} — \si{Épist.} {\bf 1.} Tout ce qui est susceptible de
{\it plus} et de {\it moins} : « Toute recherche mathématique à pour objet de
déterminer des grandeurs inconnues d'après les relations qui existent entre
elles et des grandeurs connues » (Comte). Cf. {\it Extensif}*.
$->$ {\it Dist.} quantité*.

— \si{Ps. path.} {\bf 2.} Folie ou {\it Délire des grandeurs} (syn. :
{\it Mégalomanie}) : délire caractérisé par des sentiments de puissance et
d’euphorie* ; le sujet se croit milliardaire, roi, empereur, Dieu, etc.

\ib{Grand terme} — \si{Log.} \si{form.} Dans en syllogisme (syn. :
{\it majeur}$^1$) attribut de la conclusion$^3$ (parce qu'il a {\it gén.}
la plus grande extension*).

\ib{Granulaire (Théorie)} — \si{Phys.} Celle selon laquelle la matière ou
l’énergie est composée de {\it grains} : « Dans la science moderne,
l'élément simple et indivisible, c’est le grain de matière ou de lumière,
neutron, électron ou photon » ({\it L.} de Broglie).

\ib{Graphique (Méthode)} — \si{Épist.} Celle qui consiste à représenter les phénomènes ou leurs relations abstraites par des figures géométriques 
%85
(diagrammes, échelles, courbes), soit pour présenter à l'œil un tableau
schématique plus saisissant, soit pour mettre en évidence les relations
constantes entre les faits.

\ib{Gratuit} — Sans justification : « Une supposition gratuite », « L'on est
bien éloigné de recevoir des principes gratuits » [en parlant des axiomes$^3$]
(Leibniz, {\it N. E.}, IV, 12, 6). {\it Ext.}, « acte gratuit », sans motif :
« Commettre gratuitement le crime, commettre un crime parfaitement
immotivé » (A. Gide).

\ib{Grégaire} [L. {\it grex, gregis,} troupeau]. — {\bf 1.} Qui vit en
troupe : « Animaux grégaires ». — {\bf 2.} Se dit d’un état social temporaire
et diffus* ({\it opp.} vie sociale stable et organisée) : « État grégaire ».
— {\bf 3.} \si{Psycho.} {\it Tendance grégaire} : tendance de certains êtres
vivants à se rassembler et à vivre ensemble.

\ib{Groupe} — \si{Math.} {\bf 1.} Système de termes dont chacun se tire du
précédent selon une loi définie ({\it p. e.} suite des nombres entiers) :
« C’est Galois qui a inventé la Théorie des Groupes pour généraliser la
résolution des équations algébriques. » — \si{Soc.} {\bf 2.} Ensemble de
personnes formant un tout* (v. ce mot) en ce sens qu’elles participent aux
mêmes sentiments, représentations et jugements de valeur et présentent les
mêmes types de comportement « Les groupes sociaux sont l’objet propre de la
sociologie ». $->$ Un groupe$^2$ n'est pas nécessairement {\it organisé}* :
il peut être {\it diffus}* ou {\it virtuel}* (p. e. un {\it public}$^2$).

\ib{Groupement} — \si{Log.} {\bf 1.} Système d'opérations logiques fermé
et réversible, {\it p. e.} une classification, une table à double entrée (Piaget).
%86

— \si{Soc.} {\bf 2.} Groupe$^2$ volontairement constitué, association$^4$.

\begin{center}
\huge{H}
\end{center}

\ib{Habitude} — \si{Biol.} et \si{Psycho.} Manière d’être permanente,
contractée par un être vivant à l'égard d’une influence ou d’un acte et qui
fait que cette influence ou cet acte n’exigent plus de lui, pour la supporter
ou accomplir, le même effort qu’auparavant.

\ib{Hallucination} — \si{Ps. path.} Image$^3$ prise pour une perception
réelle, {\it Halluc. négative} : absence anormale de perception d’un objet
présent.

\ib{Haphi-esthésimètre} [G. {\it haphé}, toucher]. — \si{Ps. phys.}
Instrument servant à mesurer l’acuité* tactile.

\ib{Harmonie préétablie} — \si{Hist.} {\it Chez Leibniz} ({\it Mon.}, 56 et
78) : accord établi par Dieu entre les substances créées et qui explique la
concordance de leurs perceptions sans influence sur elles d’une substance
corporelle et sans action réciproque de ces substances les unes sur les
autres.

\ib{Hasard} — \si{Vulg.} Ce qui n’est pas prévisible : {\bf 1.} soit qu’on
suppose dans les choses une indétermination$^2$ radicale ; — {\bf 2.} soit
qu'il s'agisse d'événements si complexes (cf. {\it Fortuit}*) qu’on ne puisse
en connaître toutes les conditions : « Il n’y a pas incompatibilité entre le
rôle de ce que nous appelons le hasard et l’établissement de lois
scientifiques » (Borel) ; — {\bf 3.} soit qu’on ignore le déterminisme$^1$ du
phénomène ; — {\bf 4.} soit que, se plaçant au point de vue de la finalité*,
on n’en aperçoive
% 86
pas les raisons d’être : « Ce qui est hasard à l’égard des hommes
est dessein à l'égard de Dieu » (Bossuet). $->$ Terme très équivoque.

\ib{Hébéphrénie} [G. {\it hêbê}, adolescence, et {\it phrên}, esprit].
— \si{Ps. path.} (Syn. : {\it démence précoce}). Psychose qui apparaît gén.
de 10 à 25 ans et qui consiste en « une destruction de la cohésion intime de
la personnalité, avec lésions prédominantes de l’affectivité et de la
volonté » (Dumas), qqsf. avec hallucinations ou délire métaphysique
({\it p. e.} le {\it Louis Lambert} de Balzac).

\ib{Hédonisme} [G. {\it hêdonê}, plaisir]. — \si{Mor.} \fsb{S. norma.}
Doctrine morale selon laquelle le plaisir est le souverain bien
(Aristippe de Cyrène, Épicure).

\ib{Héraclitéisme} — \si{Hist.} \fsb{S. norma.}1. Doctrine du philosophe grec
Héraclite. — {\it Ext.} {\bf 2.} Toute doctrine qui privilégie le devenir$^2$
et la mobilité des choses : « L’héraclitéisme bergsonien » (Bayer).

\ib{Hérédité} — \si{Biol.} Transmission des caractères génériques,
spécifiques et individuels des êtres vivants à leurs descendants.

\ib{Herméneutique} [G. {\it herméneuein}, interpréter]. — \si{Épist.}
Interprétation des textes anciens, {\it spéc.} des textes bibliques.

\ib{Hétérogène} — (Ctr. {\it homogène}*).
Composé d’éléments de nature différente.

\ib{Hétéronomie} — \si{Mor.} (Ctr. {\it autonomie}*). État de la volonté qui
reçoit passivement sa loi d’une autorité extérieure ou d’une impulsion
étrangère à la raison.

\ib{Heuristique} [G. {\it heuriskein}, trouver].
— A) Adjectif : {\bf 1.} \si{Épist.} Qui sert à
% 87
la découverte : « Hypothèse heuristique ». — {\bf 2.} \si{Péd.} Méthode
heuristique (opp. {\it dogmatique}$^4$) : celle qui fait découvrir ce qu’on
enseigne, par l'élève lui-même, — B) Nom : {\bf 3.} \si{Épist.} En histoire,
la recherche des documents.

\ib{Hiérarchie} — Ordre de dépendance des personnes, des idées ou des
phénomènes, établi soit d’un point de vue normatif* (religieux, juridique,
moral, logique, etc.), soit du point de vue d’une simple dépendance naturelle
(hiérarchie des fonctions physiologiques en Biol). — \si{Épist.}
{\it Hiérarchie des sciences} : classification des sciences fondée sur
« l’enchaînement naturel » (Comte) et l’ordre de complexité des phénomènes
qu'elles étudient.

\ib{Histoire} — {\bf 1.} \fsb{S. concr.} [Cf. all. : {\it Geschichte}]
Ensemble des états par lesquels passe un être qui change : « Chacun de nous a
une histoire qui nous a faits ce que nous sommes » (Berger) ; « L'esprit est
histoire » (Blanché). — \fsb{S. abstr.} \si{Épist.} [Cf. all. :
{\it Historie}]. {\bf 2.} {\it Lato.} Toute étude à caractère descriptif :
« L'histoire naturelle ». — {\bf 3.} {\it Str.} Étude chronologique des faits
sociaux considérés dans leurs particularités de temps et de lieu : « Pour
bien écrire l’histoire, il faut être dans un pays libre » (Voltaire).

\ib{Histologie} [G. {\it histos}, tissu]. — (Syn. : {\it anatomie fine}).
\si{Épist.} Partie de l'anatomie qui a pour objet l'étude des tissus.

\ib{Historicité} — {\bf 1.} \fsb{S. objec.} \si{Épist.} Caractère de
réalité historique : « L’historicité de
Jésus ». — {\bf 2.} \fsb{S. subje.} \si{Méta.} Dans le {\it lang.
existentialiste} : condition de l’existant humain qui, tout en étant
engagé dans le temps et solidaire de son passé, s’en détache en se
% 
situant par rapport à lui et se projette librement vers l'avenir.

\ib{Holisme} [G. {\it holos}, total]. — \fsb{S. norma.} \si{Méta.} 1. Terme
inventé en 1926 par J. C. Smuts pour désigner la tendance de l’univers à
construire des unités formant un tout et de complication croissante.
— {\it D'où} : \si{Biol.} {\bf 2.} Doctrine qui considère l’organisme vivant
comme un tout indécomposable.

\ib{Homaloïdal} [G. {\it homalos}, plan]. — \si{Math.} Sans courbure; où l’on
peut, par suite, tracer des figures semblables à n'importe quelle échelle :
« L'espace euclidien est homaloïdal ».

\ib{Homogène} — {\bf 1.} Dont toutes les parties sont de même nature.

— \si{Math.} {\bf 2.} Un milieu est dit {\it homogène} quand on peut y
déplacer une figure sans déformation (milieu sans courbure [homaloïdal*] ou
à courbure constante [{\it p. e.} surface sphérique]).

\ib{Hormè} — \si{Ps. an.} Terme grec qqfs. employé comme syn. de
{\it pulsion}* ou d’{\it instinct} au sens {\bf 3.}

\ib{Humaniser} — {\bf 1.} {\it Laud.} Rendre humain, {\it i. e.} bon et
doux : « Humaniser les mœurs de la nation » (Voltaire). — {\bf 2.} {\it Péj.}
Réduire aux dimensions de l’homme : « Les hommes humanisent toutes
choses » [et même Dieu] (Malebranche, {\it Entr.}, VIII, 9).

\ib{Humanisme} — \si{Hist.} {\bf 1.} Mouvement des « humanistes » de la
Renaissance (Erasme, Budé) : « L’humanisme n’est pas seulement le goût de
l’antiquité : il en est le culte » (Ph. Monnier).

— \si{Crit.} {\bf 2.} \fsb{S. norma.} Forme de pragmatisme* professée par
F. C. S. Schiller (d'Oxford) et selon laquelle toute
% 88
connaissance est subordonnée aux conditions de l'expérience humaine.

— \si{Mor.} {\bf 3.} \fsb{S. norma.} Doctrine qui ne reconnaît aucune valeur
supérieure à l'être humain ({\it p. e.} Feuerbach, Nietzsche) : « L’humanisme
n’a de sens que s’il est possible de fixer l'humanité dans une nature {\it ne
varietur} » (Le Senne). — {\bf 4.} « Intérêt majeur témoigné au problème de
l’homme, de sa nature, de son origine, de sa destinée, de sa situation dans
le monde » (A. Etcheverry). $->$ Terme équivoque.

\ib{Hylè} [mot grec = matière]. — \si{Méta.}
{\it Chez Husserl} : matière$^2$ de la sensation considérée comme pur donné,
indépendamment de son sens intentionnel*.

\ib{Hylètique} — ({\it Adj.}) [Opp. : {\it éïdétique}$^1$]. 1. Qui concerne
la hylè* : « Les vécus hylétiques » — (Nom fém.). {\bf 2.} Théorie de la
hylè* : « L’hylètique pure se subordonne à la phénoménologie* de la
conscience transcendantale » (Husserl).

\ib{Hylèmorphisme} [G. hylê, et morphê, forme]. — \si{Hist.} \fsb{S. norma.}
Doctrine (d'Aristote et des Scolastiques) selon laquelle les corps$^1$ sont à
la fois {\it matière}$^1$ (d’où leurs propriétés quantitatives) et
{\it forme}$^1$ (d’où leurs propriétés qualitatives).

\ib{Hylozoïsme} [G. {\it hylê}, et {\it zôon}, être vivant]. — \si{Hist.}
\fsb{S. norma.} Conception selon laquelle la matière même et l'univers sont
doués de vie (v. {\it Précis}, Ph. II, p. 482-483).

\ib{Hyperbolique (Doute)} — \si{Hist.} {\it Chez Descartes} : le doute
provisoire : « Ce doute général et universel que j'ai moi-même appelé
hyperbolique et métaphysique et duquel j'ai dit qu'il ne fallait point se
servir pour... la conduite de la vie » (7$^\text{es}$ {\it Rép.}).

\ib{Hyperespace} — \si{Math.} Espace différent de l’espace euclidien*.

\ib{Hyperesthésie} — \si{Ps. path.} Augmentation anormale de la 
sensibilité$^4$.

\ib{Hypermnésie} — Ps, path. Surexcitation morbide de la reviviscence
des images (v. {\it Précis}, Ph I, p. 226).

\ib{Hypnagogiques (États)} [G. {\it hypnos}, sommeil, et {\it agôgé},
conduite]. — \si{Psycho.} États psychiques intermédiaires entre ceux de la
veille et ceux du sommeil (ceux du réveil sont qqfs. appelés {\it états
hypnopompiques}).

\ib{Hypnose} — \si{Ps. path.} (Syn. : {\it sommeil hypnotique}).
Somnambulisme* artificiellement provoqué ou qui se produit en dehors du
sommeil normal.

\ib{Hypnotisme} — \si{Psycho.} Ensemble
des moyens par lesquels on provoque l’hypnose*,

\ib{Hypostase} — \si{Méta.} Nom grec de la {\it substance}* ; d’où : {\bf 1.}
{\it Chez Plotin} : l'Un, l'Intelligence et l'Ame : « L'hypostase naît quand
la puissance émanée de l'Un se recueille en quelque sorte sur elle-même et se
fixe » (Bréhier) ; — {\bf 2.} {\it Chez les auteurs chrétiens} : les trois
personnes de la Trinité. Plus gén., {\it chez saint Thomas} : « Les
substances individuelles sont appelées hypostases ou substances
premières » ({\it S. Th.}, I, 29).

\ib{Hypostasier} — Ériger à l'état de substance : « La réalité métaphysique,
telle que Plotin la conçoit, est la vie spirituelle hypostasiée » (Bréhier).
Souvent péj., ériger à l’état d’entité$^3$ : « Il n’est pas nécessaire
d’hypostasier la conscience collective » (Durkheim).
% 89

\ib{Hypostatique (Union)} — Theol. Celle de la nature divine et de la nature
humaine dans la personne du Christ.

\ib{Hypothèse} — \si{Vulg.} {\bf 1.} {\it Lato.} Supposition ; proposition
admise sans égard à sa vérité ou à sa fausseté et sans intention de la
soumettre à vérification. $->$ Dire plutôt en ce sens {\it conjecture}.

— \si{Épist.} {\bf 2.} En \si{Math.} (Opp. : {\it conclusion}) : données$^1$
d’un problème, ou proposition admise comme point de départ dans un théorème :
« On construit, la conséquence* avec l'hypothèse » (Goblot). — {\bf 3.} Dans
les Sc. expérimentales : proposition admise provisoirement et destinée à être
soumise au contrôle de l’expérience : « Une idée anticipée ou une hypothèse
est le point de départ nécessaire de tout raisonnement expérimental » (Cl.
Bernard). {\it Grandes hypothèses} ou {\it hypothèses générales} : syn. de
{\it théories}$^2$ : « L’hypothèse doit toujours être soumise à la
vérification » (Poincaré).

— \si{Théol.} {\bf 4.} (Opp. : {\it thèse}). Application d’une vérité
générale à des circonstances particulières et plus ou moins contingentes.

\ib{Hypothétique} — \si{Épist.} Conditionnel*, {\it }Proposition hypothétique
(opp. {\it catégorique} et {\it disjonctive}) celle où l’assertion est
subordonnée à une condition$^2$ ({\it p. e.} : « Si deux droites sont
parallèles, elles sont équidistantes »). {\it Syllogisme hypothétique} :
celui dont une prémisse est hypothétique.

\ib{Hystérie} — \si{Ps. path.} Névrose* qui se manifeste psychologiquement
par « le rétrécissement du champ$^3$ de la conscience personnelle et la
tendance à la dissociation des fonctions qui par leur synthèse$^3$
constituent la personnalité » (Janet). Voir {\it Pithiatique}* et
{\it Suggestibilité}*.

	\end{itemize}
