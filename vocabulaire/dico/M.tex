	\begin{itemize}[leftmargin=1cm, label=\ding{32}, itemsep=1pt]
\ib{Macrocosme} [G. macros, grand, et
cosmos, monde]. — Hist. L'univers
considéré comme correspondant au
% 111
microcosme où « petit monde » de
l’organisme.

\ib{Macro...} — Préfixe souvent usité dans
l'Épist. contemporaine pour désigner les phénomènes vus à grande
échelle, vg. : Phys. « échelle macroscopique », celle de nos sens ;
— Éc. pol. : « L'approche macroéconomique
comporte l'étude des ensembles, des
groupes, des comportements collectifs, des macroquantités, des macrodécisions... Comment passer du micro
au macro, de l'individuel au collectif ? » (A. Marchal).

\ib{Magie} — Hist. 1. Ensemble de pratiques occultes par lesquelles (surtout dans les sociétés primitives) on
prétend agir sur la nature, ces pratiques se distinguant des pratiques
religieuses, soit par leur caractère
coercitif alors que celles-ci visent à
se concilier les puissances cachées
(Frazer), soit par leur caractère
para-social, illicite, étranger aux
rites organisés (Mauss, Durkheim),
soit enfin parce que leur caractère
mystique s’est dégradé en technique de la terre alors que la religion évoluait en économie du salut
et de la vie spirituelle (Pradines). —
2. Magie naturelle : nom donné au
{\footnotesize XVI}$^\text{e}$ siècle aux premières expériences de physique,

— Ext. 3. Attitude qui implique
des conduites différentes de celles
que nous tenons en face du réel :
« L’acte d'imagination est un acte
magique : c’est une incantation
destinée à faire apparaître la chose
qu’on désire » (Sartre). Cf. Textes
choisis, I, p. 243.

\ib{Maïeutique} [G. maieutèr, accoucheur].
— Hist. Méthode par laquelle Socrate se vantait d’ « accoucher » les
esprits, i. e. de leur faire découvrir
la vérité qu'ils portent en eux.
% 111

\ib{Maître et esclave} — Hist. Chez Hegel :
rapport de deux consciences dont
l’une s'affirme « comme existence
pour soi absolue contre et pour
l’autre » (le maitre), tandis que
l’autre (l’esclave) abdique son moi
individuel et préfère la vie à la
liberté. Cf. Esclaves*.

\ib{Majeur} — Log. form. Dans un syllogisme : 1. Grand* terme. — 2. (Au
fém.). Majeure, prémisse* qui contient le majeur$^1$.

\ib{Mal} — Méta. et Mor. 1. « Le mal
métaphysique consiste dans la simple
imperfection ». — 2. « Le mal physique, dans la souffrance ». — 3. « Le
mal moral, dans le péché » (Leibniz,
Théod., I, 21). — 4. Problème du
mal : celui qui consiste à concilier
l'existence du mal avec celle d’un
Dieu bon et tout-puissant.

\ib{Malheur de la conscience} — Hist.
Chez Hegel : état de la conscience
« scindée à l'intérieur de soi » par la
contradiction entre la « conscience
immuable » et la « conscience changeante » et par son vain effort pour
s’élever à l’objectivité.

\ib{Mana} —— Soc. Principe impersonnel
qui, dans les croyances des sociétés
primitives, donne l'efficacité à tout
ce qui agit.

\ib{Manichéisme} — A. Hist. 1. Doctrine
de l’hérétique Manès ({\footnotesize III}$^\text{e}$ siècle) :
« Un Dieu qui se plairait au mal
d'autrui ne saurait être distingué du
mauvais principe du Manichéisme »
(Leibniz, Théod., préf.). — Ent. 2.
Toute doctrine qui admet le dualisme* d’un principe du bien et
d’un principe du mal.

\ib{Manie} — Ps. path. Surexcitation
mentale caractérisée par l'euphorie*
et l’hyperactivité, avec périodes
% 112
d’agitation motrice et de délire. Cf.
Circulaire*.

\ib{Manqué (Acte)} [trad. all. Fehlleisiung].
— Ps. an. Conduite telle que lapsus,
oubli, perte d'objet, etc., qui, selon
Freud, résulte de l'interférence de
deux intentions et du refoulement*
de l’une des deux.

\ib{Marginaux (États)} [Angl. margin,
frange]. — Psycho. Halo d’inconscient qui, selon James, forme la
« frange » des états conscients.

\ib{Marginalisme} — A. Éc. pol. Théorie
de la valeur$^3$ selon laquelle celle-ci
se détermine d’après l'utilité marginale (syn. : utilité limite ou finale),
i. e. celle de l’objet d'une certaine
nature (vg. un seau d’eau) qui sert
à satisfaire le besoin le moins urgent
(vg. arroser des fleurs).

\ib{Masochisme} — Ps. path. Perversion
dans laquelle le sujet prend plaisir
à s'infliger des souffrances à luimême,

\ib{Matérialisme} — A (Ctr. : spiritualisme*). Méta. 1. Str. Doctrine
selon laquelle la matière$^4$ est la
réalité première et qui nie l'existence originale de l'esprit$^5$. Dist. :
a) le matérialisme mécaniste ({\footnotesize XVIII}$^\text{e}$
et début du {\footnotesize XIX}$^\text{e}$ siècles) qui réduit
tout aux phénomènes matériels les
plus simples, les phénomènes mécaniques$^4$ ; — b) le malérialisme dialectique (nom donné ultérieurement
par F. Engels au matérialisme de
K. Marx) selon lequel les choses
elles-mêmes se développent « dialectiquement », i. e. selon un processus analogue à celui dont Hegel
avait fait la loi de la pensée (v. Dialectique$^5$), ce mouvement dialectique
de la pensée étant selon Marx
« le reflet du monde réel ». — 2. Lato.
loute doctrine qui « explique le
% 112
supérieur par l’inférieur » (Comte).

— Mor. A. 3. État d'esprit et
direction pratique de la vie orientée
vers la recherche des jouissances et
des biens matériels. $->$ Dist. ce
sens des précédents avec lesquels
il n’est pas nécessairement lié.

— Soc. À. 4. Matérialisme historique : doctrine de K. Marx et
F. Engels selon laquelle la superstructure sociale, politique et idéologique$^3$ de la vie est déterminée en
dernière analyse par sa base$^2$ économique, sans préjudice des actions
réciproques entre l’une et l’autre.

\ib{Matériel} — Qui se rapporte à la ma
tière* : 1. (opp. : formel$^3$) aux sens 1,
2 ou 3 de ce mot. Vérité matérielle :
celle qui consiste dans l’accord de
la pensée avec les données de l’expérience; — 2. (Opp. : spirituel$^1$) au
sens 4 : « La substance étendue est
ce qu’on nomme proprement le
corps$^1$ ou la substance des choses
matérielles » (Descartes, Princ.,
II, 1).

\ib{Mathématiques} — Épist. Ensemble
des sciences de la quantité* et de
l’ordre$^2$. Math. pures ou abstraites :
arithmétique*, algèbre*, calcul des
fonctions$^1$ et infinitésimal*. Math.
concrètes : géométrie* et mécanique$^1$.
Math. appliquées : trigonométrie*,
géométrie* descriptive, calcul des
probabilités*. Sciences  physicomathématiques : mécanique$^1$, astronomie*, etc.

\ib{Matière} — Hist. 1. (Syn. : cause$^4$ matérielle. Opp. : forme$^1$). Chez Aristote
et les Scolastiques (qqfs. matière première) : sujet$^3$ indéterminé qui est le
support de la forme$^1$ et qui en fait
une réalité concrète : « Un des principes d’Aristote est que la matière par elle-même, est informe » (Buffon). — 2. Chez Kant, voir Forme$^2$ :
% 113
« Notre activité intellectuelle élabore
la matière (Stoff) brute des impressions sensibles en une connaïssance
des objets » (R. pure, introd., I).

— Épist. 3. (Opp. : forme$^6$). Contenu (d'un jugement, d’un raisonnement, d’une connaissance) : « L’esprit mathématique dédaigne la matière pour ne s’attacher qu'à la
forme pure » (Poincaré). D'où, ext. :
objet$^3$ ou question dont on traite.

— Méta. 4. (Opp. : esprit$^5$), Substance qui constitue les corps$^1$ : « La
matière dont la nature consiste en
cela seul qu’elle est une chose étendue, occupe tous les espaces imaginables » (Descartes, Princ., II, 22) ;
« L’étendue et la matière ne sont
qu'une même substance » (Malebranche, Entr., I, 2) ; « Les éléments
de la matière peuvent se ramener à
l'étendue et au mouvement » (Boutroux). Chez saint Thomas : « matière sensible », la matière corporelle
« en tant que sujet$^3$ des qualités
sensibles » ; « matière intelligible »,
la même « en tant que sujet de la
quantité » (S. th., I, 85, 1). Chez Descartes : « matière subtile », sorte de
fluide formé des parties les plus
fines et les plus mobiles de la matière (cf. Princ., IV, 25).

\ib{Matrice} — Math. Nombre complexe$^2$
dont les termes groupés en tableau
rectangulaire permettent des opérations algébriques applicables not. à
la théorie de l'atome (atome matriciel de Heisenberg).

\ib{Maxime} — Vulg. 1. Proposition exprimant un précepte : « Les maximes
sont d’un grand usage en morale et
en politique » (Condillac). — Hist.
2. Chez Kant (opp. : loi$^4$) : « Les principes pratiques sont subjectifs, ce
sont des maximes, quand la condition n’est regardée comme valable
% 113
par le sujet que pour sa propre volonté; ils sont objectifs, ce sont des
lois pratiques, s’ils sont reconnus
comme valables pour la volonté de
tout être raisonnable » (R. pratique,
début).

\ib{Mécanicisme} — Syn. de Mécanisme
aux sens 3 ou 4.

\ib{Mécanique (nom)} — Épist. — 1. Science
mathématique du mouvement et des
causes (forces$^4$) qui le déterminent.
$->$ Dis. l'art des machines qu'il
est impropre auj. d’appeler de ce
nom. — 2. Mécanique céleste : partie
de l'astronomie constituée par la
théorie mathématique des mouvements des astres.

\ib{Mécanique (adj.)} — Vulg. 3. Qui concerne les machines : « Arts mécaniques ».

— Épist. 4. Qui se ramène aux
notions en usage dans la mécanique$^1$
et exclut par suite toute finalité* :
« Théorie mécanique », « Explication mécanique ». Déterminisme
mécanique : celui dont les termes,
restant extérieurs les uns aux autres,
conformément au principe de l’indépendance des mouvements, s’enchaînent d’une façon mécanique$^4$.
Soc. Voir Solidarité$^3$.

\ib{Mécanisme} — A Vulg. 1. Combinai
son d'organes en vue de la production de certains mouvements. —
2. Ext. Combinaison de fonctions :
« Le mécanisme de la perception ».

— À, Phys. 3 Théorie scientifique qui explique les phénomènes
physiques par le mouvement (cf.
Cinétique*). Qqfs., Biol, syn. de :
théorie physico-chimique* de la vie.
— Méta. 4. (Ctr. : dynamisme).
Système philosophique selon lequel
la matière$^4$ est distincte de la force$^6$
(ou de l'énergie) et qui explique
% 114
l’ensemble des phénomènes matériels par le mouvement : « Le mécanisme est devenu dans ces derniers
temps le signe distinctif des Cartésiens (Mairan).

\ib{Médiat} — (Ctr. : immédiat*). Qui
comporte quelque intermédiaire. —
Log. Inférences médiates : celles où
Pon passe de la proposition qui sert
de point de départ, à la conclusion
par au moins une proposition intermédiaire (vg. syllogisme).

\ib{Médiation} — Vulg. 1. Action de
servir de médiateur, d’intermédiaire.
Spéc., Théol. : « La médiation du
Christ » (entre Dieu et l'homme).

— ©. Méta. 2. Chez Hegel : acte
de négation et de dépassement à la
fois (cf. Aufheben*) qui établit le
lien entre le sujet et l’objet, le temps
et l'éternité, le fini et l'infini : « La
médiation est le moment du moi qui
est pour-soi, la pure négativité »
(Phén., préf., II) ; « Après avoir dans
Ja Phénoménologie ouvert la voie à
la médiation psychologique en montrant que la prise de conscience du
moi comme sujet enveloppe la présence de l’autre$^1$, Hegel conçoit la
médiation comme la relation idéale
reliant entre eux les différents moments d’un tout; finalemert, il
reconnaît en elle l'expression de
l'identité entre la logique et l’histoire » (Niel). — 3. Plus gén. : « Nul
ne réalise sa propre vie tout seul,
mais seulement par la médiation des
autres hommes » (Lavelle). —
© 4. Ce qui fait office de médiation® :
« L’une de ces médiations [entre la
valeur et notre visée*] est la pensée
conceptuelle... Nous faisons rentrer
ce que nous visons dans l’extension
d’un concept : l’idéal » (Le Senne).

\ib{}Médiatiser — 1. Rendre médiat*. —
2. Servir de médiationt : « Dès
% 114
qu'elle médiatise la grâce$^2$, la dualité du toi et du moi le cède à la
communion de l'existence divine et
de l’existence humaine » (Le Senne).

\ib{Médium} — Psycho. Personne qui
prétend être en relation avec les
esprits$^4$.

\ib{Méga...} — Voir Micro...*.

\ib{Mégalomanie} — Voir Grandeur$^2$.

\ib{Mélancolie} — Psycho. 1. Tristesse
vague dans laquelle le sujet se complaît. — Ps. path. 2. Tristesse morbide accompagnée d’une dépression mentale générale avec anxiété,
dégoût de la vie et souvent idées
délirantes d’auto-accusation. Cf. Circulaire* et Culpabilité*.

\ib{Même (Le)} — Méta. L'identique. Cf.
Autre$^2$ et Identité$^3$.

\ib{Mémoire} — Biol. et Psycho. À. Lato.
Persistance d’une modification correspondant à une action passée avec
faculté de la reproduire : en ce sens,
il y a une mémoire organique ou
biologique : « La mémoire est une
fonction générale du système nerveux » (Ribot) ; « Le rôle biologique
de la mémoire apparaît indispensable à la conservation des espèces »
(Pièron). $->$ Ce sens a même été
étendu jusqu’au domaine purement
physique. De même qu’on parlait
autref., à propos de l'hystérésis
magnétique, d’une « mémoire de la
matière », on donne auj. le nom de
« mémoire », dans les servomécanismes dont use la cybernétique*,
au dispositif qui enregistre et met
en œuvre les informations passées.

— Psycho. Ensemble de fonctions psychiques parmi lesquelles il
y a lieu de distinguer : 2. la mémoire immédiate qui « manifeste la puissance
de conservation, la persistance qui
% 115
est en chaque état de conscience »
(Delacroix) ; — 3. la mémoire élémentaire, simple reviviscence* du passé
sans qu'il y ait nécessairement
reconnaissance* : vg. résurrection
des images$^3$ ou des sentiments (« mémoire affective ») ; — 4. la mémoire
ppt. dite ou prise de conscience du
passé comme tel, qui implique :
a) la fixation$^1$; b) l’évocation* ou
rappel*, qui en réalité est souvent
reconstruction volontaire ; c) la reconnaissance* et la localisation$^3$ du
souvenir : « Nous voyons la mémoire
pénétrée d'organisation  intellectuelle » (Delacroix) ; « La mémoire
est une réaction sociale dans la
condition d'absence » (Janet).

— Méta. 3. Chez Bergson : propriété fondamentale de la conscience,
essentiellement distincte de la mémoire!-habitude : « Toute conscience
est mémoire, conservation et accumulation du passé dans le présent »
(E. S., I). On peut, en ce sens, distinguer
« deux mémoires, dont l’une imagine [images-souvenirs] et dont
l’autre répète [« mécanismes tout
montés » dans le corps]... De ces
deux mémoires, la première paraît
bien être la mémoire par excellence.
La seconde. est l'habitude éclairée
par la mémoire plutôt que la mémoire
même » (Mat. et Mém., II).

\ib{Mémoration, Mémorisation} — Psycho.
Fonction de la mémoire par laquelle
le sujet fixe les souvenirs et spéc.
apprend par cœur.

\ib{Mental} — Psycho. 1. Lato. Syn. :
psychique$^1$ : « La vie mentale »; « Les
maladies mentales ». — 2. Str. Syn. :
intellectuel$^1$ : « Les opérations mentales. »

\ib{Mentalité} — Psycho., Soc. Ensemble
des représentations et habitudes mentales$^2$ d’un individu ou d’un groupe.
% 115

\ib{Mentisme} — Ps. path. Exaltation men
tale$^2$ morbide qui produit un défilé
rapide des idées dans lequel le sujet
ne se sent plus maître de sa pensée.

\ib{Mérite} — Mor. 1. Valeur morale im
pliquant un effort de volonté pour
vaincre les obstacles (cf. « bien
mériter ») : « Avoir du mérite »; « Le
mérite est désintéressé ou il n'est
pas » (Le Senne). — 2. Caractère
d’une personne ou d’un acte qui a
ou donne droit à récompense (cf.
« mériter une récompense »). En ce
sens, surtout au pluriel : « S’acquérir
des mérites »,

\ib{Mésologique} — Biol., Soc. Qui concerne le milieu$^1$.

\ib{Mesure} — Épist. Rapport d’une
grandeur à une autre grandeur
prise comme unité « On peut
définir la science mathématique en
lui assignant pour but la mesure
indirecte des grandeurs » (Comte).

\ib{Méta...} — Préfixe souvent usité dans
l'Épist. contemporaine pour désigner un mode de pensée qui est au
delà du savoir ordinaire : « Si l’on
veut exprimer le savoir implicitement utilisé dans le travaii d’axiomatisation de la Logique$^2$, ce n’est
pas à l’intérieur de la Logique qu’on
pourra le faire, mais dans une discipline nouvelle qui prendrait pour
objet les formules de la Logique
axiomatisée et les règles de leur
maniement. La métalogique joue
ainsi, par rapport à la logique$^2$, le
même rôle que la métamathématique
par rapport à la mathématique... Au
calcul formel, langue objective, vient
ainsi se superposer une mélalangue
qui comprend not. les règles de
syntaxe du calcul formel » (Blanché).

\ib{Métagéométrie} — Épist. Géométrie
de l'hyperespace* : « Les géométries
% 116
non-euclidiennes ne sont que des
cas particuliers de la métagéométrie » (Brunschvicg).

\ib{Métamorale} — Mor. « Dans nos systèmes de morale$^2$ théorique se trouvent souvent confondues des observations de faits, et des conceptions
métaphysiques que l’on pourrait
plus précisément appeler métamorales » (L. Lévy-Bruhl).

\ib{Métamorphose} — Voir Métempsycose*.

\ib{Métaphysique (nom)} [Sur l'origine du
mot, v. Textes choisis, II, p. 259]. —
1. (Sens usuel. Syn. : ontologie).
Connaissance de « l'être en tant
qu'être », i. e. de l'être absolu, et des
principes premiers : « La métaphysique traite des choses les plus
immatérielles, comme de l'être en
général et en particulier de Dieu et
des êtres intellectuels faits à son
image » (Bossuet) ; « J'entends par
Métaphysique les vérités générales
qui peuvent servir de principes aux
sciences particulières » (Malebranche,
Entr., VI, 2) ; « Pour ces esprits [les
platoniciens], la philosophie est vraiment une méla-physique, un mouvement au-delà, un effort non pour
saisir des réalités qui expliquent,
bien qu’analogues, celles de la nature, mais pour comprendre, d’un
point de vue supérieur, la loi même
en vertu de laquelle l'esprit pose
spontanément les unes et les autres »
(Lagneau). — 2. Systématisation
générale et réfléchie de la pensée :
« Il [Leibniz] saisissait dans tout les
principes les plus élevés et les plus
généraux, ce qui est le caractère de
la métaphysique » (Fontenelle) ;
« Faire de la métaphysique, ce n’est
pas autre chose qu'organiser des
idées » (Dunan). D'où : conception
d’ensemble du monde et de la vie :
« Toute civilisation importante a sa
% 116 — MÉT
métaphysique » (Eucken), — 3.
« Étude des conditions générales
d'une œuvre telles qu’elles résultent
de l'analyse critique qu’on peut
faire par avance de son objet et de
ses présuppositions » (Lalande)

« Je veux mourir s’il y a dans ces
têtes-là le premier mot de la métaphysique de leur art » (Diderot).

— Hist. 4 Chez Kant : ensemble
des connaissances obtenues par la
faculté de connaître a priori : « La
métaphysique, connaissance spéculative de la raison totalement isolée
qui s'élève tout à fait au-dessus de
l’enseignement de l'expérience et
cela par de purs concepts... »
(R. pure, préf. 2$^\text{e}$ éd.). — 5. Chez
Condillac, d’Alembert et les Idéologues$^1$ : théorie de l’origine des
idées$^4$ : « La métaphysique a pour
but d'examiner la genèse de nos
idées » (D’Alembert). — 6. Chez
Bergson : connaissance intuitive de
l'absolu (opp. à la pensée discursive
qui « tourne autour » de l’objet) et,
de façon privilégiée, de l'esprit : « La
métaphysique est la science qui
prétend se passer de symboles »
(P. M., VI, 1903) ; « Nous assignons
à la métaphysique un objet limité,
principalement l'esprit » (ib, II,
1934). — 7. Chez les existentialistes :
recherche où le problème « empiète»
(Wahl) sur celui même qui le pose :
« Aucune question métaphysique ne
peut être posée sans que le questionneur, comme tel, ne soit luimême pris dans cette question »
(Heidegger). Cf, Problème*.

\ib{Métaphysique (adj.)} — 8. Qui relève
de la métaphysique (surtout au
sens 1) : « Il faut en venir d’une
nécessité physique$^1$ ou hypothétique
à qqc. qui soit une nécessité absolue
ou métaphysique » (Leibniz). —
9. Qui est d’ordre intelligible, et non
% 117
sensible : « La vérité des choses métaphysiques, lesquelles ne dépendent point des sens » (Descartes,
2$^\text{es}$ Rép.) ; « Tâchez de vous accoutumer aux idées métaphysiques et
de vous élever au-dessus de vos
sens » ((Malebranche, Entr., I, 10).
D'où : purement théorique (v. Hyperbolique*), ou : très abstrait : «Je ne
gais si je dois vous entretenir des
méditations que j'y ai faites; car
elles sont si métaphysiques et peu
communes... » (Descartes, Méth.,
IV). D'où qqfs. péj. : « La question
[du pur amour] devint si subtile et
si métaphysique... » (Fontenelle).

— Hist. 10. Chez Comte : « état
métaphysique », celui dans lequel
l'homme s'efforce d'expliquer la
nature intime des choses par des
entités® et où domine la tendance à
argumenter au lieu d'observer. —
11. Chez les existentialistes : qui
relève de la métaphysique au sens 7 :
« Il semble que l'inquiétude métaphysique puisse s’interpréter comme
un certain refus d’abdiquer, l’objet
étant précisément ce devant quoi
j'abdique » (G. Marcel).

\ib{Métapsychique} — Épist. Étude des
phénomènes parapsychiques*.

\ib{Métempirique} — Crit. Qui, sans être
ppt. métaphysique$^8$, est cependant
au-delà de l'expérience sensible.

\ib{Métempsycose} — Hist. Doctrine (vg.
chez Pythagore) de la transmigration* des âmes : « L’âme ne change
de corps que peu à peu. et il y a
souvent métamorphose dans les
animaux, mais jamais métempsychose ni transmigration des âmes »
(Leibniz, Mon., 72).

\ib{Méthode} — Épist. O 1. « Art de bien
disposer une suite de plusieurs
pensées ou pour découvrir, la vérité
% 117
quand nous l’ignorons ou pour la
prouver aux autres quand nous la
connaissons déjà » (Port-Royal). —
©. 2. Procédé spécial : « La méthode
des variations* concomitantes ».

\ib{Méthodologie} — Étude des méthodes
propres aux différentes sciences.
$->$ Dist. épistémologie* qui, même
au sens 1, est plus général.

\ib{Micro...} — Préfixe souvent usité dans
l'Épist. contemporaine pour désigner les phénomènes à très petite
échelle : vg. Phys., « échelle microscopique », au-dessous de 1µm ;
« microphysique », étude des phénomènes (atomiques et nucléaires) à
l'échelle microscopique; — Biol.,
« micro-évolution », celle qui correspond à la diversification des espèces
(opp. : « méga-évolution », formation
des groupes supérieurs) ; — Éc. pol.,
Soc., « microéconomie », « micro
sociologie », étude des phénomènes
économiques ou sociaux à l'échelle
individuelle.

\ib{Milieu} — Biol., Psycho., Soc. 1. Ensemble des êtres et des phénomènes
avec lesquels un être vivant se trouve
en rapport.

— Log. 2. Principe du milieu
(syn. : du tiers) exclu : de deux propositions contradictoires$^1$, si l’une
est vraie, l’autre est nécessairement
fausse et réciproquement, et il n’y a
pas de troisième solution possible
(cf. Alternative$^3$).

\ib{Mimique} — [G. mimeisthai, imiter]. —
Psycho. Ensemble des gestes, jeux
de physionomie, etc, qui imitent
les réactions spontanées et par lesquels s'expriment plus ou moins
volontairement les faits de conscience : « La mimique vocale suffit,
par de simples inflexions de voix,
pour changer le sens d’une phrase »
(Dumas).
% 118

\ib{Mineur} — Log. form. Dans un syllogisme : 1. Petit* terme. — 2. (Au
fém.). Mineure, prémisse* qui contient le mineur$^1$.

\ib{Minimum sensible} — Voir Seuil*.

\ib{Miracle} — Théol. : 1. « Quelque chose
de difficile et d'insolite, surpassant
la puissance de la nature et l'attente
du spectateur » (saint Thomas,
S. th., I, 105, 7). — Vulg. 2. Fait
extraordinaire ou qui paraît contraire aux lois de la nature : « L'attraction et la direction de l’aimant
sont des miracles continuels » (Voltaire).

\ib{Misonéisme} — Psycho. Éloignement
pour tout ce qui est nouveau : « Le
misonéisme des sociétés primitives
est une conséquence immédiate de
leur conformisme » (L. Lévy-Bruhl).

\ib{Mnémonique} — Psycho. Qui concerne
la mémoire : « Les fonctions mnémoniques. »

\ib{Mnémotechnie} — Psycho. Procédés
artificiels destinés à faciliter le
rappel des souvenirs.

\ib{Mobiles} — Psycho. 1. Lato. Tout ce
qui pousse à l’action (y compris les
motifs*) : idées, sentiments, ou
tendances. — 2. Str. Éléments actifs$^2$
et affectifs* (tendances, sentiments,
désirs) qui poussent à l’action.

\ib{Mobilité sociale} — Soc. Aptitude
d'une société au changement, soit
dans la hiérarchie* des individus ou
des groupes qui la composent (mobilité verticale), soit dans les rapports
des éléments de même niveau (mob.
horizontale).

\ib{Modal} — Épist. 1. Qui concerne les
modes$^1$, Chez Descartes (Princ., I,
61) : « distinction modale », celle qui
se fait soit entre un mode et la substance
% 118
qu'il diversifie, soit entre
deux modes d’une même substance
(opp. distinction réelle et celle qui
se fait seulement par la pensée). —
2. Propositions modales : celles où
« l'affirmation ou la négation est
modifiée par l'un de ces quatre
modes : possible, contingent, impossible, nécessaire » (Port-Royal)

\ib{Modalité} — Crit. Propriété qui affecte
la valeur de l’assertion* dans un
jugement. Chez les classiques : voir
Modal$^2$. Chez Kant : ce qui fait que
le jugement est assertorique*, problématique* ou apodictique* (v. ces
mots). Chez Husserl : « modalités
doxiques », celles qui affectent les
divers modes de croyance par rapport à la « croyance-mère » ou fondamentale : vg. certitude, supputation, conjecture, doute, ete. — Voir
Précis, Ph. II, p. 26.

\ib{Mode (masc.)} — Méta. 1. (Syn. : modi
fication). Toute détermination$^1$ d’un
sujet$^3$; manière d’être : « Lorsque je
dis ici façon ou mode, je n’entends
rien que ce que je nomme ailleurs
attribut ou qualité. Mais, lorsque je
considère que la substance en est autrement disposée ou diversifiée, je
me sers du nom de mode ou façon »
(Descartes, Princ., I, 56). — 2. Chez
Spinoza : « les affections de la substance ; autrement dit, ce qui est en
une autre chose par le moyen de
laquelle il est aussi conçu » (Eth. I,
déf. 5) ; en ce sens, mode s’opp. à
attribut$^3$ qui est une propriété essentielle : vg. les modes de l'étendue
sont les corps$^1$.

— Log. form. 3. Forme que prend
un syllogisme selon la quantité$^2$ et
la qualité$^3$ de ses propositions.
$->$ Dist. figures$^5$.

\ib{Mode (fém.)} — Soc. 4. Str. Ensemble
d'usages qui règnent dans une société
% 119
donnée : « Cette mode des
Éthiopiens était fort bizarre et
incommode; mais c'était la mode :
on la suivait avec joie » (Malebranche, R. V. II, 3, 2) ; « Le goût,
qui est personnel, est bien différent
de la mode, qui est fait social »
(Goblot). — 5. Lato. Chez Tarde : imitation$^1$ des contemporains
(opp. coutume$^1$ ou imitation du
passé).

\ib{Mœurs} — Soc. et Mor. 1. Ensemble
des pratiques, sentiments et jugements relatifs au bien$^2$ et au mal$^3$
et à la conduite en gén. : « Pour les
mœurs, il est besoin qqfs. de suivre
des opinions qu'on sait être fort
incertaines » (Descartes, Méth., IV) ;
« ll y a cette diflérence entre les
lois$^1$ et les mœurs que les lois règlent
plus les actions du citoyen et les
mœurs les actions de l’homme »
(Montesquieu, Lois, XIX, 16) ; « La
raison pure donne à l’homme une
loit universelle que nous appelons
la loi des mœurs [Sittengesetz] »
(Kant, R. pr., I, 1, 1, § 7). Science
(ou Physique) des mœurs : étude
sociologique et positive des mœurs
(L. Lévy-Bruhl, Durkheïm).

— Biol. 2. Comportement général : « Les mœurs des abeilles ».

\ib{Moi} — Le sujet*, considéré : 1. Psycho.
(moi empirique) comme identique à
la conscience$^1$; — 2. Méta. (moi
gubstantiel) comme âme$^2$ distincte
de la conscience empirique; — 3.
Crit. (moi sujet$^4$) comme pensée
s'opposant à l'objet$^5$ ou non-moi :
« Le moi ne peut se onnaître que
dans un rapport immédiat à quelque
impression qui le modifie » ((Biran) ;
— k, comme détermination$^1$ du Je*
(v. ce mot) : « L’acte essentiel de
conscience consiste pour le je à distinguer et opposer en lui-même CET
moi [le moi universel (cf. Valeur*))
% 119
et le moi particulier ou empirique] » (Le Senne, Bull. 1932, p. 8).

— Hist. 5. Chez Kant : « moi nouménal », le moi en tant qu’ « il a
conscience de lui-même comme
chose en$^3$ soi » (R. pr. I, 3, ad fin.). —
6. Chez Fichte : « Moi absolu », acte
constitutif du sujet$^4$ qui, en se posant lui-même, pose à la fois le moi
et le non-moi. — Cf. Ego*.

\ib{Molaire} — Biol. Qui est le résultat
d’une action d'ensemble (opp. moléculaire : qui concerne les actions de
détail des éléments de la cellule).
Ext. qui concerne l’ensemble (en
qq. domaine que ce soit) : vg. certains psychologues opposent le point
de vue molaire de la forme$^4$ ou du
comportement* total à l’atomisme$^3$
psychologique et à la psycho. des
éléments.

\ib{Monade} [G. monas, unité]. — Hist.
1. Chez Platon : terme d’origine
pythagoricienne, appliqué . aux
Idées$^1$ : « On discute la question de
savoir s’il faut admettre de telles
monades véritablement existantes;
puis comment chacune, tout en
restant une et toujours la même,
sans génération ni dépérissement,
peut être avec une parfaite constance la même unité » (Philèbe, 15 b).
— 2. Chez Leibniz : substance$^1$
simple, inétendue, indivisible, active,
douée de perception$^3$ et d’appétition* et qui constitue l'élément dernier des choses : « Les monades n’ont
point de fenêtres par lesquelles qqc.
y puisse entrer ou sortir » (Mon., 7).
Cf. Atome$^4$ et Entéléchie$^2$. $->$ Le
terme a été repris en un sens voisin
par Renouvier dans sa Nouvelle
Monadologie, —— par Husserl (v.
Textes choisis, I, p. 100), etc.

\ib{Monadisme} — Hist. A. Doctrine de
Leibniz sur les monades$^2$.
% 120

\ib{Mondain} — Méta. Qui concerne le
monde extérieur$^2$. $->$ On dit aussi :
intramondain (opp. extramondain*).

Monisme\ib{} [G. monos, seul]. — A, Méta.
Nom générique des doctrines qui
n’admettent qu’un seul principe là
où d’autres en admettent deux ou
plusieurs. Notamment : 1. (Opp. :
dualisme$^2$). Système philosophique
qui ramène tout ce qui existe, soit
à la matière$^4$ (monisme matérialiste) soit à l’esprit$^5$ (mon. spiritualiste) soit à l’idée$^1$ (mon. idéaliste).
— 2. (Opp. : pluralisme*). Doctrine
qui considère la multiplicité du
devenir comme superficielle et admet
l'unité et l’intelligibilité de l’être :
vg. celle de Bradley (cf. W. James,
Philo. de l’ Expérience, leçon II).

— Spéc. 3. Système de Hegel
(parce que la thèse$^2$ et l’antithèse$^2$
s'y dépassent dans une synthèse$^2$
supérieure). — 4. Système de Hæckel
(forme de monisme$^1$ et de panthéisme* qui pose l’unité de l'esprit$^1$
et de la matière$^1$ et l'identité de
Dieu et du monde).

— 5. Doctrine qui, dans un domaine particulier, vg. Mor., n’admet
qu'un seul principe : « Les morales$^1$
classiques relèvent toutes d’une
sorte de monisme moral » (Parodi).

\ib{Monogénisme} — Biol. À. Doctrine
selon laquelle : 1° toutes les races
humaines se rattachent à une espèce
unique; 2° tous les êtres humains
descendent d’un couple originel
unique : « La question du monogénisme est moins simple qu’on ne l’a
cru » (Le Roy).

\ib{Monothéisme} — Méta. A. Système
religieux ou doctrine philosophique
qui affirme l'existence d’un Dieu
unique distinct du monde.

\ib{Moral} — Mor. 1. Qui concerne les
mœurs$^1$ : « La conscience$^3$ morale »;
% 120
« L'obligation morale$^1$ » ; « L’essentiel dans la valeur morale [sittlich]
des actions, c’est que la loi morale
[moralisch] détermine immédiatement la volonté » (Kant, R. pr., I,
1, 3) ; « Ces deux caractéristiques de
la vie morale [l’obligation$^1$ et la désirabilité] se retrouvent partout où il
y a fait moral » (Durkheim), — 2.
Qui concerne la Morale$^2$ : « Les doctrines morales des philosophes » ;
« Toutes les théories morales constatent que l'individu ne peut vivre
uniquement pour lui-même »
(Guyau). — 3. (Ctr. immoral).
Laud. Conforme aux règles morales$^1$ :
« Il n’est rien de si facile que de se
donner l'air très moral » (Staël).

— Psycho. et Ébpist. 4. (Opp.
physique$^4$). Qui concerne l’esprit$^4$, la
pensée$^1$ : « Nos maux moraux sont
tous dans l'opinion » (Rousseau) :
« Le droit est un pouvoir moral
[= idéal$^1$, spirituel] » (Leibniz) ; « La
société est une personne morale »
(Durkheim). Le moral : ensemble
des facultés psychiques, et spéc. de
celles qui permettent de faire face
aux épreuves : « Si le physique va
trop bien, le moral se corrompt »
(Rousseau) ; « Remonter le moral à
qqn. » Sciences morales : celles qui
étudient « le moral » de l’homme :
psychologie, histoire, sociologie,
morale$^2$. — 5. (Opp. : logique$^7$, théorique). D'ordre sentimental ou pratique : « Je distinguerai ici deux
sortes de certitude : la première est
appelée morale, i. e. suffisante pour
régler nos mœurs ou aussi grande
que celle des choses dont nous
n'avons point coutume de douter
touchant la conduite de la vie »
(Descartes, Princ., IV, 205).

\ib{Morale} — Soc., Psycho. 1. (Syn. :
moralité$^1$, mœurs$^1$). Ensemble des
mœurs$^1$ et jugements moraux$^1$ d’un
% 121
individu ou d’une société : « Bien
qu’il y ait une morale du groupe,
chaque homme a sa morale à soi »
(Purkheim) ; « Une morale relâchée »; 
« La morale chinoise ».

— Mor. À, 2. (Syn. : éthique*).
Théorie, gén. conçue sous forme
normative*, de l’action humaine en
tant qu’elle est soumise au devoir$^6$
et a pour but le bien$^2$ : « La Morale
est la science des fins, la science de
ce que la raison veut invinciblement,
la science de l’ordre idéal de la vie »
(Rauh). — 3. Un système particulier de Morale$^2$ : « La morale utilitaire » ; « La morale de Kant » ; « À
elles deux, la morale hellénique et la
morale chrétienne paraissent embrasser tout l'idéal humain : l’une
est la morale de l'intelligence, l’autre
est la morale de la volonté » (Boutroux).

\ib{Moralisme} — Mor. (Gén. péj.) Attitude morale : a) qui substitue « l’honnêteté de surface » à la vertu authentique : « La corruption de la morale,
c'est le moralisme » (Guitton), —
ou : b) qui sacrifie à la valeur morale
toutes les autres valeurs : « Le moralisme consiste à traiter la valeur
morale comme si elle devait se confondre avec la valeur absolue » (Le
Senne).

\ib{Moralité} — Soc. et Mor. 1. @. (Syn. :
mœurs$^1$ ou morale$^1$). Croyances et
pratiques morales$^1$ effectives d’une
société : « La science des mœurs a
pour objet la moralité positive. »

— Mor. 2. @. Vie morale personnelle d’un individu : « La moralité n’est pas autre chose que la volonté rationnelle » (Hamelin) ; « La
moralité déborde la morale$^2$... [qui]
s’y oppose comme la rigueur à la
souplesse » (Le Senne). — 3. 0.
Valeur morale; caractère de ce qui
est moral$^3$ : « Nos actions tirent leur
moralité du rapport qu’elles ont
avec l’ordre$^{11}$ immuable » (Malebranche, Entr., XIV, 7). Chez Kant :
opp. légalité$^2$ (v. ce mot).

\ib{Morphème} — Ling. Élément formatif
d'un mot, tel qu'affixe, suffixe,
désinence, etc.

\ib{Morphologie} [G. morphé, forme, et
logos, étude]. — Épist. Science des
formes, not. : a) en Ling., partie de
la grammaire qui étudie les morphèmes* et leurs rapports (opp.
syntaxe*) ; — b) en Biol., description et classification des espèces$^3$
végétales et animales (botanique,
zoologie) etétude deleurs transformations (morphologie dynamique) ; —
c) en Soc., étude des différents types
de sociétés classées d’après leur
volume* et leur densité*.

\ib{Moteur} — Hist. (nom). 1. Chez Aristote : « premier moteur », Dieu en
tant qu'acte$^2$ pur et cause de tout
mouvement$^2$.

— Ps. phol. (adj.). 2. Qui se rapporte au mouvement$^1$ : « Sensations
motrices (syn. kinesthésiques) ;
« Nerfs moteurs »; — ou : qui tend
au mouvement$^1$ : « La reviviscence
possible d’une représentation est
en raison directe des éléments moteurs qu’elle contient » (Ribot).

\ib{Motif} — Psycho. Mobile$^1$ d'ordre intellectuel : « Le conflit des mobiles$^2$ et
des motifs ».

Motilité, Motricité\ib{} — Ps. phol. Faculté
de se mouvoir soi-même.

Mouvement\ib{} — Str. Phys. 1. Changement de position dans l’espace en
fonction du temps : « Le mouvement,
pris selon l’usage commun, n’est
autre chose que l’action par laquelle
un corps passe d’un lieu en un
autre » (Descartes, Princ., II, 24).
%122
Chez les cartésiens : « quantité de
mouvement », produit mv de la
masse par la vitesse : « Ils ont cru que
ce qui peut se dire de la force$^5$ [i. e.
qu’elle est constante] pourrait aussi
se dire de la quantité de mouvement » (Leibniz, Disc. méta., 17).
— Lato. Méta. 2. Changement
en général. Spéc., chez Aristote :
« entéléchie$^1$ de l’être en puissance$^2$ »,
i. e. la réalité en devenir$^1$, comprenant : a) le mouvement spatial;
b) le changement qualitatif ou altération ; c) le changement quantitatif
[accroissement ou décroissance] ;
d) la génération et la corruption.
— Psycho. 3. Impulsion spirituelle :
« Les mouvements de la grâce »
(Pascal, 507) ; « On ne saurait rien
demander à Dieu qu'il n’en donne
le mouvement » (Bossuet) ; « L'esprit
a du mouvement pour aller plus
loin [que l'impression particulière] »
(Malebranche, R. V., I, 1, 2) ; « Un
mouvement de pitié » — Soc. 4.
Changement dans l’ordre social :
« Le parti du mouvement » ; « Le
mouvement des idées ».

\ib{Moyen} — Vulg. (nom). 1. Ce qui sert
à réaliser une fin$^2$ : « Le passé et le
présent sont nos moyens ; le seul
avenir est notre fin » (Pascal, 172).
— Log. form. (adj). 2 Moyen
terme : dans un syllogisme, terme
qui sert à établir le rapport entre le
petit* et le grand* terme. Ext., tout
intermédiaire entre deux concepts.

\ib{Mutation} — Biol. (Opp. fluctuation*). Transformation brusque d’un
type morphologique, ayant son origine dans les conditions internes de
l'organisme.

\ib{Mystère} [G. muein, garder le silence].
— Hist. 4 Dans l'antiquité : culte
ésotérique : « Les mystères d'Éleusis ». — Théol. 2. Dans la religion
% 122
chrétienne : vérité révélée d'ordre
transrationnel* et que nous ne pouvons comprendre : « Le mystère de
la Trinité ». — Méta. 3. Plus gén.,
difficulté que nous ne pouvons
résoudre : « Toutes choses couvrent
quelque mystère » (Pascal) ; « Je
suis un être non transparent pour
lui-même, i. e. à qui son être même
apparaît comme un mystère »
(G. Marcel). Cf. Problème*.

\ib{Mysticisme} — Psycho. À. 1. État
psychique où le sujet a le sentiment
d'entrer en rapport direct avec Dieu :
« S'il y a un mysticisme faux et
périlleux, il y a un mysticisme vrai
et salutaire. Ce dernier part de ce
principe que nous ne pouvons pas
développer en dehors de Dieu l’être
que nous tenons de Dieu » (Wehrlé).

— Hist. À. 2. Doctrine fondée
sur le sentiment et l'imagination
plus que sur la raison et l’expérience
sensible (qqfs. péj., et avec l’idée
qu’elle repose sur des notions confuses) : « Le mysticisme consiste à
prétendre connaître autrement que
par l'intelligence » (Goblot).

\ib{Mystique} — Psycho. 1. Qui concerne
ou pratique le mysticisme$^1$ : « Les
états mystiques »; « Le mystique est
celui qui croit appréhender immédiatement le divin » (Delacroix).
Qqfs. péj. : « On vient avec nos mystiques [les quiétistes*] à faire un
dogme de l'indifférence du salut »
(Bossuet). Nom fém. : (la) mystique,
étude de la spiritualité mystique :
« La théologie, dont la mystique est
une branche... » (Bossuet).

— Crit 2. Qui concerne le mysticisme$^2$, ésotérique, caché : « Des
notions mystiques »; « Il y à deux
sens parfaits [de l'Écriture], le littéral et le mystique » (Pascal).
Qqfs. péj. : « C'est un mystique
(= un rêveur, un utopiste) ». Nom
% 123
fém. : (une) mystique, attachement
passionnel à une idéologie$^4$ : « La
mystique de la révolution »; « Une
mystique de l'humanité » (Ruyssen).

— Soc. 3. Chez L. Lévy-Brulhl :
« pensée mystique », type de pensée
répandu surtout dans les sociétés
primitives$^2$ et fondé sur la « croyance
à des forces, des influences, des
actions imperceptibles aux sens et
cependant réelles ». Cf. Purticipation$^2$.

\ib{Mythe} [G. muthos, récit ou parole]. —
Soc. 4. « Récit fabuleux d’origine
populaire » (Lalande) : « Les mythes
cosmogoniques » ; « Le mythe est
vécu, avant d'être formulé, fixé
dans une mythologie et revivifié par
un rituel » (Leenhardt).

— Hist. 2. Lxposé d’une doctrine sous forme de récit allégorique :
« Les mythes platoniciens ».

— Vulg. 3. Représentation idéale
de l'avenir. Chez G. Sorel, opp. à
l'utopie qui est une construction
intellectuelle, tandis que le mythe
(vg. de la grève générale, de la révolution) exprime l'instinct profond
d’une classe infériorisée.

\ib{Mythomanie} — Ps. path. Tendance
pathologique (not. dans l’hystérie,
l'infantilisme, etc.) à la fabulation*
et à la simulation.

	\end{itemize}
