
	\begin{itemize}[leftmargin=1cm, label=\ding{32}, itemsep=1pt]

\ib{Macrocosme} — [G. {\it macros}, grand, et {\it cosmos}, monde] — \si{Hist.} L'univers considéré comme correspondant au
% 111
{\it microcosme} où « petit monde » de l’organisme.

\ib{Macro...} — Préfixe souvent usité dans l'\si{Épist.} contemporaine pour
désigner les phénomènes vus à grande échelle, {\it p. e.} : \si{Phys.}
« échelle macroscopique », celle de nos sens ; — \si{Éc. pol.} : « L'approche
{\it macroéconomique} comporte l'étude des ensembles, des groupes, des
comportements collectifs, des {\it macroquantités}, des
{\it macrodécisions}... Comment passer du {\it micro} au {\it macro}, de
l'individuel au collectif ? » (A. Marchal).

\ib{Magie} — \si{Hist.} {\bf 1.} Ensemble de pratiques occultes par
lesquelles (surtout dans les sociétés primitives) on prétend agir sur la
nature, ces pratiques se distinguant des pratiques religieuses, soit par leur
caractère {\it coercitif} alors que celles-ci visent à se {\it concilier} les
puissances cachées (Frazer), soit par leur caractère para-social, illicite,
étranger aux rites organisés (Mauss, Durkheim), soit enfin parce que leur
caractère mystique s’est dégradé en technique de la terre alors que la
religion évoluait en économie du salut et de la vie spirituelle (Pradines). —
{\bf 2.} {\it Magie naturelle} : nom donné au {\footnotesize XVI}$^\text{e}$
siècle aux premières expériences de physique,

— {\it Ext.} {\bf 3.} Attitude qui implique des conduites différentes de
celles que nous tenons en face du réel : « L’acte d'imagination est un acte
magique : c’est une incantation destinée à faire apparaître la chose qu’on
désire » (Sartre). Cf. {\it Textes choisis}, I, p. 24 {\bf 3.}

\ib{Maïeutique} — [G. {\it maieutèr}, accoucheur] — \si{Hist.} Méthode par
laquelle Socrate se vantait d’ « accoucher » les esprits, {\it i. e.} de leur
faire découvrir la vérité qu'ils portent en eux.
% 111

\ib{Maître et esclave} — \si{Hist.} {\it Chez Hegel} : rapport de deux
consciences dont l’une s'affirme « comme existence pour soi absolue contre et
pour l’autre » (le {\it maitre}), tandis que l’autre (l’{\it esclave})
abdique son moi individuel et préfère la vie à la liberté. Cf.
{\it Esclaves}*.

\ib{Majeur} — \si{Log.} \si{form.} Dans un syllogisme : {\bf 1.} Grand*
terme. — {\bf 2.} (Au fém.). {\it Majeure}, prémisse* qui contient le
majeur$^1$.

\ib{Mal} — \si{Méta.} et \si{Mor.} {\bf 1.} « Le {\it mal métaphysique}
consiste dans la simple imperfection ». — {\bf 2.} « Le {\it mal physique},
dans la souffrance ». — {\bf 3.} « Le {\it mal moral}, dans le
péché » (Leibniz, {\it Théod.}, I, 21). — {\bf 4.} {\it Problème du mal} :
celui qui consiste à concilier l'existence du mal avec celle d’un Dieu bon et
tout-puissant.

\ib{Malheur de la conscience} — \si{Hist.} {\it Chez Hegel} : état de la
conscience « scindée à l'intérieur de soi » par la contradiction entre la
« conscience immuable » et la « conscience changeante » et par son vain
effort pour s’élever à l’objectivité.

\ib{Mana} — \si{Soc.} Principe impersonnel qui, dans les croyances des
sociétés primitives, donne l'efficacité à tout ce qui agit.

\ib{Manichéisme} — \fsb{S. norma.} \si{Hist.} {\bf 1.} Doctrine de
l’hérétique Manès ({\footnotesize III}$^\text{e}$ siècle) : « Un Dieu qui se
plairait au mal d'autrui ne saurait être distingué du {\it mauvais principe}
du Manichéisme » (Leibniz, {\it Théod.}, préf.). — {\it Ext.} {\bf 2.} Toute
doctrine qui admet le dualisme* d’un principe du bien et d’un principe du mal.

\ib{Manie} — \si{Ps. path.} Surexcitation mentale caractérisée par
l'euphorie* et l’hyperactivité, avec périodes
% 112
d’agitation motrice et de délire. Cf. {\it Circulaire}*.

\ib{Manqué (Acte)} — [trad. all. {\it Fehlleistung}] — \si{Ps. an.} Conduite
telle que lapsus, oubli, perte d'objet, etc., qui, selon Freud, résulte de
l'interférence de deux intentions et du refoulement* de l’une des deux.

\ib{Marginaux (États)} — [Angl. {\it margin}, frange] — \si{Psycho.} Halo
d’inconscient qui, selon James, forme la « frange » des états conscients.

\ib{Marginalisme} — \fsb{S. norma.} \si{Éc. pol.} Théorie de la valeur$^3$
selon laquelle celle-ci se détermine d’après l'{\it utilité marginale}
(syn. : {\it utilité limite} ou {\it finale}), {\it i. e.} celle de l’objet
d'une certaine nature ({\it p. e.} un seau d’eau) qui sert à satisfaire le
besoin le moins urgent ({\it p. e.} arroser des fleurs).

\ib{Masochisme} — \si{Ps. path.} Perversion dans laquelle le sujet prend
plaisir à s'infliger des souffrances à lui-même,

\ib{Matérialisme} — \fsb{S. norma.} (Ctr. : {\it spiritualisme}*). \si{Méta.}
{\bf 1.} {\it Str.} Doctrine selon laquelle la matière$^4$ est la réalité
première et qui nie l'existence originale de l'esprit$^5$. {\it Dist.} :
{\it a)} le {\it matérialisme mécaniste} ({\footnotesize XVIII}$^\text{e}$ et
début du {\footnotesize XIX}$^\text{e}$ siècles) qui réduit tout aux
phénomènes matériels les plus simples, les phénomènes mécaniques$^4$ ; —
{\it b)} le {\it matérialisme dialectique} (nom donné ultérieurement par F.
Engels au matérialisme de K. Marx) selon lequel les choses elles-mêmes se
développent « dialectiquement », {\it i. e.} selon un processus analogue à
celui dont Hegel avait fait la loi de la pensée (v. {\it Dialectique}$^5$),
ce mouvement dialectique de la pensée étant selon Marx « le reflet du monde
réel ». — {\bf 2.} {\it Lato.} loute doctrine qui « explique le
% 112
supérieur par l’inférieur » (Comte).

— \si{Mor.} \fsb{S. posit.} {\bf 3.} État d'esprit et direction pratique de
la vie orientée vers la recherche des jouissances et des biens matériels. $->$
{\it Dist.} ce sens des précédents avec lesquels il n’est pas nécessairement
lié.

— \si{Soc.} \fsb{S. norma.} {\bf 4.} {\it Matérialisme historique} : doctrine
de K. Marx et F. Engels selon laquelle la {\it superstructure} sociale,
politique et idéologique$^3$ de la vie est déterminée en dernière analyse par
sa {\it base}$^2$ économique, sans préjudice des actions réciproques entre
l’une et l’autre.

\ib{Matériel} — Qui se rapporte à la matière* : {\bf 1.} (opp. :
{\it formel}$^3$) aux sens 1, 2 ou 3 de ce mot. {\it Vérité matérielle} :
celle qui consiste dans l’accord de la pensée avec les données de
l’expérience; — {\bf 2.} (Opp. : {\it spirituel}$^1$) au sens 4 : « La
substance étendue est ce qu’on nomme proprement le corps$^1$ ou la substance
des choses matérielles » (Descartes, {\it Princ.}, II, 1).

\ib{Mathématiques} — \si{Épist.} Ensemble des sciences de la quantité* et de
l’ordre$^2$. \si{Math. pures} ou {\it abstraites} : arithmétique*, algèbre*,
calcul des fonctions$^1$ et infinitésimal*. \si{Math. concrètes} : géométrie*
et mécanique$^1$. \si{Math. appliquées} : trigonométrie*, géométrie*
descriptive, calcul des probabilités*. {\it Sciences physicomathématiques} :
mécanique$^1$, astronomie*, etc.

\ib{Matière} — \si{Hist.} {\bf 1.} (Syn. : {\it cause$^4$ matérielle}. Opp. :
{\it forme}$^1$). {\it Chez Aristote et les Scolastiques} (qqfs. {\it matière
première}) : sujet$^3$ indéterminé qui est le support de la forme$^1$ et qui
en fait une réalité concrète : « Un des principes d’Aristote est que la
matière par elle-même, est informe » (Buffon). — {\bf 2.} {\it Chez Kant},
voir {\it Forme}$^2$ :
% 113
« Notre activité intellectuelle élabore la matière ({\it Stoff}) brute des
impressions sensibles en une connaïssance des objets » ({\it R. pure},
introd., I).

— \si{Épist.} {\bf 3.} (Opp. : {\it forme}$^6$). Contenu (d'un jugement, d’un
raisonnement, d’une connaissance) : « L’esprit mathématique dédaigne la
matière pour ne s’attacher qu'à la forme pure » (Poincaré). D'où,
{\it ext.} : objet$^3$ ou question dont on traite.

— \si{Méta.} {\bf 4.} (Opp. : {\it esprit}$^5$), Substance qui constitue les
corps$^1$ : « La matière dont la nature consiste en cela seul qu’elle est une
chose étendue, occupe tous les espaces imaginables » (Descartes,
{\it Princ.}, II, 22) ; « L’étendue et la matière ne sont qu'une même
substance » (Malebranche, {\it Entr.}, I, 2) ; « Les éléments de la matière
peuvent se ramener à l'étendue et au mouvement » (Boutroux). {\it Chez saint
Thomas} : « matière sensible », la matière corporelle « en tant que sujet$^3$
des qualités sensibles » ; « matière intelligible », la même « en tant que
sujet de la quantité » ({\it S. th.}, I, 85, 1). {\it Chez Descartes} :
« matière subtile », sorte de fluide formé des parties les plus fines et les
plus mobiles de la matière (cf. {\it Princ.}, IV, 25).

\ib{Matrice} — \si{Math.} Nombre complexe$^2$ dont les termes groupés en
tableau rectangulaire permettent des opérations algébriques applicables
{\it not.} à la théorie de l'atome (atome {\it matriciel} de Heisenberg).

\ib{Maxime} — \si{Vulg.} {\bf 1.} Proposition exprimant un précepte : « Les
maximes sont d’un grand usage en morale et en politique » (Condillac). —
\si{Hist.}  {\bf 2.} {\it Chez Kant} (opp. : {\it loi}$^4$) : « Les principes
pratiques sont subjectifs, ce sont des {\it maximes}, quand la condition
n’est regardée comme valable
% 113
par le sujet que pour sa propre volonté; ils sont objectifs, ce sont des
{\it lois} pratiques, s’ils sont reconnus comme valables pour la volonté de
tout être raisonnable » ({\it R. pratique}, début).

\ib{Mécanicisme} — Syn. de {\it Mécanisme} aux sens 3 ou {\bf 4.}

\ib{Mécanique (nom)} — \si{Épist.} — {\bf 1.} Science mathématique du
mouvement et des causes (forces$^4$) qui le déterminent. $->$ {\it Dist.}
l'art des machines qu'il est impropre auj. d’appeler de ce nom. — {\bf 2.}
{\it Mécanique céleste} : partie de l'astronomie constituée par la théorie
mathématique des mouvements des astres.

\ib{Mécanique (adj.)} — \si{Vulg.} {\bf 3.} Qui concerne les machines :
« Arts mécaniques ».

— \si{Épist.} {\bf 4.} Qui se ramène aux notions en usage dans la
mécanique$^1$ et exclut par suite toute finalité* : « Théorie mécanique »,
« Explication mécanique ». {\it Déterminisme mécanique} : celui dont les
termes, restant extérieurs les uns aux autres, conformément au principe de
l’indépendance des mouvements, s’enchaînent d’une façon mécanique$^4$.
\si{Soc.} Voir {\it Solidarité}$^3$.

\ib{Mécanisme} — A \si{Vulg.} {\bf 1.} Combinaison d'organes en vue de la
production de certains mouvements. —  {\bf 2.} {\it Ext.} Combinaison de
fonctions : « Le mécanisme de la perception ».

— \fsb{S. norma.} \si{Phys.} {\bf 3.} Théorie scientifique qui explique les
phénomènes physiques par le mouvement (cf. {\it Cinétique}*). {\it Qqfs.},
Biol, syn. de : théorie physico-chimique* de la vie. — \si{Méta.} {\bf 4.}
(Ctr. : {\it dynamisme}$^2$). Système {\it philosophique} selon lequel la
matière$^4$ est distincte de la force$^6$ ({\it ou} de l'énergie) et qui
explique
% 114
l’ensemble des phénomènes matériels par le mouvement : « Le mécanisme est
devenu dans ces derniers temps le signe distinctif des Cartésiens (Mairan).

\ib{Médiat} — (Ctr. : {\it immédiat}*). Qui comporte quelque intermédiaire. —
\si{Log.} {\it Inférences médiates} : celles où Pon passe de la proposition
qui sert de point de départ, à la conclusion par au moins une proposition
intermédiaire ({\it p. e.} syllogisme).

\ib{Médiation} — \si{Vulg.} {\bf 1.} Action de servir de médiateur,
d’intermédiaire. {\it Spéc.}, \si{Théol.} : « La médiation du Christ » (entre
Dieu et l'homme).

— \fsb{S. abstr.} \si{Méta.} {\bf 2.} {\it Chez Hegel} : acte de négation et
de dépassement à la fois (cf. {\it Aufheben}*) qui établit le lien entre le
sujet et l’objet, le temps et l'éternité, le fini et l'infini : « La
médiation est le moment du moi qui est pour-soi, la pure négativité »
({\it Phén.}, préf., II) ; « Après avoir dans la {\it Phénoménologie} ouvert
la voie à la médiation psychologique en montrant que la prise de conscience
du moi comme sujet enveloppe la présence de l’autre$^1$, Hegel conçoit la
médiation comme la relation idéale reliant entre eux les différents moments
d’un tout; finalemert, il reconnaît en elle l'expression de l'identité entre
la logique et l’histoire » (Niel). — {\bf 3.} Plus gén. : « Nul ne réalise sa
propre vie tout seul, mais seulement par la médiation des autres
hommes » (Lavelle). — \fsb{S. concr.} {\bf 4.} Ce qui fait office de
médiation$^3$ : « L’une de ces médiations [entre la valeur et notre visée*]
est la pensée conceptuelle... Nous faisons rentrer ce que nous visons dans
l’extension d’un concept : l’idéal » (Le Senne).

\ib{Médiatiser} — {\bf 1.} Rendre médiat*. —  {\bf 2.} Servir de
médiation$^4$ : « Dès
% 114
qu'elle médiatise la grâce$^2$, la dualité du toi et du moi le cède à la
communion de l'existence divine et de l’existence humaine » (Le Senne).

\ib{Médium} — \si{Psycho.} Personne qui prétend être en relation avec les
esprits$^4$.

\ib{Méga...} — Voir {\it Micro}...*.

\ib{Mégalomanie} — Voir {\it Grandeur}$^2$.

\ib{Mélancolie} — \si{Psycho.} {\bf 1.} Tristesse vague dans laquelle le
sujet se complaît. — \si{Ps. path.} {\bf 2.} Tristesse morbide accompagnée
d’une dépression mentale générale avec anxiété, dégoût de la vie et souvent
idées délirantes d’auto-accusation. {\it Cf.} {\it Circulaire}* et
{\it Culpabilité}*.

\ib{Même (Le)} — \si{Méta.} L'identique. Cf. {\it Autre}$^2$ et
{\it Identité}$^3$.

\ib{Mémoire} — \si{Biol.} et \si{Psycho.} À. {\it Lato.} Persistance d’une
modification correspondant à une action passée avec faculté de la
reproduire : en ce sens, il y a une {\it mémoire organique} ou
{\it biologique} : « La mémoire est une fonction générale du système
nerveux » (Ribot) ; « Le rôle biologique de la mémoire apparaît indispensable
à la conservation des espèces » (Pièron). $->$ Ce sens a même été étendu
jusqu’au domaine purement physique. De même qu’on parlait {\it autref.}, à
propos de l'hystérésis magnétique, d’une « mémoire de la matière », on donne
{\it auj.} le nom de « mémoire », dans les servomécanismes dont use la
cybernétique*, au dispositif qui enregistre et met en œuvre les informations
passées.

— \si{Psycho.} Ensemble de fonctions psychiques parmi lesquelles il y a lieu
de distinguer : {\bf 2.} la {\it mémoire immédiate} qui « manifeste la
puissance de conservation, la persistance qui
% 115
est en chaque état de conscience » (Delacroix) ; — {\bf 3.} la {\it mémoire
élémentaire}, simple reviviscence* du passé sans qu'il y ait nécessairement
reconnaissance* : {\it p. e.} résurrection des {\it images}$^3$ ou des
{\it sentiments} (« mémoire affective ») ; — {\bf 4.} la {\it mémoire ppt.
dite} ou prise de conscience du passé comme tel, qui implique : {\it a)} la
fixation$^1$; {\it b)} l’évocation* ou rappel*, qui en réalité est souvent
reconstruction volontaire ; {\it c)} la reconnaissance* et la localisation$^3$
du souvenir : « Nous voyons la mémoire pénétrée d'{\it organisation}
intellectuelle » (Delacroix) ; « La mémoire est une réaction sociale dans la
condition d'absence » (Janet).

— \si{Méta.} {\bf 3.} {\it Chez Bergson} : propriété fondamentale de la
conscience, essentiellement distincte de la mémoire!-habitude : « Toute
conscience est mémoire, conservation et accumulation du passé dans le
présent » ({\it E. S.}, I). On peut, en ce sens, distinguer « deux mémoires,
dont l’une {\it imagine} [images-souvenirs] et dont l’autre {\it répète}
[« mécanismes tout montés » dans le corps]... De ces deux mémoires, la
première paraît bien être la mémoire par excellence. La seconde. est
l'{\it habitude éclairée par la mémoire} plutôt que la mémoire même »
({\it Mat.} et {\it Mém.}, II).

\ib{Mémoration, Mémorisation} — \si{Psycho.} Fonction de la mémoire par
laquelle le sujet fixe les souvenirs et {\it spéc.} apprend par cœur.

\ib{Mental} — \si{Psycho.} {\bf 1.} {\it Lato.} Syn. : {\it psychique}$^1$ :
« La vie mentale »; « Les maladies mentales ». — {\bf 2.} {\it Str.} Syn. :
{\it intellectuel}$^1$ : « Les opérations mentales. »

\ib{Mentalité} — \si{Psycho.}, \si{Soc.} Ensemble des représentations et
habitudes mentales$^2$ d’un individu ou d’un groupe.
% 115

\ib{Mentisme} — \si{Ps. path.} Exaltation mentale$^2$ morbide qui produit un
défilé rapide des idées dans lequel le sujet ne se sent plus maître de sa
pensée.

\ib{Mérite} — \si{Mor.} {\bf 1.} Valeur morale impliquant un effort de
volonté pour vaincre les obstacles (cf. « bien mériter ») : « Avoir du
mérite » ; « Le mérite est désintéressé ou il n'est pas » (Le Senne). —
{\bf 2.} Caractère d’une personne ou d’un acte qui a ou donne droit à
récompense (cf. « mériter une récompense »). En ce sens, surtout au pluriel :
« S’acquérir des mérites »,

\ib{Message} — Terme «souvent usité {\it auj.} pour désigner le contenu
significatif, soit de la pensée d’un auteur : « Le message de Barrès », soit
d’un fait ou d’un document : « Leur message [des documents historiques]
échappe à l’emprise*$^2$ des règles fondées sur l'observation de certaines
constantes » (Marrou).

\ib{Mésologique} — \si{Biol.}, \si{Soc.} Qui concerne le milieu$^1$.

\ib{Mesure} — \si{Épist.} Rapport d’une grandeur à une autre grandeur prise
comme unité « On peut définir la science mathématique en lui assignant pour
but la mesure indirecte des grandeurs » (Comte).

\ib{Méta...} — Préfixe souvent usité dans l'\si{Épist.} contemporaine pour
désigner un mode de pensée qui est {\it au delà} du savoir ordinaire : « Si
l’on veut exprimer le savoir implicitement utilisé dans le travail
d’axiomatisation de la Logique$^2$, ce n’est pas à l’intérieur de la Logique
qu’on pourra le faire, mais dans une discipline nouvelle qui prendrait pour
objet les formules de la Logique axiomatisée et les règles de leur maniement.
La {\it métalogique} joue ainsi, par rapport à la logique$^2$, le même rôle
que la {\it métamathématique} par rapport à la mathématique... Au calcul
formel, langue objective, vient ainsi se superposer une {\it métalangue} qui
comprend {\it not.} les règles de syntaxe du calcul formel » (Blanché).

\ib{Métagéométrie} — \si{Épist.} Géométrie de l'hyperespace* : « Les
géométries
% 116
non-euclidiennes ne sont que des cas particuliers de la
métagéométrie » (Brunschvicg).

\ib{Métamorale} — \si{Mor.} « Dans nos systèmes de morale$^2$ théorique se
trouvent souvent confondues des observations de faits, et des conceptions
métaphysiques que l’on pourrait plus précisément appeler
{\it métamorales} » ({\it L.} Lévy-Bruhl).

\ib{Métamorphose} — Voir {\it Métempsycose}*.

\ib{Métaphysique (nom)} — [Sur l'origine du mot, v. {\it Textes choisis}, II,
p. 259] — {\bf 1.} (Sens usuel. Syn. : {\it ontologie}). Connaissance de
« l'être en tant qu'être », {\it i. e.} de l'être absolu, et des principes
premiers : « La métaphysique traite des choses les plus immatérielles, comme
de l'être en général et en particulier de Dieu et des êtres intellectuels
faits à son image » (Bossuet) ; « J'entends par Métaphysique les vérités
générales qui peuvent servir de principes aux sciences
particulières » (Malebranche, {\it Entr.}, VI, 2) ; « Pour ces esprits [les
platoniciens], la philosophie est vraiment une {\it méta-physique}, un
mouvement au-delà, un effort non pour saisir des réalités qui expliquent,
bien qu’analogues, celles de la nature, mais pour comprendre, d’un point de
vue supérieur, la loi même en vertu de laquelle l'esprit pose spontanément
les unes et les autres » (Lagneau). — {\bf 2.} Systématisation générale et
réfléchie de la pensée : « Il [Leibniz] saisissait dans tout les principes
les plus élevés et les plus généraux, ce qui est le caractère de la
métaphysique » (Fontenelle) ; « Faire de la métaphysique, ce n’est pas autre
chose qu'organiser des idées » (Dunan). {\it D'où} : conception d’ensemble du
monde et de la vie : « Toute civilisation importante a sa
% 116 — MÉT
métaphysique » (Eucken), — {\bf 3.} « Étude des conditions générales d'une
œuvre telles qu’elles résultent de l'analyse critique qu’on peut faire par
avance de son objet et de ses présuppositions » (Lalande) : « Je veux mourir
s’il y a dans ces têtes-là le premier mot de la métaphysique de leur
art » (Diderot).

— \si{Hist.} {\bf 4.} {\it Chez Kant} : ensemble des connaissances obtenues
par la faculté de connaître {\it a priori} : « La {\it métaphysique},
connaissance spéculative de la raison totalement isolée qui s'élève tout à
fait au-dessus de l’enseignement de l'expérience et cela par de purs
concepts... » ({\it R. pure}, préf. 2$^\text{e}$ éd.). — {\bf 5.} {\it Chez
Condillac, d’Alembert et les Idéologues}$^1$ : théorie de l’origine des
idées$^4$ : « La métaphysique a pour but d'examiner la genèse de nos
idées » (D’Alembert). — {\bf 6.} {\it Chez Bergson} : connaissance intuitive
de l'absolu ({\it opp.} à la pensée discursive qui « tourne autour » de
l’objet) et, de façon privilégiée, de l'esprit : « La métaphysique est la
science qui prétend se passer de symboles » ({\it P. M.}, VI, 1903) ; « Nous
assignons à la métaphysique un objet limité, principalement l'esprit »
({\it ib.}, II, 1934). — {\bf 7.} {\it Chez les existentialistes} : recherche
où le problème « empiète» (Wahl) sur celui même qui le pose : « Aucune
question métaphysique ne peut être posée sans que le questionneur, comme tel,
ne soit luimême pris dans cette question » (Heidegger). Cf. {\it Problème}*.

\ib{Métaphysique (adj.)} — {\bf 8.} Qui relève de la métaphysique (surtout au
sens 1) : « Il faut en venir d’une nécessité physique$^1$ ou hypothétique à
qqc. qui soit une nécessité absolue ou métaphysique » (Leibniz). — {\bf 9.}
Qui est d’ordre intelligible, et non
% 117
sensible : « La vérité des choses métaphysiques, lesquelles ne dépendent
point des sens » (Descartes, 2$^\text{es}$ {\it Rép.}) ; « Tâchez de vous
accoutumer aux idées métaphysiques et de vous élever au-dessus de vos
sens » ((Malebranche, {\it Entr.}, I, 10). {\it D'où} : purement théorique
(v. {\it Hyperbolique}*), ou : très abstrait : «Je ne gais si je dois vous
entretenir des méditations que j'y ai faites; car elles sont si métaphysiques
et peu communes... » (Descartes, Méth., IV). {\it D'où} qqfs. {\it péj.} :
« La question [du pur amour] devint si subtile et si
métaphysique... » (Fontenelle).

— \si{Hist.} {\bf 10.} {\it Chez Comte} : « état métaphysique », celui dans
lequel l'homme s'efforce d'expliquer la nature intime des choses par des
entités$^3$ et où domine la tendance à argumenter au lieu d'observer. —
{\bf 11.} {\it Chez les existentialistes} : qui relève de la métaphysique au
sens 7 : « Il semble que l'inquiétude métaphysique puisse s’interpréter comme
un certain refus d’abdiquer, l’objet étant précisément ce devant quoi
j'abdique » (G. Marcel).

\ib{Métapsychique} — \si{Épist.} Étude des phénomènes parapsychiques*.

\ib{Métempirique} — \si{Crit.} Qui, sans être ppt. métaphysique$^8$, est
cependant au-delà de l'expérience sensible.

\ib{Métempsycose} — \si{Hist.} Doctrine ({\it p. e.} chez Pythagore) de la
transmigration* des âmes : « L’âme ne change de corps que peu à peu. et il y
a souvent métamorphose dans les animaux, mais jamais métempsychose ni
transmigration des âmes » (Leibniz, {\it Mon.}, 72).

\ib{Méthode} — \si{Épist.} \fsb{S. abstr.} {\bf 1.} « Art de bien disposer
une suite de plusieurs pensées ou pour découvrir, la vérité
% 117
quand nous l’ignorons ou pour la prouver aux autres quand nous la connaissons
déjà » (Port-Royal). — \fsb{S. concr.} {\bf 2.} Procédé spécial : « La
méthode des variations* concomitantes ».

\ib{Méthodologie} — Étude des méthodes propres aux différentes sciences. $->$
Dist. {\it épistémologie}* qui, même au sens 1, est plus général.

\ib{Métrologie} — \si{Techn.} Science et technique de la mesure*.

\ib{Micro...} — Préfixe souvent usité dans l'\si{Épist.} contemporaine pour
désigner les phénomènes à très petite échelle : {\it p. e.} \si{Phys.},
« échelle microscopique », au-dessous de 1µm ; « microphysique », étude des
phénomènes (atomiques et nucléaires) à l'échelle microscopique; — \si{Biol.},
« micro-évolution », celle qui correspond à la diversification des espèces
({\it opp.} : « méga-évolution », formation des groupes supérieurs) ; —
\si{Éc. pol.}, \si{Soc.}, « microéconomie », « micro sociologie », étude des
phénomènes économiques ou sociaux à l'échelle individuelle.

\ib{Milieu} — \si{Biol.}, \si{Psycho.}, \si{Soc.} {\bf 1.} Ensemble des êtres
et des phénomènes avec lesquels un être vivant se trouve en rapport.

— \si{Log.} {\bf 2.} {\it Principe du milieu} (syn. : {\it du tiers exclu}) :
de deux propositions contradictoires$^1$, si l’une est vraie, l’autre est
nécessairement fausse et réciproquement, et il n’y a pas de troisième
solution possible (cf. {\it Alternative}$^3$).

\ib{Mimique} — [G. {\it mimeisthai}, imiter] — \si{Psycho.} Ensemble des
gestes, jeux de physionomie, etc, qui imitent les réactions spontanées et par
lesquels s'expriment plus ou moins volontairement les faits de conscience :
« La mimique vocale suffit, par de simples inflexions de voix, pour changer
le sens d’une phrase » (Dumas).
% 118

\ib{Mineur} — \si{Log.} \si{form.} Dans un syllogisme : {\bf 1.} Petit*
terme. — {\bf 2.} (Au fém.). {\it Mineure}, prémisse* qui contient le
mineur$^1$.

\ib{Minimum sensible} — Voir {\it Seuil}*.

\ib{Miracle} — \si{Théol.} : {\bf 1.} « Quelque chose de difficile et
d'insolite, surpassant la puissance de la nature et l'attente du
spectateur » (saint Thomas, {\it S. th.}, I, 105, 7). — \si{Vulg.} {\bf 2.}
Fait extraordinaire ou qui paraît contraire aux lois de la nature :
« L'attraction et la direction de l’aimant sont des miracles
continuels » (Voltaire).

\ib{Misonéisme} — \si{Psycho.} Éloignement pour tout ce qui est nouveau :
« Le misonéisme des sociétés primitives est une conséquence immédiate de leur
conformisme » ({\it L.} Lévy-Bruhl).

\ib{Mnémonique} — \si{Psycho.} Qui concerne la mémoire : « Les fonctions
mnémoniques. »

\ib{Mnémotechnie} — \si{Psycho.} Procédés artificiels destinés à faciliter le
rappel des souvenirs.

\ib{Mobiles} — \si{Psycho.} {\bf 1.} {\it Lato.} Tout ce qui pousse à
l’action (y compris les motifs*) : idées, sentiments, ou tendances. —
{\bf 2.} {\it Str.} Éléments actifs$^2$ et affectifs* (tendances, sentiments,
désirs) qui poussent à l’action.

\ib{Mobilité sociale} — \si{Soc.} Aptitude d'une société au changement, soit
dans la hiérarchie* des individus ou des groupes qui la composent (mobilité
{\it verticale}), soit dans les rapports des éléments de même niveau (mob.
{\it horizontale}).

\ib{Modal} — \si{Épist.} {\bf 1.} Qui concerne les modes$^1$. {\it Chez
Descartes} ({\it Princ.}, I, 61) : « distinction modale », celle qui se fait
soit entre un mode et la substance
% 118
qu'il diversifie, soit entre deux modes d’une même substance ({\it opp.}
distinction {\it réelle} et celle qui se fait seulement {\it par la pensée}).
—  {\bf 2.} Propositions {\it modales} : celles où « l'affirmation ou la
négation est modifiée par l'un de ces quatre modes : possible, contingent,
impossible, nécessaire » (Port-Royal)

\ib{Modalité} — \si{Crit.} Propriété qui affecte la valeur de l’assertion*
dans un jugement. {\it Chez les classiques} : voir {\it Modal}$^2$. {\it Chez
Kant} : ce qui fait que le jugement est assertorique*, problématique* ou
apodictique* (v. ces mots). {\it Chez Husserl} : « modalités doxiques »,
celles qui affectent les divers modes de croyance par rapport à la
« croyance-mère » ou fondamentale : {\it p. e.} certitude, supputation,
conjecture, doute, ete. — Voir {\it Précis}, Ph. II, p. 2 {\bf 6.}

\ib{Mode (masc.)} — \si{Méta.} {\bf 1.} (Syn. : {\it modification}). Toute
détermination$^1$ d’un sujet$^3$; manière d’être : « Lorsque je dis ici façon
ou mode, je n’entends rien que ce que je nomme ailleurs attribut ou qualité.
Mais, lorsque je considère que la substance en est autrement disposée ou
diversifiée, je me sers du nom de mode ou façon » (Descartes, {\it Princ.},
I, 56). — {\bf 2.} {\it Chez Spinoza} : « les affections de la substance ;
autrement dit, ce qui est en une autre chose par le moyen de laquelle il est
aussi conçu » ({\it Eth.} I, déf. 5) ; en ce sens, mode s’{\it opp.} à
attribut$^3$ qui est une propriété essentielle : {\it p. e.} les {\it modes}
de l'étendue sont les corps$^1$.

— \si{Log.} \si{form.} {\bf 3.} Forme que prend un syllogisme selon la
quantité$^2$ et la qualité$^3$ de ses propositions. $->$ {\it Dist.}
figures$^5$.

\ib{Mode (fém.)} — \si{Soc.} {\bf 4.} {\it Str.} Ensemble d'usages qui
règnent dans une société
% 119
donnée : « Cette mode des Éthiopiens était fort bizarre et incommode; mais
c'était la mode : on la suivait avec joie » (Malebranche, R. V. II, 3, 2) ;
« Le goût, qui est personnel, est bien différent de la mode, qui est fait
social » (Goblot). — {\bf 5.} {\it Lato.} {\it Chez Tarde} : imitation$^1$
des contemporains ({\it opp.} coutume$^1$ ou imitation du passé).

\ib{Mœurs} — \si{Soc.} et \si{Mor.} {\bf 1.} Ensemble des pratiques,
sentiments et jugements relatifs au bien$^2$ et au mal$^3$ et à la conduite
en gén. : « Pour les mœurs, il est besoin qqfs. de suivre des opinions qu'on
sait être fort incertaines » (Descartes, {\it Méth.}, IV) ; « ll y a cette
différence entre les lois$^1$ et les mœurs que les lois règlent plus les
actions du citoyen et les mœurs les actions de l’homme » (Montesquieu,
{\it Lois}, XIX, 16) ; « La raison pure donne à l’homme une loit universelle
que nous appelons la loi des mœurs [{\it Sittengesetz}] » (Kant, {\it R.
pr.}, I, 1, 1, § 7). {\it Science} (ou {\it Physique}) {\it des mœurs} :
étude sociologique et positive des mœurs ({\it L.} Lévy-Bruhl, Durkheïm).

— \si{Biol.} {\bf 2.} Comportement général : « Les mœurs des abeilles ».

\ib{Moi} — Le sujet*, considéré : {\bf 1.} \si{Psycho.} (moi empirique) comme
identique à la conscience$^1$; — {\bf 2.} \si{Méta.} (moi substantiel) comme
âme$^2$ distincte de la conscience empirique; — {\bf 3.} \si{Crit.} (moi
sujet$^4$) comme pensée s'opposant à l'objet$^5$ ou non-moi : « Le {\it moi}
ne peut se onnaître que dans un rapport immédiat à quelque impression qui le
modifie » ((Biran) ; — k, comme détermination$^1$ du {\it Je}* (v. ce mot) :
« L’acte essentiel de conscience consiste pour le {\it je} à distinguer et
opposer en lui-même deux {\it moi} [le {\it moi} universel (cf.
{\it Valeur}*))
% 119
et le {\it moi} particulier ou empirique] » (Le Senne, {\it Bull.} 1932, p. 8).

— \si{Hist.} {\bf 5.} {\it Chez Kant} : « moi nouménal », le moi en tant
qu’ « il a conscience de lui-même comme chose en$^3$ soi » ({\it R. pr.} I,
3, {\it ad fin.}). —  {\bf 6.} {\it Chez Fichte} : « Moi absolu », acte
constitutif du sujet$^4$ qui, en se posant lui-même, pose à la fois le moi et
le non-moi. — Cf. {\it Ego}*.

\ib{Molaire} — \si{Biol.} Qui est le résultat d’une action d'ensemble (opp.
{\it moléculaire} : qui concerne les actions de détail des éléments de la
cellule). {\it Ext.} qui concerne l’ensemble (en qq. domaine que ce soit) :
{\it p. e.} certains psychologues opposent le point de vue {\it molaire} de
la forme$^4$ ou du comportement* total à l’atomisme$^3$ psychologique et à la
psycho. des éléments.

\ib{Monade} — [G. {\it monas}, unité] — \si{Hist.} 1. {\it Chez Platon} :
terme d’origine pythagoricienne, appliqué . aux Idées$^1$ : « On discute la
question de savoir s’il faut admettre de telles monades véritablement
existantes; puis comment chacune, tout en restant une et toujours la même,
sans génération ni dépérissement, peut être avec une parfaite constance la
même unité » ({\it Philèbe}, 15 b). — {\bf 2.} {\it Chez Leibniz} :
substance$^1$ simple, inétendue, indivisible, active, douée de perception$^3$
et d’appétition* et qui constitue l'élément dernier des choses : « Les
monades n’ont point de fenêtres par lesquelles qqc. y puisse entrer ou
sortir » ({\it Mon.}, 7). Cf. {\it Atome}$^4$ et {\it Entéléchie}$^2$. $->$
Le terme a été repris en un sens voisin par Renouvier dans sa {\it Nouvelle
Monadologie}, —— par Husserl (v. {\it Textes choisis}, I, p. 100), etc.

\ib{Monadisme} — \si{Hist.} \fsb{S. norma.} Doctrine de Leibniz sur les
monades$^2$.
% 120

\ib{Mondain} — \si{Méta.} Qui concerne le monde extérieur$^2$. $->$ On dit
aussi : {\it intramondain} ({\it opp.} {\it extramondain}*).

\ib{Monisme} — [G. {\it monos}, seul] — A, \si{Méta.} Nom générique des
doctrines qui n’admettent qu’un seul principe là où d’autres en admettent
deux ou plusieurs. Notamment : {\bf 1.} ({\it Opp.} : {\it dualisme}$^2$).
Système philosophique qui ramène tout ce qui existe, soit à la matière$^4$
(monisme matérialiste) soit à l’esprit$^5$ (mon. spiritualiste) soit à
l’idée$^1$ (mon. idéaliste). — {\bf 2.} (Opp. : {\it pluralisme}*). Doctrine
qui considère la multiplicité du devenir comme superficielle et admet l'unité
et l’intelligibilité de l’être : {\it p. e.} celle de Bradley (cf. W. James,
{\it Philo. de l’ Expérience}, leçon II).

— {\it Spéc.} {\bf 3.} Système de Hegel (parce que la thèse$^2$ et
l’antithèse$^2$ s'y dépassent dans une synthèse$^2$ supérieure). — {\bf 4.}
Système de Hæckel (forme de monisme$^1$ et de panthéisme* qui pose l’unité de
l'esprit$^1$ et de la matière$^1$ et l'identité de Dieu et du monde).

— {\bf 5.} Doctrine qui, {\it dans un domaine particulier}, {\it p. e.}
\si{Mor.}, n’admet qu'un seul principe : « Les morales$^1$ classiques
relèvent toutes d’une sorte de monisme moral » (Parodi).

\ib{Monogénisme} — \si{Biol.} \fsb{S. norma.} Doctrine selon laquelle : 1°
toutes les races humaines se rattachent à une espèce unique; 2° tous les
êtres humains descendent d’un couple originel unique : « La question du
monogénisme est moins simple qu’on ne l’a cru » (Le Roy).

\ib{Monolithique} — \si{Soc.} \si{Pol.} Se dit de la structure des États ou
des partis totalitaires (v. {\it Totalitarisme}*) parce que ces États ou ces
partis forment « un seul bloc » et excluent les groupes intermédiaires.

\ib{Monothéisme} — \si{Méta.} \fsb{S. norma.} Système religieux ou doctrine
philosophique qui affirme l'existence d’un Dieu unique distinct du monde.

\ib{Moral} — \si{Mor.} {\bf 1.} Qui concerne les mœurs$^1$ : « La
conscience$^3$ morale » ;
% 120
« L'obligation morale$^1$ » ; « L’essentiel dans la valeur morale
[{\it sittlich}] des actions, c’est que la loi morale [{\it moralisch}]
détermine immédiatement la volonté » (Kant, {\it R. pr.}, I, 1, 3) ; « Ces
deux caractéristiques de la vie morale [l’obligation$^1$ et la désirabilité]
se retrouvent partout où il y a fait moral » (Durkheim), — {\bf 2.} Qui
concerne la Morale$^2$ : « Les doctrines morales des philosophes » ; « Toutes
les théories morales constatent que l'individu ne peut vivre uniquement pour
lui-même » (Guyau). — {\bf 3.} (Ctr. {\it immoral}). {\it Laud.} Conforme aux
règles morales$^1$ : « Il n’est rien de si facile que de se donner l'air très
moral » (Staël).

— \si{Psycho.} et Épist. {\bf 4.} (Opp. {\it physique}$^4$). Qui concerne
l’esprit$^4$, la pensée$^1$ : « Nos maux moraux sont tous dans
l'opinion » (Rousseau) : « Le droit est un pouvoir moral [= idéal$^1$,
spirituel] » (Leibniz) ; « La société est une personne morale » (Durkheim).
{\it Le moral} : ensemble des facultés psychiques, et {\it spéc.} de celles
qui permettent de faire face aux épreuves : « Si le physique va trop bien, le
moral se corrompt » (Rousseau) ; « Remonter le moral à {\it qqn.} »
{\it Sciences morales} : celles qui étudient « le moral » de l’homme :
psychologie, histoire, sociologie, morale$^2$. — {\bf 5.} (Opp. :
{\it logique}$^7$, {\it théorique}). D'ordre sentimental ou pratique : « Je
distinguerai ici deux sortes de certitude : la première est appelée morale,
{\it i. e.} suffisante pour régler nos mœurs ou aussi grande que celle des
choses dont nous n'avons point coutume de douter touchant la conduite de la
vie » (Descartes, {\it Princ.}, IV, 205).

\ib{Morale} — \si{Soc.}, \si{Psycho.} \fsb{S. posit.} {\bf 1.} (Syn. :
{\it moralité}$^1$, {\it mœurs}$^1$). Ensemble des mœurs$^1$ et jugements
moraux$^1$ d’un
% 121
individu ou d’une société : « Bien qu’il y ait une morale du groupe, chaque
homme a sa morale à soi » (Purkheim) ; « Une morale relâchée » ; « La morale
chinoise ».

— \si{Mor.} \fsb{S. norma.} {\bf 2.} (Syn. : {\it éthique}*). Théorie,
{\it gén.} conçue sous forme normative*, de l’action humaine en tant qu’elle
est soumise au devoir$^6$ et a pour but le bien$^2$ : « La Morale est la
science des fins, la science de ce que la raison veut invinciblement, la
science de l’ordre idéal de la vie » (Rauh). — {\bf 3.} Un système
particulier de Morale$^2$ : « La morale utilitaire » ; « La morale de
Kant » ; « À elles deux, la morale hellénique et la morale chrétienne
paraissent embrasser tout l'idéal humain : l’une est la morale de
l'intelligence, l’autre est la morale de la volonté » (Boutroux).

\ib{Moralisme} — \si{Mor.} (Gén. {\it péj.}) Attitude morale : {\it a)} qui
substitue « l’honnêteté de surface » à la vertu authentique : « La corruption
de la morale, c'est le moralisme » (Guitton), — ou : {\it b)} qui sacrifie à
la valeur morale toutes les autres valeurs : « Le moralisme consiste à
traiter la valeur morale comme si elle devait se confondre avec la valeur
absolue » (Le Senne).

\ib{Moralité} — \si{Soc.} et \si{Mor.} {\bf 1.} \fsb{S. concr.} (Syn. :
{\it mœurs}$^1$ ou {\it morale}$^1$). Croyances et pratiques morales$^1$
effectives d’une société : « La science des mœurs a pour objet la moralité
positive. »

— \si{Mor.} {\bf 2.} \fsb{S. concr.} Vie morale personnelle d’un individu :
« La moralité n’est pas autre chose que la volonté rationnelle » (Hamelin) ;
« La moralité déborde la morale$^2$... [qui] s’y oppose comme la rigueur à la
souplesse » (Le Senne). — {\bf 3.} \fsb{S. abstr.} Valeur morale; caractère
de ce qui est moral$^3$ : « Nos actions tirent leur
% 121
moralité du rapport qu’elles ont avec l’ordre$^{11}$ immuable » (Malebranche,
{\it Entr.}, XIV, 7). {\it Chez Kant} : {\it opp.} légalité$^2$ (v. ce mot).

\ib{Morphème} — \si{Ling.} Élément formatif d'un mot, tel qu'affixe, suffixe,
désinence, etc.

\ib{Morphologie} — [G. {\it morphê}, forme, et {\it logos}, étude] —
\si{Épist.} Science des formes, {\it not.} : {\it a)} en \si{Ling.}, partie
de la grammaire qui étudie les morphèmes* et leurs rapports ({\it opp.}
{\it syntaxe}*) ; — {\it b)} en \si{Biol.}, description et classification des
espèces$^3$ végétales et animales (botanique, zoologie) etétude deleurs
transformations (morphologie dynamique) ; — {\it c)} en \si{Soc.}, étude des
différents types de sociétés classées d’après leur volume* et leur densité*.

\ib{Moteur} — \si{Hist.} (nom). {\bf 1.} {\it Chez Aristote} : « premier
moteur », Dieu en tant qu'acte$^2$ pur et cause de tout mouvement$^2$.

— \si{Ps. phol.} (adj.). {\bf 2.} Qui se rapporte au mouvement$^1$ :
« Sensations motrices (syn. {\it kinesthésiques}) ; « Nerfs moteurs »; — ou :
qui tend au mouvement$^1$ : « La reviviscence possible d’une représentation
est en raison directe des {\it éléments moteurs} qu’elle contient » (Ribot).

\ib{Motif} — \si{Psycho.} Mobile$^1$ d'ordre intellectuel : « Le conflit des
mobiles$^2$ et des motifs ».

Motilité, Motricité\ib{} — \si{Ps. phol.} Faculté de se mouvoir soi-même.

\ib{Mouvement} — {\it Str.} \si{Phys.} {\bf 1.} Changement de position dans
l’espace en fonction du temps : « Le mouvement, pris selon l’usage commun,
n’est autre chose que l’action par laquelle un corps passe d’un lieu en un
autre » (Descartes, {\it Princ.}, II, 24).
%122
{\it Chez les cartésiens} : « quantité de mouvement », produit mv de la masse
par la vitesse : « Ils ont cru que ce qui peut se dire de la force$^5$
[{\it i. e.} qu’elle est constante] pourrait aussi se dire de la quantité de
mouvement » (Leibniz, {\it Disc. méta.}, 17). — {\it Lato.} \si{Méta.}
{\bf 2.} Changement en général. {\it Spéc.}, {\it chez Aristote} :
« entéléchie$^1$ de l’être en puissance$^2$ », {\it i. e.} la réalité en
devenir$^1$, comprenant : {\it a)} le mouvement spatial; {\it b)} le
changement qualitatif ou altération ; {\it c)} le changement quantitatif
[accroissement ou décroissance] ; {\it d)} la génération et la corruption. —
\si{Psycho.} {\bf 3.} Impulsion spirituelle : « Les mouvements de la
grâce » (Pascal, 507) ; « On ne saurait rien demander à Dieu qu'il n’en donne
le mouvement » (Bossuet) ; « L'esprit a du mouvement pour aller plus loin
[que l'impression particulière] » (Malebranche, {\it R. V.}, I, 1, 2) ; « Un
mouvement de pitié » — \si{Soc.} {\bf 4.} Changement dans l’ordre social :
« Le parti du mouvement » ; « Le mouvement des idées ».

\ib{Moyen} — \si{Vulg.} (nom). {\bf 1.} Ce qui sert à réaliser une fin$^2$ :
« Le passé et le présent sont nos moyens ; le seul avenir est notre
fin » (Pascal, 172). — \si{Log.} \si{form.} (adj). 2 {\it Moyen terme} : dans
un syllogisme, terme qui sert à établir le rapport entre le petit* et le
grand* terme. {\it Ext.}, tout intermédiaire entre deux concepts.

\ib{Mutation} — \si{Biol.} (Opp. {\it fluctuation}*). Transformation brusque
d’un type morphologique, ayant son origine dans les conditions internes de
l'organisme.

\ib{Mystère} — [G. {\it muein}, garder le silence] — \si{Hist.} {\bf 1.} Dans
l'antiquité : culte ésotérique : « Les mystères d'Éleusis ». — \si{Théol.}
{\bf 2.} Dans la religion
% 122
chrétienne : vérité révélée d'ordre transrationnel* et que nous ne pouvons
comprendre : « Le mystère de la Trinité ». — \si{Méta.} {\bf 3.} {\it Plus
gén.}, difficulté que nous ne pouvons résoudre : « Toutes choses couvrent
quelque mystère » (Pascal) ; « Je suis un être non transparent pour lui-même,
{\it i. e.} à qui son être même apparaît comme un mystère » (G. Marcel). Cf.
{\it Problème}*.

\ib{Mysticisme} — \si{Psycho.} \fsb{S. posit.} {\bf 1.} État psychique où le
sujet a le sentiment d'entrer en rapport direct avec Dieu : « S'il y a un
mysticisme faux et périlleux, il y a un mysticisme vrai et salutaire. Ce
dernier part de ce principe que nous ne pouvons pas développer en dehors de
Dieu l’être que nous tenons de Dieu » (Wehrlé).

— \si{Hist.} \fsb{S. norma.} {\bf 2.} Doctrine fondée sur le sentiment et
l'imagination plus que sur la raison et l’expérience sensible (qqfs.
{\it péj.}, et avec l’idée qu’elle repose sur des notions confuses) : « Le
mysticisme consiste à prétendre connaître autrement que par
l'intelligence » (Goblot).

\ib{Mystique} — \si{Psycho.} {\bf 1.} Qui concerne ou pratique le
mysticisme$^1$ : « Les états mystiques »; « Le mystique est celui qui croit
appréhender immédiatement le divin » (Delacroix). Qqfs. {\it péj.} : « On
vient avec nos mystiques [les quiétistes*] à faire un dogme de l'indifférence
du salut » (Bossuet). Nom fém. : ({\it la}) {\it mystique}, étude de la
spiritualité mystique : « La théologie, dont la mystique est une
branche... » (Bossuet).

— Crit {\bf 2.} Qui concerne le mysticisme$^2$, ésotérique, caché : « Des
notions mystiques » ; « Il y a deux sens parfaits [de l'Écriture], le
littéral et le mystique » (Pascal). Qqfs. {\it péj.} : « C'est un mystique
(= un rêveur, un utopiste) ». Nom
% 123
fém. : ({\it une}) {\it mystique}, attachement passionnel à une
idéologie$^4$ : « La mystique de la révolution »; « Une mystique de
l'humanité » (Ruyssen).

— \si{Soc.} {\bf 3.} {\it Chez L. Lévy-Bruhl} : « pensée mystique », type de
pensée répandu surtout dans les sociétés primitives$^2$ et fondé sur la
« croyance à des forces, des influences, des actions imperceptibles aux sens
et cependant réelles ». Cf. {\it Participation}$^2$.

\ib{Mythe} — [G. {\it muthos}, récit ou parole] — \si{Soc.} {\bf 4.} « Récit
fabuleux d’origine populaire » (Lalande) : « Les mythes cosmogoniques » ;
« Le mythe est vécu, avant d'être formulé, fixé dans une mythologie et
revivifié par un rituel » (Leenhardt).

— \si{Hist.} {\bf 2.} Exposé d’une doctrine sous forme de récit allégorique :
« Les mythes platoniciens ».

— \si{Vulg.} {\bf 3.} Représentation idéale de l'avenir. {\it Chez G. Sorel},
{\it opp.} à l'{\it utopie} qui est une construction intellectuelle, tandis
que le mythe ({\it p. e.} de la grève générale, de la révolution) exprime
l'instinct profond d’une classe infériorisée.

\ib{Mythomanie} — \si{Ps. path.} Tendance pathologique ({\it not.} dans
l’hystérie, l'infantilisme, etc.) à la fabulation* et à la simulation.

	\end{itemize}
