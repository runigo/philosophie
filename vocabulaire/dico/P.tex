
	\begin{itemize}[leftmargin=1cm, label=\ding{32}, itemsep=1pt]

\ib{Paléontologie} — \si{Épist.} Étude des
espèces$^3$ disparues, d'après les fossiles.

\ib{Pancalisme} — [G. pan, tout, et kanlon,
beau] — \si{Hist.} À. Doctrine (de
Baldwin) d’après laquelle le beau$^1$
est la valeur fondamentale, dont
dépendent toutes les autres.

\ib{Panlogisme} — \si{Hist.} À. {\bf 1.} Syn. de
Logicisme* : « Pour caractériser cette
métaphysique [de Leibniz] d’un mot,
c’est un panlogisme » (Couturat). —
 {\bf 2.} Doctrine (de Hegel) d’après laquelle tout le réel est rationnel, en
ce sens qu'il dérive du développement dialectique$^5$ de l’Idée$^1$.

— {\bf 3.} {\it Ext.} À. Doctrine selon laquelle l’action$^4$ elle-même se ramène
à la pensée$^1$ : « C’est une sorte de
panlogisme que je propose, une
réintégration finale de la vie dans
la pensée » (Blondel, {\it Bull.}, 1902,
p. 190).

\ib{Panthéisme} — [G. pan, le Tout, et theos,
Dieu] — \si{Méta.} À. {\bf 1.} ({\it Opp.} : athéisme* et théisme*). Système philosophique selon lequel tout est, non
seulement par Dieu, mais en Dieu
(cf. Immanent$^1$ et Dieu B$^\text{b}$). Ainsi
Dieu se confond : a) soit avec « l'âme
du monde » [panth. stoïcien]; b) soit
avec l’Un* absolu d’où émanent les
% 134
deux autres hypostases$^1$ de l’Intelligence et de l’Ame formatrice du
monde [panth. de Plotin]; c) soit
avec la Substance$^2$ unique dont les
modes$^2$ sont les âmes et les corps
[panth. de Spinoza] ; d) soit avec
l'Esprit$^5$ absolu où s'achève le développement dialectique de l’Idée$^1$ :
« Le panthéisme prend pour principe la consubstantialité éternelle et
nécessaire du fini et de l'infini, de
Dieu et de la nature » (Saisset).

— \si{Vulg.} À. {\bf 2.} Doctrine (vulgaire
ou littéraire) qui identifie Dieu et la
nature et aboutit au matérialisme
({\it p. e.} chez Diderot, d'Holbach, etc.) :
« Le panthéisme est la nature divinisée » (Staël). $->$ Bien dist. ces
deux sens. En aucun cas, le panthéisme ne peut être défini par la
formule : « Tout est Dieu. »

\ib{Paradigme} — [G. paradeigma, exemple]
— {\bf 1.} (En grammaire). Modèle,
exemple type : « Aimer est le paradigme des verbes du premier
groupe. »

— \si{Méta.} {\bf 2.} Type exemplaire.
Chez Platon, appliqué à l’Idée$^1$
« Peut-être se trouve-t-il au ciel un
paradigme pour celui qui, l'ayant
aperçu, veut s'y établir » (République, 592 b). {\it Cf.} Exemplarisme*.

\ib{Parallélisme} — \si{Psycho.} À. Théorie
selon laquelle les phénomènes psychiques d’une part, les phénomènes
physiologiques ({\it spéc.} nerveux) de
l'autre, constituent deux séries
telles : a) qu'aucune action ne
s’exerce de l’une sur Fautre; b) qu’à
tout phénomène psychique correspond un phénomène nerveux et
inversement.

\ib{Paralogisme} — [G. para, faussement, et
logismos, raisonnement] — \si{Log.} {\bf 1.}
Faute* formelle de raisonnement :
« Il y a des hommes qui se méprennent
% 134
en raisonnant et y font des
paralogismes » (Descartes, Méth.,
IV).— \si{Hist.} {\bf 2.} Chez Kant : « paralogismes de la raison pure », ceux qui
sont à la base de la psychologie$^2$
rationnelle et qui prétendent conclure du « je pense » à la substantialité, à la simplicité, à la personnalité de l’âme$^2$ et à l’idéalité de son
rapport avec les phénomènes extérieurs ({\it R. pure}, {\it Dial.}, II, 1).

\ib{Paramètre} — \si{Math.} {\bf 1.} Quantité
entrant dans l'équation d’une courbe
et qu’on peut faire varier sans
changer la nature de cette courbe
({\it p. e.} rayon de la circonférence dans
$x^2+ y^2=R^2$). — {\it Ext.} {\bf 2.} Tout élément dont la variation de valeur
change la solution d’un problème
sans en changer la nature.

\ib{Paramnésie} — \si{Ps. path.} (Syn. : illusion du déjà-vu). Trouble de la perception caractérisé par le sentiment
qu’on a déjà vécu un état de conscience global qui est en réalité
nouveau. $->$ {\it Dist.} pseudomnésie*.

\ib{Paranoïa} — \si{Ps. path.} Délire d’interprétation*.

Paraphasie. — \si{Ps. path.} Trouble du
langage dans lequel les mots sont
employés les uns pour les autres.

\ib{Paraphrénie} — \si{Ps. path.} Délire chronique qui prend souvent une forme
extravagante et fantastique.

\ib{Parapsychique} — \si{Psycho.} Qualifie les
faits psychiques d'apparence supranormale, tels que télépathie*, prémonitions*, transmission de pensée, etc.

\ib{Parénétique} — [G. parainein, exhorter]
— {\bf 1.} (Nom) Chez les Stoïciens : partie
de la philosophie qui traite des préceptes et des devoirs particuliers
({\it opp.} celle qui traite des principes).
% 135
— {\bf 2.} ({\it Adj.}). Qui exhorte à la vertu
ou à la piété : « J’ai été pressant,
onctueux et même parénétique »
(A. France) ; « Théologie parénétique ».

\ib{Parenthèses (Mise entre)} — \si{Méta.}
Dans le lang. phénoménologique
opération par laquelle le philosophe fait abstraction de certains
problèmes, {\it p. e.} de l'existence substantielle du monde extérieur ou des
conditions empiriques où se situe
l’Ego* transcendantal.

\ib{Paresseux (Argument)} — Raisonne
ment des fatalistes qui concluent de
leur doctrine à l’inutilité de l'effort
(Leibniz, {\it Théod.}, préf.).

\ib{Parfait} — [{\it L.} per-fectus, achevé] — \si{Vulg.}
1. (Souvent, mais non nécessairement {\it laud.}). Complet, achevé dans
un ordre d'idées particulier : « L'unité
peut être parfaite, la multitude ne
peut l'être » (Fénelon) ; « Une parfaite indifférence ». — {\it Spéc.} \si{Mor.}
 {\bf 2.} Qui possède éminemment toutes
les qualités morales : « Dieu est le
souverain bien ; de-là, il est parfait »
(Bossuet) ; « Dieu possédant la sagesse suprême et infinie agit de la
manière la plus parfaite, non seulement au sens métaphysique, mais
encore moralement parlant » (Leibniz, Disc. méta., I).

— \si{Méta.} {\bf 3.} Complet, achevé sous
tous les rapports; qui possède la
plénitude de l’être : « Je doutais et
par conséquent mon être n'était pas
tout parfait » (Descartes, Méth., IV) ;
« Le parfait est le premier et en soi
et dans nos idées, et l’imparfait n’en
est que la dégradation » (Bossuet).
D'où : « L'Être parfait », Dieu : « La
substance que nous entendons être
souverainement parfaite et dans laquelle nous ne concevons rien qui
enferme qq. défaut* ou limitation de
% 135
perfection, s'appelle Dieu » (Descartes, 2$^\text{e}$ {\it Rép.}, déf. 8). Plus ou
moins parfait : qui approche plus ou
moins de la plénitude de l'être : « Il
n’y a pas moins de répugnance* que
le plus parfait soit une suite et une
dépendance du moins parfait qu'il
y en a que de rien procède qqc. »
(id., Méth., IV).

\ib{Pari (Argument du)} — \si{Hist.} Celui
par lequel Pascal (Pensées, 233)
invite l’incrédule à « gager que Dieu
est » en comparant les chances de
perte et de gain pour son salut dans
les deux hypothèses.

\ib{Parousie} — [G. parousia, présence] —
\si{Théol.} Dans le christianisme primitif : retour du Christ attendu
comme prochain.

\ib{Par soi} — \si{Méta.} Est « par soi » l'être
qui n’a pas besoin d’autre chose
pour exister (cf. Aséité* et Substance$^2$). — Chez Descartes : « Connues par soi » [{\it L.} per se nota], les
« natures$^2$ simples », dont la notion
n’a pas besoin d’être définie à l’aide
d’une autre.

\ib{Parti} — \si{Math.} {\bf 1.} Règle des partis,
ancien nom du calcul des probabilités$^2$ : « Par les partis, vous devez
rechercher la vérité » (Pascal, 286;
ci. Pari*). — \si{Psycho.} {\bf 2.} Décision,
résolution : « Prendre parti » —
\si{Pol.} {\bf 3.} « Réunion d'hommes qui
professent la même doctrine politique » (B. Constant).

\ib{Participation} — \si{Hist.} {\bf 1.} Chez Platon,
rapport que soutiennent : a) les
êtres sensibles avec les Idées$^1$ ; b) les
Idées$^1$ entre elles. — {\bf 2.} {\it Ext.} \si{Méta.}
Union de la partie au tout, de l’être
fini à l'infini : « L’essence de Dieu
est participable par les créatures »
(Malebranche) ; « L'existence dela partie est toujours participée » (Lavelle).
% 136

— \si{Psycho.} {\bf 3.} Loi de participation : « Dans les représentations de la
mentalité primitive, les objets, êtres,
phénomènes peuvent être à la fois
eux-mêmes et autre chose qu’eux-mêmes » ({\it L.} Lévy-Bruhl) : {\it p. e.} le
primitif est à la fois lui-même et son
totem*.

\ib{Particulier} — \si{Vulg.} {\bf 1.} Syn. soit de
singulier*, soit de spécial* : {\it p. e.}
\si{Math.} « cas particuliers » = cas
spéciaux. $->$ Emploi très impropre,
quoique courant, à éviter autant
que possible :

— \si{Log.} \si{form.} {\bf 2.} (Seul sens propre). Se dit des propositions ou jugements dans lesquels l’attribut$^1$
n'est affirmé ou nié que d’une
partie de l’extension du sujet$^2$ : {\it p. e.}
« Certains hommes sont instruits »;
« Tous les hommes ne sont pas
égoïstes ».

\ib{Partie} — Ce qui est plus petit (dans
l'espace, le temps ou l'extension$^3$
logique) que le tout sans être plus
simple. $->$ {\it Dist.} Élément*, et cf.
Division*.

\ib{Passion} — {\it Latiss.} \si{Méta.} {\bf 1.} ({\it Opp.} :
action$^1$). Impression reçue par un
sujet$^3$ : qui « Ce est passion au
regard d’un sujet est toujours action
à qq. autre égard » (Descartes). —
\si{Psycho.} {\it Lato.} {\bf 2.} {\it Autref.}, tout état
affectif : « Les passions de l'âme » ;
« J'appelle passions toutes les émotions$^1$ que l’âme ressent naturellement à l’occasion des mouvements
extraordinaires des esprits$^2$ animaux » (Malebranche, R. V., V, 1).
— {\it Str.} {\bf 3.} {\it Auj.}, inclination devenue
presque exclusive et dont la prédominance amène une rupture de
l'équilibre psychique et une transformation générale de la personnalité : « La passion du jeu »; « L'amour-passion »,
% 136

\ib{Pathologique} — [G. pathos, affection, et
logos] — {\bf 1.} Chez Kant : affectif (cf.
Passion$^2$) : « Ce sentiment [du respect] n’est comparable à aucun sentiment pathologique » ({\it R. pr.}, I,
1, 3). — {\bf 2.} (Ctr. : normal*). Maladif,
morbide : « [Il faut] concevoir tous
les phénomènes de l’état pathologique comme un prolongement des
phénomènes de l'état normal, exagérés ou atténués au-delà de leurs
limites ordinaires de variation »
(Comte, {\it Cours}, 45$^\text{e}$ leçon). — {\bf 3.} Qui
étudie les cas pathologiques$^2$ : « Anatomie pathologique »; « Psychologie
pathologique », celle qui étudie les
troubles mentaux en vue d'en dégager les lois de l'état sain ({\it Dist.}
Pathologie mentale, partie de la médecine qui étudie les maladies mentales).

\ib{Patient} — {\it Cf.} Agent* et Intellect$^2$.

\ib{Patron} — \si{Méta.} Modèle, type idéal :
« Il faut à la fin parvenir à une première idée dont la cause soit comme
un patron ou un original... » (Descartes, \si{{\it Méd.}}, III) ; « Notre esprit ne
peut former par lui-même cette
image de l'infini qui n’aurait aucun
patron » (Fénelon).

\ib{Pattern} — Mot anglais qqîfs employé
comme syn. de type, {\it not.} en \si{Soc.} :
« Les patterns de culture » = types
de civilisation]. En \si{Psycho.}, à peu
près syn. de Forme$^4$.

\ib{Péché} — \si{Mor.} Faute* morale considérée
comme entachant l'âme elle-même : « La faute devient péché
au moment où elle n’est plus éprouvée comme transgression d’une
règle, mais comme diminution de
l’être même du moi. Ce qui compte,
ce n’est plus la gravité de la faute,
mais la négation par le moi de la loi
spirituelle qui fait le fond de son
être » (Nabert).
% 137

\ib{Pédagogie} — \si{Épist.} Système ou
théorie de l’éducation.

\ib{Pédologie} — \si{Épist.} {\bf 1.} [G. pedon, sol].
Étude des sols au point de vue de
leur fertilité. — {\bf 2.} [G. pais, paidos,
enfant]. Science de l’enfant au point
de vue physiologique et psychologique.

\ib{Pensée, Penser} — \si{Psycho.}, \si{Méta.} O.
Ces termes peuvent s’appliquer :
1. {\it Lato.} à l’activité psychique dans
son ensemble : « Je suis une chose
qui pense, {\it i. e.} qui doute, qui
affirme, qui nie, qui connaît peu de
choses, en ignore beaucoup, qui
aime, qui hait, qui veut, qui ne veut
pas, qui imagine aussi et qui sent »
(Descartes, \si{{\it Méd.}}, 111) ; « La pensée
toute seule est l'essence de l'esprit »
(Malebranche, R. V., III, 1, 1) ;
« J'appelle pensée tout ce que l’âme
éprouve » (Condillac) ; — {\bf 2.} {\it Str.} à
la pensée$^1$ réfléchie et organisée
« Travaillons à bien penser : voilà le
principe de la morale » (Pascal, 347) ;
« Penser est un art qui s’apprend,
comme tous les autres, même plus
difficilement » (Rousseau) ; « Penser,
c’est connaître par concepts »
(Kant, {\it R. pure}, {\it Analyt.}, I, 1, 1).

— @. 3 Une idée$^4$ particulière :
« Hasard donne les pensées » (Pascal,
370) ; « La pensée de la mort ».

\ib{Percentilage} — \si{Ps. métr.} Méthode
d’étalonnage* des tests (cf. Précis,
Ph. II, p. 225; Sc. et M., p. 334).

\ib{Perception} — \si{Psycho.} O. {\bf 1.} Fonction
par laquelle l'esprit se forme une
représentation des objets extérieurs :
« La perception est la représentation de choses situées dans l’espace à
travers de simples impressions sensibles » (Pradines). — @. {\bf 2.} Exercice
ou résultat de cette fonction : « Une
perception tactile »; « Nos jugements
% 137
ont plus d’étendue que nos perceptions » (Malebranche, R. V., III,
2, 9, 1).

— \si{Hist.} {\bf 3.} Chez les cartésiens :
toute opération de l'intelligence :
« Il y a en nous deux sortes de pensées$^1$, à savoir la perception de l’entendement et l’action de la volonté »
({\it Princ.}, I, 32) ; « Il y a une perception que nous tenons du ouï-dire... »
(Spinoza; cf. Textes choisis, II
p. 275). — 4, {\it Spéc.}, chez Leibniz :
affection$^1$ de la substance : « L'état
passager qui enveloppe et représente une multitude dans l’unité ou
dans la substance simple n’est autre
chose que ce qu’on appelle la perception, qu’on doit distinguer de
l’aperception$^1$ ou de la conscience »
({\it Mon.}, 14). D'où : « petites perceptions », les états subconscients.

\ib{Perfection} — Caractère de ce qui est
parfait* : au sens 1 : « C’est une plus
grande perfection de connaître que
de douter » (Descartes, Méth., IV) ;
— au sens 2 : « Qu'on n'aille pas
penser que la perfection$^2$ morale ou
bonté est ici confondue avec la perfection$^1$ métaphysique ou grandeur »
(Leibniz) ; — au sens 3 : « Il s'ensuit
de la perfection suprême de Dieu
que... » (id.).

\ib{Périodique (Amnésie)} — \si{Ps. path.}
Amnésie* lacunaire portant sur une
ou plusieurs périodes de la vie du
sujet$^5$.

\ib{Péripatétisme} — [G. peripateuein, se pro
mener, parce qu’Aristote enseignait
en se promenant] — \si{Hist.} À. École
ou doctrine d’Aristote.

\ib{Périphérique} — \si{Phol.} {\bf 1.} L’extrémité
périphérique d’un nerf est celle qui
innerve les organes musculaires ou
sensoriels ({\it opp.} à l'extrémité centrale
qui part des centres nerveux). —
\si{Psycho.} {\bf 2.} Voir Attention*.
%138

\ib{Persécution (Délire de)} — \si{Ps. path.}
Celui dans lequel le malade se croit
en butte à l’action malveillante de
personnages réels ou de forces imaginaires.

\ib{Persévération} — Psych. path. Sorte
d'inertie ou d’ « enlisement » mental
(dans l'hébéphrénie*, la démence
sénile, qqfs. aussi chez l'enfant) qui
se traduit par la persistance des
expressions avec impuissance à les
modifier selon les besoins de la situation.

\ib{Persona} — (mot latin] — \si{Ps. an.} Chez
Jung : sorte de « masque » qu'impose
à l’individualité son adaptation au
milieu social.

\ib{Personnalisme} — A. \si{Méta.} {\bf 1.} Chez
Renouvier : doctrine qui pose la notion de personne* comme base de
toute la philosophie. — \si{Mor.} et \si{Pol.}
 {\bf 2.} {\it Auj.}, doctrine qui pose la personne* humaine comime la valeur
fondamentale.

\ib{Personnage} — \si{Psycho.} L'homme défini par son rôle social : « La nature
du personnage combine les impulsions individuelles et les influences
externes (modèles et contrôles sociaux) » (Maisonneuve). {\it Cf.} Précis,
Ph. I, p. 509.

\ib{Personnalité, Personne} — \si{Psycho.}
1. Forme que prend.la vie psychique
chez l'homme normal et qui suppose : 1° l’individualité (cf. Individu$^1$) ; 2° la conscience$^1$; 3° une
fonction de synthèse$^4$ qui établit une
unité et une continuité dans la vie
mentale : « Dans la mesure où le moi
est conscient, il doit être toujours
plus que simple sujet$^4$ ; et nous l’appellerons personne quand nous devrons mettre en évidence qu'il est
le maître, et non l’esclave de ce qui
lui est donné » (Le Senne). — {\bf 2.} Personnalité
% 138
de base (notion élaborée
par l'anthropologie$^3$ américaine
basic personality) : « configuration
psychologique propre aux membres
d’une société donnée et qui se manifeste par un certain style de vie »
(Dufrenne).

\ib{Perspectivisme} — \si{Hist.} À. {\bf 1.} Doctrine
de Nietzsche, selon laquelle la
connaissance est relative aux besoins
vitaux de l’homme. — {\bf 2.} Doctrine qui
considère une idéologie$^4$ comme
valable seulement selon une certaine perspective : « Nous entendons
par perspective la façon globale qu’a
le sujet de concevoir les choses, en
tant que déterminée par sa position
sociale et historique » (Mannheim).

\ib{Persuader} — \si{Psycho.} Faire accepter
une opinion à {\it qqn.} en faisant appel
à ses dispositions personnelles et à
des mobiles d’ordre affectif plutôt
qu'intellectuel. Voir Conviction*, et
cf. Précis, Ph. I, p. 531.

\ib{Pessimisme} — (Ctr. : oplimisme*).
\si{Vulg.} à. © {\bf 1.} Disposition à voir le
mauvais côté des choses, les risques
d'échec. — \si{Méta.} A. {\bf 2.} Doctrine
({\it p. e.} de Schopenhauer) selon laquelle,
dans l’univers, le mal l’emporte sur
le bien : « Le pessimisme exprime
une protestation de la conscience
contre le réel au nom de la valeur »
(Lavelle).

\ib{Petit} — \si{Psycho.} {\bf 1.} Chez Malebranche :
personnes qui ont « l'esprit petit »,
celles à qui « leur capacité d’esprit »
ne permet pas « de penser à plusieurs choses en même temps »
(R. V., IL, 3, 1, 5). — {\bf 2.} (\si{Vulg.})
Mesquin : « Un petit esprit ». — {\bf 3.}
Chez Leibniz : « petites perceptions »,
v. Perception*.

— \si{Log.} \si{form.} {\bf 4.} Petit terme (syn. :
mineur$^1$) : dans un syllogisme, le
% 139
sujet$^2$ de la conclusion$^3$ (parce qu’il
a gén. la plus petite extension$^3$).

\ib{Pétition de principe} — \si{Log.} Paralogisme* où l’on prend pour principe$^1$
ce qu'il s’agit de démontrer,

\ib{Phantasme} — \si{Psycho.} Représentation imaginaire qui se produit dans
certains états névropathiques.
$->$ {\it Dist.} hallucination*.

\ib{Phénoménal} — \si{Crit.} Chez Kant : qui
est de l’ordre des phénomènes$^2$.

\ib{Phénomène} — [G. phaïnomenon, ce qui
apparaît] — \si{Épist.} {\bf 1.} Tout ce qui
se manifeste aux sens ou à la conscience, et gén. tout fait qui peut
être objet de science : « Je ferai ici
une brève description des principaux
phénomènes dont je prétends rechercher les causes » (Descartes, {\it Princ.},
III, 4) ; « Les phénomènes sociaux ».
Voir Fait*.

— \si{Crit.} {\bf 2.} Chez Kant ({\it Trad.} all. :
Erscheinung ou Phänomen. {\it Opp.}
noumène*) : tout ce qui est « objet
d'une expérience possible » dans
l’espace ou dans le temps : « Dans
le phénomène, les objets et les manières d'être que nous leur attribuons sont toujours regardés comme
qqc. de réellement donné » ({\it R. pure},
{\it Esth.}, § 8). $->$ Remarquer que
Kant dist. avec soin le phénomène
de « la simple apparence » (Schein) :
« Les prédicats du phénomène peuvent être attribués à l’objet même,
dans son rapport à nos sens. »

\ib{Phénoménisme} — \si{Hist.} A. Doctrine
qui rejette la réalité de la substance$^1$
et du noumène* et n’admet que des
phénomènes$^2$.

\ib{Phénoménologie} — {\it Lato.} {\bf 1.} (Rare
auj. en ce sens). Simple description
des phénomènes$^1$ : « La philosophie
n’est pas, comme la physique expérimentale,
% 139
une phénoménologie superficielle » (Ravaisson). — \si{Hist.}
 {\bf 2.} Chez Hegel : « phénoménologie de
l'esprit », sorte d’autobiographie de
l'esprit qui passe de la connaissance
sensible au savoir véritable, genèse
de la science$^1$. — {\bf 3.} Chez Husserl, et
gén. auj. : méthode philosophique
qui vise à saisir, par delà les êtres
empiriques et individuels, les
essences absolues de tout ce qui est.
{\it Ext.} : « La phénoménologie contemporaine [qui] a perdu la pureté husserlienne,... donne, comme allant de
soi, une primauté au senti, au perçu,
voire à l’imaginé » (Bachelard).

\ib{Philologie} — \si{Épist.} Étude spéciale d’une
langue. $->$ {\it Dist.} linguistique*.

\ib{Philosophie} — \si{Vulg.} {\bf 1.} Conception
générale plus ou moins raisonnée de
l’univers et de la vie : « La foi tient
lieu de philosophie aux chrétiens »
(Bossuet) ; « Chacun, au cours de sa
vie, adopte une philosophie » (Roustan). {\it Spéc.}, sagesse, modération
dans les désirs, force d'âme : « Supporter les épreuves avec philosophie »; « Il est bon d’être philosophe,
mais il est triste d’être obligé de se
servir de sa philosophie » (Voltaire).

— \si{Hist.} {\bf 2.} {\it Autref.}, tout savoir
rationnel, ensemble des sciences$^2$ :
« Au lieu de cette philosophie spéculative qu’on enseigne dans les écoles,
on en peut trouver une pratique,
par laquelle, connaissant la force et
les actions du feu, de l’eau, de l'air,
des astres. » (Descartes, Méth., VI;
cf. Physique$^5$) ; « L'expression philosophie naturelle est usitée en Angleterre pour désigner l’ensemble des
diverses sciences d'observation »
(Comte, {\it Cours}, avert.)). — {\bf 3.} {\it Auj.}
type de connaissance qui diffère de
la science$^2$ ppt. dite : A) soit en
nature : a) parce qu’elle fait appel à
des facultés supérieures de l’esprit
%140
humain permettant de saisir les
causes premières et la réalité absolue : « Chercher les causes premières
et les vrais principes..., ce sont particulièrement ceux qui ont travaillé
à cela qu’on a nommés philosophes »
[Descartes, {\it Princ.}, préf.] ; — b) ou
bien parce que, comme l’art, elle est
individuelle et n’est faite que d'opinions probables : « Je ne crois pas
que la philosophie soit une science :
c’est une distraction utile pour
l'esprit » (Cl. Bernard) ; — B) soit
seulement en degré, {\it not.} parce qu’elle
atteint une généralité plus haute :
« Par philosophie positive, j'entends
l’étude propre des généralités des
différentes sciences » (Comte) ; « La
philosophie est le savoir totalement
unifié » (Spencer) ; — C) soit enfin
par son objet qui porte sur les valeurs
humaines et leur fondement$^2$ : « S’enquérir toujours du fondement dernier de sa connaissance ou de son
action, c’est proprement ce que l’on
appelle philosophie » (Le Senne).
Philosophie première : chez Aristote,
la métaphysique$^1$, connaissance de
« l'être en tant qu'être », des principes premiers$^3$ et des causes premières$^4$. — {\bf 4.} Un système particulier de philosophie$^3$ : « La philosophie de Descartes », ou ensemble des
systèmes d'un pays où d’une époque : « La philosophie anglaise » ;
« La philosophie moderne ». {\it Spéc.},
au {\footnotesize XVIII}$^\text{e}$ siècle, rationalisme$^3$ des
« philosophes » qui rejetaient la
révélation : « La superstition met le
monde en flammes; la philosophie
les éteint » (Voltaire).

\ib{Phobie} — \si{Ps. path.} Crainte obsédante
et angoissante qui s'attache sans
raison à un objet déterminé ({\it not.}
dans la psychasthénie).

\ib{Phonation} — \si{Phol.} Fonction physiologique du langage articulé.
% 140

\ib{Phonème} — \si{Ling.} Élément sonore du
langage, son significatif : « Un phonème distinct est l'articulation qui
peut servir à distinguer un sens »
(Cohen).

\ib{Phonétique} — \si{Épist.} Science de la
phonation*. $->$ {\it Dist.} phonologie*.

\ib{Phonologie} — \si{Ling.} Partie de la linguistique* qui étudie les phonèmes.

\ib{Phosphène} — \si{Phol.} Impression lumineuse produite par une pression du
globe oculaire.

\ib{Phrénologie} — \si{Hist.} A. Théorie (de
Gall et Spurzheim, 1808) selon laquelle le développement des facultés
mentales se traduirait par celui des
circonvolutions cérébrales correspondantes.

\ib{Phylétique} — \si{Biol.} Qui concerne les
phyla* : « Évolution phylétique ».

\ib{Phylogénie} — \si{Biol.} ({\it Opp.} : ontogénie*). Évolution du phylum*
« L’ontogénie* reproduit la phylogénie » (loi biogénétique*, auj. contestée).

\ib{Phylum} — \si{Biol.} Série d'espèces dérivant Jes unes des autres ou liées les
unes aux autres « Le phylum
humain n’est pas un phylum comme
les autres » (Teilhard de Chardin).

\ib{Physicalisme} — \si{Épist.} À. Doctrine de
l’école de Vienne (cf. Précis, Ph. IT,
p. 24), selon laquelle il n’est d'autre
critère de vérité que la vérification
positive et empirique.

\ib{Physico...} — Voir Chimique*, Mathématique*, Téléologique*.

\ib{Physiologie} — \si{Épist.} Science des
fonctions$^2$ des organismes vivants.

\ib{Physique (adj.)} — {\bf 1.} ({\it Opp.} : rationnel$^2$
ou métaphysique$^1$). Qui concerne la
nature, telle qu’elle se révèle dans
% 141
l'expérience : « La conséquence n’est
pas toujours nécessaire necessitate
metaphysica, souvent elle n’est que
physique » (Leibniz). Qqfs {\it opp.} à
mathématique : « Astronomie physique ». — {\bf 2.} ({\it Opp.} : chimique). Qui
se rapporte à la Physique$^6$. — {\bf 3.}
({\it Opp.} : biologique et psychologique).
« Sciences physiques » (ou « physicochimiques ») : la Physique$^6$ et la
Chimie. — {\bf 4.} ({\it Opp.} : moral$^4$). Qui se
rapporte au corps$^3$ ou à la matière$^4$ :
« Souffrance physique »; « Liberté$^1$
physique » (= absence de contrainte
matérielle). Le physique : le corporel.
Au {\footnotesize XVII}$^\text{e}$ siècle, souvent écrit le
Physic : « Que de sagesse dans les
combinaisons du Physic avec le
Moral ! » (Malebranche, {\it Entr.}, XII).

\ib{Physique (nom fém.)} — \si{Épist.} {\bf 5.}
{\it Autref.}, théorie philosophique de la
nature : « La physique des stoiciens ». Chez les cartésiens : toutes
les sciences de la nature : « Toute la
philosophie est comme un arbre
dont les racines sont la métaphysique, le tronc la physique » (Descartes, {\it Princ.}, préf.). D'où même
auj. : étude entreprise d’un point de
vue positif, comme les sciences de
la nature : « Physique sociale »
(Comte), syn. de Sociologie* ; « Physique des mœurs et du droit » (Durkheim), étude positive des faits moraux et juridiques. — {\bf 6.} {\it Auj.},
science des phénomènes matériels en
tant qu'ils n’altèrent pas la structure moléculaire des corps : « La
Physique, comme toutes les sciences
de la nature, progresse par deux
voies : l’expérience, la théorie »
({\it L.} de Broglie).

\ib{Pithiatique} — [G. peïthô iatos, curable
par la persuasion] — \si{Ps. path.} Se
dit des malades et des maladies
(hystérie*, mythomanie*) guérissables par la suggestion$^1$.
% 141

\ib{Plaisir} — \si{Psycho.} {\bf 1.} {\it Lato.} Syn.
d’agréable et, en ce sens, indéfinissable. $->$ Impropre, quoique fréquent, en ce sens (voir Précis, Ph. I,
p. 379). — {\bf 2.} {\it Str.} Au sens propre,
le terme s'applique a) soit au
plaisir physique, b) soit au plaisir
moral (Précis, ib., p. 390). $->$ Dist,
bonheur* et joie*.

\ib{Plébiscite} — \si{Pol.} Consultation du
peuple sur une question à laquelle
il doit répondre par oui ou par non.
$->$ Le plébiscite a souvent servi à
légitimer le pouvoir personnel (plébiscites napoléoniens) : il est alors
une abdication de souveraineté,
Quand il porte sur une question
générale ({\it p. e.} rattachement d'une
province à une nation), il doit être
appelé referendum*.

\ib{Ploutocratie} — [G. ploutos, riche, et
cratos, pouvoir] — \si{Pol.} Régime politique où le pouvoir est exercé ou
dominé par les riches.

\ib{Pluralisme} — \si{Méta.} ({\it Opp.} : monisme*).
Nom générique des doctrines qui,
en un ordre d’idées quelconque,
posent une pluralité de principes
irréductibles, {\it spéc.} de certaines doctrines anglo-saxonnes qui posent
les êtres individuels comme irréductibles à une substance unique :
« Le pluralisme permet aux choses
d’exister individuellement ou d’avoir
chacune sa forme particulière »
(James) ; « Le pluralisme n'est
jamais un beau rêve : il est la réclamation de l’état de fait contre la
tendance naturelle de l'esprit à
transformer l'idéal en une réalité »
(Lalande).

\ib{Pneumatique, Pneumatologie} — [G.
pneuma, esprit] — His. Nom
donné autref. : {\bf 1.} {\it Lato.} à la science
des choses spirituelles ou à Ja
croyance aux « esprits » : « La pneumatologie
%142
de nos jours se nomme
spiritisme* » (Bersot) ; — {\bf 2.} {\it Str.} à
la psychologie : « Les perceptions
insensibles sont d’un grand usage
dans la pneumatique » (Leibniz,
N. E., préf., 11).

\ib{Pneumatisme} — \si{Hist.} Ancien nom
du monisme$^1$ spiritualiste$^1$ : « Ni ce
concept [de la dualité de l'âme et
du corps] ni le pneumatisme qui s’y
oppose d’un côté ni le matérialisme
qui s’y oppose de l’autre... » (Kant,
{\it R. pure}, {\it Dial.}, II, 1, 4).

\ib{Polysyllogisme} — \si{Log.} \si{form.} Ensemble de plusieurs syllogismes
enchaînés de sorte que la conclusion de l’un (prosyllogisme) serve
de prémisse au suivant (épisyllogisme).

\ib{Polytélisme} — [G. poly, plusieurs, et telos,
fin]. — « Multiplicité des fins qu’un
même moyen permet d'atteindre »
(Bouglé).

\ib{Polythéisme} — \si{Soc.} Système religieux qui admet une pluralité de
dieux ({\it p. e.} paganisme ancien).

\ib{Polyvalentes (Logiques)} — \si{Log.} Celles
qui, outre le vrai et le faux, admettent d’autres valeurs logiques, {\it p. e.}
le probable*, l’indéterminé, etc. {\it Cf.}
Précis, Ph. II, p. 2 {\bf 5.}

\ib{Positif} — \si{Vulg.} {\bf 1.} (Ctr. : imaginaire).
Réel, palpable : « Des avantages
positifs » ; « Le mot positif désigne le
réel par {\it opp.} au chimérique » (Comte,
Disc. esprit pos., I, III, 1, 2). —
 {\bf 2.} (Ctr. : vain). « Ce terme indique le
contraste de l’utile à l'oiseux »
(Comte, ib.). D'où (en parlant des
personnes) : utilitaire : « Un homme
très positif. » — {\bf 3.} Ctr. de négatif*
en tous les sens du terme. D'où :
\si{Méta.} « Ce qu’il y a de positif et de réel
dans les mouvements$^3$ de la concupiscence »
% 142
(Malebranche, Ecl. I) ; « La
privation des qualités fait le néant :
pour être, il faut avoir qqc. de
positif » (Condillac) ; \si{Math.} « Nombre
positif », affecté du signe +;
\si{Soc.} et \si{Pol.} « La société ne vit point
d'idées négatives, mais d'idées
positives » (Saint-Simon, 1821) ;
« La philosophie positive est destinée
par sa nature, non à détruire,
mais à organiser » (Comte, ib.).

— \si{Épist.} {\bf 4.} Donné dans l’expérience, ou : qui s’appuie sur les faits :
« Un fait positif » ; « Il [d’Alembert]
proscrivait tout ce qui ne tendait
pas à la découverte de vérités positives » (Condorcet) ; « Les savants
adonnés à l’étude des sciences positives, les physiologistes, les chimistes, les physiciens et les géomètres » (St-Simon). Chez Comte
({\it opp.} : théologique$^2$ et métaphysique$^{10}$) : « état positif » (cf. Elats*),
celui où l’esprit humain, renonçant
à découvrir l’origine, la destinée et
la nature intime des êtres, recherche
uniquement les lois$^5$ des phénomènes. — D'où: {\bf 5.} (Ctr. : douteux).
Certain, bien établi : « Ce qui paraît
si positif à nous, ne paraît pas tel à
tout le monde » (Massillon).

— \si{Soc.}, \si{Jur.} {\bf 6.} ({\it Opp.} : naturel).
Établi, institué : « J’appellerai la
valeur de la monnaie valeur positive parce qu’elle peut être fixée par
une loi » (Montesquieu, {\it Lois}, XII,
10). Droit positif : cf. Droit$^1$. Religions positives ({\it opp.} : religion naturelle$^2$) : celles qui existent en fait
(Christianisme, Judaïsme, Islam,
Bouddhisme, etc.). Théologie positive : celle qui étudie les documents
de la foi reçus dans l’Église (Écriture, décisions des conciles et des
papes, etc.).

\ib{Position (Géométrie de)} — \si{Épist.}
Syn. de Topologie*.
% 143

\ib{Positionnel} — Voir Thétique*.

\ib{Positivisme} — \si{Hist.} À. {\bf 1.} {\it Str.} Doctrine d'Aug. Comte. — {\bf 2.} {\it Lato.}
Doctrine (de Comte, Littré, Taine,
Goblot, et même Mill et Spencer)
qui rejette la métaphysique et fonde
la connaissance sur les faits : « Le
Positivisme cherche à atteindre
l’unité de la pensée en partant des
données réelles » (Höffding). —
 {\bf 3.} Positivisme logique : doctrine de
l’école de Vienne. Voir Physicalisme*

— \si{Vulg.} À. {\bf 4.} Souci de se tenir
aussi près que possible de la réalité :
« Envisager les choses avec positivisme. »

\ib{Possibilité, Possible} — \si{Méta.} {\it Dist.} :
1. la possibilité rationnelle (ce qui
est possible logiquement ou en droit),
{\it i. e.} ce qui n'implique pas contradiction : « La possibilité d’une chose
nous est connue a priori quand nous
résolvons la notion en ses éléments
et quand rien entre eux n’est incompatible » (Leibniz) ; « Il suffit de
prouver qu'il [Dieu] est possible
pour prouver qu’il est » (id.) ; « Les
deux contradictoires sont souvent
possibles toutes deux [séparément];
est-ce qu'elles peuvent aussi exister
toutes deux [elles ne sont pas « compossibles »] ? » (id., {\it Théod.}, 409) ; —
 {\bf 2.} I. la possibilité physique (ce qui
est possible empiriquement ou en
fait), {\it i. e.} ce qui n’est pas en contradiction avec les lois de la nature :
« Prendrons-nous notre faible connaïssance pour la mesure des possibilités physiques ? » (Bonnet) ; « La
probabilité$^2$ mathématique est la
mesure de la possibilité physique »
(Cournot) ; — {\bf 3.} © la possibilité
subjective, {\it i. e.} ce que l’on considère
comme possible$^2$ : « Il est possible
qu'il vienne ». {\it Cf.} Probabilité au
sens {\bf 1.}

% 143
\ib{Post hoc, ergo propter hoc} — [après cela,
donc à cause de cela] — \si{Log.} Paralogisme* où l’on conclut, de ce qu’un
fait B succède à un fait A, à un lien
causal entre A et B.

\ib{Postulat} — \si{Épist.} Proposition qui
n’est ni évidente ni démontrable,
mais qu’on admet comme principe
indispensable d'un système déductif
({\it p. e.} \si{Math.} : voir Euclidien*), d’une
opération logique, d’une théorie de
la pratique$^4$. — Chez Kant : « postulats de la raison pratique », la liberté, l’immortalité de l'âme et
l’existence de Dieu, parce que,
quoique indémontrables, ce sont des
croyances? indispensables à la loi$^4$
morale.

\ib{Posturale (Sensibilité)} — \si{Ps. phol.}
Partie de la sensibilité proprioceptive* qui correspond aux attitudes
ou positions du corps. Cf Précis,
Ph. I, p. 6 {\bf 8.}

\ib{Potentialité} — \si{Méta.} Puissance$^2$ ; possibilité d’être ou de n'être pas telle
chose : « La potentialité fait apparaître la dimension du futur
(Sartre).

\ib{Potentiel} — \si{Méta.} {\bf 1.} (Syn. : virtuel:
etc. : actuel$^2$). Qui existe en puissance$^2$ : « Le temps appartient à ce
type de réalités potentielles... »
(Guitton).

— \si{Phys.} {\bf 2.} Énergie potentielle :
voir Énergie*.

\ib{Pour-soi} — \si{Méta.} {\bf 1.} Chez Hegel : l'être,
en tant que, par la conscience, il
s’oppose à l’objet et rentre en soi. —
 {\bf 2.} {\it Lato.} La conscience$^1$ en gén. : « Le
pour soi ou la conscience : telle est
la synthèse à laquelle nous aspirons » (Hamelin) ; « Le pour-soi, c’est
l’en-soi se perdant comme en-soi
pour se fonder comme conscience »
(Sartre).
% 144

\ib{Pouvoir} — \si{Phys.} {\bf 1.} Propriété physique, force$^3$ : « Pouvoir absorbant » ;
« Pouvoir inducteur ». — \si{Psycho.} et
\si{Mor.} {\bf 2.} Faculté morale : « Le pouvoir de la volonté ». En notre pouvoir : expression d'origine stoïcienne
pour désigner ce qui
dépend de notre volonté : « Il n’y a
rien qui soit entièrement en notre
pouvoir que nos pensées » (Descartes,
Méth., III). — {\bf 3.} Puissance$^5$, autorité : « Et sur lui la raison a repris
son pouvoir » (Corneille) ; « Le pouvoir de l'imagination est sans
borne » (Condillac). — \si{Soc.} et \si{Pol.}
 {\bf 4.} Corps constitué qui, dans une
société, détient l'autorité : « Il y a
dans chaque État trois sortes de
pouvoirs : la puissance législatrice,
la puissance exécutrice des choses
qui dépendent du droit des gens, et
la puissance exécutrice de celles qui
dépendent du droit civil » (Montesquieu, {\it Lois}, XI, 6) ; « On n’a jamais
vu encore de pouvoir sans flatteurs »
(Alain).

\ib{Pragmatique} — \si{Hist.} Chez Kant ({\it opp.}
pratique$^2$) : qui concerne l'action
ulililaire, qui vise le succès ou Île
bien-être.

\ib{Pragmatisme} — \si{Hist.} A. Svstème
philosophique (de James, Schiller,
Dewey, etc.) selon lequel la vérité
se définit par la « réussite » en des
sens divers de ce terme. {\it Cf.} Précis,
Ph. II, p. 418; Sc., p. 419; M.,
p. 415-416; et Textes choisis, II,
p. 28 {\bf 8.}

\ib{Pratique (adj.)} — {\bf 1.} ({\it Opp.} : théorique,
spéculatif). Qui concerne l’action$^3$
utilitaire : « On en peut trouver une
[philosophie] pratique, par laquelle...
nous pourrions nous rendre comme
maîtres et possesseurs de la nature »
(Descartes, Méth., VI). — {\bf 2.} ({\it Opp.}:
pragmatique*). Qui concerne l’action
% 144
morale. {\it Spéc.}, chez Kant : « La Raison
pratique » ; « La règle pratique est
inconditionnée, donc représentée a
priori comme proposition catégoriquement pratique » ({\it R. pr.}, I, 1,
1, § 7). {\it Cf.} Théorétique$^2$.

\ib{Pratique (nom)} — (Sens général).
 {\bf 3.} (Ctr. : théorie$^1$, spéculation*).
L'action$^3$, au sens le plus large : « La
pratique n’admet pas toujours les
lenteurs de la spéculation » (Fontenelle) ; « La pratique conduit parfois
à la théorie ». — \si{Mor.} et \si{Soc.} {\bf 4.}
« Règles de la conduite individuelle
et collective, système des devoirs et
des droits, rapports moraux des
hommes entre eux » ({\it L.} Lévy-Bruhl). — \si{Vulg.} {\bf 5.} Observance ou
exercice habituel : « La pratique
des règles morales, d’une méthode,
d’une langue, des affaires, »

\ib{Précision} — \si{Épist.} {\bf 1.} S’est dit autref.
pour abstraction$^1$ : « La précision
est l’action que fait notre esprit en
séparant des choses inséparables »
(Bossuet). — {\bf 2.} {\it Auj.}, qualité de ce
qui est précis (ctr. : vague), {\it i. e.} nettement déterminé$^1$ : « Les mesures
ont atteint une précision extraordinaire en électromagnétisme et en
optique » (Langevin) ; « La précision
d'une définition » $->$ {\it Dist.}
1° exactitude qui exclut toute approximation; la précision ne requiert
qu’une approximation aussi rigoureuse que possible; — 2° vérité ou
certitude$^2$ : « La précision et la certitude sont deux qualités en ellesmêmes fort différentes » (Comte,
{\it Cours}, II).

\ib{Pré-conscient} — \si{Ps. an.} Chez Freud :
partie de l'inconscient qui est « capable de devenir consciente ».

\ib{Prédestination} — \si{Théol.} Décret divin
qui fait que chaque individu est
destiné de toute éternité à être
% 145
sauvé ou damné : « Il y a en lui
[Dieu] des raisons de la prédestination des élus » (Malebranche, {\it Entr.},
IX, 12) ; « Il y a une question à
l'égard de la prédestination à la vie
éternelle, savoir si cette destination est absolue ou respective »
(Leibniz, {\it Théod.}, I, 81).

\ib{Prédicat} — [{\it L.} prædicare, attribuer]
— \si{Log.} \si{form.} Syn. d'attribut$^1$ :
« Dans une proposition vraie, la
notion du prédicat est toujours contenue dans le sujet » (Leibniz). Voir
Quantification$^2$.

\ib{Prédicatif (Jugement)} — \si{Log.} \si{form.}
(Syn. : jug. de prédication). Celui qui
affirme ou nie un prédicat* d’un
sujet$^2$ : « Toute prédication véritable
a qq. fondement dans la nature des
choses » (Leibniz, Disc. méta., 8).
Voir Inhérence$^2$.

\ib{Préférence} — \si{Mor.} {\bf 1.} Chez Scheler :
acte émotionnel$^1$ et intentionnel,
supérieur au sentiment pur, mais
inférieur à l'amour et par lequel
s'établit une hiérarchie entre les
valeurs. — {\bf 2.} {\it Lato.} « La préférence est
l'acte même de l'évaluation : elle
est l’attribution de la valeur, l’opération par laquelle se constitue cet
ordre hiérarchique qui montre la
valeur à l’œuvre » (Lavelle).

\ib{Préformation} — \si{Biol.} À. (Syn. : emboîtement des germes. {\it Opp.} : épigénèse). Théorie (auj. abandonnée)
selon laquelle : 1° tous les organes
de l'être vivant sont préformés dans
l'embryon; 2° tous les êtres d’une
lignée sont préformés dans l'œuf des
premiers géniteurs, de sorte que
« chaque semence contient toute
l'espèce qu’elle peut conserver »
(Malebranche, {\it Entr.}, X, 3).

\ib{Prégnant} — \si{Vulg.} Significatif, expres
sif. D'où : \si{Psycho.} Dans la Gestalttheorie :
% 145
« loi de prégnance », prédominance d’une forme$^4$ privilégiée,
plus stable et plus fréquente parmi
toutes les autres possibles.

\ib{Préjugé} — \si{Crit.} Opinion$^3$ admise sans
jugement$^1$ explicite : « Les préjugés ne se détruisent pas en les
heurtant de front » (D’Alembert) ;
« Préjugé est synonyme de jugement précipité » (Destutt).

\ib{Prélèvement} — \si{Épist.} Chez Lalande
« méthode des prélèvements successifs », celle qui, dans les sciences
expérimentales, consiste « à décomposer idéellement les faits dont on
recherche la cause, de façon à isoler
des éléments entre lesquels il y ait
identité ou du moins équivalence
quantitative ».

\ib{Prélogique} — \si{Soc.} Qualificatif attri
bué par {\it L.} Lévy-Brubhl, dans ses
premiers ouvrages, à la mentalité
primitive$^2$, parce qu'elle « ne s’astreint pas avant tout, comme la
nôtre, à s'abstenir de la contradiction ». {\it Cf.} Participation$^3$.

\ib{Premier} — Qui n’est précédé par rien :
1. dans l’ordre chronologique (syn. :
primitif$^1$) : « Ce qui est premier dans
l’ordre de la psychologie, c’est le
syncrétisme$^2$ » ; — {\bf 2.} \si{Log.} selon la
relation de principe$^1$ à conséquence :
« Les principes premiers de la démonstration mathématique sont les
axiomes$^3$ ; » — {\bf 3.} \si{Crit.} eu égard à la
valeur de nos connaissances et de
leurs fondements principes premiers de la raison, notions ou vérilés
premières, ceux qui sont évidents
par eux-mêmes et qui constituent
la base de toutes nos connaissances :
« Ce qu'ils appellent le premier principe de la connaissance : il est impossible que la même chose soit et
ne soit pas » (Port-Royal, IV, 7) ; —
% 146
 {\bf 4.} \si{Méta.} dans l'ordre ontologique :
cause première, celle qui se suffit à
elle-même et qui a en elle sa raison
d’être (({\it opp.} : cause seconde, celle qui
est l’effet d’une autre cause). —
 {\bf 5.} Philosophie première : voir Philo
sophie$^3$. — {\bf 6.} Qualités premières :
voir Qualité$^2$, — {\bf 7.} Matière première :
voir Matière. — {\bf 8.} \si{Mor.}, \si{Esth.}
Fondamental ou supérieur dans
l’ordre de la valeur : « Le premier
mérite auprès des hommes, c’est de
leur être utile » (D’Alembert) ; « Le
premier de tous les biens est la
liberté» (Rousseau). Cf Primat*,

\ib{Prémisses} — \si{Log.} \si{form.} Dans un syllogisme les deux propositions principes$^1$ : majeure* et mineure*.

\ib{Prémonition} — \si{Psycho.} Pensée,
rève, etc., qui semble nous annoncer
l'avenir.

\ib{Prémotion physique} — \si{Théol.} Action
par laquelle, selon certains théologiens, Dieu agirait directement sur
la volonté humaine tout en la laissant libre.

\ib{Préoccupation} — [{\it Trad.} all. Besorgen]
— \si{Méta.} Chez Heidegger : caractère
fondamental du Dasein$^2$, qui résulte
du fait qu'il est lié au monde.

\ib{Préperception} — \si{Psycho.} Représentation anticipée d’un objet : « Nulle
part il n'y a perception sans préperception, {\it i. e.} sans conscience
préalable d’un schéma intuitif. »
(Le Roy).

\ib{Pré-réflexif} — Voir Cogito$^3$.

\ib{Prescience} — \si{Théol.} Attribut de Dieu
qui lui permet de connaître l’avenir
ou plutôt de tout voir dans « un
éternel présent ».

\ib{Présence} — \si{Hist.} {\bf 1.} Chez Plotin : ce
terme s'applique à l'Un lorsque
% 146
l’âme s’unit à lui dans l'extase :
« Tant que dure cette présence,
aucune distinction n’est possible...
La présence est meilleure que la
science » (Ennéades, VI). — {\bf 2.} Terme
fréquent dans le lang. philosophique
contemporain : « La présence totale
de l’être est déjà impliquée dans la
simple expérience que le moi fait
de sa propre existence » (Lavelle) ;
« Ce sentiment de présence qui est la
définition même de la réalité religieuse » (Brunschvicg).

\ib{Présent} — Se dit de ce qui est immédiatement saisi par l'esprit : « J'appelle claire la connaissance qui est
présente et manifeste à un esprit
attentif » (Descartes, {\it Princ.}, I, 45).

\ib{Presque-néant} — \si{Hist.} {\bf 1.} Terme em
ployé par Leibniz ({\it Théod.}, I, 19)
pour réduire le rôle du mal* dans
le monde : « La proportion de la
partie de l'univers que nous connaïissons se perdant presque dans le
néant au prix de ce qui nous est
inconnu... il se peut que tous les
maux ne soient qu’un presque-néant
en comparaison des biens ». — {\bf 2.} En
un sens très différent, V. Jankélévitch a nommé presque-rien le « passage du rien à l’être » et le « passage
de l'être au rien » (l'événement, la
mort, l'instant, etc.) : « Le presquerien correspond à une expérience$^1$
concrète » ({\it Bull.}, 1954, p. 66).

\ib{Préternaturel} — Voir Surnaturel*.

\ib{Preuve} — \si{Épist.} Ce qui amène notre
esprit à reconnaître la vérité d’une
proposition : on prouve soit par
démonstration$^1$ ({\it p. e.} en \si{Math.}) soit
par vérification* ({\it p. e.} en \si{Phys.})

« Les preuves ne convainquent que
l'esprit$^7$ » (Pascal, 252) ; « Une
preuve est un fait purement intellectuel ou un ensemble de faits purement
% 147
intellectuels, qui est condition suffisante d’un autre fait intellectuel » (Goblot).

\ib{Primaire} — \si{Psycho.} {\bf 1.} {\it Autref.}, « état
primaire », la sensation* comme première présentation à l'esprit ({\it opp.} à
l’image$^3$ ou état secondaire). —
\si{Méta.} {\bf 2.} Voir Qualité$^2$.

\ib{Primarité} — \si{Car.} ({\it Opp.} : secondarité*).
Trait de certains caractères$^3$ chez
lesquels l'effet des représentations
est immédiat et sans retentissement*.

\ib{Primitif} — Syn. de premier (v. ce
mot) en ses divers sens : {\bf 1.} (Sens
chronologique). « Les temps primitifs »; « L’axiome qui voudrait
que le primitif fût toujours le fondamental » (Bachelard) ; — 2 et {\bf 3.} \si{Log.}
et \si{Crit.} « Toute vérité est ou primitive ou dérivée : les vérités primitives sont celles dont on ne peut
rendre raison, et telles sont soit les
vérités identiques$^4$, soit les immédiates$^2$ » (Leibniz) ; — {\bf 4.} \si{Méta.} À la
fois fondamental et primitif au
sens 1 : « Les faits primitifs du sens
intime » (Biran) ; « Le fait [de
l'effort] est bien primitif puisque
nous ne pouvons en admettre aucun
autre avant lui dans l’ordre de la
connaissance » (id.).

— {\bf 5.} \si{Soc.} Un système social,
religieux, etc, est dit primitif :
1° quand il « se rencontre dans des
sociétés dont l’organisation n'est
dépassée par aucune autre en simplicité » ; 2° quand il est « possible
de l'expliquer sans faire intervenir
aucun élément emprunté à un [système] antérieur. Dans le même sens,
nous dirons de ces sociétés qu’elles
sont primitives et nous appellerons
primitif l'homme de ces sociétés »
(Durkheim) ; « La mentalité primitive » ({\it L.} Lévy-Bruhl).

\ib{Principe} — Ce dont les autres choses
découlent, ou ce qui leur sert de
norme* directrice : A) \si{Log.} Dans
l'ordre logique : {\bf 1.} dans une déduction*, proposition d’où l’on tire
d’autres propositions, dites conséquences, qui en résultent nécessairement$^\text{1b}$ : « Ceux qui sont accoutumés... à raisonner par principes »
(Pascal, 3) ; « Les principes de la
géométrie » ; — {\bf 2.} Principes rationnels ou directeurs de la connaissance :
axiomes$^1$ purement formels$^3$ qui
règlent l'exercice de la pensée logique (cf. Jdentité*, Contradiction*.
Contrariété*, Causalité*, Milieu$^2$) ;
— B) \si{Épist.} En sciences expérimentales : {\bf 3.} proposition très générale$^3$
dont les lois déjà découvertes peuvent être considérées comme les
conséquences : « Principe de la conservation de l'énergie »; « Principe
de relativité »; « Quand une loi$^5$ a
reçu une confirmation suffisante de
l'expérience, on peut ériger en
principe » (Poincaré) ; — {\bf 4.} proposition fondamentale : « Principes
de biologie, de psychologie, etc. »
(Spencer) ; — C) \si{Méta.} Dans l’ordre
ontologique : {\bf 5.} ce qui est source
d'existence ou d’action : « J’ai tâché
de trouver les principes ou premières$^4$
causes de ce qui est » (Descartes,
Méth., VI) ; « Le principe vital » ;
« Le secret principe de nos actions
[l'âme] » (Bossuet); « Les manichéens* enseignaient deux principes »
(id.) ; — {\bf 6.} d'où : ce qui fait agir,
source, cause, mobile : « Nous avons
un autre principe d'erreur : les maladies » (Pascal, 82) ; « Le travail est
le principe de toute richesse » (Vauban) ; « Il y a cette différence entre
la nature du gouvernement et son
principe, que sa nature est ce qui le
fait être tel, et son principe ce qui le
fait agir [{\it p. e.} l'honneur dans la monarchie, la vertu$^3$ dans la démocratie] »
% 148
(Montesquieu, {\it Lois}, III, 1) ;
— {\bf 7.} élément constitutif ou essentiel : « Les proportions des principes
qui entrent dans la composition du
sucre » (Lavoisier) ; « Les principes
nutritifs »; — D) \si{Esth.} et \si{Mor.}
 {\bf 8.} Norme fondamentale, règle :
« Les principes de l’art, de la morale »; « Une croyance ne devient
ppt morale que du jour où elle
résiste à toutes les raisons, où elle
est devenue un principe » (Rauh) ;
« Un homme sans principes ».

\ib{Privatif} — \si{Log.} {\bf 1.} Terme privatif :
celui qui exprime l'absence d’une
qualité ({\it p. e.} inobservable, aveugle).
— \si{Méta.} {\bf 2.} Négatif$^1$, qui implique
un manque d’être : « La nature privative du mal » (Leibniz, {\it Théod.},
préf.).

\ib{Probabilisme} — \si{Hist.} À. {\bf 1.} Doctrine
selon laquelle il n’y a que des opinions$^1$ plus ou moins probables$^2$, non
des propositions certaines$^2$. — {\bf 2.}
Probabilisme scientifique : auj. doctrine selon laquelle les lois scientifiques, ayant un caractère statistique$^2$, n’ont, en ce qui regarde les
faits singuliers, qu'une signification probabilitaire (cf. Précis, Ph. II,
p. 146; Sc. et M., p. 262).

\ib{Probabilité} — \si{Épist.} {\bf 1.} ©. Caractère
de ce qui nous parait vraisemblable,
de ce qui nous semble devoir se
réaliser de préférence à d’autres
possibles ou avoir le plus de chances
d’être vrai, sans cependant qu’on
puisse le prouver; en ce sens, la
probabilité caractérise l’opinion$^1$ :
« La probabilité, comme toute autre
modalité de la pensée, est un caractère essentiellement subjectif de nos
jugements » (Couturat). — {\bf 2.} I.
(Sens mathématique). « La probabilité est le rapport du nombre des
% 148
cas favorables au nombre total des
événements » (Borel). Calcul des
probabilités : règles à l'aide desquelles on calcule la probabilité$^2$
d’un événement futur. Lois de probabilité : les lois statistiques$^2$ (cf.
Probabilisme$^2$) : « La nouvelle Physique ne nous fournit que des lois de
probabilité » ({\it L.} de Broglie).

\ib{Probable} — {\it Autref.}, {\bf 1.} digne d'être
approuvé : « La doctrine des opinions probables » (Pascal, {\it Prov.}, 5),
celles qui, selon les casuistes, méritent d’être suivies sans cependant
être certaines.

— {\it Auj.} {\bf 2.} ©. Qui présente de la
probabilité au sens 1; qui a une
apparence de vérité : « Les actions
de la vie ne souffrant aucun délai,...
lorsqu'il n’est pas en notre pouvoir
de discerner les plus vraies opinions,
nous devons suivre les plus probables » (Descartes, Méth, III). —
 {\bf 3.} M Qui présente une certaine probabilité au sens 2 : « L'erreur probable, »

\ib{Problématique (Jugement)} — \si{Crit.}
Chez Kant ({\it opp.} : assertorique* et ;
apodictique*) : celui qui exprime une
simple possibilité$^2$.

\ib{Problème} — Question à résoudre, en
gén. : « Les problèmes scientifiques ».
« Les problèmes politiques ». $->$
G. Marcel a dist. le problème du
mystère$^3$ : « Un problème est qqc. que
je trouve tout entier devant moi,
mais que je puis par là-même cerner
et résoudre, au lieu qu’un mystère$^3$
[v. ce mol] est qqc. en quoi je suis
moi-même engagé. »

\ib{Procession} — \si{Hist.} {\bf 1.} Chez Plotin :
progrès$^1$ selon lequel une hypostase$^1$
naît d’une autre : « Les hommes
de la fin de l'antiquité et du moyen
âge pensent les choses sous la catégorie
% 149
de procession, comme ceux
du {\footnotesize XIX}$^\text{e}$ et du {\footnotesize XX}$^\text{e}$ sous la catégorie
d'évolution » (Bréhier). — {\bf 2.} Dans
la théol. chrétienne : l'Esprit Saint
« procède » du Père et du Fils, auxquels il est consubstantiel.

\ib{Prochain} — ({\it Adj.}). {\bf 1.} \si{Log.} \si{form.}
Genre prochain : le genre$^1$ immédiatement supérieur en extension$^3$
à l'espèce$^2$ considérée. — {\bf 2.} Ébist.
Cause prochaine : celle qui précède
immédiatement l'effet. — {\bf 3.} \si{Théol.}
Pouvoir prochain : pouvoir d’agir
selon la volonté de Dieu, avec l’assistance de la grâce$^2$ (cf. Pascal, {\it Prov.},
I). — (Nom). {\bf 4.} Le prochain : tous
nos semblables.

\ib{Profil psychologique} — \si{Ps. métr.} Graphique figurant les diverses aptitudes d’un sujet. {\it Cf.} Précis, Ph. IT,
p. 226 ; Sc. et M., p. 33 {\bf 5.}

\ib{Profit} — \si{Éc. pol.} Bénéfice propre de
l'entrepreneur ou du commerçant,
distinct de la rémunération du travail de direction.

\ib{Profond, Profondeur} — \si{Psycho.} Ces
termes, très usités auj., sont extrémement équivoques. Ils peuvent
désigner : {\bf 1.} la profondeur vers le
bas, {\it i. e.} l’infra-conscient et l'illogique : « L’absurde, que l’on baptise
profondeur » (Jaspers). Psychologie
des profondeurs ({\it trad.} all. Tiefenpsychologie) : celle du subconscient
ou la psychanalyse*. Chez Bergson :
« moi profond » ou « moi fondamental » ({\it D. I.}, 11), encore appelé « moi
d’en bas » ({\it D. I.}, III), celui de la
durée$^2$ pure, « confus, infiniment
mobile, inexprimable » et caractérisé par « une absurdité fondamentale » ({\it opp.} moi superficiel, tourné
vers l’action pratique$^1$) ; — {\bf 2.} la profondeur vers le haut : « La profondeur est le terme de la réflexion.
Quiconque a l'esprit véritablement
% 149
profond doit avoir la force de fixer
sa pensée fugitive » (Vauvenargues).

\ib{Progrès} — À. {\bf 1.} Développement (sans
jugement de valeur) : « Les progrès
de l'alcoolisme, de la criminalité » ;
« [La dynamique sociale$^4$] constitue
la théorie positive du progrès [ou
« développement ») social qui, en
écartant toute vaine pensée de perfectibilité absolue et illimitée... »
(Comte, {\it Cours}, 48$^\text{e}$ leçon). — À. {\bf 2.}
Transformation graduelle dans le
sens d'un mieux : « Les progrès de
l'esprit humain » (Condorcet) ; « Le
progrès de la conscience » (Brunschvicg). Le Progrès : mouvement général de la civilisation vers le mieux :
« L'idée de progrès s’est désagrégée :
on ne considère plus qu'il y ait une
liaison nécessaire entre le progrès
intellectuel et le progrès technique
d'une part, le progrès politique ou
le progrès moral de l’autre » (Hubert).

\ib{Projection} — \si{Ps. an.} Processus par
lequel « le sujet se décharge de ses
propres mouvements affectifs en les
attribuant à autrui » (Baudouin).

\ib{Pro-jet} — [{\it Trad.} all. Entwurf] — \si{Méta.}
Chez Heidegger : propriété du Dasein$^2$ d’être sans cesse jeté-en-avant
de lui-même par la préoccupation*.

\ib{Prolégomènes} — Préliminaires, introduction : {\it p. e.} Kant, Prolégomènes
à toute métaphysique future.

\ib{Propédeutique} — Étude préparatoire
à une science : « La logique, comme
propédeutique, ne forme qu'une
sorte de vestibule des sciences »
(Kant, {\it R. pure}, préf. 2° éd.).

\ib{Proposition} — \si{Log.} Énoncé verbal
d’un jugement$^2$. {\it Cf.} Assertion* et
Lexis*. — Voir aussi Fonction$^1$.

\ib{Propre} — \si{Log.} Qui appartient à
l’objet considéré et à lui seul : « La
% 150
définition$^3$ doit être propre » ({\it i. e.}
convenir au seul défini$^2$).

\ib{Propriété} — \si{Épist.} {\bf 1.} Caractère$^2$ qui
appartient à un être ({\it opp.} : Faculté$^1$).
D'où : caractère$^1$ distinctif : « Les
qualités qui sont tellement propres
à une chose qu'elles ne sauraient
convenir à d’autres se nomment
propriétés : être terminé par trois
côtés est une propriélé du triangle »
(Condillac).

— Éc, pol. et \si{Jur.} {\bf 2.} Droit$^4$ de
posséder : « La propriété est le droit
de jouir et de disposer des choses
de la manière la plus absolue pourvu
qu’on n’en fasse pas un usage prohibé
par les lois ou les règlements » ({\it C. C.},
544). $->$ {\it Dist.} le simple fait de la
possession.

\ib{Proprioceptive (Sensibilité)} — \si{Ps. phol.}
Celle qui nous renseigne sur l’activité propre de notre corps (sensations kinesthésiques* et posturales*).

\ib{Protocole} — \si{Épist.} Chez les logisticiens
({\it opp.} : Tautologie$^2$) : énoncé formulant une assertion expérimentale
vérifiable.

\ib{Protopathique} — \si{Ps. phol.} ({\it Opp.}
épicritique*). Se dit de la sensibilité
tactile profonde, qui n’apporte que
des indications vagues, mais suscite
de fortes réactions affectives.

\ib{Protreptique} — \si{Hist.} Genre littéraire
ayant pour objet la consolation
dans la douleur ou l’exhortation
à la vertu : « Aristote a écrit un
Protreptique ».

\ib{Proversion} — \si{Mor.} Mouvement qui
nous pousse « à tourner le dos au
passé pour nous porter vers l'avenir
encore indéterminé en vue de le
marquer au sceau de l'idéal » (Le
Senne) : « La morale est proversive ».
%150

\ib{Providence} — \si{Méta.} et \si{Théol.} Action
que Dieu exerce sur le monde : 1° par
l'établissement de lois$^3$ fixes (providence générale) ; 2° par des interventions particulières (providence
particulière, miracles).

\ib{Providentialisme} — \si{Méta.} À. Doctrine qui consiste à tout expliquer
dans l’univers par l'intervention de
la Providence*.

\ib{Prudence} — \si{Mor.} {\bf 1.} (Syn. : sagesse$^2$)
{\it Autref.}, une des vertus cardinales*,
consistant dans « la pénétration et
le discernement du vrai » (Cicéron,
Off., I, 5). — {\bf 2.} {\it Auj.}, circonspection.

\ib{Pseudomnésie} — \si{Psycho.} Erreur de
la mémoire qui consiste à croire
reconnaître qqc. qu’on n'a pas déjà
perçu ou inversement à prendre
pour nouveau ce qu’on à déjà vu.
$->$ {\it Dist.} paramnésie*.

\ib{Psittacisme} — [{\it L.} psittacus, perroquet]
— Pensée verbale où l’on « raisonne
en paroles sans avoir les objets
mêmes dans l’esprit », où l’on « récite
sur la foi d'autrui » (Leibniz, N. E.,
IT, 21, 31).

\ib{Psychanalyse} — \si{Épist.} Méthode thérapeutique, puis système psychologique, inaugurés par Freud, continués avec des variantes par Adler,
Jung, etc. (voir Précis, Ph. I, p. 176 ;
Sc. et M., p. 327) et qui consiste
en une interprétation des rêves,
propos spontanés, actes manqués*
d’un sujet en vue d’explorer son
inconscient et {\it spéc.} de déceler les
complexes$^3$ psychiques qui causent
chez lui des troubles mentaux ou
physiques. $->$ Le sens du terme
est qqfs. élargi jusqu’à désigner la
psychologie* clinique tout entière.
{\it Cf.} Refoulement* et Transfert*.

\ib{Psychasthénie} — \si{Ps. path.} Nom
donné par Janet à l’ensemble des
% 151
névroses* caractérisées par les obsessions*, doutes$^2$, phobies*, sentiments
d’incomplétude*, l’abaissement de
la tension* psychologique et l’affaiblissement de la « fonction du réel$^3$ ».
$->$ Janet écrit psychasténie.

\ib{Psychè} — \si{Ps. an.} « Ensemble de tous
les processus psychiques conscients
et inconscients » (Jung).

\ib{Psychiatrie} — [G. psychè, âme, et iatreia,
médecine] — \si{{\it Méd.}} Partie de la
médecine qui consiste dans l'étude
et la thérapeutique des maladies
mentales.

\ib{Psychique} — \si{Psycho.} {\bf 1.} (Syn. : mental). Qui concerne l'esprit$^1$, la
pensée$^1$ : « La Psychologie est la
science des faits psychiques » —
 {\bf 2.} Chez certains, se dit {\it spéc.} des
phénomènes parapsychiques*.

\ib{Psychodiagnostic} — \si{Psycho.} Méthode
d'exploration de la personnalité
fondée sur l'interprétation du test de
Rorschach (test des taches d’encre).

\ib{Psycho-analyse} — Syn. : psychanalyse*.

\ib{Psychodrame} — \si{Ps. path.} Improvisation d'actions dramatiques, sur
un thème donné, par un groupe de
sujets pour les guérir de troubles
psychiques analogues. {\it Cf.} Soeiométrie*.

\ib{Psycholepsie} — \si{Ps. path.} Chute de
tension* psychologique (Janet).

\ib{Psychologie} — [G. psychè, âme, et logos]
— \si{Épist.} {\bf 1.} Science positive de la vie
psychique et de la conduite*. Psychologie de conscience celle qui
prend pour objet les faits de conscience$^1$ et fait surtout appel à l’introspection*. \si{Psycho.} de comportement ou de réaction (syn. : behaviourisme$^1$) : celle qui étudie le comportement*,
% 151
les réactions$^2$ des êtres
vivants. \si{Psycho.} clinique : celle qui
repose sur « l'investigation systématique$^1$ et aussi complète que possible des cas individuels » (Lagache).
\si{Psycho.} expérimentale celle qui
utilise les techniques expérimentales (méthodes de laboratoire,
tests*, etc.). \si{Psycho.} objective : celle
qui, sans rejeter l’introspection, fait
surtout appel à l'observation extérieure (qqfs. syn. de psycho. de réaction au sens étroit, avec rejet de l’introspection : {\it p. e.} Bechterev). \si{Psycho.}
pathologique : cf. Pathologique$^3$.

— \si{Méta.} {\bf 2.} Psychologie rationnelle (ou ontologique) : étude métaphysique de l'âme$^2$ en tant que
réalité substantielle.

\ib{Psychologique} — {\bf 1.} Qui concerne la
psychologie : « Les méthodes, les
théories psychologiques ». — {\bf 2.} Souvent, mais imppt, syn. de psychique$^1$ :
« Les faits psychologiques ».

\ib{Psychologisme} — A. Tendance à faire
prédominer le point de vue psychologique$^1$, soit sur le point de vue
logique ou critique, soit sur le point
de vue sociologique : « Le psychologisme de Tarde » (cf. Précis, Ph. II,
p. 192; Se. et M., p. 192).

\ib{Psychométrie} — \si{Épist.} Étude quantitative des faits psychiques$^1$, au
point de vue de leur intensité*, de
leur durée$^1$, etc.

\ib{Psychopathie} — \si{Ps. path.} État psychique morbide, maladie mentale.

\ib{Psycho-physiologie} — \si{Épist.} Étude
des rapports entre les faits psychiques et les faits physiologiques.

\ib{Psychophysique} — \si{Épist.} {\bf 1.} Étude
expérimentale et quantitative des
rapports entre l’excitation* et la
sensation$^1$ (voir Fechner*). — {\bf 2.}
% 152
Lato, syn. de psycho-physiologie.

\ib{Psychose} — \si{Ps. path.} Maladie mentale
avec trouble des fonctions intellectuelles. $->$ {\it Dist.} névrose*.

\ib{Psychosomatique} — Qui concerne à la
fois l'esprit et le corps. Médecine
psychosomatique : celle qui se fonde
sur l’union intime du psychique et
du corporel.

\ib{Psychotechnique} — \si{Techn.} Étude
scientifique, pour des buts pratiques
(orientation professionnelle, organisation du travail, etc.), du comportement de l'individu devant une
tâche à accomplir.

\ib{Psychothérapie} — \si{{\it Méd.}} « Médecine
psychologique » (Janet), traitement
des maladies par des moyens psylogiques (suggestion, isolement, confiance, etc). $->$ {\it Dist.} psychiatrie*
où l'esprit est l’objet du traitement,
tandis qu'ici il en est le moyen.

\ib{Public} — {\bf 1.} ({\it Adj.}) Qui concerne l’ensemble des citoyens. {\it Spéc.}, \si{Jur.}
Droit public ({\it opp.} : droit privé),
celui qui concerne l'État et ses rapports avec les citoyens (cf. Droit$^3$)
— {\bf 2.} (Nom) \si{Soc.} Le public : groupe
diffus ayant pour base l'opinion$^2$ :
« L'opinion$^2$ est l'essence d’un type
de groupement social original, qui ne
peut se comprendre que par elle : le
public » (Stœtzel). — 3, Se dit qqfs
d'un public particulier : « Le public
du cinéma, des courses ».

\ib{Puérilisme} — \si{Ps. path.} Retour aux
attitudes extérieures et au langage
de l'enfant (minauderies, emploi des
diminutifs, etc.), que l’on constate
dans les tumeurs du lobe frontal, la
démence sénile, etc.

\ib{Puissance} — \si{Vulg.} {\bf 1.} Possibilité,
faculté : « Peut-être qu’il y a en moi
% 152
qq. faculté ou puissance propre à
produire ces idées » (Descartes,
\si{{\it Méd.}}, III) ; « Cet état [de l’homme]
qui tient le milieu entre deux
extrêmes, se trouve en toutes nos
puissances » (Pascal, 72) ; « Les puissances trompeuses » [sens, passions,
imagination] (id., 83).

— \si{Méta.} {\bf 2.} Chez Aristote ({\it opp.} :
Acte$^2$) : l'être à l’état virtuel*, en
voie de devenir. En puissance, virtuellement : « Peut-être que toutes
les perfections que j'attribue à la
nature d'un Dieu sont en moi en
puissance » (Descartes, \si{{\it Méd.}}, III),
— {\bf 3.} Pouvoir d'agir, causalité elficace : « La puissance de Dieu »:
« Que si l’on vient à considérer l’idée
que l'on a de cause$^1$ ou de puissance
d'agir, on ne peut douter que cette
idée ne représente qqe. de divin »
(Malebranche, R. V. VI, 2, 3) ;
« Rien n'est plus sacré que la puissance » (id., {\it Entr.}, VII, 14).

— \si{Soc.} et \si{Pol.} {\bf 4.} Pouvoir$^4$ politique, souveraineté : « Il y a dans
chaque nation un esprit général sur
lequel la puissance même est fondée »
(Montesquieu) ; « Quand on a la
puissance, on croit tout possible »
(Condillac). — {\bf 5.} D'où : {\it Ext.} Autorité, domination : « La puissance de
la volonté sur les passions »; « La
puissance de l'argent. »

\ib{Pulsion} — [{\it Trad.} all. Trieb] — \si{Ps. an.}
Instinct$^3$ (au sens freudien) en tant
qu'il pousse à agir. Cf Textes, I,
p. 217, n. {\bf 1.}

\ib{Pur} — {\bf 1.} Sans mélange d’un élément
étranger. Plaisir pur (Épicure, Bentham) : celui qui n’est pas mêlé de
peine ou de besoin. — {\bf 2.} Sans mélange d’éléments empiriques ou
sensibles : « L’entendement ou l’esprit pur » (Malebranche, R. V., III),
{\it i. e.} « l'esprit considéré en lui-même
et sans aucun rapport au corps ».
% 153
Quantité pure : le nombre, abstraction faite de toute grandeur concrète. Mathématiques pures : celles
qui ont pour objet la quantité pure,
arithmologie*, {\it Spéc.}, chez Kant :
« Je nomme pures, au sens transcendantal, toutes les représentations dans lesquelles il ne se trouve
rien qui appartienne à l'expérience
sensible [Empfindung] » ({\it R. pure},
{\it Esth.}, § 1) ; d’où : « raison pure »,
«intuitions pures » de l'espace et du
temps, « concepts purs de l’entendement » (= catégories*). — {\bf 3.} Pur
amour : voir Amour$^1$.

— Mor {\bf 4.} (Ctr. : impur). Sans
souillure : « Ames pures et innocentes » (Bossuet)) ; « Tout est pur
à ceux qui sont purs » (Massillon,
d’après saint Paul).

\ib{Pyrrhonisme} — \si{Hist.} À. {\bf 1.} Doctrine
de Pyrrhon. — {\it Ext.} {\bf 2.} Scepticisme*
radical : « Une idée de la vérité,
invincible à tout le pyrrhonisme »
(Pascal, 395). D'où : « les pyrrhoniens », les sceptiques.

\ib{Pythagorisme} — \si{Hist.} À. {\bf 1.} Doctrine
de Pythagore. — {\it Ext.} {\bf 2.} Tendance
à faire du nombre la loi suprême des
choses, idéalisme$^4$ et symbolisme
d'inspiration mathématique.

	\end{itemize}
