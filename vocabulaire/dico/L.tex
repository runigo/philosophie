
	\begin{itemize}[leftmargin=1cm, label=\ding{32}, itemsep=1pt]

\ib{Lacunaire} — \si{Ps. path.} {\bf 1.} Malade
présentant des lésions circonscrites
et gén. multiples des centres nerveux sous forme de petites cavités
dans le tissu cérébral; d’où troubles
moteurs et psychiques (démence,
gâtisme). — {\bf 2.} Amnésie lacunaire :
voir Amnésie*.

\ib{Lallation} (Syn. : babillage). — \si{Psycho.}
Sorte de pré-langage, où le tout
jeune enfant commence à émettre
des sons semi-articulés sans signification ; il précède le gazouillis*.

\ib{Langage} — {\bf 1.} Laio. Tout système de
signes$^4$ : « Le langage algébrique ».
Langage naturel ou l. d'action :
expression naturelle des états psychiques par les gestes, la physionomie, les cris, etc. — {\bf 2.} {\it Str.} Le
langage! vocal, la parole. Langage
intérieur : suite d'images verbales
et de mouvements esquissés qui
accompagnent la pensée.

\ib{Latent} — \si{Ps. an.} Chez Freud : le
« contenu latent » du rêve est sa
signification profonde, masquée par
le « contenu manileste », {\it i. e.} par les
images qui la symbolisent.

\ib{Lattice} [mot anglais = treillis, réseau]
— \si{Log.} Système logique « partiellement ordonné dans lequel toute
paire d'éléments possède une borne
supérieure et une borne inférieure »
(Piaget). {\it P. e.} pour deux classes$^1$ A
et B, la borne supérieure est le
genre* prochain dans lequel A et B
% 107
sont contenues, et la borne inférieure la plus grande des classes
contenues à la fois en A et en B.

\ib{Laxisme} — \si{Théol.} À. Doctrine impliquant une morale$^3$ relâchée, trop
indulgente.

\ib{Légalité} — \si{Jur.} {\bf 1.} (Sens usuel). Conformité aux lois$^1$ positives : « Comme
les individus, les groupes, les masses
sont tenus au respect de la légalité » (H. Michel).

— \si{Mor.} {\bf 2.} Chez Kant ({\it opp.} : moralité$^3$) : conformité extérieure à la
loi$^4$ morale : « Si la volonté se détermine conformément à la loi morale,
mais non par respect de la loi,
l’action possédera bien de la légalité, mais non de la moralité »
(R. pratique, III, début).

— Épist, {\bf 3.} Principe de légalité :
celui d’après lequel la nature obéit
à des lois$^5$. {\it Cf.} déterminisme$^2$.

\ib{Lexis} — \si{Log.} Simple énoncé d’un
contenu intellectuel sans assertion*.

\ib{Libéralisme} — \si{Vulg.} A. {\bf 1.} Attitude
générale de tolérance$^2$ et de respect
de l'indépendance d’autrui.

— \si{Pol.} À. {\bf 2.} Doctrine qui préconise la liberté$^3$ politique ou la liberté
de conscience$^4$ (par {\it opp.} à l'autorité
de l'État ou de l'Église). — {\bf 3.} Doctrine selon laquelle le meilleur
moyen de sauvegarder la liberté$^1$ et
les droits de l'initiative privée est
de restreindre le plus possible les
attributions de l’État.

— \si{Éc. pol.} À. 4, Doctrine qui préconise la liberté du travail$^2$ et des
échanges et la non-intervention de
l'État en matière économique.

\ib{Libertaire} — \si{Pol.} Partisan de l’anarchisme$^2$. $->$ {\it {\it Dist.}} libéral.

\ib{Liberté} — A) \si{Soc.} et \si{Pol.} « La Hberté
est la propriété de soi; on distingue
% 107
trois sortes de liberté : la liberté
naturelle, la liberté civile et la
liberté politique, {\it i. e.} la liberté de
l’homme (1), celle du citoyen (2) et
celle du peuple (3) » (Raynal). D'où:
1. (Liberté externe, celle qui manque
au prisonnier, à l’impotent, à qui
agit sous la menace, etc.). Absence
d’entrave ou de contrainte; pouvoir
d’agir selon sa nature ou sa volonté :
« Liberté de conscience$^4$ »; « Liberté
du travail$^2$ ». {\it Spéc.}, \si{Éc. pol.} (Liberté
économique; cf. Libéralisme$^4$).
Absence d'intervention de l'État en
matière économique : « La police
[organisation] du commerce la plus
sûre, la plus profitable à la nation
et à l'État consiste dans la pleine
liberté de la concurrence$^1$ » (Quesnay). — 2, Liberté civile : état de
l'individu qui jouit de ses droits
civils* : « Sous ce nom de liberté,
les Romains se figuraient un État
où personne ne fut sujet que de la
loi$^1$ » (Bossuet)) ; « La liberté est le
droit de faire tout ce que les lois
permettent » (Montesquieu, {\it Lois},
XI, 3). — {\bf 3.} Liberté politique : état de
l'individu qui jouit de ses droits
civiques*, {\it i. e.} qui contribue à la
confection des lois$^1$ : « Faire la loi
et lui obéir volontairement, n'est-ce
pas la plus haute expression de la
liberté? » (Lacordaire): « La souveraineté du peuple est une condition
nécessaire de la liberté » (H. Michel).

— B) \si{Psycho.} et \si{Méta.} {\bf 4.} (Liberté
morale. {\it Opp.} : impulsion), État de
l'être qui agit avec pleine conscience
et après réflexion : « Si je connaissais toujours ce qui est vrai et bon,...
je serais entièrement libre sans
jamais être indillérent » (Descartes,
\si{{\it Méd.}}, IV) ; « Toute substance a une
parfaite spontanéité, qui devient
liberté dans les substances intelligentes » (Leibniz, Disc. méta., 22). —
% 108
 {\bf 5.} (Liberté du sage. {\it Opp.} : esclavage
des passions, de l'ignorance). État
de l'être qui agit conformément au
bien et à la raison : « Il y a d’autant
plus liberté qu’on agit davantage
selon la raison » (Leibniz) ; « Dieu
seul est parfaitement libre, et les
esprits créés ne le sont qu’à mesure
qu'ils sont au-dessus des passions »
(id., {\it N.E.}, II, 21, 8) ; « Notre vraie
liberté consiste à faire prévaloir
les bons penchants sur les mauvais »
(Comte). — {\bf 6.} (Syn. : libre arbitre.
{\it Opp.} : déterminisme$^\text{a}$). Indétermination$^2$ de la volonté, celle-ci étant
considérée : a) soit comme pouvoir
d’agir sans motif (lib. d’indifférence$^2$) ;
b) soit comme pouvoir créateur
auquel le déterminisme est inapplicable; c) soit enfin comme pouvoir, propre à l'être conscient, de
se choisir tel ou tel : « Pour sentir
évidemment notre liberté$^\text{a}$, il en
faut faire l'épreuve dans les choses
où il n’y à aucune raison qui nous
penche d’un côté plutôt que d’un
autre » (Bossuet) ; « On appelle liberté$^\text{b}$ le rapport du moi concret à
l'acte qu'il accomplit; ce rapport
est indéfinissable » (Bergson, D. I,
III) ; « La liberté$^\text{c}$ n’a pas d'essence,
c’est elle au ctr. qui fait le fondement de toutes les essences » (Sartre).

Libido (mot latin = plaisir). — Ps.
an. Chez Freud : À. {\it Str.} Recherche
instinctive du plaisir, {\it spéc.} du plaisir
sexuel. — {\bf 2.} Lalo. Énergie vitale
du ça* se répartissant entre le moi
(libido narcissique) et les objets ou
les personnes (libido objectale). —
Chez Jung {\bf 3.} L'énergie vitale
en général.

\ib{Lieu} — \si{Méta.} {\bf 1.} Situation qu’occupe
un corps dans l’espace : « Le lieu
nous marque plus expressément la
situation que la grandeur ou la
% 108
figure » (Descartes, {\it Princ.}, IT, 14).
— {\bf 2.} Par anal. : « Dieu est le lieu
des esprits, de même que les espaces
sont, en un sens, le lieu des corps »
(Malebranche, R. V., III, 2, 6) ; « Le
vrai lieu des intelligences, c’est le
monde intelligible » (id., Entretiens
sur la mort, II).

— \si{Log.} 3 « Ce que les rhétoriciens et les logiciens appellent lieux,
loci argumentorum, sont certains
chefs généraux auxquels on peut
rapporter toutes les preuves dont
on se sert dans les diverses matières
que l’on traite » (Port-Royal). Lieux
de logique : le genre, l’espèce, la différence, le propre, l'accident, etc.; on
y joint « certaines maximes communes ». Lieux de métaphysique : les
causes, les effets, le tout, les parties, etc. — D'où : lieux communs,
vérités gén. reçues, banalités : « Les
lieux communs mènent le monde »
(Tocqueville).

\ib{Lignée} — \si{Biol.} Suite des individus$^1$
de même espèce$^3$ dérivant des
mêmes géniteurs. $->$ {\it {\it Dist.}} phylum*.

\ib{Limitatif} — Chez Kant : {\bf 1.} « Jugement limitatif » (syn. : « indéfini ») :
jugement affirmatif dont l’attribut
est négatif : {\it p. e.} « L’âme est immortelle » (cf. Catégories*). — 2, « Concept limitatif » : concept (tel celui de
noumène*) « qui a pour but de restreindre les prétentions de la sensibilité et qui n’est que d'un usage
négatif » ({\it R. pure}, {\it Analyt.}, II, 2, 3).

\ib{Limite} — \si{Vulg.} {\bf 1.} Ce qui borne une
portion d’espace ou de temps, l’exercice d’un pouvoir, etc. : « Les limites
de la connaissance ».

— \si{Math.} {\bf 2.} Limite d'une variable :
grandeur constante telle que la
différence entre elle et la variable
puisse devenir et rester moindre que
toute grandeur désignée.

% 109
— {\it Ext.} \si{Psycho.}, \si{Mor.} {\bf 3.} Idéal dont
on approche sans jamais l’atteindre.
— \si{Épist.} {\bf 4.} Passage à la limite :
acte intellectuel par lequel on passe
de la notion d’un progrès! continu
et indéfini à celle du terme idéal de
ce progrès. Concept-limile : celui qui
résulte d’un passage à la limite ({\it p. e.}
concepts mathématiques).

\ib{Linguistique} — \si{Épist.} Étude générale
et comparée des langues, visant à
déterminer les lois de leur évolution.
 $->$ {\it {\it Dist.}} philologie*.

\ib{Localisation} — \si{Psycho.} Opération
par laquelle nous rapportons : {\bf 1.}
nos sensations à un certain point de
notre corps (localisation des sensations corporelles) ; — 2, l’origine
de certaines de nos sensations à un
point de l’espace extérieur : {\it p. e.}
localisation du son dans l’objet sonore (loc. des sensations ou des perceptions dans l’espace) ; — {\bf 3.} nos
souvenirs à un certain moment du
passé (loc. des souvenirs).

— \si{Ps. phol.} {\bf 4.} Localisations cérébrales : correspondance entre certaines fonctions cérébrales ayant leur
siège dans des régions déterminées
du cerveau et certaines fonctions
psychiques.

\ib{Logicisme} — \si{Épist.} A. Doctrine qui
ramène toutes les relations$^1$, {\it not.}
les relations mathématiques, à des
relations de logique pure (cf. Précis,
Ph. II, p. 22 et 97 ; Sc., p. 214 ; M,
p. 214 et 446).

\ib{Logique (nom)} — \si{Vulg.} {\bf 1.} Enchaîne
ment régulier ou cohérent des idées
ou des faits : « La logique d’une
argumentation »; « La logique des
événements ».

— \si{Log.} À. {\bf 2.} (Sens usuel en philosophie). Étude normative des conditions, surtout formelles$^3$, de la vérité :
% 109
« La logique ne conduit pas le
raisonnement : elle est simplement
la théorie du raisonnement » (Goblot) ; « Nous conviendrons d'appeler
épistémologie$^2$ l'étude de la connaissance en tant que rapport entre le
sujet et l’objet et de réserver le
terme de logique pour l'analyse formelle$^3$ de la connaissance » (Piaget).

— \si{Psycho.} À. {\bf 3.} Pensée logique$^7$,
raisonnement, que celui-ci soit ou
non conforme aux règles de la Logique$^2$ : « La logique qui peut seule
donner la certitude, est l’instrument de la démonstration » (Poincaré) ; « La logique de l’enfant »;
« La logique des sentiments ».

— \si{Hist.} {\bf 4.} {\it Autref.}, psychologie
de l’intelligence$^2$. Chez Baldwin
« logique génétique », étude positive et génétique de la connaissance.
— {\bf 5.} Chez Kant : « logique transcendantale », étude « de l’entendement pur et de la connaissance de
raison par laquelle nous pensons
des objets entièrement a priori »
({\it R. pure}, I, 2). — {\bf 6.} Chez Hegel :
« science de l'Idéel$^\text{1b}$ pure, {\it i. e.} de
l’Idée dans l'élément abstrait de la
pensée »; elle se confond avec la
métaphysique.

\ib{Logique (adj)} — {\bf 7.} (Syn. : intellectuel$^2$). Qui se rapporte à l’entendement$^3$ : « Les opérations logiques ».
— {\bf 8.} (Souvent {\it opp.} psychologique).
Qui se rapporte à la Logique$^2$. — 9,
(Ctr. : illogique). {\it Laud.} Conforme
aux règles de la Logique$^2$.

\ib{Logistique} — \si{Épist.} {\bf 1.} Chez les Grecs :
art du calcul, {\it opp.} à l’arithmétique
théorique. — {\bf 2.} {\it Auj.}, Logique$^2$ mise
sous forme d’algorithme*.

\ib{Logos} — [mot grec, qui signifie à la fois
parole et raison] — \si{Hist.} {\bf 1.} Chez
les Stoïciens : « logos universel »
(koinos logos), un des noms de la
% 110
divinité suprême qui est la « raison
commune » de toutes les parties de
l'univers. — {\bf 2.} Chez les Néo-Platoniciens (et {\it spéc.} Philon) : être intelligible intermédiaire entre Dieu et
le monde, à la fois force cosmique et
parole divine. — {\bf 3.} Dans la théol.
chrétienne : le Verbe* (v. ce mot).

— {\it Ext.} \si{Méta.} {\bf 4.} La Raison (immanente ou transcendante) qui
gouverne le monde : « [Chez Hegel],
le logos des Grecs est mis en mouvement et conçu comme appartenant à l’histoire » (Scheler) ; « Le
moi tient à un principe plus haut
que lui, à une raison suprême ou
logos » (Biran, 1823).

\ib{Loi} — \si{Jur.} {\bf 1.} (Loi positive$^4$). Règle
impérative promulguée par l'autorité souveraine d’une société : « Les
lois doivent être propres au peuple
pour lequel elles sont faites » (Montesquieu, {\it Lois}, I, 3), ou ensemble de
telles règles : « La Loi ».

— {\it Ext.} Règle impérative : {\bf 2.} existant à l’état diffus* dans une société
ou un ensemble de sociétés : « Les
lois de l'honneur » ; « Les lois de la
mode » ; — {\bf 3.} exprimant la volonté
de Dieu comme législateur, soit de
la nature : « Dieu donne des lois à la
nature et les renverse quand il veut »
(Bossuet), soit de l'action humaine :
« L'ancienne et la nouvelle loi »
(celle de l'Ancien Testament et celle
de FÉvangile) ; « La loi n'a pas détruit la nature... la grâce n’a pas
détruit la loï » (Pascal, 520).

— \si{Mor.}, \si{Log.}, \si{Esth.} {\bf 4.} À Norme*
morale, logique ou esthétique : « La
loi morale » ; « L'impératif catégorique seul a la valeur d’une loi pratique » (Kant). Voir Marime$^2$.

— \si{Épist.} {\bf 5.} À [Loi naturelle ou
scientifique). Énoncé d'un rapport
constant entre phénomènes ou éléments d’un phénomène : « Les lois
% 110
du pendule » ; « Les lois sont les rapports nécessaires$^2$ qui dérivent de la
nature des choses » (Montesquieu,
{\it Lois}, I, 1) ; « Le caractère fondamental de la philosophie positive
est de regarder tous les phénomènes
comme assujettis à des lois naturelles invariables » (Comte). Qqfs,
formule générale résumant certains
faits: {\it p. e.} en psycho., « la loi d’intérêt$^3$ ».

\ib{Ludique} — \si{Psycho.} Qui concerne le
jeu : « La pensée ludique ».

\ib{Lumière naturelle} — \si{Méta.} La raison
en tant qu’elle procède de Dieu et
illumine* l'esprit humain : « La
connaissance que nous avons par la
raison naturelle, requiert deux
choses : des images$^4$ reçues des
choses sensibles et la lumière naturelle de l’intellect$^2$ grâce à laquelle
nous en retirons des conceptions
intelligibles » (Saint Thomas, S. th,
I, 12, 13) ; « La faculté de connaître
que Dieu nous a donnée, que nous
appelons lumière naturelle, n'aperçoit jamais aucun objet qui ne soit
vrai en ce qu’elle l'aperçoit » (Descartes, {\it Princ.}, I, 30).

\ib{Lumières (Philosophie des)} — [{\it Trad.}
all. Aufklärung] — \si{Hist.} Doctrines
de certains philosophes allemands de
la seconde moitié du {\footnotesize XVIII}$^\text{e}$ siècle,
caractérisées par l'appel au sens
commun, l'optimisme naïf, l’eudémonisme* et la croyance aux causes
finales et au progrès des « lumières »,
{\it i. e.} de la raison.

\ib{Lycée} — \si{Hist.} École philosophique
d’Aristote.

	\end{itemize}
