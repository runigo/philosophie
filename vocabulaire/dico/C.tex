
	\begin{itemize}[leftmargin=1cm, label=\ding{32}, itemsep=11pt]

\ib{Ça} (Trad. all. : Es). — \si{Ps. an.} Chez
Freud : ensemble des instincts$^3$ qui
constituent le fond de la personnalité :
« Le noyau de notre être est constitué
par le ténébreux ça qui ne communique pas directement avec le
monde extérieur ». $->$ On traduit
aussi Es par Soi.

\ib{Calcul} — \si{Math.} Art d'exécuter des
opérations à l’aide d’un système de
signes. — Calcul différentiel*, des
fonctions*, infinilésimal*, intégral*,
des probabilités* : voir ces mots.
$->$ Dist. arithmétique*.

\ib{Canon} — [G. canôn, règle] — \si{Log.}
Règle : « Les canons de la méthode
expérimentale » (J. S. Mill).

\ib{Canonique} — \si{Hist.} Chez les Épicuriens : la logique$^2$.

\ib{Capacité} — Voir Aptitude*.

\ib{Capital} — \si{Éc. pol.} {\bf 1.} Richesse ne
servant pas à la consommation
immédiate et mise en réserve, soit
pour être consommée plus tard, soit
pour produire des richesses nouvelles. —  {\bf 2.} Richesse qui produit un
revenu (rente, loyer, bénéfice, intérêt) à son propriétaire indépendamment du travail de celui-ci.

\ib{Capitalisme} — \si{Éc. pol.} À Régime économique caractérisé par la grande
production (industrialisme, machinisme, production en série, spécialisation
% 30
du travail) et par la propriété individuelle des capitaux.

\ib{Caractère} — \si{Log.} et \si{Épist.} {\bf 1.} Str.
(caractère propre, distinctif). Ce qui
distingue un individu$^3$ ou une espèce$^2$
des autres individus ou espèces. —
Spéc. \si{Biol.} Caractère dominateur :
dans la systématique classique,
caractère plus général qui conditionne un caractère subordonné : vg.
« vertébré » par rapport à « mammifère », « oiseau », « reptile », etc. —
Caratière dominant : dans la théorie
mendélienne, celui qui reparaît seul
à la première génération et qui
semble effacer les autres, dits récessifs, qui reparaissent cependant aux
générations suivantes (cf. Précis,
Ph. II, p. 164; Sc. et M., p. 280). —
 {\bf 2.} {\it Qqfs. Lato.} (syn. propriété). Tout
ce qui peut être attribut$^1$ d’un individu$^3$ ou d’un genre$^1$ considéré
comme sujet$^2$.

— Car.  {\bf 3.} Type de structure psychique d’un individu$^2$, (Toutefois
certains auteurs réservent ce nom
aux dispositions innées : vg. Le
Senne : « Par caractère, on entend
le noyau de dispositions foncières,
congénitales*, qui constitue la substructure somato-psychologique d’un
individu »). —  {\bf 4.} Spéc., disposition
psychique où domine l'énergie de
la volonté : « On dit d’un homme
qu’il n’a point de caractère lorsque
les traits de son âme sont faibles,
légers, changeants; mais cela même
fait un caractère$^3$ » (Vauvenargues).

\ib{Caractériel} — \si{Car.} (. Qui concerne le
caractère$^3$ : « Les types caractériels ». —  {\bf 2.} Qui présente des troubles
du caractère$^3$ : « Un enfant caractériel ».

\ib{Caractérologie} — \si{Épist.} Branche de la
psychologie qui classe et étudie les
caractères$^3$.

\ib{Cardinales (Vertus)} — \si{Mor.} La prudence$^1$, le courage, la tempérance
et la justice (cf. Cicéron, De officiis,
I, 5) : « Les quatre vertus cardinales
ont disparu avec les temps d’innocence » (Voltaire).

\ib{Carnot (Principe de)} — Voir Dégradation*.

\ib{Cartésien} — (de Cartesius, nom latinisé
de Descartes) — \si{Hist.} {\bf 1.} Qui se
rapporte à Descartes : « La doctrine
cartésienne ». —  {\bf 2.} Disciple de Descartes : « Malebranche est, de tous
les cartésiens, celui qui a le mieux
aperçu la cause de nos erreurs »
(Condillac).

\ib{Cas} — \si{Soc.} {\bf 1.} Méthode des cas particuliers (angl. : case study). Étude
d’un phénomène social chez un individu ou un groupe particulier (cf.
Précis, Ph. II, p. 197; Sc. et M.
p. 307). — \si{Mor.}  {\bf 2.} Cas de conscience.
Difficulté à laquelle se heurte la
conscience$^3$ en cas de conflit de
devoirs (cf. Précis, Ph. II, p. 308;
Sc. et M., p. 166). Qqfs., par abréviation : cas : « On ne saurait, dit le
Père, particulariser trop les cas »
(Pascal, Prov., 8).

\ib{Casuiste} — \si{Mor.} {\bf 1.} Théologien qui
s'occupe de casuistique*. —  {\bf 2.} Ext. :
« La conscience est le meilleur des
casuistes » (Rousseau).

\ib{Casuistique} — \si{Mor.} Étude des cas$^2$
de conscience.

\ib{Caste} — \si{Soc.} Voir Classe$^3$.

\ib{Catégorielle (Attitude)} — \si{Psycho.}
Celle qui implique « une appréhension$^1$ conceptuelle des rapports » et
dégage, par suite, la signification
des situations concrètes,

\ib{Catégories} — [G. catégorein, attribuer]
— \si{Crit} Concepts très généraux
exprimant les diverses relations que
% 31
nous pouvons établir entre nos idées.
Chez Aristote, il y a dix catégories :
l'essence, la qualité, la quantité, la
relation, l’action, la passion, le lieu,
le temps, la situation, la manière
d’être. Chez Kant, il y a douze « catégories de l’entendement » se distribuant sous les quatre chefs : quantité*, qualité*, relation*, modalité*
(voir le tableau dans notre Précis,
Ph. I, p. 487).

\ib{Catégorique} — (Ctr. : hypothétique*).
\si{Log.} {\bf 1.} Proposition catégorique :
celle où l’assertion ne dépend d’aucune condition, i. e. qui n’est ni
hypothétique*, ni disjonctive*.

— \si{Mor.}  {\bf 2.} Impératif catégorique
(Kant) : le Devoir$^6$, parce qu'il
commande sans condition. Les
autres impératifs (ceux de l’habileté
ou de Ja prudence) sont hypothétiques : Vg. « si vous voulez la santé,
soyez tempérants »; le Devoir dit :
« soyez tempérants » sans condition.

\ib{Catharsis} — [mot grec] — \si{Hist.} {\bf 1.} Chez
Arisiote : purgation des passions par
le moyen de l’art$^3$, qui leur permet
de s’épancher sur des objets fictifs.
— \si{Ps. an.}  {\bf 2.} (Syn. : méthode cathartique). Thérapeutique psychanalytique qui consiste à débarrasser le
sujet de ses troubles, soit par rappel
à la conscience de l’idée dont le
refoulement* les a causés, soit par
abréaction*.

\ib{Catoptrique} — [G. catoptris, miroir] —
\si{Épist.} Autref. : partie de l'Optique
traitant de la réflexion de la lumière
(opp. Dioptrique, traitant de la
réfraction) : « Une seule expérience
sur la réflexion de la lumière donne
toute la Catoptrique, ou science des
propriétés des miroirs; une seule sur
la réfraction de la lumière produit.
toute la Dioptrique ou science des

propriétés des verres concaves et
convexes » (D’Alembert).

\ib{Causalité} — \si{Épist.} Rapport de cause*
à effet. — Principe de causalité :
« Tout a une cause et, dans les
mêmes conditions, la même cause
est suivie du même effet. »

\ib{Causation} — \si{Méta.} Action par laquelle la cause$^1$ produit son effet.

\ib{Cause} — Méta. {\bf 1.} Force$^2$ productrice,
engendrant l'effet et se prolongeant
en lui. Cf. Efficace* et Occasionnelle*. — Épist.  {\bf 2.} Antécédent$^1$
constant (Hume) et inconditionnel
(J. S. Mill). —  {\bf 3.} Phénomène lié au
phénomène considéré par une relation fonctionnelle : « La cause n’est
jamais vraiment empirique » (Bachelard) $->$ Dans la science, l'explication par les forces productrices
(sens 1) fait place de plus en plus à
l’explication par les relations fonctionnelles (sens 3). Aussi, tandis
que F. Bacon disait que « savoir
vraiment, c’est savoir par les causes »
(sens 2), A. Comte a pu écrire
(Cours, I) que la science renonce.à
la recherche des causes (sens 1), ce
qui est d'ailleurs auj. discuté.

—— \si{Hist.}  {\bf 4.} Aristote distingue
4 espèces de causes : a) la cause matérielle (og, dans une statue, la
matière dont elle est faite); — b) la
cause formelle (la figure que la statue
représente; cf. Formel); — c) la
cause efficiente, {\bf 1.} e. la cause au sens 1
(le sculpteur); — d) la cause finale$^1$
(le but : désir de la gloire ou du gain,
visé par le sculpteur).

— \si{Méta.}  {\bf 5.} Cause première : voir
Premier$^4$.

\ib{Caverne (Allégorie de la)} — \si{Hist.}
Fiction par laquelle Platon (Rép.
VII) dépeint la condition de l’homme
prisonnier du corps et qui prend les
%
choses sensibles pour les vraies réalités, par comparaison avec des
captifs relégués dans une caverne
où ils ne voient que des ombres
qu'ils croient réelles.

\ib{Cécité} — [L. cæcus, aveugle] — État
de celui qui est aveugle, \si{Ps. path.} :
Cécité psychique, agnosie* visuelle,
— Cécité verbale (syn. : alexie),
agnosie visuelle verbale (le sujet
voit les caractères écrits, mais ne les
reconnaît pas : il ne sait plus lire).
— Cécité morale, absence anormale
de tout sentiment moral.

\ib{Cénesthésie ou Cœnesthésie} [G. koïné
aisthésis, sensibilité générale]. —
Psycho. Sensibilité organique$^3$, ensemble des sensations internes.
$->$ Dist. sensations cinesthésiques
ou kinesthésiques*.

\ib{Censure} — \si{Psycho.} {\bf 1.} Pouvoir de
contrôle que l'esprit adulte et normal possède vis-vis de lui-même
et qui lui permet de distinguer le
réel de l’irréel, le praticable de l’impraticable, ete. — \si{Ps. an.}  {\bf 2.} Chez
Freud : fonction par laquelle certains désirs ou images sont repoussés
dans l'inconscient. Voir Refoulement*.

\ib{Central} — \si{Phol.} (Ctr. : périphérique*).
Qui se rapporte aux centres* nerveux, ou qui vient de ces centres*.

\ib{Centre} — \si{Phol.} Centres nerveux
chez l’homme, cerveau$^2$, cervelet,
bulbe et moelle épinière. Sont dits
centrifuges (syn. : efférents) les nerîs
(moteurs ou sécréteurs) qui conduisent l'incitation des centres nerveux
vers Ja périphérie; centripèles (syn. :
aférents) les nerfs (sensitifs) qui
conduisent l'excitation de la périphérie vers les centres. — On a distingué dans le cerveau des cenires
sensoriels (syn. : de projection) et des
centres d'associalion$^2$.
% 32

\ib{Cercle} — \si{Log.} {\bf 1.} Cercle vicieux : sorte
de pétition de principe consistant
à prouver une proposition P par
une proposition R qui ne peut se
prouver elle-même que par P.

— \si{Hist.}  {\bf 2.} Cercle cartésien : celui
que Gassendi reproche à Descartes
en ces termes : « Vous admettez
qu’une idée claire et distincte est
vraie parce que Dieu existe, qu'il
est l’auteur de cette idée et qu'il
n'est pas trompeur; et d'autre part,
vous admettez que Dieu existe, qu'il
est créateur et vérace, parce que
vous en avez une idée claire. »

\ib{Cérébelleux} — \si{Phol.} Qui concerne
le cervelet.

\ib{Cérébral} — \si{Phol.} {\bf 1.} Qui concerne le
cerveau$^2$ : « Les hémisphères cérébraux ». $->$ Ne pas confondre avec
cervical [L. cervix, nuque], qui concerne le cou. — \si{Car.}  {\bf 2.} Qui vit surtout par la pensée et l'imagination :
« Baudelaire était un cérébral. »

\ib{Cérébration} — \si{Ps. phol.} Activité
cérébrale. Cérébration inconsciente :
travail intellectuel qui se fait sans
que le sujet en soit conscient.

\ib{Certain} — \si{Psych.} {\bf 1.} © En parlant des
personnes : qui se croit en possession de la vérité : « Si l'homme qui se
trompe dit, au moment où il se
trompe : je suis certain, quand il a
reconnu son erreur il dit : je me
croyais certain » (Brochard).

— \si{Log.} M En parlant des propositions :  {\bf 2.} Qui est assurément vrai:
« Il n’y a eu que les seuls mathématiciens qui ont pu trouver quelques
démonstrations, c’est-à-dire quelques
raisons certaines et évidentes »
(Descartes, Méth., II); « Ce qui n’est
certifié que par les hommes, peut
être cru comme vraisemblable, mais
non pas comme certain » (Bossuet);
« Pour autant que les propositions
% 33 — cHi
de la mathématique se rapportent
à la réalité, elles ne sont pas certaines » (Einstein). Qqfs., en un sens
plus fort : démontré : « S’il ne fallait
rien faire que pour le certain, on ne
devrait rien faire pour la religion:
car elle n’est pas certaine » (Pascal,
234). —  {\bf 3.} Dont on est plus ou
moins assuré : « Toutes les autres
choses dont ils se pensent peut-être
plus assurés, comme d'avoir un
corps [etc.], sont moins certaines
[que Pexistence de Dieu]; car, encore
qu’on ait une assurance morale de
ces choses... » (Descartes, Méth., IV).
Cf. Moral$^5$.

\ib{Certitude} — O. \si{Psych.} {\bf 1.} © (Opp.
doute$^1$ et opinion$^1$). État de l’esprit
qui « se croit en possession de la
vérité » (Goblot), qui donne son
assentiment* sans réserve aucune :
« Certitude, certitude, sentiment,
joie, paix » (Pascal, mémorial); « La
certitude n'existe que par l’harmonie de la nature et de l'esprit »
(Lagneau); « L’enthousiasme a toujours engendré la certitude » (Espinas), $->$ Cf. Croyance et Moral$^5$.
— \si{Épist.}  {\bf 2.} IN Caractère de ce qui
est certain au sens 2 : « C’est à la
simplicité de leur objet que les mathématiques sont redevables de leur
certitude » (D'Alembert). $->$ Terme
équivoque comme le précédent : les
confusions sont fréquentes entre le
sens 1 et le sens  {\bf 2.} Cf. Conviction*.

— ©  {\bf 3.} Proposition, croyance
ou opinion certaine$^2$, ou que l’on
croit telle : « La jeunesse veut des
certitudes. »

\ib{Cerveau} — \si{Phol.} {\bf 1.} Lato. Encéphale.
—  {\bf 2.} Str. Les hémisphères cérébraux
(qui, avec le cervelet et le bulbe, forment l’encéphale).

— \si{Vulg.}  {\bf 3.} Esprit : « Un puissant cerveau ». $->$ Très impropre
au sens  {\bf 3.}

%Cuvillier. — Vocabulaire philosophique.

\ib{Champ} — \si{Phys.} {\bf 1.} « Ensemble des
propriétés physiques qui caractérisent à chaque instant les divers
points de l’espace et qui s'expriment
par des fonctions gén. continues des
coordonnées d’espace et de temps »
(L. de Broglie); « Le champ [physique] n’est plus une chose, mais un
système de rapports entre des forces
ponctuelles : le champ va peu à peu
éclipser la substance » (P. Valéry).
— \si{Phol.}  {\bf 2.} Champ visuel : étendue totale qu’un œil peut voir sans bouger.

— Anal. \si{Psycho.}  {\bf 3.} Champ de la
conscience : « nombre le plus grand
de phénomènes simples qui peuvent
être réunis, à chaque moment, à
notre personnalité dans une même
perception personnelle » (Janet).
Par ext. du sens 1 : « On peut parler
d’un champ psychologique, qui est
un comportement*  systématisé,
composé d’un cours de mouvements
et d'images, orientés par une tendance » (Bréhier). —  {\bf 4.} \si{Soc.} Champ
social (notion, inspirée à la fois de
celle de la forme$^4$ et de celle de
l’espace topologique*, que K. Lewin
a introduite en psycho-sociologie) :
réseau des relations et positions respectives des différents éléments
(individus, sous-groupes, tabous sociaux, règles, etc.) qui constituent
un groupe et représentent sa structure et son orientation dynamique.

\ib{Charité} — [L. caritas, amour] — \si{Mor.} {\bf 1.}
Amour du prochain*. —  {\bf 2.} Bienfaisance, et spéc. aumône : « Faire la
charité »; « Il faut toujours rendre
justice avant d'exercer la charité »
(Malebranche). — $->$ Sur l'équivoque du sens 2, cî. Précis, Ph, II,
p. 334; Se. et M., p. 289; Textes
choisis, p. 192-19 {\bf 6.}

\ib{Chiffre} — \si{Math.} {\bf 1.} Signe figuré d’un
nombre. — \si{Méta.}  {\bf 2.} Signe secret à
déchiffrer : « Les langues sont des
% 34
chiffres » (Pascal, 45). Spéc., chez
Jaspers : signes de la Transcendance
cachés dans l’histoire, les mythes,
les doctrines philosophiques, etc. :
« Il n'est rien qui ne puisse être
chiffre » (Jaspers).

\ib{Chimie} — \si{Épist.} Science des propriétés spéciales des corps et des
modifications de structure interne
qu'ils peuvent subir. — Théorie
chimique (ou physico-chimique) de
la vie : celle qui ramène les phénomènes vitaux à des phénomènes
physiques et chimiques.

\ib{Choix} — \si{Psycho.} {\bf 1.} Sir. Décision volontaire par laquelle un parti est élu
entre plusieurs possibles. —  {\bf 2.} Lato.
Sélection purement spontanée :
« La conscience est une activité de
choix ».

\ib{Chose} — \si{Méta.} {\bf 1.} Réalité, en gén. :
« L'âme est une chose qui pense »
(Descartes). —  {\bf 2.} Réalité objective :
« Il faut traiter les faits sociaux
comme des choses » (Durkheim). —
 {\bf 3.} Réalité statique : « Il n’y a pas
de choses, il n’y a que des actions »
(Bergson, E. C., III); « La chose est
l’idole d’une pensée que hante un
souci de fabrication manuelle »
(Le Roy).

\ib{Chosisme} — \si{Crit.} (Péj.). Tendance à
traiter le réel comme une chose au
sens 3 ou « comme un donné brut
extérieur ou plutôt hétérogène à la
pensée » (Le Roy).

\ib{Cinématique} — \si{Épist.} Partie de la
Mécanique où l’on étudie le mouvement, abstraction faite des forces$^4$
qui le causent.

\ib{Cinétique} — [G. kinésis, mouvement] —
\si{Phys.} Qui se rapporte au mouvement : « Énergie* cinétique ». —
Théories cinétiques (syn. : mécanisme$^3$) : celles qui ramènent les
phénomènes physiques à du mouvement (vg. théorie cinélique des gaz :
celle qui explique les propriétés des
gaz par le mouvement des molécules
dont ils sont composés).

\ib{Circulaire (Folie)} — \si{Ps. path.} État
psychique morbide caractérisé par
une alternance régulière de l’excitation (manie*) et de la dépression
(mélancolie$^2$) : c’est une forme de la
folie intermittente ou psychose maniaque-dépressive.

\ib{Civil} — \si{Pol.} {\bf 1.} Chez Rousseau : « état
civil » (opp. : état de nature), condition morale et juridique de l’homme
dans une société organisée.

— \si{Jur.} et \si{Pol.}  {\bf 2.} Qui se rapporte
à l'individu comme personne. Droits
civils : droits de propriété, de se
marier, d’hériter, de tester, d’être
tuteur, témoin, etc. : « L'exercice
des droits civils est indépendant de
la qualité de citoyen » (C. C., 7).

— \si{Hist.}  {\bf 3.} Chez Hegel : « société
civile » (all. : bürgerliche Gesellschaft),
la société, considérée surtout dans
sa structure économique et politique
et en tant qu’elle implique l’indépendance des personnes : « La personne concrète est le principe de la
société civile ».

\ib{Civilisation} — \si{Soc.} {\bf 1.} Lato. Ensemble
des institutions*, techniques*, coutumes$^1$, croyances$^2$, etc., qui caractérisent l’état d’une société : « La
civilisation des Guarani »; « La
civilisation minoenne ». —  {\bf 2.} Str.
Il y a plus spéc. « phénomène de
civilisation » lorsqu'une civilisation$^1$
s’étend à une aire très large, comprenant plusieurs peuples distincts, et
se hausse à la notion d’un « ensemble
de valeurs$^2$ susceptibles d’être appliquées à la totalité de l'espèce
humaine » (R. Hubert) : « La civilisation méditerranéenne, »
% 35

— \si{Mor.}  {\bf 3.} (Opp. : barbarie.) État
supérieur de la civilisation$^2$, apprécié d’après certains critères plus ou
moins variables (avancement de la
technique, état social et moral, développement intellectuel) : « La Civilisation, »

\ib{Civique} — \si{Jur.} et \si{Pol.} Qui se rapporte à l'individu comme citoyen.
Droits civiques : droits de vote,
d'éligibilité, de prétendre « à toutes
dignités, places et emplois publics ».

\ib{Clair} — \si{Log.} (Ctr. : obscur). Une idée
est claire quand on ne confond son
objet avec aucun autre, quand on
connaît bien l’extension* de cette
idée. Cf, Distinct* et Présent*.

\ib{Clan} — \si{Soc.} {\bf 1.} Lato. Toute société
où la parenté résulte de la communauté du nom (vg. la gens romaine).
—  {\bf 2.} Str. Le clan totémique : forme
sociale où tous les individus, se
considérant comme issus d’un même
totem*, portent tous le nom de ce
totem.

\ib{Classe} — \si{Log.} form. {\bf 1.} Ensemble
d'êtres, objets ou faits en nombre
indéterminé, possédant tous et possédant seuls un ou plusieurs caractères$^2$ communs : les genres$^1$ et les
espèces$^2$ sont des classes. $->$ Dist.
classe au sens scolaire : en Log., une
classe d’élèves est une collection*.

— \si{Biol.}  {\bf 2.} Groupe morphologique supérieur à l’ordre$^8$ et inférieur à l’embranchement : « La classe
des mammifères. »

— \si{Soc.}  {\bf 3.} Groupe d'individus
caractérisé, dans une société donnée,
par son niveau de vie, ses droits ou
ses privilèges, et surtout son rôle
dans la production économique :
«a La classe ouvrière ». $->$ Dist,
caste, groupe héréditaire, impliquant
une stratification à base religieuse.

\ib{Classification} — \si{Épist.} Opération qui
consiste à répartir les concepts ou
les objets étudiés en classes$^1$ hiérarchiquement ordonnées, en genres$^1$
et en espèces$^2$. — Classification artificielle : fondée sur un caractère
choisi arbitrairement et gén. apparent, mais secondaire. Classification
naturelle : fondée sur les caractères
essentiels* (dans les sciences biol.
psycho., soc., elle vise à reproduire
l’enchaînement naturel, la généalogie des formes à classer).

\ib{Clinamen} — [mot latin] — Voir Déclinaison*.

\ib{Clos} — Voir Ouvert*.

\ib{Cœur} — {\bf 1.} Autref., not. chez Pascal :
intuition, sentiment$^2$ immédiat :
« Nous connaissons la vérité non
seulement par la raison, mais encore
par le cœur : c’est de cette sorte que
nous connaissons les premiers principes.. Le cœur sent qu’il y a trois
dimensions dans l’espace et que les
nombres sont infinis » (Pensées, 282).
—  {\bf 2.} Dispositions intérieures, conscience : « Ouvrir son cœur à qqn »;
« Il ne faut pas juger des hommes
sur une première vue : il y a un
intérieur$^4$ et un cœur qu'il faut
approfondir » (La Bruyère, XII);
« Son juge [de l'homme] est dans
son cœur » (Voltaire). —  {\bf 3.} Sentiments$^4$, côté aflectif de l'âme
« L'esprit est toujours la dupe du
cœur » (La Rochefoucauld); « Ce
n'est pas la raison de l’homme qui
le séduit, c’est son cœur » (Malebranche, Éd. XIII) ; « J’appelle héros
ceux qui furent grands par le cœur »
(Romain Rolland).

\ib{Cogent} — Contraignant pour l'esprit,
rationnellement nécessaire$^2$ : « Pour
lui [Descartes], toutes les vérités se
trouvent seulement sur le plan de la
vérité cogente » (Jaspers).
%

\ib{Cogito} — \si{Hist.} {\bf 1.} Expression abrégée
pour désigner le « Je pense, donc je
suis » (cogito, ergo sum) de Descartes
(Méth., IV, et Méd., II). —  {\bf 2.} Chez
Kant, le « je pense » (cf. Transcendantale* [Aperception]) devient la
condition qui rend possible l’expérience. —  {\bf 3.} Chez Husserl : « Cogito
pré-réflexif », la pensée comme condition transcendantale de l’expérience, alors qu’elle n'a pas encore
pris conscience de son intentionalité*
et n’a pas encore pour elle-même
qualité d'objet.

\ib{Cognitif} — \si{Psycho.} Qui concerne la
connaissance$^1$

\ib{Coïncidence} — Concomitance accidentelle de deux ou plusieurs phénomènes.

\ib{Collectif} — \si{Log. form.} {\bf 1.} (Opp. : singulier et général$^2$). Qui désigne une
collection* (voir ce mot).

— \si{Psycho.} et \si{Soc.}  {\bf 2.} Psychologie
collective (ou sociale) : psychologie
des groupes humains. Conscience
collective : manières de sentir, de
penser et d’agir propres à un groupe
social déterminé et qui diffèrent des
manières de sentir, de penser et
d'agir individuelles (cf. Précis, Ph. I,
p. 533). Inconscient collectif : v. ce
mot.

\ib{Collection} — \si{Log. form.} Pluralité
d'êtres, objets ou faits en nombre
déterminé, vg. : « Les membres de
telle association$^4$ » $->$ Dist. classe$^1$,

\ib{Collectivisme} — Soc. À {\bf 1.} Organisation économique où la propriété
individuelle des instruments de production* (mais non des objets de
consommation*) est remplacée par
la propriété de l’État ou de la nation.

— \si{Éc. soc.} 2 A. Doctrine qui préconise cette organisation : « Le collectivisme diffère du communisme$^2$
% 36
en ce qu'il ne réclame pas l'abolition
générale de la propriété individuelle
et prétend même Ja rendre plus
solide en lui donnant pour fondement le travail personnel. Mais pour
cela il faut [selon lui] la restreindre
aux produits et l’abolir en ce qui
concerne les instruments de production » (Ch. Gide).

\ib{Colligation} — \si{Log.} « Opération par
laquelle nous réunissons les faits
sous une idée » (Whewell) : vg. les
positions des planètes sous l’idée
d’orbite planétaire. Pour Whewell,
elle se confond avec l’induction*;
pour Mill, elle lui est préliminaire.

\ib{Commun} — \si{Log.} {\bf 1.} Notions communes : celles qui, faisant partie intégrante de la raison, sont communes
à tous les hommes : « Un philosophe serait bien savant s’il voyait
tout ce qui est dans les notions
communes » (Condillac). —  {\bf 2.} Sens
commun : voir Sens*.

— \si{Math.}  {\bf 3.} Une quantité est
commune mesure entre deux autres
quand elle est comprise un nombre
entier de fois dans chacune d'elles :
celles-ci sont dites alors commensurables.

\ib{Communauté} — \si{Crit.} {\bf 1.} Chez Kant :
syn. : réciprocité*.

— \si{Soc.} {\bf 2.} O Caractère de ce qui
est commun dans un groupe d’individus : « une communauté d’opinions ». Spéc., se dit du régime économique où la propriété individuelle n'existe pas (syn. : communisme$^1$) : « L'état de communauté
[des biens] est le seul juste » (G. Babeuf). — {\bf 3.} @ Groupe, gén. à base
religieuse, dont les membres sont
unis par la même foi et font abandon
de leurs biens personnels : « Les
Perses eurent des communautés de
cénobites » (Voltaire). — {\bf 4.} Type
% 37
d'organisation sociale défini par
F. Tônnies comme reposant sur les
liens du sang et de l'instinct et opp.
par lui à la Société$^3$ reposant sur la
volonté réfléchie et l'intérêt (voir
Précis, Ph. II, p.336; Sc. et M., p.191).

\ib{Communication (des consciences)} —
\si{Psycho.} Phénomène par lequel les
consciences individuelles, au lieu de
rester closes et isolées, sympathisent
et communiquent entre elles (voir
Précis, Ph. I, chap. VIII; Sc. et M.,
chap. XI) : « Je ne crois pas à la
transmigration* des âmes, mais je
ne puis m'empêcher de croire à leur
communication » (Buffon); « La
communication avec un autre être
ne peut se produire que grâce à ce
mouvement par lequel chacun d’eux
ne pense plus à soi, mais au prochain
afin de l’appeler à une vie supérieure » (Lavelle).

\ib{Communion} — \si{Psycho.} {\bf 1.} Fusion parfaite des âmes dans les mêmes sentiments : « Toute communion n’est
qu’un leurre si elle ne nous donne en
même temps une conscience plus
vive de notre existence séparée »
(Lavelle).

— \si{Théol.} {\bf 2.} Communion des
saints : union spirituelle de tous les
fidèles qui les fait participer aux
mêmes mérites. — {\bf 3.} Église, confession religieuse : « Les communions
protestantes. »

\ib{Communisme} — Soc. À {\bf 1.} Organisation économique reposant sur la
communauté$^2$ des biens et sur le
principe : « À chacun selon ses besoins ». Lalo. Type d'organisation
sociale « qui absorbe l’individu dans
le groupe » (Durkheim).

— \si{Éc. soc.} À. {\bf 2.} Autref., doctrine
qui préconise cette organisation (cf.
Collectivisme$^2$) : « Le communisme
ne pense ni ne raisonne,... il croit »
% 37
(Proudhon); « Ce communisme [de
Platon] est très complet et très ascétique » (Ruyer). — {\bf 3.} Auj., collectivisme$^2$ qui se réclame de l’idéologie marxiste : « Par suite d’une
confusion regrettable, on a pris
l'habitude de parler du communisme russe, alors que le régime
soviétique exclut tout à fait l’attribution des biens selon la maxime :
A chacun selon ses besoins » (B. Lavergne).

\ib{Commutative (Justice)} — [L. commuiare,
échanger] — \si{Mor.} (Opp. : distributive*). Celle qui préside aux échanges,
contrats, etc., et est fondée sur
l'équivalence des choses, abstraction faite des inégalités entre les
personnes (voir Précis, Ph. II,
p. 331; Sc. et M., p. 186).

\ib{Comparaison} — \si{Psycho.} et \si{Épist.}
Opération réfléchie par laquelle on
établit les ressemblances et les différences entre deux termes. — La méthode comparative est celle qui use
systématiquement de la comparaison : vg. en Soc. l'observation comparative supplée à la difficulté
d’expérimenter (voir Précis, Ph. II,
p. 197; Sc. et M., p. 307) : « Dans les
sciences sociales, le seul moyen pour
appliquer le mode de raisonnement
expérimental est de comparer les
diverses formes de la vie sociale »
(A.-R. Brown). — Anatomie comparée : celle qui compare les organismes et les modifications que
subit chaque organe dans la série
des êtres vivants. — Psychologie
comparée : celle qui compare la vie
psychique chez l’homme et les animaux ou chez les divers groupes
humains, dans les peuples, les
classes, les sexes, etc.

\ib{Compensation} — \si{Ps. an.} Chez Adler :
« mécanismes de compensation »,
% 38
ceux par lesquels le sujet « essaie de
compenser ses infériorités* par des
efforts pour se mettre au premier
plan dans un autre sens » (Baruk) :
vg. Démosthène bègue devenant
orateur.

\ib{Complémentaire} — \si{Phys.} {\bf 1.} Couleurs
complémentaires celles dont le
mélange donne du blanc (vg. rouge
et bleu-vert).

— \si{Psycho.} {\bf 2.} Voir Consécutive*.

\ib{Complexe (adj)} — \si{Log.} {\bf 1.} (Ctr.
simple*). Composé d’un grand nombre d'éléments. Terme complexe :
celui dont la compréhension$^2$ est
très riche. — Math. {\bf 2.} Nombre complexe : voir Imaginaire$^3$.

\ib{Complexe (nom)} — \si{Vulg.} {\bf 3.} (Syn. :
complexus). Système complexe$^1$ de
phénomènes se mêlant intimement.

— \si{Ps. an.}  {\bf 4.} « Ensemble de contenus représentatifs ou de situations qui, à la suite d'expériences
spéciales des années d'enfance, possèdent pour le sujet une forte charge
émotive et qui produisent leurs
effets consciemment ou inconsciemment au cours du développement
psychique » (J. Nuttin). Ces effets
consistent en fortes réactions émotives, qqfs. en troubles psychiques.
— Cf, infériorité*, Œdipe*.

\ib{Comportement} — [Trad. anglais behaviour] — \si{Psycho.} Ensemble des
actes par lesquels l’homme et les
animaux réagissent aux impressions
reçues du monde extérieur; manière
dont ils se conduisent. Cf. Behaviourisme*. $->$ Ce mot avait déjà
été employé par Poscal : « Pour
reconnaître si c'est Dieu qui nous
fait agir, il vaut bien mieux s’examiner par nos comportements au
dehors que par nos motifs au-dedans »
(L. à Périer, 1661).

\ib{Composé} — \si{Log. form.} Proposition
composée : celle qui a plusieurs
sujets$^2$ ou plusieurs attributs$^1$ (vg.
une disjonctive*). — Syllogisme
composé : raisonnement formé de
plusieurs syllogismes enchaînés (vg.
polysylogisme*).

\ib{Composite (Portrait)} — \si{Techn.} Celui
qu’on obtient par superposition
de photographies analogues sur
un même cliché (voir Précis, Ph. I,
p. 306).

\ib{Compossible} — \si{Méta.} Chez Leibniz :
qui est non seulement possible,
mais compatible avec l'ensemble de
l'univers. Voir Possible*.

\ib{Compréhensif} — \si{Épist.} On a nommé
« psychologie » ou « sociologie compréhensive » (all. : versiehende) celle
qui cherche à comprendre au sens 4,
i. e. à saisir par sympathie plutôt
qu’à expliquer (voir Précis, Ph. IT,
p. 194 et 239; Sc. et M. p. 348;
Textes choisis, II, p. 118-121).

\ib{Compréhension} — \si{Vulg.} {\bf 1.} (Syn. :
intelligence$^4$). Action de comprendre*
(en tous les sens de ce terme) : « La
compréhension met toujours en
œuvre tout l'esprit » (Delacroix).

— \si{Log. form.}  {\bf 2.} (Opp. : extension). Ensemble des caractères, propriétés ou qualités qui constituent
un concept, dont il est le sujet$^2$
(vg. pour « oiseau » : animal, vertébré, à sang chaud, ovipare, etc.).
— Voir Précis, Ph. I, p. 316, note.

\ib{Comprendre} — \si{Psycho.} A) Au sens
intellectuel : {\bf 1.} Comprendre un
signe, un symbole, une langue, c'est
être capable de lui substituer les
concepts, images, etc., qui constituent sa signification : « Comprendre
une phrase, c’est comprendre et le
sens des mots et la construction de
la phrase » (Delacroix). —  {\bf 2.} Comprendre
%39
un fait, une idée, un raisonnement, c’est l'intégrer dans un système logique (comprendre le fait,
c'est connaître sa cause ou sa loi;
l’idée, c’est saisir ses relations avec
d’autres idées; le raisonnement,
c'est concevoir l’ensemble des rapports qui le constituent) : « On comprend dès qu’on systématise » (Delacroix). —  {\bf 3.} Saisir pleinement par
la pensée l'essence de ce que l'on
connaît : « Si je vous comprenais,
mon Dieu, vous ne seriez plus ce que
vous êtes » (Bourdaloue). Cf, incompréhensible$^2$.

B) Au sens affectif ou sympathique :  {\bf 4.} Entrer en communication*, voire en communion* avec
(autrui, une œuvre spirituelle, etc.) :
« Ils [Corinne et Oswald] se comprenaient mutuellement d’une façon
merveilleuse » (Staël) ; « Comprendre
amoureusement un thème musical,
c'est s'ouvrir à la suggestion d'évasion qu'il contient » (Pradines).
$->$ (Certains auteurs, not. allemands, opposent, en ce sens, comprendre à expliquer : « La nature,
nous l'expliquons [erklären]; la vie
de l'âme, nous la comprenons [verstehen] » (Dilthey). Cf. compréhensif*.

\ib{Conation, Conatus} — [L. conari, s’efforcer] — \si{Méta.} {\bf 1.} Chez Leibniz : éléments ponctuels d'énergie, qui composent : a) la pensée : « La pensée
consiste dans le conatus [effort]
comme le corps dans le mouvement » (L. à Arnauld); puis b) le
mouvement lui-même : « Non seulement les actions intérieures volontaires de notre esprit suivent de ce
conatus, mais encore les extérieures,
i. e. les mouvements volontaires de
notre corps ». — \si{Psycho.} {\bf 2.} Qqfs.
employé auj. pour désigner les faits
d'activité$^2$ : « Dans la vie mentale de
l'homme, les conations se transforment
% 39
couramment en activités volontaires » (Warren).

\ib{Concept} — \si{Psycho, Ébpist., Crit.} {\bf 1.}
Lato. Idée abstraite et générale :
« Les concepts élaborés par la masse
et ceux qu’élaborent les savants ne
sont pas de nature essentiellement
différente » (Durkheim). — {\bf 2.} Str.
Se dit spéc. des idées les plus
abstraites et les plus générales :
« Les catégories* sont les vrais
concepts primitifs de l’entendement
pur» (Kant, À. pure, Analyt., I, 1, 3).
— Voir Schème$^4$.

\ib{Conception} — \si{Psycho.} {\bf 1.} O. Opération ou faculté consistant à concevoir* (au sens 1 ou 2) : « Je remarque
premièrement la diflérence qui est
entre l'imagination et la pure intellection ou conception » (Descartes,
Méd., VI). — {\bf 2.} @. Produit de cette
opération; d’où ext., façon de penser,
théorie : « La conception empiriste
de la causalité »; « Les conceptions
modernes de la matière ».

\ib{Conceptualisation} — \si{Épist.} Réduction
du donné empirique à des concepts$^1$,
du sensible à l'intelligible : « La
science est une conceptualisation de
la nature. »

\ib{Conceptualisme} — \si{Crit.} (Opp. : nominalisme$^1$ et réalisme$^3$). À. Doctrine
selon laquelle les concepts$^1$ ou universaux* existent, à titre d'idées$^3$,
dans notre esprit, mais non en tant
que réalités à part des objets individuels$^2$ : « Sans aboutir à un conceptualisme nettement caractérisé, Abélard en approche aussi près que possible » (Gilson).

\ib{Concevoir} — \si{Psycho.} {\bf 1.} Latiss. Se
disait autref. gén. de toute appréhension$^1$ d’un objet par l'esprit : « On
appelle concevoir la simple vue que
nous avons des choses qui se présentent
% 40
à notre esprit » (Port-Royal).
On dit encore auj. : « concevoir un
projet » (le former dans son esprit),
« concevoir de la haine pour qqn »
(commencer à l’éprouver), etc. —
{\bf 2.} Lato. Comprendre$^2$ : « Concevoir
un raisonnement », Ou encore : se
représenter comme possible : « Cela
ne se conçoit pas ». —  {\bf 3.} Sir. (Opp. :
imaginer]. Former le concept$^1$ d’une
chose. $->$ Le sens propre est le
sens {\bf 3.}

\ib{Conclusion} — \si{Log.} {\bf 1.} Proposition
qui clôt un raisonnement* ou une
inférence* et que ceux-ci ont pour
but d'établir. — Spéc. {\bf 2.} (Math)
Dans un théorème, proposition à
démontrer (opp. : hypothèse$^2$). —
3 (Log. form.). Dans un syllogisme, troisième et dernière proposition, implicitement contenue dans
les deux premières (opp. : prémisses*). $->$ Dist. conséquence*.

\ib{Concomitant} — Qui se produit en
même temps.

\ib{Concordance (Méthode de)} — \si{Log.}
Une des quatre méthodes expérimentales de J. Stuart Mil (voir
Précis, Ph. II, p. 122; Sc. et M.
p. 238).

\ib{Concret} — \si{Psycho.} et \si{Log.} (Ctr. :
abstrait). Un objet est concret quand
il est considéré tel qu'il est donné
dans l’expérience, soit externe, soit
interne : un fait, qu'il soit physique,
psychique ou social, est concret; au
ctr., une relation mathématique est
abstraite. Le concret est singulier,
individuel (cf. cependant Universel*),
par opp. à l’abstrait qui, seul, est
général. Une représentation est concrète (vg. sensation, image, perception) quand elle reproduit son
objet tel qu'il est donné dans l’expérience; elle est abstraite (vg. idée)
%40
quand on ne considère qu’un élément pris à part. $->$ Dist. sensible : tout ce qui est sensible est
concret, mais non réciproquement.

\ib{Concupiscence} — \si{Théol.} Appétit$^1$ déréglé de la volenté corrompue par
le péché : « Tout ce qui est au
monde est concupiscence de la chair,
ou concupiscence des yeux, ou
orgueil de la vie : libido sentiendi,
libido sciendi, libido dominandi »
(Pascal, 458, d’après saint Jean,
Épitres, I, 2, 16).

\ib{Concupiscible} — Voir Appétit$^1$.

\ib{Concurrence} — \si{Éc. pol.} {\bf 1.} Dans le
phénomène de la concurrence économique, dist. : a) la liberté du travail
et des échanges [ctr. : monopole];
b) la lutte économique ((ctr. : coopération$^1$).

— \si{Biol.} 2, Concurrence vitale
sorte de « lutte pour la vie » gén.
inconsciente qui, selon Darwin,
s'établit entre les êtres vivants et
élimine les moins forts ou les moins
bien adaptés (voir Précis, Ph. II,
p- 160; Sc. et M., p. 275).

\ib{Condensation} — \si{Ps. an.} Procédé de
la pensée onirique* par lequel
« 1° certains éléments du rêve latent*
sont éliminés; 2° le rêve maniteste
ne reçoit que des fragments de certains ensembles du rêve latent;
3° des éléments latents ayant des
traits communs se trouvent fondus
ensemble » (Freud).

\ib{Condition} — {\bf 1.} (Au pluriel). Circonstances dans lesquelles un phénomène se produit ou dans lesquelles une persorne se trouve
« Conditions de température et de
pression », « Conditions de vie », —
 {\bf 2.} (Au pluriel ou au singulier). Circonstance nécessaire$^2$, celle sans
laquelle le phénomène ne se produirait
% 41
pas : « Une des conditions de
l'ébullition est que la pression soit
inférieure au point critique ».

\ib{Conditionnel} — \si{Log.} (Syn. : hypothétique*). Qui dépend d'une condition$^2$.

\ib{Conduite} — Ce terme qui enveloppait
autref. un sens moral$^1$, est souvent
employé auj. comme syn. de Comportement* : « La Psychologie est la
science de la conduite » (Lagache).

\ib{Confus}. — Ctr. de distinct*.

\ib{Confusion mentale}. — \si{Ps. path.} État
pathologique caractérisé par de
l’obtusion intellectuelle, de l’amnésie* et une incapacité générale de
coordonner les sensations et les
idées (voir Précis, Ph. I, p. 280).

\ib{Congénital} — \si{Biol.} (Ctr. : acquis$^1$).
Que l’être apporte en naissant.

\ib{Congruence} — \si{Épist.} Égalité$^2$ géométrique : deux figures sont congruentes quand elles sont superposables.

\ib{Conjecture} — Voir Hypothèse$^1$.

\ib{Conjonction} — \si{Épist.} Liaison fortuite. Voir connexion*.

\ib{Connaissance} — \si{Psycho.} {\bf 1.} O (Opp. :
affectivité$^2$ et activité$^2$), Fonction de
la vie psychique qui se manifeste
par des phénomènes ayant un caractère représentatif* et objectif*. Voir
Connaître$^1$, — \si{Épist.}  {\bf 2.} @ (Souvent
au pluriel). Résultat de cette fonction : « Acquérir des connaissances »,
« Dans l’état actuel de nos connaissances ».

— Crit.  {\bf 3.} Théorie de la connaissance : voir Crilique$^1$.

\ib{Connaître} — \si{Psycho.} IH {\bf 1.} Prendre
connaissance$^1$ de... La connaissance
proprement dite suppose la distinction du sujet et de l’objet : « Connaître, c’est sortir de soi » (Lagneau).
—  {\bf 2.} Reconnaître, discerner :
« Connaître le bien et le mal ». —
 {\bf 3.} Spéc., reconnaître comme ami,
comme parent : « Albe vous a nommé.
je ne vous connais plus » (Corneille).
— ©.  {\bf 4.} Se connaître : avoir conscience de soi-même, de ses forces,
de sa valeur : « Il faut se connaître
soi-même : quand cela ne servirait
pas à trouver le vrai, cela au moins
sert à régler sa vie » (Pascal, 66); « Et
nul ne se connaît tant qu'il n’a pas
souffert » (Musset).

\ib{Connaturel} — \si{Méta.} Dans le lang.
scolastique : qui est inhérent à la
nature d’un être ou en accord avec
elle, D'où : {\bf 1.} inné* : « Science connaturelle » (saint Thomas) ; —  {\bf 2.} connaissance par connaturalité, celle
qui repose sur l’identité de nature
du connaissant et du connu, vg. la
connaissance d'autrui.

\ib{Connexion} — Liaison (avec une idée
de nécessité) : « La cause et l'effet
sont liés, non par une connexion
nécessaire [angl. : connected], mais
par une conjonction constante [ang].
conjoined] » (Hume).

\ib{Connotation} — Log. form. Propriété
que possède un terme de désigner
certains attributs$^1$ constituant la
compréhension$^2$ du concept correspondant.

\ib{Conscience} — \si{Psycho.} {\bf 1.} O Intuition$^1$
plus ou moins claire et explicite
que prend l'individu$^2$ humain des
faits qui se passent dans son propre
esprit. Conscience simple ou spontanée : sentiment intérieur, intuition
immédiate qu’a le sujet$^4$, de ses
états psychiques et dans laquelle il
ne s'oppose pas à ce qui n’est pas lui,
Conscience réfléchie : celle qui implique
% 42
la connaissance$^1$ des états psychiques par un moi qui s’en distingue. Conscience de soi : sentiment
de la personnalité. —  {\bf 2.} @ Ensemble
des faits de conscience au sens 1, ou
sujet de ces faits : « Toute conscience
est mémoire. Une conscience qui ne
conserverait rien de son passé, »
(Bergson), ou plus précisément encore : telle structure particulière de
ce sujet : « Nous userons du terme
conscience non pour désigner la monade* et l’ensemble de ses structures
psychiques, mais pour nommer chacune de ces structures dans sa particularité concrète » (Sartre).

— \si{Mor.}  {\bf 3.} Conscience morale :
fonction pratique$^2$ de la conscience$^1$
qui nous permet de distinguer le
bien et le mal et qui juge$^2$ nos actes :
« Il ne faut pas forcer les hommes à
agir contre leur conscience » (Malebranche); « L’approbation et la
réprobation, voilà l'essence bipolaire de la consrience morale » (Le
Senne). —  {\bf 4.} Liberté de conscience :
absence de contrainte en ce qui
concerne les croyances et pratiques
religieuses : « La liberté de conscience
est la seule qui soit absolue : nul ne
peut être incriminé pour ses croyances » (Renouvier).

\ib{Conscient} — {\bf 1.} © Qui possède : a) soit
simplement la conscience$^1$ spontanée; b) soit la conscience$^1$ réfléchie.
—  {\bf 2.} M Qui est objet de conscience$^1$
(même distinction). —  {\bf 3.} IN A priori$^2$;
que l'esprit connaît sans avoir besoin
de recourir à l’expérience : « Il y a
deux ordres de vérités ou de notions,
les unes conscientes, intérieures el
subjectives; les autres inconscientes,
extérieures ou objectives » (Cl. Bernard). $->$ Très impropre au
sens  {\bf 3.}

\ib{Consécution} — Simple succession de
faits donnée dans l'expérience$^2$ :
% 42
« La mémoire fournit une espèce de
consécution aux âmes, qui imite la
raison, mais qui doit en être distinguée » (Leibniz, Mon. 26).

\ib{Consécutive (Sensation)} — \si{Psycho.}
(Syn. : complémentaire) Sensation
(qqfs appelée image) qui se produit
après la cessation de la sensation
visuelle primaire et qui est comme
le négatif de celle-ci. — Voir Précis,
Ph. I, p. 20 {\bf 3.}

\ib{Consensus} — {\bf 1.} \si{Hist.} Chez Maine de
Biran : accord, communication*
des consciences : « L'homme est en
rapport avec son semblable par une
sympathie$^2$ naturelle très bien
nommée consensus » (Biran).

—  {\bf 2.} \si{Biol.} Consensus vital : accord, coopération des fonctions vitales dans l'organisme.

— {\bf 3.} \si{Soc.} Chez Comte : « consensus
social », solidarité des phénomènes
sociaux, tous « profondément connexes » (Cours, 48$^\text{e}$ leçon).

\ib{Consentement} — \si{Psycho.} {\bf 1.} Auiref.
Assentiment* donné à un jugement : « On ne doit jamais donner de
consentement entier qu'aux propositions qui paraissent évidemment
vraies » (Malebranche, R. V., I, 2). —
 {\bf 2.} Auj., accord donné à un projet
émanant d'autrui : « Je vous réponds
déjà de son consentement » (Racine).

— \si{Crit.}  {\bf 3.} Consentement universel : accord de tous les hommes
sur un jugement ou une croyance.

\ib{Conséquence} — \si{Log.} Conclusion logiquement nécessaire$^{1b}$ ; proposition
qu’il est impossible de nier sans
contradiction, une fois les principes$^1$
admis.

\ib{Conséquent} — (Opp. : antécédent).
Épist, {\bf 1.} Fait qui suit un autre fait.
Cf. Consécution*. —  {\bf 2.} \si{Log.} Dans
une proposition hypothétique, partie
%43
de la proposition qui résulte de
l’antécédent$^2$.

\ib{Conservation} — \si{Psycho.} {\bf 1.} Conservation des souvenirs : fonction par
laquelle les souvenirs sont censés se
« conserver » dans la mémoire.

Phys.  {\bf 2.} Principe de la conservation de l’énergie : l'énergie$^1$ totale
d’un système de corps demeure toujours la même si aucune force n’agit
sur ce système.

\ib{Conservatrice (Activité)} — \si{Psycho.}
Nom donné par Pierre Janet à l’activité psychique qui se borne à conserver le passé et à le restituer intégralement, sans adaptation au présent. Voir Précis, Ph. I, p. 70.

\ib{Consommation} — Éc. pol. Fonction
de la vie économique consistant
dans l'utilisation directe des richesses produites. Objets de consommation : tous ceux qui sont directement utilisables (opp. : instruments
de production), vg. aliments, vêtements, habitation, etc.

\ib{Constatif} — \si{Épist.} (Opp. : appréciatif
ou normalif). Qui exprime une
simple constatation ou un rapport
de fait : « La raison n’est pas constative, mais normative » (Lalande).

\ib{Contiguité} — \si{Psycho.} Deux états de
conscience sont « en contiguité »
dans l'esprit quand ils s’y produisent simultanément ou en succession
immédiate. Association par contiguïté : un des modes de l’association$^3$ des idées.

\ib{Contingence} — \si{Épist.} et \si{Méta.} {\bf 1.} Caractère de ce qui est contingent$^1$ :
« La contingence des lois de la
nature » ((Boutroux). Dans les doctrines existentialistes, caractère de
ce qui est sans raison, gratuité :
« L’être est sans raison, sans cause et
% 43
sans nécessité; la définition même
de l’être nous livre sa contingence
originelle » (Sartre). —  {\bf 2.} (Syn. :
indétermination$^2$). Caractère de ce
qui est contingent$^2$; absence de déterminisme : « La probabilité, en
Physique quantique, ne résulterait
plus [selon le probabilisme$^2$] d’une
ignorance : elle serait de la contingence pure » (L. de Broglie). —  {\bf 3.}
Preuve par la contingence du monde
(syn. : argument cosmologique) : une
des preuves classiques de l’existence de Dieu : « L'observation et
le raisonnement nous montrent de
la contingence$^1$ dans le monde... Or
le contingent$^1$), par définition, ne se
suffit point, mais réclame l’existence antérieure du nécessaire » (Le
Roy).

\ib{Contingent} — \si{Épist.} et \si{Méta.} {\bf 1.} (Ctr. :
nécessaire$^1$), Ce qui n'est pas de nécessité logique : « Il y a deux sortes
de vérités : les unes sont nécessaires
et les autres contingentes » (Malebranche); « Les lois de la statique
et de la dynamique, telles que l’expérience les donne, sont de vérité
contingente » (D’Alembert); « Ce
n’est pas la recherche scientifique,
c’est uniquement la prétention d’arriver, à se passer de l'expérience qui
est condamnée par la doctrine des
variations contingentes »  (Boutroux). —  {\bf 2.} Indéterminable
«Cette réduction [qu’opère la science]
du particulier au général est aussi
un passage du contingent au nécessaire » (Liard). — Futurs contingents : événements à venir imprévisibles : « Les philosophes conviennent aujourd'hui que la vérité des
futurs contingents est déterminée »
(Leibniz, Théod., 36).

\ib{Continu} — \si{Épist.} {\bf 1.} Qui ne comporte
pas d'intervalles ou d'éléments
actuellement distincts : « La classique
% 44
opposition de l'élément simple
et indivisible [cf. Atome$^1$] avec le
continu étendu et divisible » (L. de
Broglie); « Dans les théories nouvelles, l'étendue continue et divisible, c'est essentiellement le champ$^1$»
(id.). Cf. Quantum*.

— \si{Math.}  {\bf 2.} Grandeur ou quantité
continue : celle qui varie par différences infiniment petites. Fonction
continue : celle qui est susceptible de
varier aussi peu qu’on voudra pour
des variations suffisamment petites
des variables.

— \si{Ps. path.}  {\bf 3.} Amnésie continue
((syn. : antérograde) : amnésie de
fixation! où le sujet « oublie à mesure » tout ce qu'il perçoit (voir
Précis, Ph. I, p. 225).

\ib{Continuité} — \si{Méta.} Principe de continuité : « La nature ne fait pas de
sauts » (Leibniz), i. e. il n’y a pas de
solution de continuité entre les êtres
ou les phénomènes de la nature.

\ib{Contradiction} — \si{Log. form.} {\bf 1.} O
Action d'affirmer et de nier en même
temps une même chose. Principe de
contradiction : « Deux propositions
contradictoires$^1$ ne peuvent être à la
Lois toutes deux vraies ni toutes deux
fausses » (cf. Alternative$^3$). Impliquer contradiction : voir Contradictoire$^3$. — 2, @ (Au pluriel) Propositions contradictoires$^1$ : « On trouve
de pareilles contradictions chez
tous les peuples... les circonstances
introduisent d'âge en âge des usages
et des opinions contradictoires »
(Condillac).

\ib{Contradictoire} — \si{Log. form.} A. Relativement : À. Propositions contradictoires (entre elles) : propositions
opposées* différant à la fois en
quantité* et en qualité*, vg. « Tous
les hommes sont mortels » et « Certains hommes ne sont pas mortels ».
% 44

—  {\bf 2.} Concepts contradictoires (entre
eux) : concepts tels que l'affirmation
de l’un implique la négation de
l’autre et que la négation de l’un
implique l'affirmation de l’autre
(vg. « coloré » et « non coloré »). —
B. Absolument :  {\bf 3.} Une proposition
ou un concept est contradicioire (en
soi) ou « implique contradiction »
quand on peut la cou le décomposer
en deux propositions contradictoires$^1$ ou en deux concepts contradictoires$^2$.

\ib{Contraire} — \si{Log. form.} {\bf 1.} Propositions contraires : propositions opposées*, toutes deux universelles, l’une
affirmative, l’autre négative, vg.
« Tous les hommes sont mortels » et
« Aucun homme n’est mortel » —
 {\bf 2.} Concepts contraires : concepts tels
que l'affirmation de l’un implique
la négation de l’autre, mais sans que
la négation de l'un implique l’affirmation de l’autre (vg. « blanc » et
« noir »). $->$ Dist. contradictoire*
(confusion fréquente).

\ib{Contraposition} — \si{Log. form.} Mode
de conversion* consistant à affecter
d’une négation le sujet et l’attribut
d'une proposition et à les faire permuter (schéma : « tout A est B,
donc tout non-B est non-A »).

\ib{Contrariété (Principe de)} — \si{Log. form.} « Deux propositions contraires$^1$
ne peuvent être toutes deux vraies »
(mais elles peuvent être toutes deux
fausses).

\ib{Contraste} — \si{Psycho.} Opposition de
deux qualités. Loi de contraste : le
contraste renforce l'intensité des
états de conscience, not. les états
affectifs et les sensations (vg. contraste des couleurs : simultané lorsqu’elles sont juxtaposées, successif
quand on regarde l’une après avoir
observé l’autre). Association par contraste :
% 45
un des modes de l’association$^3$ des idées.

\ib{Contrat} — \si{Jur.} {\bf 1.} « Convention par
laquelle une ou plusieurs personnes
s’obligent, envers une ou plusieurs
autres, à donner, à faire ou à ne
pas faire qq. ch. » (C. C., 1101). Cf.
Statutaire*.

— \si{Pol.}  {\bf 2.} Chez J.-J. Rousseau,
«contrat social »: contrat fictif qui
constitue le fondement idéal du
droit politique (cf. Contrat social,
l. I, ch. vi).

\ib{Convaincre} — \si{Psycho.} {\bf 1.} M (Au sens
propre). Obtenir  l’assentiment*
d'autrui à l’aide d'arguments d'ordre
purement intellectuel : « On ne peut
réellement convaincre sans être
convaincu soi-même: car la conviction réelle est la suite de l'évidence »
(La Bruyère). $->$ Dist. persuader*,
et cf. Conviction$^2$. —  {\bf 2.} Lato (et
improprement)). © Rendre certain$^1$;
persuader : « Il l’exhorte, il le redresse, il le convainc » (Fléchier).
$->$ Terme équivoque, comme le
mot certitude*.

\ib{Conversion} — \si{Log. form.} Mode de
déduction immédiate* qui consiste,
une proposition étant posée, à former une proposition nouvelle (dite
converse) ayant pour sujet$^2$ l'attribut$^1$ de la première et pour attribut le sujet de la première. Conversion simple : celle où l’on se borne à
faire permuter le sujet et l’attribut,
sans autre modification. Conversion
par négation : celle qui consiste à
transformer une particulière négative en particulière affirmative (en
faisant porter la négation sur l’attribut) et à convertir ensuite celle-ci
simplement. — Cf. Accident$^2$ et
Contraposition*.

\ib{Convertible} — \si{Log. form.} Susceptible de conversion* : « Les universelles
% 45
affirmatives ne sont convertibles que par accident$^2$. »

\ib{Conviction} — \si{Psycho.} {\bf 1.} O Action de
convaincre, spéc. au sens 1 : « La
conviction agit sur l’entendement,
et la persuasion sur la volonté »
(D’Aguesseau). —  {\bf 2.} @ M Certitude$^1$
résultant de ce qu’on a été « convaincu » au sens 1 : « J’exigerais
qu'ils [les libertins] eussent des raisons claires et qui emportent conviction » (La Bruyère). —  {\bf 3.} @ ©
Croyance$^3$ certaine$^1$ résultant de ce
qu’on a été « convaincu » au sens 2 :
« Ce qui pense, c’est-à-dire ce qui
est à l’homme même une conviction
qu’il n’est point matière » (La
Bruyère), $->$ Terme équivoque
comme convaincre. La même ambiguïté existe vg. en all. Cf. Kant,
R. pure, Methodenlehre, II, 3 :
« Quand la croyance [Fürwahrhalten] est valable pour tout le monde,
pourvu seulement qu’on ait de la
raison, son fondement est objectivement suffisant, et la croyance
s’appelle alors conviction$^2$ [Ueberzeugung]. Si elle n’a son fondement
que dans l’état particulier du sujet,
elle est nommée persuasion [Ueberredung]. » Mais un peu plus loin, ibid. :
« La suffisance subjective s'appelle
conviction$^3$ [Ueberzeugung] (pour
moi-même), la suffisance objective
s’appelle certitude [Gewissheit] (pour
tout le monde). »

\ib{Coopération} — {\bf 1.} O Action de coopérer* (en gén.) : « C’est impiété de
n'attendre de Dieu nul secours simplement sien et sans notre coopération » (Montaigne). Spéc. \si{Soc.} Processus social, volontaire ou spontané, par lequel les actions individuelles coopèrent* au lieu de s’opposer; entr’aide, solidarité.

— \si{Éc. soc.}  {\bf 2.} @ Organisation de
la vie économique qui repose sur le
% 46
principe « de faire remise du profit
*au consommateur » (B. Lavergne).
Ceci définit surtout les sociétés
coopératives de consommation. Mais
il existe aussi des associations coopératives de production (celles où les
ouvriers s'associent pour entreprendre des travaux à leur propre
compte au lieu de travailler pour le
compte d'un patron) et des coopératives de crédit (où les adhérents
constituent une caisse commune
destinée à leur fournir des capitaux).

\ib{Coopérer} — Conjuguer, volontairement ou non, son action avec d’autres en vue d'un but à atteindre :
« Les impies mêmes coopèrent au
bien des élus » (Massillon).

\ib{Coordonné} — \si{Math.} {\bf 1.} Coordonnées :
éléments qui déterminent la position d’un point sur une surface ou
dans l’espace.

— \si{Biol.}  {\bf 2.} Caractères coordonnés :
ceux qui sont liés de telle sorte que
la présence, l’absence ou la variation
des uns entraîne la présence, l’absence ou la variation des autres : pg.
« la forme de la dent entraîne la
forme du condyle, celle de l’omoplate, celle des ongles » (Cuvier).

\ib{Copule} — \si{Log. form.} Mot qui, dans
une proposition, exprime le rapport
entre le sujet$^2$ et l’attribut$^1$ (le verbe
« être » dans les prop. d’inhérence$^2$).

\ib{Corporation} — \si{Éc. Pol.} {\bf 1.} Corps de
métier. —  {\bf 2.} Autref. Groupement
hiérarchique et gén. local des maîtres et ouvriers d’un corps de métier.
—  {\bf 3.} Auj. Groupement hiérarchique
et national des patrons et ouvriers
d'un corps de métier.

\ib{Corps} — {\bf 1.} Lato. Tout ce qui tombe
sous nos sens : « La nature m'enseigne que plusieurs autres corps
% 46
existent autour du mien (Descartes,
Méd., VI); « Il n’y a que la foi qui
puisse nous convaincre qu'il y a
effectivement des corps » (Malebranche). En \si{Méta.}, un corps est
une substance jouissant de deux
propriétés essentielles : l'étendue et
l'impénétrabilité. En \si{Phys.}, la propriété fondamentale des corps est la
masse. —  {\bf 2.} Str. Espèce chimique :
« Les corps simples ».

—  {\bf 3.} Organisme de l’homme ou
de l’animal : « Il n’y a rien que ma
nature m'enseigne plus expressément sinon que j’ai un corps » (Descartes, Méd., VI) : « Il est légitime
de dire : je suis mon corps pour
autant que je reconnais ce corps
comme n'étant pas assimilable à
un objet : c’est ainsi qu’on est amené
à faire intervenir le corps-sujet »
(G. Marcel).

—  {\bf 4.} Anal. Groupe social considéré comme une unité vivante : « Le
corps social »; « Les corps constitués »; « Cet acte d'association [le
contrat$^2$ social] produit un corps
moral et collectif, composé d’autant
de membres que l’assemblée à de
voix » (Rousseau). D’où, en Théol.,
« le corps mystique de J.-C. » :
l'Église. — Esprit de corps : sentiment d'unité d’un groupe social :
« Ce n’est pas seulement dans le militaire qu'on prend l'esprit de corps »
(Rousseau).

\ib{Corrélation} — \si{Math.} {\bf 1.} Lato. Liaison
numérique empiriquement constatée
entre deux ou plusieurs caractères
biologiques, psychologiques ou sociologiques (vg. corrélation de la taille
et du poids) : « La notion de corrélation généralise celle de liaison
fonctionnelle » (Darmois). Coefficient de corrélation : nombre variable de - 1 à + 1 qui mesure cette
liaison. —  {\bf 2.} Str. Liaison fonctionnelle
% 47
impliquant un rapport de causalité entre les variables.

— \si{Biol.}  {\bf 3.} Corrélation des formes :
principe morphologique* d’après lequel les parties d’un organisme
vivant sont liées de telle sorte que
la forme de l'une entraîne la forme
des autres (cf. Coordonné$^2$).

\ib{Cortical}. — \si{Phol.} Qui. appartient à
l’écorce* cérébrale ou cortex.

\ib{Cosmique} — \si{Méta.} Qui se rapporte au
cosmos* : « La matière cosmique » ;
« La signification cosmique de la
conscience » (Gusdorf).

\ib{Cosmogonie} — \si{Épist.} Théorie de
l’origine de l'univers.

\ib{Cosmographie}. — \si{Épist.} Astronomie
élémentaire et descriptive.

\ib{Cosmologie} — \si{Épist.} {\bf 1.} Science positive des lois générales de la matière
(peu usité auj. en ce sens). — \si{Méta.}
 {\bf 2.} Partie de la métaphysique traitant de l’essence de la matière et de
la vie (cosmologie rationnelle).

\ib{Cosmologique (Argument)} — Voir
Contingence$^3$.

\ib{Cosmos} — [G. kosmos, ordre; d'où :
monde] — Le monde, la nature,
considérés comme un tout organisé
et harmonieux.

\ib{Courage} — {\bf 1.} Autref., tous les sentiments du « cœur » : « Ce malheureux
visage D'un chevalier romain captiva
le courage » (Corneille).

— \si{Mor.}  {\bf 2.} Une des vertus cardinales* : domination de soi-même,
caractérisée par l'empire de la volonté sur la sensualité, l'émotivité
ou les « passions : Un courage d’esprit, rare même parmi ceux qui ont
le courage du cœur » (Fontenelle);
« D'une âme faible, elle [la philosophie] ne saurait faire une âme
% 47
forte : il y a bien des sortes de courages » (Condillac).

\ib{Courant (de conscience, de pensée)} —
[trad. angl. stream of conscionsness,
of thought] — \si{Psycho.} Nom donné
par W. James au flux continuel des
états psychiques dans la conscience$^2$,
voir Précis, Ph. I, p. 80-8 {\bf 2.}

\ib{Coutume} — \si{Soc.} {\bf 1.} Pratique générale
et traditionnelle dans un groupe
social donné. Cf. Mode$^5$ et voir
Précis, Ph. I, p. 53 {\bf 7.} — \si{Jur.}  {\bf 2.} Coutume$^1$ ayant force de loi (l’ensemble
de ces coutumes forme le droit coutumier) : « On peut définir la coutume juridique comme une règle de
droit obligatoire traditionnelle qui
prend naissance spontanément, en
dehors de tout organisme spécialisé » (H. Lévy-Brubl).

\ib{Création} — \si{Théol.} et \si{Méta.} Action par
laquelle Dieu donne l'être à l’univers, le tire du néant. — A Création
continuée théorie selon laquelle
« la conservation des choses par
Dieu ne se fait pas par une nouvelle action, mais par la continuation de l’action par laquelle il leur
donne l’être » (saint Thomas, S. th.,
I, 104, 1). Cf. Descartes, Méth., V :
« C'est une opinion communément
reçue entre les théologiens que
Paction par laquelle il [Dieu] le
conserve [le monde] est toute la
même que celle par laquelle il l’a
créé », et voir Émanation$^2$.

\ib{Créationnisme} — \si{Méta.} À Doctrine
selon laquelle l'univers a été, non
seulement organisé, mais appelé à
l'existence par Dieu. Ci. Théisme*.

\ib{Créatrice (Activité ou Imagination)}
— Voir Imagination*.

\ib{Crédulité} — \si{Psycho.} Tendance à
croire$^3$ sans réflexion, à accepter les
% 48
assertions sans critique : « La crédulité des peuples, qui est toujours au-dessus du ridicule et de l’extravagant... » (Montesquieu) : « Un vif
intérêt$^2$ engendre la crédulité »
(Condorcet).

\ib{Crime} — \si{Soc.} {\bf 1.} Violation des règles
que la société considère comme
indispensables à son existence : « Un
acte est criminel quand il offense les
états forts et définis de la conscience
collective » (Durkheim). — \si{Jur.}
 {\bf 2.} « L’infraction que les lois punissent d’une peine afflictive [mort,
travaux forcés, déportation, détention, ou réclusion] ou infamante
[bannissement, dégradation civique]
est un crime » (Code pénal).

— \si{Mor.}  {\bf 3.} Faute morale grave :
« Le crime est d’obéir à des ordres
injustes » (Voltaire).

\ib{Criminologie} — \si{Épist.} Partie de la
sociologie qui étudie la criminalité,
i. e. la fréquence des crimes$^1$, et leurs
causes.

\ib{Cristallisation} — \si{Psycho.} Chez Sitendhal : phénomène par lequel, dans
la passion, l'imagination transfigure
l’objet de celle-ci : « Ce que j'appelle
cristallisation, c’est l'opération de
l'esprit qui tire de tout ce qui se
présente la découverte que l’objet
aimé a de nouvelles perfections. »

\ib{Critérium} — [G. krinein, juger] — \si{Crit.}
Signe qui permet de distinguer avec
sécurité une chose parmi d’autres :
« Le critérium de la vérité. »

\ib{Criticisme} — \si{Hist.} A. Doctrine de
Kant selon laquelle le problème
critique$^1$ doit être le centre de toute
recherche philosophique : elle aboutit
au relativisme*.

\ib{Critique} — {\bf 1.} \si{Épist.} et \si{Crit.} Appréciation d’une chose du point de vue
de sa valeur$^2$ : « critique d’un témoignage ». Critique historique : partie
du travail de l'historien consistant
dans l'analyse des documents et
l'appréciation de leur valeur$^2$ (cf.
Externe$^3$). Spéc. : problème critique,
problème de la valeur$^2$ de la connaissance humaine; critique (syn.
théorie) de la connaissance, partie de
la philosophie qui concerne le problème critique.

— \si{Psycho.}  {\bf 2.} Esprit critique (ctr.
crédulité) : disposition d'esprit qui
permet de ne pas accepter sans contrôle une assertion. $->$ Dist. esprit
de critique (i. e. de dénigrement).

\ib{Croire, Croyance} — \si{Psycho.} © Ces
termes peuvent s’appliquer : {\bf 1.} à
une opinion$^1$ fondée sur une simple
probabilité$^1$ : « Je ne croyais pas que
tout fût perdu » (Sévigné); « Deux
sortes d'hommes : les uns justes qui
se croient pécheurs, les autres pécheurs qui se croient justes » (Pascal,
534); en ce sens, qqfs. opp. à savoir:
« Nous ne pouvons pas croire ce que
nous savons, et nous ne pouvons
pas savoir ce que nous croyons »
(Pradines); —  {\bf 2.} (syn. : foi$^4$) à une
certitude$^1$ qui ne résulte pas uniquement d'une démonstration rationnelle, soit qu’elle se fonde sur l’autorité$^2$ et le témoignage, soit qu'elle
repose sur des motifs affectifs (sentiments)) et actifs (aspirations, inclinations, désirs) ou qu'elle relève des
exigences de la « raison pratique$^2$ »,
soit enfin (foi$^5$ religieuse) qu’elle
dépasse la raison : « Elle croit, elle
qui jugeait la foi impossible »
(Bossuet); « Il me fallut abolir le
savoir [Wissen] afin d'obtenir une
place pour la croyance (Glauben] »
(Kant, R. pure, préf. 2$^\text{e}$ éd.); « Une
religion est d’autant plus crue qu’elle
suscite davantage les sentiments
profonds » (Delacroix); « On croit en
Dieu plus qu’on ne le prouve » (Le
% 49
Roy); —  {\bf 3.} lato : à l'assentiment* en
gén. : « Nier, croire et douter bien
sont à l’homme ce que courir est au
cheval » (Pascal, 259); « Toute aperception$^2$ suppose affirmation implicite, au sens de croyance, même si
elle était unique, simple... Si elle
est multiple, elle est croyance à la
liaison de ses parties » (Lagneau);
« La croyance est un genre dont
la certitude$^2$ est une espèce » (Brochard).

—  {\bf 4.} M Objet de la croyance aux
sens 1, 2 ou 3 : « Les croyances religieuses » ; « La croyance à la liberté ».

\ib{Crucial} — [L. crux, croix; d’où : poteau
placé à un carrefour pour indiquer
la route] \si{Épist.} — Chez Bacon :
« faits cruciaux » ou « expériences$^3$
cruciales », cas décisifs qui permettent de se prononcer à coup sûr entre
deux hypothèses.

\ib{Culpabilité (Sentiment de)} — \si{Ps. path.} Sorte de « complexe inconscient
d’accusation » (Baruk) qui fait que,
dans certaines psychopathies* (mélancolie*, schizophrénie*), le malade
éprouve de la douleur morale pour
certaines fautes que souvent il n’a
pas commises.

\ib{Cutanées (Sensations)} — \si{Psycho.} Sensations tactiles de la peau.

\ib{Cybernétique} — [G. kubernêtês, pilote] —
Techn. Technique du fonctionnement et du contrôle des commandes
électro-magnétiques et des transmissions électroniques dans les machines à calculer et les automates
modernes. On l’a étendue par la suite
à l’étude des connexions nerveuses
dans les organismes vivants, et on a
même prétendu l'appliquer aux
groupes humains et en faire une
science du gouvernement des hommes réduits à l’état de « robots ».
% 49
$->$ Éviter d'employer à ce propos
les termes : « machines à percevoir,
à se souvenir, à penser ». Cette façon
de parler est tout à fait impropre.

\ib{Cyclothymie} — \si{Car.} Type psychologique caractérisé par la tendance
aux changements périodiques d’humeur, En s’accentuant, il conduit à
la folie circulaire*.

\ib{Cynisme} — \si{Hist.} {\bf 1.} À École et doctrine d’Antisthène, de Diogène de
Sinope, etc.

—  {\bf 2.} \si{Mor.} À. Attitude qui repose
sur « le mépris des conventions sociales et même de la morale communément admise » (Lalande).

\ib{Cyrénaïsme} — \si{Hist.} À. École et doctrine d’Aristippe de Cyrène; hédonisme*.

	\end{itemize}
