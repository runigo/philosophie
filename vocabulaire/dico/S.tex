
	\begin{itemize}[leftmargin=1cm, label=\ding{32}, itemsep=1pt]

\ib{Sacré} — \si{Théol.} {\bf 1.} Qui inspire un respect religieux. —
\si{Soc.} {\bf 2.} Qui est séparé des choses et êtres ordinaires (dits
{\it profanes}) par un système d’interdits : « Les choses sacrées sont celles
que les interdits protègent et isolent » (Durkheim). {\it Cf.}
{\it Numineux}*. — {\it Ext.} {\bf 3.} Respectable, inviolable :
% 164
« Les droits sacrés de la personne humaine. »

\ib{Sadisme} — \si{Ps. an.} Perversion qui consiste à trouver le plaisir
dans la souffrance infligée à autrui.

\ib{Sagesse} — {\bf 1.} {\it Autref.}, savoir, science$^1$ : « La sagesse
accumulée des siècles » (Diderot) ; « Le plus ancien nom de la philosophie
fut sagesse » (Renouvier). — {\it Auj.}, \si{Mor.} {\bf 2.} (Syn. :
prudence$^1$) La vertu de l'intelligence : « La sagesse est la vertu propre
de lentendement... Le mal le plus contraire à la sagesse, c’est exactement
la sottise » (Alain). — {\bf 3.} (\si{Vulg.}). Modération dans les désirs.

— {\bf 4.} \si{Théol.} La Sagesse$^2$ parfaite, le Verbe$^2$ divin : « La
Raison s'est incarnée : les hommes ont vu de leurs yeux la Sagesse
éternelle » (Malebranche, {\it Entr.}, V, 9).

\ib{Sainteté} — \si{Mor.} Perfection morale, dans une conception religieuse
de la vie : « La sainteté est en Dieu une incompatibilité essentielle avec
tout péché » (Bossuet) ; « Le propre de la sainteté, c’est de nous découvrir
la relation entre les deux mondes, le matériel et le spirituel » (Lavelle).

\ib{Salut} — \si{Théol.} Destinée bienheureuse des justes$^1$ : « Dieu veut
le salut de tous les hommes » (Malebranche).

\ib{Sanction} — [L. {\it sancire}, consacrer] — \si{Jur.} et \si{Soc.}
{\bf 1.} Confirmation d’une loi par le chef de l'État qui la rend exécutoire
en la promulguant. {\it Ext.} Confirmation en gén. : « Un terme qui n’a pas
encore reçu la sanction de l'usage. » — {\bf 2.} Peine établie par la loi :
« Les sanctions pénales. »

— \si{Mor.} {\bf 3.} Récompense ou punition appliquée à l’agent moral : « La
sanction est une conséquence de l’acte qui ne résulte pas du contenu
% 164
de l'acte, mais de ce que l’acte [est ou] n’est pas conforme à une règle
préétablie » (Durkheim). — {\it Ext.} {\bf 4.} Toute conséquence résultant,
pour notre sensibilité, des actes que nous avons commis. {\it Sanction
intérieure} : celle du remords* ou de la satisfaction de conscience.
{\it Sanction naturelle} : celle qui résulterait du jeu même des lois
naturelles, {\it p. e.} la maladie conséquence de l’intempérance.
$->$ Impropre en ce dernier sens.

\ib{Scalaire} — \si{Math.} Se dit des grandeurs : {\bf 1.} non dirigées,
{\it p. e.} le travail$^1$ (opp. : {\it vectoriel}, {\it p. e.} la
force$^4$) ; — {\bf 2.} variant de façon discontinue. $->$ Impropre au sens 2.

\ib{Scepticisme} — [G. {\it skeptomai}, j'examine] — \si{Crit.}
\fsb{S. norma.} {\bf 1.} (Opp. : {\it dogma\-tisme}$^2$). {\it Str.} Doctrine
selon laquelle l'esprit humain ne peut rien connaître avec certitude$^2$ et
qui conclut à la suspension* du jugement et au doute$^1$ permanent : « Le
scepticisme se nie en se posant comme vrai » (Lagneau). — {\bf 2.} (Opp. :
{\it dogmatisme}$^3$). {\it Lato.} Toute doctrine qui nie la possibilité de
la connaissance de l'absolu {\it p. e.} {\it a)} celle de Hume, ainsi
qualifiée par lui-même en tant qu’elle aboutit au doute à l'égard : 1° de
l’existence des objets extérieurs; 2° de la connexion* nécessaire des causes
et des effets (Entend. humain, IV) ; — {\it b)} celle de Kant en tant qu’elle
déclare la connaissance entachée de relativité$^1$. $->$ Tout à fait impropre
au sens 2 : la doctrine de Hume est une sorte de {\it positivisme} ; celle de
Kant, un {\it relativisme} ; même certains « sceptiques » de l'antiquité
étaient plutôt des empiristes, {\it p. e.} Sextus Empiricus : « Les
sceptiques ne détruisent pas les apparences. » Le seul sens propre est le
sens {\bf 1.}

\ib{Sceptique} — \si{Hist.} \fsb{S. norma.} Qui professe le scepticisme : {\bf 1.} au
sens 1 : « Les sceptiques qui ne doutent que pour douter » (Descartes,
{\it Méth.}, III) ; —  {\bf 2.} au sens 2 : « M. Hume, ce fameux
sceptique... » (Voltaire) ; « Kant, le
% 165
plus profond et le plus original des sceptiques modernes... » (Saisset).
$->$ Très impropre au sens {\bf 2.}

— \fsb{S. posit.} {\bf 3.} Qui adopte une attitude de doute à l’égard de
la {\it foi}$\,^6$ religieuse : « Faut-il l’abandonner [Dieu]... aux langues
du sceptique et du blasphémateur? » (Lamartine).

\ib{Schéma} — \si{Techn.} {\bf 1.} Figure ou diagramme simplifié. —
\si{Psycho.} {\bf 2.} {\it Chez Bergson} : « schéma dynamique »,
« représentation simple, développable en images multiples » et « dont les
éléments s’entrepénètrent » « Le sentiment de l'effort intellectuel se
produit sur le trajet du schéma à l’image » ({\it E. S.}, VI).

\ib{Schématisme} — \si{Hist.} {\it Chez Kant} : « schématisme des concepts
purs de l’entendement » ({\it R. pure}, {\it Analyt.} II, 1), fonction des
schèmes$^5$ transcendantaux.

\ib{Schème} — {\bf 1.} Syn. de {\it Schéma}$^1$ : « Le schème n’est qu’une
figure, simplifiée et concrète, représentant les traits essentiels d’un objet
également concret ou d’un mouvement » (Dumas). — \si{Psycho.} {\bf 2.}
« Résumés ou abréviations qui conditionnent les opérations des sens, du
sentiment, de l'esprit » (Revault d’Allonnes) : « Les schèmes de
l’attention ». — {\bf 3.} (Syn. : {\it tendance formatrice}). « Forme$^2$
qui s’inscrit progressivement dans une matière et qui l’organise »
(Burloud). — Spéc., {\it schème moteur} : ensemble de rapports spatiaux et
temporels qui organise une suite de mouvements « Le schème de la valse » —
{\bf 4.} {\it Schème opératoire} : forme$^2$ qui sert aux opérations de
l’entendement, {\it not.} à la formation des concepts : « Le concept général
n’est ni un simple signe, ni une idée véritable,
% 165
{\it eïdos} : il consiste dans un schème opératoire de notre
entendement » (Lalande).

— \si{Hist.} {\bf 5.} {\it Chez Kant} : « schème transcendantal »,
représentation intermédiaire entre les catégories* et les intuitions$^1$
empiriques : « Le schème pur de la quantité est le nombre » ({\it R. pure},
Analyt., II, 1).

\ib{Schizoïdie} — \si{Car.} Orientation du caractère prédisposant à la schizophrénie* et caractérisée par l’amour de la solitude et le repliement sur soi.

\ib{Schizophrénie} — ([G. {\it schizein}, partager, et {\it phrên},
esprit] — \si{Ps. path.} Nom donné par Bleuler à une incoordination
psychique proche de la démence* précoce et caractérisée par l'autisme* et
l'affaiblissement de l’affectivité.

\ib{Science} — \si{Épist.} {\bf 1.} {\it Lato.} Savoir en {\it gén.},
connaissance : « J'entends par la science du monde l’art de se conduire avec
les hommes » (D’Alembert) ; « Savoir de science certaine... ». — {\bf 2.}
{\it Str.} Ensemble de connaissances et de recherches méthodiques ayant pour
but la découverte des lois$^5$ des phénomènes : « La science a trouvé dans
l’expérience un principe propre et immanent$^2$ d’où elle tire, sans autre
auxiliaire que l'activité intellectuelle commune, les faits, matériaux de son
œuvre, et les lois$^5$ à l'aide desquelles elle coordonne les
faits » (Boutroux).

\ib{Scientisme} — \si{Épist.} ({\it Péj.}) Conception déformée de la
science$^2$ qui consiste : {\bf 1.} soit à en faire une connaissance
dogmatique$^2$, un système clos et définitif ; — {\bf 2.} soit à lui
demander la solution de tous les problèmes.

\ib{Scolastique} — \si{Hist.} {\bf 1.} Qui se rattache
à la philosophie de « l'Ecole », {\it i. e.}
% 166
celle qu'on enseignait au moyen âge et jusqu'au
{\footnotesize XVII}$^\text{e}$ siècle dans les Universités (celle d’Aristote
adaptée au dogme chrétien) : « Il y a dans les sentiments des philosophes et
théologiens scolastiques bien plus de solidité qu'on ne s'imagine » (Leibniz,
{\it Disc. méta.}, 11).

— {\it Ext.} {\bf 2.} ({\it Péj.}). Formel$^5$, verbal et figé dans les
cadres traditionnels : « Une argumentation trop scolastique. »

\ib{Second, Secondaire} — Voir {\it Premier}$^4$,
{\it Primaire}* et {\it Secondarité}*.

\ib{Secondarité} — \si{Car.} (Opp. : {\it primarité}*). Trait de caractère$^3$
où domine le retentissement* : « Chez le secondaire, le passé ne sert pas
uniquement à appuyer le présent : il le prédétermine, l’oriente, le dessine
d'avance; il en refuse certains aspects et le prolonge dans
l’avenir » (Berger).

\ib{Ségrégation} — {\bf 1.} Processus par lequel des êtres ou objets de même
nature sont isolés les uns des autres et mis à part. — {\bf 2.} \si{Psycho.}
Dans la {\it Gestalttheorie} : « ségrégation des unités », processus par
lequel se constituent, dans le champ perceptif total, des unités perceptives
(cf. {\it Précis}, Ph. I, p. 115).

\ib{Sélection} — {\bf 1.} Processus par lequel, parmi plusieurs êtres ou
objets de même nature, certains seulement sont conservés.

— {\it Spéc.} \si{Biol.} {\bf 2.} {\it Sélection naturelle} (Darwin) : fait
que seuls les êtres vivants les mieux armés pour l'existence dans un milieu
déterminé sont appelés à survivre et à perpétuer leur espèce. Cf.
{\it Concurrence}$^2$.

\ib{Sémantème} — [G. {\it sêmainein}, signifier] — \si{Ling.} Élément du
mot qui exprime sa signification (racine, etc.) : « Les
% 166
sémantèmes portent les notions; et les morphèmes*, les rapports »
(Delacroix).

\ib{Sémantique} — \si{Épist.} Partie de la linguistique qui étudie les
significations et l'évolution du sens des mots. $->$ On a dit aussi
{\it séméiotique} et {\it sémiologie}.

\ib{Semences de vérité} — \si{Hist.} Expression employée par Descartes
({\it Méth.}, VI) pour désigner les principes innés* des mathématiques. Cf.
{\it Reg.}, IV : « L'âme humaine a je ne sais quoi de divin, où ont été
déposées les premières semences des vérités utiles. » St. Augustin avait déjà
parlé d’une {\it ratio inseminata}.

\ib{Séminales (Raisons)} — [{\it Trad.} G. : {\it logospermatikoï}] —
\si{Hist.} Notion d’origine stoïcienne, reprise, mais modifiée par Plotin,
puis par St Augustin, et qui désigne les germes dans lesquels auraient été
{\it préformés}* dès l’origine les êtres vivants.

\ib{Sens} — \si{Psycho.} {\bf 1.} ({\it Adj.} correspondant :
{\it sensible}$^2$ ou {\it sensitif}$^1$). Fonction psycho-physiologique qui
consiste à éprouver une certaine classe de sensations* : « Le sens de la vue
» ; « Les sens ne nous sont donnés que pour la conservation de notre corps,
non pour apprendre la vérité » (Malebranche, R. V., I, 10, 5). On a
{\it dist.} des sens {\it internes} et des sens {\it externes}, des sens
impressionnables par {\it contact} direct et des sens impressionnables à
{\it distance} (Sherrington), des sens du besoin et des sens de la
{\it défense} (Pradines). {\it Cf.} aussi {\it Epicritique}* et
{\it Protopathique}*. — {\bf 2.} {\it Anal.} (Adj. corresp. :
{\it sensible}$^3$). Intuition, connaissance spontanée et immédiate :
« Avoir le sens de l'opportunité, le sens pratique, le sens du vrai. »
{\it Sens intime} : nom donné autref. à la conscience$^1$ psychologique :
% 167
« Cet effort primitif est un fait de sens intime, car il se constate lui-même
intérieurement » (Biran). {\it Sens moral} : la conscience$^3$ morale,
{\it spéc.} lorsqu'elle est regardée comme donnant une intuition immédiate du
bien et du mal : « Croire avec Hutcheson, Smith et d'autres que nous ayons un
sens moral propre à discerner le bon et le beau, c'est une vision...
» (Diderot). —  {\bf 3.} Jugement [cf. L. {\it sensus, sententia}] : « Pour
ce qui touche les mœurs, chacun abonde si fort en son sens... » (Descartes,
{\it Méth.}, VI) ; « Diverses sortes de sens droit » (Pascal, 2) ; « À mon
sens... ».

— {\bf 4.} \si{Phol.} (Adj. corresp. : {\it sensoriel}). Organe des
sens$^1$ : « Ce qui se fait dans les sens et dans le cerveau » (Port-Royal).

— {\bf 5.} \si{Mor.} (Adj. corresp. : {\it sensuel}). Au plur. :
impulsions de la vie animale : « Les sens l’ont emporté [l’homme] à la
recherche des plaisirs » (Pascal, 430) ; «Il ne faut rien accorder aux sens
quand on veut leur refuser qqc. » (Rousseau).

— {\bf 6.} \si{Épist.} Signification : « Un même sens change selon les
paroles qui l’expriment » (Pascal, 50) ; « Le principe essentiel du
changement de sens est dans l’existence de groupements sociaux où la langue
est parlée » (Meillet).

— {\bf 7.} \si{Phys.} Orientation : « Sens d’un mouvement, des aiguilles
d’une montre, »

$->$ Ces diverses acceptions ont-elles une unité ? Cette phrase de Claudel le
suggère, mais ne la dévoile pas : « Le temps est le sens de la vie, — sens,
comme on dit le sens d’un cours d’eau, d’une phrase, d’une étoffe, le sens de
l'odorat. »

\ib{Sens commun} — \si{Hist.} {\bf 1.} {\it Chez les
Scolastiques} (lat. : {\it sensorium commune} :
%167
{\it a)} organe central où viendraient se combiner les impressions$^2$ reçues
par les différents sens$^4$ et qui serait aussi l'organe de
l'imagination$^1$ : « Le sens externe étant mis en mouvement par l'objet, la
figure qu’il reçoit est transportée à une autre partie du corps appelée sens
commun... Le sens commun joue aussi le rôle d’un sceau pour former dans la
fantaisie* ou imagination$^1$ les mêmes figures ou idées qui viennent des
sens externes » (Descartes, {\it Reg.}, XII) ; « J'ai cru la connaître [la
cire] par le moyen des sens extérieurs ou, à tout le moins, du sens commun,
ainsi qu'ils [les scolastiques] appellent, {\it i. e.} de la puissance
imaginative » (id., {\it Méd.}, II) ; — {\it b)} fonction de l'esprit
correspondante : « Cette faculté de l’âme qui réunit les sensations, en tant
qu'elle fait un seul objet de tout ce qui frappe ensemble nos sens, est
appelée le sens commun » (Bossuet).

— \si{Vulg.} {\bf 2.} Bon* sens (au sens 2), intelligence$^4$ élémentaire :
« Le sens commun n’est pas une qualité si commune que l’on pense »
(Port-Royal) ; « {\it Sens commun} ne signifie que le bon sens, raison
grossière, raison commencée, première notion des choses ordinaires
» (Voltaire). —  {\bf 3.} Ensemble des opinions professées sur une question
par « le commun » des hommes : « Le sens commun est plein de préjugés* » ; «
La science$^2$ et le sens commun sont ici d'accord [pour morceler la durée]
» (Bergson, {\it P. M.}, I) ; « Le sens commun a raison contre l'idéalisme et
le réalisme des philosophes » ({\it ib.}, VI).

\ib{Sensation} — \si{Psycho.} {\bf 1.} Fait de conscience élémentaire
provoqué par la modification d'un sens$^4$, externe ou interne : « Les
sensations ne sont pas les qualités mêmes des objets ;
% 168
elles ne sont que des modifications de notre âme » (Condillac). $->$ Sur la
dist. entre {\it sensation} et {\it perception}, cf. {\it Précis}, Ph. I,
p. 99 et 112.

— {\bf 2.} (Imppt). {\it Faire sensation} : causer qq. émotion : « Nous
sommes dans un temps où rien ne fait une grande sensation » (Voltaire).

\ib{Sensibilité} — \fsb{S. objec.} \si{Phol.} {\bf 1.} (Syn. : {\it
excitabilité} ou {\it irritabilité}). Propriété des tissus vivants de réagir
d’une certaine manière aux excitants extérieurs : « La sensibilité d’un
muscle au courant électrique ». {\it Sensibilité différentielle} (J. Lœb) :
mouvements par lesquels un être vivant ({\it p. e.} un insecte) réagit aux
variations d'intensité d’un excitant ({\it p. e.} la lumière) : cf.
{\it Précis}, Ph. I, p. 439.

— \fsb{S. subje.} \si{Psycho.} {\bf 2.} Faculté d’éprouver des
sensations$^1$ : « La sensibilité représentative devient esthétique et
symbolique » (Pradines). — {\bf 3.} (Syn. : {\it affectivité}). Faculté
d’éprouver des sentiments, des émotions, des états agréables ou
désagréables : « La sensibilité a pour objet tout ce qui peut affecter l’âme
en bien ou en mal » (D'Alembert). — {\bf 4.} Syn. d'{\it acuité}* : « La
sensibilité de l’ouïe. » — \si{Car.} {\bf 5.} ({\it Opp.} :
{\it insensibilité}). Disposition de caractère qui fait qu’on éprouve
facilement des sentiments$^5$ : « La sensibilité est la condition de la vraie
culture » (Roustan).

\ib{Sensible} — {\bf A)} \fsb{S. subje.} Doué de sensibilité* en tous les sens du
terme : {\bf 1.} « Tous les éléments du corps vivant sont sensibles par
essence » (Biran). —  {\bf 2.} « Les animaux sont des êtres sensibles. » —
{\bf 3.} « Se montrer sensible aux reproches » — {\bf 4.} « L’œil humain est
sensible à une variation de 1/150 en lumière blanche ». —  {\bf 5.} « Une âme
si sensible et si délicate» (Bossuet). — {\bf 6.} \si{Car.} Syn. de {\it
sensuel}$^1$ : « Les hommes sont devenus
% 168
sensibles, grossiers, charnels » (Malebranche, {\it Médiations chrét.}, Il).

— {\bf B)} \fsb{S. objec.} (Opp. {\it intelligible})  {\bf 7.} Perceptible
par les sens$^1$ : « Les qualités sensibles »; « L’étendue sensible »; « Le
sensible, chez Platon, n’est que l’image de l'intelligible ». {\it Chez les
Scolastiques} : « sensibles propres », ceux qui sont particuliers à chaque
sens; « sensibles communs », ceux qui sont communs à plusieurs sens. —
{\bf 8.} Saisissable par le cœur ou par l'intuition : « Dieu sensible au cœur
» (Pascal, 278) ; « Amitiés sensibles qui font une impression vive sur le
cœur » (Bourdaloue) ; « Pour rendre ma pensée sensible. »

\ib{Sensitif} — (Opp. : {\it intellectuel}, dans tous les sens). —
\si{Psycho.} {\bf 1.} Qui concerne les sens : « Les opérations sensitives
» (Bossuet). — \si{Hist.} {\bf 2.} {\it Chez les Scolastiques} : « âme
sensitive », voir Ame$^1$. — {\bf 3.} {\it Chez Maine de Biran} : « système
sensitif » (opp. : {\it système perceptif} et {\it système aperceptif} ou
{\it réflexif}), celui qui est constitué par l’union du {\it moi} aux
impressions de la vie animale.

— {\bf 4.} \si{Car.} Qui a une « sensibilité vive, généralement mobile,
passagère et par conséquent assez superficielle » (Malapert).

\ib{Sensoriel} — \si{Phol.} Qui concerne les
sens$^4$ : « Les organes sensoriels. »

\ib{Sensorium} — Voir {\it Sens commun}.

\ib{Sensualisme} — \si{Hist.} \fsb{S. norma.} Nom imppt. donné (pour la
discréditer) par les éclectiques$^2$ à la doctrine de la « sensation
transformée » de Condillac, selon laquelle toutes nos connaissances et
facultés viennent de la sensation$^1$

\ib{Sensuel} — \fsb{S. subje.} {\bf 1.} Attaché aux plaisirs
des sens$^5$ : « Homme sensuel, ne
% 169
sauras-tu jamais aimer? » (Rousseau). — \fsb{S. objec.} {\bf 2.} Qui concerne les
sens$^5$ : « Les plaisirs sensuels ». —  {\bf 3.} Qui concerne les
sensations$^1$ (avec l’idée d’une valeur propre accordée à l'élément
sensitif$^1$) : « Une piété sensuelle » (Massillon) ; « L'élément sensuel de
l’art ».

\ib{Sentiment} — \si{Vulg.} {\bf 1.} Conscience$^1$ : « Mes pleurs du
sentiment lui rendirent l’usage » (Racine) ; « Perdre le sentiment » (=
s’évanouir). — {\bf 2.} Intuition* (en tous les sens du terme) : « Nous n’avons
point d'idée de notre âme, mais seulement sentiment intérieur » (Malebranche,
{\it Écl.}, XI) ; « J’ai un sentiment clair de ma liberté » (Bossuet) ; « Le
sentiment engendre l’idée ou l'hypothèse expérimentale » (Cl. Bernard) ; « Le
sentiment des convenances ». — {\bf 3.} (Syn. : {\it sens}$^3$. Cf. L.
{\it sententia}). Jugement, opinion : « Il était lui-même dans ce
sentiment » (Pascal, {\it Prov.}, 1) ; « Heurter de front ses
sentiments... » (Molière) ; « A mon sentiment... ».

— \si{Psycho.} {\it Lato.} {\bf 4.} Tout état affectif* : {\it p. e.} :
« Psychologie des sentiments » (Ribot), et même {\it autref.} la
sensation$^1$ : « Croire qu'il y a je ne sais quoi dans les objets qui cause
ces pensées confuses qu’on nomme sentiments » (Descartes, {\it Princ.}, I,
70) ; « Ces philosophes jugent des qualités sensibles par les sentiments
qu'ils en reçoivent » (Malebranche, {\it R. V.}, VI, 2, 2). — {\it Str.}
{\bf 5.} État affectif* complexe et stable dont les causes sont surtout
d'ordre moral$^4$ : « Un sentiment de tristesse » ; « Les sentiments d’une piété sincère » (Bossuet). — {\bf 6.} (Syn. : {\it inclination}).
Disposition affective* durable : « Le sentiment du devoir, de l'honneur ». —
{\bf 7.} {\it Spéc.}, tendresse, altruisme : « Perfectionner la raison par le
sentiment » (Rousseau) ; « Le
% 169
langage du sentiment » ; « Prendre {\it qqn.} par les sentiments ».

\ib{Série} — \si{Épist.} {\bf 1.} Suite de termes ordonnés suivant une
loi$^5$. {\it Spéc.} \si{Math.} : « Les séries convergentes », « La série de
Taylor ». — \si{Biol.} {\bf 2.} {\it Série naturelle} : hiérarchie des êtres
vivants dans la nature.

\ib{Sérieux} — \fsb{S. subje.} {\bf 1.} Grave, qui ne plaisante pas : « Un
homme qui n’a de l'esprit que dans une certaine médiocrité est sérieux » (La
Bruyère) ; « On est sérieux par tempérament; par trop ou trop peu de
passions, trop ou trop peu d'idées : par timidité, par
habitude » (Vauvenargues). — \fsb{S. objec.} {\bf 2.} « Le sérieux, c’est ce
qui, en se présentant comme systématique et constructif, satisfait les
tendances les plus profondes de l'esprit » (Le Senne).

— {\bf 3.} {\it Chez les existentialistes} : « L'homme sérieux est l’homme
d’une seule chose à laquelle il dit oui » (Merleau-Ponty) ; « L'esprit de
sérieux a pour double caractéristique de considérer les valeurs comme des
données transcendantes, indépendantes de la subjectivité humaine, et de
transférer le caractère « désirable » de la structure ontologique des choses
à leur simple constitution matérielle » (Sartre) ; « Il y a sérieux dès que
la liberté$^1$ se renie au profit de fins qu'on prétend absolues » (S. de
Beauvoir).

\ib{Servage} — \si{Soc.} Condition du travailleur manuel qui, tout en
possédant la personnalité juridique (dist. {\it esclavage}*), reste attaché à
la terre qu'il cultive et « corvéable à merci », i.e. à la disposition
constante du propriétaire qui l’emploie.

\ib{Seuil} — \si{Phol.}, \si{Psycho.} Intensité minima d’un excitant
nécessaire pour provoquer une réaction : « Le seuil d’excitation d'un
muscle ». {\it Seuil de la sensation} (syn. : {\it minimum sensible}) : la
plus petite excitation (lumineuse, sonore, etc.) capable de provoquer une
sensation (visuelle, auditive, etc). {\it Seuil différentiel} (syn. : {\it
minimum de différence sensible}) :
% 170
la plus petite variation d’excitation perceptible.

\ib{Signal} — \si{Psycho.} Geste ou symbole impliquant un ordre ou un
avertissement : « Donner le signal du départ, du combat »; « Les signaux de
circulation ».

\ib{Signe} — \si{Psycho.} {\bf 1.} (Syn. : {\it indice}). Phénomène sensible
qui permet d'affirmer la présence d’un objet ou d’un autre phénomène non
perçu actuellement ({\it p. e.} la fumée, signe du feu) où non perceptible
(le cri, signe de douleur ; la pâleur, signe d'émotion) ; « Les signes
extérieurs de richesse ». — {\bf 2.} Réaction volontaire d’un sujet conscient
destinée à {\it signifier}, {\it i. e.} à faire comprendre qqc. à autrui : «
Elle fit signe qu’elle ne voulait aucun soulagement » (Fénelon) ; « L’enfant
ne commence à avoir des signes que lorsqu'il transforme ses cris en signes de
réclame » (Biran). — {\bf 3.} Signal* : « Il lui fit signe d'entrer » ; « Le
roi fit à l'ambassadeur un signe de tête qui lui fit comprendre qu'il ne
voulait point de réplique » (Sévigné). — \si{Psycho.} et \si{Épist.} {\bf 4.}
Symbole* (au sens 2) : « Les signes de l’écriture »; « Les signes de
ponctuation »; « Les signes arithmétiques, algébriques ». $->$ La distinction
entre {\it signes naturels} (ceux qui sont liés à la chose signifiée par une
loi de la nature) et {\it signes artificiels} (ceux qui y sont liés par une
convention) correspond à peu près à celle des sens 1 et 2 d’une part et des
sens 3 et 4 de l'autre.

\ib{Similitude} — \si{Épist.} {\bf 1.} Ressemblance, identité partielle dans
l’ordre qualitatif : « Nous ne concevons presque rien que par
similitude » (Voltaire). — \si{Math.} {\bf 2.} Propriété de deux ou plusieurs
figures « qui ne diffèrent
% 170
que par l'échelle sur laquelle elles sont construites » (Cournot).

\ib{Simple} — \si{Méta.} {\bf 1.} Indivisible ; où l’on ne distingue ni
parties ni éléments : « ... En commençant par les objets les plus simples...
pour monter peu à peu jusques à la connaissance des plus
composés » (Descartes, {\it Méth.}, II) ; « La monade$^2$ n’est autre chose
qu’une substance simple, {\it i. e.} sans parties » (Leibniz, {\it Mon.},
1) ; « Le point mathématique est simple » (Voltaire) ; « Il n’est pas sûr que
la nature soit simple » (Poincaré) ; « Il n'y a pas de phénomènes simples :
le phénomène est un tissu de relations » (Bachelard). {\it Chez Descartes} :
« natures simples », voir {\it Nature}$^2$.

— \si{Log.} {\bf 2.} {\it Terme simple} (Ctr.
{\it complexe}) : celui dont la compréhension$^2$ est pauvre.

\ib{Simplicité des voies} — \si{Hist.} Principe posé par Malebranche d’après
lequel, « non content que l'univers l’honore par son excellence, [Dieu] veut
que ses voies le glorifient par leur simplicité » ({\it Entr.}, IX, 10). Cf.
Leibniz, Disc. \si{Méta.}, 5 : « Pour la simplicité des voies de Dieu, elle a
lieu ppt. à l'égard des moyens, comme au ctr. la variété, richesse ou
abondance y a lieu à l'égard des fins ou effets. »

\ib{Singulier} — \si{Log.} Qui s'applique à un sujet unique : « Terme
singulier » ({\it p. e.} Socrate). {\it Proposition singulière} : celle qui a
pour sujet un terme singulier. $->$ Dist. {\it particulier}\,*.

\ib{Situation} — \si{Méta.} {\bf 1.} Une des catégories* d'Aristote :
{\it p. e.} être couché ou assis. — {\bf 2.} {\it Auj.}, ensemble des
conditions concrètes dans lesquelles se trouve l'existant humain : « On peut
se représenter le moi au carrefour de deux ensembles idéals. Celui par lequel
il éprouvera des
% 171
difficultés, bref un {\it complexe d’obstacles}, définira sa {\it situation}
» (Le Senne) ; « Le point de départ de la philosophie est dans notre
situation » (Jaspers).

\ib{Social} — {\bf 1.} (Ctr. : {\it individuel}). Qui concerne la société en
tant que telle : « Est fait social toute manière de faire, fixée ou non,
susceptible d'exercer sur l'individu une contrainte extérieure, ou bien : qui
est générale dans l'étendue d’une société donnée tout en ayant une existence
propre, indépendante de ses manifestations individuelles. » (Durkheim) ; «
Tout fait social est un moment d’une histoire d’un groupe d'hommes.
» (Mauss) ; « L’interdépendance est l'essence du social » (Lewin). {\it
Psychologie sociale} : celle des groupes sociaux en tant qu'ils possèdent,
comme groupes, une vie psychique propre (cf. {\it Précis}, Ph. I, p. 525).
{\it Sciences sociales} : nom générique désignant à la fois la science
positive de la société (sociologie*), des études abstraites (droit, économie
politique) ou descriptives (histoire, géographie humaine) et même des
disciplines normatives (politique, économie* sociale, {\it qqfs.} morale
sociale), etc. — {\bf 2.} Qui vit en société : « L'homme est un animal social
» ; « Les insectes sociaux. »

— \si{Pol.} {\bf 3.} (Opp. : {\it politique}). Qui
concerne {\it spéc.} les problèmes du travail et de la vie économique : « La
question sociale »; « Les lois sociales ».

\ib{Socialisme} — \si{Soc.} \fsb{S. posit.} {\bf 1.} (Sens primitif, {\it p. e.} chez
Pierre Leroux. Opp. : {\it individualisme}$^6$). Tendance à tout subordonner
à la société : « Nous ne voulons pas sacrifier la personnalité au socialisme,
pas plus que ce dernier à la personnalité » ({\it Le Globe}, journal
saint-simonien, 1831). — {\bf 2.} Mouvement de la société qui tend à
« rattacher toutes les fonctions économiques ou certaines d’entre elles
actuellement diffuses* aux centres directeurs et conscients de la
société » (Durkheim).

— \si{Éc. soc.} ou \si{pol.} \fsb{S. norma.} {\bf 3.} Nom générique des
doctrines qui, soit au nom d’un idéal de justice ou de fraternité (socialisme
dit « utopique » ou idéaliste : {\it p. e.} saint-simonisme, fouriérisme,
proudhonisme, socialisme chrétien), soit au nom d’une interprétation de
l’évolution économique (socialisme dit « scientifique » ou marxisme),
préconisent ou prévoient, à la place de l’organisation capitaliste, « une
organisation concertée aboutissant à des résultats non seulement plus
équitables, mais plus favorables au plein développement de la personne
humaine » (Lalande). Cf. {\it Collectivisme}*.

\ib{Société} — \fsb{S. abstr.} {\bf 1.} {\it Latiss.} Ensemble de rapports
réciproques de communication* ou de communion* : « Nous pouvons faire avec
les hommes deux sortes de sociétés : une société de qqs. années et une
société éternelle : une société de commerce et une société de religion : je
veux dire une société animée par les passions$^1$... et une société réglée
par la Raison » (Malebranche, {\it Traité de Mor.}, II, 6) ; « Les esprits
sont capables d'entrer dans une manière de société avec Dieu » (Leibniz,
{\it Mon.} 84) ; « La société des enfants de Dieu » (Bossuet). — \si{Soc.}
{\bf 2.} {\it Lato.} État de l’homme qui vit d'une vie commune avec ses
semblables: ensemble des groupes humains : « Il n’y a société que là où
s’exerce une action générale et combinée » (Comte) ; « En même temps qu’elle
est transcendante par rapport à nous, la
% 172
société nous est immanente » (Durkheim) ; « On diminue la société quand on ne
voit en elle qu'un corps organisé... Dans ce corps vit une âme » (id.). —
{\bf 3.} {\it Str.} (Opp. : {\it communauté}*). {\it Chez Tœnnies} : type
d'organisation sociale fondé sur la volonté réfléchie et l’échange.

— \fsb{S. concr.} \si{Soc.} {\bf 4.} Un type particulier de société$^2$
concrète, ou un groupe social particulier : « Plus les sociétés sont vastes,
plus elles ont besoin de réflexion pour se conduire » (Durkheim) ; « La
famille est une sorte de société complète » (id.) ; « Les sociétés archaïques
» ; « Les sociétés modernes ». — \si{Jur.} {\bf 5.} « Contrat par lequel deux
ou plusieurs personnes conviennent de mettre qqc. en commun dans la vue de
partager le bénéfice qui pourra en résulter » ({\it C. C.}, 1832).

\ib{Sociodrame} — \si{Soc.} Sorte de psychodrame* qui s'adresse à un groupe
et a pour but une catharsis* collective. — Cf. {\it Sociométrie}*.

\ib{Sociogramme} — \si{Soc.} Graphique qui représente, en sociométrie*, les
attractions et répulsions entre membres d’un même groupe. Cf.
{\it Précis}, Ph. I, p. 529).

\ib{Sociologie} — Epist. Science positive de la vie sociale, de ses types$^2$
et de ses lois$^5$ : « La sociologie ne pouvait apparaître avant qu’on n’eût
acquis le sentiment que les sociétés sont soumises à des lois » (Durkheim).

\ib{Sociométrie} — \si{Soc.} Forme de recherches sociologiques (ou plutôt
interpsychologiques) créée par J. Moreno et qui prétend mesurer les rapports
de « distance psychique » (voir {\it Tèlé}\,*) entre « atomes sociaux »,
{\it i. e.} entre individus impliqués en des rapports interpersonnels. Cf.
{\it Psychodrame}* et {\it Sociodrame}*.

\ib{Socius} — \si{Soc.} Terme introduit dans la sociologie américaine par
Giddings (1898) et qui désigne « l’unité d'investigation en sociologie,
{\it i. e.}
% 172
l'individu en tant que compagnon, apprenti, maître ou co-travailleur ».

\ib{Soi} — \si{Ps. an.} Autre nom du {\it Çà}*.

\ib{Solidarisme} — \si{Mor.} \fsb{S. norma.} Doctrine qui pose la
solidarité$^4$ comme principe de la morale, de la politique et de l’économie
sociale.

\ib{Solidarité} — [L. {\it solidus}, massif, qui forme bloc] — {\bf A)}
\fsb{S. posit.} {\bf 1.} \si{Jur.} Le fait, pour des débiteurs, d’être «
obligés à une même chose de manière que chacun puisse être contraint pour la
totalité » ({\it C. C.}, 1200). — {\bf 2.} \si{Phys.}, \si{Biol.}, \si{Soc.},
\si{Psycho.} Dépendance unilatérale d'une partie d'un mécanisme à l'égard
d’une autre, d’un organe à l'égard d'un autre, d’une génération à l'égard des
précédentes et {\it gén.} du présent à l'égard du passé : « L'hérédité est
une forme de solidarité »; « La {\it solidarité personnelle} est double :
d’une part, selon que la personne s’est déterminée dans le passé, elle veut
encore se déterminer dans l'avenir; ... d’autre part, la nature morale acquise
devient un élément des déterminations de la personne actuelle » (Renouvier).
— {\bf 3.} \si{Biol.}, \si{Soc.}, \si{Psycho.} Dépendance {\it réciproque}
des éléments et des fonctions dans un organisme, une société, etc. : « L’être
vivant se définit par la solidarité des fonctions qui lie les parties
distinctes » (Ch. Gide) ; « C’est la répartition continue des différents
travaux humains qui constitue la solidarité sociale » (Comte). {\it Spéc.}
\si{Soc.} {\it Chez Durkheim} : « solidarité mécanique », solidarité qui
existe dans les sociétés peu différenciées et qui repose sur la {\it
similitude} des unités qui les composent (conformisme, etc.) ; « solidarité
organique », celle qui est créée par la division$^3$ du travail et qui repose
sur les {\it différences} des fonctions,
% 173
devenues nécessaires les unes aux autres.

— {\bf B)} \fsb{S. norma.} {\bf 4.} Devoir ou vertu résultant, soit de la
solidarité$^2$ d’une génération à l'égard des précédentes envers qui elle se
reconnaît une {\it dette}, soit de la solidarité$^3$ des individus qui
prennent conscience de leurs {\it obligations} réciproques en tant que
membres d’un même corps : « Le mot {\it solidarité} a pris depuis qqs. années
un sens nouveau... Il exprime alors la {\it notion d’un devoir} à observer
par tout homme vis-à-vis de ses semblables » ({\it L.} Bourgeois) ; « Un acte
de solidarité ».

\ib{Solipsisme} — [L. {\it solus ipse}, seul moi-même] — \si{Crit.} Attitude
philosophique qu'on a qqfs. donnée comme la conséquence de l'idéalisme$^4$
extrême et selon laquelle « tout esprit est comme un monde à part, suffisant
à lui-même » (Leibniz, qui d’ailleurs n’emploie pas ce terme) : « Autant le
solipsisme est déraisonnable quand on y voit une doctrine, autant il est
incontestable lorsqu'il se présente simplement comme l'expression de ce fait
que les images sont relatives à un sujet individuel » (Blanché) ; « Si le
solipsisme doit pouvoir être réfuté, c'est que mon rapport à autrui est
fondamentalement une relation d’être à être » (Sartre).

\ib{Solution de continuité} — Interruption
de ce qui est ou devrait être continu.

\ib{Somatique} — [G. {\it sôma}, corps] — Qui concerne le corps$^3$.

\ib{Somnambulisme} — \si{Ps. path.} État dans lequel un sujet endormi exécute
des actes dont il ne garde aucun souvenir au réveil.

\ib{Sophisme} — [de {\it sophiste}$^2$] — \si{Log.} {\bf 1.} Raisonnement
captieux, paralogisme*
% 173
fait avec l'intention de tromper. — {\it Ext.} {\bf 2.} Paralogisme* en
gén. : « Ce sophisme (l'argumentation du parallélisme*] n’a rien de voulu
» (Bergson, {\it E. S.}, VII).

\ib{Sophistes} — [G. {\it sophia}, sagesse] — \si{Hist.} {\bf 1.} Primitivement,
professeurs de sagesse. — {\bf 2.} Plus tard., péj. ; philosophes grecs
contemporains de Socrate, dénoncés par leurs adversaires [Platon] comme des
rhéteurs de mauvaise foi. — {\bf 3.} {\it Ext.}, (Au sing.) Homme qui use de
sophismes$^1$.

\ib{Sorite} — [G. {\it sôros}, tas] — \si{Log.} \si{form.} Polysyllogisme* où
l’attribut$^1$ de la première proposition devient sujet$^2$ de la seconde,
l’attribut de la seconde sujet de la troisième, etc., et où la conclusion$^3$
unit le sujet de la première et l’attribut de la dernière ({\it p. e.} le
raisonnement du renard qui, de ce que la rivière fait du bruit, conclut
qu’elle remue, donc qu’elle n’est pas gelée, donc qu’elle est liquide, donc
qu'elle ne peut le porter).

\ib{Sotériologie} — [G. {\it sôtêr}, sauveur] —
\si{Théol.} Doctrine du salut*.

\ib{Souci} — [{\it Trad.} all. Sorge] — \si{Méta.} {\it Chez Heidegger} :
l'être même du {\it Dasein}$^2$ en tant qu’anticipation de soi, que
« pro-jeté en avant de lui-même ». Cf. {\it Préoccupation}* et {\it Pro-jet}*.

\ib{Souvenir} — \si{Psycho.} {\bf 1.} {\it Lato.} Tout état ancien qui est
censé « se conserver » dans la mémoire et être susceptible de reparaître dans
l'esprit : « Un souvenir affectif » Spéc., {\it chez Bergson} : « souvenir
pur », notre passé qui « dure » dans l'inconscient : « Le souvenir$^2$
actualisé en image diffère profondément de ce souvenir pur... [Celui-ci],
impuissant tant qu'il demeure inutile, reste pur de tout mélange avec la
sensation sans attache avec le présent »
% 174
({\it Mat. et Mém.}, III). Qqfs. appelé « souvenir personnel » : « Les
souvenirs personnels, essentiellement fugitifs, ne se matérialisent que par
hasard » ({\it ib.}, II), ou « souvenir-fantôme » : « Parmi les
souvenirs-fantômes qui aspirent à se lester de matérialité, ceux-là seuls y
réussissent qui... » ({\it E. S.}, IV). $->$ Bergson nomme aussi « souvenir »
l'habitude : « À ce moment Précis je sais ma leçon par cœur : on dit qu’elle
est devenue souvenir » [inexact : le lang. courant ne le dit pas), mais dist.
radicalement ce souvenir-habitude de l'{\it image-souvenir} : « ... Deux
mémoires théoriquement indépendantes : la première enregistrerait, sous forme
d’images-souvenirs, tous les événements de notre vie » ({\it Mat. et Mém.},
II). — {\bf 2.} {\it Str.} État passé qui revient à la conscience et est, en
outre, reconnu comme tel et souvent rapporté à un moment déterminé du passé
(cf. {\it Mémoire}$^2$) : « Le souvenir n’est pas l’image, mais un jugement
sur l’image dans le temps » (Delacroix). $->$ Très impropre au sens {\bf 1.}
Le seul sens propre est le sens {\bf 2.}

\ib{Spatial} — Qui a les caractères de l’espace ou se localise dans
l’espace : « Les perceptions spatiales. »

\ib{Spécial} — \si{Log.} Un terme général$^2$ est dit {\it spécial} par
rapport à un autre (dit « plus général$^3$ ») quand son extension est
comprise dans celle de ce dernier : {\it p. e.} « losange » par rapport à «
parallélogramme » « Certaines démonstrations [mathématiques] vont du spécial
au général » (Goblot).

\ib{Spécieux} — \si{Épist.} {\bf 1.} Qui présente une apparence, le plus
souvent trompeuse, de vérité : « Un argument spécieux ». — {\bf 2.} (Au
fém.). L’algèbre :
% 174
« L'analyse ou l’algèbre spécieuse est assurément la plus belle de toutes les
sciences » (Malebranche, R. V., IV, 11, 2). {\it Chez Leibniz} : « spécieuse
universelle », sorte de Logique algorithmique*.

\ib{Spécificité} — \si{Épist.} {\bf 1.} Caractère spécifique$^2$. — \si{Phol.}
{\bf 2.} {\it Spécificité des sens} : le fait que la qualité d’une sensation
dépend de l'organe impressionné, et non de la qualité de l’excitant (voir
{\it Précis}, Ph. I, p. 103).

\ib{Spécifique} — \si{Épist.} {\bf 1.} Qui concerne l’espèce* (aux sens 2 ou
3) ou lui appartient en propre. Cf. {\it Différence}$^1$. — {\bf 2.} Qui a
une nature propre : « Ces faits spécifiques [les faits sociaux] résident dans
la société même qui les produit » (Durkheim).

\ib{Spéculatif} — {\bf 1.} Qui concerne la spéculation$^1$ pure : « Notre âme
n'est guère attentive aux choses purement spéculatives, mais beaucoup plus à
celles qui la touchent » (Malebranche, R. V., IV, 11). — {\bf 2.} {\it Chez
Kant} : qui concerne la spéculation au sens {\bf 2.}

\ib{Spéculation} — {\bf 1.} (Syn. : {\it théorie}$^1$, Opp. :
{\it pratique}$^3$). Activité intellectuelle désintéressée ayant pour seul
but de connaître : « Cela est permis dans la spéculation, mais je n’en
approuve pas la pratique » (Pascal, {\it Prov.}, XIII) ; « Tous les travaux
humains sont ou de spéculation ou d'action » (Comte, {\it Cours}, II). —
{\bf 2.} ({\it Opp.} : expérience$^1$). {\it Chez Kant} : recherche portant
sur ce qui est inaccessible à l'expérience : « Le but final auquel se
rapporte la spéculation de la raison dans l’usage transcendantal concerne
trois objets : la liberté de la volonté, l’immortalité de l'âme et
l'existence de Dieu » ({\it R. pure}, Methodenlehre, II, 1). Cf.
{\it Postulat}$\,^2$.
% 175

\ib{Spiritisme} — Ensemble de pratiques qui visent à entrer en communication
avec les esprits$^4$ des morts.

\ib{Spiritualisme} — \si{Méta.} \fsb{S. norma.} {\bf 1.} (Opp. : {\it
matérialisme}$^1$). Doctrine selon laquelle l'{\it esprit}$^5$ ou
l'{\it âme}$^2$ constitue une réalité substantielle distincte de la
matière$^4$ et du corps$^3$ : « Si le matérialisme est insuffisant à
expliquer mon existence, le spiritualisme ne l’est pas moins » (Kant,
{\it R. pure}, Dial., I, 1, 4, 2$^\text{e}$ éd.) ; « Le spiritualisme est un
système d’après lequel il y a des êtres réels, véritables substances et
véritables causes, dont les modes d’existence sont absolument différents de
ceux des modes d'existence des corps et qui ne peuvent être perçus par les
sens » (Franck). — {\bf 2.} (Opp. : {\it matérialisme}$^2$, Syn. : {\it
philosophie de l'esprit}). Toute doctrine selon laquelle la vie de l'esprit
est irréductible à la matière : « Le spiritualisme moderne n’est plus, à ppt.
parler, un spiritualisme de l’Idée, c’est un spiritualisme de la conscience
» (Brunschvicg).

— \fsb{S. posit.} \si{Mor.} {\bf 3.} (Opp. : {\it matérialisme}$^3$).
Pratique de la vie spirituelle, tendance à placer les biens
spirituels$^2$ au-dessus des biens matériels.

$->$ Terme mis en honneur dans le lang. philosophique, aux sens 1 et 2, par
l’école de Cousin. Au {\footnotesize XVII}$^\text{e}$ siècle, il est pris au
sens 3, et plutôt {\it péj.} pour désigner les pratiques de « spiritualité
outrée » (Bossuet).

\ib{Spirituel} — \si{Méta.} {\bf 1.} (Opp. : {\it matériel}). Qui concerne
l'esprit$^5$, ou qui est de la nature de l'esprit : « Nos maladies
spirituelles » (Bossuet) ; « Ceux-là se trompent, qui croient que la
rébellion du corps n'est cause que des vices grossiers, et non de ceux
% 175
qu’on appelle spirituels, comme l’orgueil et l’envie » (Malebranche,
{\it Écl.}, VIII, rép. 11). — \si{Mor.} {\bf 2.} (Opp. : {\it charnel}). Qui
appartient aux fonctions supérieures de l’esprit$^6$ : « Ce qui est
proprement spirituel, c’est ce qui est intellectuel » (Lachelier, d’après
Bossuet) ; « Primauté du spirituel » (Maritain).

— (En parlant des personnes).  {\bf 3.} Qui s’adonne à la vie spirituelle : «
Il y a de faux spirituels » (Bossuet). — {\bf 4.} \si{Car.} Qui a de l'esprit
au sens 8: « Elle se croit intelligente et spirituelle » (Bourdaloue).

\ib{Spontané} — {\bf 1.} (Opp. : {\it réceptif}). Qui implique une initiative
de la part de l’agent : « Le caractère de la nature qui fait la vie, est la
prédominance de la spontanéité sur la réceptivité » (Ravaisson) ; « Il ne
nous paraît pas douteux que le dernier mot de son système (de Kant) soit dans
la spontanéité pure, {\it i. e.} dans le volontarisme, la finalité
» (Hamelin). — {\bf 2.} (Opp. : {\it provoqué}). Qui se produit de lui-même,
sans sollicitation étrangère : « Les mouvements spontanés des organes »; «
Des aveux spontanés ». — {\bf 3.} {\it Spéc.}, \si{Psycho.} (Opp. : {\it
réfléchi}). Qui n’implique aucun retour de la conscience sur elle-même : « La
conscience spontanée » (syn. {\it conscience$^1$ simple}) ; « Une association
d'idées spontanée ».

\ib{Statique (adj.)} — {\bf 1.} (Opp. : {\it dynamique$^1$, mouvant}). Qui a
les caractères d’un état, d’un repos ou d’un équilibre. Auj.,
souvent {\it péj.} : « Le concept n’est pas qqc. de statique et de mort
» (Hamelin). — {\bf 2.} \si{Psycho.} {\it Sens statique} (syn. :
{\it ampullaire, labyrinthique}) : celui qui a son organe dans l'oreille
interne et qui nous donne les sensations de rotation, de verticalité,
d'équilibre, etc.
% 176

\ib{Statique (nom)} — \si{Épist.} {\bf 3.} Partie de la Mécanique$^1$ qui
étudie l’équilibre des forces$^4$. — {\bf 4.} {\it Statique sociale} : cf.
{\it Dynamique}$^4$.

\ib{Statistique} — [De l'italien {\it statisia}, homme d'Etat] — \si{Épist.}
{\bf 1.} (Nom). D'abord science qualitative, puis quantitative de l'Etat.
Auj., {\it gén.}, toute étude « qui a pour objet de recueillir et de
coordonner des faits nombreux dans chaque espèce$^2$, de manière à obtenir
des rapports numériques, sensiblement indépendants des anomalies du hasard
» (Cournot) : « Tout comptage, même d’un grand nombre d'unités ou de cas,
n’est pas une statistique... La première condition est que nos expressions
statistiques$^2$ soient établies sur une base présentant une certaine
homogénéité » (Simiand). — {\bf 2.} ({\it Adj.}) Qui utilise les procédés de
la statistique$^1$ : « Les calculs statistiques ». {\it Déterminisme} ou {\it
loi statistique} ceux qui s'appliquent à un ensemble global de phénomènes
({\it p. e.} les mouvements des molécules d’un gaz) sans permettre la
prévision du détail ({\it p. e.} le mouvement d’une molécule déterminée) :
 cf. {\it Textes choisis}, II, p. 78.

\ib{Statutaire (Droit)} — \si{Jur.} Celui qui repose sur le {\it status},
{\it i. e.} sur la situation des individus ou des groupes dans la hiérarchie
sociale au point de vue civil ou politique (opp. {\it droit
contractuel}, fondé sur la volonté des intéressés).

\ib{Stéréotypes} — \si{Psycho.} \si{Soc.} Images préconçues des choses et des
êtres, que se fait l'homme moyen d’un certain milieu social (W. Lippmann).

\ib{Stimulus} — Voir {\it Excitant}*.

\ib{Stochastique} — [G. {\it stochos}, conjecture] — \si{Épist.} Se dit des
phénomènes
% 176
dont le détail relève du hasard$^2$ ({\it p. e.} rencontre de deux molécules
dans un gaz) et à propos desquels on ne peut énoncer que des probabilités
d'ensemble fondées sur des lois statistiques$^2$.

\ib{Stratification} — \si{Phys.} {\bf 1.} En Géologie, disposition des
terrains par couches superposées. — {\it Anal.} \si{Soc.} {\bf 2.} {\it
Stratification sociale} : disposition hiérarchique des éléments d’une société
en groupes (castes, classes$^3$, etc.) de niveaux différents.

\ib{Structuration} — Organisation en structures$^2$, {\it p. e.} \si{Soc.} :
« La vie en groupe est immédiatement et nécessairement structurée, à tel
point que la meilleure définition de la société est sans doute de dire que la
société est un phénomène de structuration et même peut-être le phénomène
essentiel de structuration » (Davy).

\ib{Structure} — {\bf 1.} \fsb{S. abstr.} Organisation des parties ou des
éléments qui forment un tout : « La structure d’un organe, d'une phrase,
d’une théorie » —  {\bf 2.} \fsb{S. concr.} Le tout lui-même en tant qu'unité
organisée : {\it spéc.} \si{Psycho.} forme$^4$ ou {\it Gestalt} ; \si{Soc.},
type sociologique, {\it pattern}* : « De tels rapports [entre individus ou
groupements] s'inscrivent dans une structure qui les commande » (Davy).
{\it Ext.} \si{Log.} : « Nous appellerons {\it structure} toute liaison
logique susceptible de jouer, alternativement ou simultanément, le rôle de
forme$^1$ et celui de contenu » (Piaget).

\ib{Subalternes (Propositions)} — \si{Log.} \si{form.} Propositions opposées*
différant seulement en quantité$^2$ : {\it p. e.} « Tous les A ne sont pas B
» et « Certains A sont B ». {\it Règle} : de la vérité de l’universelle on
conclut (a fortiori) à la vérité de la particulière ;
% 177
et de même, de la fausseté de la particulière à la fausseté de l’universelle.
Mais, de la fausseté de l’universelle ou de la vérité de la particulière, on
ne peut rien conclure.

\ib{Subconscient} — \si{Psycho.} {\bf 1.} Faiblement conscient : « Ces
phénomènes [demi-morbides] se rattachent à l’activité subconsciente, que nous
distinguons soigneusement de l'inconscient pur » (Ribot). — {\bf 2.} « Isolé
de la conscience totale de l’individu » et par suite « ignoré par le sujet
même qui l’éprouve et, en apparence, inconscient » (Janet) : {\it p. e.} les
« actes subconscients » des somnambules, hystériques, etc.

\ib{Subcontraires (Propositions)} — \si{Log.} \si{form.} Propositions
opposées*, toutes deux particulières, différant en qualité$^3$ : {\it p. e.}
« Certains A sont B » et « Certains A ne sont pas B ». {\it Règle} : de la
fausseté de l’une, on conclut (a fortiori) à la vérité de l’autre. Mais, de
la vérité de l’une, on ne peut rien conclure.

\ib{Subjectif} — \si{Crit.} et \si{Méta.} (Opp. {\it objectif}$\,^2$). {\bf 1.}
Qui se rapporte au sujet*, dans les différentes acceptions du sens 4 :
{\it a)} au sujet {\it transcendantal} : « L'espace et le temps, comme
conditions nécessaires de toute expérience, sont des conditions purement
subjectives de toute notre intuition » (Kant, {\it R. pure}, {\it Esth.}, §
8, I) ; — {\it b)} au sujet {\it empirique} et {\it psychologique} : « La
raison devient subjective par son rapport au moi volontaire et libre, siège
et type de toute subjectivité » (Cousin) ; « La subjectivité est un caractère
qui distingue les phénomènes psychiques » (Hamelin) ; « Le subjectif n’est
pas mesurable » (Lagneau). {\it D'où} : intérieur : « Le retour
% 177
à la subjectivité, tel que l'entend notre philosophie [contemporaine] est une
conversion du dehors au dedans, un appétit de vie intérieure » (Bréhier).

— \si{Épist.} {\bf 2.} (Opp. : {\it objectif}$\,^3$, et souvent {\it péj.})
Individuel ; qui dépend du point de vue personnel : « Une interprétation
toute subjective » ; « Ce qui est subjectif, c’est ce qui est isolé dans le
sujet pensant, dans le moi, et à quoi les semblables ne font pas écho
» (Alain). {\it D'où} : apparent, illusoire : « sensations subjectives »,
celles qui se produisent sans stimulus extérieur ({\it p. e.} couleurs
complémentaires, sensations consécutives). — {\bf 3.} (Opp. :
{\it objectif}$\,^4$). Fondé sur l'étude des phénomènes subjectifs$^1$,
intérieurs à la conscience. Psychologie subjective : celle qui ne fait appel
qu’à la « méthode subjective », {\it i. e.} à l’introspection*.

\ib{Subjectivisme} — \fsb{S. posit.} {\bf 1.} {\it Péj.} Propension à tout
juger d’un point de vue subjectif, au sens 2 : « Les adversaires de ce
mouvement [vers l’intériorité : cf. {\it Subjectif}$\,^1$, dernier texte]
l’accusent de {\it subjectivisme}, reproche qui veut dire que, restant dans
les limites de ce que nous connaissons de nous-même, nous croyons atteindre
des réalités distinctes, alors que nous entendons seulement l’écho informe de
réalités physiques et physiologiques sous-jacentes » (Bréhier). — \fsb{S. norma.} {\bf 2.}
Doctrine d’après laquelle notre connaissance ne peut rien avoir
d’objectif$^3$ : « Le Kantisme a souvent été accusé, à tort, de
subjectivisme. »

\ib{Sublimation} — \si{Ps. an.} Processus par lequel les instincts$^4$
inférieurs sont transformés en tendances supérieures : {\it p. e.} selon
Freud les tendances sexuelles en tendances esthétiques.
% 178

\ib{Sublime} — \si{Esth.} Le beau, dans un ordre qui nous dépasse : « Est
sublime ce qui, du fait même qu'on le conçoit, est l'indice d’une faculté de
l’âme qui surpasse toute mesure des sens » (Kant, {\it Jug.}, § 25).

\ib{Subliminal} — [L. {\it sub}, sous, et {\it limen}, seuil]. — \si{Psycho.}
Inférieur au seuil de la conscience, subconscient : « Le moi subliminal. »

\ib{Subordonné} — Voir {\it Caractère}$^1$.

\ib{Subsomption} — \si{Épist.} Opération qui consiste à {\it subsumer},
i. e. à faire rentrer un individu$^3$ ou une espèce$^2$ dans un
genre$^1$, un fait sous une loi$^5$.

\ib{Substance} — [L. {\it substantia $=$ quod stat sub}, ce qui est dessous].
— \si{Méta.} {\bf 1.} (Opp. : {\it attribut}$\,^2$). Ce qui est {\it en soi}$\,^2$ ;
réalité permanente qui sert de support aux attributs$^2$ changeants : «
Parce qu’entre les choses créées qqs.-unes sont de telle nature qu’elles ne
peuvent exister sans qqs. autres, nous les distinguons d’avec celles qui
n’ont besoin que du concours ordinaire de Dieu, en nommant celles-ci des
substances, et celles-là des qualités ou attributs de ces substances
» (Descartes, {\it Princ.}, 1, 51) ; « Ne pouvant concevoir comment des
qualités pourraient subsister seules, nous supposons qu’elles existent dans
qq. objet commun qui en est le support, et c'est ce support que nous
désignons par le nom de substance » (Locke). {\it Principe de substance} :
celui d'après lequel toute qualité requiert le support d'une substance$^1$.
{\it Chez Leibniz} : substance$^1$ singulière et active : « Lorsque
plusieurs prédicats* s’attribuent à un même sujet et que ce sujet ne
s’attribue à aucun autre, on l’appelle substance individuelle » ({\it Disc.
méta.}, 8) ; « La notion de {\it vis} ou de {\it virtus}, que les Allemands
% 178
appellent {\it Kraft}, les Français {\it la force}, apporte beaucoup de
lumière à la vraie notion de substance » (id., 1694). — {\bf 2.} Ce qui est
{\it par soi}\,* : « Lorsque nous concevons la {\it substance}, nous concevons
une chose qui existe en telle façon qu’elle n’a besoin que de soi-même pour
exister : à ppt. parler, il n’y a que Dieu qui soit tel » (Descartes,
{\it ibid.}; cf. {\it Univoque}) ; « Par {\it substance}, j'entends ce qui
est en soi et est conçu par soi, {\it i. e.} ce dont le concept n'a pas
besoin du concept d’une autre chose dont il doive être formé » (Spinoza,
{\it Éth.}, 1, déf. 3).

— \si{Vulg.} {\bf 3.} L'essentiel : « La substance d’un discours » ; « En
substance ». — {\bf 4.} Un corps$^2$ : « La substance que les chimistes
appellent carbone » (Cournot).

\ib{Substantialisme} — \si{Hist.} ({\it Opp.} : {\it phénoménisme}*).
\fsb{S. norma.} Doctrine qui admet l’existence d’une ou de plusieurs
substances : « Le substantialisme cartésien ». Qqfs. {\it péj.} : « Il
[Kant] fait à peu près [en alléguant une pensée intemporelle] ce que font
des substantialistes naïfs » (Hamelin).

\ib{Substantialité} — \si{Méta.} Caractère substantiel$^1$ : « La
substantialité du changement » (Bergson, {\it P. M.}, V). {\it Chez Kant} :
« paralogisme de la substantialité », celui qui consiste à faire de l'âme
une substance$^2$ pensante ({\it R. pure}, Dial., II, 1, 1).

\ib{Substantiel} — \si{Méta.} {\bf 1.} Qui est une substance ou a les
caractères d'une substance : « Donner à l'étendue une existence
substantielle ». {\it Forme substantielle} : v. {\it Forme}$^1$. —
\si{Vulg.}  {\bf 2.} Qui contient beaucoup de substance au sens 3 : « Un
exposé substantiel. »

\ib{Substrat, Substratum} — \si{Méta.} {\bf 1.} La
substance$^2$, en tant qu’elle sert de
% 179
support aux attributs : « Distinguer l'âme de la matière, relativement à son
{\it substrat} » (Kant, {\it R. pure}, Dial., Il, 1, 2). — {\it Ext.}
{\bf 2.} Ensemble de phénomènes$^1$ qui conditionnent d’autres phénomènes :
« [Nos habitudes] constituent, réunies, le substrat de notre activité libre
» (Bergson, {\it D. I.}, III). {\it Chez Durkheim} : « substrat social »,
base morphologique* de la société consistant 1° dans la répartition des
groupes sur le sol (géographie humaine) ; 2° dans les variations de volume*
et de densité* de ces groupes : « La sociologie ne peut se désintéresser de
ce qui constitue le {\it substrat} de la vie collective. »

\ib{Subsumer} — Voir {\it Subsomption}*.

\ib{Subtil} — Voir {\it Matière}$^4$. Qqfs., nom : « Cette corde peut donc
ébranler l’air qui l’environne, et même le subtil qui en pénètre les pores ?
» (Malebranche, {\it Entr.}, III, 10).

\ib{Suggestibilité} — \si{Psycho.} État de l’individu qui subit facilement
des sug\-gestions$^2$ ({\it p. e.} l’hystérique).

\ib{Suggestion} — \si{Ps. path.} {\bf 1.} {\it Lato.} « Il y a suggestion
quand un sujet obtient par son autorité morale la confiance aveugle et
docile ou l'obéissance passive d’un autre » (Dumas). — {\bf 2.} {\it Str.} «
Il y a encore suggestion lorsqu’un sujet agit par sa volonté sur
l’automatisme d’un autre sujet au point de lui faire réaliser certaines
idées ou certains actes sans qu'il en ait la volonté ou même la conscience
» (id.).

— \si{Psycho.} {\bf 3.} \fsb{S. concr.} Idée que l’on a {\it suggérée},
{\it i. e.} inspirée par insinuation ; conseil : « J’accepterai volontiers
vos suggestions ». Autref. {\it péj.} « Les suggestions du démon » (Bossuet).
% 179

\ib{Sujet} — \fsb{S. objec.} \si{Vulg.} {\bf 1.} Ce sur quoi porte la
réflexion, le sentiment, etc. : « Le sujet d’un discours »; « Un sujet de
mécontentement ». — \si{Log. form.}  {\bf 2.} (Opp. : {\it attribut}$\,^1$ ou
{\it prédicat}\,*). Dans une proposition le terme dont on affirme ou nie
qqc. : « La logique moderne a été amenée à considérer le sujet comme une
variable dont le prédicat est une fonction » (Couturat). — \si{Méta.}
{\bf 3.} (Opp. : {\it attribut}$\,^2$). L'être réel qui sert de substrat* aux
attributs : « Toute qualité a son sujet d’inhérence » (Cousin). {\it Ext.}
\si{Jur.} « Sujet de droit », la personne qui en est investie.

— \fsb{S. subje.} \si{Crit.} et \si{Épist.} {\bf 4.} (Opp. :
{\it objet}$\,^5$). L'esprit qui connaît, par {\it opp.} à la chose connue. Ce
« sujet connaissant » peut être entendu : {\it a)} soit comme le sujet {\it
épistémologique} ou {\it transcendantal} (sujet pur), qui {\it p. e.} chez
Kant n’est autre que l’ensemble des lois {\it universelles} a priori de la
pensée : « Le sujet des catégories ne peut recevoir, du fait seul qu'il les
pense, un concept de lui-même comme objet » (Kant, {\it R. pure}, Dial, II,
1, 4, 2° éd.) ; « C’est un sujet tout réduit à sa fonction d’objectivation
» (Bréhier) ; « Fonder la nécessité des lois physiques sur leur origine
subjective qui permettrait de les déterminer a priori, cela ne se justifie
que si l'on vise le sujet pensant en général, principe d'universalité, et
non le sujet individuel » (Blanché) ; — {\it b)} soit comme le sujet {\it
empirique}, {\it i. e.} le moi individuel : « Une connaissance {\it
subjective} relative à la nature du sujet connaissant n’est pas pour cela
moins valable... C'est au contraire le caractère partial, incarné,
conditionné d’une connaissance qui fonde sa valeur » (E. Grimal, {\it R.
ph.}, 1945, p. 243). $->$ Se
% 180
tenir en garde contre la confusion des sens {\it a} et {\it b}, nettement
accusée dans le dernier texte : « L’irréalité du sujet pur n’excuse point
cette confusion : le sujet pur est pour l’épistémologue une idéalisation
aussi légitime que le point inétendu pour le géomètre » (Blanché). {\it Cf.}
Bachelard qui {\it dist.} « le sujet individuel » du « sujet quelconque » :
« Ce sujet quelconque ne saurait être le sujet empirique livré à l’empirisme
de la connaissance... C'est le sujet rationnel... le sujet de la cité
scientifique ». — \si{Méta.} {\bf 5.} L’existant individuel : « Les sujets
personnels incluent aussi les animaux » (Heidegger).

— \si{Psycho.} {\bf 6.} L'individu soumis
à une observation ou à une expérience : « Interroger un sujet. »

— \si{Pol.} {\bf 7.} L’individu, soumis à l'autorité souveraine de l'État :
« Le peuple est à l’égard des nobles ce que les sujets sont à l'égard du
monarque » (Montesquieu, {\it Lois}, III, 4).

\ib{Superstructure} — \si{Soc.} Voir {\it Base}$\,^2$ :
« L'ensemble des rapports de production constitue la base réelle sur
quoi s'élève la {\it superstructure} juridique de la politique » (K. Marx).

\ib{Suppôt} — [L. {\it suppositum}, placé sous] — Se disait autref. de la
substance comme sujet$^3$ de ses attributs, et {\it not.} de la personne
humaine : « Comment connaîtrions-nous la matière puisque notre suppôt qui
agit en cette connaissance est en partie spirituel ? » (Pascal, 72).

\ib{Surdétermination} — \si{Psycho.} Une des formes de syncrétisme* de la
pensée de l'enfant qui, au lieu de définir rigoureusement ses concepts, les
caractérise par plusieurs attributs à la fois (Piaget).

\ib{Surdité psychique} — \si{Ps. path.} Agnosie* auditive. — {\it Surdité
verbale}. Cf. {\it Aphasie}*.

\ib{Sur-moi} — \si{Ps. an.} (Syn. : {\it moi idéal}). {\it Chez Freud} :
secteur de la personnalité, né du complexe d'Œdipe et qui est « la source de
toutes les réalisations culturelles supérieures de l’homme » (art,
littérature, droit, morale, religion).

\ib{Surmonter} — Voir {\it Aufheben}*.

\ib{Surnaturel} — \si{Théol.} Ce qui dépasse la nature des êtres créés. Les
théologiens dist. : {\bf 1.} le surnaturel {\it quant au mode} (syn. : {\it
préternaturel}), fait naturel en soi, mais qui dépasse la puissance de toute
nature créée par la façon dont il se produit : {\it p. e.} guérison
instantanée d'un malade; — {\bf 2.} le surnaturel {\it quant à la chose} :
ce qui dépasse toute nature créée ou créable par son essence même :
{\it p. e.} la grâce$^2$, la vision béatifique : « Nos actions ne tirent
leur dignité surnaturelle que par J.-C. » (Malebranche, {\it Entr.}, XIV,
7). {\it Vérités surnaturelles} celles qui ne sont connues que par la
révélation.

— Ext. \si{Vulg.} {\bf 3.} Extraordinaire : « C'est un aveuglement
surnaturel de vivre sans chercher ce qu'on est » (Pascal, 495).

\ib{Surréalisme} — \si{Esth.} Forme de littérature et d'art qui prétend
atteindre par la « déréalisation* du monde quotidien » un univers qui aurait
« plus de réalité que l’univers logique et objectif » (Alquié).

\ib{Survivance} — {\bf 1.} Action de survivre : « La survivance de l'âme après la
mort » — {\bf 2.} Organe, usage, façon de penser qui survit aux causes qui
lui ont donné naissance : « Cette mentalité ne subsiste plus auj. qu’à
l'état de survivance » (Durkheim).

%181
\ib{Suspension du jugement} — \si{Hist.} Action par laquelle les Sceptiques
({\it spéc.} les pyrrhoniens$^2$) s’abstenaient de juger. Cf. {\it Epoché}\,*.

\ib{Syllogisme} — \si{Log.} \si{form.} Type de déduction formelle$^1$ tel
que, deux propositions appelées {\it prémisses}* étant posées, on en tire
une troisième appelée {\it conclusion}$^3$ qui y est logiquement impliquée.
Cf. {\it Précis}, Ph. II, p. 33.

\ib{Symbiose} — \si{Biol.} Phénomène d’association entre deux êtres vivants
qui vivent de la même vie organique : {\it p. e.} l’algue et le champignon
dans le lichen.

\ib{Symbole} — \si{Épist.} {\bf 1.} {\it Latiss.} Substitut du réel : « La
science positive travaille avant tout sur des symboles... La métaphysique
est la science qui prétend se passer de symboles » (Bergson, {\it P. M.},
VI). — {\bf 2.} {\it Lato.} Signe$^4$, le plus souvent artificiel et
conventionnel, inclus dans un système organisé (cf. {\it Algorithme}*) et
destiné à servir à des combinaisons opératoires : « Les symboles algébriques
». — {\bf 3.} {\it Str.} Représentation concrète liée par une correspondance
analogique naturelle avec l’abstraction ou la réalité mentale ou morale
qu’elle représente : {\it p. e.} le serpent réchauffé, « symbole des ingrats
» (La Fontaine). $->$ Cf. {\it Précis}, Ph. I, p. 384, et dist. {\it
Allégorie} et {\it Emblème}*.

— \si{Théol.} {\bf 4.} Formulaire des articles de la foi : « Le symbole de
Nicée ».

\ib{Symbolique} — Qui use de symboles* (en tous les sens du terme) : « La
philosophie de Pythagore était énigmatique et symbolique » (Diderot). {\it
Logique symbolique} : la Logistique$^2$. {\it Pensée symbolique} : celle qui
procède par symboles$^2$ ({\it p. e.} celle d’un calculateur
% 181
qui effectue ses opérations sans songer expressément à ce qu’elles
représentent) ou bien par symboles$^3$ ({\it p. e.} celle du poète qui pense
par images ou analogies).

\ib{Sympathie} — {\bf A)} (Sympathie {\it avec}...). \si{Phys.} {\bf 1.}
{\it Autref.}, affinité entre certains corps : « J'ai peur que la sympathie
[entre les cordes vibrantes] ou qq. autre chimère ne vous empêche de suivre
le principe des idées claires » (Malebranche, {\it Entr.}, III, 16). —
\si{Phol.} {\bf 2.} Sorte de contagion {\it physiologique} qui fait qu'un
être reproduit les attitudes ou mouvements d’un autre : {\it p. e.} rire,
bâillement. — \si{Psycho.} {\bf 3.} Sorte de contamination {\it mentale} qui
« crée chez deux ou plusieurs individus des dispositions affectives
analogues » (Ribot) : {\it p. e.} contagion de la peur dans la panique. —
{\bf 4.} Participation plus ou moins {\it volontaire} à la joie ou à la
douleur d'autrui ({\it Mitgefühl} de Scheler : cf. {\it Précis}, Ph. I, p.
193).

— {\bf B)} (Avoir de la sympathie {\it pour}...). {\bf 5.} Attachement fondé sur
une certaine communauté d'idées ou de caractère.

\ib{Symptomatique (Acte)} — \si{Ps. an.} Acte machinal, sans finalité
consciente ({\it p. e.} à table, jouer avec son couteau, pétrir de la mie de
pain). Il est ainsi appelé parce que, selon Freud, il manifeste des
tendances inconscientes.

\ib{Syncrétisme} — {\bf 1.} \fsb{S. norma.} {\it Péj.} « Réunion factice
d’idées ou de thèses d’origine disparate » (Lalande). — {\bf 2.}
\fsb{S. posit.} \si{Psycho.} État d’esprit dans lequel les différents
éléments d’un ensemble complexe ne sont pas encore distingués; vue générale
et confuse : « L’esprit humain dans sa marche traverse trois états : le
syncrétisme, l'analyse, la synthèse » (Renan) ;
% 182
« Le syncrétisme enfantin » (Piaget) ; « Les perceptions syncrétiques
» (Claparède) : cf. {\it Précis}, Ph. I, p. 84 et 120.

\ib{Syndérèse} — \si{Hist.} {\it Chez les Scolastiques} : conscience$^3$
morale : « Saint Basile dit que la conscience ou syndérèse est la loi de
notre intellect en tant que contenant les préceptes de la loi naturelle
» (St. Thomas, {\it S. th.}, I$^\text{a}$ II$^\text{ae}$, 94, 1).

\ib{Synergie} — [G. {\it sun}, avec, et {\it ergon}, travail] — {\bf 1.}
Association dynamique de plusieurs fonctions. — {\bf 2.} Accord des
tendances.

\ib{Synesthésie} — \si{Psycho.} Fusion intime
de deux sensations de qualité différente : {\it p. e.} audition* colorée.

\ib{Synnomique} — [G. {\it sun}, avec, et {\it nomos}, loi] — \si{Épist.} Se
dit des jugements « conçus par ceux qui les énoncent comme valables en droit
pour tous les autres esprits avec lesquels ils peuvent entrer en société
» (Lalande) : « Les normes de conduite synnomiques. »

\ib{Syntaxe} — \si{Ling.} Partie de la grammaire qui étudie la construction
des propositions* et les rapports logiques des phrases. {\it Syntaxe
logique} : nom donné par l’école de Vienne ({\it Précis}, Ph. II, p. 24) à la
Logique$^2$ conçue comme une théorie « des formes propositionnelles et autres
créations grammaticales » du langage scientifique.

\ib{Synthèse} — \fsb{S. abstr.} \si{Épist.} {\bf 1.} (Opp. :
{\it analyse}$\,^3$). Opération qui consiste à recomposer un tout à l’aide de
ses éléments : « Les unir entre eux [les corps simples] et reformer par leur
combinaison ces mêmes principes naturels qui constituent tous les êtres
matériels : tel est l’objet de la synthèse chimique » (M. Berthelot).
{\it D'où}, en gén. : marche du simple au complexe : « La synthèse...
démontre clairement ce qui est contenu en ses conclusions et se sert d’une
longue suite de définitions, de demandes, d’axiomes, de théorèmes et de
problèmes, afin que, si on lui en nie qqs. conséquences, elle fasse voir
comment elles sont contenues dans les antécédents » (Descartes,
% 182
2$^\text{es}$ {\it Rép.}) ; « On arrive souvent à de belles vérités par la
synthèse, en allant du simple au composé. » (Leibniz, {\it N. E.}, IV, 2, §
7). {\it Spéc.}, « synthèse historique » : partie du travail de l’historien
qui consiste à combiner les résultats de l’analyse des documents pour
reconstituer le passé : « Pour un jour de synthèse, il faut des années
d'analyse » (Fustel de Coulanges). — \si{Méta.} {\bf 2.} Dans la
dialectique$^6$ hegelienne et hamelinienne : dépassement de la thèse et de
l’antithèse en un terme qui les combine d’un point de vue supérieur : « La
synthèse qui concilie les opposés ne les nie pas » (Hamelin).

— \fsb{S. concr.} {\bf 3.} Ensemble complexe où il est difficile de
distinguer des éléments, ceux-ci se pénétrant et se fondant les uns dans les
autres. {\it Spéc.}, \si{Psycho.} : « La synthèse mentale » : « Le
psychologue se trouve en présence de synthèses dont les éléments n'existent
pas en dehors de ces synthèses mêmes. » (Dwelshauvers). — {\bf 4.} {\it Chez
Janet} : « activité de synthèse », activité mentale ({\it opp.} à l’activité
{\it conservatrice}*) qui « réunit des phénomènes donnés plus ou moins
nombreux en un phénomène nouveau différent des éléments » ({\it p. e.} dans
l'attention) et qui est la source de notre adaptation au réel (voir {\it
Réel}$\,^3$) et des facultés créatrices de l'esprit.

\ib{Synthétique} — {\bf 1.} Qui se rapporte à la synthèse (en tous les sens
du terme) : « L'avantage de la méthode synthétique est double : elle se
prête à une exposition plus claire,... elle est plus probante » (Goblot).
{\it Déduction synthétique} : voir le texte de Descartes à
{\it Synthèse}$^1$. {\it Idéalisme synthétique} : celui de Hamelin : voir
{\it Synthèse}$^2$, et cf. {\it Précis}, Ph. II, p. 461.
% 183
— \si{Crit.} {\bf 2.} Proposition {\it synthétique} (opp. :
{\it analytique}$^4$) : celle où l’attribut$^1$ ajoute à la compréhension$^2$
du sujet : {\it p. e.} « Ce corps est un métal. » {\it Chez Kant} : «
jugements synthétiques a priori », ceux qui ne dérivent pas de
l'expérience : {\it p. e.} « 7 + 5 = 12 », « la quantité de matière est
invariable », etc.

\ib{Syntone} — \si{Car.} (Syn. : {\it extraverti}). Qui est en accord avec son
entourage.

\ib{Systématique} — ({\it Adj.}). {\bf 1.} {\it Laud.} Méthodiquement (et
souvent consciemment) organisé : « La constitution systématique du Grand
Être$^2$ » (Comte) ; « La composition systématique de la religion » (id.) ;
« Le savoir est essentiellement systématique » (Hamelin), ou (en parlant des
personnes) qui organise méthodiquement sa pensée : « Le véritable esprit
systématique, qu'il faut se garder de prendre pour l'esprit de système
» (D'Alembert). — {\bf 2.} {\it Péj.} Dominé par des idées préconçues : « Le
savant systématique » {\it opp.} au « savant expérimentateur » (Cl. Bernard).

— (Nom fém.) {\bf 3.} \si{Épist.} En \si{Biol.}, classification et
description des êtres vivants.

\ib{Système} — \fsb{S. concr.} {\bf 1.} Ensemble organisé dont les parties
ou éléments sont interdépendants ou obéissent à une loi unique : « Le
système solaire »; « Le système nerveux »; « Tout fait psychique est un
système » (Paulhan). — \fsb{S. abstr.} {\bf 2.} Combinaison d'idées ou de
procédés coordonnés et ramenés à un petit nombre de principes : « Le système
héliocentrique »; « Les systèmes philosophiques »; « Le système métrique ».
Qqfs. {\it péj.}, spéc. dans l'expression « esprit de système » : voir {\it
Systématique}$^2$, et cf. {\it Textes choisis}, II, p. 38.

	\end{itemize}
