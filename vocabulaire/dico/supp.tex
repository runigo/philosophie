
	\begin{itemize}[leftmargin=1cm, label=\ding{32}, itemsep=1pt]

\ib{Athymhormie} — [G. a priv.; thymos,
cœur; hormè, élan] \si{Ps. path.}
Diminution générale de la vitalité
qui s’observe dans l'hébéphrénie*
et la schizophrénie* et qui se manifeste par l'indifférence affective,
le dégoût de tout effort et le désintérêt à l'égard d'autrui et du
monde extérieur.

\ib{Communication de masses} — [{\it Trad.}
anglais : mass communications]
\si{Soc.} Ensemble des procédés
(presse, livres, tracts, radio, télévision, Cinéma, etc.) qu'utilisent
l'information et a propagande
pour agir sur l'opinion$^2$ publique.

\ib{Culpabilité (Sentiment de)} — \si{Ps. path.} « Hyperesthésie de la conscience
morale », (Baruk) qui se rencontre
souvent dans la mélancolie*, la
paranoïa* et autres psyvchopathies*
et qui torture le malade en le poussant à s’accuser de fautes véritables
ou imaginaires.

\ib{Cyclique (Théorie)} — À. \si{Hist.}
Théorie selon laquelle le genre
humain repasse sans cesse par les
mêmes âges ou états de civilisation
({\it p. e.} Vico, Spengler). {\it Cf.} Retour*.

\ib{Demande} — \si{Épist.} Ancien nom des
postulats* : « Une longue suite de
définitions, de demandes, d’axiomes, de théorèmes... » (Descartes,
2$^\text{es}$ {\it Rép.})

\ib{Démythologiser} — Dépouiller [la
religion] de ses éléments mythiques
pour atteindre, croit-on, à travers
eux une vérité plus profonde (Bultmann).

\ib{Diachronique} — \si{Hist. Ling.} ({\it Opp.} :
synchronique). Considéré du point
de vue de son développement
dans le temps.

\ib{Disponible} — ({\it Opp.} : engagé*).
Qualifie auj. l'attitude de eelui
qui se refuse à être « entravé par
lui-même » (A. Gide) et demeure
« disposé à l'accueil » pour les
orientations nouvelles.

\ib{Dominateur (Argument)} — [En grec :
kurieuôn logos] — \si{Hist.} Argument
employé par les Mégariques et qui
se résume ainsi : l’impossible ne
peut procéder du possible ; or le
passé, étant révolu, ne peut être
autre qu’il n’est ; il n’a donc jamais
été possible autrement qu'il n'a
été, ce qui exclut toute indétermination (Schuhl).

\ib{Émotivisme} — A. \si{Méta.} Nom donné
par les philosophes de langue anglaise aux doctrines selon lesquelles la vraie réalité est de
nature affective, et non d'ordre
intellectuel ou rationnel.

\ib{Emprise} — {\bf 1.} {\it Autref.}, entreprise,
action chevaleresque. — {\bf 2.} Se dit
auj., par contresens, pour action$^1$
ou influence (voir Message*).

\ib{Engagé} — ({\it Opp.} : disponible*).
{\it Auj.}, « pensée engagée » : celle qui
reconnaît ses attaches à une situation$^2$, prend parti sur les problèmes
qui en résultent et reste fidèle
à ce parti.

\ib{Environnement} — \si{Soc.} Nom donné
par les sociologues américains à la
% 199
fois au milieu naturel (geographical
environment) et au milieu social
(social ou sociological environment)
dans lequel un groupe ou un individu se trouve plongé et avec lequel
il entretient des rapports d'action
réciproque.

\ib{Éon} — \si{Hist.} Voir Plérôme*.

\ib{Expiation} — \si{Mor.} {\bf 1.} Au sens religieux : satisfaction offerte à Dieu
pour la réparation d’un péché*.

— {\bf 2.} {\it Lato.} Toute souffrance destinée
à racheter une faute* et à purifier
ou guérir l'âme.

\ib{Générosité} — \si{Mor.} {\bf 1.} Chez Descartes : sorte de grandeur d’âme
qui est, selon lui, « la clef de toutes
les autres vertus » et qui fait que
l’homme « s’estime au plus haut
point » en tant qu'il possède la
« libre disposition de ses volontés »
et qu'il « sent en soi-même une
ferme et constante résolution d'en
bien user » (Passions, art. 153). ——
 {\bf 2.} \si{Vulg.} Disposition à donner avec
libéralité.

\ib{Information} — \si{Techn.} Ensemble
des données, en principe imprévisibles, que reçoit du milieu extérieur, soit une machine électronique, soit un être vivant (et {\it spéc.}
l’homme) par ses sens. Quantité
d’information : inverse du logarithme de la probabilité p du
signal reçu (i = - log p).

\ib{Message} — Terme «souvent usité
auj. pour désigner le contenu significatif, soit de la pensée d’un auteur : « Le message de Barrès »,
soit d’un fait ou d’un document :
« Leur message [des documents
historiques] échappe à l’emprise*$^2$
des règles fondées sur l'observation
de certaines constantes » (Marrou).
% 199

\ib{Métrologie} — \si{Techn.} Science et technique de la mesure*.

\ib{Monolithique} — \si{Soc.} \si{Pol.} Se dit de
la structure des États ou des partis totalitaires (v. Totalitarisme*)
parce que ces États ou ces partis
forment « un seul bloc » et excluent les groupes intermédiaires.

\ib{Nocturne} — Qqfs employé auj.
comme synonyme d’inconscient*
et d’irrationnel* : « Le côté nocturne de l’âme humaine ».

\ib{Opérationalisme} — A. \si{Épist.} Forme
renouvelée du pragmatisme* selon
laquelle les concepts doivent se
définir en termes d'opérations
physiques. {\it Cf.} Opérationnel*.

\ib{Orgueil} — \si{Mor.} {\bf 1.} Sentiment exagéré de sa propre valeur qui pousse
l'individu à se préférer aux autres
et à les dédaigner : « L’orgueil a
deux effets, dont l’un est de vouloir en tout exceller au-dessus des
autres ; l’autre, de s’attribuer à
soi-même sa propre excellence »
(Bossuet), $->$ {\it Dist.} vanité*. —
 {\bf 2.} Qqfs {\it laud.} Sentiment légitime
de la dignité personnelle : « (Un
guerrier) peut mettre l’orgueil
même à pardonner l'offense »
(Voltaire).

\ib{Phoronomie} — \si{Épist.} Science des
lois de l'équilibre et du mouvement des corps (à l’exclusion de
l’idée de force$^4$) : « La mécanique
cartésienne est une phoronomie ».

\ib{Plérôme} — {\bf 1.} \si{Hist.} Chez les Gnostiques* : être suprême d'où émanent les éons, i.e. les intelligences
éternelles qui président à la formation de l’univers et reviendront,
à la fin des temps, s’y confondre.
— {\bf 2.} {\it Lato.} Ensemble des êtres.
% 200

\ib{Précarité} — \si{Mor.} Caractère attribué
par Eug. Dupréel aux valeurs
morales et qui fait : 4° qu'il n’y a
de valeur que pour un sujet ; 2°
que ces valeurs sont sans cesse
menacées par des actes qui les
nient.

\ib{Primat, Primauté} — A. \si{Mor.} \si{Esth.}
Caractère de ce qui est premier$^8$,
fondamental dans l’ordre de la
valeur : « Le primat de la raison
pratique » (Kant), « Primauté du
spirituel » (Maritain).

\ib{Quantophrénie} — \si{Épist.} Terme inventé par le sociologue américain
Sorokin pour désigner ironiquement la tendance excessive à introduire la quantité et la mesure dans
les sciences de l'esprit (psychométrie, sociométrie, statistiques,
etc.).

\ib{Retour éternel} — A. \si{Méta.} Théorie
de certains philosophes anciens
(Héraclite, Pythagore, Stoïciens)
selon laquelle l'univers repasse
toujours, au terme de plusieurs
milliers d'années, par les mêmes
phases. La théorie a été reprise
par Nietzsche.

\ib{Rôle social} — A. \si{Soc.} Ensemble des
comportements présentant une
certaine unité qui caractérisent
dans la société un individu qui y
occupe une situation$^2$ particulière
({\it p. e.} père, mari, médecin, professeur, chef de groupe) ou qui
cherche à incarner une valeur
particulière ({\it p. e.} patriote, honnête
homme).

\ib{Scalaire} — \si{Math.} Se dit des grandeurs : {\bf 1.} non dirigées, {\it p. e.} le travail$^1$ ({\it opp.} : vectoriel, {\it p. e.} la force$^4$) ;
— {\bf 2.} variant de façon discontinue.
$->$ Impropre au sens {\bf 2.}

\ib{Servage} — \si{Soc.} Condition du travailleur manuel qui, tout en possédant la personnalité juridique
(dist. esclavage*), reste attaché à
la terre qu'il cultive et « corvéable
à merci », i.e. à la disposition
constante du propriétaire qui l’emploie.

\ib{Sociogramme} — \si{Soc.} Graphique qui
représente, en sociométrie*, les
attractions et répulsions entre
membres d’un même groupe. {\it Cf.}
Précis, Ph. I, p. 529).

\ib{Syntaxe} — \si{Ling.} Partie de la grammaire qui étudie la construction
des propositions* et les rapports
logiques des phrases. Syntaxe logique : nom donné par l’école de
Vienne (Précis, Ph. II, p. 24) à la
Logique$^2$ conçue comme une théorie « des formes propositionnelles
et autres créations grammaticales »
du langage scientifique.

\ib{Tensions} — Terme souvent usité
pour désigner les oppositions internes qui se manifestent ou existent à l’état latent dans une réalité
humaine : « Les tensions immanentes à toute réalité sociale. »

\ib{Tout (nom)} — La totalité$^2$ considérée
comme ensemble organique$^4$ et
original : « Un tout n’est pas identique à la somme de ses parties,
il est qqc. d’autre et dont les propriétés diffèrent de celles que présentent les parties dont il est composé » (Durkheim).

\ib{Transsexualisme} — \si{Ps. path.} Sentiment obsédant qu’éprouve un sujet
d’appartenir au sexe autre que le
sien, avec le désir d'en changer
pour vivre conformément à l’image
qu'il se fait de lui-même.
% 201

\ib{Tychisme} — [G. tyché, hazard] —
À \si{Épist.} Doctrine qui affirme l’existence, dansle monde, d’un hasard$^1$
radical.

\ib{Vanité} — {\bf 1.} \si{Vulg.} Caractère de ce
qui est vain, i.e. sans importance,
sans valeur : « La vanité du
monde » (Pascal, 161). — {\bf 2.} \si{Mor.}
Sentiment exagéré de la valeur
personnelle, qui diffère de l’orgueil*
en ce qu'il s'attache surtout aux
petites choses et en ce qu'il recherche l'approbation d'autrui : « Curiosité n’est que vanité : on ne veut
savoir que pour en parler » (Pascal,
152).

\ib{Zététique} — [G. zêtein, chercher] — \si{Hist.}
1. Qualificatif autref. appliqué aux
Sceptiques, {\it spéc.} aux disciples de
Pyrrhon. — \si{Épist.} {\bf 2.} Qui concerne
ou constitue une recherche. Analyse zététique : celle qui consiste à
supposer le problème résolu pour
trouver la solution.

	\end{itemize}
