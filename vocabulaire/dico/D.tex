
	\begin{itemize}[leftmargin=1cm, label=\ding{32}, itemsep=1pt]

\ib{Daltonisme} — [De Dalton, qui décrivit
le premier ce phénomène en 1794]
— \si{Ps. phol.} Dyschromatopsie* consistant dans l'incapacité de distinguer certaines couleurs complémentaires ({\it not.} le rouge et le vert).

\ib{Darwinisme} — \si{Biol.} À Doctrine de
Darwin et de ses disciples : c'est
une forme du transformisme*, caractérisée par la théorie de la sélection$^2$
naturelle et de la concurrence$^2$
vitale.

\ib{Dasein} — [mot allemand = être-là] —
\si{Méta.} {\bf 1.} Chez Kant : existence en
tant qu’elle s’oppose simplement au
non-être (cf. Catégories*), — {\bf 2.} Chez
Heidegger : l'être de l'existant
humain en tant qu’existence singulière concrète : « Le propre du Dasein,
c’est de ne jamais former un tout
achevé. »

% 51
\ib{Décision} — \si{Psycho.} (Syn. : choix$^1$,
détermination$^3$, résolution$^2$). Phase
terminale qui, selon la description
classique de l’acte volontaire, succède à la délibération*. {\it Cf.} Volition*.

\ib{Déclinaison} — [{\it Trad.} {\it L.} clinamen*] —
\si{Hist.} Chez Épicure et Lucrèce : mouvement par lequel les atomes$^1$, en
tombant dans le vide, s’écartent
légèrement de la verticale (voir
Textes choisis, II, p. 302).

\ib{Déduction} — \si{Log.} Raisonnement
par lequel on passe du ou des principes$^1$ à la conséquence*, ou : dans
lequel, une ou plusieurs propositions (principes) étant posées, on en
tire une autre proposition (conséquence) qui en résulte nécessairement$^{1b}$, Déduction formelle : celle où 3
la conclusion, implicitement con- î
tenue dans les principes, n’ajoute
rien à ceux-ci. Déd. constructive
(syn. : démonstration$^1$) : celle où la
conclusion constitue un gain pour la
pensée théorique : « Déduire, c’est
construire. On démontre qu’une
chose est conséquence* d’une autre.
Pour cela, on construit la conséquence avec l'hypothèse$^2$ » (Goblot).

Déduction. . .
 Formelle. . .
  Immédiate. . .
   Opposition”. 
   Conversion*.
  Médiate (syllogisme*).
 Constructive (démonstration).

\ib{Défaut} — \si{Méta.} Chez Descartes :
manque d’être : « Je pouvais croire
que, si elles [mes pensées] étaient
vraies, c’étaient des dépendances
de ma nature en tant qu'elle avait
quelque perfection$^2$, et, si elles ne
l’étaient pas, que je les tenais du
néant, c'est-à-dire qu’elles étaient
en moi pour ce que j'avais du défaut»
(Méth., IV).

\ib{Défini} — \si{Log.} {\bf 1.} Qui a fait l'objet
d’une définition. — {\bf 2.} (Syn. : membrum definitum). Dans une définition$^2$, terme sujet dont il s’agit de
déterminer la compréhension ({\it p. e.}
« l'homme » dans : « L'homme est un
animal raisonnable »).

\ib{Définissant} — \si{Log.} (Syn. : membrun
definiens). Dans une définition$^2$,
terme attribut qui détermine$^1$ la
compréhension du défini ({\it p. e.} « animal raisonnable » dans : « L’homme
est un animal raisonnable »).

\ib{Définition} — [{\it L.} finis, borne, limite]
— \si{Log.} O {\bf 1.} Opération qui consiste à
déterminer$^1$ les « limites » de l’extension$^3$ d’un concept, ce qui se fait par
une analyse de sa compréhension$^2$,
$->$ {\it Dist.} description*. — {\bf 2.} @ Proposition par laquelle s’exprime cette
opération : « Il [Leibniz] pose des
définitions exactes qui le privent de
l’agréable liberté d’abuser des termes
dans les occasions » (Fontenelle). —
 {\bf 3.} Syn. de définissant* : « La définition doit convenir à tout le défini$^2$
et au seul défini ».

\ib{Dégradation de l'énergie} — \si{Phys.}
(Syn. : principe de Carnot-Clausius).
L'énergie$^1$, tout en restant, à travers ses transformations, constante
en quantité (conservation$^2$ de l’énergie), tend à déchoir de ses formes
supérieures à celles qui ne sont pas
utilisables, du moins en totalité
(chaleur), de sorte qu’une certaine
% 51
quantité d'énergie se trouve pratiquement perdue dans toute transformation.

\ib{Déisme} — \si{Méta.} (Syn. : religion naturelle). À. Doctrine qui admet l’existence de Dieu, mais nie la révélation
à et qqfs. même la Providence : « La
voie ouverte au déisme, c’est-à-dire
à un athéisme déguisé » (Bossuet).

\ib{Déjà-vu (Sentiment du)} — \si{Psycho.}
Celui que nous ressentons en présence d’une perception que nous
avons déjà éprouvée. — Illusion du
déjà-vu : cf. Paramnésie*.

\ib{Délibération} — \si{Psycho.} Une des
phases de l’acte volontaire selon la
description classique : elle consiste 
dans le conflit et l'examen des motifs* et des mobiles? favorables et
contraires à l’acte projeté.

\ib{Délire} — \si{Ps. path.} Trouble mental
temporaire ou progressif où la conscience est envahie par des images
hallucinatoires et désordonnées. Délire aigu : « agitation confusionnelle
violente avec désorientation, égarement, fièvre élevée, signes infectieux [azotémie] » (Baruk). — Voir
Grandeurs*, Interprétation*, Onirique*, Persécution*.

\ib{Démence} — \si{Ps. path.} Trouble mental
profond caractérisé par « une destruction organique et définitive de
l'intelligence » (Janet). $->$ {\it Dist.}
névrose*. — Démence précoce : voir
Hébéphrénie*. — {\it Ext.} : « Une passion sans intervalles est démence »
(Buffon).

\ib{Démérite} — \si{Mor.} Diminution de
valeur personnelle résultant d’une
faute contre la loi morale.

\ib{Démiurge} — [G. démiourgos, ouvrier]
— \si{Hist.} Chez Platon : divinité organisatrice (mais non créatrice) de
l'univers.

\ib{Démocratie} — \si{Soc.} Type d’organisation politique où dominent les
tendances égalitaires ({\it opp.} régimes
où dominent les castes, les classes*
ou aussi un parti politique à l'exclusion de tout autre) : « La démocratie suppose à sa base des volontés
individualisées* entre lesquelles
s'établit l'accord qui fonde l’ordre
social » (R. Hubert); « Le conformisme forcé du parti unique est
en opposition radicale avec les principes de la démocratie » (Henri
Lévy-Bruhl).

\ib{Démographie} — \si{Soc.} Étude statistique des mouvements de la population (naissances, mariages, décès, émigration, etc.).

\ib{Démon de Socrate} — \si{Hist.} Génie par
lequel Socrate se disait inspiré.

\ib{Démoniaque} — {\bf 1.} Diabolique. — {\bf 2.}
Dont l'inspiration est plus ou moins
irrationnelle : « Nous ne savons pas
assez combien la raison est démoniaque et nous prenons son démon
pour une voix » (Pradines).

\ib{Démonstration} — \si{Log.} {\bf 1.} {\it Str.} Déduction* constructive ({\it opp.} déduction
purement formelle) : « La démonstration mathématique ». — {\bf 2.} Lato
(Syn. : preuve*). Toute manière de
prouver, soit déductivement, soit
inductivement : « La démonstration
expérimentale ». $->$ Impropre au
sens {\bf 2.}

\ib{Dénombrement imparfait} — \si{Log.}
(Syn. : énumérulion incomplète).
Paralogisme* qui consiste, dans un
raisonnement exigeant l’énumération de tous les cas d'espèce ({\it p. e.}
induction$^2$ formelle, alternative$^2$), à
omettre un des cas possibles.
% 52

\ib{Dénotation} — \si{Log.} \si{form.} ({\it Opp.} : connotation*). Propriété que possède un
terme de désigner certains sujets$^2$
constituant l'extension$^3$ du concept
correspondant.

\ib{Densité sociale} — \si{Soc.} Degré d'unité
matérielle et morale d’une société,
se mesurant à la fois par la concentration de la population et par le
nombre des individus qui vivent
d’une vie morale commune.

\ib{Déontologie} — (G. ta deonta, les devoirs,
et logos] — \si{Mor.} {\bf 1.} Nom donné par
J. Bentham à sa morale (Deontology
or Science of morality, 1834). —
 {\bf 2.} {\it Auj.}, étude des devoirs spéciaux
à une situation déterminée : « La
déontologie du médecin ».

\ib{Dépasser} — Voir Aufheben*.

\ib{Dépersonnalisation} — \si{Ps. path.}
Trouble conscient de la personnalité
dans lequel le sujet éprouve vis-à-vis
de lui-même et du monde extérieur
des sentiments d’étrangeté.

\ib{Déplacement} — \si{Ps. an.} Procédé de la
pensée onirique* par lequel : « 1° un
élément latent* est remplacé par
une allusion ; 2° l'accent psychique
est transféré d’un élément sur un
autre, peu important, de sorte que
le rêve apparaît étrange » (Freud).

\ib{Déréalisation} — \si{Méta.} Tendance à
dépouiller le monde usuel de sa
réalité prétendue et, par suite, à
situer la réalité authentique au-delà ou en dehors de ce monde :
« Les métaphysiques apparaissent
toujours, par qq. côté comme des
entreprises de déréalisation du
Monde » (Alquié).

\ib{Déréliction} — [{\it trad.} all. Geworfenheit] —
\si{Méta.} Chez Heidegger : état d'abandon
et de solitude de l’être humain « jeté
dans le monde ».

\ib{Désagréable} — Voir Douleur*.

\ib{Désagrégation mentale} — \si{Ps. path.}
État pathologique caractérisé par la
dissolution plus ou moins complète
de la synthèse$^3$ mentale (dans
l'hébéphrénie*, la schizophrénie*,
etc.).

\ib{Description} — \si{Log.} La description
se dist. de la définition$^1$ en ce qu’elle
n’est fondée que sur des caractères
sensibles et extérieurs; on « décrit »
des individus$^3$ ; on ne « définit » que
des genres$^1$ et des espèces$^2$.

\ib{Désir} — \si{Psycho.} {\bf 1.} © Tendance devenue consciente d'elle-même et
accompagnée de la représentation
de son but (voir Précis, Ph. I, p. 469).
— 2, © Aspiration profonde : « La
raison est elle-même au service d’un
désir, auquel aucun désir de notre
être donné ne peut s'égaler » (Nabert). — {\bf 3.} I Objet du désir$^1$ : « Tout
ce qui de mon cœur fut l’unique
désir » (Racine).

\ib{Destin} — {\bf 1.} Puissance, plus ou moins
personnifiée, qui est censée gouverner les événements et l'existence de
Fhomme : « Des arrêts du destin
l’ordre est invariable » (Corneille). —
 {\bf 2.} Sort fatal : « L'homme de destin
ne croit pas à sa liberté » (Wahl). —
 {\bf 3.} Destinée$^2$ : « Le moi et son destin»
(Lavelle).

\ib{Destinée} — « Le mot destinée a deux
sens. [1] Ce terme équivoque désigne le développement nécessaire
de la vie, indépendamment de toute
intervention de l’homme dans la
trame des événements qui se déroulent en lui et hors de lui [cf. Destin$^2$] ;
— [2] et il désigne, en même temps,
la façon personnelle dont nous parvenons à nos fins dernières, selon
l'usage même de la vie et l'emploi
de nos volontés » (M. Blondel).
%53

\ib{Détermination} — \si{Log.} et \si{Méta.} {\bf 1.}
@ Tout caractère$^2$, qualitatif ou quantitatif, qui fait qu'un être, un concept est ce qu'il est. {\it Spéc.}, chez Le
Senne : aspect déterminé$^1$ de l’expérience$^1$ (dont l’autre aspect est la
valeur$^2$) qui fait obstacle* à la
« spontanéité naïve » du je*. —
 {\bf 2.} O Action de déterminer$^1$.

— \si{Psycho.} {\bf 3.} Syn. de décision* :
« Liberté, c'est choix, détermination volontaire au bien ou au mal »
(La Bruyère).

\ib{Déterminer} — \si{Log.} {\bf 1.} Délimiter,
fixer, établir : « Déterminer le sens
d'un mot, la cause d’un accident ».
— {\bf 2.} (\si{Vulg.}). En parlant d’une
cause : produire, faire exister
(l’effet) : « Déterminer un accident ».
— {\bf 3.} (Dans la langue philosophique). Conditionner d’une façon nécessaire$^2$ (cf. Déterminisme$^2$).

\ib{Déterminisme} — \si{Épist.} @ {\bf 1.} Ensemble des conditions nécessaires$^2$
d’un phénomène : « Connaître le
déterminisme d’une maladie ». —
O. {\bf 2.} Principe de la science expérimentale selon lequel il existe entre
les phénomènes des relations nécessaires$^2$ (lois$^5$), de telle sorte que tout
phénomène est rigoureusement conditionné par ceux qui le précèdent
ou l’accompagnent : « Le déterminisme est absolu aussi bien dans les
phénomènes des corps vivants que
dans ceux des corps bruts » (Claude
Bernard). — Voir Statistique.

— \si{Méta.} À {\bf 3.} Doctrine philosophique selon laquelle tout, dans
l'univers, même les décisions de la
volonté humaine, est le résultat nécessaire des conditions antérieures
ou concomitantes. $->$ {\it Dist.} fatalisme* et cf. Mécanique$^4$.

\ib{Devenir} — \si{Méta.} {\bf 1.} @ Série des changements concrets par lesquels passe
% 53
un être$^2$ : « Pourquoi ne pas identifier l’être au devenir ? » (Le Roy,
{\it R. M. M.}, 1907, p. 150) ; « C’est la
réaction contre la pensée du devenir
qui explique en grande partie le
développement de la métaphysique
en Occident après Héraclite »
(Wahl). — {\bf 2.} O Le changement lui-même : « Sur la continuité d’un certain devenir$^1$, j'ai pris une série de
vues que j'ai reliées entre elles par
le devenir$^2$ en général » (Bergson,
{\it E. C.}, IV).

\ib{Dévoilement} — ({\it Trad.} all. Entbergung)
— \si{Méta.} Chez Heidegger : caractère
de la vérité qui, selon l’étymologie
du mot grec a-lètheia, non-voilement, dé-couverte, est la manifestation de l’étant qui cesse d’être
caché par les préoccupations de
l'existence quotidienne.

\ib{Devoir (verbe)} — \si{Vulg.} « Ce qui doit
être » peut signifier : {\bf 1.} ce qui sera
(marque simplement le futur) : « Je
devais partir demain » ; — {\bf 2.} ce qui
est, a été ou sera probablement
(marque une probabilité$^1$) : « Il doit
être arrivé, à cette heure » ; « C’est
elle qui doit avoir fait cela » ; — {\bf 3.} ce
qui ne peut pas ne pas être (marque
une nécessité) : « Tous les hommes
doivent mourir » ; — {\bf 4.} ce qui est
désirable, ce qui vaut mieux (marque
une convenance) : « On devrait
planter des arbres ici » ; — {\bf 5.} ce que
l’on est tenu de réaliser (marque une
obligation morale ou sociale) : « Fais
ce que dois » ; « Je dois payer mes
impôts à telle date ». $->$ Bien dist,
ces différents sens.

\ib{Devoir (nom)} — \si{Mor.} {\bf 6.} O L’obligation$^1$ morale : le devoir est la forme$^2$
de la loi morale (ci. Catégorique$^2$)
comme le bien$^2$ en est la matière$^2$. —
 {\bf 7.} @ Objet de cette obligation : « Le
seul viatique utile pour la traversée
% 54
de la vie, c’est un grand devoir »
(Amiel). Ext, obligation sociale,
rituelle, etc. : « Le devoir fiscal » ;
« Le devoir pascal ».

\ib{Dialectique} — [G. dialegesthai, converser, et dialegein, trier, distinguer]
— \si{Hist.} {\bf 1.} Chez les Socratiques : art
de discuter par questions et réponses ; d'où : art de classer les concepts,
et {\it spéc.}, chez Platon, art de s'élever
des connaissances sensibles aux
connaissances intelligibles, aux
Idées$^1$. — {\bf 2.} Chez Aristote : art des
raisonnements qui portent sur de
simples opinions$^2$, logique du probable. — {\bf 3.} Au moyen âge : les subtilités de la logique formelle : « La
dialectique renverse le bon sens au
lieu de le raffermir » (Descartes). —
 {\bf 4.} Chez Kant : « dialectique transcendantale », logique de « l’apparence transcendantale* », {\bf 1.} e. étude
de cette illusion qui porte notre
esprit à dépasser par ses raisonnements les limites de toute expérience possible ({\it R. pure}, I, II, 2,
introd.). — {\bf 5.} Chez Hegel : marche de
la pensée procédant par thèse, antithèse et synthèse$^2$ et qui reproduit
le mouvement même de l’Être
absolu ou Idée$^1$. — {\bf 6.} Dans le lang.
marziste (non chez Marx lui-même) :
matérialisme dialectique. Voir Matérialisme$^{1b}$. — {\bf 7.} {\it Péj.}, art de discuter
à l’aide de raisonnements subtils et
vides : « Leurs preuves [des pragmatistes] ont le plus souvent un caractère
dialectique; tout se réduit à une pure
construction logique » (Durkheim).

— {\bf 8.} {\it Auj.}, le mot s'emploie en
une multitude de sens, pour désigner, soit (sens ancien) « toute suite
de pensées ordonnées qui dépendent
logiquement l’une de l’autre » (Lalande), soit (sens hégélien) une idée
plus ou moins vague de mouvement
discontinu dans la pensée ou dans
% 54
l’être. {\it P. e.} Croce parle d’une histoire
« dialectique » de la pensée humaine;
Laveile propose une « dialectique de
la participation » où « l'esprit pénètre
le monde en faisant jaillir en nous
une pluralité de puissances auxquelles le réel ne cesse de répondre » ;
dans sa « Dialectique de la durée »
Bachelard oppose une conception
discontinuiste de la durée à la conception continuiste de Bergson;
K. Jaspers déclare que « tout ce qui
est vivant se développe avec une
multiplicité de significations dialectique ». — {\it Spéc.}, \si{Théol.}, le terme
dialectique évoque une opposition
entre l'homme et Dieu. Chez Kierkegaard, cette opposition se révèle
dans l’angoisse$^2$ du péché : « dialectique » signifie alors « inquiétude »;
Berdiaev offre une « dialectique
existentielle du divin et de l’humain »; X. Barih enfin propose une
« théologie dialectique » dans laquelle « le non doit toujours être
expliqué par le oui, et le oui par le
non », le langage humain étant toujours déficient quand il s’agit de
Dieu. $->$ Pratiquement, ce terme
est devenu si équivoque, qu'il est
indispensable de toujours préciser
en quel sens on l’emploie.

\ib{Diallèle} — [G. di’ allêlôn, les uns par les
autres] — \si{Log.} Argument des
sceptiques anciens d’après lequel
toutes nos connaissances, se démontrant les unes par les autres, forment un cercle* vicieux.

\ib{Dialogue} — Ce terme s'emploie souvent auj. dans le style pseudo-philosophique au sens de relation réciproque : « La recherche scientifique
est un dialogue entre l'esprit et la
nature » (Bergson, {\it P. M.}, VII).

\ib{Dichotomie} — [G. dicha, en deux, et
tomè, coupure] — \si{Log.} Division
% 55 — DIE
du concept d’un genre$^1$ en deux espèces$^2$ qui en épuisent l’extension
({\it p. e.} « animal » en « vertébré » et
« invertébré »).

\ib{Dictum de omni et nullo} — \si{Log.}
\si{form.} Principe logique suivant lequel ce qui est affirmé ou nié de
tout un genre$^1$ peut être affirmé ou
nié de toutes les espèces$^2$ et de tous
les individus$^3$ de ce genre.

\ib{Didactique} — [G. didaskein, enseigner]
— Qui se rapporte à l’enseignement.

\ib{Dieu} — Méta et \si{Théol.} A) E. Le Roy
(se référant à Belot, R. ph., déc.1908)
distingue 3 sens de l'idée de Dieu :
« {\bf 1.} le Dieu populaire et anthropomorphe, objet d'imagination collective, symbole et facteur ou plutôt
centre d'unité sociale» [« Les hommes
s'imaginent que les dieux sont engendrés comme eux, qu'ils ont des
vêtements, une voix, un corps
pareils aux leurs » (Xénophane);
« Tout était dieu, excepté Dieu lui-même » (Bossuet)]; — {\bf 2.} le Dieu
moral, connu par expérience de vie
intérieure, avec lequel on cherche à
entrer en communion spirituelle »
(« Dieu m'est plus intime que mon
intimité même » (saint Augustin);
« Dieu sensible au cœur » (Pascal).
À la limite, c’est le Dieu mystique* :
« On veut toujours dire qqe. digne
de Dieu... Mais on ne peut pas même
expliquer combien il est ineffable*,
ni comprendre combien il est incompréhensible* » (Bossuet)]; —
« {\bf 3.} le Dieu philosophique, premier
principe d’existence et centre d’unité
intelligible » (voir B). — Ces 3 notions ne coïncident pas nécessairement : cf. Pascal, mémorial : « Dieu
d'Abraham, d’Isaac et de Jacob,
non des philosophes et des savants »,
et Bergson, Deux Sources, ch. III :
« Quand la philosophie parle de
Dieu,... il s’agit si peu du Dieu
auquel pensent la plupart des
hommes que, si, par miracle et
contre l'avis des philosophes, Dieu
ainsi défini descendait dans le champ
de l'expérience, personne ne Île
reconnaîtrait. » Tandis que les conceptions 2 et 3 sont gén. liées au
monothéisme, la 1$^\text{ère}$ l’est souvent au
polythéisme (v. ces mots), ou bien à
une religion nationale : cf. Deutéronome, VII, 6 : « Yahvé, ton Dieu,
t'a choisi pour être son peuple parmi
tous les peuples de la terre. »

— B) La conception philosophique (3) se rattache elle-même
soit au théisme, soit au panthéisme
(v. ces mots). a) Dans le premier,
Dieu, tout en étant distinct du
monde (cf. Créationnisme*), est :
$\alpha$. principe d'existence, être par soi
(cf. Aséité*), infini*, éternel*, nécessaire$^1$. « Par le nom de Dieu j’entends une substance infinie, éternelle, immuable, indépendante,
toute connaissante, toute puissante
et par laquelle moi-même et toutes
les choses qui sont, ont été créées et
produites » (Descartes, \si{{\it Méd.}}, III);
« Dieu est celui en qui le non être
n’a point de lieu » (Bossuet) ; « Par
la Divinité nous entendons tous
l'Infini, l’Être sans restriction, l'Être
infiniment parfait » (Malebranche,
{\it Entr.}, VIII, 1). {\it Cf.} Immensité*,
Ubiquité* ; — $\beta$. principe d'intelligibilité : « Les vérités* éternelles
sont quelque chose de Dieu ou
plutôt sont Dieu même » (Bossuet) ;
« Dieu seul est notre lumière » (Malebranche, o. c., IV, 14) — $\gamma$. principe
de perfection morale : « Dieu est le
Bien universel qui comprend tous
les biens » (Malebranche); « En tant
que la valeur absolue doit posséder
éminemment la personnalité, l’Absolu*
% 56
doit être appelé Dieu » (Le
Senne).

— b) Dans le panthéisme, Dieu
est la Substance unique : « En dehors
de Dieu, aucune substance ne peut
être donnée ni être conçue. Tout ce
qui est, est en Dieu» (Spinoza, Éth.,
I, 14-15).

\ib{Différence} — \si{Log.} {\bf 1.} Difjérence spécifique : caractère qui distingue une
espèce$^2$ des autres espèces du même
genre$^1$ ({\it p. e.} « allaitant ses petits » est
la différence spécifique de « mammifère » dans le genre « vertébré »).
— {\bf 2.} Méthode de différence : une
des quatre méthodes expérimentales
de J. Stuart Mill (voir Précis, Ph. II,
p. 122 ; Sc. et M., p. 2388).

\ib{Différenciation} — \si{Biol.} {\bf 1.} Processus par lequel des organes ou des
fonctions primitivement semblables
se transforment progressivement en
organes ou fonctions différents, d’où
résulte une division$^4$ du travail. —
Ext, {\bf 2.} \si{Psycho.} et \si{Soc.} Même sens.
— Plus gén. encore : {\bf 3.} \si{Méta.} « Passage de l’homogène à l’hétérogène »
(Spencer).

\ib{Différentiation} — \si{Math.} Opération
fondamentale du calcul difiérentiel$^1$.

\ib{Différentiel} — \si{Math.} {\bf 1.} Calcul différentiel : partie du calcul infinitésimal* qui permet, grâce à la considération des accroissements infiniment$^2$ petits subis par certaines
variables*, de déterminer les relations entre certaines grandeurs,
quoiqu'il n'existe pas de commune$^3$
mesure entre les données et l’inconnue.

— \si{Ps. métr.} {\bf 2.} Seuil différentiel :
voir Seuil*,

\ib{Diffus} — Non organisé. {\it Spéc.} \si{Soc.} :
« Nous dirons des fonctions économiques, dans l’état où elles se trouvent, qu’elles sont diffuses, la diffusion consistant dans l'absence d’organisation » (Durkheim).

\ib{Diffusionisme} — \si{Soc.} A. Théorie
({\it opp.} à l’évolutionnisme$^1$), soutenue
par certains anthropologistes anglosaxons, qui explique l'expansion
des civilisations$^1$ par la diffusion,
{\it i. e.} par « le transfert de certains
traits culturels d'une aire de civilisation à une autre » (Krœber).

\ib{Dilemme} — \si{Log.} Raisonnement qui
consiste à poser une alternative$^2$ et
à montrer que, dans les deux cas,
la même conclusion s'impose : c’est
un syllogisme disjonctif*. Schéma :
« De deux choses l’une, ou A est B,
ou C est D. Si A est B, R est S. Si
C est D, Rest S. Donc R est S ».

\ib{Dilettantisme} — \si{Psycho.} État d'esprit de celui qui se complaît au jeu
des idées sans rechercher la vérité.

\ib{Dimension} — M \si{Vulg.} {\bf 1.} Mesures :
« Les dimensions d’un corps ». —
\si{Math.} {\bf 2.} Quantité qui détermine
la position d’un point sur une ligne,
dans un plan ou dans un espace :
« La géométrie classique est une
géométrie à trois dimensions ». —
\si{Phys.} {\bf 3.} Relation d’une unité ({\it p. e.}
vitesse) aux unités fondamentales
({\it p. e.} longueur parcourue et temps).

— {\it Ext.} {\bf 4.} © Système de références, point de vue : « La dimension de la morale et la dimension
de l’histoire ne sauraient être conciliées » (Alquié).

\ib{Dionysiaque} — \si{Hist.} Chez Nietzsche,
le point de vue dionysiaque est celui
de l’exaltation tragique et pathétique de la vie, {\it opp.} à la sérénité
apollinienne*.

\ib{Dioptrique} — Voir Catoptrique*.

\ib{Discernement (Temps de)} — Ps.
métr. Temps nécessaire à un sujet
% 57
pour distinguer une impression
simple ({\it p. e.} le noir) d’une autre impression simple ({\it p. e.} le blanc).

\ib{Discipline} — {\bf 1.} Ensemble de connaissances méthodiques : « Les
disciplines philosophiques » — {\bf 2.}
Règle de conduite : « La discipline
des mœurs. »

\ib{Discontinu} — \si{Épist.} {\bf 1.} Formé d'éléments distincts : « L'intervention
des quanta* a conduit à introduire
partout le discontinu dans la Physique atomique » ({\it L.} de Broglie). {\it Cf.}
Granulaire*.

— \si{Math.} {\bf 2.} Quantité discontinue
(ou discrète) : celle qui varie par passage brusque d’une valeur à une
autre : {\it p. e.} le nombre entier.

\ib{Discours} — \si{Épist.} {\bf 1.} Pensée discursive* : « L'intuition s'oppose au discours, comme la pensée qui se concentre à la pensée qui se détend »
(Le Roy).

— \si{Log.} 2 Univers du discours.
Voir Univers*.

\ib{Discrimination} — \si{Psycho.} Fonction
psychique grâce à laquelle la conscience$^1$ distingue l’un de l'autre
deux de ses états, {\it spéc.} deux sensations (cf. acuité* sensorielle).

\ib{Discursif} — \si{Épist.} ({\it Opp.} : intuitif).
Qui va d’une idée ou d'un jugement
à un autre en passant par un ou plusieurs intermédiaires : « Le raisonnement est le type de la pensée
discursive. »

\ib{Disjonctif} — \si{Log.} \si{form.} Proposition
disjonctive : celle qui exprime une
alternative$^2$. Syllogisme disjonctif :
celui qui a pour majeure une proposition disjonctive. Schéma : « A
est R ou S. Or A est R. Donc A n'est
pas S. Ou bien : Or A n’est pas R.
Donc A est S ».

\ib{Dissociation} — \si{Psycho.} Fonction par
laquelle des éléments psychiques
d’abord confondus sont séparés par
la conscience$^1$.

\ib{Dissolution} — Voir Involution*.

\ib{Distinct} — \si{Vulg.} {\bf 1.} © Net, précis :
« La vision distincte » ; « Avoir un
souvenir distinct de quelque chose ».
— {\bf 2.} M Que l’on discerne bien :
« Les objets devinrent distincts ».

— \si{Log.} (Ctr. confus) {\bf 3.} Chez
Descartes : « J’appelle distincte celle
[la connaissance] qui est tellement
précise et différente de toutes les
autres, qu'elle ne comprend en soi
que ce qui paraît manifestement à
celui qui la considère comme il faut »
({\it Princ.}, I, 45); « La connaissance
peut qqfs. être claire sans être distincte » (ib., 46). — {\bf 4.} Chez Leibniz :
« Lorsque je puis expliquer les marques que j'ai [de sa vérité], la connaissance s’appelle distincte : telle
est la connaissance d’un essayeur
qui discerne le vrai ou le faux par
le moyen de certaines épreuves ou
marques qui font la définition de
l'or » (Disc. méta, 24). $->$ Pour
Descartes, une idée est distincte
relativement aux autres idées; pour
Leibniz, elle l’est en elle-même,
quand on connaît bien ses éléments
constitutifs, {\it i. e.} sa compréhension$^2$.

\ib{Distorsion} — {\bf 1.} \si{Techn.} Déformation
des images visuelles ou auditives par
certains appareils optiques ou radiophoniques. — {\it Ext.} {\bf 2.} \si{Psycho.}, \si{Soc.}
Altération d’une fonction par action
d’un secteur perturbé sur un autre.

\ib{Distributive (Justice)} — \si{Mor.} ({\it Opp.} :
commutative*). Celle qui préside à la
répartition des biens, des récompenses et des châtiments et repose
par suite sur l'inégalité des mérites.
%58

\ib{Divertissement} — Chez Pascal
tout ce qui détourne l’homme de
penser à sa condition et à son salut.

\ib{Division} — \si{Log.} {\bf 1.} Décomposition
d'un tout en ses parties. $->$ {\it Dist.}
analyse$^3$. — {\bf 2.} Opération logique
qui consiste à distinguer les espèces$^2$
d'un genre$^1$ donné.

\ib{Division du travail} — \si{Éc. pol.} {\bf 1.} (Division du travail professionnelle). Spécialisation du travail par corps de
métiers, la même industrie se subdivisant à son tour en différentes
branches. — {\bf 2.} (Division du travail
technique et {\it spéc.} industrielle.) Décomposition, dans le travail industriel, d’un mouvement complexe en
mouvements simples.

— \si{Soc.} {\bf 3.} (Division du travail
social.) Spécialisation des fonctions,
au point de vue non seulement
économique, mais aussi politique,
juridique, etc.

— {\it Ext.} Biol, {\bf 4.} Spécialisation
des fonctions physiologiques dans un
organisme vivant.

\ib{Dogmatique} — \si{Car.} {\bf 1.} En parlant
des personnes : qui. affirme d’une
façon tranchante et sans preuves
(cf. « un ton dogmatique »).

— \si{Crit.} {\bf 2.} ({\it Opp.} : sceptique*). En
parlant des théories ou des méthodes : qui se rapporte au dogmatisme? : « La science doit toujours
être dogmatique, {\it i. e.} strictement
démonstrative » (Kant., {\it R. pure}
préf. 2$^\text{e}$ éd.). — {\bf 3.} ({\it Opp.} : relativiste*). En parlant des doctrines :
qui se rapporte au dogmatisme$^3$ :
« La philosophie dogmatique ».

— \si{Péd.} {\bf 4.} ({\it Opp.} : heuristique* ou
historique). En parlant des méthodes : qui consiste à exposer les vérités scientifiques, non dans l’ordre
où elles ont été découvertes, mais
d'une manière logique et en partant des principes (cf. sens 2).

— \si{Théol.} {\bf 5.} Qui concerne les
dogmes, {\it i. e.} les vérités révélées,
objet de la foi$^5$ : « La théologie dogmatique ».

\ib{Dogmatisme} — \si{Car.} {\bf 1.} Tendance à
affirmer sans preuve ni critique
préalable,

— Crit, {\bf 2.} {\it Lato.} (Ctr. : scepticismel). {\it Autref.}, toute doctrine affirmant la possibilité de la connaissance vraie. — {\bf 3.} {\it Str.} ({\it Opp.} : agnosticisme*, criticisme*, relativisme*).
{\it Auj.}, doctrine affirmant la possibilité et la validité de la connaissance
de l’absolu$^1$.

\ib{Dolorisme} — \si{Mor.} ({\it Opp.} : hédonisme*).
Apologie de la douleur (voir Précis,
Ph. I, p. 394, note).

\ib{Domestique} — \si{Soc.} Qui se rapporte à
la famille comme groupe de parenté :
« La société domestique ».

\ib{Donné} — \si{Psycho.} Immédiatement
présent à la conscience avant toute
élaboration (cf. Données$^2$).

\ib{Données} — \si{Math.} {\bf 1.} ({\it Opp.} : inconnues). Dans un problème : quantités
connues.

— \si{Psycho.} {\bf 2.} Données de la connaissance : éléments de la connaissance fournis immédiatement, soit
par les sens, soit par la raison pure,
— {\bf 3.} Données immédiates de la conscience (Bergson) : propriétés fondamentales de la vie psychique données dans l'intuition$^4$ et dégagées de
toute la superstructure de concepts
et d’habitudes qui les déforment.

\ib{Douleur} — \si{Psycho.} {\bf 1.} {\it Lato.} Syn. de
désagréable et, en ce sens, indéfinissable. $->$ Impropre, quoique fréquent, en ce sens (voir Précis, Ph. I,
p. 379). — {\bf 2.} {\it Str.} Au sens propre,
le terme s’applique : a) soit à la douleur physique qui s'apparente à la
sensation (Précis, ib., p. 386); —
b) soit à la douleur morale qui s’apparente au sentiment$^5$ et à l’émotion$^3$
(Précis, ib., p. 389).
% 59

\ib{Doute} — \si{Psycho.} {\bf 1.} ({\it Opp.} :
assentiment ou croyance*). État de l'esprit
qui suspend son assentiment*.
$->$ {\it Dist.} opinion$^2$ — \si{Ps. path.}
 {\bf 2.} Folie du doute : incapacité de
croire$^3$ (de donner son assentiment*)
ou de prendre des décisions*.

— \si{Hist.} {\bf 2.} Chez Descartes : « doute
méthodique », méthode philosophique qui consiste à révoquer en
doute tout ce qu’on a admis antérieurement et à n’accepter pour vrai
que ce qui est évident, afin de fonder
la connaissance sur des bases certaines : « Je pensai qu'il fallait que
je rejetasse comme absolument faux
tout ce en quoi je pourrais imaginer
le moindre doute » (Méth., IV). {\it Cf.}
Husserl, Médit. cartésiennes, introd. :
« Ne connaissant d’autre but que
celui d’une connaissance absolue,
il [Descartes] s’interdit d'admettre
comme existant ce qui n’est pas à
l’abri de toute possibilité d’être mis
en doute ». — {\bf 4.} Doute scientifique :
attitude du savant qui révoque en
doute ses hypothèses$^2$ tant qu’elles
ne sont pas confirmées par l’expérience* : « Le grand principe expérimental est le doute philosophique
% 59
qui laisse à l’esprit sa liberté et son
initiative » (Claude Bernard).

\ib{Doxique} — [G. doxa, opinion, croyance]
— Qui concerne la croyance$^3$. Voir
modalité*.

\ib{Droit (adj.)} — \si{Vulg.} {\bf 1.} Par anal. avec
« ligne droite » : juste$^4$, honnête, sans
détours : « La droite raison »; « Une
conscience droite »; « La philosophie
fait un cœur droit comme la géométrie fait l'esprit juste » (Voltaire).

\ib{Droit (nom)} — \si{Mor.} et \si{Jur.} A) Le
droit : {\bf 2.} {\it Lato.} ({\it Opp.} : fait). Ce qui
doit être, par {\it opp.} à ce qui est; ce
qui est légitime, au point de vue
moral, juridique ou même logique.
En droit s’oppose en ce sens à en fait
({\it p. e.} \si{Log.} : « nécessaire$^1$ en droit »).
— Droit naturel (syn. : droit moral,
idéal, rationnel) : celui qui appartient
à l’homme du seul fait qu'il est
l’homme, indépendamment de toute
convention ou législation : « Le droit
est la limitation de la force par la
raison » (Parodi)) ; cf. inhérence$^3$. Droit
positif : celui qui résulte des coutumes établies (droit coutumier) ou
des lois (droit écrit). — {\bf 3.} {\it Str.} {\it Spéc.}
le droit positif (se subdivisant comme
l'indique le tableau ci-dessous) :
« Le droit est l’ensemble des règles
obligatoires qui déterminent Îles
rapports sociaux tels que la volonté
collective du groupe se les représente
à tout moment » (H. Lévy-Brubhl).

Droit.
 National. . .
  Privé. . .
   Droit civil.
   Droit commercial.
  Public. . .
   Droit pénal.
   Droit administratif.
   Droit constitutionnel.
 International (droit des gens).
  Privé.
  Public.
%60

— B) Les droits : {\bf 4.} (Avoir droit
à...) Ce qui est exigible en vertu
du droit positif ({\it p. e.} « Droit de réponse ») ou du droit naturel ({\it p. e.}
« Droit à la vie »). — {\bf 5.} (Avoir Le
droit de...) Ce qui est permis par le
droit positif ({\it p. e.} « Droit de tester »),
par les conventions ou règlements
({\it p. e.} « Droit de passage sur un terrain ») ou par la morale ({\it p. e.} « On
n’a pas le droit de se venger »)) : « Un
peuple inorganisé n’a pas encore en
lui-même le droit [d'indépendance]
réel » (Alain).

\ib{Dualisme} — {\bf 1.} {\it Lato.} Toute doctrine
qui, dans un ordre d'idées quelconque, pose deux principes$^5$ absolument irréductibles : {\it p. e.} \si{Théol.} :
« Dualisme de la nature et de la
grâce » ; \si{Psycho.} : « Dualisme de la
volonté et de l’entendement » ;
\si{Phys.} : « Dualisme de la matière et
de l’énergie » ; Méta : « Dualisme de
l’âme et du corps ».

— {\bf 2.} Sir. \si{Méta.} ({\it Opp.} : monisme$^1$
et pluralisme*). À Système métaphysique qui admet dans l’univers deux
substances$^1$ ou deux mondes irréductibles : « Le véritable dualisme
est celui qui pose, non deux principes du même monde, mais deux
mondes » (S. Pétrement). {\it Cf.} Manichéisme*.

\ib{Duplicité} — \si{Méta.} À. Au sens étymologique : caractère double, dualité :
« Cette duplicité de l’homme est si
visible qu’il y en a qui ont pensé
que nous avions deux âmes » (Pascal,
417); la « duplicité de l'obligation »
[en tant que relation entre celui qui
oblige et celui qui est obligé] (Le
Senne).

— \si{Car.} et \si{Mor.} {\bf 2.} {\it Péj.} Manque de
sincérité : « Ils ne servent qu’à nous
montrer la duplicité de votre cœur »
(Pascal, {\it Prov.}, 13).
% 60

\ib{Durée} — \si{Vulg.} {\bf 1.} Portion finie et gén.
mesurée du temps* : « La durée
d'un phénomène ». Souvent avec
une idée de continuité : « La durée
de chaque chose est un mode ou une
façon dont nous considérons cette
chose en tant qu’elle continue d’être»
(Descartes, {\it Princ.}, I, 55).

— \si{Psycho.} et \si{Méta.} {\bf 2.} © Chez
Bergson (durée pure, vécue, concrête) : « Forme que prend la succession de nos états de conscience
quand le moi se laisse vivre »; « Succession de changements qualitatifs
qui se fondent, sans contours précis,... sans aucune parenté avec le
nombre » ({\it D. I.}, II).

\ib{Dynamique (adj.)} — {\bf 1.} (Ctr. : statique$^1$). Qui implique un mouvement, une transformation. — {\bf 2.}
(Ctr. : mécanique). Qui implique une
tendance* vers..., une finalité* : Le
« progrès dynamique » de la conscience ; « Le rapport de causalité
interne est purement dynamique »
(Bergson).

\ib{Dynamique (nom)} — \si{Math.} {\bf 3.} Partie
de la Mécanique$^1$ qui étudie le mouvement dans ses rapports avec les
forces$^4$ qui le produisent.

— \si{Soc.} {\bf 4.} Dynamique sociale
(Comte) : partie de la sociologie qui
traite du progrès$^1$ de la société ({\it opp.} :
Statique sociale, science des lois
générales de l’ordre social).

\ib{Dynamisme} — {\bf 1.} @ Activité$^1$ : « Le
dynamisme mental ». Ext, en parlant de l’homme, « avoir du dynamisme » être entreprenant. —
$->$ On abuse beaucoup de ce terme
auj. : cf. J. Benda : « Une nouvelle
idole : le dynamisme. »

— \si{Méta.} O À {\bf 2.} (Ctr. : mécanisme$^4$). Doctrine qui suppose des
forces irréductibles au mouvement
et qui, au lieu de considérer la
% 61
matière comme inerte, l'identifie
avec la force$^6$ et l'énergie : « Le
dynamisme de Leibniz » (voir Précis,
Ph. II, p. 479). — {\bf 3.} Syn. de vitalisme$^2$ ({\it p. e.} Ravaisson : v. Précis
ib., p. 486).

\ib{Dyschromatopsie} — \si{Ps. phol.} Trouble
de la vision des couleurs ({\it p. e.} daltonisme*).

\ib{Dysgnosie} — [G. dys, difficilement, et
gnôsis, connaissance] — \si{Ps. path.}
Trouble de la perception caractérisé 
par des sentiments d’incomplétude*
({\it p. e.} cas cités dans notre Précis,
Ph. I, p. 144).

	\end{itemize}
