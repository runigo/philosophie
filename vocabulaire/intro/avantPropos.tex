
\section{Avant propos}

{\it Parmi les difficultés qui attendent le débutant au seuil des études philosophiques,
celles qui viennent du langage ne sont pas les moindres.}

{\it De tout temps, les philosophes ont eu leur langue spéciale, et l’on trouvera
par exemple dans cet ouvrage l'indication du sens} scolastique {\it de certains
termes qui ne sont plus aujourd’hui d'usage très courant, encore que certains
d'entre eux (par exemple quand on parle de quelque chose qui n'existe qu’ « en
puissance »), persistent — et d’ailleurs fort utilement — dans notre langage
actuel. On a souvent contesté à la philosophie le droit d’user de ce langage
spécial et l’on a écrit des livres entiers pour démontrer qu’elle devait parler
« comme tout le monde ». Nous n’hésitons pas à dire qu’il y a là, selon nous, une
erreur, et une erreur très grave : car elle implique une complète méconnaissance
de la tâche propre de la pensée philosophique. À notre sens, la philosophie
est essentiellement un effort pour se libérer des confusions verbales et pour
atteindre à la pensée claire, c’est-à-dire à la probilé intellectuelle : car la
pensée confuse est une pensée qui ne croit pas à sa propre valeur, et qui, par
suite, ne fait pas l'effort nécessaire pour se clarifier. Qu’on veuille bien se
reporter à quelques articles de ce} Petit Vocabulaire {\it tels que les articles} cœur,
croyance, esprit, idée, liberté, sentiment, {\it et même — qu’on nous pardonne !}
— exister, {\it et l’on constatera la multitude de sens différents, parfois opposés,
qu'un même mot peut véhiculer avec lui. Le « sens commun » ne se soucie
guère de distinctions rigoureuses et le langage courant ne fait que refléter
cette confusion de pensée. Il resie donc vrai que la philosophie, autant et plus
que toute autre discipline intellectuelle, a besoin d’un langage technique.}
%6 AVANT-PROPOS

{\it Encore ne faut-il point en abuser. Certains auteurs contemporains semblent
se complaire à entourer leur pensée de barbelés terminologiques qui en défendent
l’accès aux profanes. Déjà, lorsque les logiciens nous parlent de « discours
apophantique » ou d’ « induction épagogique », nous concevons fort bien
qu'un débutant reste perplexe. Mais ce sont surtout la phénoménologie et
l'existentialisme qui ont fait preuve, en ce domaine, d’une virtuosité qui ne
fut pas toujours du meilleur aloi. Ce n’est pas, en tout cas, sans une sérieuse
initiation préalable, qu'on peut arriver à se débrouiller parmi les} contenus
hylètiques, {\it les} modalités doxiques, {\it la} noëse {\it et le} noème, {\it l’}existentiel {\it et
l'}existential, {\it l'}ontique {\it et l'}ontologique, {\it etc.}

{\it Il est pourtant, dans le langage philosophique contemporain, un autre
obstacle plus grave encore, parce que moins évident. Comme le remarquait
récemment M. Léon Bérard dans son livre} Science et humanisme, {\it publié
en commun avec le professeur Pasteur Vallery-Radot, l'usage s’est établi
d'employer « quantité de mots tirés de notre vieux vocabulaire, mais pris
dans un sens nouveau et mystérieux :} présence, témoignage, engagement,
structures », {\it à tel point qu’ « on s'applique, pourrait-on croire, à ne plus
appeler les choses par leur nom ». Même les mots techniques du langage philosophique
traditionnel sont aujourd'hui galvaudés en des significations presque
directement contraires à celles qu'on leur donnait jusqu'ici (voir par eremple
les articles} Profond {\it et} Transcendant{\it ).}

{\it Les philosophes eux-mêmes, ceux du moins qui demeurent fidèles aux
exigences de la pensée claire en même temps qu'au génie de cet « idiome réputé
pour sa loyauté, sa probité vigoureuse » qu’est la langue française, n’ont pas
manqué de dénoncer ces équivoques. Dans l’un de ses derniers livres} Héritage
de mots, héritage d'idées, {\it Léon Brunschvicg parlait de ce mot de} dialectique
{\it que l’on charge aujourd’hui « de significations suffisamment obscures et
diverses pour qu'y soit sous-entendu le pouvoir de tout contredire comme de
tout concilier » et il signalait ce « désir de total égarement » qui pousse certains
de nos contemporains à brouiller la valeur des termes en jouant avec « la scintillation
de leurs sens ». Notre maître André Lalande, dans un morceau que
nous avons reproduit dans nos} Textes choisis {\it (tome 1, p. 196), a protesté
lui aussi contre ce goût de l’équivoque et cette « obscurité verbale qui donne
l'illusion de la profondeur ».}

{\it La chose se complique encore du fait de cette} défrancisation {\it du langage
philosophique qui amène certains de nos jeunes philosophes à ne plus pouvoir
s’erprimer, semble-t-il, qu’en farcissant un français approtimatif de termes
calqués sur l'anglais et surtout l'allemand. On ne dit plus} recherche : {\it on dit}
approche {\it (par imitation de l'anglais} approach{\it ); on ne dit plus} état de fait,
{\it on dit} facticité {\it (par imitation de l'allemand} Faktizität{\it ), et j'imagine le
nombre de contre-sens qu'a dû entraîner chez les non-initiés ce terme ainsi
employé en un sens exactement opposé à celui qu’il a en bon français !}

%AVANT-PROPOS 7

{\it Nous nous sommes efforcé, dans ce} Nouveau Vocabulaire, {\it d'aplanir pour
les débutants — et peut-être aussi pour d’autres qui veulent y voir clair — toutes
ces difficultés, en donnant des définitions abordables tant des principaut
termes philosophiques que de certains mots du langage courant qui peuvent
être pris en un sens philosophique. Nous avons même introduit quelques
termes étrangers, tels que} aufheben, Erlebnis, Dasein, Gestalt, pattern, {\it etc,
dans leur langue d’origine, parce qu’ils sont parfois employés aunst, ce qui,
après tout, vaut peut-être mieux que de les accoutrer en un français douteux.
Nous avons enfin, le plus souvent, éclairé ces définitions par de courts exemples
précisant leur emploi. Il va de soi que nous avons utilisé pour ce travail les
ouvrages plus considérables tels que le si précieux } Vocabulaire technique
et critique de la philosophie {\it d'}A. \textsc{Lalande} {\it et le} Vocabulaire de la psychologie
{\it d'}H. \textsc{Piéron}, {\it et nous leur avons même fait quelques emprunts directs,
indiqués par le nom de leur auteur entre parenthèses.}

