
%%%%%%%%%%%%%%%%%%%%%
\chapter{Énoncés de croyance}
%%%%%%%%%%%%%%%%%%%%%

%%%%%%%%%%%%%%%%%%%%%%%%%
%\section{Encyclopédie de la philosophie}
%%%%%%%%%%%%%%%%%%%%%%%%%
% 350
%{\bf énoncés de croyance}
Énoncés de forme
« X croit A », sous-classe des énoncés dits
(depuis Bertrand Russell) d’{\it attitude
propositionnelle}
parce qu’ils expriment l’attitude
d’un sujet (X) à l'égard d’une
proposition (P) (« X pense que P », « X
désire que P », « X sait que P »). Déjà
George E. Moore et Ludwig Wittgenstein
avaient relevé une particularité sémantique
de ces énoncés : ils semblaient avoir
un comportement différent à la première
personne et aux autres personnes.
« Georges croit qu’il pleut, mais il ne
pleut pas » est parfaitement acceptable,
alors que « Je crois qu’il pleut, mais il ne
pleut pas » est perçu comme contradictoire
(c’est le « paradoxe de Moore »).
Les énoncés de croyance ont été au centre
de la discussion en philosophie du langage
à partir des années 1950, parce qu’ils semblaient
violer le principe de compositionalité
(Gottlob Frege) selon lequel la
signification d’un énoncé complexe
dépend de ses composants. Si la signification
d’un énoncé est identifiée à ses
conditions de vérité, comme dans la
sémantique modéliste, le principe n’est
pas respecté, parce qu’il n’est pas vrai, de
façon générale, que les conditions de
vérité de « X croit que P » dépendent systématiquement
de celles de « P ». Par
exemple, les deux énoncés ({\it a}) « Hugo
croit que Giorgione a été un grand peintre »
et ({\it b}) « Hugo croit que Giorgione
Barbarelli a été un grand peintre » n’ont
pas les mêmes conditions de vérité : ({\it a})
peut être vrai et ({\it b}) peut être faux, bien
que les deux énoncés aient les mêmes
conditions de vérité. La raison intuitive en
est, naturellement, que Hugo peut ne pas
savoir que Giorgione et Giorgione Barbarelli
sont la même personne. Différentes
solutions ont été proposées pour résoudre
la difficulté (par des philosophes comme
Rudolf Carnap, Willard Quine, Kaarlo
J. Hintikka, et par de nombreux autres) ;
mais aujourd’hui encore cette difficulté ne
semble pas pouvoir être résolue par une
sémantique d’orientation modéliste.

%%%%%%%%%%%%%%%%%%%%%%%%%%%%%%%%%%%%%%%%%%%%%%%%%%%%%%%%%%%%%%%%%%%%%%%%%%%%%%%%%%%%%
