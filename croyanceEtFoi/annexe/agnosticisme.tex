


\subsection{Agnosticisme}

{\footnotesize
\begin{itemize}[leftmargin=1cm, label=\ding{32}, itemsep=1pt]
\item {\bf \textsc{Étymologie} :} grec {\it agnôstos},
« inconnu ».
\item {\bf \textsc{Sens ordinaire} :} refus de se
prononcer sur des principes absolus.
\item {\bf \textsc{Philosophie} :} refus de se
prononcer sur l’existence ou la non-existence de Dieu.
\end{itemize}
}

Le terme agnosticisme est surtout utilisé
pour désigner une attitude face à la religion ;
mais il peut être employé dans un
sens plus général. Il est alors synonyme
de scepticisme radical.

{\footnotesize
\begin{itemize}[leftmargin=1cm, label=\ding{32}, itemsep=1pt]
\item {\bf \textsc{Termes voisins} :} scepticisme.
\item {\bf \textsc{Termes opposés} :} athéisme ;
croyance ; foi.
\item {\bf \textsc{Corrélats} :} Dieu ;
religion.
\end{itemize}
}

%%%%%%%%%%%%%%%%%%%%%%%%%%%%%%%%%%%%%%%%%%%%%%%%%%%%
