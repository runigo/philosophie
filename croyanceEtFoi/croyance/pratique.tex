%%%%%%%%%%%%%%%%%%%%%%%%%%%%%%%%%%%%
\section{Pratique de la philosophie}
%%%%%%%%%%%%%%%%%%%%%%%%%%%%%%%%%%%%

\subsection{Croyance}

{\footnotesize
\begin{itemize}[leftmargin=1cm, label=\ding{32}, itemsep=1pt]
\item {\bf \textsc{Étymologie} :} latin {\it credere}, « croire ».
\item {\bf \textsc{Sens ordinaire} :} disposition
 de l'esprit qui admet quelque
chose, qui adhère à une opinion,
une doctrine, une idéologie, etc.
\item {\bf \textsc{Philosophie} :} adhésion incertaine,
par opposition au savoir ou à la foi.
\end{itemize}
}

D'une façon générale, la croyance est
adhésion à une idée, une pensée, une
affirmation, une théorie, un dogme... En
ce sens, la naïveté, le préjugé, l'erreur,
la foi, l’opinion, aussi bien que le
savoir sont des modes différents de
croyance. Comme le montre Descartes
({\it Quatrième Méditation métaphysique}),
la croyance est un effet de la volonté :
l'entendement conçoit les idées, la
volonté y adhère, les refuse ou les met
en doute. Toutefois, la notion de
croyance est le plus souvent utilisée par
opposition au savoir et, dans une
moindre mesure, à la foi. La croyance
est alors surtout considérée comme une
adhésion plus où moins hasardeuse.

{\footnotesize
\begin{itemize}[leftmargin=1cm, label=\ding{32}, itemsep=1pt]
\item {\bf \textsc{Termes voisins} :} confiance ; foi.
\item {\bf \textsc{Termes opposés} :} agnosticisme ;
doute.
\item {\bf \textsc{Corrélats} :}  certitude ;
foi ; opinion.
\end{itemize}
}

%%%%%%%%%%%%%%%%%%%%%%%%%%%%%%%%%%%%%%%%%%%%%%%%%%%%

\subsection{Doute}

{\footnotesize
\begin{itemize}[leftmargin=1cm, label=\ding{32}, itemsep=1pt]
\item {\bf \textsc{Étymologie} :} latin {\it dubitare},
« balancer ».
\item {\bf \textsc{Sens ordinaire} :} état
d’esprit provenant d’une absence de certitude.
\item {\bf \textsc{Philosophie} :} attitude
réfléchie, volontaire et critique ; suspension
du jugement devant ce qui
se présente comme une vérité, afin
de l'examiner et d'en mettre à
l'épreuve le bien-fondé.
\item {\bf \textsc{Épistémologie} :} selon Claude Bernard,
qualité fondamentale de l’investigation
scientifique, qui vise à ne pas
prendre des conclusions momentanées
pour des vérités absolues.
\end{itemize}
}

D'un point de vue philosophique, il faut
distinguer deux sortes de doute: le
doute sceptique et le doute méthodique.

{\bf 1.} Le doute sceptique (cf. Scepticisme)
est une suspension radicale et définitive
du jugement. La pensée chrétienne, en
particulier avec Pascal, a repris certains
aspects de la tradition sceptique : en
mettant en évidence la faiblesse de notre
raison, le doute sceptique peut être aussi
un auxiliaire de la foi. À la suite de
Hume ({\footnotesize XVIII}$^\text{e}$ siècle),
le doute sceptique
devient plus modéré : il consiste moins
à suspendre son jugement qu’à ne pas
prendre nos croyances, mêmes les plus
crédibles, pour des certitudes, et à se
défendre contre l'enthousiasme des passions
et contre le dogmatisme.

{\bf 2.} Le doute méthodique est le point de
départ de la philosophie de Descartes.
S'il consiste dans le projet de faire table
rase de toutes les opinions que nous
avons reçues jusqu'ici comme étant
vraies, c'est en vue de trouver celles qui
leur résisteront. Le doute méthodique
diffère donc du doute sceptique parce
qu'il est un moyen en vue d’une fin, qui
est la certitude. Provisoire et délibéré, le
doute cartésien est également radical : il
révoque ce qui est simplement vraisemblable
et n’admet pas d'intermédiaire
entre le vrai et le faux. Il est, de ce fait,
hyperbolique, c'est-à-dire excessif. C’est
pourquoi, à la fin de la {\it Première Méditation
métaphysique}, Descartes avance
la fiction d'un « malin génie » qui lui
permet de se persuader que tout est
faux. Cette fiction a essentiellement un
rôle psychologique. En effet, les raisons
de douter sont logiquement suffisantes,
mais elles ne sont pas psychologiquement
assez parfaites pour maintenir
l'esprit dans sa résolution de douter. En
se persuadant, grâce au « malin génie »,
que tout est faux {\bf --} et non plus seulement
douteux {\bf --} cette résolution pourra
plus aisément se maintenir.

{\footnotesize
\begin{itemize}[leftmargin=1cm, label=\ding{32}, itemsep=1pt]
\item {\bf \textsc{Termes voisins} :} embarras ; hésitation ;
incertitude.
\item {\bf \textsc{Termes opposés} :} certitude.
\item {\bf \textsc{Corrélats} :} méthode ; philosophie ;
scepticisme ; vérité.
\end{itemize}
}

%%%%%%%%%%%%%%%%%%%%%%%%%%%%%%%%%%%%%%%%%%%%%%%%%%%%

\subsection{Certitude}

{\footnotesize
\begin{itemize}[leftmargin=1cm, label=\ding{32}, itemsep=1pt]
\item {\bf \textsc{Étymologie} :} latin {\it certitudo},
de {\it certus}, «assuré ».
\item {\bf \textsc{Sens ordinaire} :} état d'esprit
de celui qui est assuré de détenir une vérité.
\item {\bf \textsc{Philosophie} :} assurance intellectuelle
ou morale fondée sur les
conclusions d’une démonstration,
sur l'expérience, sur une évidence
ou sur une très grande probabilité.
\end{itemize}
}

Si la certitude tire sa force de la vérité,
elle ne se confond pas totalement avec
elle. La certitude réside en effet dans la
double assurance que l’on détient à la fois
la vérité et les critères qui nous garantissent
qu'il s’agit bien de la vérité. Son
caractère subjectif la rapproche de la
conviction : mais à celle-ci manquent
précisément les critères qui en fonderaient
à coup sûr la vérité. On ne peut
que persuader autrui de partager une
conviction. La certitude au contraire interdit
en principe le doute. Toutefois à côté
de la certitude de ce premier type, Descartes
par exemple admet la possibilité
de la certitude « morale », qui porte sur
« des choses dont nous n'avons point
coutume de douter touchant la conduite
de la vie, bien que nous sachions qu'il se
peut faire, absolument parlant, qu'elles
soient fausses » ({\it Principes de la
philosophie}, 205). Ce sens se rapproche d'un
usage courant du terme, qui distingue
assez peu la certitude de la conviction,
comme lorsqu'on dit avoir «la certitude
que telle personne viendra demain ».

{\footnotesize
\begin{itemize}[leftmargin=1cm, label=\ding{32}, itemsep=1pt]
\item {\bf \textsc{Termes voisins} :} conviction.
\item {\bf \textsc{Termes opposés} :} doute.
\item {\bf \textsc{Corrélats} :} démonstration ;
évidence ; preuve ; vérité.
\end{itemize}
}

%%%%%%%%%%%%%%%%%%%%%%%%%%%%%%%%%%%%%%%%%%%%%%%%%%%%
