
%%%%%%%%%%%%%%%%%%%%%
\section{Pratique de la philosophie}
%%%%%%%%%%%%%%%%%%%%%

\subsection{Croyance}

{\footnotesize
\begin{itemize}[leftmargin=1cm, label=\ding{32}, itemsep=1pt]
\item {\bf \textsc{Étymologie} :} latin {\it credere}, « croire ».
\item {\bf \textsc{Sens ordinaire} :} disposition
 de l'esprit qui admet quelque
chose, qui adhère à une opinion,
une doctrine, une idéologie, etc.
\item {\bf \textsc{Philosophie} :} adhésion incertaine,
par opposition au savoir ou à la foi.
\end{itemize}
}

D'une façon générale, la croyance est
adhésion à une idée, une pensée, une
affirmation, une théorie, un dogme... En
ce sens, la naïveté, le préjugé, l'erreur,
la foi, l’opinion, aussi bien que le
savoir sont des modes différents de
croyance. Comme le montre Descartes
({\it Quatrième Méditation métaphysique}),
la croyance est un effet de la volonté :
l'entendement conçoit les idées, la
volonté y adhère, les refuse ou les met
en doute. Toutefois, la notion de
croyance est le plus souvent utilisée par
opposition au savoir et, dans une
moindre mesure, à la foi. La croyance
est alors surtout considérée comme une
adhésion plus où moins hasardeuse.

{\footnotesize
\begin{itemize}[leftmargin=1cm, label=\ding{32}, itemsep=1pt]
\item {\bf \textsc{Termes voisins} :} confiance ; foi.
\item {\bf \textsc{Termes opposés} :} agnosticisme ;
doute.
\item {\bf \textsc{Corrélats} :}  certitude ;
foi ; opinion.
\end{itemize}
}

%%%%%%%%%%%%%%%%%%%%%%%%%%%%%%%%%%%%%%%%%%%%%%%%%

\subsection{Préjugé}

{\footnotesize
\begin{itemize}[leftmargin=1cm, label=\ding{32}, itemsep=1pt]
\item {\bf \textsc{Étymologie} :} latin {\it praejudicare},
« juger préalablement ».
\item {\bf \textsc{Sens ordinaire} :} Opinion admise sans
jugement ni raisonnement.
\end{itemize}
}

Le terme préjugé est souvent employé
dans un sens péjoratif, pour dénoncer
l'erreur ou au moins l'absence de
réflexion qui conduit un individu à
adhérer à une idée fausse {\bf --} dont il n’a
pas pris la peine de contrôler le bien-fondé {\bf --}
voire à la défendre contre des
idées justes, ou à condamner des individus
au nom de cette idée (par
exemple, les opinions racistes sont des
préjugés).

{\footnotesize
\begin{itemize}[leftmargin=1cm, label=\ding{32}, itemsep=1pt]
\item {\bf \textsc{Termes voisins} :} opinion.
\item {\bf \textsc{Termes opposés} :} savoir ; science.
\item {\bf \textsc{Corrélats} :} certitude ; croyance ;
dogme ; doute ; foi.
\end{itemize}
}

  
%%%%%%%%%%%%%%%%%%%%%%%%%%
%%%%%%%%%%%%%%%%%%%%%%%%%%%%%%%%%%%%%%%%%%%%%%%%%

\subsection{Erreur}

{\footnotesize
\begin{itemize}[leftmargin=1cm, label=\ding{32}, itemsep=1pt]
\item {\bf \textsc{Étymologie} :} latin {\it error}, « course
à l'aventure », de {\it errare}, «errer».
\item {\bf \textsc{Logique et sciences} :} affirmation
fausse, c'est-à-dire non conforme
aux règles de la logique, et/ou en
contradiction avec les données
expérimentales.
\item {\bf \textsc{Psychologie} :} état
de l'esprit qui tient pour vrai ce
qui est faux, et réciproquement
(ex. : « être dans l’erreur »).
\end{itemize}
}

L'erreur doit être soigneusement distinguée
aussi bien de la faute (qui engage
plus nettement notre responsabilité) que
de l’illusion (qui n’est pas vaincue par
le savoir). L'erreur procède toujours de
notre jugement : elle résulte, selon Descartes,
d’un décalage permanent entre
notre volonté, qui est infinie, et notre
entendement, qui ne l'est pas. Nous
nous trompons parce que nous outrepassons
nos possibilités intellectuelles,
par étourderie ou vanité : l'erreur n’est
donc qu'une privation de connaissance.
L'épistémologie contemporaine, au
contraire, donne à l’erreur un tout autre
statut, plus « positif ». Bachelard, notamment,
montre que les « vérités » scientifiques
ne sont jamais que provisoires,
qu'elles doivent constamment être remaniées
et corrigées. La connaissance
scientifique ne peut pas faire l'économie
de l’erreur.

{\footnotesize
\begin{itemize}[leftmargin=1cm, label=\ding{32}, itemsep=1pt]
\item {\bf \textsc{Termes voisins} :} fausseté ; illusion ;
incorrection.
\item {\bf \textsc{Termes opposés} :} vérité.
\item {\bf \textsc{Corrélats} :} connaissance ; Évidence ;
faute ; illusion ; jugement.
\end{itemize}
}

%%%%%%%%%%%%%%%%%%%%%%%%%%%%%%%%%%%%%%%%%%%%%%%%%

\subsection{Opinion}

{\footnotesize
\begin{itemize}[leftmargin=1cm, label=\ding{32}, itemsep=1pt]
\item {\bf \textsc{Étymologie} :} latin {\it opinari},
« émettre une opinion ».
\item {\bf \textsc{Sens ordinaire} :} avis,
jugement porté sur
un sujet, qui ne relève pas d'une
connaissance rationnelle vérifiable,
et dépend donc du système de
valeurs en fonction duquel on se
prononce.
\item {\bf \textsc{Philosophie} :} jugement
sans fondement rigoureux,
souvent dénoncé dans la mesure où
il se donne de façon abusive les
apparences d’un savoir.
\end{itemize}
}

L'interrogation sur la nature de la vérité
et les moyens de l’atteindre a conduit
nombre de philosophes à distinguer,
entre les différents types de connaissance
possibles, ceux qui conduisent effectivement
à la vérité, et ceux qui en éloignent.
En un premier sens, l’opinion est ainsi
traditionnellement considérée comme un
genre de connaissance peu fiable, fondée
sur des impressions, des sentiments, des
croyances où des jugements de valeur
subjectifs. Pour Spinoza, par exemple,
elle est forcément « sujette à l'erreur et n’a
jamais lieu à l'égard de quelque chose
dont nous sommes certains mais à l'égard
de ce que l’on dit conjecturer ou supposer »
({\it Court traité}, chap. II). Depuis Platon,
et jusque chez de nombreux penseurs
contemporains, l'opinion est
dénoncée comme a priori douteuse, illusoire
ou fausse, voire dangereuse, lorsqu’elle
cherche à s'imposer en dissimulant
la faiblesse de ses fondements sous
les apparences de la plus claire certitude.
Selon Adorno ({\it Modèles critiques}, 1963),
« l'opinion s’approprie ce que la connaissance
ne peut atteindre pour s’y substituer »,
elle rassure à bon compte, parce
qu’« elle offre des explications grâce auxquelles
on peut organiser sans contradiction
la réalité contradictoire ». Tel est bien
le « fonctionnement psychique » qui soustend,
par exemple, les opinions racistes :
pour être plus crédible, la peur de l’autre
prend le masque de l'affirmation de son
infériorité ou de la mise en garde contre
le danger qu'il est censé représenter. La
justesse de ces analyses ne doit pas faire
oublier qu'en un autre sens, l'opinion
constitue une forme de connaissance
utile, voire un type de jugements éminemment
respectables. Dans le {\it Ménon},
Platon reconnaît aux opinions droites la
faculté, sur les sujets qui ne relèvent ni de
la science ni de la simple conjecture,
d'éclairer l’action humaine. Dans le
domaine moral par exemple, à défaut de
vérités certaines, des intuitions justes
relatives au bien peuvent guider efficacement
l'éducation ou l’action, en leur
fixant pour but la satisfaction d'intérêts
conformes aux exigences de la réflexion,
et non à la soumission aux apparences ou
au plaisir immédiat. Enfin, sur toutes les
questions qui engagent des choix individuels
qu'aucune autorité ne peut légitimement
contraindre {\bf --} la religion, la
préférence politique, l'adhésion à une
conception du monde {\bf --} la liberté d’opinion
est un droit fondamental, dans les
sociétés démocratiques en tout cas, dès
l'instant où ceux auxquels elle est garantie
n'en usent pas au détriment de la
liberté d'autrui.

Analysée dans le {\it Traité
théologico-politique}, où Spinoza insiste
sur la nécessité d'une indépendance
absolue des opinions religieuses et de
leur expression par rapport à l'État, la
liberté d'opinion est proclamée dans la
Déclaration des droits de l'homme et du
citoyen de 1789. Et depuis près d'un
siècle, elle est au cœur du principe de la
laïcité qui garantit (en particulier en
France) la séparation entre l'Église et
l'État.

{\footnotesize
\begin{itemize}[leftmargin=1cm, label=\ding{32}, itemsep=1pt]
\item {\bf \textsc{Termes voisins} :} avis ; croyance.
\item {\bf \textsc{Termes opposés} :} science.
\end{itemize}
}

\subsubsection{Opinion publique}

Ensemble fluctuant de prises de positions
portant sur des questions politiques,
 morales, économiques... Les
« sondages d'opinion » prétendent en
constituer une sorte de baromètre.

{\footnotesize
\begin{itemize}[leftmargin=1cm, label=\ding{32}, itemsep=1pt]
\item {\bf \textsc{Corrélats} :} connaissance ;
conviction ; croyance ; doute ; foi ;
jugement ; préjugé.
\end{itemize}
}

%%%%%%%%%%%%%%%%%%%%%%%%%%{\it }{\bf --}
%%%%%%%%%%%%%%%%%%%%%%%%%%%%%%%%%%%%%%%%%%%%%%%%%

\subsection{Certitude}

{\footnotesize
\begin{itemize}[leftmargin=1cm, label=\ding{32}, itemsep=1pt]
\item {\bf \textsc{Étymologie} :} latin {\it certitudo},
de {\it certus}, «assuré ».
\item {\bf \textsc{Sens ordinaire} :} état d'esprit
de celui qui est assuré de détenir une vérité.
\item {\bf \textsc{Philosophie} :} assurance intellectuelle
ou morale fondée sur les
conclusions d’une démonstration,
sur l'expérience, sur une évidence
ou sur une très grande probabilité.
\end{itemize}
}

Si la certitude tire sa force de la vérité,
elle ne se confond pas totalement avec
elle. La certitude réside en effet dans la
double assurance que l’on détient à la fois
la vérité et les critères qui nous garantissent
qu'il s’agit bien de la vérité. Son
caractère subjectif la rapproche de la
conviction : mais à celle-ci manquent
précisément les critères qui en fonderaient
à coup sûr la vérité. On ne peut
que persuader autrui de partager une
conviction. La certitude au contraire interdit
en principe le doute. Toutefois à côté
de la certitude de ce premier type, Descartes
par exemple admet la possibilité
de la certitude « morale », qui porte sur
« des choses dont nous n'avons point
coutume de douter touchant la conduite
de la vie, bien que nous sachions qu'il se
peut faire, absolument parlant, qu'elles
soient fausses » ({\it Principes de la
philosophie}, 205). Ce sens se rapproche d'un
usage courant du terme, qui distingue
assez peu la certitude de la conviction,
comme lorsqu'on dit avoir «la certitude
que telle personne viendra demain ».

{\footnotesize
\begin{itemize}[leftmargin=1cm, label=\ding{32}, itemsep=1pt]
\item {\bf \textsc{Termes voisins} :} conviction.
\item {\bf \textsc{Termes opposés} :} doute.
\item {\bf \textsc{Corrélats} :} démonstration ;
évidence ; preuve ; vérité.
\end{itemize}
}

%%%%%%%%%%%%%%%%%%%%%%%%%%{\it }{\bf --}
%%%%%%%%%%%%%%%%%%%%%%%%%%%%%%%%%%%%%%%%%%%%%%%%%

\subsection{Doute}

{\footnotesize
\begin{itemize}[leftmargin=1cm, label=\ding{32}, itemsep=1pt]
\item {\bf \textsc{Étymologie} :} latin {\it dubitare},
« balancer ».
\item {\bf \textsc{Sens ordinaire} :} état
d’esprit provenant d’une absence de certitude.
\item {\bf \textsc{Philosophie} :} attitude
réfléchie, volontaire et critique ; suspension
du jugement devant ce qui
se présente comme une vérité, afin
de l'examiner et d'en mettre à
l'épreuve le bien-fondé.
\item {\bf \textsc{Épistémologie} :} selon Claude Bernard,
qualité fondamentale de l’investigation
scientifique, qui vise à ne pas
prendre des conclusions momentanées
pour des vérités absolues.
\end{itemize}
}

D'un point de vue philosophique, il faut
distinguer deux sortes de doute: le
doute sceptique et le doute méthodique.

{\bf 1.} Le doute sceptique (cf. Scepticisme)
est une suspension radicale et définitive
du jugement. La pensée chrétienne, en
particulier avec Pascal, a repris certains
aspects de la tradition sceptique : en
mettant en évidence la faiblesse de notre
raison, le doute sceptique peut être aussi
un auxiliaire de la foi. À la suite de
Hume ({\footnotesize XVIII}$^\text{e}$ siècle),
le doute sceptique
devient plus modéré : il consiste moins
à suspendre son jugement qu’à ne pas
prendre nos croyances, mêmes les plus
crédibles, pour des certitudes, et à se
défendre contre l'enthousiasme des passions
et contre le dogmatisme.

{\bf 2.} Le doute méthodique est le point de
départ de la philosophie de Descartes.
S'il consiste dans le projet de faire table
rase de toutes les opinions que nous
avons reçues jusqu'ici comme étant
vraies, c'est en vue de trouver celles qui
leur résisteront. Le doute méthodique
diffère donc du doute sceptique parce
qu'il est un moyen en vue d’une fin, qui
est la certitude. Provisoire et délibéré, le
doute cartésien est également radical : il
révoque ce qui est simplement vraisemblable
et n’admet pas d'intermédiaire
entre le vrai et le faux. Il est, de ce fait,
hyperbolique, c'est-à-dire excessif. C’est
pourquoi, à la fin de la {\it Première Méditation
métaphysique}, Descartes avance
la fiction d'un « malin génie » qui lui
permet de se persuader que tout est
faux. Cette fiction a essentiellement un
rôle psychologique. En effet, les raisons
de douter sont logiquement suffisantes,
mais elles ne sont pas psychologiquement
assez parfaites pour maintenir
l'esprit dans sa résolution de douter. En
se persuadant, grâce au « malin génie »,
que tout est faux {\bf --} et non plus seulement
douteux {\bf --} cette résolution pourra
plus aisément se maintenir.

{\footnotesize
\begin{itemize}[leftmargin=1cm, label=\ding{32}, itemsep=1pt]
\item {\bf \textsc{Termes voisins} :} embarras ; hésitation ;
incertitude.
\item {\bf \textsc{Termes opposés} :} certitude.
\item {\bf \textsc{Corrélats} :} méthode ; philosophie ;
scepticisme ; vérité.
\end{itemize}
}

%%%%%%%%%%%%%%%%%%%%%%%%%%
%%%%%%%%%%%%%%%%%%%%%%%%%%%%%%%%%%%%%%%%%%%%%%%%%

\subsection{Agnosticisme}

{\footnotesize
\begin{itemize}[leftmargin=1cm, label=\ding{32}, itemsep=1pt]
\item {\bf \textsc{Étymologie} :} grec {\it agnôstos},
« inconnu ».
\item {\bf \textsc{Sens ordinaire} :} refus de se
prononcer sur des principes absolus.
\item {\bf \textsc{Philosophie} :} refus de se
prononcer sur l’existence ou la non-existence de Dieu.
\end{itemize}
}

Le terme agnosticisme est surtout utilisé
pour désigner une attitude face à la religion ;
mais il peut être employé dans un
sens plus général. Il est alors synonyme
de scepticisme radical.

{\footnotesize
\begin{itemize}[leftmargin=1cm, label=\ding{32}, itemsep=1pt]
\item {\bf \textsc{Termes voisins} :} scepticisme.
\item {\bf \textsc{Termes opposés} :} athéisme ;
croyance ; foi.
\item {\bf \textsc{Corrélats} :} Dieu ;
religion.
\end{itemize}
}

%%%%%%%%%%%%%%%%%%%%%%%%%%
