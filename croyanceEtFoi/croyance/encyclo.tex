
%%%%%%%%%%%%%%%%%%%%%
\section{Encyclopédie de la philosophie}
%%%%%%%%%%%%%%%%%%%%%
% 349
\subsection{Croyance}
État subjectif qui s'oppose au
doute, mais qui reste inférieur à l’état de
certitude. Locke la définit comme la
forme la plus intense que peut avoir l’assentiment
concernant ce qui est considéré
comme seulement probable. Il s’agit, quoi
qu'il en soit, de la croyance naturelle, par
opposition à la croyance d'inspiration surnaturelle,
ou « foi ». Pour Platon, la
croyance ({\it pistis}) est le deuxième degré de
la connaissance. Elle a pour objet les
choses sensibles, c’est pourquoi elle est
située à un degré plus élevé que l’« imagination »,
mais elle appartient encore au
% 350
domaine de l’« opinion », par opposition
à la science, ou connaissance idéale, qui a
pour objet les vérités éternelles. Hume,
qui a reconnu à l'imagination un rôle fondamental
dans la vie psychique, oppose la
croyance à l'imagination. Pour Hume,
l'imagination est un mécanisme psychique
inconscient, un instinct naturel, qui investit
une « idée » avec une intensité ou une
vivacité tout à fait particulières : ainsi,
l'idée acquiert une intensité qui est initialement
celle des « impressions » et qui
distingue celles-ci des idées, comme l’original
de la copie. Mais c’est dans l’analyse
de la causalité que Hume fait intervenir la
croyance avec une valeur déterminante ;
l'expérience répétée de successions de
faits s’étant reproduits de façon uniforme
par le passé nous induit à penser, en présence
d’un fait déterminé, que se produira
le fait qui en avait découlé par le passé :
l'idée de ce fait que nous attendons se
présente avec une telle vigueur qu’elle
nous fait « croire » qu’il se produira. La
signification pratique de la croyance a été
particulièrement développée par le pragmatisme ;
la formulation extrême en est la
« volonté de croire » soutenue par William
James. D’après cette doctrine, hors
de toute considération de vérité ou d’erreur,
l’adhésion à une croyance est la
condition de la réalisation de celle-ci, qu’il
s'agisse des valeurs morales à mettre en
pratique dans la vie, ou de l’idée même
d’un but pour lequel la vie vaut la peine
d’être vécue. Notre vie ne peut avoir de
but que si nous « croyons » que la vie
mérite d’être vécue.

%%%%%%%%%%%%%%%%%%%%%%%%%%%%%
