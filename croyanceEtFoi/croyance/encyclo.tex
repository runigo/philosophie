
%%%%%%%%%%%%%%%%%%%%%
\section{Encyclopédie de la philosophie}
%%%%%%%%%%%%%%%%%%%%%
% 349
\subsection{Croyance}
État subjectif qui s'oppose au
doute, mais qui reste inférieur à l’état de
certitude. Locke la définit comme la
forme la plus intense que peut avoir l’assentiment
concernant ce qui est considéré
comme seulement probable. Il s’agit, quoi
qu'il en soit, de la croyance naturelle, par
opposition à la croyance d'inspiration surnaturelle,
ou « foi ». Pour Platon, la
croyance ({\it pistis}) est le deuxième degré de
la connaissance. Elle a pour objet les
choses sensibles, c’est pourquoi elle est
située à un degré plus élevé que l’« imagination »,
mais elle appartient encore au
% 350
domaine de l’« opinion », par opposition
à la science, ou connaissance idéale, qui a
pour objet les vérités éternelles. Hume,
qui a reconnu à l'imagination un rôle fondamental
dans la vie psychique, oppose la
croyance à l'imagination. Pour Hume,
l'imagination est un mécanisme psychique
inconscient, un instinct naturel, qui investit
une « idée » avec une intensité ou une
vivacité tout à fait particulières : ainsi,
l'idée acquiert une intensité qui est initialement
celle des « impressions » et qui
distingue celles-ci des idées, comme l’original
de la copie. Mais c’est dans l’analyse
de la causalité que Hume fait intervenir la
croyance avec une valeur déterminante ;
l'expérience répétée de successions de
faits s’étant reproduits de façon uniforme
par le passé nous induit à penser, en présence
d’un fait déterminé, que se produira
le fait qui en avait découlé par le passé :
l'idée de ce fait que nous attendons se
présente avec une telle vigueur qu’elle
nous fait « croire » qu’il se produira. La
signification pratique de la croyance a été
particulièrement développée par le pragmatisme ;
la formulation extrême en est la
« volonté de croire » soutenue par William
James. D’après cette doctrine, hors
de toute considération de vérité ou d’erreur,
l’adhésion à une croyance est la
condition de la réalisation de celle-ci, qu’il
s'agisse des valeurs morales à mettre en
pratique dans la vie, ou de l’idée même
d’un but pour lequel la vie vaut la peine
d’être vécue. Notre vie ne peut avoir de
but que si nous « croyons » que la vie
mérite d’être vécue.

%%%%%%%%%%%%%%%%%%%%%
\subsection{Doute}
%%%%%%%%%%%%%%%%%%%%%{\it }
%

Au sens courant, sentiment psychologique
et état d’incertitude que le sujet
éprouve lorsque ses convictions ou ses
croyances sont remises en cause par des
événements extérieurs. C’est un doute
subi, éprouvé par le sujet souvent contre
%
son gré. Dans l’usage philosophique, on
distingue le doute sceptique (Pyrrhon
d'Élis, Montaigne) et le doute méthodique
(Augustin, Bacon, Descartes). Le
doute sceptique est un acte volontaire par
lequel on suspend le jugement dans le
processus de connaissance tant que celui-ci
ne conduit à aucune connaissance assurée,
le but étant d'obtenir un état d’indifférence.
On définit habituellement
comme {\it absolu} le doute typique de l’attitude
sceptique : avec Pyrrhon déjà, le
scepticisme soutient qu’il est impossible
de parvenir à une quelconque connaissance
certaine, et qu’il est donc nécessaire
de douter de toute affirmation et de toute
théorie (Diogène Laërce, {\it Vies et doctrines
des philosophes illustres}, VIII, 103).
Divers auteurs, à commencer par saint
Augustin, opposent au doute absolu des
sceptiques un usage {\it méthodique} du doute.
Il est alors un moment préliminaire de la
connaissance, destiné à éliminer de l’investigation
les préjugés qui en entravent
le déroulement correct. Il est utilisé en ce
sens par F. Bacon dans sa théorie des
« idoles » (ou {\it pars destruens} de la
méthode), par laquelle il soumet à la critique
les préjugés les plus courants afin de
restaurer une approche directe des données
de l’expérience ({\it Novum Organum}, 1,
38-68). Le doute cartésien est un instrument
au service de la connaissance qui
consiste à rejeter comme faux tout ce en
quoi il est possible d’imaginer ou de supposer
le moindre doute. C’est grâce à ce
doute « méthodique » que Descartes
dégage le {\it cogito} comme première vérité
absolument indubitable et fondement de
toutes les autres ({\it Discours de la méthode},
IV$^\text{\,e}$ partie, {\it Méditations métaphysiques} I et
II). En effet, Descartes cherche à
atteindre, précisément à travers l’usage
systématique du doute, une évidence certaine
et indubitable, qui puisse être prise
comme point de départ et critère pour
toute vérité ultérieure. Après avoir exclu
les connaissances sensibles, puis les
connaissances rationnelles, et allant jusqu’à
admettre l’hypothèse d’un « mauvais
génie » qui éprouve du plaisir à induire
l’homme en erreur ({\it Méditations métaphysiques},
Première Méditation). Mais le fait
de douter est pourtant quelque chose et
ainsi Descartes peut-il conclure que si je
doute, je pense, et si je pense, je suis : telle
est l'évidence absolue qui était recherchée
(Seconde Méditation). En un sens, Husserl
% 422
reprend à son compte le doute radical
cartésien, puisqu'il soutient la nécessité
de « suspendre » la validité de toute théorie
ou jugement préconçu, afin de dégager
le terrain pour la pure description phénoménologique
des données de l’expérience.
Charles Sanders Peirce s’était au
contraire tourné vers une analyse de la
dimension psychologique et pragmatique
du doute. Il refuse le doute cartésien,
dans la mesure où personne ne peut
sérieusement douter de ce qu’il n’a pas de
raison valable de mettre en doute ou de
ce qu’il accueille comme une croyance
inconsciente. Descartes alla en effet jusqu’à
douter de l’existence de Dieu, mais
non du fait, allant pour lui de soi, que le
mode d’être de la vérité est l’évidence
claire et distincte, et que celui de la réalité
est la substance ({\it Comment rendre nos
idées claires}, \S 1). Le doute, pour Peirce,
est une interruption de l’action qui
advient quand nos croyances sont remises
en cause par l'expérience; doute et
croyance sont deux moments, étroitement
liés entre eux, de l'expérience de la vérité,
qui n’est jamais abstraite, mais vérification
concrète, et toujours faillible, des
conséquences pratiques découlant de nos
habitus et comportements ({\it L'Établissement
de la croyance}, \S 3-4).

%—> épochè + pragmatisme + scepticisme

%%%%%%%%%%%%%%%%%%%%%
\subsection{Certitude}
%%%%%%%%%%%%%%%%%%%%%{\it }
%
État de conviction subjective
considéré en général comme l'effet d’une
évidence. A côté de la notion subjective de
certitude, que l’on trouve dans l’emploi
commun du terme et qui le fait coïncider
en partie avec l’assentiment, on peut lire
chez les différents philosophes l’analyse
des données objectives en vertu desquelles
une vérité peut être dite « certaine ». Sous
ce deuxième aspect, l’histoire du concept
de certitude suit les avatars de celui de
« vérité » et des critères adoptés pour
l’évaluer. Dans le monde antique, c’est
l'analyse des aspects objectifs du concept
qui prévaut : pour Platon et pour Aristote,
la certitude s’identifie à la stabilité, puisqu’il
n’est de connaissance certaine que
des choses stables, alors que le mouvant
donne lieu à une connaissance probable
(Platon, {\it Philèbe}, 59b ; {\it Timée}, 29b-c).
Aristote, en particulier, en donna une
formulation logique, en affirmant que c’est
%
seulement dans le raisonnement apodictique,
dont la négation est impensable ou
contradictoire, qu’existe la garantie d’une
persuasion objectivement fondée. En cela,
Aristote s’opposait à la thèse des sophistes
selon laquelle toute certitude étant produite
par les modalités formelles du discours,
elle serait liée à l’habileté rhétorique
de celui qui l’énonce.

Les deux aspects, celui, subjectif, de
l'assentiment et celui, objectif, du critère
de la vérité, restent complémentaires jusqu’au
Moyen Age, lorsque l’insistance sur
la foi comme adhésion à la vérité révélée
et comme soumission à l'autorité des
Pères qui l’ont interprétée détermine aux
débuts de la scolastique la prédominance
de l'attention portée à l’aspect subjectif.
Thomas d'Aquin établit une distinction
entre la certitude des vérités de la foi,
dans lesquelles prédomine l’élément subjectif,
c’est-à-dire la volonté d’accepter
l’autorité du Dieu de la Révélation ; et la
certitude sur les vérités de la raison fondée
sur l’évidence ({\it Summa theologica}, 
II-IIae, quest. 2, art. 1 ; et {\it In Sententiarum},
dist. 23, quest. 2, art. 2, quest. Ia, 2c).

A l’Âge classique, Descartes, avec la
règle de l’évidence, identifia vérité et certitude
et unifia les aspects subjectifs et
objectifs du concept : l’évidence signifie à
la fois la clarté et la distinction des idées
et l’assentiment subjectif à celles-ci.
L'identité cartésienne de la « vérité » et
de la « certitude » demeure présente chez
Locke, qui y ajoute la distinction entre la
« certitude de la vérité », c’est-à-dire
l’adéquation entre les mots et les idées, et
la « certitude de la connaissance », c’est-à-dire
l’accord ou la discordance entre les
seules idées ({\it Essai sur l’entendement
humain}, livre quatre, VI, 3). On retrouve
cette identité de la vérité et de la certitude
chez Leibniz, qui introduit le concept de
« certitude morale » que l’on atteint grâce
aux preuves des vérités religieuses ({\it Essais
de théodicée}, {\it Discours de la conformité de
la foi avec la raison}, \S 5). Mais cette identité
est critiquée par Vico, pour qui le
« vrai » s’identifie au « fait », alors que la
« conscience du certain » est donnée par
« l'autorité du jugement humain » ({\it La
Science nouvelle}, dignité X). Pour Kant,
« certitude » et « conviction » expriment
respectivement l’aspect objectif et l’aspect
subjectif de la « science » entendue
comme croyance « suffisamment » garantie
comme vraie ({\it Critique de la raison
%
pure}, Doctrine transcendantale de la
méthode, chap. II, section II). Pour
Hegel, dans {\it La Phénoménologie de l’esprit},
la certitude sensible apparaît seulement
comme le savoir le plus immédiat,
qui n’atteste toutefois que la présence
d’un Moi singulier face à une chose singulière,
sans qu’intervienne aucune
conscience subjective. Mais, à partir de la
certitude sensible, naît la {\it différence} entre
le Moi et la chose, et donc la perception
que le « pur rapport immédiat » entre l’un
et l’autre implique en tout cas une médiation
({\it La Phénoménologie de l'esprit}, A, 1).

Le concept de « certitude scientifique »,
apparu au {\footnotesize XVII}$^\text{\,e}$ s.
en étroite relation avec le
principe de la vérification scientifique, a
subi dans la pensée scientifique contemporaine
de sensibles mutations. C’est l’expérience,
ou si l’on préfère l’expérimentation,
qui permet de saisir dans une technique
opératoire l’unité de la certitude issue du
constat de la perception sensible et la certitude
induite par la formalisation logico-mathématique
d’une théorie. En excluant
de son champ de recherche la prétention de
parvenir à une certitude absolue et universelle,
une grande partie des épistémologues
considère les critères de la vérité comme
des paramètres changeants, relatifs au système
choisi pour les mettre en œuvre. De
ce relativisme de la « vérité » découle le
concept d’une certitude conventionnellement
déterminée, qui n’est plus marquée
par des traits psychologiques ou subjectifs
(dont la pensée contemporaine, dans le sillage
de Nietzsche et de Freud, a appris à
douter) ni par des traits universellement
objectifs, mais qui varie avec les critères
formels adoptés pour l’atteindre.

% --> évidence + vérification (principe
%de) e vérité

 

%%%%%%%%%%%%%%%%%%%%%%%%%%%%%
