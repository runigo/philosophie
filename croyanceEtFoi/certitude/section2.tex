
%%%%%%%%%%%%%%%%%%%%%
\section{Titre de la section 2}
%%%%%%%%%%%%%%%%%%%%%

 
\subsection{Foi}

%{\bf }{\it }{\bf --}{\footnotesize X}$^\text{e}$

\begin{itemize}[leftmargin=1cm, label=\ding{32}, itemsep=1pt]
\item {\bf \textsc{Étymologie} :} latin {\it },.
\item {\bf \textsc{Sens ordinaire} :} .
\item {\bf \textsc{Théologie} :} .
\end{itemize}

\begin{itemize}[leftmargin=1cm, label=\ding{32}, itemsep=1pt]
\item {\bf \textsc{Terme voisin} :} .
\item {\bf \textsc{Terme opposé} :} .
\item {\bf \textsc{Corrélats} :} .
\end{itemize}


CERTITUDE

(n. f.) @ ÉryM.: latin certitudo, de
certus, «assuré ». @ SENS ORDINAI-
RE: état d'esprit de celui qui est
assuré de détenir une vérité.
© PHILOSOPHIE : assurance intellec-
tuelle ou morale fondée sur les
conclusions d’une démonstration,
sur l'expérience, sur une évidence
ou sur une très grande probabilité.

Si la certitude tire sa force de la vérité”,
elle ne se confond pas totalement avec
elle. La certitude réside en effet dans la
double assurance que l’on détient à la fois
la vérité et les critères qui nous garan-
tissent qu'il s’agit bien de la vérité. Son
caractère subjectif la rapproche de la
conviction* : mais à celle-ci manquent
précisément les critères qui en fonde-
raient à coup sûr la vérité. On ne peut
que persuader autrui de partager une
conviction. La certitude au contraire inter-
dit en principe le doute*. Toutefois à côté
de la certitude de ce premier type, Des-
cartes* par exemple admet la possibilité
de la certitude «morale», qui porte sur
«des choses dont nous n'avons point
coutume de douter touchant la conduite
de la vie, bien que nous sachions qu'il se
peut faire, absolument parlant, qu'elles
soient fausses » (Principes de la philoso-
phie, 205). Ce sens se rapproche d'un
usage courant du terme, qui distingue
assez peu la certitude de la conviction,
comme lorsqu'on dit avoir «la certitude
que telle personne viendra demain ».

© TERMES VOISINS: Conviction ;

| vérité. @ TERME OPPOSÉ: doute.
© CORRÉLATS : démonstration ; Évi-
dence ; preuve ; vérité.

 
%%%%%%%%%%%%%%%%%%%%%%%%%%%%%%%%%%%%%%%%%%%%%%%%%%%%%%%%%%%%%%%%%%%%%%%%%%%
