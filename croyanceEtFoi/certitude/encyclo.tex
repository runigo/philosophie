
%%%%%%%%%%%%%%%%%%%%%
\section{Certitude}
%%%%%%%%%%%%%%%%%%%%%{\it }
%
{\bf évidence}
certitude état de conviction subjective
considéré en général comme l'effet d’une
évidence. A côté de la notion subjective de
certitude, que l’on trouve dans l’emploi
commun du terme et qui le fait coïncider
en partie avec l’assentiment, on peut lire
chez les différents philosophes l’analyse
des données objectives en vertu desquelles
une vérité peut être dite « certaine ». Sous
ce deuxième aspect, l’histoire du concept
de certitude suit les avatars de celui de
« vérité » et des critères adoptés pour
l’évaluer. Dans le monde antique, c’est
l'analyse des aspects objectifs du concept
qui prévaut : pour Platon et pour Aristote,
la certitude s’identifie à la stabilité, puis-
qu’il n’est de connaissance certaine que
des choses stables, alors que le mouvant
donne lieu à une connaissance probable
(Platon, {\it Philèbe}, 59b ; {\it Timée}, 29b-c).
Aristote, en particulier, en donna une
formulation logique, en affirmant que c’est
%
seulement dans le raisonnement apodictique,
dont la négation est impensable ou
contradictoire, qu’existe la garantie d’une
persuasion objectivement fondée. En cela,
Aristote s’opposait à la thèse des sophistes
selon laquelle toute certitude étant produite
par les modalités formelles du discours,
elle serait liée à l’habileté rhétorique
de celui qui l’énonce.

Les deux aspects, celui, subjectif, de
lPassentiment et celui, objectif, du critère
de la vérité, restent complémentaires jusqu’au
Moyen Age, lorsque l’insistance sur
la foi comme adhésion à la vérité révélée
et comme soumission à l'autorité des
Pères qui l’ont interprétée détermine aux
débuts de la scolastique la prédominance
de l'attention portée à l’aspect subjectif.
Thomas d'Aquin établit une distinction
entre la certitude des vérités de la foi,
dans lesquelles prédomine l’élément subjectif,
c’est-à-dire la volonté d’accepter
l’autorité du Dieu de la Révélation ; et la
certitude sur les vérités de la raison fondée
sur l’évidence ({\it Summa theologica}, 
II-IIae, quest. 2, art. 1 ; et {\it }In Sententiarum,
dist. 23, quest. 2, art. 2, quest. Ia, 2c).

A l’Age classique, Descartes, avec la
règle de l’évidence, identifia vérité et certitude
et unifia les aspects subjectifs et
objectifs du concept : l’évidence signifie à
la fois la clarté et la distinction des idées
et l’assentiment subjectif à celles-ci.
L'identité cartésienne de la « vérité » et
de la « certitude » demeure présente chez
Locke, qui y ajoute la distinction entre la
« certitude de la vérité », c’est-à-dire
l’adéquation entre les mots et les idées, et
la « certitude de la connaissance », c’est-à-dire
l’accord ou la discordance entre les
seules idées ({\it Essai sur l’entendement
humain}, livre quatre, VI, 3). On retrouve
cette identité de la vérité et de la certitude
chez Leibniz, qui introduit le concept de
« certitude morale » que l’on atteint grâce
aux preuves des vérités religieuses ({\it Essais
de théodicée}, {\it Discours de la conformité de
la foi avec la raison}, \S 5). Mais cette identité
est critiquée par Vico, pour qui le
« vrai » s’identifie au « fait », alors que la
« conscience du certain » est donnée par
« l'autorité du jugement humain » ({\it La
Science nouvelle}, dignité X). Pour Kant,
« certitude » et « conviction » expriment
respectivement l’aspect objectif et l’aspect
subjectif de la « science » entendue
comme croyance « suffisamment » garantie
comme vraie ({\it Critique de la raison
%
pure}, Doctrine transcendantale de la
méthode, chap. II, section II). Pour
Hegel, dans {\it La Phénoménologie de l’esprit},
la certitude sensible apparaît seulement
comme le savoir le plus immédiat,
qui n’atteste toutefois que la présence
d’un Moi singulier face à une chose singulière,
sans qu’intervienne aucune
conscience subjective. Mais, à partir de la
certitude sensible, naît la {\it différence} entre
le Moi et la chose, et donc la perception
que le « pur rapport immédiat » entre l’un
et l’autre implique en tout cas une médiation
({\it La Phénoménologie de l'esprit}, À, 1).

Le concept de « certitude scientifique »,
apparu au {\footnotesize XVII}$^\text{e}$ s. en étroite relation avec le
principe de la vérification scientifique, a
subi dans la pensée scientifique contemporaine
de sensibles mutations. C’est l’expérience,
ou si l’on préfère l’expérimentation,
qui permet de saisir dans une technique
opératoire l’unité de la certitude issue du
constat de la perception sensible et la certitude
induite par la formalisation logico-mathématique
d’une théorie. En excluant
de son champ de recherche la prétention de
parvenir à une certitude absolue et universelle,
une grande partie des épistémologues
considère les critères de la vérité comme
des paramètres changeants, relatifs au système
choisi pour les mettre en œuvre. De
ce relativisme de la « vérité » découle le
concept d’une certitude conventionnellement
déterminée, qui n’est plus marquée
par des traits psychologiques ou subjectifs
(dont la pensée contemporaine, dans le sillage
de Nietzsche et de Freud, a appris à
douter) ni par des traits universellement
objectifs, mais qui varie avec les critères
formels adoptés pour l’atteindre.

% --> évidence + vérification (principe
%de) e vérité

 

 

%%%%%%%%%%%%%%%%%%%%%%%%%%%%%%%%%%%%%%%%%%%%%%%%%%%%%%%%%%%%%%%%%%%%%%%%
