\subsection{Opinion}
 — \si{Psycho.} et \si{Crit.} {\bf 1.} \fsb{S. subje.}
Assentiment* partiel ; croyance, au sens 1 : « Quand le pénitent suit une
opinion probable, le confesseur le doit absoudre » (Pascal, {\it Prov.}, 5).
Spéc., {\it chez Platon} (grec {\it doxa}) : type de connaissance inférieur à
la science et à la pensée discursive et qui comprend la croyance ({\it
pistis}) et la pensée par images ({\it eïkasia}) : « Ce qu'est l'être au
devenir, ainsi est la connaissance intellectuelle ({\it noêsis}) à l'opinion
» ({\it République}, VI).

 — \si{Psycho.} et \si{Soc.} \fsb{S. abstr.} {\bf 2.} Type de
% 131
pensée sociale qui consiste à prendre position, plus ou moins fermement, sur
les problèmes politiques, moraux, philosophiques, religieux : « L'opinion
fait des hommes ce qu'elle veut » (Lacombe) ; « Les valeurs sont choses
d'opinion » (Durkheim) ; « Il existe deux formes de l'opinion, l'opinion
publique et l'opinion privée. La première est d'ordre sociologique: ... la
seconde, d'ordre psychologique », toutefois même celle-ci « répond à une
question sociale, est elle-même une réponse sociale » (Stœtzel). Cf. {\it
Public}$^2$.

 — \fsb{S. concr.} {\bf 3.} Objet de l'opinion$^2$ : « L'opinion
est un groupe plus ou moins logique de jugements qui, répondant à des
problèmes actuellement posés, se trouvent reproduits en nombreux exemplaires
dans des personnes du même pays, du même temps, de la même société
» (Tarde) ; « Ainsi se vont les opinions, succédant du pour au contre
» (Pascal, 337) ; « Tout le mécanisme social repose sur des opinions
» (Comte, {\it Cours}, I).

