\subsection{Foi}
 — \si{Vulg.} {\bf 1.} Garantie : « Sur la
foi des traités ». — {\bf 2.} Fidélité à un
%81
engagement : « C’est parce que nous sommes civilisés que nous nous
imposons le respect de la foi que nous avons jurée » (Davy). {\it Bonne
foi} : sincérité. {\it Mauvaise foi} : duplicité$^2$ ; {\it spéc.},
{\it chez Sartre} : attitude de la conscience qui se masque à elle-même la
vérité, mensonge à soi-même. — {\bf 3.} Confiance : « Quoiqu’à leur nation
[les voleurs] bien peu de foi soit due,... » (Molière).
— {\bf 4.} Syn. de {\it croyance}$^2$ : « Ajouter foi à... ». {\it Chez
Kant} : « foi morale », croyance rationnelle, quoique non
démontrable, à la liberté, à l’existence de Dieu et à la vie future.

— \si{Théol.} {\bf 5.} \fsb{S. abstr.} Adhésion aux
dogmes d’une Église, à des vérités considérées comme révélées : « La
foi est différente de la preuve : l’une est humaine, l’autre est un don de
Dieu » (Pascal, 248). — {\bf 6.} \fsb{S. concr.} Objet
de la foi$^5$, les dogmes : « Je ne croirai jamais que la vraie philosophie
soit opposée à la foi » (Malebranche, {\it Entr.}, VI, 2).
\subsubsection{Duplicité}
 — \si{Méta.} {\bf 1.} Au sens étymologique : caractère double, dualité :
« Cette duplicité de l’homme est si
visible qu’il y en a qui ont pensé
que nous avions deux âmes » (Pascal,
417); la « duplicité de l'obligation »
[en tant que relation entre celui qui
oblige et celui qui est obligé] (Le
Senne).

— \si{Car.} et \si{Mor.} {\bf 2.} {\it Péj.} Manque de
sincérité : « Ils ne servent qu’à nous
montrer la duplicité de votre cœur »
(Pascal, {\it Prov.}, 13).
