
( \bi{Note de la numérisation :} Les abréviations utilisées dans le {\it mouveau vocabulaire philosophique} sont intégralement reproduites en annexe de ce document. Ce paragraphe en est un résumé suffisant pour ce chapitre. )

\vspace{0.211cm}
1° les abréviations suivantes indiquent les
disciplines au langage desquelles le mot est emprunté :

\vspace{0.211cm}
\hfill
\begin{minipage}[c]{.45\linewidth}
\si{Car.} — Caractérologie, psychologie
des caractères.

\si{Crit.} — Critique ou théorie de la
connaissance.

\si{Épist.} — Épistémologie.

\si{Hist.} — Histoire de la philosophie.

\si{Log.} — Logique.

\si{Mor.} — Morale.
\end{minipage}
\hfill
\begin{minipage}[c]{.45\linewidth}
\si{Méta.} — Métaphysique, philosophie
générale.

\si{Pol.}. — Politique.

\si{Psycho.} — Psychologie.

\si{Ps. path.} — Psychologie pathologique.

\si{Soc.} — Sociologie.

\si{Théol.} — Théologie,

\si{Vulg.} — Sens vulgaire, courant.
\end{minipage}

\vspace{0.211cm}
2° Les chiffres en caractères gras ({\bf 1, 2}) distinguent les différentes acceptions
du mot ;

3° Le signe * indique les mots définis à leur ordre alphabétique et auxquels
il y a lieu de se reporter pour plus complète explication; lorsque ces mots présentent plusieurs acceptions, l'étoile est remplacée par un chiffre mis en exposant (ex. : {\it absolu}$^2$) qui détermine le sens qu'il convient de choisir ;

4° Les termes contraires (Ctr.), opposés (Opp.) ou synonymes (Syn.) sont
indiqués eutre parenthèses ;

5° Le signe $->$
signale les impropriétés, confusions, incorrections, le plus
souvent commises et contre lesquelles on doit se tenir en garde;

6° Les abréviations suivantes indiquent certaines nuances de sens :

\vspace{0.211cm}
\hfill
\begin{minipage}[c]{.45\linewidth}
\fsb{S. abstr.} — Sens abstrait

\fsb{S. subje.} — Sens subjectif

\end{minipage}
\hfill
\begin{minipage}[c]{.45\linewidth}
\fsb{S. concr.} — Sens concret

\fsb{S. objec.} — Sens objectif

\end{minipage}


\vspace{0.211cm}

\subsection {Autres abréviations}

\begin{minipage}[c]{.45\linewidth}
{\it Adj.} — Adjectif.

{\it Cf.} — Se reporter à.

{\it Dist.} — Distinguer (de), ne pas confondre (avec).

{\it Péj.} — Avec un sens péjoratif.

{\it Qqfs.} — Quelquefois.
\end{minipage}
\hfill
\begin{minipage}[c]{.45\linewidth}
{\it Gén.} — Généralement, en général.

{\it Lato.} — Au sens large.

{\it Not.} — Notamment.

{\it Opp.} — Par opposition à.

{\it Spéc.} — Spécialement.

\end{minipage}

