\subsection{Pratique (adj.)}
 — {\bf 1.} ({\it Opp.} : {\it théorique, spéculatif}).
Qui concerne l’action$^3$ utilitaire : « On en peut trouver une [philosophie]
pratique, par laquelle... nous pourrions nous rendre comme maîtres et
possesseurs de la nature » (Descartes, {\it Méth.}, VI). — {\bf 2.}
({\it Opp.}: {\it pragmatique}*). Qui concerne l’action
% 144
{\it morale}. Spéc., {\it chez Kant} : « La Raison pratique » ; « La règle
pratique est inconditionnée, donc représentée {\it a priori} comme
proposition catégoriquement pratique » ({\it R. pr.}, I, 1, 1, § 7). Cf.
{\it Théorétique}$^2$.

\subsubsection{Pragmatique}
 — \si{Hist.} {\it Chez Kant} (opp. {\it pratique}$^2$) : qui
concerne l'action utililaire, qui vise le succès ou le bien-être.

