\subsection{Doute}
 — \si{Psycho.} {\bf 1.} (Opp. :
{\it assentiment} ou {\it croyance}$^3$). État de l'esprit
qui suspend son assentiment*.
$->$ {\it Dist.} opinion$^2$ — \si{Ps. path.}
 {\bf 2.} {\it Folie du doute} : incapacité de
croire$^3$ (de donner son assentiment*)
ou de prendre des décisions*.

— \si{Hist.} {\bf 2.} {\it Chez Descartes} : « doute
méthodique », méthode philosophique qui consiste à révoquer en
doute tout ce qu’on a admis antérieurement et à n’accepter pour vrai
que ce qui est évident, afin de fonder
la connaissance sur des bases certaines : « Je pensai qu'il fallait que
je rejetasse comme absolument faux
tout ce en quoi je pourrais imaginer
le moindre doute » ({\it Méth.}, IV). Cf.
Husserl, {\it Médit. cartésiennes}, introd. :
« Ne connaissant d’autre but que
celui d’une connaissance absolue,
il [Descartes] s’interdit d'admettre
comme existant ce qui n’est pas à
l’abri de toute possibilité d’être mis
en doute ». — {\bf 4.} {\it Doute scientifique} :
attitude du savant qui révoque en
doute ses hypothèses$^2$ tant qu’elles
ne sont pas confirmées par l’expérience* : « Le grand principe expérimental est le doute philosophique
% 59
qui laisse à l’esprit sa liberté et son
initiative » (Claude Bernard).

\subsubsection{Décision}
 — \si{Psycho.} (Syn. : {\it choix$^1$,
détermination$^3$, résolution}$^2$). Phase
terminale qui, selon la description
classique de l’acte volontaire, succède à la délibération*. Cf. {\it Volition}*.

\subsubsection{Volition}
 — \si{Psycho.} \fsb{S. concr.} Acte de volonté$^1$ : « Vouloir,
c’est agir : la volition est un passage à l'acte » (Ribot).

