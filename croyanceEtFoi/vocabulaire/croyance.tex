\subsection{Croire, Croyance}

 — \si{Psycho.} \fsb{S. subje.} Ces
termes peuvent s’appliquer : {\bf 1.} à
une opinion$^1$ fondée sur une simple
probabilité$^1$ : « Je ne croyais pas que
tout fût perdu » (Sévigné); « Deux
sortes d'hommes : les uns justes qui
se croient pécheurs, les autres pécheurs qui se croient justes » (Pascal,
534); en ce sens, qqfs. opp. à {\it savoir} :
« Nous ne pouvons pas croire ce que
nous savons, et nous ne pouvons
pas savoir ce que nous croyons »
(Pradines); — {\bf 2.} (syn. : {\it foi}$^4$) à une
certitude$^1$ qui ne résulte pas uniquement d'une démonstration rationnelle, soit qu’elle se fonde sur l’autorité$^2$ et le témoignage, soit qu'elle
repose sur des motifs affectifs (sentiments) et actifs (aspirations, inclinations, désirs) ou qu'elle relève des
exigences de la « raison pratique$^2$ »,
soit enfin ({\it foi$^5$ religieuse}) qu’elle
dépasse la raison : « Elle croit, elle
qui jugeait la foi impossible »
(Bossuet); « Il me fallut abolir le
savoir [{\it Wissen} ] afin d'obtenir une
place pour la croyance [{\it Glauben} ] »
(Kant, {\it R. pure}, préf. 2$^\text{e}$ éd.); « Une
religion est d’autant plus crue qu’elle
suscite davantage les sentiments
profonds » (Delacroix); « On {\it croit} en
Dieu plus qu’on ne le {\it prouve} » (Le
% 49
Roy); — {\bf 3.} {\it lato.} : à l'{\it assentiment}* en
 {\it gén.} : « Nier, croire et douter bien
sont à l’homme ce que courir est au
cheval » (Pascal, 259); « Toute aperception$^2$ suppose affirmation implicite, {\it au sens de croyance}, même si
elle était unique, simple... Si elle
est multiple, elle est {\it croyance} à la
liaison de ses parties » (Lagneau);
« La croyance est un genre dont
la certitude$^2$ est une espèce » (Brochard).

— {\bf 4.} \fsb{S. objec.} Objet de la croyance aux
sens 1, 2 ou 3 : « Les croyances religieuses » ; « La croyance à la liberté ».

