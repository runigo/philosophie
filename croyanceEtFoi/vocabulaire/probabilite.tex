\subsection{Probabilité}
 — \si{Épist.} {\bf 1.} \fsb{S. subje.} Caractère de ce qui
nous parait vraisemblable, de ce qui nous semble devoir se réaliser de
préférence à d’autres possibles ou avoir le plus de chances d’être vrai,
sans cependant qu’on puisse le prouver ; en ce sens, la probabilité
caractérise l’{\it opinion}$^1$ : « La probabilité, comme toute autre
modalité de la pensée, est un caractère essentiellement subjectif de nos
jugements » (Couturat).

 — {\bf 2.} \fsb{S. objec.} (Sens mathématique). « La
probabilité est le rapport du nombre des
% 148
cas favorables au nombre total des événements » (Borel). {\it Calcul des
probabilités} : règles à l'aide desquelles on calcule la probabilité$^2$
d’un événement futur. {\it Lois de probabilité} : les lois statistiques$^2$
(cf. {\it Probabilisme}$^2$) : « La nouvelle Physique ne nous fournit que
des lois de probabilité » (L. de Broglie).

