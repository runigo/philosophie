
%%%%%%%%%%%%%%%%%%%%%
\section{Encyclopédie de la philosophie}
%%%%%%%%%%%%%%%%%%%%%
%{\bf }{\bf --}{\it }{\footnotesize X}$^\text{e}$

\subsection{Vérité}

Terme fondamental, non seulement
dans le discours philosophique qui
cherche plus particulièrement à en trouver
le sens, mais aussi dans le langage
courant et scientifique. La question qui se
pose est : que veut dire qualifier quelque
chose de « vrai » (ou de « faux ») ? Une
question plus importante encore se pose
à propos de la vérité : à quelles conditions
est-il justifié de qualifier quelque chose de
vrai (ou de faux) ? Cette deuxième question,
traditionnellement appelée question
du critère (ou des critères) de la vérité,
n’en fait qu’une avec celle de la connaissance.
Ce problème a été envisagé sous de
nombreux points de vue et les solutions
apportées occupent une des places les
plus importantes dans l’histoire de la pensée
occidentale. On se limitera ici à la
question du sens de la notion de vérité en
renvoyant pour le reste aux entrées sur les
philosophes, ainsi qu’aux entrées générales
« gnoséologie » et « science (philosophie de la) ».

\subsection{Vérité de la proposition
et vérité ontologique}

Historiquement, la recherche sur le
sens de la notion de vérité remonte à
Aristote ({\it Métaphysique} et début du {\it De
interpretatione}). Il limite l’application sensée
de la qualification de « vrai » et de
« faux » au discours apophantique (déclaratif),
autrement dit aux propositions
affirmatives et négatives, en l’excluant
pour les éléments du discours isolés les
uns des autres (comme un nom ou un
verbe seul) comme pour les discours qui
ne sont pas apophantiques (les prières ou
les ordres, par exemple). L’alternative
vrai ou faux s’applique donc seulement à
une partie de ce qui est porteur de sens
(puisque les éléments du discours pris de
façon isolée comme les discours qui ne
sont pas apophantiques peuvent eux aussi
être porteurs de sens). Le caractère signifiant
reste une condition préalable nécessaire
pour que cette alternative soit
pertinente (on trouve déjà de façon implicite
l’idée que, pour pouvoir être fausse,
une proposition doit être porteuse de sens
et qu’elle ne peut être fausse si elle n’est
pas porteuse de sens).

Cette façon de considérer la vérité
comme une éventuelle propriété des propositions
a traversé l’ensemble de la pensée
occidentale. On a cependant contesté
qu’elle soit la seule légitime. Dans la tradition
d’inspiration platonicienne (même
si elle n’est pas explicitement présente
chez Platon), on trouve cette acception
ontologique de la notion de vérité
comprise comme totalement indépendante
de la pensée et des discours des
hommes, c’est le cas de l’Étre « vrai » ou
de la Réalité « vraie » (dans le sens où le
langage commun parle de l’or véritable
par opposition à ce qui n’en a que l’aspect
ou dit de quelqu'un qu’il est un vrai
héros). Dans ce sens, « vrai » ne s’oppose
pas à ce qui est faux ou erroné mais à ce
qui n’est qu'apparent (comme les rêves
contre les perceptions de l’état de veille)
ou bien doté d’un degré de réalité inférieur
et dérivé (comme les choses sensibles,
dans le platonisme, sont des copies
des Idées). Avec le christianisme, on
assista à l’identification de la vérité avec
Dieu lui-même (Évangile selon saint
Jean : « Je suis la voie, la vérité, la vie »)
et, en particulier dans le cadre du dogme
de la Trinité, avec le Verbe ou Logos
divin. Le dogme de la Création imposait
l'attribution d’un caractère de vérité (au
sens ontologique) aux choses sensibles et
tout élément de la création, même le plus
modeste, étant nécessairement conforme
à l’idée archétype, présente dans l’esprit
% 1646
divin, sur la base de laquelle il a été créé.
Le terme « vrai » fait ainsi partie, pour la
scolastique, des termes dits transcendantaux.
Dans le vocabulaire scolastique, il
s’agit de la vérité « métaphysique » que
l’on doit distinguer de la vérité « logique »
(relative à la connaissance humaine) et de
la vérité « morale » (la simple sincérité ou
la véracité). Saint Thomas pressentait
déjà que le sens le plus propre de la
notion de vérité était le second, celui
qu'avait défini Aristote. Dans la pensée
moderne, l’acception ontologique fut
réfutée de façon plus définitive, entre
autres, par Hobbes, Spinoza et Locke. Ce
sens est revenu avec Hegel. Au sommet
de sa logique, il identifie l’« Idée » à la
vérité mais il montre qu’il ne s’agit pas de
la vérité que l’on peut trouver dans les
propositions (on peut dans ce cas parler
plus précisément d’« exactitude »), mais
du « Vrai en soi et pour soi » ou encore
de l’Absolu.


\subsection{La vérité comme correspondance}
Aristote définit le contenu de la notion
de vérité comme la conformité d’une proposition
avec la réalité : une proposition
est vraie si les faits sont tels qu'elle les
décrit ou s’ils ne sont pas comme elle dit
qu'ils ne sont pas. Elle est fausse dans les
autres cas. Aristote souligne que, même si
la vérité d’une proposition implique que
les faits soient tels qu’elle les décrit, la raison
pour laquelle elle est vraie est que les
faits sont comme elle dit qu’ils sont. C’est-à-dire
qu’il y a une asymétrie radicale
entre le discours et la réalité, en rapport
à la vérité. Cette conception de la vérité,
qualifiée de « réaliste » et communément
appelée « conception de la vérité
comme correspondance » (saint Thomas
employait la formule {\it aedequatio rei et
intellectus}), a longtemps prévalu sur
toutes les autres. Elle n’a commencé à
être contestée qu’au {\footnotesize XX}$^\text{e}$ s.
tout en conservant
quelques défenseurs (B. Russell et
A. Tarski). Dans {\it Sens et vérité}, Russell
soutient en particulier que le domaine de
la vérité est beaucoup plus étendu que
celui de la connaissance et que la notion
de vérité n’est pas réductible à celle de
vérification. Une proposition ne serait ni
vraie ni fausse dans le cas où elle ne serait
ni vérifiée ni infirmée, alors que, conformément
au principe du tiers exclu, toujours
selon Russell, une proposition qui
n’est pas dépourvue de sens est, de fait,
%
vraie ou fausse (il n’admet pas d’autre
possibilité) même s’il n’existe aucun
moyen de s’en assurer, et même aussi si
l’on n’a pas d’opinion à son égard, comme
cela se produit du reste pour les questions
innombrables auxquelles on ne réfléchit
jamais. C’est la présence ou l’absence
d’une relation déterminée communément
appelée « correspondance » qui rend les
propositions vraies ou fausses (ainsi que,
à l’origine, les croyances qui sont exprimées
par ces propositions). Cette correspondance
s'établit entre la proposition
(ou la croyance) et un ou plusieurs faits
totalement indépendants de l'expérience
ou de la non-expérience que l’on en a.
Ainsi, on est certainement convaincu de
quelque chose sur la base d’expériences
données, qu’elles soient directes ou indirectes
(des documents historiques, par
exemple), mais cette conviction n’est
vraie ou fausse qu’« à cause » des faits
auxquels elle se réfère. Tarski, dans {\it Le
Concept de la vérité dans les langages formalisés}
(1933) puis dans {\it La Conception
sémantique de la vérité} (1944), a élaboré
une formulation rigoureuse de la conception
de la vérité comme correspondance.
Cette conception retient la thèse que la
proposition « {\it A} est {\it B} » est vraie si et seulement
si {\it A} est {\it B}. De cela vient que
« vrai » (ou « faux ») est un prédicat de
second niveau en ce qu’il se rapporte à la
mention d’une proposition (ou à son
« nom », comme ce que l’on obtient par
exemple avec la convention de l'écrire
entre guillemets). En effet, une proposition
peut être utilisée, comme c’est couramment
le cas, pour parler de ce dont
elle parle, mais elle peut également être
« mentionnée » comme c’est le cas lorsque
l’on parle d’elle justement en tant que
proposition. C’est ce qui se produit quand
l’on dit d’une proposition qu’elle est vraie
ou (fausse). Il est donc nécessaire de distinguer
le langage-objet, dans lequel on
utilise une proposition donnée pour parler
de quelque chose, du métalangage que
l’on utilise pour parler de la proposition
elle-même. Par cette distinction, qui
n'existe pas dans les langages naturels, on
peut par exemple éviter le fameux paradoxe
du menteur (quand quelqu'un dit
« je mens », ce qu’il dit est vrai si et seulement
si cela n’est pas vrai), qui depuis
l'Antiquité représentait  l’antinomie
typique par rapport à la notion de vérité.
Pour Tarski, la conception de la vérité
%1647
comme correspondance est totalement
neutre du point de vue philosophique,
c’est-à-dire en ce qui concerne des présupposés
épistémologiques ou métaphysiques.
Elle signifie seulement que
l’assertion (ou le refus) de la proposition
{\it p} implique l’assertion (ou le refus) de la
proposition métalinguistique « {\it p} est
vraie », si bien que nier cette conception
de la vérité équivaudrait à soutenir que
« {\it A} est {\it B} » est vraie si et seulement si {\it A}
n’est pas {\it B}. Et en effet, la conception de
la vérité comme correspondance opère
alors aussi à l’intérieur des conceptions
qui se présentent comme ses alternatives.

\subsection{Les alternatives à la conception
de la vérité comme correspondance}

Ces conceptions sont celles de la vérité
comme cohérence, celle de la vérité
comme conformité à la règle, celle du
caractère pragmatique de la vérité et ce
que l’on appelle la conception « performative »
de la vérité. La conception de la
vérité comme cohérence recouvre cependant
deux conceptions radicalement différentes :
l’une est métaphysique et l’autre
est empirique. La première, soutenue par
les néohégéliens anglais aux {\footnotesize XIX}$^\text{e}$
et {\footnotesize XX}$^\text{e}$ s.
(F. H. Bradley par exemple), considère
que seul l’ Absolu est totalement vrai (en
ce qu’il est totalement cohérent) et privé
de multiplicité (nous sommes donc dans
le cas de l’acception ontologique de la
notion de vérité). La seconde est soutenue
par (ou mieux, attribuée à) des penseurs
néopositivistes comme O. Neurath
ou C. G. Hempel. Pour ces derniers, une
proposition ne peut être confrontée qu’à
d’autres propositions (c’est la thèse du
caractère non transcendantal du langage).
La conception de la vérité comme conformité
aux règles est parfois attribuée, bien
que de façon erronée, à Kant. En effet, il
partageait la conception de la vérité
comme correspondance, tout en la considérant
philosophiquement peu  pertinente.
La conception des néokantiens
(W. Windelband et H. Rickert) réduit
l’objectivité de la connaissance humaine
aux caractéristiques de l’universalité et de
la nécessité. Il existe également deux
variantes sensiblement différentes de la
conception pragmatique de la vérité.
C.S. Peirce et J. Dewey ont soutenu cette
conception, en référence au contexte de la
recherche empirique et donc à la méthode
expérimentale de vérification, avec son
%
caractère connu et vérifiable dans le cadre
d’une critique des métaphysiques. En
revanche, William James et
F. C. S. Schiller ont réduit la question de
la vérité d’une croyance à celle de son utilité
(vitale, psychologique) pour l’individu,
et ce à cause des croyances
métaphysiques et théologiques reconnues
pourtant comme invérifiables. La théorie
de la « volonté de croire » trouve ici son
origine. Nietzsche propose des formulations
sensiblement équivalentes. La plus
récente conception de la vérité a été
défendue par P. F. Strawson : dire qu’une
proposition est vraie ne signifie pas lui
attribuer un prédicat qui décrive une de
ses propriétés mais bien quelque chose
comme « je suis d'accord », « j’approuve ».
On déclare ainsi que l’on est d’accord.
Dans le même ordre d’idée, lorsque
quelqu'un dit « je le promets », il ne décrit
rien mais il accomplit directement l’action
de promettre. On est ici dans la description
de la conception de la vérité comme
étant performative ou exécutive. De cette
conception, largement contestée, on tire
cependant une conclusion incontestable
sur la notion de vérité en général (de
même que, quelle que soit la conception
que l’on retient, en qualifiant une proposition
de « vraie » ou de « fausse », on
n’ajoute rien à son sens). Cette constatation
a conduit jusque-là à soutenir le
caractère non indispensable du terme
« vrai » (dire « {\it p} est vrai » serait une
façon redondante de dire simplement {\it p}).
Elle confirme l'interprétation de ce terme
comme caractère métalinguistique dans la
droite ligne de ce qu’avançait Tarski, pour
illustrer la conception classique de la
vérité.

\subsection{La vérité dans l’existentialisme}
Pour être complet, il faudrait mentionner
également une conception existentialiste
de la vérité dans la mesure où cette
question a été étudiée par Heidegger (en
particulier dans {\it L’Essence de la vérité},
1943) comme par Karl Jaspers ({\it Sur la
vérité}, 1947). Plus que de conceptions de
la notion de vérité (comparables à celles
déjà citées tout en étant très différentes),
il s’agit de théories sur le rapport de
l’homme à la vérité. Heidegger insiste sur
le sens étymologique du mot qui, en grec,
signifie littéralement « le fait de ne pas
cacher » ({\it aletheia}). Elle consiste ainsi en
% 1648
une sorte d’autorévélation toujours
incomplète de l’Être. Pour Heidegger, en
effet, même quand il se révèle, l’Être se
soustrait à l’être. Pour Jaspers, la vérité
est l’autorévélation de l'existence singulière,
toute existence est sa propre vérité.
Du moment que les individus entretiennent
des rapports mutuels, chacun est en
rapport avec la vérité des autres. C’est ce
qui se produit dans le processus de la
communication en ce qu’il est toujours
ouvert. L’unicité et la pluralité de la vérité
s’accordent, selon Jaspers, dans la tension
continue vers la Vérité, totale et définitive,
qui reste impossible à atteindre (seul
le dogmatique peut être en désaccord
avec cette idée).

%%%%%%%%%%%%%%%%%%%%%%%%
