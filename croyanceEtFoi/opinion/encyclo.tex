
%%%%%%%%%%%%%%%%%%%%%
\section{Encyclopédie de la philosophie}
%%%%%%%%%%%%%%%%%%%%%
%{\bf }{\bf --}{\it }

\subsection{{\it Doxa}}

Terme grec qu’on traduit le plus souvent
par « Opinion », mais qui recouvre en
grec un éventail de significations beaucoup
plus large. Le mot vient du verbe
{\it dokeô}, « j’apparais », « je me manifeste ».
Dans l’un des sens courants, {\it doxa} désigne
la considération dont jouit une personne
(équivalent grec de la {\it fama} des Latins),
l’opinion qu’on a d’elle. Si cette opinion
porte sur quelque chose de véritablement
considérable, alors {\it doxa} peut être rendu
par « gloire » et la {\it doxa tou théou} est la
« gloire de Dieu ».

%—> doxographes  paradoxe  Parménide
 
%%%%%%%%%%%%%%%%%%%%%%%%
