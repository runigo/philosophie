%%%%%%%%%%%%%%%%%%%%%%%%%%%%%%%%%%%%
\section{Pratique de la philosophie}
%%%%%%%%%%%%%%%%%%%%%%%%%%%%%%%%%%%%

\subsection{Opinion}

{\footnotesize
\begin{itemize}[leftmargin=1cm, label=\ding{32}, itemsep=1pt]
\item {\bf \textsc{Étymologie} :} latin {\it opinari},
« émettre une opinion ».
\item {\bf \textsc{Sens ordinaire} :} avis,
jugement porté sur
un sujet, qui ne relève pas d'une
connaissance rationnelle vérifiable,
et dépend donc du système de
valeurs en fonction duquel on se
prononce.
\item {\bf \textsc{Philosophie} :} jugement
sans fondement rigoureux,
souvent dénoncé dans la mesure où
il se donne de façon abusive les
apparences d’un savoir.
\end{itemize}
}

L'interrogation sur la nature de la vérité
et les moyens de l’atteindre a conduit
nombre de philosophes à distinguer,
entre les différents types de connaissance
possibles, ceux qui conduisent effectivement
à la vérité, et ceux qui en éloignent.
En un premier sens, l’opinion est ainsi
traditionnellement considérée comme un
genre de connaissance peu fiable, fondée
sur des impressions, des sentiments, des
croyances où des jugements de valeur
subjectifs. Pour Spinoza, par exemple,
elle est forcément « sujette à l'erreur et n’a
jamais lieu à l'égard de quelque chose
dont nous sommes certains mais à l'égard
de ce que l’on dit conjecturer ou supposer »
({\it Court traité}, chap. II). Depuis Platon,
et jusque chez de nombreux penseurs
contemporains, l'opinion est
dénoncée comme a priori douteuse, illusoire
ou fausse, voire dangereuse, lorsqu’elle
cherche à s'imposer en dissimulant
la faiblesse de ses fondements sous
les apparences de la plus claire certitude.
Selon Adorno ({\it Modèles critiques}, 1963),
« l'opinion s’approprie ce que la connaissance
ne peut atteindre pour s’y substituer »,
elle rassure à bon compte, parce
qu’« elle offre des explications grâce auxquelles
on peut organiser sans contradiction
la réalité contradictoire ». Tel est bien
le « fonctionnement psychique » qui soustend,
par exemple, les opinions racistes :
pour être plus crédible, la peur de l’autre
prend le masque de l'affirmation de son
infériorité ou de la mise en garde contre
le danger qu'il est censé représenter. La
justesse de ces analyses ne doit pas faire
oublier qu'en un autre sens, l'opinion
constitue une forme de connaissance
utile, voire un type de jugements éminemment
respectables. Dans le {\it Ménon},
Platon reconnaît aux opinions droites la
faculté, sur les sujets qui ne relèvent ni de
la science ni de la simple conjecture,
d'éclairer l’action humaine. Dans le
domaine moral par exemple, à défaut de
vérités certaines, des intuitions justes
relatives au bien peuvent guider efficacement
l'éducation ou l’action, en leur
fixant pour but la satisfaction d'intérêts
conformes aux exigences de la réflexion,
et non à la soumission aux apparences ou
au plaisir immédiat. Enfin, sur toutes les
questions qui engagent des choix individuels
qu'aucune autorité ne peut légitimement
contraindre {\bf --} la religion, la
préférence politique, l'adhésion à une
conception du monde {\bf --} la liberté d’opinion
est un droit fondamental, dans les
sociétés démocratiques en tout cas, dès
l'instant où ceux auxquels elle est garantie
n'en usent pas au détriment de la
liberté d'autrui.

Analysée dans le {\it Traité
théologico-politique}, où Spinoza insiste
sur la nécessité d'une indépendance
absolue des opinions religieuses et de
leur expression par rapport à l'État, la
liberté d'opinion est proclamée dans la
Déclaration des droits de l'homme et du
citoyen de 1789. Et depuis près d'un
siècle, elle est au cœur du principe de la
laïcité qui garantit (en particulier en
France) la séparation entre l'Église et
l'État.

{\footnotesize
\begin{itemize}[leftmargin=1cm, label=\ding{32}, itemsep=1pt]
\item {\bf \textsc{Termes voisins} :} avis ; croyance.
\item {\bf \textsc{Termes opposés} :} science.
\end{itemize}
}

\subsubsection{Opinion publique}

Ensemble fluctuant de prises de positions
portant sur des questions politiques,
 morales, économiques... Les
« sondages d'opinion » prétendent en
constituer une sorte de baromètre.

{\footnotesize
\begin{itemize}[leftmargin=1cm, label=\ding{32}, itemsep=1pt]
\item {\bf \textsc{Corrélats} :} connaissance ;
conviction ; croyance ; doute ; foi ;
jugement ; préjugé.
\end{itemize}
}

%%%%%%%%%%%%%%%%%%%%%%%%%%%%%%%%%%%%%%%%%%%%%%%%%%%%
\subsection{Préjugé}

{\footnotesize
\begin{itemize}[leftmargin=1cm, label=\ding{32}, itemsep=1pt]
\item {\bf \textsc{Étymologie} :} latin {\it praejudicare},
« juger préalablement ».
\item {\bf \textsc{Sens ordinaire} :} Opinion admise sans
jugement ni raisonnement.
\end{itemize}
}

Le terme préjugé est souvent employé
dans un sens péjoratif, pour dénoncer
l'erreur ou au moins l'absence de
réflexion qui conduit un individu à
adhérer à une idée fausse {\bf --} dont il n’a
pas pris la peine de contrôler le bien-fondé {\bf --}
voire à la défendre contre des
idées justes, ou à condamner des individus
au nom de cette idée (par
exemple, les opinions racistes sont des
préjugés).

{\footnotesize
\begin{itemize}[leftmargin=1cm, label=\ding{32}, itemsep=1pt]
\item {\bf \textsc{Termes voisins} :} opinion.
\item {\bf \textsc{Termes opposés} :} savoir ; science.
\item {\bf \textsc{Corrélats} :} certitude ; croyance ;
dogme ; doute ; foi.
\end{itemize}
}

%%%%%%%%%%%%%%%%%%%%%%%%%%%%%%%%%%%%%%%%%%%%%%%%%%%%

\subsection{Erreur}

{\footnotesize
\begin{itemize}[leftmargin=1cm, label=\ding{32}, itemsep=1pt]
\item {\bf \textsc{Étymologie} :} latin {\it error}, « course
à l'aventure », de {\it errare}, « errer ».
\item {\bf \textsc{Logique et sciences} :} affirmation
fausse, c'est-à-dire non conforme
aux règles de la logique, et/ou en
contradiction avec les données
expérimentales.
\item {\bf \textsc{Psychologie} :} état
de l'esprit qui tient pour vrai ce
qui est faux, et réciproquement
(ex. : « être dans l’erreur »).
\end{itemize}
}

L'erreur doit être soigneusement distinguée
aussi bien de la faute (qui engage
plus nettement notre responsabilité) que
de l’illusion (qui n’est pas vaincue par
le savoir). L'erreur procède toujours de
notre jugement : elle résulte, selon Descartes,
d’un décalage permanent entre
notre volonté, qui est infinie, et notre
entendement, qui ne l'est pas. Nous
nous trompons parce que nous outrepassons
nos possibilités intellectuelles,
par étourderie ou vanité : l'erreur n’est
donc qu'une privation de connaissance.
L'épistémologie contemporaine, au
contraire, donne à l’erreur un tout autre
statut, plus « positif ». Bachelard, notamment,
montre que les « vérités » scientifiques
ne sont jamais que provisoires,
qu'elles doivent constamment être remaniées
et corrigées. La connaissance
scientifique ne peut pas faire l'économie
de l’erreur.

{\footnotesize
\begin{itemize}[leftmargin=1cm, label=\ding{32}, itemsep=1pt]
\item {\bf \textsc{Termes voisins} :} fausseté ; illusion ;
incorrection.
\item {\bf \textsc{Termes opposés} :} vérité.
\item {\bf \textsc{Corrélats} :} connaissance ; Évidence ;
faute ; illusion ; jugement.
\end{itemize}
}

%%%%%%%%%%%%%%%%%%%%%%%%%%%%%%%%%%%%%%%%%%%%%%%%%%%%
