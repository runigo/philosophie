
%%%%%%%%%%%%%%%%%%%%%
\section{Encyclopédie de la philosophie}
%%%%%%%%%%%%%%%%%%%%%
% 585
\subsection{Foi}
Terme qui, dans sa plus large acception,
indique les formes de conscience que
ne peuvent garantir ni les contrôles empiriques
ni les procédés rationnels. Ils s’en
% 586
remettent donc soit à des intuitions subjectives
probantes, soit à des postulats
adoptés comme principes de démonstration,
soit encore à des témoignages dignes
de confiance, Dans un sens plus étroit, la
foi apparaît comme la croyance en des
principes ou vérités religieuses, en particulier
lorsque leur révélation est d’ordre
surnaturel.

Dans la religion biblique et chrétienne,
le terme de « foi » acquiert une signification
particulière. La foi apparaît dans la
Bible comme l’acceptation de la Révélation
de Dieu. Cette révélation, quand
bien même d'attribution divine, est un
événement qui s'inscrit dans la réalité du
monde et de l’histoire. La foi est donc un
acte et un processus par lesquels
l’homme, interpellé par les paroles et les
interventions divines {\bf --} dont témoigne la
Bible {\bf --}, décide de sa propre existence.
Dans la conscience chrétienne, le centre
et le sommet de cette présence divine est
Jésus de Nazareth : la foi en le Christ
implique l’engagement total de soi, qui
s'étend à tous les moments de l’existence,
et conduit aussi à la forme visible et
communautaire de la confession de foi.
L'Église est le lieu qui historiquement
atteste de la foi, laquelle en est par conséquent
la norme et la mesure. La crédibilité
et la valeur de l'Église ne lui
appartiennent pas en propre, mais seulement
en fonction de la foi pour laquelle
elle existe. Si la foi est foi en Jésus-Christ,
sa totalité et son unité dépendent premièrement
de l’adhésion au Christ, et deuxièmement {\bf --}
sur la base de leur référence
spécifique au Christ {\bf --} de l’acceptation des
propositions de foi ou des dogmes, ou des
instances qui les définissent. Pour le christianisme
(mais aussi pour le judaïsme), la
foi n’est pas un acte qui vaille devant
Dieu, mais au contraire une acceptation
de Dieu comme grâce absolue. Mais tout
ceci n’advient que simultanément à l’acte
de l’homme qui croit, à l’acte de foi, qui
à son tour n’est pas détachable des autres
actes qui constituent la trame de la vie
humaine. L’acte de foi est au contraire
(ainsi que le souligne toute la pensée
chrétienne, depuis saint Paul jusqu’à
Søren Kierkegaard et à Karl Barth) une
décision par laquelle l’homme {\bf --} s’en
remettant à Dieu qui se révèle {\bf --} projette
le sens de sa propre existence. En tant
que décision, la foi est un acte libre qui
implique aussi la raison. D’où la possibilité
d’une réflexion {\it a posteriori} sur la décision
de foi, au sein de laquelle est
thématisé ce qui a tout d’abord été vécu
et compris sous une forme préréflexive.
Là réside le fondement d’une potentielle
expression de la foi en formules linguistiques
qui prétendent à la vérité, et de
l’évolution d’une théologie.

La foi s’en remet tout particulièrement
à la théologie pour se confronter aux positions
culturelles qui contestent sa prétention
à être la plus haute valeur humaine,
ou qui tendent à la réduire à des formes
d’aliénation psychique ou sociale, ou à
toute autre attitude irrationnelle.

%%%%%%%%%%%%%%%%%%%%%%%%
