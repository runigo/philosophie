
%%%%%%%%%%%%%%%%%%%%%
\section{Pratique de la philosophie}
%%%%%%%%%%%%%%%%%%%%%

\subsection{Foi}

%\begin{itemize}[leftmargin=1cm, label=\ding{32}, itemsep=1pt]
%\item {\footnotesize \bf  :} .\end{itemize}

%{\footnotesize V}$^\text{e}$ siècle avant Jesus-Christ

%{\it }

\begin{itemize}[leftmargin=1cm, label=\ding{32}, itemsep=1pt]
\item {\bf \textsc{Étymologie} :} latin {\it fides},
« confiance », « crédit », « loyauté »,
« engagement ».
\item {\bf \textsc{Sens ordinaire} :} degré d'adhésion que l'on peut
accorder à une idée, une parole, un
comportement ou un homme.
\item {\bf \textsc{Théologie} :} mode religieux de la
croyance.
\end{itemize}

La foi, qui n’est pas un savoir, ne se
réduit pas à une simple croyance et
même, souvent, s'oppose à celle-ci.
Comme le dit Alain ({\it Propos}), la
croyance crédule, « c’est pensée agenouillée
et bientôt couchée », tandis
que dans la foi, « il faut croire d’abord,
et contre l’apparence ; la foi va devant,
la foi est courage » (en ce sens,
l’athéisme conscient peut aussi relever
de la foi). Ce en quoi on a foi n’est pas
démontrable, mais exige un degré de
confiance au moins égal à celui que
produirait une démonstration. La foi est
un engagement qui se veut lucide,
contrairement à la croyance, le plus
souvent naïve. Pour qu'il y ait foi, il
faut donc qu'il y ait des raisons de
croire. La foi c’est, par exemple, l’exigence
qu'on s'impose à soi-même de
croire en l'autre lorsqu'il a pris un
engagement, mais sans en méconnaître
les risques. Le mot désigne donc à la
fois une obligation qui se traduit par
un comportement volontaire (sens
objectif) et un régime de croyance
(sens subjectif).

Au sens théologique, la foi désigne la
confiance absolue qu'on accorde à
Dieu, même lorsque la raison n'y saurait
donner un quelconque appui. Chez
Pascal, par exemple, la foi relève de
l'ordre de la grâce ; le moyen de la
croyance est ici le « cœur » : « Voilà ce
que c’est que la foi. Dieu sensible au
cœur, non à la raison » ({\it Pensées}, Laf.
424). Kierkegaard montre bien
comment la foi suppose une confiance
au-delà de ce que la raison peut calculer
ou démontrer, à la limite de
l’absurde, mais sans faire l’économie
de l’angoisse que cela suscite (cf. Kierkegaard
et Abraham). Chez Kant, les
« postulats de la raison pratique » (la
croyance en l'immortalité de l'âme et
en l'existence d'un législateur
suprême) relèvent aussi de la foi, précisément
parce que la philosophie
morale donne en la matière des raisons
de croire, tout en affirmant qu’on ne
peut pas savoir.

Tandis que la croyance conduit à la
crédulité et au sommeil de l'esprit, la
foi se présente donc comme une
croyance consciente d’être croyance,
reposant sur des principes et engageant
une décision de la volonté.

\begin{itemize}[leftmargin=1cm, label=\ding{32}, itemsep=1pt]
\item {\bf \textsc{Terme voisin} :} croyance.
\item {\bf \textsc{Terme opposé} :} savoir.
\end{itemize}

\subsection{Mauvaise foi}

L'expression désigne généralement une
attitude d’esprit inspirant des comportements
ou des propos dans lesquels il est
clair que le sujet ne respecte pas ses
engagements, explicites ou tacites. On
tient des propos « de mauvaise foi » lorsqu'on
n'est plus fidèle à sa propre
volonté de vérité (par exemple, lorsqu'on
refuse de reconnaître qu’on a
tort).

Chez Sartre ({\it L'Être et le Néant}, première
partie, chap. 2), l'expression désigne
plus spécialement l'attitude par laquelle
la conscience d’un sujet cherche à se
tromper elle-même, afin de se voiler ses
responsabilités et d'échapper à l'angoisse
que celles-ci pourraient provoquer.
« L'homme n'est rien d'autre que
ce qu'il se fait » ({\it L'Existentialisme est un
humanisme}) ; il est de mauvaise foi dès
qu'il se dissimule ce qu'il fait et que c’est
lui qui le fait (cf. Conscience).

\begin{itemize}[leftmargin=1cm, label=\ding{32}, itemsep=1pt]
\item {\bf \textsc{Corrélats} :} athéisme ; conviction ;
engagement ; religion ;
volonté.
\end{itemize}

%%%%%%%%%%%%%%%%%%%%%%%%%%%%%%%%%%%%%%%%%%%%%%%%%%%%%%%%%%%%%%%%%%%%%%%%%%%
