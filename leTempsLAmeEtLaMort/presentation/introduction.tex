
\thispagestyle{empty}

\begin{center}
\Large
%Introduction
Préambule
\normalsize
\end{center}
\vspace{3cm}

Ce document est une compilation d'articles provenant de trois ouvrages : un dictionnaire encyclopédique de poche, {\it La pratique de la philosophie} destiné aux lycéens, et une encyclopédie de la philosophie destinée aux néophytes. 

\vspace{1.3cm}

Chaque chapitre contient les articles correspondant à une notion particulière. Ces notions ont été choisies en raison de leurs liens avec la question de l'âme. Ces choix ont été guidés : {\bf 1.} Par les renvois vers d'autres articles présent dans les ouvrages. {\bf 2.} Mes propres choix, liés à ma subjectivité. {\bf 3.} La volonté d'obtenir une quantité raisonnable d'information.

\vspace{1.3cm}

J'ai reproduit en annexe l'article de l'encyclopédie de la philosophie concernant Aufhebung.
Les articles compilés dans ce document comportent donc les choix "discutables" réalisés dans les trois ouvrages utilisés. Il s'agit donc d'un document de travail destiné à apporter quelques éléments de réflexion et une synthèse relativement élémentaire des points de vues philosophiques.

\vspace{1.3cm}

%Les trois premiers chapitres abordent les thèmes .
% Les chapitres suivants élargissent le champ de vision philosophique en abordant les thèmes .

\vspace{2.3cm}

\hfill Stephan Runigo

%%%%%%%%%%%%%%%%%%%%%%%%%%%%%%%%%%%%%%%%%%%%%%%%
