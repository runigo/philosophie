
%%%%%%%%%%%%%%%%%%%%%
\chapter{Aufhebung}
%%%%%%%%%%%%%%%%%%%%%

{\it Aufhebung} : terme allemand difficilement
traduisible (on utilise quelquefois les mots
de dépassement, d’assomption, voire de
« Sursomption ») parce qu’il a un double
sens : celui de « conserver, garder », et en
même temps celui de « mettre fin, enlever,
dépasser ». Hegel compare le verbe {\it aufheben} au latin {\it tollere} (“\,élever\;”), et illustre l’ambivalence
de l’{\it Aufhebung} par le jeu de mots latin
« {\it tollenda est Octavium} » qui signifie à la
fois qu’il faut porter Octave aux nues et,
aussi, le supprimer. C’est précisément en
raison de cette présence simultanée de
deux significations opposées en un même
vocable que Hegel le choisit pour désigner
la dialectique, c’est-à-dire pour indiquer
une négation qui ne soit pas anéantissement ou suppression complète, mais processus d’avènement du négatif. On parle
d'{\it Aufhebung} lorsqu'une détermination
conceptuelle est pensée conjointement à
son contraire ({\it La Science de la logique}) et
que de cette relation d’opposition naît une
troisième entité qui est la vérité de l’opposition dépassée ({\it aufgehoben}). Cette unité
est le concret logique : autrement, aussi
longtemps que l’on prétend penser isolément une détermination conceptuelle,
celle-ci demeure de ce fait abstraite. Toute
la logique de Hegel est construite sur ce
modèle : elle part du couple être et néant,
qui « passe » dans le « devenir », puis dans
l’« être-là » ({\it Dasein}, existence), pour se
déployer dans l’opposition de la subjectivité et l’objectivité, et trouve sa vérité dans
l’Idée absolue. L’{\it Aufhebung} implique
donc la reconnaissance de l’opposition en
tant que structure fondamentale de la
logique et du monde, et elle signifie la
réduction des déterminations conceptuelles qui s'opposent les unes aux autres
par « moments », en entendant par là
qu'elles n’ont précisément pas de vérité
indépendamment l’une de l’autre. L’{\it Aufhebung} est par conséquent pour Hegel
l'opération typique de la « Raison », qui,
par opposition à l’intellectualisme, permet
de reconnaître le caractère dialectique de
la pensée et du réel. Après Hegel, ce terme
n'est présent que chez les philosophes qui
s’inspirent de sa pensée.


%%%%%%%%%%%%%%%%%%%%%%%%%%%%%%%%%%%%%%%%%%%%%%%%%%%%%%%%%%%%%%%%%%%%%%%%%%%%%%%%%%%%%
