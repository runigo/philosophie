
%%%%%%%%%%%%%%%%%%%%% 36
\section{L'âme}
%%%%%%%%%%%%%%%%%%%%%
%{\it }
L'âme {\it }(anima, psuchè) est ce qui anime le corps : ce qui lui permet de se
mouvoir, de sentir et de ressentir. Pour le matérialiste, c’est donc le
corps lui-même, considéré dans sa motricité, dans sa sensibilité, dans son affectivité
propres. Dans son intellectualité ? Pas nécessairement. Un animal peut
sentir et ressentir (avoir une âme) sans être capable pour autant de penser ou de
raisonner abstraitement (sans être esprit). C’est ce qu’on appelle une bête.
Mieux vaut en effet parler d’esprit {\it }(mens, noûs), plutôt que d’âme, pour désigner
cette partie du corps qui a accès au vrai ou à l’idée. Cela fait, entre les
deux, une autre différence. Perdre l'esprit, comme chacun sait, n’est pas la
même chose que perdre son âme. Perdre l'esprit, c’est perdre la raison, le bon
sens, qui est le sens commun : c’est perdre l’accès à l’universel et devenir du
même coup prisonnier de son âme. Le fou n’est pas moins 504 que n'importe
qui, pas moins particulier, pas moins singulier, bien au contraire : il n’est plus
que soi, et c’est cet enfermement — coupé du vrai, coupé du monde, coupé de
tout — qui le rend fou. L'esprit ouvre la fenêtre, et c’est ce qu’on appelle la
raison.

L'âme est toujours individuelle, singulière, incarnée (il n’y a pas d’âme du
monde, ni de Dieu). L’esprit serait plutôt anonyme ou universel, voire objectif
ou absolu (si l’univers pensait, il serait Dieu ; si Dieu existait, il serait esprit).
Nul, par exemple, ne peut ressentir à ma place, ni tout à fait comme moi. Mon
âme est unique, tout autant que mon corps. Alors qu’une idée vraie, en tant
qu’elle est vraie, est la même en moi et en tout autre (en moi et en Dieu, disait
Spinoza). Cet accès — pour nous toujours relatif — à l’universel ou à l'absolu,
c'est ce que j'appelle l'esprit : c’est notre façon d’habiter le vrai en nous libérant
de nous-mêmes. L’Âme serait plutôt notre façon, toujours singulière, toujours
déterminée, d’habiter le monde : c’est notre corps en acte, dit à peu près Aristote,
en tant qu'il a la vie (la motricité, la sensibilité, l’affectivité) en puissance.

Ainsi c’est l'esprit qui est libre, non l’âme, ou plutôt c’est l'esprit qui libère,
et c’est, pour l’âme, le seul salut, toujours inachevé.

%%%%%%%%%%%%%%%%%%%%%%%%%%%%%%%%%%%%%%%%%%%%%%%%%%%%%%%%%%%%%%%%%%%%%%%%%%%%%%%%%%%%%
