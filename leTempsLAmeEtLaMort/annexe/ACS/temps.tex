
%%%%%%%%%%%%%%%%%%%%%
\section{Le temps}
%%%%%%%%%%%%%%%%%%%%%
% 579
%{\it }
« Le temps, disait Chrysippe, se prend en deux acceptions. » Il est
d’usage de les confondre, et c’est cette confusion, presque toujours,
qu’on appelle le temps.

Le temps, c’est d’abord la durée, mais considérée indépendamment de ce
qui dure, autrement dit abstraitement. Non un être, donc, mais une pensée.
C’est comme la continuation indéfinie et indéterminée d’une inexistence : ce
qui continuerait encore, c’est du moins le sentiment que nous avons, si plus
rien n'existait.

Ce temps abstrait — l’{\it aiôn} des stoïciens — peut se concevoir, et se conçoit
ordinairement, comme la somme du passé, du présent et de l’avenir. Mais ce
présent n’est alors qu’un instant sans épaisseur, sans durée, sans temps (s’il
durait, il faudrait le diviser en passé et en avenir), et c’est en quoi il n’est rien,
ou presque rien. En ce sens, et comme disait encore Chrysippe, « aucun temps,
n’est rigoureusement présent ». C’est ce qui le distingue de la durée. À le considérer
abstraitement, le temps est constitué essentiellement de passé et d’avenir
(alors qu’on ne peut durer qu’au présent), et pour cela indéfiniment divisible
(ce que le présent n’est jamais) et mesurable (ce que le présent n’est pas davantage).
C’est le temps des savants et des horloges. « Pour déterminer la durée,
écrit Spinoza, nous la comparons à la durée des choses qui ont un mouvement
invariable et déterminé, et cette comparaison s’appelle le temps. » Comparaison
n’est pas raison : le présent, incomparable et indivisible, n’en continue pas
moins.

Quant au temps concret ou réel — le {\it }chronos des stoïciens —, ce n’est que la
durée de tout, autrement dit la continuation indéfinie de l’univers, qui
demeure toujours le même, comme disait à peu près Spinoza, bien qu’il ne
cesse de changer en une infinité de manières. C’est la seconde acception du
mot : non plus une pensée, mais l’être même de ce qui dure et passe. Non la
somme d’un passé et d’un avenir, mais la perduration du présent. C’est le
temps de la nature ou de l’être : le devenir en train de devenir, le changement
continué des étants. Le passé ? Ce n’est rien de réel, puisque ce n’est plus.
L'avenir ? Ce n’est rien de réel, puisque ce n’est pas encore. Dans la nature, il
n’y a que du présent. C’est ce qu'avait vu Chrysippe (« seul le présent existe »),
et c’est ce que Hegel, à sa façon, confirmera : « La nature, où le temps est le
maintenant, ne parvient pas à différencier d’une façon durable ces dimensions
du passé et du futur : elles ne sont nécessaires que pour la représentation subjective,
le souvenir, la crainte ou l’espérance » ({\it Précis de l'Encyclopédie}, \S 259).
Comment mieux dire qu’elles ne sont nécessaires que pour l'esprit, point pour
le monde? Le temps de l’âme n’est qu’une {\it distension}, comme disait saint
Augustin, entre le passé et l’avenir (c’est ce qu’on appelle la temporalité). Le
temps de la nature, qu’une tension ({\it tonos}), qu’un effort ({\it conatus}) ou un acte
% 580
{\it (energeia)}, dans le présent. Ces deux temps, toutefois, ne sont pas sur le même
plan : l’âme fait partie du monde, comme la mémoire et l’attente font partie du
présent. Le temps, dans sa vérité, est donc celui de la nature : ce n’est qu’un
perpétuel, quoique multiple et changeant, {\it }maintenant. C’est en quoi il ne fait
qu’un avec l'éternité.

Deux sens, donc : une abstraction ou un acte. La durée, abstraction faite de
ce qui dure, ou l’être même, en tant qu’il continue. Une pensée, ou un devenir.
La somme du passé et de l’avenir, qui ne sont rien, ou la continuation du présent,
qui est tout. Un non-être, ou l’être-temps. Ce qui nous sépare de l’éternité,
ou l'éternité même.


%%%%%%%%%%%%%%%%%%%%%
\subsection{Temps perdu}
%%%%%%%%%%%%%%%%%%%%%
C'est le passé, en tant qu’il n’en reste rien, ou le présent, en
tant qu’il n’est que l’attente de l’avenir. Aussi est-ce le
contraire de l'éternité. Misère de l’homme. Le temps perdu, c’est le temps
même.

%%%%%%%%%%%%%%%%%%%%%
\subsection{Temps retrouvé}
%%%%%%%%%%%%%%%%%%%%%
C’est une espèce d’éternité de la mémoire, où le temps
soudain se révèle (« un peu de temps à l’état pur », dit
Proust), dans sa vérité, et par là (en cet instant «affranchi de l’ordre du
temps ») s’abolit. Voilà que le présent et le passé ne font qu’un, ou plutôt, pour
différents qu’ils demeurent (la madeleine dans le thé et la madeleine dans la
tisane sont deux), voilà qu’ils se rencontrent dans un même présent, qui est
celui de l'esprit, qui est celui de l’art, voilà qu’ils libèrent « l’essence permanente
et habituellement cachée des choses », qui est simplement leur vérité, toujours
présente, ou leur éternité. Car la vérité ne passe pas, tout est là, car le temps ne
passe pas (c’est nous, dirait Proust comme Ronsard, qui passons en lui), et cette
contemplation, quoique fugitive, est d’éternité. Le temps retrouvé est ainsi la
même chose que le temps perdu («la vraie vie, la vie enfin découverte et
éclaircie, la seule vie par conséquent réellement vécue. »), et pourtant son
contraire.

% 578
%%%%%%%%%%%%%%%%%%%%%
\subsection{Temporalité}
%%%%%%%%%%%%%%%%%%%%%
C’est une dimension de la conscience : sa façon d’habiter
le présent en retenant le passé et en anticipant l'avenir.
Elle n’est pas la vérité du temps, montre Marcel Conche, mais sa négation (elle
fait exister ensemble, comme «unité ek-statique du passé, du présent et du
futur », ce qui ne saurait en vérité coexister). Ce n’est pas le temps réel, mais
notre façon de le vivre ou de l'imaginer.

%%%%%%%%%%%%%%%%%%%%%%%%%%%%%%%%%%%%%%%%%%%%%%%%%%%%%%%%%%%%%%%%%%%%%%%%%%%%%%%%%%%%%
