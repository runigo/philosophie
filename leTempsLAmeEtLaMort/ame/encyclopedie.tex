
\section{Encyclopédie de la philosophie}

{\bf Âme} : terme d'origine latine (Du latin {\it anima}, qui a la même racine que le grec {\it ànemos}, "vent", "soufle" et le même sens que le latin {\it spiritus}, correspondant au grec {\it pnêuma}, "air", "souffle", "respiration"), qui désigne le principe de l'activité consciente de l'homme et, de façon plus large, le principe de vie de tout être vivant ou animé. La recherche anthropologique a révélé que dans les civilisations dites primitives, il existe une conception de l'âme entendue surtout comme principe vital. Ce principe peut se matérialiser et on peut l'identifier à des organes ou à des parties du corps (par exemple à la partie grasse des reins chez les indigènes d'Australie), dans la respiration ou même dans l'ombre, etc, ou bien encore dans un objet ou un animal qui apparaît alors comme une "âme extérieur" : cette dernière notion est présente dans différentes formes de culture primitive aussi bien que dans certains mythes de la culture indo-européenne. La conception de l'âme comme principe distinct du corps et principe survivant au corps se retrouve dans de nombreuses doctrines religieuses de l'Antiquité, telles que l'orphisme, les religions babylonienne, égyptienne, iranienne, ainsi que dans certaines religions orientales comme le boudhisme et le brahmanisme, où l'on admet également la transmigration des âmes. La pensée philosophique et religieuse chrétienne accentue la distinction de l'âme et du corps, et elle a tendance à concevoir l'homme comme une résultante de deux principes, l'un matériel (le corps) et l'autre spirituel (l'âme). Les anthropologie préphilosophiques des peuples indo-européens, au contraire, n'insiste pas sur l'interprétation dualiste de l'existence de l'homme : chez Homère, le mot {\it psyché} ne désigne pas l'âme par opposition au corps, mais le "souffle vital" qui, lors de la mort, s'échappe du corps et continue de vivre comme une ombre inconsistante. Dans l'entropologie hébraïque tel qu'on peut la conjecturer à la lecture de l'Ancien Testament, l'homme est conçu de façon unitaire : l'âme est désigné principalement par le mot {\it nèfesh} qui indique aussi bien la vie végétative que la vie spirituelle (la connaissance et les sentiments religieux), et dont la signification se rapproche de celle de {\it rûah}, "respiration", "soufle" : les deux termes désigne la force vitale unique à laquelle se rapportent les manifestations de l'homme de caractère spirituel, sensitif et végétatif. Même les termes néotestamentaires qui leur correspondent excluent une vision dualiste partagée entre l'âme et le corps : on attribue à la {\it psyché} , "âme", les sentiments de l'appétit "de concupiscence" (joie et tristesse); au {\it pnêuma}, "esprit", on attribue les sentiments de l'appétit "irrascible" (la colère, la patience, la douceur).

La conception dualiste de l'homme apparaît pour la première fois de façon très nette chez Platon, qui conçoit l'âme comme un principe de nature différente de celle du corps, liée au monde des idées, préexistant au corps et immortelle. La même conception, transmise par les philosophies stoïcienne et néoplatonicienne, a été reprise par la plupart des Pères de l'Église, en particulier par Augustin, qui définit l'âme comme une "substance doué de raison, destinée à guider le corps" : comme l'âme est une substance indépendante du corps, il est logique qu'elle ne doive pas mourir avec lui, mais qu'elle continue à vivre jusqu'à la résurection finale, quand elle se réappropriera le corps. Une critique radicale de l'anthropologie platonicienne a vu le jour avec Aristote, pour qui l'âme doit être entendue comme forme ({\it entélékheia}) du corps, c'est-à-dire comme principe qui détermine et spécifie le corps, auquel il donne vie. Comme les phénomènes de la vie supposent des opérations déterminées, constantes et nettement différenciées, l'âme doit avoir, selon Aristote, des capacités, des fonctions ou des parties qui président à ces opérations : l'âme {\it végétative} préside à la génération, à l'alimentation et à la croissance; l'âme {\it sensitive} préside à l'activité sensitive et l'âme {\it locomotrice} au mouvement local des animaux; l'âme  {\it intellective} ou  {\it rationnelle} préside à la connaissance, à la délibération et au choix. Il suffit, dit Aristote, que l'une ou l'autre des diverses fonctions des organismes vivants soit remplie pour qu'on puisse parler d'âme. Il n'y a donc pas plusieurs âmes ou plusieurs types d'âme, mais des fonctions spécifiques et distinctes caractérisant l'âme comme forme du corps. La doctrine psychologique d'Aristote a le mérite de dépasser le dualisme de l'âme et du corps, mais elle se heurte à une nouvelle difficulté : quel est le rapport de l'âme intellective avec les autres ? On connaît la réponse d'Aristote : l'intellect, qui est le principe par lequel l'homme connaît et réfléchit, n'est pas mêlé au corps mais en est séparé par nature, il est imortel, "divin", et vient du dehors se loger dans l'embryon (la doctrine du  {\it nous thûraten} évoqué dans le  {\it De generatione animalium}). Cette question du statut de l'intellect (grec {\it nous}) dans ses rapports avec les autres fonctions vitale de l'âme a donner lieu à de vives discussions tout au long de l'histoire de l'aristotélisme : l'un des enjeux, et non des moindres, était l'affirmation ou non par Aristote d'une immortalité de l'âme. Pourtant, Aristote semble avoir été étranger à cette problématique et il ne précise pas si cet intellect est individuel ou non, ni quel rapports l'unissent avec les parties sensibles et avec le comportement moral de l'homme. Certains aristotéliciens, en soulignant l'unité intime et l'impossibilité de séparer le corps et l'âme, en arrivèrent à nier l'immortalité de l'âme individuelle (Alexandre d'Aphrodise et Averroès), alors qu'à l'inverse d'autre soutiennent l'immortalité individuelle (Thémistius, Avicenne).

C'est à Thomas d'Aquin que revient le mérite d'avoir tenté de faire de la théorie aristotélicienne un système plus organisé et, en particulier, de s'être efforcé de concilier la théorie de l'âme comme forme du corps avec la conception platonicienne et chrétienne de l'âme comme substance autonome. L'âme intellective est l'unique forme substancielle de l'homme et c'est donc l'unique principe de l'existence humaine. Il n'y a pas en l'homme une âme végétative et une âme sensitive distinctes de l'âme intellective, parce que celle-ci, dans la mesure où elle est d'une perfection plus accomplie, peut s'acquitter des fonctions des formes inférieures : les manifestations corporelles sont des déploiements et des réalisations de la vie spécifiquement humaine. Pour Thomas, toutefois, l'âme est aussi une substance spirituelle, parce que c'est une forme  {\it subsistante}, qui, à la différence des autres formes qui meuvent la matière, a un être propre et ne particpe pas uniquement de l'être de l'individu : ceci est démontré par le fait que l'âme a des activités (la connaissance des universaux et la conscience de soi) auquelles le corps ne participe pas.

À la Renaissance, on voit réapparaître les deux positions antagonistes : les platoniciens soutiennent la spiritualité et l'immortalité de l'âme (Marsile Ficin), alors que les aristotéliciens, reprenant les doctrines d'Alexandre d'Aphrodise et d'Averroes, la nient (Pietro Pomponazzi). Dans la philosophie moderne, on insiste de moins en moins sur l'unité de l'homme, ce en faveur d'un dualisme renouvelé : chez Descartes, la dualité du corps ({\it res extensa}) et de l'âme ({\it res cogitans}) est radicale, parce que le corps existe et se meut en vertu de principes propres, purement matériels, alors que l'âme est conscience pure, qu'elle constitue l'essence de l'homme en tant qu'il est sujet pensant, et qu'en tant que telle elle peut même exister sans le corps. La distinction du corps et de l'âme est radicalisée par l'occasionalisme de Malebranche (il voit dans la pensé et dans la volonté de pures occasions d'action de Dieu sur le corps) et par la théorie de l'harmonie préétablie de Leibniz. La critique des divers dualismes soutenus par les pensées classiques trouve une expression ferme chez un philosophe comme Hume. Il se fonde sur une vision de l'âme comme faisceau de faits ou d'évènements psychiques en perpétuel mouvement; c'est de là que provient l'interprétation matérialiste ultérieur qui considère les phénomènes de conscience comme un reflet intérieur des processus physiologiques (parallélime psychophysique).

La critique postidéaliste du concept de l'esprit comme opposé au corps ou comme négation de celui-ci a relancé, au {\footnotesize XX}$^{\text{e}}$ siècle, un discours philosophique voyant dans l'âme un principe vital et non pas purement spirituel et rationnel; l'âme est ici un principe inconscient, nocturne et naturel, qui s'oppose non au corps mais à l'esprit en tant que principe abstrait.

%%%%%%%%%%%%%%%%%%%%%%%%%%%%%%%%%%%%%%%%%%%%%%%%%%%%%%%%%%%%%%%%%%%%%%%%%%%%
