
\section{Pratique de la philosophie}

« La vie est l'ensemble des fonctions qui résistent à la mort ». Par cette affirmation, M. F. X. Bichat — médecin et anatomiste français du XVIII{$^\text{e}$} — entendait insister sur le fait que la mort est la règle et la vie l'exception, par définition menacée. Les recherches actuelles tendent à montrer que la mort est programmée génétiquement, ruinant ainsi définitivement le vieux rêve d'immortalité. En effet, la mort n'est pas seulement, pour l'homme du moins, un simple fait biologique. L'homme est en effet le seul animal qui sache qu'il doit mourir, le seul pour qui la mort est ce qui inscrit la vie dans la précarité. Aussi est-elle source d'angoisse. C'est pourquoi l'homme a cherché à y opposer une croyance en un au-delà, croyance qui est au fond de toute religion. De son coté, la philosophie, dans sa quête de la vérité, se place « {\it sub specie aeternitatis} » (“de toute éternité”) ou avec Épicure, cherche à nous convaincre que « la mort n'est rien pour nous ». Pourtant, la pensée contemporaine et notamment l'existentialisme, loin de se détourner de la mort, la met au centre de sa reflexion, dans la mesure où c'est sur l'horizon de la mort que toute vie humaine peut prendre sens.

\subsection{Philosopher c'est apprendre à mourir}

Le {\it Phédon} est un dialogue de Platon qui met en scène Socrate, à la veille de sa mort, discutant avec ses amis de la question de l'immortalité de l'âme, et qui s'achève sur cette idée que c'est un beau risque à courir. C'est que l'âme est parente et amie des idées, réalités intelligibles, éternellement identiques à elles-mêmes, non soumises à la corruption et au chagement comme le sont les choses sensibles qui en sont la copie imparfaite et grossière. La patrie du philosophe est le « ciel des idées », qu'il cherche à atteindre et que la mort lui permet d'approcher. À l'affirmation paradoxale de Platon selon laquelle « philosopher c'est apprendre à mourir », semble répondre, pour la contredire, cette proposition de Spinoza au livre IV de l'éthique, selon laquelle la philosophie est « une méditation non de la mort mais de la vie ». À vrai dire, Spinoza vise ici l'attitude morbide qui se complaît dans la fascination de la mort et condamne l'homme à l'impuissance et à la tristesse. Si Spinoza invite ainsi à se détourner de la mort, c'est qu'elle est, selon lui, une pensée inutile. Par-delà le christianisme, il rejoint ainsi la sagesse antique du stoïcisme et de l'épicurisme.

\subsection{La mort n'est rien pour nous}

Épicure, dans sa {\it Lettre à Ménécée}, exprime avec une force d'argumentation convaincante cette idée que la mort ne doit pas nous faire renoncer au bonheur, puisque « la mort n'est rien pour nous ». À travers cette affirmation paradoxale, il entend rejeter au loin la crainte de la mort, qu'il juge absurde. En effet, « tant que nous existons, la mort n'est pas, et quand la mort est là nous ne sommes plus ». Pour Épicure, l'âme est un corps subtil, voué, comme le corps à la désagrégation. Elle est le siège de la sensibilité, et lorsqu'elle meurt, meurt aussi la sensibilité. La mort ne saurait donc faire l'objet d'aucune expérience « vécue », elle ne peut être éprouvée. Et si nous sommes convaincus que la mort est la fin de tout, nous n'aurons ni à redouter ni à espérer une autre vie. Cette vie est au contraire la seule qui puisse nous apporter le bonheur, pourvu qu'elle soit sereine face à la mort.

\subsection{La mort à l'horizon de la vie}

Pourtant, s'il est vrai que notre propre mort est un évènement auquel nous n'assisterons pas, et qu'elle n'est rien pour nous, la mort d'autrui, ou la mort « en seconde personne », comme la qualifie Vladimir Jankélévitch, nous place devant un scandale qui est celui de la perte d'un être irremplaçable et unique. Mais c'est sa propre mort que l'homme doit prendre en charge, car nul autre ne le peut pour lui. C'est pourquoi Heidegger nous invite, dans une perspective pourtant très différente de penseurs religieux comme Pascal ou Kierkeggard, à prendre au sérieux notre propre mort, non comme évènement toujours à venir et en même temps absolument sûr, mais comme horizon à partir duquel surgit la pensée du néant, inscrit au sein même de l'existence. Et si cette conscience du néant s'éprouve dans un sentiment d'angoisse, c'est qu'avec elle nous sommes jetés au monde. L'angoisse ne doit pas être confondue par conséquent avec la simple crainte de mourir. L'angoisse exprime au contraire le fait que notre existence n'a de sens que parce qu'elle est « pour-la-mort », en quelque sorte « orienté » par elle. La mort est, fondamentalement, ce à partir de quoi la vie peut prendre un sens. Assumer notre condition d'être mortel nous oblige à prendre en charge la responsabilité de notre propre vie, mais ne signifie donc nullement l'obligation de méditer sur la vanité de toutes choses.

\begin{itemize}[leftmargin=1cm, label=\ding{32}, itemsep=1pt]
\item {\bf Textes clés} : Épicure, {\it Lettre à Ménécée}; Platon, {\it Phédon}; M. Heidegger, {\it Être et temps} (première partie); V. Jankélévitch, {\it La Mort}.
\item {\bf Termes opposés} : immortalité; vie
\item {\bf Terme voisin} : néant.
\item {\bf Corrélats} : âme; crime; existence; mal; sens; temps; vivant.
\end{itemize}


