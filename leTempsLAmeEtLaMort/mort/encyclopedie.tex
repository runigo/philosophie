

\section{Encyclopédie de la philosophie}

La langue française distingue la mort du {\it décès}, cette dernière expression, notament employée en français dans le contexte juridique, désigne en général la mort d'une personne, d'un être humain (On ne dira pas d'un animal qu'il est décédé). La mort se définit généralement dans sa relation nécessaire avec la vie. Ce lien de la vie et de la mort, est du reste reconnu par la science, elle est un fait biologique universel. C'est en ce sens que le médecin français Xavier Bichat, dans la première partie de ses {\it Recherche physiologique sur la vie et la mort} (1800), définissait la vie « l'ensemble des fonctions qui résiste à la mort ». Et Claude Bernard ajoute dans son {\it Cahier de notes} que l'élément primordial de cette définition était moins l'opposition des deux forces en lutte, que l'affirmation de leur liaison nécessaire : « nous ne distinguont la vie que par la mort et inversement ». Mais la mort n'est bien sur pas uniquement un phénomène naturel et l'objet d'investigation scientifique. Pour les vivants que nous sommes, elle est aussi le thème d'un savoir paradoxalement certain et confus (la certitude de sa propre mortalité naît au contact de la mort des « autres » en général), l'occasion d'une souffrance déchirante (le décés d'un être cher), l'aliment d'une préoccupation anxieuse (« ma » mort, à la fois certaine et imprévisible).

\subsection{Le discours philosophique sur la mort}

Sans être toujours indépendant du discours sur la mort propre à la culture à laquelle elle est historiquement liée, la réflexion des philosophes ne se contente jamais de transposer ou d'intégrer ces éléments mythiques religieux ou rituels dans son discours sur la mort. L'exemple le plus frappant de la transposition philosophique d'éléments mythiques est sans doute à chercher dans la pensée platonicienne qui lie l'idée de l'immortalité de l'âme à sa position à l'égard de l'existence des Formes intelligibles. Certes, à parler en rigueur, on ne trouvera pas dans le {\it Phédon} de preuve de l'immortalité de l'âme. Il semble au contraire que la réminiscence ou l'anamnèse viennent étayer une conviction (gr. {\it pistis}), « une opinion vraie accompagné de raison » (gr. {\it aléthès doxa meta logou}), certes pas un savoir irréfutable qui procurerait à son détenteur une certitude telle qu'il se trouverait en sécurité par rapport à sa propre mortalité. Pour autant, l'idée que l'exercice de la philosophie a quelque chose à voir avec un détachement à l'égard du sensible, une épuration des sentiments et des affects liés à la vie périssable, bref que « philosopher, c'est avoir soin de son âme », qui est comme en prison dans cette vie {\it Phédon}, sont autant de thèmes issus de la réflexion platonicienne qui ont pour effet de réintroduire le rapport à la mort au c{\oe}ur de la reflexion sur l'existence. La philosophie {\it ars vivandi} est aussi un {\it ars moriendi}. Comme l'écrit Françoise Dastur ({\it La mort}, 1994) : « L'invention de la philosophie coïncide avec celle d'un autre discours sur la mort que celui que propose la mythologie ou la théologie, ce qui implique d'emblée une homologie entre la mort et la philosophie, laquelle sera l'horizon de tout le discours platonicien sur la mort ». Cependant ce discours n'est pas celui d'un savoir, mais celui d'un risque, d'une sorte de pari. Parier sur l'existence de l'idée du Bien et croire à l'immortalité de l'âme sont au total deux conjectures rationnelles qui sont liées et enracinées dans une même démarche consistant à réintégrer au c{\oe}ur de l'existence le rapport à la mort, fût-ce de manière ambivalente chez Platon dans la mesure ou elle est envisagée comme une libération de l'âme. À l'inverse, la démarche d'Épicure tend au contraire à vider de tout contenu anxiogène le fait de la mortalité : la mort ne saurait en aucun cas être objet de crainte pour qui que ce soit, puisque lorssqu'elle est là, le sujet n'est plus là et tant qu'il est là, ma mort n'est pas là. « La mort n'est rien pour nous » ({\it Lettre à Ménécée} 125), déclare le sage épicurien, cela signifie d'abord que la mort n'est essentiellement inexpérimentable et que se déprendre du sentiment d'effroi qu'elle fait peser sur l'existence des hommes est l'une des tâches de la sagesse. Il faut pour le sage épicurien jouir de son existence mortelle et lutter contre l'idée d'immortalité dans la mesure où elle tend à attiser la crainte de la mort. Une autre manière de déréaliser la mort, caractéristique du mécanisme de l'Âge classique, est de l'envisager d'un coté comme un phénomène matériel qui correspond au dérèglement et à la désagrégation du corps machine, de l'autre la mort n'est pas complète et le dualisme de Descartes ou de Malebranche laisse une place à l'immortalité de l'âme dont on peut montrer la possibilité rationnelle. Le monisme de Spinoza semble aller contre l'expérience courante lorsqu'il affirme que « nous sentons et expérimentons par expérience que nous sommes éternel » ({\it Éthique} V, prop. XXIII, scolie). Il commente cette formule surprenante en précisant que « l'âme, en tant qu'elle enveloppe l'essence du corps avec une sorte d'éternité est éternelle ». Certes, ce n'est l'affirmation claire de l'immortalité personnelle, mais Spinoza soutient qu'il y a dans l'âme ou dans l'esprit quelque chose qui est éternel. Ici la croyance en l'impérissabilité se fonde dans la croyance dans l'indépendance de l'âme et de la pensée et à l'égard du temps. L'optimisme de Leibniz, en pensant la naissance et la mort comme un développement et un enveloppement, participe aussi de ce qu'on pourrait appeler une mise entre parenthèse du rapport à la mort dans le discours philosophique. Il va jusqu'à affirmer dans {\it Les principes de la nature et de la grâce} (1714) que « les animaux ne naissent pas entièrement dans la conception ou {\it génération}; et ne périssent pas entièrement dans ce que nous appelons mort » et que « non seulement les âmes mais aussi les animaux sont ingénérable et impérissable : ils ne sont que développés, enveloppés, revêtus, dépouillés, transformés » . La raison philosophique classique, sans exclure à proprement parler la mort de son discours, ne la place au centre de ses préocupations, allant jusqu'à nier en divers manière qu'elle soit à l'horizon de toute expérience humaine.

\subsection{La mort re-pensée comme prix de la vie}

Le retour de la mort au c{\oe}ur des réflexions philosophique peut-être daté de la fin du {\footnotesize XVIII}$^{\text e}$ siècle et de l'émergence des courant idéalistes et romantiques. On trouve ainsi chez Novalis une affirmation qui aurait certainement été peu intelligible aux hommes de l'{\it Aufklärung} : L'acte philosophique authentique est le suicide ; c'est là le commencement réel de toute philosophie, c'est là que tendent tous les besoins du futur philosophe et seul cet acte est conforme aux conditions et aux caractéristiques d'une action transcendantaliste« » ({\it Fragments}). George Wilhelm friedriche Hegel va s'efforcer de ressaisir ce que la raison classique avait plus ou moins rejeté comme son autre. La mort  devient la condition de la vie vraiment vivante. Ainsi qu'il le formule avec force dans la Préface de sa {\it phénoménologie de l'esprit} (1807) : « ce n'est pas cette vie qui recule d'horreur devant la mort et se préserve pure de la destruction, mais la vie qui porte la mort et se maintient dans la mort même qui est la vie de l'esprit ». La dénégation de la mort doit céder la place à une pensée qui tire sa force de son rapport à sa propre limitation. Séjourner auprès de son propre anéantissement, faire face à sa finitude, faire ce que Hegel appel une {\it Aufhebung} de la mort, devient la condition de toute réalisation effective de l'esprit. « Ce séjour est le pouvoir magique qui convertit le négatif en être. » C'est en se faisant familier de la mort, en l'apprivoisant selon le mot de Montaigne, que la raison humaine finie peut s'accomplir. Pour Arthur Schopenhauer, dans {\it Le monde comme volonté et comme représentation} (suppléments c. LVI, 1844), le renversement du discours philosophique à propos de la mort est encore plus net puisque Schaupenauer fait de la mort « le génie inspirateur ou le musagète de la philosophie » et de préciser aussitôt que « sans la mort, il n'y aurait sans doute pas de philosophie ». Il soutient la thèse d'une connexion nécessaire entre l'apparition de la raison chez l'homme et la certitude effrayante de la mort. La crainte de la mort n'est, pour schopenhauer, que le « revers de la volonté de vivre ». Cette thèse n'est pas sans préfigure les représentations que Freud a pu donner, du point de vue de la psychanalyse, du rapport de l'homme des origines à la mort ({\it Totem et tabou}, 1913; {\it Actuelles sur la guerre et sur la mort}, 1915) : la naissance de l'idée d'esprit ou d'âme immortelle serait la traduction psychique de l'ambivalence des sentiments exprimés à l'égard du défunt. En fait, si l'on suit la critique de Paul Ric{\oe}ur dans {\it De l'interprétation} (1965), il semblerait que Freud ait davantage réintroduit le schème tragique dans le matériau de l'ethnologie historique, alors naissante, plutôt qu'il ait véritablement mis au jour l'archéologie des conduites à l'égard de la mort. 

\subsection{Mort propre et mort d'autrui}

À la différence du discours de la philosophie classique sur la mort, l'existentialisme et la phénoménologie expriment des courants qui ne visent pas a surmonter la mort par une construction mythologique, théologique, religieuse ou philosophique, ni a l'objectiver, mais a interpréter le sens de l'être mortel dans le contexte plus général d'une herméneutique de l'existence humaine. Si l'expérience de la mort ne se donne pas sur un autre mode que la mort de l'autre, le sens même de l'être mortel fait partie intrinséque de l'existant humain que Heidegger nomme {\it Dasein}. Le rapport à la mort doit être assumé par chaque {\it Dasein} qui doit prendre sur lui son mourir propre : « La mort, pour autant qu'elle “soit”, est a chaque fois essentiellement la mienne » ({\it Être et Temps}). L'être pour la mort, le {\it Sein zum Tode}, ne dit pas que le {\it Dasein} vit de manière morbide le rapport à sa propre fin, plutôt que c'est pour lui la condition d'un rapport plus authentique aux possibilités d'existence et pour cette possibilité ultime qu'est la mort pour un {\it Dasein}.
Cette dernière a ceci de particulier qu'elle est une pure possibilité qui est toujours recouverte par la vie quotidienne du {\it Dasein} dans la mesure où elle est une possibilité de l'absence de tout possible, la possibilité de l'impossibilité de tout exister. Avoir un rapport authentique à la mort consiste alors pour Heidegger à laisser se déployer cette possibilité comme telle, c'est-à-dire comme possibilité, la « liberté du {\it Dasein} par rapport à la mort » étant à ce prix.

\subsection{Approches anthropologique}

Les historiens des religions et les ethnologues ont abordé le phénomène de la mort, surtout dans les cultures archaiques et primitives, comme un « fait social », un événement qui détermine une crise, non seulement a l'intérieur du groupe familial, mais aussi dans celui, plus ample, de l'ascendance, de la descendance, du clan, de la tribu. Il s'agissait de montrer comment les structures sociales réagissent à la mort à travers une série de procédés mythiques et rituels qui amènent les individus a vivre la mort selon les paradigmes offerts par la société. Cette approche de la mort, marquée par l'anthropologie et l'ethnologie historique, ne saurait étre complète si elle ne tenait pas compte des  diverses maniéres dont le discours philosophique, classique ou traditionnel a envisagé le phénomène de la mort, non pas tant en montrant comment les groupes humains ou les cultures réagissent face au décès, mais en essayant de penser en quoi le caractère fini de l'existence humaine joue un rôle déterminant pour la compréhension de son sens.

\subsection{Mort, vengeance, au-delà}

La mort est interprétée par les ethnologues et les anthropologues comme une crise du groupe qui, a cause de la superposition des motivations mythiques et rituelles qui l'accompagnent, déterminent des réactions opposées, mais, au fond, logiques : la douleur, la perte — imputées à un ennemi ou a un événement qui a troublé l'ordre — entrainent comme conséquence la recherche d'actions concrétes de vengeance pour rétablir la situation normale. Puisqu'il conçoit la mort comme un événement non naturel, le groupe tend à en déterminer les causes par les méthodes divinatoires propres à chaque culture. Connaître la raison de la mort sert a reprendre en quelque sorte le
contrôle de la situation et a fixer les actions qui doivent être faites, étant donné que la responsabilité du deuil peut retomber sur le défunt lui-même (a cause de la violation d'un tabou, a cause d'un péché ou d'une faute commise dans le déroulement d'un rite), ou sur d'autres membres du groupe (des membres de la famille, des ennemis ou des adversaires du mort) qui auraient agi directement ou par le truchement d'opérations de magie noire ou de sorcellerie. Le diagnostic propose a la famille ou au groupe un devoir de vengeance, qu'il faut réaliser dans la réalité ou magiquement contre le responsable. Les témoignages ethnologiques sur les processus divinatoires sont trés nombreux. Les aborigénes australiens, par exemple, procédent a une enquête presque a chaque nouveau cas de mort (A. P. Elkin, {\it Aboriginal Men of High Degree. Initiation and Sorcery in the World's Oldest Tradition}, 1938), et toujours, lorsque le défunt est un jeune adulte de sexe masculin (ce qui s'explique aisément du moment que, dans la réalité des faits, le groupe est formellement constitué par des adultes masculins initiés, et que la mort de l'un de ses membres frappe par conséquent le groupe, comme si ce dernier avait été mutilé ou mis en danger dans son ensemble). Dans de nombreux cas, la personne qui est soupconnée en tout premier lieu est l'épouse, soit parce qu'elle appartient — là où s'applique la norme exogamique — à un groupe étranger, et par conséquent en quelque sorte potentiellement hostile, soit parce qu'elle a l'occasion, estime-t-on — ne serait-ce que par la non-observance des tabous menstruels —, de provoquer la mort du mari (Marilyn Strathern, {\it Women in Between : Female Roles in a Male World}, 1972). Dés que l'assassin a été identifié avec certitude par la mise en {\oe}uvre de plusieurs méthodes, la vengeance est directement exercée sur le responsable ou sur le groupe auquel il appartient ; tandis que, si la responsabilité remonte au défunt lui-même, pour avoir violé quelque norme rituelle, nous sommes en présence de la « malemort » (mauvaise mort), décès soudain et terrible, qui frappe le violeur (nombreux sont les exemples de ce type cités par H. Webster, {\it Le Tabou}, 1942, trad. fr. 1952).

La mort n'est pas vue toutefois comme la fin de l'existence : l'individu quitte ce monde pour entrer dans le monde de l'au-delà, qui est percu comme « puissant », source de terreur, ce qui confére par conséquent au mort une agressivité virtuelle ou, au contraire, une force bienveillante. Cette idée que la mort est un passage, le passage d'un monde à un autre, ce que le mot francais « trépas » évoque particuliérement bien. Transformé en « double », en « fantôme », en « ombre », en « spectre », le mort devient redoutable, car il a été arraché par une surabondante énergie vitale, a laquelle il reste attaché. Se constitue ainsi un rapport particulier entre le mort et les survivants, qui sont tenus de nourrir grace à des offrandes et d'apaiser la soif de vivre qui est encore présente dans le « double » ou « fantôme », en courant le risque de s'exposer a ses violences maléfiques ou destructrices au cas où les obligations prescrites ne sont pas accomplies. Dans la majorité des cas, le mythe et le rite visent à affaiblir la personnalité du défunt, qui se manifesterait surtout dans la période qui suit immédiatement le décès. C'est de là que tire son origine également la coutume des différents types de sépulture qui se sont succédé. Dans de nombreux cas, le défunt deviendra l'ancêtre, dont la fonction est de protéger son groupe et sa dynastie ; ou bien il sera effacé de la mémoire, comme cela a lieu par exemple avec l'abandon du cadavre, ou carrément parfois avec l'abandon et la mise à feu de tout le village. Pour ce qui concerne les femmes, il est presque certain qu'elles se transforment rarement en ancétres (même si l'utilisation courante du pluriel masculin par les ethnologues permet difficilement de distinguer les cas particuliers) ; mais dans la société des {\it vedda} (île de Ceylan), il existe, à côté du terme masculin {\it yaka}, un terme féminin particulier pour indiquer la puissante ancêtre, {\it yakini}, chez ces populations, les esprits féminins apparaissent souvent comme maléfiques, préts à enlever et tuer des enfants, à leur communiquer des maladies, alors que les esprits masculins sont toujours bienveillants (C. G. Seligman, {\it The Veddas}, 1911).

\subsection{Origine et représentations de la mort}

La mort étant considérée comme un phénoméne étranger a la nature originelle de homme, nombreux sont les mythes
qui expliquent comment elle est entrée dans le monde et a modifié une condition primordiale de plénitude vitale. Ce changement dépend du péché (ainsi dans le judaisme et dans le christianisme), ou de la violation d'un tabou imposé a l'origine, ou enfin de quelques événements mythiques qui introduisent la mort dans le monde, indépendamment de la volonté, ou de la responsabilité, des hommes. Dans le mythe, ce n'est pas tant le pourquoi de l'origine de la mort qu'on s'efforce d'établir, que la maniére dont elle a été introduite et le moment où elle l'a été. Trés souvent l'intermédiaire par lequel la mort est entrée dans le monde est la femme, ou encore, la mort elle-même assume l'image de la femme. Cette conception, trés répandue dans des aires culturelles variées, est liée à la physiologie particuliére de la femme, qui est interprétée comme le signe que la femme est à la limite, à la frontière entre la « nature » et l'au-delà, un au-delà qui existe toujours comme monde des morts, comme monde qui précéde la vie et vient aprés la mort. Chez les Cagabas, par exemple, peuplade amérindienne de la Sierra Nevada de Santa Marta, c'est la déesse {\it Gautéovan}, la Mère originelle, qui crée avec le sang de ses menstrues d'abord le soleil, puis toutes les autres choses, y compris les esprits de la maladie et de la mort (J. Curtin, {\it Creation Myths of Primitive America}, 1899). Dans toute l'aire indo-européenne, la Déesse Mére est liée à la mort et au monde des morts. Chez les Grecs, Hécate, divinité des Enfers, reine des spectres et des ombres, apparait comme une particulière épiphanie lunaire d'Artémis, divinité néfaste et vengeresse. Artémis frappe mortellement de ses fléches, elle est la patronne de la mort soudaine. Perséphone elle-même, la jeune fille du mythe de l'enlévement qui est a la base des cultes d'Eleusis, est une figure de la mort, instrument de communication et de passage entre le monde terrestre et le monde des Enfers. En Gréce toujours, les Erynnies, divinités infernales, sont représentées sous la forme de serpents, car le serpent symbolise les esprits de la mort (G. Thomson, {\it Eschyle et Athénes}, 1940). La triade des Moires, qui apparait pour la première fois dans l'{\oe}uvre d'Hésiode ({\it Théogonie}), préside au destin de l'homme et fixe le moment de sa mort. La {\it Moira}, divinité qui tisse le fil de la vie et
y met fin en le brisant, est commune aux Romains et aux Germains (chez lesquels les Moires apparaissent sous le nom de Parques et de Nornes). Dans le domaine de la religion juive, c'est par l'intermédiaire d'Eve que l'homme a été condamné a mort : c'est ainsi que, même dans de trés nombreux mythes indo-américains, la mort est souvent introduite dans le monde par le geste inconsidéré d'une femme. La connexion mort-femme assumé par l'image de la mort, aspect présent dans de nombreuses aires culturelles. Chez les Bambara du Haut Niger, la mort est une « femme », objet d'union sexuelle, car c'est dans le coït que réside l'énigme de la douleur (D. Zahan, {\it Société d'initiation bambara}, 1960). Le caractére féminin de la mort est aussi présent dans le lien fréquemment fait entre la femme et la Lune, ainsi que dans la bipolarité lumière / ténèbres ; et, comme l'a noté V. Propp ({\it Les Racines historiques des récits de fées}, 1946), la sorcière des fables est elle aussi un personnage qui vient du monde des ombres et des morts.

\subsection{Significations de la mort}

Dans le modèle mythique le plus fréquent, la mort assume le sens d'un « passage », et parfois celui d'une « épreuve », grâce à laquelle on accède a une condition différente, mais qui, en tout cas, assure la continuité de l'existence dans une autre vie. Cette condition peut consister dans le retour a l'intégrité physique et psychique et a la perfection originelle, dans laquelle l'homme jouissait de l'immortalité comme de son état naturel; elle peut consister aussi dans le passage a une nouvelle existence, plus ou moins correspondante à l'existence terrestre, qu'elle continue à l'infini (et l'homme peut vivre cette nouvelle vie soit dans sa plénitude, soit sous la forme d'une ombre). Parfois la mort peut représenter l'affranchissement des limites de l'individualité : en conséquence de quoi, l'homme (sous une forme non individuelle) ou son esprit (affranchi des caractères personnels) est absorbé dans le tout. Une autre représentation eschatologique concoit la mort comme le moment ou l'on acquiert une dimension absolument différente de la dimension terrestre, une dimension affranchie de la corruption et du péché inhérents à la chair, où l'on revêt un nouveau corps glorieux ou
« pneumatique ». Il peut y avoir enfin l'identification du défunt (de son âme ou de son double) avec le dieu, modèle exemplaire de l'immortalité, intemporel et incorruptible. Souvent, dans les cas où la mort est conçue comme passage a un autre état, apparaît nécessaire un comportement spécifique de l'homme (observation des rites, purification du péché, innocence, etc.) ou même une « révélation » de type initiatique, qui permette à l'homme de connaître la réalité fondamentale qui se cache derrière la mort, sa fonction déterminante dans le nouveau cycle de vies. On a ainsi différentes conceptions qui font disparaitre l'angoisse et la crise liées à la mort, dans une perspective individuelle ou collective qui peut se réaliser {\it una tantum} (unique, exceptionnel), pour chaque mort en particulier ou pour les morts dans leur totalité (jugement individuel ou jugement final), perspective qui est située dans un temps éloigné, indéfini; on a également des conceptions religieuses, qui se fondent sur la croyance dans la réincarnation et dans la transmigration des âmes, et qui font donc de la mort le passage a une nouvelle forme de vie. Nouvelle forme de vie qui ne constitue pas encore l'affranchissement définitif pour l'homme de sa condition mortelle, mais ne fait que consister dans l'écoulement d'un certain cycle et dans le tarissement d'une certaine charge de mal et de négativité, également d'ordre moral (comme par exemple dans les mystères orphiques).

Dans la seconde moitié du {\footnotesize XX}${^\text{e}}$ siècle, les historiens francais, influencés par l'anthropologie, se sont tournés vers l'étude de la mort dans la culture occidentale, et en ont fait apparaître certaines caractéristiques, de refoulement, d'éloignement; ou, au contraire, des raisons d'espérer dans une « réanimation » future, qui peut être obtenue grace aux conquêtes de la science (par exemple, l'hibernation, qui est a la mode aux Etats-Unis). Edgar Morin ({\it L'Homme et la mort}, 1951), W. Fuchs ({\it Les Images de la mort dans la société moderne}, 1969), F. Lebrun ({\it Les Hommes et la mort en Anjou aux {\footnotesize XVII\;}${^\text{e}}$ et {\footnotesize XVIII\;}${^\text{e}}$ siècles}, 1971), L. V. Thomas ({\it Anthropologie de la mort}, 1975), J. Baudrillard ({\it L'Echange symbolique et la mort}, 1976), P. Aries ({\it L'Homme devant la mort}, 1977) : voila certains des auteurs les plus importants qui se sont occupés du problème de la mort, tel qu'il est vécu en
Occident, après qu'a été dépassée (partiellement tout au moins) la conception chrétienne d'un « passage » à une autre vie.


\begin{center}
\setlength{\fboxsep}{3mm}
\fbox{
$\to$ {\it Aufhebung} $\bullet$ existencialisme $\bullet$ facticité $\bullet$ finitude $\bullet$ immortalité $\bullet$ métempsycose
}
\end{center}

%%%%%%%%%%%%%%%%%%%%%%%%%%%%%%%%%%%%%%%%%%%%%%%%%%%%%%%%%%%%%%%%%%%%%%%%%%%%%%%%%%%%%
