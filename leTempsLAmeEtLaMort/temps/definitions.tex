
\section{Dictionnaire encyclopédique}

\begin{itemize}[leftmargin=1cm, label=\ding{32}, itemsep=1pt]
\item {\bf 1} : Durée dans laquelle se succède les évènements, les actions, les jours et les nuits; cette durée mesurable.
\item {\bf 2} : Moment d'une action, période de l'année, de l'histoire.
\item {\bf 3} : Moment propice; occasion.
\item {\bf 4} : État de l'atmosphère. {\it Un temps orageux}.
\item {\bf 5} : {\footnotesize \sf MUS.} Division de la mesure servant à régler le rythme.
\item {\bf 6} : {\footnotesize \sf GRAM.} Série des formes du verbe marquant le temps (présent, passé ou futur)
\item {\footnotesize \bf Loc.} : {\it À temps} : au moment convenable, voulu. {\it Être de son temps} : se conformer aux usages de son époque. {\it En même temps} : simultanément. {\it De tout temps} : depuis toujours. {\it Tout le temps} : sans cesse. {\it De temps en temps} : quelquefois. {\it Gros temps} : tempête. {\it De mon temps} : dans ma jeunesse. {\it Perdre son temps} : ne rien faire. {\it Gagner du temps} : temporiser. {\it N'avoir qu'un temps} : être de courte durée. {\it Avoir fait son temps} : être dépassé.
\end{itemize}

%%%%%%%%%%%%%%%%%%%%%%%%%%%%%%%%%%%%%%%%%%%%%%%%%%%%%%%%%%%%%%%%%%%%%%%%%%%%%%%%%%%%%
