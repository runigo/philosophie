
%%%%%%%%%%%%%%%%%%%%%
\section{Repères biographiques}
%%%%%%%%%%%%%%%%%%%%%

Emmanuel Kant (1724-1804) est né en Prusse, à
Kônigsberg, où il devient professeur
et qu'il ne quittera jamais. Il reçoit de
sa mère, luthérienne piétiste, une
éducation morale très rigoureuse.
Son existence est austère et d’une
régularité qui, raconte-t-on, ne sera
troublée qu’en deux occasions : la
lecture de l'{\it Émile}, de Rousseau, et la
nouvelle de la Révolution française.
Homme engagé dans les débats de
son siècle (sur les Lumières, les
droits de l’homme...), Kant laisse une
œuvre considérable. Celle-ci aura sur
la modernité une influence déterminante.

%%%%%%%%%%%%%%%%%%%%%%%%%%%%%%%%%%%%%%%%%%%%%%%%%%%%%%%%%%%%%%%%%%%%%%%%%%%
