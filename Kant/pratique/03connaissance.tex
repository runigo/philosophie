%%%%%%%%%%%%%%%%%%%%%
\section{La philosophie de la connaissance}
%%%%%%%%%%%%%%%%%%%%%


\subsection{Criticisme, dogmatisme et scepticisme}

La raison est naturellement dogmatique ;
c'est-à-dire qu’elle use de son pouvoir
de connaître sans s'interroger préalablement
sur les conditions de légitimité de
ce pouvoir. Faute d’une telle interrogation,
la métaphysique traditionnelle s’est
perdue dans des contradictions interminables,
là où les autres sciences ont
assuré leurs principes et leurs résultats.
D'où la réaction sceptique (celle de
Hume par exemple) qui, devant les
échecs dogmatiques de la raison, la proclame
impuissante. Le scepticisme a le
mérite de réveiller la raison de son
« sommeil dogmatique »; mais il a le
défaut de la condamner plutôt que de la
juger, de lui assigner des bornes plutôt
que de lui prescrire des limites à l’intérieur
desquelles son pouvoir serait
garanti. Aussi le criticisme veut-il instituer
un tribunal de la raison et lui poser
la question de droit (et non seulement
de fait) : à quelles conditions sa prétention
à connaître des objets est-elle autorisée ?

\subsection{La théorie des facultés}

Les philosophes ont communément
opposé deux genres de sciences : d’un
côté les sciences expérimentales ou
« empiriques », de l’autre les sciences
déductives, exemplairement représentées
par les mathématiques. Kant récuse
doublement cette distinction. Les mathématiques
sont évidemment {\it a priori},
%
mais elles ne sont pas analytiques et ne
se réduisent pas à la logique. Quant aux
sciences physiques, elles sont bien sûr
synthétiques (elles fournissent des
informations sur le réel), mais elles ne
sont pas entièrement fondées sur l’expérience :
en effet les lois formulées
dans le cadre des sciences de la nature
valent universellement et nécessairement ;
or l'expérience n’enseigne ni la
nécessité, ni l’universalité. Quel que
soit donc le type de science physique
considéré, ses principes reposent sur
des jugements à la fois synthétiques et {\it a
priori}.

Nous connaissons donc quelque chose
{\it a priori} des objets. Mais quoi ? Kant
répond en distinguant deux grandes
facultés de l'esprit humain : la sensibilité
et l’entendement. Par la première,
les objets nous sont donnés dans des
intuitions sensibles ; par le second, ils
sont pensés, mis en relations, de sorte
qu’existe pour nous une nature soumise
à un ordre et à des lois. La connaissance
a donc des conditions « subjectives »,
c'est-à-dire liées au sujet connaissant.
Comment, alors, rendre compte de sa
valeur objective ? Par le fait que le sujet
qui porte un jugement n’est pas Pierre,
Paul ou Jacques — la connaissance ne
varie pas avec chacun — mais que c’est
un sujet « transcendantal » : autrement
dit, c’est l'esprit humain en général qui
est organisé de cette façon, et les conditions
de la connaissance sont en même
temps subjectives et les mêmes pour
tous.

\subsection{La révolution copernicienne en philosophie}

Kant fait donc du sujet le centre de la
connaissance. Ce n'est pas le sujet
(l'esprit) qui doit se régler sur les objets,
mais l'inverse: la connaissance des
objets dépend des structures {\it a priori} de
la sensibilité et de l’entendement. Ce
changement de perspective, Kant le
compare à celui opéré par Copernic en
astronomie, lorsque celui-ci a affirmé
que ce n'était pas la terre, mais le soleil
qui était le centre immobile du mouvement
circulaire des planètes. Cette
« révolution copernicienne en philosophie »
doit permettre à la métaphysique
de s'engager enfin dans la voie sûre
d’une science.

\subsection{La critique de la métaphysique}

L'âme, le monde comme totalité, Dieu,
ont été les préoccupations traditionnelles
de la métaphysique, mais ce ne
%241
sont pas des objets de connaissance : ce
sont des « idées de la raison » ou « idées
transcendantales ». La métaphysique,
qui veut transformer ces idées en objets,
est donc une illusion ; mais c’est une
illusion inévitable, car ces idées de la raison
représentent chacune un
inconditionné dont la connaissance
permettrait d'achever l'unité du savoir.
Au plus fort de sa critique, Kant ne
renonce jamais au projet métaphysique.
D'une part, il incombera à la morale de
le réaliser ; d'autre part, la métaphysique
comme science reste possible : il suffit,
grâce à la « révolution copernicienne »
d’en infléchir l'objet. La {\it Critique de la
raison pure} constitue ainsi une sorte de
préface à un système de la raison pure
qui sera la métaphysique scientifique
future.

\subsection{La « critique de la faculté de juger »}

Les idées « transcendantales » possèdent
une autre vertu : même si aucun objet
de connaissance ne leur correspond,
elles gardent une fonction « régulatrice »,
en faisant tendre le savoir vers l'unité
qu'il recherche. C'est cette fonction
régulatrice de la connaissance que Kant
explore dans la {\it Critique de la faculté de
juger}, écrite dix ans après la {\it Critique de
la raison pure}. Celle-ci réduisait les
jugements de connaissance à des « jugements
déterminants » (qui déterminent
un objet au moyen d’un concept lui servant
de règle). Mais Kant découvre peu
à peu l'existence d’un autre type de
jugement, qu’il appelle « réfléchissant ».
Celui-ci ne produit pas une connaissance
objective, mais il est l'expression
subjective d’un ordre que nous devons
admettre dans les objets pour les
comprendre. Tel est le jugement
téléologique (de finalité) ; tel est également
le jugement de goût, c'est-à-dire
portant sur le beau.

%%%%%%%%%%%%%%%%%%%%%%%%%%%%%%%%%%%%%%%%%%%%%%%%%%%%%%%%%%%%%%%%%%%%%%%%%%%
