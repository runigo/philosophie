
\subsection{LE DEVOIR}

{\it Kant prétend, au départ, seulement rapporter une opinion commune autant que
banale : ce qu'on tient pour véritablement moral, c'est une bonne volonté. Les
autres dispositions (les talents de l'esprit, les qualités de caractère) ne peuvent
jamais être considérées comme bonnes en elles-mêmes, mais dépendent de l'usage
que notre volonté en fait : on peut user de son intelligence ou de son courage à des
fins mauvaises. La seule chose qui ait vraiment une valeur pour elle-même, c'est
une volonté bonne, c'est-à-dire qui agit par devoir.}

« Il est, par exemple, conforme au devoir que l’épicier ne fasse pas un prix plus
élevé au client inexpérimenté, et là où il y a beaucoup de négoce le marchand avisé
s'abstient de le faire, mais établit au contraire un même prix général pour tous, si
bien qu’un enfant achète chez lui au même prix que n'importe qui d'autre. On est
donc {\it honnêtement} servi chez lui ; pourtant, c’est loin d’être assez pour qu’on
puisse croire que le marchand a agi par devoir et par principe d’honnêteté ; son
intérêt l’exigeait ; en revanche, on ne peut pas admettre ici qu’il aurait dû, de plus,
éprouver une inclination immédiate pour ses clients, pour ne donner, en quelque
sorte par amour, aucun avantage de prix à l’un par rapport aux autres. L'action n’a
donc été accomplie ni par devoir, ni par une inclination immédiate, mais simplement
dans une intention intéressée [...]. Une action accomplie par devoir a sa
valeur morale {\it non dans le dessein}, qui doit être réalisé par son moyen, mais dans la
maxime, d’après laquelle elle a été décidée ; elle ne dépend donc pas de la réalité
de l’objet de l’action, mais seulement du {\it principe du vouloir} d'après lequel l’action
a été produite, indépendamment de tous les objets de la faculté de désirer.

\begin{flushright}
Kant, {\it Fondement pour la métaphysique des mœurs} (1785), I$^\text{re}$ section,

trad. O. Hansen-Lôve, Paris, Hatier, coll. « Les classiques Hatier de la philosophie », 2000,

pp. 21 et 25.
\end{flushright}

\subsubsection{Moralité et légalité}

Dans la première partie du texte, Kant donne l’exemple du marchand qui pratique
les mêmes prix pour tous ses clients. Pourquoi celui-ci se refuse-t-il à la malhonnêteté ?
Si l’on examine les mobiles de son action, on n’y trouvera certainement
pas le sentiment humanitaire (le commerce ne repose sans doute pas sur l'amour
%243
inconditionnel des clients). Le marchand serait-il alors inspiré par un respect pour
la loi morale qui prescrit l'honnêteté ? On comprend pourtant vite que c'est plutôt
par calcul et intérêt qu'il procède ainsi : une réputation de voleur lui ferait perdre
sa clientèle. Il agit conformément au devoir et non par devoir.

\subsubsection{Une morale de l'intention}

Cet exemple nous permet de comprendre qu'on ne décide pas de la valeur morale
d’un acte en considérant simplement son aspect extérieur : il faut savoir pourquoi
le sujet agit ainsi. C’est dans la disposition intérieure du vouloir, dans l'intention
qui préside à l’action que peut se mesurer sa moralité.

%%%%%%%%%%%%%%%%%%%%%%%%%%%%%%%%%%%%%%%%%%%%%%%%%%%%%%%%%%%%%%%%%%%%%%%%%%%
