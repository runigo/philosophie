
%%%%%%%%%%%%%%%%%%%%%
\section{Le projet critique}
%%%%%%%%%%%%%%%%%%%%%

Dans la première partie de sa vie intellectuelle,
Kant développe une
métaphysique proche de celle des
grands systèmes de son temps (Leibniz
et Christian Wolff). Mais, à partir de
1770, sa pensée va connaître un tournant
décisif, à partir duquel se construira
la philosophie proprement kantienne. Le
projet de cette philosophie est d’être une
critique. Ce mot signifie ici: examen
des pouvoirs de la raison, définition du
domaine à l’intérieur duquel ces pouvoirs
peuvent légitimement s'exercer.
Kant qualifie également sa philosophie
de « transcendantale », ce qui veut dire :
qui s'interroge sur les conditions de possibilité
de la connaissance (« raison théorique »)
%240
et de l'action (« raison pratique »).

%%%%%%%%%%%%%%%%%%%%%%%%%%%%%%%%%%%%%%%%%%%%%%%%%%%%%%%%%%%%%%%%%%%%%%%%%%%
