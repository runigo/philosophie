
%%%%%%%%%%%%%%%%%%%%%
\section{La raison pratique}
%%%%%%%%%%%%%%%%%%%%%

\subsection{Le devoir, la loi morale}
%{\bf }{\it }{\bf --}{\footnotesize X}$^\text{e}$

La seule action moralement bonne l’est
dans sa forme, non dans sa matière.
C'est celle qui procède d’une intention
pure, c'est-à-dire qui est accomplie par
devoir. Il faut bien distinguer ici l’action
simplement conforme au devoir (le
commerçant peut être honnête par intérêt)
et celle accomplie par devoir, sans
aucune considération pour l'intérêt
qu'on en espère ou la satisfaction qu’on
en tire. L'action morale procède donc
d'une loi morale qui s'exprime sous la
forme d’un devoir (« Tu dois ») et qui est
%241
une loi universelle de la raison pratique
(qui agit en se représentant ce qui doit
être). Mais comme nous sommes des
êtres sensibles, la loi morale se manifeste
sous la forme d’un commandement,
d’un impératif. L'impératif catégorique
fondamental commande d’agir
de telle sorte que la maxime de notre
action puisse être érigée en règle universelle.
Tel est le fondement nécessaire
de toute moralité rationnelle, mais aussi
de toute politique, laquelle doit être soumise
à l'exigence morale, loin de se
réduire à une technique de pouvoir.

\subsection{Unité de l'humanité, finalité de l’histoire}
%2. 

Mais l’homme ne peut se réaliser vraiment,
c’est-à-dire atteindre le plein
développement de toutes ses dispositions,
que dans la société. Pour cette raison,
la nature a sagement privé l’homme
d’instinct et l’a mis au monde dans la
nudité pour le forcer à s'élever par lui-même
(par le travail et la culture) et à
établir un ordre régi par des lois. L’histoire,
conçue comme un progrès, est
l’éducatrice de l'humanité, qu’elle oblige
à s'améliorer sans cesse en vue de la
liberté partagée. Kant affirme à la fois
que le devenir de notre espèce a pour
finalité le règne de la loi et la paix universelle,
et que, pourtant, l’établissement
de la justice publique — le « plus
grand problème pour l'espèce
humaine », le « plus difficile » — ne peut
jamais être considéré comme une affaire
%
réglée. Seul l'établissement d’une
« société des nations », soumise à une
législation internationale, permettra à
l’homme d'accéder à la paix et à l’ordre
juridique (condition de toute véritable
autonomie) et de surmonter véritablement
sa sauvagerie originelle ({\it Projet de
paix perpétuelle}).

\subsection{La réalisation de la métaphysique par
la morale}
%3. 

L'action morale nous introduit dans ce
que la connaissance ne peut atteindre :
elle réalise l’inconditionné qui se dérobe
à la raison spéculative. L'existence de
Dieu, l’incorruptibilité de l’âme et la
liberté ne sont pas des objets de
connaissance, mais ce sont des postulats
nécessaires de la raison pratique. Outre
la métaphysique, les principes de la religion
sont alors sauvés, mais pour des
raisons autres que religieuses : parce
qu'ils correspondent à l'exigence rationnelle
qui est au cœur de la morale kantienne.

\begin{itemize}[leftmargin=1cm, label=\ding{32}, itemsep=1pt]
\item {\bf \textsc{Principaux écrits} :} {\it Critique de la
raison pure} (1781) ; {\it Idée d'une histoire
universelle au point de vue
cosmopolitique} (1784) ; {\it Fondement
pour la métaphysique des mœurs}
(1785); {\it Critique de la raison pratique}
(1788) ; {\it Critique de la faculté
de juger} (1790) ; {\it La Religion dans
les limites de la simple raison}
(1793); {\it Projet de paix perpétuelle}
(1795).
\end{itemize}

%%%%%%%%%%%%%%%%%%%%%%%%%%%%%%%%%%%%%%%%%%%%%%%%%%%%%%%%%%%%%%%%%%%%%%%%%%%
