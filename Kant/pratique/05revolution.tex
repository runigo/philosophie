
\subsection{« LA RÉVOLUTION COPERNICIENNE EN PHILOSOPHIE »}


{\it La philosophie critique de Kant commence par un constat : l'histoire des sciences
mathématiques et physiques est celle de révolutions réussies. Les découvertes de
Thalès, de Galilée ont en effet ouvert la voie de la science et ont donné lieu aux plus
grandes conquêtes de l'esprit. L'histoire de la métaphysique, au contraire, reste un
terrain de bataille sans cesse dévasté : on n’y voit toujours que ruine et
contradictions. Il n’y a pas ici progrès du savoir mais perpétuelle remise en cause.
Et pourtant, la métaphysique continue d’édicter des vérités, comme si elle
demeurait, malgré ce constat millénaire d'échec, une exigence irréductible de la
raison. Confronté à ce paradoxe, Kant va tenter de lui donner une solution
théorique.}

« On a admis jusqu'ici que toutes nos connaissances devaient se régler sur les
objets ; mais, dans cette hypothèse, tous nos efforts pour établir à l'égard de ces
objets quelque jugement {\it a priori} et par concept qui étendit notre connaissance
n’ont abouti à rien. Que l’on cherche donc une fois si nous ne serions pas plus heureux
dans les problèmes de la métaphysique, en supposant que les objets se règlent
sur notre connaissance, ce qui s'accorde déjà mieux avec ce que nous désirons
[démontrer], à savoir la possibilité d’une connaissance {\it a priori} de ces objets qui
établisse quelque chose à leur égard, avant même qu'ils nous soient donnés. Il en
est ici comme de la première idée de Copernic : voyant qu'il ne pouvait venir à
%242
bout d'expliquer les mouvements du ciel en admettant que toute la multitude des
étoiles tournait autour du spectateur, il chercha s’il n’y réussirait pas mieux en supposant
que c’est le spectateur qui tourne et que les astres demeurent immobiles. »

\begin{flushright}
Kant, {\it Critique de la raison pure} (1781), trad. J. Barni,

Paris, GF-Flammarion, 1976, pp. 41-42.
\end{flushright}

\subsubsection{La recherche métaphysique}


La métaphysique procède depuis toujours comme si son objet pré-existait à son
investigation. Il y aurait ainsi les objets du monde, comme domaine d’application
de la connaissance scientifique, et les objets qui ne nous sont pas directement
donnés dans l'expérience sensible (Dieu, l'âme, l’un, l'être...) et qui constituent le
terrain d'investigation de la pensée métaphysique. Mais, malheureusement, toutes
ces recherches pour établir une connaissance pure, {\it a priori} (c'est-à-dire qui ne
recourt pas à l'expérience), ne donnent jamais des résultats assurés.

\subsubsection{La révolution copernicienne}

Kant propose comme hypothèse que les objets ne pré-existent pas à la connaissance
qu’on en prend. Le processus de connaissance ne doit pas être en effet
pensé comme un faisceau de lumière qui donnerait à voir des objets jusqu'alors
plongés dans l'obscurité : la connaissance ne dévoile pas son objet, elle le constitue
et l’élabore. C'est là ce que Kant nomme sa « révolution copernicienne » :
l'objet à connaître tourne autour du sujet connaissant comme la terre autour du
soleil. En somme, le sujet qui veut connaître le monde ne doit plus se faire
(comme dans les conceptions traditionnelles) le miroir de l’objet, car l’objet n’est
jamais que le miroir du sujet.

%%%%%%%%%%%%%%%%%%%%%%%%%%%%%%%%%%%%%%%%%%%%%%%%%%%%%%%%%%%%%%%%%%%%%%%%%%%
