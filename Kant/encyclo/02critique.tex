
\section{{\it La} Critique de la raison pure}
%%%%%%%%%%%%%%%%%%%%%%%%%%%%%
%{\bf }{\it }{\bf --}{\footnotesize X}$^\text{e}$

Kant apporte une correction fondamentale
à la métaphysique leibnizienne et
wolffienne en rejetant l’idée qu’il n’existe
qu'une différence de degré entre les
représentations sensibles et les représentations
intellectuelles, du confus au distinct ;
une différence qui ne concerne ni
le contenu ni l’origine de nos représentations.
Au contraire, Kant perçoit dans la
sensibilité et dans l’entendement deux
sources spécifiques de la connaissance,
irréductibles l’une à l’autre. Concernant
la sensibilité, sa spécificité réside dans
l’immédiateté de la référence aux objets
(intuition) : la sensibilité est pure passivité,
réceptivité. Par opposition, l’entendement
procède selon une démarche
discursive (par concepts). Pure spontanéité,
il opère la synthèse du divers donné
par l'intuition sensible. L'homme, dans sa
finitude (en tant qu’{\it intellectus ectypus}), ne
peut être affecté que par des objets qui se
donnent comme phénomènes de l’intuition
sensible, et ne peut donc accéder à
une intuition suprasensible (laquelle est
un attribut de Dieu, {\it intellectus archetypus}).
L'association des intuitions et des
concepts est indispensable pour qu’il se
forme une connaissance, c’est-à-dire une
pensée dotée d’un contenu. Kant admet
donc que toute connaissance commence
avec l’expérience, tout en précisant
qu’elle ne provient pourtant pas entièrement
de l’expérience. Ce que l'expérience
ne fournit pas, par définition, c’est un critère
de validité universel et nécessaire.
Or, même Hume reconnaît une telle validité,
tout au moins aux propositions des
mathématiques, qui sont vraies par la
seule vertu des opérations de la pensée,
indépendamment de l'expérience : ce sont
des connaissances {\it a priori}, dans le langage
kantien. Tout en tenant compte des objections
sceptiques de Hume, Kant entend
garantir à la raison un pouvoir plus
étendu que celui que lui accordait Hume,
fondé sur des bases différentes. C’est ainsi
qu'il affirme que le domaine de l’{\it a priori}
ne se limite pas aux propositions où aux
jugements qu’il nomme analytiques.

