
Kant Emmanuel (Königsberg 1724 - {\it id.}
1804), philosophe allemand. Principal
représentant allemand des Lumières, il
est à l’origine d’une véritable révolution
philosophique. Kant réfute en effet la
métaphysique dogmatique, que représentait
à ses yeux le système de Christian
Wolff et de son école, en procédant à une
critique de la raison, qui détermine les
conditions de possibilité et les limites des
capacités cognitives de l’homme, selon ses
facultés (faculté de connaître, de désirer,
d’éprouver du plaisir ou du déplaisir). Les
principaux représentants de l’idéalisme
allemand se sont confrontés à sa position :
Fichte, Schelling et Hegel ont fait référence
à Kant, chacun à sa manière, même
s’ils ont ensuite orienté leur réflexion
dans des directions qu’il avait refusé d’envisager.
Par ailleurs, l’œuvre de Kant,
dans la mesure où elle procède, contrairement
à celle des idéalistes, d’un examen
approfondi des particularités et des problèmes
de la science moderne de la nature
— envisagée dans une perspective newtonienne —,
a suscité un important débat
quant au développement successif des
sciences et de la réflexion épistémologique.
