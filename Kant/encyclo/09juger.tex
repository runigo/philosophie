
\section{{\it La} Critique de la faculté de juger}
%%%%%%%%%%%%%%%%%%%%%%%%%%%%%
%{\bf }{\it }{\bf --}{\footnotesize X}$^\text{e}$

Dans cet ouvrage, qui allait connaître
un immense succès auprès des romantiques
allemands, Kant se propose de
compléter le système de la critique en
recherchant la fonction de la faculté de
juger, à laquelle il attribue une position
intermédiaire entre l’entendement et la
raison (c’est-à-dire entre la nature et la
liberté). En général, le jugement est la
faculté de penser le particulier comme
contenu dans l’universel. Cela peut se
faire de deux façons : soit en subsumant,
à partir d’une règle, le cas particulier sous
un universel donné ; soit en cherchant un
universel pour un cas particulier donné.
Le premier est le jugement « déterminant »,
le second est le jugement « réfléchissant ».
Kant a déjà analysé le premier
dans la {\it Critique de la raison pure} (en particulier
à propos du schématisme), tandis
qu’il évoque maintenant la question de
savoir si la recherche de régularités empiriques,
et en général la réflexion sur des
objets pour lesquels il ne nous est pas
donné de loi, doit également être
conduite selon des principes, liés à une
application particulière du concept de fin.
La {\it Critique de la faculté de juger} s’articule
%
en deux parties, consacrées respectivement
à une « critique du jugement esthétique »
et à une « critique du jugement
téléologique ».

La première traite du jugement lié à
une faculté déterminée de l'esprit, le sentiment
de plaisir et de déplaisir. Kant
entend fournir, à proprement parler, une
« critique du goût », un examen attentif
de notre capacité de jugement esthétique
(esthétique non pas au sens de l’Esthétique
transcendantale, mais dans le sens
courant d’esthétique tel qu’il est employé
à partir de Baumgarten). D’après Kant,
les jugements de goût ne contribuent pas
à la connaissance des choses ; ils instaurent
néanmoins une relation immédiate
entre le sentiment de plaisir-déplaisir et la
faculté cognitive. Ils ont ceci de particulier
que, tout en reposant sur le sentiment,
ils sont communicables, et exigent l’approbation
de chacun, faisant appel à une
sorte de « sens commun » esthétique. Ils
se caractérisent non pas par une universalité
objective, fondée sur un concept, mais
par une universalité {\it sui generis}, « subjective ».
Quand nous disons qu’un objet est
beau, nous sommes conscients que cette
affirmation n’équivaut pas seulement à
« il me plaît ». Autrement dit, nous présupposons
l'existence d’une humanité
capable de s’accorder sur ce jugement.
C’est une différence que les empiristes ne
saisissent pas, en ramenant entièrement le
beau à la sensation individuelle. Si les
empiristes n’expliquent pas la différence
entre le beau et l’agréable, selon Kant, le
point de vue rationaliste qui considère le
beau comme un concept confus de la perfection
d’un objet se fourvoie tout autant.
Kant lui objecte que le beau n’est pas le
parfait, car il n’a aucun rapport avec la
matière, mais seulement avec la forme
d’un objet. Nous appelons beau un objet
qui nous plaît, en vertu d’une finalité qui
n’est pas imputable à un but ou à une
volonté, mais uniquement au jeu harmonique
qui s’instaure entre l'imagination
(que l’on nomme dans ce cas « productrice »)
et l’entendement, lorsqu'on se
représente un objet donné. Kant définit
donc la beauté comme la forme de la finalité
d’un objet, ne s’accompagnant pas de
la représentation d’un but; et ceci vaut
aussi bien pour le beau dans la nature que
pour le beau dans l’art. Pour juger des
beaux objets, le goût est nécessaire. Pour
produire de belles œuvres d’art, le génie
%
est nécessaire, et il est conçu comme la
capacité de l’artiste à travailler sans être
soumis à la contrainte de règles intentionnelles,
c’est-à-dire comme le moyen par
lequel la nature dicte sa règle à l’art, qui
est à l’image de la nature législatrice.

Kant s'inscrit dans la tradition du
{\footnotesize XVIII}$^\text{e}$ s., en complétant son traité du beau
par un traité du sublime. Il définit le
sublime comme ce qui est grand dans l’absolu
(et non comparativement), c’est-à-dire
infini, soit par sa grandeur (sublime
mathématique), soit par sa puissance
(sublime dynamique), et il le ramène à un
rapport non harmonique de l'imagination
et de la raison.

Si la critique kantienne du goût fait partie
des rares épisodes décisifs de la philosophie
de l’art, et est habituellement
reconnue comme un {\it terminus a quo}, la
seconde partie de la {\it Critique de la faculté
de juger}, consacrée au jugement téléologique,
apparaît davantage liée à la problématique
de l’époque, et elle a exercé une
influence déterminante sur l’idéalisme.
Kant est le premier à faire une distinction
précise entre la finalité « externe » (qui
s'était attiré tant de critiques durant les
deux siècles précédents) et une finalité
« interne », identifiable dans les êtres
organisés. Les principales caractéristiques
de cette finalité sont : 1) la capacité de
l'être vivant à se générer soi-même, à la
fois comme individu (assimilation et
accroissement) et comme espèce (reproduction) ;
2) le lien réciproque entre la
cause et l'effet, entre le moyen et le but
des parties d’un « produit naturel organisé »
(à commencer par le brin d’herbe)
entre elles ; et 3) le lien de ces parties avec
le tout. Le jugement téléologique sert à
établir, à différents niveaux, des lois particulières
et taxinomiques de la nature,
dans la mesure où celles-ci sont différentes
des lois universelles émanant de
l'entendement. D'où l’idée d’une unité,
d’un ordre de la nature, comme si, tout
entière, elle était produite pour obéir à
une finalité. Il existe donc un principe
régulateur pour le jugement réfléchissant
qui, bien que ne remplaçant pas le mécanisme,
peut lui être associé. Toutefois ce
principe demeure « subjectif » : la conformation
de notre faculté cognitive nous
amène à penser que la nature est possible
en vertu d’une finalité objective. D'où les
conditions particulières de cette {\it Critique},
qui n’admet aucun développement doctrinal :
%878
Kant propose une critique du jugement,
et non une science.

