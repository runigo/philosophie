
\section{{\it La conception de l’histoire}}
%%%%%%%%%%%%%%%%%%%%%%%%%%%%%
%{\bf }{\it }{\bf --}{\footnotesize X}$^\text{e}$

Au centre des réflexions de Kant sur
l’histoire des hommes, se trouve l’idée
d’un progrès culturel de l’humanité, susceptible
d’être interrompu, mais pas
arrêté. À côté de l’étude concrète de l’histoire,
Kant fait place à une réflexion philosophique,
où l’idée d’une destination de
l'homme sert de fil conducteur pour
s'orienter dans une expérience qui ne
livre pourtant que des équivalents intermittents
et partiels de cette destination.
Partant en effet du constat que les
hommes agissent sur la scène du monde
suivant des plans et des buts différents,
Kant cherche à définir un « dessein de la
nature », à partir de la notion de « dispositions
naturelles » chez l’homme, destinées
à un développement complet. Puisque
l'homme est un être qui n’est pas purement
instinctif, mais qui procède par des
tentatives imparfaites, s’aidant de la raison,
un tel développement ne peut se vérifier
que pour l’espèce, le long d’un
parcours qui ne va pas du bien au mal,
mais du pire au meilleur. On peut définir
le genre humain comme « la totalité d’une
série de générations qui vont vers l’infini ».
Mais le cours des générations ne se
rapproche pas seulement de façon asymptotique
de la ligne de sa « destination » :
il rencontre bien cet horizon, en un point
indéfini du point de vue de l’individu singulier,
mais déterminé si on le considère
selon la « totalité » de la série des générations.

Kant voit dans l’antagonisme des
hommes en société (« insociable sociabilité »)
le ressort de la civilisation. C’est
grâce à cet antagonisme que les talents
individuels se manifestent, et que les
hommes quittent le stade de la paresse
animale, pour passer petit à petit de la
barbarie à la culture, et finir par auto-discipliner
l’insociabilité. Au lieu d’une
union pathologique (c’est-à-dire ancrée
dans la sensibilité) « coercitive », la
société tend ensuite à devenir un
ensemble moral. Et c’est enfin dans le
cadre d’une organisation civile élaborée
(juridico-politique) de la société que les
fins de la nature deviennent réalisables, et
que l’humanité peut déployer ses facultés
particulières en réalisant sa destination. Il
s’agit donc d’instaurer une constitution
%879
civile parfaite. Pour cela, il faut envisager
de définir un cadre juridique, qui permette
de résoudre les relations entre
Etats, où nous voyons se perpétuer les
effets de l’insociabilité naturelle. Kant
préfigure une fédération de peuples dans
laquelle chaque État serait protégé. Pour
chimérique que cela paraisse, observe-t-il,
c’est la seule solution aux maux qui affligent
l'humanité. La fin suprême de la
nature est donc une organisation cosmopolitique,
qui implique le progrès des
Lumières, à condition qu’elles s’étendent
jusqu'aux trônes. C’est dans cette perspective
que vient s’insérer le célèbre projet
« pour une paix perpétuelle », qui
couronne la nouvelle proposition kantienne
d’idéaux caractéristiques du
{\footnotesize XVIII}$^\text{e}$ s.

