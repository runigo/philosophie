
\section{{\it La Dialectique transcendantale}}
%%%%%%%%%%%%%%%%%%%%%%%%%%%%%
%{\bf }{\it }{\bf --}{\footnotesize X}$^\text{e}$


Lorsque la faculté cognitive prétend
étendre l’usage des concepts {\it a priori} au-delà
des limites de l’intuition, elle est victime
d’une illusion. Expliquer cette illusion,
cependant, ne revient pas à la
dissiper, puisqu'il s’agit d’une « illusion
naturelle de la raison ». Celle-ci est liée à
la tentative qu’effectue la raison de
remonter dans la série des conditions à la
recherche d’un inconditionné, d’une totalité
absolue des conditions ; par une telle
tentative, la raison parvient à des
concepts auxquels aucun objet d’expérience
ne peut correspondre. Ces
concepts n’ont aucune fonction constitutive,
mais ils ont seulement un usage « régulateur »,
qui conduit à l’élargissement
et à l’unification systématique du savoir.
Kant nomme ces concepts « idées », et il
définit la raison comme la faculté de produire
des idées. Ces idées sont l’âme, le
monde, Dieu, auxquelles correspondent
autant de doctrines de la « {\it metaphysica
specialis} » de Wolff — la psychologie, la
cosmologie et la théologie rationnelles —
que Kant s’applique à démolir scrupuleusement.
En particulier pour ce qui
concerne la cosmologie, il montre qu’elle
s’enlise dans des antinomies, à savoir des
paires de propositions antithétiques qui
diffèrent des propositions contradictoires,
par le fait qu’elles ne sont pas nécessairement
l’une vraie, l’autre fausse. La doctrine
des antinomies est habituellement
considérée comme la partie la plus
ancienne de la {\it Critique de la raison pure},
et comme celle qui en contient la source
d'inspiration.
