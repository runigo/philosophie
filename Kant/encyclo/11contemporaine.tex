
\section{{\it Les interprétations de Kant dans la philosophie contemporaine}}
%%%%%%%%%%%%%%%%%%%%%%%%%%%%%
%{\bf }{\it }{\bf --}{\footnotesize X}$^\text{e}$

L'influence de Kant a été considérable
sur la tradition philosophique. Elle irrigue
l’idéalisme  transcendantal de Fichte,
Schelling et Hegel jusqu’au néokantisme
des {\footnotesize XIX}$^\text{e}$ et {\footnotesize XX}$^\text{e}$ s., et jusqu’à différents courants
de la philosophie contemporaine.
Dans le cadre de la philosophie contemporaine,
il est possible de dégager
quelques interprétations-clés de la pensée
kantienne, qui se réclament de la philosophie
kantienne ou qui cherchent en elle
des instruments de justification. On peut
ainsi distinguer, outre le néokantisme : 1)
l'interprétation de tendance sceptique et
basée sur l’idée de fiction avancée par
H. Vaihinger, qui s'inspire aussi de
Nietzsche, et qui considère les idées et les
catégories comme des instruments créés
par l'esprit humain en vue d’une maîtrise
pratique du réel ; 2) l'interprétation psychologique,
qui, en s'appuyant sur les
affirmations de J. F. Fries en Allemagne
au {\footnotesize XIX}$^\text{e}$ s., interprète l’{\it a priori} comme une
structure psychologique, parvenant ainsi à
une vue empiriste de Kant ; 3) l’interprétation
de type « pratique » ou moraliste
(Erich Adickes), qui place l'inspiration
fondamentale de Kant dans ses doctrines
éthico-religieuses, en considérant sa critique
de la métaphysique comme un mouvement
nécessaire pour garantir à
l’homme l’accès à la sphère éthico-religieuse
dans sa pureté ; 4) l'interprétation
Spiritualiste, qui cherche à utiliser les
thèses kantiennes pour le fondement critique
d’une métaphysique spiritualiste ; 5)
l'interprétation néo-idéaliste, qui reprend
les thèses de l’idéalisme romantique et
considère Kant comme celui qui aurait
inauguré une nouvelle métaphysique de
l'esprit, censée remplacer la vieille métaphysique
de l'être (Bertrando Spaventa,
Giovanni Gentile); 6) l'interprétation
existentialiste, qui voit chez Kant celui
qui, en montrant la passivité fondamentale
de la connaissance et l'impossibilité
pour la pensée humaine de transcender le
temps considéré comme une forme générale
de l’existence, fraie un chemin vers
une nouvelle ontologie du fini (Heidegger) ; 7)
une interprétation nouvelle dans
l'optique des Lumières, pour laquelle le
véritable esprit du kantisme ne serait pas
l’idéalisme, mais les Lumières : la philosophie
de Kant se présente comme la
recherche dans tous les domaines des
véritables possibilités de l’homme, dans
des limites que la philosophie a justement
pour tâche de déceler ; 8) l'interprétation
linguistique de Karl Otto Apel qui
conçoit l’{\it a priori} comme une fonction
communicative et propose, sur les traces
de Peirce, une « sémiotisation du kantisme ».

Les recherches kantiennes les plus
significatives de notre époque tendent
cependant à dépasser ces interprétations
du kantisme (unilatérales dans la mesure
où elles se contentent de souligner tour à
tour quelques-uns seulement des multiples
éléments qu’il comporte) pour restituer
la doctrine kantienne dans son
contexte philosophique, ainsi que dans la
perspective générale de la pensée européenne
moderne.

%—> catégorie + critique + esthétique
%+ génie + néokantisme + transcendantal
