
\section{La vie et l’œuvre}
%%%%%%%%%%%%%%%%%%%%%%%%%%%%%
%{\bf }{\it }{\bf --}{\footnotesize X}$^\text{e}$


Issu d’une famille modeste, Kant a reçu
une éducation moralement imprégnée du
piétisme de sa mère. Après avoir quitté la
discipline sévère du Collegium Fridericianum,
il s'inscrit en 1740 à l’université
albertine de Königsberg, où se mêlent des
influences aristotéliciennes, piétistes et
wolffiennes. Son premier écrit est une
intervention dans la dispute entre leibniziens
et cartésiens au sujet du concept de
force : {\it Gedanken von der wahren Schätzung
der lebendingen Kräfte} ({\it Pensées sur
la véritable évaluation des forces vives},
1746, paru en 1749). Après la mort de son
père (1746), Kant devient précepteur,
pendant de nombreuses années. Il obtient
en 1755 son doctorat avec la dissertation
{\it De igne} et le titre d’assistant avec la dissertation
{\it Principiorum primorum cognitionis
metaphysicae nova dilucidatio}.
Toujours en 1755, son {\it Histoire naturelle et
théorie du ciel} paraît anonymement. Il y
applique la théorie newtonienne des
forces d’attraction et de répulsion pour
exposer l’hypothèse de la formation du
système solaire à partir d’une nébuleuse
originelle. Cette hypothèse passe inaperçue
à l’époque, mais elle est proche de
celle formulée en 1796 par le physicien
Pierre Simon Laplace, si bien qu’elle est
communément désignée comme l’« hypothèse
de Kant-Laplace ». Commence
alors une période d’une quinzaine d’années,
marquée par une activité sociale
assez intense, durant laquelle Kant présente
de très nombreuses leçons dans différentes
disciplines, de la logique à la
géographie physique. La {\it Monadologie
physique} date de 1756 ; c’est une tentative
pour concilier le monadisme de Leibniz et
la doctrine de l’attraction newtonienne
sur la question de la divisibilité de l’étendue.
La {\it Fausse Subtilité des quatre figures
syllogistiques} et {\it L'Unique Fondement
possible d’une démonstration de l’existence
de Dieu} datent de 1762 ; suivis, entre
autres, de l’{\it Essai pour introduire dans la
philosophie le concept des quantités négatives}
(1763); l’{\it Étude sur l'évidence des
principes de la théologie naturelle et de la
morale} (1764); les {\it Observations sur le
sentiment du beau et du sublime} (1764) qui
témoignent de son intérêt pour la pensée
des moralistes anglais; les {\it Rêves d’un
visionnaire} (1766), contre Emmanuel
Swedenborg, de tonalité ironique et sceptique
à l’égard de la métaphysique. Enfin,
en 1770, grâce à la dissertation {\it De mundi
sensibilis atque intelligibilis forma et principiis},
Kant est nommé professeur titulaire
de la chaire de logique et de
métaphysique de l’université de Kôünigsberg,
où il continue d’enseigner jusqu’en
1796, en refusant les offres d’autres universités.

Son projet, mûrement réfléchi, de
remettre en question la métaphysique
dans la perspective d’une critique de la
raison trouve, après un parcours non
linéaire, sa première réalisation fondamentale
dans la {\it Critique de la raison pure}
(1781). Entre la première édition de l’ouvrage,
considéré comme l’œuvre majeure
de Kant, et la seconde édition (1787),
considérablement modifiée, paraissent les
{\it Prolégomènes à toute métaphysique future
qui pourra se présenter comme science}
(1783), qui présentent sa théorie sous la
forme d’« exercices préparatoires », et les
%872
{\it Premiers Principes métaphysiques de la
science de la nature} (1786), une analyse du
concept général de matière, qui rende
compte de cette partie de la physique qui
est pure sans être mathématique. Kant
poursuit son examen critique dans le
domaine de la morale, avec la {\it Critique de
la raison pratique} (1788), inaugurée par
les {\it Fondements de la métaphysique des
mœurs} (1785) et suivie par la {\it Métaphysique
des mœurs} (1797), qui comporte
deux parties : la {\it Doctrine du droit}, et la
{\it Doctrine de la vertu}. En 1790 paraît le
troisième et dernier volet du triptyque critique,
la {\it Critique de la faculté de juger},
consacrée au jugement esthétique et au
jugement téléologique (qui développe
l’hypothèse d’une finalité intrinsèque de
la nature).

Il convient de rappeler, parmi la vaste
production qui accompagne ses œuvres
principales : l’{\it Idée d’une histoire universelle
du point de vue cosmopolitique}
(1784); {\it Qu'est-ce que les Lumières ?}
(1784) ; les {\it Comptes rendus de l'ouvrage
de Herder : Idées sur la philosophie de
l’histoire de l'humanité} (1785) ; {\it Qu'est-ce
que s'orienter dans la pensée ?} (1786) ; la
réponse à Eberhard dirigée contre Leibniz
{\it Sur une découverte selon laquelle toute
nouvelle critique de la raison pure serait
rendue superflue par une plus ancienne}
(1790) ; {\it Sur l'expression courante : il se
peut que cela soit juste en théorie, mais en
pratique cela ne vaut rien} (1793) ; le {\it Projet
de paix perpétuelle} (1795); enfin,
l’{\it Anthropologie du point de vue pragmatique}
(1798), synthèse de trente années de
cours, qui connaît rapidement un vif
succès. En 1794, suite à la parution de {\it La
Religion dans les limites de la simple raison}
(1793) et de {\it La Fin de toutes choses}
(1794), Kant se voit sommé, par ordonnance
royale, d'abandonner les parties de
sa doctrine qui sont soupçonnées de dénigrer
le christianisme. En 1799, il affirme
que la {\it Doctrine de la science} de Fichte est
un « système insoutenable », étant donné
la fortune croissante de l’œuvre qui se
présente comme un prolongement cohérent
et unitaire de sa pensée. Gravement
affaibli, Kant se consacre durant les dernières
années de sa vie à un traité sur le
passage de la métaphysique de la science
de la nature à la physique, c’est-à-dire
pratiquement à un remaniement de son
système, resté toutefois inachevé : des
fragments paraissent, intitulés {\it Opus postumum},
%873
en 1882-1884. De son vivant ont
paru ses leçons de logique (1800), de géographie
physique (1802) et de pédagogie
(1803) ; d’autres séries de leçons ont été
publiées après sa mort.

