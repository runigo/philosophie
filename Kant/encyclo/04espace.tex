
\section{{\it La conception kantienne de l’espace et du temps}}
%%%%%%%%%%%%%%%%%%%%%%%%%%%%%
%{\bf }{\it }{\bf --}{\footnotesize X}$^\text{e}$

La compréhension de la théorie kantienne
de l’espace et du temps s’avère
essentielle à cette fin. Kant conçoit l’espace
et le temps comme des conditions {\it a
priori} de l’expérience, qui s’enracinent
dans la subjectivité. En effet, ils participent
de notre constitution subjective en
tant que formes {\it a priori} de la sensibilité :
le temps comme le « sens interne » et l’espace
le « sens externe ». Ce qui ne les
prive pas de validité objective, mais fait
en sorte que leur validité soit limitée aux
objets qui sont en rapport avec nous
(« phénomènes »). Les « objets pour
nous » renvoient à, ou présupposent, des
objets en soi ; mais ce dernier aspect nous
demeure inaccessible. Les choses en soi
ne font donc pas partie du champ de notre
connaissance ; elles se situent à sa limite,
comme des objets possibles de la pensée
pure (« noumènes »). C’est à l’intérieur
des limites posées par la distinction entre
les phénomènes et les choses en soi que
la physique newtonienne se révèle inattaquable,
avec ses lois qui sont en parfaite
correspondance avec les vérités géométriques.
Partant de ces vérités, Kant
remarque que l’espace (au même titre
que le temps) constitue une totalité
unique qui ne s’épuise pas et ne s’identifie
à aucune portion d’espace déterminée ; et
que c’est précisément en tant que tel qu’il
est capable de contenir une multitude infinie
de représentations. Il est alors possible
de dissocier de ces intuitions
spatiales empiriques une intuition pure,
conçue exclusivement en termes de figure
et d'extension. C’est ainsi que la géométrie
peut exposer les propriétés de l’espace,
en construisant des propositions
synthétiques {\it a priori} grâce à une intuition
pure. Ainsi, le concept de droite n’est pas
inclus au préalable dans celui de la ligne
la plus courte ; il est le résultat d’un lien
{\it a priori}, en référence à l’intuition spatiale.
De façon analogue, l’arithmétique
construit ses concepts dans l'intuition
pure du temps, sans laquelle l’entendement
ne serait pas en mesure de parvenir,
à partir de la simple notion de somme, à
un résultat précis.

