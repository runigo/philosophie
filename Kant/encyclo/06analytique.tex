
\section{{\it L’Analytique transcendantale}}
%%%%%%%%%%%%%%%%%%%%%%%%%%%%%
%{\bf }{\it }{\bf --}{\footnotesize X}$^\text{e}$


L’« Analytique » se compose de deux
livres : l’« Analytique des concepts » et
l’« Analytique des principes ». Kant
détermine tout d’abord les conditions
dans lesquelles il est possible de penser
en général un objet, même sans en avoir
l'intuition. Elles résident dans l’entendement,
sous la forme des catégories ou des
concepts purs. Puisque la fonction unificatrice
exercée par l’entendement se manifeste
dans le jugement, Kant établit dans
un premier temps, en correspondance
avec la table logique des jugements, en
quatre rubriques, une table des catégories,
également quadripartite, qui propose
un nouvel ordonnancement des catégories
aristotéliciennes. Kant entend par « déduction »,
au sens juridique du terme, la
justification de la validité des catégories
(remaniée pour la deuxième édition de la
{\it Critique}), et montre que tout usage de
l’entendement, à savoir toute synthèse du
multiple, présuppose toujours une activité
unificatrice de la part du sujet de la
connaissance, c’est-à-dire qu’elle s’accompagne
toujours de la représentation pure
du « Je pense ». Il cherche ensuite à expliquer
comment les conditions subjectives
de la pensée, ainsi définies, peuvent revêtir
une validité objective. Elles parviennent
à l’atteindre dans la mesure où elles
s’appliquent à des données de l'intuition
sensible. L’Analytique des principes
contient précisément les règles {\it a priori} de
cet usage de l’entendement : pour l’expliquer,
Kant introduit la doctrine du « schématisme ».
On entend par « schème » une
figuration pure (différente de la simple
image), qui concrétise la catégorie, et en
restreint en même temps l’usage, dans le
cadre de conditions extra-intellectuelles,
c’est-à-dire sensibles. Ce sont, précisément,
%875
les déterminations pures du temps
(comme forme du sens interne et, partant,
comme condition de toutes nos représentations)
qui servent de schèmes. Par
exemple, le schème de la quantité est le
nombre, compris comme l’addition successive
d’unités homogènes. Les schèmes correspondant
aux catégories de la relation
(substance-accident, cause-effet, action
réciproque) sont la permanence, la succession,
la simultanéité. Et l’existence des
phénomènes dans le temps se détermine
grâce aux principes qui régissent cet
ensemble de schèmes. S’opposant à
Hume, Kant revendique en particulier la
nécessité du principe de causalité, c’est-à-dire
celui qui autorise la distinction entre
une succession purement subjective, relative
au flux de nos représentations, et une
succession objective, valable par rapport
aux seuls phénomènes. Ainsi, une chose
est de percevoir les différentes parties
d’une maison selon un ordre réversible ;
une autre chose est de percevoir un navire
qui occupe successivement des lieux distincts
dans un cours d’eau, auquel cas il
n'y a pas dans ce cas de réversibilité.
Autrement dit, la fonction constitutive est
à attribuer à l’entendement, par rapport à
l'expérience.

