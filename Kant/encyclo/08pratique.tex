
\section{{\it La} Critique de la raison pratique}
%%%%%%%%%%%%%%%%%%%%%%%%%%%%%
%{\bf }{\it }{\bf --}{\footnotesize X}$^\text{e}$

La seconde critique traite de la raison
dans son usage pratique. Si la raison pure
n’est pas théorétique en soi, ou « spéculative » — 
puisqu'elle s’adosse à la sensibilité
et qu’elle est tributaire de ses limites —, la
raison est en revanche pratique en soi, et
capable de déterminer la volonté à agir.

Par conséquent, ce n’est pas la raison
{\it pure} pratique qui doit faire l’objet d’une
critique, mais seulement la raison pratique.
Un « être raisonnable fini » tel que
l’homme est spontanément capable d’agir
selon des principes : la question est de
savoir si l’on peut établir des principes {\it a
priori} de l’action, et sur quelle base. Kant
exclut l'hypothèse que l’on puisse déduire
ces principes de n'importe quel objet
— comme de nombreux philosophes ont
tenté de le faire — en tant qu'il serait la
cause déterminante de la volonté. Ce faisant,
on ne disposerait plus que de principes
matériels et empiriques, aussi variés
que possible, mais tous réductibles au
principe fondamental de l’amour-propre
ou du bonheur. S’opposant à l’eudémonisme
d’une grande partie de la pensée
morale du {\footnotesize XVIII}$^\text{e}$ s., Kant affirme que seul
un principe pratique et formel garantit la
validité universelle de notre volonté. Il y
parvient en établissant une distinction
entre les « maximes », principes valables
pour un sujet particulier (bien que susceptibles
de généralité, c’est-à-dire d’être
valables dans la majeure partie des cas),
et les « lois », principes dotés d’une validité
objective, c’est-à-dire en mesure de
déterminer l’action de tout être rationnel.
En effet, les « lois morales » sont la cause
déterminante de notre volonté : elles
constituent le fondement idéal de toutes
nos actions de même que les lois de
la nature régentent les phénomènes.
C’est pourquoi l’une des formulations de
l'impératif kantien recommande d’agir
%
« {\it comme si} la maxime de ton action devait
être érigée par ta volonté en loi universelle
de la nature » ({\it Fondements de la
métaphysique des mœurs}, « Deuxième
section »). Aussi, le rapport de l’homme
avec la loi morale ne peut être qu’un rapport
de dépendance, étant donné que
l’homme est effectivement capable de
maximes contraires à la loi. On ne peut
pas lui attribuer une volonté « sainte »,
qui ne soit pas exposée à l’opposition
entre les vertus et les inclinations, mais
seulement une volonté « bonne », qui se
fonde sur la représentation du devoir.
Cette dépendance s'exprime sous la
forme d’un commandement, d’un impératif
que Kant nomme « catégorique »,
c’est-à-dire inconditionné, pour le différencier
d’autres types de conseils ou de
préceptes pratiques relatifs au rapport
moyens-but (« impératifs hypothétiques »,
qui s’expriment sous la forme conditionnée
{\it si... alors}). Kant pense avoir fourni de
cette façon une formule rigoureuse pour
ce qu’il nomme le « fait » de la raison, ce
dernier permettant à l’idée de liberté, qui
peut être pensée mais qui ne peut pas être
démontrée sur le plan spéculatif, d’acquérir
une réalité objective. La loi morale
implique en effet l’autonomie de la
volonté, qu'il faut aussi bien comprendre
dans un sens négatif, précisément comme
une indépendance des causes matérielles,
que dans un sens positif, comme une
autodétermination rationnelle spontanée.
L’autonomie du sujet moral est la condition
pour que les maximes puissent s’accorder
à la loi. D’un côté, donc, la liberté
est la {\it ratio essendi} de la loi morale. D’un
autre côté, la loi morale est la {\it ratio
cognoscendi} de la liberté : c’est à travers
la reconnaissance de la loi que nous parvenons
à la conscience de la liberté. La
« voix de la raison » se fait entendre
même chez l’homme le plus ordinaire, et
même plus distinctement que dans les
spéculations des philosophes. C’est là un
sujet (l’appel à la « conscience » de
l’homme ordinaire) pour lequel Kant
reconnaît sa dette envers Rousseau.
Entre le désir irrépressible de bonheur
et la vertu, il existe une contradiction qui
n’est pas simplement logique mais « pratique »,
et que le principe de la morale
résout par une subordination rigoureuse
du premier terme au second. Seul celui
qui est vertueux est digne du bonheur.
C’est le problème qui occupe la Dialectique
%876
de la {\it Critique de la raison pratique}.
L’antinomie de la raison pratique se présente
à propos de l’objet et du but final
imposé par la règle morale. Ce but ne se
réduit pas à la vertu, qui constitue pourtant
le bien suprême : ce n’est toutefois
pas tout le bien auquel peut légitimement
aspirer un être raisonnable fini. Le bonheur
participe également du « souverain
bien » — expression par laquelle Kant
désigne l’objet de la raison pratique dans
sa totalité inconditionnée. Les grandes
philosophies antiques, le stoïcisme ou
l’épicurisme, ont soulevé ce problème,
mais l’ont laissé irrésolu, en identifiant les
deux termes du problème, quoique de
façon opposée. En affirmant au contraire
leur différence, et en recherchant le lien
synthétique qui les unit, Kant parvient à
formuler l’antinomie du bonheur et de la
vertu : le désir du bonheur conditionne
l’être vertueux (thèse) ; la vertu est la
cause efficiente du bonheur (antithèse).
La première affirmation est absolument
fausse, à la lumière de l’Analytique. La
seconde affirmation est également fausse,
cependant pas de façon absolue, mais seulement
dans la mesure où elle se réfère
au déterminisme des phénomènes. Kant
indique un moyen de dépasser l’antinomie,
en se référant au christianisme, sous
sa forme la plus rationnelle. Dans la perspective
d’un ordre supérieur, suprasensible,
dont nous participons tous en tant
que sujets moraux, il introduit certains
postulats, à savoir des propositions non
démontrables, mais qui se rendent nécessaires
sur le plan pratique, car ils contiennent
les conditions de l’objet nécessaire
de la volonté : l’immortalité de l’âme,
pour qu’il soit possible de s’ajuster à la loi
morale, dans une progression à l'infini;
l'existence de Dieu, car il faut supposer
qu’un être intelligent et saint est l’auteur
de la nature pour que coïncident bonheur
et vertu. Ces idées, qui concernent respectivement
le bien suprême et le souverain
bien, expriment une nécessité morale subjective
(« je veux qu'il y ait un Dieu », dit
l’homme juste), un besoin pratique de la
raison fondé sur la conscience de devoir
promouvoir le souverain bien. C’est ainsi
que Kant peut affirmer la primauté de la
raison pratique, en vue de son intérêt
« supérieur ». D'où l’idée d’une destination
pratique de l’homme, dont on
connaît la fortune ultérieure, et qui n’est
qu’un aspect du retentissement durable
%877
qu'a eu l'éthique formelle du devoir et de
l'intention.

De cette revendication du caractère
autonome de la morale dépend également
la conception kantienne de la religion, qui
trouve son fondement dans la moralité et
qui s’y ramène. D’après Kant, il ne faut
pas considérer que certaines actions sont
obligatoires parce qu’elles seraient des
ordres de Dieu, sous peine de tomber
dans l’hétéronomie — et donc aller à l’encontre
de la morale et de la liberté ; il faut
au contraire considérer comme des ordres
de Dieu les actions auxquelles nous
sommes moralement tenus. La religion
« dans les limites de la simple raison »
devient la connaissance de tous les
devoirs compris comme des ordres divins,
mais non comme des « sanctions » ou des
commandements en soi arbitraires et
contingents, dictés par une volonté extérieure.
C’est pourquoi il n’y aura pas à
proprement parler des religions différentes,
mais seulement une religion
morale universelle, une sorte d'Église
invisible qui regroupe tous ceux qui révèrent
Dieu d’une manière pure et inconditionnée.

