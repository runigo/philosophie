
\section{{\it Les jugements analytiques et synthétiques}}
%%%%%%%%%%%%%%%%%%%%%%%%%%%%%
%{\bf }{\it }{\bf --}{\footnotesize X}$^\text{e}$

Sont analytiques les jugements dans lesquels
le concept de prédicat est déjà
pensé dans le concept de sujet (par exemple
« tous les corps sont étendus »). Régis
uniquement par le principe d'identité ou
de non-contradiction (qui est d’ailleurs
valable pour n’importe quelle proposition),
ils ont une valeur explicative, mais
n’accroissent en rien le contenu de la
connaissance. Kant nomme au contraire
synthétiques les jugements qui ajoutent
au concept du sujet un prédicat qui n’était
pas du tout pensé en lui; ils accroissent
donc notre connaissance. La validité d’un
jugement tel que « les corps sont lourds »
est déterminée par les conditions de l’expérience,
et constitue un exemple de jugement
synthétique {\it a posteriori} ou extensif.
Or, il existe, selon Kant, des jugements
également synthétiques et  extensifs,
valables {\it a priori}, c'est-à-dire qui répondent
aux conditions requises de l’universalité
et de la nécessité (par exemple
« tout changement a une cause ») ; c’est
sur ce type de jugements synthétiques {\it a
priori} que se fondent les mathématiques
et la physique. Kant cherche ainsi à maintenir
les acquis de la science newtonienne
à l'abri des objections soulevées par
Hume, en prenant le problème du synthétique
{\it a priori} comme le fil conducteur
d’un examen de la raison, entendue
comme la faculté des connaissances {\it a
priori}.
