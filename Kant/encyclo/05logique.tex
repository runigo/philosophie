
\section{{\it La logique transcendantale}}
%%%%%%%%%%%%%%%%%%%%%%%%%%%%%
%{\bf }{\it }{\bf --}{\footnotesize X}$^\text{e}$

Kant expose la conception de l’espace
et du temps comme des formes {\it a priori} de
la sensibilité (il définit plus précisément
sa position comme un idéalisme « formel »,
ou encore « transcendantal », par
opposition à l’idéalisme de Berkeley, qu’il
qualifie de « matériel » ou « empirique »)
dans la première partie de la {\it Critique de
la raison pure}, sous le titre « Esthétique
transcendantale » (« esthétique » au sens
de doctrine de la sensibilité). Elle constitue,
avec la « Logique transcendantale »,
la « Doctrine des éléments » (intuitions et
concepts), suivie de la « Doctrine de la
méthode », beaucoup plus courte. Kant
distingue ensuite la logique « générale »
ou formelle, qu’il estime réalisée pour
l'essentiel avec Aristote, et la logique
transcendantale, qui est un apport original
de sa réflexion. L’attribut « transcendantal »
désigne une connaissance qui ne
concerne pas directement les objets, mais
« notre façon de les connaître, en tant que
celle-ci doit être possible {\it a priori} ». Aussi,
%874
à la différence de la logique générale, la
logique transcendantale a pour but de
rechercher la possibilité de nos connaissances
sans faire abstraction de la référence
aux objets, en tant qu’elle est
possible {\it a priori}. Les deux grandes subdivisions
de la logique transcendantale,
l'« Analytique» et la « Dialectique »,
représentent respectivement les moments
positif et négatif de sa thèse : la première
coïncide avec la logique de la vérité, régie
par l’entendement ({\it Verstand}) ; la seconde
avec la logique de l'apparence, régie par
la raison ({\it Vernunft}).

