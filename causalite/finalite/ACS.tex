
%%%%%%%%%%%%%%%%%%%%%
\section{André Comte-Sponville}
%%%%%%%%%%%%%%%%%%%%%
\subsection{Fin}

Le mot a deux sens très différents, qu’il importe de ne pas confondre :
il peut désigner la limite ou le but, le terme ou la destination, la {\it finitude}
ou la {\it finalité}. Par exemple que la mort soit la {\it fin} de la vie, cela fait partie
de sa définition ; mais ne nous dit pas si elle est son {\it but}, comme le voulait
Platon, ou simplement, et comme disait Montaigne, son {\it bout} ({\it Essais}, III, 12,
1051). On dira que les deux peuvent ne faire qu’un : tel serait le cas de la ligne
d’arrivée dans une course. Je ne suis pas sûr que l’exemple soit pertinent (le but,
dans une course, est moins d’arriver que de vaincre), ni, encore moins, qu’on
puisse le généraliser. Ce n’est pas pour son dernier mot qu’on écrit un livre, pas
pour son dernier jour qu’on vit tous les autres, pas pour sa clôture qu’on cultive
un jardin.

Ce qui reste vrai, en revanche, et que les Grecs savaient mieux que nous,
c’est que ces deux sens sont à la fois liés et asymétriques : la finalité suppose la
finitude, non la finitude la finalité. L’infini, par définition, ne peut aller nulle
part, ni tendre vers quoi que ce soit. Imaginez une autoroute infinie : {\it où} pourrait-elle
aller ? Un univers infini : {\it vers quoi} pourrait-il tendre ? Un Dieu infini :
{\it à quoi} pourrait-il servir ? Alors que le moindre de nos actes, pour fini qu’il soit,
n’en a pas moins une finalité (le but que nous visons à travers lui). L’infini n’est
pas un but plausible, ni n’en peut avoir. Seul le fini vaut la peine qu’il se donne.

\subsection{Finale (cause —)}

Une cause, c’est ce qui répond à la question {\it « Pourquoi ? »} ;
une cause finale, ce qui y répond par l’énoncé
d’un {\it but}. Par exemple, explique Aristote, la santé est la {\it cause finale} de la promenade,
si, à la question : « Pourquoi se promène-t-il ? », on peut répondre
légitimement : « Pour sa santé » ({\it Physique}, II, 3). Ainsi la cause finale est {\it ce en
vue de quoi} quelque chose existe, qui n’existerait pas sans cela.

Mais alors pourquoi ne se promènent-ils pas tous, ni toujours ? La santé
n'est-elle une fin que pour quelques-uns, et à quelques moments ? Non pas ;
mais elle n’agit, comme fin, qu’à condition qu’un désir actuel et actif la vise.
C’est où l’on échappe au finalisme — et à Aristote — par Spinoza : la santé n’est
%— 248 
une cause finale que pour autant que le désir de santé est une cause efficiente
({\it Éthique}, IV, Préface).

\subsection{Finalisme}

Toute doctrine qui accorde aux causes finales un rôle effectif.
On l’illustre souvent par les exemples délicieusement outrés de
Bernardin de Saint-Pierre. Pourquoi les melons sont-ils divisés en côtes ? Pour
qu’on puisse plus aisément les manger en famille. Pourquoi Dieu nous a-t-il
donné des fesses ? Pour que nous puissions plus confortablement nous asseoir
et méditer sur les merveilles de sa création. Mais cela ne doit pas faire oublier
que la plupart des grands philosophes — depuis Platon et Aristote jusqu’à
Bergson et Teilhard de Chardin — ont été finalistes. Au reste, comment y
échapper, si l’on croit en un Dieu créateur ou ordonnateur ? Même Descartes,
qui fit tant pour que les scientifiques cessent de chercher des causes finales, ne
contestait pas qu’elles puissent exister en Dieu, mais seulement que nous puissions
les connaître ({\it Principes}, III, 1-3). Quant à Leibniz, il y voyait « le principe
de toutes les existences et des lois de la nature, parce que Dieu se propose toujours
le meilleur et le plus parfait » ({\it Discours de métaphysique}, 19). Pourquoi
aurions-nous des yeux, si ce n’était {\it pour voir} ?

Ce dernier exemple, repris par Leibniz et tant d’autres, suggère bien l’essentiel.
Pourquoi voyons-nous ? Parce que nous avons des yeux. Pourquoi avons-nous
des yeux ? Pour voir. Ainsi les yeux sont la cause efficiente de la vue ; la
vue, la cause finale des yeux. Mais laquelle de ces deux causes existe d’abord ?
Est-ce la fonction qui crée l’organe (ce qui est une forme de finalisme), ou
l'organe qui crée la fonction ? Les matérialistes, presque tous, presque seuls,
choisissent résolument le deuxième terme de l'alternative. Penser que nous
avons des yeux pour voir, explique Lucrèce, « c’est faire un raisonnement qui
renverse le rapport des choses, c’est mettre partout la cause après l’effet » (IV,
823 sq.). Mais pourquoi, alors, avons-nous des yeux ? Par hasard ? Non pas ;
mais seulement par des causes efficientes, qui renvoient au passé de l’espèce
(par l’hérédité et la sélection naturelle), non à l’avenir de l’individu.

Spinoza, sur cette question comme sur tant d’autres, est du côté des matérialistes.
Le finalisme, pour lui aussi, « renverse totalement la nature : il considère
comme effet ce qui en réalité est cause, et met après ce qui de nature est
avant » ({\it Éthique}, I, Appendice). C’est là le préjugé fondamental, dont tous les
autres dérivent : « Les hommes supposent communément que toutes les choses
de la nature agissent, comme eux-mêmes, en vue d’une fin » ({\it ibid.}). S'agissant
de la nature, c’est clairement une illusion. Mais s'agissant d’eux-mêmes ? C'en
est une aussi, du moins si l’on y voit la cause de leurs actes : cette illusion
s'appelle alors le libre arbitre. Soit par exemple cette maison que je construis.
%— 249 —
Pourquoi le fais-je ? Si on répond {\it « pour l'habiter »}, comme beaucoup le feront
spontanément, cela signifie que ce qui n’est pas encore (l’habitation) explique
et produit ce qui est (le travail de construction), que c’est la fin, comme dira
Sartre, qui « éclaire ce qui est » : tel est le paradoxe de la liberté ({\it L'Étre et le
Néant}, p. 519-520, 530, 577-578...). Mais comment ce qui n’est pas encore
pourrait-il produire ou expliquer quoi que ce soit ? Que les hommes agissent
toujours « en vue d’une fin », comme le reconnaît Spinoza, cela ne prouve pas
que cette fin soit la cause de l’action : ce qu’on prend pour une cause finale
n’est rien d’autre qu’un désir déterminé (ici le désir d’habiter cette maison),
et ce désir «est en réalité une cause efficiente » ({\it Éthique}, IV, Préface). Les
hommes se trompent en ce qu’ils se figurent être libres ; cette illusion n’est
pas autre chose qu’un finalisme à la première personne : ils ont conscience de
viser une fin, point des causes efficientes qui les déterminent à le faire
(I, Appendice).

Ainsi il n’y a pas de finalité du tout : il n’y a que la puissance aveugle de la
nature, et celle, qui peut être éclairée, du désir.

\subsection{Finalité}

Le fait de tendre vers une fin ou un but. C’est le cas par exemple
de la plupart de nos actions, et même, selon Freud, de nos actes
manqués. Cela ne prouve pas que cette fin soit la {\it cause} de l’acte. C’est où la
finalité, qui est une donnée de la conscience ou un fait, se distingue du finalisme,
qui est une doctrine et un contresens. On agit {\it pour} une fin, mais {\it par} un
désir : la finalité n’est elle-même qu’un effet de l'efficience du désir.

On pourrait en dire autant de la finalité du vivant, qui est un fait d’expérience,
mais dont rien ne prouve qu’elle soit une {\it cause}. C’est pourquoi nos biologistes
parlent plutôt, pour la désigner, de téléonomie, qui est une finalité sans
finalisme (une finalité pensée comme effet, point comme cause).

%%%%%%%%%%%%%%%%%%%%%%%%%%%%%%%%%%%%%%%%%%%%%%%%%%%%%%%%%%%%%%%%%%%%%%%%%%%
