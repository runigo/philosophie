
%%%%%%%%%%%%%%%%%%%%%
\section{Pratique de la philosophie}
%%%%%%%%%%%%%%%%%%%%%

{\bf }

\begin{itemize}[leftmargin=1cm, label=\ding{32}, itemsep=1pt]
\item {\footnotesize ÉTYMOLOGIE} : latin {\it }, .
\item {\footnotesize SENS ORDINAIRES ET PHYSIQUE} : état de choses qui ne peut pas ne pas exister.
\item {\footnotesize MÉTAPHYSIQUE} : puissance (parfois divinisée) qui gouverne le cours
de la réalité.
\item {\footnotesize LOGIQUE} : {\bf 1}. Caractère de ce qui ne peut être faux, de
ce qui est universellement vrai. {\bf 2}. Relation inévitable entre deux propositions.
\item {\footnotesize SENS DÉRIVÉ} : le besoin ; ce dont un être ne peut pas
se passer.
\end{itemize}

\begin{itemize}[leftmargin=1cm, label=\ding{32}, itemsep=1pt]
\item {\footnotesize TERMES VOISINS} : besoin ; destin ; fatalité ; universalité.
\item {\footnotesize TERMES OPPOSÉS} : contingence.
\item {\footnotesize CORRÉLATS} : Catégorie ; déterminisme ; liberté ; modalité.
\end{itemize}

%%%%%%%%%%%%%%%%%%%%%%%%%%%%%%%%%%%%%%%%%%%%%%%%%%%%%%%%%%%%%%%%%%%%%%%%%%%
