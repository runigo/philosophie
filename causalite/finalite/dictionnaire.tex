
%%%%%%%%%%%%%%%%%%%%%
\section{Dictionnaires}
%%%%%%%%%%%%%%%%%%%%%
\subsection{Encyclopédique}
{\bf finalité} ({\it nf}\,) {\bf 1} Caractère de ce qui tend vers un but.

\subsection{Étymologique}
{\bf 1. fin}, {\it substantif},lat {\it finis}, terme.
{\footnotesize (final, fnalisme, finaliste, finalité, finale, finir, finisseur, finissage, finition, finer, finance, financier, financer, autofinancement, autofinancer, afin de, afin que, enfin)}

{\bf 2. fin}, {\it adjectif}, 1080 Roland,  « qui est au point extrême », d'où « accompli », par extension « délicat ».
{\footnotesize (fine, finement, finesse, finasser, finasserie, finassier, finasseur, finaud, finet, finette, fingnoler, fignolage, fignoleur, affiner, affinement, affinage, affinerie, affineur, rafiner, raffinage, raffinement, raffineur, raffinerie, superfin, surfin)}


\subsection{Vocabulaire}

{\bf Fin} — \si{Épist.} et \si{Méta.} {\bf 1.} ({\it Opp.} : {\it moyen}$^1$).
Ce pour quoi une chose est ou se fait ; ce vers quoi elle tend consciemment
ou inconsciemment : « Vous ne détournerez nul être de sa fin » (La
Fontaine) ; « Il est, dans la nature, des fins que la raison ne saurait
méconnaître » (Ch. Bonnet), — {\bf 2.} But*, ce vers quoi tend un
acte consciemment et intentionnellement$^1$ : « Ce qui est désiré pour
l'amour de soi-même et à cause de
sa propre bonté s’appelle fin » (Bossuet); « Nous sommes sujets à nous
abuser quand nous voulons déterminer les fins ou conseils de Dieu »
(Leibniz, {\it Disc. méta.}, XIX). — \si{Vulg.} 3 Terme, cessation :
« Toutes les choses de ce monde prennent fin » (Sévigné).
%80

{\bf Final} — \si{Épist.} et \si{Méta.} {\bf 1.} Qui se rapporte à une fin$^1$.
{\it Cause finale} ({\it opp.} : {\it efficiente}*) : la fin$^1$ elle-même,
considérée comme la raison d’être de la chose (cf. {\it Cause}$^4$) : « La
connaissance des causes finales n’est pas nécessaire dans la physique »
(Malebranche, {\it Méditations chrétiennes}, XI) ; « La voie des causes
efficientes est plus profonde, mais la voie des finales est plus aisée »
(Leibniz, {\it Disc. méta.}, 22) ; « La science reste défavorable aux causes
finales que nulle part elle ne discerne avec évidence » (Le Roy).
{\it Argument des causes finales} (syn. : {\it physico-téléologique}*) :
celui d’après lequel les faits naturels, « disposés avec ordre, intelligence,
prévoyance pour les besoins et le bien de chaque être », prouvent
« l'existence d’une cause intelligente et souverainement bonne » (Franck).

— \si{Vulg.} 2 ({\it Opp.} : {\it initial}). Terminal, dernier.

{\bf Finalisme} — \si{Méta.} \fsb{S. norma.} Doctrine qui
admet dans l’univers : {\bf 1.} une finalité$^1$ fondée sur l'argument des
causes finales$^1$ (syn. : {\it providentialisme}*); — {\bf 2.} de la
finalité* en un sens quelconque : « Le finalisme renverse l’ordre naturel :
il explique le présent par l'avenir » (Goblot).

{\bf Finalité} — \si{Épist.} et \si{Méta.} {\bf 1.} Caractère de ce qui tend
vers une fin$^2$ de façon consciente : « La volonté est toujours
indissolublement finalité par les desseins, mécanicité par les moyens
 » (Pradines), — {\bf 2.} Adaptation des moyens$^1$ à une fin$^1$, des
organes aux besoins, soit par une activité intelligente, soit par une
« finalité sans intelligence » (Goblot) : « Nier la finalité organique, c'est
le plus audacieux des paradoxes » (id.) ; « La finalité du plaisir et de la
douleur ».
% 80
— {\bf 3.} Dépendance des parties ou éléments à l'égard d’un tout :
« La finalité n’est pas la conformité à l’idée : elle est l'idée » (Hamelin).
— {\bf 4.} {\it Finalité externe} : celle où la fin est extérieure à l'être
considéré. {\it F. interne} : celle où la fin est l'être lui-même (sens 3).
— {\bf 5.} {\it Principe de finalité} : « La nature ne fait rien en
vain » (Aristote) ou « Tout être à une fin$^1$. »
%%%%%%%%%%%%%%%%%%%%%%%%%%%%%%%%%%%%%%%%%%%%%%%%%%%%%%%%%%%%%%%%%%%%%%%%%%%
