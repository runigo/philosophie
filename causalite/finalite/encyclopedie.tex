
%%%%%%%%%%%%%%%%%%%%%
\section{Encyclopédie de la philosophie}
%%%%%%%%%%%%%%%%%%%%%
%
\subsection{Finalité}

{\bf finalité} : caractéristique que l’on reconnaît
à tout événement ou processus que l’on
considère intéressé ou guidé par une fin.
En général, on attribue cette caractéristique au comportement humain, dans la
mesure où l’homme tend consciemment à
des buts précis; seuls quelques philosophes, parmi lesquels Hobbes et Spinoza, ont retenu cette caractérisation du
comportement humain comme insignifiante et trompeuse. La finalité s'oppose
à l’instinct ; de célèbres expériences de
psychologie animale donnent les premiers
signes de cette opposition avec l’utilisation d’une chose, chez certains animaux
supérieurs, comme moyen pour atteindre
un but. Toute tentative d’extension de la
notion de finalité au-delà du champ
représenté par l’action de l’homme s’appelle finalisme.

\subsection{Finalisme}

{\bf finalisme} : conception qui affirme l’existence de la finalité dans des secteurs de la
réalité autres que celui où elle est unanimement reconnue, c’est-à-dire en dehors
même de l’activité consciente, propre aux
hommes, appliquée à des buts déterminés.
C’est précisément l’intentionnalité de
l'acte de l’homme, et par conséquent le
rapport de subordination des moyens aux
fins, qui constitue le modèle que le finalisme se propose, par analogie, d’étendre
plus ou moins largement et de généraliser
à l’ensemble de l’univers.
Dans la philosophie grecque, le finalisme est né en réaction au physicisme des
présocratiques. Les structures physiques
du réel sont subordonnées, pour Platon, à
l'intelligence ordonnatrice : qui dirige
toute chose et le tout « de la meilleure
façon ». Jusqu'à Leibniz et au-delà, on
reviendra toujours à l’idée que le finalisme n’exclut pas les rapports mécaniques entre les choses, mais que ces
rapports sont subordonnés, instrumentaux, et non pas exhaustifs. Dans la {\it Physique},
%582
Aristote élabore le finalisme en
réponse à une hypothèse d’Empédocle,
celle d’une évolution biologique liée au
hasard. Dans les temps anciens, la vie animale serait née selon des modes d’organisation divers : certains animaux auraient
disparu tout de suite, alors que d’autres
auraient été aptes à la survie. Il voit en
la permanence des races un témoignage
décisif du finalisme. D’un animal, naît
toujours un autre animal de la même
espèce ; ce fait, ainsi que tout ce qui arrive
« toujours » dans la nature, ne s’expliquerait pas si la nature était gouvernée par
le hasard. La finalité est justement, selon
Aristote, l’unique alternative au hasard.
Pour Aristote, le finalisme est valable
aussi pour les corps inorganiques, selon la
doctrine des différents « lieux naturels »
en direction desquels les corps se déplaceraient spontanément tant qu’ils n’en sont
pas empêchés par la violence, et pour les
astres, dont le mouvement serait déterminé
par l’amour et le désir de la perfection
divine. Comme Platon, Aristote se sert du
modèle de l’artisan, selon l'exemple célèbre
du sculpteur, pour démontrer que la fin
est une des « causes » ou « principes »
nécessaires à l’explication complète de la
nature ; toutefois, le trait caractéristique
du finalisme aristotélicien est que la finalité de la nature est intrinsèque, et qu’elle
ne dépend pas d’une intelligence providentielle. Après Aristote, l'opposition au
finalisme, qui a suscité une conception
mécaniste du monde, même si elle n’était
pas déterministe, vient de l’empirisme :
partout dans la nature dominent le hasard
et la nécessité aveugle, sauf à envisager
les intentions conscientes des hommes
pour lesquelles, seulement, on peut parler
de fins.

Avec le christianisme, le concept de
providence divine devient le fondement
du finalisme, comme il l’avait été pour les
stoïciens. L'homme, en particulier, serait
la finalité de la création, tout le reste
étant destiné à le servir ; de façon générale, chaque niveau de la réalité est organisé en fonction de niveaux supérieurs : la
nature inorganique en fonction de l’organique, les plantes en fonction des animaux, et ainsi de suite; et chaque
phénomène simple à une finalité identifiable, suivant le principe selon lequel
Dieu ne fait rien en vain. À partir de
Kant, on a caractérisé cette conception
par l’expression de « finalisme externe »,
%583
justement parce que de cette façon les
choses ont leur propre finalité en dehors
d’elles-mêmes, dans l’autre ; alors que le
terme « interne » s'applique au finalisme
biologique, fondé sur le principe de
l'unité et de la conservation de l’organisme, car tel serait le but immanent de
toutes les parties et de tous les processus
vitaux de l’organisme.

\subsubsection{La critique du finalisme
dans la science moderne}

La bataille contre le finalisme, interne
ou externe, sera reprise par le mécanisme
moderne, à commencer par Galilée, Francis Bacon et Descartes qui insistent sur la
vanité des prétentions explicatives du
finalisme et situent au centre de la science
naturelle la notion de causalité, plus ou
moins comprise dans le sens de la cause
efficiente d’Aristote. Spinoza, en particulier, argumente sa critique dans l’important appendice de la première, partie de
L'{\it Éthique}. Le finalisme est un préjugé
désastreux, qui naît de l’ignorance naturelle des hommes et d’une attitude utilitariste tout aussi spontanée ; à l’illusion
vaine mais rassurante que tout est fait
pour l’homme, s'ajoute la mentalité
anthropomorphique commune qui, interprétant tout selon le modèle de l'artisan,
s’interdit la connaissance de la nécessité
absolue (qui, pour Spinoza, domine non
seulement le monde mais Dieu lui-même), conduisant ainsi à la superstition
du dieu personnel, libre et créateur.
D'autre part, un mouvement de réaction
pour la défense du finalisme naît à la fin
du {\footnotesize XVII}$^\text{e}$ s., avec Boyle, Malebranche,
Newton, Leibniz. Ce mouvement se poursuivra dans les différentes tendances
déistes, avec l’image du Dieu-Architecte.
C’est substantiellement dans cette même
ligne de pensée que se situe Kant ; mais il
ramène, dans la {\it Critique de la faculté de
juger}, le finalisme biologique au type de
jugement qu’il nomme « réfléchissant »,
en opposition au jugement « déterminant » (objectivement valide) qu’il estime
propre aux seules sciences mathématiques
de la nature. Selon lui, le cerveau humain
ne peut concevoir les organismes vivants
comme  réductibles aux mécanismes
aveugles de la matière, justement parce
que, face à leurs spécificités, il ne peut les
penser hors du concept de finalité : prédominance harmonieuse du tout sur les parties « comme si » le tout était le but de
%
celles-ci. Pour Kant, l’argument physicothéologique n’a aucune valeur théorique
pour la démonstration de l’existence de
Dieu ; cet argument, formalisé par saint
Thomas comme la « cinquième voie » (la
finalité évidente partout dans le monde
implique une intelligence suprême ordonnatrice), avait été très souvent repris.
Ainsi Kant à la fois prend acte de la critique serrée à laquelle avait été soumis le
finalisme aux {\footnotesize XVII}$^\text{e}$ et {\footnotesize XVIII}$^\text{e}$ s. et ouvre la
voie à la dernière version du finalisme
dans la pensée occidentale, en particulier
le finalisme allemand du {\footnotesize XIX}$^\text{e}$ s. La
conjonction d’un finalisme interne, de
type aristotélicien, avec la théorie de la
providence, de dérivation chrétienne,
dans une version immanentiste, est
typique de Hegel. Selon la théorie hégélienne, l’histoire du monde a une fin
intrinsèque qui la guide et rend intelligibles les événements. L'Esprit, originellement « en soi », en puissance, comme
une graine, se développe jusqu’à sa réalisation comme rationalité et liberté dans le
monde politique et jusqu’à l'acquisition
de la conscience de soi d’un point de vue
philosophique (ce qui advient justement
dans le système hégélien). Au {\footnotesize XX}$^\text{e}$ s., le
finalisme réapparaît surtout au sein des
courants de la pensée biologique, mais il
est contesté de façon déterminante par
l’évolutionnisme dérivé de la pensée darwinienne.

%%%%%%%%%%%%%%%%%%%%%%%%%%%%%%%%%%%%%%%%%%%%%%%%%%%%%%%%%%%%%%%%%%%%%%%%
