\documentclass[12pt, a4paper]{report}
%\documentclass[11pt, a4paper]{article}

%====================== PACKAGES ======================


\usepackage[french]{babel}

\frenchbsetup{StandardLists=true}
\usepackage{enumitem}
\usepackage{pifont}

\usepackage[utf8x]{inputenc}
%\usepackage[latin1]{inputenc} % Feynman ?

%pour gérer les positionnement d'images
\usepackage{float}
\usepackage{amsmath}
\DeclareMathOperator{\dt}{dt}
\usepackage{graphicx}
\usepackage{tabularx}
\usepackage[colorinlistoftodos]{todonotes}
\usepackage{url}
%pour les informations sur un document compilé en PDF et les liens externes / internes
\usepackage[pdfborder=0]{hyperref}
\hypersetup{
	colorlinks = true
	}
%pour la mise en page des tableaux
\usepackage{array}
\usepackage{tabularx}
\usepackage{multirow}
\usepackage{multicol}
\setlength{\columnsep}{50pt}
%pour utiliser \floatbarrier
%\usepackage{placeins}
%\usepackage{floatrow}
%espacement entre les lignes
\usepackage{setspace}
%modifier la mise en page de l'abstract
\usepackage{abstract}
%police et mise en page (marges) du document
\usepackage[T1]{fontenc}
\usepackage[top=2cm, bottom=2cm, left=2cm, right=2cm]{geometry}
%Pour les galerie d'images
\usepackage{subfig}

\usepackage{pdfpages}
\usepackage{tikz}
%\usepackage{tikz}
\usetikzlibrary{trees}
\usetikzlibrary{decorations.pathmorphing}
\usetikzlibrary{decorations.markings}
\usetikzlibrary{decorations.pathreplacing,calligraphy}
%\usetikzlibrary{decorations}
\usetikzlibrary{angles, quotes}
\usepackage{verbatim}

\usepackage{appendix}

\usepackage{comment}

\usepackage{xcolor}


%\PreviewEnvironment{tikzpicture}
%\setlength\PreviewBorder{0pt}%

%====================== INFORMATION ET REGLES ======================

%rajouter les numérotation pour les \paragraphe et \subparagraphe
\setcounter{secnumdepth}{4}
\setcounter{tocdepth}{4}

\hypersetup{							% Information sur le document
pdfauthor = {Stephan Runigo},			% Auteurs
pdftitle = {Documentation},			% Titre du document
pdfsubject = {Documentation},		% Sujet
pdfkeywords = {Document},	% Mots-clefs
pdfstartview={FitH}}	% ajuste la page à la largeur de l'écran
%pdfcreator = {MikTeX},% Logiciel qui a crée le document
%pdfproducer = {} % Société avec produit le logiciel
%======================== DEFINITION COMMANDES ========================
\newcommand{\si}[1]{\textsf{\textit {#1}}}
\newcommand{\fsb}[1]{\textsf{\textbf {\footnotesize #1}}}
\newcommand{\bi}[1]{\textbf{\textit {#1}}}
%======================== DEBUT DU DOCUMENT ========================
%
\begin{document}
%
%
% Titre, résumé, ... %
%
%
\begin{titlepage}
%
~\\[1cm]

\begin{center}
%\includegraphics[scale=0.5]{./presentation/chambreABulle}
\end{center}

\textsc{\Large }\\[0.5cm]

% Title \\[0.4cm]
\HRule

\begin{center}
{\huge \bfseries Spiritualisme et\\
matérialisme\\[0.4cm] }
\end{center}

\HRule \\[1.5cm]


\vfill

\hfill
\begin{minipage}{0.4\textwidth}
\begin{flushright} \large
%\emph{Auteur:}\\
%Stephan \textsc{Runigo}
Extraits de dictionnaires et d'encyclopédies
\end{flushright}
\end{minipage}

\vfill
{\sf \footnotesize
\begin{itemize}[leftmargin=1cm, label=\ding{32}, itemsep=1pt]
\item {\bf Objet : } Étudier les concepts liés au spiritualisme et au matérialisme.
\item {\bf Contenu : } Définitions et philosophie encyclopédique.
\item {\bf Public concerné : } Néophyte.
\end{itemize}
}

\vfill

% Bottom of the page
{\large \today}

\end{titlepage}

\newpage
%\begin{center}
\Large
Résumé
\normalsize
\end{center}
\vspace{3cm}
\begin{itemize}[leftmargin=1cm, label=\ding{32}, itemsep=21pt]
\item {\bf Objet : éclairer } .
\item {\bf Contenu : définitions} .
\item {\bf Public concerné : néophyte} .
\end{itemize}

\vspace{3cm}

résumé.

\vspace{3cm}

Ce document contient des définitions provenant de divers dictionnaires.




\thispagestyle{empty}

\begin{center}
\Large
%Introduction
Préambule
\normalsize
\end{center}
\vspace{3cm}

Ce document est une compilation d'articles provenant de quatre ouvrages : un dictionnaire encyclopédique de poche, {\it La pratique de la philosophie} destiné aux lycéens, une encyclopédie de la philosophie destinée aux néophytes, et le dictionnaire philosophique d'André Comte-Sponville.
les chapitres contiennent les articles des trois premiers ouvrages, les articles du dictionnaire de philosophie d'André Comte-Sponville sont reproduits en annexe.

\vspace{1.3cm}

Chaque chapitre contient les articles correspondant à une notion particulière. Ces notions ont été choisies en raison de leurs liens avec la question du hasard. Ces choix ont été guidés : 1. Par les renvois vers d'autres articles présent dans les ouvrages, 2. Mes propres choix, liés à mes connaissances, 3. La volonté d'obtenir une quantité raisonnable d'information.

\vspace{1.3cm}

Dans {\it La pratique de la philosophie}, l'article concernant la nécessité renvoit à des textes de Spinoza dont le choix reste subjectif à l'ouvrage. J'ai néanmoins reproduit en annexe ces textes ainsi que l'article concernant Spinoza. (les autres annexes sont d'autres renvois de cet ouvrage)

Les articles compilés dans ce document comportent donc les choix "discutables" réalisés dans les trois ouvrages utilisés. Il s'agit donc d'un document de "travail" destiné à apporter quelques points de vues philosophiques de manière relativement élémentaire.

\vspace{1.3cm}

Les trois premiers chapitres abordent les thèmes du hasard, de la nécessité puis du vitalisme. Les chapitres suivants élargissent le champ de vision philosophique en abordant les thèmes du déterminisme, de la contingence et de la providence.

\vspace{2.3cm}

\hfill Stephan Runigo

%%%%%%%%%%%%%%%%%%%%%%%%%%%%%%%%%%%%%%%%%%%%%%%%

%
\newpage
%
%

\thispagestyle{empty}
\begin{center}
\Large
%Introduction
Les ouvrages utilisés
\normalsize
\end{center}

\vspace{1.3cm}

Premières de couverture

\begin{center}
\includegraphics[scale=0.43]{./presentation/dictionnaire-1}
\hfill
\includegraphics[scale=0.43]{./presentation/pratique-1}
\hfill
\includegraphics[scale=0.43]{./presentation/encyclopedie-1}
\end{center}

\vspace{1.3cm}

Quatrièmes de couverture

\begin{center}
\includegraphics[scale=0.43]{./presentation/dictionnaire-2}
\hfill
\includegraphics[scale=0.43]{./presentation/pratique-2}
\hfill
\includegraphics[scale=0.43]{./presentation/encyclopedie-2}
\end{center}



%\newpage
%

%
% Table des matières
\setcounter{page}{0}
\thispagestyle{empty}
\tableofcontents
%
%espacement entre les lignes des tableaux
\renewcommand{\arraystretch}{1.5}
%
%====================== INCLUSION DES PARTIES ======================
%
~
\thispagestyle{empty}
%recommencer la numérotation des pages à "1"
\setcounter{page}{0}
\newpage

%%%%%%%%%%%%%%%%%%%%%
\section{Origine de l'idee de causalité}
%%%%%%%%%%%%%%%%%%%%%
%{\footnotesize XVIII}$^\text{e}$ siècle
%{\it }

Supposez qu’un homme, pourtant doué des plus puissantes
facultés de raison et de réflexion, soit soudain
transporté dans ce monde ; il observerait immédiatement,
certes, une continuelle succession d’objets, un événement
en suivant un autre ; mais il serait incapable de découvrir
autre chose. Il serait d’abord incapable, par aucun raisonnement,
d’atteindre l’idée de cause et d’effet, car les
pouvoirs particuliers qui accomplissent toutes les opérations
naturelles n’apparaissent jamais aux sens; et il
n’est pas {\it raisonnable} de conclure, uniquement parce
qu’un événement en précède un autre dans un seul cas,
que l'un est la cause et l’autre effet. Leur conjonction
peut être arbitraire et accidentelle. Il n’y a pas de raison
d'inférer l’existence de l'un de l’apparition de l’autre.
En un mot, un tel homme, sans plus d’expérience, ne
ferait jamais de conjecture ni de raisonnement sur aucune
%54
%{\it }
question de fait; il ne serait certain de rien d’autre que
de ce qui est immédiatement présent a sa mémoire et
a ses sens.

Supposez encore que cet homme ait acquis plus d’expérience
et qu’il ait vécu assez longtemps dans le monde
pour qu’il ait remarqué la conjonction-constante d’objets
ou d’événements familiers; que résulte-t-il de cette
expérience ? Il infére immédiatement l’existence d’un
des objets de l’apparition de l’autre. Il n’a pourtant acquis,
par toute son expérience, aucune idée, aucune connaissance
du pouvoir caché par lequel l’un des objets produit
l'autre; et ce n’est par aucun progrés de raisonnement
qu’il est engagé a tirer cette conclusion. Mais il se trouve
toujours déterminé à la tirer ; et, même-si on le convainquait
que son entendement n’a aucune part dans l’opération,
il continuerait pourtant le même cours de pensée. Il y a un
autre principe qui le détermine a former une telle conclusion.

Ce principe, c’est l'accoutumance, l’habitude. Car,
toutes les fois que la répétition d’une opération ou d’un
acte particulier produit une tendance a renouveler le
même acte ou la même opération sans l’impulsion d’aucun
raisonnement ou progrés de l’entendement, nous disons
toujours que cette tendance est effet de l’accoutumance.
En employant ce mot, nous ne prétendons pas que nous
avons donné la raison derniére d’une telle tendance.
Nous désignons seulement un principe de la nature
humaine, universellement reconnu et bien connu par ses
effets. Peut-être pouvons-nous ne pas pousser plus loin
nos recherches, ni prétendre donner la cause de cette
cause ; mais il faut que nous nous en contentions comme
du principe dernier que nous puissions assigner pour nos
conclusions tirées de l’expérience. C’est une satisfaction
suffisante que de pouvoir aller aussi loin ; et il n’y a pas
à nous irriter de l’étroitesse de nos facultés parce qu’elles
ne veulent pas nous conduire plus loin. Assurément, nous
avons ici, au moins, une proposition trés intelligible,
sinon une vérité, quand nous affirmons que, après la
%55
%{\it }
constante conjonction de deux objets — chaleur et flamme,
par exemple, ou poids et solidité —, nous sommes déterminés
par la seule accoutumance a attendre l’un quand
parait l’autre. Cette hypothèse, semble-t-il, est la seule
qui explique la difficulté ; pourquoi tirons-nous de mille
cas une conclusion que nous étions incapables de tirer
d’un seul cas, qui ne différe à aucun égard des précédents ?
La raison est incapable de varier de pareille maniére. Les
conclusions qu’elle tire de la considération d’un cercle
sont les mêmes que celles qu’elle formerait à l’examen de
tous les cercles de univers. Mais si on n’a vu qu’un seul
corps se mouvoir sous l’impulsion d’un autre, personne
n’inférerait que tout autre corps se mouvra sous une
impulsion analogue. Toutes les conclusions tirées de
l'expérience sont donc des effets de l'accoutumance et
non des effets du raisonnement.

Alors, l'accoutumance est le grand guide de la vie
humaine. C’est ce seul principe qui fait que notre expérience
nous sert, c’est lui seul qui nous fait attendre, dans
le futur, une suite d’événements semblables a ceux qui ont
paru dans le passé. Sans l’action de l’accoutumance, nous
ignorerions complétement toute question de fait en dehors
de ce qui est immédiatement présent a la mémoire et aux
sens. Nous ne saurions jamais comment ajuster des moyens
en vue de fins, ni comment employer nos pouvoirs naturels
pour produire un effet. Ce serait a la fois la fin de toute
action aussi bien que de presque toute spéculation. ({\it Enquête
sur l’entendement humain}, trad. André L{\footnotesize EROY}, section V.)

%%%%%%%%%%%%%%%%%%%%%%%%%%%%%%%%%%%%%%%%%%%%%%%%%%%%%%%%%%%%%%%%%%%%%%%%


%%%%%%%%%%%%%%%%%%%%% 14
\section{La finalité n'est pas une providence (discours de Philon)}
%%%%%%%%%%%%%%%%%%%%%
%{\footnotesize XVIII}$^\text{e}$ siècle
%{\it }

Pourquoi y a-t-il si peu que ce soit de misère dans le
monde ? Ce n’est pas par hasard assurément. C’est donc
par quelque cause. Est-ce par l’intention de la Divinité ?
Mais elle est parfaitement bienveillante. Est-ce contraire
a son intention ? Mais elle est toute-puissante. Rien ne
peut ébranler la solidité de ce raisonnement, si court,
si clair, si décisif, A moins d’affirmer que ces sujets passent
toute humaine capacité et que nos mesures ordinaires
de la verité et de la fausseté n’y sont pas applicables :
théorie que j’ai soutenue d’un bout a Vautre, mais que
%79
%{\it }
vous avez, depuis le commencement, rejetée avec dédain
et indignation.

Mais je consentirai à me retirer encore de ce retranchement,
car je nie que vous puissiez jamais m’y forcer ;
j’accorderai que l’existence de la peine ou de la misère
chez l'homme soit {\it compatible} avec celle d’un pouvoir et
d'une bonté infinis chez la Divinité, même selon le sens
que vous donnez à ces attributs : en quoi toutes ces concessions
vous avancent-elles ? Une compatibilité simplement
possible ne suffit pas. Vous devez {\it prouver} l’existence de
ces attributs purs, sans mélange et sans borne, d’aprés
les présents phénoménes, mélangés et confus, et d’aprés
ceux-la seulement. Entreprise pleine d’espoir! Si purs
et si peu mélangés que fussent les phénoménes, encore, étant
finis, seraient-ils insuffisants dans ce but. Combien davantage
alors qu’ils sont en outre si disparates et si discordants !

Ici, Cléanthe, je me sens à l’aise dans mon argumentation.
Ici je triomphe. Auparavant, quand nous argumentions
touchant les attributs naturels d’intelligence
et de dessein, j’avais besoin de toute ma subtilité sceptique
et métaphysique pour échapper a vos prises. En bien des
spectacles de l’univers et de ses parties, de ces derniéres
surtout, la beauté et la convenance des causes finales
nous frappent avec une force si irrésistible, que toutes les
objections paraissent — ce que je crois qu’elles sont en
effet — arguties et sophismes purs ; et nous ne pouvons
alors imaginer comment il nous fut jamais possible d’y
accorder quelque poids. Mais il n’est pas de spectacle de
la vie humaine ou de la condition de l’humanité, d’où
nous puissions, sans la plus grande violence, inférer les
attributs moraux, ou apprendre à connaitre cette infinie
bienveillance, jointe à un infini pouvoir et a une sagesse
infinie, qu’il nous faut découvrir uniquement par les
yeux de la foi. C’est votre tour maintenant de tirer la
rame fatigante, et de soutenir vos subtilités philosophiques
contre les clairs préceptes de la raison et de l’expérience.
({\it Dialogues sur la religion naturelle}, 10$^\text{e}$ partie.)

%%%%%%%%%%%%%%%%%%%%%%%%%%%%%%%%%%%%%%%%%%%%%%%%%%%%%%%%%%%%%%%%%%%%%%%%

%%
\begin{appendix}
%

%%%%%%%%%%%%%%%%%%%%%
\chapter{La vie}
%%%%%%%%%%%%%%%%%%%%%
%{\it }{\oe}

David Hume est né le 26 août 1711 à Édimbourg
où son pére exerçait la profession d’avocat. Ce
dernier étant mort en 1714, Mrs Hume se retira
avec ses trois enfants John, Katherine et David
dans la propriété familiale de Ninewells, le domaine
des « Neuf Sources » situé dans la pittoresque campagne
(avec ses falaises, ses ruisseaux et ses bois)
du Berwickshire. L’oncle de David, pasteur du
village voisin de Chirnside, dirigea sa toute première
éducation. L’enseignement religieux que reçut le
jeune David semble avoir été particulièrement
austère et maladroit (le Révérend George Hume
se plaisait, dans ses sermons, à humilier publiquement
les jeunes filles dont la grossesse révélait les
péchés charnels). L’antipathie précoce de Hume
pour le christianisme vient en partie de là.

Cependant le petit David échappa assez rapidement
à cette atmosphère déprimante. Élève dès
l'âge de onze ans du collège d’Édimbourg (renommé
à juste titre et qui deviendra plus tard université),
il se trouve dans une ambiance intellectuelle beaucoup 
plus stimulante. Il y écoute les cours de « philosophie
naturelle », c’est-a-dire de physique, de
Robert Stewart (disciple de Newton aprés avoir été
cartésien) et se souviendra certainement de ses
%6
%{\it }{\oe}
leçons lorsqu’il rêvera d’appliquer la méthode expérimentale
à la morale et à la métaphysique. Mais
la formation de Hume au collège fut essentiellement
littéraire. C’est de cette époque que date son goût
de Virgile et de Cicéron. Ce sont les textes (notamment
le {\it De Natura Deorum}) où Cicéron résume les
débats philosophiques des Stoïciens et des Epicuriens
qui découvrent à Hume le monde des discussions
métaphysiques. Revenu à Ninewells dès sa quinzième
année, le jeune David se livre avec passion à la
lecture des anciens et des modernes. Il dévore
Montaigne, Bacon, Malebranche, Bayle, mais aussi
Milton, Pope, Swift, Shaftesbury. À l’âge de vingt
ans, il a déja rempli un gros cahier de réflexions sur
le problème religieux, sur la psychologie, sur l’histoire.
Cette activité intellectuelle bouillonnante, les
leçons de scepticisme qu’il tire de lectures aussi
diverses, un conflit avec sa famille qui voudrait,
malgré lui, l'orienter vers des études juridiques
provoquent une crise de dépression passagère dont
nous trouvons le témoignage dans un curieux brouillon
de lettre à un médecin célèbre (qui n’est pas
comme on l’a cru George Cheyne, mais le D$^\text{r}$ Arbuthnot).
David qui n’a hérité de son pére que d’une
toute petite rente doit de toute urgence prendre
un état. Après un bref essai dans le commerce (au
service d’un marchand de Bristol), David Hume
décide de ne plus résister à sa vocation : Il sera
philosophe et homme de lettres, et il entend conquérir
la gloire. Pour pouvoir subsister, il se rend en France
(où la vie est à l’époque beaucoup moins chère),
s’installe à Reims en 1734 à l'hôtel du {\it Perroquet
vert}, puis à La Flèche en Anjou, tout près du collège
%7
%{\it }{\oe}
de Jésuites ou Descartes fut élève. Il y rédige, à
peine âgé de vingt-trois ans, son chef-d’{\oe}uvre :
le {\it Traité de la nature humaine}.

Revenu à Londres en 1737, il a la chance de trouver
un éditeur, et la prudence (ou si l’on veut la faiblesse)
de supprimer les chapitres sur la religion
(il espère la protection de l’évèque Butler). Les deux
premiers livres du traité ainsi publiés « tombérent
mort-nés de la presse », raconta plus tard Hume dans
sa courte {\it Autobiographie}. Ce n’est pas tout à fait
vrai. En fait l’{\oe}uvre intéressa quelques critiques,
mais n’atteignit pas le grand public (qui seul donne
la notoriété). A cette époque, Hume entre en relations
avec Hutcheson, professeur à Glasgow, qui
lui présente son jeune étudiant Adam Smith (qui
restera toujours l’ami de Hume) et lui trouve un
éditeur pour les deux livres suivants du {\it Traité de
la nature humaine} dont le succès n’est pas plus
grand. Hume cependant ne doute pas de sa valeur.
Son échec vient de la présentation trop lourde et
trop savante de sa philosophie et non pas du fond
({\it more from the manner than the matter} dira l'{\it Autobiographie}).
Hume décide alors d’écrire des essais
courts et brillants et publie en 1741 a Édimbourg
ses {\it Essais moraux et politiques} (humilié par ses
insuccés il présente d’ailleurs cet ouvrage comme son
premier livre !). Cette fois les lecteurs sont nombreux,
et Hume croit pouvoir présenter sa candidature à
la chaire de philosophie morale de l’Université de
Glasgow. L’opposition des chrétiens empêche sa
nomination. En 1746, Hume devient le secrétaire
particulier du général Saint-Clair, un Écossais qui
est son parent éloigné, et l’accompagne dans une
%8
%{\it }{\oe}
mission diplomatique à Vienne et à Turin. Pendant
son voyage paraissent ses {\it Essais philosophiques sur
l'entendement humain} (plus tard {\it Enquête sur l’entendement
humain}), qui reprennent dans un style nouveau
les deux premiers livres du {\it Traité de la nature
humaine}. Cette fois les chapitres sur le miracle et
sur la providence particulière paraissent avec le
reste (1748). A son retour c’est le troisième livre du
Traité que Hume reprend avec l’{\it Enquête sur les
principes de la morale} (1751). Dès lors la notoriété
de Hume s’affirme. Il entre en relations épistolaires
avec Montesquieu à propos de l’{\it Esprit des lois}.
Et s'il échoue une nouvelle fois (1751) dans sa candidature
à l’Université de Glasgow, il devient conservateur
de la bibliothéque de la « Faculté des Avocats »
à Edimbourg. Il trouve là tous les documents
nécessaires pour écrire de 1754 à 1761 sa monumentale
{\it Histoire d’ Angleterre}, de Jules César à Jacques II.
Le premier volume qui traite des régnes de Jacques
I$^\text{er}$ et de Charles I$^\text{er}$ déclenche un petit scandale
dans les milieux religieux, et Hume n’améliore pas
son cas auprès des chrétiens en publiant ses {\it Quatre
dissertations} (dont son {\it Essai sur l'histoire naturelle
de la religion} et son {\it Essai sur le suicide}). Il est vrai
qu'il s’empresse de retirer l’{\it Essai sur le suicide}
ainsi qu’un {\it Essai sur l’immortalité de l’âme} pour les
remplacer par un {\it Essai sur la règle du goût}. Hume
subit aussi avec patience les tracasseries des avocats
d’Édimbourg qui lui reprochent d’avoir acheté
pour la bibliothéque les {\it Contes} de La Fontaine, et
l'{\it Histoire amoureuse des Gaules} de Bussy-Rabutin !

C’est en France que Hume devait connaître la
gloire. A l’appel de lord Hertford, ambassadeur
%9
%{\it }{\oe}
d’Angleterre à Paris, il va exercer de 1763 à 1766
les fonctions de secrétaire d’ambassade. Il sera
même quelque temps — lorsque lord Hertford
est rappelé et en attendant son successeur — « chargé
d’affaires », c’est-à-dire en fait ambassadeur, grâce
à la puissante protection de la comtesse de Boufflers.

A Edimbourg Hume inquiète, à Londres il n’est
qu’un Écossais, qu’un intellectuel provincial. A
Paris, le petit monde des philosophes qui a lu ses
Essais, qui connaît l’opposition de Hume à la
« superstition » et au « fanatisme » le tient pour un
philosophe de premier plan. Il deviendra très vite
l'ami de d’Alembert, de Diderot, d’Helvétius, du
baron d’Holbach. Certes, officiellement, Hume est
déiste, à un diner du baron d’Holbach il confesse
même n’avoir jamais rencontré d’athée (« Regardez
autour de vous, répond le baron : Il y en a quinze
autour de cette table ! »). En fait ses positions anti-religieuses,
son Essai sur les miracles le rendent
sympathique aux encyclopédistes qui le tiendront
désormais pour un « frére ». A Paris, c’est un véritable
triomphe. Les plus grandes dames s’arrachent ce
quinquagénaire bedonnant. C’est la duchesse de
La Vallière qui tient à le voir dès son arrivée à Paris,
avant même qu’il ait pu changer de costume !
C’est Mme Du Deffand, Mme Geoffrin, Mlle de Lespinasse
qui le fêtent dans leurs « salons ». même
l'apparence physique de Hume, assez ingrate (il
est obèse, son visage empâté est peu expressif)
qui lui avait valu naguére, tandis qu’il voyageait
en Italie avec le général Saint-Clair, les railleries
du jeune James Caulfield, futur lord Charlemont,
et quelques déboires amoureux, est maintenant
%10
%{\it }{\oe}
trouvée sympathique. Son visage un peu lourd, son
fort accent écossais donnent au grand philosophe
un air débonnaire, une simplicité de bon aloi
(Mme Du Deffand l'appelle « mon cher paysan »,
Mme Geoffrin « mon gros drôle, mon gros coquin »).
Hume vole de succès en succès, écrit à son ami le
D$^\text{r}$ Robertson : « Je ne mange que de l’ambroisie,
je ne bois que du nectar, je ne respire que de l’encens,
je ne marche que sur des fleurs. »

Marie-Charlotte Hippolyte de Campet de Saujeon,
comtesse de Boufflers et maîtresse en titre du prince
de Conti, avait dès 1761 écrit à Hume pour lui dire
toute son admiration pour sa philosophie « sublime ».
Elle avait tenté de le rencontrer au cours d’un voyage
à Londres, sans succés car Hume n’avait pas voulu
quitter Édimbourg. A Paris elle l'invite tout de
suite à ses lundis, puis à ses vendredis plus intimes.
Il semble que cette jolie femme de trente-cinq ans
ait été quelque peu amoureuse du gros philosophe
vieillissant. Hume la considéra toujours comme une
amie très chère (le philosophe, cinq jours avant sa
mort, lui écrit encore le 20 août 1776 pour lui
annoncer qu’il se sent perdu et lui dire une dernière
fois son « affection » et son « respect »), mais il ne
paraît pas que leurs relations aient jamais été plus
intimes. Nous connaissons mal la vie intime de Hume,
mais il semble que le philosophe, resté célibataire
(à Edimbourg, sa s{\oe}ur Katherine tenait son ménage)
se soit, à cause d’anciens déboires, ou par souci de
préserver son indépendance et son travail, toujours
méfié des passions amoureuses.

En 1766, le nouvel ambassadeur d’Angleterre, le
duc de Richmond, arrive à Paris et Hume repart
% 11
%{\it }{\oe}
en Angleterre. C’est ici que se situe un des épisodes
les plus mal éclaircis de la vie de Hume. Dès ses
lettres de 1761 la comtesse de Boufflers avait intéressé
Hume à Jean-Jacques Rousseau. Pendant le
séjour à Paris, la marquise de Verdelin demande au
philosophe écossais de chercher pour Jean-Jacques
un refuge en Angleterre. Celui-ci, proscrit de Genéve,
sa ville natale, interdit de séjour en France, persécuté
par les habitants de Môtiers-Travers, en butte à la
haine des encyclopédistes qui le tiennent pour un
dévot, est dans une des périodes les plus critiques de
son existence. Hume, ému par les malheurs de
Jean-Jacques, et tout d’abord enthousiasmé par sa
simplicité et sa franchise (Rousseau est un Socrate
moderne, dira-t-il), part avec lui le 4 janvier 1766.
Ils arrivent à Londres le 13 ou Jean-Jacques est
fété et reçoit de Hume mille témoignages d’amitié.
Mais Rousseau qui déteste le monde et cherche la
solitude n’entend rester à Londres que jusqu’à
l'arrivée de sa servante-maîtresse Thérèse Le Vasseur.
En fait c’est seulement le 19 mars 1766 qu’il part
pour Wooton, maison de campagne dans les bois du
Derbyshire qu’un ami de Hume, Davenport, avait
mise à sa disposition.

Moins de trois mois ont suffi pour altérer la belle
amitié de Hume et de Rousseau. Désormais Rousseau
considére Hume comme un traître et comme un
malhonnête homme. Que s’est-il passé ? Il est certain
que Rousseau a toujours eu un caractère soupçonneux,
une tendance paranoiaque au délire d’interprétation
que les réelles persécutions dont il fut
victime n’ont fait qu’aggraver. Avant même d’atteindre Calais,
le premier soir du voyage, un curieux
% 12
%{\it }{\oe}
incident se produit. Hume, Rousseau et Luze, un
ami suisse qui va à Londres pour ses affaires, descendent
dans un hotel de Senlis et couchent dans une
chambre à trois lits. Jean-Jacques qui cherche en
vain le sommeil entend soudain Hume dire à plusieurs
reprises « à pleine voix » et avec « une véhémence extrême » :
« Je tiens Jean-Jacques Rousseau. »
Hume n’est-il pas l’ami intime des ennemis les plus
acharnés de Rousseau, les encyclopédistes athées,
Diderot, d’Alembert et d’Holbach ? Ne s’est-il
pas mis d’accord avec cette clique pour faire de
Rousseau en quelque sorte son prisonnier ? Dès le
départ ce rêve ou cette hallucination (car Luze
n’a pas été éveillé par ces voix véhémentes) met
Rousseau sur ses gardes.

Cela suffit-il pour que nous n’accordions aucun
crédit aux accusations formulées ultérieurement par
Jean-Jacques ? Certaines assurément sont délirantes.
A Londres, Rousseau est exaspéré par les
compliments de Hume qui a toujours à son chevet
un tome de la {\it Nouvelle Héloïse}. Pure hypocrisie,
juge Rousseau, car Hume ne peut aimer ce roman !
En réalité Hume, qui fait effectivement des réserves
sur les idées de Jean-Jacques, apprécie son roman,
le tient pour le chef-d’{\oe}uvre du philosophe français
(lettre à Blair du 25 mars 1766). Mais il y a plus
grave. Hume n’ignore pas qu’à Paris son ami Walpole
a fait à Rousseau une plaisanterie trés méchante
(écrivant et faisant publier une lettre d’invitation a
Rousseau, au nom du roi de Prusse qui lui promet à
son choix faveurs ou persécutions, puisqu’il paraît
aimer les persécutions !). Or tandis que Rousseau se
préoccupe de faire venir à Londres des papiers pour
% 13
%{\it }{\oe}
la rédaction de ses {\it Confessions}, papiers restés en
France, Hume lui propose Walpole comme commissionnaire !
Hume fait faire une enquête à la banque
Rougemont sur les vraies ressources de Jean-Jacques
qui crie toujours misère. D’autre part Hume veut
toujours se charger d’expédier et de retirer de la
poste le courrier de Jean-Jacques, et celui-ci se
plaindra que son courrier lui parvienne toujours
décacheté et grossièrement refermé. Hume répondra
plus tard que s’il tenait à se charger du courrier de
Rousseau, c’est parce que ce dernier, alléguant sa
pauvreté, ne voulait payer le port d’aucune lettre
(a l’époque, c’est le destinataire qui payait) et que
lui, David Hume, ne tenait pas a laisser trop longtemps
le courrier à la poste, mais désirait le soustraire
promptement {\it from the curiosily and indiscretion
of the clerks of Post-office} !

Quoi qu’il en soit de toutes ces accusations, il
reste que Hume porte la responsabilité d’avoir mis
la querelle sous les yeux du public en laissant éditer
à ses amis français l’{\it Exposé succinct} de son différend
avec Rousseau, livrant ainsi Jean-Jacques à de
nouvelles railleries. Ce fut peut-être en cette affaire
sa seule faute, une entreprise, écrira-t-il à Adam
Smith le 17 octobre 1767, que « l’un et l’autre nous
avons été enclins a blâmer parfois, à regretter
toujours ». Si Hume ne fut pas le traître qu’a imaginé
Rousseau, peut-être manqua-t-il ici de patience et
de générosité. Une des faiblesses de Hume est d’avoir
été toujours trés préoccupé de sa réputation. Et
il craignait que Rousseau, dans ses {\it Confessions}, ne
raconte l'histoire à sa manière (en réalité le récit de
Rousseau s’arrétera avant les événements de 1766).
% 14
%{\it }{\oe}

Hume, grace à son fidèle protecteur, lord Hertford,
devient sous-secrétaire d’État en 1767. Par un
étrange retour des choses c’est lui qui, entre autres
affaires, est chargé de régler les conflits et de décider
de l’avancement des pasteurs de cette Église
d’Écosse qui naguére avait tenté d’entraver sa
carriére !

En 1769 Hume rentre à Édimbourg désormais
riche et considéré. Il choisit de finir ses jours dans
sa ville natale, intense foyer de culture, qui est bien
plutôt que Londres la capitale intellectuelle de
l’Angleterre de ce temps. A Édimbourg, cette
« Athènes du Nord », Hume retrouve en effet des
amis éminents, Adam Smith, le juriste lord Kames,
Ferguson, et aussi des adversaires loyaux et courtois
comme le théologien George Campbell qui avait
rédigé une critique de l’{\it Essai sur les miracles}, critique
fort appréciée par Hume lui-même. Hume
s’emploie à corriger ses {\oe}uvres pour de prochaines
éditions, à répondre à son courrier, et il rejoint
fréquemment ses amis écossais au {\it Poker Club} et a
la {\it Select Society}, club philosophique et littéraire
qu’il avait fondé lui-même en 1754.

Très rapidement sa santé décline. Il souffre d’une
tumeur de l'intestin, et dès le début de 1776 il se
sait perdu. En avril il rédige son testament. Il a
dans ses papiers un ouvrage inédit commencé
dès 1751 et dont il a déja soumis à cette époque les
premiers chapitres à son ami Gilbert Elliot of Minto :
{\it Les dialogues sur la religion naturelle}. Adam Smith
n'est pas très favorable à la publication de cet
ouvrage. C’est donc le neveu de Hume qui sera
chargé de cette édition posthume. {\it Les Dialogues}
% 15
%{\it }{\oe}
paraîtront en 1779, plus de deux ans après la mort
de Hume.

Hume est mort avec la plus grande sérénité.
Dans sa brève autobiographie, rédigée le 18 avril 1776
il déclare : « Il est difficile d’être plus détaché de la
vie que je ne le suis à présent. » Le 13 août, il dit qu'il
se console d’abandonner des amis, car « {\it hélas, on
ne laisse que des mourants}, comme Ninon de Lenclos
le dit sur son lit de mort. La mort m'apparaît si
peu terrible maintenant qu’elle approche, que je
dédaigne de citer des héros et des philosophes comme
exemples de courage. Le témoignage d’une femme de
plaisir qui néanmoins était également philosophe
est suffisant ». Hume s’éteignit sans angoisse dans
l'après-midi du dimanche 25 août 1776. C’était la
veille de son soixante-cinquième anniversaire.
%%%%%%%%%%%%%%%%%%%%%%%%%%%%%%%%%%%%%%%%%%%%%%%%%%%%%%%%%%%%%%%%%%%%%%%%

%
\newpage
%

%%%%%%%%%%%%%%%%%%%%%
\chapter{L'{\oe}uvre}
%%%%%%%%%%%%%%%%%%%%% \textsc{}

%%%%%%%%%%%%%%%%%%%%%%%%%
\section{Ouvrages publiés du vivant de Hume}
%%%%%%%%%%%%%%%%%%%%%%%%%
% 
%{\it }
{\it A treatise of Human Nature} (1739-1740, 3 vol.).

{\it Essays moral and political} (3 vol., 1741-1742, 1748).

{\it Philosophical essays concerning Human Understanding}
(1748 ; à partir de 1758 le mot {\it Inquiry} remplace {\it Philosophical essays}).

{\it An Inquiry concerning the principles of Morals} (1751).

{\it Political discourses} (1752).

{\it The History of Great Britain} (1754-1757).

{\it Four Dissertations : 1. The natural history of Religion ;
II. Of the passions ; III. Of tragedy ; IV. Of the standard
of Taste (1757).} En 1755, Hume avait remis à l’éditeur
les trois premiéres de ces dissertations et une quatriéme
consacrée aux mathématiques et à la physique. Hume
la retire sur le conseil d’un ami mathématicien. Il la
remplace par deux dissertations : {\it On suicide} et {\it On the
immortality of soul.} Tandis que ces deux essais sont
déja en vente, Hume les retire et les remplace par une
seule dissertation : {\it Of the Standard of Taste. }

{\it The history of England} (1759 a 1767).

{\it Exposé succinct de la contestation qui s’est élevée entre
M. Hume et M. Rousseau} (1766).

%%%%%%%%%%%%%%%%%%%%%%%%%
\section{Ouvrages posthume}
%%%%%%%%%%%%%%%%%%%%%%%%%

{\it The life of David Hume written by himself} (1777).
{\it Two essays (On Suicide et The Immortality of the Soul)}
(1777).

%50 HUME
%{\it }
 

{\it Dialogues concerning Natural Religion} (1779).

J. H. \textsc{Burton}, {\it Life and Correspondance of David Hume,}
Edinburgh, 1846, 2 vol.

J. Y. T. \textsc{Greig}, {\it The letters of David Hume}, Oxford, 2 vol.,
1932.

R. \textsc{Klibansky} et E. C. \textsc{Mossner}, {\it }New letters of David
Hume, Oxford, 1954.

Rappelons qu’une édition anglaise classique comprend
toute l’{\oe}uvre philosophique de Hume : {\it The philosophical
works of David Hume}, éd. T. H. Green and T. H. Grose,
London, 1874-1875, 4 vol.

%%%%%%%%%%%%%%%%%%%%%%%%%
\section{Traductions françaises}
%%%%%%%%%%%%%%%%%%%%%%%%%{\it }

{\it {\oe}vres philosophiques choisies (Enquéte sur l'entendement,
Traité de la nature humaine, Dialogues de la religion
naturelle)}  traduites par Maxime \textsc{David} avec préface
de \textsc{Lévy-Bruhl}, Paris, Alcan, 1912. La traduction
Maxime \textsc{David} des {\it Dialogues sur la religion naturelle}
a été rééditée en 1964 chez J.-J. Pauvert dans la collection 
« Libertés » avec une présentation et des notes
de Clément \textsc{Rosset}.

{\it Traité de la nature humaine}, préfacé et traduit par André
\textsc{Leroy}, Editions Aubier (1$^{\text re}$ éd., 1946), 2 vol.

{\it Enquête sur l'entendement humain} (trad. \textsc{Leroy}, Aubier,
1947).

{\it Enquête sur les principes de la morale. Les quatre philosophes} 
(trad. \textsc{Leroy}, Aubier, 1947).

Il existe une traduction frangaise de 1788 des quatre
dissertations {\it (L’histoire naturelle de la. religion, Les passions, 
La tragédie, La règle du goût)}. Une nouvelle traduction 
des {\oe}uvres de Hume est en cours d’édition.

 

 
%%%%%%%%%%%%%%%%%%%%%%%%%%%%%%%%%%%%%%%%%%%%%%%%%%%%%%%%%%%%%%%%%%%%%%%%

%
\newpage
%

%%%%%%%%%%%%%%%%%%%%%
\chapter{Bibliographie}
%%%%%%%%%%%%%%%%%%%%%

%%%%%%%%%%%%%%%%%%%%%%%%%
\section{En anglais}
%%%%%%%%%%%%%%%%%%%%%%%%%{\it }

Hendel, {\it Studies on the philosophy of D. Hume}, Princeton,
1925.

A. E. Taylor, {\it David Hume and the miraculous}, Gambridge, 1927.

J. Laird, {\it Hume’s Philosophy of Human Nature}, London,
1932.

{\it Hume and present day problems}, Aristotelian Society,
suppl., vol. XVIII, London, 1939 (4 symposia sur
l'identité du moi, sur les concepts {\it a priori}, sur l’éthique,
sur la religion naturelle avec des articles de Taylor,
de Lairp, de Jessop).

Norman Kemp Smith, {\it Philosophy of David Hume}, London,
1941.

D. G. C. Mac Nabb, {\it David Hume, His theory of Knowledge
and Morality}, London, 1954.

E. G. Mossner, {\it The life of David Hume}, London, 1954.

%%%%%%%%%%%%%%%%%%%%%%%%%
\section{En français}
%%%%%%%%%%%%%%%%%%%%%%%%%

G. Compayré, {\it }La philosophie de D. Hume, Paris, 1873.

G. Lechartier, {\it David Hume sociologue et moraliste},
Paris, 1900.

L. Levy-Bruhl, Orientation de la pensée de D. Hume,
{\it Revue de métaphysique et de morale}, 1909.

A. Leroy, {\it Critique et religion chez D. Hume}, Paris, 1931.

%92
Laporte, Le septicisme du Hume, {\it Revue philosophique,}
1933-1934.

G. Brercer, Husserl et Hume, {\it Revue internationale de
philosophie}, 1939.

{\it Mélange David Hume}, divers articles, {\it Revue internationale
de philosophie}, Bruxelles, 1952.

DELEUZE, Empirisme et subjectivité, Paris, Presses universitaires 
de France, 1953.

A. Leroy, {\it David Hume}, Paris, Presses Universitaires de
France, 1953.

O. Brunet, Philosophie et Esthétique chez D. Hume,
Paris, Librairie A.-G. Nizet, 1965.

%%%%%%%%%%%%%%%%%%%%%%%%%%%%%%%%%%%%%%%%%%%%%%%%%%%%%%%%%%%%%%%%%%%%%%%%

%
%
%%%%%%%%%%%%%%%%%%%%%
\section{Encyclopédie de la philosophie}
%%%%%%%%%%%%%%%%%%%%%
%
\subsection{Nécessité}
Est dit nécessaire « ce qui ne peut être autrement qu'il n'est »,
telle est du moins la définition aristotélicienne de la notion (grec {\it anankaion}).
Elle s'applique
notamment pour caractériser la nature de
la relation qui relie entre elles les propositions (prémisses et protases) du syllogisme scientifique ou démonstratif. Est
nécessaire ce qui appartient à un sujet
partout et toujours : ainsi la propriété
d’avoir la somme de ses angles intérieurs
égale à deux droits est une propriété
nécessaire pour tout triangle en tant que
tel. Mais, outre le sens logique et ontologique, il y a un sens psychologique, de
contrainte inévitable. Ce qui a conduit certains à émettre l'hypothèse que le concept
est peut-être né d’une projection anthropomorphique de l’idée originelle de coercition,
ce qui aurait pour conséquence que la
notion de nécessité déontologique (c’est-à-dire d’obligation) précéderait historiquement toutes les autres formes de nécessité.
La notion de nécessité physique et causale comme subordination aux lois de la
nature et celle de nécessité logique
comme propriété de ce qui est « forcément » vrai en vertu des lois logiques
seraient apparues par la suite, selon un
ordre inversé par rapport à celui qui est
parfois considéré logiquement rationnel.

\subsection{Nécessité et modalité}

La notion leibnizienne de nécessité
comme vérité dans tous les mondes possibles à été reprise dans l’analyse sémantique contemporaine des modalités.
D'après ces analyses, il apparaît que la
notion modale fondamentale est la notion
de possibilité plutôt que celle de nécessité. À côté de cette notion leibnizienne
de nécessité absolue, la logique. contemporaine a caractérisé différentes notions
de nécessité relative, en-isolant des sous-classes pertinentes à l’intérieur de la
classe de tous les mondes possibles,
comme celle des mondes où sont valides
les lois de la nature ou les lois de code
pénal. Toute logique de la nécessité,
quelle qu’elle soit, a comme condition
minimum d'éviter ce que l’on appelle
« l’effondrement des modalités », à savoir
la démonstration de l’équivalence entre
une proposition nécessaire et une proposition dépourvue d’opérateurs modaux
(cette équivalence est présente dans toute
philosophie fataliste ou strictement déterministe : il suffit de penser à l’assimilation
opérée par Hegel entre ce qui est rationnel et ce qui est réel). Dans une acception
élargie, la nécessité logique est parfois
assimilée à l’analycité. Les énoncés analytiques résultent nécessaires dans la
mesure où les règles linguistiques
« créent » la signification des termes que
l’on y trouve, que ces termes soient des
constantes logiques (« et », « ou »,
« tous », etc.) ou des mots du langage
ordinaire comme « célibataire » ou « non-
marié ». Du point de vue pragmatiste
d'auteurs comme W.V.O. Quine, la nécessité d’un énoncé quelconque consiste dans
son immunité à l’intérieur du système des
connaissances prouvées, c’est-à-dire dans
le fait que le renoncement à l’énoncé en
question est trop coûteux pour le système
dans son ensemble. Tout en excluant qu’il
existe une distinction historiquement définitive entre analytique et synthétique et
entre nécessaire et contingent, Quine.
pense toutefois que la nécessité logique
peut être justement attribuée. à des
énoncés dont la vérité dépend uniquement des constantes logiques qui sont
présentes en lui. Il s’agit d’un pas en avant
par rapport à la conception du Tractatus
de L. Wittgenstein, qui assimilait les
nécessités logiques aux tautologies du calcul propositionnel, excluant de cette
manière les vérités logiques qui dépendent de la présence de quantificateurs. De
toute façon, cette conviction est partagée
par presque toute  l'épistémologie
contemporaine, surtout par l’épistémologie empiriste : que la nécessité des lois
scientifiques et des inférences garanties
par elles ne dépend pas de l’existence de
connexions nécessaires dans la nature,
mais doit être indirectement référée à la
nécessité logique ou au concept d’implication logiquement nécessaire.

%\vspace{0.35cm}
%$\to$ existence ; mondes possibles ; possibilité ; quantificateurs ; Quine

%%%%%%%%%%%%%%%%%%%%%%%%%%%%%%%%%%%%%%%%%%%%%%%%%%%%%%%%%%%%%%%%%%%%%%%%

%%%%%%%%%%%%%%%%%%%%%%%%%%%%%%%%%%%%%
\section{Pratique de la philosophie}
%%%%%%%%%%%%%%%%%%%%%%%%%%%%%%%%%%%%

\subsection{Opinion}

{\footnotesize
\begin{itemize}[leftmargin=1cm, label=\ding{32}, itemsep=1pt]
\item {\bf \textsc{Étymologie} :} latin {\it opinari},
« émettre une opinion ».
\item {\bf \textsc{Sens ordinaire} :} avis,
jugement porté sur
un sujet, qui ne relève pas d'une
connaissance rationnelle vérifiable,
et dépend donc du système de
valeurs en fonction duquel on se
prononce.
\item {\bf \textsc{Philosophie} :} jugement
sans fondement rigoureux,
souvent dénoncé dans la mesure où
il se donne de façon abusive les
apparences d’un savoir.
\end{itemize}
}

L'interrogation sur la nature de la vérité
et les moyens de l’atteindre a conduit
nombre de philosophes à distinguer,
entre les différents types de connaissance
possibles, ceux qui conduisent effectivement
à la vérité, et ceux qui en éloignent.
En un premier sens, l’opinion est ainsi
traditionnellement considérée comme un
genre de connaissance peu fiable, fondée
sur des impressions, des sentiments, des
croyances où des jugements de valeur
subjectifs. Pour Spinoza, par exemple,
elle est forcément « sujette à l'erreur et n’a
jamais lieu à l'égard de quelque chose
dont nous sommes certains mais à l'égard
de ce que l’on dit conjecturer ou supposer »
({\it Court traité}, chap. II). Depuis Platon,
et jusque chez de nombreux penseurs
contemporains, l'opinion est
dénoncée comme a priori douteuse, illusoire
ou fausse, voire dangereuse, lorsqu’elle
cherche à s'imposer en dissimulant
la faiblesse de ses fondements sous
les apparences de la plus claire certitude.
Selon Adorno ({\it Modèles critiques}, 1963),
« l'opinion s’approprie ce que la connaissance
ne peut atteindre pour s’y substituer »,
elle rassure à bon compte, parce
qu’« elle offre des explications grâce auxquelles
on peut organiser sans contradiction
la réalité contradictoire ». Tel est bien
le « fonctionnement psychique » qui soustend,
par exemple, les opinions racistes :
pour être plus crédible, la peur de l’autre
prend le masque de l'affirmation de son
infériorité ou de la mise en garde contre
le danger qu'il est censé représenter. La
justesse de ces analyses ne doit pas faire
oublier qu'en un autre sens, l'opinion
constitue une forme de connaissance
utile, voire un type de jugements éminemment
respectables. Dans le {\it Ménon},
Platon reconnaît aux opinions droites la
faculté, sur les sujets qui ne relèvent ni de
la science ni de la simple conjecture,
d'éclairer l’action humaine. Dans le
domaine moral par exemple, à défaut de
vérités certaines, des intuitions justes
relatives au bien peuvent guider efficacement
l'éducation ou l’action, en leur
fixant pour but la satisfaction d'intérêts
conformes aux exigences de la réflexion,
et non à la soumission aux apparences ou
au plaisir immédiat. Enfin, sur toutes les
questions qui engagent des choix individuels
qu'aucune autorité ne peut légitimement
contraindre {\bf --} la religion, la
préférence politique, l'adhésion à une
conception du monde {\bf --} la liberté d’opinion
est un droit fondamental, dans les
sociétés démocratiques en tout cas, dès
l'instant où ceux auxquels elle est garantie
n'en usent pas au détriment de la
liberté d'autrui.

Analysée dans le {\it Traité
théologico-politique}, où Spinoza insiste
sur la nécessité d'une indépendance
absolue des opinions religieuses et de
leur expression par rapport à l'État, la
liberté d'opinion est proclamée dans la
Déclaration des droits de l'homme et du
citoyen de 1789. Et depuis près d'un
siècle, elle est au cœur du principe de la
laïcité qui garantit (en particulier en
France) la séparation entre l'Église et
l'État.

{\footnotesize
\begin{itemize}[leftmargin=1cm, label=\ding{32}, itemsep=1pt]
\item {\bf \textsc{Termes voisins} :} avis ; croyance.
\item {\bf \textsc{Termes opposés} :} science.
\end{itemize}
}

\subsubsection{Opinion publique}

Ensemble fluctuant de prises de positions
portant sur des questions politiques,
 morales, économiques... Les
« sondages d'opinion » prétendent en
constituer une sorte de baromètre.

{\footnotesize
\begin{itemize}[leftmargin=1cm, label=\ding{32}, itemsep=1pt]
\item {\bf \textsc{Corrélats} :} connaissance ;
conviction ; croyance ; doute ; foi ;
jugement ; préjugé.
\end{itemize}
}

%%%%%%%%%%%%%%%%%%%%%%%%%%%%%%%%%%%%%%%%%%%%%%%%%%%%
\subsection{Préjugé}

{\footnotesize
\begin{itemize}[leftmargin=1cm, label=\ding{32}, itemsep=1pt]
\item {\bf \textsc{Étymologie} :} latin {\it praejudicare},
« juger préalablement ».
\item {\bf \textsc{Sens ordinaire} :} Opinion admise sans
jugement ni raisonnement.
\end{itemize}
}

Le terme préjugé est souvent employé
dans un sens péjoratif, pour dénoncer
l'erreur ou au moins l'absence de
réflexion qui conduit un individu à
adhérer à une idée fausse {\bf --} dont il n’a
pas pris la peine de contrôler le bien-fondé {\bf --}
voire à la défendre contre des
idées justes, ou à condamner des individus
au nom de cette idée (par
exemple, les opinions racistes sont des
préjugés).

{\footnotesize
\begin{itemize}[leftmargin=1cm, label=\ding{32}, itemsep=1pt]
\item {\bf \textsc{Termes voisins} :} opinion.
\item {\bf \textsc{Termes opposés} :} savoir ; science.
\item {\bf \textsc{Corrélats} :} certitude ; croyance ;
dogme ; doute ; foi.
\end{itemize}
}

%%%%%%%%%%%%%%%%%%%%%%%%%%%%%%%%%%%%%%%%%%%%%%%%%%%%

\subsection{Erreur}

{\footnotesize
\begin{itemize}[leftmargin=1cm, label=\ding{32}, itemsep=1pt]
\item {\bf \textsc{Étymologie} :} latin {\it error}, « course
à l'aventure », de {\it errare}, «errer».
\item {\bf \textsc{Logique et sciences} :} affirmation
fausse, c'est-à-dire non conforme
aux règles de la logique, et/ou en
contradiction avec les données
expérimentales.
\item {\bf \textsc{Psychologie} :} état
de l'esprit qui tient pour vrai ce
qui est faux, et réciproquement
(ex. : « être dans l’erreur »).
\end{itemize}
}

L'erreur doit être soigneusement distinguée
aussi bien de la faute (qui engage
plus nettement notre responsabilité) que
de l’illusion (qui n’est pas vaincue par
le savoir). L'erreur procède toujours de
notre jugement : elle résulte, selon Descartes,
d’un décalage permanent entre
notre volonté, qui est infinie, et notre
entendement, qui ne l'est pas. Nous
nous trompons parce que nous outrepassons
nos possibilités intellectuelles,
par étourderie ou vanité : l'erreur n’est
donc qu'une privation de connaissance.
L'épistémologie contemporaine, au
contraire, donne à l’erreur un tout autre
statut, plus « positif ». Bachelard, notamment,
montre que les « vérités » scientifiques
ne sont jamais que provisoires,
qu'elles doivent constamment être remaniées
et corrigées. La connaissance
scientifique ne peut pas faire l'économie
de l’erreur.

{\footnotesize
\begin{itemize}[leftmargin=1cm, label=\ding{32}, itemsep=1pt]
\item {\bf \textsc{Termes voisins} :} fausseté ; illusion ;
incorrection.
\item {\bf \textsc{Termes opposés} :} vérité.
\item {\bf \textsc{Corrélats} :} connaissance ; Évidence ;
faute ; illusion ; jugement.
\end{itemize}
}

%%%%%%%%%%%%%%%%%%%%%%%%%%%%%%%%%%%%%%%%%%%%%%%%%%%%

%
%\newpage
%
\end{appendix}
%

%
%
%====================== INCLUSION DE LA BIBLIOGRAPHIE ======================
%
%récupérer les citation avec "/footnotemark" : 
\nocite{*}
%
% choix du style de la biblio
\bibliographystyle{plain}
%
% inclusion de la biblio
\cleardoublepage
\addcontentsline{toc}{chapter}{Bibliographie}
\bibliography{bibliographie.bib}
%
\end{document}
%%%%%%%%%%%%%%%%%%%%%%%%%%%%%%%%%%%%%%%%%%%%%%%%%%%%%%%%%%%%%%%%%%%%%%%%%%%%%%%%%
