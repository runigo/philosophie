\chapter{Abréviation du Cuvillier}
\section{Signes et abréviations}
\subsection {En tête des articles}
L’étymologie est indiquée entre [ ]:

\hfill
\begin{minipage}[c]{.45\linewidth}
[G. signifie : Du grec.

[L. signifie : Du latin.
\end{minipage}
\hfill
\begin{minipage}[c]{.45\linewidth}
[All signifie : De l'allemand.

[Angl. signifie : De l’anglais.
\end{minipage}

\vspace{0.311cm}

\subsection {Dans le corps des articles}
1° les abréviations suivantes indiquent les
disciplines au langage desquelles le mot est emprunté :

\vspace{0.211cm}
\hfill
\begin{minipage}[c]{.45\linewidth}
\textsf{\textit {Biol.}} — Biologie.

\textsf{\textit {Car.}} — Caractérologie, psychologie
des caractères.

\textsf{\textit {Crit.}} — Critique ou théorie de la
connaissance.

\textsf{\textit {Éc. pol.}} — Économie politique.

\textsf{\textit {Éc. soc.}} — Économie sociale.

\textsf{\textit {Épist.}} — Épistémologie.

\textsf{\textit {Esth.}} — Esthétique,

\textsf{\textit {Hist.}} — Histoire de la philosophie.

\textsf{\textit {Jur.}} — Droit.

\textsf{\textit {Ling.}} — Linguistique.

\textsf{\textit {Log.}} — Logique.

\textsf{\textit {Log. form.}} — Logique formelle.

\textsf{\textit {Math.}} — Mathématiques.

\textsf{\textit {Méd.}} — Médecine.

\textsf{\textit {Mor.}} — Morale.
\end{minipage}
\hfill
\begin{minipage}[c]{.45\linewidth}
\textsf{\textit {Méta.}} — Métaphysique, philosophie
générale.

\textsf{\textit {Péd.}} — Pédagogie.

\textsf{\textit {Phol.}} — Physiologie.

\textsf{\textit {Phys.}} — Sciences physiques.

\textsf{\textit {Pol.}}. — Politique.

\textsf{\textit {Psycho.}} — Psychologie.

\textsf{\textit {Ps. an.}} — Psychanalyse.

\textsf{\textit {Ps. métr.}} — Psychométrie.

\textsf{\textit {Ps. path.}} — Psychologie pathologique.

\textsf{\textit {Ps. phol.}} — Psycho-physiologie.

\textsf{\textit {Ps. phys.}} — Psychophysique.

\textsf{\textit {Soc.}} — Sociologie.

\textsf{\textit {Techn.}} — Technique.

\textsf{\textit {Théol.}} — Théologie,

\textsf{\textit {Vulg.}} — Sens vulgaire, courant.
\end{minipage}

\vspace{0.211cm}
2° Les chiffres en caractères gras ({\bf 1, 2}) distinguent les différentes acceptions
du mot ;

3° Le signe * indique les mots définis à leur ordre alphabétique et auxquels
il y a lieu de se reporter pour plus complète explication; lorsque ces mots présentent plusieurs acceptions, l'étoile est remplacée par un chiffre mis en exposant (ex. : {\it absolu}$^2$) qui détermine le sens qu'il convient de choisir ;

4° Les termes contraires (Ctr.), opposés (Opp.) ou synonymes (Syn.) sont
indiqués eutre parenthèses ;

5° Le signe $->$
signale les impropriétés, confusions, incorrections, le plus
souvent commises et contre lesquelles on doit se tenir en garde;

6° Les abréviations suivantes indiquent certaines nuances de sens :

\hfill
\begin{minipage}[c]{.45\linewidth}
\fsb{S. abstr.} — Sens abstrait

\fsb{S. subje.} — Sens subjectif

\fsb{S. norma.} — Doctrine, théorie, ou : sens nor-   
matif. — Signifie, équivalent à.
\end{minipage}
\hfill
\begin{minipage}[c]{.45\linewidth}
\fsb{S. concr.} — Sens concret

\fsb{S. objec.} — Sens objectif

\fsb{S. posit.} — État de fait, ou : sens positif.
\end{minipage}

\vspace{0.211cm}

7° Les références aux textes sont données à l’aide des abréviations suivantes:

Bergson, {\it D. I.}, Données immédiates de la conscience.

\ \ \ — {\it 2 Sources}, Les deux Sources de la morale et de la religion,

\ \ \ — {\it E. C.}, L'Évolution créatrice.

\ \ \ — {\it E. S.}, L'Énergie spirituelle.

\ \ \ — {\it P. M.}, La Pensée et le mouvant.

{\it Bull.}, Bulletin de la Société française de Philosophie, A. Colin édit.

{\it C. C.}, Code Civil (le chiffre est le numéro de l’article du Code).

Comte, {\it Cours}, Cours de philosophie positive.

Descartes, {\it Méd.}, Méditations métaphysiques,

\ \ \ — {\it Méth.}, Discours de la méthode,

\ \ \ — {\it Princ.}, Principes de la philosophie,

\ \ \ — {\it Reg.}, Regulæ ad directionem ingenii.

\ \ \ — {\it Rép.}, Réponses aux Objections Méditations).

Kant, {\it Jug.}, Critique du jugement.

\ \ \ —  {\it R. pr.}, Critique de la raison pratique.

\ \ \ — {\it R. pure}, Critique de la raison pure.

\ \ \ — \ \ \ — {\it Analyt.}, Analytique transcendantale,

\ \ \ — \ \ \ — {\it Esth.}, Esthétique transcendantale,

\ \ \ — \ \ \ — {\it Dial.}, Dialectique transcendantale.

\ \ \ — \ \ \ — {\it Log.}, Logique transcendantale, introduction.

Leibniz, {\it Mon.}, Monadologie.

\ \ \ —  {\it N.E.}, Nouveaux Essais,

\ \ \ — {\it Théod.}, Théodicée,

Malebranche, {\it Écl.}, Éclaircissements à la Recherche de la vérité.

\ \ \ — {\it Entr.}, Entretiens sur la Métaphysique.

\ \ \ — {\it R. V.}, Recherche de la vérité.

Montesquieu, {\it Lois}, De l'Esprit des lois.

Pascal, « Pensées » (le chiffre indique le n° du fragment dans l'édition Brunschvieg).

\ \ \ —  {\it Prov.}, Provinciales.

Port-Royal, Logique de Port-Royal.

{\it R. M. M.}, Revue de Métaphysique et de Morale, A. Colin édit.

{\it R. Ph.}, Revue philosophique, P. U. F. édit.

Spinoza, {\it Eth.}, Éthique.

St Thomas, {\it S. th.}, Somme théologique.

\subsection {Autres abréviations}

\begin{minipage}[c]{.45\linewidth}
{\it Adj.} — Adjectif.

{\it Anal.} — Par analogie.

{\it Auj.} — Aujourd'hui.

{\it Autref.} — Autrefois.

{\it Cf.} — Se reporter à.

{\it Dist.} — Distinguer (de), ne pas confondre (avec).

{\it Ext.} — Par extension.

{\it Ibid.} — [{\it }Ibidem] Même référence.

{\it Id.} — [{\it }Idem] Même auteur.

{\it Péj.} — Avec un sens péjoratif.

{\it Ppt.} — Proprement.

{\it Qq.} — Quelque.

{\it Qqc.} — Quelque chose.

{\it Qqfs.} — Quelquefois.
\end{minipage}
\hfill
\begin{minipage}[c]{.45\linewidth}
{\it Gén.} — Généralement, en général.

{\it I. e.} — [{\it }Id est] C'est-à-dire.

{\it Lang.} — Langage.

{\it Lato.} — Au sens large.

{\it Latiss.} — Au sens très large.

{\it Laud.} — Avec un sens laudatif, élogieux.

{\it L.} — Lettre.

{\it Not.} — Notamment.

{\it Opp.} — Par opposition à.

{\it Qqn.} — Quelqu'un.

{\it Spéc.} — Spécialement.

{\it Str.} — Au sens étroit, précis.

{\it Trad.} — Traduction de.

{\it P. e.} — Par exemple.
%{\it Vg.} — [{\it Verbi gratia}] Par exemple.

\end{minipage}

\vspace{0.211cm}
Les références à nos {\it Précis de Philosophie} ont été indiquées à l’aide des
abréviations suivantes :
{\it Précis}, Ph I {\it ou} II. — Édition pour la Classe de Philosophie, tome I {\it ou} tome II.

{\it Précis}, Sc. — Édition pour les Classes de Sciences Expérimentales et de
Technique-Économique.

{\it Précis}, M. — Édition pour les Classes de Mathématiques et de
Mathématiques-Technique.

