
%%%%%%%%%%%%%%%%%%%%%
\section{Pratique de la philosophie}
%%%%%%%%%%%%%%%%%%%%%
%1

\subsection{Causalite}

\begin{itemize}[leftmargin=1cm, label=\ding{32}, itemsep=1pt]
\item {\footnotesize ÉTYMOLOGIE}: latin {\it causa}, « cause ».
\item {\footnotesize PHILOSOPHIE}: principe
en vertu duquel un phénomène
donné est rattaché à un autre qui est
perçu comme en étant la condition.
\end{itemize}

Le principe de causalité, selon lequel
« tout événement a une cause », fonde
l'idée du déterminisme naturel qui est au
cœur de la science moderne.

À partir du {\footnotesize X}$^\text{e}$ siècle, une réflexion critique
a mis en évidence que la causalité
n'était pas dans les choses, mais dans
l'esprit qui les pense. Ce déplacement
de l’objet vers le sujet connaissant a
deux versions :

1. L'empirisme et le scepticisme de
Hume : la causalité est une habitude de
l'esprit née de l'observation d’une
conjonction constante entre deux phénomènes.
Le développement de cette
idée conduisit Auguste Comte à récuser
le concept de cause au profit de celui de
lot physique, c'est-à-dire de relation
constante entre deux phénomènes, sans
qu'on soit en droit de s'interroger sur le
mécanisme producteur existant entre
l'un et l’autre.
%

2. Le criticisme de Kant. Ce dernier
refuse de faire de la causalité une simple
habitude associative. Elle est une structure
a priori de l’entendement, un
concept pur qui organise l'expérience et
permet de la connaître : si nous repérons,
dans la nature, des causes et des effets, et
non une succession chaotique de phénomènes,
c’est que nous savons a priori
que « tout événement a une cause ».

Le développement des sciences cybernétiques
modernes a permis d'élaborer
un modèle de causalité complexe et non
mécanique, selon lequel les phénomènes
se conditionnent réciproquement
à l’intérieur d’un système ; la causalité
est alors non linéaire mais circulaire,
avec retour de l'effet sur la cause.

D'un point de vue métaphysique, le problème
de la causalité rencontre celui de
la liberté : peut-il exister une causalité
libre dans le monde et, si oui, comment
déterminisme et spontanéité peuvent-ils
se concilier ?

\begin{itemize}[leftmargin=1cm, label=\ding{32}, itemsep=1pt]
\item {\footnotesize TERME VOISIN} : déterminisme.
%\item {\footnotesize TERME OPPOSÉ} : 
\end{itemize}


\begin{itemize}[leftmargin=1cm, label=\ding{32}, itemsep=1pt]
\item {\footnotesize CORRÉLATS} : Catégorie; cause ;
étiologie ; habitude ; hasard ; raison.
\end{itemize}

\subsection{Cause}

\begin{itemize}[leftmargin=1cm, label=\ding{32}, itemsep=1pt]
\item {\footnotesize ÉTYMOLOGIE}: latin {\it causa}, « raison », « cause ».
\item {\footnotesize SENS ORDINAIRES}: {\bf 1.} Ce qui produit un effet. {\bf 2.} Ce qui
constitue l’antécédent constant d’un
phénomène. {\bf 3.} Principe explicatif ;
raison où motif (ex. : « la cause d’un
échec »).
\end{itemize}

Aristote distinguait quatre causes : la
cause matérielle, où le support de la
transformation (ex. : le bloc de marbre
par rapport à la statue) ; la cause efficiente
où l'agent de la transformation
(en l'occurrence, l’action du burin pour
fabriquer la statue) ; la cause finale, où
le but en vue duquel s’accomplit la
transformation (ici l'intention du sculpteur) ;
la cause formelle, ou l'idée qui
organise l’objet transformé selon une
forme déterminée.

Cette classification fut battue en brèche
au fur et à mesure du développement de
la science moderne. Celle-ci évacue les
causes finales (cf. Finalisme) ; la cause
et l'effet sont désormais conçus comme
unis dans un rapport d’antécédent à
conséquent, conforme à l’idée de déterminisme
naturel (cf. Causalité) et dont
le modèle le plus simple (mais non le
seul) est la transmission du mouvement
dans une machine. Le positivisme ira
jusqu’à dissocier nettement la cause (ou
antécédent constant) et la raison (ou
principe d’intelligibilité) qui n’a plus sa
place dans les sciences de la nature.

\begin{itemize}[leftmargin=1cm, label=\ding{32}, itemsep=1pt]
\item {\footnotesize TERME VOISIN} : antécédent ;
condition ; raison.
\item {\footnotesize TERME OPPOSÉ} : effet.
\end{itemize}


\begin{itemize}[leftmargin=1cm, label=\ding{32}, itemsep=1pt]
\item {\footnotesize CORRÉLATS} : cause ; causalité ; destin ;  empirisme ;  fatalisme ; hasard ; induction ; liberté ; loi ; nature ; nécessité ; science ; volonté.
\end{itemize}

\subsection{Cause occasionelle}
Selon la théorie « occasionnaliste » de
Malebranche : agent qui produit, ou qui
« occasionne » un effet, ou un événement.
En fonction du principe de la
« création continuée », Dieu est la seule
cause directe et efficace de tout ce qui
arrive dans le monde. Les éléments du
réel (ou causes « antécédentes ») ne sont
que la condition, ou l'instrument de la
volonté divine. Cf. Malebranche.

\begin{itemize}[leftmargin=1cm, label=\ding{32}, itemsep=1pt]
\item {\footnotesize TERME VOISIN} : causalité.
\item {\footnotesize TERME OPPOSÉ} : indéterminisme.
\end{itemize}


\begin{itemize}[leftmargin=1cm, label=\ding{32}, itemsep=1pt]
\item {\footnotesize CORRÉLATS} : causalité ; déterminisme ;
explication ; finalité ; finalisme ; nature.
\end{itemize}

%%%%%%%%%%%%%%%%%%%%%%%%%%%%%%%%%%%%%%%%%%%%%%%%%%%%%%%%%%%%%%%%%%%%%%%%%%%
