
%%%%%%%%%%%%%%%%%%%%%
\section{Encyclopédie de la philosophie}
%%%%%%%%%%%%%%%%%%%%%
%1
\subsection{Causalité}
Relation de cause à effet. Le
principe de causalité consiste à affirmer
que rien n’arrive sans cause.

—> cause, déterminisme

 

\subsection{Cause}
Le mot est issu du latin {\it carere} qui a
aussi donné « chose » en français. Il
semble que comme l’allemand {\it Sache}, il ait
d’abord été employé dans un contexte
juridique. On retrouve d’ailleurs trace de
cet usage dans des expressions comme
« plaider sa cause » ou « mettre en cause ».
% 249
La notion philosophique de cause est
particulièrement complexe et ses développements historiques multiples. On peut
dire qu’elle est indissociable d’un besoin
de la raison consistant à vouloir expliquer
l'apparition d’une chose en en donnant la
cause, c’est-à-dire ce qui fait qu’elle est
apparue, et qu’elle est apparue ainsi et
pas autrement.

\subsubsection{La notion de cause dans
l'Antiquité et au Moyen Age (gr. {\it aition}
ou {\it aitia}, lat. {\it causa})}


Si l’on s’en tient à la présentation qu’en
ont donnée les Anciens, l’idée de cause est
indissociable de la volonté de rendre raison des phénomènes ({\it didonai logos}) caractéristique des débuts de la philosophie
grecque et de ceux qu’Aristote a appelés
les « physiologues ». L'idée est d'expliquer
ce qui est sans avoir recours à un récit imaginaire (mythe) ou à une construction faisant appel à des forces surnaturelles. Le
concept, sans doute né dans le contexte des
réflexions de l’école atomiste grecque
(Leucippe, Démocrite, {\footnotesize V}$^\text{e}$ s. av. J.-C.),
caractérise tout le mouvement de la pensée
philosophique et scientifique occidentale.
Depuis l’Antiquité, la notion de causalité
ou de relation causale postule l’existence
d’un lien nécessaire réglant la succession
des phénomènes donnés dans l’expérience. Ce lien peut renvoyer à des causes
purement physiques et mécaniques
(comme le concevaient les atomistes) ou
recourir à des causes intelligibles et premières (Anaxagore, Platon, les stoïciens).
La formulation la plus élaborée du concept
de cause, et qui influença toute la pensée
antique et médiévale, se trouve chez Aristote ({\it Physique} A 1, 184a, 10). Aristote
identifie expressément la connaissance
scientifique à la recherche des causes et en
présente quatre types : 1) la {\it matière} (en gr.
la {\it hylè}, ce dont une chose est faite) que les
interprètes d’Aristote ont au Moyen Age
appelée la {\it causa materialis}, 2) la {\it forme}
(l’{\it eidos} ou le {\it logos}, parfois aussi appelée
{\it ousia} chez Aristote) est le modèle ou
même l'essence d’une chose, ce qui fait
qu’elle est ce qu’elle est, 3) la {\it causa formalis}, le principe du mouvement
({\it è archè tès kineseôs}, littéralement ce qui est à l’origine
de la chose, ce peut être le sculpteur pour
la statue, ou, pour être plus précis, la
{\it tékhné} du sculpteur) ; ce type de cause a
été modifié profondément par les Latins
sous l'influence de la conception stoïcienne
%250
de la causalité et du créationnisme
issu des monothéismes, la tradition la
nomme souvent par l'expression {\it causa efficiens}, 4) ce en vue de quoi la chose
est produite (gr. {\it to hou eneka}) désignant la fin (gr.
{\it telos}) en vue de laquelle une chose est produite et qui a été désigné par l’expression
{\it causa finalis}. La scolastique développe son
interprétation du schème aristotélicien de
la causalité en y adjoignant des distinctions
consacrées aux causes directes et indirectes, univoques et équivoques, et donne
un relief particulier au problème de la
{\it cause première}, identifiée à Dieu et avancée comme preuve de son existence
(argument cosmologique).

\subsubsection{Le passage de la notion de cause à celle de loi}

La pensée classique se focalise au
contraire sur la notion de cause efficiente.
Le mécanisme porte en lui l’idée qu’on
doit se représenter la nature comme une
machine dans laquelle les parties agissent
sans qu'il soit besoin de faire appel à
l'idée de finalité interne. Cette primauté
accordée à l’Age classique à l’aspect d’efficience de la causalité est à l’origine de
l'émergence progressive et complexe du
concept scientifique moderne de loi qui,
chez Auguste Comte ({\footnotesize XIX}$^\text{e}$ s.), est même
censé venir se substituer au « stade positif » à la relation causale. Cette évolution,
la lente incubation de la notion de loi
scientifique, et la formation du concept
mathématique afférent de fonction
$f(x) = y$, n’a été possible que parce que la
relation de cause à effet a pu se traduire
plus particulièrement en une relation
entre des grandeurs, mathématiquement
mesurables (processus de mathématisation de la physique dont les hérauts furent
Kepler, Galilée, Descartes). Ce modèle,
qui demeure au fond celui de toute la
physique classique de Newton à Laplace,
est à la base du mécanisme et du déterminisme qui caractérisent la conception classique de la nature.

\subsubsection{La critique de l’induction causale}

La validité universelle de la relation de
cause à effet, telle qu’on la trouve appliquée au domaine de l'expérience, en
physique notamment, a toutefois été fortement remise en question au {\footnotesize XVIII}$^\text{e}$ s. par
Hume, avec des arguments s’inspirant
pour une part des thèses de Sextus Empiricus et de l’école sceptique en général.
Ces thèses, selon lesquelles il n’existe
% 251
aucun lien nécessaire entre cause et effet,
mais seulement une corrélation de faits,
et selon lesquelles la connaissance d’une
chose comme cause n’implique pas la
connaissance d’une chose différente
comme effet, avaient aussi trouvé des
défenseurs au Moyen Age avec, par
exemple, le philosophe al-Ghazali ({\footnotesize XI}$^\text{e}$ s.)
et les différents penseurs soutenant l’occasionnalisme. Pour Hume, la nécessité
causale et la présence de lois universelles
dans la nature sont une hypothèse, qu’explique uniquement l'habitude mentale
d'opérer des associations. En d’autres
termes, il est impossible d’induire ou d’inférer à partir de l’expérience une relation
de causalité ayant un caractère de nécessité stricte. C’est ce problème que Kant
entend résoudre en faisant de la relation
causale une catégorie transcendantale de
l'entendement, c’est-à-dire une forme {\it a
priori} de l'expérience, universellement
valide, susceptible de justifier le caractère
nécessaire des lois scientifiques. Ce débat
a perdu une grande partie de son intérêt
scientifique dans la mesure où, à partir de
la première moitié du {\footnotesize XX}$^\text{e}$ s., la crise du
modèle mécanique de compréhension de
la nature a mis en lumière, notamment
avec la mécanique quantique et la théorie
des relations d’incertitude, qu’il n’était
pas possible de s’en tenir à une conception déterministe de la nature, sur le
modèle de Laplace. Le caractère statistique des lois scientifiques, la discontinuité et le principe d’indétermination
amènent de nouveaux modèles d’explication scientifique des phénomènes. Ainsi,
avec la physique quantique, le modèle
déterministe strict tend à céder la place à
un modèle probabiliste dans lequel le lien
cause / effet ne peut plus être déterminé
selon les critères de l’objectivité classique.
Parallèlement, la question du statut ontologique des lois et des théories scientifiques devient le centre des discussions
des épistémologues. Doivent-elles être
entendues comme des traductions réalistes des phénomènes naturels, ou
comme des schèmes commodes et fonctionnels permettant de résumer de façon
conventionnelle les mesures quantitatives
et les prévisions expérimentales des
savants ? Ce débat, amorcé par certains
philosophes et certains savants comme
Ernst Mach et Hermann von Helmholtz
entre la seconde moitié du {\footnotesize XIX}$^\text{e}$ s. et le
début du {\footnotesize XX}$^\text{e}$ s., aboutit progressivement
% 251
à l'élimination du concept de causalité ou
de relation causale dans le domaine scientifique et à son remplacement par celui
de « loi descriptive » : les lois scientifiques
viseraient donc à relever la constance et
l’uniformité des phénomènes au moyen
de « descriptions synthétiques » (comme
l’écrivait Mach). La connaissance scientifique se réduit ainsi à la formulation de
séquences uniformes qui exigent une vérification continuelle par l'expérience et qui
n’ont plus la prétention d'{\it expliquer} les
phénomènes par le recours à des relations
supposées de causalité objective. Ces
conceptions, articulées sur une théorie
conventionnelle de la science, comme par
la suite également l’opérationnisme de
Percy Williams Bridgman, ont cependant
soulevé de nombreuses critiques. La
physique théorique et l’épistémologie
contemporaines, tout en admettant la
valeur des lois descriptives, sont toutefois
principalement orientées vers l’élaboration d’un nouveau concept de causalité,
plus modulable et plus souple.

 

—> Aristote, Bridgman, conventionnalisme, déterminisme, finalisme,
mécanisme, opérationnisme
%%%%%%%%%%%%%%%%%%%%%%%%%%%%%%%%%%%%%%%%%%%%%%%%%%%%%%%%%%%%%%%%%%%%%%%%%%
