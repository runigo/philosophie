
%%%%%%%%%%%%%%%%%%%%%
\section{André Comte-Sponville}
%%%%%%%%%%%%%%%%%%%%%
%1

\subsection{Causalité}

C’est une relation entre deux êtres ou deux événements, telle
que l'existence de l’un entraîne celle de l’autre et l'explique.

La causalité se déduit ordinairement de la succession : si à chaque fois
qu’un phénomène {\it a} apparaît, le phénomène {\it b} le suit, on en conclura que {\it a} est
la cause de {\it b}, qui serait son effet. C’est passer du constat empirique d’une
conjonction constante (à notre échelle) à l’idée d’une connexion nécessaire.
Hume n’aura pas de mal à montrer que ce passage reste intellectuellement mal
fondé. Car une succession, aussi répétée qu’elle puisse être, ne saurait en toute
rigueur prouver quoi que ce soit : l’idée d’une connexion nécessaire, entre la
cause et l’effet, n’est que le résultat en nous d’une accoutumance très forte, qui
nous pousse à passer, presque inévitablement, de l’idée d’un objet à celle d’un
% 101 
autre objet qui le précède ou le suit. La causalité n’apparaît pas dans le monde
(entre les choses) mais dans l'esprit (entre les idées). Ou elle n’apparaît dans le
monde, préciserait Kant, que parce qu’elle est d’abord dans l'esprit : c’est une
catégorie de l’entendement, qui ne saurait venir de l'expérience puisqu'elle la
rend possible. On n’échappe à l’empirisme que par le transcendantal ; au transcendantal,
que par l’empirisme ou le matérialisme.

Quoi qu'il en soit du statut de l’idée de causalité ({\it a priori} ou {\it a posteriori}),
on accordera à ces deux auteurs que la causalité en tant que telle n’est jamais
perçue {\bf --} on ne perçoit que des successions ou des simultanéités {\bf --} ni démontrée.
C’est peut-être ce qui explique que les sciences modernes, comme l'avait
remarqué Auguste Comte, s'intéressent moins aux {\it causes} qu'aux {\it lois}. C’est
renoncer au {\it pourquoi}, pour ne plus dire que le {\it comment}, et préférer la {\it prévision}
à {\it l'explication}. L'action y trouve son compte ; mais l'esprit, non. Car quelle est
la cause des lois ?

\subsection{Causalité (principe de --)}

Le principe de causalité stipule que tout
fait a une cause et que, dans les mêmes
conditions, la même cause produit les mêmes effets. C’est parier sur la rationalité
du réel et sur la constance de ses lois. Pari sans preuve, comme ils sont tous,
et jusqu'ici sans échec.

On ne confondra pas le principe de causalité avec le déterminisme absolu,
qui suppose non seulement la constance des lois de la nature mais aussi l’unité
et la continuité des séries causales dans le temps, de telle sorte qu’un état donné
de l'univers découle de ses états antérieurs et entraîne nécessairement la totalité
de ses états ultérieurs (ce pour quoi le déterminisme, en ce sens fort, est en
vérité un prédéterminisme : tout est joué ou écrit à l'avance). L’indéterminisme,
de même, ne suppose pas une violation du principe de causalité, mais
simplement la pluralité et la discontinuité des chaînes causales. Ainsi, chez
Lucrèce, le {\it clinamen} n’est pas sans cause (sa cause est l’atome) ; mais il est sans
cause antécédente : c’est un pur présent, qu'aucun passé n’explique ni ne contenait.
Il se produit {\it incerto tempore, incertisque locis} (il est indéterminé dans
l’espace et le temps). Comme le remarque Marcel Conche, « le principe de causalité
n’en est pas pour autant contredit, car, comme tel, il n’implique pas que
toute cause doive produire son effet sous des conditions de lieu et de temps ».
C’est en quoi le clinamen vient briser « le destin des physiciens » (le prédéterminisme),
non la rationalité du réel (la causalité).

On remarquera que toute cause, étant elle-même un fait, doit avoir une
cause, qui doit à son tour en avoir une, et ainsi à l’infini. C’est ce qu’on appelle,
à propos de Spinoza, la chaîne infinie des causes finies ({\it Éthique}, I, prop. 28).

% 102
Le déterminisme suppose la continuité de cette chaîne ; l’indéterminisme, sa
discontinuité. Mais il y a plus. D’un point de vue métaphysique, l’application
indéfiniment réitérable du principe de causalité semble exiger {\bf --} si l’on veut
échapper à la régression à l’infini {\bf --} une cause première, qui serait sans cause ou
cause de soi. Le principe de causalité, pris absolument, aboutit ainsi à sa propre
violation (une cause sans cause) ou au cercle (une cause qui se causerait elle-même).
Si Dieu est cause de tout, quelle est la cause de Dieu ? S’il n’y a pas de
Dieu, quelle est la cause de tout ?

\subsection{Cause}

Ce qui produit, entraîne ou conditionne autre chose, autrement dit
ce qui permet de l'expliquer : sa condition nécessaire et suffisante,
s’il en est une, ou l’ensemble de ses conditions.

Une cause est ce qui répond à la question {\it « Pourquoi ? »}. Comme on peut
répondre de plusieurs points de vue différents, il y a différents types de cause.
Aristote en distinguait quatre : la cause formelle, la cause matérielle, la cause
efficiente, la cause finale ({\it Métaphysique}, A, 3, {\it Physique}, II, 3 et 7). Soit par
exemple cette statue d’Apollon. Pourquoi existe-t-elle ? Bien sûr parce qu’un
sculpteur l’a taillée ou modelée (c’est sa cause efficiente : par exemple Phidias).
%{\bf --} 103 
Mais elle n’existerait pas, du moins telle qu’elle est, sans la matière dont elle est
faite (c’est sa cause matérielle : par exemple le marbre), ni sans la forme qu’elle
a ou qu’elle est (non la forme supposée d’Apollon, que personne ne connaît,
mais la forme réelle de la statue elle-même : c’est sa cause formelle, son essence
ou quiddité), ni enfin sans le but en vue duquel on l’a sculptée (qui est donc sa
cause finale : par exemple la gloire, la dévotion ou l’argent). Les Modernes, de
ces quatre causes, ne retiennent guère que la cause efficiente. Ils ne croient plus
qu’en l’action, qui n’a pas besoin de croire.

\subsection{{\it Causa sui}}

Cause de soi. La notion est évidemment paradoxale : elle ne
s'applique légitimement qu’au libre arbitre et à Dieu, s’ils existent,
ou au Tout, s’il est nécessaire. « J'entends par {\it cause de soi}, écrit Spinoza,
ce dont l’essence enveloppe l'existence, autrement dit ce dont la nature ne peut
être conçue sinon comme existante. » Ce sont les toutes premières lignes de
l’{\it Éthique}, qui commence ainsi par un abime. Comment se causer soi, puisqu'il
faudrait, pour en être capable, exister déjà, donc n’avoir plus besoin de cause ?
Mais c’est passer à côté de l'essentiel, qui est l’éternité, qui est la nécessité (être
cause de soi, c’est exister « par la seule nécessité de sa nature » : {\it Éthique}, I, 24,
dém.), qui est l’immanence, qui est « l'affirmation absolue de l’existence d’une
nature quelconque » (premier scolie de l’Éthique), qui est la puissance d’exister
de tout, comme le {\it conatus} de la nature ou « l’auto-productivité même du Réel »
(cette dernière expression est de Laurent Bove, à propos de Spinoza). Un
abîme ? Si l’on veut, mais absolument plein : abîme de l'être, non du néant, et
qui fait comme un sommet indépassable. « La {\it causa sui} n’est pas un principe
abstrait, souligne Laurent Bove : elle est la position du réel (en son essence
identique à sa puissance) comme “affirmation absolue” ou comme autonomie »
({\it La stratégie du conatus}, Vrin, 1996, p. 7). C’est la puissance d’exister de tout,
ou du Tout, sans quoi aucune cause jamais ne serait possible.

%%%%%%%%%%%%%%%%%%%%%%%%%%%%%%%%%%%%%%%%%%%%%%%%%%%%%%%%%%%%%%%%%%%%%%%%%%%
