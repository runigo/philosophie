
%%%%%%%%%%%%%%%%%%%%%
\section{Dictionnaires}
%%%%%%%%%%%%%%%%%%%%%
\subsection{Encyclopédique}
{\bf Cause} ({\it nf}\,) {\bf 1.} Ce qui fait que qqch est ou se fait; raison, motif. Les causes d'une guerre. {\bf 2.} Parti à soutenir; proçès, intérêts. Défendre sa cause.

\subsection{Étymologique}
{\bf Cause} fin {\footnotesize XII}$^\text{e}$ s. {\it Rois} du lat. {\it causa}, cause et procès ; 1549, sens politique et religieux.
{\footnotesize (causal du lat. {\it causalis} ; causalité, causatif, causer, causant, causeur, causerie, causette, causeuse, recauser)}

\subsection{Vocabulaire}

{\bf Causalité} — \si{Épist.} Rapport de cause*
à effet. — Principe de causalité :
« Tout a une cause et, dans les
mêmes conditions, la même cause
est suivie du même effet. »

{\bf Causation} — \si{Méta.} Action par laquelle la cause$^1$ produit son effet.

{\bf Cause} — \si{Méta.} {\bf 1.} Force$^2$ productrice,
engendrant l'effet et se prolongeant
en lui. {\it cf.} {\it Efficace}* et {\it Occasionnelle}*. — \si{Épist.} {\bf 2.} Antécédent$^1$
constant (Hume) et inconditionnel
(J. S. Mill). — {\bf 3.} Phénomène lié au
phénomène considéré par une relation fonctionnelle : « La cause n’est
jamais vraiment empirique » (Bachelard) $->$ {\it Dans la science}, l'explication par les forces productrices
(sens 1) fait place de plus en plus à
l’explication par les relations fonctionnelles (sens 3). Aussi, tandis
que F. Bacon disait que « savoir
vraiment, c’est savoir par les causes »
(sens 2), A. Comte a pu écrire
(Cours, I) que la science renonce à
la recherche des causes (sens 1), ce
qui est d'ailleurs auj. discuté.

— \si{Hist.} {\bf 4.} {\it Aristote} distingue
4 espèces de causes : a) la cause matérielle ({\it p. e.} dans une statue, la
matière dont elle est faite); — b) la
cause formelle (la figure que la statue
représente; {\it cf.} Formel); — c) la
cause efficiente, {\it i. e.} la cause au sens 1
(le sculpteur); — d) la cause finale$^1$
(le but : désir de la gloire ou du gain,
visé par le sculpteur).

— \si{Méta.} {\bf 5.} Cause première : voir
Premier$^4$.

%%%%%%%%%%%%%%%%%%%%%%%%%%%%%%%%%%%%%%%%%%%%%%%%%%%%%%%%%%%%%%%%%%%%%%%%%%%
