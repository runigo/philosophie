%TU {\it }
TABOU C’est comme un interdit sacré. De là l’envie de le violer, par curiosité,
par défi, par bravade. Mieux vaut une loi claire, librement
acceptée et discutée.

TALENT Plus qu’un don, moins que du génie. Un enfant qui est doué pour

les mathématiques ou le dessin, on ne dira pas forcément qu’il a du
talent. Et un artiste talentueux ou génial, comme Cézanne, peut n'être que
moyennement doué. Le don est une facilité à apprendre. Le talent, une puissance
de créer. Le don est donné à la naissance : il touche à la génétique. Le
talent se conquiert davantage pendant l’enfance et l’adolescence : il touche à
l’histoire, à la psychologie, à l'aventure d’être soi ou de le devenir. Le don est
impersonnel. Le talent serait plutôt la personne même, quand elle parvient à
s'exprimer de façon créatrice et singulière.

On sait que le mot vient d’une métaphore. Dans la fameuse {\it parabole des
talents}, Jésus compare implicitement les capacités que chacun a reçues à des
pièces de monnaie (des « talents »), qu’il doit faire fructifier. L'important, c’est
moins le talent qu’on a que ce qu’on en fait : ce n’est plus talent, mais œuvre
ou gâchis.

TAUTOLOGIE Une proposition qui est toujours vraie, soit parce que le prédicat
ne fait que répéter le sujet (« Dieu est Dieu »), soit
parce qu’elle reste valide indépendamment de son contenu et même de la
valeur de vérité des éléments qu’elle met en œuvre. La logique formelle est faite
de tautologies : « Si {\it p} implique {\it q}, et si {\it non-q}, alors {\it non-p} » (c'est ce qu’on
%— 574 —
appelle le {\it modus tollens}) est toujours vrai, quels que soient le contenu et la
valeur de vérité de {\it p} et de {\it q}.

On remarquera que le mot de tautologie, pris en ce sens, n’a rien de péjoratif,
Mais même pris au sens de répétition, il ne l’est pas forcément. Quand
Parménide nous dit que l’être est, il fait une tautologie. Mais cela, loin de le
réfuter, le rend irréfutable.

TECHNIQUE Un ensemble d'instruments (outils, machines, logiciels.) et
de savoir-faire, permettant d’obtenir un certain résultat.

Le mot vient du grec {\it tekhnê}, qui est l’équivalent du latin {\it ars}. Mais les deux
vocables, en français, ont évolué en des sens opposés. La technique se distingue
de l’art, et même de l’artisanat, par son efficacité impersonnelle : un objet technique
peut être fabriqué identiquement par tous les individus compétents et
convenablement outillés ; un produit artisanal ou une œuvre d’art, non. C’est
pourquoi l’art est plus singulier, plus contrasté, plus précieux. Et la technique,
plus efficace.

Les techniques constituent un élément essentiel du progrès, aussi bien pour
l'individu (« sans technique, chantait Brassens, un don n’est rien qu’une sale
manie ») que pour la société (c’est à elles qu’on doit le développement des
forces productives, comme disait Marx, dont tout le reste découle). Elles n’en
ont pas moins leurs dangers ; mais elles valent mieux, presque toutes, presque
toujours, que leur contraire, qui est soumission aveugle à la nature. Tailler un
silex, fût-ce pour s’en faire une arme, cela vaut mieux que se laisser dévorer ou
massacrer.

On a beaucoup reproché à Descartes d’avoir voulu, selon lexpression
fameuse du {\it Discours de la méthode}, nous rendre « comme maîtres et possesseurs
de la nature ». De là viendrait tout le mal, la mise en coupe réglée de la nature,
son pillage, son saccage (son {\it arraisonnement}, dit Heidegger), le travail à la
chaîne, la bombe atomique, la dégradation irréversible de l’environnement.
L’oubli de l’être déboucherait sur le culte de lutile et du rendement, l’humanisme
sur le machinisme, le rationalisme sur la barbarie et la déraison.… C’est
oublier le « {\it comme} », qui maintenait une certaine distance (s’il s’agit de nous
rendre {\it comme} maîtres et possesseurs de la nature, c’est que nous ne le sommes
pas et ne le serons jamais). C’est oublier, surtout, d’où nous venons. Un de mes
amis, écologiste radical, ne jure que par le paléolithique : la révolution néolithique
serait la faute première, dont toutes les autres découlent. Les hommes,
m’explique-t-il, cessent alors de vivre en harmonie avec la nature : ils commencent
à la transformer, à la pressurer, à la défigurer.. Pour d’autres ce serait la
révolution industrielle, les technosciences, la révolution informatique. Ils ne
%— 575 —
me feront pas regretter la préhistoire, ni le Moyen-Âge, ni même le {\footnotesize XIX}$^\text{\,e}$ siècle.
Une imprimerie-vaut mieux qu’un stylet. Un ordinateur vaut mieux qu’un
boulier. Une machine à laver, mieux qu’un lavoir. Un vaccin, mieux qu’un
grigri.

Le danger n’en demeure pas moins, ou plutôt il ne peut que s’aggraver,
d’une civilisation technicienne, qui prendrait les moyens pour des fins. Nos
techniques, aujourd’hui, sont bien davantage que des outils. Ce sont des pensées,
mais objectivées, mais instrumentalisées. Ce sont des sciences, mais appliquées.
Or quoi de plus normal que de se soumettre au vrai ? Et quoi de plus
vrai que les sciences ? Double contresens : double idolâtrie. La vérité sans la
charité n’est pas Dieu, disait Pascal. Les sciences sans l'humanité sont inhumaines.

Les techniques, historiquement, sont antérieures aux sciences, mais depuis
longtemps transformées par elles : les nôtres en dépendent de plus en plus, au
point d’en être à peu près indissociables (c’est ce qu’on appelle les techno-sciences).
De là une puissance démultipliée, qui devient en effet inquiétante, et
d'autant plus qu’elle s’autonomise davantage : les moyens tendent à nous
imposer leurs fins, ou plutôt à valoir comme telles (l’efficacité devient une
valeur en soi). Nos techniques nous gouvernent, au moins autant que nous les
gouvernons. Âvec un marteau, dit-on souvent, on peut faire ce qu’on veut,
planter un clou ou fracasser un crâne. Sans doute. Mais on peut aussi ne rien
faire, et tel est le cas le plus fréquent. Ce n’est plus vrai de nos machines, dont
beaucoup fonctionnent jour et nuit, qu’il faut amortir, qu’il faut rentabiliser,
qui fabriquent d’autres machines, qui créent jusqu'aux besoins qu’elles viennent
satisfaire, qui nous font vivre, et qu’on ne peut plus arrêter, bien souvent,
sans mettre en cause l’existence même de nos sociétés. Nos voitures menacent
l’environnement, ou plutôt elles font plus que le menacer. Mais on ne
reviendra pas à la traction hippomobile. Nos télévisions menacent l’intelligence.
Mais on ne reviendra pas au règne presque exclusif de l'écrit. Il faut
avancer toujours, comme en vélo, mais en essayant de rester maître au moins
de sa vitesse et de sa direction. Personne n’a décidé de faire la révolution industrielle,
ni la révolution informatique et communicationnelle. Comment
quelqu'un pourrait-il les arrêter ou les abolir ? L'histoire des techniques est irréversible,
comme l’histoire des sciences, et pour la même raison. Pas question de
revenir en arrière, et c’est tant mieux. Mais pas question non plus de laisser le
marché ou les machines décider à notre place. Que les techniques créent des
besoins, c’est entendu. Mais comment pourraient-elles remplacer des volontés ?
L'issue, malgré Heidegger, n’est ni dans la technophobie ni dans la contemplation
fascinée de l’être ou des origines, mais dans la soumission résolue des
moyens que nous nous sommes donnés aux fins que nous nous fixons — ce qui
%— 576 —
est morale, pour l'individu, et politique, pour les citoyens. Si le peuple est souverain,
il est exclu que les machines ou les technocrates le soient.

TECHNOCRATIE Le pouvoir de la technique, ou plutôt des techniciens.
C’est une forme de barbarie, qui voudrait soumettre la
politique et le droit à l’ordre techno-scientifique : tyrannie des experts. On y
parvient insensiblement, dès lors qu’on veut que les plus compétents gouvernent
ou décident. Contre quoi il faut rappeler que la démocratie non seulement
n’en a pas besoin mais l’exclut : ce n’est pas parce que le peuple est compétent
qu’il est souverain, c’est parce qu’il est souverain qu'aucune compétence
ne saurait, politiquement, valoir sans lui ou contre lui. Les experts sont là pour
l'éclairer, non pour décider à sa place.

TÉLÉOLOGIE L'étude des finalités ({\it telos}, en grec, c’est la fin). Cette étude

peut être utile et légitime, montre Kant, mais à condition de
ne considérer le concept de finalité que comme un concept régulateur, qui ne
vaut que pour la faculté de juger réfléchissante : il s’agit de faire {\it comme} si la
nature poursuivait un but, tout en sachant qu’on ne pourra jamais montrer que
c’est en effet le cas ({\it Critique de la faculté de juger}, II, 2, \s 75 et 76), voire en
pensant qu’il n’en est rien. La téléologie ne débouche sur une théologie que
subjectivement, dit Kant, et pour ceux-là seuls, ajouterai-je, qui en sont dupes.

TÉLÉONOMIE Une finalité sans finalisme, donc sans causes finales : une

finalité pensée comme effet de causes efficientes (par
exemple, dans le darwinisme, comme effet de l’évolution des espèces et de la
sélection naturelle).

TÉMÉRITE Un courage disproportionné face au danger : le téméraire prend
des risque exagérés, pour un enjeu qui ne les justifie pas. C’est
moins un excès de courage qu’un manque de prudence.

TÉMOIGNAGE C’est dire ce qu’on sait ou ce qu’on croit, quand on n’a
aucun moyen de le prouver. Tient lieu de preuve, dans certaines
questions de fait, quand les témoins sont multiples, et quand les preuves
font défaut.

%— 577 —
TEMPÉRAMENT Étymologiquement, c’est un mélange, un équilibre, une
proportion. Se dit en musique du système identifiant
deux notes très voisines (par exemple’un do dièse et un ré bémol) afin de
diviser l’octave, pour l’adapter aux instruments à sons fixes, en douze demi-tons
égaux : c’est en ce sens que Bach nomme l’un de ses chefs-d’œuvre {\it Le
clavier bien tempéré}. Mais le mot désigne surtout un certain type de constitution
individuelle, à la fois physique et psychologique, dont on crut longtemps,
depuis Hippocrate et Galien, qu’il dépendait du mélange, plus ou
moins équilibré ou déséquilibré, de quatre humeurs : la lymphe, le sang, la
bile, l’atrabile (ou bile noire). De là quatre tempéraments traditionnels,
fondé sur la prédominance de l’une des quatre : le lymphatique, le sanguin,
le bilieux, enfin l’atrabilaire ou le mélancolique. Cette classification est
aujourd’hui abandonnée. Mais l’idée de tempérament suit son cours, qui
suppose une typologie des individualités : c’est « un ensemble de traits généraux,
comme dit Lalande, qui caractérisent la constitution physiologique
individuelle d’un être », mais en tant, ajouterai-je, qu’elle a des retombées
psychologiques et qu’elle est susceptible d’entrer dans une certaine classification.
C’est une façon particulière, mais commune à des millions de gens,
d’être son corps. Se distingue par là du caractère, plus individuel, moins physiologique.
Et contribue à le façonner.

TEMPÉRANCE La modération dans les plaisirs sensuels. C’est une exigence
de la prudence, mais aussi de la dignité. L’intempérant est
esclave, et c’est la liberté qui est bonne.

Il ne s’agit pas de ne pas jouir (tempérance n’est pas ascétisme), mais de
jouir mieux — ce qui suppose qu’on reste maître de ses désirs. Ainsi le gourmet,
contre le goinfre qu’il porte en lui. L’amateur de vins, contre l’ivrogne.
L'amant, contre le violeur ou le goujat.

La tradition y voit une vertu cardinale. C’est qu’il n’y a pas de vertu sans
maîtrise de soi, ni de maîtrise de soi sans tempérance.

TEMPORALITÉ C’est une dimension de la conscience : sa façon d’habiter
le présent en retenant le passé et en anticipant l'avenir.
Elle n’est pas la vérité du temps, montre Marcel Conche, mais sa négation (elle
fait exister ensemble, comme «unité ek-statique du passé, du présent et du
futur », ce qui ne saurait en vérité coexister). Ce n’est pas le temps réel, mais
notre façon de le vivre ou de l’imaginer.

%— 578 —
TEMPS  « Le temps, disait Chrysippe, se prend en deux acceptions. » Il est
d’usage de les confondre,.et c’est cette confusion, presque toujours,
qu’on appelle le temps.

Le temps, c’est d’abord la durée, mais considérée indépendamment de ce
qui dure, autrement dit abstraitement. Non un être, donc, mais une pensée.
C’est comme la continuation indéfinie et indéterminée d’une inexistence : ce
qui continuerait encore, c’est du moins le sentiment que nous avons, si plus
rien n'existait.

Ce temps abstrait — l’{\it aiôn} des stoïciens — peut se concevoir, et se conçoit
ordinairement, comme la somme du passé, du présent et de l’avenir. Mais ce
présent n’est alors qu’un instant sans épaisseur, sans durée, sans temps (s’il
durait, il faudrait le diviser en passé et en avenir), et c’est en quoi il n’est rien,
ou presque rien. En ce sens, et comme disait encore Chrysippe, « aucun temps,
n’est rigoureusement présent ». C’est ce qui le distingue de la durée. À le considérer
abstraitement, le temps est constitué essentiellement de passé et d’avenir
(alors qu’on ne peut durer qu’au présent), et pour cela indéfiniment divisible
(ce que le présent n’est jamais) et mesurable (ce que le présent n’est pas davantage).
C’est le temps des savants et des horloges. « Pour déterminer la durée,
écrit Spinoza, nous la comparons à la durée des choses qui ont un mouvement
invariable et déterminé, et cette comparaison s’appelle le temps. » Comparaison
n’est pas raison : le présent, incomparable et indivisible, n’en continue pas
moins.

Quant au temps concret ou réel — le {\it chronos} des stoïciens —, ce n’est que la
durée de tout, autrement dit la continuation indéfinie de l’univers, qui
demeure toujours le même, comme disait à peu près Spinoza, bien qu’il ne
cesse de changer en une infinité de manières. C’est la seconde acception du
mot : non plus une pensée, mais l'être même de ce qui dure et passe. Non la
somme d’un passé et d’un avenir, mais la perduration du présent. C’est le
temps de la nature ou de l’être : le devenir en train de devenir, le changement
continué des étants. Le passé ? Ce n’est rien de réel, puisque ce n’est plus.
L'avenir ? Ce n’est rien de réel, puisque ce n’est pas encore. Dans la nature, il
n’y a que du présent. C’est ce qu'avait vu Chrysippe (« seul le présent existe »),
et c’est ce que Hegel, à sa façon, confirmera : « La nature, où le temps est le
{\it maintenant}, ne parvient pas à différencier d’une façon durable ces dimensions
du passé et du futur : elles ne sont nécessaires que pour la représentation subjective,
le souvenir, la crainte ou l’espérance » ({\it Précis de l'Encyclopédie}, \s 259).
Comment mieux dire qu’elles ne sont nécessaires que pour l’esprit, point pour
le monde ? Le temps de l’âme n’est qu’une {\it distension}, comme disait saint
Augustin, entre le passé et l’avenir (c’est ce qu’on appelle la temporalité). Le
temps de la nature, qu’une tension ({\it tonos}), qu’un effort ({\it conatus}) ou un acte
— 579 —
({\it energeia}), dans le présent. Ces deux temps, toutefois, ne sont pas sur le même
plan : l’âme fait partie du monde, comme la mémoire et l'attente font partie du
présent. Le temps, dans sa vérité, est donc celui de la nature : ce n’est qu’un
perpétuel, quoique multiple et changeant, {\it maintenant}. C’est en quoi il ne fait
qu’un avec l'éternité.

Deux sens, donc : une abstraction ou un acte. La durée, abstraction faite de
ce qui dure, ou l’être même, en tant qu’il continue. Une pensée, ou un devenir.
La somme du passé et de l'avenir, qui ne sont rien, ou la continuation du présent,
qui est tout. Un non-être, ou l’être-temps. Ce qui nous sépare de l’éternité,
ou l'éternité même.

TEMPS PERDU C'est le passé, en tant qu’il n’en reste rien, ou le présent, en

tant qu’il n’est que l'attente de l’avenir. Aussi est-ce le
contraire de l'éternité. Misère de l’homme. Le temps perdu, c’est le temps
même.

TEMPS RETROUVÉ C’est une espèce d’éternité de la mémoire, où le temps

soudain se révèle (« un peu de temps à l’état pur », dit
Proust), dans sa vérité, et par [à (en cet instant «affranchi de l’ordre du
temps ») s’abolit. Voilà que le présent et le passé ne font qu’un, ou plutôt, pour
différents qu’ils demeurent (la madeleine dans le thé et la madeleine dans la
tisane sont deux), voilà qu’ils se rencontrent dans un même présent, qui est
celui de l'esprit, qui est celui de l’art, voilà qu’ils libèrent « l’essence permanente
et habituellement cachée des choses », qui est simplement leur vérité, toujours
présente, ou leur éternité. Car la vérité ne passe pas, tout est là, car le temps ne
passe pas (c’est nous, dirait Proust comme Ronsard, qui passons en lui), et cette
contemplation, quoique fugitive, est d’éternité. Le temps retrouvé est ainsi la
même chose que le temps perdu («la vraie vie, la vie enfin découverte et
éclaircie, la seule vie par conséquent réellement vécue. »), et pourtant son
contraire.

TENDANCE La direction d’un être ou d’un processus quelconque. Se dit
spécialement de l'orientation du désir, mais en tant qu’elle est
plutôt naturelle (à la différence de l’inclination) et collective (à la différence du
penchant). Disons que c’est la pente naturelle de l’espèce, comme le penchant
est celle de l’individu. L’équivalent à peu près de l’{\it hormè} des stoïciens (la tendance,
la pulsion), qui est comme un {\it conatus} biologique (« l’{\it hormè} fondamentale
%— 580 —
de tout être vivant est de se conserver soi-même » : Chrysippe, cité par
Diogène Laërce, VII, 85).

TENDRESSE La douceur pour ceux qu’on aime, et l’amour de cette douceur.

TERME (GRAND, MOYEN OU PETIT -) On appelle ainsi les trois éléments,
unis deux à deux et
intervenant chacun deux fois, d’un syllogisme. Dans l’exemple canonique,
« mortels » est le grand terme, « Socrate » le petit terme, et « hommes » le moyen
terme. Contrairement à ce que pourrait laisser croire cet exemple, ce n’est pas leur
extension qui les définit (ce ne serait pertinent que dans certains modes du syllogisme).
Le grand terme, qui est le prédicat de la conclusion, figure dans la
majeure ; le petit terme, qui est le sujet de la conclusion, dans la mineure ; enfin
le moyen terme figure dans les deux prémisses, mais pas dans la conclusion.

TERRORISME Ce n’est pas régner par la terreur, comme fait le despotisme,

mais combattre, par la terreur qu’on suscite, le règne d’un
autre. C’est utiliser la violence à des fins politiques, contre un pouvoir qu’on ne
peut vaincre démocratiquement ou militairement.

Le terrorisme est l’arme des faibles ; c’est ce qui peut parfois le justifier, mais
seulement au service d’une cause juste et contre un adversaire qu'on ne saurait
affronter autrement. Les nazis appelaient « terroristes » nos Résistants, et après
tout pourquoi pas ? Ceux-ci combattaient sans uniforme, ils faisaient exploser des
bombes, qui pouvaient tuer des civils, et plusieurs n'auraient pas hésité, s’ils
l'avaient pu, à semer la terreur à Berlin ou à Vienne. Mais ce qui peut être légitime
contre Hitler et en temps de guerre ne l’est pas contre un État démocratique
et en temps de paix. Cela laisse une marge d’appréciation (où commence la
démocratie ? où finit la paix ?), et c’est pourquoi la dénonciation du terrorisme ne
saurait tenir lieu d’analyse politique. Mais la politique, contre le terrorisme, ne.
suffit pas davantage. Les terroristes sont des combattants de l'ombre, qui ne se
soucient pas des lois de la guerre et qui n’hésitent pas à frapper, le cas échéant, des
civils ou des innocents. C’est une raison suffisante, en tout État démocratique,
pour le refuser et pour le combattre, y compris militairement. Contre le fanatisme,
la raison. Contre la violence aveugle, la force lucide.

{\it TETRAPHARMAKON} Mot grec, signifiant littéralement « quadruple remède »,
ou « remède à quatre ingrédients ». En philosophie,
il s’agit en l'occurrence de quatre maximes qu’un épicurien du {\footnotesize II}$^\text{\,e}$ siècle après
%— 581 —{\it }
Jésus-Christ — Diogène d’'Œnoanda — avait fait graver, pour l'édification des passants
et de la postérité, sur un mur, où elles furent en effet redécouvertes au {\footnotesize XIX}$^\text{\,e}$
siècle. Elles sont souvent citées depuis, à juste titre, comme l’un des meilleurs
résumés de la pensée d’Épicure, telle qu’elle apparaît aussi bien dans la {\it Lettre à
Ménécée} (dont je suis ici l’ordre) que dans les quatre premières {\it Maximes capitales} :
« Il n’y a rien à craindre des dieux ;

Il n’y a rien à craindre de la mort ;

On peut supporter la douleur ;

On peut atteindre le bonheur. »

On n’oubliera pas qu’il s’agit d’un remède philosophique, ou de la philosophie
comme remède : l'important n’est pas de le répéter (ce n’est pas un
mantra) mais de le {\it méditer}, comme dit Épicure ({\it Lettre à Ménécée}, \s 135),
autrement dit d’essayer de le comprendre et de le vivre.

THÉISME Toute doctrine qui affirme l'existence d’un Dieu personnel,
transcendant et créateur. C’est le contraire de l’athéisme. Le mot
ne faisant référence à aucune religion particulière, on peut dire aussi bien qu’il les
contient toutes (le christianisme ou l’Islam sont deux théismes en ce sens) ou
qu'il ne saurait se réduire à aucune. C’est ce qui explique que le mot, en pratique,
prenne souvent un sens plus déterminé ou plus polémique, qui est une croyance
en Dieu indépendante de quelque religion positive que ce soit, voire les récusant
toutes. C’est la position de Voltaire ou du Vicaire savoyard de Rousseau.

Par différence avec le déisme, le théisme suppose qu’on peut connaître —
que ce soit par analogie, par raisonnement ou par révélation — au moins
quelques-uns des attributs de Dieu (par exemple qu’il est tout-puissant, omniscient,
créateur, parfaitement bon et juste, aimant et miséricordieux..….). Le
déisme n’affirme qu’une existence ; le théisme croit connaître aussi, ou reconnaître,
bien sûr partiellement, une essence. La différence entre les deux reste
pourtant fluctuante et n’interdit pas les degrés intermédiaires. Le déisme est un
théisme vague. Le théisme, un déisme déterminé.

THÉISTE Celui qui croit en Dieu, spécialement s’il ne se reconnaît dans

aucune religion établie. Le théiste, explique Voltaire, « n’embrasse
aucune des sectes qui toutes se contredisent ». C’est un croyant sans rites, sans
Église, sans théologie. « Faire le bien, voilà son culte ; être soumis à Dieu, voilà
sa doctrine. Le mahométan lui crie : “Prends garde à toi si tu ne fais pas le pèlerinage
de La Mecque !” “Malheur à toi, lui dit un récollet, si tu ne fais pas un
voyage à Notre-Dame-de-Lorette !” Il rit de Lorette et de La Mecque ; mais il
%— 582 —
secourt l’indigent et il défend l'opprimé » ({\it Dictionnaire}, article « Théiste »).
Sa foi serait donc une morale ? Pas seulement, puisqu’un athée peut se passer
de celle-là sans renoncer à celle-ci. Le théiste ne se contente pas de faire le bien ;
il croit que c’est le Bien qui l’a fait et qui doit au bout du compte le juger. C’est
pourquoi il se soumet à Dieu, comme dit Voltaire. Mais pourquoi faudrait-il se
soumettre à ce qu’on ne comprend pas ?

THÉODICÉE C’est un mot forgé par Leibniz, qui en fit le titre d’un de ses
livres ({\it Essais de théodicée : Sur la bonté de Dieu, la liberté de
l’homme et l'origine du mal}). Il exprime moins la justice de Dieu, malgré l’étymologie
({\it dikè}, en grec, c’est la justice), que sa justification par nous. C’est une
espèce de plaidoirie. Il s’agit de montrer que Dieu est innocent, comme disait
Platon, et que l'existence du mal n’est pas un argument insurmontable contre
son existence et sa bonté. Le livre, quoique moins éblouissant que le {\it Discours de
métaphysique}, est un chef-d'œuvre. Mais c’est un chef-d'œuvre agaçant, par la
volonté de justifier l’injustifiable. À ne pas lire quand on souffre trop. Cela rendrait
injuste avec Leibniz.

THÉOLOGALES (VERTUS -) Ce sont les trois vertus principales de la tradition
chrétienne, qui touchent moins à la
morale qu’à la religion : la foi, l'espérance, la charité. On les dit {\it théologales}
parce qu’elles auraient Dieu même pour objet. On remarquera avec saint Paul,
qui ne les appelle pas ainsi, que la « plus grande des trois », et la seule qui « ne
passera pas », est la charité (I, Co, 13). C’est suggérer ce que saint Augustin et
saint Thomas diront expressément : que la foi et l'espérance n’ont de sens que
provisoire et pour autant seulement que nous sommes séparés du Royaume. Au
paradis, elles seront obsolètes. Il n’y aura plus lieu de croire en Dieu, puisque
nous Le verrons face à face, et plus rien à espérer. D'ailleurs, remarque saint.
Thomas, « il y eut dans le Christ une charité parfaite, et il n’eut cependant ni
la foi ni l'espérance » ({\it Somme théologique}, Ia IIæ, quest. 65, 5). C’est qu’il était
Dieu, bien sûr, et que Dieu n’a pas à croire ni à espérer quoi que ce soit
(puisqu'il est à la fois omniscient et omnipotent). Cela n’en donne pas moins
un sens singulier, et singulièrement fort, à ce qu’un livre fameux appelait, c’est
son titre, « limitation de Jésus-Christ ». Comment imiter en lui ce qu’il n’avait
pas ? Cela laisse une chance aux athées. C’est l’amour qui sauve, point la foi,
point l'espérance. C’est ce qu’on peut appeler le Royaume, où rien n’est à
croire, puisque tout est à connaître, où rien n’est à espérer, puisque tout est à
%— 583 —
faire ou à aimer. Nous y sommes : le Royaume est en nous, comme dit Jésus,
ou nous en lui, et il n’y en a pas d’autre.

THÉOLOGIE La « science » de Dieu ? Même avec des guillemets, l’expression
serait contradictoire. La théologie est moins une science
qu'une étude : c’est un discours rationnel (un {\it logos}), tenu par des hommes,
mais portant sur le divin. Elle s’appuie le plus souvent, du moins dans les religions
révélées, sur ce que Dieu est censé avoir dit de lui-même. S'il avait été
plus clair, la théologie n’existerait pas ou serait inutile.

On parle de théologie apophatique, ou négative, quand elle ne procède
que par négations : non en disant ce que Dieu est (ce qui serait le ramener à
nos catégories humaines), mais en disant ce qu’il n’est pas. C’est un antidote
contre l’anthropomorphisme. L’athéisme en est un autre, plus simple et plus
efficace.

THÉOLOGIEN Celui qui consacre sa vie à l’étude de Dieu, c’est-à-dire, en
pratique, à l'étude de ce que les hommes en ont dit ou, s’il
croit à la Révélation, de ce que Dieu aurait dit de lui-même. Les résultats sont
impressionnants par la masse, par l'intelligence, par l’érudition, parfois par la
profondeur. Toutefois Dieu, après ces dizaines de milliers de pages, n’en reste
pas moins incertain et incompréhensible. « J’ai connu un vrai théologien, rapporte
Voltaire : plus il fut véritablement savant, plus il se défia de tout ce qu'il
savait. [...] À sa mort, il avoua qu’il avait consumé inutilement sa vie. »
Anticléricalisme ? Peut-être. Mais saint Thomas lui-même, tout à la fin de sa
vie, écrivit à son ami Réginald, qui s’enquérait de ses travaux, ceci : « Je ne puis
plus écrire. J'ai vu des choses auprès desquelles mes écrits ne sont que de la
paille. » Ce qu’il avait vu, nul ne le sait. Reste la paille.

THÉORÈME Une proposition démontrée, à l’intérieur d’un système hypothético-déductif.
En philosophie, cela n’existe donc pas : on
parlera plutôt de {\it thèses} (voir ce mot).

THÉORÉTIQUE Le mot, calqué du grec, fait presque toujours référence à
son usage aristotélicien. Est {\it théorétique} ce qui relève de la
{\it théôria}, c’est-à-dire de la connaissance pure ou désintéressée. Les {\it sciences théorétiques}
— mathématique, physique, théologie — sont celles qui se contentent de
%— 584 —
connaître (par différence avec les sciences pratiques ou poiètiques, qui servent
à l’action ou à la productiori : {\it Métaphysique}, E, 1). L’{\it intellect théorétique} est
celui qui connaît ou contemple, indépendamment de quelque action que ce
soit ; il « ne pense rien qui ait rapport à la pratique, et n’énonce rien sur ce qu’il
faut éviter ou poursuivre » ({\it De anima}, III, 9). La {\it vie théorétique}, qui est le
sommet de la sagesse et du bonheur ({\it Éthique à Nicomaque}, X, 7-8), est la vie
contemplative : c’est à la fois une activité (« l’activité de l’intellect ») et une joie
(« la joie de connaître », X, 7).

{\it THÉÔRIA} Mot grec, signifiant {\it vision} ou {\it contemplation}. Contrairement à ce
qu’on croit parfois, l’étymologie ne renvoie pas à la vision qu’on
aurait de Dieu ({\it théos}), mais simplement à la contemplation (oros, celui qui
observe) d’un spectacle ({\it théa}, qui donnera notre théâtre). Il n’en reste pas
moins que le spectacle en question est d’abord un oracle ou une fête religieuse,
et que le mot, à son origine, est associé par là, ou peut l'être, à la religion.
Platon nommera ainsi la contemplation des Idées. Mais la fortune du mot doit
surtout à Aristote, qui y verra l’activité propre de l’intellect, donc le sommet du
bonheur et de la vertu: l’activité contemplative, c’est-à-dire la «joie de
connaître », est « le parfait bonheur de l’homme » ({\it Éthique à Nicomaque}, X, 7)
et la seule activité de Dieu ({\it ibid.}, X, 8).

THÉORICISME C'est accorder trop de crédit à la pensée théorique ou
abstraite : croire par exemple qu’elle suffira à changer le
monde, la vie, les hommes. C’est l’inverse de l’activisme, et une autre faute.

THÉORIE Le mot, en français, s’est éloigné de son étymologie grecque (voir

l'article « {\it Théôria} »). La théorie, pour nous, relève moins de la
contemplation que du travail, moins de la joie de connaître que de l'effort de
penser. Qu’est-ce qu’une théorie ? Un ensemble, en principe cohérent, de concepts
et de propositions, qui vise à produire un effet de connaissance ou à
rendre compte d’au moins une partie du réel. Si les propositions qui la composent
sont des axiomes et des théorèmes, il s’agit d’une théorie hypothético-déductive.
Si ce sont des hypothèses vérifiées ou falsifiables, d’une théorie
inductive ou expérimentale. Elle n’en est pas moins abstraite dans les deux cas.
Cela ne veut bien sûr pas dire qu’elle soit déconnectée de toute pratique. Il faut
au contraire y voir, soulignait Althusser, « une forme spécifique de la pratique,
appartenant elle aussi à l’unité complexe de la “pratique sociale” d’une société
%— 585 —
humaine déterminée ». C’est pourquoi on peut parler de {\it pratique théorique} : « La
pratique théorique rentre sous la définition générale de la pratique [comme processus
de transformation]. Elle travaille sur une matière première (des représentations,
concepts, faits) qui lui est donnée par d’autres pratiques » (qu’elles soient
empiriques, techniques ou idéologiques) et les transforme ({\it Pour Marx}, VI, 1).

THÈSE Une proposition indémontrable, mais qui peut faire l’objet d’une
argumentation. Se dit spécialement, dans une démarche dialec-
tique, d’une proposition considérée comme première, par rapport et par opposition
à une deuxième (l'antithèse), en attendant qu’une troisième (la synthèse)
vienne dépasser leur contradiction.

Le mot, dans son usage philosophique contemporain, doit beaucoup à
Louis Althusser. Une thèse, explique-t-il, n’est pas une proposition scientifique :
elle ne relève pas de la connaissance, mais de la pratique ; elle a moins
un {\it objet} qu’un {\it enjeu} ; elle n’est ni vraie ni fausse (je dirais plutôt : ni démontrable
ni falsifiable), mais elle peut être {\it juste} ou non. Parce qu’elle correspondrait
à la justice ? Non pas ; mais parce qu’elle fait preuve de {\it justesse} (c’est-à-dire
d’un rapport opératoire à la pratique, fût-ce à la pratique théorique) et
peut faire l’objet de « justifications rationnelles » ({\it Philosophie et philosophie
spontanée des savants}, I et II). C’est l'élément de base d’une philosophie : une
position pratique (une thèse doit transformer quelque chose ou produire
quelque effet) dans la théorie.

TIERS EXCLU (PRINCIPE DU -) Il stipule que, de deux propositions contradictoires,
l’une est vraie et l’autre fausse,
nécessairement, ce qui exclut toute autre possibilité. {\it P} ou {\it non-P}. Dieu est un
tuyau d’arrosage ou Dieu n’est pas un tuyau d’arrosage. Cela n’exclut pas qu’il
soit autre chose, mais que cet {\it autre chose} puisse offrir une troisième issue à la
question de savoir s’il est ou non un tuyau d’arrosage. Le principe du tiers-exclu,
contrairement à ce qu’on croit parfois, n’interdit ni la finesse, ni la subtilité,
ni les compromis, ni la complexité. Il interdit la confusion et la bêtise.
Le principe du tiers exclu entraîne que deux propositions contradictoires ne
peuvent être fausses toutes les deux (alors que le principe de non-contradiction
entraîne qu'elles ne peuvent être vraies toutes les deux), de telle sorte que la
fausseté de l’une suffit à prouver la vérité de l’autre. C’est ce qui fonde les raisonnements
par l'absurde, où le vrai brille encore, jusque dans son absence.
{\it Verum index sui}, disait Spinoza, {\it et falsi}. Le faux, miroir du vrai.

%— 586 —
TIMIDITÉ C’est une sensibilité exagérée au regard de l’autre, comme une
peur d’être jugé, comme une honte d’être soi, mais sans culpabilité,
et sans autre raison de rougir ou de bafouiller que cette rougeur même ou
cet embarras de la parole. Les uns le ressentent surtout devant une foule ;
d’autres, dont je suis, dans le tête-à-tête. C’est peut-être que les premiers craignent
surtout d’être vus ; les seconds, d’être devinés.

TOLÉRANCE Tolérer, c’est laisser faire ce qu’on pourrait empêcher ou
punir. Cela ne vaut pas comme approbation, ni même
comme neutralité. Ce comportement que je tolère (le sectarisme, la superstition,
la bêtise...), je peux aussi le combattre, en moi ou en autrui. Mais je
m'interdis de l’interdire : je ne le combats que par les idées, point par la loi ou
la force. C’est aimer la liberté plus que son propre camp, le débat plus que la
contrainte, la paix plus que la victoire.

Doit-on tout tolérer ? Bien sûr que non, puisqu'il faudrait pour cela tolérer
l'intolérance, y compris quand elle menace la liberté, et laisser les plus faibles
sans défense : ce serait abandonner le terrain aux fanatiques et aux assassins !

Il y a de l’intolérable : c’est tout ce qui rendrait la tolérance suicidaire ou
coupable.

Tolérance n’est ni laxisme ni faiblesse. Il n’est pas interdit d’interdire, mais
seulement d'interdire ce qui doit être protégé (la liberté de conscience et
d'expression, le libre affrontement des arguments et des idées...) ou ce qu'on
pourrait combattre, sans danger pour la liberté, autrement qu’en l’interdisant.
On dira que cela laisse, en pratique, une marge importante d’appréciation. Cela
même doit être toléré. Dans quelles limites ? Celles de l’État de droit. Pourquoi
interdire un groupuscule totalitaire, tant qu’il n’agite que des idées ou des
imbéciles ? Mais qu’il viole la loi ou verse dans le terrorisme, voilà qui appelle
une sanction immédiate. Le tolérer, ce serait s’en rendre complice.

TOPIQUE Qui concerne les lieux ({\it topoi}), et notamment les lieux communs

(au sens non péjoratif du terme : c’est en ce sens qu’on parle des
{\it Topiques} d’Aristote, pour désigner l’un des six traités de sa logique ou de son
{\it Organon}, en l'occurrence celui consacré aux lieux communs de l’argumentation
dialectique). Cette acception n’a plus d’usage qu’historique. Au sens moderne
du terme, une {\it topique} est une espèce de schéma, ou plutôt de modèle schématisable,
qui sert à représenter dans l’espace, et comme différents lieux, ce qui
n'existe ou n’est connu que de façon non spatiale. Il est arrivé à Althusser d’utiliser
le mot à propos de Marx (concernant la distinction entre l’infrastructure
%— 587 —
et la superstructure, ainsi que les différents niveaux de l’une et de l’autre),
comme à moi à propos des quatre ordres (voir l’article « Distinction des
ordres ») qui me paraissent structurer toute vie sociale. Mais le mot, dans son
usage ordinaire, est presque toujours d'inspiration psychanalytique. Freud proposa
en effet, à quelque vingt ans d’intervalle, deux modèles différents pour
penser les différents « lieux » de l'appareil psychique. La première topique distingue
le {\it conscient}, le {\it préconscient} et l'{\it inconscient} ; la seconde, qui n’est ni superposable
à la première ni incompatible avec elle, distingue le {\it ça}, le {\it moi} et le
{\it surmoi} (voir ces mots). La première est surtout descriptive ; la seconde, davantage
explicative.

TORTURE C'est imposer à quelqu'un, volontairement, une souffrance
extrême, parfois par pure cruauté, plus souvent pour en obtenir
quelque aveu ou dénonciation. Comportement spécifiquement humain, qui en
dit long sur notre espèce. On évitera pourtant d’en faire un argument en faveur
de la misanthropie. Ce serait donner raison aux tortionnaires, contre leurs victimes,
et tenir pour rien l’héroïsme de celles, même rares, qui sont mortes sans
parler.

TOTALITARISME Pouvoir total (d’un parti ou de l’État) sur le tout (d’une
société) : c’est la forme moderne et bureaucratique de la
tyrannie. Le totalitarisme constitue un système politique où tous les pouvoirs
appartiennent en fait à un même clan, qui impose partout son idéologie, son
organisation, ses hommes. Règne ordinairement au nom du bien et du vrai ;
gouverne par le mensonge et la terreur.

Le mot, qui est apparu dès les années 20, sert surtout à désigner ce que les
dictatures fascistes et communistes pouvaient avoir en commun : un parti de
masse, une idéologie d’État, un contrôle absolu des moyens d’information et de
propagande, la suppression des libertés individuelles, l’absence de toute vraie
séparation entre les pouvoirs, un régime inquisitorial et policier, qui débouche
sur la terreur et culmine dans les camps. Ces traits communs, qui sont incontestables,
ne signifient pas que ces deux régimes soient foncièrement identiques,
pas plus que leurs différences, qui sont tout aussi incontestables, ne sauraient
annuler ces convergences objectives. Nazisme et stalinisme sont-ils
comparables ? Bien sûr, puisqu'ils ont des traits communs, et puisqu’on ne
pourrait répondre non qu’à la condition de les comparer d’abord ! Sont-ils
identiques ? Bien sûr que non, puisqu'ils cesseraient alors d’être deux et de pouvoir
%— 588 —
être comparés ! Moyennant quoi le débat, sur leurs ressemblances et différences,
peut durer toujours.

Les convergences sont surtout objectives et organisationnelles ; les différences,
surtout subjectives et idéologiques. Les deux systèmes s’imposent à peu
près de la même façon, mais au nom d’idéologies opposées. Non, certes, parce
que l’un aurait fait le mal pour le mal, comme le croient les naïfs, quand l’autre
ne l’aurait fait qu’au nom du bien, voire par accident ou par erreur. Le nazisme,
pour un nazi, est un bien, et l’enfer totalitaire, qu’il soit de droite ou de
gauche, n’est pavé, comme dit Todorov, que de bonnes intentions. Si ces deux
systèmes s'opposent, ce n’est pas comme le bien et le mal, mais comme deux
conceptions opposées du bien, qui débouchent sur deux maux parallèles. L’un
veut imposer le pouvoir d’une race ou d’un peuple, sur d’autres races ou sur
d’autres peuples. L'autre, le pouvoir d’une classe sur d’autres classes, mais
pour les abolir toutes : pour qu’il n’y ait plus que humanité heureuse et libre.
La pensée du premier est essentiellement biologique, hiérarchique, guerrière ;
celle du second, essentiellement historique, égalitaire, universaliste. C’est ce
qui rend le communisme plus sympathique, plus trompeur (lécart entre son
discours et ses actes est plus grand), et plus dangereux peut-être. Du moins
c’est ce qui peut sembler, une fois que le nazisme a été vaincu par la force. Le
communisme ne le sera que par la fatigue, l'impuissance et le ridicule — que
par lui-même.

TOTALITÉ Tous les éléments d’un ensemble, mais en tant que celui-ci constitue
une unité. La totalité, qui est chez Kant l’une des trois catégories
de la quantité, est ainsi définie par lui à partir des deux autres, dont elle
opère la conjonction : une {\it totalité}, c’est l'{\it unité} d’une {\it pluralité}. C'est ce qui
autorise à parler de {\it la} totalité comme d’un absolu : ce serait l’unité de toutes les
pluralités (l’ensemble de tous les ensembles : la {\it summa summarum} de Lucrèce).
Mais nous n’en avons d’expérience, par définition, que partielle.

TOUT Le sens du substantif change en fonction de l’article. {\it Un} tout, c’est
un ensemble unifié ou ordonné. {\it Le} Tout, surtout quand on l'écrit
avec une majuscule, c’est l’ensemble de tous les ensembles. On pourra dire par
exemple qu’un monde est un tout, qu’il peut en exister une infinité, dont
l’ensemble serait le Tout. En ce dernier sens, qui correspond au {\it to pan} d’Épicure
ou à la {\it summa summarum} de Lucrèce (la somme des sommes), c’est un
synonyme d’univers, au sens philosophique du terme. L'idée qu’il en existe plusieurs
serait contradictoire.

%— 589 —
TRAGIQUE Ce n’est pas le malheur ou le drame. Ce n’est pas la catastrophe
— ou bien c’est la catastrophe humaine, celle d’être soi, celle de
se savoir mortel. Le tragique, c’est tout ce qui résiste à la réconciliation, aux
bons sentiments, à l’optimisme béat ou bélant. C’est la contradiction insoluble,
mais existentielle plutôt que logique (par exemple entre notre finitude et notre
désir d’infini). C’est une espèce de dialectique, si l’on veut, mais sans synthèse
ni dépassement — une dialectique sans pardon. C’est le divorce, mais sans
réconciliation. C’est le conflit sans issue, en tout cas sans issue satisfaisante,
entre deux points de vue l’un et l’autre légitimes, du moins dans leur ordre, et
qui n’en sont que davantage opposés. Par exemple le conflit entre les lois de
l’État et celles de la conscience (Antigone), entre le destin et la volonté
(Œdipe), entre les dieux et les hommes (Prométhée), entre la passion et le
devoir (spécialement chez Corneille), ou entre deux passions (spécialement
chez Racine). Je ne prends ces exemples littéraires que par commodité. La
tragédie, comme genre littéraire, se nourrit du tragique ; elle ne l’épuise pas.
C’est qu’il est une dimension de la condition humaine et de l’histoire : l’incapacité
où nous sommes de trouver, même intellectuellement, une solution pleinement
satisfaisante au problème que constitue, au moins à nos yeux, notre
existence. C’est pourquoi la mort est tragique. C’est pourquoi la vie est tragique.
Parce qu'elles nous confrontent toutes deux à l'impossible ou à
l'absurde, à l’inacceptable, à l’inconsolable. Parce que toute vie est l’histoire
d’un échec, comme dit Sartre, et parce que la mort en est un autre. Parce que
toute vie est un combat, mais sans victoire ni repos.

Cela vaut aussi pour les peuples. « Le tragique, disait un homme politique,
c’est quand tout le monde a raison : la situation au Proche-Orient est
tragique. » Les Palestiniens ont raison de vouloir vivre chez eux, de vouloir
leur indépendance, leur souveraineté, leur État. Les Israéliens ont raison de
vouloir leur sécurité. Mais on ne voit pas comment ces deux légitimités pourraient
n’en faire qu’une, ni même cohabiter sans renoncer l’une et l’autre à
une part — au moins une part — de leur bon droit. Il faudra donc des
compromis : il faudra accepter une solution qui ne sera pleinement satisfaisante
pour personne, qui ne sera pas juste, mais qui vaudra mieux pourtant
que la guerre et le terrorisme. Ce sera sortir de la tragédie, non du tragique.
De la guerre, non du conflit. De la haine, non de l’amertume. Le tragique est
le goût même du réel — parce qu’il ne nous obéit pas, parce qu’il n’est jamais
tout à fait à notre goût.

On remarquera que la catastrophe nazie n’était pas {\it tragique}, en ce sens.
Elle ne pourrait sembler l’être qu’à ceux qui jugeraient le nazisme légitime,
ou nieraient qu'une issue parfaitement satisfaisante était envisageable, au
moins en théorie, au moins en droit, qui était la défaite, le plus tôt possible,
%— 590 —
du nazisme. Mais que la raison et le droit n’aient pu y suffire, cela confirme
ce qu’il y a d’irréductiblement tragique dans l’histoire humaine : qu’il ne
suffit pas d’avoir raison pour vaincre, que le droit ne peut rien sans la force,
qu’on ne peut combattre le mal, presque toujours, que par un autre mal (la
violence, la guerre, la répression), certes moindre, mais qui ne saurait non
plus tout à fait nous satisfaire. Hitler n’est pas un personnage tragique. Ni
ceux, s’il en existe, qui n’opposèrent au nazisme que leurs bons sentiments.
Mais Churchill, si. Mais Cavaillès, si. Le tragique, ce n’est pas le conflit entre
le bien et le mal ; c’est le conflit entre deux biens, ou entre deux maux.

On parle de {\it philosophie tragique} lorsqu’une pensée, loin de vouloir nous
satisfaire totalement, nous confronte à de l’inacceptable, à de l’injustifiable, à ce
que Clément Rosset appelle la {\it logique du pire} (contre la logique du meilleur de
Leibniz) et nous voue par là à l’insatisfaction ou au combat. Ainsi Pascal ou
Nietzsche. La frontière peut traverser une même école, voire un même individu.
Par exemple Lucrèce est un penseur tragique ; Épicure, non. Marc Aurèle
est un penseur tragique ; Épictète, non. Spinoza et Kant le sont parfois, point
toujours. À chacun de trouver ses maîtres, en fonction du tragique qu’il accepte
ou requiert.

Le grand théoricien du tragique, et l’un de ses plus illustres représentants,
est évidemment Nietzsche. Qu'est-ce que le tragique ? C’est la vie telle qu’elle
est, sans justification, sans providence, sans pardon, c’est la volonté de
l’affirmer toute, de l’accepter toute, avec la souffrance dedans, avec la joie
dedans, sans ressentiment, sans mauvaise conscience, sans nihilisme, c’est
l'amour du destin ou du hasard, du devenir et de la destruction, c’est « {\it le oui
par excellence} », sans religion, sans « moraline », c’est le sentiment que le réel est
à prendre ou à laisser, joint à la volonté joyeuse de le prendre. « L'artiste tragique
n’est pas un pessimiste, il dit {\it oui} à tout ce qui est problématique et terrible,
il est {\it dionysien} » ({\it Le crépuscule des idoles}, III, 6). C’est qu’il aime la vie
comme elle est, comme elle vient, comme elle passe ({\it amor fati}). C’est qu’il n’a
pas besoin d’autre chose, ni espérance ni consolation. C’est qu’il n’a même pas
besoin d’y croire tout à fait. Le contraire du tragique, ce n’est pas le comique,
c’est l’esprit de sérieux.

TRANSCENDANCE C'est l’extériorité et la supériorité absolues : l’ailleurs
de tous les ici (et même de tous les ailleurs), et leur
dépassement. L'absence suprême, donc, qui serait aussi le comble de la présence
— le point de fuite du sens. Car « le sens du monde, écrit Wittgenstein,
doit se trouver en dehors du monde ». La transcendance est ce {\it dehors} ou le suppose.
C’est le Royaume absent, qui nous voue à l’exil.

%— 591 —
Ce sens premier ou général est susceptible de variations. Est transcendant
tout ce qui se trouve {\it au-delà de}. Mais au-delà de quoi ? Au-delà de la conscience
(c’est le sens phénoménologique : l'arbre que je perçois n’est pas {\it dans} la
conscience ; il est un objet transcendant {\it pour} la conscience) ; au-delà de l’expérience
possible (c’est le sens kantien) ; au-delà du monde ou de tout (c’est le
sens classique).

Peut désigner aussi le mouvement qui y mène, comme un dépassement,
mais réservé au {\it Dasein}, de tout donné ou de toute limite. La liberté, spécialement,
serait ce pouvoir de {\it transcender} toute situation, tout conditionnement,
tout déterminisme. Ce serait une façon d’être extérieur à sa propre histoire, à
son propre corps, à sa propre situation, ou de pouvoir en sortir. Il y a du
miracle dans la transcendance, dès lors qu’elle prétend s’expérimenter de l’intérieur
ou ici-bas.

TRANSCENDANT Au sens classique : ce qui est extérieur et supérieur au
monde. Dieu, en ce sens, serait transcendant, et lui seul
peut-être.

Chez Kant, ce qui est extérieur à l'expérience, et hors de sa portée.

Chez Husserl et les phénoménologues, ce qui est extérieur à la conscience,
vers quoi elle se projette ou « s'éclate ». C’est en ce sens que Sartre parle d’une
{\it transcendance de l'ego} : le Moi ne fait pas partie de la conscience, il n’est que
l’un de ses objets ; il n’est pas {\it de} la conscience, mais {\it pour} la conscience ; il n’est
pas « dans la conscience », mais « dehors, dans le monde ».

Dans la philosophie contemporaine, on parle aussi de {\it transcendance} pour
désigner tout ce qui est irréductible à la matière, à la nature ou à l’histoire. Pour
Luc Ferry, par exemple, l'homme est un être transcendant, non parce qu’il
serait extérieur au monde ou à la société, mais parce qu’il ne saurait totalement
s’y réduire : il y a en lui une capacité d’excès, d’arrachement, de liberté absolue,
et cette « transcendance dans l’immanence » (l'expression est de Husserl) en fait
une espèce de Dieu, dont l’humanisme serait la religion.

Pour le matérialiste, au contraire, rien n’est transcendant : il n’y a que la
nature, il n’y a que l’histoire, il n’y a que tout, dont l’homme fait partie et ne
saurait se libérer totalement.

TRANSCENDANTAL Ce n’est pas un synonyme de transcendant.

Le mot est d’origine scolastique : il désigne les attri-
buts qui {\it transcendent} les genres de l’être ou les catégories d’Aristote, c’est-à-dire
%— 592 —
qui les dépassent et peuvent pour cela convenir à tout être : l’Un, le Vrai, le
Bien et l’Être lui-même sont des transcendantaux.

Mais aujourd’hui, et depuis deux siècles, le mot est presque toujours pris en
un sens kantien : est {\it transcendantal} tout ce qui concerne les conditions a priori de
l'expérience, ainsi que les connaissances qui, prétendument, en découlent. C’est
l’inempirique de lempiricité. «J'appelle {\it transcendantale}, écrit Kant, toute
connaissance qui, en général, s’occupe moins des objets que de nos concepts {\it a
priori} des objets », ou, précise la seconde édition, que « de notre manière de les
connaître, en tant que ce mode de connaissance doit être possible {\it a priori} »
({\it C. R. Pure}, Introd., VII). Rien à voir donc avec le transcendant, auquel le transcendantal
s’opposerait plutôt. Est transcendant, en effet, tout ce qui est {\it au-delà}
de l’expérience ; transcendantal, tout ce qui est ex deçà, et qui la permet. « Le mot
{\it transcendantal}, insiste Kant, ne signifie pas quelque chose qui s’élève au-dessus de
toute expérience, mais ce qui certes la précède ({\it a priori}) sans être destiné cependant
à autre chose qu’à rendre possible uniquement une connaissance empirique.
Si ces concepts dépassent l’expérience, leur usage se nomme transcendant, et il est
distingué de l’usage immanent, c’est-à-dire borné à l’expérience » ({\it Prolégomènes…},
Appendice). Transcendant s'oppose à immanent, mais de l’extérieur :
ce qui est transcendant n’est pas immanent, ce qui est immanent n’est pas transcendant.
Transcendantal s'oppose à empirique, mais de l’intérieur. Dans une
connaissance empirique, dès lors qu’elle est nécessaire et universelle, il y a forcément
du transcendantal, c’est-à-dire de l’inempirique. « Que toute connaissance
commence {\it avec} l'expérience, cela n’entraîne pas qu’elle vienne toute {\it de}
l'expérience » ({\it C. R. Pure}, Introd., I). Par exemple quand je dis que 7 + 5 = 12 ou
que tout fait a une cause. En tant que ces connaissances sont universelles et nécessaires,
elles ne sauraient dériver tout entières de l’expérience : il faut qu’elles aient
une source {\it a priori} (l'entendement, ses catégories, ses principes). Même chose
pour l’espace et le temps : ce sont des formes {\it a priori} de la sensibilité, qui rendent
l'expérience possible et ne sauraient pour cela en résulter. Leur {\it idéalité transcendantale}
(le fait qu’ils n’existent qu’à titre de conditions subjectives de l'intuition.
sensible) est le gage de leur {\it réalité empirique} (pour tout objet d’une expérience
possible), mais n’en relève pas : les conditions {\it a priori} de l'expérience ne sauraient
par définition être objets d’expérience. C’est en quoi le transcendantal fait
comme une transcendance paradoxale, à l’intérieur même de l’immanence — d’où
certains voudront tirer une religion de l’homme ou de l'esprit (c’est ce que Luc
Ferry, reprenant une expression de Husserl, appelle «la transcendance dans
l’immanence »). Mais cela Kant, à ma connaissance, ne l’a jamais fait. Le vrai
Dieu pour lui est transcendant, ce qui exclut qu’on puisse le connaître, certes,
mais aussi que le transcendantal soit Dieu.

%— 593 —
TRANSFERT C'est un déplacement : un changement de lieu ou d’objet. Se
dit spécialement, en psychanalyse, du report d’un certain
nombre d’affects inconscients’ (désir ou rejet, amour ou haine) sur une autre
personne que celle qui les suscita à l’origine, spécialement pendant la petite
enfance. Ce processus, qui est omniprésent (« il s’établit spontanément, écrit
Freud, dans toutes les relations humaines »), agit pourtant de façon plus spectaculaire
durant la cure analytique : le patient déverse sur son analyste « un trop
plein d’excitations affectueuses, souvent mélées d’hostilité, qui n’ont leur
source ou leur raison d’être dans aucune expérience réelle ; la façon dont elles
apparaissent, et leurs particularités, montrent qu’elles dérivent d’anciens désirs
du malade devenus inconscients » ({\it Cinq leçons...}, V). De là une espèce d’accélération
du travail psychique, dans lequel le psychanalyste, grâce au transfert,
joue le rôle d’un « ferment catalytique, qui attire temporairement sur lui les
affects qui viennent d’être libérés » ({\it ibid.}). C’est dire que les patients accordent
presque toujours trop d'intérêt à leur analyste, jusqu’à en être, quand ils en parlent,
quelque peu ridicules ou fatigants. Mais cette survalorisation fait partie de
la cure (ils lui accordent trop d'importance, mais cela même lui en donne),
comme le retour à la lucidité, quand il advient, fait partie de la guérison.

TRANSSUBSTANTIATION Cest la première fois, sauf oubli de ma part,
que j'écris ce mot. Que Voltaire ait jugé bon
de le mettre dans son Dictionnaire en dit long sur les enjeux de son époque, et
sur ce qui nous en sépare. {\it Sic transit gloria Dei...}

Qu'est-ce que la transsubstantiation ? La transformation d’une substance
en une autre, et spécialement la transformation du pain et du vin, lors de
l'eucharistie, en corps et sang du Christ. Croyance absurde ? C’est ce que pensait
Voltaire, qui s’étonnait de « ce vin changé en sang, et qui a le goût du vin,
de ce pain changé en chair, et qui a le goût du pain », enfin de ces croyants qui
« mangent et boivent leur dieu, qui chient et pissent leur dieu ».. Étonnement
compréhensible, devant l’incompréhensible. Mais que serait une religion,
répondrait Pascal, qui n’étonnerait pas et qu’on pourrait comprendre ? C’est
qu'il s’agit moins d’une absurdité, pour les catholiques, que d’un mystère ou
d’un miracle. Et après tout, pourquoi pas ? Si Dieu a pu créer le monde, on
aurait tort de le chipoter sur les détails. On s’étonne que Voltaire, qui trouve la
création si plausible, soit tellement choqué par la présence réelle du Christ dans
le pain et le vin. Créer quelque chose à partir de rien, cela fait, me semble-t-il,
une transsubstantiation autrement étonnante.

%— 594 —
TRAVAIL C’est une activité fatigante ou ennuyeuse, qu’on fait en vue d’autre
chose. Qu’on puisse l’aimer ou y trouver du plaisir, c’est entendu.
Mais ce n’est un travail, non un jeu, que parce qu’il ne vaut pas par lui-même,
ni pour le seul plaisir qu’on y trouve, mais en fonction d’un résultat qu’on en
attend (un salaire, une œuvre, un progrès.) et qui justifie les efforts qu’on lui
consacre. Ce n’est pas une fin en soi : ce n’est qu’un moyen, qui ne vaut qu’au
service d’autre chose. C’est ce que prouvent les vacances et le salaire.
Travailler ? Il le faut bien ? Mais qui le ferait gratuitement ? Qui ne désire le
repos, les loisirs, la liberté ? Le travail, pris en lui-même, ne vaut rien. C’est
pourquoi on le paie. Ce n’est pas une valeur. C’est pourquoi il a un prix.

Une valeur, c’est ce qui vaut par soi. Ainsi l’amour, la générosité, la justice,
la liberté... Pour aimer, vous demandez combien ? Ce ne serait plus amour
mais prostitution. Pour être généreux, juste, libre, il faut qu’on vous paie ? Ce
ne serait plus générosité mais égoïsme, plus justice mais commerce, plus liberté
mais esclavage. Pour travailler ? Vous demandez quelque chose, vous avez évidemment
raison, et vous pestez, bien souvent, de ne pas obtenir davantage.

Une valeur, c’est ce qui n’est pas à vendre. Comment aurait-elle un prix ?
C’est une fin, pas un moyen. À quoi bon aimer ? À quoi bon être généreux,
juste, libre ? Il n’y a pas de réponse. Il ne peut y en avoir. À quoi bon travailler ?
Il y a une réponse, ou plutôt il y en a plusieurs excellentes : pour gagner sa vie,
pour être utile, pour s'occuper, pour s’épanouir, pour s'intégrer dans la société,
pour montrer de quoi on est capable... Même le bénévole n’y échappe pas. S’il
travaille, c’est pour autre chose que le travail (pour le plaisir, pour le groupe,
pour une certaine idée de l'humanité ou de soi...). Cela met le travail à sa
place, qui n’est pas la première.

On m'objectera le chômage de longue durée, et certes je ne conteste pas
qu'il y ait là une tragédie. Mais point du tout, comme on le dit parfois, parce
que le chômeur y perdrait sa dignité. Où avez-vous vu que la dignité d’un
homme dépende de son travail ? Et pourquoi, si tel était Le cas, ne pas plaindre
aussi le milliardaire, qui n’a plus besoin, le pauvre, de travailler ? Mais il n’en
est rien. Si tous les hommes sont égaux en droits et en dignité, comme nous le
voulons, comme nous avons raison de le vouloir, il est exclu que la dignité des
uns et des autres soit proportionnée à la quantité de travail, bien sûr inégale,
qu’ils fournissent. Si le chômage est un malheur, et c’en est évidemment un, ce
n'est pas par l'absence de travail. C’est par l’absence d’argent, c’est par la
misère, c’est par l'isolement ou l'exclusion. Mieux vaut être rentier que smicard,
et cela en dit long sur le travail.

Aristote, dans son génial bon sens, a dit l’essentiel : « Le travail tend au
repos, et non pas le repos au travail. » Il n’est pas vrai qu’on se repose le week-end
pour pouvoir travailler toute la semaine, ni qu’on prenne des vacances,
%— 595 —
comme le voudraient les patrons, pour mieux travailler toute l’année. C’est
l'inverse. On travaille pour gagner sa vie et son repos, pour pouvoir profiter de
ses soirées, de ses week-ends, de ses vacances, bref on travaille pour vivre, alors
qu’il serait fou de vivre pour travailler !

« Je n’ai vu personne, me disait une infirmière, qui regrette, sur son lit de
mort, de ne pas avoir travaillé une heure de plus. » Mais de ne pas avoir assez
vu ses enfants, mais de ne pas avoir assez vécu, réfléchi, aimé, combien sont-ils,
sur leur lit de mort, à le regretter amèrement ?

On parle de « salle de travail », dans nos maternités. C’est que le mot a
d’abord désigné une souffrance, un tourment, une peine.

Dans la Bible, le travail est un châtiment, et telle est aussi l’étymologie du
mot (le {\it trepalium}, d’où vient {\it travail}, était un instrument de torture) comme
encore son sens chez Montaigne. C’est moins vrai aujourd’hui. C’est l’un des
progrès que nous devons au machinisme et aux luttes syndicales. Ce n’est pas
une raison, tordant le bâton dans l’autre sens, pour faire du travail une récompense
ou une valeur. Ce n’est qu’un moyen, jy insiste, qui ne vaut qu’à proportion
du résultat qu’il obtient ou vise. De l’argent ? Pas toujours. Pas seulement.
Le travail est apparu bien avant la monnaie. Et combien de travaux non
rémunérés ? L’humanité doit d’abord produire les moyens de sa propre existence,
comme disait Marx, ce qui ne va pas sans transformation de la nature et
de soi — sans travail. « En même temps qu'il agit par ce travail sur la nature extérieure
et la modifie, soulignait Marx, l’homme modifie sa propre nature et les
facultés qui y sommeillent » ({\it Le Capital}, I, chap. 7). C’est humaniser l’homme en
humanisant le monde. Mais c’est l'humanité qui vaut, non le travail. Aussi le travail
devient-il inhumain, ou déshumanisant, quand le moyen qu’il est tend à
l'emporter sur la fin qu’il vise, ou doit viser. C’est ce que Marx appelle
l'aliénation : quand le travailleur se nie dans son travail, au lieu de s’y réaliser.

TRISTESSE L'un des affects fondamentaux : le contraire de la joie, aussi difficile
qu’elle à définir. C’est comme une souffrance, mais qui
serait de l’âme. Comme une déperdition d’être, de puissance, de vitalité. Comme
une fatigue, mais qu'aucun repos ne suffirait à abolir. « La tristesse est le passage
de l’homme d’une plus grande à une moindre perfection », écrit Spinoza, autrement
dit une diminution de sa puissance d’agir ({\it Éthique}, III, déf. 3 des affects).
C'est exister moins, et le sentir, et en souffrir. Se distingue toutefois du malheur
par l’inconstance ou la mobilité : la tristesse est moins un état qu’un {\it passage}, en
effet ; le malheur, moins un passage qu’un état. Les tristesses vont et viennent,
comme les joies ; le malheur est ce qui reste, quand toute joie semble impossible.
Le malheur est une tristesse qui s’installe. La tristesse, un malheur qui passe.

%— 596 —
TROISIÈME HOMME (ARGUMENT DU -) C’est un argument d’Aristote,
contre Platon, ou de
Platon, déja, contre lui-même. Il apparaît dès le {\it Parménide} (132 a-b). Une idée
est ce qu’il y a de commun entre plusieurs individus (par exemple la grandeur,
entre plusieurs objets grands). Mais si elle existe en elle-même (le Grand en soi),
elle est à son tour un être individuel ; il faut dès lors, pour penser le rapport entre
les objets grands et le Grand en soi, quelque chose de commun, qui ferait une
troisième entité. Puis, pour assurer l’unité entre cette troisième et les deux autres,
une quatrième, et ainsi à l'infini : l'unité ne cesse de fuir et les Idées vont se multiplier
à l'infini. Comment Platon se sortait-il de l’objection ? Sans doute par
l’'unicité de chaque Idée, mais postulée plutôt que démontrée (voir {\it République}, X,
597 c). Il en fallait plus pour impressionner le Stagirite, qui reprend l’argument
dans sa {\it Métaphysique} (A 9, Z 7, M 4). Si l’on se donne un Homme en soi pour
penser ce qui est commun aux différents hommes, on aura besoin d’un {\it troisième
homme} pour penser l’unité entre les hommes et l'Homme en soi, puis d’un quatrième
pour penser l’unité entre ces trois types d’êtres, et ainsi à l'infini : prêter à
l’idée d’homme une existence séparée (Homme intelligible, l'Homme en soi), ce
n’est pas se donner les moyens de penser l’unité des hommes sensibles (celle du
genre humain) ; c’est au contraire la perdre dans une multiplication indéfinie
d’abstractions hypostasiées. Il faut donc renoncer à substantialiser l’universel, ce
qui revient à rompre avec le platonisme. « Je suis lami de Platon, disait Aristote,
mais plus encore de la vérité » (voir {\it Éthique à Nicomaque}, L, 4, 1096 a).

TROPE Une figure de style ou de logique : c’est jouer avec les mots ou les
idées.

Toute figure de style est-elle un trope ? Non pas. Le trope joue avec le sens
des mots plutôt qu'avec leur place ou leur arrangement: c’est une figure
sémantique. Par exemple la métaphore et la métonymie sont des tropes ; le
chiasme et l’accumulation, non.

En philosophie, le mot est pris en un sens logique plutôt que rhétorique. Le
trope est un type d’argument (il joue avec les idées, non avec les mots), comme
une figure de la pensée, comme un raisonnement prédécoupé ou en kit. Le mot
peut désigner, par exemple, tel ou tel mode ou figure du syllogisme. Mais les
tropes les plus célèbres sont ceux d’Énésidème et d’Agrippa, qui relèvent de la
tradition sceptique et tendent à imposer la suspension du jugement. C’est une
espèce de machine de guerre contre tout dogmatisme. On en trouve la liste
chez Diogène Laërce ({\it Vies et doctrines}, IX) et Sextus Empiricus ({\it Hypotyposes
pyrrhoniennes}, I). Ceux d’Énésidème sont au nombre de dix, qui visent à montrer
que toutes nos représentations sont relatives : elles varient en fonction du
% 597
sujet qui perçoit (homme ou animal, tel homme ou tel autre, avec tel organe
sensoriel ou tel autre, etc.), mais aussi en fonction des circonstances (position,
distance, mélange, quantité.….), des relations, des fréquences et des modes de
vie. Ceux d’Agrippa, plus ramassés, plus frappants, sont au nombre de cinq : il
y a le trope de la discordance (les opinions s'opposent, chez les philosophes
comme chez les profanes : pourquoi privilégier l’une d’entre elles ?), celui de la
régression à l'infini (toute preuve devant être prouvée par une autre, et ainsi à
l'infini, on n’en aura jamais fini de prouver quoi que ce soit : tout reste donc
douteux), celui de la relation (rien ne peut être appréhendé en soi : toute représentation
est relative au sujet et aux circonstances), celui des principes (qui ne
sont que des hypothèses indémontrables : puisqu'il faut en poser pour démontrer
quoi que ce soit, toute démonstration reste ainsi incertaine), enfin celui du
diallèle ou cercle vicieux (qui prétend démontrer une proposition à partir d’une
autre qui en dépend : les deux sont donc sans valeur). Ces cinq tropes, comme
les dix d’Énésidème, aboutissent à la suspension du jugement : le mieux,
puisqu'on ne peut trancher, est de s’interdire toute assertion dogmatique.

TRUISME Une vérité évidente et sans portée. À ne pas confondre avec la
tautologie, qui n’est pas toujours évidente et rarement sans portée.

TYRANNIE Cest exercer le pouvoir au-delà de son domaine légitime
(Locke), et spécialement dans un ordre où l’on n’a aucun titre
légitime à le faire (Pascal) : ainsi le roi qui veut être aimé ou cru, quand il ne
mérite, en tant que roi, que d’être obéi ({\it Pensées}, 58-332). C’est vouloir régner
sur les esprits par la force (barbarie), ou sur la force par l'esprit (angélisme), et
c’est en quoi toute tyrannie, même effrayante, est ridicule (voir ce mot) : c’est
le ridicule au pouvoir, ou la confusion des ordres érigée en système de gouvernement.
La faute du tyran est de « vouloir régner partout », ce que nul ne peut,
et « hors de son ordre », ce que nul ne doit ({\it ibid.}).

En un sens plus général, on désigne par tyrannie le pouvoir absolu d’un
seul, quand il est illégitime, violent ou arbitraire. Le mot, dans son usage
moderne, vaut toujours comme condamnation. C’est que nous y voyons, à
juste titre, le contraire de l’État de droit: « Là où le droit finit, la tyrannie
commence » (Locke, {\it Deuxième traité du gouvernement civil}, chap. XVII).

%598
UBIQUITÉ La faculté d’être présent partout à la fois. Ce serait le propre de
Dieu, s’il existe. Mais alors, pourquoi dit-on qu’il est aux
cieux ? C’est qu’il n’est présent ici-bas, répondrait Simone Weil, que sur le
mode de l'absence ou du retrait (il n’est là qu’en tant qu’il n’y est pas). C’est un
Dieu caché : son ubiquité nous en apprend moins sur lui que sa transcendance.

UN Le premier élément d’une énumération (le zéro, qui fut inventé beaucoup
plus tard, sert moins à nombrer qu’à calculer). Peut désigner à ce
titre aussi bien l’{\it unité} (l’un des éléments d’une pluralité possible : un parmi
d’autres) que l’{\it unicité} (quand il n’y a pas de pluralité : un seul). Ces deux
sens sont d’autant moins incompatibles que le second suppose le premier.
Par exemple si j'entends, me réveillant la nuit, une horloge sonner un coup :
rien, lorsque ce coup retentit, ne me permet de savoir s’il est une heure ou
plus : seuls me l’apprendront d’autres coups ou le silence. L’unicité n’est
qu’une unité sans suite ; la pluralité, qu’un ensemble d’unités. C’est donc
l'unité qui est première, dont l’unicité et la pluralité ne sont que des occurrences.
Cela semble donner raison à Parménide ou à Plotin. Mais non, pourtant,
puisque rien ne prouve que cette unité première soit unique (cela peut
donner raison tout autant à Démocrite : les atomes sont des unités en
nombre infini), ni même qu’elle {\it soit} première : elle ne l’est que pour la
pensée ; pourquoi la matière, qui ne pense pas, devrait-elle s’y soumettre ? Il
n’est pas impossible qu’il n’y ait d’abord qu’une multiplicité indéfinie, sans
unités, sans êtres, sans substances : qu’il n’y ait que des flux et des processus.
L'Un, alors, serait moins leur principe que leur ensemble : c’est ce que nous
appelons l’univers.

%— 599 —
UNICITÉ Le fait d’être unique. On peut accorder à Leibniz que c’est le
propre de tout être (principe des indiscernables), mais inégalement :
deux feuilles d’un même arbre, quoique différentes l’une de l’autre, sont
pourtant moins uniques qu'un être qui ne ressemble à aucun autre ni n’entre,
comme élément, dans aucune multiplicité. Dieu ou le Tout sont plus uniques,
en ce sens, que ce qu'ils créent ou contiennent, et eux seuls, peut-être, le sont
absolument.

UNION Le devenir-un d’une multiplicité, qui reste pourtant hétérogène ou
« plurielle », comme on dit aujourd’hui. C’est ce qui distingue
l'{\it union} (le fait d’être unis) de l’{\it unité} (le fait d’être un). Les militants progressistes,
dans ma jeunesse, distinguaient traditionnellement l’union de la gauche
(qui supposait une alliance de classes et de partis) et l’unité de la classe ouvrière
(qui supposait, ou aurait supposé, qu’on revienne en amont de la scission de
1920). Je ne sais si les militants d’aujourd’hui font encore ces distinctions. Philosophiquement,
on retiendra surtout que l’union est un processus ou un
combat ; l’unité, un état ou un idéal.

UNITÉ Le fait d’être un. À ne pas confondre avec l’unicité (le fait d’être un
seul), ni avec l’union (le fait d’être unis), qui la supposent l’une et
l’autre. Il n’y aurait pas {\it un seul}, ou on ne pourrait le penser, s’il n’y avait
d’abord {\it un}. Mais il n’y aurait pas davantage pluralité, ni donc union : il n’y
aurait pas {\it plusieurs} s’il n’y avait pas d’abord {\it un}, et encore {\it un}, et encore {\it un}.
C’est pourquoi on peut compter sur ses doigts : parce que chaque doigt est un.
« Le pluriel suppose le singulier », écrivait Leibniz : il n’y aurait pas plusieurs
êtres s’il n’y avait plusieurs fois un être. Et « ce qui n’est pas véritablement #r
être, continuait-il, n’est pas non plus véritablement un {\it être} ». Ainsi l’unité est
première, au moins pour la pensée. De là, sans doute, le privilège métaphysique
de l’Un. Mais si la nature ne pense pas ?

UNIVERS Pour la plupart des philosophes, c’est l’ensemble de tout ce qui
existe ou arrive. Il est donc exclu qu’il y en ait plusieurs : si c’était
le cas, l’univers serait leur somme.

Que penser alors de hypothèse, parfois évoquée par les physiciens contemporains,
d’une pluralité d’univers ? Qu’elle correspond à l’idée philosophique
d’une pluralité des {\it mondes}, et rend les deux mots à peu près synonymes. Quand
%— 600 —
on veut éviter l'ambiguïté, mieux vaut, philosophiquement, distinguer le
monde et le Tout (voir ces mots), et laisser l’univers aux physiciens.

UNIVERSAUX (QUERELLE DES —) C’est un débat qui traverse et structure
toute la pensée du Moyen-Âge. Il
s’agit de savoir quel type de réalité accorder aux idées générales ou universelles.
Sont-ce des êtres réels, comme le voulait Platon (réalisme), ou bien de simples
conceptions de notre esprit (conceptualisme), voire de purs mots (nominalisme) ?
Le matérialisme n’a bien sûr le choix qu’entre ces deux dernières solutions,
qui s'opposent peut-être moins qu’elles ne se complètent.

UNIVERSEL Qui vaut pour l’univers entier, ou pour la totalité d’un
ensemble donné. C’est en ce dernier sens que les droits de
l’homme sont universels : non parce que l’univers les reconnaîtrait (pourquoi
l'univers serait-il humaniste ?), mais parce qu’ils valent, en droit, pour tout être
humain. On voit que si {\it universel} s'oppose à {\it particulier}, ce n’est pas de façon
simple. Les droits de l’homme sont une particularité humaine (ils ne valent que
pour l'humanité), mais n’en sont pas moins universels pour autant (ils doivent
s'appliquer à tout être humain, y compris s’il ne les respecte pas).

L’universel, remarque Alain, est le lieu des pensées. Une vérité qui ne
serait pas vraie, en droit, pour tous, ne serait pas une vérité du tout. Cela,
remarquons-le, ne dépend pas du degré de généralité de la pensée considérée.
Que tu sois en train de lire cet article, c’est un fait très singulier. Mais
il n’y a pas un seul point de l’univers où l’on puisse nier cette vérité sans
faire preuve d’ignorance ou de mauvaise foi. Et comme tout est vrai, toujours,
tout est universel : le plus petit de nos mensonges est universellement
mensonger.

« La pensée, disait encore Alain, ne doit pas avoir d’autre chez soi que tout
l'univers ; c’est là seulement qu’elle est libre et vraie. Hors de soi ! Au-dehors ! »
L’universel, pour l’esprit, est la seule intériorité vraie.

UNIVOQUE Qui ne se prend qu’en un seul sens, quel que soit l'emploi ou
le contexte et y compris quand on l’applique à des objets différents.
Sorti du langage scientifique, et encore, c’est beaucoup moins la règle
que l’exception. S’oppose à équivoque (qui a au moins deux significations différentes)
et parfois à plurivoque (qui en a plusieurs).

%— 601 —
URBANITÉ La politesse des villes. C’est supposer qu’on trouve une politesse
aussi à la campagne, et qu’elle n’est pas la même. Quand
on croise chaque jour des milliers d’inconnus, la politesse inévitablement
devient plus nécessaire, plus superficielle, plus systématique. La foule impose sa
loi, qui est d’anonymat et de prudence.

USAGE/USURE User, c’est d’abord se servir de. Puis détériorer peu à peu,
affaiblir, amoindrir. L'usage est antérieur à l’usure, et l’entraîne.
Ainsi une paire de souliers : on ne peut en user sans l’user. Le lien, entre
ces deux notions, n’est pourtant pas sans quelques notables exceptions. On
peut se servir de son corps et de son cerveau sans les user d’abord, et ils s’usent
d'autant plus, semble-t-il, qu’on s’en sert moins. C’est qu’on est dans l’ordre du
vivant, qui résiste à l’usure par l’exercice et la régénération (comme une
machine, disait Leibniz, qui réparerait d’elle-même ses rouages). Toutefois cela
ne dure qu’un temps. La matière ou l’entropie imposent leur loi, peu à peu, qui
est de dégradation et de mort. C’est ce qu’on appelle le vieillissement, usure
biologique. « Tout faiblit peu à peu, écrit Lucrèce, tout marche vers la mort,
usé par la longueur du chemin de la vie » ({\it De rerum natura}, II, 1173-1174).
Cela donne tort aux optimistes, non aux vivants.

UTILE Ce qui sert à quelque chose d’autre, à condition que ce quelque
chose soit bon ou jugé tel. Notion relative, donc : ce qui est utile
aux uns peut être nuisible aux autres, voire à la fois utile et nuisible pour les
mêmes (par exemple la voiture, utile pour les voyages, nuisible pour l’environnement).
Il n’y a pas d’utilité absolue : l’utile n’est pas une fin en soi ; ce n’est
qu'un moyen efficace, en vue d’une fin désirée. C’est pourquoi l’utilitarisme
aura besoin de se donner une fin ultime, qui sera presque toujours le bonheur
du plus grand nombre : est utile ce qui le favorise, nuisible ce qui lui fait obstacle.
Mais cette fin est elle-même sujette à caution. Si vous mettez la vérité ou
la justice plus haut que le bonheur, les frontières de l’utile et de l’inutile vont
se déplacer. Mais n’en existeront pas moins.

UTILITARISME Toute doctrine qui fonde ses jugements de valeur sur l’utilité.
Un égoïsme ? Non pas, puisque l’utilité est définie,
chez la plupart des utilitaristes (spécialement chez Bentham et John Stuart
Mill), comme ce qui contribue au bonheur du plus grand nombre. Rien
n'exclut donc qu’un utilitariste se sacrifie pour les autres, s’il considère que la
%— (602 —
quantité globale de bonheur en est augmentée (s’il juge, cela revient au même,
que son sacrifice est utile). Et il est bien difficile, qu’on soit utilitariste ou pas,
de se sacrifier inutilement (serait-ce encore un sacrifice ?) ou même, sauf rigorisme
particulier, sans avoir le sentiment que le bonheur de l'humanité en est
au moins possiblement augmenté. Par quoi l’utilitarisme est moins une morale
particulière qu’une philosophie particulière de la morale : les comportements,
en pratique, seront souvent les mêmes, mais pensés ou justifiés différemment.

Jean-Marie Guyau a bien montré qu’Épicure était une espèce d’utilitariste
avant la lettre, comme Spinoza (s'agissant de ce dernier, voir par exemple
{\it Éthique}, IV, prop. 20 et 24), ou plutôt que « c’est l’épicurisme, uni au naturalisme
de Spinoza, qui renaît chez Helvétius et d’Holbach », avant de « susciter
dans la patrie de Hobbes des partisans plus nombreux encore » et de prendre
«sa forme définitive » chez Bentham et Mill ({\it La morale d'Épicure et ses rapports
avec les doctrines contemporaines}, 1878, Introduction). Qu’on en juge :

« La doctrine qui donne comme fondement à la morale l'utilité ou le principe du
plus grand bonheur, affirme que les actions sont bonnes ou sont mauvaises dans la
mesure où elles tendent à accroître le bonheur, ou à produire le contraire du bonheur.
Par “bonheur” on entend le plaisir et l’absence de douleur ; par “malheur”, la douleur
et la privation de plaisir. [...] Cette théorie de la moralité est fondée sur une conception
de la vie selon laquelle le plaisir et l’absence de douleur sont les seules choses désirables
comme fins, et que toutes les choses désirables ne le sont que pour le plaisir qu’elles
donnent elles-mêmes ou comme moyens de procurer le plaisir et d’éviter la douleur »
(John Stuart Mill, {\it L'utilitarisme}, II).

Faut-il alors renoncer à toute élévation, à toute spiritualité ? Aucunement,
puisque celles-ci peuvent être (comme le dit Mill de la vertu) un moyen ou une
partie du bonheur. C’est que le bonheur ({\it happiness}) est bien autre chose que la
satisfaction ({\it content}) des instincts ou des appétits. Chacun a les plaisirs qu’il
mérite, qui fondent aussi le bonheur qu’il ambitionne. Pour un individu aux
aspirations élevées, note Mill, toutes les satisfactions ne se valent pas. De là
cette vigoureuse formule, qui m’a toujours rendu son auteur extrêmement
sympathique : « Il vaut mieux être un homme insatisfait qu’un porc satisfait ; il
vaut mieux être Socrate insatisfait qu’un imbécile satisfait. Et si l’imbécile ou le
porc sont d’un avis différent, c’est qu’ils ne connaissent qu’un côté de la
question : le leur. L’autre partie, pour faire la comparaison, connaît les deux
côtés » ({\it ibid.}).

On remarquera pourtant que l'utilité peut être un critère de valeur, mais
non de vérité. C’est ce qui distingue, ou qui peut distinguer, l’utilitarisme du
pragmatisme, voire de la sophistique. Une vérité inutile, et même nuisible (une
vérité qui ne serait pas « avantageuse pour la pensée », comme disait William
%— 603 —
James), n’en serait pas moins vraie pour cela. Et un mensonge utile ou une
erreur avantageuse, pas moins faux. C’est en quoi l’utilitarisme, même moralement
justifié, ne saurait tenir lieu de philosophie : ce n’est pas parce qu’une
idée est favorable au bonheur du plus grand nombre qu’il faut la penser (ce ne
serait plus philosophie mais sophistique, plus utilitarisme mais méthode
Coué) ; c’est parce qu’elle semble vraie. Un utilitariste pourrait objecter que la
vérité, même désagréable, est plus utile, au bout du compte, qu’une illusion,
même confortable, et que l’utilitarisme est sauvé par là de la sophistique. Dont
acte. Mais il n’échappe au cercle qu’en acceptant que la vérité de cette dernière
idée ne dépend pas de son utilité (puisque son utilité, si elle est vraie, en
dépend). La vérité ne peut être utile que si ce n’est pas son utilité qui fait sa
vérité. Elle ne peut être une valeur que si elle n’a pas besoin de valoir pour être
vraie. Par quoi l’utilitarisme n’est acceptable qu’à l’intérieur du rationalisme,
non contre lui.

Il reste que l’utilitarisme, même s'agissant des actions, pèche peut-être par
optimisme. « Si les hommes n’avaient en vue que l’utile, tout s’arrangerait, écrit
Alain. Mais il n’en est pas ainsi. » C’est qu’ils agissent par passion, beaucoup
plus que par intérêt. De là les guerres, nuisibles pour presque tous. L’amour-propre
est un moteur plus puissant que l’égoïsme, et plus dangereux.

UTOPIE Ce qui n’existe nulle part (en aucun lieu : {\it u-topos}). Un idéal ? Si
l’on veut, mais programmé, mais organisé, mais planifié, souvent
avec un soin maniaque des détails : c’est un idéal qui ne se résigne pas à en être
un, qui se prend pour une prophétie ou un mode d’emploi. Se dit spécialement
des sociétés idéales ; l’utopie est alors une fiction politique, qui sert moins à
condamner la société existante (pas besoin d’utopie pour cela) qu’à en proposer
une autre, déjà conçue dans ses détails, qui n’aurait plus qu’à être réalisée. Ainsi
chez Platon, Thomas More (qui inventa le mot) ou Fourier.

Le mot peut se prendre positivement ou négativement : il peut désigner ce
qui n’existe pas encore mais existera un jour ; ou bien ce qui n’existe pas et
n’existera jamais. Dans le premier cas, c’est un but, vers lequel il faut tendre ;
dans le second, une illusion, à laquelle il vaut mieux renoncer. Dans l’usage
courant, c’est ce second sens qui tend à l'emporter : une utopie, c’est un but ou
un programme qu'on juge irréalisable. Parce qu’on manque d’imagination,
d’audace, de confiance ? C’est ce que certains suggèrent : l'utopie d’aujourd’hui
serait la réalité de demain. Et d’évoquer les congés payés, la sécurité sociale, la
télévision, Internet, qu’on aurait jugé utopiques il y a quelques siècles. Mais
c’est confondre utopie et science-fiction, Thomas More et Jules Verne. Les
grandes utopies du passé (depuis la République de Platon jusqu’aux socialismes
%— 604 —
utopistes du {\footnotesize XIX}$^\text{\,e}$ siècle), nous semblent aussi irréalisables aujourd’hui qu’hier,
et plus dangereuses. C’est qu’on voit trop ce qu’elles supposeraient de
contraintes et de bourrage de crâne (de totalitarisme). Une utopie, ce n’est pas
seulement un projet de société qui semble présentement impossible ; c’est une
société parfaite, qui ne laisserait rien à transformer. Ce serait la fin de l’histoire,
la fin des conflits, comme une espèce de paradis collectif : cela ressemble à un
Club Méditerranée définitif, autrement dit à la mort.

