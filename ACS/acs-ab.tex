%{\footnotesize XIX$^\text{e}$} siècle — {\it }
\chapter{AB}
%\section{}

\section{Abbé}
%ABBÉ
Le mot signifie père ({\it abba} en araméen puis dans le grec et le latin
ecclésiastiques). Voltaire voulait pour cela que les abbés fassent des
enfants, ce qui les rendrait au moins utiles à quelque chose. C'était pousser
un peu loin l’amour de l’étymologie.

Un abbé est d’abord le supérieur d’un monastère, ou, aujourd’hui, un
prêtre quelconque : c’est le père spirituel de ses moines ou de ses ouailles. Voltaire
leur reproche leur richesse, leur vanité, leurs abus. « Vous avez profité des
temps d’ignorance, de superstition, de démence, pour nous dépouiller de nos
héritages et pour nous fouler à vos pieds, pour vous engraisser de la substance
des malheureux : tremblez que le jour de la raison n’arrive » ({\it Dictionnaire philosophique},
article « Abbé »). On ne saurait lui donner tort, s’agissant des abbés
de son temps. Mais aujourd’hui, où tant d’abbayes sont vides, ou presque vides,
il m'est arrivé bien souvent, sous leurs voûtes sublimes et désertées (à Noirlac,
à Sénanque, à Fontenay...), de regretter cette désaffection, cette déshérence, et
de ressentir, vis-à-vis de ces formidables bâtisseurs, et d’une si évidente élévation
spirituelle, je ne sais quel mélange de gratitude, d’admiration et de nostalgie...
Le jour de la raison est-il arrivé ? Il n’arrivera jamais. Pourtant Voltaire,
s’il revenait parmi nous, aurait le sentiment de l'avoir emporté, contre les
prêtres et les inquisiteurs. Combien d’abbayes rasées par la Révolution ? Combien
d’autres transformées en fermes, en entrepôts ou, aujourd’hui, en musées ?
De moins en moins de moines. De plus en plus de touristes. Combien
sommes-nous, parmi ceux-ci, à nous sentir indignes de ceux-là ? Nous construisons
des hôtels plutôt que des abbayes, des hôpitaux plutôt que des couvents,
des lycées plutôt que des églises. On aurait tort de le regretter. Mais
pourquoi faut-il qu’ils soient si laids, si insignifiants, si désolants de platitude ?
Qu'ils parlent si peu au cœur ou à l’âme ?

% 13
Qu'on se soit débarrassé de l’Inquisition et de la dîme, des abbés de cour,
de cette alliance obscène du trône et de l’autel, du despotisme et de la superstition,
c’est évidemment heureux. Nous le devons, au moins pour une part, à
Voltaire et à ses amis. Grâce soit rendue aux Lumières ! Mais faut-il pour cela
s’illusionner sur notre époque ? Prendre le tourisme pour une spiritualité, l’art
pour une religion, les loisirs pour un art? Adorer laudimat, le CAC 40 et
l’équipe de France de football ? Ce n’est pas le jour de la raison qui est venu.
C’est celui du capitalisme triomphant, dont Voltaire rêvait, mais qui domine
aujourd’hui jusqu’au marché de la culture et de l'information, c’est celui des
médias, du spectacle mercantile, de la communication universelle et narcissique.
{\it « Parlez-moi de moi, y a que ça qui m'intéresse. »} Et de filmer sa vie en
continu pour Internet. Cela vaut mieux que l’Inquisition ? Sans doute. Mais
ne suffira pas à sauver une civilisation.

Les temps changent. Rencontrer un abbé, aujourd’hui, même pour le libre-penseur
que je suis, c’est plutôt une bonne surprise : en voilà au moins un, se
dit-on, qui n’a pas oublié complètement l'essentiel, qui n’a pas vendu sa vie au
plus offrant, avec qui j'aurai de vrais désaccords, et pas ce mélange de lassitude
et d’agacement que m'inspirent tant de mes contemporains. Le combat
continue, pour les Lumières, pour les droits de l’homme, pour le bonheur.
Mais les adversaires ont changé. Raison de plus pour entreprendre — c'est le
premier nom que Voltaire avait donné au sien — un nouveau {\it Dictionnaire philosophique
portatif...}

\section{Abnégation}
%ABNÉGATION
C’est la vertu du sacrifice : l'oubli ou le don de soi-même,
quand il n’est pas pathologique. L’inverse donc de l’égoïsme,
comme un altruisme en creux ; c’est moins vivre pour autrui que renoncer à soi.
Vertu toujours suspecte (est-ce dévouement ou masochisme ? pulsion de vie ou
pulsion de mort ?), qui ne vaut vraiment que par la joie qu’elle comporte ou
permet — ce qui n’est plus abnégation mais amour ou générosité.

\section{Aboulie}
%ABOULIE
Impuissance de la volonté. Le mot désigne une pathologie (l'incapacité
à agir de façon volontaire et réfléchie), ou bien n’est
qu’un euphémisme savant pour dire la veulerie, la lâcheté, la paresse. Un syndrome,
donc, ou une faute.

\section{Absence}
%ABSENCE
Ce n’est pas rien, puisque c’est l'absence {\it de quelque chose}, comme
un néant déterminé, délimité, défini. Par quoi ? Par une présence,
% 14
où plusieurs : ce qui est absent ici est présent 1à-bas ({\it abesse}, en latin, c'est
être éloigné de), ou l’a été, ou le sera, ou pourrait l’être, comme autre chose,
maintenant, est présent à sa place. La conscience a horreur du vide : il n’est
absence que d’une présence.

Il n’y a pas d’absence absolue (ce ne serait plus absence mais néant) ni totale
(ce ne serait rien). Il n’y a donc pas d’absence en vérité : il n’y a que la présence
de tout — le monde — et notre incapacité à nous en satisfaire. {\it Tout}, pour nous, ce
n'est pas assez ! Il faut donc autre chose que tout, c’est le secret de l’idéalisme, que
le matérialisme, depuis vingt-cinq siècles, essaie de démasquer. Platon toujours
renaît : l'être est ailleurs ; l'être est ce qui manque ; il ne brille, ici-bas, que par son
absence ! Et Démocrite renaît avec lui, contre lui : l’être est partout, qui ne
manque de rien. Transcendance ou immanence : absence ou présence de l'être.

On parle aussi des {\it absences} de celui qui est distrait, inattentif, presque
inconscient : parce qu’il semble n’être pas, en esprit, là où son corps se trouve.
D'où cette formule d’un enseignant, sur le livret scolaire d’un de ses élèves :
{\it « Souvent absent, même quand il est là... »} Cela suggère, par différence, ce qu’est
l'attention.

Toute absence est absence de quelque chose, et pour quelqu’un. C’est le
contraire de la présence, et c’est pourquoi ce n’est rien de réel, ou presque rien :
la conscience présente de ce qui ne l’est pas.

C’est pourquoi aussi c’est un piège. Nous sommes au monde : la vraie vie
est présente.

\section{Absolu}
%ABSOLU
Comme adjectif, caractérise tout ce qui est complet ({\it absolutus}),
total, sans restriction ni réserve. On dira par exemple : pouvoir
absolu, confiance absolue, savoir absolu. C’est en général un abus de langage.

L'humanité, pour tout esprit lucide, fait une restriction suffisante.

Mais le mot, philosophiquement, vaut surtout comme substantif. L’absolu,
c'est alors ce qui existe indépendamment de toute condition, de toute limite,
de tout point de vue, donc de façon autonome ou séparée.

L’absolu doit être cause de soi (il serait autrement dépendant de sa cause) ou
exister par soi (il serait autrement relatif) : il ne peut être que Dieu ou tout. Ce
dont tout dépend ne dépend de rien. L'ensemble des relations n’est pas relatif.

C’est un autre nom pour l'être en soi et par soi. Que nous n’y ayons pas
accès, sinon relativement, n’empêche pas qu’il nous contienne.

\section{Absolution}
%ABSOLUTION
Vaut parfois, dans le langage juridique, par différence avec
l’acquittement : on acquitte un innocent ; on absout un
% 15
coupable, lorsqu'on ne peut (parce que la loi ne le prévoit pas) ou qu'on ne
veut (si elle ne l’impose pas) le punir.

Mais c’est surtout le nom sacramentel du pardon. Dieu seul, s’il existe, peut
nous absoudre, en ce sens, c’est-à-dire effacer ou remettre nos péchés. On
remarquera que c’est nous supposer coupables ; cela en dit long sur la religion,
et peut-être aussi sur l’athéisme.

\section{Abstraction}
%ABSTRACTION
«Il n’y a de science que du général, disait Aristote, et d’existence
que du singulier. » C’est pourquoi toute science, par
définition, est abstraite : parce qu’elle vise à la généralité d’une loi, d’une relation
ou d’un concept, et non à la singularité d’une existence — parce qu’elle
n'existe, comme science, qu’à la condition de se détacher ({\it abstrahere}) du réel
immédiat. Cela vaut aussi pour la philosophie, comme pour tout effort théorique.
Il n’y a pas de pensée concrète : une pensée concrète serait le monde, qui
ne pense pas, ou Dieu, que nous ne pouvons penser. C’est ce qui nous en
sépare : Dieu et le monde, pour nous, ne sont que des abstractions.

Abstraire, c’est séparer, isoler par la pensée ce qui n’existe qu'avec autre
chose, ou au contraire rassembler ce qui n'existe que séparément. Par exemple
une couleur, si je la considère indépendamment — on dira aussi : {\it abstraction
faite} — de tout objet coloré (le rouge est une abstraction), une forme, si je la
considère indépendamment de l’objet dont elle est la forme, voire de tout objet
matériel (le triangle, le cube, la sphère. sont des abstractions) ; ou bien un
ensemble d’objets, si je laisse de côté leurs différences (l’ensemble des objets
triangulaires, ou cubiques, ou sphériques, l'ensemble des hommes ou des
vivants sont aussi des abstractions). De là la géométrie, la physique, la biologie,
et toutes les sciences.

C’est comme un détour par la pensée, qui serait un raccourci vers le vrai,
comme une simplification obligée. Le réel est inépuisable, toujours ; mais la
pensée fatigue. D’où cette commodité — pour ne pas dire cette paresse — de
l’abstraction. Aurait-on besoin autrement d’un dictionnaire ? Mais aucun dictionnaire
n’est le monde, ni aucune langue.

Essayez de décrire complètement un caillou ; vous verrez, par l'impossible,
ce que c’est qu’une abstraction : une idée qui ne correspond à son objet qu’en
renonçant à le contenir tout entier, et même à lui ressembler. Ainsi l’idée de
caillou, et même de ce caillou-ci. L’abstraction est le lot de toute pensée finie,
ou la finitude, pour nous, de toute pensée.

Toute idée, même vraie, est abstraite, puisque aucune ne ressemble à son
objet (Spinoza : le concept de chien n’aboie pas, l’idée de cercle n'est pas
ronde), ni ne saurait reproduire en nous, s’il existe en dehors d’elle, son inépuisable 
% 16
réalité. Mais elles le sont inégalement : {\it couleur} est plus abstrait que {\it rouge},
moins abstrait qu’{\it apparence}. Surtout, il y a un bon et un mauvais usage des abstractions,
selon qu’elles ramènent au réel ou en éloignent, le dévoilent ou le
masquent.

On parle aussi de peinture abstraite, quand elle renonce à la figuration. Il y
a, dans ce renoncement, une part de nécessité : reproduire, c’est toujours
choisir, transformer, séparer, rapprocher, simplifier — abstraire. Toute figuration
est abstraite au moins en quelque chose, non quoiqu’elle soit figurative,
mais parce qu’elle l'est. C’est la peinture non figurative qui pourrait être dite
concrète, n'étant séparée de rien qu’elle imite ou reproduit : quoi de plus
concret qu'une tache de couleur sur une toile ? Si on l'appelle pourtant peinture
abstraite, c’est qu’elle semble séparée du monde réel (ce n’est bien sûr
qu'une illusion : le vrai est qu’elle en fait partie) et renonce en effet à l’imiter.
Cela fait comme une abstraction redoublée, qui sépare la peinture de tout objet
extérieur pour ne plus la livrer qu’à elle-même ou à l'esprit. C’est peut-être
confondre la peinture et la pensée, comme un philosophe qui voudrait dessiner
ses concepts.

\section{Absurde}
%ABSURDE
Ce n’est pas l’absence de sens. Une éclipse ne veut rien dire, mais
n'est pas absurde pour cela. Et une phrase quelconque ne peut
être absurde, inversement, qu’à la condition de signifier quelque chose.
Ainsi, je ne fais qu’assembler des exemples traditionnels : {\it « Sur une montagne
sans vallée, près d’un cercle carré, d'incolores idées vertes dormaient
furieusement... »} Cela ne veut rien dire ? Si, pourtant, puisqu'on comprend
ce qu'il y a là d’impossible à penser ou à comprendre tout à fait. C’est pourquoi
nous pouvons juger que cette phrase est absurde, ce que nous ne saurions
dire, me semble-t-il, d’une phrase, si c’en est une, totalement
dépourvue de sens (par exemple: {\it « Ofym idko rufiedy ud kodziriaku »}).
L’absurde est plutôt insensé qu’insignifiant. Ce n’est pas l’absence de sens,
mais son inversion, son explosion, sa contradiction, sa déconstruction, si
l’on peut dire, de l’intérieur. Est absurde ce qui est contraire au bon sens ou
au sens commun : contraire à la raison, à la logique ou à l'humanité ordinaires.
C’est pourquoi on y trouve parfois une espèce de poésie onirique ou
un peu folle, qui fit le charme, un temps, du surréalisme. C’est par quoi
aussi l'absurde est essentiel à un certain type d’humour, quand l’insignifiance,
en toute chose, ennuie. Voyez Woody Allen ou Pierre Dac. Non
l’absence de sens, donc, mais un sens trop paradoxal ou trop contradictoire
pour pouvoir être pensé ou accepté totalement. Par exemple ceci, qui est de
Woody Allen : « L’éternité c’est long, surtout vers la fin. » C’est qu’il n’y a
% 17
pas de fin. Ou cela, qui est de Pierre Dac : « À l’éternelle triple question,
toujours demeurée sans réponse : “Qui sommes-nous ? D'où venons-nous ?
Où allons nous ?”, je réponds : “En ce qui me concerne personnellement, je
suis moi, je viens de chez moi, et j'y retourne”. » Cela ne répond pas à la
question ? Si, pourtant, mais en l'invalidant. L’absurde n’est pas tout
l'humour, ni toujours humoristique, pas plus que tout humour n'est
absurde. Mais ils se rejoignent en ceci que leur sens n’est pas le bon ou le
commun. C’est pourquoi ils font rire ou font peur.

Le monde est-il absurde ? Il ne pourrait l’être qu’à la condition d’avoir
un sens, qui ne serait pas le nôtre. L’absurde, remarque Camus, naît toujours
d’une comparaison entre deux ou plusieurs termes disproportionnés,
antinomiques ou contradictoires, et « l’absurdité sera d’autant plus grande
que l'écart croîtra entre les termes de la comparaison ». Par exemple « si je
vois un homme attaquer à l’arme blanche un groupe de mitrailleuses, je
jugerai que son acte est absurde ; mais il n’est tel qu’en vertu de la disproportion
qui existe entre son intention et la réalité qui l'attend, de la contradiction
que je puis saisir entre ses forces réelles et le but qu’il se propose ».
Il n’y a pas d’absurde en soi, ni par soi — pas d’absurde absolu. « L’absurde
est essentiellement un divorce. Il n’est ni dans l’un ni dans l’autre des éléments
comparés. Il naît de leur confrontation. » Le monde n’est donc pas
absurde : ce qui est absurde, explique {\it Le mythe de Sisyphe}, c'est « cette
confrontation entre l’appel humain et le silence déraisonnable du monde ».
Que le monde n’ait pas de sens, cela ne le rend absurde que pour nous, qui
en cherchons un. C’est pourquoi l’absurde est « un point de départ », non
un point d'arrivée. Pour qui saurait accepter le monde, son silence, son
indifférence, sa pure et simple réalité, l'absurde disparaîtrait : non parce que
nous aurions trouvé un sens, mais parce qu'il aurait cessé de nous manquer.
C’est la sagesse ultime de {\it L'étranger}: « Vidé d’espoir, devant cette nuit
chargée de signes et d’étoiles, je m’ouvrais pour la première fois à la tendre
indifférence du monde. De l’éprouver si pareil à moi, si fraternel enfin, j'ai
senti que j'avais été heureux, et que je l’étais encore... » Cela dit assez ce
qu’est l'absurde : non l'absence du sens, mais son échec ou son manque. Et
ce qu'est la sagesse : l’acceptation comblée, non d’un sens, mais d’une présence.

La vie est-elle absurde ? Pour autant seulement que nous lui cherchons un
sens, qui ne pourrait exister — sens, c’est absence — qu’en dehors delle. Et quel
autre {\it dehors}, pour la vie, que la mort ? « Chercher le sens de la vie, écrit François
George, c’est faire un contresens sur la vie. » C’est en effet vouloir l'aimer
pour autre chose qu’elle, qui serait son sens, quand tout sens au contraire la
suppose. Si la vie « doit être elle-même à soi sa visée », comme disait Montaigne,
% 18
c'est qu’elle n’est ni absurde ni sensée : réelle, simplement, merveilleusement
réelle — et aimable, si nous l’aimons.

C’est le plus difficile : non comprendre la vie, comme si elle était une
énigme à résoudre, mais l’accepter telle qu’elle est — fragile, passagère —, et
joyeusement si l’on peut. Sagesse tragique : sagesse non du sens mais de la
vérité, non de l'interprétation mais de l’amour et du courage.

Arthur Adamov, sur ce point, a dit l'essentiel : « La vie n’est pas absurde ;
elle est seulement difficile, très difficile. »

\section{Absurde (raisonnement par l'—)}
%ABSURDE (RAISONNEMENT PAR L-)
C'est un raisonnement qui prouve
la vérité d’une proposition par
l’évidente fausseté d’une au moins des conséquences de sa contradictoire.
Pour démontrer {\it p}, on fait l'hypothèse de {\it non-p}, dont on montre qu’elle
aboutit à une absurdité. C’est ainsi, par exemple, qu'Épicure démontrait
l'existence du vide {\it (p)} : S'il n’y avait pas de vide {\it (non-p)}, il n’y aurait pas de
mouvement (les corps n’ayant pas la place de se mouvoir) ; or cette conséquence
est évidemment fausse (puisque démentie par l’expérience) ; le vide
existe donc. Ce type de raisonnement, qu’on appelle aussi apagogique,
repose comme on le voit sur le principe du tiers exclu ({\it p} ou {\it non-p} : si une
proposition est fausse, sa contradictoire est vraie). Il est formellement valide,
mais n’est probant que si la contradiction, la fausseté et la conséquence sont
assurées — ce qui, en philosophie, est rarement le cas. Le raisonnement
d’ Épicure, sur le vide, n’a pas convaincu les stoïciens, ni ne convaincra les
cartésiens.

\section{Absurde (réduction à l'—)}
%ABSURDE (RÉDUCTION À L’—)
C’est comme une variété négative du précédent,
qui serait aussi son commencement :
la réduction à l'absurde démontre la fausseté d’une proposition par celle
d’une au moins de ses conséquences, dont on montre qu’elle est contradictoire
ou absurde. C’est suivre son adversaire, pour le réfuter — ou plutôt c’est
l’accompagner jusqu’au bout, où il se réfute lui-même.

Logiquement valide, le raisonnement n’a guère d’efficacité philosophique.
Sur les conséquences et sur l’absurdité, on peut presque toujours discuter.

\section{Abus}
%ABUS
Tout type d’excès. Mais le mot désigne spécialement un excès dans le
droit : l’abus est comme une injustice légale, ou qui paraît l’être (c’est
moins violer la loi que lutiliser indûment), et l'inverse par là de l'équité.

% 19
\section{Académiciens}
%ACADÉMICIENS
Membres d’une académie, ou de l’Académie (celle de
Platon puis de ses successeurs). Dans la langue philosophique
des {\footnotesize XVI$^\text{e}$} et {\footnotesize XVII$^\text{e}$} siècles,
le mot désigne surtout une variété de sceptiques,
par référence à la Nouvelle Académie, celle d’Arcésilas, Carnéade et Clitomaque.
Ceux-ci, se souvient Montaigne, « ont désespéré de leur quête, et jugé
que la vérité ne se pouvait concevoir par nos moyens ». La différence avec les
pyrrhoniens ? Elle est double, explique Montaigne : les Académiciens affirment
l'incertitude de tout (quand Pyrrhon n’affirme rien), tout en reconnaissant
qu’il y a du plus ou moins vraisemblable (quand Pyrrhon ne reconnaît rien).
Scepticisme plus extrême et plus modéré à la fois : c’est comme un scepticisme
dogmatique («une ignorance qui se sait », écrit Montaigne, ou qui prétend se
savoir), qui déboucherait sur un dogmatisme sceptique (un dogmatisme du
probable). La position des pyrrhoniens est à l’inverse : scepticisme sceptique
(une ignorance qui « s’ignore soi-même »), lequel ne débouche que sur le doute
ou le silence.

\section{Académie}
%ACADÉMIE
Le nom propre désigne d’abord l’école de Platon (parce qu’il
enseignait dans les jardins d’{\it Akadémos}, au nord-ouest d’Athènes),
laquelle devint plus tard, contre l'orientation de son fondateur, un foyer du
scepticisme. C’était une façon peut-être de revenir à Socrate, en s’éloignant de
Platon.
Par extension, le nom commun peut désigner tout rassemblement d’esprits
savants ou habiles, ou supposés tels.

\section{Académique}
%ACADÉMIQUE
Propre à l’école ou à l’université. Se prend le plus souvent
en mauvaise part («un style académique »). Synonyme à
peu près de scolaire, la prétention en plus. Ou de scolastique, la théologie en
moins.

\section{Académisme}
%ACADÉMISME
Soumission exagérée aux règles de l’école ou de la tradition,
au détriment de la liberté, de l’originalité, de l’invention,
de l’audace. Propension à imiter, chez les maîtres, ce qui est en effet imitable
(la doctrine, la manière, les tics), plutôt que ce qui importe vraiment, qui
ne l’est pas. En prose, goût immodéré pour le style savant ou universitaire.
C’est une façon de s'adresser à ses collègues plutôt qu’au public. En vain : les
collègues sont des rivaux, qui s’ennuient autant que les autres et haïssent davantage.

% 20
\section{Acceptation}
%ACCEPTATION
Accepter, c’est faire sien : c’est accueillir, recevoir, consentir,
c’est dire {\it oui} à ce qui est ou arrive. C’est la seule façon
de vivre {\it homologoumenos}, comme on disait en grec, c’est-à-dire en accord,
indissolublement, avec la nature et avec la raison. Refuser ? À quoi bon,
puisque cela ne change rien à ce qui est ? Mieux vaut accepter et agir.

On ne confondra pas l’acceptation avec la tolérance (qui suppose un reste
de refus ou de distance), ni avec la résignation (qui suppose un reste de tristesse).
L’acceptation vraie est joyeuse. C’est en quoi elle est le contenu principal
de la sagesse. Ainsi, chez Montaigne : « J’accepte de bon cœur, et reconnaissant,
ce que nature a fait pour moi, et m'en agrée et m’en loue. » Ou chez
Prajnânpad : « Ce que j'ai à vous dire est très simple et peut se résumer en un
seul mot : {\it Oui}. Oui à tout ce qui vient, à tout ce qui arrive. La voie c’est de
goûter les fruits et les richesses de la vie. » La voie, c’est de comprendre qu’il
n'y en a qu’une, qui est le monde, et qu’elle est à prendre ou à laisser. Accepter,
c’est prendre.

L’acceptation n’est pas non plus la même chose que la volonté : on veut ce
qui dépend de nous, comme disaient les stoïciens, on {\it accepte} ce qui n’en
dépend pas. Cela soulève un problème majeur, où le stoïcisme vient buter :
l'acceptation elle-même dépend-elle de nous ? Épictète répondait que oui, et
c'est ce que la vie, hélas, ne cesse de démentir. Cela ne dépend pourtant de personne
d’autre : l’acceptation dépend, non de ce que nous voulons, mais de ce
que nous sommes. Et qui se choisit soi ? Du moins peut-on se connaître, se
transformer, progresser, avancer. L’acceptation ne se décrète pas, mais
s’'apprend, mais s’entretient, mais se cultive. Travail sur soi, qui est le vrai
métier de vivre.

Peut-on tout accepter ? Même le mal ? Même le pire ? Il le faut, puisqu'on
ne pourrait autrement l’affronter. Comment se soigner, sans accepter d’être
malade ? Comment combattre, sans accepter le conflit ? Aimer ses ennemis,
cela suppose qu’on en ait, et qu’on accepte d’en avoir. Sagesse des Évangiles :
aimer, c’est dire oui. Mais ce n’est pas renoncer à agir, à affronter, à modifier.
Ainsi le sculpteur doit-il d’abord accepter le marbre, comme l’homme d’action
doit accepter le monde, pour le transformer. Ainsi les parents acceptent tout de
leurs enfants (ce qui signifie qu’ils les aiment comme ils sont, sans rien renier
ni rejeter), mais ne renoncent pas pour autant à les élever ni même, parfois, à
les punir. Il n’est pas interdit d’interdire, mais seulement de mébpriser, de
rejeter, de haïr. Il n’est pas interdit de dire {\it non}, si c’est pour un {\it oui} plus libre
ou plus clair. Acceptation n’est pas faiblesse. C’est force lucide et généreuse.

Son contraire est refus, ressentiment, déni, dénégation, forclusion. C’est
dire non au monde, et le commencement de la folie. L’acceptation, à l’inverse,
est le commencement de la sagesse.

% 21
\section{Accident}
%ACCIDENT
Ce qui arrive ({\it accidere} : tomber sur) à quelque chose ou à quelqu’un.
L'accident se distingue de {\it ce à quoi} il arrive (une substance,
un sujet), de {\it ce sans quoi} le sujet ne peut pas exister (son essence), enfin
des qualités spécifiques ou permanentes, qui demeurent et n’arrivent pas (les
propres). Par exemple, qu’un homme soit assis, c’est un accident (il pourrait
être debout ou couché sans cesser d’être homme). Qu'il soit humain, c’est
son essence. Qu'il soit capable de raison, de politique ou de rire, c’est son
propre.

C’est pourquoi Épicure disait que le temps est l'accident des accidents : à
tout ce qui arrive (par exemple être assis), il arrive de durer plus ou moins longtemps.
Le présent, en revanche, est le propre de l'être, comme l'être est
l'essence du présent.

Il en découle que tout est accidentel, y compris les propres, les essences, les
substances, puisque tout est dans le temps. Contingence de l’être, nécessité du
devenir : il n’y a que l’histoire.

\section{Acomisme}
%ACOSMISME
Le mot est forgé sur le modèle d’{\it athéisme} : ce serait ne pas
croire au monde. Hegel y voyait la position de Spinoza, qui
ne croirait qu’en Dieu ({\it Encyclopédie...}, I, \S 50). C’est bien sûr un contre-sens.
Si Dieu et la Nature sont une seule et même chose, la Nature existe
donc. Et le monde, qu’on le définisse comme l’ensemble infini des modes
finis (la nature naturée) ou comme le mode infini médiat de lattribut
étendue (la {\it facies totius universi} de la {\it Lettre LXIV}), existe lui aussi : il n’est
pas Dieu (puisqu'il est {\it en} Dieu et en résulte), mais il n’est pas rien. Le spinozisme
n’est ni un athéisme ni un acosmisme : la réalité du monde découle
nécessairement de la puissance de Dieu ou de la nature ({\it Éthique}, I, 16), et
la suppose (I, 15 et dém.). Ou pour le dire plus simplement, et comme faisait
Spinoza : « Plus nous connaissons les choses singulières, plus nous connaissons
Dieu » (V, 24).

\section{Acquis}
%ACQUIS
C’est un accident durable : quelque chose qui arrive (ce n'est ni un
propre ni une substance) et qui demeure. En pratique, s’oppose le
plus souvent à {\it inné} : est acquis tout ce que l’éducation, l’histoire ou la culture
— et non l’hérédité ou la nature — font de nous.

La querelle de l’inné et de l’acquis fit les beaux jours des années 1960-1970.
L’inné était censé être de droite, parce qu’il ne laissait aucune prise à la politique,
à la justice ou à l’histoire, ce qui le vouait, presque par définition, au
conservatisme ; l’acquis, inversement, semblait de gauche, parce qu’il ouvrait
% 22
une porte à l’action, au changement, au progrès. Un hebdomadaire titrait par
exemple : « Les dons n’existent pas. » C'était un hebdomadaire de gauche. Un
autre faisait sa une sur l’hérédité de l'intelligence : c'était un hebdomadaire de
droite. Deux demi-vérités ; une même erreur. L’acquis n’est pas moins réel que
l’inné, ni moins injuste.

Un phénomène humain un peu complexe se situe toujours à la charnière
entre l’un et l’autre. Par exemple la capacité langagière est innée ; et toute
langue est acquise. L'intelligence ? Les dons ? Ils supposent bien sûr un soubassement
biologique, donc inné, mais aussi une histoire, un développement, une
éducation — de lacquis. Double chance, ou malchance.

Mozart, si on ne lui avait pas appris la musique, n'aurait jamais écrit ses
opéras. Mais on pourrait m’apprendre toute la musique du monde, que je ne
serais pas Mozart pour autant. L’inné et l’acquis vont ensemble, toujours,
tantôt pour se renforcer mutuellement, tantôt pour s’équilibrer ou se nuire. On
naît humain, puis on le devient.

Cela vaut pour l'individu autant que pour l’espèce. L'histoire naturelle est
une histoire, comme l’histoire des hommes fait partie de la nature. Ainsi tout
est acquis, y compris l’inné.

\section{Acroamatique}
%ACROAMATIQUE
Synonyme savant ou aristotélicien d’ésotérique. Les écrits
acroamatiques d’Aristote sont ceux qui s'adressent à ses
élèves, par opposition aux écrits exotériques, presque tous perdus, qui s’adressaient
au grand public. La lecture des premiers donne une haute idée des élèves
d’Aristote ; et fait regretter la disparition des seconds, que les Anciens admiraient
fort.

\section{Acte}
%ACTE
Ce qui est fait ({\it actum}, du verbe {\it agere}, faire). Au sens psychologique
ou éthique, c’est un synonyme d’action, à ceci près qu’il peut exister
des actes involontaires (les actes manqués, les tics, les maladresses) : ce qui est
fait s’oppose à ce qui est seulement subi ou ressenti. Au sens ontologique, l’acte
s'oppose à la puissance, comme le réel (ce qui est fait) au possible (ce qui peut
l'être). « L'acte, disait Aristote, est le fait pour une chose d'exister en réalité. »
Par exemple la statue est en puissance dans le marbre, mais en acte seulement
quand le sculpteur l’a terminée.
Les deux notions sont bien sûr relatives. Le chêne est en puissance dans le
gland, qui existe en acte, comme le gland en puissance dans le chêne. Et le
marbre est en acte (réel, complet, achevé) avant comme après la statue. L'acte
% 23
est pourtant la notion première, comme Aristote l’avait vu : le possible naît du
réel, non le réel du possible.

Au présent, d’ailleurs, l’acte et la puissance ne font qu’un : rien n’est possible,
ici et maintenant, que ce qui est. C’est l'être même, qui est puissance en
acte {\it (energeia, conatus)}.

\section{Acte manqué}
%ACTE MANQUÉ
C’est un acte qui échoue à atteindre le résultat que visait
son auteur, alors que cela ne présentait aucune difficulté,
et qui en atteint un autre, qu'il ne visait pas, du moins consciemment : casser
le vase qu’on voulait ranger, égarer un objet qu’on voulait conserver, oublier un
rendez-vous où l’on voulait aller, dire un mot à la place d’un autre (voir
« lapsus »).. La psychanalyse, qui ne croit pas au hasard, y voit la manifestation
d’un désir refoulé, qui vient perturber l’enchaînement conscient et volontaire
de nos actes : tout acte manqué serait un discours réussi. À quoi je ne vois
guère à reprendre que le «tout». Pourquoi l'inconscient devrait-il réussir
toujours ? Pourquoi le corps n’aurait-il ses échecs, ses ratés, ses cafouillages ?
Pourquoi cette omniprésence du sens ? Au demeurant, cela n'importe guère.

L’inconscient, pour la raison, n’est jamais qu’un hasard parmi d’autres.

\section{Action}
%ACTION
C’est un effet de la volonté. Ni une volonté sans effet n’est action, ni
un effet sans volonté. Agir c’est faire ce qu’on veut, et en cela c’est
être libre.

Mais qui veut ? L'âme. Et qui fait ? Le corps, pour autant qu’on puisse,
illusoirement, les distinguer l’un de l’autre. L’action est ainsi soumission du
corps à l’âme, et libre encore en ce sens. Elle s'oppose à la passion, où le
corps commande (où l’âme, comme le mot l’indique, {\it pâtit}), et où la liberté
se perd.

Une action supposant toujours un sujet, avec son corps et son histoire,
toute volonté est pourtant déterminée avant d’être déterminante. C’est pourquoi
nulle action n’est libre, absolument parlant, mais seulement libérée plus
ou moins des contraintes et des déterminations extérieures. Est-ce possible
totalement ? Il semble que la pensée y parvienne, parfois, quand elle raisonne.
Mais la raison est sans effet, et ne dispense pas d’agir.

\section{Activisme}
%ACTIVISME
Confiance exagérée en l’action et en ses pouvoirs. C’est le contraire
du théoricisme. Leur remède commun ? L'action lucide et
réfléchie : la pensée en acte.

% 24
\section{Actualisme}
%ACTUALISME
Doctrine pour laquelle tout ce qui existe est actuel, ou en acte.
Parce qu’il n’y a pas de possible ? Non pas. Mais parce que le
possible et le réel, au présent, ne font qu’un. Le stoïcisme et le spinozisme, par
exemple, sont des actualismes ; c’est ce qui fait une partie essentielle, à mes
yeux, de leur vérité. Il n’y a pas d’être en puissance : il n’y a que la puissance de
l'être, et son perpétuel {\it passage à l'acte}, qui est le monde ou le devenir.

\section{Adaptation}
%ADAPTATION
C’est changer ce qui peut l'être, pour affronter ce qui ne le
peut. Par exemple changer ses désirs, comme disait Descartes,
plutôt que l’ordre du monde. Ou changer la société, comme pourrait
dire un marxiste intelligent, plutôt que la nature humaine.
C’est pourquoi la vie est adaptation : parce que le réel lui impose sa loi, qui
est de changement ou de mort.

\section{Adéquation}
%ADÉQUATION
C’est une correspondance parfaite, ou supposée telle, entre
deux êtres, spécialement entre une idée et son objet. Correspondance
mystérieuse, puisque ces deux êtres sont réputés différents, et toujours
impossible à vérifier absolument. Le seul moyen que nous aurions de vérifier
cette adéquation serait de comparer l’objet et l’idée. Mais nous ne les
connaissons l’un et l’autre que par les idées que nous nous en faisons.

Saint Thomas, après Avicenne et Averroës, définissait la vérité comme {\it ad{\ae}quatio
rei et intellectus}, adéquation entre la chose et l’entendement. Mais cette
adéquation n’est possible que pour la même raison qui la rend nécessaire :
parce que la chose et l’entendement sont deux. On évitera de la penser en
termes de ressemblance. Si l’idée de cercle n’est pas ronde, si l’idée de chien
n’aboie pas, elles ne sauraient ressembler à un chien ou à un cercle. Mais elles en
disent, sans sortir de la pensée, la vérité — que ni le cercle ni le chien ne
connaissent.

« Par {\it idée adéquate}, écrit Spinoza, j'entends une idée qui, en tant qu’elle est
considérée en elle-même, sans relation à un objet, a toutes les propriétés ou
dénominations intrinsèques d’une idée vraie.» On ne pourrait autrement
savoir qu’elle est vraie (puisqu'on ne pourrait la comparer à son objet qu’à la
condition qu’il fût en nous, ce qui n’est pas), et c’est pourquoi on ne le peut
jamais absolument. Il n’en reste pas moins qu’« une idée vraie doit s’accorder
avec l’objet dont elle est l’idée », comme dit encore Spinoza. Cette adéquation
est le vrai mystère de la pensée. Est-ce l’univers qui est adéquat aux mathématiques,
ou les mathématiques qui sont adéquates à l’univers ?

% 25
De ce mystère, Dieu serait la solution. Mais nous n’en avons, malgré Spinoza,
aucune idée adéquate.

\section{Admiration}
%ADMIRATION
Le mot, en français, signifia d’abord étonnement. Par exemple
chez Descartes : « L’admiration est une subite surprise de
l’âme, qui fait qu’elle se porte à considérer avec attention les objets qui lui semblent
rares et extraordinaires. » C’est ce sens, aujourd’hui tombé en désuétude,
qui faisait dire à Montaigne que « l'admiration est le fondement de toute
philosophie ». J’y vois une espèce de leçon, qui mène au sens moderne. Rien ne
surprend comme la grandeur, et c’est l’admiration vraie : l’étonnement joyeux
ou reconnaissant devant ce qui nous dépasse.
Son contraire est mépris ; son absence, petitesse.

\section{Adoration}
%ADORATION
« Tu n’adoreras que Dieu. » Ce commandement vaut comme

définition : l’adoration est un amour exclusif, pour un objet
qui le justifie et le dépasse absolument. Appliqué aux êtres d’ici-bas, c’est idolâtrie,
fanatisme ou jobardise. C’est l’amour qui croit en la perfection de son
objet, et qui l'aime pour cela : excès de crédulité, défaut d'amour.

Mieux vaut la tendresse, qui n’aime rien tant que la faiblesse de son objet.
Ou l'amour, qui donne de la valeur à ce qu’il aime et n’en dépend pas.

On remarquera que l’adoration, en bonne théologie, n’est pas réciproque
(Dieu nous aime, il ne nous adore pas), alors que la charité peut l’être. Adorer,
c’est prendre modèle sur les prêtres, plutôt que sur le Christ. Toute adoration
est religieuse ; seule la charité est divine.

\section{Adulte}
%ADULTE
Celui dont le corps a cessé de grandir — qui ne peut plus grandir
que par l’âme. C’est être fidèle à l'enfance, en refusant de s’y
enfermer. Car tous les enfants veulent grandir. L’infantilisme est une maladie
de vieillards.

\section{Affect}
%AFFECT
C’est le nom commun et savant des sentiments, des passions, des
émotions, des désirs — de tout ce qui nous {\it affecte} agréablement ou
désagréablement. On objectera que le corps est sensible aussi. Certes. Mais un
affect est comme l’écho en nous de ce que le corps fait ou subit. Le corps sent ;
lâme ressent, et c’est ce qu’on appelle un affect.

% 26
Que serait la douleur, si elle n’était ressentie par personne ? Une pure réaction
physiologique, qui ne serait plus une {\it douleur} à proprement parler. Même
chose, bien sûr, pour le plaisir. Ce sont les deux affects fondamentaux. La joie ?
C’est un plaisir de l’âme. La tristesse ? Une souffrance de l’âme. Le désir ? La
polarisation qui résulte, en nous, de leur opposition réelle ou imaginaire. De là
le principe de plaisir, comme dit Freud, qui est la grande loi de notre vie affective.

« Par affect {\it (affectum)}, écrit Spinoza, j'entends les modifications du corps
{\it (corporis affectiones)}, par lesquelles la puissance d’agir de ce corps est accrue ou
diminuée, secondée ou réduite, et en même temps les idées de ces affections »
({\it Éthique}, III, déf. 3). L'existence n’est pas un absolu : on existe {\it plus ou moins},
tels sont nos affects, et le plus qu’on peut. L'âme et le corps sont une seule et
même chose : rien ne se passe dans celle-là qui ne se passe aussi dans celui-ci, et
réciproquement. L’affect est le nom de cette unité, en tant qu’elle exprime une
augmentation ou une diminution de notre puissance d’exister et d’agir. C’est
l'effort de vivre (le {\it conatus}), considéré dans ses fluctuations positives ou négatives.
Nos affects sont souvent des passions (quand ces fluctuations ne dépendent
pas de nous, ou n’en dépendent que partiellement), parfois des actions
(quand nous en sommes la cause adéquate : {\it Éthique}, III, déf. 2). Toute joie est
bonne, mais toutes les joies ne se valent pas.

\section{Affectation}
%AFFECTATION
L’imitation prétentieuse d’un affect : c’est faire semblant de
ressentir ce qu'on ne ressent pas, pour se distinguer ou se
faire valoir. Ainsi le snob imite la noblesse qu’il n’a pas (il s’agit bien sûr de la
noblesse du cœur : un aristocrate peut être snob) ou la culture qui lui manque,
comme le tartufe affiche une dévotion feinte. C’est le contraire du naturel ou
de la simplicité.

\section{Affection}
%AFFECTION
Chez Spinoza, c’est une modification de la substance ou du
corps (on évitera donc, malgré certains traducteurs, de confondre
l’{\it affectio} et l’{\it affectus}, l'{\it affection} et l'{\it affect} : voir ce mot). Mais cette acception
n'est d'usage que chez les spécialistes. Dans le langage courant, l’affection est
un affect particulier, et particulièrement doux : c’est un amour sans passion,
sans violence, sans jalousie. Sans désir ? Pas forcément. Mais Le désir s’y ajoute,
le cas échéant, plutôt qu’il n’en vient ou ne l’engendre — à moins que l’affection,
cela arrive aussi, ne s'ajoute au désir.
L’affection peut porter sur plusieurs personnes, point sur toutes (c’est ce
qui la distingue de la charité). Son lieu de prédilection est la famille, et tout ce
%— 27
qui lui ressemble : c’est l'amour pour les proches ou les intimes, comme une
amitié tendre.

\section{Affirmer}
%AFFIRMER
C'est dire ce qu’on croit, ou sait, ou veut être vrai. Ainsi l’assentiment,
chez les stoïciens. C’est aussi le prendre à son compte,
l’approuver, s’en réjouir. Ainsi l’{\it amor fati}, chez Nietzsche : « l'affirmation dionysiaque
de l’univers tel qu’il est, sans possibilité de soustraction, d'exception
ou de choix ». C’est l'acceptation entière, avec quelque chose de plus, et peut-être
de trop — comme un {\it oui} plus résolu ou plus enthousiaste au réel. Voyez
Zarathoustra, qui « dit {\it oui} et {\it Amen} d’une façon énorme et illimitée », qui se
veut lui-même « l’éternelle affirmation de toutes choses ».. Le réel lui est
monté à la tête.

La phrase la plus affirmative de l’histoire de la philosophie ? Celle-ci, qui
est de Spinoza : « Par réalité et par perfection, j'entends la même chose. »

Il m'est arrivé, écoutant Mozart, de le comprendre à peu près. Et plus souvent,
écoutant Schubert, de ne pouvoir même essayer.

Dire {\it oui}, il le faut bien. Dire {\it Amen}, c’est trop nous demander.

\section{\it Agapé}
%{\it AGAPÈ}
C’est le nom chrétien, en grec, de l'amour de charité (voir ces mots).
C’est l'amour qui donne, sans avoir besoin pour cela de recevoir,
d’être aimé, ni même d’espérer. Le pur amour : l’amour à l’état pur. Il n’est pas
fondé sur la valeur de son objet (au contraire d’{\it Éros}), mais lui en donne: il
n’aime pas ce qui est aimable, il rend aimable ce qu’il aime. Il n’est pas fondé
sur la joie du sujet (au contraire de {\it Philia}), mais le réjouit : il n'aime pas parce
qu'il est joyeux, il est joyeux parce qu’il aime. Par quoi il est universel et
désintéressé : libéré de l'{\it ego} et de l’égoïsme. Ce serait l'amour de Dieu, s'il
existe ({\it o Théos agapè estin}, lit-on chez saint Jean : « Dieu est amour »), et ce qui
lui ressemble le plus, si Dieu n’existe pas. Que nous en soyons capables est
douteux ; mais quand bien même ce ne serait qu’un rêve ou qu’un idéal, il
indique une direction, qui est celle de l'amour sans mesure, comme disait saint
Augustin, sans attache, comme disait Pascal, enfin sans appartenance, comme
dit Bobin, et presque sans sujet : c’est l'amour dépris de soi et de tout.

\section{Agnosticisme}
%AGNOSTICISME
Nous ne savons pas si Dieu existe ; nous ne pouvons le
savoir. C’est ce qui justifie la foi et l’athéisme, qui sont
deux croyances. C’est ce qui justifie aussi lagnosticisme, qui refuse de croire ce
qu’il ignore. Position respectable, cela va de soi, et qui semble de bon sens.

% 28
Pourquoi faudrait-il choisir sans savoir ? Il se peut toutefois que l’apparence ici
soit trompeuse. Si l’on savait, la question du choix ne se poserait plus. Et qui
peut vivre sans croyance ?

{\it Agnôstos}, en grec, c’est l'inconnu ou l’inconnaissable. Être agnostique, c’est
prendre cet inconnu au sérieux, et refuser d’en sortir : c’est reconnaître ou
affirmer qu’on ne sait pas. Le mot, qui serait susceptible d’une extension plus
large, n’est guère utilisé qu’en matière de religion. C’est que Dieu est l’inconnaissable
absolu, comme la mort l’inconnaissable ultime. L’agnostique ne
prend position ni sur celle-ci ni sur celui-là. Il laisse la question ouverte. La
mort fermera la porte ou allumera la lumière.

La faiblesse de la notion tient à son évidence : sa limite est de n’en pas
avoir. Puisque personne ne sait si Dieu existe, nous devrions tous être agnostiques.
Mais cet aveu d’ignorance cesserait alors d’être une position particulière,
pour devenir un trait général de la condition humaine. Que resterait-il de
l’agnosticisme ? C’est dire qu’il n’existe que par différence : être agnostique,
c'est moins reconnaître ne pas savoir (beaucoup d’athées et de croyants le
reconnaissent également), que vouloir s’en tenir à cette ignorance. Que cette
position soit plus juste que les autres, c’est pourtant ce que nul savoir ne
garantit. Il faut donc y croire, et c’est en quoi l’agnosticisme, lui aussi, est une
espèce de foi, mais seulement négative : c’est croire qu’on ne croit pas.

\section{Agonie}
%AGONIE
{\it Agônia}, en grec, c’est l’angoisse. Mais {\it Agôn}, c’est le combat. L’agonie
est le dernier: c’est le dernier combat perdu de vivre. D'où
l'angoisse, chez presque tous. Et l’acceptation, chez les plus sages. La seule victoire
ici est la paix. Heureux ceux qui l’auront connue de leur vivant. Se battre
jusqu’au bout ? À quoi bon, si c’est pour mourir en état de guerre ? Mieux vaut
quitter la vie, quand elle nous quitte, avec douceur et dignité. Merci, amis
médecins, le moment venu, de nous y aider.

\section{\it Agora}
%{\it AGORA}
La place publique, en Grèce et spécialement à Athènes : c’est là que
Socrate philosophait. Mais c'était aussi, et avant tout, le centre de
la vie sociale et politique. Aussi le mot est-il devenu, dans la plupart des langues
modernes, le symbole du débat démocratique. C’est ainsi qu’un collègue, lors
d’un colloque, me reprocha d’avoir « déserté l’{\it agora} » (parce que je revenais au
vieux thème de la sagesse, au lieu de m’engager dans les combats du jour). Le
même me reprochera, quelques secondes plus tard, d’être « un intellectuel
médiatique », parce qu’il m'avait vu à la télévision. Que lui répondre, sinon
que les médias, aujourd’hui, font partie de l’agora, et qu’il n’y a nulle contradiction
%—29
entre chercher la sagesse, comme philosophe, et la justice, comme
citoyen ? Socrate, qui philosophait sur la place publique, n’a jamais confondu
la pensée avec un bulletin de vote.

\section{Agréable}
%AGRÉABLE
L'histoire est fameuse. C’est un fou qui se tape sur la tête, à grands
coups de marteau. On l’interroge sur ses motivations. Il répond :
« Ça fait tellement de bien quand on s'arrête ! »

Qu'est-ce qui est agréable ? Tout ce qui cause du plaisir ou du bien-être. La
cause peut être positive ou négative, apporter une jouissance ou diminuer une
souffrance, l'effet n’en est pas moins agréable pour autant. Par exemple retirer
ses chaussures quand on a mal aux pieds, plonger dans un bain chaud, se laisser
masser ou caresser. « Est agréable, écrit Kant, ce qui plaît aux sens dans la
sensation. » Mais ce qui plaît à l'esprit ne l’est pas moins. Voyez l'humour ou
l'amour. Ainsi tout ce qui plaît est agréable, et cette tautologie vaut comme
définition : l’agréable c’est ce qui nous plaît ou nous agrée au moins en quelque
chose. Notion par nature relative, mais sans laquelle aucune relation ne saurait
tout à fait nous satisfaire. Triste amitié, qu’une amitié sans plaisir.

L’agréable n’est pas le bien (un plaisir peut être coupable, une souffrance peut
être méritoire), mais il est {\it un} bien. Le premier, sans doute, et le principe de tous.
Si le bonheur n’était agréable, à quoi bon le bonheur ? Et que vaudrait une vertu
qui ne serait agréable pour personne ? Cessez de vous taper sur la tête.

\section{Agressivité}
%AGRESSIVITÉ
Disposition à la violence, physique ou verbale, avec une propension
à attaquer le premier. C’est à la fois une force et une
faiblesse : c’est la force des faibles. Ils croient que la meilleure défense, c’est
l'attaque. Ils ont raison. Mais pourquoi ont-ils tellement besoin de se défendre ?

\section{Aléas}
%ALÉAS
Ce sont des causes sans raison, sans but, sans lien — des causes sans
prétention. Elles sont pour cela imprévisibles, et fâcheuses le plus
souvent. Le réel n’a que faire de nos désirs, et c’est ce que les aléas nous rappellent.
Ce sont les grains de sable de l'inconnu. Le mot latin désignait les {\it dés}, où
le hasard se joue. Le danger est d’y voir une fatalité ou une providence. Les dés
ne jouent pas à Dieu.

\section{Aléatoire}
%ALÉATOIRE
Tout ce qui relève du hasard, de la rencontre, de l’imprévisible
— des aléas. On ne le confondra ni avec l’indéterminé ni avec
% 30
l’inexplicable. Rien de plus déterminé qu’un dé qui roule. Rien de plus aléatoire
que son résultat à venir. Rien ? Si, pourtant : ce que feront deux dés qui
roulent, ou trois, ou quatre... L’aléatoire se calcule (voyez les probabilités).
L’aléatoire s'explique (voyez la météorologie). Mais il ne se prévoit pas, ou mal.
Il faut donc en tenir compte sans le connaître : de là une vertu, qui est la prudence,
et un danger, qui est la superstition. Mieux vaut veiller à ce qui dépend
de nous, que vouloir deviner ou influencer ce qui n’en dépend pas. L’un
n'empêche pas l’autre ? Sans doute. Mais n’en tient pas lieu non plus. La proportion,
entre les deux, varie en fonction des individus comme des situations
(selon qu’elles dépendent {\it plus ou moins} de nous). C’est pourquoi les joueurs
sont superstitieux, presque toujours. Et les hommes d’action, prudents.

Il y a des degrés dans l’aléatoire. Le temps qu’il fera demain est moins aléatoire
que celui qu’il fera dans dix jours (il peut même arriver, dans certaines
conditions climatiques, qu’il ne le soit presque pas) ; le bridge, moins aléatoire
que la roulette. Mais l’aléatoire, dans tous les cas, ne porte que sur l’avenir. Dès
que la bille s'arrête, le résultat cesse d’être aléatoire, ou ne l’est que rétrospectivement.
Le temps qu’il fera est aléatoire ; le temps qu’il fait ne l’est pas. Un
parapluie, dans tous les cas, est plus utile qu’un talisman. Mais il ne sert à rien
de l’ouvrir quand il fait beau.

\section{\it Alèthéia}
%{\it ALÈTHÉIA}
Le nom grec de la vérité. Souvent opposé, depuis Heidegger, à la
{\it veritas}, qui est son nom latin et scolastique. L’{\it alèthéia} est de
l'être : c’est son dévoilement, ou plutôt c’est l'être même, en tant que non voilé.
La {\it veritas} est de l’esprit ou du discours : c’est la correspondance, la conformité
ou l’adéquation entre ce qui est pensé ou dit et ce qui est. Distinction commode
et légitime. On remarquera pourtant que ni les Grecs ni les Latins ne la
faisaient en ces termes. Et que la faire ne saurait autoriser à rejeter l’une de ces
deux acceptions, qui renvoient l’une à l’autre. Que saurions-nous de l'être, si
notre pensée ne pouvait lui correspondre ? Et quel sens y aurait-il à juger
qu’une pensée est adéquate à ce qui est, si l’être n’était d’abord adéquat à lui-même ?
Par quoi l’identité est peut-être la vérité première : « Même chose,
disait Parménide, se donne à penser et à être. » La vérité, que ce soit comme
{\it alèthéia} où comme {\it veritas}, est ce {\it même} qui se donne à l'esprit ({\it veritas}) et au
monde ({\it alèthéia}). Les deux idées, même solidaires, même indissociables en
toute rigueur (puisque la {\it veritas} suppose l’{\it alèthéia} et permet seule de la penser),
n’en demeurent pas moins différentes. L’{\it alèthéia} est vérité de la présentation ;
la {\it veritas}, de la représentation. Ainsi c’est l’{\it alèthéia} qui est première ; mais on
ne peut la penser que par la {\it veritas}. La {\it veritas} est en nous, mais parce que nous
sommes d’abord dans l’{\it alèthéia}. Nous n’avons accès à la vérité que parce que
% 31
nous sommes déjà dans le vrai ; l'esprit ne se donne à soi qu’en s’ouvrant au
monde.

\section{Aliénation}
%ALIÉNATION
Aliéner, c’est perdre : perdre ce qu’on possède (par exemple
par don ou vente : c’est le sens juridique), ce qu’on fait ou
produit (par exemple parce qu’on est exploité : c’est le sens socio-économique),
voire ce qu’on est (par exemple dans l’aliénation mentale : c’est le sens psychiatrique).
Quant au sens philosophique, il peut traverser ou inclure les trois précédents.
De là un certain flou, qui rend la notion commode et suspecte. On
parle d’aliénation, dans la philosophie moderne, quand quelqu'un devient
comme étranger {\it (alienus)} à lui-même, quand il ne s’appartient pas ou plus,
quand il ne se comprend plus, ne se maîtrise plus, quand il est dépossédé de son
essence ou de sa liberté. C’est le supposer d’abord propriétaire de lui-même,
voué à la libre et familière transparence de soi à soi. Qui peut y croire ? Ou bien
il faut supposer quelque chose comme une aliénation originaire (le fameux « Je
est un autre » de Rimbaud), dont il s’agit de s’affranchir. Être aliéné, c’est être
soumis à ce qui n'est pas soi. Et nul ne devient {\it soi} autrement. Le concept d’aliénation
ne vaut que par celui, symétrique, de libération. « Là où {\it ça} était, écrit
Freud, {\it je} dois avenir. »

Chez Hegel, l’Idée s’aliène dans la nature, comme l'Esprit dans l’espace et
le temps. Chez Feuerbach, l’homme s’aliène en Dieu. Chez Marx (surtout celui
des {\it Manuscrits de 1844}), le prolétaire s’aliène dans le travail salarié (« l’ouvrier
se vend pour vivre »), dans son œuvre (qui lui échappe et le domine), puis tous
dans l’idéologie dominante, qui masque et exprime à la fois cet asservissement.
C‘était s'approcher de la vérité. Aliéner c’est perdre, ou se perdre. Mais nul ne
possède ce qu’il est. On ne possède, dans le meilleur des cas, que ce qu’on fait.

\section{Allégorie}
%ALLÉGORIE
C'est exprimer une idée par une image ou un récit : l’inverse de
l’abstraction, comme une pensée figurative. L’allégorie, d’un
point de vue philosophique, ne prouve jamais rien. Et il n’y a guère que chez
Platon qu’elle ne soit pas simplement ridicule.

\section{Allégresse}
%ALLÉGRESSE
Une joie redoublée par ses signes ou par la conscience qu’on
en prend, et pour cela d’autant plus vive qu’elle se manifeste
ou se sait davantage. Joie d’être joyeux. C’est le rire silencieux de l'esprit, ou
bruyant, de la foule.

% 32
\section{Alphabet}
%ALPHABET
Les lettres, mises en ordre par le hasard ou la coutume. L’ordre
alphabétique, qui n’est ainsi qu’un désordre parmi d’autres, a ce
mérite singulier de ne présenter d'ordre, puisqu'il en faut un, que réduit à sa
plus simple expression, sans aucune prétention au sens. L’alphabet, c’est sa
vertu, ne veut rien dire. Et ce livre, par exemple, est adéquat à son objet, en ce
qu’il ne peut être lu, comme le monde ou la vérité, que {\it dans le désordre}. Un dictionnaire
est le contraire d’un système, et vaut mieux.

\section{Altération}
%ALTÉRATION
Le fait de devenir autre, ce qui suppose qu’on reste le même
(on devient autre, non {\it un} autre) tout en perdant un certain
nombre de ses qualités ou en en acquérant de nouvelles. Le mot se prend le
plus souvent en mauvaise part : l’altération se dit surtout d’un changement
négatif, d’une déperdition, d’une corruption, d’une détérioration. Il n’y a à
cela aucune nécessité étymologique ni conceptuelle. Pourquoi l’autre {\it (alter)}
serait-il moins bien que le même ? Mais c’est que le même nous rassure davantage.
Et que le changement ordinaire du vivant prend en effet la forme, très tôt,
d’une détérioration. Grandir, c’est devenir soi (individuation). Vieillir, c’est
devenir autre tout en restant le même (altération). Et qui ne préférerait rester
jeune ?

Ainsi l’altération est la loi du vivant, comme l'identité la loi de l’être. Cela
confirme — puisque le vivant fait partie de l'être — que les deux notions ne sont
ni des contraires ni des contradictoires. Comment le seraient-elles ? L’altération
est la forme vivante de l’identité : c’est le devenir autre du même (le symétrique
en cela du mimétisme, qui est le devenir même de l’autre).

\section{Altérité}
%ALTÉRITÉ
Caractère de ce qui est autre, ou un autre. L’altérité, à la différence
de l’altération, suppose une relation entre deux êtres distincts
ou supposés tels. C’est le contraire de l'identité, comme l’autre est le
contraire du même. On pourrait en faire un principe : toute chose étant identique
à soi (principe d'identité) est aussi différente de toutes les autres (principe
d’altérité). La tradition parle plutôt de principe des indiscernables. Mais les
deux principes sont différents. Quand bien même il existerait deux êtres parfaitement
semblables, ils n’en seraient pas moins numériquement différents l’un
de l’autre. Les scolastiques parlaient, en un sens voisin, de principe d’individuation,
Mais celui-ci vaut au sein d’une même espèce, quand le principe d’altérité
vaut absolument. C’est ce qui interdit à deux êtres de ne faire qu’un, et le principe,
pour nous, de la solitude.

% 33
\section{Alternative}
%ALTERNATIVE
C'est un choix imposé, entre deux termes, de telle sorte qu’on
ne puisse ni vouloir ni refuser les deux à la fois. Par exemple :
être ou ne pas être. L’alternative est un choix où l’on n’a pas le choix du choix.
Se dit spécialement, en logique, d’un couple de propositions, dont l’une est
vraie, l’autre fausse, nécessairement. C’est donc une disjonction exclusive : {\it p ou
q}, et non {\it p et q}. On appelle parfois « principe de l'alternative » le principe qui
stipule que deux propositions contradictoires constituent toujours une alternative.
Mais ce n’est pas un principe à proprement parler: ce n'est que la
conjonction des deux principes de non-contradiction [non ({\it p} et {\it non-p})| et du
tiers exclu [{\it p} ou {\it non-p}]. Deux propositions contradictoires ne peuvent être
vraies toutes les deux (principe de non-contradiction), ni fausses toutes les deux
(principe du tiers exclu) : l’une est donc vraie, l’autre fausse, nécessairement
(« principe » de l'alternative). On remarquera toutefois que cela ne vaut que des
propositions. Des discours qui ne sont ni vrais ni faux (par exemple une prière)
ne sauraient constituer une alternative nécessaire. C’est qu’ils ignorent la
logique, comme la logique les ignore.

\section{Altruisme}
%ALTRUISME
C'est « vivre pour autrui », disait Auguste Comte, autrement dit
tenir compte des intérêts de l’autre plus que des siens propres,
ce qui n’arrive presque jamais, ou autant, ce qui est déjà bien difficile. C’est
donc le contraire de l’égoïsme ; c’est pourquoi c’est si rare. Le contraire ? Et si
ce n’était que son expression déguisée ? « Sœur Emmanuelle ou l'Abbé Pierre,
me disait un ami, quand ils font du bien aux autres, cela leur fait plaisir à eux-mêmes :
leur altruisme n’est qu’une forme encore d’égoïsme ! » Soit. Mais cela
ne prouve rien contre l’altruisme. Trouver son plaisir dans le plaisir de l’autre,
loin que cela récuse l’altruisme, ce serait plutôt sa définition. On ne sort pas du
principe de plaisir, ni de l’égoïsme. Mais certains s'y enferment, quand
d’autres, sans en sortir, y trouvent le secret de la liberté. « Aimer, disait Leibniz,
c’est se réjouir du bonheur d’un autre. » C’est l’altruisme vrai, ou sa forme la
plus pure. Il s’agit non de vaincre l'{\it ego}, mais de l'ouvrir — devenir, explique Prajnânpad,
comme « un cercle devenu si large qu’il ne peut plus rien entourer, un
cercle d’un rayon infini : une ligne droite ».

Le mot, qui fut forgé par Auguste Comte, gêne pourtant par ce qu'il a
d’abstrait, de prétendument explicatif ou théorique. On se trompe si l'on voit
dans l’altruisme un instinct ou un système. Prendre en compte les intérêts de
l’autre, autant ou plus que les siens propres, cela ne va pas, selon les cas, sans
effort, sans tristesse ou sans joie : cela ne va pas sans générosité, sans compassion,
sans amour. Deux vertus, une grâce. L’altruisme, sans elles, n’est qu'une
abstraction ou un mensonge.

% 34
\section{Ambiguïté}
%AMBIGUÏTÉ
Une expression ou un comportement sont ambigus s’ils autorisent
deux ou plusieurs interprétations différentes, voire opposées.
L’ambiguïté est donc une dualité ou une pluralité de significations possibles.
On ne la confondra pas avec l’ambivalence, qui est une dualité — et
presque toujours une opposition — de valeurs ou de sentiments réels.

L’ambiguïté suppose une certaine complexité : elle est un fait de l’homme
ou du discours. C’est ce qui la distingue de la polysémie, qui est un fait de la
langue et ne se dit guère que pour un seul signifiant. Une phrase peut être
ambiguë ; un mot pris en lui-même ne sera, s’il a plusieurs sens, que polysémique.
Mais la polysémie d’un mot peut rendre une expression ambiguë : si je
parle du « sens de l’histoire », ai-je en vue sa direction ou sa signification ?

L’ambiguïté, à la différence de l’amphibologie, n’est pas toujours fautive et
ne saurait être éliminée totalement. C’est comme un halo de sens, dans la nuit
du réel. Ce n’est pas une raison pour la cultiver à dessein. Le halo naît de la
lumière, mais n’en tient pas lieu.

\section{Ambition}
%AMBITION
C'est la passion de réussir, tournée surtout vers l’avenir et l’action.
L’ambition porte moins sur ce qu’on est (comme l’orgueil) ou fait (comme
le zèle) que sur ce qu’on sera ou fera. C’est un goût immodéré pour les succès
à venir. Encore faut-il qu’elle se donne au moins certains des moyens nécessaires
à sa réalisation : c’est ce qui distingue l’ambitieux du rêveur puis du raté.
Passion prospective, mais actuelle et active.

Il y a un âge pour l’ambition. Utile dans la jeunesse ou la maturité (c’est
une passion tonique), elle devient vaine chez les vieilles gens, qui n’ont plus
d’ambitions que posthumes (la postérité, la descendance, le salut...) ou sordides
(conserver la vie, l'argent, le pouvoir...). Aie l'ambition, plutôt, de te
guérir de l’ambition. Le succès y aide. Aussi l’échec et la fatigue.

\section{Ambivalence}
%AMBIVALENCE
C'est la coexistence, chez un même individu et dans sa relation
à un même objet, de deux affects opposés : plaisir et
souffrance, amour et haine (voir par exemple Spinoza, {\it Éthique}, IIL, 17 et
scolie), attirance et répulsion... Loin d’être une exception, l’'ambivalence serait
plutôt la règle de notre vie affective, comme l'ambiguïté est celle de notre vie
communicationnelle. La simplicité, dans les deux cas, est l’exception.
On notera que l’ambivalence, qui ne vaut que pour les sentiments, ne saurait
nous dispenser de respecter la logique, qui vaut pour les idées. Par exemple
l'inconscient, disait Freud, «n’est pas soumis au principe de non-contra-diction ».

% 35
Mais la psychanalyse l’est. Elle ne serait autrement qu’un délire ou
un symptôme de plus.

\section{Âme}
%ÂME
L'âme {\it (anima, psuchè)} est ce qui anime le corps : ce qui lui permet de se
mouvoir, de sentir et de ressentir. Pour le matérialiste, c’est donc le
corps lui-même, considéré dans sa motricité, dans sa sensibilité, dans son affectivité
propres. Dans son intellectualité ? Pas nécessairement. Un animal peut
sentir et ressentir (avoir une Âme) sans être capable pour autant de penser ou de
raisonner abstraitement (sans être esprit). C’est ce qu’on appelle une bête.
Mieux vaut en effet parler d'esprit {\it (mens, noûs)}, plutôt que d’âme, pour désigner
cette partie du corps qui a accès au vrai ou à l’idée. Cela fait, entre les
deux, une autre différence. Perdre l'esprit, comme chacun sait, n’est pas la
même chose que perdre son âme. Perdre l'esprit, c’est perdre la raison, le bon
sens, qui est le sens commun : c’est perdre l'accès à l’universel et devenir du
même coup prisonnier de son âme. Le fou n’est pas moins {\it soi} que n'importe
qui, pas moins particulier, pas moins singulier, bien au contraire : il n'est plus
{\it que soi}, et c’est cet enfermement — coupé du vrai, coupé du monde, coupé de
tout — qui le rend fou. L'esprit ouvre la fenêtre, et c’est ce qu’on appelle la
raison.

L'âme est toujours individuelle, singulière, incarnée (il n’y a pas d’âme du
monde, ni de Dieu). L'esprit serait plutôt anonyme ou universel, voire objectif
ou absolu (si l’univers pensait, il serait Dieu ; si Dieu existait, il serait esprit).
Nul, par exemple, ne peut ressentir à ma place, ni tout à fait comme moi. Mon
âme est unique, tout autant que mon corps. Alors qu’une idée vraie, en tant
qu’elle est vraie, est la même en moi et en tout autre (en moi et en Dieu, disait
Spinoza). Cet accès — pour nous toujours relatif — à l’universel ou à l'absolu,
c’est ce que j'appelle l'esprit : c’est notre façon d’habiter le vrai en nous libérant
de nous-mêmes. L'âme serait plutôt notre façon, toujours singulière, toujours
déterminée, d’habiter le monde : c’est notre corps en acte, dit à peu près Aristote,
en tant qu’il a la vie (la motricité, la sensibilité, l’affectivité) en puissance.

Ainsi c’est l'esprit qui est libre, non l’âme, ou plutôt c’est l'esprit qui libère,
et c’est, pour l’âme, le seul salut, toujours inachevé.

\section{Ami}
%AMI
Celui que vous aimez et qui vous aime, indépendamment de tout attachement
familial, passionnel ou érotique. Non qu'on ne puisse être
l'ami d’un parent ou d’un amant, mais en ceci qu’on n'est son ami que si
l'amour partagé ne s'explique suffisamment ni par les liens du sang ni par ceux
du désir ou de la passion. On choisit ses amis. On ne choisit pas ses parents, ni

% 36
d’être amoureux ou troublé. C’est ce qui rend l'amitié plus légère et plus libre.
Cela me fait penser à cette leçon de morale, qu’on apprenait autrefois dans les
écoles : « Un frère est un ami donné par la nature. » La formule m’éclaire plus
sur l'amitié que sur la famille. Un ami est un frère, qu’on se donne, et qui se
donne, par liberté.

« Ce n’est pas un ami, disait Aristote, celui qui est l’ami de tous. » C’est ce
qui distingue l’amitié de la charité. Celle-ci, dans son principe, est universelle.
Toute amitié est particulière. On n’en conclura pas que charité et amitié
seraient incompatibles (lami est aussi un prochain, rien n’empêche que le prochain
devienne un ami), mais que l’une ne saurait tenir lieu de l’autre. Voyez
Jésus et Jean. Aimer son prochain, pour autant qu’on le puisse, n'empêche pas
de préférer ses amis, ni n’en dispense.

\section{Amitié}
%AMITIÉ
La joie d’aimer, ou l’amour comme joie, qu’on ne saurait réduire ni
au manque ni à la passion. Non qu’il les exclue : on peut manquer
de ses amis comme de tout, et il n’est pas rare qu’on les aime passionnément.
Mais ce n’est pas toujours le cas — il y a des amitiés pleines et douces, d’autant
plus fortes qu’elles sont plus sereines —, ce qui interdit de confondre l’amitié
{\it (philia)} avec le manque ou la passion {\it (éros)}. Il n’y a pas d’amour heureux, tant
qu’il est manque, ni serein, tant qu’il est passion. Cela dit, par différence, ce
qu'est l’amitié : ce qui, dans l’amour, nous réjouit, nous comble ou nous
apaise. C’est aimer ce qui ne manque pas, ou plus : l'amitié est un amour heureux,
ou le devenir heureux de l'amour.

Une autre différence est la réciprocité nécessaire. Nous pouvons aimer qui
ne nous aime pas, c’est ce qu’on appelle un chagrin d’amour, mais point tout à
fait être l’ami de celui que nous ne croirions pas être le nôtre. En ce sens, il n’y
a pas d’amitié malheureuse (le malheur ne vient à l’amitié que de l'extérieur, ou
de sa fin), ni de bonheur, sans doute, sans amitié. « Aimer, c’est se réjouir »,
écrit merveilleusement Aristote. Cela, qui n’est pas toujours vrai de la passion,
tant s’en faut, dit l'essentiel de l'amitié : qu’elle est une joie réciproque, que
chacun tire de l’existence et de l’amour de l’autre. C’est pourquoi les amis ont
plaisir à se connaître, à se fréquenter, à se parler, à s’entraider... Sans ce plaisir-là,
qui trouverait la vie agréable ? « L'amitié est ce qu’il y a de plus nécessaire
pour vivre, écrit Aristote, car sans amis personne ne choisirait de vivre. » Je ne
sais si cela est tout à fait vrai. Mais rien ne me rend Aristote et l'amitié plus
aimables.

Il n’y a pas à choisir entre l'amitié et la passion, puisque celle-ci tend ordinairement
vers celle-là. Ainsi dans le couple ou la famille (« la famille est une
amitié », écrit Aristote), quand ils sont heureux.

% 37
Ni entre le désir et l'amitié, puisque rien n’interdit de désirer ses ami(e)s, et
puisque l'amitié, comme dit encore le Philosophe, « est désirable par elle-même ».

Mais ni la passion ni le désir n’y sont nécessaires, pas plus qu’ils n’y suffisent.
Rien n’y suffit que l’amour : l’amitié, même réciproque, « consiste plutôt
à aimer qu’à être aimé », souligne l’{\it }Éthique à Nicomaque, et c'est en quoi
«aimer est la vertu des amis ». L'amitié est à la fois un besoin et une grâce, un
plaisir et un acte, une vertu et un bonheur. Qui dit mieux ? Ce n'est pas
lamour qui prend {\it (éros)}, ni seulement l'amour qui donne {\it (agapè)}. C'est
l'amour qui se réjouit et partage.

\section{Amoral}
%AMORAL
Le {\it a}, ici, est purement privatif : être amoral, c’est être sans morale,
ou ne relever d’aucune. Ainsi la nature est amorale, ce qui signifie
qu’elle ne fait aucune différence entre le bien et le mal : voyez la pluie, le soleil
ou la foudre.

C’est ce qui distingue l’{\it amoralité} de l’{\it immoralité}. Être immoral, c’est aller
contre la morale, ce qui suppose qu’on en a une ou, à tout le moins, qu'on
pourrait ou devrait en avoir une. Voyez le viol, la torture, le racisme. Il se peut
que l’immoralité, en ce sens, soit le propre de l’homme. Et que l’amoralité soit
le propre du vrai.

\section{Amour}
%AMOUR
« Aimer, écrit Aristote, c’est se réjouir » ({\it Éthique à Eudème}, VI, 2).
Quelle différence alors entre la joie et l’amour ? Celle-ci, qu'énonce
Spinoza : « L'amour est une joie qu’accompagne l’idée d’une cause extérieure »
({\it Éthique}, III, déf. 6 des affects) ou, ajouterai-je, intérieure. Aimer, c’est se
réjouir {\it de}. Ou plus exactement (puisqu'on peut aussi aimer un mets ou un
vin) : {\it jouir ou se réjouir de}. Tout amour est joie ou jouissance. Toute joie, toute
jouissance — dès lors qu’on les rapporte à leur cause — est amour. Aimer Mozart,
c’est jouir de sa musique ou se réjouir à l’idée qu’elle existe. Aimer un paysage,
c’est jouir ou se réjouir de sa vue ou de son existence. S’aimer soi, c'est être
pour soi-même cause de joie. Aimer ses amis, c’est se réjouir de ce qu'ils sont.
Si l’on ajoute que tout en nous a une cause et que plaisir sans joie n'est pas tout
à fait amour (la chair est triste quand le plaisir du corps ne réjouit pas aussi
l’âme : quand on fait l'amour, par exemple, sans aimer au moins le faire), on
rejoint les deux définitions d’Aristote et de Spinoza, en ce point lumineux où
elles se rejoignent : il n’est joie que d’aimer ; il n’est amour que de joie.
L'accord de ces deux génies me réjouit : ce m'est une occasion supplémentaire
de les aimer.

% 38
Mais que prouve une joie ? Et que vaut cette définition joyeuse ou aimable
contre tant d’amours tristes, angoissées, malheureuses — tant d’amours sans
plaisir ou sans joie — qu’atteste la littérature et que confirme, hélas, notre
expérience ? Que pèse Aristote contre un chagrin d'amour ? Spinoza, contre un
deuil ou une scène de ménage ? Le réel a toujours raison, puisque c’est lui qu’il
s’agit de penser. Mais {\it quid}, alors, de notre définition ?

Une autre se propose, qui vient de Platon. L'amour est désir, explique-t-il
dans {\it Le Banquet}, et le désir est manque : « Ce qu’on n’a pas, ce qu’on n’est pas,
ce dont on manque, voilà les objets du désir et de l’amour. » Le malheur, avec
une telle définition, ne s'explique que trop bien. Comment serait-on heureux
en amour, puisqu'on n’aime que ce qui manque, que ce qu’on n’a pas, puisque
amour n'existe que par ce vide qui l’habite ou le constitue ? Il n’y a pas
d'amour heureux, et c’est l'amour même, qui manque toujours, par définition,
de ce qui {\it ferait} son bonheur.

Parce que aucun manque n’est jamais satisfait ? Non pas. La vie n’est pas
difficile à ce point. Mais parce que la satisfaction du manque l’abolit comme
manque et donc (puisque l'amour en est un) comme amour. Cela ne laisse
guère le choix qu'entre deux situations : tantôt nous aimons ce que nous
n'avons pas, et nous souffrons de ce manque ; tantôt nous avons ce qui dès lors
ne nous manque plus, et que nous devenons pour cela (puisque l’amour est
manque) incapables d’aimer... L'amour s’exalte dans la frustration, s’endort ou
s'éteint dans la satisfaction. Cela vaut spécialement pour notre vie amoureuse.
Le manque dévorant de l’autre (la passion) semble n’avoir d’avenir heureux que
dans la possession de son objet. Cette possession fait-elle défaut ? C’est le malheur
assuré, au moins un certain temps. Advient-elle ? Dure-t-elle ? Le bonheur
vient s’user, en même temps que le manque, dans la présence de celui ou celle
qui devait l’assurer. Qui peut manquer de ce qu’il a, de celui ou celle qui partage
sa vie, qui est là tous les soirs, tous les matins, si présent, si familier, si
quotidien ? Comment la passion survivrait-elle au bonheur ? Comment le bonheur,
à la passion ? « Imaginez Madame Tristan », disait Denis de Rougemont.
Ce ne serait plus Iseut, ou elle ne serait plus amoureuse. Comment aimer passionnément
l'ordinaire ? Quel philtre contre l’habitude, l'ennui, la satiété ?

Être heureux, explique Platon avant Kant, c’est avoir ce qu’on désire. C’est
ce qui rend le bonheur impossible : comment aurait-on ce qu’on désire, si on
ne peut désirer que ce qu’on n’a pas ? Schopenhauer, en génial disciple de
Platon, tirera la conclusion qui s'impose : « Ainsi toute notre vie oscille,
comme un pendule, de droite à gauche, de la souffrance à l'ennui. » Souffrance
parce que nous désirons ce que nous n’avons pas, et que nous souffrons de ce
manque ; ennui parce que nous avons ce qui dès lors ne nous manque plus et
que nous nous découvrons pour cela incapables d'aimer... C’est ce que Proust
% 39
appellera les intermittences du cœur, ou du moins les deux pôles entre lesquels
elles se jouent. Albertine présente, Albertine disparue... Quand elle n’est pas là,
il souffre atrocement : il est prêt à tout pour qu’elle revienne. Quand elle est là,
il s'ennuie ou rêve à d’autres : il est prêt à tout pour qu’elle s’en aille... Qui n'a
vécu ces oscillations ? Qui n’y reconnaît quelque chose de sa vie, de son malheur,
de son inconstance ? Il faut aimer celui ou celle que nous n'avons pas,
c’est ce qu’on appelle un chagrin d’amour, ou bien avoir celui ou celle qui ne
nous manque plus, qu’on aime pour cela de moins en moins, et c’est ce qu'on
appelle un couple.

Cela rejoint une chanson fameuse de Nougaro : {\it « Quand le vilain mari tue
le prince charmant... »} C’est le même individu pourtant, mais dans deux situations
opposées : le prince charmant, c’est le mari qui manque ; le vilain mari, le
prince charmant qui a cessé de manquer.

Ces deux définitions de l’amour présentent des avantages et des inconvénients
symétriques. Celle d’Aristote ou de Spinoza vient buter contre l'échec de
l'amour, contre son malheur, sa tristesse, ses angoisses. Celle de Platon échoue
plutôt devant ses réussites : elle explique fort bien nos souffrances et nos déceptions
amoureuses, mais point l'existence, parfois, de couples heureux, où
chacun se réjouit non du manque de l’autre — comment serait-ce possible ? —
mais de son existence, mais de sa présence, mais de cet amour même qui les
unit et qu’ils partagent. Tout couple heureux est une réfutation du platonisme.
Ce m'est une raison supplémentaire d’aimer les couples et le bonheur, et de
n’être pas platonicien. Mais comment, si l'amour échoue, rester spinoziste ?

Commençons par le plus facile. Que l'amour puisse être obscurci d’angoisse
ou de souffrance, rien là de mystérieux. Si l'existence de mes enfants me
réjouit, comment ne serais-je pas triste, atrocement triste, s'ils viennent à
mourir ? Comment ne serais-je pas angoissé, atrocement angoissé, à l’idée —
hélas toujours plausible — qu’ils peuvent souffrir ou mourir ? Si leur existence
me réjouit, l'imagination de leur inexistence, ou de l’amoindrissement de leur
existence (leur maladie, leur souffrance, leur malheur), ne peut que m’angoisser
ou m’attrister. C’est ce que Spinoza explique suffisamment ({\it Éthique} III, propositions
19 et 21, avec leurs démonstrations), sur quoi il est inutile de s’attarder.
Aimer, c’est trembler — non parce que l’amour est crainte, mais parce que la vie
est fragile. Ce n’est pas une raison pour renoncer à aimer, ni à vivre.

Le couple est plus difficile à penser. Qu'il commence ordinairement dans le
manque, c’est une donnée qui est moins physiologique (la frustration n’a
jamais suffi à rendre quiconque amoureux) que psychologique, mais qui n’en
est pas moins avérée. {\it I need you}, chantaient les Beatles : je t’aime, je te veux, tu
me manques, j'ai besoin de toi... L'amour, en ses commencements, donne
raison à Platon, presque toujours. C’est ce que les Grecs appelaient {\it éros} :
% 40
l'amour qui manque de son objet, l'amour qui prend ou qui veut prendre,
l’amour qui veut posséder et garder, l’amour passionnel et possessif.. C’est
n’aimer que soi (l'amant aime l’aimé, écrivait Platon dans le {\it Phèdre}, comme le
loup aime l’agneau), ou l’autre seulement en tant qu’il nous manque, en tant
qu'il nous est nécessaire ou qu’on l’imagine être tel, et c’est pourquoi c’est si
fort, si facile, si violent... Amour de concupiscence, disaient les scolastiques :
aimer l’autre pour son bien à soi. Voyez l'enfant qui prend le sein. Voyez
l’amant avide ou brutal. Voyez l’amoureux exalté. Manquer est à la portée de
n'importe qui. Rêver est à la portée de n’importe qui. Mais quand le manque
disparaît ? Quand les rêves viennent se briser contre la présence continuée de
l’autre ? Quand le mystère se fait transparence ou opacité ? Certains ne pardonneront
jamais à l’autre de n'être que ce qu'il est, et point le miracle qu'ils
avaient d’abord imaginé. C’est ce qu’on appelle le désamour, qui a le goût
amer, presque toujours, de la vérité. « On aime quelqu'un pour ce qu’il n’est
pas, disait Gainsbourg, on le quitte pour ce qu’il est. » Mais tous les couples ne
se séparent pas, ni ne vivent tous dans l’ennui ou le mensonge. C’est que certains
ont su apprendre à aimer l’autre tel qu’il est, disons tel qu’il se donne à
connaître, à côtoyer, à expérimenter, jusqu’à se réjouir de sa présence, de son
existence, de son amour, et d’autant plus qu’il ne manque pas mais qu’il est là,
mais qu'il se donne, ou qu’il ne manque, dans la joyeuse répétitivité du désir,
que pour mieux manifester sa présence, sa disponibilité, sa puissance, sa douceur,
sa sensualité, sa tendresse, son habileté, son amour... Cet amour qui ne
manque de rien, c’est ce que les Grecs appelaient {\it philia}, qu’on peut traduire
par « amitié », si l’on veut, mais à condition d’y inclure la famille et le couple,
comme faisait Aristote, et spécialement ce que Montaigne appelait « l'amitié
maritale » : c’est l’amour de celui ou celle qui ne manque pas mais qui réjouit,
mais qui comble, mais qui conforte et réconforte. Que l'érotisme puisse aussi y
trouver son compte, c'est ce que les couples savent bien, et qui leur donne
raison. Comme la vérité des corps et des âmes est plus excitante, pour deux
amants, que le rêve ! Comme la présence de l’autre — son corps, son désir, son
regard — est plus troublante que son absence ! Comme le plaisir est plus plaisant
que le manque ! Mieux vaut faire l'amour que le rêver. Mieux vaut jouir et se
réjouir de ce qui est qu’en manquer ou que souffrir de ce qui n’est pas.

Entre {\it éros} et {\it philia}, entre le manque et la joie, entre la passion et l'amitié,
on évitera pourtant de choisir. Ce ne sont pas deux mondes, qui s’excluraient,
ni deux essences séparées. Plutôt deux pôles, mais dans un même champ. Deux
moments, mais dans un même processus. Voyez l'enfant qui prend le sein,
disais-je. C’est {\it éros}, l'amour qui prend, et tout amour commence là. Et puis
voyez la mère, qui le donne. C’est {\it philia}, l'amour qui donne, l'amour qui protège,
l'amour qui se réjouit et partage. Chacun comprend que la mère a été
% 41
enfant d’abord : elle a commencé par prendre ; et que l’enfant devra apprendre
à donner. Ainsi {\it éros} est premier, toujours, et le demeure. Mais {\it philia} en émerge
peu à peu, qui le prolonge. Que tout amour soit sexuel, comme le veut Freud,
cela ne veut pas dire que la sexualité soit le tout de l’amour. Qu’on commence
par s'aimer soi, comme l'avaient vu les scolastiques, n’empêche pas — mais
permet au contraire — qu’on aime aussi, parfois, quelqu'un d’autre. D'abord le
manque, puis la joie. D’abord l'amour de concupiscence (aimer l’autre pour
son bien à soi), puis l’amour de bienveillance (aimer l’autre pour son bien à
lui). D'abord l'amour qui prend, puis l’amour qui donne. Que le second
n’efface pas le premier, c’est ce que chacun peut expérimenter. Le chemin n’en
est pas moins clair, qui mène de l’un à l’autre, et c’est un chemin d’amour, ou
l'amour comme chemin.

Jusqu'où va-t-il ? Aimer ce qui me réjouit, ce qui me fait du bien, ce qui me
comble ou m’apaise, c’est encore m’aimer moi. Par quoi la bienveillance
n'échappe pas à la concupiscence, ni {\it philia} à {\it éros}, ni l'amour à l’égoïsme ou à
la pulsion de vie. Peut-on aller plus loin ? C'est ce que requièrent les Évangiles.
Aimer son prochain, c’est aimer n’importe qui : non celui qui me plaît, mais
celui qui est là. Non celui qui me fait du bien, mais jusqu’à ceux qui me font
du mal. Aimer ses ennemis, c’est par définition sortir de l’amitié, au moins
dans sa définition égologique ou montanienne (« parce que c'était lui, parce
que c'était moi»), peut-être aussi de la logique (les Grecs n’y auraient vu
qu’une contradiction ou une folie : comment être l’ami de ses ennemis ?). Les
premiers chrétiens, pour désigner en grec un tel amour, ne pouvaient utiliser ni
{\it éros} ni {\it philia} : ils forgèrent le néologisme {\it agapè} (du verbe {\it agapan}, aimer,
chérir), que les Latins traduisirent par {\it caritas} et qui donnera notre {\it charité}. Ce
serait bienveillance sans concupiscence, joie sans égoïsme (comme une amitié
libérée de lego), et pour cela sans rivage : l'amour désintéressé, le pur amour,
comme disait Fénelon, l'amour sans possession ni manque, l’amour sans
convoitise, comme dit Simone Weil, celui qui n’espère rien en retour, celui qui
n’a pas besoin d’être réciproque, celui qui n’est pas proportionné à la valeur de
son objet, celui qui donne et s’abandonne. Ce serait l'amour que Dieu a pour
nous, que Dieu est pour nous ({\it « o Théos agapè estin »}, dit l'Évangile de Jean), et
cela dit assez sa valeur, au moins imaginaire, et combien il nous dépasse. En
sommes-nous capables ? J'en doute fort. Mais cela n’interdit pas d’y tendre, d'y
travailler, de s’en approcher peut-être. Plus on s’éloigne de l’égoïsme — plus on
s'éloigne de soi -, plus on s’approche de Dieu. Cela dit peut-être, sur l’amour
de charité, l'essentiel : ce serait une joie, comme aurait pu dire Althusser, {\it sans
sujet ni fin}.

Ainsi tout commence par le manque, et tend vers la joie — vers une joie de
plus en plus vaste et libre. C’est pourquoi le trait commun entre ces trois
% 42
amours — qui serait donc l’amour même, ou son genre prochain — est la joie. Il
faut se réjouir, fantasmatiquement, à l’idée qu’on pourrait posséder ce qui nous
manque {\it (éros)}, ou bien se réjouir de ce qui ne nous manque pas et qui nous fait
du bien {\it (philia)}, ou bien encore se réjouir, purement et simplement, de ce qui
est {\it (agapè)}.

On peut aussi n’aimer rien (c’est ce que Freud appelle la mélancolie : « la
perte de la capacité d’aimer »), et constater que la vie dès lors n’a plus ni saveur
ni sens. Plusieurs en sont morts ou en mourront : on ne se suicide que lorsque
l'amour échoue, ou lorsqu'on échoue à aimer. Tout suicide, même légitime, est
un échec, comme l’a vu Spinoza, ou la marque d’un échec, ce qui devrait dissuader
de le condamner — nul n’est tenu de réussir toujours — comme d’en faire
l'apologie. Un échec n’est ni une faute ni une victoire.

La vie vaut-elle la peine d’être vécue ? Il n’y a pas de réponse absolue. Rien
ne vaut en soi, ni par soi : rien ne vaut que par la joie qu’on y trouve ou qu’on
y met. La vie ne vaut que pour qui l’aime. L'amour ne vaut que pour qui
l'aime. Ces deux amours vont ensemble. Non seulement parce qu’il faut être
vivant pour aimer, mais aussi parce qu’il faut aimer pour prendre goût à la vie,
et même — puisque le courage ne peut suffire — pour continuer à vivre.

C’est l'amour qui fait vivre, puisque c’est lui qui rend la vie aimable. C’est
l'amour qui sauve, et c’est donc lui qu’il s’agit de sauver.

\section{Amour-propre}
%AMOUR-PROPRE
C'est l’amour de soi sous le regard de l’autre : le désir d’en
être aimé, approuvé, admiré, l’horreur d’en être détesté
ou méprisé. La Rochefoucauld y voyait la principale de nos passions, et le ressort
de toutes. Rousseau, plus généreusement, plus justement, le distinguait de
l'amour de soi : « L'amour de soi-même est un sentiment naturel qui porte tout
animal à veiller à sa propre conservation, et qui, dirigé dans l’homme par la
raison et modifié par la pitié, produit l'humanité et la vertu. L’amour-propre
n’est qu’un sentiment relatif, factice, et né dans la société, qui porte chaque
individu à faire plus de cas de soi que de tout autre, qui inspire aux hommes
tous les maux qu’ils se font mutuellement, et qui est la véritable source de
l’honneur » ({\it Discours sur l'origine de l'inégalité}, note XV). Entre les deux, la
transition ne s’explique que trop bien. Nous ne vivons d’abord que pour nous,
mais qu'avec et par les autres. Comment n’aimerions-nous pas en être aimé ?
L’amour-propre est cet amour de l’amour, centré sur soi, médiatisé par autrui.
C’est n’aimer l’autre que pour soi, et soi que par l’autre. Double erreur, ou
double piège, qui explique que l’amour-propre, comme le disait Alain, soit un
amour malheureux. Ce ne sont pourtant que petites blessures, à côté des grands
% 43
drames de l'existence. Le vrai malheur, parfois, en guérit. Le vrai bonheur aussi
peut-être.

\section{Amour nommé socratique}
%AMOUR NOMMÉ SOCRATIQUE
C’est le nom que Voltaire, dans son {\it Dictionnaire},
donne à l'homosexualité masculine —
qui ne saurait être socratique, selon lui, que par abus de langage. Il est
contre : il y voit « un vice destructeur du genre humain, s’il était général, et un
attentat infâme contre la nature ». Cela fait deux reproches différents.

Le premier pourrait prendre une forme quasi kantienne : l'homosexualité
n’est pas universalisable, puisqu'elle aboutirait, si elle était exclusive, à la disparition
de l'espèce, donc de l'homosexualité. Mais le mensonge, le suicide et la
chasteté ne le sont pas davantage ; cela ne prouve pas qu’ils soient toujours
immoraux. Kant, qui n’eut pas d’enfants, qui mourut peut-être puceau, était
bien placé pour le savoir. Contradiction ? Pas forcément : c’est la {\it maxime} d’une
action qui doit pouvoir, selon lui, être universalisée sans contradiction, non
l’action elle-même. Pourquoi ne serait-ce pas le cas de la maxime « J'ai le droit
de faire l'amour avec tout partenaire adulte consentant, quel que soit son
sexe » ? Kant n’en condamne pas moins l'homosexualité, comme il condamne
la masturbation et la liberté sexuelle ({\it Doctrine du droit}, \S 24, {\it Doctrine de la
vertu}, \S  5 à 7). Je crains que l’universel n’y soit pas pour grand-chose — que
cette condamnation, chez Kant comme chez Voltaire, relève plus de l’état des
mœurs et de la société (donc du particulier) que de la raison. Cela vaut aussi
pour nous ? Sans doute : que nous soyons plus tolérants ou plus ouverts qu’il y
a deux siècles, cela ne prouve pas que nous soyons plus intelligents que nos
deux auteurs. On ne m’ôtera pourtant pas de l’idée qu’il s’agit là d’un progrès :
c’est un peu de haine, de mépris et de pudibonderie en moins.

Le deuxième argument est encore plus faible. Un attentat contre la nature ?
Je ne suis pas sûr que la notion ait un sens. Comment ce qui existe {\it dans la
nature} pourrait-il être {\it contre nature} ? Mais quand bien même cela serait, la
question de la moralité ou de l’immoralité de l’homosexualité ne s’en poserait
pas moins. Contre nature, la chasteté l’est sans doute davantage. Est-elle pour
autant immorale ? Rien de plus naturel, à l'inverse, que l’égoïsme. Faut-il en
faire pour cela une vertu ?

Le droit à la différence, comme nous disons aujourd’hui, fait partie des progrès
importants de ces dernières décennies. Il est clair qu’il n’est pas sans limite
(la différence du pédophile, du violeur ou de l’assassin ne leur donne aucun
droit). Il est clair qu’il n’est pas suffisant. Mais aucun droit ne l’est.

« Justement, me dit une amie, c’est ce qui me gêne, chez les homosexuels :
ils se réclament très fort du droit à la différence, mais ils fuient la différence
% 44
principale, qui est celle des sexes. Moralement, je ne leur reproche rien ; mais
enfin ils vont vers le plus facile. » C’est trop dire sans doute, tant l’homosexualité,
même aujourd’hui, reste un choix socialement inconfortable. Mais
cela nous rappelle que l’hétérosexualité n’est pas non plus, c’est le moins que
l’on puisse dire, de tout repos.

Les gens de ma génération ont beaucoup de chance : peu d’époques auront
été, sexuellement, aussi tolérantes et libérées que la nôtre. C’est évidemment
tant mieux ; mais cela nous donne aussi de nouvelles responsabilités. L’homosexualité
n’est pas une faute ? C’est devenu une évidence, pour la quasi-totalité
de nos contemporains. C’est qu’elle ne fait de tort — entre partenaires adultes et
consentants — à personne. On ne peut en dire autant de l'oppression des
femmes et des enfants, du viol, du proxénétisme, de l’égoisme, du mépris, de
l’irresponsabilité, de l’asservissement... L'orientation sexuelle ne relève pas de
la morale, mais n’en dispense pas non plus.

\section{Amphibologie}
%AMPHIBOLOGIE
Ambiguïté fautive (parce qu’on aurait pu et dû l’éviter) ou
drôle (si elle est délibérée et piquante) dans le discours.
Par exemple dans ce dialogue imaginaire, qui nous amusait enfants :

« — Papa, j'aime pas grand-mère !

« — Tais-toi, et mange ce qu’on te donne!»

Kant appelle « amphibologie transcendantale » l'erreur de raisonnement
qui consiste à confondre l’objet pur de l’entendement (le noumène) avec celui
de la sensibilité (le phénomène). C’est l’erreur commune et symétrique de Leibniz,
qui « intellectualisait les phénomènes », et de Locke, qui « avait sensualisé
tous les concepts de l’entendement ». C’était confondre la sensibilité et l’entendement,
au lieu de les utiliser ensemble — ce qui suppose qu’on les distingue —
dans la connaissance.

L’amphibologie est donc une ambiguïté savante, ou le nom savant d’une
ambiguïté. Deux raisons pour éviter, sauf pour rire, et le mot et la chose.

\section{Analogie}
%ANALOGIE
C’est une identité de rapports (par exemple en mathématiques :
a/b = c/d), ou une équivalence fonctionnelle ou positionnelle
(qui s'explique moins par chacun des termes que par leur place ou leur fonction
dans un ensemble). Par exemple quand Platon écrit que l’être est au devenir ce
que l'intelligence est à l'opinion, quand Épicure compare les atomes aux lettres
de l’alphabet, ou quand Maine de Biran écrit que « Dieu est à l’âme humaine
ce que l’âme est au corps », ils font des analogies. En philosophie, c’est souvent
% 45
une façon de penser l’impensable, ou de faire semblant. Difficile de s’en passer,
et de s’en contenter.

Kant en donne une belle définition : « Ce mot ne signifie pas, comme on
l'entend ordinairement, une ressemblance imparfaite entre deux choses, mais
une ressemblance complète de deux rapports entre des choses tout à fait
dissemblables » ({\it Prolégomènes...}, \S 58). Surtout, il distingue l’{\it analogie mathématique}
de l’{\it analogie philosophique}. La première exprime « l'égalité de deux rapports
de grandeur », de telle sorte que, lorsque trois membres en sont donnés
(12/3 = 8/x), le quatrième l’est par là même (l’analogie est donc constitutive).
Dans la philosophie, au contraire, mais aussi en physique, « l’analogie n’est pas
l'égalité de deux rapports {\it quantitatifs}, mais bien de deux rapports {\it qualitatifs},
dans lesquels, trois membres étant donnés, je ne puis connaître et donner {\it a
priori} que le {\it rapport} à un quatrième, mais non ce quatrième membre lui-même »
({\it C. R. Pure}, Analytique des principes, II, 3). Les analogies de l’expérience,
qui sont des principes {\it a priori} de l’entendement correspondant aux catégories
de la relation, ne valent que de façon régulatrice. Elles ne disent pas ce
qu’est ce quatrième membre (c’est pourquoi on ne saurait faire une physique {\it a
priori}), mais « fournissent une règle pour le chercher dans l'expérience » (par
quoi il y a de l’{\it a priori} dans toute physique scientifique). Elles sont au nombre
de trois, qui correspondent au trois modes du temps que sont la permanence,
la succession et la simultanéité, mais aussi aux trois catégories de la relation : le
principe de la permanence de la substance, de la succession dans le temps suivant
la loi de causalité, enfin de l’action réciproque. Ils ne valent tous les trois
que pour l'expérience, qu’ils rendent possible « par la représentation d’une
liaison nécessaire des perceptions » ; ils ne sauraient en tenir lieu.

En métaphysique, {\it a fortiori}, l'analogie ne saurait valoir comme preuve. Je
peux bien me représenter l’univers comme une horloge dont Dieu serait
l’horloger ; cela ne prouve pas que Dieu existe, ni ne me dit ce qu’il est ({\it Religion...},
II, 1, en note). On ne peut penser Dieu que par analogie (le Dieu
artisan, le Dieu souverain, le Dieu Père...). C’est ce qui nous voue tous à
l’anthropomorphisme ; l’athée même n’y échappe pas (pour ne pas croire en
Dieu, il faut s’en faire une idée). Mais cet anthropomorphisme, explique Kant,
doit rester {\it symbolique}, non {\it dogmatique}. Il « ne concerne que le langage, et non
Pobjet lui-même » : il dit ce qu’est Dieu pour nous, ou ce que nous entendons
par ce mot ; il ne dit rien sur ce que Dieu est en soi, ni s’il est ({\it Prolégomènes}...,
\S 57).

\section{Analyse}
%ANALYSE
Analyser, c’est décomposer un tout en ses éléments ou parties
constitutives, ce qui suppose ordinairement qu’on le divise et
% 46
qu'on les sépare, au moins provisoirement ou intellectuellement. Le contraire
donc (mais souvent aussi la condition) de la synthèse, qui rassemble, compose
ou recompose. On peut ainsi analyser un corps quelconque (faire ressortir les
éléments physiques ou chimiques qui le constituent), une idée complexe
(qu'on ramènera à une somme d’idées simples), une société (analyse sociologique,
qui distinguera par exemple plusieurs classes ou courants), un individu
(analyse psychologique, ou psychanalytique), un problème, une œuvre d’art,
un rêve, bref n'importe quoi — sauf l’absolument simple, s’il existe. Descartes
en a fait une règle de sa méthode : « Diviser chacune des difficultés que j’examinerai,
en autant de parcelles qu’il se pourrait et qu’il serait requis pour les
mieux résoudre » ({\it Discours de la méthode}, II). C’est vouloir ramener le complexe
au simple, pour le comprendre. Démarche légitime et nécessaire, tant
qu'elle ne fait pas oublier la complexité de l’ensemble. Pascal, en une formule
qu'Edgar Morin aime à citer, nous le rappelle : « Toutes choses étant causées et
causantes, aidées et aidantes, médiates et immédiates, et toutes s’entretenant
par un lien naturel et insensible qui lie les plus éloignées et les plus différentes,
je tiens pour impossible de connaître les parties sans connaître le tout, non plus
que de connaître le tout sans connaître particulièrement les parties. » On ne se
dépêchera pas trop d’opposer pour cela Pascal à Descartes. Que tout soit dans
tout et réciproquement, comme dit encore Edgar Morin, cela n’interdit pas
l'analyse ; c’est au contraire ce qui la rend nécessaire et interminable.

\section{Analytiques (jugements —)}
%ANALYTIQUES (JUGEMENTS -)
Un jugement est analytique, explique Kant,
quand le prédicat est contenu, même de
manière cachée ou implicite, dans le sujet, et peut donc en être tiré par analyse.
Par exemple : {\it « Tous les corps sont étendus »} (la notion d’étendue est incluse dans
celle de corps : un corps sans étendue serait contradictoire). Les jugements analytiques,
qui sont fondés sur l'identité, sont seulement explicatifs : « ils n’étendent
pas du tout nos connaissances », souligne Kant, mais ne font que développer
ou expliciter nos concepts. Si nos connaissances s'étendent, comme
nous constations qu’elles le font, c’est donc qu’il y a d’autres jugements, que
Kant appelle les {\it jugements synthétiques} (voir ce mot).

\section{Anamnèse}
%ANAMNÈSE
C’est comme une réminiscence, mais volontaire, mais laborieuse :
le travail sur soi de la mémoire, la quête de ce qui fut,
de ce qui est encore, mais en nous, comme une Atlantide intérieure. L’oubli,
parfois, vaudrait mieux.

% 47
\section{Anarchie}
%ANARCHIE
L'absence de pouvoir ou le désordre. Cette ambiguïté en dit
long sur l’ordre (qu’il ne va pas sans obéissance) et sur la liberté
(qu’elle ne va pas sans contraintes). « Tout pouvoir est militaire », disait Alain.
C’est pourquoi les anarchistes ont horreur de l’armée; et les militaires, de
l'anarchie. Les démocrates se méfient et de l’une et de l’autre : ils savent bien
que le désordre, presque toujours, fait le jeu de la force ; et qu'aucune force ne
vaut, qu’au service de la justice ou de la liberté.

Le mot est pris le plus souvent en mauvaise part. C’est donner raison à
Goethe, qui préférait l'injustice au désordre. Seuls les anarchistes y voient un
idéal, qu’ils croient accessible. C’est se tromper sur l’homme ou vouloir le
transformer. Une erreur, donc, ou une utopie.

La justice sans la force n’est qu’un rêve. Ce rêve est l’anarchie. La force sans
la justice est une réalité : c’est la guerre, c’est le marché, c’est la tyrannie des
plus puissants ou des plus riches. Les deux modèles peuvent pourtant se nourrir
d’un même rejet de l’État, du droit, de la République (de l’ordre démocratiquement
imposé). Cela explique que les jeunes anarchistes, souvent, fassent de
vieux libéraux.

\section{Anarchisme}
%ANARCHISME
C'est la doctrine des anarchistes, quand ils en ont une : ainsi
chez Proudhon, Bakounine ou Kropotkine. L’anarchisme
prône la suppression de l’État, toujours ; de la religion (« Ni Dieu ni maître »),
presque toujours ; enfin de la propriété privée, le plus souvent. C’est ce qui le
classe à gauche. Mais on trouve aussi des anarchistes de droite (ils se réclament
parfois de l’individualisme de Stirner), voire des anarcho-capitalistes : ainsi, aux
États-Unis, le mouvement libertarien, qui est comme un libéralisme extrémiste.
C’est mettre la liberté plus haut que tout, et trop haut. Comment pourrait-elle
se passer de la force, de la contrainte, de l’ordre imposé et contrôlé ? Comment
tiendrait-elle lieu de droit, d'égalité, de justice ? L’anarchie ferait un parfait
régime pour des anges ; c’est ce qui la rend suspecte de bêtise (Pascal : « Qui
veut faire l’ange fait la bête ») ou d’angélisme.

\section{Âne de Buridan}
%ÂNE DE BURIDAN
Jean Buridan, philosophe français du {\footnotesize XIV$^\text{e}$} siècle, n’est
plus guère connu aujourd’hui que par cet âne dont on
lui prête l'invention, quand bien même il n’est évoqué dans aucun de ses écrits
conservés. De quoi s’agit-il ? D’une fable, ou d’une expérience de pensée. Imaginons
un âne ayant également faim et soif, et placé à égale distance d’un seau
d’eau et d’une ration d’avoine, qu’il aime également. N'ayant aucune raison
d’aller plutôt d’un côté que de l’autre, il serait incapable de choisir : il mourrait
% 48
de faim et de soif. On évoque parfois cette histoire pour montrer que le libre
arbitre est impossible (chacun est déterminé par le bien qui lui semble le
meilleur, le plus nécessaire ou le plus accessible), parfois pour montrer qu'il
existe (puisque la fable de Buridan, appliquée à l’homme, semble absurde). On
en discute depuis six siècles. L’âne est toujours vivant.

\section{Ange}
%ANGE
« Un être intermédiaire, disait Voltaire, entre la divinité et nous » : ce
serait un messager ({\it angelos} en grec) de Dieu. On s'étonne qu'il en ait
besoin.

\section{Angélisme}
%ANGÉLISME
L'abus des bons sentiments, aux dépens de la lucidité. Plus précisément,
et d’un point de vue topique, j'entends par {\it angélisme}
un ridicule particulier, qui confond les ordres, comme tout ridicule (voir ce
mot), mais au bénéfice d’un ordre supérieur, dans la quadripartition que j'ai
proposée (voir l’article « Distinction des ordres »), et en prétendant annuler par
R les pesanteurs ou les contraintes d’un ou de plusieurs ordres inférieurs.
L’angélisme veut abolir le plus bas au nom du plus haut. Mais il ne peut y parvenir
que par aveuglement ou violence. Voyez la dictature de la vertu, chez
Saint-Just, la Révolution culturelle en Chine ou, aujourd’hui, l’intégrisme islamiste.
L’angélisme peut prendre des formes radicalement différentes, depuis
l'utopie la plus généreuse jusqu’à la terreur la plus sanguinaire. Il passe
d’ailleurs volontiers de l’une à l’autre. Mais toujours au nom d’idéaux, de
valeurs, ou d’un Bien transcendant. C’est une tyrannie, dirait Pascal, des ordres
supérieurs. Par exemple : prétendre annuler la logique et les contraintes de
l’économie au nom de la politique ou du droit (angélisme politique ou
juridique : volontarisme ou juridisme). Ou bien : prétendre annuler la légitimité
et les contraintes de la politique ou du droit au nom de la morale (angélisme
moral : le politiquement correct n’est le plus souvent, en tout cas en
France, qu'un moralement correct). Ou encore: prétendre annuler les
contraintes de la morale, voire des trois ordres inférieurs, au nom de l'amour
(angélisme éthique : idéologie {\it « Peace and love »}). Enfin, pour ceux qui y
croient, prétendre annuler les contraintes ou les exigences de chacun de ces
ordres au nom d’un ordre divin ou surnaturel (angélisme religieux : intégrisme).
Tout cela s'explique, et par les ordres inférieurs plutôt que supérieurs
(voir l’article « Primat/primauté »), mais doit aussi se combattre : qui veut faire
l’ange fait la bête, disait Pascal, et tout ange, ajoutait Rilke, est effrayant. À la
gloire de la laïcité.

% 49
\section{Angoisse}
%ANGOISSE
Peur vague ou indéterminée, sans objet réel ou actuel, et qui n’en
est que plus prégnante : parce qu’elle est aussi — faute d’un
danger effectif à combattre ou à fuir — sans riposte possible. Comment vaincre
le néant ? Comment échapper à ce qui n’est pas ou pas encore ? C’est une peur
intempestive et envahissante, qui nous étouffe ({\it angere}, en latin, signifie serrer,
étrangler) ou nous submerge. Le corps s’affole ; l’âme se noie.

Vous avez peur d’un chien qui est là, qui grogne, qui vous semble menaçant...
C’est moins une angoisse qu’une crainte, qui n’appelle que prudence et
courage. Le chien vous attaque : la crainte redouble, qui justifie la fuite ou le
combat.

Mais si vous avez peur d’être attaqué par un chien quand aucun chien n’est
présent ou ne vous menace, c’est plutôt une angoisse. Vous êtes, contre elle,
davantage démuni. Que peuvent la fuite ou le combat contre l'absence d’un
chien ? contre un danger inexistant ou purement imaginaire ? Ici, point de
riposte efficace (qui agirait sur le danger) ; tout au plus un remède (qui n’agit
que sur votre peur).

Entre la crainte et l'angoisse, la limite est bien sûr incertaine, approximative,
fluctuante. Cette ombre, là-bas, est-ce un chien ou une ombre ? Mais la
limite est floue aussi entre la santé et la maladie, qui n’en sont pas moins deux
états différents.

Psychologiquement, l'angoisse porte le plus souvent sur l'avenir (elle est
«en relation avec l'attente », écrit Freud). C’est ce qui la rend si difficile à
vaincre : comment se prémunir, ici et maintenant, contre ce qui n'est pas
encore, contre ce qui {\it peut} être ? L'avenir est hors d’atteinte ; la sérénité aussi,
tant qu’on vit dans l'attente.

Philosophiquement, l’angoisse est le sentiment du néant : sentiment nécessairement
sans objet (le néant n’est pas), et pour cela sans limites. Sans objet ?
Disons sans objet effectif. « Il n’y a rien contre quoi combattre », comme disait
Kierkegaard, et c’est pourquoi c’est une angoisse, non une crainte. « Qu'est-ce
donc ? Rien. Mais quel effet produit ce rien ? Il engendre l'angoisse » ({\it Le
concept d'angoisse}, I). L’angoissé a peur, exactement, de {\it rien} (ce qui le distingue
de l’anxieux, qui a plutôt peur de tout), et n’en est que plus effrayé. D'où, pour
le corps, cette sensation de vide ou de vague, qui peut aller jusqu’à l’étouffement.
L’angoissé {\it manque d'être} comme on manque d’air. Le néant lui fait peur,
et c’est l’angoisse même : le sentiment effrayé du néant de son objet.

Mais quel néant ? Qu'il ne soit pas, cela fait partie de sa définition. Encore
faut-il — pour qu’il y ait angoisse — que nous en ayons pourtant une certaine
expérience. Mais laquelle ? Ou lesquelles ? Quelle perception du néant ? Quelle
réalité, pour nous, du non-être ? J'en vois au moins quatre : le vide, le possible,
la contingence, la mort.

% 50
L'expérience du vide est vertige : rien ne fait obstacle au rien, et tout le
corps en est malade. C’est comme une angoisse physiologique (de même que
l'angoisse est comme un vertige psychologique ou métaphysique), mais qui a
tôt fait d’étreindre aussi l’âme. Voyez Montaigne et Pascal : « Le plus grand
philosophe du monde, sur une planche plus large qu’il ne faut, s’il y a au-dessous
un précipice. » Le vertige angoisse, et c’est pourquoi l’on a raison de le
craindre. Le vertige, en montagne, est plus dangereux que le vide.

L'expérience du possible est liberté. C’est pourquoi la liberté angoisse :
parce qu’elle a ce pouvoir de faire être ce qui n’est pas et de néantiser, comme
dit Sartre, ce qui est. « L’angoisse, écrivait déjà Kierkegaard, est la réalité de la
liberté comme possibilité offerte à la possibilité » : elle est «le vertige de la
liberté ». Être libre, c’est n’être pas prisonnier du réel, puisque c’est pouvoir le
changer, ni de soi, puisque c’est pouvoir choisir. C’est où la liberté touche au
néant, par l'imaginaire, voire en relève : « La réalité humaine est libre, écrit
Sartre, dans l’exacte mesure où elle a à être son propre néant. » D’où l’angoisse,
pour qui assume («langoisse est la saisie réflexive de la liberté par elle-même »),
ou la mauvaise foi, pour qui le dénie. On n’aurait le choix qu’entre le
néant et le mensonge.

La contingence est comme un possible qui se serait réalisé : est contingent
ce qui est et qui {\it aurait pu} ne pas être. C’est pourquoi tout être est contingent,
et c’est ce que l’angoisse obscurément perçoit ou manifeste, comme l’ombre
portée du néant sur l’évidence soudain fragilisée de l’être. « Dans l’angoisse,
écrit Heidegger, l’étant dans son ensemble devient branlant. » C’est qu’il perd
sa nécessité, sa plénitude, sa justification. Pourquoi y a-t-il quelque chose
plutôt que rien ? Il n’y a pas de réponse : tout être est contingent, tout être est
{\it de trop}, comme dira Sartre, tout être est {\it absurde}, comme dira Camus, qui
n'apparaît qu’en se détachant — mais pourquoi ? mais comment ? — sur le fond
imperceptible du néant. Il n’y aurait pas d’être autrement, ou plutôt nous ne
saurions autrement le penser. « Dans la nuit claire du néant de l’angoisse, écrit
encore Heidegger, se montre enfin la manifestation originelle de l’étant comme
tel : à savoir qu’il y ait de l’étant — et non pas rien. » Mais le rien {\it perce}, comme
disait Valéry, ou du moins c’est ce que l'angoisse, obscurément, semble nous
faire éprouver.

Enfin, la mort. C’est le néant le plus réel peut-être, mais aussi le plus
impossible à expérimenter — puisqu'il n’est d'expérience, par définition, que
pour un vivant. La mort ne serait donc rien ? C’est la position d’Épicure, et la
plus raisonnable que je connaisse. Mais enfin nous n’en mourrons pas moins,
et ce néant-là — être mortel — ne cessera, tant que nous vivrons, de nous accompagner.
Néant toujours possible et toujours nécessaire. C’est l’ombre de la
mort, sur la clairière de vivre. Ombre imaginaire ? Sans doute, puisque nous
% 51
vivons. Mais réelle pourtant, puisque toute vie est mortelle. C’est ce qui nous
voue à l’angoisse ou au divertissement.

Avoir peur de la mort, c’est avoir peur de rien. Cette idée vraie ne suffit
pourtant pas à nous rassurer. Comment le pourrait-elle, puisque le {\it rien}, dans
l'angoisse, est justement ce qui nous effraie ?

Ainsi la peur de la mort est le modèle de toute angoisse, et l’origine, selon
Lucrèce, de toutes.

Elle en indique aussi le remède. Si l'angoisse est sentiment du néant, elle ne
peut être combattue que par une certaine expérience de l'être. Mieux vaut
penser ou affronter ce qui est qu’imaginer ce qui n’est pas : la connaissance et
l’action valent mieux que l’angoisse, et en guérissent.

Elles seules ? Non pas. Toujours ? Non plus. Car l'angoisse est aussi un état
du corps, qui peut résister à toute pensée, et contre lequel nous disposons
aujourd’hui, grâce à la médecine, de traitements efficaces. On aurait tort de
s’en plaindre, et tort aussi de s’en contenter.

Contre l’angoisse ? Le réel (la connaissance, l’action, la sagesse), ou un petit
morceau du réel (un anxiolytique). Philosophie ou médecine, et parfois l’une et
l’autre. La santé n’a jamais suffi à la sagesse, ni la sagesse à la santé.

\section{Animaux}
%ANIMAUX
Qu'ils aient une âme {\it (anima)}, c’est ce que suggère l’étymologie,
et que confirme l'observation. L'animal est un vivant {\it animé},
c’est-à-dire capable de sentir et de se mouvoir.

Peut-il aussi penser ? Bien sûr, puisque l’homme pense, qui est un animal,
et puisque l'intelligence des bêtes, pour inférieure qu’elle soit ordinairement à
la sienne, est susceptible de degrés, qui se mesurent. Un chimpanzé est plus
intelligent qu’un chien, qui est plus intelligent qu’une huître. L'intelligence,
toutefois, n’est pas essentielle à la notion. Un débile profond n’est pas moins
animal qu’un génie, ni davantage. Est-il moins humain ? Non plus, puisqu'il
appartient à la même espèce. C’est où la biologie est un guide plus sûr que
l'anthropologie, et plus contraignant. L'homme n’est pas un animal qui pense ;
c’est un animal qui est né de deux êtres humains. Du moins jusqu’à présent, et
c’est une de mes raisons d’être hostile au clonage. La filiation, qui suppose la
différence, vaut mieux que la répétition, qui voudrait s’en dispenser.

Les naturalistes distinguent traditionnellement trois règnes, qui sont le
minéral, le végétal et l'animal. La transition, de l’un à l’autre, est moins évidente
qu’on ne le croit parfois. Voyez les coraux ou les éponges. Mais la différence,
dans son principe, reste claire. Le minéral est sans vie. Le végétal, sans
âme. Seul l'animal est animé : lui seul sent qu’il vit. C’est ce qui le voue au
% 52
plaisir et à la souffrance, et qui nous donne des devoirs envers lui. Descartes,
humaniste et inhumain.

\section{Animaux machines (théorie des —)}
%ANIMAUX MACHINES (THÉORIE DES -)
C'est une théorie de Descartes
et des cartésiens, qui voulaient
que les bêtes ne soient que mécanisme, sans rien qui pense ou qui sente. Un
chien gémit quand on le bat ? C’est comme un réveil qui sonne ou une porte
qui grince. Et qui plaindrait une porte ou un réveil ? Le bons sens et la biologie
eurent vite raison de ces billevesées philosophiques. Non qu’il n’y ait
rien de mécanique en l’animal, mais en ceci qu’il n’est animal que pour
autant qu'un mécanisme lui permet de sentir ou de ressentir. Un animal
insensible ne serait plus un animal ; ce serait un robot naturel. À l'inverse, un
robot sensible, comme on en voit dans nos films de science-fiction, serait un
animal artificiel. La notion n’est pas plus contradictoire que la chose, en
droit, n’est impossible.

\section{Animisme}
%ANIMISME
Au sens étroit, c’est expliquer la vie par la présence, en chaque
organisme, d’une âme. S’oppose alors au matérialisme (qui
l'explique par la matière inanimée) et se distingue du vitalisme (qui ne
l'explique pas).

En un sens plus général, c’est imaginer partout de l’âme {\it (anima)} ou de
l'esprit {\it (animus)}, y compris dans les êtres qui semblent dépourvus de toute
sensibilité : dans l'arbre, dans le feu, dans la rivière, dans les étoiles. C’est la
première superstition, et le principe peut-être de toutes. Mais c’est aussi, pour
Auguste Comte, le commencement nécessaire de l'esprit. Il faut croire avant de
connaître. Et quoi de plus facile à croire que l'esprit, que toute croyance
suppose ?

Auguste Comte utilisait plutôt le mot {\it fétichisme}, que nous réservons à
un autre usage. Il y voyait le premier stade de l’âge théologique, qu’il jugeait
à la fois plus spontané et plus logique que les deux autres (le polythéisme et
le monothéisme). « Concevoir tous les corps extérieurs quelconques, naturels
ou artificiels, comme animés d’une vie essentiellement analogue à la
nôtre », comme il disait, c’est certes une erreur, mais c’est aussi un premier
pas vers le réel et la compréhension du réel. Mieux vaut se tromper sur ce
monde qu’en inventer un autre. Les esprits sont moins encombrants que les
dieux.

Ou bien il faut que les dieux s’en aillent tout à fait, très loin, comme les
dieux d’Épicure, comme le Dieu de Simone Weil, et laissent enfin le monde à
% 53
la matière sans esprit — sourde aux prières, comme dit Alain, fidèle aux mains.
Le contraire de l’animisme, comme de toutes les religions, c’est le travail, la
connaissance et l’action.

\section{Anomie}
%ANOMIE
Absence de loi ou d'organisation. Chez Durkheim, c’est une espèce
de dérèglement social, qui vient briser ou mettre à mal la cohésion
ou la « solidarité organique » d’une société. L’individu se trouve alors abandonné
à lui-même, sans loi, sans {\it repères}, comme on dit aujourd’hui, sans
limites, sans garde-fous. C’est ce qui le voue à l’angoisse, à la démesure, à la violence —
ou au suicide.

\section{Antéprédicatif}
%ANTÉPRÉDICATIF
Ce qui est antérieur à tout jugement prédicatif, c’est-à-dire
à toute attribution d’un prédicat à un objet. S'il
n’y avait, spécialement dans la sensation, quelque chose de tel, que resterait-il à
juger ? Ainsi c’est le silence qui rend le discours possible.

\section{Anthropique (principe —)}
%ANTHROPIQUE (PRINCIPE -)
Puisque nous existons, l'univers a nécessairement
un certain nombre de caractéristiques
sans lesquelles notre existence serait impossible. De là le {\it principe anthropique},
qui permet en quelque sorte de remonter de l’homme vers l'univers, de
la biologie vers la physique, enfin du présent vers le passé. N'est-ce pas renverser
l’ordre des causes ? La réponse dépend de l'interprétation qu’on donne
d’un tel principe, voire de sa formulation. Il peut en effet être énoncé sous deux
formes différentes. Dans sa forme faible (Dicke, 1961), il stipule que « puisqu’il
y a des observateurs dans l’univers, ce dernier doit posséder des propriétés
qui permettent l'existence de tels observateurs ». Ce qu'on ne peut guère
contester : du fait que l'humanité fait partie du réel, on peut évidemment
conclure que l’univers est tel que l'humanité soit possible. Dans sa forme forte
(Carter, 1973), en revanche, le principe semble beaucoup plus discutable. II
affirme que « l'univers doit être constitué de telle façon dans ses lois et son
organisation qu’il ne manque pas de produire un jour un observateur ». C’est
passer du possible au nécessaire, ce que rien n’autorise, et considérer l'humanité
comme le but au moins partiel de l’univers : c’est un principe anthropo-téléologique,
voire anthropo-théologique, qui excède de très loin ce qu’on peut
demander à la physique. Mais enfin les physiciens ont bien le droit, eux aussi,
de faire de la métaphysique.

% 54
\section{Anthropocentrisme}
%ANTHROPOCENTRISME
C'est mettre l’homme au centre, non des valeurs,
comme fait l’humanisme, mais des êtres : parce
que l'univers n'aurait été créé que pour nous, ou tournerait autour. La notion
est aussi facile à comprendre, d’un point de vue psychologique (c’est comme un
narcissisme de l'espèce), que difficile, d’un point de vue rationnel, à accepter.
Pourquoi ce privilège exorbitant de l'humanité ? Il y faut le secours de la religion,
qui est un anthropocentrisme paradoxal (le vrai centre reste Dieu), ou du
criticisme, qui est un anthropocentrisme gnoséologique. La « Révolution
copernicienne », que Kant nous propose, est en vérité une contre-révolution : il
s’agit de remettre l’homme au centre, d’où les progrès des sciences l'avaient
chassé. Au centre de ses connaissances, certes, par le transcendantal ; mais au
centre aussi de la création (comme son but final), par la liberté. C'était accepter
les Lumières sans renoncer à la foi. La question « Qu'est-ce que l’homme »,
disait Kant, est la question centrale de la philosophie, à laquelle toutes les autres
se ramènent. J'y vois de l’anthropocentrisme philosophique, et une raison forte
de n’être pas kantien.

Freud, sur ce sujet, m’éclaire davantage. Dans un passage fameux de ses
{\it Essais de psychanalyse appliquée}, il évoque les trois blessures narcissiques que
l’humanité, du fait des progrès scientifiques, a subies : la révolution copernicienne,
la vraie, celle de Copernic, qui chasse l’homme du centre de l’univers
(c’est l’humiliation cosmologique) ; l’évolutionnisme de Darwin, qui le réintroduit
dans le règne animal (c’est l’humiliation biologique) ; enfin la psychanalyse
elle-même, qui montre que «le moi n’est pas maître dans sa propre
maison » (c'est l’humiliation psychologique). J’ajouterais volontiers Marx,
Durkheim et Lévi-Strauss, qui montrent que l'humanité n’est pas davantage
maîtresse d’elle-même ou de l’histoire. Il faut certes rappeler, avec Rémi Brague
que la position centrale, chez les Anciens, était loin d’être privilégiée (voyez le
corps humain, disait Plotin, voyez la sphère, disait Macrobe : le centre, dans les
deux cas, serait plutôt le plus bas..), que la Terre, jusqu’à la Renaissance, est
plutôt considérée comme le cul-de-basse-fosse de l’univers, mais peu importe.
L'essentiel, en l’occurrence, et qui donne malgré tout raison à Freud, c’est que
toute notre modernité épistémique s’est jouée {\it contre} l’anthropocentrisme. Que
croyez-vous qu’il advint ? Le narcissisme s’est trouvé des consolations : philosophiques
(Kant, Husserl), scientifiques ou supposées telles (le principe anthropique),
enfin et surtout psychanalytiques. Le moi n’est plus maître dans sa
propre maison ? Qu'importe, puisque l’inconscient, absurdement, fait comme
un autre moi, qui fascine encore plus que l’autre ! Qu'il y ait là un contresens
sur la psychanalyse, c’est ce que je crois, mais qui n’échappe pas à la règle habituelle
du succès, qui est de malentendu. Narcisse à quitté sa fontaine ; il
s’allonge sur le divan. « Qu'est-ce que je suis intéressant ! Quelle profondeur !
% 55
Quelle complexité! Et mon père! Et ma mère! Quel abîme de sens, de
drames, de fantasmes, de désirs ! » Et voilà que la psychanalyse, de blessure narcissique
qu’elle se voulait d’abord, n’est plus qu’une consolation narcissique
comme une autre, simplement un peu plus prétentieuse et bavarde que la plupart.
Heureusement qu’elle nous guérit, parfois, d’elle-même. Quand tu as
cessé de t’intéresser, la cure est finie.

\section{Anthropologie}
%ANTHROPOLOGIE
Étymologiquement, c’est la connaissance {\it (logos)} de
l’homme {\it (anthropos)}. Le terme est vague ; la chose
aussi. S'agit-il de philosophie? De science? Mais alors de laquelle, ou
desquelles ? Beaucoup de ce que nous savons de l’homme nous est appris par
des sciences (la physique, la biologie, la paléontologie...), dont il n’est nullement
l’objet spécifique. Quant aux sciences dites humaines (Pethnologie, la
sociologie, la psychologie, la linguistique, l’histoire...), elles échouent à se constituer
en une science unique, qui serait justement l’anthropologie — ou plutôt
elles n’existent, les unes et les autres, que par le refus de se fondre dans un discours
unique, qui perdrait ce que chacune d’entre elles a de radical et de tranchant.
L'unité de l'espèce n’est pas en question; mais son autonomie, si.
« L'homme n’est pas un empire dans un empire », disait Spinoza. C’est ce qui
interdit à l’humanisme de valoir comme religion, et à l’anthropologie de valoir
comme science.

\section{Anthropomorphisme}
%ANTHROPOMORPHISME
C’est donner forme d’homme à ce qui n’est pas
humain, spécialement aux animaux ou aux
dieux. Ainsi dans les fables ou les religions. « Si Dieu nous a fait à son image,
écrit Voltaire, nous le lui avons bien rendu. »

\section{Anthropophages}
%ANTHROPOPHAGES
Un nom savant, pour désigner les cannibales : ceux
des humains qui ne répugnent pas à manger de la
chair humaine. Le fait est bien avéré, dans la quasi-totalité des civilisations primitives.
Il relève du rituel, presque toujours, davantage que de la gastronomie.
Cela nous choque, mais n'empêche pas que nous fassions pire : « Nous tuons
en bataille rangée ou non rangée nos voisins, remarque Voltaire, et pour la plus
vile récompense nous travaillons à la cuisine des corbeaux et des vers. C’est Là
qu'est l’horreur, c’est là qu’est le crime ; qu'importe quand on est tué d’être
mangé par un soldat, ou par un corbeau ou un chien ? Nous respectons plus les
% 56
morts que les vivants. Il aurait fallu respecter les uns et les autres » ({\it Dictionnaire...},
art. « Anthropophages » ; voir aussi Montaigne, {\it Essais}, I, 31).

\section{Anticipation}
%ANTICIPATION
C'est être en avance sur le présent. Cela vient ordinairement
du passé. Ainsi, chez Épicure, l’anticipation ou prénotion
{\it (prolèpsis)} est-elle une idée générale, qui résulte de la répétition d’expériences
singulières. Par exemple cet animal devant moi. Si je dis {\it « C'est un chien »},
c'est que j'en avais déjà l’idée avant de le percevoir, ce qui me permet seul de
reconnaître que c’en est un. Cette idée, qui résulte de la répétition de perceptions
antérieures, c’est ce qu'Épicure appelle la {\it prolèpsis}, qu’on traduit habituellement
par anticipation : c’est avoir une idée d’avance sur le réel, c’est-à-dire une idée.
Mais elle n’est possible que parce que le réel toujours la précède.

\section{Antimatière}
%ANTIMATIÈRE
Les physiciens appellent ainsi des particules, dites « antiparticules »,
qui seraient symétriques — parce que de même
masse et de charge électrique opposée — aux particules qui constituent la
matière ordinaire, celle qui nous constitue et nous environne. Philosophiquement,
c’est bien sûr un abus de langage : si cette antimatière existe objectivement,
indépendamment de l'esprit ou de la pensée, elle est aussi matérielle que
le reste.

\section{Antinomie}
%ANTINOMIE
Contradiction nécessaire, entre deux thèses également vraisemblables
ou prétendument démontrées. Kant appelle {\it antinomies
de la raison pure} les conflits dans lesquels la raison entre inévitablement
avec elle-même, tant qu’elle prétend atteindre l’inconditionné. Il en retient
quatre : on peut démontrer que le monde a un commencement dans le temps
et une limite dans l’espace, comme on peut démontrer qu’il n’en a pas ; que
tout est composé de parties simples ou qu’il n’existe rien de simple dans le
monde ; qu’il existe une causalité libre ou que tout arrive, au contraire, suivant
les lois de la nature ; enfin qu’il existe un être absolument nécessaire, ou qu’il
n’en existe aucun ({\it C. R. Pure}, Des raisonnements dialectiques, II). Ces quatre
antinomies condamnent le scientisme autant que la métaphysique dogmatique,
et justifient, selon Kant, le criticisme.

\section{Antiquité}
%ANTIQUITÉ
Tout ce qui est très ancien, et spécialement (avec une majuscule)
la longue période qui sépare la fin de la préhistoire
du début du Moyen Âge : depuis l’invention de l'écriture, il y a quelque
5 000 ans, jusqu’à la chute de l’empire romain — c’est du moins la convention
qui s’est imposée en Europe -, soit environ trente-cingq siècles d’histoire.. La
notion est par nature relative et rétrospective. Aucune époque, jamais, ne s’est
vécue comme antique. Les Grecs eux-mêmes se voyaient plutôt comme des
tard-venus, des héritiers, des continuateurs, voire, si l’on en croit Platon,
comme des « enfants » (l’Antiquité, pour eux, était égyptienne). Il n’y a pas
d’antiquité absolue, ni présente. Il n’y a que l'actualité de tout, et l’immensité
de l’histoire.

L'idée d’ancienneté, qui est le sens premier du mot, ne doit pas être confondue
avec celle de vieillesse. Si la vieillesse, comme le remarquait Pascal,
est l’âge le plus distant de l'enfance, il faut en conclure — contre Platon — que
« ceux que nous appelons anciens étaient véritablement nouveaux en toutes
choses, et formaient l'enfance des hommes proprement » ; c’est nous, à côté,
qui sommes des vieillards. De là le charme, pour les Modernes, de l’art
antique, qui est celui, suggère Marx, d’une enfance préservée et perdue :
nous admirons d’autant plus sa beauté qu’elle nous est définitivement interdite.

\section{Antithèse}
%ANTITHÈSE
Pour la rhétorique, c’est une simple opposition. Pour les philosophes,
c’est le plus souvent une thèse qui s'oppose à une
autre (par exemple, chez Kant, dans les antinomies de la raison pure). C’est
aussi le deuxième moment de la dialectique hégélienne, qui est d'inspiration
ternaire : l’antithèse s'oppose à la thèse, mais cette opposition est elle-même
appelée à être « dépassée » — à la fois conservée et supprimée — par la synthèse.
Ainsi l’opposition de l’être et du néant, dans le devenir.

\section{Antitrinitaires}
%ANTITRINITAIRES
Ceux qui ne croient pas à la Trinité. Voltaire, dans son
{\it Dictionnaire}, a beau jeu de montrer que la raison est de
leur côté. Mais pourquoi faudrait-il que Dieu soit rationnel ? Trois personnes
en une seule essence, c’est inintelligible. Mais pas plus que l’idée d’une personne
infinie et toute-puissante. Celui qui dit « Dieu », c’est qu’il a déjà renoncé à comprendre.
Que ce Dieu soit un, trois ou quarante-douze n’y change rien.

\section{Anxiété}
%ANXIÉTÉ
Le mot vaut souvent comme synonyme d’angoisse (beaucoup de
langues ne distinguent pas les deux notions), spécialement dans
le langage médical (qui les distingue de moins en moins). L’anxiété, toutefois,
% 58
tire davantage vers la psychologie, et moins vers la philosophie. C’est un trait
de caractère plutôt qu’une position existentielle, un état plutôt qu’une expérience,
une disposition pathologique plutôt qu’ontologique. C’est comme une
angoisse sans prétention, qui porte moins sur le néant que sur Le possible, et
qui ressemble pour cela davantage à la crainte. C’est la peur vague de quelque
chose de précis, et la peur de cette peur, et la propension à la ressentir.
L’anxieux a toujours une peur d’avance : il vérifie trois fois qu’il a fermé sa
porte, redoute d’être suivi ou agressé, craint toujours — pour lui-même
comme pour ses proches — la maladie, l’accident, le malheur... Il prend,
contre sa propre peur, un luxe de précautions, qui l’augmentent. Il a peur
d’avoir peur, et s’en effraie.

L’anxiété, même pathologique, n’est pas toujours sans fondement. Elle
résulte plutôt de la conscience aiguë des dangers que nous courons en effet,
mais en exagère la probabilité et ne cesse, c’est son supplice, d’anticiper sur eux.
C’est une peur intempestive et disproportionnée.

Son contraire est la confiance ; ses remèdes, la médecine ou l’action.

\section{Apagogique (raisonnement —)}
%APAGOGIQUE (RAISONNEMENT —)
C’est le nom savant ou cuistre du
raisonnement par l’absurde (voir ce
mot). Peut aussi désigner un raisonnement qui prouve la vérité d’une proposition
par la réfutation non de la seule contradictoire (comme dans le raisonnement
par l’absurde) mais de toutes les propositions qu’on pourrait légitimement
lui substituer, dans la solution d’un même problème. La lourdeur de la
procédure n’excuse qu’à peine celle du mot.

\section{Apathie}
%APATHIE
C'est l'absence de passion, de volonté ou d’énergie. Cette polysémie,
toutefois, est propre aux Modernes, qui croient volontiers
que toute énergie ou toute volonté est passionnelle. Ils voient souvent dans
l’apathie un symptôme (notamment dans les états schizophréniques ou dépressifs),
et sans doute ils n’ont pas tort. Mais n’est-ce que cela ? Si l’on prend le
mot en son sens originel ou étymologique (absence de passion, de trouble, de
pathos), la perspective change du tout au tout : l’{\it apatheia}, chez les stoïciens,
n'était pas une faiblesse mais une vertu. C’est qu’ils croyaient au courage
davantage qu'aux passions, et n’avaient pas besoin, pour agir, de se laisser
emporter. Peut-être étaient-ils plus lucides que nous sur les passions, comme
nous le sommes davantage qu’eux sur la volonté ? On ne connaît vraiment que
ce qu'on surmonte. Chaque époque a les lucidités qu’elle mérite.

% 59
\section{Aperception}
%APERCEPTION
C'est percevoir qu’on perçoit, ou se percevoir soi-même
percevant : conscience de soi, donc, sans laquelle il n’y
aurait pas de conscience du tout.

Kant appelle {\it aperception transcendantale} la conscience de soi-même, mais
comme conscience « pure, originaire et immuable », grâce à laquelle l'unité du
{\it « je pense »} peut et doit accompagner toutes nos représentations, et sans laquelle
celles-ci ne sauraient être nôtres ({\it C. R. Pure}, Déduction des concepts, \S 16 à
21). Cette unité synthétique de l’aperception est « le point le plus élevé auquel
il faut rattacher tout l’usage de l’entendement», ou plutôt elle est
« l’entendement même », lequel n’est « rien de plus que le pouvoir de lier {\it a
priori} et de ramener le divers de représentations données à l’unité de
l’aperception ; c’est là le principe suprême dans la connaissance humaine tout
entière » ({\it ibid.}, \S 16). C’est qu’il n’y a de connaissance que pour une conscience,
et dans la mesure seulement où elle est consciente de soi. Si ma calculette
se savait calculante, elle ne serait plus une calculette. Mais elle s’ignore
elle-même ; comment pourrait-elle connaître quoi que ce soit ? Elle calcule ;
elle ne sait pas compter.

\section{Aphasie}
%APHASIE
C’est être pathologiquement incapable de parler, mais pour des
raisons neurologiques ou mentales plutôt que physiologiques,
sensorielles ou motrices. Trouble du cerveau, plutôt que de l’ouïe ou des cordes
vocales.

À ne pas confondre avec l’{\it aphasia} de Pyrrhon, qui pourrait parler, qui parle
parfois, mais n’en a plus besoin.

L’aphasie est une prison, qui nous enferme dans le silence ; l’{\it aphasia}, une
liberté, qui nous y ouvre. C’est que ce n’est pas le même silence : l’un en deçà
de la parole, l’autre au-delà et les contenant toutes.

\section{Apocalypse}
%APOCALYPSE
Une révélation {\it (apokalupsis)} ou la fin des temps, telle qu’elle
est annoncée dans le livre éponyme attribué à saint Jean. Que
le mot ait fini par désigner une catastrophe particulièrement épouvantable est
révélateur : la peur, même chez les croyants, l'emporte sur l’espérance.
D'ailleurs, s’ils n'avaient pas peur, auraient-ils besoin à ce point d’espérer ?

\section{Apodictique}
%APODICTIQUE
Désigne une nécessité logique, telle qu’on en trouve dans
les démonstrations (le mot vient du grec {\it apodeiktikos},
démonstratif).

% 60
C'est aussi une des modalités du jugement : une proposition quelconque
peut être assertorique (si elle énonce un fait), problématique ou hypothétique
(si elle énonce une possibilité), enfin apodictique (si elle énonce une nécessité).
Il est important de distinguer ces deux sens, car le premier vaut comme certitude,
et point du tout le second. La certitude d’une proposition dépend non de
la modalité du jugement qu’elle énonce, mais de la validité de sa démonstration.
Une proposition assertorique (« Dieu existe »), problématique (« Il se peut
que Dieu existe ») ou apodictique (« Dieu existe nécessairement ») n’est certaine
que si sa démonstration, elle, est apodictique — autrement dit que si elle
est vraiment une démonstration. C’est ce qui explique qu’on puisse douter
d’une nécessité ou d’un fait, et être certain d’une possibilité.

\section{Apollinien}
%APOLLINIEN
L’un des deux principes, selon Nietzsche, de l’art grec, et peut-être
de tout art. Le principe apollinien, c’est le principe de
l'individuation, par lequel chaque être est ce qu’il est, mais aussi de l'équilibre
et de la mesure, par quoi il s’en contente. S’oppose au principe dionysiaque,
qui est celui de la démesure, de la fusion, du devenir, de l’illimité — du tragique.
Ces deux principes sont complémentaires : la plupart des chefs-d’œuvre relèvent
des deux. Mais cela n’interdit pas des différences d’accents ou de degrés.
Le principe apollinien, qui est celui de la belle forme, règne sur les arts plastiques
et culmine dans le classicisme. Le dionysiaque, qui est celui de l’élan, est
surtout chez lui dans la musique : il culmine dans le baroque ou le romantisme.
C’est le dionysiaque, pour Nietzsche, qui est premier. L'équilibre, la
mesure ou le classicisme ne sont jamais donnés, toujours à conquérir. D’abord
l'ivresse, puis la lucidité.

\section{Apophantique}
%APOPHANTIQUE
{\it Apophansis}, en grec, c’est la proposition. Un discours
apophantique est un discours qui asserte, c’est-à-dire qui
affirme ou nie, et peut par conséquent être vrai ou faux. Par extension, le mot
se dit aussi de tout ce qui concerne le jugement ou en fait la théorie (ainsi l’apophantique
formelle, chez Husserl).

\section{Apophatique}
%APOPHATIQUE
Du grec {\it apophanai}, dire non. Un discours apophatique
est celui qui ne procède que par négations. Se dit spécialement
de la théologie. La théologie négative ou apophatique est celle qui se
reconnaît incapable de dire ce qu’est Dieu, sans renoncer tout à fait à en parler :
elle ne le connaît que comme inconnaissable ; elle ne le dit que comme indicible.

% 61
Mais c’est une manière encore de l’affirmer. Faute de pouvoir l’atteindre
ou le comprendre en disant ce qu’il est, elle tente de le cerner (si l’on peut dire)
négativement, en disant ce qu’il n’est pas. Il peut sembler que le silence vaudrait
mieux. Mais le silence ne fait pas une théologie.

\section{Aporie}
%APORIE
C'est une contradiction insoluble, ou une difficulté, pour la pensée,
insurmontable. Par exemple la question de l’origine de l’être
est une aporie : parce que toute origine suppose l'être et ne saurait donc
l'expliquer. L’aporie est une espèce d’énigme, mais considérée d’un point de
vue logique plutôt que magique ou spirituel. C’est un problème qu’on
renonce à résoudre, au moins provisoirement, ou un mystère qu’on refuse
d’adorer.

\section{\it A posteriori}
%{\it A POSTERIORI}
Tout ce qui est postérieur à l’expérience et en dépend.
S’oppose à l’{\it a priori}, le suppose (selon Kant) et vaut mieux
(selon lusage courant). On n’a jamais raison qu'après coup. Même un calcul
ou une démonstration — qui sont comme des expériences de la pensée — ne sont
vrais, pour nous, qu’une fois qu’on les a faits.

\section{Apparence}
%APPARENCE
Tout ce qui se donne à voir ou à sentir par l’un quelconque de
nos sens, voire, plus généralement, par notre conscience
elle-même. Cette feuille de papier devant moi, sa forme, sa blancheur, ou
bien ce bouquet de fleurs, ou le bruit au loin de la rue, ce sont autant
d’apparences. Cela ne prouve pas qu’il n’y ait pas en effet une feuille, un
bouquet, une rue, mais ne saurait non plus attester leur existence objective,
ni qu'ils aient les caractères qu’ils me semblent avoir. Il se peut que je rêve,
ou que je sois fou, ou que la matière n'existe pas, que mon propre corps ne
soit qu’une illusion trompeuse, bref qu’il n’y ait partout que des apparences...
On dira que rien n’apparaîtrait si rien n’était. Cette évidence ne
fait qu’une apparence de plus ({\it evidens}, en latin, c’est ce qui se voit...) ; mais
admettons-la pourtant. Que l’être soit, qu'est-ce que cela nous apprend s’il
est pour nous hors d’atteinte, si l’on ne connaît jamais que des apparences,
dont on ignore si elles sont vraies ou fausses ? On ne pourrait en effet le
savoir qu’en confrontant ce qui apparaît à ce qui est. Mais cette confrontation
n’est elle-même possible qu’à la condition que ce qui est — le réel —
apparaisse d’une façon ou d’une autre : ce ne serait pas confronter l’apparence
à l’être, mais simplement une apparence à une autre apparence,
% 62
comme on fait toujours, comme il faut faire de toute façon. Ainsi l’apparence
est non seulement le point de départ obligé, mais le seul point
d'arrivée qui nous soit accessible. C’est un autre nom pour le réel, en tant
qu'on ne le connaît jamais immédiatement ni absolument.

Kant distingue l'apparence {\it (Schein)} du phénomène {\it (Erscheinung)}. L’apparence,
c'est ce qui, dans l’expérience (apparence empirique) ou dans la pensée
(apparence transcendantale), relève de l'illusion. Ainsi le bâton qui dans l’eau
paraît brisé, ou la métaphysique dogmatique qui paraît démontrer des propositions
antinomiques — par exemple sur le monde ou Dieu —, quand elle est incapable,
en vérité, de démontrer quoi que ce soit qui dépasse le champ d’une
expérience possible. L’apparence est une erreur de jugement, sur ce que les sens
ou l’entendement proposent. Le phénomène, lui, n’a rien d’une erreur : il serait
plutôt la réalité même, non certes telle qu’elle est en soi, ce que nul ne connaît,
mais telle qu’elle se donne dans l’expérience. Disons que c’est une apparence
vraie, comme l’apparence est un phénomène trompeur, ou plutôt sur lequel on
se trompe.

Cette distinction a de moins en moins cours dans la philosophie contemporaine.
D'abord parce que si l’on ne perçoit que des phénomènes, rien n’autorise
à affirmer qu’ils soient autre chose que des apparences. C’est la revanche de
Hume, si l’on veut, contre Kant. Ensuite parce que les phénoménologues nous
ont habitués à rejeter, comme dit Sartre, « le dualisme de l'être et du paraître ».
Si « l'être d’un existant, c’est précisément ce qu’il {\it paraît} », si le phénomène ne
renvoie plus à une chose en soi mais à d’autres phénomènes, et ainsi à l'infini,
alors l'apparence retrouve sa légitimité ontologique : « L’apparence ne cache
pas l'essence, elle la révèle : elle {\it est} l'essence » ({\it L'être et le néant}, introd.). Enfin
parce que l’apparence alors « est le tout », comme dit Marcel Conche, puisqu'il
n'y a rien d’autre, ou puisque rien d’autre, en tout cas, ne se donne. Cette
apparence n'est ni manifestation ni illusion. Elle n’est pas apparence {\it de} (ce qui
supposerait autre chose que l’apparence, qui serait caché derrière elle), ni apparence
{\it pour} (ce qui serait enfermer l'apparence dans le sujet, quand il n’est lui-même
qu'une apparence parmi d’autres), mais «apparence pure et
universelle », comme dit encore Marcel Conche, voire « apparence absolue ».

C’est le monde même, quand on a renoncé à le connaître absolument. Mais
faut-il renoncer ?

\section{Appétence}
%APPÉTENCE
Mot savant pour dire l’appétit en général (par différence avec
le sens moderne et particulier d’appétit, qui désigne de plus
en plus le désir de nourriture). On y perd en saveur, me semble-t-il, plus qu’on
n’y gagne en clarté.

% 63
\section{Appétit}
%APPÉTIT
Le désir, considéré dans sa matérialité.

L’appétit est au corps ce que le désir est à l’âme : la puissance de
jouir de ce qui nous est nécessaire, utile ou agréable. Mais si âme et le corps
sont une seule et même chose, comme dit Spinoza et comme je le crois, le désir
et l'appétit ne se distinguent que par le point de vue qui les considère : surtout
physiologique ici, davantage psychologique là. Ce sont deux façons de dire une
même attirance pour ce qui nous permet d’exister plus ou mieux, une même
tendance, une même pulsion — deux aspects, donc, de notre {\it conatus}. « L’appétit
n’est rien d’autre que l’essence même de l’homme, de laquelle suit nécessairement
ce qui sert à sa conservation ; et l’homme est ainsi déterminé à le faire »
({\it Éthique}, III, 9, scolie). Non qu’on ne puisse faire autre chose ; mais on ne le
peut, ou on ne le pourrait, qu’à la condition d’un autre appétit.

Dans le langage courant, le mot désigne surtout le désir de manger. C’est
que la physiologie règne ici, sinon seule (un état d’âme peut augmenter ou
diminuer l’appétit), du moins en maître. On évitera pourtant de confondre
l'appétit et la faim : la faim est un manque, une faiblesse, une souffrance ;
l'appétit, une force et, déjà, un plaisir.

{\it Appetere}, en latin, c’est approcher, chercher à atteindre, tendre vers. Qu’on
puisse tendre vers ce qui manque, c’est entendu. Mais on peut tendre aussi vers
ce qui est là, qui ne manque pas, qui est offert ou disponible. Manger {\it de bon
appétit} n’est pas la même chose que souffrir de la faim.

On peut parler d’appétit sexuel, ou d’appétence, pour désigner ce qu’il y a,
dans la sexualité, de physiologique (quant à ses causes) et d’indéterminé (quant
à son objet). Le désir porte sur telle ou telle femme en particulier, sur tel ou tel
homme, alors que l’appétence se suffit du coït, indépendamment de l’objet
interchangeable — à la fois nécessaire et indifférent — qui pourrait nous permettre
d’y atteindre.

On parlera d’appétit, dans le même esprit, pour désigner l’envie indéterminée
de (bien) manger. C’est la joie du convive quand il se met à table, avant
même la lecture du menu. Reste à choisir son plat, ce qui suppose le désir d’un
mets en particulier.

Cela dit quelque chose sur la gastronomie comme sur l'érotisme. II s’agit de
transformer l’appétit en désir, et le désir en plaisir ou en joie — en amour. Cela
ne va pas sans art ni, peut-être, sans artifice.

\section{\it A priori}
%{\it A PRIORI}
Tout ce qui, dans l'esprit, est indépendant de l’expérience, et spécialement
ce qui la rend possible (le transcendantal), qui doit
donc, au moins logiquement, la précéder. Ainsi, chez Kant, les formes {\it a priori}
de la sensibilité (l’espace et le temps) et de l’entendement (les catégories).

% 64
On ne confondra pas l’{\it a priori} et l’inné. L’inné désigne une antériorité
chronologique ou de fait ; l’{\it a priori}, une antériorité logique ou de droit. L’inné
relève de la métaphysique, de la psychologie ou, de plus en plus, de la biologie ;
l’{\it a priori}, de la gnoséologie ou théorie de la connaissance. L’{\it a priori} peut être
acquis (pourvu qu’il soit tiré non de l’expérience, dirait Kant, mais de l’activité
même de l'esprit) ; l’inné, par définition, ne le peut. Enfin l’inné, quant au
corps, existe évidemment : c’est le corps lui-même, et spécialement le cerveau.
L’{\it a priori}, quant à l'esprit, reste douteux : il n’existe que si l’esprit est autre
chose que le corps.

« Que toute connaissance débute {\it avec} l'expérience, écrit Kant, cela ne
prouve pas qu’elle dérive toute {\it de} l’expérience » ({\it C. R. Pure}, introd.). Certes.
Mais cela ne l’exclut pas non plus. Il se pourrait que ce soient Locke et les
empiristes qui aient raison, autrement dit que rien ne soit dans l'esprit qui
ne vienne des sens ou de l’expérience. C’est la fameuse formule : {\it « Nihil est
in intellectu, quod non fuerit in sensu »} (il n’y a rien dans l’entendement qui
n'ait été d’abord dans les sens). À quoi Leibniz ajoutait, en guise d’objection :
{\it nisi ipse intellectus}, « si ce n’est l'esprit lui-même ». L’{\it a priori}, s’il
existe, est cette antériorité logique de l'esprit par rapport à toutes les données
empiriques qui lui donnent l’occasion de s'exercer. Les matérialistes y
voient plutôt une faculté du cerveau. Rien n’existe dans l'esprit, diraient-ils
volontiers, indépendamment de l’expérience que le cerveau fait du monde et
de soi — rien, si ce n’est le cerveau lui-même : ce n’est plus de l’{\it a priori}, c’est
de l’inné.

Dans le langage courant, l'expression {\it a priori} désigne une hypothèse qui
reste à vérifier, voire un préjugé ou un parti pris. Cet usage, qui prête à confusion,
est à proscrire dans le discours philosophique. C’est ce qui interdit de dire
que le concept d’{\it a priori} (au sens technique) est un {\it a priori} (au sens trivial).
Mais cela n’interdit pas de le penser.

\section{Arbitraire}
%ARBITRAIRE
Qui ne dépend que d’une volonté sans raison ou sans justification.
Par exemple le libre arbitre, s’il était absolument
indéterminé, serait arbitraire. Mais alors il ne serait plus libre.

\section{Arbitre}
%ARBITRE
Un individu neutre, chargé, avec leur accord, d’assurer l'équilibre
ou de résoudre les conflits entre plusieurs parties en présence.
C’est ce qui rend la notion de {\it libre arbitre} paradoxale. Si c’est moi qui décide,
comment serait-il neutre ? Si ce n’est pas moi, comment serais-je libre ?

% 65
\section{Achétype}
%ARCHÉTYPE
C’est un modèle {\it (tupon)} qui sert de principe {\it (arkhê)}, voire
une forme primordiale, dont le réel serait la copie. Ainsi les
Idées, chez Platon, ou les structures de l'inconscient collectif, chez Jung. C’est
comme une pensée d’avant la pensée, qui lui servirait de modèle. Mais si l’on
avait des modèles, aurait-on besoin de penser ?

\section{Argument}
%ARGUMENT
C’est une idée qui tend à en justifier une autre, sans suffire
pourtant à l’imposer. L’argument n’est pas une preuve, mais
ce qui en tient lieu quand les preuves font défaut.

\section{Argumentation}
%ARGUMENTATION
Un ensemble ordonné d’arguments, qui tend à justifier
une thèse de façon rationnelle (la prière n’est pas
une argumentation) mais non probante (ce serait autrement une démonstration).
Essentielle, pour ces deux raisons, à toute philosophie digne de ce nom.

\section{Aristocratie}
%ARISTOCRATIE
C’est le pouvoir des meilleurs {\it (arstoï)}, ou supposés tels.
L'étymologie justifie qu’on la distingue de l’oligarchie, qui
est le pouvoir de quelques-uns, quelle que soit leur valeur propre. Comme on
ne sait jamais quels sont les meilleurs, et comme il est peu probable qu’ils gouvernent,
les deux notions tendront pourtant, en pratique, à se rejoindre. Toute
aristocratie prétendue est une oligarchie réelle.

\section{Art}
%ART
L'ensemble des procédés et des œuvres qui portent la marque d’une
personnalité, d’un savoir-faire et d’un talent particuliers. Cette triple
exigence distingue l’art de l'artisanat (qui a moins besoin de personnalité et de
talent) et de la technique (qui peut s’en passer tout à fait).

Le mot, aujourd’hui, se dit surtout des beaux-arts : ceux qui ont la beauté,
l'expression ou l'émotion pour but. Mais rien de tout cela n’est pleinement
artistique sans une certaine vérité, fût-elle subjective (et parce qu’elle l’est),
sans une certaine {\it poésie}, au sens de René Char (« poésie et vérité, nous le
savons, sont synonymes »), disons sans un certain effet de connaissance ou de
re-connaissance. Shakespeare, Chardin ou Beethoven nous en ont plus appris,
sur l’homme et sur le monde, que la plupart de nos savants. Au reste les découvertes
de ces derniers, s’ils étaient morts à la naissance, eussent été faites,
quelques années ou décennies plus tard, par tel ou tel de leurs collègues. Mais
qui aurait remplacé Rembrandt ou Bach ? Qui écrira les œuvres que Schubert
% 66
n'a pas eu le temps d'écrire? Une œuvre d’art est irremplaçable, comme
l'individu qui l’a créée, et c’est à quoi elle se reconnaît. IL s’agit d’exprimer
« l’irremplaçable de nos vies », comme dit Luc Ferry, et d’autant plus peut-être
qu'elles sont plus ordinaires. Que la beauté soit au rendez-vous est le miracle de
Part.

En son sommet, l’art touche à la spiritualité : c’est comme la célébration —
voire la création — de l'esprit par lui-même. Dieu se tait ; l’artiste lui répond.

\section{Articulation (double —)}
%ARTICULATION (DOUBLE —)
Articuler, c’est diviser. Les linguistes appellent
{\it double articulation}, depuis André Martinet,
le fait qu’un discours puisse se diviser deux fois : selon le sens (il se divise
alors en monèmes) et selon le son (il se divise en phonèmes). Par exemple le mot
{\it rembarquons} est composé de quatre monèmes ({\it r-em-barqu-ons} ; si on change
lune de ces unités, le sens n’est plus le même : rembarquons n’a pas le même sens
que redébarquez où remballons) et de sept phonèmes (7-em-b-a-r-qu-ons). Mais
comme les mêmes phonèmes reviennent dans plusieurs monèmes différents, qui
reviennent eux-mêmes dans plusieurs mots différents (lesquels reviendront dans
plusieurs phrases différentes...), la double articulation s'avère un principe d’économie
formidablement efficace : les dizaines de milliers de mots d’une langue
(donc aussi bien l’ensemble indéfini des livres réels ou possibles dans cette langue)
ne sont constitués que de quelques milliers de monèmes, qui ne sont eux-mêmes
composés que de quelques dizaines de phonèmes. Sans cette double articulation,
il faudrait être capable de pousser autant de cris différents que nous pouvons
avoir d’idées différentes : nos cordes vocales ne seraient pas à la hauteur de notre
cerveau. Au lieu qu’une quarantaine de cris minimaux (les phonèmes de chaque
langue), habilement agencés, suffiraient à parler, sans se répéter jamais, jusqu’à la
fin des temps. C’est notre cerveau, cette fois, qui n’est pas à la hauteur.

\section{Ascèse}
%ASCÈSE
C’est un exercice {\it (askèsis)} qui peut être physique, mais dont la visée
est spirituelle. Par exemple Diogène, nu, étreignant en plein hiver
une statue gelée. C’est violenter le corps, pour forger l'âme. L’esprit ? Il est au-dessus
de ces petitesses ; c’est ce qui explique que les vrais maîtres ne soient pas
dupes de l’ascétisme.
ne pratiquer qu'avec modération.

\section{Ascétique (idéal —)}
%ASCÉTIQUE (IDÉAL —)
Chez Nietzsche, c’est l’idéal — fait de ressentiment
et de mauvaise conscience — des forces réactives,
% 67
qui ne savent vivre que {\it contre}. L'idéal ascétique transforme la souffrance en
châtiment, l’existence en culpabilité, la mort en salut, enfin la volonté de puissance
en « volonté d’anéantissement ». Triomphe du nihilisme : c’est vouloir
sauver la vie en la niant. Un tel idéal triomphe dans le christianisme, selon
Nietzsche, mais aussi en tous ceux, pâles athées et autres rachitiques de l'esprit,
« qui croient encore à la vérité » ({\it Généalogie de la morale}, III). Ce serait, avec
l'alcool et la syphilis, l’un des trois « fléaux » qui rongent l’Europe.

\section{Ascétisme}
%ASCÉTISME
L’ascèse érigée en règle de vie ou en doctrine. La règle est
outrancière, la doctrine erronée. Le plaisir nous en apprend
davantage.

\section{Aséité}
%ASÉITÉ
Le fait d’être ou de subsister par soi {\it (a se)} : ainsi la substance ou
Dieu. C’est un autre nom, un peu oublié et moins paradoxal, pour
désigner ce qui est {\it causa sui}, où qui semble l’être.

\section{Assentiment}
%ASSENTIMENT
C’est approuver ce qui paraît juste ou vrai. Spécialement,
dans le stoïcisme, l’assentiment représente ce qu’il y a
d’actif et de volontaire dans le jugement : c’est adhérer — à la fois librement et
nécessairement — à ce qu’une représentation nous propose, ou nous impose.
Par quoi toute vérité, comme dira Alain, est « de volonté » : parce qu’elle n’est
vraie, en nous, que par l'effort volontaire de penser.

Cela ne suffit pourtant pas à la reconnaître. Comment savoir si nous voulons
le vrai (assentiment), ou si nous croyons vrai ce que nous voulons
(illusion) ? S'il y avait un critère de la vérité, on n’aurait plus besoin
d’assentiment ; c’est en quoi l’assentiment ne saurait en être un.

\section{Assertion}
%ASSERTION
Le fait d’asserter, c’est-à-dire d’affirmer ou de nier. Toute affirmation
est donc une assertion, mais toute assertion n’est pas
une affirmation (elle peut être négative).

\section{Assertorique}
%ASSERTORIQUE
L’une des trois modalités du jugement selon Kant : celle
qui correspond à la catégorie de l’existence ou de l’inexistence.
Un jugement {\it assertorique} est un jugement qui affirme ou nie la réalité de
ce qu’il énonce : c’est une proposition de fait. Se distingue par là des jugements
% 68
{\it problématiques}, qui n’énoncent qu’une possibilité, et des jugements {\it apodictiques}
(voir ce mot), qui affirment une nécessité.

\section{Assurance}
%ASSURANCE
C'est à la fois une disposition de l’âme et une technique, l’une
et l’autre face au danger: il s’agit de l’affronter avec
confiance, soit qu’on pense pouvoir en triompher (l’assurance comme disposition
de l’âme : « marcher avec assurance en cette vie », disait Descartes), soit
qu’on se soit donné les moyens, par la mutualisation des risques, d’en réduire
la gravité ou les effets (l’assurance comme technique : « souscrire une assurance »).
C’est faire ce qui dépend de nous, dans les deux cas, pour affronter ce
qui n’en dépend pas ; et agir au présent, pour préparer l'avenir.
Au premier sens, l’assurance a surtout à voir avec le courage et la prudence.
Au second, avec le calcul et la solidarité. La première est une vertu. La seconde,
un marché. Ce serait abuser de celle-là que de vouloir se passer de celle-ci.

\section{Ataraxie}
%ATARAXIE
L’absence de trouble : la paix de l’âme. C’est le nom grec (spécialement
chez Épicure et les stoïciens) de la sérénité.

Est-ce un état purement négatif, comme on le croit presque toujours ? Non
pas. Car dans cette absence de trouble, ce qui se donne c’est la présence du
corps, de la vie, de tout, et c’est la seule positivité qui vaille. Le 4 privatif ne doit
pas tromper : l’ataraxie n’est pas privation mais plénitude. C’est le plaisir en
repos de l’âme (Épicure) ou le bonheur en acte (Épictète).

C’est aussi une expérience d’éternité : « Car il ne ressemble en rien à un être
mortel, écrit Épicure, l'homme qui vit dans des biens immortels » ({\it Lettre à
Ménécée}, 135). Par quoi l’ataraxie, en tant qu’expérience spirituelle, est l’équivalent
de la béatitude, chez Spinoza, ou du nirvâna, dans le bouddhisme.

\section{Athéisme}
%ATHÉISME
Le a privatif, ici, dit l’essentiel : être athée, c’est être {\it sans dieu}
{\it (a-théos)}, soit parce qu’on ne croit en aucun, soit parce qu’on
affirme l’inexistence de tous.

Il y a donc deux façons d’être athée: ne pas croire en Dieu (athéisme
négatif), ou croire que Dieu n’existe pas (athéisme positif, voire militant).
Absence d’une croyance, ou croyance en une absence. Absence de Dieu, ou
négation de Dieu.

Le premier de ces deux athéismes est très proche de l’agnosticisme, dont il
ne se distingue guère que par un choix, fût-il négatif, plus affirmé. L’agnostique
ni ne croit ni ne croit pas : il doute, il s’interroge, il hésite, ou bien il refuse de
% 69
choisir. Il coche la case «sans opinion» du grand sondage métaphysique
(« Croyez-vous en Dieu ? »). L’athée, lui, répond clairement {\it non}. Ses raisons ?
Elles varient bien sûr selon les individus, mais convergent, le plus souvent, vers
un refus d’adorer. L’athée ne se fait pas une assez haute idée du monde, de
l'humanité et de soi pour juger vraisemblable qu’un Dieu ait pu les créer. Trop
d’horreurs dans le monde, trop de médiocrité dans l’homme. La matière fait
une cause plus plausible. Le hasard, une excuse plus acceptable. Puis un Dieu
bon et tout-puissant (un Dieu Père !) correspond tellement bien à nos désirs les
plus forts et les plus infantiles, qu’il y a lieu de se demander s’il n’a pas été
inventé pour cela — pour nous rassurer, pour nous consoler, pour nous donner
à croire et à obéir. Dieu, par définition, est ce qu’on peut espérer de mieux.
C’est ce qui le rend suspect. L'amour infini, l'amour tout-puissant, l'amour
plus fort que la mort et que tout... Trop beau pour être vrai.

L’athée, plutôt que de se raconter des histoires, préfère affronter comme il
peut l’angoisse, la détresse, le désespoir, la solitude, la liberté. Non qu'il
renonce à toute sérénité, à toute joie, à toute espérance, à toute loi. Mais il ne
les envisage qu’humaines, et pour cette vie seulement. Cela lui suffit-il ? Point
forcément, ni souvent. Le réel ne suffit qu’à qui s’en contente. C’est ce qu’on
appelle la sagesse, qui est la sainteté des athées.

\section{Atome}
%ATOME
Étymologiquement, c’est une particule indivisible, ou qui n’est divisible
que pour la pensée : un élément insécable {\it (atomos)} de matière.
Tel est le sens du mot chez Démocrite et Épicure. Nos scientifiques savent
aujourd’hui qu’il n’en est rien : ils ont appris à rompre les atomes pour en
libérer l'énergie. Cela ne change rien d’essentiel à l’atomisme, qui n’a que faire
de l’étymologie.

\section{Atomisme}
%ATOMISME
C'est une théorie physique ou métaphysique, selon les cas, qui
explique l’ordre et la complexité (le monde) par les interactions
hasardeuses de particules élémentaires (les atomes, mais aussi bien les quarks, leptons
et autres bosons...). Quand cette théorie se prétend suffisante, l’atomisme
est une forme — peut-être la plus radicale — de matérialisme. C’est expliquer le
plus haut par le plus bas, l'esprit par la matière, l’ordre par le désordre. Le
contraire en cela de la religion, comme les atomes sont le contraire des monades.

\section{Attente}
%ATTENTE
Ce qui nous sépare de l’avenir. C’est donc le présent même, mais
comme vidé de l’intérieur par le manque en lui — ou en nous —
% 70
de ce que nous désirons ou redoutons. « Plus que trois jours », se dit-on. Et cela
fait comme trois jours de néant. Ou bien : « Encore une heure » ; et c’est une
heure de trop. Ce qui nous sépare de l’avenir, dans l'attente, finit ainsi par nous
séparer paradoxalement du présent. Comme si ces trois jours ou cette heure,
qui sont devant nous, venaient creuser le présent où nous sommes, où nous
nous enfonçons, où nous nous engloutissons, où nous nous noyons.. C’est
toujours au présent qu’on attend, mais on n'attend jamais que l'avenir.
L’attente est cette absence de l’avenir dans le présent, mais sentie, mais vécue,
mais subie : c’est la présence, dans l’âme, de cette absence, et le principal obstacle
qui nous sépare de la sagesse, c’est-à-dire du réel, du présent, de tout.
Son remède est l’action. Son contraire, l'attention.

\section{Attention}
%ATTENTION
C'est la présence de l'esprit à la présence d’autre chose (attention
transitive) ou de lui-même (attention réflexive). La
deuxième attitude, moins naturelle, est plus fatigante et peut-être impossible à
garder absolument. L’introspection nous en apprend moins sur nous-mêmes
que l’action ou la contemplation.

« L’attention absolument pure est prière », écrit Simone Weil. C’est qu’elle
est pure présence à la présence, pure disponibilité, pur accueil.

Lorsqu'il vint passer quelques mois en France, Svâmi Prajnânpad eut
l’occasion de rencontrer la Supérieure d’un couvent de religieuses. « N’est-il pas
vrai qu’il faut prier sans cesse ? », lui demande-t-elle. Et Swamiji de répondre :
« Oui, absolument. Mais qu'est-ce que cela veut dire ? Prier, c’est rester présent
à ce qui est. » Attention silencieuse, plutôt que bavarde ou implorante.

L’attention absolument impure, ajouterai-je, est voyeurisme : fascination
pour l’obscène ou lobscur. Ce sont les deux extrêmes de l'attention, son
sommet et son abîme, d’ailleurs l’un et l’autre délectables, et, pour l’âme,
comme deux façons de s’oublier.

Que ces deux extrêmes puissent se rejoindre, c’est ce que je ne crois guère
(malgré Bataille). Mais qu’ils aient une source commune, c’est ce que Freud
suggère et que je ne peux guère m'empêcher de penser.

\section{Attribut}
%ATTRIBUT
Tout ce qui peut être dit d’un sujet ou d’une substance, autrement
dit tout ce qu’on peut lui {\it attribuer} (on préférera en ce
sens « prédicat »), et spécialement une qualité essentielle, c’est-à-dire constitutive
de son être même. Ainsi, chez Spinoza, la pensée et l’étendue sont les deux
attributs que nous connaissons de la substance ou de Dieu, parmi une infinité
d’autres que nous ne connaissons pas. Mais toutes ces distinctions, précise Spinoza,
% 71
« ne sont que de raison » : les attributs ne se distinguent pas plus réellement
entre eux qu'ils ne se distinguent de la substance ({\it Pensées métaphysiques},
I, 3, et II, 5 ; voir aussi Gueroult, {\it Spinoza}, t. 1, \S XIV-XV, p. 47-50). Les attributs
ne sont pas extérieurs à la substance : ils constituent son essence même
({\it }Éthique, 1, déf. 4), qu’ils expriment, les uns et les autres, différemment ({\it Éth.} I,
scolie de la prop. 10), sans cesser pour autant d’être la même substance ni donc
de contenir «les mêmes choses » ({\it Éth.} II, scolie de la prop. 7). La pensée et
l'étendue ne sont pas des prédicats de la substance, ni un point de vue sur elle,
mais son être même. On dit souvent que ces attributs sont parallèles (parce que
les chaînes causales y suivent le même ordre, tout en restant internes à chaque
attribut : un corps n’agit pas sur une idée, ni une idée sur un corps). Mais
toutes ces parallèles en vérité sont confondues et n’en font qu’une, qui est la
nature elle-même : « substance pensante et substance étendue, c’est une seule et
même substance » ({\it Éth.} II, scolie de la prop. 7), comme l’âme et le corps, en
l’homme, « sont une seule et même chose » ({\it Éth.} III, scolie de la prop. 2).
L'union de l’âme et du corps n’est qu’un faux problème, qui résulte de ce qu'on
n’a pas pensé leur identité. L'union des attributs, de même : ils n’ont pas à être
unis, n’ayant jamais été séparés.

\section{Audace}
%AUDACE
C'est un courage extrême face au danger, qui reste proportionné
aux enjeux (c’est moins et mieux que la témérité) tout en excédant
quelque peu la raison ordinaire (c’est plus que la hardiesse). Vertu limite et unilatérale.
Davantage de courage que de prudence; plus d’action que de
réflexion.

On remarquera que l’audace est moralement neutre. Elle peut se mettre au
service du mal comme du bien, de l’égoïsme comme de la générosité. On ne la
confondra pas avec l’héroïsme, qui ne vaut pas seulement face au danger (mais
aussi face à la souffrance, à la mort, à la fatigue...), et qui ne se dit que du courage
désintéressé ou généreux.

\section{Au-delà}
%AU-DELÀ
Plus loin, après, de l’autre côté. Le substantif désigne ce qu’il y a
après la mort (donc au-delà de cette vie), s’il y a quelque chose.
Les Anciens s’en faisaient une vision souvent effrayante ou sinistre (le royaume
des ombres). Les classiques, une vision contrastée (salut ou damnation). Les
Modernes, qui croient de moins en moins à l’enfer et de plus en plus au
confort, une vision presque toujours rassurante ou dérisoire : ils n’hésitent plus
guère qu'entre le paradis et la réincarnation ! Après moi, quoi ? Moi-même, en
plus heureux ou en plus jeune. On a l'au-delà que l’on mérite.

% 72
L’athée préfère penser qu’il n’y a pas d’au-delà : que cette vie et ce monde
sont les seuls, définitivement, qui nous soient donnés. Après moi, quoi ? Les
autres. Et pour moi ? Rien, ou plutôt moins que rien, puisque je ne serai plus
là pour m’en rendre compte ni même pour n’y être pas. Le néant, si l’on veut,
mais qui ne serait néant pour personne, comme un sommeil sans rêve et sans
dormeur. L'idée est une forte incitation à vivre, quand on est heureux. Et une
consolation, quand on ne l’est pas. Le bonheur ni la souffrance ne dureront
toujours.

\section{\it Aufhebung}
%{\it AUFHEBUNG}
Voir « Dépassement ».

\section{Autarcie}
%AUTARCIE
C'est le nom grec de l’indépendance ou de l’autosuffisance {\it (autarkeia)}.
Les Anciens y voyaient une caractéristique du sage. Parce
qu’il se suffit à lui-même ? C’est ce que suggère l’étymologie ({\it arkein}, suffire),
mais qu’on aurait tort d’absolutiser. L’autarcie n’est pas l’autisme : le sage, pour
Aristote comme pour Épicure, préfère la société à l’isolement, et l’amitié à la
solitude. Mais il peut se passer de tout, y compris de lui-même. C’est pourquoi
l’{\it autarkeia} est un si grand bien : son vrai nom est liberté (Épicure, {\it Sentences
vaticanes}, 77 ; voir aussi {\it Lettre à Ménécée}, 130).

\section{Authenticité}
%AUTHENTICITÉ
La vérité sur soi, et de soi à soi. C’est le contraire de la
mauvaise foi. Un synonyme, donc, de la bonne ? Plutôt
son nom moderne et quelque peu prétentieux. Les deux notions ne se recouvrent
pas totalement. Être de bonne foi, c’est aimer la vérité plus que soi. Être
authentique, pour beaucoup de nos contemporains, c’est plutôt aimer la vérité
qu’on est. {\it « Be yourself »}, dit-on outre-Atlantique. La psychologie remplace la
morale ; le développement personnel tient lieu de religion. Je suis lâche,
égoïste, brutal ? Sans doute, mais reconnaissez-moi au moins ce mérite de l’être
authentiquement ! Je suis ce que je suis : est-ce ma faute à moi si je ne peux être
quelqu'un d’autre ? L’authenticité est une vertu confortable ; c’est ce qui fait
douter qu’elle en soit une. C’est une bonne foi narcissique, ou un narcissisme
de bonne foi. Mais la bonne foi n’excuse pas tout.

Chez les philosophes contemporains, spécialement chez Heidegger et les
existentialistes, l’authenticité désigne plutôt le statut d’une conscience qui se
sait solitaire (par opposition à l’inauthenticité du « on »), libre (par opposition
à la mauvaise foi), enfin vouée à l’angoisse et à la mort — au néant. Beaucoup
de bruit pour rien.

% 73
\section{Automate}
%AUTOMATE
Qui se meut soi-même. La première idée est donc celle de spontanéité :
Leibniz considérait chaque organisme vivant comme
«une espèce de machine divine ou d’automate naturel », et l’âme comme un
«automate spirituel ». Mais l’idée de mécanisme finit par l'emporter sur celle
de spontanéité : un automate, même s’il se meut par lui-même, reste prisonnier
de ce qu’il est, de son programme (fût-il partiellement aléatoire), de l’agencement
déterminé et déterminant de ses pièces. Ce n’est pas un sujet, c’est une
machine qui imite la subjectivité. Reste à savoir si les sujets ne sont pas des
machines qui s’ignorent. Le cerveau, automate matériel. Que reste-t-il de l'âme
et de Leibniz ?

\section{Autonomie}
%AUTONOMIE
C'est obéir à la loi qu’on s’est prescrite, comme disait Rousseau,
et c’est en quoi c’est être libre.

Le mot, dans son usage philosophique, doit surtout à Kant. L’autonomie
est pouvoir de soi sur soi (liberté), mais par la médiation d’une loi {\it (nomos)}
que la raison s'impose à elle-même, et nous impose, qui est la loi morale. La
volonté est autonome, explique Kant, quand elle ne se soumet qu'à sa
propre législation (en tant que raison pratique), indépendamment de toute
détermination sensible ou affective, indépendamment du corps, donc, mais
aussi du moi, dans sa particularité contingente, et même de quelque but ou
objet que ce soit. C’est n’obéir qu’à la pure forme d’une loi, autrement dit
qu’à l’universel qu’on porte en soi, qui est soi (c’est pourquoi il s’agit de
liberté) mais en tant que raisonnable et législateur (c’est pourquoi il s’agit
d'autonomie).

Les deux concepts d’autonomie et de liberté, bien sûr solidaires, ne se
recouvrent pourtant pas totalement. Celui qui fait le mal agit certes librement,
mais sans autonomie : il se soumet librement à cette partie de lui qui n’est pas
libre (ses instincts, ses passions, ses faiblesses, ses intérêts, ses peurs). Cela dit,
par différence, ce qu’est l'autonomie. C’est la liberté pour le bien : être autonome,
c’est obéir à la partie de soi qui est libre, « sans égard à aucun des objets
de la faculté de désirer », comme dit Kant, et indépendamment même du
« cher moi », autrement dit de l’individu particulier qu’on est. C’est pourquoi
l’autonomie est le principe de la morale : l’égoïsme est le fondement de tout
mal ; la raison, qui n’a pas d’{\it ego}, de tout bien. Ainsi le seul devoir est d’être
libre, et c’est ce que signifie l’autonomie : c’est obéir au devoir de se gouverner.

Le mot, quand on n’est pas kantien, vaut surtout comme idéal. Il n’indique
pas un fait mais un horizon, un processus, un travail. Il s’agit de se libérer le
plus qu’on peut de tout ce qui, en nous, n’est pas libre. La raison seule le

% 74
permet, comme on voit chez Spinoza (mais aussi chez Marx ou Freud), et c’est
en quoi l’idée d'autonomie garde un sens : « Un homme libre, écrit Spinoza,
c’est-à-dire qui vit suivant le seul commandement de la raison... » Mais cette
autonomie n’est jamais donnée : elle est à faire et, toujours, à refaire. Il n’y a
pas d’autonomie ; il n’y a qu’un processus, toujours inachevé, d’{\it autonomisation}.
On ne naît pas libre ; on le devient.

\section{Autoritarisme}
%AUTORITARISME
Abus d'autorité, le plus souvent fondé sur la croyance
naïve qu’elle pourrait suffire. C’est trop demander à
l’obéissance, toujours nécessaire, jamais suffisante.

\section{Autorité}
%AUTORITÉ
Le pouvoir légitime ou reconnu, ainsi que la vertu qui sert à
l'exercer. C’est le droit de commander et l’art de se faire obéir.

\section{Autorité (argument d’—)}
%AUTORITÉ (ARGUMENT D’—)
C'est prendre une autorité (un pouvoir, une
tradition, un auteur reconnu ou consacré...)
pour un argument. Double faute : contre l’esprit, qui se moque de
l'autorité, et contre l'autorité, qui doit avoir de meilleurs arguments. Si le pape
pouvait nous convaincre, aurait-il besoin du dogme de linfaillibilité
pontificale ? Et à quoi bon ce dogme, s’il ne nous convainc pas ?
L'autorité mérite obéissance, non créance. Tout argument d’autorité est
donc sans valeur. Quand lesprit se met à obéir, que reste-t-il de l'esprit ?

\section{Autre}
%AUTRE
Le contraire du même : ce qui est numériquement ou qualitativement différent.

On distinguera donc {\it l'autre numérique} (si je décide d’acheter une autre voiture,
tout en choisissant la même, c’est-à-dire ici le même modèle de la même
marque), et {\it l'autre qualitatif ou générique} (quand je choisis une voiture différente,
par son modèle ou sa marque, de la précédente). Deux jumeaux ou deux
clones parfaitement semblables n’en seraient pas moins numériquement
différents : chacun resterait l’autre de l’autre (ils seraient deux, non pas un),
quand bien même cet autre, par hypothèse ou par impossible, lui serait parfaitement
identique.

On remarquera que l’autre humain hésite entre ces deux altérités : c’est à
la fois un autre homme (altérité numérique), et un homme autre (altérité
qualitative). C’est ce qu’on appelle autrui : un autre individu de la même
humanité, mais réputé différent de tous les autres. D’où le droit à la différence,
qui ne saurait pourtant occulter celui, plus essentiel encore, à l’identité
générique. Un être humain différent — et ils le sont tous — est d’abord un être
humain.

\section{Autrui}
%AUTRUI
C'est l’autre en personne : non un autre moi-même (un alter ego),
mais un moi autre (un moi qui n’est pas moi). N’importe qui,
donc, en tant qu’il est quelqu'un.
C’est le prochain possible et encore indéterminé : objet non d’amour, cela
ne se peut (comment aimer n’importe qui ?), mais, par anticipation, de respect.
N'importe qui, ce ne saurait être n’importe quoi.

\section{Avarice}
%AVARICE
C'est l’amour exagéré de l’argent, spécialement (par différence
avec la cupidité) de celui qu’on possède déjà. L’avare a peur de
perdre ou de manquer, et la passion, à l'inverse, de conserver, d’accumuler, de
retenir. Personnalité anale, dirait un psychanalyste, quand la cupidité relèverait
plutôt de l’oralité. C’est ce qui rend peut-être l’avarice plus antipathique. Il faut
dire aussi qu’elle est plus absurde ; à quoi bon s'enrichir, si lon ne sait pas
dépenser ?

\section{Avenir}
%AVENIR
Le mot, dans son étymologie transparente, tient presque lieu de
définition : l'avenir, c’est ce qui est à venir. Cette tautologie est
plus problématique qu’il n’y paraît. Elle suggère en effet deux questions. Si
c’est {\it à venir}, d'où cela viendra-t-il ? Si {\it c'est} à venir, en quoi n’est-ce pas déjà
du présent (puisque cela est). Les deux questions se rejoignent en une même
aporie : comment l’avenir pourrait-il exister, et où, puisque, s’il existait, il
serait du présent ? Saint Augustin indépassable ici : l'avenir comme le passé
ne peut exister « qu’en tant que présent ». La topologie de l’avenir (où est-il ?)
commande son ontologie (qu’est-il ?). Mais l’une et l’autre soumises à
son concept, qui est paradoxal : où qu’il soit, et quoi qu’il soit, l’avenir ne
peut s’y trouver (comme présent) qu’en tant qu’il n’est pas encore. Cela
commande à peu près la réponse à nos deux questions. L'avenir ne peut être
présent (comme à-venir) que dans l’âme ou la conscience, qui ont seules
cette capacité de se représenter ce qui n’est pas. C’est ce qu’on peut appeler
l’anticipation, la protention, la futurition, ou, plus simplement, l’attente.
Cela suppose l'imagination. Cela suppose aussi, et sans doute d’abord, la
mémoire. Quel avenir imaginable, pour le nouveau-né, sinon la répétition
% 76
ou l’absence de ce qui fut ? Sa mère était là, puis plus ; puis là à nouveau,
puis plus, puis là, puis plus... Comment ne pas attendre (désirer, espérer,
prévoir) qu’elle revienne ? Comment ne pas craindre qu’elle ne revienne
pas ? Elle est donc présente, si l’on peut dire, en tant qu’elle ne l’est plus
(mémoire) ou pas encore (anticipation). L’avenir n’a d’existence que pour
l'esprit, non en soi : ce n’est pas parce qu’il existe (dans le monde) qu’on
l'attend ; c’est au contraire parce qu’on l’attend qu’il existe (dans la conscience).
Ainsi l’avenir n’est pas un être ; c’est le corrélat imaginaire d’une
conscience en attente.

L'avenir n'existe pas : seule existe la conscience présente que nous pouvons
avoir, ici et maintenant, de son absence actuelle et de sa venue attendue. C’est
une vue de l’esprit, si l’on veut, mais sur lui-même en train d’attendre. Tout
avenir est donc subjectif : rien ne nous attend ; c’est nous qui nous attendons à
quelque chose, au point souvent (voyez {\it La bête dans la jungle}, de Henry James)
d’être incapables de le vivre quand il advient. « Ainsi nous ne vivons jamais,
disait Pascal, nous espérons de vivre. » C’est ce qui nous voue au temps, au
manque, à l’impatience, à l’angoisse. Nous ne sommes séparés du présent ou de
l'éternité que par nous-mêmes.

\section{Aventure}
%AVENTURE
Une histoire intéressante et risquée — et intéressante, le plus
souvent, {\it parce que} risquée. Étymologiquement, c’est ce qui doit
arriver. Mais il n’y a aventure que parce qu’on l’ignore. C’est ce qui est en train
d’advenir, mais dont on ne connaît pas encore la fin, qui reste imprévisible et
hasardeuse. Cela fait comme un destin inachevé ou en suspens.

Entre un homme et une femme, c’est une liaison sans lendemain. Mais on
ne le sait qu'après coup. Qui peut être certain, lorsqu’elle commence, qu’elle ne
va pas bouleverser notre vie entière ? C’est la différence qu’il y a entre une aventure,
toujours risquée, et une bonne fortune, sans risque intrinsèque ou sentimental
(sans autre risque, à la limite, et peut-être sans autre fonction, que sanitaire).
Il n’y a aventure que là où il y a danger. C’est pourquoi la vie en est une,
et la seule.

\section{Aversion}
%AVERSION
Moins le contraire du désir, malgré Lalande, qu’un désir négatif :
c’est désirer s'éloigner de quelque chose, ou l’éloigner de
soi. (Le contraire du désir, ce serait plutôt l'indifférence. Ne pas désirer un mets
ou un individu, cela ne signifie pas qu’on l’ait en aversion ; l'absence d’un désir
n’est pas le désir d’une absence.)

% 77
\section{Aveu}
%AVEU
C'est dire sur soi-même une vérité coupable, ou plutôt (car la vérité,
en elle-même, est toujours innocente) c’est dire la vérité sur sa propre
culpabilité, ou sur ce qu’on juge être tel.

L’aveu suppose la conscience d’une faute au moins possible. Les Résistants
qui parlèrent sous la torture n’ont pas avoué : ils ont trahi ou craqué, comme
nous aurions presque tous fait, et c’est ce que certains, plus tard, avoueront. En
revanche, il n’est pas exclu que certains communistes aient avoué, lors des
procès de Moscou, les crimes qu’on leur imputait : c’est qu’on avait réussi à les
convaincre que la dissidence en était un.

On ne confondra pas l’aveu et la confidence : d’abord parce qu’un aveu
peut être public, ensuite parce qu’une confidence peut être innocente. Avouer,
au contraire, c’est se reconnaître coupable : c’est faire preuve de mauvaise conscience
et de bonne foi. Cela vaut mieux que la tartuferie, qui fait preuve de
bonne conscience et de mauvaise foi. Mais moins, parfois, que le silence.

\section{Avoir}
%AVOIR
Le mot est vague, et d’autant plus qu’il sert davantage : il peut désigner
une possession (« j’ai une voiture »), un affect (« j’ai de l'amour
pour lui »), une représentation (« j'ai une idée »), une sensation (« j'ai faim »),
une propriété (« le triangle a trois côtés »), bref toute relation, mais interne ou
intériorisée, entre un individu et ce qui n’est pas lui, ou qui n’en est qu’une
artie. Si « j'ai un corps », par exemple, c’est que je ne suis pas {\it que} mon corps.
C’est en quoi lavoir s’oppose à l’être, et le suppose.

\section{Axiologie}
%AXIOLOGIE
L'étude ou la théorie des valeurs. Elle peut se vouloir objective
(si elle considère les valeurs comme des faits) ou normative (si
elle y souscrit). Celle-ci relève de celle-là ; celle-là ne vaut que pour celle-ci.

\section{Axiomatique}
%AXIOMATIQUE
Ensemble d’axiomes, et parfois, par extension, des propositions
qu’on en peut déduire, sans qu’intervienne pour ce
faire aucun élément empirique. Une axiomatique est un système formel hypothético-déductif.
Les mathématiques, par exemple, sont une axiomatique, ou
plutôt plusieurs, et c’est ce qui justifie qu’on en parle au pluriel. La logique ? Si
elle n’était qu’une axiomatique, elle ne pourrait prétendre être vraie. Et que resterait-il,
alors, de nos vérités ?
Qu’une axiomatique ne vaille qu’à proportion de sa rationalité, c’est ce qui
interdit de considérer la raison elle-même comme une axiomatique, et de
prendre quelque axiomatique que ce soit pour la raison.

% 78
\section{Axiome}
%AXIOME
Proposition indémontrable, qui sert à en démontrer d’autres. Les
axiomes sont-ils vrais ? C’est ce qu’on crut pendant longtemps :
un axiome, pour Spinoza ou Kant, est une vérité évidente par elle-même, qui
n'a donc pas besoin d’être démontrée. Les mathématiciens ou logiciens
d’aujourd’hui y voient plutôt de pures conventions ou hypothèses, qui ne peuvent
l'être. La vérité, dès lors, n’est plus dans les propositions (si les axiomes ne
sont pas vrais, aucun théorème ne l’est), mais dans les rapports d’implication
ou de déduction qui les unissent. C’est dire qu’il n’y a pas d’axiomes, au sens
traditionnel du terme, mais seulement des postulats (voir ce mot). Cela toutefois
est un postulat, non un axiome.

\section{Baptême}
%BAPTÊME
Le mot grec, note Voltaire, signifie immersion : « Les hommes,
qui se conduisent toujours par les sens, imaginèrent aisément que
ce qui lavait le corps lavait aussi l'âme. » De là le baptême, qui consiste à «se
mettre dans le bain du sacré ». C’est bien plus qu’un symbole : c’est un rite, et
surtout, pour les croyants, c’est un sacrement, qui nous fait entrer dans l’Église.
Qu'on l’impose à des nouveau-nés m’a longtemps choqué : pourquoi leur
imposer une appartenance qu’ils n’ont pas demandée, qu’ils ne peuvent refuser
ni comprendre ? Je me dis à présent que ce n’est pas si grave, ni si singulier : ils
n’ont pas demandé non plus de vivre, ni d’être français, ni de s'appeler Dupont
ou Martin. Faut-il pour cela les considérer comme apatrides, anonymes et à
naître jusqu’à leur majorité ? Nul ne choisit ce qu’il est, ni son pays, ni son
nom, ni sa foi. On ne choisit, et encore, que d’en changer ou pas. À la gloire
des convertis, des hérétiques et des apostats.

\section{Barbare}
%BARBARE
Chez les Grecs, c'était l'étranger, en tant qu’il suscitait le mépris
ou la peur. L’ethnocentrisme ne date pas d’hier. Par dépassement
puis critique de ce sens premier, on appelle {\it barbare} toute personne qui semble
violer non seulement telle ou telle civilisation particulière, mais la civilisation
elle-même ou l’idée que l’on s’en fait — c’est-à-dire, de plus en plus, les droits
de l’homme. On parlera par exemple de la barbarie nazie. Cela ne prouve rien
contre la civilisation allemande. Quoique.

\section{Barbarie}
%BARBARIE
Comportement de barbare, et tout ce qui l’évoque. Le mot, qui
vaut presque toujours comme condamnation (je ne connais
% 80
guère que Nietzsche, parmi les philosophes, qui l'utilise parfois positivement),
n’a bien sûr de sens que relatif : il suppose une civilisation de référence, dont la
barbarie serait l’absence ou le saccage. C’est presque toujours donner raison à
son propre camp : le barbare, c’est d’abord l’autre, l'étranger, celui qui n’a pas
la même civilisation que nous, dont on croit pour cela qu’il n’en a aucune, ou
que celle qu’il a, si l’on consent à lui en reconnaître une, ne vaut rien. « Chacun
appelle barbarie, disait Montaigne, ce qui n’est pas de son usage. » C’est pourquoi
il faut se méfier du mot, toujours suspect d’ethnocentrisme ou de bonne
conscience, Mais cela ne dispense pas de se méfier aussi, et bien davantage, de
la chose. Aucune civilisation n’est immortelle. Pourquoi la civilisation elle-même
le serait-elle ?

Il m'arrive aussi de prendre {\it barbarie} en un sens plus singulier, pour désigner
l'inverse ou le symétrique de l’angélisme. C’est alors une confusion des ordres,
comme l’angélisme, mais au bénéfice cette fois d’un ordre inférieur : c’est vouloir
soumettre le plus haut au plus bas. Tyrannie, dirait Pascal, des ordres inférieurs.
Par exemple : vouloir soumettre la politique ou le droit aux sciences, aux techniques,
à l’économie (barbarie technocratique : tyrannie des experts ; ou bien, il y
a deux écoles, barbarie libérale : tyrannie du marché). Ou encore : vouloir soumettre
la morale à la politique ou au droit (barbarie politique ou juridique : barbarie
totalitaire, chez un Lénine ou un Trotski, ou bien barbarie démocratique,
qui menace davantage chez nous ; tyrannie des militants, du suffrage universel ou
des juges). Ou enfin: vouloir soumettre l'amour à la morale (barbarie
moralisatrice : tyrannie de l’ordre moral). On peut aussi sauter des ordres : par
exemple vouloir soumettre la morale aux sciences (scientisme, darwinisme moral,
etc.) ou bien l'amour à l'argent (prostitution, mariages arrangés ou intéressés.)
ou au pouvoir (culte de la personnalité, fanatisme, eugénisme....). Tout cela
s'explique, et par les ordres inférieurs plutôt que supérieurs, mais est aussi à
combattre : c’est la seule façon de sauver la politique (contre la tyrannie des
experts ou du marché), la morale (contre la tyrannie des partis, des assemblées ou
des tribunaux) et l’amour (contre la tyrannie des moralistes, de l'opinion
publique ou de l'argent). On n’en a jamais fini. Les groupes, presque inévitablement,
tendent vers le bas (même dans une Église, les rapports de pouvoir comptent
davantage que la morale ou l'amour) : tout groupe, si on le laisse faire, tend
à la barbarie, autrement dit à la dictature, quelles qu’en soient les formes, de ce
que Platon appelait le {\it gros animal}. Il n’y a que les individus, quand ils ne tombent
pas dans l’angélisme, qui aient la force, parfois, de résister.

\section{Baroque}
%BAROQUE
L’art maximum, ou qui se voudrait tel : esthétique de l’excès et
de l’étonnement. L’art y va jusqu’au bout de sa nature, qui est
% 81
d'ornement, et s’enivre de sa propre richesse. Car tout art est excessif (le sens
est toujours de trop), et étonnant déjà d’exister : l’art est le baroque du monde.
Le baroque est ainsi la règle, pour l’art, dont le classicisme serait l'exception.

Le mot, qui relève de l’histoire de l’art, désigne à la fois une période (en gros :
de la fin du {\footnotesize XVI$^\text{e}$} siècle au début du {\footnotesize XVIII$^\text{e}$}) et un style, fait de complexité, d’audace,
de surabondance, qui privilégie presque toujours les courbes, le mouvement, les
formes déséquilibrées ou pathétiques, avec un faible pour le spectaculaire et
l'étrange, voire pour le trompe-l’œil ou Partifice. On y voit souvent le contraire
du classicisme, comme l’hyperbole est le contraire de la litote. Disons que c’est
son autre, qui peut le suivre (en Italie) ou le précéder (en France), mais sans
lequel le classicisme n’aurait pas cette rigueur ou cet équilibre qui servent — par
contraste et rétrospectivement — à le définir. Le classicisme, écrit admirablement
Francis Ponge, est « la corde la plus tendue du baroque ».

Cela dit aussi, par différence, ce qu’est le baroque : c’est un classicisme qui
se détend ou qui se prépare, qui s’amuse à se faire valoir, qui renonce à la perfection
pour le plaisir d’impressionner ou de surprendre, enfin qui se cherche
ou se laisse aller. Le classicisme ne passe pour la règle, répétons-le, que parce
qu'il est d’abord une exception, qui ne surprend que par ses réussites. Le
baroque, c’est sa limite, a besoin de bizarrerie, d’excès ou de virtuosité pour
n'être pas banal.

\section{Bassesse}
%BASSESSE
Au sens le plus général, c’est bien sûr le contraire de l'élévation :
c’est suivre sa pente, qui est la pente commune, mais en la descendant.

En un sens plus technique, on peut y voir l’équivalent de la {\it micropsuchia}
chez Aristote, et de l’{\it abjectio} chez Spinoza : c’est « faire de soi, par tristesse,
moins de cas qu’il n’est juste » (Spinoza, {\it Éthique}, III, déf. 29 des affects ; voir
aussi Aristote, {\it Éthique à Nicomaque}, IV, 9). L'homme bas manque à la fois de
fierté et de dignité : il se croit incapable de toute action un peu haute ou désintéressée,
et l’est en effet, par cette croyance. Son erreur est de croire, quand il
s’agit de vouloir.

À ne pas confondre avec l'humilité. On peut avoir conscience de sa petitesse
(humilité) sans l’exagérer et sans s’y enfermer (bassesse). La bassesse
décourage l’action : c’est le contraire de la grandeur d’âme. L’humilité la
désillusionne : c’est le contraire de l’orgueil et de la bonne conscience.

\section{Bavardage}
%BAVARDAGE
La parole dévaluée par l’excès ou la superficialité. C’est avoir
peur du silence ou du vrai.

% 82
\section{Béatitude}
%BÉATITUDE
« La béatitude, écrit saint Augustin, c’est la joie dans la vérité. »

Cette définition parfaite suggère un bonheur plus vaste et
plus lucide que nos bonheurs ordinaires, toujours pétris de petitesses et d’illusions —
toujours pétris de nous-mêmes. La vérité n’a pas d’ego ; comment
serait-elle égoïste ? Tout mensonge la suppose ; comment serait-elle mensongère ?

La béatitude est un vrai bonheur, ou un bonheur vrai, qui serait pour cela
éternel (la vérité l’est toujours), complet (la vérité ne manque de rien), enfin
comme un autre nom du salut. On évitera d’y trop rêver. La béatitude est éternelle,
explique Spinoza ; comment pourrait-elle {\it commencer} ? Il est donc vain de
l’espérer : nul ne l’atteint qu’en cessant de l’attendre.

«Si la joie consiste dans le passage à une perfection plus grande, écrit
encore Spinoza, la béatitude doit consister en ce que l’âme est douée de la perfection
même » ({\it Éth.} V, 33, scolie). Mais la perfection n’est autre que la réalité
({\it Éth.} II, déf. 6) : la béatitude est l’état normal de l’âme, dont nous ne sommes
séparés que par nos illusions et nos mensonges. Elle n’est pas la récompense de
la vertu, mais la vertu elle-même ({\it Éth.} V, prop. 42).

Plus simplement, c’est le bonheur du sage, ou la sagesse elle-même comme
bonheur. Son contenu est de joie, donc d’amour. Son objet est la vérité, donc
tout. C’est l’amour vrai du vrai.

Que nous soyons incapables de l’habiter n'empêche pas de découvrir, parfois,
qu’elle nous habite.

\section{Beau}
%BEAU
Tout ce qui est agréable à voir, à entendre ou à comprendre, non à
cause de quelque autre chose qu’on désire ou qu’on attend (comme la
vue d’une fontaine plaît à l’homme assoiffé), mais en soi-même, et indépendamment
de quelque utilité ou intérêt que ce soit. Le beau se reconnaît d’abord
au plaisir qu’il suscite (être beau, c’est plaire), mais se distingue de la plupart
des autres plaisirs par le fait qu’il ne suppose ni convoitise ni possession : il est
l’objet d’une jouissance contemplative et désintéressée. C’est pourquoi peut-être
on ne parle de beauté que pour la pensée (une belle théorie, une belle
démonstration) ou, s’agissant des sens, que pour la vue et l’ouïe — comme si le
toucher, le goût ou l’odorat, trop corporels, trop grossiers, étaient incapables de
jouir sans posséder ou consommer. Cela, toutefois, relève plus des contraintes
du langage que de la nécessité du concept. Un aveugle peut trouver belle la
statue qu’il palpe, et rien n’interdit, philosophiquement, de parler d’un beau
parfum ou d’une belle saveur. Le langage ne pense pas ; c’est ce qui rend la
pensée possible et nécessaire.

% 83
« Est beau, écrit Kant, ce qui plaît universellement et sans concept. » Mais
l'universalité n’est jamais donnée en fait, et il peut arriver que la beauté, pour
tel ou tel, passe par la médiation d’une pensée. Nul n’est tenu de trouver beau
ce qui plaît à ses voisins, ni laid ce qui leur déplaît, ni d'admirer ce qu’il ne
comprend pas. La jouissance esthétique est tout aussi solitaire, en fait, qu’elle
semble universelle, en droit. Nul ne peut aimer, ni admirer, ni comprendre ou
jouir à ma place. C’est qu'aucune vérité ici ne règne. « Les choses considérées
en elles-mêmes ou dans leur rapport à Dieu ne sont ni belles ni laides », écrit
Spinoza (lettre 54, à Hugo Boxel), et c’est ce que Kant, à sa façon, confirmera
(« sans relation au sentiment du sujet, la beauté n’est rien en soi», {\it C. F. J.}, I,
\S 9). Il n’y a pas de beauté objective ou absolue. Il n’y a que le plaisir de percevoir
et la joie d'admirer.

\section{Beauté}
%BEAUTÉ
La qualité de ce qui est beau, ou le fait de l’être. Faut-il distinguer
les deux notions ? C’est ce que suggère Étienne Souriau, dans son
{\it Vocabulaire d'esthétique} : « Quand on parle du {\it beau}, on est conduit à chercher
une essence, une définition, un critère. Tandis que la {\it beauté}, étant une qualité
sensible, peut être l’objet d’une expérience directe et même unanime. » Cette
distinction, sans s’être absolument imposée, correspond à peu près à l'usage. Le
beau est un concept ; la beauté, une chance.

\section{Béhaviorisme}
%BÉHAVIORISME
De l’anglo-américain {\it behavior}, conduite, comportement.
C’est un autre nom pour la psychologie du comporte-
ment (voir ce mot).

\section{Bénédiction}
%BÉNÉDICTION
Une parole qui dit le bien, et qui le fait être au moins par
À. L'erreur serait d’y croire tout à fait, autrement dit d’en
attendre un résultat autre que performatif, par exemple une aide ou une protection.
Dire le bien ne dispense pas de le faire.

\section{Besoin}
%BESOIN
Nécessité vitale ou vécue. Un manque ? Pas forcément. Toute plante
a besoin d’eau, qu’elle en ait ou pas suffisamment. Le petit enfant a
besoin de ses parents. Cela ne prouve pas qu’il soit orphelin. Enfin rien ne vous
empêche de chanter {\it « I need you »} à l'homme ou à la femme qui partage votre
vie, qui ne vous manque pas mais qui est nécessaire à votre bonheur.

% 84 —
On a besoin non seulement de ce qui manque, mais de ce qui manquerait
s’il n’était pas là. Le besoin n’est pas toujours une absence ; c’est une condition,
qui peut être satisfaite ou non : il porte sur tout ce qui m'est nécessaire, autrement
dit sur tout ce sans quoi je ne puis vivre ou vivre bien. C’est la condition
{\it sine qua non} de la vie ou du bonheur.

On distingue ordinairement le besoin, qui serait objectif, du {\it désir}, qui serait
subjectif. Mais la limite entre les deux reste floue. Le sujet, pour illusoire qu’il
soit toujours, n’en existe pas moins objectivement. C’est ce qui explique que
nous ayons besoin d’amour autant, ou presque autant, que de nourriture.

\section{Bestialité}
%BESTIALITÉ
Manque d'humanité, au sens normatif du terme, autrement
dit d'éducation : c’est vivre ou agir comme une bête. Il en
découle que la bestialité est le propre de l’homme, quand il oublie ce qu’il est
ou doit être.

\section{Bête}
%BÊTE
Tout animal qui ne fait pas partie du genre humain.

Comme adjectif, et en un sens figuré, le mot désigne aussi un homme
qui manque d'intelligence. C’est que l'intelligence, croit-on, est le propre de
l'espèce humaine. Disons que c’est son lieu ordinaire d’excellence. Sauf pathologie
ou accident, l’homme le plus bête l’est pourtant moins que le chimpanzé
le plus intelligent. La pensée touche au corps, par le cerveau, et donc à l’espèce,
par les gènes. Que cela ne tienne pas lieu d'éducation, c’est bien clair. Mais
quelle éducation sans cerveau ? L’acquis suppose l’inné ; la culture fait partie de
la nature.

Les bêtes ont-elles des droits ? Pas les unes pour les autres (le lion ne viole
pas les droits de la gazelle qu’il dévore, ni le moineau ceux du ver de terre).
Mais nous avons des devoirs envers elles : le devoir de ne pas les faire souffrir
inutilement, celui de ne pas les exterminer, de ne pas les humilier, de ne pas les
martyriser... La souffrance commande, c’est ce que signifie la compassion. Les
bêtes en sont-elles capables ? Guère, semble-t-il. Cela ne nous dispense pas
d’être humains avec elles, qui ne le sont pas.

\section{Bêtise}
%BÊTISE
Manque d’intelligence : c’est penser comme une bête, ou comme on
suppose, plutôt, que les bêtes pensent, disons mal ou trop peu.
Notion par nature relative. La bêtise, comme l'intelligence, ne se dit que par
comparaison et degrés. Cela laisse une chance aux imbéciles (il y a toujours plus
bête qu’eux) et aux génies (il y a toujours de la bêtise en eux à surmonter). « J'ai
% 85
été bête », dit-on souvent. C’est supposer non qu’on ne l’est plus, mais qu’on
l'est moins.

\section{Bien}
%BIEN
Tout ce qui est bon absolument. Si toute valeur est relative, comme je
le crois, le bien n’est donc qu’une illusion : c’est ce qui reste d’un juge-
ment de valeur positif, quand on méconnaît les conditions subjectives qui le
rendent possible. On dira par exemple que la santé, la richesse ou la vertu sont
des biens, ce qui laisse entendre qu’ils valent par eux-mêmes, quand ils ne
valent en vérité que pour autant que nous les désirons. Qu'importe la santé au
suicidaire, la richesse au saint, la vertu au salaud ? « Il n’y a pas de Bien ni de
Mal, disait Deleuze à propos de Spinoza, il n’y a que du {\it bon} et du {\it mauvais}
(pour nous). » Le bien est le bon, quand on croit qu’il l’est en soi.

Le mot, en français, est pourtant difficile à éviter. On dit « faire le bien »,
non « faire le bon ». La langue reflète nos illusions en même temps qu’elle les
renforce. En l’occurrence cette expression dit quelque chose d’important : le
bien n'existe pas ; il est à faire. Ce n’est pas un être ; c’est un but. Non une
Idée, malgré Platon, mais un idéal. Non un absolu, malgré Kant, mais une
croyance. C’est le corrélat hypostasié de nos désirs.

« Le bien, écrivait Aristote, est ce à quoi toutes choses tendent » ({\it Éthique à
Nicomaque}, I, 1). C'était penser la nature sur le modèle humain, et l’homme
selon le finalisme. Un matérialiste dirait plutôt : « Les hommes appellent bien
ce vers quoi ils tendent. » C’est l’esprit de Hobbes ({\it Léviathan}, chap. VI). C’est
l'esprit de Spinoza : « Par {\it bien}, j'entends tout genre de joie, ainsi que tout ce
qui y mène, et principalement ce qui satisfait un désir, quel qu’il soit ; par {\it mal},
tout genre de tristesse, et principalement ce qui frustre un désir » ({\it Éthique}, III,
39, scolie ; voir aussi III, 9, scolie, et IV, Préface). C’est pourquoi les biens sont
multiples — parce que les hommes ne tendent pas tous vers les mêmes choses,
ni vers une seule. Voyez Diogène et Alexandre. La convergence des désirs est
pourtant la règle la plus ordinaire, qui nous voue au conflit (quand nous désirons
les mêmes choses, que nous ne pouvons tous posséder) ou à l’émulation.
Que le pouvoir ne soit pas un bien, pour le sage, n'empêche pas que la sagesse
en soit un, pour l’ambitieux. « Si je n’avais été Alexandre, disait l'élève d’Aristote,
j'aurais voulu être Diogène. »

\section{Bien (souverain —)}
%BIEN (SOUVERAIN —)
« Dans toute action, dans tout choix, écrit Aristote,
le bien est la fin, car c’est en vue de cette fin qu’on
accomplit tout le reste » ({\it Éthique à Nicomaque}, I, 5). Mais la plupart des fins
que nous visons ne valent pas par elles-mêmes : ce ne sont en vérité que des
% 86 
moyens, pour d’autres fins. Par exemple le travail n’est une fin (donc un bien)
que par l’argent qu’il permet de gagner, comme l'argent n’est un bien que par
le confort ou le luxe qu’il autorise. Mais si toute fin n’était ainsi qu’un moyen
pour une autre fin, qui ne serait à son tour qu’un moyen pour une autre, et
ainsi à l'infini, notre désir serait par définition insatiable. On dira que c’est en
effet le cas, et c’est ce que les Grecs pourtant n’ont pas voulu accepter. Aristote
encore : « Si donc il y a, de nos activités, quelque fin que nous souhaitions par
elle-même, et les autres seulement à cause d’elle, et si nous ne choisissons pas
indéfiniment une chose en vue d’une autre (car alors on procéderait ainsi à
l'infini, de sorte que le désir serait futile et vain), il est clair que cette fin-là ne
saurait être que le bien, le souverain bien » ({\it ibid.} I, 1). Cela vaut comme définition
au moins formelle : le souverain bien serait la {\it fin finale}, comme dit Aristote,
autrement dit la fin qui n’est le moyen d’aucune fin, et dont toutes les
autres fins ne sont que les moyens. Il serait pour cela le but ultime de tous nos
actes.

Mais qu’est-il ? Qu'est-ce qui est bon par-dessus tout, que nous désirons
pour lui-même, et en vue de quoi nous désirons tout le reste ? Aristote répond :
« le bonheur, car nous le choisissons toujours pour lui-même et jamais en vue
d’une autre chose » ({\it ibid.} I, 5 ; voir aussi X, 6). Épicure répondrait plutôt « le
plaisir » : car le bonheur ne vaut que pour autant qu’il est agréable, quand le
plaisir, sans le bonheur, continue de valoir. Les stoïciens répondraient plutôt
« la vertu », puisqu’elle seule nous rend heureux et vaut mieux qu’un bonheur,
d’ailleurs impossible, qui prétendrait s’en passer.

Entre ces trois éthiques, centrées sur trois souverains biens, on évitera de
trop forcer l'écart. L’eudémonisme est le lot commun des sagesses grecques :
que le bonheur puisse n’être pas agréable ou vertueux, ce n’était pour Épicure
ou Zénon qu’une pure hypothèse d’école, qu’ils n’ont jamais envisagée sérieusement.
Le bonheur pour tous est le but : c’est l’activité conforme à la vertu
(Aristote), le plaisir en repos de l’âme (Épicure) ou la vertu en acte (Zénon).

C’est ce qui donne raison à Kant, quelque deux mille ans plus tard, quand
il leur donne tort. Le mot {\it souverain} est équivoque, remarque-t-il dans la
« Dialectique de la raison pratique » : il peut signifier {\it suprême} ou {\it parfait}. Or,
quand bien même la vertu serait, comme le veut Kant, « la condition suprême
de tout ce qui peut nous paraître désirable », elle ne sera « le bien complet et
parfait » qu’à la condition d’être accompagnée du bonheur. De fait, si le souverain
bien est le désirable absolu, chacun accordera qu’il ne peut aller ni sans
bonheur ni sans vertu : ce que nous désirons, c’est la conjonction ou, comme
dit Kant, « l’exacte proportion » des deux. C’est bien ce que cherchaient les épicuriens
(pour qui le bonheur est la vertu) comme les stoïciens (pour qui la vertu
est le bonheur). Ils se trompaient les uns et les autres. L’union du bonheur et

% 87
de la vertu est synthétique, explique Kant, non analytique : ce sont deux concepts
« tout à fait distincts », dont la conjonction, sur cette terre, n’est jamais
garantie et presque jamais donnée ({\it C. R. Pratique}, I, II, chap. 2). Il faut donc
renoncer au souverain bien, ou croire en Dieu. C’est où la modernité commence,
en nous séparant du bonheur.

\section{Bien (tout est —)}
%BIEN (TOUT EST —)
C'est l’expression qui résume, pour Voltaire, l’optimisme
leibnizien, dont il s’est si efficacement moqué
dans {\it Candide}. La formule semble absurde ou choquante, tant elle est démentie
par l'expérience. Elle est pourtant irréfutable. Si Dieu existe, s’il est à la fois
tout-puissant et parfaitement bon, comment tout n'irait-il pas pour le mieux
dans le meilleur des mondes possibles ? Le principe du meilleur explique tout,
mais c’est pourquoi aussi il n’explique rien (puisqu'il expliquerait aussi bien un
monde complètement différent du nôtre : par exemple un monde où le cancer
n’existerait pas, ou bien un autre dans lequel Hitler aurait gagné la guerre...).
Le vrai c’est qu’on ne sait rien d’un tel Dieu, ni d’un autre monde possible, ni
de l’origine, si Dieu existe, du mal. Ce n’est « qu’un jeu d’esprit, conclut Voltaire,
pour ceux qui disputent : ils sont des forçats qui jouent avec leurs
chaînes ».

« Tout est bien », c’est aussi le dernier mot d'Œdipe, chez Sophocle, et de
Sisyphe, chez Camus. Ce n’est plus religion, c’est sagesse. Plus optimisme, mais
tragique. Plus foi, mais fidélité : « Sisyphe enseigne la fidélité supérieure, qui
nie les dieux et soulève les rochers. Lui aussi juge que tout est bien. Cet univers
désormais sans maître ne lui paraît ni stérile ni futile. [...] La lutte elle-même
vers les sommets suffit à remplir un cœur d’homme. Il faut imaginer Sisyphe
heureux » ({\it Le mythe de Sisyphe}, Conclusion).

Si tout est bien, au moins en un certain sens, ce n’est donc pas parce que le
mal n’existerait pas ou serait au service, comme le voulait Leibniz, d’un plus
grand bien. C’est plutôt qu’il n’y a que le réel, qui n’est ni bien ni mal (ou qui
ne l’est que pour nous), et qu’il faut accepter en entier pour pouvoir le transformer
au moins en partie. L’optimisme n’est qu’un mensonge ; le pessimisme,
qu’une tristesse. Le monde n’est pas un supermarché, où l’on aurait le choix des
produits. Il est à prendre ou à laisser, et nul ne peut le transformer qu’à la
condition d’abord de le prendre. Tout est bien ? Ce serait trop dire. Disons que
tout est vrai. On n’est plus chez Leibniz ; on est chez Spinoza. Philosophie non
du meilleur des mondes possibles, mais du seul réel. C’est retrouver Camus, au
point exact où il rencontre Spinoza : « Ce qui compte, c’est d’être vrai, et alors
tout s’y inscrit, l'humanité et la simplicité. Et quand donc suis-je plus vrai que
lorsque je suis le monde ? Je suis comblé avant d’avoir désiré. L’éternité est là

% 88
et moi je l’espérais. Ce n’est plus d’être heureux que je souhaite maintenant,
mais seulement d’être conscient » ({\it L'envers et l'endroit}, p. 49 dans la Pléiade).

\section{Bienfaisance}
%BIENFAISANCE
Capacité à faire le bien, ou du bien, autrement dit à bien
agir. Se dit presque exclusivement par rapport à autrui
(quand on se fait du bien à soi, ce peut être vertu, ce n’est pas bienfaisance), et
souvent avec réserve ou ironie. Quand quelqu'un fait du bien à autrui, on
soupçonne, souvent légitimement, quelque calcul ou condescendance. Aussi le
personnage du « bienfaiteur » n’est-il guère sympathique. Cela, toutefois, ne
saurait excuser ni l’égoïsme ni l’inaction.

\section{Bienséance}
%BIENSÉANCE
L’art ou la capacité de bien se conduire en public, de bien se
{\it tenir}, comme on dit, ce qui ne va pas sans contrainte ni artifice.
La bienséance concerne moins la morale que la politesse, moins l’être que
le paraître, moins la vertu que les convenances. Il s’agit d’abord de ne pas choquer.
Diogène, se masturbant en public, manquait assurément à la bienséance.
C'était sa façon à lui d’être vertueux.

\section{Bienveillance}
%BIENVEILLANCE
C'est vouloir du bien à quelqu'un, donc lui en faire (la
bienveillance se confond alors avec la bienfaisance) ou en
espérer pour lui. Cette seconde forme de bienveillance ne vaut que ce que vaut
l'espérance, qui ne vaut guère, mais mieux tout de même que la malveillance.

\section{Bilieux}
%BILIEUX
L'un des quatre tempéraments d’Hippocrate (avec le Iymphatique,
le sanguin et le nerveux). C'était aussi l’une des catégories,
aujourd’hui hors d’usage, de la caractérologie traditionnelle. Le bilieux aurait le
corps maigre, le teint jaune, l'esprit soucieux, le cœur fidèle... Cela s’expliquait,
selon Hippocrate, par un excès de bile dans l'organisme. On dirait
aujourd’hui plutôt l'inverse (si vous avez trop de bile, c’est à cause de vos
soucis). Une superstition chasse l’autre. Le foie et les soucis demeurent.

\section{Biologie}
%BIOLOGIE
La science de la vie ou du vivant. On remarquera qu’elle ne
donne aucune raison de vivre, ni même de faire de la biologie.
Toutes nos raisons pourtant en dépendent, qu’elle doit pouvoir, au moins en
droit, expliquer. Mais expliquer n’est pas juger, et n’en dispense pas.

% 89
\section{Bivalence}
%BIVALENCE
Le fait, pour une proposition quelconque, de ne pouvoir
prendre que deux valeurs, le vrai ou le faux. La logique classique
est une logique bivalente.

On appelle parfois {\it principe de bivalence} le principe qui stipule que toute
proposition a une valeur de vérité déterminée : elle est soit vraie soit fausse.
C’est une sorte d’alternative, mais appliquée à une seule proposition. Il semble
en résulter un fatalisme logique, duquel Aristote et Épicure eurent bien du mal
à s'échapper. Soit par exemple la proposition : {\it « Je me suiciderai demain. »} Le
principe de bivalence entraîne qu’elle est soit vraie soit fausse. Mais si elle est
vraie, je ne suis donc pas libre de me suicider ou non demain (puisqu'il est déjà
vrai que je le ferai) ; et pas davantage si elle est fausse (puisqu'il est déjà vrai que
je ne le ferai pas). Je ne suis donc libre ni dans un cas ni dans l’autre, qui sont
les deux seuls cas possibles. Comme le raisonnement peut être indéfiniment
répété à propos de n’importe quelle proposition, il en découle que tous les
futurs se produisent nécessairement, et qu’il n’y a dans le monde ni contingence
ni liberté. Si tout est vrai ou faux, tout est écrit. L'esprit de Parménide
souffle là, dans le désert de Mégare.

Il semble qu’on puisse en sortir pourtant, en posant (c’est le présupposé
d’existence) qu’une proposition ne peut être vraie ou fausse que pour autant
que son objet existe : si Dieu n'existe pas, il n’est ni vrai ni faux qu'il ait une
barbe. L'application du principe de bivalence aux événements futurs ne serait
donc légitime qu’à supposer que ces événements existent déjà, ce qui constituerait
à la fois une pétition de principe (tout ce qui existe existe nécessairement)
et une contradiction (puisqu’un futur qui existerait déjà ne serait pas un futur).
La vraie question est donc de savoir si le futur aussi est éternel.

\section{Bon}
%BON
Tout ce qui plaît ou semble devoir plaire. On dira par exemple : une
bonne action, un bon repas, une bonne idée. Mais pourquoi cela plaît-il ?
Parce que nous le désirons, répond Hobbes : « L'objet, quel qu’il soit, de
l'appétit ou du désir d’un homme, est ce que pour sa part celui-ci appelle {\it bon} ;
et il appelle {\it mauvais} l’objet de sa haine ou de son aversion. » Le bon et le mauvais
(à la différence supposée du bien et du mal) n’existent que subjectivement :
«Ces mots s'entendent toujours par rapport à la personne qui les emploie,
continue Hobbes, car il n'existe rien qui soit tel, simplement et absolument »
({\it Léviathan}, chap. VI). Même idée chez Spinoza : « Nous ne désirons aucune
chose parce que nous la jugeons bonne, mais au contraire nous appelons bonne
la chose que nous désirons ; conséquemment, nous appelons mauvaise la chose
que nous avons en aversion ; chacun juge ainsi ou estime, selon ses affects,
quelle chose est bonne, quelle mauvaise, quelle meilleure, quelle pire, quelle
% 90
enfin la meilleure ou quelle la pire » ({\it Éthique}, III, 39, scolie ; voir aussi III, 9,
scolie, et IV, Préface). Le bon n’est pas un absolu ; mais ce n’est pas non plus
un pur néant. C’est le bien désillusionné : le corrélat reconnu et assumé de nos
désirs.

C’est ce qui distingue le relativisme du nihilisme. « Par-delà le Bien et le
Mal, remarquait Nietzsche, cela ne veut pas dire par-delà le bon et le mauvais. »
Il avait raison. C’était sa façon à lui d’être spinoziste.

\section{Bonheur}
%BONHEUR
On croit parfois que c’est la satisfaction de tous nos désirs. Mais
si tel était le cas, nous ne serions jamais heureux, et ce serait
Kant, hélas, qu’il faudrait suivre : le bonheur serait « un idéal, non de la raison,
mais de l’imagination ». Comment tous nos désirs seraient-ils satisfaits, puisque
le monde ne nous obéit pas, puisque nous ne savons désirer, presque toujours,
que ce qui nous manque ? Ce bonheur-là n’est qu’un rêve, qui nous empêche
de latteindre.

D’autres veulent y voir une joie continue ou constante. Mais comment la
joie — qui est passage, jaillissement, turbulence — pourrait-elle l'être ?

Le bonheur n’est ni la satiété (la satisfaction de tous nos penchants), ni la
félicité (une joie permanente), ni la béatitude (une joie éternelle). Il suppose la
durée, comme l’avait vu Aristote (« une hirondelle ne fait pas le printemps, ni
un seul jour le bonheur »), donc aussi les fluctuations, les hauts et les bas, les
intermittences, comme en amour, du cœur ou de l'âme... Être heureux, ce
n'est pas être toujours joyeux (qui peut l'être ?), ni ne l’être jamais : c’est {\it pouvoir}
l'être, sans qu’on ait besoin pour cela que rien de décisif n’advienne ou ne
change. Qu'il ne s'agisse que de {\it possibilité} laisse place à l’espérance et à la
crainte, au manque, à l’à-peu-près.. qui le distinguent de la béatitude. Le bonheur
appartient au temps : c’est un état de la vie quotidienne. État subjectif,
bien sûr, relatif évidemment, dont on peut pour cela contester jusqu’à l’existence.
Mais qui a connu le malheur n’a plus de ces naïvetés, et sait, au moins
par différence, que le bonheur aussi existe. Le confondre avec la félicité, c’est se
l’interdire. Avec la béatitude, c’est y renoncer. Péchés d’adolescent et de philosophe.
Le sage n’est pas si bête.

On peut appeler {\it bonheur}, c’est en tout cas la définition que je propose, tout
laps de temps où la joie est perçue, fût-ce après coup, comme immédiatement
possible. Et {\it malheur}, inversement, tout laps de temps où la joie paraît immédiatement
impossible (où l’on ne pourrait être joyeux, c’est du moins le sentiment
que l’on a, que si quelque événement décisif changeait le cours du
monde).

% 91
Parce qu’il s’agit d’une joie seulement possible, le bonheur est un état imaginaire.
C’est donner raison à Kant ? Pas forcément. Car cela n’empêche pas
d’être heureux (c’est un état, non un idéal), ni que la joie advienne (le réel fait
partie du possible), et même c’est une raison déjà d’être heureux (l’imaginaire
fait partie du réel) et de se réjouir (quel bonheur de n’être pas malheureux !).
Ainsi la joie est le contenu — tantôt effectif, tantôt imaginaire — du bonheur,
comme le bonheur est le lieu naturel de la joie. C’est une espèce d’écrin :
l'erreur est de le chercher pour lui-même, quand il ne vaut que pour la perle.

L'erreur est même de le {\it chercher} tout court. C’est l’espérer pour demain,
où nous ne sommes pas, et s’interdire de le vivre aujourd’hui. Occupe-toi
plutôt de ce qui compte vraiment : le travail, l’action, le plaisir, l'amour — le
monde. Le bonheur viendra par surcroît, s’il vient, et te manquera moins, s’il
ne vient pas. On l’atteint d’autant plus facilement qu’on a cessé d’y tenir. « Le
bonheur est une récompense, disait Alain, qui vient à ceux qui ne l’ont pas
cherchée. »

\section{Bon sens}
%BON SENS
La raison ordinaire, comme à portée d’homme, et moins universelle
peut-être que commune. C’est le rapport humain au vrai :
«la puissance, disait Descartes, de bien juger et distinguer le vrai d’avec le
faux ».

Montaigne suggérait par boutade que tous en ont assez, puisque nul ne se
plaint d'en manquer ({\it Essais}, II, 17, p. 657). Descartes en conclura, peut-être
ironiquement, que « le bon sens est la chose du monde la mieux partagée », qui
serait « naturellement égale en tous les hommes » ({\it Discours de la méthode}, I).
C'était confondre non seulement le bon sens et la raison, comme Descartes fait
explicitement, mais aussi l’universel, qui est de droit, avec une égalité, qui serait
de fait. Que nous ayons tous la même raison, puisqu'il n’y en a qu’une, ne
prouve pas que nous en ayons tous autant ni assez. C’est ce que le bon sens sait
bien, qui le distingue de la raison et leur interdit à tous deux de prétendre à
l'absolu.

Une raison absolue serait de Dieu. Le bon sens, qui est une raison
humaine, nous en sépare et en tient lieu.

\section{Bonne foi}
%BONNE FOI
L’amour ou le respect de la vérité : c’est aimer ce qu’on croit
vrai, ou s’y soumettre. C’est en quoi la bonne foi se distingue
de la foi (qui croit vrai ce qu’elle aime) et vaut mieux. C’est la seule foi qui
vaille.

% 92
\section{Bonté}
%BONTÉ
Le propre de l’homme bon. C’est moins une vertu particulière que
la rencontre, en un même individu, de plusieurs vertus différentes
et complémentaires : la générosité, la douceur, la compassion, la bienveillance,
amour parfois. Que de tels êtres existent, même rares, même imparfaits,
c’est une vérité d'expérience, comme celle qu’il existe des salauds. La différence,
entre les uns et les autres, suffit à donner à la morale son sens, fût-il relatif, et
son enjeu.

On remarquera qu’un amour sans bonté — voyez la concupiscence ou la
jalousie — n’est plus une vertu du tout, alors qu’une bonté sans amour (faire du
bien à ceux qui nous sont indifférents, voire à ceux que l’on déteste) ne cesse
pas pour autant d’être bonne. Cela met l’amour à sa juste place, qui n’est la
première que sous réserve de sa bonté.

\section{Bornes de l'esprit humain}
%BORNES
DE L'ESPRIT HUMAIN C'est une expression traditionnelle,
pour dire que nous ne sommes pas
Dieu ni en état de connaître absolument l'absolu. J’en suis évidemment
d’accord. Mais est-ce le mot de {\it bornes} qui convient ? Voltaire lui consacre un
chapitre, plutôt décevant, de son {\it Dictionnaire}. Il se réclame de Montaigne.
Mais Montaigne, sur le sujet, est plus subtil. Que notre esprit soit limité, c’est
une évidence ; mais qui ne nous autorise pas à lui fixer une borne : «Il est
malaisé de donner bornes à notre esprit ; il est curieux et avide, et n’a point
occasion de s’arrêter plutôt à mille pas qu’à cinquante » ({\it Essais}, II, 12). C’est
toute la différence qu’il y a entre un être fini, comme nous sommes tous, et un
être borné, ce qui vaut comme reproche ou condamnation. Un de mes amis,
dont le jardin très étroit donnait directement sur l’à-pic d’une falaise, donc sur
la mer et le ciel, disait joliment que son jardin était « infini au moins par un
côté » ; puis il ajoutait : « comme l'esprit peut-être ». Cela me fit rêver longtemps.
Pourquoi un esprit fini ne pourrait-il concevoir l'infini, s’y plonger, s’y
perdre ? Les mathématiques en donnent l’exemple. La philosophie aussi peut-être.
À ceux qui lui reprochaient de borner sa philosophie à l’homme, Alain
répondait simplement : « C’est que je ne vois point la borne » (Propos du
25 déc. 1928).

\section{Boudhisme}
%BOUDDHISME
La doctrine du Bouddha, qui enseigne que tout est douleur
et les moyens de s’en libérer : par le renoncement à soi (le
sujet n’est qu’une illusion), à la stabilité (tout est impermanent), au manque
(tout est là), et même au salut ({\it nirvâna} et {\it samsâra} sont une seule et même
chose).

Le Bouddha n’est pas un dieu et ne se réclame d’aucun. Son enseignement
constitue moins une religion qu’une spiritualité, moins une philosophie qu’une
sagesse, moins une théorie qu’une pratique, moins un système qu’une thérapie.
Quatre bonnes raisons, pour les philosophes, de s’y intéresser.

\section{Bravoure}
%BRAVOURE
Le courage face au danger, avec quelque chose de spectaculaire
qui le redouble. « Le courageux a du courage, disait Joubert, le
brave aime à le montrer. »

\section{Brutalité}
%BRUTALITÉ
Un penchant pour la violence. Le brutal ne manque pas seulement
de douceur (il ne serait alors que violent) ; il manque
aussi d'intelligence, de finesse, de contrôle, de respect, de compassion... Triste
brute.

\section{Bureaucratie}
%BUREAUCRATIE
Le pouvoir des bureaux, c’est-à-dire (par métonymie) de
ceux qui y travaillent. S’oppose en cela à la démocratie

(les bureaux ne sont pas le peuple) mais aussi, plus généralement, au pouvoir
politique. C’est ce qui distingue la bureaucratie de l'administration. L’administration
sert le souverain ; la bureaucratie le dessert, s’en sert ou s’y substitue.

Le propre de la bureaucratie est d’être impersonnelle et irresponsable. Un
bureaucrate peut toujours en remplacer un autre. Mais qui pourrait renverser la
bureaucratie ? Ces gens-là n’ont pas été élus ; que leur font les électeurs ?

Cela n'empêche pas que les bureaucrates soient aussi des individus, qui doivent
répondre personnellement de leurs actes. Le règlement n’excuse pas tout,
ni l’obéissance, ni la hiérarchie. Eichmann était un bureaucrate.

\section{But}
%BUT
Ce qu'on vise, ce qu’on poursuit, ce qu’on veut atteindre ou obtenir.
Les stoïciens distinguaient la fin ({\it telos}) et le but ({\it skopos}). Le but est
extérieur à l’action : c’est par exemple la cible que vise l’archer. La fin est
intérieure : il s’agit non d’atteindre la cible (cela ne dépend pas de nous : un
coup de vent soudain peut faire dévier la flèche), mais de la viser bien. Le but
est toujours à venir; la fin, toujours présente. Le but est l’objet d’une
espérance ; la fin, d’une volonté.

Ainsi le sage agit sans but (c’est ce que les Orientaux appellent « le détachement
par rapport au fruit de lacte »), mais non sans fin. Et il vise d’autant
mieux la cible qu’il se moque de l'atteindre. L'action lui suffit.
%{\footnotesize XIX$^\text{e}$} siècle — {\it }

