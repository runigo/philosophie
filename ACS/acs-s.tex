%S{\it }
\chapter{S}
%— 604
%{\footnotesize XIX$^\text{e}$} siècle — {\it }

\section{Sacré}
%SACRÉ
Ce qui vaut absolument, au point de ne pouvoir être touché, sauf
précautions particulières, sans sacrilège. Le sacré est un monde à
part, comme le représentant, dans celui-ci, de l’autre. Il est séparé, ou doit
l'être, du quotidien, du laïque, du simplement humain. C’est en quoi le mot
dit plus que {\it dignité} — et dit sans doute trop. Le sacré mérite davantage que du
respect : il mérite, ou plutôt il exige, vénération, adoration, « crainte et
tremblement », comme un mélange d’effroi et de fascination. Le mot, en ce
sens strict, appartient au vocabulaire religieux : le sacré s'oppose au profane
comme le divin à l’humain ou comme le surnaturel à la nature. S'il n’y a ni
dieux ni surnaturel, comme je le crois, ce sacré-là n’est qu’un mot, que nous
mettons sur des sentiments archaïques ou illusoires.

En un sens plus général et plus vague, on appelle parfois {\it sacré} ce qui
semble avoir une valeur absolue, qui mérite pour cela un respect
inconditionnel : ainsi parle-t-on du caractère sacré de la personne humaine,
voire (dans la déclaration des droits de l’homme de 1789) de la propriété
privée comme étant « un droit inviolable et sacré ».. Le sacré, en ce sens
large, c’est ce qui peut être profané et ne le doit, qui mérite pour cela qu’on
se {\it sacrifie} pour lui. C’est ce qui fait dire à mon ami Luc Ferry que tout être
humain est sacré, et il m'est arrivé, quoique rarement, de le dire aussi. Reste
à savoir s’il l’est au sens strict (auquel cas l’humanisme est une religion :
voyez {\it L'homme-Dieu} de Luc Ferry) ou au sens large (auquel cas l’humanisme
n'est qu’une morale). Je penche trop résolument vers le deuxième terme de
l'alternative pour ne pas juger ce mot de sacré, appliqué à l'humain, quelque
peu excessif. C’est moins un concept, au moins dans ma bouche, qu’une
métaphore. Elle est parfois éclairante ; ce n’est pas une raison pour y croire
tout à fait. Le respect suffit, et vaut mieux.

%— 516 —
\section{Sacrement}
%SACREMENT
Un rite qui rend sacré, ou par lequel le sacré opère : c’est
comme un miracle institutionnel. Si le mariage est un sacrement,
par exemple, il devient sacré: nul ne peut s’en libérer, fût-ce d’un
commun accord, sans sacrilège. Le divorce prouve le contraire, ou plutôt ce
n’est qu’à supposer le contraire que le divorce devient admissible. Comment
l'Église pourrait-elle l’accepter ? Raison de plus pour ne pas laisser l'Église
décider de nos amours.

On remarquera que la naissance n’est pas un sacrement, alors que le baptême
en est un. Non bien sûr qu’un membre de l’Église soit plus sacré, même
aux yeux des croyants (au moins aujourd’hui), qu’un autre être humain, mais
en ceci que la naissance ne suppose aucune intervention surnaturelle. Or quoi
de plus émouvant qu’une naissance ? Quoi de plus kitsch, presque toujours,
qu’un baptême ? La vie est plus précieuse que les sacrements, et n'en a pas
besoin.

\section{Sacrifice}
%SACRIFICE
Une offrande qu’on fait au sacré, le plus souvent sous la forme
d’un animal ou d’un être humain qu’on immole. La plupart des
religions considèrent aujourd’hui que les sacrifices humains sont sacrilèges, ou
plutôt que nul n’a le droit de sacrifier que soi. C’est soumettre la religion à la
morale, comme il faut faire en effet, et l’une des marques les plus sûres de la
modernité. On dira qu’à ce compte la modernité commencerait à Abraham.
Pourquoi non ? Mais ce n’est que depuis Kant, peut-être bien, qu'on commence
à le comprendre.

En un sens plus général, le sacrifice est un don que l’on fait pour quelque
chose ou quelqu'un que l’on aime ou respecte. Le sacrifice ultime est celui de
sa propre vie — non qu’on la juge sans valeur, mais parce qu’on juge qu’elle ne
vaut qu’au service d’autre chose, qui la dépasse, ou de quelqu'un d’autre, qu'on
ne peut abandonner sans se trahir. Ainsi font les héros, et c’est à quoi, une fois
qu'ils sont morts, on les reconnaît.

\section{Sacrilège}
%SACRILÈGE
Une offense faite au sacré. Par exemple, au sens strict, cracher
sur un crucifix (ainsi Giordano Bruno, sur le bûcher). Ou bien,
au sens large : le viol, la torture, l'assassinat (le sacré, c’est ce qui peut être
profané : le corps humain est sacré). On voit que tous les sacrilèges ne se valent
pas. Celui de Giordano Bruno est l’un des actes les plus admirables que je
connaisse.

%— 517 —
\section{Sadisme}
%SADISME
Une perversion, qui consiste à jouir, comme on voit chez Sade,
de la souffrance d’autrui. Se distingue de la cruauté par une
charge érotique plus forte. Cela toutefois n’excuse rien, sauf consentement
explicite du partenaire.

\section{Sage}
%SAGE
Celui qui n’a pas besoin, pour être heureux, de se mentir, ni de se
raconter des histoires, ni même d’avoir de la chance. On dirait qu’il
se suffit à lui-même ; c’est en quoi il est libre. Maïs la vérité est qu’il se suffit de
tout, ou que tout lui suffit. Cela le distingue assez de l’ignorant, pour lequel
{\it tout}, ce n'est jamais assez. C’est que l’ignorant veut prendre, posséder, garder,
quand le sage se contente de connaître, de goûter ({\it sapere}, d’où vient {\it sapiens},
c'est avoir du goût), de se réjouir. C’est moins un savant qu’un connaisseur.
Moins un expert qu’un amateur (au double sens du terme : celui qui aime,
celui qui ne fait pas profession). Moins un propriétaire qu’un homme libre (le
{\it jivan mukta} des Orientaux : le libéré vivant). Le sage est sans maître, mais aussi
sans maîtrise, sinon sur soi, sans Église, sans appartenance, sans attaches, sans
attachements (ce qu’il aime, il ne le possède pas, ni n’en est possédé). Même
son bonheur ne lui appartient pas : ce n’est qu’un peu de joie dans le grand
vent du monde. Il est dépris de lui-même et de tout. C’est pourquoi peut-être
il est heureux : parce qu’il n’a plus besoin de l'être. Et sage : parce qu'il ne croit
plus à la sagesse.

\section{Sagesse}
%SAGESSE
L'idéal d’une vie réussie — non parce qu’on aurait réussi {\it dans} la vie,
ce qui ne serait que carriérisme, mais parce qu’on aurait réussi sa
vie elle-même. C’est le but, depuis les Grecs, de la philosophie. Toutefois ce
n'est qu’un idéal, dont il importe de se libérer aussi. Le vrai sage n’a que faire
de réussir quoi que ce soit : sa vie ne lui importe pas plus, ni moins, que celle
d'autrui. Il se contente de la vivre, et il y trouve un {\it contentement} suffisant, qui
est la seule sagesse en vérité. « Pour moi j'aime la vie », disait Montaigne. C’est
en quoi il était sage : parce qu’il n’attendait pas que la vie soit aimable (facile,
agréable, réussie...) pour l'aimer. Question de tempérament ? Question de
doctrine ? Sans doute un peu des deux. On est plus ou moins doué pour la vie,
plus ou moins sage ; ceux qui le sont moins ont donc besoin, j’en sais quelque
chose, de philosopher davantage. Mais nul n’est sage absolument, ni tout entier :
tous ont besoin de philosopher, ne serait-ce que pour se déprendre de la philosophie
elle-même. De la sagesse ? Bien sûr : on ne l’atteint qu’à condition de
cesser d’y croire. L'homme le plus sage du monde, un caillot ou un virus suffit
à le rendre fou. Ou un chagrin plus fort que les autres et que sa sagesse. Il le
%— 518 —
sait, et d'avance l’accepte. Ses échecs ne sont pas moins vrais que ses succès.
Pourquoi seraient-ils moins sages ? La sagesse, la vraie sagesse, n’est pas une
assurance tous risques, ni une panacée, ni une œuvre d’art. C’est le repos, mais
joyeux et libre, dans la vérité. Un savoir ? Tel est en effet le sens du mot, chez
les Grecs ({\it sophia}) comme chez les Latins ({\it sapientia}). Mais c’est un savoir très
particulier. « La sagesse ne peut être ni une science ni une technique », disait
Aristote : elle porte moins sur ce qui est vrai ou efficace que sur ce qui est bon,
pour soi et pour les autres. Un savoir ? Certes. Mais c’est un savoir-vivre.

Les Grecs distinguaient la sagesse théorique ou contemplative ({\it sophia}) de la
sagesse pratique ({\it phronèsis}). Mais l’une ne va guère sans l’autre, ou la vraie
sagesse, plutôt, serait la conjonction des deux. Elle se reconnaît à une certaine
sérénité, mais plus encore à une certaine joie, à une certaine liberté, à une certaine
éternité (le sage vit au présent : il sent et expérimente, comme disait Spinoza,
qu'il est éternel), à un certain amour... « De tous les biens que la sagesse
nous procure pour le bonheur de la vie tout entière, soulignait Épicure, l'amitié
est de beaucoup le plus grand » ({\it Maximes capitales}, XXVII). C’est que l'amour-propre
a cessé de faire obstacle. Que la peur a cessé de faire obstacle. Que le
manque a cessé de faire obstacle. Que le mensonge a cessé de faire obstacle. Il
n’y a plus que la joie de connaître : il n’y a plus que l'amour et la vérité. C’est
pourquoi nous avons tous nos moments de sagesse, quand l’amour et la vérité
nous suffisent. Et de folie, quand ils nous déchirent ou nous font défaut. La
vraie sagesse n’est pas un idéal ; c’est un état, toujours approximatif, toujours
instable (il n’est éternel, comme l'amour, que tant qu’il dure), c’est une expérience,
c’est un acte. Ce n’est pas un absolu, malgré les stoïciens (on est {\it plus ou
moins sage}), mais un maximum (comme tel relatif) : c’est le maximum de bonheur,
dans le maximum de lucidité. Il dépend de la situation de tel ou tel, des
capacités de tel ou tel (la sagesse n’est pas la même à Auschwitz ou à Paris, pour
Etty Hillesum ou pour Cavaillès), bref de l’état du monde et de soi. Ce n’est
pas un absolu ; c’est la façon, toujours relative, d’habiter le réel, qui est le seul
absolu en vérité. Cette sagesse-là vaut mieux que tous les livres qu’on a écrits
sur elle, qui risquent de nous en séparer. À chacun d'inventer la sienne.
« Quand bien même nous pourrions être savants du savoir d’autrui, disait
Montaigne, au moins sages ne pouvons-nous être que de notre propre sagesse »
({\it Essais}, I, 25).

\section{Saint}
%SAINT
Le mot se prend principalement en deux sens, l’un religieux, l'autre
moral.
Pour la religion, Le saint est celui qui est uni à Dieu (qui seul est saint absolument)
par la foi, l’espérance et la charité : il aime Dieu plus que tout, et que
%— 519 —
soi. Aussi est-il déjà sauvé, par cet amour même, déjà bienheureux, déjà dans le
Royaume, qu'il ne quittera plus. Il agit bien sûr moralement, mais par amour
et foi plutôt que par devoir.

Du point de vue moral, le saint est celui dont la volonté se conforme en
tout à la loi morale, au point que celle-ci, pour lui, ne vaille plus comme obligation
ou devoir (ce qui suppose la contrainte), mais bien comme liberté
(comme autonomie en acte). Ce n’est vraiment possible, selon Kant, qu’en
Dieu ({\it C. R. Pratique}, T, Analytique, \S 7, scolie) ou qu'après la mort ({\it op. cit.},
Dialectique, IV). Toutefois nul n’est tenu d’être kantien, ni dispensé, dès cette
vie, de faire son devoir : on peut appeler {\it saint}, en un sens plus général, celui qui
le ferait toujours.

Rien n'empêche, en ce dernier sens, qu’un saint soit athée, ou qu’un athée
soit saint. Toutefois ce n’est pas la règle. Non seulement parce que la médiocrité
est toujours le plus probable, chez les croyants comme chez les incroyants,
mais aussi parce que la plupart des saints (pour autant qu’il en existe, ou dans
la mesure où ils le sont) auront tendance à croire vrai ce qu’ils aiment ou qui
les meut. Ils ont raison, puisque cette vérité est en eux, ou puisqu'ils sont cette
vérité-là. Mais tort, peut-être, de croire qu’elle existe aussi hors d’eux-mêmes,
hors du monde et absolument. Que nous fassions parfois le bien, c’est difficilement
contestable, même pour les pécheurs que nous sommes. Mais pourquoi
faudrait-il que le Bien existe indépendamment de nous, et nous fasse ?

Au sens moral comme au sens religieux, le saint se distingue du sage, qui
n’a pas besoin de croire, ni d’espérer, ni d’obéir. Un Dieu ? Le salut ? La Loi ?
Il y a bien longtemps que le sage ne se soucie plus de ces abstractions ! Il les
laisse aux philosophes, qui en ont besoin.

Le sage et le saint voisinent, comme dirait Heidegger, sur des monts
séparés.

Le saint, sommet de la foi ou de la morale.

Le sage, sommet de l'éthique.

De l'extérieur, ils se ressemblent tellement qu’on pourrait les confondre, et
d’ailleurs rien n’interdit, au moins en droit, qu’un même individu soit les deux
à la fois. Mais le saint n’a que faire d’être sage, ni le sage d’être saint.

Qu'ils n'existent ni l’un ni l’autre absolument, c’est une évidence (comment
un sommet serait-il absolu ?), mais qui ne les réfute pas. Cela donnerait
plutôt raison au saint, par l'humilité. Et au sage, par l'humour.

\section{Sainteté}
%SAINTETÉ
La perfection morale ou religieuse. Elle n'appartient absolument
qu'à Dieu, s’il existe, mais peut se dire, par extension, de
ceux qui sont unis à lui ou qui respectent en tout la loi morale. Ce n’est qu’un
%— 520 —
%{\footnotesize XIX$^\text{e}$} siècle — {\it }
idéal, comme la sagesse : nul ne saurait y prétendre sans s’en éloigner. Mais les
deux idéaux sont très différents : idéal de soumission dans un cas (soumission à
Dieu, soumission à la loi morale : les saints sont des {\it muslims}, comme on dirait
en arabe, c’est-à-dire, étymologiquement, des soumis) ; idéal de liberté dans
l’autre (le sage est un {\it jivan mukta}, comme on dit en Inde, un libéré vivant).
Toutefois cette opposition reste abstraite, comme ces idéaux eux-mêmes. Comment
se libérer sans se soumettre à la nécessité ? Comment obéir à la loi morale
(c’est-à-dire à la liberté en soi : autonomie) sans être libre ? Sages et saints,
lorsqu'ils se rencontrent, préfèrent parler d’autre chose ou sourire en silence.

\section{Salaud}
%SALAUD
Le nom commun du méchant, ou plutôt du mauvais. Il ne fait pas
le mal pour le mal, mais par intérêt, par lâcheté ou par plaisir,
autrement dit par égoïsme : il fait du mal aux autres, pour son bien à soi. Le
salaud, ce serait donc l’égoïste ? Pas seulement, car alors nous le serions tous.
C’est l’égoïste sans frein, sans scrupules, sans douceur, sans compassion. La vulgarité
du mot traduit la bassesse de la chose, et se justifie par là.

Chez Sartre, le salaud est le gros plein d’être, celui qui se prend au sérieux,
celui qui se croit, celui qui oublie sa propre contingence, sa propre responsabilité,
son propre néant, celui qui fait semblant de n’être pas libre (c’est ce que
Sartre appelle la mauvaise foi), enfin qui fait le mal, lorsqu'il y trouve son
intérêt, en étant persuadé de sa propre innocence ou, s’il se sent parfois coupable,
d’innombrables circonstances atténuantes, qui l’excusent.

Ces deux définitions se rejoignent. Qu'est-ce qu’un salaud ? C’est un égoïste
qui a bonne conscience. Aussi est-il persuadé que le salaud, c’est l’autre. Il
s’autorise le pire, au nom du meilleur ou de soi — d’autant plus salaud qu’il se
croit justifié de l’être, et pense donc ne l'être pas. Comment s’imposerait-il
quelque frein que ce soit ? Pourquoi devrait-il se repentir ? Saloperie : égoïsme
de bonne conscience et de mauvaise foi.

\section{Salut}
%SALUT
Le fait d’être sauvé : « Il ne dut son salut qu’à la fuite ». En philosophie,
et pris absolument, le mot indique pourtant davantage qu’une
survie, toujours provisoire. Le vrai salut serait complet et définitif. Ce serait une
existence libérée de la souffrance et de la mort : la vie éternelle et parfaite. C’est
donc un mythe, qui relève comme tel du vocabulaire religieux. Ou bien il faut
considérer que l'éternité n’est pas autre chose que le présent, ni la perfection
autre chose que la réalité. C’est ce qui m’a fait dire parfois que nous sommes
déjà sauvés. Non parce que nous cesserions par là d’être perdus, mais parce que
le salut et la perte sont une seule et même chose. C’est ce que j’appelle le tragique,
%— 521 —
%{\footnotesize XIX$^\text{e}$} siècle — {\it }
et la seule sagesse qui ne mente pas. L’éternité, c’est maintenant : le salut
n'est pas une autre vie, mais la vérité de celle-ci. Nous sommes déjà dans le
Royaume. Aussi est-il vain de l’attendre, et même de l’espérer. C’est l'esprit de
Nagarjuna : « Tant que tu fais une différence entre le nirvâna et le samsâra, tu
es dans le samsära. » Tant que tu fais une différence entre ta vie telle qu’elle est
et le salut, tu es dans ta vie telle qu’elle est. C’est l'esprit de Prajnânpad : « La
vérité ne viendra pas ; elle est ici et maintenant. » Ce n’est plus religion, mais
sagesse. Plus promesse, mais don. Plus espérance, mais expérience. « La béatitude
est éternelle, écrit Spinoza, et ne peut être dite commencer que
fictivement » ({\it Éthique}, V, 33, sc.). Le salut est cette fiction, ou cette éternité.

\section{Sanguin}
%SANGUIN
L'un des quatre tempéraments selon Hippocrate et Gallien. Embonpoint,
teint vif, irritabilité, violence. On sait aujourd’hui que le
sang n'y est pour rien. Cela n’a pas suffi à les apaiser, ni à les faire maigrir.

\section{Santé}
%SANTÉ
« La santé est un état précaire, qui ne présage rien de bon. » Le docteur
Knock avait évidemment raison. Celui qui n’est pas malade
peut toujours le devenir, et même, sauf accident mortel, le deviendra inévitablement.
Il n’y a pas de santé absolue, pas de santé définitive : il n’y a que le
combat contre la maladie, contre la mort, contre l’usure, et c’est la santé même.
Elle n’est pas seulement l’absence de maladies (puisqu’on peut être en {\it mauvaise
santé}), mais la force en nous qui leur résiste, autrement dit la vie elle-même,
dans son équilibre fonctionnel et efficace. « La vie est une victoire qui dure »,
disait Jean Barois, et chacun sait bien qu’elle ne durera pas toujours. La santé
n'est pas son triomphe, mais son combat continué.

Il faut citer, parce qu’elle est absurde, la définition qu’en donne l'Organisation
Mondiale de la Santé: « La santé n’est pas seulement l'absence de
maladie ou d’infirmité. C’est un état de complet bien-être physique, psychique
et social. » L'Union soviétique avait donc raison d’interner ses dissidents en
hôpital psychiatrique : ils étaient malades, puisque leur bien-être, spécialement
psychique et social, n’était pas complet... Quant à moi, si j'ai eu, depuis que je
suis né, trois jours de santé, au sens de l’'O.M.S., c’est un maximum. Les
moments de bien-être, cela m'arrive assez souvent. Mais complets, c’est une
autre histoire. « Il y a toujours quelque pointe qui va de travers », comme disait
Montaigne, toujours quelque souci, quelque douleur, quelque angoisse.
« Docteur, ce matin, j'ai pensé à la mort. Cela m'inquiète. Mon état de bien-être
n’est pas complet : vous ne pourriez pas me donner quelque chose ? » C’est
%— 522 —
%{\footnotesize XIX$^\text{e}$} siècle — {\it }
confondre la santé et le salut, donc la médecine et la religion. Dieu est mort :
vive la Sécu !

Pourtant il est vrai, je le disais en commençant, que la santé n’est pas seulement
l’absence de maladies. Car alors les morts et les pierres seraient en
bonne santé. Non l’absence de maladies, donc, mais la puissance en nous, toujours
finie, toujours variable, toujours {\it précaire}, en effet, qui leur résiste ou les
surmonte. C’est pourquoi c’est le bien le plus précieux. Plus que la sagesse ?
Bien sûr, puisque aucune sagesse n’est possible sans une santé (notamment
mentale) au moins minimale. La santé n’est pas le souverain bien (elle ne tient
lieu ni de bonheur ni de vertu) ; mais elle est le bien le plus important —
puisqu'elle est la condition de tous. Ce n’est pas un salut ; c’est un combat. Pas
un but, un moyen. Pas une victoire, une force. C’est le {\it conatus} d’un vivant, tant
qu’il réussit à peu près.

\section{Sauvagerie}
%SAUVAGERIE
C’est comme une barbarie individuelle ou native, qui pour
cela inquiète moins. Il peut y avoir de bons sauvages ; il n’y
a pas de bons barbares.

La sauvagerie est proximité avec la nature («chez moi, en pays sauvage »,
écrit Montaigne : cela veut dire à peu près qu’il vit à la campagne). La barbarie
est distance d’avec la civilisation. Le sauvage n’est pas encore civilisé. Le barbare
ne l’est plus. Le sauvage est derrière nous. Le barbare, devant.

\section{Savoir}
%SAVOIR
Comme substantif, c’est un synonyme à peu près de connaissance.
Si on veut les distinguer, on peut dire que la connaissance serait
plutôt un acte, dont le savoir serait le résultat. Ou que les connaissances sont
multiples ; le savoir serait plutôt leur somme ou leur synthèse. Ces différences
restent pourtant approximatives et fluctuantes : l’usage ne les impose ni ne les
interdit.

Comme verbe, en revanche, la différence est plus nette : je sais lire et
écrire ; je connais (plus ou moins) le vocabulaire, la grammaire, l'orthographe.
Je sais conduire ; je connais le code de la route. Je connais la musique pour
piano de Schubert ; je ne sais pas la jouer, ni la lire. Je connais plus ou moins
la vie ; je sais plus ou moins vivre. Il me semble que toutes ces expressions vont
à peu près dans le même sens, qui indique au moins une direction. Le {\it connaître}
porte sur un objet ou une discipline ; le {\it savoir} porte plutôt sur une pratique ou
un comportement. Connaître, c’est avoir une idée vraie ; savoir, c’est pouvoir
faire. C’est pourquoi il ne suffit pas de savoir penser pour connaître, ni de
connaître pour savoir penser.

%— 523 —
%{\footnotesize XIX$^\text{e}$} siècle — {\it }
\section{Scélératesse}
%SCÉLÉRATESSE
Se conduire comme un scélérat, c’est-à-dire comme un criminel
ou, plus souvent, comme un salaud. (Le mot est
utile, parce que saloperie, en français, a un tout autre sens : une saloperie, c’est
l’acte d’un salaud ; la scélératesse, son caractère ou sa disposition).

\section{Scepticisme}
%SCEPTICISME
Le contraire du dogmatisme, au sens technique du terme.
Être sceptique, c’est penser que toute pensée est douteuse —
que nous n’avons accès à aucune certitude absolue. On remarquera que le scepticisme,
sauf à se détruire, doit donc s’inclure dans le doute général qu’il
instaure : tout est incertain, y compris que tout soit incertain. À la gloire du
pyrrhonisme, disait Pascal. Cela n’interdit pas de penser, et même c’est ce qui
oblige à penser toujours. Les sceptiques cherchent la vérité, comme tout philosophe
(c’est ce qui les distingue des sophistes), mais ne sont jamais certains de
l'avoir trouvée, ni même qu’on le puisse (c’est ce qui les distingue des
dogmatiques). Cela ne les chagrine pas. Ce n’est pas la certitude qu’ils aiment,
mais la pensée et la vérité. Aussi aiment-ils la pensée en acte, et la vérité en
puissance : c’est la philosophie même, et c’est en quoi, disait Lagneau, « Le scepticisme
est le vrai ». Il en découle que nul n’est tenu d’être sceptique, ni autorisé
à l’être dogmatiquement.

\section{Sciences}
%SCIENCES
Mieux vaut en parler au pluriel qu’au singulier. {\it La} science
n'existe pas : il n’y a que {\it des} sciences, et elles sont toutes différentes,
par leur objet ou leur méthode. Toutefois le pluriel, ici comme ailleurs,
suppose le singulier. Nul ne peut savoir ce que sont {\it les} sciences, s’il ne sait pas
ce que c’est qu’{\it une} science.

Disons d’abord ce que ce n’est pas. Ce n’est pas une connaissance certaine,
malgré Descartes, ni toujours une connaissance démontrée (puisqu’une hypothèse
peut être scientifique, puisqu'il n’y a pas de science sans hypothèses), ni
même une connaissance vérifiable (il est plus facile de vérifier la fermeture de
votre braguette que la non-contradiction des mathématiques, ce que nul ne
peut : cela ne retire rien à la scientificité des mathématiques, ni ne rend scientifique
votre comportement vestimentaire). Ce n’est pas non plus un ensemble
d'opinions ou de pensées, fût-il cohérent et rationnel — car alors la philosophie
serait une science, ce qu’elle n’est ni ne peut être.

Toute science, pourtant, relève bien de la pensée rationnelle ; disons que
c'est le genre prochain, dont les sciences sont une certaine espèce. Reste à
trouver leurs différences spécifiques. Qu'est-ce qu’une science? C’est un
ensemble de connaissances, de théories et d’hypothèses portant sur le même

%— 524 —
%{\footnotesize XIX$^\text{e}$} siècle — {\it }
objet ou le même domaine (par exemple la nature, le vivant, la Terre, la
société), qu’elle construit plutôt qu’elle ne le constate, historiquement produites
(toute vérité est éternelle, aucune science ne l’est), logiquement organisées
ou démontrées, autant qu’elles peuvent l'être, collectivement reconnues, au
moins par les esprits compétents (c’est ce qui distingue les sciences de la philosophie,
où les esprits compétents s’opposent), enfin — sauf pour les mathématiques —
empiriquement falsifiables. Si l’on ajoute à cela que les sciences s’opposent
ordinairement à l’opinion (une connaissance scientifique, c'est une
connaissance qui ne va pas de soi), on peut risquer une définition simplifiée :
{\it une science est un ensemble ordonné de paradoxes testables, et d'erreurs rectifiées}. Le
progrès fait partie de son essence ; non que les sciences avancent de certitude en
certitude, comme on le croit parfois, mais parce qu’elles se développent par
« conjectures et réfutations ».

Karl Popper, à qui j'emprunte cette dernière expression, s’est longtemps
battu, en un temps où c'était nécessaire, pour montrer que le marxisme et la
psychanalyse ne sont pas des sciences (aucun fait empirique n’est susceptible de
les réfuter). Il avait évidemment raison. Mais cette irréfutabilité, qui donne tort
à la plupart des marxistes et des psychanalystes, ne saurait valoir elle-même
comme réfutation, ou ne réfute que la prétendue scientificité des deux théories
en question. On évitera d’en conclure que marxisme et psychanalyse seraient
sans intérêt ou sans vérité. Tout ce qui est scientifique n’est pas vrai, tout ce qui
est vrai (ou possiblement vrai) n’est pas scientifique : la notion d’erreur scientifique
n’est pas contradictoire, celle de vérité scientifique n’est pas pléonastique.
C’est pourquoi la philosophie reste possible, et les doctrines, nécessaires.

\section{Scientisme}
%SCIENTISME
La religion de la science, ou la science comme religion. C’est
vouloir que les sciences disent l'absolu, quand elles ne peuvent
atteindre que le relatif, et qu’elles commandent, quand elles ne savent que
décrire ou (parfois) expliquer. C’est ériger la science en dogme, et le dogme en
impératif. Que resterait-il de nos doutes, de notre liberté, de notre
responsabilité ? Les sciences ne sont soumises ni à la volonté individuelle ni au
suffrage universel. Que resterait-il de nos choix? Que resterait-il de nos
démocraties ? Le mathématicien Henri Poincaré, contre cette niaiserie dangereuse,
a dit ce qu’il fallait : « Une science parle toujours à l'indicatif, jamais à
l'impératif. » Elle dit ce qui est, dans le meilleur des cas, plus souvent ce qui
paraît ou peut être, parfois ce qui sera, jamais ce qui {\it doit} être. C’est pourquoi
elle ne tient pas lieu de morale, ni de politique, ni, encore moins, de religion.
C’est ce que le scientisme méconnaît, et qui le condamne. Le positivisme, à
tout prendre, vaudrait mieux.

%— 525 —
%{\footnotesize XIX$^\text{e}$} siècle — {\it }
\section{Scolastique}
%SCOLASTIQUE
La doctrine et les procédés de l’École, c’est-à-dire, selon l’acception
la plus usuelle, des universités européennes au
Moyen-Âge : mélange de théologie chrétienne et de philosophie grecque
(d’abord platonicienne, du fait de l'influence de saint Augustin, puis de plus en
plus aristotélicienne), de logique et d'arguments d’autorité, de rigueur et de
monotonie..…. C’est un beau moment de l’esprit, où l'Occident s’invente, mais
qui finit par paralyser la pensée en l’enfermant dans des querelles aussi érudites
que stériles. Déjà Montaigne ne l’évoque que pour s’en moquer ou s’en
plaindre. Descartes, que pour l’enterrer. Tout indique pourtant qu’il y avait là
des trésors d’intelligence. Mais à quoi bon un trésor, quand on n’en a plus
l'usage ?

En un sens plus général et plus péjoratif, on appelle souvent {\it scolastique} la
doctrine d’une école, quelle qu’elle soit, dès lors qu’elle s’enferme dans une
orthodoxie déjà constituée (quitte à en complexifier indéfiniment les détails),
au point que c’est la pensée du Maître — et non plus l’accord avec le réel ou
l'expérience — qui décide de la vérité possible d’une proposition. Dogmatisme,
psittacisme, maniérisme. C’est ainsi qu’on a pu parler de la scolastique freudienne
(ou, en France, lacanienne), de la scolastique marxiste-léniniste, de la
scolastique heideggérienne.. Beaucoup d'intelligence dans les trois cas. C’est
ce qui rend la scolastique si dangereuse. Rien de tel pour stériliser un bel esprit.
J'en connais plusieurs qui sont passés pour cela à côté d’une œuvre possible.

\section{Sectarisme}
%SECTARISME
Un certain type de comportement intellectuel, digne d’une
secte, indigne d’un esprit libre. C’est un mélange d’étroitesse,
d’intolérance et de conviction : certitude d’avoir raison, même contre tous,
mépris ou rejet des autres positions, toujours suspectées d’aveuglement ou de
mauvaise foi, culte du chef, de la doctrine ou de l’organisation. C’est le dogmatisme
des imbéciles.

\section{Secte}
%SECTE
« Toute secte, disait Voltaire, est le ralliement du doute et de l’erreur. »
C’est qu’on ne dispute que sur ce qu’on échoue à connaître.
Il n’y a point de secte en géométrie, continuait Voltaire : « on ne dit point un
euclidien, un archimédien »... Même l'invention des géométries non euclidiennes
n’y a rien changé. Les sciences n’ont pas besoin d’absolu. L’universel
leur suffit. Toute religion, à l'inverse, est particulière : « Vous êtes mahométan,
donc il y a des gens qui ne le sont pas, donc vous pourriez bien avoir tort »
(Voltaire, {\it Dictionnaire philosophique}, article « Secte »). C’est ce qui énerve les
sectaires. Ils sentent bien que la pluralité des sectes, qui fait partie du concept,
%— 526 —
%{\footnotesize XIX$^\text{e}$} siècle — {\it }
est un formidable argument contre chacune d’entre elles. Vous êtes chrétien ;
c’est donc que tous ne le sont pas. Pourquoi auriez-vous raison davantage que
les autres ?

Qu'est-ce qu’une secte ? C’est une Église, vue par ceux qui n’en font pas
partie et qui la jugent sectaire. Le mot, en son sens moderne, vaut donc comme
rejet ou condamnation : la secte, c’est l’Église de l’autre. Ce n’est pas une raison
pour les interdire, tant qu’elles respectent la loi. Autant interdire la bêtise ou la
superstition.

On s'interroge sur la différence entre une secte et une Église. Je reprendrais
volontiers une formule qui n’est pas de moi : « Une Église, c’est une secte qui
a réussi. » Cela dit, par différence, ce qu’est une secte : une Église en gestation
ou en échec. Ses membres sont persuadés que le temps travaille pour eux, s’irritent
que cela aille si lentement, nous en veulent de ne rien faire pour accélérer
le processus. Ils sont pleins d’impatience, de mépris, de colère, de certitude.
C’est ce qui les rend sectaires. Redoutable engeance.

\section{Sélection}
%SÉLECTION
Un choix par élimination. Par exemple la sélection naturelle
des espèces, selon Darwin, par l'élimination des moins aptes.
Ou la sélection des meilleurs, à l’Université, par l'élimination des plus faibles,
des plus pauvres ou des moins studieux. Nos étudiants sont contre, ils l'ont fait
vertement savoir en de multiples occasions, et nos politiques, qui sont gens
prudents, ont renoncé à en parler. Cela n’a jamais empêché que la sélection se
fasse, bien sûr par l'échec (quelle sélection autrement ?), mais n’aide pas à la
faire dans de bonnes conditions — sur des critères purement scolaires, comme il
faudrait, et non en fonction des moyens financiers des parents. J'aimerais
mieux des examens plus sévères, et des bourses plus généreuses : cela serait
moins injuste et plus efficace.

\section{Sens}
%SENS
{\it Sens} se dit principalement en trois sens : comme sensibilité (le sens de
l’odorat), comme direction (le sens d’un fleuve), comme signification
(le sens d’une phrase). Un sens, c’est ce qui sent ou ressent, ce qu’on suit ou
poursuit, enfin ce qu’on comprend.

Le premier de ces sens est défini ailleurs (voir les articles « Sensation » et
« Sensibilité »). Les deux autres sont liés, au moins pour nous : le but d’une
action lui donne aussi une signification (si vous courez pour aller plus vite, cela
signifie vraisemblablement que vous êtes pressés) ; et la signification d’une
phrase, c’est ce qu’elle veut dire ou obtenir, autrement dit le but que poursuit
celui qui l’énonce ou vers lequel, même inconsciemment, il tend. Avoir un
%— 527 —
%{\footnotesize XIX$^\text{e}$} siècle — {\it }
sens, c'est {\it vouloir dire} ou {\it vouloir faire}. Cette volonté peut être explicite ou
implicite, consciente ou inconsciente, elle peut même n’être que l’apparence
d’une volonté ; cela nuance, mais n’annule pas, cette caractéristique générale :
il n’est de sens que là où intervient une volonté ou quelque chose qui lui ressemble
(un désir, une tendance, une pulsion). La sphère du sens et celle de
l’action se recouvrent : toute parole est un acte ; tout acte est un signe ou peut
être interprété comme tel.

Il en résulte qu’il n’est de sens que pour un sujet (que pour un être capable
de désirer ou de vouloir), et par lui. Un sens objectif ? C’est une contradiction
dans les termes. Un sens absolu ? Cela supposerait un Sujet absolu, qui serait
Dieu. On parle pourtant, je le signalais en commençant, du sens d’un fleuve.
Mais ce n’est, précisément, qu’une façon de parler. Si je dis par exemple que la
Loire coule d’Est en Ouest, ou qu’elle se dirige vers l’océan, cela ne suffit pas à
lui donner un sens : non seulement parce que la Loire ne veut rien dire (elle n’a
pas de signification), mais aussi parce qu’elle ne se dirige en vérité vers rien :
elle ne fait que suivre la pente. De même quand on dit qu’une fleur se dirige
vers le Soleil. Cela ne fait sens que pour nous, point pour elle : son phototropisme
doit tout à la nature, rien à la finalité ou à l’herméneutique.

Faut-il dire alors qu’il n’est de sens qu’humain ? Je n’en suis pas sûr. Des
animaux peuvent poursuivre un but et interpréter, même à l’état sauvage, le
comportement de tel ou tel de leurs congénères. Les éthologues, là-dessus, nous
renseignent suffisamment. Faut-il dire qu’il n’est de sens que pour une
conscience ? Pas davantage : la psychanalyse nous a assez éclairés sur la signification
inconsciente de tel ou tel de nos actes, de nos rêves ou de nos symp-
tômes. Je dirais plutôt qu’il n’est de sens que pour un être capable de désirer,
donc sans doute capable de souffrir et de jouir. C’est où l’on retrouve le mot
« sens » en sa première acception : il n’est de sens (comme signification ou
direction) que pour un être doué de sens (comme sensibilité), et proportionnellement
sans doute à cette faculté. La frontière est floue ? Pourquoi ne le serait-elle
pas ? L'homme n’est pas un empire dans un empire. Le sens non plus.

On remarquera que dans ces trois acceptions principales, et spécialement
dans les deux qui nous occupent (comme direction et comme signification), le
sens suppose une extériorité, une altérité, disons une relation à autre chose qu’à
soi. Prendre l’autoroute en direction de Paris n’est possible qu’à condition de
{\it n'être pas} à Paris. Et un signe n’a de sens que dans la mesure où il renvoie à
autre chose qu’à ce signe. Quel mot qui se signifie soi ? Quel acte qui se signifie
soi ? Tout mot signifie autre chose que lui-même (une idée : son signifié ; ou
un objet : son référent). Tout acte signifie autre chose que lui-même (son but,
conscient ou inconscient, ou le désir qui le vise). Pas de sens qui soit purement
intrinsèque : vouloir dire ou vouloir faire, c’est toujours vouloir autre chose que
%— 528 —
%{\footnotesize XIX$^\text{e}$} siècle — {\it }
soi. C’est ce qu'avait vu Merleau-Ponty : « Sous toutes les acceptions du mot
{\it sens}, nous retrouvons la même notion fondamentale d’un être orienté ou polarisé
vers ce qu'il n’est pas » ({\it Phénoménologie de la perception}, III, 2). Le sens
d’un acte n’est pas cet acte. Le sens d’un signe n’est pas ce signe. C’est ce qu’on
peut appeler la structure extatique du sens (il est toujours ailleurs). Nul ne peut
aller où il se trouve, ni se signifier soi. C’est ce qui nous interdit le confort,
l’autoréférence satisfaite, peut-être même le repos. On ne s’installe pas dans le
sens comme dans un fauteuil. On ne le possède pas comme un bibelot ou un
compte en banque. On le cherche, on le poursuit, on le perd, on l’anticipe…
Le sens n’est jamais là, jamais présent, jamais donné. Il n’est pas où je suis, mais
où je vais ; non ce que nous sommes ou faisons, mais ce que nous voulons faire,
ou qui nous fait. Il n’est sens, à jamais, que de l’autre.

Le sens de la vie ? Ce ne pourrait être qu'autre chose que la vie : ce ne pourrait
être qu’une autre vie ou la mort. C’est ce qui nous voue à l’absurde ou à la
religion. Le sens du présent ? Ce ne peut être que le passé ou l'avenir. C’est ce
qui nous voue au temps. Un fait quelconque n’a de sens, ici et maintenant, que
pour autant qu’il annonce un certain avenir (c’est la logique de l’action, tout
entière tendue vers son résultat) ou résulte d’un certain passé (c’est la logique
de l'interprétation, par exemple en archéologie ou en psychanalyse). Le sens de
ce qui est, c’est ce qui n’est plus ou pas encore : le sens de l'être, c’est le temps.
C’est ce qui justifie la belle formule de Claudel, dans {\it L'art poétique} : « Le temps
est le sens de la vie ({\it sens} : comme on dit le sens d’un cours d’eau, le sens d’une
phrase, le sens d’une étoffe, le sens de l’odorat). » Mais c’est aussi pourquoi le
sens, comme le temps, ne cesse de nous fuir, et d’autant plus qu’on le cherche
davantage : le sens du présent n’est jamais présent. Aussi le sens, comme le
temps, ne cesse-t-il de nous séparer de nous-mêmes, du réel, de tout. On le
trouve parfois, mais le sens qu’on trouve, comme dit Lévi-Strauss, « n’est
jamais le bon » ({\it La pensée sauvage}, IX), ou n’a lui-même de sens que par autre
chose, qui n’en a pas ou que l’on cherche. La quête du sens est par nature
infinie. C’est ce qui nous condamne à l’insatisfaction : toujours cherchant autre
chose, qui serait le sens, toujours cherchant le sens, qui ne peut être qu'autre
chose. Mais comment autre chose que le réel (son sens) serait-il réel ? « Le sens
du monde doit se trouver hors du monde », disait à juste titre Wittgenstein.
Mais hors du monde, quoi, sinon Dieu ? Le sens du présent, pareillement, doit
se trouver en dehors du présent. Mais hors du présent, quoi, sinon le passé ou
l'avenir, qui ne sont pas ? Sens, c’est absence : il n’est là (pour nous) qu'en tant
qu’il n’y est pas (en soi). Il y a donc du sens dans ma vie, puisque je me projette
vers l'avenir, puisque je reste marqué par mon passé, puisque j'essaie d'agir et
de comprendre. Mais comment ma vie elle-même aurait elle un sens, si elle ne
peut avoir que celui qu’elle n’a plus ou pas encore ?

%— 529 —
%{\footnotesize XIX$^\text{e}$} siècle — {\it }
« Quand le doigt montre la lune, l’imbécile regarde le doigt. » Ce proverbe
oriental va plus loin qu’il ne paraît. Cet imbécile nous ressemble, ou c’est nous,
bien souvent, qui lui ressemblons. Que fait-il ? Il regarde ce qui a du sens (le
doigt) plutôt que ce que le sens désigne, qui n’en a pas (la lune). Il se trompe
sur le sens, qui le fascine, et méconnaît le réel. Ainsi faisons-nous, à chaque fois
que nous sacrifions ce qui est à ce que cela pourrait signifier ou annoncer. Renversons
plutôt les priorités. Le sens ne vaut qu’au service d’autre chose, qui n’en
a pas. Comment serait-il le Tout (puisque le Tout, par définition, n’a pas
d’autre) ? Comment serait-il l’essentiel ? Rien de ce qui importe vraiment n’a
de sens. Que signifient nos enfants ? Que signifie le monde ? Que signifie
l'humanité ? Que signifie la justice ? Ce n’est pas parce qu’ils ont du sens que
nous les aimons ; c’est parce que nous les aimons que notre vie, pour nous,
prend sens. Une illusion ? Non pas, puisqu'il est vrai que nous les aimons.
L’illusion serait d’hypostasier ce sens, d’en faire un absolu, de croire qu’il existe
hors de nous et de sa quête. Ce serait Dieu — et la question se poserait d’ailleurs
de savoir ce qu’il pourrait bien {\it signifier}. Mais à quoi bon ? L'action suffit. Le
désir suffit. Il n’y a pas de sens du sens, ni de sens absolu, ni de sens en soi.
Tout sens, par nature, est relatif : ce n’est pas une substance, c’est un rapport ;
ce n'est pas un être, c’est une relation. C’est toujours la logique de l’altérité :
tout ce que nous faisons, qui a du sens, ne vaut qu’au service d’autre chose, qui
n'en a pas. Ce n’est pas le sens qu’il faut poursuivre, c’est ce qu’on poursuit qui
fait sens.

Souvenons-nous du Laboureur et de ses enfants. Il n’y a pas de trésor
caché, montre La Fontaine, mais le travail en est un, et le seul. Je dirais volontiers
la même chose du sens : il n’y a pas de sens caché, mais la vie en produit
et elle seule. Le sens n’est pas à chercher, ni à trouver, comme s’il existait déjà
ailleurs, comme s’il nous attendait. Ce n’est pas un trésor ; c’est un travail. Il
n'est pas tout fait: il est à faire (mais toujours en faisant autre chose), à
inventer, à créer. C’est la fonction de l’art. C’est la fonction de la pensée. C’est
la fonction de l'amour. Le sens est moins la source d’une finalité que le résultat
ou la trace d’un désir (or le désir, rappelle Spinoza, est cause efficiente). Moins
l’objet d’une herméneutique que d’une poésie — ou il ne peut y avoir herméneutique,
plutôt, que là où il y a eu d’abord {\it poièsis}, comme on dirait en grec,
c'est-à-dire création : dans nos œuvres, dans nos actes, dans nos discours. Le
sens n'est pas un secret, qu'il faudrait découvrir, ni un Graal, qu’il faudrait
atteindre. Il est un certain rapport, mais en nous, entre ce que nous sommes et
avons été, entre ce que nous sommes et voulons être, entre ce que nous désirons
et faisons. Ce n’est pas parce que la vie a un sens qu’il faut l’aimer ; c’est parce
que nous l’aimons, ou dans la mesure où nous l’aimons, qu’elle prend, pour
nous, du sens.

%— 530 —
%{\footnotesize XIX$^\text{e}$} siècle — {\it }
La vie a-t-elle un sens ? Aucun qui la précède ou la justifie absolument.
« Elle doit être elle-même à soi sa visée », comme dit Montaigne (III, 12,
1052). Elle n’est pas une énigme, qu’il faudrait résoudre. Ni une course, qu’il
faudrait gagner. Ni un symptôme, qu’il faudrait interpréter. Elle est une aventure,
un risque, un combat — qui vaut la peine, si nous l’aimons.

C’est ce qu’il faut rappeler à nos enfants, avant qu’ils ne crèvent d’ennui ou
de violence.

Ce n’est pas le sens qui est aimable ; c’est l'amour qui fait sens.

\section{Sens commun}
%SENS COMMUN C'est le bon sens installé : moins une puissance de juger
que son résultat socialement disponible et reconnu, autrement
dit un ensemble d’opinions ou d’évidences qu’il serait déraisonnable,
croit-on, de contester. L'expression, qui valait d’abord positivement (voyez le
{\it Lalande}), tend de plus en plus à devenir suspecte, voire péjorative. Nous avons
appris à nous méfier des évidences : le soupçon, face à l’unanimité, est notre
première réaction. C’est notre sens commun à nous.

\section{Sensation}
%SENSATION
Une perception élémentaire, ou l'élément d’une perception
possible. Il y a sensation lorsqu'une modification physiologique,
d’origine le plus souvent externe, excite l’un quelconque de nos sens. Par
exemple l’action de la lumière sur la rétine, ou des vibrations de l'air sur le
tympan, entraînent des modifications, via le système nerveux, jusqu’au
cerveau : c’est ce qui nous permet de prendre conscience de ce que nous voyons
ou entendons.

La perception est plutôt du côté de la conscience ; la sensation, du côté du
corps : elle fournit la matière que la perception mettra en forme. C’est pourquoi
c’est une abstraction, qui n’existe jamais seule. Nous n’avons affaire qu'à
des sensations plurielles, liées, organisées — qu’à des perceptions. Celles-ci sont
en quelque sorte au-delà du corps. La sensation, en deçà de l'esprit. C’est ce qui
faisait dire à Lagneau que « la sensation n’est pas une donnée de la conscience ».
Mais il ny aurait pas de conscience sans elle, ou ce ne serait qu’une conscience
vide.

La perception suppose la sensation ; elle ne s’y réduit pas. On ne peut percevoir
sans sentir ; mais il est possible de sentir sans percevoir. Par exemple le
contact du sol sous mes pieds : il est vraisemblable que je le sens toujours, du
moins quand je suis debout ou assis ; je ne le perçois qu’assez rarement. Ou le
bruit au loin de la rue : il est vraisemblable que je l’entends toujours ; je ne le
perçois (je ne m'aperçois que je l’entends) que lorsqu'il est spécialement fort ou
%— 531 —
%{\footnotesize XIX$^\text{e}$} siècle — {\it }
lorsque je l'écoute. C’est que la perception suppose une activité ou une attention,
au moins minimale, de l'esprit ; la sensation se suffit d’un esprit passif, ou
de la seule activité du corps. Ainsi quand je dors : j'entends, puisqu’un bruit
peut me réveiller. Mais je ne perçois aucun son. Cela dit, par différence, ce
qu'est la perception : non forcément une sensation active (percevoir un son
n'est pas forcément l’écouter), mais une sensation, ou plus souvent un
ensemble de sensations, dont on prend conscience ou auxquelles on prête
attention. La sensation est la même chose, abstraction faite de cette attention et
même de cette conscience. Mais il est vraisemblable que cette {\it abstraction} existe
d’abord et très concrètement : c’est l’ouverture du corps au monde, comme la
perception est ouverture de l’esprit au corps et à tout.

\section{Sensibilité}
%SENSIBILITÉ
La faculté de sentir ou de ressentir. Le mot peut désigner la
condition en nous d’un phénomène physique (la sensation),
affectif (le sentiment), voire intellectuel (le bon sens, comme sensibilité au vrai
ou au réel). Kant nous a habitués à considérer la sensibilité comme purement
réceptive ou passive. « La capacité de recevoir (réceptivité) des représentations
grâce à la manière dont nous sommes affectés par les objets se nomme {\it sensibilité}.
Ainsi, c’est au moyen de la sensibilité que des objets nous sont {\it donnés}, seule
elle nous fournit des {\it intuitions} ; mais C’est l’entendement qui pense ces objets,
et c'est de lui que naissent les {\it concepts} » ({\it C. R. Pure}, Esthétique transcendantale,
\S 1). Mais ce n’est passivité que pour l'esprit ; le corps, lui, fait activement son
travail, qui est de réagir aux excitations extérieures ou intérieures. C’est pourquoi
un bruit, une lumière ou une douleur peuvent nous réveiller. Parce que la
sensibilité, elle, ne dort jamais. C’est le travail du corps, et le repos de l'esprit.

\section{Sensible}
%SENSIBLE
Qui est doué de sensibilité, ou qui peut être perçu par les sens.
En philosophie, cette seconde acception est plus fréquente : le
monde sensible s'oppose au monde intelligible, depuis Platon, comme ce qui
est perçu par les sens à ce qui est connu par l’esprit. Mais notre monde, le seul
que nous puissions expérimenter et connaître, est l’unité des deux.

\section{Sensualisme}
%SENSUALISME
La doctrine qui veut ramener toutes nos connaissances aux
sensations. Le mot est souvent pris péjorativement, mais à
tort. L’épicurisme, par exemple, est un sensualisme : les trois critères de la
vérité — les sensations, les anticipations et les affections — se ramènent au premier
d’entre eux (Diogène Laërce, X, 31-34), si bien que les sens, comme on
%— 532 —
voit chez Lucrèce, sont la source, le fondement et la garantie de toute connaissance
vraie ({\it De rerum}, IV, 479-521). Reste à ne pas transformer ce sensualisme
en sottise. Épicure ni Lucrèce n’ont jamais dit qu’on pouvait sentir la vérité
elle-même, ni qu’il suffisait de regarder pour comprendre. Ils ont même dit fort
clairement le contraire : les yeux ne peuvent connaître la nature des choses (IV,
385), pas plus qu’aucun sens ne peut percevoir les atomes ou le vide, qui sont
pourtant la seule réalité. Sensualisme paradoxal, donc, mais sensualisme : toute
vérité est insensible, mais toute vérité vient des sensations. Il ne suffit pas de
sentir pour connaître : le sensualisme d’Épicure est aussi un rationalisme (au
sens large) ; mais aucune connaissance, sans la sensation, ne serait possible :
c’est en quoi le rationalisme d’Épicure est d’abord un sensualisme. Connaître
est plus que sentir ; mais ce {\it plus} lui-même (la raison, les anticipations.….) est
issu des sensations et en dépend (D.L., X, 32 ; {\it De rerum}, IV, 484). Sensualisme
rationaliste, donc, qui suppose une théorie sensualiste de la raison.

\section{Sentiment}
%SENTIMENT
Ce qu’on ressent, c’est-à-dire la conscience qu’on prend de
quelque chose qui se passe dans le corps, qui modifie notre
puissance d’exister et d’agir, comme dit Spinoza, et spécialement notre joie ou
notre tristesse. C’est le nom ordinaire des affects (voir ce mot), en tant qu'ils
sont durables (par différence avec les émotions) et concernent l'esprit ou le
cœur davantage que le corps ou les sens (par différence avec les sensations).

Le corps sent, l'esprit ressent : c’est à peu près la différence qu'il y a entre
une sensation et un sentiment. La sensation est une modification ({\it affectio}) du
corps ; le sentiment, un affect ({\it affectus}) de l'âme. On évitera pourtant de trop
forcer l’opposition. Si l’âme et le corps sont une seule et même chose, comme
dit Spinoza et comme je le crois, la différence entre les sentiments et les sensations
est plus de point de vue que d’essence : point de vue organique ou physiologique
dans un cas, affectif ou psychologique dans l’autre. Subjectivement,
pourtant, cette différence reste importante : ce n’est pas la même chose qu'avoir
mal et être triste, que se cogner contre un mur ou contre ses angoisses, que de
voir un visage ou d’en tomber amoureux. La sensation est un rapport au corps
et au monde ; le sentiment, un rapport à soi et à autrui.

\section{Sérieux}
%SÉRIEUX
Ce qui mérite attention et application, ou celui qui en fait preuve.
À ne pas confondre avec la dignité, qui mérite respect, ni avec la
gravité, qui se confronte davantage au tragique. Le sérieux, c'est ce avec quoi on
ne doit pas plaisanter, ou celui qui ne plaisante pas. C’est pourquoi le mot est
souvent péjoratif, soit parce qu’on le confond avec l’esprit de sérieux, soit parce

%— 533
%{\footnotesize XIX$^\text{e}$} siècle — {\it }
qu'on y voit un manque d’humour ou de légèreté. Cela peut arriver. Mais le
sérieux peut aussi marquer les limites de l'humour, qui le séparent de la
frivolité : qu’on puisse rire de tout, comme c’est en effet le cas, cela ne saurait
dispenser quiconque de faire son devoir. Le sérieux, dans certaines circonstances,
est une exigence éthique : celle de la responsabilité, de la constance, de
« l'engagement, comme disait Mounier, sans duperie et sans avarice ». Les
parents le savent bien : comme ils sont devenus sérieux, dès leur premier
enfant ! Cela ne les empêche pas de rire, et de leur sérieux même. Mais ils
savent bien qu’aucun rire désormais ne saurait les exempter de leurs responsabilités.
Cela vaut, plus généralement, dans toute situation qui nous confronte à
nos devoirs. Nul n’est tenu d’être un héros ; nul n’est dispensé d’assumer ses
responsabilités. « Il ne s’agit pas d’être sublime, disait Jankélévitch, il suffit
d’être fidèle et sérieux » ({\it L'imprescriptible}, p. 55).

\section{Sérieux (esprit de —)}
%SÉRIEUX (ESPRIT DE)
Se prendre soi-même au sérieux, ou ériger en
absolu les valeurs dont on se réclame. C’est
oublier le néant qu’on est et qui nous attend, mais aussi sa propre liberté (selon
Sartre), sa propre fragilité, sa propre dépendance, sa propre contingence.
Manque de lucidité et d'humour : c’est pécher, doublement, contre l'esprit.

\section{Servilité}
%SERVILITÉ
L’attitude d’un esclave ({\it servus}) ou d’un inférieur, quand il a
intériorisé sa propre soumission au point de la croire légitime.
« C’est une flatterie en action », disait Alain : « Tout fait signe qu’on exécutera
et qu'on approuvera. La servilité n'attend pas les ordres, elle les espère, elle se
précipite au-devant » {\it (Définitions)}. C’est une obéissance sans résistance, sans
révolte, sans dignité. Mauvaise obéissance. Il est indigne de la manifester, mais
plus encore de la susciter ou de l’encourager. La servilité est une soumission qui
se voudrait flatteuse, et qui est en vérité insultante. De quel droit le traiterais-je
comme un esclave ? De quel droit me traite-t-il comme un esclavagiste ?

SERVITUDE Soumission de fait, sans choix et sans limites, à un pouvoir
extérieur. C’est le contraire de la liberté, de l'indépendance, de
l'autonomie, mais aussi de la citoyenneté (qui est soumission de droit à un souverain
légitime, dont on fait partie) et même de la simple obéissance à une
autorité que lon s’est choisie ou que l’on accepte, dans des limites qui sont
celles de la dignité et de la responsabilité.

%— 534 —
%{\footnotesize XIX$^\text{e}$} siècle — {\it }
La Boétie, s'agissant de politique, parle de {\it servitude volontaire} : non que
personne choisisse d’être esclave, mais parce que nul tyran ne pourrait régner
sans le soutien, ou en tout cas l’acceptation, du plus grand nombre. Toutefois
cette volonté doit moins à un libre choix qu’à un système de croyances et d’illusions.
C’est ce qu’a vu Spinoza : « Le grand secret du régime monarchique et
son intérêt majeur est de tromper les hommes et de colorer du nom de religion
la crainte qui doit les maîtriser, afin qu’ils combattent pour leur servitude
comme s’il s'agissait de leur salut. On ne peut, par contre, rien concevoir ni
tenter de plus ficheux dans une libre république, puisqu'il est entièrement
contraire à la liberté commune que le libre jugement propre soit asservi aux
préjugés ou subisse aucune contrainte » ({\it T. Th. P.}, Préface).

Dans le champ de la philosophie éthique, on parle aussi de servitude pour
désigner la soumission d’un individu à ses passions : c’est obéir à son corps ou
à ses affects, au lieu de les commander ou d’essayer (en les comprenant) de s’en
libérer. C’est ainsi que Spinoza intitule la quatrième partie de son {\it Éthique} « De
la servitude humaine ou de la force des affects » : « J'appelle {\it servitude}, explique-t-il
dans sa préface, l'impuissance de l’homme à gouverner ses affects ; tant qu'il
leur reste soumis, en effet, l’homme ne relève pas de lui-même mais de la fortune,
dont le pouvoir est tel sur lui que souvent il est contraint, voyant le
meilleur, de faire le pire. » On sait que la cinquième partie s’intitulera « De la
puissance de l’entendement ou de la liberté humaine ». Que ce soit politique
ou morale, on ne sort de la servitude que par la raison, qui n’obéit à personne.

\section{Sexe}
%SEXE
C’est une partie du corps (les organes génitaux) en même temps
qu’une fonction, elle-même multiple (d’excitation, de plaisir, de
copulation, de reproduction…), l’une et l’autre divisant la plupart des espèces
animales en deux genres — on dit aussi en deux sexes —, qui sont les femelles et
les mâles. C’est notre façon d’appartenir à l'espèce (être humain, c’est être
femme ou homme), de pouvoir en jouir (par l’orgasme) et la prolonger (par la
reproduction). Beaucoup de plaisirs et de soucis en perspective.
« Le bas-ventre, disait Nietzsche, est cause que l’homme ait quelque peine
à se prendre pour un Dieu » ({\it Par-delà le bien et le mal}, IV, 141). C’est qu'un
Dieu doit être libre, et que nul ne l’est de son sexe : nous ne choisissons pas
d’en avoir un, ni lequel, pas plus que nous ne décidons de la force ou de la faiblesse
de ses désirs, ni de notre puissance, ou de notre impuissance, à leur
résister ou à les satisfaire... Les philosophes, pour cette raison, ont souvent
parlé du sexe avec suspicion ou dédain, quand ce n’est pas avec sottise ou pudibonderie.
Tant pis pour eux. Je ne sais guère que Montaigne qui en ait parlé
comme il faut, avec plaisir et humour, simplicité et vérité (voir spécialement

%— 535 —
%{\footnotesize XIX$^\text{e}$} siècle — {\it }
l’admirable chapitre 5 du livre III, « Sur des vers de Virgile »). « Chacune de
mes pièces me fait également moi que toute autre, disait-il. Et nulle autre ne
me fait plus proprement homme [ou femme] que cette-ci. » C’est que le désir,
non la liberté ou la raison, est l’essence de l'humanité, et que ce désir, sans être
uniquement sexuel, est toujours et tout entier sexué. Cela même qui nous
empêche de nous prendre pour un Dieu nous oblige à nous reconnaître animaux,
et à {\it devenir} humains. Jouir de l’autre, s’il y consent, ou le faire jouir, si
nous en sommes capables, cela ne saurait nous autoriser à l’asservir. Le désirer,
cela ne saurait nous dispenser de l’aimer et de le respecter.

\section{Sexisme}
%SEXISME
Une forme de racisme, fondée sur la différence sexuelle. Mieux
toléré que le racisme ordinaire. C’est qu’il est plus fréquent.

\section{Sexualité}
%SEXUALITÉ
Tout ce qui concerne le sexe, et spécialement les plaisirs qu’on
y trouve ou qu'on y cherche. C’est moins un instinct qu’une
fonction, moins une fonction qu’une puissance : puissance de jouir, et de faire
jouir. C’est le désir même, en tant qu’il est sexué. L’essence de l’homme, donc,
et de la femme, en tant qu’ils n’ont pas la même.

\section{Signal}
%SIGNAL
Définition parfaite chez Prieto : « Un signal est un fait qui a été
produit artificiellement pour servir d'indice ». Il annonce un fait
ou ordonne une action. Se dit surtout, en pratique, des signes non linguistiques.

\section{Signe}
%SIGNE
Tout objet susceptible d’en représenter un autre, auquel il est lié par
ressemblance ou par analogie (on parle alors d’icône ou de symbole),
par une relation causale (on parle alors d’indice ou de symptôme), ou encore et
surtout par convention (les Anglo-Saxons parlent alors de {\it symbol} ; mieux vaut
parler en français de signe conventionnel, ou de signe strictement dit). Le signe
linguistique entre bien sûr dans cette dernière catégorie. Il n’unit pas une chose
et un nom, montre Saussure, mais un concept (le signifié) et une image acoustique
(le signifiant). Le signe est l’unité des deux ; et c’est cette unité intra-linguistique
qui peut éventuellement désigner autre chose à l'extérieur du langage
(le référent). Le lien unissant le signifiant au signifié est purement
conventionnel : c’est ce que Saussure appelle « l'arbitraire du signe » ({\it Cours de
linguistique générale}, Y, chap. 1). Et celui unissant le signe à son référent, sauf
%— 536 —
%{\footnotesize XIX$^\text{e}$} siècle — {\it }
exception (les onomatopées), l’est tout autant. Cela ne signifie pas qu’on puisse
utiliser n'importe quel signifiant pour signifier n'importe quelle idée, ni
n'importe quel signe pour désigner n’importe quoi, mais que ce rapport est fixé
par une règle, non imposé par la nature ou suggéré par une ressemblance.

\section{Signifiant/signifié}
%SIGNIFIANT/SIGNIFIÉ
Les deux faces du signe, spécialement linguistique :
côté son, côté sens. Le {\it signifiant}, c’est la réalité
matérielle, ou plutôt sensorielle, du signe (le son que lon émet quand on le
prononce). Le {\it signifié}, c’est sa réalité intellectuelle ou mentale : le concept ou la
représentation que le signifiant véhicule (ce qu’on veut dire ou qu'on comprend
grâce à lui). Le signe est l'unité indissoluble des deux (De Saussure,
{\it Cours de linguistique générale}, TX, chap. 1). On remarquera que le rapport entre
le signifiant et le signifié, qui est arbitraire, reste interne au signe ; c’est ce qui
distingue le {\it signifié} du {\it référent}, et la {\it signification} de la {\it désignation}.

\section{Signification}
%SIGNIFICATION
C'est un rapport, interne au signe, entre un signifiant et
un signifié. À distinguer de la {\it désignation} (ou dénotation),
qui est un rapport entre le signe et ce à quoi il renvoie à l'extérieur de lui-même
(son référent). Par exemple quand je dis : « Il y a un oiseau sur la branche. » Le
rapport entre le signifiant (la réalité sonore et sensorielle du mot « oiseau », ce
que Saussure appelle son image acoustique) et le signifié (le concept d'oiseau)
reste interne au signe, qui est l’unité indissociable des deux : c’est ce rapport
qu’on appelle la {\it signification}. Le rapport de {\it désignation}, au contraire, tout en
restant intralinguistique (il ne vaut qu’à l’intérieur d’une langue donnée), unit
un signe à un objet qui existe en dehors du langage et, le plus souvent, indépendamment
de lui : l'oiseau n’est pas un signe, et n’habite que le silence.

\section{Silence}
%SILENCE
Au sens où je prends le mot, c’est l’absence non de sons mais de
sens. Un bruit peut donc être silencieux, comme un silence peut
être sonore. Ainsi le bruit du vent, ou le silence de la mer.

Le silence, c’est ce qui reste quand on se tait — c’est-à-dire tout, moins le
sens que nous lui prêtons (y compris donc ce sens même, quand nous cessons
de lui en chercher un autre). Ce n’est qu’un autre nom pour le réel, en tant
qu’il n’est pas un nom.

C'est aussi l’état ordinaire du vivant. « La santé, c’est le silence des
organes » (Paul Valéry). La sagesse, le silence de l'esprit.

%— 537 —
%{\footnotesize XIX$^\text{e}$} siècle — {\it }
À quoi bon interpréter toujours, parler toujours, signifier toujours ? Écoute
plutôt le silence du vent.

\section{Simple}
%SIMPLE
Ce qui est indivisible ou indécomposable (« simple, dit Leibniz,
c'est-à-dire sans parties »). Se dit aussi, mais par abus de langage, de
ce qui est facile à comprendre ou à faire. De là peut-être un troisième sens, qui
désigne une espèce de vertu, comme une facilité à vivre et à être soi. Être
simple, en ce dernier sens, c’est exister tout d’une pièce, sans duplicité, sans
calcul, sans composition : c’est être ce qu’on est, sans se soucier de le paraître,
sans s’efforcer d’être autre chose, c’est ne pas faire semblant, c’est n’être ni snob
ni intéressé, ni hystérique ni manipulateur. Je ne connais pas de vertu plus
agréable, et c’est à quoi peut-être les simples se reconnaissent le mieux : ils sont
faciles à vivre, à comprendre, à aimer.

\section{Sincérité}
%SINCERITÉ
Le fait de ne pas mentir. Ce n’est pas toujours une vertu (il
arrive que le mensonge vaille mieux), mais c’en est une que d’y
tendre.

\section{Singulier}
%SINGULIER
Qui ne vaut que pour un seul élément d’un ensemble donné.
S’oppose à ce titre à {\it universel} (qui vaut pour tous), à {\it général}
(qui vaut pour la plupart) et à {\it particulier} (qui vaut pour quelques-uns).

Dans le langage courant, le mot est souvent un synonyme de rare ou
d’étrange. Cet usage, en philosophie, est à éviter. L’individu le plus banal n’en
est pas moins {\it singulier} pour autant : la singularité est une caractéristique universelle
des individus.

\section{Situation}
%SITUATION
On y voit parfois l’une des dix catégories d’Aristote, d’ailleurs
point toujours la même : mieux vaut, pour éviter cette ambi-
guïté, parler de {\it lieu} ou de {\it position} (voir ces mots).

La situation d’un être, au sens courant du terme, c’est la portion d’espace-temps
qu'il occupe (son ici-et-maintenant propre), donc aussi son environnement
et sa place, le cas échéant, dans une hiérarchie. C’est également, s'agissant
d’un être humain, ce qu’il y fait. Par exemple quand on dit de quelqu’un qu’il
a « une belle situation » : cela désigne moins un lieu qu’un métier, qu’une fonction,
qu’un certain {\it rang} dans une hiérarchie sociale ou professionnelle. Toutefois
l’usage philosophique du mot tend de plus en plus à se concentrer sur son
%— 538 —
%{\footnotesize XIX$^\text{e}$} siècle — {\it }
acception sartrienne : être en {\it situation}, c’est être soumis à un certain nombre de
données et de contraintes que l’on n’a pas choisies (être un homme ou une
femme, grand ou petit, d’origine bourgeoise ou prolétarienne, dans tel ou tel
pays, à telle ou telle époque....), mais que l’on reste libre d’assumer ou non. La
situation, écrit Sartre, est un phénomène ambigu : c’est le « produit commun
de la contingence de l’en-soi et de la liberté ». C’est donc notre lot, définitivement.
Il y a toujours un monde, un environnement, des contraintes, des obstacles.
Toujours la possibilité de les affronter ou de les fuir. C’est ce que Sartre
appelle « le paradoxe de la liberté : il n’y a de liberté qu’en situation, et il n’y a
de situation que par la liberté » ({\it L'être et le néant}, IV, I, 2, p. 568-569). Mais
qu’en est-il alors de la liberté elle-même ? Elle n’est pas l'effet du donné, ni
conditionnée par lui (puisque le donné « ne peut produire que du donné »).
Elle n’est pas un être ; comment serait-elle déterminée par ce qui est ? Elle n’est
que néant et pouvoir de néantisation : elle échappe à l’être par la conscience
qu’elle en prend et la fin qu’elle projette. C’est bien commode. Il suffit
d'appeler « situation » tout ce que les sciences humaines considèrent comme
des déterminismes (le corps, l'inconscient, l'éducation, le milieu social.) pour
que le libre arbitre soit sauvé par là. Reste pourtant à savoir ce qui fait que je
choisis ce que je choisis. Si c’est ce que je suis (puisqu’un autre choisirait autrement),
tout choix est donc déterminé par quelque chose que je n’ai pas choisi.
Aussi faut-il, pour sauver la liberté, que mon choix s'explique non par ce que je
suis, mais {\it par ce que je ne suis pas} : c’est où la liberté retombe sur ses pieds ou
sur son néant. Qu'est-ce qu'être en situation ? C’est être confronté, comme
néant, à un être déterminé mais non déterminant (puisqu’une détermination
ne peut porter que sur l'être, point sur le néant), qu’on reste libre pour cela
d'assumer ou non. La situation est donc le corrélat objectif (déterminé, non
déterminant) de ma subjectivité : c’est l’être propre de mon néant.

La notion ne sauve la liberté, comme on le voit, que pour ceux qui
ne-sont-pas-ce-qu'ils-sont-et-sont-ce-qu'ils-ne-sont-pas, autrement dit qui échappent,
ou croient échapper, au principe d'identité. Pour ceux qui sont ce qu’ils sont et
ne sont pas ce qu'ils ne sont pas (non un néant, donc, mais un être), la situation
n’est qu’une façon de parler : c’est un déterminisme qui n’ose pas dire son
nom.

\section{Snobisme}
%SNOBISME
C’est prendre modèle sur une élite, ou supposée telle, faute de
pouvoir lui appartenir. Le snob mime une distinction qu’il n’a
pas, qu’il ne peut avoir. Il veut moins devenir ce qu'il imite (ce ne serait plus
snobisme mais émulation) qu’en prendre l'apparence. Son art est tout de naïveté
et de faux-semblants : c’est un simulateur sincère, comme l’hystérique
%— 539 —
%{\footnotesize XIX$^\text{e}$} siècle — {\it }
qu'il est parfois, et crédule, comme le superstitieux qu'il est souvent : c’est un
idolâtre de la forme, un adorateur des signes. Très proche ici du dandy,
l’humour en moins, le ridicule en plus. C’est un dandy qui se croit, quand le
dandy serait plutôt un snob qui se sait.

Une étymologie douteuse mais suggestive voudrait que le mot {\it snob}, bien
sûr d’origine anglaise, vienne du latin {\it sine nobilitate}. Le snob est sans noblesse
et cherche à le masquer ; il voudrait passer pour le noble qu’il n’est pas, en affichant
les manières qu’il croit être celles de l'aristocratie. C’est le syndrome de
M. Jourdain : un bourgeois qui veut passer pour gentilhomme. Ce n’est qu’un
snobisme parmi d’autres, aujourd’hui plutôt anachronique. Nos snobs ont
d’autres modèles. Mais ils ne sont snobs, aujourd’hui comme du temps de
Molière, que par incapacité à en prendre autre chose que les apparences. Soit
par exemple la culture : étaler celle qu’on a, c’est être cuistre ou pédant ;
simuler celle qu’on n’a pas, c’est être snob. La richesse ? Exhiber la sienne, c’est
être vaniteux, vulgaire, ostentatoire ; vouloir passer pour riche, c’est être snob.
Les relations, les conquêtes ? En afficher de vraies, c’est être mondain ou
goujat ; les exagérer ou en inventer de fausses, c’est être snob.

C’est où le snobisme touche à la mauvaise foi : être snob, c’est vouloir
passer pour ce qu'on n’est pas, en imitant ceux que l’on admire, que l’on envie
ou à qui l’on voudrait ressembler. Le snob se distingue par là de l’hypocrite, qui
n'imite que par intérêt, point par envie, qui ne veut pas ressembler mais
tromper, qui n'admire pas mais qui utilise. Le snob est ordinairement la première dupe —
et parfois la seule — de son personnage. C’est ce qui le rend plus
sympathique, plus ridicule, et moins rare. Pour un Tartufe, combien de bourgeois
gentilshommes et de précieuses ridicules ? L’hypocrisie est l” exception ; le
snobisme, [a règle. Qui peut être certain de lui échapper toujours ? Écrirait-on,
si l’on ne voulait passer pour écrivain ? Lirait-on, si l’on ne voulait que cela se
sache ? Chacun commence par faire semblant, par imiter ce qui lui manque, et
il n’y aurait pas de culture autrement. Le snobisme n’est qu’un premier pas, qui
voudrait donner l'illusion du parcours entier. L'erreur n’est pas d’imiter, mais
de faire semblant, et c’est le snobisme même: se contenter d’un jeu réglé
d’apparences au lieu de s'imposer un travail effectif, qui permettrait seul
d'avancer véritablement. Que M. Jourdain prenne des cours de musique ou de
philosophie, ce n’est pas moi qui le lui reprocherai. Mais on ne peut pas faire
semblant de penser, ni de chanter. Le snob est un mauvais élève ; il joue au professeur
au lieu de faire ses exercices.

\section{Socialisme}
%SOCIALISME
C'est d’abord une conception de la société et de la politique,
qui vise à mettre celle-ci au service de celle-là. En ce sens,
%— 540 —
%{\footnotesize XIX$^\text{e}$} siècle — {\it }
c'est presque un pléonasme, ce pourquoi tout le monde peut s’en réclamer (y
compris, dans l’Allemagne des années 1930, le parti National-socialiste).
Quelle politique voudrait aller contre l'intérêt de la société ? Pourtant le mot,
lorsqu'il apparaît, au début du xIx siècle, est loin de faire l’unanimité. Pierre
Leroux, qui l’inventa peut-être, du moins en français, l’opposait à l’individualisme.
Il y voyait un danger au moins autant qu’une promesse. Le socialisme
veut le bien de la société. Mais à quel prix ? Ce serait une politique collective,
sinon collectiviste, qui servirait le groupe plutôt que les individus. Comment
ne pas craindre qu’elle les écrase ?

En un sens plus restreint et plus rigoureux, on appelle {\it socialisme} tout système
fondé sur la propriété collective des moyens de production et d’échange :
c’est le contraire du capitalisme. Marx y voyait une période de transition,
devant mener au communisme. Certains de ses disciples, au Xx° siècle, voulurent
l’imposer par la révolution et la dictature du prolétariat. D’autres, par des
réformes et la démocratie. Cette opposition entre les révolutionnaires et les
réformistes portait moins sur le but que sur les moyens : il s’agissait, dans les
deux cas, de rompre avec le capitalisme.

L’échec tragique du marxisme-léninisme, dans tous les pays où il parvint au
pouvoir, mais aussi l’incapacité des sociaux-démocrates à triompher, même graduellement,
du capitalisme, rend ce deuxième sens presque obsolète. Les socialistes
d’aujourd’hui ont renoncé à sortir du capitalisme. Ils veulent seulement le
gérer de façon plus sociale, c’est-à-dire dans l’intérêt de l’ensemble de la société
et, spécialement, des plus pauvres. Ils ont fini par accepter l’économie de
marché, sans renoncer toutefois à l’encadrer : ils croient moins au « libre jeu des
initiatives et des intérêts individuels », comme dit Lalande, qu’à l’organisation
de l’État et de la société. Ce socialisme-là a renoncé à toute visée collectiviste.
Ce n’est plus le contraire du capitalisme : c’est son régulateur, et le contraire de
l'ultra-libéralisme.

\section{Société}
%SOCIÉTÉ  {\it Socius}, en latin, c’est le compagnon, l'associé, l’allié : vivre en
société, c’est vivre en compagnie, mais aussi dans un système
structuré d'associations et d’alliances.

« Humaine ou animale, une société est une organisation, notait Bergson :
elle implique une coordination et généralement aussi une subordination d’éléments
les uns aux autres » ({\it Les deux sources}, I). C’est le contraire de la solitude,
ou plutôt de l'isolement, de la dispersion, de la guerre, comme disait
Hobbes, de chacun contre chacun. C’est pourquoi les humains ont besoin de
société. Parce qu’ils ne peuvent vivre seuls, ni seulement les uns contre les
autres. Parce qu’ils ne peuvent s’isoler, comme disait Marx, qu’au sein de la
%— 541 —
%{\footnotesize XIX$^\text{e}$} siècle — {\it }
société. Mais leurs sociétés sont autrement fragiles que celles des insectes. C’est
que les règles y sont culturelles. C’est que les individus y sont libres de les violer
ou pas. C'est où commence la politique. C’est où commence la morale. Il peut
y avoir des sociétés sans État, sans pouvoir, sans hiérarchie. Mais il n°y a pas de
société sans solidarité, ni d’ailleurs de solidarité sans société.

\section{Sociobiologie}
%SOCIOBIOLOGIE
La science, ou prétendue telle, qui veut expliquer les
faits sociaux par des faits biologiques. C’est une espèce
de darwinisme social, mais revisité à la lumière des progrès de la génétique et
de l’éthologie.

Qu’une telle explication soit parfois possible ou nécessaire, spécialement
chez les animaux, nul ne le conteste. La sélection naturelle retient les gènes les
plus efficaces, parce que les mieux à même de se transmettre. Comment cela
n'aurait-il pas de répercussion sur les comportements sociaux, qu’ils soient
individuels ou collectifs ? Mais ce n’est alors qu'une partie de la biologie, bien
sûr légitime, qui nous en apprend davantage sur l’espèce que sur la société. Que
l’égoïsme et l’altruisme, par exemple, soient des comportements génétiquement
déterminés, on peut assurément l’admettre. Mais cela ne saurait expliquer leur
fonctionnement différencié, dans telle ou telle société, ni, encore moins, permettre
de choisir entre l’un et l’autre. La sociobiologie ne saurait donc tenir
lieu ni de sociologie ni de morale.

On a vu dans la sociobiologie une caution possible pour certaines thèses de
l'extrême droite. Si la société est soumise à la sélection naturelle, c’est-à-dire à
l'élimination des plus faibles au bénéfice des plus forts ou des plus aptes, on
aurait tort, nous dit-on, de se plaindre qu’elle soit injuste : l'inégalité serait un
avantage sélectif, qu’il faudrait préserver soigneusement. On n'échappe pas à
cet argument en contestant toute détermination biologique de l’humanité, ce
qui serait faire un bien beau cadeau à l'extrême droite, mais en refusant à cette
détermination toute pertinence politique et morale. Que nous soyons des animaux,
ce n'est pas une grande découverte, ou elle n’est pas récente. Cela ne
nous dispense pas de devenir humains, ni de le rester.

\section{Sociologie}
%SOCIOLOGIE
La science de la société, ou {\it des} sociétés. Le mot est inventé
par Auguste Comte. Mais c’est Durkheim qui fonde vraiment
la discipline. Il s’agit d'étudier les faits sociaux comme des choses,
explique-t-il, autrement dit comme extérieurs à notre intelligence et indépendants
de notre volonté. C’est pourquoi l’introspection ne suffit pas, ni l’observation
des comportements individuels. « Le groupe pense, sent, agit tout autrement
%— 542 —
%{\footnotesize XIX$^\text{e}$} siècle — {\it }
 que ne feraient ses membres, s’ils étaient isolés. Si donc on part de ces
derniers, on ne pourra rien comprendre à ce qui se passe dans le groupe » ({\it Les
règles de la méthode sociologique}, V, 2). C’est dire que la sociologie ne saurait se
réduire à la psychologie, ni même en dépendre. Un fait social ne peut et ne doit
s'expliquer que par un autre fait social : la société est « une réalité {\it sui generis} »,
explique Durkheim, avec ses caractères et ses déterminismes propres, qu’on
« ne retrouve pas sous la même forme dans le reste de l’univers ». La sociologie
est la science qui prend en compte cette réalité, ce qui n’est possible qu’à la
condition de la constituer d’abord comme objet. Il ne suffit pas d’observer la
société réelle pour faire de la sociologie. Car cette observation, faisant partie de
la société, risque fort d’en reproduire les préjugés, les illusions, les évidences.
C’est à peu près ce qui distingue la sociologie du journalisme.

\section{Sociologisme}
%SOCIOLOGISME
C’est vouloir tout expliquer par la sociologie. Cela ne
semble pas sans quelque pertinence. Le neurologue, le
physicien ou le philosophe vivent dans une société, dira-t-on : le sociologue a
donc vocation à les étudier et à rendre compte de leur travail. Et après tout
pourquoi pas ? Une sociologie des sciences ou de la philosophie est assurément
possible. Mais la sociologie ne saurait pour autant nous apprendre ce que sont
le cerveau ou l'univers, ni quelle philosophie est la meilleure ou la plus vraie.
D'ailleurs l'argument vaudrait aussi bien, ou aussi mal, dans l’autre sens. Le
sociologue a un cerveau et fait partie de l’univers : c’est donc aux neurobiologistes
et aux physiciens, pourrait suggérer tel ou tel, de nous dire ce qu’il en est
de la société et de la sociologie. Tous ces {\it ismes} sont ridicules, et se détruisent
mutuellement.

Au reste, le sociologisme, même à le considérer isolément, s’enferme dans
une aporie comparable à celles du biologisme ou du psychologisme. Si tout est
déterminé socialement, la sociologie l’est aussi: ce n’est qu’un fait social
comme un autre, qui n’a pas plus de valeur que n’importe quelle idéologie.
Que reste-t-il du sociologisme ? Pour que la sociologie puisse prétendre à la
vérité, il faut que la raison, qu’elle met en œuvre, lui échappe — que la vérité ne
soit pas un fait social. À la gloire du rationalisme, qui est la seule doctrine dont
Durkheim ait accepté de se réclamer ({\it Les règles de la méthode sociologique}, Préface
de la 1$^{\text{re}}$ éd.)

\section{Soi}
%SOI
Le sujet, considéré dans son objectivité. Cette contradiction le rend
insaisissable, voire impossible. Le bouddhisme enseigne qu’il n’y a pas
de soi, ni en moi (pas d’{\it atman}) ni en tout (pas de {\it Brahman}). Il n’y a que des

%— 543 —
%{\footnotesize XIX$^\text{e}$} siècle — {\it }
agrégats et des processus, qui sont tous conditionnés et impermanents. C’est
dire que le sujet n’est pas une substance, mais une histoire. Pas une essence,
mais un accident. Pas un principe mais un résultat, toujours éphémère. Tel est
aussi l’esprit, me semble-t-il, de nos sciences humaines. Cela met l'amour de soi
à sa place, qui n’est pas la première.

\section{Solidarité}
%SOLIDARITÉ
L'abus du mot, depuis des années, tend à lui faire perdre
toute signification rigoureuse. Ce n’est le plus souvent qu’une
générosité qui n’ose pas dire son nom (par exemple quand on donne de l'argent
à une organisation humanitaire) ou qui ne sait se manifester que sous la
contrainte (par exemple quand on crée un Impôt de Solidarité sur la Fortune).
Mais pourquoi faudrait-il avoir honte — quand on l’est, et on l’est si
rarement — d’être généreux ? Et comment la contrainte pourrait-elle suffire à la
solidarité ?

À parler de solidarité à tout bout de champ, nos politiques et nos belles
âmes la vident de tout contenu. C’est, avec la tolérance, la vertu {\it politiquement
correcte} par excellence. Cela ne la condamne pas, mais rend son usage malaisé.
Ce n’est plus un concept, c’est un slogan. Plus une idée, un idéal. Plus un outil,
une incantation. On voudrait l’abandonner aux meetings ou aux journaux. On
aurait tort. La confusion du langage, même politiquement correcte, est toujours
politiquement dangereuse.

Mieux vaut revenir au sens précis du mot, tel que le suggère l’étymologie.
{\it Solidaire} vient du latin {\it solidus}. Dans un corps solide, les différentes parties sont
solidaires en ceci qu’on ne peut agir sur l’une sans agir aussi sur les autres. Par
exemple une boule de billard : un choc sur un seul de ses points fait rouler la
boule entière. Ou dans un moteur: deux pièces, même séparées l’une de
l’autre, sont solidaires si elles ne peuvent bouger qu’ensemble. La solidarité
n’est pas d’abord un sentiment, encore moins une vertu. C’est une cohésion
interne ou une dépendance réciproque, l’une et l’autre objectives et dépourvues,
au moins en ce premier sens, de toute visée normative. Une boule de
billard ovale serait sans doute moins commode, mais n’en serait pas moins
solide pour autant.

C’est ce qui donne son sens, dans le latin juridique, à lexpression « {\it in
solido} », qui signifie en bloc ou pour le tout. Des débiteurs sont solidaires
quand chacun peut être tenu pour responsable (au cas où les autres s’avéreraient
insolvables) de la totalité de la somme empruntée. C’est bien sûr une
garantie, pour le prêteur, et un risque, pour chacun des emprunteurs. Par
exemple dans un couple marié sous le régime de la communauté des biens :
chacun des deux époux peut se trouver ruiné, sans l’avoir voulu, par les dettes
%— 544 —
%{\footnotesize XIX$^\text{e}$} siècle — {\it }
de l’autre, quand bien même elles auraient été faites à son insu ou contre sa
volonté. Les époux sont donc financièrement solidaires : responsables, ensemble
et pour le tout, de ce qui peut leur arriver, même séparément, ou de ce que peut
faire, même seul, l’un d’entre eux.

Mais le mot va bien au-delà de cette acception purement juridique. Deux
individus sont objectivement solidaires, si ce qu’on fait à l’un agit aussi, inévitablement,
sur l’autre (par exemple parce qu’ils ont les mêmes intérêts), ou si ce
que l’un fait engage également le second. C’est ce qui fonde le syndicalisme :
chacun y défend ses intérêts, mais en défendant aussi ceux des autres. C’est ce
qui fonde le mutualisme, spécialement face au danger, et donc toutes nos compagnies
d’assurance (chacune est fondée, même quand elle est purement capitaliste,
sur la mutualisation des risques). Un mauvais conducteur, dans une
mutuelle, fait perdre de l’argent à tous les sociétaires ; mais même les meilleurs
sont protégés par les cotisations de tous : qu’on vole sa voiture à l’un d’entre
eux, les autres paieront, ou plutôt ont déjà payé, pour la lui rembourser.

C’est où la différence entre la {\it générosité} et la {\it solidarité} apparaît le mieux.
Faire preuve de générosité, c’est agir en faveur de quelqu'un {\it dont on ne partage
pas les intérêts} : vous lui faites du bien sans que cela vous en fasse à vous, voire
à vos dépens ; vous le servez sans que cela vous serve. Par exemple quand vous
donnez dix francs au SDF qui fait la manche (dix francs ? tant que ça?) : il a
dix francs de plus, vous avez dix francs de moins. Ce n’est pas solidarité, c’est
générosité. Vous auriez tort d’en avoir honte, mais tort aussi de vous en
contenter. Car enfin le SDF n’en reste pas moins SDF pour autant. Et qui
serait assez généreux pour l’héberger ou lui payer un loyer ?

Faire preuve de solidarité, à l'inverse, c’est agir en faveur de quelqu'un {\it dont
on partage les intérêts} : en défendant les siens, vous défendez aussi les vôtres ; en
défendant les vôtres, vous défendez les siens. Par exemple quand des salariés
font grève pour réclamer une augmentation de salaire : ils la demandent pour
tous, mais chacun sait bien qu’il se bat aussi pour lui-même. Il en va de même
quand vous adhérez à un syndicat, quand vous souscrivez une police d’assurance
ou quand vous payez votre tiers provisionnel. Vous savez bien que vous
y trouvez votre compte (même si, s'agissant des impôts, il faut qu’un système
de contrôles et de sanctions possibles vous aide à vous persuader que c’est bien
votre intérêt de les payer...). Ce n’est pas générosité ; c’est solidarité. Vous
auriez tort, [à encore, d’en avoir honte, mais tout autant de vous en contenter.
Car enfin vous n'êtes pas encore sorti de l’égoïsme. Combien de salauds syndiqués,
assurés, et qui payent leurs impôts strictement ?

La générosité, dans son principe, est désintéressée. Aucune solidarité ne
l'est. Être généreux, c’est renoncer, au moins en partie, à ses intérêts. Être solidaire,
c’est les défendre avec d’autres. Être généreux, c’est se libérer, au moins
%— 545 —
%{\footnotesize XIX$^\text{e}$} siècle — {\it }
partiellement, de l’égoïsme. Être solidaires, c’est être égoïstes ensemble et intelligemment
(plutôt que bêtement, chacun pour soi ou les uns contre les autres).
La générosité est le contraire de l’égoïsme. La solidarité serait plutôt sa socialisation
efficace. C’est pourquoi la générosité, moralement, vaut mieux. Et c’est
pourquoi la solidarité, socialement, politiquement, économiquement, est beaucoup
plus importante. Qui ne met l'abbé Pierre plus haut que la moyenne des
syndiqués, des assurés, des contribuables ? Et qui ne compte, pour défendre ses
propres intérêts, sur l’État, les syndicats et les assurances, davantage que sur la
sainteté ou la générosité de son prochain ? Cela n’empêche pas l’abbé Pierre,
faut-il le préciser, de s’assurer, de se syndiquer ni de payer des impôts, ne serait-ce
que la TVA, à laquelle nul n'échappe — pas plus que cela ne dispense un
contribuable assuré et syndiqué de faire preuve, parfois, de générosité. Mais
les deux concepts, sans être en rien incompatibles, n’en restent pas moins différents.

S’il fallait compter sur la générosité des uns et des autres pour que tous les
malades puissent se soigner, des millions d’entre eux mourraient sans soins. Au
lieu de quoi on a inventé une chose très simple, au moins moralement, bien
plus modeste que la générosité, et bien plus efficace : la Sécurité sociale, et spécialement
l'assurance maladie. Nous n’en sommes pas moins égoïstes pour
autant. Mais nous en sommes mieux soignés.

Nul ne cotise à la Sécu par générosité. II le fait par intérêt, fut-il contraint
(merci l'URSSAF), mais ne peut les défendre efficacement, dans une société
solidaire, qu’en défendant aussi et par là même ceux des autres.

Nul ne souscrit une assurance par générosité. Nul ne paye ses impôts par
générosité. Et quel étrange syndicaliste ce serait, celui qui ne se syndiquerait
que par générosité ! Pourtant la Sécu, les assurances, les syndicats et la fiscalité
ont fait beaucoup plus, pour la justice et la protection des plus faibles, que le
peu de générosité dont parfois nous sommes capables.

Primauté de la générosité ; primat de la solidarité. La générosité, pour
l'individu, est une vertu morale ; la solidarité, pour le groupe, une nécessité
économique, sociale, politique. La première, subjectivement, vaut mieux. Mais
elle est objectivement à peu près sans portée. La seconde, moralement, ne vaut
guère ; mais elle est, objectivement, beaucoup plus efficace.

C’est où morale et politique divergent. La morale nous dit à peu près :
puisque nous sommes tous égoïstes, essayons de l’être un peu moins. La politique
dirait plutôt : puisque nous sommes tous égoïstes, essayons de l’être
ensemble et intelligemment ; essayons de développer entre nous des convergences
objectives d'intérêts, qui puissent aussi, subjectivement, nous rassembler
(par quoi la solidarité, de nécessité qu’elle est d’abord, peut devenir aussi une
vertu civique ou politique). La morale prône la générosité. La politique impose
%— 546 —
%{\footnotesize XIX$^\text{e}$} siècle — {\it }
et justifie la solidarité. C’est pourquoi on a besoin des deux, bien sûr, mais {\it plus
encore de politique}. Qu'est-ce qui vaut mieux ? Vivre dans une société où tous
les individus sont égoïstes, quoique inégalement, ou vivre dans une société sans
État, sans assurances, sans syndicats, sans Sécurité sociale ? Autant se demander
si la civilisation vaut mieux que l’état de nature, le progrès mieux que la barbarie,
ou la solidarité mieux que la guerre civile.

Je reviens à nos SDF. Beaucoup nous vendent des journaux dans le métro.
Quand vous en achetez un, est-ce générosité ou solidarité ? Cela dépend de vos
motivations, qui peuvent être mêlées. Mais, pour simplifier, on peut dire que
c’est surtout solidarité si vous y trouvez votre intérêt. Pourquoi l'y trouvez-vous ?
Parce que vous vous mettez à la place du SDF ? Ce serait compassion
plutôt que solidarité. Parce que vous vous dites que l’existence de ces journaux
vous permettra, si vous venez à perdre votre emploi, d’en vendre à votre tour ?
C’est douteux, puisque cette possibilité ne dépend presque aucunement du fait
que vous achetiez ou non ce journal aujourd’hui. Le plus vraisemblable, c’est
que vous n’y trouverez votre intérêt que si le journal en question est... intéressant.
Si vous pensez que c’est le cas, par exemple parce que vous avez l’habitude
de le lire et trouvez qu’il est bien fait, vous l’achetez par intérêt : c’est solidarité,
non générosité. Si ce n’est pas le cas, si vous savez d’avance que ce journal vous
tombera des mains, si vous ne l’achetez que pour faire plaisir au SDF ou lui
rendre service, c’est générosité et non plus solidarité. Qu'est-ce qui vaut
mieux ? Moralement, c’est bien sûr la générosité. Mais celle-ci ne règle en rien
la question de l’exclusion ou de la précarité. Le SDF a quelques francs de plus,
vous quelques francs de moins ; il n’en est pas moins exclu, ni la société moins
injuste. Mieux vaudrait que ces journaux deviennent de bons journaux, que des
millions de lecteurs les achètent par intérêt, comme quand nous achetons nos
quotidiens ou nos magazines habituels, autrement dit par égoïsme. Moralement,
ce serait moins méritoire. Mais socialement, beaucoup plus efficace : le
vendeur ne serait plus un exclu, mais un marchand de journaux.

C’est le plus étonnant à penser. Quand j'achète mon journal chez le
libraire, ni lui ni moi n’agissons par générosité — pas plus d’ailleurs que les journalistes
ou les propriétaires du journal. Nous y cherchons tous notre intérêt,
mais nous ne pouvons le trouver que dans la mesure où ces intérêts convergent,
au moins pour une part, et c’est ce qu’ils font en effet (le journal, autrement,
aurait déjà disparu). C’est en quoi la Presse est aussi un marché (voir ce mot),
et cela, loin de la condamner, est ce qui la sauve. Tout marché fonctionne à
l’égoïsme. Mais il ne peut fonctionner efficacement et durablement qu’à la
condition de créer ou de maintenir des convergences objectives (qui peuvent
éventuellement devenir aussi subjectives) d’intérêts. L'égoïsme est le moteur, en
chacun. La solidarité la régulation, pour tous.

%— 547 —
%{\footnotesize XIX$^\text{e}$} siècle — {\it }
Libéralisme ? Pourquoi faudrait-il avoir peur du mot ? Dans une société de
marché, les journaux sont toujours plus intéressants que dans une société collectiviste.
Cela vaut, l'expérience le prouve, pour toute marchandise. Les vêtements
sont toujours de meilleure qualité si marchands et fabricants y trouvent
leur intérêt. Il faudra donc qu’ils le trouvent, et le marché, dans ce domaine,
s’est avéré autrement efficace que la planification et les contrôles (qui débouchent,
presque inévitablement, sur le marché noir). Mais on aurait tort, évidemment,
de croire que le marché puisse suffire à tout : d’abord parce qu’il ne
vaut que pour les marchandises (or la liberté n’en est pas une, ni la justice, ni
la santé, ni la dignité.) ; ensuite parce qu’il ne saurait suffire à sa propre régulation.
Que deviendrait le commerce, sans un droit du commerce ? Et comment
ce droit lui-même serait-il une marchandise ? Comment serait-il à
vendre ? Et comment pourrait-il suffire, s’agissant de ce qui ne s’achète pas ?
On le voit, là encore, dans la Presse. L’abandonner purement et simplement
aux lois du marché, c’est mettre en cause son indépendance (face au pouvoir de
l'argent), sa qualité, sa diversité, son pluralisme. On a donc inventé un certain
nombre de garde-fous, de même que des aides à la Presse, qui ne suppriment
pas les phénomènes de marché (un journal sans lecteurs aura toujours du mal à
survivre, et c’est très bien ainsi), mais qui les modèrent ou en limitent les effets.
L'information est aussi une marchandise. Mais la liberté d’information n’en est
pas une. Un journal s’achète. La liberté des journalistes et des lecteurs, non.

Cela vaut aussi pour la santé, pour la justice, pour l’éducation, et même, au
moins pour une part, pour la nourriture ou le logement. Aucun de ces phénomènes
n'échappe totalement au marché. Aucun, sauf à renoncer à la protection
des plus faibles, ne peut lui être totalement abandonné. Le marché crée de la
solidarité, mais il crée aussi de l’inégalité, de la précarité, de l'exclusion. C’est
pourquoi nous avons besoin d’un État, d’un droit du commerce, d’un droit
social, d’un droit de la Presse, etc., mais aussi de syndicats, d’associations,
d'organismes paritaires de contrôle et de gestion. Le marché est plus efficace
qu’une économie administrée. Mais la loi de tous (la démocratie) vaut mieux
que la loi de la jungle. La Sécu est plus efficace, socialement, que la générosité.
Mais aussi plus juste, politiquement, que de simples assurances privées. C’est
où les ultra-libéraux se trompent. Que le marché crée de la solidarité, cela
n'implique pas qu’il y suffise. C’est où le collectivisme se trompe. Que le
marché ne suffise pas à la solidarité, cela ne veut pas dire que celle-ci puisse se
passer de celui-là. À la gloire de la politique, des syndicats et de la Sécu.

\section{Solipsisme}
%SOLIPSISME
C’est ne croire qu’en sa propre existence : considérer qu’on
est soi-même ({\it ipse}) le seul ({\it solus}) existant. Le fait est que
%— 548 —
%{\footnotesize XIX$^\text{e}$} siècle — {\it }
tout le reste, n’étant connu que par les sensations, reste en quelque chose douteux,
alors que le {\it moi} semble pouvoir se recommander de la certitude immédiate
du {\it cogito}. La doctrine, pour irréfutable qu’elle soit, n’en reste pas moins
invraisemblable : comment expliquer l'existence du moi, si rien d’autre
n'existe ? Aussi le solipsisme n’a-t-il pas de partisans. C’est moins une doctrine
qu’un problème, pour les philosophes idéalistes. Si on part du sujet, comment
en sortir ?

\section{Solitude}
%SOLITUDE
Ce n’est pas la même chose que l’isolement. Être isolé, c’est être
coupé des autres : sans relations, sans amis, sans amours. État
anormal, pour l’homme, et presque toujours douloureux ou mortifère. Alors
que la solitude est notre condition ordinaire : non parce que nous n'avons pas
de relations avec autrui, mais parce que ces relations ne sauraient abolir notre
solitude essentielle, qui tient au fait que nous sommes seuls à être ce que nous
sommes et à vivre ce que nous vivons. « Dans la mesure où nous sommes seuls,
écrit Rilke, l'amour et la mort se rapprochent. » Non qu’il n’y ait pas d'amour,
ou qu’on soit seul à mourir ; mais parce que personne ne peut mourir ou aimer
à notre place. C’est pourquoi «on mourra seul », disait Pascal : non parce
qu’on devrait mourir isolé (du temps de Pascal, ce n’était presque jamais le cas :
il y avait ordinairement un prêtre, la famille, des amis...), mais parce que personne
ne peut mourir à notre place. C’est pourquoi on vit seul, toujours : parce
que personne ne peut vivre à notre place. Ainsi l'isolement est l'exception ; la
solitude, la règle. C’est le prix à payer d’être soi.

\section{Sommeil}
%SOMMEIL
C’est comme une suspension périodique de la vigilance et de
l’activité : le corps se fait presque immobile, l'esprit se ralentit, la
conscience s’oublie ou se regarde rêver. La vie fatigue et tue ; seul le sommeil
nous permet de survivre, au moins un temps. Aussi est-il « le premier besoin de
l’homme, disait Alain, plus pressant même que la faim, et qui suppose société
et veilleurs à tour de rôle, d’où toutes les institutions de police ». Il faut que les
uns montent la garde, pendant que les autres dorment : la société est fille de la
peur, disait encore Alain, davantage que de la faim ; et du sommeil, ajouterais-je,
davantage que de l’ambition.

\section{Somnolence}
%SOMNOLENCE
Un état intermédiaire entre la veille et le sommeil, qui peut
indifféremment préparer celui-ci ou celle-là, mais aussi,
parfois, nous en séparer. Elle peut être délicieuse ou insupportable, selon les
%— 549 —
%{\footnotesize XIX$^\text{e}$} siècle — {\it }
exigences du moment : c’est ce qui distingue l’insomnie de la grasse matinée, et
l'extrême fatigue du repos. « Il arrive que la somnolence soit volontaire, remarquait
Alain ; c’est alors un moyen de se reposer tout en restant ouvert aux
signes. » Mais aussi qu’elle s’impose à nous, et nous enferme.

\section{Songes}
%SONGES
Un autre mot, plus littéraire, pour dire les rêves. L'âme y est en
liberté, remarque Voltaire, et elle est folle. C’est qu’elle a perdu les
rails du réel. Heureusement que le réveil nous rend la raison en nous rendant
au monde.

\section{{\it Sophia}}
%{\it SOPHIA}
La sagesse théorique ou contemplative. Se distingue par là de la {\it phronèsis}
(la prudence, la sagesse pratique). C’est une différence que le
français ne fait pas, et il a raison : la vraie sagesse serait la conjonction des deux.

\section{Sophisme}
%SOPHISME
C'était à Montpellier, il y a une quinzaine d’années, dans la
cour, transformée en amphithéâtre, d’un bel hôtel particulier
du {\footnotesize XVIII$^\text{e}$} siècle. Je participais cet été-là, dans le cadre d’un festival organisé par
France-Culture, à un débat sur la religion, retransmis en direct par la radio.
Mon propos où mon athéisme agacent l’un des participants : « Je me demande
de quel droit vous pouvez parler de religion, me lance-t-il, puisque vous ne
croyez pas en Dieu ! »

L’argument, qui me paraît sophistique, m'’agace à mon tour. Je lui
réponds : « Autant dire qu’il faut, pour pouvoir parler légitimement de la
musique de Beethoven, être sourd, sous prétexte que Beethoven, lui, l’était ! »
Rires sur les gradins : j’avais marqué un point. Pourtant je voyais bien que ma
riposte était sophistique, au moins autant que l'attaque de l’autre, ou plutôt
davantage (ce n’était dans sa bouche, selon toute vraisemblance, qu’un paralogisme).
Mon excuse, outre l’éventuelle drôlerie de la chose, était de n’en être
pas dupe, ni d’essayer vraiment de duper : c'était polémique plutôt qu’argumentation,
défense plutôt qu’attaque, ironie plutôt que pensée. Enfin il faisait
très beau et nous étions en vacances : un sophisme, entre intellectuels, peut
faire parfois, dans certaines circonstances choisies, un divertissement acceptable.
Toutefois il convient de ne pas en abuser, sans quoi ce n’est plus débat
mais combat, plus pensée mais cirque. Je suis revenu bien vite à des choses plus
sérieuses. Si seuls les croyants pouvaient parler de religion, comment un athée
pourrait-il justifier son athéisme, et pourquoi organiser, entre croyants et
incroyants, un débat ?

%— 550 —
%{\footnotesize XIX$^\text{e}$} siècle — {\it }
Qu'est-ce qu’un sophisme ? C’est une faute volontaire, dans un raisonnement
(par différence avec le paralogisme, faute involontaire), qui vise à
tromper ou à embarrasser. Il est sans valeur quant au fond, sinon parfois par
les difficultés qu’il fait paraître. C’est une arme plus qu’une pensée, et à
double tranchant : c’est jouer avec la vérité, ou plutôt avec son apparence, au
lieu de la servir. Mes excuses, donc, cher collègue, pour ce sophisme d'il y a
quinze ans.

\section{Sophiste}
%SOPHISTE
Celui qui fait profession de sagesse ({\it sophia}) ou de sophismes.
Double faute : la sagesse n’est pas un métier ; un sophisme n’est
ni une preuve ni une excuse. Le mot, en ce sens, est toujours péjoratif : le
sophiste, c’est celui qui cherche moins la vérité que le pouvoir, le succès ou
l'argent. C’est contre eux que Socrate inventa, ou réinventa, la philosophie.

En un autre sens, plus large et plus neutre, on peut appeler {\it sophiste} toute
personne qui se réclame de la sophistique (voir ce mot) ou en relève, à commencer
par ses fondateurs (Protagoras, Gorgias, Prodicos, Antiphon.….).
C’est alors une catégorie historique, non polémique. Elle ne saurait pourtant
nous dispenser, philosophiquement, de nous situer par rapport à eux. Qu'ils
fassent partie de l’histoire de la philosophie, c’est entendu. Mais cette histoire
n’a jamais dispensé personne de philosopher. La mode, depuis plusieurs
décennies, est à la réhabilitation des sophistes. Réhabilitation sans doute
légitime. Mais faut-il pour autant, comme faisait Nietzsche, donner tort à
Socrate ?

\section{Sophistique}
%SOPHISTIQUE
Toute pensée qui se soumet à autre chose qu’à la vérité, ou
qui soumet la vérité à autre chose qu’à elle-même. C'est
considérer que la vérité n’est qu’une valeur comme une autre, réductible
comme telle au point de vue, à l'évaluation ou aux désirs qui la gouvernent. Le
mot, en ce sens, relève du langage technique : il est sans visée péjorative. Que
Protagoras et Nietzsche soient des sophistes, cela n’empêche pas qu'ils soient
aussi des génies. Toutefois cela m’empêche de les suivre absolument. « Qu'un
jugement soit faux, écrit Nietzsche, ce n’est pas, à notre avis, une objection
contre ce jugement. [..] Le tout est de savoir dans quelle mesure ce jugement
est propre à promouvoir la vie, à l’entretenir, à conserver l'espèce, voire à
l'améliorer » ({\it Par-delà le bien et le mal}, I, 4). Sophistique vitaliste. Qu’un jugement
soit faux, c’est au contraire pour moi une très forte objection contre ce
jugement ; et qu’il soit favorable à l’espèce n’y change rien.

%— 551 —
%{\footnotesize XIX$^\text{e}$} siècle — {\it }
\section{Sorcellerie}
%SORCELLERIE
Une magie mauvaise ou rituelle. On remarquera que le sorcier
est du côté du rite ; la sorcière, du côté du mal. Même
la superstition est machiste.

\section{Sottise}
%SOTTISE
« Le mal le plus contraire à la sagesse, disait Alain, c’est exactement
la sottise, j'entends l’erreur par précipitation ou prévention » ({\it Éléments
de philosophie}, VI, 8). La sottise se distingue par là de la bêtise, qui serait
plutôt le contraire de l'intelligence et se trompe par lenteur ou incapacité.

\section{Souci}
%SOUCI
C’est la mémoire de l’avenir, mais inquiète plutôt que confiante.
Disposition essentielle à l’homme, en tant qu’il est esprit (donc
mémoire) et fragilité (donc inquiétude). Heidegger avait raison de faire du
souci une structure originaire du {\it Dasein}, toujours projeté en avant de soi, toujours
préoccupé, toujours tourné vers l’avenir ou la mort. Mais les Grecs
n'avaient pas tort d’y voir le contraire de la sagesse. Il est essentiel aux humains,
et à eux seuls, de n’être pas des sages. C’est pourquoi ils ont à philosopher : pour
se rapprocher de la sagesse, au moins un peu, en s’éloignant d’eux-mêmes. En
devenant moins humains ? Non pas. Mais en devenant moins égoïstes, moins
inquiets, moins soucieux — en s’ouvrant davantage au présent, à l’action et à tout.

\section{Souhait}
%SOUHAIT
La formulation d’une espérance. Aussi ne souhaite-t-on que ce
qui ne dépend pas de nous. C’est comme une prière, mais sans
Dieu. Superstition, ou politesse. On peut se passer de la première, non de la
seconde.

\section{Souverain}
%SOUVERAIN
Le plus grand : celui qui l'emporte, ou doit l'emporter, sur
tous les autres. Par exemple le souverain bien : ce serait le bien
le plus grand, ou le bien ultime (vis-à-vis duquel les autres biens ne seraient que
des moyens). Ainsi le bonheur, selon Aristote, le plaisir, selon Épicure, ou la
vertu, selon les stoïciens. Le vrai souverain bien, s’il était possible, serait plutôt
la conjonction des trois.

Quand il est utilisé seul et comme substantif, le mot fait presque toujours
référence à la politique : il désigne le plus grand de tous les pouvoirs, dans un
territoire donné, le pouvoir premier (d’où les autres procèdent) ou ultime (celui
qui a les moyens, au moins en droit, de s’imposer à tous les autres). Concrètement,
le souverain, c’est celui qui {\it fait la loi}, comme on dit, ou qui désigne ceux
%— 552 —
%{\footnotesize XIX$^\text{e}$} siècle — {\it }
qui la font. Tel est le sens du mot chez Hobbes : le souverain est le dépositaire
de l’autorité publique, à laquelle tous, par le pacte social, ont convenu de se
soumettre ; c’est en lui que réside « l'essence de la République » ({\it Léviathan},
XVII et XVIII). Tel est le sens du mot chez Rousseau : le souverain est la République
elle-même, en tant qu’elle est active ({\it Contrat social}, I, 7), et cette souveraineté
« consiste essentiellement dans la volonté générale », qui s'exprime par
la loi ({\it ibid.}, III, 15).

Le souverain peut prendre des formes différentes : ce peut être Dieu ou le
clergé, dans une théocratie, un roi, dans une monarchie absolue, un groupe,
dans une aristocratie, ou bien, et c’est évidemment préférable, le peuple
entier, dans une démocratie — quand bien même il n’exercerait cette souveraineté,
comme presque toujours et malgré Rousseau, que par la médiation
de ses représentants. C’est bien clairement l'esprit de nos institutions, tel
qu’il est énoncé par l’article 3 de la Constitution de 1958 : « La souveraineté
nationale appartient au peuple, qui l’exerce par ses représentants et par la voie
du référendum. » Mais Hobbes, partisan de la monarchie absolue, a bien
montré qu'il ny a de monarchie ou d’aristocratie (et même, ajouterais-je, de
théocratie) qu’à la condition que le peuple d’abord y consente ; si bien que
« c’est le peuple qui règne, précise-t-il, en quelque sorte d’État que ce soit »
({\it De Cive}, XII, 8 ; voir aussi VII, 11). C’est ce qui donne raison aux démocrates
(«la démocratie, dira Marx, est l'essence de toute constitution
politique » : {\it Critique de la philosophie politique de Hegel}, I, a), mais qui ne
suffit pas à assurer leur triomphe. Combien de peuples ont préféré se soumettre
ou se démettre ?

La souveraineté, en droit, ne peut être qu’absolue. Elle cesserait autrement
d’être souveraine. Non, certes, que le peuple, dans une démocratie, ait tous les
droits. Mais parce qu’il est seul à même de les délimiter (par la constitution, par
la loi) et reste maître de modifier cette délimitation (toute constitution démocratique
prévoit les modalités démocratiques de changement de la
constitution ; sans quoi ce ne serait plus le peuple qui serait souverain mais la
constitution : on ne serait plus en démocratie mais en {\it nomocratie}). C'est pourquoi
la démocratie n’est jamais une garantie, fût-ce contre le pire ; l’histoire,
hélas, le montre assez.

Mais toute souveraineté, en fait, reste relative : c’est où l’on sort du droit
pour rentrer dans la politique, d’où le droit sort. C'est ce qu'ont compris
Machiavel et Spinoza. La souveraineté n’est qu’une abstraction, certes nécessaire,
mais qui n’en reste pas moins abstraite pour autant. La vérité, c'est qu’il
n’y a que des pouvoirs, toujours finis, toujours multiples, qui se contrecarrent
mutuellement — que des forces et des rapports de forces.

%— 553 —
%{\footnotesize XIX$^\text{e}$} siècle — {\it }
De là l’idée, chez Montesquieu et les libéraux, de la séparation des pouvoirs.
Idée légitime. Mais qui ne saurait annuler ni l’unicité de la souveraineté
(la République une et indivisible) ni la multiplicité mouvante des rapports de
forces. Entre les deux, le suffrage universel, qui est la souveraineté en acte. C’est
la mesure, en même temps que l’effectuation toujours recommencée, d’un rapport
de forces. Aucune souveraineté ne saurait dispenser les démocrates de
gagner les élections. À la gloire des partis et des militants.

\section{Spécisme}
%SPÉCISME
Ce serait l'équivalent du racisme, mais appliqué aux rapports
entre les espèces. Serait {\it spéciste} toute personne considérant que
les animaux, hommes compris, ne sont pas tous égaux en droits et en dignité.
La notion, qui part de bons sentiments (respecter les droits des animaux), est
pourtant dangereuse : c’est trop gommer la différence entre les humains et les
bêtes. Différence de degré, à ce que je crois, plutôt que de nature, mais qui justifie
qu’on les traite en effet différemment. « Si l’on n’avait pas mis les animaux
dans des wagons à bestiaux, me dit un jour une collègue, Hitler n’y aurait pas
mis les Juifs. » Sans doute. Mais cela ne fait pas du commerce de la viande
l'équivalent du nazisme, ni du nazisme un commerce.

\section{Spiritisme}
%SPIRITISME
C'est la croyance aux esprits, qui seraient les Âmes des morts,
et la prétention de les faire parler. Spiritualisme superstitieux,
ou superstition spiritualiste. L’étonnant n’est pas tant qu’une table puisse
tourner sans raison apparente (il y a bien d’autres mystères autrement étonnants),
mais que de bons esprits puissent prendre au sérieux ce qu’elle aurait à
nous dire, qui est en général d’une platitude désolante. On remarquera avec
intérêt que la table, quand c’est Victor Hugo qui la faisait tourner, parlait en
alexandrins. Mais aussi qu’elle n’a rien produit qui puisse se comparer sérieusement
aux {\it Contemplations} où à {\it La légende des siècles}.

\section{Spiritualisme}
%SPIRITUALISME
Toute doctrine qui affirme l’existence de substances pensantes
immatérielles, autrement dit d’esprits irréductibles
à quelque corps que ce soit. Le spiritualisme est le plus souvent dualiste. II
admet deux types de substances, l'esprit et la matière, ou l'âme et le corps, qui
seraient en l’homme à la fois ontologiquement distinctes et intimement unies :
ainsi chez Descartes ou Maine de Biran. Mais il peut arriver qu’il soit moniste,
s’il ne reconnaît l’existence que de substances spirituelles : c’est le cas, spécialement,
chez Leibniz ou Berkeley.

%— 554 —
%{\footnotesize XIX$^\text{e}$} siècle — {\it }
\section{Spiritualistes}
%SPIRITUALISTES
Ce sont ces gens qui prennent leur esprit ou leur âme au
sérieux : ils croient porter l’absolu au chaud de leur
conscience ! Cela ne les dispense pas d’avoir un corps. Mais justement : ils l’{\it ont},
ce qui suppose qu'ils soient autres que lui. Il est vrai qu’ils diraient aussi bien :
« J'ai une âme ». Mais quel est alors ce je qui posséderait et leur âme et leur
corps ? La somme des deux ? Mais alors il ne serait plus {\it un}. Autre chose ? Mais
alors ils ne seraient plus spiritualistes. Ce qu’ils pensent, me semble-t-il, c’est
plutôt que leur âme a un corps, que leur corps a une âme, et qu’ils sont, eux,
cette Âme ou cet esprit qu’ils connaissent de l’intérieur, dont leur corps serait
l’objet le plus proche, le plus immédiat, le plus indissociable, mais qui n’en
serait pas moins {\it objet} pour autant. Philosophes du sens intime, et du sens tout
court.

\section{Spritualité}
%SPIRITUALITÉ
La vie de l'esprit. On se trompe si on la confond avec la religion,
qui n’est qu’une des façons de la vivre. Ou avec le spi-
ritualisme, qui n’est qu’une des façons de la penser. Pourquoi les croyants
auraient-ils seuls un esprit ? Pourquoi sauraient-ils seuls s’en servir ? La spiritualité
est une dimension de la condition humaine, non le bien exclusif des
Églises ou d’une école.

Une spiritualité laïque ? Cela vaut mieux qu’une spiritualité cléricale, ou
qu'une laïcité sans esprit.

Une spiritualité sans Dieu ? Pourquoi non ? C’est ce qu’on appelle traditionnellement
la sagesse, du moins l’une de ses formes. Pourquoi faudrait-il
croire en Dieu pour que l'esprit en nous soit vivant ?

Le mot vient du latin {\it spiritus}, que les Grecs pouvaient traduire par {\it psukhê}
(l’étymologie, dans les deux langues, fait référence au souffle vital, à la respiration)
aussi bien que par {\it pneuma}. C’est dire que la frontière, entre le spirituel et
le psychique, est poreuse. L'amour, par exemple, peut appartenir aux deux. La
religion peut appartenir aux deux. La foi est un objet psychique comme un
autre. Mais c’est aussi une expérience spirituelle. Disons que tout ce qui est spirituel
est psychique, mais que tout ce qui est psychique n’est pas spirituel. Le
psychisme est l’ensemble, dont la spiritualité serait le sommet ou la pointe. En
pratique, en effet, on parle de spiritualité pour la partie de la vie psychique qui
semble la plus élevée : celle qui nous confronte à Dieu ou à l’absolu, à l'infini
ou au tout, au sens ou au non-sens de la vie, au temps ou à l'éternité, à la prière
ou au silence, au mystère ou au mysticisme, au salut ou à la contemplation.
C’est pourquoi les croyants y sont tellement à l’aise. C’est pourquoi les athées
en ont tellement besoin.

%555
%{\footnotesize XIX$^\text{e}$} siècle — {\it }
La spiritualité, pour les croyants, a un objet clairement défini (quoiqu'il
soit inconnaissable), qui serait un sujet, qui serait Dieu. La spiritualité est alors
une rencontre, un dialogue, une histoire d’amour ou de famille. « Mon Père »,
disent-ils. Est-ce spiritualité ou psychologie ? Mystique ou affectivité ? Religion,
ou infantilisme ?

L’athée est plus démuni ou moins puéril. Ce n’est pas un Père qu’il
cherche, ni qu’il trouve. Ce n’est pas un dialogue qu’il instaure. Pas un amour
qu'il rencontre. Pas une famille qu’il habite. Mais l’univers. Mais Pinfini. Mais
le silence. Mais la présence de tout à tout. Non une transcendance, donc, mais
limmanence. Non un Dieu mais l’universel devenir, qui le contient et l’emporte.
Non un sujet, mais l’universelle présence. Non un Verbe ou un sens,
mais l’universelle vérité. Qu'il n’en connaisse qu’une infime partie, cela
n'empêche pas qu’elle le contienne tout entier.

Une spiritualité sans Dieu ? Ce serait une spiritualité de l’immanence
plutôt que de la transcendance, de la méditation plutôt que de la prière, de
l'unité plutôt que de la rencontre, de la fidélité plutôt que de la foi, de l’enstase
plutôt que de l’extase, de la contemplation plutôt que de l'interprétation, de
l'amour plutôt que de l’espérance, et qui n’en déboucherait pas moins sur une
mystique, telle que je l’ai définie, c’est-à-dire sur une expérience de l'éternité,
de la plénitude, de la simplicité, de l’unité, du silence. Je ne l’ai vécu, s’agissant
de ces derniers états, que très exceptionnellement. Mais assez, toutefois,
pour que ma vie en soit définitivement transformée. Et pour que le mot de {\it spiritualité}
cesse de me faire peur.

\section{Spontanéité}
%SPONTANÉITÉ
Ce qui vient de soi-même ({\it sponte sua}, de sa propre initiative),
non d’une force extérieure ou d’une contrainte. Un
synonyme de volontaire ? Pas tout à fait, puisqu’une conduite instinctive ou
passionnelle peuvent être spontanées, sans que la volonté y soit pour rien et
parfois contre elle. Et puisqu’un acte volontaire peut n'être pas spontané
(quand je cède à la pression ou à la contrainte, j’agis bien volontairement, non
spontanément : par exemple quand je donne mon portefeuille à l'individu
armé qui me menace). La spontanéité est plutôt du côté de l’action ou de la
réaction immédiates, du désir ou de limpulsion irréfléchis, sans autre
contrainte que soi, sans autre source que soi, et qu’il y ait ou non contrôle conscient
et délibéré. La volonté peut en naître, point s’en contenter.

\section{Stoïcien}
%STOÏCIEN Qui se réclame du stoïcisme. Cela n’a jamais suffi pour être stoïque.

%— 556 —
%{\footnotesize XIX$^\text{e}$} siècle — {\it }
\section{Stoïcisme}
%STOÏCISME
École philosophique de l'Antiquité, fondée par Zénon de Citium,
renouvelée par Chrysippe, prolongée par Sénèque, Épictète et
Marc Aurèle. Son nom vient du lieu (un portique : {\it stoa}) où Zénon enseignait.
C’est dire que le fondateur n’a pas donné son nom à l’école : les stoïciens se
voulaient avant tout disciples de Socrate et des cyniques, dont ils systématisent
l’enseignement. Platon voyait en Diogène un « Socrate devenu fou » ; Zénon
serait plutôt un Diogène devenu raisonnable.

Le stoïcisme est un matérialisme volontaire et volontariste : il ne reconnaît
l'existence que des corps et n’attache de valeur qu'aux volontés. Tout ce qui ne
dépend pas de nous est moralement indifférent ; seul ce qui en dépend peut
être bien ou mal. Seule la vertu vaut donc absolument, et c’est elle, non le
plaisir, qui fait le bonheur. Le moralisme des stoïciens est ainsi à opposé de
l’hédonisme épicurien, comme leur physique continuiste est à l’opposé de l’atomisme.
Ce sont pourtant deux rationalismes. Mais la raison stoïcienne ne se
contente pas d’expliquer, comme fait l’épicurienne : elle juge, elle commande,
elle gouverne le sage comme le monde. C’est qu’elle est Dieu, ou ce qu’il y a de
divin en tout. De là cette piété stoïcienne, qui est un fatalisme mais libérateur,
et un panthéisme, mais à visée humaniste. Tout est rationnel ; à nous de
devenir raisonnables. Tout est juste ; à nous d’agir avec justice. C’est aussi un
cosmopolitisme. « La raison, qui fait de nous des êtres raisonnables, nous est
commune, écrit Marc Aurèle : c’est elle qui ordonne ce qui est à faire ou non ;
par conséquent, la loi aussi est commune ; s’il en est ainsi, nous sommes
concitoyens : nous vivons ensemble sous un même gouvernement, le monde
est comme une cité ; car à quel autre gouvernement commun pourrait-on dire
que tout le genre humain est soumis ? » ({\it Pensées pour moi-même}, IV, 4). C’est
enfin un actualisme : « Seul le présent existe », disait Chrysippe, et il suffit seul
au salut. Il n’y a donc rien à espérer : il s’agit de vouloir, pour tout ce qui
dépend de nous, et de supporter, pour tout ce qui n’en dépend pas. École de
courage, de lucidité, de sérénité. Aussi appelle-t-on {\it stoïcisme}, en un sens élargi,
tout ce qui semble relever d’une telle attitude. C’est qu’on peut être stoïque
sans être stoïcien. Pas besoin, pour faire ce qu’on doit ou supporter ce qui
advient, de croire en une quelconque providence, ni même en quelque système
que ce soit. Marc Aurèle l’a reconnu : « Si Dieu existe, tout est bien ; si les
choses vont au hasard, ne te laisse pas aller, toi aussi, au hasard » (IX, 28).

STOÏQUE Qui relève du stoïcisme, ou qui en serait digne. Le mot désigne
moins une pensée qu’une attitude. Se dit ordinairement d’un très
grand courage, spécialement contre la douleur. Nul besoin pour cela d’être
%— 557 —
%{\footnotesize XIX$^\text{e}$} siècle — {\it }
stoïcien. Épicure, face à la maladie, se montra stoïque ; mais n’en restait pas
moins épicurien pour autant.

\section{Structuralisme}
%STRUCTURALISME
Un courant de pensée, importé de la linguistique et
des sciences humaines, dont certains voulurent faire
une philosophie. Il s’agit, dans les objets qu’on étudie, de privilégier la structure
ou le système plutôt que les éléments ou leur somme, et spécialement de
penser les phénomènes humains, et l’homme lui-même, comme effets de structures
plutôt que comme création ou subjectivité. S’oppose à ce titre à l’existentialisme,
auquel il succéda, dans les années 1960, comme mode parisienne. Les
grands noms, en France, étaient Lévi-Strauss, Foucault, Lacan, Althusser…
Beaucoup de talent et de travail. L'ensemble, une fois débarrassé des oripeaux
de la mode, reste stimulant par l’ambition intellectuelle et la radicalité anti-humaniste
(au sens de l’anti-humanisme théorique). On cite souvent, pour
illustrer, le thème de {\it la mort de l'homme}, tel qu’il apparaissait, en 1966, à la
dernière page du plus fameux des livres de Foucault : « L'homme est une invention
dont l’archéologie de notre pensée montre aisément la date récente. Et
peut-être la fin prochaine » ({\it Les mots et les choses}, X, 6). Mais le texte, malgré sa
beauté, ou à cause d’elle, reste quelque peu équivoque. La maxime du structuralisme,
à mon sens, serait plutôt une formule, que je crois vraie, de Claude
Lévi-Strauss : « Le but dernier des sciences humaines n’est pas de constituer
l'homme, mais de le dissoudre » ({\it La pensée sauvage}, IX). Faut-il alors renoncer
à l’humanisme ? Pas nécessairement. Mais l’humanisme qui reste disponible est
un humanisme pratique, non théorique : il porte non sur ce que nous {\it savons} de
l’homme (qu’il fait partie de la nature : anti-humanisme théorique), mais sur ce
que nous {\it voulons} pour lui (qu’il reste humain, au sens normatif du terme). Les
sciences humaines ne sont pas humanistes. C’est donc à nous de l’être.

Le structuralisme semble d’abord s'éloigner du matérialisme : la position
d’un élément, dans une structure donnée, importe davantage que la matière
dont il est fait. Mais c’est pour déboucher sur ce que Gilles Deleuze appellera
un « nouveau matérialisme ». Ce n’est pas assez, soulignait Lévi-Strauss, que
« d’avoir résorbé des humanités particulières dans une humanité générale ; cette
première entreprise en amorce d’autres, qui incombent aux sciences exactes et
naturelles : réintégrer la culture dans la nature, et finalement la vie dans
l’ensemble de ses conditions physico-chimiques » ({\it op. cit.}, IX). L'homme n’est
pas un empire dans un empire : c’est toujours Spinoza qui renaît. S’il n’est de
sens que « de position », comme l’a montré Lévi-Strauss, si « le sens résulte toujours
de la combinaison d’éléments qui ne sont pas eux-mêmes signifiants »,
comme il dit encore, il n’y a plus de {\it sujet} du sens — ni Dieu ni homme -, ni de
%— 558 —
%{\footnotesize XIX$^\text{e}$} siècle — {\it }
{\it sens du sens} : tout sens est toujours réductible à autre chose, qui n’en a pas
(Lévi-Strauss, « Dialogue avec Ricœur », {\it Esprit}, 1963, p.637). Comme le
remarque Deleuze, qui cite ce texte, « le sens, pour le structuralisme, est toujours
un résultat, un effet: non seulement un effet comme produit, mais un effet
d'optique, un effet de langage, un effet de position » (« À quoi reconnaît-on le
structuralisme ? », in F. Châtelet, {\it Histoire de la philosophie}, t. 8). Le structuralisme
est « un kantisme sans sujet transcendantal », disait Ricœur, et Lévi-Strauss
accepte la formule. C’est qu’il n’y a plus de sujet du tout, sinon comme effet de
structures ou d'illusions : Dieu est mort, et l’homme aussi peut-être (M. Foucault,
{\it op. cit.}). Cela ne dispense pas d’être humain, ni ne suffit à le devenir.

\section{Structure}
%STRUCTURE
Du latin {\it structura}, l’arrangement, la disposition, l’assemblage.
Le mot désigne un ensemble complexe et ordonné, mais dont
l’organisation importe davantage que le contenu : c’est un système de relations
plutôt qu’une somme d’éléments, et les éléments eux-mêmes y sont moins
définis par ce qu’ils sont que par leur place dans l’ensemble et la fonction que
cette place leur assigne. Aussi y a-t-il davantage, dans un tout structuré, que la
somme de ses parties. Soit par exemple une maison. Si on ne la considère que
du point de vue de ses matériaux, elle n’est rien de plus que leur somme :
quelques milliers de briques, quelques sacs de ciment, quelques centaines de
tuiles, quelques poutres et chevrons, quelques clous, quelques tuyaux, quelques
vitres, un peu de plâtre et de peinture... Mais tout cela, entassé sur le chantier,
ne fait pas encore une maison. C’est qu’il y manque la structure, telle qu’elle est
dessinée, par exemple, sur le plan de l’architecte. On remarquera que la structure,
sans ses éléments, ne fait pas non plus une maison : on n’habite pas
davantage un plan qu’un tas de briques. Il faut donc les deux. Parler de la structure
de la maison, c’est insister sur les rapports entre les éléments, sur leur place
et leur fonction respectives, davantage que sur ces éléments eux-mêmes, lesquels
n’ont de sens ou d'utilité qu’en fonction de leur position. C’est reconnaître
qu’une maison n’est pas réductible aux matériaux qui la composent ; et
que le fait qu’elle soit construite en briques ou en pierres, sans être forcément
indifférent, importe moins que la disposition des unes ou des autres, dont
dépend aussi leur fonction. C’est pourquoi la notion de structure est spécialement
importante en linguistique : parce que les unités phoniques qu’elle rencontre
sont arbitraires, qui ne deviennent signifiantes que par leurs relations
avec d’autres, autrement dit que par leur place et leur fonction dans un
ensemble structuré (une langue). Mais c’est pourquoi aussi elle joue un rôle
majeur dans la plupart des sciences humaines : parce que rien de ce qui est proprement
humain (le langage, la culture, la politique, Part, la religion...) n’est
%— 559 —
%{\footnotesize XIX$^\text{e}$} siècle — {\it }
compréhensible indépendamment des systèmes de relations qui le rendent possible
et le constituent.

\section{Style}
%STYLE
Ce n’est pas l’homme, malgré Buffon, puisqu’un homme remarquable
peut en manquer : j'en connais quelques-uns qui écrivent
platement, et plusieurs stylistes talentueux qui m'ont semblé, à les lire ou à
les rencontrer, humainement bien médiocres. Le style est une certaine
manière d’agir, d'écrire ou de créer, qui manifeste certes une subjectivité,
mais pour autant seulement qu’elle dispose d’un certain talent, bien spécifique,
et d’un certain métier. Ce n’est pas l’homme, mais la capacité qu’il a,
et ils ne l’ont pas tous, d’inventer une expression qui lui ressemble ou le distingue
des autres. C’est ce qu’il y a de singulier dans le talent, ou de talentueux
dans la singularité. Le style est donc une force, mais aussi une limite.
Les plus grands artistes n’en ont pas, ou en ont plusieurs, ou ne cessent de
s’en libérer. Par quoi « {\it styliste} » peut devenir péjoratif. C’est attacher trop
d'importance à la forme et à la singularité. S’il avait quelque chose à dire,
attacherait-il — ou attacherions-nous — autant d'importance à la façon dont il
le dit ? Cet artiste à ce point prisonnier de sa manière ou de sa singularité,
comment serait-il universel ? Comparez par exemple Cioran, qui a un style, et
Montaigne, qui a du génie.

Que le style vaille pourtant mieux que la platitude ou la banalité, nul ne le
contestera. Mais comment le style pourrait-il suffire? Comment serait-il
l'essentiel ? Je vois bien que le Greco a un style, que Renoir a un style, que Picasso
en a plusieurs. Je ne suis pas sûr que Velézquez, qui les dépasse, en ait un.

\section{Sublimation}
%SUBLIMATION
Un changement d’état (du plus lourd au plus léger), ou
d'orientation (du plus bas vers le plus haut). Le mot, qui
désigne d’abord une élévation morale, est très vite utilisé par les alchimistes,
puis par les chimistes, pour désigner le passage d’un corps de l’état solide à
l’état gazeux. Dans la philosophie contemporaine, en revanche, c’est l’acception
freudienne qui domine. La sublimation est le processus par lequel la pulsion
sexuelle change d’objet et de niveau, trouvant ainsi à s'exprimer,
quoique indirectement, de façon socialement valorisée et en dehors de toute
satisfaction proprement érotique. C’est le cas spécialement dans l’art, dans la
pensée, dans la spiritualité, sans doute aussi dans tout amour, dès lors qu’il
ne se réduit pas à l’attirance sexuelle. Dans la sublimation, écrit Freud, « les
émotions sont détournées de leur but sexuel et orientées vers des buts socialement
supérieurs, qui n’ont plus rien de sexuel » ({\it Introduction à la psychanalyse},
%— 560 —
%{\footnotesize XIX$^\text{e}$} siècle — {\it }
1). C’est mettre les énergies du Ça au service d’autre chose, qui vaut
mieux. Au service de quoi ? De la civilisation : « C’est à l'enrichissement psychique
résultant de ce processus de sublimation, écrit Freud, que sont dues
les plus nobles acquisitions de l’esprit humain » ({\it Cinq leçons...}, V). Les désirs
infantiles peuvent ainsi « manifester toute leur énergie et substituer au penchant
irréalisable de l'individu un but supérieur, [...] un objectif plus élevé
et de plus grande valeur sociale » ({\it ibid.}), tout en procurant à l'individu des
satisfactions « plus délicates et plus élevées » ({\it Malaise dans la civilisation}, II).
Cela vaut mieux que la névrose (qui reste prisonnière des désirs infantiles
qu’elle refoule). Cela vaut mieux que la perversion (qui les satisfait). Cela
vaut mieux qu’une sexualité simplement animale (qui les ignore). C’est où
l'humanité s’invente, peut-être, en inventant des dieux. Ce n’est pas le sentiment
du sublime ; c’est le devenir sublime du sentiment.

\section{Sublime}
%SUBLIME
Ce qu'il y a de plus haut ({\it sublimis}), de plus impressionnant, de
plus admirable. Se dit surtout d’un point de vue esthétique : c’est
une beauté qui emporte ou écrase, comme si un peu d’effroi se mêlait au plaisir.
C’est qu’on se sent trop petit, face à tant de grandeur. C’est qu'on ne comprend
pas qu’une telle chose soit possible, ou comment elle Pest. C’est que
l'admiration bouscule le jeu ordinaire de nos facultés ou de nos catégories.
C’est que tant de hauteur nous élève, au moins en partie, jusqu’à nous faire
sentir douloureusement ce qui en nous reste bas ou médiocre.

« Nous nommons {\it sublime} ce qui est absolument grand », écrit Kant, « ce en
comparaison de quoi tout le reste est petit ». Aussi voulait-il que le sentiment
du sublime, même face à la nature, n’exprime que la grandeur de l'esprit
(CE, \S 23 à 29). Je dirais plutôt que le sublime est le sentiment, dans l'esprit
humain, de ce qui le dépasse, nature ou génie, et l'emporte. C’est pourquoi il
est souvent associé au beau, sans l’être toutefois nécessairement. Une tempête
est-elle belle ? Cela peut dépendre des goûts (Kant la jugeait hideuse). Mais elle
. n’en donnera pas moins le sentiment du sublime, par la démesure, par l’excès
de grandeur ou de force, par l'évidence, face à elle, de notre peritesse, de notre
impuissance, de notre fragilité. Est sublime ce qui semble absolument grand :
ce en comparaison de quoi je ne suis rien, ou presque rien. Et qui fait en moi
comme une mort heureuse.

Lors de son premier voyage en Grèce, qu’il fit tard, Marcel Conche
m'envoya une carte postale d'Athènes, représentant le Parthénon. Au dos, cette
phrase : « Si Kant avait connu le Parthénon, il n'aurait pas opposé le beau et le
sublime. » C’est que la même admiration, qui m'écrase, me réjouit.

%— 561 —
%{\footnotesize XIX$^\text{e}$} siècle — {\it }
\section{Subsomption}
%SUBSOMPTION
Le fait de subsumer, autrement dit de penser le particulier
sous le général, par exemple un objet sous un concept ou
une action sous une règle.

\section{Substance}
%SUBSTANCE
Étymologiquement, c’est ce qui est {\it sous}. Sous quoi ? Sous l’apparence,
sous le changement, sous les prédicats : la substance
est un autre mot pour dire l'essence, la permanence, le sujet, ou la conjonction
des trois.

La {\it substance} c’est l'{\it essence}, c’est-à-dire l’{\it être} : {\it ousia}, en grec, peut se traduire,
selon l’auteur ou le contexte, par l’un ou l’autre de ces trois mots. En ce sens,
seul l'être individuel est véritablement substance : lui seul est un être, à proprement
parler. Ainsi Socrate, ce caillou ou Dieu. L’humanité, la minéralité ou la
divinité ne sont que des abstractions.

La substance, c’est ce qui demeure identique à soi sous la multiplicité des
accidents ou des changements. Et il faut que quelque chose demeure, sans quoi
tout changement et tout accident deviendraient inintelligibles (puisqu'il n’y
aurait rien qui puisse changer, ni à quoi quelque chose puisse arriver). Par
exemple quand je dis que Socrate a vieilli: cela suppose qu’il est toujours
Socrate. La substance, en ce sens, est le sujet du changement, en tant que ce
sujet subsiste ou persiste.

C’est aussi le sujet d’une proposition : « ce dont tout le reste est affirmé,
comme dit Aristote, mais qui n’est pas lui-même affirmé d’autre chose » ({\it Méta-physique},
Z, 3). Le sujet de tous les prédicats, donc, qui n’est prédicat d’aucun
sujet. Par exemple quand je dis que Socrate est juste ou se promène. Ni la justice
ni la promenade ne sont des substances : ce ne sont que des prédicats, attribués
à une substance (en l’occurrence Socrate), laquelle ne saurait être le prédicat
d'aucune substance. C’est cette acception logique du mot qui explique
qu’Aristote, à propos des termes généraux, parle parfois de «substances
secondes » : l’homme ou l’humanité peuvent être le sujet d’une proposition,
qui leur attribue tel ou tel prédicat. Mais ce ne sont des substances que par
analogies : seuls les individus sont des « substances premières », c’est-à-dire des
substances proprement. C’est où Aristote s’écarte de Platon, et tel est peut-être
le principal enjeu de cette notion, aujourd’hui vieillie, de substance.

Chez Kant, la substance est l’une des trois catégories de la relation. Elle est
ce qui ne change pas dans ce qui change. Son schème est « la permanence du
réel dans le temps ». Son principe, la « première analogie de l'expérience » :
« Tous les phénomènes contiennent quelque chose de permanent ({\it substance})
considéré comme l’objet lui-même, et quelque chose de changeant, considéré
comme une simple détermination de cet objet, c’est-à-dire un mode de son
%— 562 —
%{\footnotesize XIX$^\text{e}$} siècle — {\it }
existence » ({\it C. R. Pure}, Analytique des principes). Ou bien, dans la deuxième
édition : « La substance persiste dans tout le changement des phénomènes, et sa
quantité n’augmente ni ne diminue dans la nature. » Il y a bien longtemps que
cette permanence a cessé, pour nos physiciens, d’être une évidence. Si quelque
chose se conserve, c’est l’énergie. Mais ce n’est pas une chose, ni un être individuel,
ni un sujet (si ce n’est au sens purement logique du terme). Quel sens y
aurait-il à y voir une substance ?

\section{Subsumer}
%SUBSUMER
Mettre sous ou dans. C’est inscrire un être ou une catégorie
dans un ensemble plus général. Par exemple Socrate peut être
subsumé sous le concept d'homme, qui peut à son tour être subsumé sous celui
de mammifère, qui peut l’être à son tour sous celui d’animal... On n’y gagne
ordinairement pas grand-chose : « On échange un mot pour un autre mot,
comme dit Montaigne, et souvent plus inconnu. Je sais mieux ce que c’est
qu’homme que je ne sais ce que c’est qu’animal, ou mortel, ou raisonnable.
Pour satisfaire à un doute, ils m'en donnent trois : c’est la tête de l’hydre »
(III, 13, p. 1069). C’est dire qu'aucune subsomption ne saurait suffire à
définir : l’emboîtement des généralités importe moins que l’enchaînement des
causes, des idées ou des expériences.

\section{Suggestion}
%SUGGESTION
C'est agir sur quelqu'un par des signes, sans avoir besoin
pour cela de le convaincre. C’est une espèce de magie, ou
plutôt la magie, presque toujours, n’est qu’une espèce de suggestion.

La suggestion culmine dans l’hystérie et l'hypnose, mais se manifeste, à des
degrés divers, dans tout groupe humain. Cet homme qui bâille me fait bäiller.
Cet autre qui me trouve mauvaise mine ou m’annonce ma mort me rend
presque malade, voire malade tout à fait. C’est que je suis sous influence, sans
le vouloir, parfois sans le savoir. C’est la suggestion même : une influence que
lon subit involontairement, et qui passe moins par la raison ou la volonté que
par limitation ou la soumission. Les individus y sont plus ou moins sensibles.
C’est pourquoi «il n’est permis de parler librement, disait Alain, qu’à celui
dont on prévoit qu’il résistera librement ». C’est refuser la magie ou la manipulation.

\section{Suicide}
%SUICIDE 
L’homicide de soi. C’est pourquoi certains y voient un crime,
C’est pourquoi j'y vois un droit. « Comme je n’offense les lois qui
sont faites contre les larrons quand j’emporte le mien et que je me coupe ma
%— 563 —
%{\footnotesize XIX$^\text{e}$} siècle — {\it }
bourse, écrit Montaigne, ni des boutefeux [les incendiaires] quand je brûle
mon bois, aussi ne suis-je tenu aux lois faites contre les meurtriers pour m'avoir
ôté la vie » ({\it Essais}, II, 3). Attention toutefois de ne pas faire du suicide plus de
cas qu’il ne convient. Ce n’est ni un sacre ni un sacrement. Ni une morale ni
une métaphysique. Se suicider, c’est choisir non la mort (c’est un choix que
l’on n’a pas : il faudra mourir de toute façon) mais {\it le moment} de sa mort. C’est
un acte tout d'opportunité, et point l’absolu parfois qu’on veut y voir. Il s’agit,
ni plus ni moins, de gagner du temps sur l’inévitable, de devancer le néant, de
prendre le destin, si l’on veut, de vitesse. C’est le raccourci définitif.

C’est aussi un droit, pour chacun, d’autant plus absolu qu’il se moque du
droit. « Le présent que nature nous ait fait le plus favorable, écrit encore Montaigne,
c'est de nous avoir laissé la clef des champs » ({\it ibid.}). C’est la liberté
minimale et maximale.

\section{Sujet}
%SUJET
Ce qui est {\it jeté sous} ou {\it sous-jacent}. C’est un équivalent, au moins pour
l'étymologie, de « substance » ou d’« hypostase ». Mais l’étymologie
n'a jamais suffi à faire une définition.

Qu'est-ce qu’un sujet ? Pour la logique, c’est un être quelconque, dès lors
qu'on lui attribue un prédicat. Par exemple quand je dis « La Terre est ronde » :
{\it Terre} est le sujet, ronde le {\it prédicat}. On voit que le mot, en cette acception, ne
dit rien sur la nature du sujet, sinon qu’il est un être (une substance) ou considéré
comme tel (une entité).

Pour la philosophie politique, le {\it sujet} s'oppose au {\it souverain}, comme ce qui
obéit à ce qui commande, et au {\it citoyen}, comme ce qui n’est pas libre à ce qui
l'est. Cela n'exclut pas que les mêmes individus soient à la fois sujets (membres
du peuple) et citoyens (membres du souverain) ; mais ne le garantit pas non
plus.

Dans la philosophie moderne, le mot touche davantage à la théorie de la
connaissance et à la morale, voire à la métaphysique : le sujet s’oppose à l’objet
comme ce qui connaît à ce qui est connu, ou comme ce qui veut et agit à ce qui
est fait. Ce serait l’être humain, ou plutôt une certaine façon de le penser :
comme le sujet (au sens à la fois grammatical et ontologique) de sa pensée et de
sa vie. C’est en ce sens qu’on parle — à propos de Descartes, de Kant, de
Sartre... — d’une « {\it philosophie du sujet} ». Le sujet, c’est celui qui dit {\it je}, pour
autant qu'il se désigne légitimement par là : c’est celui qui pense ou agit, mais
en tant qu'il serait le principe de ses pensées ou de ses actes, plutôt que leur
somme, leur flux ou leur résultat. Par exemple chez Descartes : « {\it Je pense, donc
Je suis} ». C’est que {\it penser} est un verbe, qui suppose un sujet. Métaphysique de
grammairien, qui prend le langage pour une preuve ou la grammaire pour une
%— 564 —
%{\footnotesize XIX$^\text{e}$} siècle — {\it }
métaphysique. Autant dire « {\it Il pleut, donc il est} ». Cela ne fera pas de la pluie
un sujet.

Hume, avant Nietzsche, a montré, dans un chapitre génial de son {\it Traité},
que nous n’avons aucune expérience, ni donc aucune connaissance, d’un tel
{\it sujet} : que nous ne connaissons de nous-mêmes « qu’un faisceau ou une collection
de perceptions différentes qui se succèdent les unes aux autres avec une
rapidité inconcevable », sans que rien nous autorise à penser que nous sommes
autre chose que leur flux ni, donc, que nous en sommes la cause, la substance
ou le principe sous-jacents ({\it Traité de la nature humaine}, I, IV, 6, « L'identité
personnelle »). C’est un des rares textes où la philosophie occidentale, sans le
savoir, s'approche du bouddhisme. Pas de sujet, pas de moi, sinon illusoire, pas
de soi ({\it anatta}) : tout n’est que flux et agrégats, qu’impermanence et processus
({\it paticca-samuppada} : production conditionnée). « Seule la souffrance existe,
mais on ne trouve aucun souffrant ; les actes sont; mais on ne trouve pas
d’acteur. Il n’y a pas de moteur immobile derrière le mouvement. Il n’y a pas
de penseur derrière la pensée » (W. Rahula, {\it L'enseignement du Bouddha}, 2 ; voir
aussi le chap. 6). Le sujet n’en demeure pas moins, comme croyance, comme
illusion, comme {\it mot} ; mais il n’explique rien. Il n’est pas ce que nous sommes,
mais ce que nous croyons être. Non une substance, mais une hypostase. Non
notre vérité, mais notre méconnaissance (l’ensemble des illusions que nous
nous faisons sur nous-mêmes). Non le principe de nos actes ou de nos pensées,
mais leur enchaînement, qui nous enchaîne. Non un principe, mais une histoire.
Non notre liberté de sujet, mais notre {\it assujettissement}.

On n’en conclura pas qu’il faudrait pour cela renoncer à la liberté. Mais
que la subjectivité ne saurait y suffire : seule la vérité libère, qui n'est pas un
sujet.

Philosophie non plus du sujet mais de la connaissance. Non plus de la
liberté, mais de la libération.

\section{Superstition}
%SUPERSTITION
C’est donner du sens à ce qui n’en a pas. Par exemple un
chat noir, un rêve, une éclipse. Notion polémique, donc
relative : on est toujours le superstitieux de quelqu'un, qui se prétend le seul
herméneute légitime. C’est en quoi la superstition se distingue de la religion,
du moins pour les croyants (parce qu’elle invente de faux signes ou de faux
dieux), et tend à l’absorber, pour les athées (puisque aucun Dieu n’est le vrai,
ni aucun sens).
On dira que la psychanalyse donne bien un sens aux rêves ou aux symptômes,
sans relever pour autant de la superstition. Sans doute. Mais c’est que
ce sens n’est que l’envers d’un processus causal : rêves et symptômes ne sont des
%— 565 —
%{\footnotesize XIX$^\text{e}$} siècle — {\it }
{\it symboles} que parce qu’ils sont d’abord des {\it indices} ou des {\it effets}. Aussi n’ont-ils
rien de surnaturel : leur interprétation finit par déboucher sur quelque chose —
la sexualité, l'inconscient — qui est dépourvu de toute signification transcendante,
et même immanente. La sémiologie renvoie à une étiologie, qui
l'explique et la borne. Il n’y a pas de sens du sens, ni de sens absolu, ni de sens
ultime : il n’y a que le réel et la pulsion, qui ne signifient rien. C’est où Freud
est le contraire d’un superstitieux, et la psychanalyse, le contraire d’une religion.
Toute superstition soumet le réel au sens : elle explique ce qui est (un
rêve, une éclipse, un chat noir) par ce que cela veut dire (par exemple un malheur
à venir). L'analyse fait l'inverse. Elle soumet le sens au réel : elle explique
ce que cela veut dire (le sens d’un rêve, d’un acte manqué, d’un symptôme) par
ce qui est (un désir refoulé, un traumatisme, une névrose). La superstition
donne du sens à ce qui n’en a pas ; la psychanalyse ramène le sens à autre chose,
qui le dissout. C’est pourquoi l’on se trompe quand on demande à la psychanalyse
le sens de sa vie. Elle ne peut donner que le sens de nos symptômes ou
de nos rêves. Ou bien ce n’est plus analyse mais superstition. Plus connaissance
(de mon histoire) mais religion (de mon inconscient). Pauvres petits analysants,
qui cherchent un sens ! Freud, lui, ne cherchait que la vérité. On dira que les
deux sont liés, que telle est la {\it voie royale} de la psychanalyse. Reste, toutefois,
à ne pas la prendre à contresens. Freud est le contraire d’un prophète. Il
n'annonce pas, il explique. Il ne parle pas, il écoute. Le sens, chez lui, n’est
qu'un chemin vers la vérité. C’est toujours superstition, à l'inverse, que de ne
voir dans la vérité qu’un chemin vers le sens. L’inconscient parle, sans doute ;
mais il n’a rien à dire. La cure est de paroles (c’est une « {\it talking cure} ») ; mais la
santé, de silence.

Notons pour finir que toute superstition tend à se vérifier. Celui qui casse
un miroir et s’en effraie, sa crainte confirme déjà le présage qui l’inspire. La
superstition porte malheur.

\section{Surhumain}
%SURHUMAIN
Ce qui excède la mesure humaine. Nietzsche y voyait un but
et un sens (« le sens de la terre ») : l’homme n’existerait que
pour être dépassé ; le surhomme serait à l’homme ce que l’homme est au singe
({\it Zarathoustra}, T, Prologue). Montaigne, contre les stoïciens et plus raisonnablement,
n’y voyait qu’une sottise : « À la vile chose, dit Sénèque, et abjecte que
l’homme, s’il ne s'élève au-dessus de l'humanité ! Voilà un bon mot et un utile
désir, mais pareillement absurde. Car de faire la poignée plus grande que le
poing, la brassée plus grande que le bras, et d’espérer enjamber plus que
l'étendue de nos jambes, cela est impossible et monstrueux. Ni que l’homme se
%— 566 —
%{\footnotesize XIX$^\text{e}$} siècle — {\it }
monte au-dessus de soi et de l'humanité » (II, 12, p. 604). Être pleinement
humain fait une tâche suffisante.

\section{Surmoi}
%SURMOI
L'une des trois instances (avec le {\it moi} et le {\it ça}) de la seconde
topique de Freud : c’est l'instance de la moralité, des idéaux, de la
Loi. Elle résulte de l’intériorisation des interdictions et valorisations parentales.
Ce qu’ils nous ont interdit, voilà que nous nous l’interdisons à nous-mêmes ;
ce qu’ils nous ont imposé, voilà que nous nous l’imposons ; ce qu’ils aimaient,
voilà que nous le jugeons aimable. Ce n’est pas toujours le cas ? Certes, puisque
cette intériorisation ne se fait ni toujours ni complètement. C’est pourquoi
nous n'avons pas tout à fait la même morale que nos parents. Il n’en reste pas
moins que chaque génération éduque la suivante, et que toute morale pour cela
vient du passé. Il n’y a pas de morale de l’avenir : il n’y a de morale que présente —
que fidèle et critique. On aurait tort d’en faire un absolu, voire d’y
croire tout à fait. Mais tort aussi de prétendre s’en exempter. Il n’est pas
interdit d'interdire, et même il est interdit de ne s’interdire rien.

Le {\it surmoi} représente le passé de la société, explique Freud, de même que
le {\it ça} représente le passé de l'espèce. Ce n’est pas une raison pour les juger l’un
et l’autre réactionnaires. Sans le ça, pas d’avenir. Sans le surmoi, pas de progrès.

\section{Surnaturel}
%SURNATUREL
Qui excéderait la puissance de la nature. Ce ne peut être
que de la magie, de la superstition ou de la religion, et c’est
pourquoi, pour un matérialiste, ce n’est rien.

\section{Syllogisme}
%SYLLOGISME
Un type de raisonnement déductif, formalisé par Aristote,
qui unit trois termes, reliés deux à deux et dont chacun apparaît
deux fois, en trois propositions. Il est susceptible de plusieurs formes ou
figures différentes. Mais l'exemple canonique, qu’on ne trouve pas chez Aristote,
est le suivant :
\begin{center}
{\it Tout homme est mortel ;

Socrate est un homme ;

Donc Socrate est mortel.}
\end{center}
Les deux premières propositions sont les prémisses (majeure et mineure) ;
la troisième, la conclusion. Les trois termes (mortel, homme, Socrate) sont
appelés respectivement grand terme, moyen terme et petit terme. On remarquera
que l’ordre des prémisses n’est pas ce qui importe, ni même toujours leur
%— 567 —
%{\footnotesize XIX$^\text{e}$} siècle — {\it }
extension. Le grand terme est celui qui sert de prédicat dans la conclusion ; la
majeure, celle des deux prémisses qui contient le grand terme. Le petit terme
est celui qui sert de sujet dans la conclusion ; la mineure, celle des deux prémisses
qui contient le petit terme. Enfin le moyen terme est le seul à apparaître
dans les deux prémisses : c’est lui qui les met en rapport et permet la conclusion,
où il ne figure pas.

Le syllogisme est-il valide ? Cela dépend bien sûr des prémisses. De leur
vérité ? Non pas. Certes, la conclusion n’est nécessairement vraie que si les prémisses
le sont ; mais cela touche moins à la validité du raisonnement qu’au
contenu des trois propositions. Soit par exemple ce syllogisme, qu’on trouve
chez Lewis Carroll :

\begin{center}
{\it 
Tous les chats comprennent le français ;

Quelques poulets sont des chats ;

Donc quelques poulets comprennent le français.}
\end{center}

Que l’inférence soit formellement valide, cela ne garantit nullement la
vérité de la conclusion. Mais l’inverse est vrai aussi : que la conclusion puisse
être fausse (puisque les prémisses le sont) n’annule pas la validité, au moins formelle,
du raisonnement. On s’en rend compte si l’on donne au syllogisme, à la
façon d’Aristote, la forme d’une implication. La proposition « {\it Si tous les chats
comprennent le français et si quelques poulets sont des chats, alors quelques poulets
comprennent le français} » est une proposition vraie. L'essentiel, d’un point de
vue logique, n’est pas dans le contenu des propositions, mais dans la légitimité
de l’inférence. Deux prémisses fausses, comme dans l'exemple de Lewis Carroll,
peuvent justifier cette inférence, alors que deux prémisses, même vraies, n’autorisent
pas forcément à conclure. D’abord parce qu’il faut qu’elles aient un
moyen terme commun, dans un même genre : Que tous les hommes soient
mortels et que Milou soit un chien, cela peut faire l’objet de deux propositions,
mais pas constituer les deux prémisses d’un syllogisme. Ensuite parce que nos
deux prémisses doivent encore respecter un certain nombre de règles, qu’on
trouve dans les manuels. Par exemple : « De deux propositions particulières, on
ne peut rien conclure » (que certains hommes soient mortels et que certains
philosophes soient des hommes, c’est assurément vrai, mais ne permet pas de
savoir si les philosophes sont mortels, ni lesquels). Ou encore : « De deux propositions
négatives, on ne peut rien conclure » (qu'aucun homme ne soit
immortel et que Socrate ne soit pas un chien, c’est très vraisemblable, mais ne
nous dit rien sur la mortalité de Socrate). Ces règles, qui sont nombreuses, sont
aussi passablement oubliées. Je ne connais guère de philosophes qui s’en servent.
Mais j’en connais encore moins, en tout cas parmi les bons, qui les transgressent.

%— 568 —
%{\footnotesize XIX$^\text{e}$} siècle — {\it }
\section{Symbole}
%SYMBOLE
Parfois synonyme de signe, voire (par influence de l’anglo-américain,
et surtout depuis Peirce) de signe conventionnel : c’est en
ce sens, par exemple, qu’on parle de symboles mathématiques.

Mais la langue résiste. Un feu rouge n’est pas un symbole. Un mot n’est pas
un symbole. La colombe en est un, pour la paix, comme la balance, pour la justice.
Les signes mathématiques ? Cela dépend lesquels. Les signes +, — ou $\surd$ ne
sont pas des symboles ; les signes > ou < en sont, même pauvres et au moins
partiellement.

Qu'est-ce qu’un symbole ? C’est un signe non arbitraire et non exclusivement
conventionnel, dans lequel le signifiant (par exemple l’image d’une
colombe, ou l’image d’une balance) et le signifié (par exemple l’idée de paix, ou
l’idée de justice) sont unis par un rapport de ressemblance ou d’analogie. Un
faucon ne ferait pas aussi bien l'affaire, ou symboliserait autre chose. C’est qu'il
y a en effet quelque chose de paisible, ou que nous jugeons tel, dans l'aspect ou
le comportement des colombes. Et que la balance doit être {\it juste}, c’est-à-dire
respecter une forme d’égalité ou de proportion entre ses deux plateaux. C'est
pourquoi les symboles sont souvent suggestifs : parce qu’ils unissent le sensible
et l’intelligible, l'imaginaire et la pensée. Aussi convient-il, en philosophie, de
s’en méfier. Le meilleur symbole ne remplacera jamais un argument.

\section{Sympathie}
%SYMPATHIE
C’est sentir avec, ensemble ou de la même façon. Le mot dit
la même chose, en grec, que {\it compassion} en latin. Mais cela ne
fait pas, en français, deux synonymes. C’est que la sympathie est affectivement
neutre : on peut sympathiser dans la joie comme dans la tristesse. Alors que la
compassion, en français, ne se dit que négativement : on compatit avec la souffrance
ou le malheur d’autrui, non avec sa joie ou son bonheur. C’est ce qui
rend la sympathie plus sympathique, plus plaisante, et plus équivoque. Qui
voudrait partager la joie du méchant ou le plaisir du tortionnaire ? Toute souffrance
mérite compassion. Toute joie ne mérite pas sympathie.

\section{Symptôme}
%SYMPTÔME
C’est un effet qui désigne sa cause. D’où l'illusion d’un sens,
quand il faudrait n’y voir que causalité. Une fièvre ne veut rien
dire. Mais elle a une cause, qu’on peut connaître et combattre.

Le mot sert surtout pour désigner les signes — c’est-à-dire les effets observables
et reconnaissables — des maladies. Mais il est susceptible, spécialement
depuis Freud, d’un emploi plus vaste. Vous bâillez ? C’est un symptôme de
fatigue, ou d’ennui, ou de dépression. Vous avez oublié votre parapluie ?
Symptôme. Vous êtes en retard ? Symptôme. En avance ? Symptôme. Juste à
%— 569 —
%{\footnotesize XIX$^\text{e}$} siècle — {\it }
l'heure ? Symptôme. C’est la forme moderne de l’herméneutisme ou de la
superstition. Tout est symptôme en nous, ou peut l'être, qu’on interprète, faute
d'en connaître les causes, pour en comprendre le sens. Et en effet c’est toujours
possible. Mais à quoi bon ? Si tout est symptôme, et l'interprétation elle-même,
c'est que tout a une cause, qu’on ignore le plus souvent et qu’on ne peut en
conséquence qu'imaginer.. Labyrinthe de l'imaginaire, dont la vérité seule,
dirait Spinoza, guérit. La santé est le silence des organes. La sagesse, le silence
de l'esprit. Tout peut avoir un sens, mais le sens n’en a pas. Tout peut s’interpréter,
mais à proportion seulement de l’ignorance qu’on en a.

\section{Syncrétsme}
%SYNCRÉTISME
C’est comme un éclectisme sans choix ou sans rigueur : la
juxtaposition de plusieurs thèses mal coordonnées, empruntées
à des doctrines incompatibles ou disparates.

Piaget et Wallon ont appelé {\it syncrétisme} une tendance de la perception et de
la pensée du jeune enfant, qui perçoit l’ensemble plutôt que les détails : il relie
tout à tout de façon globale et confuse. La précision et la rigueur ne viendront
que plus tard.

\section{Syndrome}
%SYNDROME
Ensemble de symptômes. Une maladie ? Pas forcément. Pas toujours.
Ce pourrait être plusieurs maladies différentes. Ce sera
au diagnostic d’en décider. Le syndrome est son point de départ, qui relève de
l'observation. La maladie, comme concept, serait plutôt son point d’arrivée,
qui relève de l'explication. La guérison ? Ce n’est plus diagnostic, mais pronostic
ou thérapie.

SYNTHÈSE {\it Sunthesis}, en grec, c’est la réunion, la composition, l’assemblage :
synthétiser, c’est poser ({\it tithenai}) ensemble ({\it sun}). La synthèse
s’oppose par là à l'analyse, qui sépare ou décompose.

La synthèse va du simple au composé, disait Leibniz. Elle constitue ou
reconstitue un tout à partir d'éléments déjà donnés. L'analyse va du composé
au simple. Elle décompose un tout en ses éléments. Cela peut s'entendre
chimiquement : on peut produire une eau de synthèse, en assemblant des
atomes d’oxygène et d'hydrogène. Et obtenir par analyse, à partir d’une eau
quelconque, des atomes d'oxygène et d'hydrogène.

Mais le mot « synthèse », en philosophie, sert surtout à désigner un processus
intellectuel, qui ne compose guère que des idées. C’est le cas, spécialement,
chez Descartes. La synthèse constitue, après celles de l'évidence et de
%— 570 —
%{\footnotesize XIX$^\text{e}$} siècle — {\it }
l'analyse, la troisième règle de sa méthode : « Conduire par ordre mes pensées,
en commençant par les objets les plus simples et les plus aisés à connaître, pour
monter peu à peu, comme par degrés, jusqu'à la connaissance des plus
composés » ({\it Discours de la méthode}, II). Il en résulte que l’analyse est première
(le simple n’est pas donné d’abord : il faut le conquérir), et plus proche de « la
vraie voie par laquelle une chose a été méthodiquement inventée » (Réponses
aux IF objections, AT, 121). La synthèse sert surtout à démontrer ce qu’on
connaît déjà : c’est une méthode d’exposition plutôt que de découverte. Comparez
par exemple, chez Descartes, les {\it Méditations}, qui sont écrites selon l’ordre
analytique, et les {\it Principes de la philosophie}, qui suivent l’ordre synthétique.

Mais la synthèse n’est pas seulement un ordre ou une méthode. C’est
aussi un moment, chez Hegel ou Marx, de la dialectique : celui où les deux
contraires sont réunis en un troisième terme, qui les dépasse (c’est-à-dire les
supprime tout en les conservant). Ainsi le devenir, après l’être et le néant
(Hegel, {\it Logique}, I, 1). Le fruit, après la graine et la plante (Engels,
{\it Anti-Dühring}, XI). Ou le communisme, après la lutte des classes (le prolétariat « ne
l'emporte qu’en s’abolissant lui-même et en abolissant son contraire », qui est
la propriété privée : Marx et Engels, {\it La sainte famille}, IV ; voir aussi {\it Le Capital},
I, 8$^\text{\,e}$ section, chap. 32). Négation de la négation : nouvelle affirmation. C’est
un beau moment, dans le travail du négatif : celui du repos, mais en mouvement,
et de la fécondité. Trop beau pour être vrai ? C’est ce qu’on peut penser.
On ne lui est pas fidèle, en tout cas, en le réduisant, comme souvent nos étudiants
dans leurs dissertations, à une apologie du juste milieu, de la mollesse ou
de l’indécision. Je pense, on l’a compris, au fameux plan « Thèse, antithèse,
synthèse ». C’est un plan comme un autre, qui n’est ni obligatoire ni interdit.
Encore faut-il le comprendre dialectiquement. {\it Thèse}, {\it antithèse}, {\it synthèse}, cela ne
saurait vouloir dire : {\it Blanc}, {\it Noir}, {\it Gris}. Et pas davantage : {\it Oui}, {\it Non}, {\it Peut-être}.
Ou bien ce n’est plus une synthèse, mais une échappatoire ou un compromis.
Plus de la dialectique, mais du chewing-gum.

\section{Synthétiques (Jugements —)}
%SYNTHÉTIQUES (JUGEMENTS -)
Ce sont les jugements qui « ajoutent
au concept du sujet un prédicat qui
n'était pas du tout pensé dans le sujet, et qu'aucune analyse de celui-ci n'aurait
pu en tirer » (C. R Pure, introd., IV). S’opposent aux jugements analytiques.
Par exemple, explique Kant, « {\it Tous les corps sont étendus} » est un jugement analytique
(la notion d’étendue est incluse dans celle de corps: un corps sans
étendue serait contradictoire) ; alors que « {\it Tous les corps sont pesants} » est un
jugement synthétique (la notion de poids n’est pas comprise dans celle de
corps : l’idée d’un corps sans poids n’est pas intrinsèquement contradictoire ;
%— 571 
%{\footnotesize XIX$^\text{e}$} siècle — {\it }
seule l’expérience nous apprend qu’ils en ont tous un). Les jugements d’expérience
sont tous synthétiques, souligne Kant, mais tous les jugements synthétiques
ne sont pas d'expérience. Il en est certes qui sont {\it a posteriori} (c’est le cas
de « Tous les corps sont pesants »), mais d’autres qui sont {\it a priori} (« {\it Tout ce qui
arrive a une cause} »). Expliquer la possibilité de ces derniers — donc la possibilité
des sciences — est l’un des enjeux principaux de la {\it Critique de la raison pure}.

\section{Système}
%SYSTÈME
Un assemblage ordonné, où chaque élément est nécessaire à la
cohésion de l’ensemble et en dépend. Ainsi parle-t-on du système
nerveux, du système solaire, d’un système informatique. En philosophie, se
dit le plus souvent d’un ensemble d’idées, « mais en tant qu’on les considère
dans leur cohérence, comme dit Lalande, plutôt que dans leur vérité ». C’est
que la pluralité même des systèmes, qui sont incompatibles (puisque chacun
prétend dire la vérité sur le tout), interdit de les accepter tous comme de se
satisfaire de l’un d’entre eux. Les systèmes sont tous faux, disait Alain, et le système
des systèmes, qui serait l’hégélianisme (parce qu’il fait de la contradiction
entre les systèmes le moteur de son propre développement) l’est tout autant. La
cohérence n’est pas une preuve. Elle n’est même pas un argument. Combien de
délires cohérents ? La paranoïa, disait Freud, est « un système philosophique
déformé » ; un système philosophique, ajouterais-je volontiers, est une paranoïa
réussie. Jaime mieux les contradictions de la vie et les aspérités du réel.

Que beaucoup de grandes philosophies soient des systèmes n’est pourtant
pas niable. Presque toutes y tendent, et pour des raisons nécessaires. Il faut bien
faire tenir ensemble ce qu’on croit vrai. Il faut bien penser tout, ou le tout. Le
système est l'horizon de la philosophie : c’est une pensée où tout se tient,
comme une synthèse supérieure, comme un tout organique, et cela vaut mieux
qu'une pensée qui se délite ou se contredit. Reste à n’en être pas dupe, à ne pas
prendre cette cohérence pour une preuve, à ne pas s’y enfermer. Pourquoi faudrait-il
te soumettre à ce que tu as déjà pensé ? C’est la vérité qui importe, non
la cohérence. La pensée d’aujourd’hui, non celle d’hier. Le réel, non le système.
Quelle tristesse que de ne penser que pour se donner raison ! Quelle folie que
de prétendre posséder l’horizon ! Les sciences donnent un meilleur exemple,
qui font tout pour être contredites, et qui avancent par là. Platon donne un
meilleur exemple, et Montaigne, et Pascal. L’horizon est devant nous ; il faut
donc avancer. Le système serait la philosophie de Dieu. Mais Dieu n’est pas
philosophe.

Il y a quelque chose de pathétique chez les auteurs de système. Ils croient
penser le tout ; ils ne font que bricoler leurs petites idées. Comment pourraient-ils
contenir l’univers, puisqu'ils en font partie ? Le monde continue sans
%— 572 —
%{\footnotesize XIX$^\text{e}$} siècle — {\it }
eux. La philosophie continue sans eux, et c’est tant mieux. Si un système réussissait,
c’en serait fini de la philosophie. Mais ils ont tous échoué, même les plus
grands. Le cartésianisme est mort. Le leibnizianisme est mort. Le spinozisme est
mort. Raison de plus pour lire Descartes, Leibniz ou Spinoza, qui valent mieux
que leurs systèmes. Battez les cartes et les idées. Le jeu n’est pas fait ; il est à
faire.

Un système philosophique est comme un château de cartes : vous en retirez
une, tout le reste s’écroule. C’est que chaque carte ne tient que par l’ensemble,
et le tient. J'aime mieux le jeu ouvert et vivant, les cartes qui volent, les joueurs
qui s'affrontent, les plis qui se font ou se défont, jusqu’à la victoire, jusqu’à la
défaite, jusqu’à la prochaine partie. J’aime qu’on batte les cartes à chaque fois.
Qu'on invente ses coups à chaque fois, en fonction du jeu et des adversaires.
Que chaque partie soit neuve et incertaine. L’infini est là, non dans le château
de cartes dérisoire et figé.

