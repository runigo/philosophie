%{\footnotesize XIX$^\text{e}$} siècle — {\it }

\chapter{D}

DANDYSME C’est une esthétique qu’on prend pour une éthique, ou qui
voudrait en tenir lieu. Le dandy rêve de faire de sa vie une
œuvre d'art. Il n’y parvient pas et s’en console comme il peut, par coquetterie
et dérision. L’élégance lui paraît une vertu suffisante ; suivre la mode, la seule
sagesse nécessaire. C’est un virtuose de l’apparence. Son corps, ses vêtements, sa
parole... sont autant d’instruments, qu’il utilise pour se faire valoir ou remarquer.
Le dandysme, écrit Baudelaire, est « une espèce de culte de soi-même ».
C’est sa limite. Quel dieu plus dérisoire ? Le dandy n’échappe au ridicule que
par l'humour. S’il se prend au sérieux, ce n’est plus un dandy : c’est un snob.

DANSE C’est une espèce de gymnastique, mais qui serait aussi un art : elle
tend moins à la santé qu’au plaisir, moins à la force qu’à la beauté
ou à la séduction. La musique ordinairement l’accompagne, ou même la suscite.
C’est que le rythme y est roi, et cela fait partie de sa définition : la danse
est un mouvement rythmé des corps, quand il se suffit de soi ou de sa beauté.
De là une impression de liberté paradoxale (puisqu'il n’est pas de danse sans
contraintes), que ceux qui ne dansent pas envient et voudraient imiter. Ce qui
leur manque ? Un peu de talent, d’inconscience ou d’abandon.

DARWINISME Ia théorie de Darwin et de ses disciples : c’est une façon
d’expliquer l’apparente finalité du vivant par l’évolution des
espèces et la sélection naturelle — par le jeu aveugle des mutations, de la reproduction
et de la mort. On en retient surtout que l’homme descend du singe, ou
que les singes, à tout le moins, sont nos cousins. Le scandale retombé, c’est
%— 143 —
devenu une espèce d’évidence, qui nous réintroduit heureusement dans la
nature. Reste un type d’explication plus général, qui fait aujourd’hui l’enjeu
essentiel du darwinisme. C’est une façon d’expliquer l’ordre par le désordre, et
le sens, au moins apparent, par le hasard. Un changement survient, sur tel ou
tel chromosome : il a d’autant plus de chances de se répandre qu’il est plus
favorable à l’espèce (il constitue un avantage sélectif), d'autant moins qu’il lui
est plus néfaste (auquel cas ceux qui le possèdent voient leur vie ou leur fécondité
diminuer). Même chose si c’est l’environnement qui change : certaines
espèces vont disparaître, d’autres ne s’en multiplieront que mieux, et les jobards
pourront toujours s'étonner, avant comme après, de l’étonnante harmonie qui
règne dans la nature, malgré les massacres ou plutôt grâce à eux... Certains y
verront une main invisible. Cela ressemble plutôt à une loterie paradoxale, où
les gagnants seraient le gros lot. Comment, s’ils ont une métaphysique, ne croiraient-ils
pas en Dieu ? C’est ériger la loterie en religion. De cette superstition
habituelle des gagnants, le darwinisme heureusement nous dissuade. La nature
ne choisit pas, jamais ; simplement les individus transmettent ou non leurs
gènes, et ils le font, selon leurs aptitudes, plus ou moins efficacement, plus ou
moins abondamment... Cela fait une espèce de tri — par la mort, par la reproduction —,
qui prend la forme, pour nous, d’une finalité quasi providentielle,
mais qui n’est que le jeu du hasard (des gènes, des mutations) normé par la
nécessité (le combat pour la vie et l'élimination des plus faibles ou des moins
aptes à se reproduire). Ainsi le darwinisme pousse à l’athéisme. La nature joue
aux dés ; c’est donc qu’elle n’est pas Dieu.

{\it DASEIN} Mot allemand du langage courant, où il correspond à peu près à
notre « existence ». Dans l’usage philosophique, qui lui valut ses
lettres de noblesse internationales, le mot peut se traduire également par
« existence » (notamment chez Kant), par « être-là » (notamment chez Hegel)
ou par « réalité humaine » (chez Heidegger, du moins chez ses premiers traducteurs
français). En ce dernier sens, qui est seul usuel en français, le {\it Dasein} est
simplement cet étant que nous sommes, mais considéré dans sa dimension existentielle
(comme être-au-monde-et-pour-la-mort) plutôt qu’anthropologique
ou psychologique. C’est l’étant pour lequel {\it il y a} de l'être, et qui a {\it à} être.
L'homme ? Si l’on veut. Mais comme ouverture (de l’être et à l’être) davantage
que comme sujet ou comme intériorité. Nos heideggériens ont essayé plus
récemment, selon une indication de Heidegger lui-même, de le rendre par
« être-le-là » (homme est le « là » de l'être : l'ouverture pour laquelle, et par
laquelle, il y a de l'être), puis ont renoncé à le traduire. Renonçons plutôt à
l'utiliser.

%— 144 —
DÉBAT C'est une discussion publique, donc aussi un spectacle. C’est ce qui
rend le débat nécessaire, spécialement dans une démocratie, et
presque toujours décevant. Le souci de plaire ou de convaincre tend à
l'emporter sur les exigences de la raison. Et l’amour du succès, sur l’amour de
la vérité. Toute démocratie, on le sait depuis les Grecs, pousse à la sophistique.
Cela ne condamne pas la démocratie, ni n’excuse les sophistes.

DÉCADENCE Le début de la fin, et le contraire par là du progrès : la décadence
est une évolution lente, qui semble irréversible, vers
le pire ou vers rien. Il en résulte ordinairement un climat de pessimisme et de
langueur, qui ne va pas, chez les artistes, sans beaucoup de raffinements et
d’artifices. Les décadents sont volontiers esthètes : ils mettent l’art plus haut
que le réel (Mallarmé : « Le monde est fait pour aboutir à un beau livre »), plus
haut que la vie (Villiers de l’Isle-Adam : « Vivre ? Les serviteurs feront cela pour
nous »), plus haut que la vérité (Nietzsche : « L'art a plus de valeur que la
vérité »), plus haut que tout. Le réel se venge ; la décadence sombre dans le ridicule
ou l’ennui. Les barbares, déjà, se préparent.

DÉCENCE C’est comme une pudeur minimale, que la société nous impose.
Relève moins de la morale que de la politesse : l'apparence de
décence est une décence suffisante.

DÉCENTREMENT Le fait de changer de perspective, en adoptant, au moins
à titre provisoire ou méthodique, un autre point de vue
que le sien. C’est se mettre à la place d’un autre, ou à une autre place que la
sienne propre, ce qu’on ne peut faire, toujours incomplètement, que par sympathie
ou intelligence. Le contraire par là de l’égocentrisme, du fanatisme et de
la bêtise.

DÉCEPTION La fin d’une espérance, quand le réel lui donne tort — presque
toujours. On espérait un avenir, on se découvre incapable
d’aimer le présent : soit parce qu’il ne correspond pas à notre attente (souffrance,
frustration), soit parce que notre désir ne survit pas à sa propre satisfaction
(ennui). On s’en console ordinairement par une nouvelle espérance, qui
sera déçue à son tour. « Nous espérons que notre attente ne sera pas déçue en
cette occasion comme en l’autre, explique Pascal, et ainsi, le présent ne nous

%— 145 —
satisfaisant jamais, l'expérience nous pipe, et de malheur en malheur nous
mène jusqu’à la mort, qui en est un comble éternel » ({\it Pensées}, 148-425).

Le remède serait de n’espérer rien, et d’aimer tout. Mais qui le peut ? La
déception est la marque de notre finitude, face à l'infini du réel. Un être omniscient,
tout-puissant ou tout amour ne la connaîtrait jamais. Mais il serait
Dieu. C’est dire assez que la déception fait partie de notre humanité. Il faut
donc l’accepter aussi : cesse d’espérer n’être jamais déçu.

DÉCISION  L’acte de la volonté, quand elle choisit entre plusieurs possibles.
C’est passer du conditionnel à lindicatif, de l’imaginaire au
réel, de la délibération à l’action. Moment de crise et de résolution.

DÉCOUVERTE Faire apparaître ce qui existait déjà (c’est ce qui distingue
la découverte de l'invention), mais était inconnu. Ainsi
Christophe Colomb découvrant l'Amérique, ou Newton la gravitation universelle.
Notion presque toujours relative : l'Amérique n’était pas inconnue de
tous, Newton n’a pas découvert la gravitation tout seul. Au reste, une découverte
absolue n’en serait plus une : ce serait invention ou création.

DÉDUCTION  Déduire, c’est mener de propositions vraies ou supposées
(principes ou prémisses) à une ou plusieurs autres, qui en
découlent nécessairement. « Par déduction, écrit Descartes, nous entendons
toute conclusion nécessaire tirée d’autres choses connues avec certitude » ou,
ajouterions-nous aujourd’hui, posées à titre d’hypothèses. La déduction est
donc un raisonnement, qui suppose, comme l'écrit encore Descartes, « une
sorte de mouvement ou de succession » ({\it Règles pour la direction de l'esprit}, III).
S’oppose en cela à l’{\it intuition}, qui est la saisie séparée d’une vérité évidente. Sans
la déduction, nous ne pourrions jamais passer d’une vérité à une autre : nous
serions prisonniers de l’évidence actuelle. Mais sans l'intuition, pas davantage :
il n’y aurait plus d’évidence du tout.

On oppose également la {\it déduction} à l'{\it induction}. La déduction irait du
général au particulier (du principe à ses conséquences) ; l'induction, du particulier
au général (du fait à la loi). Cela indique en effet deux directions, et deux
faiblesses. La faiblesse de l’induction, c’est que la particularité des faits, aussi
nombreux soient-ils, ne pourra jamais justifier l’universalité d’une loi (quand
bien même j'aurais vu dix mille cygnes blancs, cela ne m’autorisera jamais à
affirmer que tous les cygnes sont blancs). La faiblesse de la déduction, c’est
%— 146 —
qu’elle n’est vraie que si les principes le sont — ce que ni la déduction ni l’induction
ne suffisent à attester. Ainsi toute induction est abusive, et toute déduction
incertaine. À la gloire du pyrrhonisme.

DÉFENSE Il n’y a pas d’être invincible : pour toute chose existante, explique
Spinoza, il en existe une autre plus forte par laquelle la première
peut être détruite ({\it Éthique}, IV, axiome). Cela vaut spécialement pour les
vivants, qui ne survivent qu’à la condition toujours de se défendre. On préférerait
la paix. On a raison. Mais la paix suppose la défense et ne saurait en tenir
lieu.

Qu'est-ce que la défense ? La tendance de tout être à persévérer dans son
être, en tant qu’elle se heurte à un autre qui le menace ou pourrait le
menacer. C’est le {\it conatus} en situation de danger, comme il est toujours. Ainsi
les défenses immunitaires d’un individu, la défense nationale, la défense d’un
accusé, ou, chez Freud, les mécanismes de défense du moi. On remarquera
que cela ne garantit nullement, dans chacun de ces cas, que la défense retenue
soit la bonne. Il y a des maladies auto-immunes et des névroses de défense.
Qu'il faille toujours se défendre ne prouve pas que toute défense soit opportune.

On parle de légitime défense quand la sauvegarde d’un individu le
contraint à accomplir un acte ordinairement interdit par la loi. C’est une auto-défense
justifiée et proportionnée au danger, ce qui rappelle qu’elles ne le sont
pas toutes.

Dans le {\it Traité des passions}, Descartes écrit qu’il y a « toujours plus de sûreté
en la défense qu’en la fuite ». Ce n’est vrai que si l’on a les moyens de se
défendre. C’est une raison forte pour se les donner.

DÉFÉRENCE Le respect à l'égard des supérieurs. Ce n’est un sentiment moral
que si la supériorité l’est également. À l’égard des puissants,
ce n’est que lâcheté ou bassesse.

DÉFI {\it « T'es pas cap ! »} C’est le défi premier : l’enfant est mis en demeure de
prouver de quoi il est capable. Seuls les actes comptent ici ; il faut
accepter le risque ou le déshonneur. Cette alternative, quand elle est imposée
par autrui, est le défi même. Le sage, légitimement, s’en défie. Pourquoi se laisserait-il
imposer sa conduite par moins sage que lui ?

%— 147 —
DÉFIANCE Une occurrence de la prudence, dans les rapports à autrui :
c'est un doute préalable, qui refuse de se fier à ce qu’on ne
connaît pas.

Sur ce qui la distingue de la méfiance, Littré a dit l’essentiel : « La méfiance
fait qu’on ne se fie pas du tout ; la défiance fait qu’on ne se fie qu'avec précaution.
Le défiant craint d’être trompé ; le méfiant croit qu’il sera trompé. La
méfiance ne permettrait pas à un homme de confier ses affaires à qui que ce
soit ; la défiance peut lui faire faire un bon choix. » La méfiance est un défaut
(c’est manquer de confiance) ; la défiance, une vertu : c’est vouloir ne se fier ou
ne se confier qu’à bon escient, autrement dit, la formule est encore de Littré,
« qu'après examen et réflexion ». Vertu peu sympathique, je l'accorde, mais
nécessaire. Il faut bien que les enfants sachent qu’il existe des pédophiles et des
assassins.

DÉFINITION Tout énoncé qui fait connaître ce qu’est une chose, disait
Aristote, ou ce que signifie un mot. On parlera dans le premier
cas de {\it définition réelle}, dans le second de {\it définition nominale}. Mais on n’a
accès aux premières que par les secondes.

Définir, c’est établir la compréhension d’un concept (souvent en indiquant
son genre prochain et ses différences spécifiques), et permettre par là de le comprendre.
On se souviendra pourtant que les concepts ne sont pas le réel, et
qu'aucune définition ne saurait pour cela tenir lieu de connaissance. « Dieu, dit
Spinoza, ne connaît rien abstraitement, ni ne forme de définitions générales. »
En quoi définir est un exercice d’humilité : c’est assumer, sans en être dupe,
notre lot de sens et d’abstraction.

DÉGÉNÉRESCENCE C’est comme une décadence naturelle (de même
que la décadence serait une dégénérescence cultu-
relle). Le dégénéré est victime de ses gènes ; le décadent, de son éducation ou
de ses goûts.

Le mot, dont les nazis abusèrent, est devenu pour cela malsonnant. Rappelons
pourtant que la dégénérescence fait bien partie des évolutions, ou des
involutions, possibles du vivant. L'erreur des nazis, comme de tous les racistes,
fut d’en voir la source dans le mélange des races, qui la combattrait plutôt (la
dégénérescence peut venir d’un excès d’endogamie). C’est la répétition du
même, non la rencontre de l’autre, qui fait dégénérer. Contre quoi la nature a
inventé le sexe et la mort ; et la culture, la prohibition de l’inceste. Trois façons,
pour combattre la dégénérescence, de sortir de soi, de sa famille et de tout.

%— 148 —
DÉGOÛT Impuissance momentanée à jouir, et même à désirer, qui va
parfois jusqu’à l’aversion. Le dégoût peut naître d’un excès préalable,
mais aussi d’une maladie, d’une fatigue, d’un chagrin, d’une angoisse...
Quand le dégoût devient général, il ne se distingue de la mélancolie que par la
durée. La mélancolie serait un dégoût généralisé ou permanent ; le dégoût, une
mélancolie ponctuelle ou provisoire. C’est le jusant de vivre : le désir à marée
basse.

DÉISME  Croyance en Dieu, sans la prétention de le connaître. Mais si on
ne le connaît pas du tout, comment savoir qu’il est Dieu ?

DÉLASSEMENT Une activité qui repose ou détend, par le libre jeu, toujours
agréable, de telle ou telle faculté : c’est le repos en
acte et en mouvement. Proche du loisir, par le temps libre qu’il suppose, il s’en
distingue par une fatigue préalable et un travail prévisible. C’est comme un
loisir nécessaire, entre deux fatigues ; le vrai loisir serait plutôt un délassement
superflu ou gratuit, entre deux repos. Le loisir tend au plaisir ; le délassement,
au travail ou à l’effort. C’est donc le loisir qui est bon, et le délassement qui est
nécessaire.

DÉLATION Une dénonciation coupable. En est-il d’innocentes ? Oui : celles
qui viennent de la victime, quand il y en a une, ou qui ne sont
inspirées que par les exigences de la justice. Ce n’est plus délation, mais plainte
ou témoignage. La différence, souvent ténue ou incertaine, est plutôt morale
que juridique. Une délation, même intéressée ou haineuse, peut parfois servir
la justice. Elle en est plus utile. Elle n’en est pas moins méprisable.

DÉLECTATION Se délecter, c’est plus que jouir ou que se réjouir, parce
que c’est l’un et l’autre, inséparablement : c’est jouir
joyeusement, c’est se réjouir de son plaisir et jouir de sa joie. Ce sont les
meilleurs moments de vivre. Poussin y voyait le but de l’art. Ce n’est sans doute
pas le seul. Mais c’est le plus délectable.

DÉLIBÉRATION L'examen avant la décision ou l’action. Les linguistes ne
savent trop si le mot vient de {\it libra}, la balance, ou de
%— 149 —
{\it liber}, libre. Cette hésitation est significative. Délibérer, c’est peser le pour et le
contre; mais c'est aussi le propre de l’homme libre. L’esclave n’a pas à
délibérer ; il lui suffit d’obéir à son maître ou à ses impulsions.

L'usage moderne et ordinaire du mot suppose une discussion à plusieurs :
toute délibération, en ce sens, serait collective. Mais dans la langue philosophique
(où {\it délibération} traduit la {\it bouleusis} d’Aristote), la délibération est souvent
un examen qui reste purement intérieur. Le mot, dans les deux cas, évoque
pourtant une certaine pluralité conflictuelle, qui est celle des arguments : il n’y
a pas délibération si tous les arguments vont dans le même sens.

Aristote, dans l’{\it Éthique à Nicomaque} (III, 5), remarque que nous ne délibérons
que « sur les choses qui dépendent de nous et que nous pouvons réaliser ».
La délibération est de l’ordre de l’action. On ne délibère pas sur le vrai et le
faux, mais seulement sur la décision que la situation, pour autant qu’on la
connaisse, semble justifier ou imposer. On ne délibère pas davantage sur les fins
elles-mêmes, ajoute Aristote, mais sur les moyens de les atteindre. Un médecin,
par exemple, ne se demande pas s’il doit guérir son malade, mais comment il
peut y parvenir. C’est dire qu’on ne délibère jamais sur l’essentiel, mais seulement
sur l'important.

DÉLICATESSE Mélange de douceur et de finesse, devant la fragilité de l’autre.
Une vertu, donc, et un talent : le contraire exactement de
la brute épaisse. Son danger serait qu’elle nous rende incapables de rudesse,
lorsque celle-ci est nécessaire. Ce ne serait plus délicatesse, mais pusillanimité.

DÉLINQUANCE L'ensemble des crimes et des délits, mais considéré plutôt
d’un point de vue sociologique. Les délinquants font
partie de la société ; ils en vivent, bien ou mal, ils la subissent ou en profitent,
comme tout le monde, ils lui ressemblent. Ce sont les hors-la-loi d’aujourd’hui,
mais sans autre programme que de vivoter ou de s'enrichir. Coupables ?
Victimes ? L’un et l’autre, presque toujours. C’est ce qui rend la répression
insuffisante et nécessaire.

On parle beaucoup de délinquance juvénile. C’est que la plupart seront
calmés par le mariage et les enfants. Les vrais délinquants sont ceux qui ne se
calment pas. On les met en prison, dirait Rousseau, pour les obliger à être
libres. Mais le plus souvent ils n’ont pas lu Rousseau, et se sentent plus libres
dehors.

%— 150 —
DÉLIRE Dérèglement de la pensée, quand elle cesse de se soumettre au vrai.
C’est moins une perte de la raison (le délire paranoïaque, par
exemple, est volontiers raisonneur) qu’une perte, qui peut être provisoire, du
réel ou du bon sens. Quand la raison tourne à vide, elle tourne mal.

{\it Delirare}, en latin, c’est sortir du sillon ({\it lira}) ; on dirait aujourd’hui extravaguer
ou dérailler. Cela suggère que le seul sillon, pour la pensée, c’est le réel ou
le vrai, disons l’univers ou l’universel, qu’on ne peut tester ou attester que dans
la rencontre de l’altérité : le monde, qui ne raisonne pas, ou la raison {\it des} autres.
Le délire est une pensée singulière ; c’est par quoi le génie, parfois, lui ressemble.
Mais le délire reste prisonnier de sa singularité, quand le génie, même
fou, est plutôt ouverture à l’universel.

Par extension, on peut appeler {\it délire} toute pensée qui enferme : ainsi dans
le fanatisme, la passion, la superstition. C’est mettre le sens plus haut que le
vrai, ses croyances plus haut que la raison, ses désirs plus haut que le réel. Au
bout les bûchers, l'asile, ou le mariage.

DÉLIT Violation du droit. On remarquera que tout délit n’est pas une faute
(quand la loi est injuste, il peut être juste de la violer), que toute
faute n’est pas un délit (aucune loi ne vous interdit d’être égoïste ou méchant),
enfin qu’il y a davantage de fautes légales que de délits vertueux. C’est que la
morale est plus exigeante. Les fautes sont innombrables ; les délits restent
l'exception — non que nous soyons bons ou justes, mais parce que la loi et la
police sont bien faites.

DÉMAGOGUE Celui qui veut guider le peuple en le suivant. Comme l'hystérique
selon Lacan, il cherche un maître sur qui régner.

Les armes habituelles du démagogue sont la flatterie, le mensonge, les promesses
inconsidérées, l'appel aux sentiments les plus bas ou les plus violents,
spécialement la peur, l'envie, la haine. Il nourrit les passions, et s’en nourrit.

Son contraire serait l’homme d’État, qui donnerait à raisonner, à vouloir, à
agir. Mais s’il s’interdisait toute démagogie, pourrait-il parvenir au pouvoir ?

DÉMENCE Perte de la raison, ou incapacité à s’en servir. Plus grave en cela
que le délire, qui est une raison déréglée, et même que la folie,
qui peut laisser intactes certaines des capacités intellectuelles du sujet. La
démence est une détérioration globale — que les psychiatres considèrent le plus
souvent comme irréversible — des facultés cognitives, affectives et normatives.
%— 151 —
C’est comme une désintégration de la personne. Le dément n’a pas seulement
perdu l'esprit ; il s’est perdu lui-même.

DÉMESURE  L’excès déraisonnable : l’{\it hubris} des Grecs, toujours preuve d’arrogance
ou d’aveuglement, toujours source de violences ou
d’injustices. C’est un propre de l’homme (les animaux sont mesurés par l’instinct),
qui fait de la mesure, pour nous, une vertu.

DÉMIURGE  « Si l’idée d’un Dieu créateur était rationnelle, me dit un jour
Marcel Conche, les Grecs l’auraient eue... » Leurs dieux, de
fait, ne créent pas le monde : soit parce que celui-ci est éternel (par exemple
chez Aristote), soit parce qu’il résulte d’un agencement hasardeux (chez Épicure)
ou divin (chez Platon) d’une matière préexistante. C’est ici qu’intervient
le démiurge : tel est le nom que Platon, dans le {\it Timée}, donne à ce dieu qui ne
crée pas le monde sensible à partir de rien, mais qui le fabrique en ordonnant
le « réceptacle » (la matière, l’espace) à partir du modèle que lui offre l’éternel
ordonnancement des Idées. C’est une espèce de dieu artisan (le {\it « démiourgos »},
étymologiquement, c’est l'artisan qui travaille pour le peuple), qui ne crée ni la
matière ni les Idées, mais qui fait passer dans celle-là, autant que faire se peut,
quelque chose de la perfection de celles-ci. Tel est resté, depuis Platon, le sens
du mot : un démiurge, c’est un dieu ordonnateur plutôt que créateur, médiateur
plutôt que transcendant, habile plutôt que parfait. L'idée est-elle pour cela
plus rationnelle ? Je n’en suis pas sûr. Quand bien même les Grecs auraient eu
toutes les idées rationnelles possibles, ce qui serait tout de même surprenant,
cela ne prouverait pas qu’ils n’aient eu que cela. Il arrive que la déraison, elle
aussi, parle grec.

DÉMOCRATIE Le régime où le peuple est souverain. Cela ne signifie pas
qu’il gouverne, ni même qu’il fait la loi, mais que nul ne
peut gouverner ou légiférer sans son accord ou hors de son contrôle. S’oppose
à la monarchie (souveraineté d’un seul), à l’aristocratie (souveraineté de
quelques-uns), enfin à l'anarchie ou à l’ultralibéralisme (pas de souverain).

On ne confondra pas la démocratie avec le respect des libertés individuelles
ou collectives. Quand la Convention, en 1793, décréta « la terreur jusqu’à la
paix », elle ne sortait pas pour cela de la démocratie. Si le peuple est souverain,
il peut fixer souverainement des limites à telle ou telle liberté, et même il le fait

%— 152 —
nécessairement, mais plus ou moins. C’est ce qui donne son sens à l'expression
« démocratie libérale » — parce qu’elles ne le sont pas toutes.

On ne confondra pas non plus la démocratie avec la république, qui serait
plutôt sa forme pure ou absolue — une et indivisible, laïque et égalitaire, nationale
et universaliste... « La Démocratie, écrit joliment Régis Debray, c’est ce
qui reste de la République quand on éteint les Lumières. » Disons que la démocratie
est un mode de fonctionnement ; la république, un idéal. Cela confirme
que la démocratie, même impure, est la condition de toute république.

DÉMON Un petit diable ou (chez les Grecs) un petit dieu. Les démons sont
innombrables : leur nom est légion. Le diable tend davantage à
l'unicité d’un principe ou d’un prince.

DÉMON DE CHANGEUX C'est une fiction, que j'ai proposée, dans {\it Une
éducation philosophique}, par analogie avec le
démon de Laplace : celle d’un neurobiologiste surdoué, dans dix mille ans, qui
pourrait tout connaître du cerveau de ses contemporains, jusqu’à y lire à livre
ouvert. Imaginons qu’il étudie, grâce à une imagerie médicale hyper développée,
les cerveaux de deux individus. Dans le cerveau de M. X, il lit que ce
dernier est persuadé qu'il a existé, il y a dix mille ans, une horreur qu’on a
appelée la Shoah. Dans le cerveau de M. Y, il lit au contraire qu’il n’en est rien :
que ce n’est qu’un mythe des âges obscurs. Lequel des deux a raison ? Voilà
ce que notre neurobiologiste ne peut lire dans aucun des deux cerveaux : il
faudra, pour en décider, qu’il fasse autre chose que de la neurobiologie (en
l'occurrence de l’histoire). Quelle différence y a-t-il, d’un point de vue neurobiologique,
entre une idée vraie et une idée fausse ? Aucune, à ce que je crois,
et c’est pourquoi la neurobiologie ne peut tenir lieu de pensée : la connaissance
neurobiologique d’une idée vraie peut tout connaître d’elle, sauf sa vérité.
Connaître une idée (comme objet) ne dispense pas de la penser (comme idée).
Notre neurobiologiste pourrait aussi lire, dans les cerveaux de MM. X et Y,
un certain nombre de valeurs, parfois communes, parfois opposées. Mais il
n'aurait aucun moyen, en ne faisant que de la neurobiologie, de juger de la
valeur de ces valeurs. M. Y est plus heureux ? Peut-être, mais qu'est-ce que cela
prouve ? M. X est davantage attaché à la justice, à la compassion, au devoir de
mémoire ? Cela ne prouve pas davantage qu’il ait raison. Bref, le démon de
Changeux peut tout connaître de nos valeurs, sauf leur valeur. Connaître une
valeur (comme objet) ne dispense pas de la juger (comme valeur).

%— 153 —
Cela ne retire rien à la neurobiologie, comme science, mais souligne les
limites du neurobiologisme, comme idéologie.

DÉMON DE LAPLACE L'expression fait allusion à un texte fameux de l’ {\it Essai
philosophique sur les probabilités}, de Pierre-Simon
Laplace :

\vspace{0.5cm}
{\footnotesize
« Nous devons envisager l’état présent de l’univers comme l'effet de son état antérieur,
et comme la cause de celui qui va suivre. Une intelligence qui, pour un instant
donné, connaîtrait toutes les forces dont la nature est animée et la situation respective
des êtres qui la composent, si d’ailleurs elle était assez vaste pour soumettre ces données
à l’analyse, embrasserait dans la même formule les mouvements des plus grands corps
de l’univers et ceux du plus léger atome : rien ne serait incertain pour elle, et l’avenir
comme le passé serait présent à ses yeux. »
}
\vspace{0.5cm}

Le démon de Laplace serait cette intelligence hypothétique, dont le point
de vue suffirait à abolir la différence entre le passé et l’avenir, donc l’idée même
de possible et par conséquent le libre arbitre. Le déterminisme, si tel était le cas,
mènerait nécessairement au prédéterminisme. C’est comme une personnification
de ce qu'Épicure appelait, pour le critiquer, « le destin des physiciens ».
On considère ordinairement qu’il est réfuté (comme il l'était, chez les épicuriens,
par le {\it clinamen}) par l’indéterminisme de la physique quantique et par les
processus chaotiques. L'avenir n’est pas davantage contenu dans le présent que
celui-ci ne l'était dans le passé. Il y a de l’irréductiblement nouveau, de l’imprévisible,
du chaos : avenir est ouvert. Même une intelligence infinie ne saurait
transformer le futur en passé. C’est donc que le présent, qui les sépare, est
quelque chose, ou plutôt qu’il est tout.

Cela, toutefois, ne règle pas la question de la liberté. Que l’avenir et le passé
soient différents (celui-là possible, celui-ci nécessaire) ne dit pas encore ce qu’il
en est du présent. Tant que je n’ai pas agi, je peux sans doute agir autrement ;
une fois que j'ai agi, je ne peux plus faire que cela n’ait pas été : l’action future
serait libre, l’action passée ne le serait pas. Mais l’action présente ? Comment
pourrait-elle ne pas être, quand elle est, ou être différente de ce qu’elle est ? Et
quelle autre action que présente ? Ainsi le démon de Laplace est mort : tout
n’était pas écrit de toute éternité. Mais cela ne suffit pas à sauver le libre arbitre.

DÉMON DE SOCRATE C’est un bon démon, une espèce d’ange gardien,
mais qui ne sait que parler et uniquement de
façon négative : il ne dit jamais ce qu’il faut faire, seulement ce qu’il faut éviter
%— 154 —
ou s’interdire (voir par exemple Platon, {\it Apologie de Socrate}, 31 d et 40 a-c).
Ceux qui ne croient ni aux démons ni aux anges y verront une image assez juste
de la conscience morale. Elle ne sait guère dire que non. C’est qu’elle s’oppose
à nos désirs égoïstes. Quant à ce qu’il faut faire de positif, c’est à l’intelligence,
plus qu’à la morale, d’en décider.

DÉMONSTRATION Un raisonnement probant. Cela suppose que la raison
puisse valoir comme preuve — ce qui est sans preuve,
mais que toute preuve suppose. « Il se peut faire qu’il y ait de vraies démonstrations,
écrit Pascal, mais cela n’est pas certain. » Cela, en effet, ne se démontre
pas.

Toute démonstration est donc incomplète (Montaigne : « Aucune raison
ne s’établira sans une autre raison : nous voilà à reculons jusques à l’infini »).
Une démonstration incomplète est pourtant autre chose qu’une conviction
indémontrable.

DÉNÉGATION Nier ce qu’on sait — fût-ce inconsciemment — être vrai.
C’est une espèce de mensonge ou d’erreur (nier le vrai,
c’est forcément affirmer le faux), mais plutôt défensif et de soi à soi. Le mot sert
surtout, en psychanalyse, pour désigner un mécanisme de défense, qui consiste
à formuler un désir ou un sentiment refoulés tout en niant les ressentir. {\it « Ce
n'est pas que je souhaite la mort de mon père, mais... »} C’est ouvrir la soupape de
l'inconscient, sans ôter le couvercle et pour aider à le maintenir. Le concept est
toutefois susceptible d’une extension plus large. Une phrase commençant par
{\it « Je ne suis pas raciste, mais... »} est souvent une dénégation.

DÉNI Une négation qui ne porte pas sur un affect inconscient (comme la
dénégation) mais sur le réel lui-même. Mécanisme psychotique,
selon Freud, plutôt que névrotique. L’inconscient n’a pas toujours tort, mais le
réel a toujours raison : le nier, c’est la perdre.

DÉONTOLOGIE C’est une espèce de morale professionnelle : l’ensemble
des devoirs ({\it to déon}, ce qu’il faut faire) qui ne s'imposent
que dans certains métiers. Une déontologie est à la fois conditionnelle et
particulière : c’est moins une morale, à strictement parler, qu’un code. Par
exemple l'honnêteté ne fait pas partie de la déontologie médicale (ce n’est pas

%— 155 —
parce qu’il est médecin qu’il doit être honnête) ; mais l’obligation de soin et de
secret, si.

DÉPASSEMENT Le fait d’aller plus loin, plus avant ou au-delà. En philosophie,
c’est surtout l'équivalent usuel pour traduire
l'{\it aufhebung} hégélienne, qui supprime et conserve à la fois ce qu’elle dépasse,
transcende, relève ou sursume (pour reprendre d’autres traductions, plus
récentes). Ainsi le chêne dépasse le gland (le supprime comme gland, le
conserve comme arbre), comme le devenir dépasse l’opposition de l'être et du
néant. C’est un processus de négation et de synthèse : le {\it happy end} de la dialectique,
qui est sans fin.

DÉPRESSION Une perte d'énergie, de désir ou de joie, comme un effondrement
du {\it conatus}. Se distingue du malheur par ses causes,
qui sont psychologiques ou morbides : c’est une espèce de tristesse endogène et
pathologique. La dépression se soigne ; le malheur se combat. Cela ne veut pas
dire qu’il n’y ait pas, entre les deux, toutes sortes de degrés intermédiaires et
d’actions réciproques (la dépression rend malheureux, le malheur déprime), qui
peuvent justifier qu’on prescrive des antidépresseurs, parfois, à quelqu’un qui
est simplement malheureux — non parce que sa souffrance serait pathologique,
mais pour éviter, peut-être, qu’elle ne devienne pathogène. Puis il y a aussi, de
la part du médecin, le simple devoir de compassion. Que celle-ci ne tienne pas
lieu de diagnostic, c’est bien clair. Mais comment un diagnostic pourrait-il en
dispenser ?

D'un point de vue clinique, la dépression s'accompagne ordinairement
d’anxiété, d’autodépréciation et d’inhibition psychomotrice. Dans ses formes
graves, elle rend la vie atrocement douloureuse, au point que la mort en devient
souvent désirable. Le suicide constitue le risque majeur de la dépression,
comme la dépression est l’une des causes principales de suicide. Raison de plus
pour la soigner. Le progrès des antidépresseurs, celui, dans une moindre
mesure, des psychothérapies, fait partie des bonnes nouvelles de ce temps. Que
certains en abusent, c’est probable. Mais il y a aussi, nous apprennent les psychiatres,
beaucoup de dépressions non soignées, et, rien qu’en France, plus de
cent mille tentatives de suicide par an, dont une sur dix, à peu près, aboutit à
la mort... Tous ces suicides ne relèvent pas de la dépression, et tous les
déprimés ne se tuent pas. L’énormité de ces chiffres doit pourtant nous rappeler
que la dépression est un problème plus grave, même dans notre pays surmédicalisé,
que l’abus des psychotropes.

%— 156 —
DÉRAISONNABLE Ce qui n’est pas conforme à la raison pratique (ce
qu’elle ne peut ni approuver ni justifier) ou à notre
désir de raison. À ne pas confondre avec l’irrationnel (voir ce mot) : il serait
irrationnel que le déraisonnable n'existe pas, et déraisonnable de croire pour
cela à l’existence de l’irrationnel.

DÉRÉLICTION C’est un état d’extrême abandon, qui fait comme une solitude
redoublée. Ainsi le Christ, sur la croix : {\it « Mon Dieu,
mon Dieu, pourquoi m'as-tu abandonné ? »} Cela suppose un reste de foi, par
quoi le mot relève le plus souvent, au moins dans son sens premier, du vocabulaire
religieux.

Au {\footnotesize XX$^\text{e}$} siècle, les existentialistes s’en sont servis pour rendre le {\it Geworfenheit}
de Heidegger, que les heiddegériens français traduisent plutôt, si on peut
appeler cela une traduction, par l’{\it être-jeté}. C’est être au monde sans y être chez
soi, et même sans y avoir été invité : sans recours, sans secours, sans justification.
La déréliction est comme une solitude métaphysique : c’est la solitude de
l’homme sans Dieu et sans place assignée.

DÉRISION Mélange d’ironie et de mépris; cela fait deux raisons de s’en
méfier. Elle n’est estimable que contre plus puissant que soi.
Contre les plus faibles, elle est méprisable. Contre les égaux, dérisoire.

DÉRISOIRE Ce qui mérite la dérision, ou qui le mériterait si nous n’en faisions
partie et si nous en étions tout à fait responsables. Pour
qui a conscience, comme Montaigne, que « notre propre et péculière condition
est autant ridicule que risible » ({\it Essais}, I, 50), il apparaît que la compassion vaut
mieux que le mépris, et l’humour que la dérision.

DÉSAMOUR Le mot parle de lui-même : l'amour s’est retiré, et cela fait
comme un grand vide où l’on voit enfin l’autre, croit-on,
comme il est... Ainsi la plage, à marée basse. Mais c’est oublier que la mer est
vraie aussi.

Belle formule de Marina Tsvetaïeva : « Quand une femme regarde un
homme dont elle n’est pas amoureuse, elle le voit tel que ses parents l’ont fait.
Quand elle l’aime, elle le voit tel que Dieu la fait. Quand elle ne l'aime plus,
elle voit une table, une chaise... »

%— 157 —
Le désamour a à voir avec la vérité: c’est un moment de désillusion.
L'amour vrai, s’il est possible, commence en ce point. Mais les amoureux n’en
veulent pas, qui préfèrent l’amour à la vérité.

C’est le vendredi saint de la passion.

DÉSENCHANTEMENT Une désillusion regrettable, ou qui laisse comme
un parfum de nostalgie.

Le mot, dans son usage philosophique, fait surtout penser à Max Weber :
le monde est {\it désenchanté} lorsqu'il apparaît sans magie, sans surnaturel, presque
sans mystère. C’est le monde des Modernes, notre monde, tout entier offert à
la connaissance et à l’action rationnelles. On évitera pourtant de confondre ce
désenchantement du monde avec sa banalisation technicienne ou mercantile.
L'existence même du monde est déjà un mystère suffisant, et un enchantement.

DÉSESPOIR Le degré zéro de l'espérance, et le contraire de la foi.

Au sens ordinaire, le mot désigne souvent le comble de la tristesse ou de la
déception : quand on ne peut plus échapper au malheur, ni espérer quelque
bonheur que ce soit. Ainsi les journaux parlent-ils souvent d’un {\it désespéré} pour
désigner celui qui s’est suicidé. Cela suppose presque toujours un espoir préalable,
qui s’est trouvé déçu (« l'espoir, m’écrivait un psychanalyste, est la principale
cause de suicide : on ne se tue guère que par déception »), et même un
espoir ultime (celui de mourir). C’est une déception insurmontable et mortifère.

Je prends le mot en un sens différent : pour désigner l’absence de tout
espoir, autrement dit de tout désir portant sur l'avenir, sur ce qu’on ignore ou
qui ne dépend pas de nous. C’est ne plus rien désirer du tout (désespoir
négatif), ou ne désirer que ce qui est, que ce qu’on connaît ou qui dépend de
nous (désespoir positif : amour, connaissance, volonté). C’est le contraire de la
foi (qui désire ce qui n’est pas ou qu’elle ignore). C’est le contraire de l’espérance
(qui désire ce qui n’est pas et qui ne dépend pas de nous). C’est pourquoi
c’est le contraire de la religion. « Le contraire de désespérer, c’est croire », écrivait
Kierkegaard. Cela se retourne : le contraire de croire, c’est désespérer.

C’est ce qui m’a autorisé à parler d’un {\it gai désespoir}. Pourquoi les croyants
auraient-ils le monopole de la joie ? Qu'il y ait, si Dieu n’existe pas, quelque
chose de désespérant dans la condition humaine, c’est une espèce d’évidence,
puisque l’on vieillit, puisque l’on souffre, puisque l’on meurt. Mais cette évidence
n’a jamais empêché de jouir du présent, ni de s’en réjouir. C’est plutôt
l'inverse qui est vrai. Combien de croyants n’ont attendu le bonheur que pour
%— 158 —
après la mort, ou ne l’ont trouvé ici-bas, comme disait Pascal, que « dans l’espérance
d’une autre vie » ? Et combien d’athées, au contraire, ont su profiter de
la vie et de ses plaisirs, combien ont su l’aimer joyeusement et désespérément,
non pour ce qu’elle annonçait mais pour ce qu’elle était ? Sagesse tragique :
sagesse du bonheur et du désespoir. Les deux peuvent aller ensemble, et même
ils le doivent. On n’espère que ce qu’on n’a pas : l'espoir du bonheur nous en
sépare. Celui qui est pleinement heureux, à l’inverse, n’a plus rien à espérer,
même pas que son bonheur continue (s’il espère le voir durer, il craint qu’il ne
cesse : son bonheur déjà est derrière lui). C’est la sagesse de l'Orient : « Seul le
désespéré est heureux, lit-on dans le {\it Sâmkhya-Sûtra}, car l'espoir est la plus
grande torture qui soit, et le désespoir la plus grande béatitude. » C’est vivre au
présent, et c’est la vérité de vivre.

En sommes-nous capables ? Disons qu’il nous arrive, par moments, d’en
faire à peu près l’expérience : par exemple dans la sexualité, la contemplation
ou l’action, quand nous ne désirons que ce qui est, que ce que nous faisons ou
qui dépend de nous. Parfois, c’est d’une plénitude telle qu’il n’y a plus rien, en
effet, à espérer — parce qu’il n’y a plus que le présent, que le réel, que le vrai.
Parce qu’il n’y a plus que tout. Ce sont des expériences rares, mais qu’on
n'oublie pas. Expériences d’éternité, dirait Spinoza, de celles, comme disait
Proust, qui rendent l’idée de la mort indifférente.

De fait, il m'est arrivé quelquefois d’être vivant, simplement.

DÉSHONNEUR Une blessure d’amour-propre, quand on la prend au sérieux
ou au tragique. C’est le principe des duels, de la vendetta,
parfois des guerres. Quelqu'un vous à traité de lâche, ou a séduit votre femme,
ou a mis en doute votre honnêteté, votre virilité, votre parole... Vous êtes
déshonoré si vous ne le tuez pas, ou du moins si vous ne prouvez publiquement
— dans l’ancien temps par un duel, aujourd’hui plutôt par une bagarre ou un
procès — qu’il a tort. C’est ainsi que plusieurs cocus, état qui n’a rien d’indigne,
devinrent des assassins, état qui n’a rien de respectable, pour échapper au
déshonneur. Ou que plusieurs peuples se sont entre-tués, indignement, pour
qu'on ne puisse pas suspecter leur courage. C’est accorder beaucoup d’importance
au regard des autres, et bien peu à leur vie.

DÉSIGNATION Le rapport d’un signe à son référent, c’est-à-dire à un objet
réel ou imaginaire qui lui est extérieur (les linguistes par-
lent aussi de {\it dénotation} ou de {\it référence}). À ne pas confondre avec la {\it signification},
qui est le rapport, interne au signe, entre le signifiant et le signifié.

%— 159 —
DÉSINTÉRESSEMENT Un acte est désintéressé lorsqu'il ne poursuit aucun
but égoïste, ou lorsque l’égoïsme, en tout cas, ne
suffit pas à l’expliquer. C’est pourquoi le désintéressement, au moins depuis
Kant, passe pour le propre de l’action morale : agir moralement, c’est agir,
comme dit Kant, «sans rien espérer pour cela ». On n’en conclura pas que
l’action morale doive forcément être triste ou dépourvue de plaisir. Vous avez
trouvé un portefeuille bien garni : vous le retournez à son propriétaire. L'action
n'est morale qu’à la condition que ne vous ne l’ayez pas fait dans l'espoir de
recevoir une récompense (si possible supérieure à ce que contenait le portefeuille...),
ni dans la crainte d’un châtiment (par exemple divin, auquel cas
votre comportement ne relève plus de la morale mais de la religion et de
l’égoïsme), ni même pour le simple plaisir d’avoir bien agi. Mais cela ne vous
interdit ni ce plaisir ni cette récompense.

On discute pour savoir si le désintéressement est possible : n’est-ce pas
violer le principe de plaisir ? La question est indécidable, mais peut-être aussi
mal posée. Jouir du bien qu’on fait à l’autre, cela ne prouve pas qu’on le fait
{\it pour} en jouir. Et quand bien même cela serait, mieux vaut cette jouissance-là
— la plus désintéressée possible, même si elle ne l’est jamais totalement — que
celle du salaud, qui ne sait jouir que du mal ou de son propre bien.

Enfin l'amour, quand il est là, rend la question dérisoire. Aucune mère ne
nourrit son enfant de façon désintéressée : tout son bonheur à elle en dépend.
Qui ne voit pourtant que son amour est supérieur au désintéressement ? C’est
où l’on passe de la morale à l'éthique, du désintéressement à l’amour, de Kant
à Spinoza.

DÉSIR Puissance de jouir ou d’agir.
On ne confondra pas le désir avec le manque, qui n’est que son
échec, sa limite ou sa frustration. Le désir, en lui-même, ne manque de rien
(c’est l’impuissance, non la puissance, qui manque de quelque chose). Pourquoi
faudrait-il manquer de nourriture pour désirer manger? Ce serait
confondre la faim, qui est une souffrance, avec l'appétit, qui est une force et,
déjà, un plaisir. Pourquoi faudrait-il être « en manque », comme on dit, pour
désirer faire l'amour ? Ce serait confondre la frustration, qui est un malheur,
avec la puissance ou l’amour, qui sont un bonheur et une chance. Le désir n’est
pas manque, malgré Platon (Le {\it Banquet}, 200), mais puissance : c’est puissance
de jouir et jouissance en puissance. Le plaisir est son acte ; la mort, son destin.
Il est la force, en chacun de nous, qui nous meut et nous émeut : c’est notre
puissance d'exister, comme dit Spinoza, de ressentir et d’agir. Le principe de
plaisir, comme dit Freud, résulte de sa définition.

%— 160 —
« Sont du désir, écrit Aristote, l’appétit, le courage et la volonté. » Il faut y
ajouter l'amour et l’espérance. Le désir, explique en effet le {\it De Anima}, est en
nous l’unique force motrice : « L’intellect ne meut manifestement pas sans le
désir », alors que le désir « peut mouvoir en dehors de tout raisonnement » ({\it De
l'âme}, II, 3, et III, 10). Or qui ne voit que l’amour et l'espérance nous
meuvent ? Il n’y a ainsi « qu’un seul principe moteur, la faculté désirante » :
c’est parce que nous désirons que nous sommes notre « propre moteur » ({\it ibid.},
III, 10).

Spinoza, qui définissait le désir comme « l’appétit avec conscience de lui-même »
(ce qui suppose qu’il y a des appétits inconscients), soulignait aussitôt
que cette définition ne disait pas l’essentiel : « Que l’homme, en effet, ait ou
n'ait pas conscience de son appétit, cet appétit n’en demeure pas moins le
même » ({\it Éthique}, III, 9, scolie, et déf. 1 des affects, explication). La vraie définition,
qui vaut donc pour le désir comme pour lappétit, est la suivante : « Le
désir est l’essence même de l’homme, en tant qu’elle est conçue comme déterminée
à faire quelque chose par une affection quelconque donnée en elle »
{\it (ibid.)}. C’est la forme humaine du {\it conatus}, et le principe par là de « tous les
efforts, impulsions, appétits et volitions de l’homme, lesquels varient suivant la
disposition variable d’un même homme et s’opposent si bien les uns aux autres
que l’homme est traîné en divers sens et ne sait où se tourner » {\it (ibid.)}. Le désir,
pour Spinoza aussi, est l’unique force motrice : c’est la force que nous sommes
ou dont nous résultons, qui nous traverse, qui nous constitue, qui nous anime.
Le désir n’est pas un accident, ni une faculté parmi d’autres. C’est notre être
même, considéré dans «sa puissance d’agir ou sa force d'exister» ({\it agendi
potentia sive existendi vis}, III, déf. générale des affects). C’est dire qu'il serait
absurde ou mortifère de vouloir supprimer le désir. On ne peut que le transformer,
que l’orienter, que le sublimer parfois, et tel est le but de l’éducation.
Tel est aussi, et plus spécialement, le but de l’éthique. Il s’agit de désirer un peu
moins ce qui n’est pas ou qui ne dépend pas de nous, et un peu plus ce qui est
ou qui en dépend : il s’agit d’espérer un peu moins, d’aimer et d’agir un peu
plus. C’est libérer le désir du néant qui le hante, en l’ouvrant au réel qui le
porte.

DÉSOBÉISSANCE C’est refuser de se soumettre à un pouvoir légitime. Ce
ne peut être qu’une exception (il n’y aurait autrement
ni pouvoir ni légitimité), mais c’est une exception nécessaire (il n’y aurait plus
autrement de liberté).

Si je refuse de me soumettre au voyou qui m’agresse ou au tyran qui
m'opprime, ce n’est pas désobéissance ; c’est combat, révolte, guerre — état de
%— 161 —
nature. Il n’y a désobéissance que lorsque je transgresse une loi dont je reconnais
par ailleurs la légitimité. Pourquoi, alors, la transgresser ? Le plus souvent
par égoïsme (parce que la loi s’oppose à mon intérêt), parfois par devoir (parce
qu'elle s'oppose à ma conscience). Cela pose le problème de la désobéissance
civique, ou plutôt permet seul de le résoudre. La question n’est pas de savoir si
telle loi est bonne ou pas. Si chaque citoyen n’obéissait qu'aux lois qu'il
approuve, il n’y aurait plus de République, donc plus de lois ni de citoyens. La
vraie question, c’est de savoir si l’on peut obéir à l'État, dans telle ou telle circonstance,
sans sacrifier quelque chose de plus essentiel encore que la République.
Si la réponse est non, il faut désobéir. Chacun comprend que ces circonstances,
dans un État de droit, ne peuvent être qu’exceptionnelles. La règle
est simple à formuler, difficile à appliquer : {\it on a le droit de désobéir, mais
seulement quand c'est un devoir.}

DÉSORDRE Jetez une poignée de cailloux sur le sol : vous aurez le sentiment
du désordre. Si, par extraordinaire, les cailloux avaient
formé une figure reconnaissable, par exemple un hexagone ou un visage, vous
auriez eu le sentiment d’un ordre. Pour les cailloux, pourtant, il n’y a pas plus
d'ordre, ni de désordre, dans un cas que dans l’autre. Cela dit à peu près ce que
c'est que le désordre : un ordre qu’on ne reconnaît pas. Et ce que c’est que
l’ordre : un désordre facile à imaginer, à mémoriser ou à utiliser. Ces notions
n'ont donc de sens que relatif. « La réalité est ordonnée, écrit Bergson, dans
l’exacte mesure où elle satisfait notre pensée », désordonnée lorsqu’elle échoue
à nous satisfaire ou lorsque nous échouons, plutôt, à nous y retrouver. L'idée
de désordre n’exprime que « la déception d’un esprit qui trouve devant lui un
ordre différent de celui dont il a besoin, ordre dont il n’a que faire pour le
moment, et qui, en ce sens, n'existe pas pour lui » ({\it L'évolution créatrice}, III ;
voir aussi {\it La pensée et le mouvant}, p. 108-109). Ainsi il n’y a pas de désordre
absolu, parce qu’il n’y a pas d’ordre absolu : il n’y a que des ordres différents et
tous relatifs. C’est ce qui interdit à la notion d’entropie (voir ce mot) de valoir
absolument. Que le désordre, dans tout système isolé, tende vers un maximum,
cela n'empêche pas que ce désordre, à le considérer en lui-même, reste un ordre
comme un autre : c’est l’ordre le plus probable, le plus stable et le moins créateur.
Tout tend vers n’importe quoi, c’est-à-dire vers rien. Mais n'importe
quoi, c’est encore quelque chose.

DESPOTISME Le pouvoir sans limite d’un seul.
Le despotisme peut être éclairé, et même légitime (c’est ce
%— 162 —
qui le distingue de la tyrannie) ; mais il est toujours injuste : s’il se soumettait
au droit, son pouvoir ne serait plus sans limite. C’est ce qui le distingue du
régime monarchique, « où un seul gouverne, remarquait Montesquieu, mais
par des lois fixes et établies ; au lieu que, dans le despotique, un seul, sans loi et
sans règle, entraîne tout par sa volonté et par ses caprices » ({\it De l'esprit des lois},
II, 1). Le despote se met au-dessus des lois (Rousseau), ou n’en connaît pas
d’autres que les siennes propres (Kant). Le despotisme est une monarchie
absolue et autoritaire. Son principe n’est pas l’honneur, comme dans une
monarchie bien réglée, ni la vertu, comme dans une démocratie, mais la crainte
({\it De l'esprit des lois}, III, 9). C’est aussi sa limite : il ne peut durer qu’à la
condition de faire peur.

DESTIN L'ensemble de tout ce qui arrive, et qui ne peut pas ne pas arriver.
Le mot s'applique spécialement à ce qui ne dépend pas de nous.
On remarquera que tout passé est donc fatal (il ne dépend plus de moi,
aujourd’hui, d’avoir fait ou non ce qui en dépendait), et tout présent, en tant
qu’il relève de l’ordre ou du désordre du monde. Cela ne signifie pas que c’était
écrit à l’avance, superstition néfaste, mais simplement que ce qui est ne peut
pas ne pas être, ni dès lors, et à jamais, ne pas avoir été. Le destin est donc le
réel même : ce n’est pas une cause de plus, c’est l’ensemble de toutes.

DESTINÉE Une fatalité qui aurait un sens. Son contraire est le hasard, qui
est une nécessité insensée (un nœud insignifiant de causes).

DÉTERMINISME Doctrine selon laquelle tout est déterminé, c’est-à-dire
soumis à des conditions nécessaires et suffisantes, qui
sont elles-mêmes déterminées. Le déterminisme, en ce sens, n’est qu’une
généralisation du principe de causalité. C’est une chaîne de causes, ou plusieurs, ou
l’ensemble de ces chaînes, à quoi rien n’échappe, ni lui-même : on peut agir sur
lui, le changer, le maîtriser, mais point en sortir. C’est le labyrinthe des causes,
ou plutôt des effets. Kant a bien vu qu’il excluait à la fois la contingence et la
fatalité ({\it C. R. Pure}, Analytique des principes, PUF, p. 208, Pléiade, p. 960). La
multiplicité des causes explique tout, mais n’impose rien.

Le déterminisme n’est qu’un autre nom pour le hasard (comme pluralité
des séries causales), en tant qu’il est connaissable. On ne le confondra pas avec
le prédéterminisme, qui suppose qu’il existe une chaîne {\it unique et continue} de
causes, de telle sorte que l'avenir serait tout entier inscrit dans le présent,
%— 163 —
comme le présent résulterait nécessairement du passé. C’est donner au temps
une efficace qu’il n’a pas. Ni avec l’idée d’une prévision possible : un phénomène
peut être intégralement déterminé tout en restant parfaitement imprévisible
(c’est le principe des jeux de hasard et des systèmes chaotiques). Le temps
qu'il fera dans six mois n’est écrit nulle part : il n’est pas {\it déjà} déterminé ; mais
il le sera dans six mois. Ainsi le déterminisme n’est pas un fatalisme : il n’exclut
ni le hasard ni l'efficacité de l’action. Il permet au contraire de les penser. De Ià
la météorologie et le parapluie.

DÉTRESSE Le malheur, quand on n’a plus les moyens de le surmonter, ni
même de le combattre — quand on ne peut plus qu’implorer
secours ou grâce. C’est l’état ordinaire du vivant, quand il naît et meurt.

DEUIL C'est la perte d’un être cher, et la douleur qui en résulte : l'amour
déchiré par l’arrachement de son objet. Blessure affective, comme
une amputation de l'essentiel. On fait alors son deuil de ce qui manque, c’est-à-dire
qu’on apprend à vivre sans. Apprentissage difficile et douloureux, toujours :
être en deuil, c’est être en souffrance. Ainsi après la mort de celui que vous aimiez
plus que tout au monde, quand vivre n’est plus que cette absence atroce, insupportable,
comme une plaie béante, quand on a le sentiment que toute joie, à
jamais, est devenue impossible... Le {\it travail du deuil}, comme dit Freud, sert à
sortir de cet état. Travail d’acceptation, de détachement progressif, de réconciliation.
Non contre l'amour, mais pour aimer autrement, puis autre chose. Il s’agit
que la joie redevienne au moins possible. Le deuil est accompli quand on y parvient.
C’est le plus dur chemin qui mène d’une vérité à un bonheur.

Le concept est susceptible d’une extension universelle. « Vivre, c’est
perdre », écrit François George, et c’est en quoi le deuil est l’état ordinaire des
vivants. On n’y échapperait qu’en cessant d’aimer, ce qu’on ne doit, de vivre,
ce qu'on ne veut, de mourir, ce qu’on ne peut. Il faut donc s’accepter mortel
(faire le deuil de soi, de son vivant), et amant de mortels. « À l'égard de toutes
les autres choses, disait Épicure, il est possible de se procurer la sécurité, mais,
à cause de la mort, nous, les hommes, habitons tous une cité sans murailles. »
Vivre est une ville ouverte, et cette ouverture — l’amour, la mort — est l’unique
demeure. Le deuil est l'horizon de l'amour, de tout amour, et l'unique chemin
vers la sagesse.

« Doux est le souvenir de l’ami disparu », disait encore Épicure ; cela
exprime assez bien ce qu’est un deuil réussi. Au bout du chemin ? Douceur,
gratitude, joie — amour. On se disait d’abord : {\it « Quelle horreur qu'il ne soit
%— 164 —
plus !»} Puis peu à peu: {\it « Comme c'est bien qu'il ait existé ! »} La vie emporte
tout, jusqu’à la mort même qui l’emporte. Cela fait comme une paix très
douce, après l’atrocité de la perte. Et si l’on n’y parvient pas ? La mort achèvera
le travail. Toute douleur cesse, où elles vont toutes.

DEVENIR Le changement, considéré dans sa globalité. C’est donc l'être même,
en tant qu'il ne cesse de changer. {\it « Panta rhei »}, disait Héraclite :
tout coule ou s'écoule, tout change, tout passe et rien ne demeure... On ne se
baigne jamais deux fois dans le même fleuve, ni même, ajoutait Cratyle, une seule
fois : le temps d’entrer dans l’eau, elle n’est plus la même.

Chez Hegel, le devenir est l’unité de l’être et du néant : le passage de l’un
dans l’autre, et de l’autre dans l’un. C’est en quoi il est la première pensée
concrète (l'être et le néant ne sont que des abstractions vides), et à ce titre la
première vérité. C'était rester fidèle à Héraclite et au réel. « Le vrai, écrit Hegel,
est le devenir de soi-même. »

DEVOIR Le verbe indique d’abord une dette ({\it debere}, en latin, vient de {\it de
habere} : c’est avoir quelque chose de quelqu’un). Le substantif, une
obligation : non plus {\it avoir de}, mais {\it avoir à}. La transition, entre les deux, relève
d’une logique de l'échange ou du don : si j’ai reçu quelque chose de quelqu'un,
je lui dois en retour autre chose. Il y a là une structure archaïque, dont le
devoir, au sens moral du terme, manifeste la permanence. Dans la plupart des
société primitives, a montré Marcel Mauss, tout don suppose un contre-don :
« Les échanges et les contrats se font sous forme de cadeaux, en théorie volontaires,
en réalité obligatoirement faits et rendus ». Le contre-don, dans ces
sociétés, est un devoir. Le devoir, dans les nôtres, comme un contre-don obligé.
Qu’avons-nous donc reçu, qui nous oblige ? Nous avons {\it tout} reçu : la vie,
l'humanité, la civilisation. De qui ? Peut-être de Dieu. Assurément de nos
parents, de la société, de l'humanité... Comment cela ne nous créerait-il pas
des devoirs ? La morale consiste à se savoir débiteur, dirais-je à la façon d’Alain,
et obligé par là : car tout don oblige. Voyez la parabole des talents. Il ne s’agit
pas seulement de rendre ce qu’on a reçu, mais de le faire fructifier au mieux.
Ainsi le premier devoir est de ne pas oublier qu’on en a.

Chez Kant, le devoir est la nécessité d’accomplir une action par pur respect
pour la loi morale, c’est-à-dire indépendamment de toute inclination sensible
ou affective (si on agit par amour ou par compassion, on n’agit plus par
devoir), et en faisant abstraction, même, de tous les objets de la faculté de
désirer, de tout plaisir, de toute fin, et spécialement de toute récompense ou
%— 165 —
châtiment attendus. Le devoir, dans son principe, est désintéressé. Par exemple
celui qui ne ferait le bien que dans l'espoir du paradis ou par crainte de l’enfer :
il agirait certes {\it conformément au devoir}, mais pas {\it par devoir} (il agirait conformément
au devoir, mais {\it par intérêt}), et son action serait pour cela moralement
sans valeur. Mais celui qui ne ferait le bien que pour le plaisir de le faire, son
action, si aimable qu’elle soit, n'aurait pas davantage «de valeur morale
véritable » : mieux vaut pour Kant un misanthrope vertueux, qui n’agit que par
devoir, plutôt qu’un philanthrope sympathique, qui n’agirait que par inclination
({\it Fondements...}, I). C’est où le devoir et la morale, tels que Kant les
conçoit, s'opposent le plus clairement à la vertu et à l’éthique, telles que les
Anciens ou Spinoza nous aident à les penser. La générosité, par exemple, est
d’autant plus morale, pour Kant, qu’on y prend moins de plaisir ; d’autant plus
vertueuse, pour Aristote ou Spinoza, qu'on en prend davantage (celui qui
donne sans plaisir n’est pas généreux : c’est un avare qui se force). De là ce
qu’on pourrait appeler une primauté de l'éthique, qui ne saurait pour autant
abolir la morale (puisque la vertu, presque toujours, fait défaut) ni même en
tenir lieu. La morale n’est bonne que pour les méchants, ou plutôt que pour les
égoïstes. C’est pourquoi elle est bonne, en pratique, pour nous tous. C’est le
contraire de l’égoïsme. C’est le contraire du mal radical. Agir moralement, c’est
n’obéir qu’à la loi que la raison en nous se donne à elle-même, et nous donne,
autrement dit qu’à l’universel. Par quoi le devoir est obligatoire, remarque
Alain, « mais non point forcé » : nul ne peut nous contraindre à agir par devoir,
nul n’agit par devoir qu’à condition de le faire librement. Cela suppose qu’on
s’affranchisse de tout ce qui n’est pas libre en nous, de tout ce qui n’est pas universel,
et d’abord du {\it « cher moi »}, comme dit Kant, de ses instincts, de ses penchants,
de ses peurs, et même de ses espérances. « La majesté du devoir, écrit
tranquillement Kant, n’a rien à faire avec la jouissance de la vie. » Non qu'il
soit immoral de jouir, la vie n’est pas difficile à ce point, mais parce qu’il le
serait de soumettre la morale au plaisir, quand c’est l'inverse qu’il faut faire (ne
chercher le plaisir que lorsque aucun devoir ne s’y oppose). Les jobards reprochent
à Kant son ascétisme. Mais lui donnent raison, à chaque fois qu'ils
s’interdisent le viol ou l'assassinat, quelque plaisir qu’ils imaginent y trouver,
ou à chaque fois qu’ils s'imposent, pour des raisons morales, une action désagréable
ou risquée. Ainsi le plaisir n’est pas tout, ni le bonheur, ni même la
sagesse, et c’est ce que signifie le devoir. Il y a en lui quelque chose de désespéré,
par quoi il échappe à l’{\it ego} : agir moralement, c’est faire ce qu’on doit parce
qu’on le doit, dût-on en souffrir et {\it « sans rien espérer pour cela »} (comme dit
Kant à propos de la bienfaisance : {\it Doctrine de la vertu}, \S 30).

Le devoir existe-t-il ? Point, certes, comme une chose ou un fait. Mais il
n’en correspond pas moins à notre expérience. Si un enfant se noie, si un innocent
%— 166 —
appelle au secours, la situation prend pour moi la forme d’une obligation,
d’un commandement, d’un impératif : je sais bien que je {\it dois} les aider, si je le
peux, quand bien même je n’y aurais aucun intérêt et fût-ce au péril de ma vie.
Par quoi Kant a au moins phénoménologiquement raison : il décrit la morale
telle qu’elle nous apparaît, comme une libre obligation, et c’est la morale
même.

Connaît-on toujours son devoir ? Disons qu’il n’y a {\it devoir}, en tout cas,
qu’à la condition qu’on le connaisse à peu près. C’est le cas, de très loin, le
plus fréquent. Pour ma part, je ne me rappelle pas m'être beaucoup interrogé
sur mon devoir. J'ai presque toujours vu fort clairement quel il était — ce qui
ne veut pas dire, tant s’en faut, que je l’aie toujours suivi. « Il n’y a jamais
d’autre difficulté dans le devoir, disait Alain, que le faire. » Difficulté pratique,
non théorique. Elle est souvent considérable ; c’est qu’il faut vaincre la
peur, l’égoïsme, la fatigue.

DIABLE Le démon principal ou principiel. Il fait le mal pour le mal. C’est
par quoi il est inhumain : l’homme ne fait jamais le mal pour le
mal, explique Kant, mais seulement par égoïsme (pour son bien à lui). Le
diable est méchant ; l’homme n’est que mauvais ({\it La religion dans les limites de
la simple raison}, I, 3).
Le diable n’est pas seulement méchant ; il est aussi très bête ou inintelligible
(pourquoi diable fait-il le mal, s’il n’y trouve pas son bien ?). On peut dire de
lui ce que Stendhal disait de Dieu : sa seule excuse, c’est qu’il n’existe pas.

DIALECTIQUE L’art du dialogue et de la contradiction, donc de la controverse.
C’est aussi une logique de l’apparence (dans le
meilleur des cas) ou l’apparence d’une logique (dans le pire). Enfin — chez
Hegel ou Marx — c’est une certaine méthode intellectuelle, fondée sur l'unité
des contraires et leur dépassement dans une synthèse supérieure.

Le mot vient des Grecs. La {\it dialectikè}, chez Platon, c’est d’abord l’art du dialogue,
tel que Socrate le pratiquait, autrement dit par questions et réponses
(voir par exemple le {\it Cratyle}, 390 c). À partir de {\it La République} (livres VI
et VII), c’est surtout la marche même de la pensée, qui part d’hypothèses,
mais reconnues comme telles, pour « s'élever jusqu’au principe universel et
anhypothétique » (dialectique ascendante), avant de redescendre, mais en
n'ayant recours qu'aux seules idées, vers ses conséquences ou applications (dialectique
descendante). C’est enfin — spécialement dans le {\it Phèdre} et le {\it Sophiste} —
Part des synthèses et des divisions, qui permet de passer du multiple à l’un et
%— 167 —
de l’un au multiple. Platon y voyait la science par excellence, celle de l’intelligible.
Cette science n’est plus la nôtre, ni cette dialectique.

Aristote nous est plus proche. La dialectique, chez lui, est la logique du
probable : non une science, mais ce qui en tient lieu lorsque toute science est
impossible ({\it Topiques}, I, 1). C’est l’art de raisonner sur des opinions opposées,
telles qu’elles peuvent s'exprimer dans un dialogue, à partir de prémisses seulement
plausibles et sans qu’aucun savoir ni aucune démonstration ne permettent,
entre ces opinions, de trancher absolument ou de façon apodictique. Cela
suppose qu’on plaide le pour et le contre, la thèse et l’antithèse, et c’est en quoi
la dialectique est à la fois universelle, quant à ses objets possibles, tout en restant
particulière (puisque objectivement insuffisante) quant au sujet qui
l’énonce. S’oppose en cela à l’analytique, qui est la science démonstrative, mais
aussi la prépare et la complète. Toute démonstration, par exemple, suppose le
principe de non-contradiction, qui est pour cela indémontrable. On ne peut le
justifier, puisqu'il le faut, que de façon dialectique : non en prouvant qu’il est
vrai, ce qui est impossible, mais en montrant que personne ne peut tenir un
discours sensé sans présupposer d’abord la vérité de ce principe ({\it Métaphysique},
$\Gamma$, 4). Ainsi le vrai suppose le probable, et la science le dialogue.

Au Moyen Âge, à la suite sans doute des stoïciens, la dialectique absorbe
la logique, ou plutôt ne fait qu’un avec elle : c’est l’art de raisonner (par opposition
à la rhétorique, qui est l’art de parler). Ce n’est plus vrai aujourd’hui :
de dialectique, nos logiciens ne se préoccupent guère. C’est que Kant, entre-temps,
revenant à Aristote, a redonné à la dialectique un sens spécifique, qui
vaut chez lui comme condamnation. Qu'est-ce que la dialectique ? C’est « la
logique de l’apparence » (par opposition à l’analytique, qui est « logique de la
vérité ») : c’est « l’art sophistique de donner à son ignorance, et même à ses
illusions délibérées, le vernis de la vérité » ({\it Critique de la raison pure}, « De la
division de la logique générale en analytique et dialectique »). Le dialecticien
voudrait se servir de la logique pour étendre ses connaissances (alors que la
logique n’enseigne que les conditions formelles de leur cohérence), ce qui ne
peut aboutir « à rien de plus qu’à un verbiage par lequel on affirme, avec
quelque apparence, ou l’on conteste, suivant son humeur, tout ce qu’on
veut ». L'exemple le plus parlant est celui des fameuses {\it antinomies} (voir ce
mot), dans lesquelles la raison s’enferre — parce qu’elle peut démontrer aussi
bien la thèse que l’antithèse — dès qu’elle essaye de raisonner sur l’inconditionné.
Kant y voyait une impasse. Cela nous mène à Hegel, qui en fera son
boulevard.

Dans le vocabulaire philosophique contemporain, lorsqu'il est utilisé sans
autre précision et s’il n’est pas un simple effet de mode ou de snobisme, le mot
{\it « dialectique »} renvoie en effet le plus souvent à la logique, ou prétendue telle,
%— 168 —
de Hegel. De quoi s'agit-il ? D'abord d’une pensée de la complexité, de l’inter-dépendance,
de la non-séparation. Pour le dialecticien, tout est dans tout et
réciproquement : « Nous appelons dialectique, écrit Hegel, le mouvement
rationnel supérieur, dans lequel des termes en apparence tout à fait séparés passent
l’un dans l’autre par eux-mêmes, par le fait même de ce qu’ils sont, et dans
lequel la présupposition de leur séparation se supprime » ({\it Logique}, I, 1 ; même
référence pour les citations qui suivent). Cela vaut spécialement pour les contraires.
Loin de s’opposer de façon extérieure et statique, comme le voudrait
l’entendement, ceux-ci n'existent en vérité qu’ensemble, dans le mouvement
même qui les oppose de l’intérieur l’un à l’autre et les dépasse. Soit, par
exemple, l'être et le néant. Si l’on commence par les séparer, comme font la
plupart des philosophes, on ne pourra jamais comprendre l’origine (le passage
du néant à l'être) ni la fin (le passage de l’être au néant). Ce que manifeste la
première antinomie de Kant, souligne Hegel, c’est d’abord l'incapacité de
l’entendement à penser le devenir : « Tant que se trouve présupposé le divorce
absolu de l’être par rapport au néant, le commencement ou le devenir sont sans
contredit quelque chose d’incompréhensible. » Pour la raison dialectique, il en
va tout autrement. Partons de l’être pur. Qu'est-il ? Une table, une chaise, une
multiplication, un tuyau d’arrosage ? Rien de tout cela, car alors il ne serait plus
pur et ne saurait valoir, comme concept, pour tout être. La vérité de l’être, en
tant qu'être pur, c’est qu'il n’est pas ceci ou cela (pas une table, pas une multiplication,
pas un tuyau d’arrosage...), ni rien de déterminé : il est l'être qui
n’est rien, par quoi « il est en fait néant, et ni plus ni moins que néant ». Mais
le néant, qu'est-ce ? « Égalité simple avec lui-même, répond Hegel, vacuité parfaite,
absence de détermination et de contenu, état de non-différenciation en
lui-même... » Il est l’être de ce qui n’est rien, ou le rien qui est. « Le néant est
donc la même détermination, ou plutôt la même absence de détermination, et,
partant, {\it absolument la même chose que l'être pur}. » Ce qu’il faut penser, pour la
dialectique, c’est donc l’unité de l’être et du néant. Après la thèse et l’antithèse,
comme on dit dans les classes, voilà le moment de la synthèse, qui n’est pas un
juste milieu mais un dépassement :

\vspace{0.5cm}
{\footnotesize
« L’être pur et le néant pur sont la même chose. Ce qui est la vérité, ce n’est ni l'être
ni le néant, mais le fait que l'être — non point passe — mais est passé en néant, et le
néant en être. Pourtant la vérité, tout aussi bien, n’est pas leur état de non-différenciation,
mais le fait qu’ils sont absolument différents, et que pourtant, tout aussi immédiatement,
chacun disparaît dans son contraire. Leur vérité est donc ce mouvement du disparaître
immédiat de l’un dans l’autre : {\it le devenir}, un mouvement où les deux sont
différents, mais par le truchement d’une différence qui s’est dissoute tout aussi
immédiatement. »
}
\vspace{0.5cm}

%— 169 —
Ce n’est qu’un exemple, qui donne autant à admirer qu’à douter. Que
l'être et le néant, {\it comme mots}, s'opposent et reviennent au même, qu'est-ce
que cela nous apprend sur le réel ou le vrai ? Et qu'est-ce que cela prouve
contre le principe de non-contradiction, que toute preuve suppose ? La
vérité, c’est que la dialectique ne prouve jamais rien, sinon, parfois, la virtuosité
de qui s’en sert. Marx, qui s’en réclama, qui en fit un usage matérialiste
et révolutionnaire, eut le mérite, au moins une fois, de le reconnaître. Il est
vrai qu'on était entre connaisseurs, et entre intimes, puisqu'il s’agit d’une
lettre à Engels, du 15 août 1857 : « Il se peut que je me fourre le doigt dans
l'œil, mais avec un peu de dialectique, on s’en tirera toujours. » La dialectique,
c’est sa fonction, a réponse à tout. Elle peut tout penser, tout expliquer,
tout justifier, aussi bien l’État prussien (chez Hegel) que la Révolution
(chez Marx), aussi bien le stalinisme que le trotskisme, la fin de l’histoire
(chez Kojève) autant que sa continuation «sans sujet ni fin» (chez
Althusser)... C’est l’art de se donner raison dans le langage, quand bien
même tout le réel nous donnerait tort. C’est bien commode. C’est bien vain.
Un dialecticien un peu talentueux est toujours invincible, au moins intellectuellement,
puisqu'il peut à chaque fois intégrer la contradiction même
qu’on lui oppose dans son propre développement, et la dépasser par là. Si
tout est contradictoire, que nous fait une contradiction ? Ainsi la dialectique
est sans fin. C’est le bavardage de la raison, qui fait mine de se contredire toujours
pour ne se taire jamais.

DIALLÈLE Faute logique, qui consiste à utiliser, pour démontrer une proposition,
une autre proposition qui la suppose — ce qui revient
à une pétition de principe, mais indirecte. C’est le nom grec ou savant du cercle
vicieux.

DIALOGUE Le fait de parler à deux ou à plusieurs, pour chercher une même
vérité. C’est donc un genre de conversation, mais tendue vers
l'universel plutôt que vers le singulier (comme dans la confidence) ou le particulier
(comme dans la discussion). On considère ordinairement que c’est là, au
moins depuis Socrate, une des origines de la philosophie. Parler à plusieurs, si
c’est pour chercher le vrai, c’est supposer en tous une raison commune, et
l'insuffisance en chacun de cette raison. Tout dialogue suppose l'esprit universel,
et notre incapacité à nous y installer. De là l’échange des arguments, et
la tentation, parfois, du silence.

%— 170 —
DICTATURE Au sens vague et moderne : tout pouvoir imposé par la force.
Au sens strict et premier : un pouvoir autoritaire ou militaire,
qui suspend les libertés individuelles et collectives, et même le fonctionnement
ordinaire de l’État, en principe à titre provisoire et en vue de l'intérêt commun.
Se distingue du despotisme par un statut moins monarchique (il peut y avoir
des dictatures collectives, voire démocratiques) ; de la tyrannie, par un registre
moins péjoratif. Une dictature peut être démocratiquement instituée, politiquement
justifiée et moralement acceptable, ce qu’une tyrannie n’est jamais.
Chez les Romains, par exemple, la dictature était un gouvernement d’exception,
légalement institué pour six mois afin de sauver la République. Chez
Marx ou Lénine, la dictature du prolétariat devait durer plus longtemps, mais
c'est qu'aussi l’objectif était plus élevé : il s'agissait de sauver non la République,
mais humanité. Cela déboucha partout sur la tyrannie ou le despotisme.
L’usage positif de la notion ne s’en est pas relevé.

DICTIONNAIRE C'est un recueil de définitions, rangées dans l’ordre ou le
désordre alphabétique. Les mots s’y définissent mutuellement
(chaque définition est faite de mots, qui doivent eux-mêmes en avoir
une), vouant ainsi tout dictionnaire à la circularité. C’est ce qui permet d’y
entrer n'importe où, et qui interdit de le prendre pour un système (qui serait
fondé, comme le voulait Descartes, sur l’ordre des raisons).

Faire un dictionnaire, c’est explorer comme on peut le désordre des mots et
des idées. C’est prendre le langage au sérieux, mais sans y croire. Aucune langue
ne pense ; c’est pourquoi elles nous permettent toutes de penser.

Un dictionnaire peut être linguistique (auquel cas il devrait contenir, en
principe, tous les mots de la langue concernée) ou thématique. Un dictionnaire
philosophique ne retiendra que les mots les plus importants, ou qui semblent
l'être, du langage philosophique. La subjectivité du choix redouble celle des
définitions : un tel dictionnaire, surtout s’il est écrit par un seul auteur, suppose
toujours une philosophie particulière. Tout philosophe pourrait faire le sien, et
ils seraient tous différents. On s'étonne qu’il n’y en ait pas davantage.

DIEU Chez les Grecs, et en général dans le polythéisme, c’est un être immortel
et bienheureux. Ces deux caractères, s'ils subsistent dans le
monothéisme, y prennent pourtant moins d'importance que les dimensions
ontologiques et morales : Dieu, c’est l’être suprême, créateur et incréé (il est
cause de soi), souverainement bon et juste, dont tout dépend et qui ne dépend
de rien. C’est l’absolu en acte et en personne.

%— 171 —
Les croyants reconnaissent ordinairement à leur Dieu quatre attributs principaux,
dont chacun serait sans limite : l’être (Dieu est infini), la puissance (il
est tout-puissant), la connaissance (il est omniscient), enfin la bonté ou l’amour
(il est parfaitement bon et infiniment aimant). C’est en quoi nous lui ressemblons,
s’il existe. Ne sommes-nous pas doués d’un peu d’être, d’un peu de puissance,
d’un peu de connaissance et d’amour ? Et c’est en quoi il nous ressemble,
s’il n'existe pas. L'homme est comme un Dieu fini et mortel ; Dieu, comme un
homme infini et immortel. L’anthropomorphisme n’est pas l'erreur des religions,
disait Alain, il en est plutôt « la vérité vivante ». Si Dieu ne nous ressemblait
en rien, comment aurait-il pu nous créer {\it à son image} ? Et comment pourrions-nous
y croire ?

Dieu est l’être suprême, aussi bien d’un point de vue théorique (c’est le
maximum de vérité possible) que d’un point de vue pratique (c’est le maximum
de valeur possible) : le {\it vrai Dieu}, le {\it bon Dieu}, c’est le même, et il ne serait pas
{\it Dieu} autrement. Cette conjonction du vrai et du bien, l’un et l’autre portés à
l'infini, est sans doute ce qui le définit le mieux, ou le moins mal. Dieu, c’est la
vérité qui fait norme, et la norme, à ce titre, de toutes les vérités : il est celui qui
connaît, juge et crée la {\it vraie valeur} de toutes choses. C’est le sens du sens, ou la
norme absolue.

Toute vérité qui prétend valoir absolument est ainsi théologique : de là
toutes sortes de religions (de Dieu, de l'Histoire, de la Science, de l’Inconscient...),
et bien peu d’athéismes.

DIEU DE SPINOZA C'est le contraire du précédent : la vérité éternelle et
infinie, mais sans valeur ni sens. La vérité, dirais-je,
{\it désespérément vraie} — le réel, sans phrases. Il se reconnaît à son silence.
Comme tout est vrai, le Dieu de Spinoza est aussi l’ensemble infini de tout
ce qui existe (la nature). C’est ce qui lui interdit de le créer.

DIFFÉRANCE Chez Jacques Derrida, qui invente cette graphie, et chez ses
disciples, qui reconnaissent que la notion est à peu près
indéfinissable, la différance conjoint les deux sens du verbe {\it différer} : être différent,
et remettre à plus tard. Disons que c’est la version derridienne de la différence
ontologique. J’y verrais volontiers la confirmation de quelque chose
d’important : que cette différence n’existe que du point de vue de la temporalité
(comme différ{\it a}nce), et s’abolit au présent, qui est tout (par quoi la différance
n’est rien).

%— 172 —
DIFFÉRENCE Je me souviens d’une devinette qui courait, lorsque j'étais
enfant, les cours de récréation :

«— Sais-tu quelle différence il y a entre un corbeau ?

— ?

— Il a les deux ailes pareilles, surtout la gauche. »

C'était une façon de rappeler que différence et non-différence supposent la
pluralité, soit dans l’espace (deux corbeaux différents, deux ailes semblables),
soit au moins dans le temps (si l’on compare un individu à ce qu’il fut ou sera).
C’est qu’elle suppose l’altérité : nul, au présent, n’est différent de ce qu’il est ;
on n’est différent que d’un autre, ou de soi à un autre moment. Que nous
soyons multiples, contradictoires, ambivalents, n’y change rien. C’est notre
façon d’être nous-même, et nous le sommes très exactement (que l’inconscient
ignore le principe d’identité, cela n’empêche pas qu’il lui reste soumis).

Mais si elle suppose l’altérité, la différence ne s’y réduit pas. Encore faut-il,
pour qu’on puisse parler pertinemment de différence, qu’il y ait, entre les
objets autres, au moins une certaine ressemblance ou identité. C’est ce
qu’indique Aristote : « {\it Différent} se dit de choses qui, tout en étant autres, ont
quelque identité, non pas selon le nombre, mais selon l’espèce, ou le genre, ou
par analogie » ({\it }Métaphysique, $\Delta$, 9). I n’y a pas de différence {\it entre un corbeau}
(s’il y a identité numérique, il n’y a plus de différence). Mais guère davantage
entre un corbeau et une machine à laver (sauf à les ranger dans un genre
commun : ce sont deux {\it choses} différentes). En revanche, il peut y avoir des différences,
et il y en a assurément, entre deux corbeaux (différence numérique,
identité spécifique), entre un corbeau et un merle (différence spécifique, identité
générique), entre un corvidé et un calomniateur (analogie). Bref, la différence
suppose la comparaison, et n’a de pertinence que dans la mesure où la
comparaison elle-même en a. C’est pourquoi elle en a toujours, au moins d’un
certain point de vue. Dire que deux choses sont incomparables, cela suppose
qu’on les compare, donc aussi qu’on les range dans un même genre (fût-ce
celui, quasiment indéterminé, d’être ou de chose). Ainsi tout est différent de
tout, sauf de soi. Ou plutôt {\it tout est différent de tout le reste} (car le tout, bien sûr,
est identique à soi), {\it y compris de soi à un autre moment} (puisque tout change) :
la différence est la règle, qui fait de tout être une exception.

La différence et l’impermanence vont ensemble, explique Prajnänpad : tout
est différent toujours (« chaque grain de sable ou de poussière est différent de
tous les autres »), et tout change toujours (« Rien ne demeure constant ; tout
est changement continuel »). Mais c’est la différence qui est le concept le plus
fondamental : « Qu'est-ce que le changement ? Un autre nom pour la différence.
Le changement, c’est la différence dans le temps » (R. Srinivasan, {\it Entretiens
avec Svâmi Prajnânpad}, L'Originel, 2$^\text{e}$ édition, p. 19 à 26). Principe des
%— 173 —
indiscernables et du devenir : il n’y a jamais eu deux grains de sable identiques,
ni deux instants. À la gloire d’Héraclite.

DIFFÉRENCE ONTOLOGIQUE Chez Heidegger et depuis : la différence
entre l’être et l’étant. Pour moi, qui ne
suis pas heideggérien, c’est distinguer un acte (l'acte d’{\it être}) du sujet ou du
résultat de cet acte (l’étant, c’est-à-dire ce qui est : cette table, cette chaise,
cette promenade...). Différence par nature insaisissable, puisque rien n’est en
dehors des étants, et pourtant irréductible, puisqu’un étant n’est ce qu’il est
qu'à la condition d’abord d’{\it être}. Cette différence, toutefois, suppose le
temps : elle est le temps même. Soit par exemple une promenade. Il est
certes légitime de distinguer le {\it se promener} du {\it se promenant}, qui sont en effet
différents dans le temps (puisque le promeneur ne se promène pas toujours,
puisque ce n’est pas toujours celui-ci qui se promène, puisqu'il ne fait jamais
deux fois la même promenade...). Mais cette différence ne saurait annuler
leur identité actuelle, et en acte : quand le {\it se promener}, le {\it se promenant} (le
promeneur) et la promenade ne font qu’un... Le présent est le lieu de leur
rencontre. Ainsi la différence ontologique, dans le temps, débouche sur l’indifférence,
au présent, de l’être et de l’étant : sur leur identité en acte, qui est
l'identité à soi du réel ou du devenir.

DIGNITÉ La valeur de ce qui n’a pas de prix, ni même de valeur quantifiable :
objet non de désir ou de commerce, mais de respect.
« Dans le règne des fins, écrit Kant, tout a un {\it prix} ou une {\it dignité}. Ce qui a
un prix peut aussi bien être remplacé par quelque chose d’autre à titre
d’équivalent ; au contraire, ce qui est supérieur à tout prix, ce qui par suite
n’admet pas d’équivalent, c’est ce qui a une dignité » ({\it Fondements de la métaphysique
des mœurs}, II). La dignité est une valeur intrinsèque absolue. C’est en
quoi, écrit encore Kant, « humanité elle-même est une dignité : l'homme ne
peut être utilisé par aucun homme (ni par autrui ni par lui-même) simplement
comme moyen, mais doit toujours être traité en même temps comme fin, et
c’est en cela que consiste précisément sa dignité ». La dignité d’un être humain,
c’est la part de lui qui n’est pas un moyen mais une fin, qui ne sert à rien mais
qu'il faut servir, qui n’est pas à vendre et que nul pour cela ne peut acheter. Si
l'esclavage et le proxénétisme sont indignes, ce n’est pas parce qu’ils supprimeraient
la dignité d’un individu, c’est un pouvoir qu’ils n’ont pas, mais parce
qu’ils la nient ou lui manquent de respect.

%— 174 —
DILEMME Au sens courant, c’est un choix difficile, entre deux possibilités
également insatisfaisantes. Au sens strict, qui est le sens logique,
c’est une espèce d’alternative, mais où les deux termes aboutissent à la même
conclusion : celle-ci s'impose donc. Les philosophes, écrit par exemple Montaigne,
«ont ce dilemme toujours en bouche pour consoler notre mortelle condition : ou
l’âme est mortelle, ou immortelle. Si mortelle, elle sera sans peine ; si immortelle,
elle ira en s’amendant » ({\it Essais}, II, 12, p. 551 ; voir aussi Pascal, {\it Pensées}, 409-220).
La mort ne serait donc à craindre dans aucun des deux cas. On voit qu'un
dilemme ne vaut que ce que valent les inférences qui le composent. Qu'est-ce qui
nous prouve, demande Montaigne, que l’âme ne va pas aller en empirant ?

DIMANCHE Le septième jour, celui du Seigneur ou du néant. C’est aussi
le jour du repos (quand le samedi serait plutôt celui du divertissement),
et cela revient au même. Le dimanche, il n’y a plus rien à faire qu’à
s'occuper de l'essentiel, qui est ce {\it rien} même peut-être. C’est le jour de vérité
— moins par le repos qu’il permet que par l'ennui qu’il impose. Les dimanches,
je l’ai senti enfant quand j'allais à la messe, sont jours d’angoisse ou de torpeur.
Oh ! le vide, dans les églises pleines ! On dirait que Dieu même, le septième
jour, perdit la foi.

DIONYSIAQUE Qui concerne Dionysos, dieu de la vigne et de la musique
— de l'ivresse. Nietzsche en a fait l’un des deux pôles, avec
Apollon, de son esthétique, qui est aussi son éthique. L’art dionysiaque, c’est
l’art de la démesure, de l’extase, du devenir instable, création et destruction
méêlées, du tragique, enfin, dirais-je volontiers, de tout ce qui n’est pas encore
éternel — « le plaisir de ce qui doit venir, de ce qui est futur, de ce qui triomphe
du présent, si bon soit-il » ({\it Volonté de puissance}, IV, 563). S’oppose à Apollon,
dieu de la lumière et de la beauté, et à l’art apollinien, tout de mesure et d’harmonie
(le plaisir de ce qui est {\it déjà} éternel). Nietzsche l’oppose aussi à Jésus-Christ
(« Dionysos contre le Crucifié »), comme la vie à la morale. C’est suggérer
qu’Apollon et Jésus sont du même côté : du côté de la vie éternelle, ici et
maintenant éternelle (la vie {\it sub specie aeternitatis} : du point de vue de l’éternité
ou du vrai). C’est où il faut choisir, malgré Deleuze, entre Nietzsche et Spinoza
— entre l’ivresse et la sagesse.

DISCOURS Souvent un synonyme de « parole ». Si on veut les distinguer,
on peut dire que la parole est un acte ou une faculté ; le discours,
%— 175 —
le résultat de l’une ou de l’autre. Les deux sont l’actualisation d’une
langue. Mais la parole est son actualisation en puissance ou en acte ; le discours,
son entéléchie, comme dirait un aristotélicien, ou son œuvre. C’est la parole
achevée ou parfaite. C’est pourquoi on est sensible surtout à ses imperfections.
Les paroles s’envolent. Les discours pèsent.

DISCURSIF Qui se fait par des discours et des raisonnements. La connaissance
discursive s'oppose en cela à la connaissance intuitive ou
immédiate. Une philosophie, par exemple, est toujours discursive. Mais il n’est
pas exclu qu’elle parte d’intuitions ou d’expériences qui ne le soient pas, et
débouche sur une sagesse qui ne le soit plus.

DISCUSSION Un échange contradictoire d'arguments, entre deux ou plusieurs
personnes. Cela suppose une raison commune aux uns
et aux autres, qui permette, au moins en droit, de trancher. C’est ce qui rapproche
la {\it discussion} du {\it dialogue}, au point que les deux notions soient souvent
interchangeables. Si lon veut pourtant les distinguer, il me semble qu’il faut
qu'on entende dans la {\it discussion} le choc que l’étymologie suggère ({\it discutere}, en
latin, c’est d’abord fracasser). Le dialogue est un échange d’idées ou d’arguments ;
la discussion serait plutôt leur affrontement. Le dialogue tend vers une vérité
commune, qu'aucun des participants ne prétend déjà détenir. La discussion, qui
est comme un dialogue contradictoire, suppose au contraire que chacun des participants
pense avoir raison, au moins sur tel ou tel point, et essaie d’en
convaincre les autres. Les deux supposent l’universel. C’est pourquoi on peut
parler d’une éthique de la discussion (chez Habermas ou Appel) comme d’une
éthique du dialogue (par exemple chez Marcel Conche) : discuter ou dialoguer,
cela n’a de sens qu’à supposer la capacité des locuteurs à penser sous l'horizon
d’une vérité au moins possible, donc leur égalité, au moins en droit, devant le
vrai. Mais ce n’est pas la même chose que de chercher l’universel ensemble (dans
le dialogue) ou les uns contre les autres (dans la discussion). Discuter, en ce sens
strict, c'est moins chercher l’universel avec d’autres, qu’essayer de les convaincre
qu’on le possède, c’est tout le paradoxe de la discussion, {\it en particulier}.

DISGRÂCE La perte d’une grâce, autrement dit d’une faveur, d’une protection,
d’un amour possible ou nécessaire... Le mot indique
que la grâce est première, ou devrait l’être. C’est que la vie en est une, comme
l'amour d’abord reçu, presque toujours, avant d’être mérité ou partagé. Toute
%— 176 —
grâce est gratuite. Toute disgrâce pourtant semble injuste (ce serait autrement
un châtiment), comme si la gratuité nous était due. C’est pourquoi peut-être
l'adjectif {\it disgracieux} exprime le plus souvent la laideur : parce que la beauté est
une chance par nature imméritée. Quelle plus claire image de l’injustice qu’une
jeune fille laide ?

DISJONCTION Une séparation ou une dissociation. Se dit spécialement, en
logique, d’une proposition composée de deux ou plusieurs
propositions reliées par le connecteur « ou » : {\it « p ou q »} est une disjonction.

On distingue les disjonctions {\it exclusives} et les disjonctions {\it inclusives}. Une
disjonction exclusive relie des propositions incompatibles : {\it « ou bien p, ou bien
q »}. Elle n’est vraie que si une seule des propositions qui la composent l’est (si
toutes ou plusieurs sont vraies, elle est fausse). Une disjonction inclusive relie
des propositions qui peuvent être vraies toutes les deux. Il suffit, pour qu’elle
soit vraie, qu’une seule des propositions qui la composent le soit (la conjonction
« p ou non p » est donc une tautologie), mais elle l’est encore si toutes le
sont.

Le langage courant, entre ces deux sortes de disjonction, est souvent équivoque.
Par exemple quand Groucho Marx, prenant le pouls d’un malade,
murmure : « Ou ma montre est arrêtée, ou cet homme est mort. » La disjonction,
qu'il présente comme exclusive, ne l’est pas ; les deux positions qu’elle
oppose pourraient être vraies toutes les deux.

DISSIMULATION C’est cacher ce qui ne devrait pas l’être : l'inverse en
cela de la pudeur (cacher ce qui doit l’être) et le
contraire de la franchise.

DISTINCT À la fois différent, séparé et précis. L’un des critères, chez Descartes,
de la vérité : une connaissance « claire et distincte » est
nécessairement vraie. De ces deux qualificatifs, les {\it Principes de la philosophie}
proposent les définitions suivantes : « J’appelle claire celle qui est présente et
manifeste à un esprit attentif [...] ; et distincte, celle qui est tellement précise et
différente de toutes les autres, qu’elle ne comprend en soi que ce qui paraît manifestement
à celui qui la considère comme il faut » (I, 45). Chez Leibniz, une
connaissance est distincte quand on peut énumérer ou expliquer les marques
qu'on en a, qui suffisent à la distinguer de toutes les autres. On remarquera
qu’une connaissance peut être claire sans être distincte, mais non l'inverse.
%— 177 —
L'exemple qu’en donne Descartes est celui de la douleur. Celui qui a mal sait clairement
qu'il souffre, sans toujours savoir distinctement où il a mal ni pourquoi
ou comment. Mais s’il le sait distinctement, il le sait clairement (I, 46). Leibniz
donne un autre exemple, qui est celui du jugement de goût. Je peux voir {\it clairement}
qu'un poème ou un tableau est réussi, sans connaître distinctement les éléments
de cette réussite, qui se ramènent alors, écrit Leibniz, à « un je ne sais quoi
qui nous satisfait ou qui nous choque » ({\it Discours de métaphysique}, \S 24).

Que clarté et distinction constituent un critère de vérité, cela m’a toujours
paru n'être ni clair ni distinct. Ce n’est pas une raison pour s’abandonner à
l’obscur ou au confus.

DISTINCTION C'est faire une différence, ou justifier qu’on la fasse. Par
exemple entre deux idées (voir « Distinct ») ou entre deux
individus. Pris absolument, le mot indique une espèce de supériorité, d’origine
le plus souvent sociale ou culturelle mais qui finit par sembler intrinsèque :
c’est l'élégance dans les manières, une certaine façon d’être remarquable sans
chercher à se faire remarquer, comme une politesse tellement intériorisée
qu’elle semble naturelle, comme un honneur qu’on rendrait — discrètement — à
soi-même ou à l’éducation qu’on a reçue. Pierre Bourdieu en a fait un concept
sociologique, qu’il applique spécialement aux jugements de goût ({\it De la distinction,
Critique sociale du jugement}, Minuit, 1979). Dans une société donnée, le
jugement de goût juge aussi celui qui l’énonce : aimer telle œuvre d’art plutôt
que telle autre, par exemple l’opéra plutôt que la variété, c’est une manière de
se {\it distinguer}, c’est-à-dire de se faire valoir en marquant sa différence. On aurait
tort d’en conclure que Michel Berger est l’égal de Mozart.

DIVERSION Cest détourner l'attention de l’autre, spécialement d’un
ennemi ou d’un adversaire. En philosophie, c’est surtout le
nom montanien du {\it divertissement} : parce que le chagrin est un adversaire aussi,
qu'on ne peut vaincre que par ruse ({\it Essais}, III, 4 : « De la diversion »). C’est
moins une faiblesse qu’une stratégie, moins une tentative d’oublier notre néant,
comme chez Pascal, qu’un refus légitime de s’en laisser accabler. Art d’esquive
et d'hygiène. Disons que c’est un divertissement lucidement assumé : c’est
refuser de prendre le tragique au sérieux.

DIVERTISSEMENT Une façon de s’occuper l'esprit pour oublier l’essentiel :
le peu que nous sommes, le néant qui nous
%— 178 —
attend. C’est comme un détournement d’attention, une distraction volontairement
entretenue, une diversion métaphysique. L’idée se trouve déjà chez Montaigne
(« peu de chose nous divertit et détourne, car peu de chose nous tient »),
mais c’est surtout l’un des grands concepts pascaliens. Le divertissement est la
marque de notre misère (qu’il suffise de si peu pour nous occuper !), en même
temps qu’une tentative pour la masquer. Son contraire n’est pas le sérieux, qui
n’est souvent qu’un divertissement comme un autre (« le chancelier est grave et
revêtu d’ornements... »), mais l’ennui ou l’angoisse ({\it Pensées}, 622-131). C’est
pourquoi nous aimons tant « le bruit et le remuement », qui nous empêchent
de penser à nous et à la mort. « Ainsi s’écoule toute la vie : on cherche le repos
en combattant quelques obstacles, et si on les a surmontés le repos devient
insupportable par l'ennui qu’il engendre. Il en faut sortir et mendier le
tumulte » (136-139, « Divertissement »). Se divertir, c’est se détourner de soi,
de son néant, de sa mort. Et cela fait comme un néant redoublé.

Notre époque n’y voit guère qu’un délassement agréable. On remarquera
pourtant que le divertissement, en ce sens philosophique, n’est pas forcément
un repos, ni même un loisir ou un amusement : on peut se divertir aussi bien,
et peut-être mieux, en s’étourdissant de travail ou de soucis. L'important est
d'oublier que rien ne l’est, hormis l'essentiel, que nous voulons oublier.

DOCIMOLOGIE La science des examens et des évaluations (de {\it dokimè},
épreuve). Sert à ceux qui notent, dans l'intérêt, en principe,
de ceux qui sont notés. En pratique, elle est ignorée de presque tous. Si
l’on se met à évaluer les évaluateurs, où va-t-on ?

DOGMATISME Au sens courant : un penchant pour les dogmes, une incapacité
à douter de ce qu’on croit. C’est aimer la certitude
plus que la vérité, au point de tenir pour certain tout ce qu’on juge être vrai.
Au sens philosophique : toute doctrine qui affirme l'existence de connaissances
certaines. C’est le contraire du scepticisme. Le mot, en ce sens technique,
est sans valeur péjorative. La plupart des grands philosophes sont dogmatiques
(le scepticisme est l’exception), et ils ont pour cela de bonnes raisons,
à commencer par la raison elle-même. Qui peut douter de sa propre existence,
de la vérité d’un théorème mathématique (s’il en comprend la démonstration),
ou, aujourd’hui, du mouvement de la Terre autour du Soleil ? On remarquera
pourtant qu’une incapacité à douter ne prouve rien (qui pouvait douter, il y a
dix siècles, de l’immobilité de la Terre ou de la vérité universelle du postulat
d’Euclide ?). C’est ce qui donne raison aux sceptiques, tout en leur interdisant
%— 179 —
de l’être dogmatiquement. La certitude que rien n’est certain serait aussi douteuse
que les autres, ou plutôt elle le serait davantage — puisqu'elle se contredit.

Le dogmatisme relève principalement de la connaissance ; mais il arrive
qu'il touche aussi, par là même, à la morale. C’est ce qui m’a amené à distinguer
deux dogmatismes différents : un {\it dogmatisme théorique}, ou dogmatisme
en général, qui porte sur les connaissances ; et un {\it dogmatisme pratique}, qui
porte sur les valeurs. En quoi ce dernier est-il un dogmatisme ? En ceci qu’il
considère que les valeurs sont des vérités, qui peuvent à ce titre être connues
avec certitude. La vérité suffit alors à juger de la valeur d’une action, et le
permet seule. Ainsi, chez Platon ou Lénine. Si le bien est à connaître, le mal
n’est qu’une erreur : nul n’est méchant ni réactionnaire volontairement. À quoi
bon la démocratie ? Une vérité, cela ne se vote pas ! À quoi bon les libertés
individuelles ? Une vérité, cela ne se choisit pas ! Par quoi le dogmatisme pratique
aboutit tout naturellement, chez Lénine (en pratique) comme déjà chez
Platon (en théorie), à ce que nous appelons aujourd’hui le totalitarisme. Ce
n'est pas vrai du dogmatisme seulement théorique, et suffit à les distinguer.
Une vérité, même à supposer qu’on la connaisse avec certitude, pourquoi faudrait-il
s’y soumettre ? Comment la connaissance de ce qui est suffirait-elle à
décider de ce qui doit être ? Pourquoi la vérité devrait-elle commander ? Comment
pourrait-elle choisir ? La biologie ne dit rien sur la valeur de la vie, ni sur
celle du suicide. Le marxisme, s’il était une science, ne dirait pas davantage sur
la valeur respective du capitalisme et du communisme. Mais alors c’est aux
individus de savoir ce qu’ils veulent, non à la science de vouloir ce qu’elle sait,
ou croit savoir.

DOGME Une vérité qu’on s’impose, ou qu’on prétend imposer aux autres.
Se distingue en cela de l’évidence (qui s'impose d’elle-même) et de
l'esprit critique (qui suppose le doute). Double aveu de fragilité, ou plutôt
double dénégation. Tout dogme est bête, et rend bête.

DON Ce qu’on donne ou qu’on reçoit gratuitement. C’est l’échange premier,
celui d’avant le commerce, ou le premier terme de l'échange.
Échange ? C’est que la gratuité, presque toujours, s'accompagne de réciprocité.
Dans son {\it Essai sur le don}, Marcel Mauss a montré qu’il y avait là tout un système
d’obligations et de contreparties. Dans la plupart des sociétés archaïques,
explique-t-il, tout don doit être suivi d’un contre-don, c’est-à-dire d’un don en
retour : « Ces prestations et contre-prestations s’engagent sous une forme
plutôt volontaire, par des présents, des cadeaux, bien qu’elles soient au fond

%— 180 —
rigoureusement obligatoires, sous peine de guerre privée ou publique. » Il m’a
fait un cadeau ou une politesse ; je suis obligé de les accepter et de lui rendre
l'équivalent. L'absence de contre-don marquerait la rupture du contrat social et
risquerait de mener à la violence. Nous n’en sommes plus là : nous avons
inventé le commerce, qui nous dispense de gratitude. Soit. Mais point tout à
fait de contre-don : puisque le prix payé en est un. Surtout, dans la sphère non
marchande des rapports humains, le don et le contre-don continuent de régner.
Il est mal venu de ne jamais rendre les invitations, de ne pas offrir un cadeau à
ses hôtes, de ne pas payer à son tour la tournée. Le don reçu nous oblige : le
contre-don, encore aujourd’hui, est une espèce de devoir. Qui sait si tout
devoir n’est pas d’abord un contre-don ?

On parle également de {\it don} pour une disposition qu’on a reçue à la naissance,
et d'autant plus qu’elle est plus rare et plus précieuse : c’est comme un
talent naturel, ou plutôt comme la base naturelle de ce qui, cultivé, travaillé,
pourra devenir un talent. D'où une espèce d’inégalité originelle, qui rend la
notion désagréable mais ne suffit pas à l’invalider. {\it « Les dons n'existent pas »},
titrait il y a quelques années un hebdomadaire de gauche ; il y voyait une leçon
de justice. C'était bien sûr une sottise. Pourquoi la nature serait-elle juste ?
Pourquoi Mozart serait-il de droite ? Mieux vaut travailler et admirer.

DOUCEUR Le refus de faire souffrir, ou la volonté de ne faire souffrir, quand
c’est absolument nécessaire, que le moins possible. Se distingue
en cela de la compassion (qui suppose une souffrance déjà existante), et la prolonge.
Être compatissant, c’est souffrir de la souffrance de l’autre. Être doux,
c’est ne rien faire pour l’augmenter.

La compassion s'oppose à l’égoïsme, à l'indifférence, à la dureté. La douceur,
à la violence, à la cruauté, à la brutalité. À la colère? Pas toujours,
puisqu'il en est de justes et de nécessaires. Le doux, remarque Aristote, est
plutôt celui qui se met en colère quand il faut, comme il faut, contre qui il faut
— celui qui n’est ni colérique ni impassible, ni sauvage ni complaisant ou mou
({\it Éthique à Nicomaque}, IV, 11). C’est que la douceur est une force, non une
faiblesse : c’est la force qui renonce à la violence, ou qui la limite le plus qu’elle
peut. Par quoi c’est la plus douce des vertus, comme il convient, mais aussi
lune des plus nécessaires et des plus émouvantes. C’est la vertu des pacifiques.

DOULEUR L'un des affects fondamentaux. C’est le contraire du plaisir, et
pourtant tout autre chose que son absence. Sur ce qu’elle est, le
corps nous en dit assez, qui vaut mieux qu’une définition. On souffre aussi de
%— 181 —
l’âme ? Sans doute, et l’expérience, là encore, tient lieu de définition. La douleur
est plus qu’une sensation pénible ou désagréable ; c’est une sensation
qu’on ne peut oublier, qui s'impose absolument, qui interdit tout bien-être,
toute détente, tout repos, enfin qu’on ne peut supporter, quand elle est vive,
que dans l’horreur ou l’héroïsme.

Faut-il la distinguer de la souffrance ? Certains s’y sont risqués. Par exemple
Michel Schneider, dans son beau livre sur Schumann : « La souffrance a un
sens, la douleur n’en a pas. La douleur est davantage physique ou métaphysique,
la souffrance, morale ou psychique... Nous avons telle ou telle souffrance,
mais c’est la douleur qui s'empare de nous. Il existe un travail possible
de la souffrance, une élaboration : le deuil. Il n’y a pas de travail de la douleur. »
Ainsi Schumann serait le musicien de la douleur, comme Schubert peut-être
celui de la souffrance. Cela me rendrait la souffrance plus sympathique, mais
ne suffit pas toutefois à me convaincre. Je n’ai connu que souffrances et douleurs
indistinctes.

DOUTE Le contraire de la certitude. Douter, c’est penser, mais sans être
sûr de la vérité de ce qu’on pense. Les sceptiques en font l’état
ultime de la pensée. Les dogmatiques, le plus souvent, un passage obligé. Ainsi,
chez Descartes : son doute méthodique et hyperbolique (c’est-à-dire exagéré : il
tient pour faux tout ce qu’il sait être douteux) n’est qu’un moment provisoire,
dans sa quête de la certitude. Descartes en sort par le {\it cogito}, qui n’est pas douteux,
et par Dieu, qui n’est pas trompeur. Mais son Dieu est douteux, quoi
qu'il en dise, et rien n’exclut que le {\it cogito} soit trompeur. Ainsi le doute renaît
toujours. On n’en sort que par le sommeil ou l’action.

DROIT Une possibilité garantie par la loi (le droit à la propriété, à la sûreté,
à l'information...) ou exigée par la conscience (les droits de
l’homme). Pris absolument, le droit est l’ensemble des lois qui limitent et
garantissent — l’un ne va pas sans l’autre — ce qu’un individu peut faire, à l’intérieur
d’une société donnée, sans encourir de sanction et sans que quiconque
puisse l’en empêcher sans en encourir. Cela suppose un système de contraintes,
donc une répression au moins possible : il n’y a de droit, en ce sens, que par la
force, et telle est la fonction de l’État. Le droit naturel n’est qu’une abstraction ;
les droits de l’homme, qu’un idéal. Seul le droit positif permet de passer, grâce
à l’État, du {\it droit} au {\it fait}. C’est une raison forte de préférer l’état civil, même
injuste, à l’état de nature : mieux vaut un droit imparfait que pas de droit du
tout.

%— 182 —
Chacun sait qu’il n’y a pas de droits sans devoirs. Mais point, comme on le
croit parfois, parce qu’on ne pourrait bénéficier de ceux-là qu’à la condition de
respecter d’abord ceux-ci. Un tortionnaire, on n’a pas le droit pour autant de le
torturer. Un voleur, pas le droit de le voler. Qu'ils n’aient pas fait leur devoir
ne nous dispense aucunement, fût-ce vis-à-vis d’eux, des nôtres. Mais il y a
plus. Un nouveau-né ou un débile profond n’ont aucun devoir, et pourtant
une multitude de droits. C’est dire assez que mon droit n’est pas défini par mes
devoirs, mais par les devoirs des autres à mon égard. Si j’ai aussi des devoirs, ce
qui est bien clair, ce n’est pas parce que j’ai des droits, mais parce que les autres
en ont. Ainsi le droit, pris absolument, fixe les droits et les devoirs de chacun,
les uns vis-à-vis des autres et de tous. C’est ce qui nous permet de vivre librement
ensemble, par la limitation, rationnellement instituée, de notre liberté.
Ma liberté s'arrête, selon la formule fameuse, où commence celle des autres.
« Le droit, écrit Kant, est la limitation de la liberté de chacun à la condition de
son accord avec la liberté de tous, en tant que celle-ci est possible selon une loi
universelle. » Mais il n’existe, c’est ce qu’on appelle le droit positif, que par des
lois particulières.

DROIT NATUREL Ce serait un droit inscrit dans la nature ou la raison,
indépendamment de toute législation positive: un
droit d’avant le droit, qui serait universel et servirait de fondement ou de
norme aux différents droits positifs. En pratique, chacun y met un peu ce qu’il
veut (par exemple, chez Locke, la liberté, l'égalité, la propriété privée, la peine
de mort...), ce qui est bien commode mais ne permet guère de résoudre
quelque problème effectif que ce soit. Que disent la nature ou la raison sur
l'avortement, l'euthanasie, la peine de mort ? Sur le droit du travail et des
affaires ? Sur le PACS et la libéralisation des drogues douces ? Sur le meilleur
type de régime ou de gouvernement ? On a pu fonder sur le droit naturel, selon
la conception qu’on s’en faisait, aussi bien la supériorité de la monarchie
absolue (chez Hobbes) que celle de la démocratie (chez Spinoza). C’est dire
assez la malléabilité de la notion. Les droits de l’homme ? Ce n’est pas la nature
qui les définit mais l’humanité, non la raison mais la volonté. Pour ma part, si
l’on tient à parler d’un droit naturel, j'aurais tendance à dire que ce n’est pas un
droit du tout, mais le simple règne des faits ou des forces. « Le droit naturel de
la nature entière et conséquemment de chaque individu s’étend jusqu'où va sa
puissance, écrit Spinoza, et donc tout ce que fait un homme suivant les lois de
sa nature propre, il le fait en vertu d’un droit de nature souverain, et il a sur la
nature autant de droit qu’il a de puissance » ({\it Traité politique}, II, 4). C’est ainsi,
précise-t-il ailleurs, que « les grands poissons mangent les petits en vertu d’un
%— 183 —
droit naturel souverain » ({\it Traité théologico-politique}, XVI). C’est la loi de la
jungle, dont seul le droit positif nous sépare.

DROIT POSITIF L'ensemble des lois effectivement instituées, dans une
société donnée, quel que soit par ailleurs le mode (coutumier
ou écrit, démocratique ou monarchique...) de cette institution. C’est un
{\it droit} qui existe {\it en fait}.

DROITE/GAUCHE Enfant, j'avais demandé à mon père ce que cela signifiait,
dans la vie politique, qu'être de droite ou de
gauche. Il me répondit : « Être de droite, c’est vouloir la grandeur de la France.
Être de gauche, c’est vouloir le bonheur des Français. » Je ne sais si la formule
était de lui. Il n’aimait pas les Français, ni les humains en général. Il me répétait
toujours qu’on n’est pas sur Terre pour être heureux. La définition, dans sa
bouche, était de droite. C’est pourquoi elle lui plaisait. Mais un homme de
gauche pourrait également s’y retrouver, s’il croit peu ou prou au bonheur.
C’est pourquoi elle ne me déplaît pas. « Car enfin, dira notre homme de
gauche, la France et la grandeur ne sont que des abstractions dangereuses. Le
bonheur des Français, voilà qui mérite autrement d’être poursuivi ! » Cela ne
prouve pourtant pas que cette définition suffise, ni même qu’elle en soit une.
Grandeur et bonheur n’appartiennent à personne.

Le temps a passé : mes enfants m'ont interrogé à leur tour... Je répondis
comme je pus, autour de quelques différences qui me paraissaient essentielles.
Sur le point de les mettre noir sur blanc, j’en perçois mieux les limites ou les
approximations. Cette logique binaire, qu’impose le principe majoritaire, ne
correspond ni à la complexité ni à la fluctuation des positions politiques effectives.
Une même idée peut être soutenue dans des camps opposés (par exemple
l’idée d’une Europe fédérale, ou son refus souverainiste, qu’on rencontre
aujourd’hui à droite comme à gauche), ou bien passer d’un camp à un autre
(ainsi l’idée de Nation, plutôt de gauche au {\footnotesize XIX$^\text{e}$} siècle, plutôt de droite au
{\footnotesize XX$^\text{e}$}). Mais faut-il pour autant renoncer à nos deux catégories, si fortement
ancrées dans la tradition démocratique, depuis 1789 (on sait qu’elles sont nées
de la disposition spatiale des députés, lors de l’Assemblée constituante, qui se
réunissaient, par affinité politique, à droite ou à gauche du président de
séance), et si omniprésentes, encore aujourd’hui, dans le débat démocratique ?
Faut-il les juger obsolètes ? Les remplacer par d’autres ? C’est ce que certains
ont tenté. L'opposition n’est plus entre la droite et la gauche, disait De Gaulle
en 1948, mais entre ceux qui sont en haut, parce qu’ils ont une vision, et ceux
%— 184 —
« qui sont en bas et qui s’agitent dans les marécages »... J'y vois une idée de
droite, comme dans toute tentative de récuser ce que cette opposition, même
schématique comme elle est forcément, garde d’éclairant, de structurant,
d’opératoire. Quel politologue pourrait s’en passer ? Quel militant ? Au reste
Alain, dès 1930, avait déjà répondu : « Lorsqu'on me demande si la coupure
entre partis de droite et partis de gauche, hommes de droite et hommes de
gauche, a encore un sens, la première idée qui me vient est que l’homme qui
pose cette question n’est certainement pas un homme de gauche » (Propos de
décembre 1930). J'ai la même réaction, et c’est ce qui m’oblige, entre la droite
et la gauche, à chercher quelques différences, même fluctuantes, même relatives,
qui donnent un sens à cette opposition.

La première différence est sociologique. La gauche représente plutôt ce que
les sociologues appellent les couches populaires, disons les individus les plus
pauvres, ou les moins riches, ceux qui ne possèdent rien, ou presque rien, les
prolétaires, comme disait Marx, qu’il vaut mieux aujourd’hui appeler les salariés.
La droite, tout en recrutant aussi dans ces milieux (il le faut bien : ils sont
majoritaires), a plus de facilité avec les indépendants, qu’ils soient ruraux ou
urbains, ceux qui possèdent leur terre ou leur instrument de travail (leur boutique,
leur atelier, leur entreprise), ceux qui font travailler les autres ou qui
travaillent pour eux-mêmes plutôt que pour un patron. Cela dessine comme
deux peuples, ou plutôt comme deux pôles : les paysans pauvres et les salariés
d’un côté ; les bourgeois, les propriétaires terriens, les cadres dirigeants, les professions
libérales, les artisans et les commerçants de l’autre. Avec tous les intermédiaires
que l’on veut, entre ces deux mondes (les fameuses « classes
moyennes »), tous les échanges que l’on veut, entre les deux camps (les transfuges,
les indécis). Que la frontière soit poreuse, et peut-être de plus en plus,
c’est une affaire entendue. Mais elle n’en est pas moins frontière pour autant.
Qu’aucun des deux camps n’ait le monopole d’aucune classe, c’est une évidence
(on se souvient que le Front national, du temps de sa sinistre splendeur,
était en passe de devenir le premier parti ouvrier de France). Mais qui ne suffit
pas, me semble-t-il, à abolir tout à fait cette dimension sociologique de la question.
Même en drainant des voix chez les plus pauvres, la droite n’a jamais
réussi, du moins en France, à pénétrer vraiment le syndicalisme ouvrier. La
gauche, chez les patrons et les grands propriétaires terriens, fait moins de 20 %
des voix. J'ai quelque peine, dans l’un et l’autre cas, à n’y voir qu’une coïncidence.

La deuxième différence est plutôt historique. La gauche, depuis la Révolution
française, se prononce en faveur des changements les plus radicaux ou les
plus ambitieux. Le présent ne la satisfait jamais ; le passé, moins encore : elle se
veut révolutionnaire ou réformiste (et la révolution est plus à gauche, bien sûr,
%— 185 —
que la réforme). C’est sa façon à elle d’être progressiste. La droite, sans être
contre le progrès (personne n’est contre), se plaît davantage à défendre ce qui
est, voire, cela s’est vu, à restaurer ce qui était. Parti du mouvement d’un côté,
parti de l’ordre, de la conservation ou de la réaction de l’autre. Avec, là encore,
tout ce qu’on veut d'échanges et de nuances entre les deux, surtout dans la dernière
période (la défense des avantages acquis tend parfois à l'emporter, à
gauche, sur la volonté réformatrice, comme la volonté de réformes libérales, à
droite, sur le conservatisme), mais qui ne suffisent pas à annuler la différence
d'orientation. La gauche se veut essentiellement progressiste. Le présent
l’ennuie ; le passé lui pèse : elle en ferait volontiers, comme le chante encore
l’Internationale, « table rase ». La droite est plus volontiers conservatrice. Le
passé lui est un patrimoine, qu’elle veut préserver, plutôt qu’un poids. Le présent
lui paraît supportable : puisse l’avenir lui ressembler ! Dans la politique, la
gauche voit surtout l’occasion d’un changement possible ; la droite, d’une
continuité nécessaire. Ils n’ont pas le même rapport au temps. C’est qu’ils n’ont
pas le même rapport au réel, ni à l'imaginaire. La gauche penche, parfois dangereusement,
vers l’utopie. La droite, vers le réalisme. La gauche est plus
idéaliste ; la droite, plus soucieuse d’efficacité. Cela n’empêche pas un homme
de gauche d’être lucide ou de se vouloir efficace, ni un homme de droite d’avoir
des idéaux généreux. Mais ils risquent alors d’avoir fort à faire, l’un et l’autre,
pour convaincre leur propre camp...

La troisième différence est proprement politique. La gauche se veut du côté
du peuple, de ses organisations (les partis, les syndicats, les associations), de sa
représentation (le Parlement). La droite, sans mépriser pour autant le peuple,
est davantage attachée à la Nation, à la patrie, au culte du terroir ou du chef.
La gauche a une certaine idée de la République. La droite, une certaine idée de
la France. La première penche volontiers vers la démagogie. La seconde, vers le
nationalisme, la xénophobie ou l’autoritarisme. Cela n'empêche pas les uns et
les autres d’être souvent de parfaits démocrates, ni de tomber parfois dans le
totalitarisme. Mais ils n’ont pas les mêmes rêves, ni les mêmes démons.

Quatrième différence : une différence économique. La gauche refuse le
capitalisme, ou ne s’y résigne que de mauvais gré. Elle fait davantage confiance
à l'État qu’au marché. Elle nationalise dans l'enthousiasme, ne privatise qu'à
regret. La droite, c’est évidemment l’inverse (au moins aujourd’hui) : elle fait
davantage confiance au marché qu’à l’État, et c’est pourquoi elle est tellement
favorable au capitalisme. Elle ne nationalise que contrainte et forcée, privatise
dès qu’elle le peut. Là encore cela n'empêche pas qu’un homme de gauche
puisse être libéral, même au sens économique du terme (voyez Alain), ni qu’un
homme de droite ait le sens de l’État ou du service public (voyez De Gaulle).
Mais la différence n’en demeure pas moins, à l’échelle des grands nombres ou
%— 186 —
des orientations fondamentales. L'État providence est à gauche ; le marché, à
droite. La planification est à gauche ; la concurrence et l’émulation, à droite.

On remarquera que la droite, sur ces questions économiques et dans la dernière
période, l’a clairement emporté, au moins intellectuellement. Le gouvernement
Jospin a privatisé davantage que ceux de Juppé ou de Balladur (il est
vrai en s’en vantant moins), et il n’y a plus guère que l’extrême gauche,
aujourd’hui, qui propose de nationaliser quelque entreprise que ce soit. On
s'étonne, dans ces conditions, que la gauche ait si bien résisté, politiquement,
voire l’ait emporté à plusieurs reprises. C’est que la sociologie lui est plutôt
favorable (il y a de plus en plus de salariés, de moins en moins
d’indépendants). C’est aussi qu’elle l'avait emporté précédemment sur d’autres
fronts, qui lui font comme un capital de sympathie. La liberté d’association,
l'impôt sur le revenu et les congés payés sont des inventions de gauche, que
personne aujourd’hui ne remet en cause. L’impôt sur la fortune, plus récemment,
est encore une invention de gauche ; la droite, qui voulut labolir, s’en
est mordu les doigts. Et qui osera toucher à la semaine de 35 heures ? Mais si
la gauche s’en sort si bien, c’est aussi, et peut-être surtout, qu’elle a compensé
cette défaite intellectuelle (dont il faut lui savoir gré d’avoir pris acte : être de
gauche, disait Coluche, cela ne dispense pas d’être intelligent) par une espèce
de victoire morale ou spirituelle. J’écrirais volontiers que toutes nos valeurs
aujourd’hui sont de gauche, puisqu'elles se veulent indépendantes de la
richesse, du marché, de la nation, puisqu'elles se moquent des frontières et des
traditions, puisqu'elles n’adorent que l'humanité et le progrès. Ce serait bien
sûr aller trop loin. Il reste qu’on est de gauche, surtout chez les intellectuels,
pour des raisons d’abord morales. On serait plutôt de droite par intérêt ou pour
des raisons économiques. « Vous n’avez pas le monopole du cœur ! », lança un
jour, lors d’un débat fameux, un homme politique de droite à son adversaire
socialiste. Qu'il ait eu besoin de le rappeler est révélateur. Nul homme de
gauche n’aurait eu l’idée d’une telle formule, tant elle lui paraîtrait évidente, ou
plutôt tant il va de soi, de son point de vue, que le cœur, en politique aussi, bat
à gauche. De là, dans le débat politique, en tout cas en France, une curieuse
asymétrie. Vous ne verrez jamais un homme de gauche contester qu’il le soit,
ni récuser la pertinence de cette opposition. Combien d'hommes de droite, au
contraire, prétendent que ces notions n’ont plus de sens, ou que la France,
comme disait l’un d’entre eux, veut être gouvernée au centre ? C’est qu'être de
gauche passe pour une vertu : la gauche serait généreuse, compatissante, désintéressée...
Être de droite, sans être un vice, passe plutôt pour une petitesse : la
droite serait égoïste, dure aux faibles, âpre au gain... Qu'il y ait là une conception
naïve de la politique, ce n’est guère niable, mais ne suffit pas à annuler
cette asymétrie. On se flatte d’être de gauche. On avoue être de droite.

%— 187 —
Cela nous conduit aux dernières différences que je voulais évoquer. Elles
sont plutôt philosophiques, psychologiques ou culturelles : elles opposent
moins des forces sociales que des mentalités ; elles portent moins sur des prorammes
que sur des comportements, moins sur des projets que sur des valeurs.
À gauche, le goût de l'égalité, de la liberté des mœurs, de la laïcité, de la défense
des plus faibles, fussent-ils coupables, de l’internationalisme, des loisirs, du
repos (les congés payés, la retraite à 60 ans, la semaine de 35 heures...), de la
compassion, de la solidarité. À droite celui de la réussite individuelle, de la
liberté d’entreprendre, de la religion, de la hiérarchie, de la sécurité, de la
patrie, de la famille, du travail, de l'effort, de l’émulation, de la responsabilité.
La justice ? Ils peuvent s’en réclamer les uns et les autres. Mais ils n’en ont pas
la même conception. À gauche, la justice est d’abord équité : elle veut les
hommes égaux, non seulement en droits mais en fait. Aussi se fait-elle volontiers
réparatrice et égalitariste. Sa maxime serait : {\it « À chacun selon ses besoins. »}
Celui qui a déjà la chance d’être plus intelligent ou plus cultivé, de faire un travail
plus intéressant ou plus prestigieux, pourquoi faudrait-il en outre qu’il soit
plus riche ? Il l’est pourtant, en tout pays, et il n’y a plus que l’extrême gauche
qui s’en étonne. Le reste de la gauche, toutefois, ne s’y résigne pas sans un peu
de mauvaise conscience. Toute inégalité lui semble suspecte ou regrettable : elle
ne la tolère qu’à regret, faute de pouvoir ou de vouloir tout à fait l'empêcher. À
‘ droite, la justice est plutôt conçue comme une sanction ou une récompense.
L'égalité des droits suffit, qui ne saurait annuler l’inégalité des talents et des
performances. Pourquoi les plus doués ou les plus travailleurs ne seraient-ils pas
plus riches que les autres ? Pourquoi ne feraient-ils pas fortune ? Pourquoi leurs
enfants ne pourraient-ils profiter de ce que leurs parents ont amassé ? La justice,
pour eux, est moins dans légalité que dans la proportion. Aussi se fait-elle
volontiers élitiste ou sélective. Sa maxime serait : {\it « À chacun selon ses mérites. »}
Protéger les plus faibles ? Soit. Mais pas au point d’encourager la faiblesse, ni
de décourager les plus entreprenants, les plus talentueux ou les plus riches !

Ce ne sont que des tendances, qui peuvent traverser chacun d’entre nous,
chaque courant de pensées (dans les Évangiles, par exemple, la parabole du
jeune homme riche est de gauche, celle des talents, de droite), mais qui me
paraissent au total assez claires pour qu’on puisse à peu près s’y retrouver. La
démocratie, parce qu’elle a besoin d’une majorité, pousse à cette bipolarisation.
Mieux vaut en prendre acte que faire semblant de l’ignorer. Non, bien sûr,
qu’un parti ou qu’un individu doive forcément, pour être de gauche ou de
droite, partager toutes les idées qui caractérisent — mais d’un point de vue régulateur
plutôt que constitutif — l’un ou l’autre courant. C’est à chacun, entre ces
deux pôles, d'inventer son chemin, sa position propre, ses compromis, ses équilibres.
Pourquoi faudrait-il, pour être de gauche, se désintéresser de la famille,
%— 188 —
de la sécurité ou de l'effort ? Pourquoi, parce qu’on est de droite, devrait-on
renoncer aux réformes ou à la laïcité ? Droite et gauche ne sont que des pôles,
je l’ai dit, et nul n’est tenu de s’enfermer dans l’un des deux. Ce ne sont que des
tendances, et nul n’est tenu de s’amputer totalement de l’autre. Mieux vaut être
ambidextre que manchot. Mais mieux vaut être manchot d’un bras que de
deux.

Reste, qu’on soit de droite ou de gauche, à l’être intelligemment. C’est le
plus difficile. C’est le plus important. L'intelligence n’est d’aucun camp. C’est
pourquoi nous avons besoin des deux, et de l’alternance entre les deux.

DROITURE La qualité de ce qui est droit, mais en un sens moral et métaphorique :
c’est une honnêteté qui va au plus court.

DUALISME Toute doctrine qui pose l’existence de deux principes irréductibles
l’un à l’autre, et spécialement de deux substances distinctes,
qui seraient la matière et l’esprit. C’est le contraire du monisme.
S’applique en particulier à l’être humain, ou plutôt à la conception que l’on
s’en fait : être dualiste, c’est affirmer que l’âme et le corps sont deux choses différentes,
qui peuvent ou pourraient, au moins en droit, exister séparément.
Ainsi chez Descartes, pour qui le corps ne peut pas davantage penser que l’âme
ne saurait être étendue, ce qui suppose (puisque le corps est étendu, puisque
l’âme pense) qu’ils soient réellement distincts. À quoi l’on oppose communément
que l’âme et le corps, loin d’être séparés, comme le voudrait Descartes,
sont au contraire en interaction étroite, comme l’enseigne l’expérience commune
et comme le confirment, aujourd’hui, les progrès des médecines dites
« psychosomatiques ». C’est méconnaître la pensée de Descartes, et lui
opposer, bien sottement, ce qu’il n’a cessé lui-même de répéter et qui lui donne
raison : « que je ne suis pas seulement logé dans mon corps, ainsi qu’un pilote
en son navire, mais outre cela que je lui suis conjoint très étroitement, et tellement
confondu et mêlé que je compose comme un seul tout avec lui » ({\it Méditations}, VI).
Que l'âme agisse sur Le corps, que le corps agisse sur l’âme, bref que
les deux substances en l’homme soient intimement unies et mêlées, la moindre
action, la moindre passion, la moindre douleur le rendent manifeste. Mais loin
que cela réfute le dualisme, cela le confirmerait plutôt : l’âme et le corps ne
peuvent être en interaction qu’à la condition d’abord d’être distincts. C’est en
quoi le sot reproche qu’on fait à Descartes manifeste aussi, et plus profondément,
une méconnaissance du psychosomatisme même dont on prétend se
réclamer. Car l’âme ne peut agir sur le corps, et le corps sur l’âme, que si l’âme
%— 189 —
et le corps sont deux choses différentes : le psychosomatisme, loin d’invalider le
dualisme, le suppose. Si l’âme et le corps sont une seule et même chose, comme
dit Spinoza et comme je le crois, la notion de phénomène psychosomatique n’a
plus de sens : autant dire psycho-psychique, ou somato-somatique, c’est-à-dire
rien. Le dualisme a encore de beaux jours devant lui.

DURÉE  Durer, c’est continuer d’être. Ainsi chez Spinoza : « La durée, écrit-il
dans l’{\it Éthique}, est une continuation indéfinie de l'existence »
({\it Éthique}, II, déf. 5). Ainsi chez Bergson : « L'univers dure », et il n’y aurait pas
de temps autrement ; « la durée immanente au tout de l’univers » doit exister
d’abord, comme nous en elle, pour que nous puissions, la découpant abstraitement,
parler de temps ({\it L'évolution créatrice}, I).

On remarquera que toute durée effective est présente (puisque le passé n’est
plus, puisque l’avenir n’est pas encore), et donc indivisible (comment diviser le
présent ?). La durée se distingue en cela :

— du temps abstrait, qui serait la somme, indéfiniment divisible, d’un passé
et d’un avenir ;

— du temps vécu ou de la temporalité, qui supposent la mémoire et
Panticipation ;

— enfin de l'instant, qui serait un présent discontinu et sans durée.

La durée est le présent même, en tant qu’il continue. Elle est la perpétuelle
{\it présentation} de la nature. C’est donc le temps réel : le temps de l’être en train
d’être, le temps de l’étant — ce que j'appelle l’être-temps.

DYNAMISME Au sens courant: manifestation d’une force, d’une puissance
{\it (dunamis)}, d’une énergie. C’est le contraire de la mollesse
ou de la langueur.

Au sens philosophique : toute doctrine pour laquelle la nature, loin de se
réduire à l’étendue et au mouvement, suppose aussi l’existence d’une force ou
d’une énergie intrinsèques. Ainsi chez Leibniz, contre Descartes (voir par
exemple le {\it Discours de métaphysique}, \S 17 et 18). On notera que le dynamisme,
en ce sens, s'oppose au mécanisme, pris en son sens étroit, mais point forcément
au matérialisme. Rien n'empêche que la matière soit énergie, ni que
l'énergie soit matérielle. Le stoïcisme, par exemple, est un matérialisme dynamiste.
%{\footnotesize XIX$^\text{e}$} siècle — {\it }

