
\chapter{IJKL}
%{\footnotesize XIX$^\text{e}$} siècle — {\it }
\section{Icône}
%ICÔNE
Une image signifiante, ou un signe imagé. Proche en ce sens de {\it symbole},
mais avec quelque chose de plus immédiatement figuratif.
C’est que les symboles portent le plus souvent sur des abstractions ; les icônes,
sur des objets ou des individus.

\section{Idéal}
%IDÉAL
C’est quelque chose qui n’existe qu’en idée, donc qui n’existe pas.

Ainsi l’homme idéal, ou la femme idéale, ou la société idéale.
Comme nos idées correspondent plus facilement à nos désirs que ne le fait la
réalité, qui n’en a cure, le mot indique aussi une quasi-perfection. L’idéal n’a
qu’un seul défaut ; c’est qu’il n'existe pas.

«Il faut croire au bien, disait Alain, car il n’est pas ; par exemple à la justice, car
elle n’est pas » ({\it 81 chapitres}, IV, 7). C’est dire que le bien et la justice ne sont que
des idéaux. On n’en condlura pas qu’il n’y a pas lieu de s’en occuper, mais au
contraire qu'ils n'existent que dans la mesure où nous nous en occupons. Rien n'est
réel, dans l'idéal, que la valeur que nous lui prêtons — que le désir, qui nous fait agir.

\section{Idéalisme}
%IDÉALISME
Le mot se prend principalement en trois sens, l’un trivial, les
deux autres philosophiques.

Au sens trivial, c’est le fait d’avoir des idéaux, autrement dit de ne pas se résigner
à la médiocrité ambiante, aux plaisirs matériels, à la réalité telle qu’elle est.
S’oppose alors au matérialisme ou au cynisme, pris eux aussi en un sens trivial.

Dans le langage philosophique, le mot peut désigner une certaine conception
de l’être (une ontologie) ou une certaine théorie de la connaissance (une
gnoséologie).

%— 287 —
D'un point de vue ontologique, il désigne l’un des deux grands camps dont
l'opposition, au moins depuis Démocrite et Platon, traverse et structure la
philosophie : est {\it idéaliste} toute doctrine pour laquelle la pensée existe indépendamment
de la matière, voire existe seule, que ce soit sous la forme d’idées
(idéalisme au sens strict) ou sous la forme d’êtres spirituels (auquel cas on parlera
plutôt de spiritualisme). C’est le contraire du matérialisme au sens philosophique.

D'un point de vue gnoséologique, l’idéalisme désigne plutôt une limite de
la connaissance : est {\it idéaliste} tout penseur pour lequel nous ne pouvons rien
connaître de la réalité en soi, soit parce qu’elle n’existe pas, soit parce que nous
ne pouvons connaître que nos représentations. C’est le contraire du réalisme,
au sens gnoséologique du terme. C’est cette dernière acception qui explique
que Kant ait pu caractériser son propre système à la fois comme {\it idéalisme transcendantal}
(nous ne connaissons que des phénomènes, jamais les choses en soi)
et comme {\it réalisme empirique} (nous connaissons effectivement les phénomènes,
qui ne sont pas de pures illusions).

On remarquera qu’on peut être idéaliste au sens ontologique sans l’être au
sens gnoséologique (c’est le cas par exemple de Descartes) ; mais qu’il est difficile
de l’être au sens gnoséologique sans l'être aussi au sens ontologique (si nous
ne connaissons que nos représentations ou notre esprit, pourquoi penser qu'il
existe autre chose, qui serait d’un autre ordre ?). Enfin, qu’on peut être idéaliste
au sens trivial et matérialiste au sens philosophique. Ainsi, Marx : le communisme
était son idéal ; le matérialisme, sa philosophie.

\section{Idée}
%IDÉE
C’est une représentation : les idées ne sont visibles ({\it idein}, en grec,
signifie voir) que pour l’esprit, et tout ce que l’esprit se représente
peut être appelé {\it idée}. La forme de cet arbre, devant moi, est son {\it eîdos} (son
aspect, sa forme visible). Mais en tant que je la perçois intérieurement, c’est
une idée. « J’appelle généralement du nom d’idée, écrit Descartes, tout ce qui
est dans notre esprit, lorsque nous concevons une chose, de quelque manière
que nous la concevions » ({\it Lettre à Mersenne}, juillet 1641). En pratique, toutefois,
le mot ne sert guère que pour les représentations les plus abstraites ou
les plus élaborées, à l'exclusion des simples images ou perceptions : on parlera
de l’idée d’arbre, plutôt que de l’idée de cet arbre-ci, et cette dernière ne sera
appelée {\it idée} qu’à la condition de comporter quelque chose de plus que la
simple sensation. L’idée, en ce sens, ce n’est pas seulement ce qui est « dans
la pensée », comme disait aussi Descartes, mais ce qui en résulte, ce que la
pensée produit ou élabore, qui est moins son objet que son effet. Penser, c’est
avoir des idées, mais on ne peut les avoir qu’à la condition de les produire ou
%— 288 —
de les reproduire — qu’à la condition de les penser -, ce qui ne va pas sans
effort ou travail. « Par idée, écrivait Spinoza, j'entends un concept de l'esprit,
que l'esprit forme parce qu’il est une chose pensante » ({\it Éth.} II, déf. 3). L'idée
n’est pas une copie des choses, mais le résultat d’un acte de penser : non
« quelque chose de muet, comme une peinture sur un panneau, souligne
encore Spinoza, mais un mode de penser, savoir l’acte même de connaître »
({\it ibid.}, scolie de la prop. 43). C’est dire, contre Platon, qu’il n’y a pas d’idées
séparées ou en soi : il n’y a que le travail de la pensée. Comment existerait-il
des idées innées ou absolues ? Ce serait pensée sans travail — pensée sans
pensée. Une idée qui n’est pensée par personne n’est pas une idée et n’est
rien.

Ce travail a ses exigences propres, qui sont de vérité plutôt que de ressemblance.
Certes, « une idée vraie doit s’accorder avec l’objet dont elle est l’idée »
({\it Éth.} I, axiome 6). Mais cet accord ne saurait prendre la forme d’une reproduction.
La pensée n’est pas un art figuratif : une idée n’est ni une peinture ni une
image ({\it Éth.} II., scolie de la prop. 48). Il s’agit de penser vrai, pas de faire ressemblant.
L'idée de cercle n’est pas ronde, l’idée de chien n’aboie pas ({\it T.R.E},
27), et aucune idée n’a d’idée.

\section{Identité}
%IDENTITÉ
Le fait d’être le même. Mais le même que quoi ? Le même que
le même : il n’y aurait pas autrement identité. Ainsi l’identité
est d’abord une relation de soi à soi (mon identité, c’est mon être moi-même),
ou, lorsqu'il ne s’agit pas de sujets, une relation entre deux objets qui n’en font
qu'un. « Pris au sens strict, ce terme est on ne peut plus précis, remarque
Quine : une chose est identique à elle-même et à rien d’autre, pas même à un
double gémellaire » ({\it Quiddités}, art. « Identité »). Deux jumeaux monozygotes,
même à les supposer parfaitement semblables, ne sont jumeaux qu’en tant qu’il
sont deux individus différents : s’ils étaient absolument le même (au sens où
l’auteur de {\it La Chartreuse de Parme} et celui de {\it Lucien Leuwen} sont le même), ils
ne seraient qu’un et il n’y aurait pas de jumeaux. Ainsi l'identité, prise en ce
sens strict, suppose l’unicité : c’est {\it être un et le même}, et nul n’est le même que
de soi.

En un sens plus large, mais bien avéré dans la tradition, il arrive pourtant
qu’on parle d'identité à propos de deux objets différents, pour marquer qu’ils
sont semblables : par exemple lorsqu'on constate, entre amis, une identité de
points de vue ou de goûts.

Ces deux sens peuvent être légitimes l’un et l’autre ; encore convient-il
de ne pas les confon\-dre. C’est pourquoi on parle souvent, pour désigner le
premier, d'{\it identité numérique} (être un et le même : « Nous habitons dans le
%— 289 —
même immeuble »), alors qu’on parlera d'{\it identité spécifique} où qualitative pour
désigner la parfaite similitude entre plusieurs objets différents (« Nous avons la
même voiture », c’est-à-dire ici deux véhicules de la même marque, du même
modèle et de la même couleur).

Cette dernière identité n’est jamais absolue (deux voitures identiques ne
sont jamais absolument indiscernables). Mais l'identité numérique l’est-elle ?
Au présent, sans doute ; mais au présent seulement. À la considérer dans le
temps, elle est aussi relative — et peut-être plus illusoire — que l’autre. Le
Stendhal qui commence {\it Lucien Leuwen}, en 1834, a quatre ans de moins que
celui qui écrira {\it La Chartreuse de Parme}. Comment lui serait-il identique ? Et
pourquoi, s’il l'était, n’écrivit-il pas le même livre ?

L'erreur serait de croire que cette notion, qui reste toute formelle, puisse
nous apprendre quoi que ce soit sur le réel. Que Stendhal, Henri Beyle et
l’auteur de {\it La vie de Henri Brulard} ne fassent qu’un, cela ne nous apprend
quelque chose que pour autant que nous savons ce que ces mots désignent, ou
plutôt c’est parce que nous le savons que nous pouvons affirmer que ces trois
personnages ne font qu’un. L'identité, pas plus que la carte du même nom, ne
se prononce sur le contenu de ce qu’elle désigne (ce n’est pas la quiddité), mais
seulement sur l'égalité de ce contenu à lui-même. A = A. L'identité n’est pas
l'essence, mais l'essence suppose l'identité.

Il se pourrait, et c’est ce que je crois, que rien, dans le temps, ne reste identique
à soi: que tout soit impermanent, comme disent les bouddhistes, et
qu'on ne se baigne jamais deux fois dans le même fleuve. Le réel ne cesserait pas
pour autant, au présent, d’être identique à soi. C’est où Parménide triomphe
d’'Héraclite, mais en vain : puisqu'il triomphe même si Héraclite a raison. Le
même est à penser, qui est ; mais ce qu'il est, la pensée ne peut l’apprendre que
de l'être, point du même. Il n’y a pas d’ontologie {\it a priori}. L'identité est un
concept nécessaire, mais vide. Elle est le nom qu’on donne à la pure présence à
soi du réel, qui n’est pas un nom.

C’est une dimension du silence, par quoi le discours est possible.

\section{Identité (Principe d'—)}
%IDENTITÉ (PRINCIPE D’)
C’est le principe qui fonde l'adéquation de la

vérité à elle-même. Tout être est ce qu’il est :
a = a, p = p ; où mieux, et comme disaient déjà les stoïciens : {\it Si a, alors a} ; {\it Si p,
alors p}. Si je vis, je vis ; si je fais ce que je fais, je le fais. D’où il ne suit rien, que
la nécessité absolue du présent, qui est tout.
Le principe d’identité est ce qui rend la pensée possible, et la vérité nécessaire.
%— 290 
\section{Idéologie}
%IDÉOLOGIE
Pour ceux des disciples de Condillac qu’on appelait, au tout
début du {\footnotesize XIX$^\text{e}$} siècle, les {\it idéologues} — et spécialement chez Destutt
de Tracy, qui inventa le mot —, c'était la science des idées, qui serait par là
même la science des sciences. Mais une telle science n’existe pas : on ne peut
connaître que le cerveau, qui pense les idées, ou une théorie particulière, qui
s’en sert ou les sert. C’est pourquoi sans doute ce sens est aujourd’hui tombé en
désuétude. Le mot, depuis des décennies, ne se prend plus guère que dans son
acception marxiste : l'idéologie est un ensemble d’idées ou de représentations
(valeurs, principes, croyances...) qui ne s'expliquent pas par un processus de
connaissance — l’idéologie n’est pas une science — mais par les conditions historiques
de leur production, dans une société donnée, et spécialement par le jeu
conflictuel des intérêts, des alliances et des rapports de forces. C’est comme une
pensée sociale, qui ne serait pensée par personne mais qui penserait en tous, ou
plutôt à l’intérieur de laquelle tous, nécessairement, penseraient. L’idéologie est
inconsciente : elle est le lieu, socialement et historiquement déterminé, de toute
conscience possible. C’est « le langage de la vie réelle » (Marx et Engels, {\it L'idéologie
allemande}, I). Elle est par nature hétéronome : son histoire est soumise à
celle de la société matérielle, elle-même dominée « en dernière instance » par
l'infrastructure économique (forces productives et rapports de production). On
n’a pas les mêmes idées à l’âge de la pierre taillée et à celui de la pierre polie,
dans une société féodale et dans une société capitaliste, à l’époque de la révolution
industrielle et à celle de la révolution informatique.

« L’idéologie n’a pas d’histoire », écrivaient les mêmes auteurs ({\it ibid.}). Il
faut entendre : pas d’histoire autonome, pas d’autre histoire que celle de la
société dont elle fait partie, qui la détermine et sur laquelle elle agit en retour.
Car l'idéologie n’est pas un simple reflet, encore moins un épiphénomène.
C’est une force agissante : la fonction pratico-sociale, soulignait Althusser,
l'emporte en elle sur la fonction théorique. Elle fait de nous des sujets, en nous
assujettissant à elle. Elle constitue « le rapport imaginaire des individus à leurs
conditions d’existence » ({\it Positions}, p. 101 ; voir aussi {\it Pour Marx}, p. 240). Elle
vise moins un effet de connaissance qu’un effet de pouvoir ou de sens.

L’idéologie dominante, disait Marx, est l'idéologie de la classe dominante :
celle-ci fait passer pour des exigences universelles — bien sûr en y croyant elle-même
— des opinions qui ne font qu’exprimer ses intérêts particuliers, tels
qu’ils résultent de sa position dans les rapports sociaux. Seule la vérité lui
échappe, qui n’a que faire de nos intérêts. C’est dire que tout ce qui n’est pas
vrai, dans une pensée donnée, est idéologique. De là l’usage souvent péjoratif
du mot, qui assimile l’idéologie à une conscience fausse. Cet usage est lui-même
idéologique. Qu’une pensée ne soit pas vraie n’implique pas, en effet,
qu’elle soit fausse. Soit par exemple la proposition : {\it « Tous les hommes sont
%— 291 —
égaux en droit et en dignité. »} Que cette affirmation ne relève pas d’une connaissance,
c’est bien clair. Mais cela même, qui lui interdit en effet d’être vraie, lui
interdit aussi d’être fausse. Cela ne signifie pas qu’elle soit sans portée ou sans
valeur. Une thèse idéologique n’est ni vraie ni fausse, explique Althusser, mais
elle peut être juste ou injuste, dans un combat donné. Et je n’en connais guère
de plus juste, dans le combat qui est aujourd’hui le nôtre, que celle-ci. Ainsi
rien n’est faux, dans l'idéologie, que sa prétention à la vérité : ce n’est pas une
conscience fausse, c’est une conscience illusoire, et nécessairement illusoire —
non un ensemble d’erreurs, mais un ensemble d'illusions nécessaires. Que la
morale en fasse partie, par exemple, n'implique pas qu’on doive ou qu’on
puisse se passer de morale : c’est ce qui le rend au contraire impossible. Le
scientisme, qui voudrait se passer de toute idéologie, n’est qu’une idéologie
parmi d’autres. «Seule une conception idéologique de la société, écrit
Althusser, a pu imaginer des sociétés sans idéologies ({\it Pour Marx}, Maspero,
1965, p. 238). Et seule une conception idéologique du marxisme, ajouterai-je,
a pu imaginer qu’il échappe à l'idéologie.

On demandera si la philosophie fait partie de l'idéologie. Il faut répondre :
oui, sauf pour ce qui, dans une philosophie donnée, relève de la vérité (vérité et
science, il faut le rappeler, ne sont pas synonymes), et quand bien même il
serait impossible, entre ces deux parts, de fixer quelque limite assurée que ce
soit. C’est ce qui explique que les pensées d’Aristote ou de Montaigne, qui
vivaient dans des sociétés si différentes de la nôtre, nous semblent encore si
vivantes, si éclairantes, si actuelles. Cela ne signifie pas qu’on puisse, au
{\footnotesize XXI$^\text{e}$} siècle, être aristotélicien ou réécrire les {\it Essais}, mais qu’on peut lire Montaigne
ou Âristote pour un intérêt autre qu’historique : parce qu'ils nous
aident, aujourd’hui, à penser. Ils avaient le même corps que nous, le même cerveau
que nous. Comment n’auraient-ils pas, pour une part, le même esprit ?
L'économie n’est pas tout. L’infrastructure biologique compte aussi, et sans
doute davantage. Au reste, si nous étions incapables de vérité (si tout était idéologie),
le marxisme n’aurait aucun sens. Il faut donc que quelque chose échappe
à l'idéologie pour que la notion d’idéologie puisse prétendre à la vérité. Parce
qu’elle est scientifique ? Je n’en suis pas sûr. Mais parce qu’elle est rationnelle.
On pourrait dire la même chose, chez Montaigne ou Aristote, de toute argumentation
rigoureuse, et on leur en doit d'innombrables. Ainsi toute philosophie
est dans l'idéologie ; mais tout, dans une philosophie, n’est pas nécessairement
idéologique.

\section{Idéologue}
%IDÉOLOGUE
C'est d’abord un praticien de l’idéologie, au sens premier du
terme, c’est-à-dire de ce qui passait, au début du {\footnotesize XIX$^\text{e}$} siècle,
%— 292 —
pour la science des idées : Cabanis et Destutt de Tracy sont les plus fameux ;
Stendhal et le jeune Maine de Biran s’en réclameront. Ce sens n’a plus d’usage
qu’historique. Aujourd’hui, on appelle plutôt {\it idéologue} toute personne qui développe
ou représente une idéologie, au sens actuel et plus ou moins marxiste du
terme, Le mot, dans cette dernière acception, est presque toujours péjoratif : un
idéologue, c’est quelqu'un qui travaille dans l'illusion, mais sans le savoir, et qui
prétend pour cela ériger en vérité universelle son propre point de vue, lequel ne
fait qu’exprimer des intérêts ou des partis pris banalement particuliers.

\section{Idiosyncrasie}
%IDIOSYNCRASIE
C’est le {\it mélange (sunkrasis) propre (idios)} à un individu
donné, autrement dit ce qu’il a de singulier, qui résulte
de la rencontre en lui d'éléments qui ne le sont pas. Ce n’est qu’un mot savant,
pour dire la banalité hétérogène d’être soi.

\section{Idiotie}
%IDIOTIE
Manque extrême d'intelligence. Dans la psychopathologie traditionnelle,
l’idiot est l'équivalent de ce qu’on appellerait aujourd’hui
un débile profond (par différence avec l’imbécile, qui correspond plutôt
au débile léger). L’idiot est incapable de parler; limbécile, de parler
intelligemment. Mais le mot a fait son entrée dans la langue proprement philosophique,
il y a une vingtaine d’années, en un sens tout à fait différent, que l’on
doit à Clément Rosset et qui renvoie à l’étymologie. {\it Idiôtès}, en grec, c’est le
simple particulier (le mot est dérivé d’{\it idios}, propre), par opposition aux magistrats
ou aux savants, qui sont supposés parler du point de vue de l’universel.
L’idiotie, en ce sens, est le propre de tout être singulier, en tant qu’il n’est que
soi : c’est la singularité brute, sans phrases, sans double, sans alternative. C’est
comme un idiotisme ontologique : la pure singularité d’exister. C’est donc le
propre de tout être (la singularité est une caractéristique universelle), et c’est ce
qu’indique bien clairement l’un des plus beaux titres de Clément Rosset et de
l’histoire de la philosophie : {\it Le réel, Traité de l'idiotie} (Éditions de Minuit,
1977 ; sur le sens du mot, voir spécialement les p. 7 et 40-51).

\section{Idolâtrie}
%IDOLÂTRIE
C’est adorer une idole, c’est-à-dire une image de la divinité
plutôt que Dieu même, ou un faux dieu plutôt que le vrai.
L'idolâtrie, en ce sens, est la religion {\it des autres}. On ne pourrait y échapper
qu’en adorant un Dieu qu’on ne puisse aucunement imaginer. Mais comment
savoir, alors, s’il est Dieu ?

%— 293 —
On peut parler d’idolâtrie, en un sens plus général, pour toute adoration
d’un objet visible ou sensible, et même d’une entité quelconque, dès lors qu’elle
est supposée exister ici-bas. Adorer la Nature, la Force, l’État, la Société,
l’Argent, la Science, l'Histoire ou l'Homme, c’est idolâtrie toujours. Simone
Weil, commentant le début du Notre Père, en apparence si dérisoire (« Notre
Père, qui êtes aux cieux. »), en a fait la remarque : « Le Père est dans les cieux.
Non ailleurs. Si nous croyons avoir un Père ici-bas, ce n’est pas lui, c’est un
faux Dieu » ({\it Attente de Dieu}, p. 215). Dieu n’est Dieu que par son absence, et
tel est le secret peut-être de la transcendance : tant que nous adorons quelque
chose de présent, nous adorons un faux Dieu ; même monothéistes ou athées,
nous sommes idolâtres. On n’y échappe qu’en adorant l'absence même, ou en
cessant d’adorer.

\section{Idole}
%IDOLE
Une image ({\it eidolon}) divine, ou un Dieu imaginaire. Par métaphore,
toute personne que l’on adore comme un dieu. Reste à
savoir si Dieu lui-même n’est pas une première métaphore.

\section{Illusion}
%ILLUSION
Ce n’est pas la même chose qu’une erreur. C’est une représentation
prisonnière de son point de vue, et qui résiste même à la
connaissance de sa propre fausseté : j’ai beau savoir que la Terre tourne autour
du Soleil, je n’en vois pas moins le Soleil se mouvoir d’est en ouest. « Est illusion,
écrit Kant, le leurre qui subsiste, même quand on sait que l’objet supposé
n'existe pas » ou est autre ({\it Anthropologie...}, \S 13). Il y a donc une positivité de
l'illusion. Si l'erreur n’est qu’une privation de connaissance (ce en quoi elle
n'est rien et s’abolit dans le vrai), l'illusion serait plutôt un excès de croyance,
d'imagination ou de subjectivité : c’est une pensée qui s'explique moins par le
réel que je connais que par le réel que je suis.

Cette subjectivité peut être purement sensorielle (les illusions des sens) ou
transcendantale (s’il existe, comme le veut Kant, des illusions de la raison).
Mais elle s'exprime plus souvent comme subjectivité désirante : se faire des illusions,
c’est prendre ses désirs pour la réalité. Tel est le sens du mot chez Freud :
« Ce qui caractérise l'illusion, écrit-il, c’est d’être dérivée des désirs humains »
({\it L'avenir d'une illusion}, VI). Toute erreur n’est donc pas une illusion, ni toute
illusion une erreur. Je peux me tromper sans que ce soit du fait de mes désirs
(c'est alors une erreur, non une illusion), et ne pas me tromper bien que ma
pensée doive plus à mes désirs qu’à une connaissance (c’est alors une illusion,
non une erreur : par exemple la jeune fille pauvre qui croit qu’un prince va
venir l’épouser ; quelques cas de ce genre, observe Freud, se sont réellement
%— 294 —
présentés). L’illusion, bien qu’elle puisse être fausse, et bien qu’elle le soit le
plus souvent, n’est donc pas un certain type d’erreur. C’est un certain type de
croyance : « Nous appelons illusion une croyance, continue Freud, quand, dans
la motivation de celle-ci, la réalisation d’un désir est prévalante », et quoi qu’il
en soit par ailleurs de son rapport à la réalité. C’est une croyance désirante, ou
un désir crédule.

Si l’on admet, avec Spinoza, que tout jugement de valeur suppose un désir
et s’y ramène ({\it Éthique}, III, 9, scolie), il en résulte que toutes nos valeurs sont
des illusions. On n’en conclura pas qu’il faudrait s’en passer, mais au contraire
qu'on ne le peut (puisque nous sommes des êtres de désirs) et qu’on ne le doit
(l'humanité n’y survivrait pas). Illusions nécessaires : on ne pourrait y échapper
que pour tomber aussitôt dans d’autres. « Seule une conception idéologique de
la société a pu imaginer des sociétés sans idéologies », écrivait Althusser. Seule
une conception illusoire de l'humanité a pu imaginer une humanité sans illusions.

\section{Image}
%IMAGE
Reproduction ou figuration sensible — qu’elle soit matérielle ou
mentale — d’un objet quelconque. L'important n’est pas que cet
objet existe ou pas réellement (on peut imaginer ou peindre une chimère aussi
bien que son voisin de palier), mais qu’il soit figurable. De à les métaphores,
symboles, allégories et autres images (mais en un sens dérivé), qui visent à
représenter ce qui n’est pas immédiatement présentable : par exemple une
balance pour la justice, une colombe pour la paix, un vieillard ou un jeune
homme pour Dieu.

\section{Imagination}
%IMAGINATION
La faculté d’imaginer, autrement dit de se représenter
intérieurement des images, y compris et surtout quand ce
qu’elles représentent est absent. Ces images sont des actes, remarquait Sartre,
non des choses : l’imagination est « une certaine façon qu’a la conscience de se
donner un objet », mais de se le donner, paradoxalement, comme absent. C’est
ce qui la rend utile et dangereuse : elle libère du réel, dont elle fait pourtant
partie, mais aussi nous en sépare. Se distingue par là de la connaissance, qui
libère sans séparer, et de la folie, qui sépare sans libérer.
On oppose couramment les classiques, qui se méfiaient de limagination
(« la folle du logis »), aux romantiques et aux modernes, qui en font la faculté
créatrice par excellence. C’est bien sûr moins simple que cela. L'imagination,
écrivait par exemple Pascal, « est cette partie dominante dans l’homme, cette
maîtresse d’erreur et de fausseté, et d’autant plus fourbe qu’elle ne l’est pas
%— 295 —
toujours ; car elle serait une règle infaillible de vérité, si elle l'était infaillible du
mensonge, » C'est ce qui permet à certains romans d’être vrais, et à tant
d’autres d’être faux. « Je ne parle pas des fous, ajoute Pascal, je parle des plus
sages, et c’est parmi eux que l'imagination a le grand droit de persuader les
hommes. La raison a beau crier, elle ne peut mettre le prix aux choses. [...]
L’imagination dispose de tout : elle fait la beauté, la justice et le bonheur, qui
est le tout du monde » ({\it Pensées}, 44-82). Maîtresse d’erreur, créatrice de valeur.
Seule la vérité lui échappe, qui ne vaut, toutefois, que pour autant qu’on l’imagine.

\section{Immanence}
%IMMANENCE
C'est la présence de tout dans tout (immanence absolue), ou
dans autre chose (immanence relative). Le contraire donc de
la transcendance. Est transcendant ce qui s'élève ({\it scandere}) au-delà ({\it trans}) ;
immanent, ce qui reste ({\it manere}) dans ({\it in}). Se dit spécialement de ce qui est
dans la nature et en dépend. Si tout est matériel, comme je le crois, s’il n'existe
rien d’autre que l'univers ou la nature (rien d’autre que tout !), il faut en
conclure que tout est immanent : la transcendance n’est qu’une extériorité imaginaire,
comme telle immanente (l'imagination fait partie de l’univers).

\section{Immanent}
%IMMANENT
Est immanent, au sens classique, ce qui est intérieur, ce qui
demeure dans ({\it in-manere}) quelque chose ou quelqu'un. On
parlera par exemple de « justice immanente » pour désigner une récompense ou
une punition incluses dans l’acte même qu’elles sanctionnent (la satisfaction du
devoir accompli, chez l’homme vertueux, la solitude du méchant, l’indigestion
du goinfre...), par opposition à une justice transcendante, qui suppose une intervention
extérieure, qu’elle soit divine ou humaine. On remarquera que l’une et
l’autre sont douteuses. Mais cela en dit plus sur la justice que sur l’immanence.
Chez Kant, tout ce qui fait partie de l’expérience et ne s'applique qu’à elle.
Chez Husserl et les phénoménologues, tout ce qui est intérieur à la conscience.
Au sens absolu, est immanent tout ce qui est intérieur au tout, ou du moins
(si l’on veut que la notion de transcendance garde un contenu) tout ce qui fait
partie de l’univers matériel, c’est-à-dire de l’univers. Le matérialisme, en ce
sens, est un immanentisme absolu : seul Dieu serait transcendant, qui n’est pas.

\section{Immanental}
%IMMANENTAL
Il m'est arrivé d’appeler {\it immanental} tout ce qui, à l’intérieur
de l'expérience, la rend possible : les conditions
%— 296 —
empiriques de l’empiricité, autrement dit son pouvoir ou son processus d’autoconstitution
historique. C’est l’équivalent, pour le matérialisme, du {\it transcendantal}
pour l’idéalisme. Est immanental tout ce qui constitue une condition de
possibilité, mais {\it a posteriori}, de la connaissance : le corps (spécialement le cerveau),
le langage et l'expérience sont des immanentaux.

On dira qu’à ce compte il y a cercle (puisque les conditions de l’expérience
résultent de l'expérience). Mais ce cercle, qui est plutôt une spirale, est celui-là
même de la pensée. « Les choses qu’il faut avoir apprises pour les faire, disait
Aristote, c’est en les faisant que nous les apprenons » ({\it Éthique à Nicomaque}, II,
1 : c'est en forgeant que l’on devient forgeron). Ainsi faut-il apprendre à
penser, et nul ne le peut qu’en pensant. L’immanental indique que l’origine de
ce processus, qui rend la pensée possible, n’est pas elle-même une pensée, mais
une expérience. À la gloire de l’empirisme.

\section{Immanentisme}
%IMMANENTISME
Doctrine pour laquelle tout est immanent, au sens
absolu du terme, ce qui suppose qu’il n'existe aucune
transcendance. Synonyme parfois de matérialisme, mais avec une extension
plus large : le spinozisme — qui n’est ni matérialiste ni idéaliste — est un immanentisme,
et le modèle, depuis trois cents ans, de tous.

\section{Immatérialisme}
%IMMATÉRIALISME
Une forme extrême et rare d’idéalisme, qui va jusqu’à
nier l'existence de la matière. La philosophie de Berkeley
est sans doute l’exemple le plus radical qu’on en puisse proposer. « Être,
c’est percevoir ou être perçu »; il n'existe que des esprits et des idées. La
«matière» n’est qu’un mot, qui ne correspond à aucune expérience réelle
(puisque nous ne pouvons expérimenter, par définition, que nos perceptions,
qui sont en nous). Qu’une pensée aussi éloignée du sens commun soit irréfutable
en dit long sur la pensée, et sur le sens commun.

\section{Immoral}
%IMMORAL
Qui s’oppose à la morale, tant que celle-ci est supposée légitime.
ne pas confondre avec l’immoralisme, qui conteste cette légitimité,
ni avec l’amoralité, qui n’en relève pas.

\section{Immoralisme}
%IMMORALISME
C’est s’opposer à la morale, le plus souvent parce qu’elle
ne serait qu’une illusion néfaste. Ainsi, chez Nietzsche :
« La morale est le danger par excellence, écrivait-il, l'instinct négateur de la vie :
%— 297 —
il faut détruire la morale pour libérer la vie. » Aussi faut-il s’efforcer de vivre
« par-delà le bien et le mal », se faire « plus fort, plus méchant, plus profond ».
C’est combattre la morale, mais au nom d’une certaine éthique : « Par-delà le
Bien et le Mal, disait encore Nietzsche, cela du moins ne veut pas dire : par-delà
le bon et le mauvais. » On ne sort pas des jugements de valeur : l’immoralisme
est le contraire d’un amoralisme. C’est ce qui autorise la plupart des
immoralistes à être de très braves gens: ce qu’ils reprochent à la morale,
presque toujours, c’est d’être immorale.

\section{Immortalité}
%IMMORTALITÉ
Est immortel ce qui ne peut mourir : ainsi l’âme selon
Platon, ou Dieu selon les croyants. On remarquera que
l’immortalité de l’âme n’est pas une idée chrétienne (si Jésus nous sauve de la
mort, et si nous pouvons ressusciter, c’est que nous pouvons mourir), ni une
idée juive. Une idée grecque ? En partie. Épicure, si elle n’avait été si répandue,
n'aurait pas consacré autant d'énergie à la combattre. Il y voyait moins une
espérance qu’une source intarissable de craintes. Être immortel, ce serait être
exposé à jamais au malheur, aux châtiments, à la répétition — à l’enfer. La mortalité
vaut mieux, qui ne nous expose qu’au néant.

\section{Impératif}
%IMPÉRATIF
Un commandement, mais qui s’énoncerait à la première personne :
non le contraire de la liberté, mais ce qu’elle s'impose
à elle-même. Ce n’est pas la même chose d’obéir à un souverain ou à un Dieu
(commandement), ou de n’obéir qu’à soi (impératif). Obéir à un commandement,
c’est se soumettre, et sans doute il le faut souvent. Obéir à un impératif,
c’est se gouverner, et il le faut toujours.

On distingue depuis Kant deux types d’impératifs : l'impératif hypothétique
et l’impératif catégorique.

Le premier reste soumis à une condition, qui est ordinairement la fin poursuivie.
Par exemple : « Si tu veux que tes amis soient loyaux avec toi, sois loyal
avec eux. » Ou encore : « Si tu veux éviter la prison, sois honnête. » Ce ne sont
que des règles de la prudence ou de l’habileté. Il s’agit de choisir des moyens
adaptés à la fin qu’on poursuit, et ils ne valent que pour autant qu’on la poursuit
effectivement.

L’impératif catégorique, au contraire, est inconditionnel. C’est qu’il n’a
que faire de quelque fin que ce soit. Par exemple : « Sois loyal avec tes amis. »
Ou bien : « Ne mens pas. » Impératifs moraux, qui commandent absolument :
ils n’ont pas à voir avec la réussite ou l'efficacité, comme la prudence ou l’habileté,
mais avec le devoir. Ainsi, explique Kant, lorsqu'il s’agit de témoigner
%— 298 —
devant un tribunal : celui qui se demande dans quel but il devrait dire la vérité
est déjà un misérable.

Les impératifs hypothétiques restent particuliers : ils ne valent que pour
ceux qui en vérifient la condition, autrement dit qui visent tel ou tel but (des
amis loyaux, la confiance, la réussite...). L’impératif catégorique, parce qu’il est
inconditionnel et ne vise aucun but, est universel : il vaut pour tout être raisonnable
fini, comme dit Kant, y compris pour ceux qui ne le respectent pas. Il est
l’'universel même, en tant que la raison l’énonce et se Le prescrit à elle-même —
non pour la pensée seule (raison théorique), mais pour laction (raison
pratique). C’est ce qui détermine sa formule bien connue et bien exigeante :
« Agis uniquement d’après la maxime qui fait que tu peux vouloir en même
temps qu'elle devienne une loi universelle » ({\it Fondements...}, II). C’est n’obéir
qu’à la raison en soi, autrement dit qu’à la partie de soi qui est libre (parce
qu’elle n’est pas soumise aux penchants ou aux instincts du « cher petit moi »).
C’est n’obéir qu’à soi (autonomie) en se libérant de soi (universalité). Ainsi la
morale ne vaut pour tous que parce qu’elle vaut pour chacun (« tout seul, disait
Alain, universellement »), et le seul devoir est d’être libre.

\section{Implication}
%IMPLICATION
C'est une relation entre deux propositions, telle que la
seconde soit une conséquence nécessaire de la première : {\it si
p, alors q}. Si la première est vraie, la seconde l’est aussi. Si la seconde est fausse,
la première également. En revanche, si la première est fausse, la seconde peut
être vraie ou fausse. À la considérer en bloc et d’un point de vue strictement
logique, une implication n’est donc fausse que si et seulement si elle relie un
antécédent vrai à un conséquent faux : « Si Paris est la capitale de la France,
alors les poules ont des dents » est une proposition fausse. Une implication
commençant par une proposition fausse, à l’inverse, est nécessairement valide.
«Si les poules ont des dents, alors je suis roi de France » est une proposition
vraie, qu’elle soit prononcée par Louis XIV ou par votre serviteur.

\section{Impression}
%IMPRESSION
C’est une espèce de perception, mais qui renvoie davantage
à l’état du sujet percevant qu’à celui de l’objet perçu. Toute
impression est subjective ; c’est sa façon à elle d’être vraie, ou de pouvoir l'être.
Ainsi dans l’impressionnisme (qui doit son nom, d’abord péjoratif, à un
tableau fameux de Monet, intitulé {\it Impression, soleil levant}) : il s’agit de peindre
non ce qu’on sait ou croit être, mais ce qu’on voit. De là une objectivité nouvelle,
plutôt phénoménologique qu’ontologique. C’est un réalisme à la première personne,
qui cherche moins la vérité des choses que leur apparence fugitive,
%— 299 —
moins l’éternité que l’instant, moins l’absolu que le mouvement ou la
lumière. Par quoi les plus grands, parfois, retrouveront — comme faisait déjà
Corot, comme fera Cézanne, dans ses meilleurs toiles — et la vérité, et l’éternité,
et l’absolu, qui sont le devenir même, dans son impermanence dévoilée ou
retrouvée.

En philosophie, toutefois, le mot renvoie moins à l'esthétique qu’à la
théorie de la connaissance, spécialement dans sa version empiriste et sceptique.
Les impressions, écrit Hume, sont « les perceptions qui pénètrent en nous avec
le plus de force et de violence » (par différence avec les idées, qui sont comme
les images effacées ou affaiblies des impressions dans nos pensées) ; « et sous ce
nom, ajoute-t-il, je comprends toutes nos sensations, passions et émotions,
telles qu’elles font leur première apparition dans l’âme » ({\it Traité...}, I, I, 1). Il en
résulte qu’on ne connaît que des impressions ou des idées, sans pouvoir jamais
les comparer à quelque modèle original qui serait l’objet même (puisqu'on ne
pourrait connaître celui-ci que par l'intermédiaire d’une impression). C’est où
l’empirisme mène au scepticisme.

\section{Impulsif}
%IMPULSIF
Celui qui ne peut résister à ses impulsions : elles sont trop fortes
pour lui ; il est trop faible pour elles.

\section{Impulsion}
%IMPULSION
Un mouvement irraisonné. À la raison de le comprendre (il
est irraisonné, point irrationnel : il a des causes) ; à la volonté
de le contrôler, s’il le faut, ou de l’utiliser, si elle le peut.

\section{Incertirude}
%INCERTITUDE
Est incertain tout ce dont on peut ou doit douter. C’est-à-dire
tout ? Oui, en un sens, puisqu'il se peut que je rêve,
que nous soyons tous fous, ou qu’un Dieu tout-puissant prenne un malin
plaisir à nous tromper toujours. Il n’en serait pas moins vrai que j’existe ?
Cette évidence suppose la validité de notre raison, qui est sans preuve (puisque
toute preuve la suppose), et n’en est donc pas une. Au reste, quand bien même
on admettrait qu’une erreur suppose quelque chose qui se trompe, cela ne saurait
prouver, tout au plus, que l’existence de... quelque chose. De [à à prétendre
que cette chose est moi. Qui sait si je ne suis pas le rêve d’un autre, ou
bien un fou qui se prend pour André Comte-Sponville, ou encore un cerveau
dans une cuve, qu’un expérimentateur de génie — ou un technicien médiocre
dans dix mille ans — programmerait en permanence, à l’aide d’électrodes et
d’ordinateurs, pour qu’il se croie philosophe et en train, par exemple, d’écrire
%— 300 
une définition de l’incertitude.. Cela est improbable ? Sans doute. Par quoi il
n’est pas certain, comme disait Pascal, que tout soit incertain. Mais cela ne fait
qu’une incertitude de plus.

Toutefois ce doute, pour légitime qu’il soit en toute rigueur, reste
métaphysique : nous serons presque tous portés, comme le Descartes de la
sixième Méditation, à le trouver quelque peu « hyperbolique et ridicule ». Aussi
ne parlera-t-on d’incertitude, en un sens plus restreint, que pour ce qui peut
être faux, quand bien même nos sens et notre raison seraient supposés à peu
près fiables. L’incertain, c’est alors ce dont on peut ou doit douter, non en
toute rigueur ou dans l’absolu, mais dans les conditions ordinaires de notre vie
et de notre pensée — ce qui est {\it particulièrement} douteux. Par exemple que
Napoléon ait été assassiné est incertain ; qu’il soit mort ne l’est pas. L'existence
d’une vie extra-terrestre est incertaine ; celle d’une vie sur Terre ne l’est pas.
Que nous ayons une Âme immatérielle est incertain ; que nous ayons un corps
matériel ne l’est pas. Cela ne prouve pas que les sceptiques aient tort, mais simplement
qu’on n’a pas besoin d’être dogmatique pour faire une différence entre
ce qui est incertain, en ce sens restreint, et ce qui ne l’est pas. Hume, quand il
jouait au trictrac, ne doutait pas de son jeu.

\section{Incertitude (relations d'—)}
%INCERTITUDE (RELATIONS D’—)
C’est une espèce de principe, qu’on
appelle parfois « principe d’indétermination »,
et qu’on doit à Heisenberg. Ce dernier a montré que les conditions
de l'observation modifiant, à l'échelle quantique, ce qu’on veut observer (en
éclairant une particule, on modifie sa trajectoire), il n’est pas possible de déterminer
à la fois la position et la vitesse d’une particule, voire qu’il n’est pas possible
de leur attribuer à la fois l’une et l’autre de ces deux caractéristiques. On
en conclut parfois que l'esprit humain est voué à l’échec, qu’il n’y a pas de
vérité, que l’idée même d’une connaissance scientifique s’écroule... C’est bien
sûr un contresens. La physique quantique est au contraire l’une des plus formidables
victoires de l'esprit humain, l’un des principaux progrès scientifiques de
tous les temps, enfin l’une des plus certaines (au sens restreint ci-dessus défini)
de nos théories. Tant pis pour les sophistes. Tant mieux pour les physiciens et
les rationalistes.

\section{Inclination}
%INCLINATION
C’est un penchant durable et plaisant, qui séduit davantage
qu’il ne contraint. On peut résister à ses inclinations (c’est
ce qui les distingue des compulsions) ; mais il est plus sage, quand elles ne sont
%— 301 —
pas déshonorantes, de s’y abandonner de temps en temps : cela évitera de les
transformer en obsessions ou en regrets.

\section{Inconditionné}
%INCONDITIONNÉ
Le mot parle de lui-même : est inconditionné ce qui ne
dépend d’aucune condition. C’est un autre nom pour
l'absolu théorique. Il est par nature inconnaissable. On ne pourrait en effet le
connaître, montre Kant (après Montaigne et Hume), qu’en le soumettant aux
{\it conditions} de nos sens et de notre esprit. Mais alors ce ne serait plus
linconditionné : ce ne serait que le réel, qui est l’ensemble indéfini de toutes les
conditions. On remarquera pourtant que cet ensemble de toutes les conditions
est lui-même nécessairement inconditionné : on ne peut pas davantage
renoncer à le penser que parvenir à le connaître.

\section{Inconditionnel}
%INCONDITIONNEL
Un de mes fils, il devait avoir sept ou huit ans, me
posa un jour la question suivante : « Qu'est-ce que je
pourrais faire, qui ferait que tu ne m'aimerais plus ? » Je n’ai pas trouvé la
réponse, ou plutôt je n’ai pu répondre que {\it « rien »}. Cela m’a surpris moi-même.
C’est la première fois que je comprenais ce qu’est un amour inconditionnel.
Je ne l'ai vécu, faut-il le préciser, qu'avec mes enfants. Mais cela m’en
a plus appris sur l’amour que tous les livres.

Est {\it inconditionnel} ce qui ne dépend d’aucune condition, mais dans l’ordre
pratique ou affectif plutôt que théorique : ce qui s'impose absolument au cœur
ou à la volonté, non parce que cela existerait de façon inconditionnée (hélas,
nos enfants dépendent de tellement de choses !), mais parce que nous ne saurions
vivre autrement. C’est ce que j'appelle l’absolu pratique : ce que nous
voulons ou aimons sans réserve et sans condition (de façon, dirait-on
aujourd’hui, « non négociable »), au point de lui sacrifier, si nécessaire, tout le
reste (du moins tout ce qui, dans le reste, n’est pas inconditionnel).

L’inconditionnel n’est donc pas nécessairement inconditionné, et même,
pour le matérialiste que je suis, il ne l’est jamais. Par exemple le refus du
racisme : qu'il y ait là une valeur inconditionnelle, cela n'empêche pas qu’elle
n’apparaisse que sous certaines conditions. C’est ce qui distingue la morale de
la religion : l’absolu pratique, pour l’athée, n'existe que {\it relativement à nous}.

\section{Inconscient}
%INCONSCIENT
Comme adjectif, c’est tout ce qui n’est pas conscient : par
exemple la circulation du sang ou les échanges électriques
entre les neurones sont des processus inconscients, comme la quasi-totalité de
%— 302 —
notre fonctionnement organique. « On ne sait pas ce que peut le corps », disait
Spinoza. C’est que l’essentiel de ce qu’il peut est inconscient.

Comme substantif, c’est tout ce qui {\it pourrait} être conscient, en droit, mais
qui ne {\it peut} l'être, en fait : le refoulement et la résistance s’y opposent. L’inconscient
est alors un inconscient {\it psychique}, comme dit Freud, et c’est ce paradoxe
qui le définit : c’est comme un esprit sans esprit, une pensée sans pensée, un
sujet sans sujet. Impossible ? Ce n’est pas sûr. Il se pourrait que l’inconscient
soit la vérité de l'esprit, dont la conscience ne serait que la pointe ultime, toujours
menacée, ou le sommet, toujours à conquérir. Si la pensée se pensait soi,
elle serait Dieu. L’inconscient est ce qui nous en sépare.

On évitera donc de l’adorer, et même d’y croire tout à fait. L’inconscient
non plus n’est pas Dieu. La psychanalyse, quand on veut en faire une religion,
n’est qu’une superstition comme une autre.

\section{Indéfini}
%INDÉFINI
Ce qui n’a pas de définition ou de fin déterminées. C’est le cas,
spécialement, des termes qui ne sont que la négation d’un autre.
Par exemple, précise Aristote, « non-homme est seulement un nom indéfini,
car il appartient pareillement à n'importe quoi, à ce qui est et à ce qui n'est
pas » ({\it De l'interprétation}, 2). Un chat, une racine carrée, un dahu ou Dieu font
partie de l’ensemble de tout ce qui n’est pas un homme ; mais cela ne nous dit
pas ce qu'est cet ensemble (sinon négativement) ni ne permet de lui fixer une
limite (cet ensemble, incluant par exemple la suite des nombres, dont aucun
n’est un homme, est bien sûr infini).

On évitera pourtant de confondre l’{\it indéfini} et l'{\it infini}. Si on laisse de côté
la question des termes négatifs, l’indéfini occupe une espèce d’entre-deux entre
le fini et l'infini. L’infini est ce qui n’a pas de limite. L’indéfini, ce dont la
limite est indéterminée ou indéterminable. Par exemple la suite des nombres
entiers est infinie ; l’histoire de l'humanité, indéfinie. L'ensemble des vérités
possibles est infini ; le progrès des connaissances, indéfini.

On notera que Descartes appelle parfois {\it indéfini} ce qui n’est infini que
d’un certain point de vue ou dans un certain ordre, et n’appelle {\it infini} que ce
qui n’a aucune limite, de quelque point de vue ou en quelque ordre que ce soit.
En ce sens, Dieu seul est infini, explique-t-il ; l’étendue des espaces imaginaires
ou la multitude des nombres ne sont qu’indéfinis. Ce sens est à connaître, mais
point, me semble-t-il, à utiliser.

\section{Indéterminisme}
%INDÉTERMINISME
Toute pensée qui nie la validité universelle du déterminisme.
Les partisans de l’indéterminisme soutiennent
%— 303 —
qu’il existe des phénomènes absolument indéterminés, autrement dit sans
causes nécessaires et suffisantes : par exemple un acte libre, chez Sartre, ou
l’effectuation spatio-temporelle du clinamen chez Lucrèce (le clinamen a bien
une cause, qui est l’atome, mais cette cause agit en un temps et en un lieu que
rien ne détermine). On y voit souvent une condition de la liberté. Mais cette
condition n’est pas suffisante. À supposer que les particules qui constituent
mon cerveau soient indéterminées, au niveau quantique, elles n’en seraient pas
moins déterminantes, au niveau neurobiologique, où plutôt elles le seraient
encore plus (si elles sont indéterminées, il est exclu que je puisse les gouverner,
non qu’elles me gouvernent). Et cette condition n’est pas non plus nécessaire.
Une autre liberté est possible, montre Spinoza, qui ne serait pas l’absence de
détermination, mais une détermination propre et par soi ({\it Éthique}, I, déf. 7:
« Cette chose est dite libre qui existe par la seule nécessité de sa nature et est
{\it déterminée par soi seule à agir} »). Il n’y a que Dieu, en ce sens, qui soit absolument
libre. Mais la raison l’est davantage en nous que la folie, nos actions
davantage que nos passions, nos vertus davantage que nos vices. Ce n’est plus
indéterminisme, mais indépendance ou autonomie. Qu'elle ne soit jamais
complète, c’est une raison pour l’accroître, non pour y renoncer.

Ainsi l’indéterminisme relève de la physique, non de la morale. Reste à
savoir s’il exprime une dimension du réel (des événements absolument indéterminés)
ou seulement une limite de nos connaissances (des événements indéterminables).
Il me semble que le deuxième terme de l’alternative, qui est avéré,
interdit d’exclure le premier tout autant que de l’ériger en certitude.

\section{Indice}
%INDICE
Un signe fondé sur un rapport de causalité : c’est un fait perceptible
qui renvoie à un autre, ordinairement imperceptible, qu’il
implique ou annonce, au point que nous utilisons celui-là comme le signe de
celui-ci. Un symptôme est l'indice d’une maladie, comme ces gros nuages noirs
peuvent être l’indice d’un orage prochain. Pourtant ni la fièvre ni les nuages ne
veulent rien dire : c’est nous qui les faisons parler en les interprétant. Un indice
n’est donc un signe que pour nous : il ne veut rien dire ; c’est nous qui le faisons
parler. Disons que c’est un fait susceptible d’une interprétation : un fait
significatif, mais sans volonté de signification.

\section{Indicible}
%INDICIBLE
Ce qui ne peut être dit, parce qu’il excéderait tout discours
possible. On pense à la dernière formule, si fameuse, du {\it Tractatus}
de Wittgenstein : « Ce dont on ne peut parler, il faut le taire. » Mais pourquoi
ce {\it « il faut »}, si l’on ne peut ? À quoi bon interdire ce dont nul n’est
%— 304 —
capable ? C’est qu’en vérité il n’y a pas d’indicible : tout peut être dit, bien ou
mal, mais il arrive que le silence, en effet, vaille mieux.

Dieu, par exemple, serait indicible. Les mystiques pourtant n’ont cessé d’en
parler, souvent fort bien, comme aussi, avec d’autres mots, les philosophes et
les théologiens. Leur discours reste inadéquat à son objet, qui l’excède de toute
part ? Sans doute. Mais c’est le cas aussi de l’univers, et de tout ce qui s’y
trouve. Essayez de dire adéquatement un caillou : l’essentiel nécessairement
échappe, qui est la différence entre le caillou et ce qui en est dit — qui est donc,
très exactement, le caillou lui-même. Est-ce à dire que le caillou soit indicible ?
Bien sûr que non, puisque vous pouvez en parler avec pertinence. Simplement
ce caillou, même dicible, même dit, est autre chose qu’un discours. C’est ce que
j'appelle non l’indicible, mais le silence. La différence entre les deux ? Le
silence, lui, peut être dit. C’est même le cas dans la plupart de nos discours. Le
métalangage, malgré nos bavards et nos sophistes, reste l’exception : le plus
souvent, et c’est heureux, on parle d’autre chose que du langage. Par exemple
ces deux vers d’Angelus Silesius : « La rose est sans pourquoi, fleurit parce
qu’elle fleurit, / N’a souci d’elle-même, ne désire être vue. » Cela parle, mais de
quelque chose qui ne parle pas. Non que la rose soit indicible, mais parce
qu’elle est silencieuse. Non qu’elle soit ineffable, mais parce qu’elle est {\it ineffante}
(si l’on m’autorise ce décalque de l’{\it infans} latin : celui qui ne parle pas). Royauté
du silence : royauté d’un enfant.

C’est donc Hegel, sur ce point, qui a raison : « Ce qu’on appelle l’indicible
n’est pas autre chose que le non-vrai, le non-rationnel, le seulement visé »
({\it Phénoménologie de l'esprit}, « La certitude sensible », III). Non que le vrai soit un
discours, mais parce que toute vérité peut être dite. Elle n’en restera pas moins
silencieuse pour autant : si le réel n’est pas un discours, comment un discours,
même vrai, pourrait-il le contenir tout entier ou le dissoudre ? Toute vérité
peut être dite, mais aucun discours n’est la vérité. Ce dont on peut parler n’en
continue pas moins à se taire. Royauté d’un enfant : royauté du silence.

\section{Indifférence}
%INDIFFÉRENCE
Ce n’est pas l’absence de différences (l'identité), mais le
refus ou l'incapacité d’en faire qui soient affectivement
significatives : l'absence non de différences, mais de préférences, de hiérarchie
ou même de normativité. Pour l’indifférent, tout n’est pas le même (tout n’est
pas identique), mais tout {\it revient au même}, comme on dit, tout est égal, ce qui
signifie que les différences, même effectives, ne sont jamais des différences de
valeur. C’est la version pyrrhonienne de l’ataraxie ({\it adiaphoria} : l'indifférence),
telle qu’elle résulte du fameux {\it ou mallon} : une chose n’est pas plus ({\it ou mallon})
qu’elle n’est pas, ni plutôt ceci que cela, ni ne vaut davantage ou moins qu’une
%— 305 —
autre. Non, répétons-le, que toutes les apparences soient identiques (Pyrrhon,
qui en vendait parfois sur les marchés, devait bien faire la différence entre un
cochon et un poulet), mais parce que aucune n’est fondée en vérité ni en valeur :
parce que tout se vaut, qui ne vaut rien. On objectera que Pyrrhon, sur les marchés,
devait bien évaluer différemment ce qu’il vendait (un cochon n’a pas le
même prix qu’un poulet). Mais c'était le problème des clients, non le sien. Être
indifférent, ce n’est pas être aveugle ou stupide. C’est être neutre et serein.

Le peut-on? Et pourquoi le faudrait-il, si la sérénité elle-même est
indifférente ? Pyrrhon est l’un des rares philosophes qui ait été vraiment nihiliste,
et qui donne envie de l’être. Mais cette envie le réfute ou nous empêche
de le suivre. Si rien ne vaut, le nihilisme ne vaut rien.

Il y a pourtant de bonnes indifférences, mais partielles ou ciblées : celles qui
refusent de considérer ce qui ne doit pas l’être ou d’attacher de l'importance à
ce qui n’en a pas. La justice, par exemple, ne va guère sans impartialité, qui est
comme une indifférence de principe. Et la charité se distingue de l’amitié en ce
qu’elle est un amour indifférencié (ce que Fénelon appelait la « sainte
indifférence »). Mais ni la charité ni la justice ne supposent pour cela qu’on soit
indifférent à tout ou à elles-mêmes : elles cesseraient autrement de valoir. Être
impartial, ce n’est pas être indifférent à la justice. C’est être indifférent à tout le
reste. Ainsi l’indifférence ne vaut, c’est son paradoxe, qu’à la condition d’être
différenciée. Il arrive même qu’elle soit une vertu: rester indifférent au
médiocre ou au dérisoire, ce n’est pas nihilisme, c’est grandeur d’âme.

\section{Indifférence (Liberté d'—)}
%INDIFFÉRENCE (LIBERTÉ D’)
Le libre arbitre, mais en tant qu’il ne
serait soumis à aucune inclination ou
préférence : c’est la liberté de l’âne de Buridan, sil en a une. Descartes n’y
voyait que « le plus bas degré de la liberté, qui fait plutôt paraître un défaut
dans la connaissance qu’une perfection dans la volonté ; car si je connaissais
toujours clairement ce qui est vrai et ce qui est bon, je ne serais jamais en peine
de délibérer quel jugement et quel choix je devrais faire, et ainsi je serais entièrement
libre sans jamais être indifférent ({\it Méditations}, IV ; voir aussi la {\it Lettre au
Père Mesland} du 9 février 1645). Celui, à l'inverse, qui serait parfaitement
indifférent, que lui importerait d’être libre ? Et quel usage pourrait-il faire de sa
liberté ?

\section{Indiscernables (principe des —)}
%INDISCERNABLES (PRINCIPE DES —)
C’est un principe leibnizien, qui
stipule que tout être réel est
intrinsèquement différent de tous les autres, autrement dit qu’il n’existe pas
% 306 
d’êtres absolument identiques ou indiscernables (qui ne se distingueraient que
numériquement ou par des données extrinsèques, comme leur position dans
l’espace et le temps). Deux gouttes d’eau, deux feuilles du même arbre ou deux
cachets d’aspirine ne semblent indiscernables que parce que nous les observons
mal : mettez-les sous un microscope, il ne sera plus possible de les confondre.
Le principe, selon Leibniz, est sans exception : tout être est unique ; l’infinie
multiplicité du réel n’est constituée que de singularités absolues (les monades).

Et si l’on n’est pas leibnizien ? Alors il reste que tout être est différent de
tous les autres — y compris de lui-même à un autre moment. C’est pourquoi il
n’y a pas d'êtres : il n’y a que des événements. On ne se baigne jamais deux fois
dans le même fleuve, ni même une seule fois. On n’est plus chez Leibniz, mais
chez Héraclite ou Montaigne.

\section{Individu}
%INDIVIDU
Un être vivant quelconque, dans une espèce quelconque, mais
en tant qu'il est différent de tous les autres. Rien de plus banal
qu’un individu, et rien de plus singulier : c’est la banalité d’être soi.

Se dit spécialement d’un être humain, mais considéré plutôt comme objet
que comme sujet, plutôt comme résultat que comme principe, plutôt comme
élément (dans un ensemble donné : une espèce, une société, une classe.) que
comme personne. N'importe qui, donc, en tant qu’il est quelqu'un.

Indivisible ? C’est ce que suggère l’étymologie : {\it individuum}, en latin, traduit
le grec {\it atomon}. Cela ne prouve pas plus dans un cas que dans l’autre (nous
savons aujourd’hui que les atomes sont sécables), mais rencontre pourtant
l'expérience commune. Non qu’on ne puisse diviser un être vivant ; mais parce
que ce qu’il y a d’individuel en lui n’est pas divisé par là. Un cul-de-jatte n’est
pas une moitié d’individu.

\section{Individualisme}
%INDIVIDUALISME
C’est mettre l'individu plus haut que l’espèce ou que la
société, voire plus haut que tout (par exemple plus
haut que Dieu ou la justice). Mais quel individu ? S'il ne s’agit que du moi,
l’individualisme n’est qu’un autre nom, moins péjoratif, pour dire l’égoïsme.
S’il s’agit de tout individu, ou plutôt de tout être humain, ce n’est qu’un autre
nom, moins emphatique, pour dire l’humanisme. Entre ces deux pôles, le mot
ne cesse de fluctuer ; il doit l'essentiel de son succès au flou qui en résulte.

\section{Induction}
%INDUCTION
C'est un type de raisonnement, qu’on définit classiquement
par le passage du particulier au général, ou des faits à la loi.
%— 307 —
S’oppose en cela à la déduction, qui va du général au particulier ou d’un principe
à ses conséquences.

On comprend que l’induction, qui est amplifiante, pose davantage de problèmes
que la déduction, qui est plutôt réductrice. Une fois admis que tous les
hommes sont mortels, il n’est pas douteux que cet homme-ci le soit : le singulier
n’est qu’un sous-ensemble de l’universel. Mais combien faut-il avoir vu
d'hommes mourir pour savoir qu’aucun n’est immortel ? En pratique ou psychologiquement,
beaucoup moins qu’il n’en meurt en effet. Mais d’un point de
vue logique ? Comment passer d’énoncés singuliers en nombre toujours fini
(« Tel homme est mort, et tel autre, et tel autre, et tel autre... ») à un énoncé
universel (« {\it Tous} les hommes sont mortels ») ? C’est ce qu’on appelle depuis
Hume le problème de l’induction. Combien faut-il avoir vu de cygnes blancs
pour savoir qu'ils le sont tous ? Combien de corps en chute libre faut-il avoir
observés pour savoir que, dans le vide, ils tombent tous à la même vitesse ? Il
faudrait avoir vu tous les cygnes et mesuré toutes les chutes, ce qui est bien sûr
impossible, ou bien supposer, au bout d’une certain nombre d’observations,
que les cas à venir ressembleront à ceux déjà observés. Mais cette dernière supposition
— que le futur ressemblera au passé — ne va pas de soi et ne peut être
démontrée ni par déduction (puisqu'il s’agit d’une question de fait) ni par
induction (puisque toute induction le suppose). Toute induction aboutit donc
à une conclusion qui excède ses capacités logiques : elle est formellement non
valide et empiriquement douteuse. La confiance que nous accordons à ce type
de raisonnement doit beaucoup plus à l’habitude, conclut Hume, qu’à la
logique ({\it Traité}, I, 3 ; {\it Enquête}, IV). Or, s'agissant de la connaissance du monde,
c’est ordinairement l'induction qui fournit à la déduction les principes généraux
dont celle-ci déduit les conséquences : si l'induction est douteuse, toute
déduction appliquée à l’expérience l’est aussi. À la gloire de Hume et du scepticisme.

À ce problème de l'induction, je ne connais qu’une seule solution satisfaisante.
C’est celle de Popper, qui résout le problème, de façon à la fois négative
et radicale, en montrant qu’{\it il n'y à pas d'induction logiquement valide}. Comment
les sciences expérimentales sont-elles pourtant possibles ? Par déduction :
on pose un principe (une hypothèse, une loi, une théorie...), dont on déduit
des conséquences (par exemple sous la forme de prévisions). Si ces conséquences
sont réfutées par l’expérience, le principe est faux. Si les conséquences
ne sont pas réfutées par l'expérience, ou tant qu’elles ne le sont pas, le principe
est considéré comme possiblement vrai : il a survécu, au moins provisoirement,
à l'épreuve du réel. Il en résulte que « seule la fausseté d’une théorie est susceptible
d’être inférée des données empiriques, et que cette sorte d’inférence est
%— 308 —
purement déductive » ({\it Conjectures et réfutations}, I, 9 ; voir aussi {\it La logique de la
découverte scientifique}, I).

L’argumentation de Popper est fondée, comme il le remarque lui-même,
sur « une {\it asymétrie} entre la vérifiabilité et la falsifiabilité, asymétrie qui résulte
de la forme logique des énoncés universels » : on ne peut conclure de la vérité
d’énoncés singuliers à celle d’un énoncé universel (dix mille cygnes blancs ne
prouvent pas qu’ils le soient tous), mais on peut conclure de leur vérité à la
{\it fausseté} d’énoncés universels (il suffit d’un cygne noir pour prouver qu’ils ne
sont pas tous blancs). « Cette manière de prouver la fausseté d’énoncés universels,
conclut Popper, constitue la seule espèce d’inférence strictement
déductive qui procède, pour ainsi dire, dans la “direction inductive”, c’est-à-dire
qui va des énoncés singuliers aux énoncés universels. » Ainsi il n’y a pas
de logique inductive, ni d’induction logiquement probante, mais il y a ce
qu’on pourrait appeler un {\it effet d'induction} (on passe bien du particulier au
général ou à l’universel) qui permet d’énoncer des lois scientifiques — par
exemple celle de la chute des corps — qui sont à la fois possiblement vraies et
empiriquement testables. Les sciences et l’action n’en demandent pas davantage.

\section{Ineffable}
%INEFFABLE
Synonyme à peu près d’indicible, en plus suave ou plus mystérieux.
L’indicible le serait plutôt par excès de force, de plénitude
ou de simplicité ; l’ineffable, par excès de délicatesse, de finesse, de subtilité...
Enfin, l’ineffable ne se dit guère que positivement (alors qu'on peut
parler d’une souffrance indicible, d’un malheur indicible). Ces nuances, sans
être tout à fait indicibles, restent toutefois quelque peu ineffables.

\section{Inertie}
%INERTIE
C’est d’abord et paradoxalement une force : la force qu’a un corps
de persévérer dans son mouvement ou son repos. Le principe
d'inertie stipule en effet qu’un objet matériel conserve de lui-même son état de
repos ou de mouvement rectiligne uniforme : qu’il ne peut être mû (s’il est en
repos), dévié ou freiné (s’il est en mouvement) que par une force extérieure.
L’inertie n’est donc pas l’immobilité (un corps en mouvement rectiligne uniforme
ne manifeste pas moins d’inertie qu’un corps immobile), ni même l’inaction
(un corps inerte peut produire quelque effet : par exemple s’il me tombe
sur le pied). L’inertie, c’est l'incapacité à changer de soi-même son mouvement,
ou à se changer soi. C’est pourquoi le mot, appliqué à un être humain, est toujours
péjoratif : se subir, c’est déchoir.
%— 309 —
\section{Inespoir}
%INESPOIR
L'absence d’espoir, mais considérée comme un état neutre et
originel : « non le deuil de l'espoir, disait Mounier, mais son
constat de défaut ». C’est ce qui le distingue du désespoir, ou qui l’en distinguerait
s’il était possible de se passer d’espoir sans en faire le deuil. Mais qui le
peut ? L'espoir est premier : il faut le perdre, d’abord, pour apprendre à s’en
passer. Ainsi l’inespoir n’est qu’un désespoir qui serait parfaitement accompli.
Non un point de départ, mais un point d’arrivée. Non un travail, mais un
repos. C’est le propre des sages et des dieux. Pour tous les autres, ce n’est qu’un
mensonge de plus.

\section{Inférence}
%INFÉRENCE
C’est passer d’une proposition tenue pour vraie à une autre
qu'on juge en conséquence l'être aussi, en vertu d’un lien
nécessaire ou supposé tel. Ce passage peut être inductif (si on passe de faits particuliers
à une conclusion plus générale) ou déductif (si on passe d’une proposition
à l’une de ses conséquences). On considère généralement que l’inférence
inductive ne peut faire passer que du vrai au probable, quand la déduction conclut
du vrai au vrai. On n’en tirera pas trop vite un argument contre les faits. Il
suffit d’un seul pour réfuter — par inférence déductive, en l’occurrence sous la
forme du {\it modus tollens} — la théorie la plus ambitieuse. C’est ce que Karl Popper
appelle la falsification ({\it La logique de la découverte scientifique}, chap. X, III et IV).

\section{Infini}
%INFINI
L’étymologie est transparente : l'infini, c’est ce qui est sans limite,
sans borne ({\it finis}), sans fin. On ne le confondra pas avec l’indéfini,
qui est sans limite connue ou connaissable.

Les illustrations les plus commodes sont mathématiques. Chacun comprend
que la suite des nombres est infinie, puisqu'on peut toujours ajouter un
nombre quelconque à celui qui serait supposé être le plus grand. On remarquera
qu’une partie d’un ensemble infini n’est pas nécessairement infinie (par
exemple les entiers situés entre 3 et 12 sont en nombre fini), mais peut l’être (la
suite des nombres pairs est aussi infinie que celle des entiers, dont elle constitue
pourtant une partie). Ainsi un ensemble infini a cette caractéristique exceptionnelle
— qui peut servir, mathématiquement, à le définir — qu’il peut être mis en
bijection (en correspondance biunivoque) avec l’un au moins de ses sous-ensembles
stricts : tout entier peut être mis en relation biunivoque avec son
carré, c'était l’exemple pris par Galilée, quand bien même la série infinie des
carrés parfaits n’est qu’un sous-ensemble de la série des entiers. Il en résulte que
le tout, s'agissant de l'infini, n’est pas nécessairement plus grand que telle de ses
parties (puisque celle-ci peut être aussi infinie). Ce qui permet de définir, par
%— 310 —
différence, la finitude : est fini tout ensemble qui est nécessairement plus grand
que l’un quelconque de ses sous-ensembles stricts (c’est-à-dire autres que lui-même).

Un exemple non mathématique ? On pense bien sûr à Dieu, dont Descartes
disait qu'il était seul infini proprement, n’ayant aucune borne d'aucune
sorte. Si on lui applique l'observation qui précède, on peut en conclure qu’il ne
serait pas nécessairement plus grand que telle de ses parties, et par exemple
qu’un Dieu trinitaire ne serait pas plus grand que chacune des trois Personnes
qui constituent l’unité de son essence (du moins si chacune est supposée
infinie). Cela toutefois ne prouve pas qu’il existe, ni qu’il soit trois.

Quant à trouver un exemple empirique, c’est ce qu’on ne peut : l’expérience
n’a accès qu’au fini, ou à l’indéfini. On pensera pourtant à la fameuse
formule de Pascal, dont on croit parfois qu’elle s’applique à Dieu, ce n’est pas
un hasard, mais que Pascal, qui reprend d’ailleurs une métaphore traditionnelle,
n'utilise qu’à propos de l’univers : « une sphère infinie dont le centre est
partout, la circonférence nulle part » ({\it Pensées}, 199-72). Mais de cette infinité,
d’ailleurs douteuse, nous n’avons qu’une idée, point une expérience.

\section{Injure}
%INJURE
C'est une dénonciation haineuse, qui s'adresse, c’est son paradoxe,
à celui-là même qu’elle dénonce. Dans quel but ? D'abord le plaisir
qu’on y trouve, qui est parfois d’hygiène. Cela vaut mieux qu’un meurtre ou
qu’un ulcère de l’estomac. Aussi une certaine exigence de vérité, voire de justice,
qu’on imposerait par la parole à celui qu’on insulte, comme s’il était
besoin de lui apprendre ce qu’il est, ou ce qu’on en pense, de le démasquer à
ses propres yeux, enfin de l’obliger à se voir, au moins une fois, dans le miroir
de notre mépris. Regarde-toi dans mon regard, juge-toi dans mon jugement :
tu es ce que je dis ! C’est comme une vérité performative, et l’on se doute que
la logique n’y trouve pas son compte. L’injure peut être fondée ou non (elle
peut être médisance ou calomnie) ; elle est toujours injuste, comme l'indique
l’étymologie, par le refus de comprendre et la volonté de blesser. Mais cette
injustice en vient corriger une autre, qui nous semble, au moins dans l'instant,
plus grave ou plus insupportable. La logique est celle d’un châtiment ; injurier,
c’est se faire juge, procureur et bourreau. La conjonction de ces trois rôles suffit
à mettre l’injure à sa place : ce n’est pas justice, mais colère.

\section{Inné}
%INNÉ
Ce qui est donné ou programmé dès la naissance. On ne confondra
pas l’inné (qui s’oppose à l’acquis) avec l’{\it a priori} (qui s’oppose à l'empirique).
L’inné relève moins du transcendantal que de l’immanental (voir ces mots).
%— 311 —
Et l’on évitera de dire trop vite que rien n’est inné en l’homme : ce serait faire
comme si nous n'avions pas de corps (lequel est par définition inné) ou comme s’il
était quantité négligeable. L'expérience et la génétique prouvent le contraire.

\section{Innéisme}
%INNÉISME
Ce n’est pas croire qu’il y a de l’inné en l’homme, ce qui n’est
guère contestable, mais que l’innéité ne se réduit pas au corps :
que certaines idées ou conduites sont inscrites en nous dès la naissance. C’est
croire en l’innéité non du corps seulement mais de l'esprit.

Pour un matérialiste, un certain innéisme va de soi : si le corps et l'esprit
sont une seule et même chose, l’innéité du corps entraîne celle de l’esprit. Le
cerveau, par définition, est inné. Cela n'empêche pas qu’il se développe et se
construise aussi {\it après} la naissance. L’inné ne s’oppose à l’acquis que pour autant
qu'il le permet, comme l’acquis ne s'oppose à l’inné que pour autant qu’il le
suppose. Par exemple le langage, comme faculté, est inné ; mais toute langue
est acquise. L'erreur de Descartes est d’avoir cru que l’innéité de l'esprit était
celle d’idées toutes faites, alors qu’il s’agit plutôt de fonctions (indissociablement
neuronales et logiques) et d’idées à faire.

\section{Inquiétude}
%INQUIÉTUDE
C’est un rapport présent à l’avenir, en tant que l’avenir
vient troubler le présent. Oscille entre le souci ({\it « Que faire ? »})
et la crainte ({\it « Comment y échapper ? »}). C’est pourquoi on n’échappe pas à l’inquiétude,
ou c’est pourquoi, plutôt, on n’y échappe que dans les rares moments où l’on
vit pleinement au présent : dans la quiétude de la contemplation ou de l’action.

\section{Inquisition}
%INQUISITION
Le mot signifie d’abord recherche ou enquête. Mais avec
une majuscule, c’est une enquête très particulière, plutôt
policière que théorique : les inquisiteurs ne cherchent plus la vérité, puisqu'ils
sont censés la connaître, mais des coupables. Qu’on ait pu torturer et massacrer
au nom des Évangiles en dit long sur la bêtise humaine, et sur les dangers du
fanatisme. C’est vouloir éclairer la splendeur de la vérité, comme dit Jean-Paul II,
par des bûchers. {\it Veritatis terror !}

\section{Insensé}
%INSENSÉ
Qui est contraire au bon sens ou à la raison.
On remarquera que ce qui est insensé est rarement insignifiant.
La folie est un état grave, qui peut être riche de sens ; et le bavardage le plus
insignifiant est rarement insensé.

%— 312 —
\section{Insignifiant}
%INSIGNIFIANT
Qui n’a pas de sens, pas de valeur ou pas d'importance.
Cette polysémie en dit long sur les humains, qui n’accordent
ordinairement d'intérêt qu’à ce qui signifie. Le doigt montre la lune. Ils
regardent le doigt.

\section{Insistance}
%INSISTANCE
C’est un mot que j'ai pris l’habitude d’opposer à l'existence,
au sens existentialiste du terme, et qui permet de la penser
autrement. Exister, c’est être dehors (hors de soi, hors de tout). Insister, c’est
être {\it dans}, et s’efforcer de s’y maintenir. Dans quoi ? Dans l’être, dans le présent,
dans tout. Comment pourrait-on {\it exister} autrement ? L’insistance, au sens
où je prends le mot, est un autre nom pour dire le {\it conatus}, en tant qu’il n'existe
que dans le temps et l’espace, en tant qu’il est toujours confronté à autre chose,
à quoi il résiste. C’est moins le contraire de l'existence que sa vérité. Nul ne
s'oppose à la nature qu’à la condition d’être dedans ; nul ne se projette dans
l'avenir qu’au présent : exister, c’est insister et résister. L’homme n’est pas un
empire dans un empire, ni un néant dans l’être. Il n’y a pas de transcendance :
il n’y a que l’immanence, que l’impermanence, et le {\it dur désir de durer}. n’y a
que des forces et des rapports de forces. Il n’y a que l’histoire. Il n’y a que l'être,
et l’insistance de l’être.

\section{Insistantialisme}
%INSISTANTIALISME
Il m'est arrivé d’utiliser ce mot, par jeu et par opposition
à l’existentialisme, pour caractériser ma propre
position ({\it L'être-temps}, VIII). Exister, C’est être {\it dehors} — hors de soi, hors de
tout. Insister, c’est être {\it dans}, et s’efforcer d’y rester : philosophie de l’immanence,
de la puissance, de la résistance (l’{\it energeia} chez Épicure, la {\it tendance} où
la {\it tension} chez les stoïciens, le {\it conatus} chez Spinoza, la volonté de puissance
chez Nietzsche, l'intérêt chez Marx, la pulsion de vie chez Freud...). C’est qu’il
n’y a pas de {\it dehors} absolu, pas de transcendance, pas d’au-delà : il n’y a que le
monde, il n’y a que tout. L’insistantialisme n’est pas un humanisme : c’est une
pensée de l’être, non de l’homme. Il enseigne que l’essence précède l'existence,
ou plutôt que rien n'existe que ce qui est (essence et existence, dans le présent
de l'être, sont bien sûr confondues ; c’est ce que Spinoza appelle {\it essence
actuelle}, qui ne fait qu’un avec la puissance en acte : {\it Éthique}, III, prop. 7 et
dém.) et continue d’être : rien n'{\it existe} que ce qui {\it insiste}. Philosophie non de
l'homme mais de la nature. Non de la transcendance mais de l’immanence.
Non du néant mais de l’être. Non du sujet mais du devenir. Non de la conscience
mais de l’acte. Non du libre arbitre, mais de la nécessité et de la libération.

%— 313 —
\section{Instant}
%INSTANT
Ce serait un point de temps : une portion de durée qui ne durerait
pas — non une durée, disait Aristote, mais une limite entre
deux durées. Ce n’est donc qu’une abstraction. Le seul instant réel, c’est le présent,
qui ne cesse de continuer. En quoi est-ce un instant ? En ceci qu'il est
indivisible (que serait un demi-présent ?) et sans durée (combien dure le
présent ? et comment pourrait-il durer sans être composé de passé et
d’avenir ?). C’est l'instant vrai : non une portion de durée qui ne durerait pas,
mais l'acte même de durer, en tant qu’il est indivisible et sans durée. C’est
l'éternité en acte.

\section{Instinct}
%INSTINCT
Un savoir-faire transmis biologiquement. L'homme en est à
peu près dépourvu : il n’a guère que des pulsions, qu’il faut
éduquer.

\section{Intellectuel}
%INTELLECTUEL
C'est celui qui vit de sa pensée, ou pour sa pensée. Il n’a
guère le choix qu’entre une petitesse (penser pour vivre)
et une illusion (vivre pour penser). Il n’y a pas de sot métier, mais pas non plus
de vanité intelligente.

\section{Intelligence}
%INTELLIGENCE
La capacité, plus ou moins grande, de résoudre un problème,
autrement dit de comprendre le complexe ou le
nouveau.

\section{Intelligible}
%INTELLIGIBLE
Au sens large : ce qui peut être compris par l'intelligence
(c’est le contraire d’inintelligible).

Au sens étroit : ce qui ne peut être compris ou visé {\it que} par l’intelligence,
jamais senti par le corps. Ainsi le {\it monde intelligible} chez Platon. C’est le
contraire du matériel ou du sensible.

\section{Intention}
%INTENTION
Une volonté présente, mais qui porte sur l’avenir ou sur la fin
qu’on poursuit. C’est le projet de vouloir, ou la visée de la
volonté.

C’est pourquoi on parle d’une {\it morale de l'intention}, pour désigner une
morale, comme celle de Kant, qui mesure la moralité d’une action non à ses
effets mais à la disposition de la volonté qui l’accomplit. L'expression, bizarrement,
%— 314 —
a souvent un sens péjoratif. J'y vois un contresens sur l’intention ou sur
Kant. Une morale de l'intention n’est pas une morale qui se contenterait
d’avoir de bonnes intentions, comme on dit: c’est une morale qui juge la
volonté en acte, non ses suites éventuelles, que personne, quand il agit, ne
connaît tout à fait. Un homme tombe à la mer. L'un des matelots lui jette un
rondin, pour l’assommer ; Le rondin flotte et le sauve. Un autre matelot lui jette
une bouée, pour qu’il s’y agrippe : la bouée lui tombe sur la tête et l’assomme ;
il se noie. Une morale de l'intention est celle qui juge l'attitude du second
marin, aussi néfaste qu’elle s’avère finalement, plus estimable que celle du
premier, fût-elle salutaire. Qui ne voit que c’est la morale même ? Nul n’est
tenu de réussir, ni dispensé d’essayer.

\section{Intentionnalité}
%INTENTIONNALITÉ
Rien à voir avec le fait d’avoir une intention, au sens
courant du terme. L’intentionnalité, en philosophie,
relève du vocabulaire technique (l’{\it intentio}, chez les scolastiques, est l’application
de l'esprit à son objet), spécialement phénoménologique. « Le mot {\it intentionnalité},
écrit Husserl, ne signifie rien d’autre que cette particularité foncière
et générale qu’a la conscience d’être conscience de quelque chose, de porter, en
sa qualité de {\it cogito}, son {\it cogitatum} en elle-même » ({\it Méditations cartésiennes},
\S 14). La conscience n’est pas une chose ou une substance, qui se suffirait de
soi ; elle est visée, relation à, éclatement vers. « Toute conscience, écrit Husserl,
est conscience {\it de} quelque chose. » Telle est l’intentionnalité. « La conscience,
écrira Sartre, n’a pas de “dedans” ; elle n’est rien que le dehors d’elle-même »
(« Une idée fondamentale de la phénoménologie de Husserl : l’intentionnalité »,
{\it Situations}, I). L’intentionnalité est cette ouverture de la conscience vers
son dehors (y compris vers l’{\it ego}, en tant qu’il est objet pour la conscience).
C’est la seule intériorité vraie : « tout est dehors, écrit Sartre, tout, jusqu’à
nous-mêmes » ({\it ibid.}).

\section{Intérêt}
%INTÉRÊT
Subjectivement, c’est une forme de désir ou de curiosité, et souvent
la conjonction des deux. Mais on peut aussi avoir objectivement
intérêt à quelque chose pour lequel on n’éprouve ni désir ni curiosité. On
peut dire, par exemple, que c’est l'intérêt de l’enfant que d’apprendre sa leçon ;
ou, avec Marx, que les travailleurs ont intérêt à la révolution : cela ne prouve
pas que tous les travailleurs soient révolutionnaires, ni que tous les élèves soient
studieux.
Qu'est-ce alors que cet intérêt prétendument objectif? Non ce qu'on
désire, mais ce qu’on {\it devrait} désirer, si l’on connaissait vraiment son bien et le
%— 315 —
chemin qui y mène. C’est le désir réfléchi ou intelligent : celui qu’on éprouve à
juste titre ou qu’un autre juge qu’on devrait éprouver. Intérêt objectif ? Seulement
pour celui qui le juge tel. Intérêt subjectif, donc, ou projectif. Car enfin
nul n’est tenu d’être marxiste, ni studieux.

La plupart des matérialistes enseignent que c’est l'intérêt qui meut les
hommes : ainsi chez Épicure, chez Hobbes, chez Helvétius (« on obéit toujours
à son intérêt », {\it De l'esprit}, I, 4 et II, 3), chez d’Holbach (« l'intérêt est l’unique
mobile des actions humaines », {\it Système de la nature}, I, 15), enfin chez Freud
(par les principes de plaisir et de réalité) ou chez Marx (« Les individus ne cherchent
que leur intérêt particulier », {{\it Idéologie allemande}, I). C’est vrai aussi (sous
l'appellation de « l’utile propre » : voir {\it Éthique}, IV, prop. 19 à 25) chez Spinoza,
qui n'est pas matérialiste, comme chez Hegel, qui ne l’est pas davantage
ou plutôt qui l’est encore moins : « C’est leur bien propre que peuples et individus
cherchent et obtiennent dans leur agissante vitalité » ; toute la « ruse de la
raison » est de mettre ces intérêts particuliers au service d’une fin générale
({\it Leçons sur la philosophie de l'histoire}, Introd., II). On se trompe évidemment si
l’on y voit une apologie de l’égoïsme. C’est tout le contraire : une tentative
pour le dépasser ou le sublimer. Il est de l'intérêt de chacun, pour tous ces
auteurs, de tendre à l'intérêt général. Qui peut être heureux tout seul ? Qui
peut s’aimer sans s’estimer ? C’est où l’amour de soi mène à l’amour de la justice
ou du prochain. « Soyez égoïstes, dit un jour le dalaï-lama : Aimez-vous les
uns les autres ! »

\section{Interprétation}
%INTERPRÉTATION
C'est chercher ou révéler le sens de quelque chose
(d’un signe, d’un discours, d’une œuvre, d’un événement...).
S'oppose par là à l'{\it explication}, qui donne non le sens mais la cause.
Les deux démarches peuvent bien sûr être légitimes ; mais il ne l’est jamais de
les confondre. Tout a une cause, et certains faits ont un sens. Mais comment
un fait pourrait-il s’expliquer par ce qu’il signifie ? On pense aux actes manqués
ou aux symptômes selon Freud : leur sens (par exemple un désir refoulé) n’est-il
pas aussi leur cause ? Sans doute, mais de deux points de vue différents. Ce
n’est pas parce qu’il signifie quelque chose qu’un lapsus se produit (car bien des
lapsus seraient tout aussi signifiants, qui ne se produisent pas) ; c’est parce qu’il
est produit par autre chose (un désir, une résistance, tel ou tel processus psychique
ou neuronal...) qu’il a un sens. Ainsi l’ordre des signes est soumis à
celui des causes, qui ne signifie rien. C’est en quoi la sexualité est bien, comme
Freud l’a dit lui-même, «le bloc de granit» de la psychanalyse. Tout sens
inconscient y renvoie. Mais elle-même ne signifie rien.

%— 316 —
\section{Intersubjectivité}
%INTERSUBJECTIVITÉ
L'ensemble des relations entre les sujets : leurs
échanges, leurs sentiments mutuels, leurs ébats et
leurs débats, leurs conflits, leurs rapports de forces ou de séduction... Il n’y
aurait pas de sujet autrement. Chacun ne se constitue que dans son rapport aux
autres : on ne se pose qu’en s’opposant, comme disait Hegel, on n’apprend à
aimer qu’en étant aimé d’abord, à penser qu’en comprenant la pensée d'autrui,
etc. Le solipsisme est une idée de philosophe, et c’est une idée creuse. Cela, toutefois,
ne supprime pas la solitude. On n'existe qu’avec les autres, mais ils ne
sauraient exister à notre place.

\section{Introspection}
%INTROSPECTION
L'observation de soi par soi, mais tournée vers l’intérieur,
comme une autocontemplation de l'esprit. Elle
est impossible à effectuer en toute rigueur (autant vouloir, disait Auguste
Comte, du haut de son balcon, se regarder passer dans la rue) et pourtant
nécessaire : il faut bien apprendre à se connaître ou à se reconnaître. Le {\it je} se
mire dans la conscience ; c’est ce qu’on appelle le {\it moi}. Mais ce n’est jamais
qu’un reflet, qu'une image sans consistance ni véritable profondeur. La
mémoire, le dialogue et l’action nous en apprennent davantage.

\section{Intuition}
%INTUITION
{\it Intueri}, en latin, c’est voir ou regarder : l’intuition serait une
vue de l'esprit, avec tout ce que cela suppose d’immédiat,
d’instantané, de simple... et de douteux. Avoir une intuition, c’est sentir ou
pressentir, sans pouvoir démontrer ni prouver. L’intuition se situe en amont du
raisonnement. Mais un esprit totalement dépourvu d’intuition serait aveugle.
Comment pourrait-il raisonner ?

Dans la langue philosophique, la polysémie du mot, qui est à peu près inépuisable,
s'organise autour de trois sens principaux : ceux de Descartes, Kant et
Bergson.

Pour Descartes, l'intuition est « une représentation qui est le fait de l’intelligence
pure et attentive, représentation si facile et si distincte qu’il ne subsiste
aucun doute sur ce que l’on comprend » ({\it Règles pour la direction de l'esprit}, III).
C’est la simple vision intellectuelle du simple : vision évidente de l'évidence.
« Ainsi, chacun peut voir par intuition qu’il existe, qu’il pense, que le triangle
est délimité par trois lignes seulement et la sphère par une seule surface. »
S’oppose en ce sens à la déduction ou au raisonnement, mais aussi les permet :
ce sont comme des chaînes de pensées, dont chaque maillon doit être vu par
intuition pour que la chaîne elle-même soit saisie ou suivie par « un mouvement
continu et ininterrompu de la pensée qui prend de chaque terme une
%— 317 —
intuition claire ». Pour comprendre que 2 et 2 font la même chose que 3 et 1,
explique Descartes, « il faut voir intuitivement, non seulement que 2 et 2 font
4, et que 3 et 1 font 4 également, mais en outre que, de ces deux propositions,
cette troisième-là se conclut nécessairement ». Ainsi l'intuition peut se passer de
raisonnement, non le raisonnement se passer d’intuition.

Chez Kant, l'intuition est aussi une façon immédiate, pour la connaissance,
de se rapporter à un objet quelconque : elle est ce par quoi un objet nous est
donné ({\it C. R. Pure}, « Esthétique transcendantale », \S 1). Mais l’homme ne dispose
d'aucune intuition intellectuelle ou créatrice. Il n’a d’intuition que sensible
et passive : l'intuition est la réceptivité de la sensibilité (elle ne contient
que « la manière dont nous sommes affectés par des objets »), dont les formes
pures sont l’espace et Le temps. S’oppose à la connaissance par concepts, qui
n'est pas intuitive mais discursive : « Des pensées sans contenu sont vides ; des
intuitions sans concepts sont aveugles » ({\it C. R. Pure}, « Logique transcendantale »,
Introd., I).

Chez Bergson, l'intuition est surtout une méthode : c’est retrouver les vrais
problèmes et les vraies différences (celles qui sont de nature, non de degré) en
se transportant «à l’intérieur d’un objet, pour coïncider avec ce qu'il a
d’unique et par conséquent d’inexprimable », spécialement avec sa durée
propre et son essentielle mobilité. Penser intuitivement, c’est penser en durée
et en mouvement. L’intuition s'oppose en ce sens à l’analyse (qui exprime une
chose «en fonction de ce qui n’est pas elle ») ou à l'intelligence (qui part de
l'immobile), comme la métaphysique s'oppose à la science (qui ne connaît
guère que l’espace) et comme l'absolu s'oppose au relatif ({\it La pensée et le mouvant},
II et VI ; voir aussi Deleuze, {\it Le bergsonisme}, PUF, 1966, chap. I).

On voit que l'intuition, dans les trois cas, est définie positivement, mais de
trois façons différentes : elle est la condition de toute pensée (Descartes), de
toute connaissance (Kant), enfin de toute saisie de l'absolu, qui est esprit,
durée, changement pur (Bergson). Elle donne à la pensée son évidence (chez
Descartes), son objet (chez Kant) ou son absoluité (chez Bergson). Elle n’a en
commun, chez ces trois auteurs, qu’une forme d’immédiateté, qui peut servir
pour cela de définition générale : on peut appeler {\it intuition} toute connaissance
immédiate, s’il en est une.

\section{Invention}
%INVENTION
Plus qu’une découverte, moins qu’une création. Inventer,
c'est faire être ce qui n’existait pas (c'est ce qui distingue
l'invention de la découverte), mais qui aurait existé tôt ou tard (c’est ce qui distingue
l'invention de la création). Ainsi, Christophe Colomb a {\it découvert} l'Amérique
(il ne l’a pas inventée : elle existait avant lui) ; Newton a {\it découvert} la gravitation
%— 318 —
universelle (même remarque) ; alors que Denis Papin a {\it inventé} la
machine à vapeur, avec d’autres, comme Edison le télégraphe, le phonographe
et la lampe à incandescence. Ces quatre instruments, avant d’être inventés,
n’existaient nulle part. Mais sans Papin ou Edison, nul doute qu’ils auraient
fini, fût-ce un peu plus tard et sous une autre forme, par exister : notre univers
technique, si ces deux inventeurs étaient morts à la naissance, serait peut-être
différent de ce qu’il est, mais plutôt dans ses détails ou le cheminement anecdotique
de son développement que dans son contenu essentiel (la révolution
industrielle et celle de la communication n’en auraient pas moins eu lieu).
Alors que si Mozart ou Michel-Ange étaient morts à la naissance, nous
n’aurions jamais eu ni les {\it Esclaves du Louvre} ni le {\it Concerto} pour clarinette : ce
n'est plus {\it invention} mais {\it création}.

\section{Involution}
%INVOLUTION
Le contraire d’une évolution (voir ce mot), ou sa forme
régressive : c’est évoluer à l'envers.

\section{Ipséité}
%IPSÉITÉ
Du latin {\it ipse} (même : au sens de {\it lui-même}, {\it moi-même}, etc.). Le fait
d’être soi. Suppose l’unité, l’unicité, l'identité, enfin (spécialement
chez les phénoménologues : voir {\it L'être et le néant}, p. 147 à 149) la conscience.
C’est l’idiotie (voir ce mot) en personne.

\section{Ironie}
%IRONIE
C'est se moquer des autres, ou de soi (dans l’autodérision) comme
d’un autre. L’ironie met à distance, éloigne, repousse, rabaisse.
Elle vise moins à rire qu’à faire rire. Moins à amuser qu’à désabuser. Ainsi
Socrate, vis-à-vis de tous les savoirs, et du sien propre. Il interroge ({\it eirôneia}, en
grec, c’est l'interrogation), quitte, parfois, à feindre d’ignorer ce qu’il sait, pour
essayer de découvrir ce qu’il ignore ou qu’on ne peut savoir. L’ironie est le
contraire d’un jeu : elle relève moins du principe de plaisir, dirait Freud, que
du principe de réalité, moins du loisir que du travail, moins de la paix que du
combat. Elle est utile : c’est là sa force en même temps que sa limite. C’est une
arme, c’est un outil, et ce n’est que cela. Un moyen, jamais une fin. Parfois
indispensable, jamais suffisante. C’est un moyen de se faire valoir, fût-ce à ses
dépens. C’est le moment du négatif, qui n’est supportable que comme
moment. L'esprit toujours nie, mais évite soigneusement, dans l'ironie, de se
nier lui-même : c’est un rire qui se prend au sérieux. Comment atteindrait-il
l'essentiel, puisqu'il nous en sépare ? « Gagnez les profondeurs, conseillait Rilke :
%— 319 —
l'ironie n’y descend pas. » Ce ne serait pas vrai de l’humour, et suffit à les distinguer.

\section{Irrationnel}
%IRRATIONNEL
Ce qui n’est pas conforme à la raison théorique : ce qu’elle
ne peut, en droit, ni connaître ni comprendre. Si la raison
a toujours raison, comme le veut le rationalisme et comme je le crois, l’irrationnel
n’est qu’une illusion ou un passage à la limite : on ne juge irrationnel
(c’est-à-dire incompréhensible en droit) que ce qu’on n'arrive pas, en fait, à
comprendre. Ainsi l’irrationnel n’existe pas. Cela suffit à le distinguer du déraisonnable,
qui n’existe que trop.

\section{Irréversibilité}
%IRRÉVERSIBILITÉ
Le fait de ne pouvoir revenir en arrière : ainsi le temps
est irréversible, et lui seul peut-être. Mais tout ce qui se
passe dans le temps — c’est-à-dire tout — l’est aussi par là même. Projeter un film
à l’envers, quand on le peut sans absurdité (parce que les phénomènes filmés
sont réversibles), n’est possible que parce que le temps, lui, continue de se
dérouler à l'endroit : la projection inversée reste pour cela tout aussi irréversible
que l’autre. C’est en quoi toute action, même physiquement réversible (encore
qu'elle ne le soit jamais absolument : il y faut une nouvelle énergie), est ontologiquement
irréversible. Ce qu’on a fait, on peut ordinairement le défaire,
mais point faire que cela n’ait pas eu lieu. Pénélope, amoureuse irréversible.

\section{Isolement}
%ISOLEMENT
L'absence de relations avec autrui. À ne pas confondre avec la
solitude, qui est une relation — à la fois singulière et inaliénable
— à soi et à tout.

\section{Isonomie}
%ISONOMIE
Une loi égale, ou légalité devant la loi. Le mot, en Grec, vient
de la terminologie juridique ou politique. Mais il est susceptible
d’un usage plus large, par exemple médical (pour dire l’équilibre des
humeurs) ou cosmologique (pour dire l’égale distribution des êtres dans l’univers).
Dans l’atomisme antique, par exemple, la notion semble avoir joué un
rôle important pour penser la répartition des atomes dans le vide (donc la pluralité
des mondes) ainsi que l’équilibre entre les forces destructrices et les forces
de conservation. C'était parier non sur la justice de la nature, mais sur la neutralité
du hasard.

% 321
\section{Jalousie}
%JALOUSIE
Parfois un synonyme d’{\it envie} ; plus souvent une de ses formes ou
de ses variantes. L’envieux voudrait posséder ce qu’il n’a pas et
qu'un autre possède ; le jaloux veut posséder seul ce qu’il croit être à lui. L'un
souffre du manque ; l’autre, du partage. Se dit surtout en matière amoureuse
ou sexuelle. C’est que toute possession vraie y est impossible ou illusoire.
Cela rend la jalousie plus cruelle. Il arrive que l'envie s’apaise, soit parce
qu'on possède enfin ce qu’on désirait, soit parce que celui qu’on enviait ne le
possède plus. La jalousie, non : tant que l’amour demeure, elle s’entretient
d'elle-même, par le soupçon ou l’interprétation interminable des signes.
L'envie est un rapport imaginaire au réel (« Qu'est-ce que je serais heureux
si... »). La jalousie serait plutôt un rapport réel à l'imaginaire (« Qu’est-ce
que je suis malheureux de. »). L’envie a davantage à voir avec l'espérance.
La jalousie, avec la crainte. C’est pourquoi elles vont ensemble, sans tout à
fait se confondre.

\section{Je}
%JE
Le sujet à la première personne et en tant que tel. Se distingue par là du
{\it moi}, qui serait le sujet en tant qu’objet (y compris pour le {\it je} : quand je
{\it me} connais, ou crois me connaître). Le {\it je} est par nature insaisissable (on ne
peut saisir, et encore, que le {\it me}), rarement saisissant. Le monde ou la vérité
sont plus intéressants ; le {\it tu} ou le {\it nous}, plus aimables. À quoi l’on objectera que
c'est toujours un {\it je} qui connaît ou aime... Sans doute. C’est ce qui le sauve en
l’abolissant — quand il n’y a plus que la vérité ou l'amour. « Le péché en moi dit
“je” », écrit Simone Weil dans {\it La pesanteur et la grâce}. Puis elle ajoute, très
proche ici du Vedânta : « Je suis tout. Mais ce “je” là est Dieu. Et ce n’est pas
un je. »

%— 321 —
\section{Jeu}
%JEU
Une activité sans autre but qu’elle-même ou que le plaisir qu'on y
trouve, sans autre contrainte que ses propres règles, enfin sans effet irréversible
(ce qu’une partie a fait, une autre partie peut toujours l’ignorer ou le
refaire). C’est pourquoi la vie, même visée pour elle-même ou pour le plaisir,
n’est pas un jeu : parce que les contraintes y sont innombrables, qui sont celles
du réel, parce qu’on n’y a pas le choix des règles ni du jeu, parce qu’on y vit et
qu’on y meurt {\it pour de vrai}, sans pouvoir jamais recommencer à neuf ni jouer à
autre chose. La vie, pourtant, vaut mieux que tous les jeux. Les enfants le savent
bien, qui jouent à être grands. Et les joueurs, lorsqu'ils jouent de l'argent. C'est
que le jeu ne leur suffit pas. Ils ont besoin d’enjeu, de sérieux, de gravité. Ils ont
besoin de la morsure du réel — non du jeu, mais du tragique. « Le contraire du
jeu n’est pas le sérieux, disait Freud, mais la réalité. » Cela n'empêche pas le jeu
d’être réel aussi, mais interdit de s’en contenter.

\section{Jeune}
%JEUNE
Celui qui n’a pas encore commencé à décliner, qui a encore des progrès
à faire, qui les fera sans doute... Notion par nature relative. Un
sportif de trente-cinq ans est un vieux sportif ; un philosophe de trente-cinq
ans, un jeune philosophe. Être jeune, c’est avoir, dans tel ou tel domaine, plus
d'avenir, du moins en principe, que de passé. Quant au présent, tous en ont
autant. C’est ce qui rend la jeunesse impatiente, et les vieillards nostalgiques. Le
présent ne leur suffit pas.

On pense à la formule fameuse de Nizan, dans {\it Aden Arabie} : « J'avais vingt
ans. Je ne laisserai personne dire que c’est le plus bel âge de la vie. » C’est qu'il
n’y a pas de plus bel âge. Les uns auront préféré l'enfance, d’autres l’adolescence,
d’autres la quarantaine. Mais enfin je n’en connais pas qui préfèrent la
vieillesse. La jeunesse, même difficile, est plus désirable. Le corps ne s’y trompe
pas. Non pourtant que les vieux soient à plaindre. Ils ont été jeunes aussi. Les
jeunes ne sont pas sûrs d’être vieux un jour, ni même de pouvoir disposer d’une
jeunesse entière. Pour moi, je ne me reconnais jamais mieux que dans mes
dix-sept ans. Mais si j'étais mort à vingt, je n’en aurais pas moins raté l’essentiel.
Mieux vaut vivre vieux que mourir jeune. Cette évidence met la jeunesse à
sa place, qui n’est la première que dans le temps. Ce n’est pas une valeur ; c'est
une étape.

\section{Jeunisme}
%JEUNISME
C’est faire de la jeunesse une valeur, voire l’ériger en valeur
suprême, y compris dans des domaines (politiques, artistiques,
intellectuels, culturels.) où elle n’a aucune légitimité particulière. Le premier
imbécile venu, à vingt ans, est ordinairement plus beau, plus fort, plus souple,
%— 322 —
plus désirable que ses parents ou ses grands-parents. Il n’en est pas moins bête
pour autant. Quant aux avantages de la jeunesse, ils sont surtout indirects. Ce
n'est pas l’âge en lui-même qui vaut ; c’est la beauté, la force, la souplesse, la
santé, l’éroticité... Qu'on s'efforce de les conserver le plus longtemps possible,
je n’ai rien contre. Ce n’est pas refuser de vieillir ; c’est vouloir vieillir le mieux
qu'on peut. Tant mieux si nos médecins peuvent nous y aider. Mais ne leur
demandons pas davantage. Si la jeunesse était le souverain bien, nos jeunes
n'auraient plus rien à désirer. Idée de vieux.

\section{Joie}
%JOIE
L'un des affects fondamentaux, en tant que tel impossible à définir
absolument. Qui ne l’aurait jamais ressentie, comment lui faire comprendre
ce qu’elle est ? Mais nous en avons tous une expérience. La joie naît
lorsqu’un désir intense est satisfait (la joie du bachelier, le jour des résultats),
lorsqu'un malheur est évité (la joie du miraculé ou du convalescent), lorsqu'un
bonheur arrive, ou semble arriver (la joie de l’amoureux, lorsqu'il se sait
aimé)... Jouissance, mais spirituelle ou spiritualisée (ré-jouissance). Elle est
l'élément du bonheur, à la fois minimum (dans le temps) et maximum (en
intensité). Élément singulier pourtant : on n’imagine pas le bonheur sans elle
(du moins sans sa possibilité), mais elle peut exister sans lui. C’est comme une
satisfaction momentanée de tout l’être — un acquiescement à soi et au monde.
Épicure dirait : plaisir en mouvement de l’âme ; et Spinoza : accroissement de
puissance ({\it passage} à une perfection supérieure). De fait, il y a dans la joie une
mobilité spécifique, qui est sa force en même temps que sa faiblesse. Quelque
chose en elle — ou en nous — lui interdit de durer. De à le désir de béatitude
(désir d’éternité) et le rêve du bonheur (dont la joie est peut-être le seul
contenu psychologique observable). La joie est ainsi notre guide et notre règle :
c’est l'étoile du berger de l’âme. Elle est l’origine, pour nous, de l’idée de salut.
« La joie, écrit Spinoza, est le passage de l’homme d’une moindre à une plus
grande perfection » ({\it Éthique}, III, déf. 2 des affects ; voir aussi le scolie de la
prop. 11). Comme la perfection elle-même n’est pas autre chose que la réalité,
cela signifie que la joie est passage à une réalité supérieure, ou plutôt à un degré
supérieur de réalité. Se réjouir, c’est exister davantage : la joie est le sentiment
qui accompagne en nous une expansion, ou une intensification, de notre puissance
d'exister et d’agir. C’est le plaisir — en mouvement et en acte — d’exister
plus et mieux.

\section{Judaïsme}
%JUDAÏSME
C'était au début des années 80. Je rencontre un ancien
condisciple de khâgne et de la rue d’Ulm, que j'avais perdu de
%— 323 —
vue depuis nos années d’études. Nous prenons un verre, nous faisons en vitesse
le bilan de nos vies. Le métier, le mariage, les enfants, les livres projetés ou en
cours. Puis mon ami ajoute :

— «Il y a autre chose. Maintenant, je retourne à la synagogue.

— Tu étais juif ?

— Je le suis toujours ! Tu ne le savais pas ?

— Comment l’aurais-je su ? Tu n’en parlais jamais.

— Avec le nom que je porte !

— Tu sais, quand on n’est ni juif ni antisémite, un nom, sauf à s'appeler
Lévy ou Cohen, cela ne dit pas grand-chose. J'ai gardé de toi le souvenir d’un
kantien athée. Ce n’est pas une appartenance ethnique ou religieuse ! »

De fait, cet ami faisait partie de cette génération de jeunes Juifs si parfaitement
intégrés que leur judéité, pour qui en était informé, semblait comme
irréelle ou purement réactive. Ils donnaient raison à Sartre : ils ne se sentaient
juifs que pour autant qu’il y avait des antisémites. Beaucoup d’entre eux, plus
tard, feront ce chemin d’une réappropriation spirituelle, qui donnera un sens
positif — celui d’une appartenance, celui d’une fidélité — au fait, d’abord contingent,
d’être juif. L’ami dont je parle fut le premier pour moi d’une longue série,
qui me donnera beaucoup à réfléchir. Peut-être avions-nous tort de dénigrer
systématiquement le passé, la tradition, la transmission ? Mais je n’en étais pas
encore là. En l'occurrence, c’est surtout la question religieuse qui me turlupinait.
Je lui demande :

— « Mais alors, maintenant. tu crois en Dieu ?

— Tu sais, me répond-il en souriant, pour un Juif, l'existence de Dieu, ce
n’est pas vraiment la question importante ! »

Pour quelqu'un qui a été élevé dans le catholicisme, comme c’est mon
cas, la réponse était étonnante : croire ou non en Dieu, c'était la seule chose,
s'agissant de religion, qui me paraissait compter vraiment ! Naïveté de goy.
Ce que je lisais, dans le sourire de mon ami, c’était tout autre chose : qu’il est
vain de centrer une existence sur ce qu’on ignore, que la question de l’appartenance
— à une communauté, à une tradition, à une histoire — est plus
importante que celle de la croyance, enfin que l'étude, l’observance et la
mémoire — ce que j’appellerai plus tard la fidélité — importent davantage que
la foi.

Le judaïsme est religion du Livre. Je sais bien qu’on peut le dire aussi du
christianisme et de l’islam. Mais pas, me semble-t-il, avec la même pertinence.
« Le judaïsme, ajoute mon ami, c’est la seule religion où le premier devoir des
parents est d’apprendre à lire à leurs enfants... » C’est que la Bible est là, qui
les attend, qui les définit. Pour un chrétien, sans doute aussi pour un
musulman, c’est Dieu d’abord qui compte et qui sauve : le Livre n’est que le
%— 324 —
chemin qui en vient et y mène, que sa trace, que sa parole, qui ne vaut absolument
que par Celui qui l’énonce ou l’inspire. Pour un Juif, me semble-t-il, c’est
différent. Le Livre vaut pour lui-même, par lui-même, et continuerait de valoir
si Dieu n'existait pas ou était autre. D’ailleurs, qu’est-il ? Nul prophète juif n’a
prétendu le savoir, mais seulement ce qu’il voulait ou ordonnait. Le judaïsme
est religion du Livre, et ce Livre est une Loi (une {\it Thora}) bien davantage qu’un
{\it Credo} : c’est ce qu’il faut faire qu’il énonce, bien plus que ce qu’il faudrait
croire ou penser ! D'ailleurs on peut croire ce qu’on veut, penser ce qu’on veut,
c'est pourquoi l'esprit est libre. Mais point faire ce qu’on veut, puisque nous
sommes en charge, moralement, les uns des autres.

Si le Christ n’est pas Dieu, s’il n’est pas ressuscité, que reste-t-il du
christianisme ? Rien de spécifique, rien de proprement religieux, et pourtant, à
mes yeux d’athée, l’essentiel : une certaine fidélité, une certaine morale — une
certaine façon, parmi cent autres possibles, d’être juif... Il m’est arrivé, quand
on m'interrogeait sur ma religion, de me définir comme {\it goy assimilé}. C’est que
je suis judéo-chrétien, que je le veuille ou pas, et d’autant plus {\it assimilé}, en effet,
que j'ai perdu la foi. Il ne me reste que la fidélité pour échapper au nihilisme
ou à la barbarie.

Il y a quelques années, lors d’une conférence à Reims ou à Strasbourg, je ne
sais plus, j’eus l’occasion de m'expliquer sur ces deux notions de {\it foi} et de {\it fidélité}.
Après la conférence, qui se passait dans une faculté ou une grande école, se
tient une espèce de cocktail. On me présente un certain nombre de collègues et
de personnalités. Parmi celles-ci, un rabbin.

«— Pendant votre conférence, me dit-il, il s’est passé quelque chose d’amusant...

— Quoi donc ?

— Vous étiez en train de parler de fidélité. Je dis à l’oreille de l'ami qui
m'accompagnait : “Cela me fait penser à une histoire juive. Je te la raconterai
tout à l’heure...”

— Et alors ?

— Alors, c’est l’histoire que vous avez racontée vous-même, quelques
secondes plus tard ! »

Voici donc cette histoire, qui me paraît résumer l’esprit du judaïsme, ou du
moins la part de lui qui me touche le plus, et qu’il me plaît de voir ainsi, en
quelque sorte, authentifiée.

C’est l’histoire de deux rabbins, qui dînent ensemble. Ils discutent de l’existence
de Dieu, et concluent d’un commun accord que Dieu, finalement,
n'existe pas. Puis ils vont se coucher. Le jour se lève. L'un de nos deux rabbins
se réveille, cherche son ami, ne le trouve pas dans la maison, va le chercher
%— 325 —
dehors, et le trouve en effet dans le jardin, en train de faire sa prière rituelle du
matin. Il va le voir, quelque peu interloqué :

— « Qu'est-ce que tu fais ?

— Tu le vois bien : je fais ma prière rituelle du matin.

— Mais pourquoi ? Nous en avons discuté toute une partie de la nuit, nous
avons conclu que Dieu n'existait pas, et toi, maintenant, tu fais ta prière rituelle
du matin?!»

L'autre lui répond simplement :

— « Qu'est-ce que Dieu vient faire là-dedans ? »

Humour juif : sagesse juive. Qu’a-t-on besoin de croire en Dieu pour faire
ce que l’on doit ? Qu’a-t-on besoin d’avoir la foi pour rester fidèle ?

Dostoïevski, à côté, est un petit enfant. Que Dieu existe ou pas, tout n’est
pas permis : puisque la Loi demeure, aussi longtemps que les hommes s’en souviennent,
l’étudient et la transmettent.

L'esprit du judaïsme, c’est l’esprit tout court, qui est humour, connaissance
et fidélité.

Comment les barbares ne seraient-ils pas antisémites ?

\section{Jugement}
%JUGEMENT
Une pensée qui vaut, ou qui prétend valoir. C’est en quoi tout
jugement est un jugement de valeur, quand bien même la
valeur en question n’est pas autre chose que la vérité (et quand bien même la
vérité, à la considérer en elle-même, n’est pas une valeur). Un jugement de réalité,
comme « La Terre est ronde », peut toujours être formulé, sans en changer
le contenu, sous la forme « Il est vrai que la Terre est ronde », où l’idée de vérité
fonctionne, pour nous, de façon normative. De même qu’un jugement normatif
peut prendre la forme d’un jugement de réalité : {\it « Cet homme est un
salaud. »} Ainsi la frontière entre les jugements normatifs et les jugements descriptifs
reste floue. Cela ne signifie pas que les normes soient réelles, ni que la
réalité soit la norme. Mais que tout jugement est humain, et subjectif par là.
Seule la vérité, qui ne juge pas, est objective. Mais nul ne la connaît que par ses
jugements, qui restent subjectifs. C’est pourquoi Dieu ne juge pas, dirait
Spinoza : parce qu’il est la vérité même. Et c’est pourquoi nous jugeons : parce
que nous ne sommes pas Dieu.

Un jugement, dans sa forme élémentaire, unit classiquement un sujet à un
prédicat, ou un prédicat à un sujet, par la médiation d’une copule : {\it « A est B »}
(dans les jugements affirmatifs) ou {\it « A n'est pas B »} (dans les jugements négatifs).
Par exemple : {\it « Socrate est mortel »} où {\it « Socrate n'est pas immortel »} sont
des jugements. On parle de jugement analytique, spécialement depuis Kant,
lorsque «le prédicat B appartient au sujet À comme quelque chose qui est
%— 326 —
contenu implicitement dans ce concept»; et de {\it jugement synthétique},
« lorsque B est entièrement en dehors du concept A, quoiqu'il soit, à la vérité,
en connexion avec lui » ({\it C. R. Pure}, Introd., IV). Par exemple, précise Kant,
{\it « Tous les corps sont étendus »} est un jugement analytique (il suffit de décomposer,
ou même de comprendre, l’idée de corps pour y trouver l'étendue) ;
alors que « Tous les corps sont pesants » est un jugement synthétique : l’idée de
poids n’est pas contenue dans celle de corps (l’idée d’un corps sans poids n’est
pas contradictoire), elle n’est unie à lui que de l’extérieur, en fonction d’autre
chose (en l'occurrence de l’expérience).

Il en suit, selon Kant :

1 — Que les jugements analytiques n’étendent pas du tout nos connaissances
(ils ne nous apprennent rien de nouveau), mais les développent, les précisent
ou les explicitent ;

2 — Que les jugements empiriques sont tous synthétiques ;

3 — Que les jugements synthétiques 4 priori, comme on voit dans les
sciences en général {\it (« Tout ce qui arrive a une cause »)} et dans les mathématiques
en particulier {\it (« 7 + 5 = 12 »)}, sont profondément mystérieux : sur quoi
peut-on bien s’appuyer pour sortir ainsi du concept et lui rattacher, de façon à
la fois universelle et nécessaire, un prédicat qu’il ne contient pas ? Tel est le problème
de la {\it Critique de la raison pure} : « Comment des jugements synthétiques
{\it a priori} sont-ils possibles ? » Kant répondra qu’un tel jugement n’est possible
que pour autant que nous nous appuyons, pour l’énoncer, sur les formes pures
de l'intuition (l’espace et le temps) ou de la pensée (les catégories de l’entendement).
Il ne vaut donc que pour nous, point en soi, et dans les limites d’une
expérience possible, point dans l’absolu. C’est résoudre le problème des jugements
synthétiques {\it a priori}, mais par l'{\it a priori} lui-même, considéré comme
antérieur au jugement et le rendant possible, autrement dit par le transcendantal.
Ce n’est pas l'expérience qui permet la connaissance ; ce sont au
contraire les formes {\it a priori} du sujet qui rendent l’expérience possible et la
connaissance nécessaire. Telle est la révolution copernicienne : c’est faire
tourner l’objet autour du jugement (ou du sujet qui juge), non le jugement
autour de l’objet.

Une autre solution, à laquelle je me range plus volontiers, serait de dire
qu'il n’y a pas de jugement à priori : c’est sortir de Kant pour revenir à Hume,
à l’empirisme et à l’histoire des sciences. Un pas en arrière, deux pas en avant.
C’est où il faut choisir entre le sujet transcendantal et le processus immanental
(voir ce mot), entre l’anhistoricité de la conscience et l’historicité des connaissances.
Ainsi, Cavaillès : «Je crois malhonnête tout recours à un {\it a priori}
quelconque », écrivait-il en 1938 à son ami Paul Labérenne, avant d’en
conclure qu’il fallait donc opérer une « rupture complète avec l’idéalisme,
%— 327 —
même brunschvicgien », considérer la logique « comme une première technique
naturelle », bref affirmer la « subordination complète » de la connaissance,
y compris des mathématiques, « à une expérience, qui n’est sans doute
pas l'expérience historique, puisqu’elle permet d’obtenir des résultats dont la
validité est hors du temps, mais qui {\it naît} de l’expérience historique ». Ainsi il
n’y a pas d’{\it a priori}, et tout jugement, même éternellement vrai, n’est rendu
possible que par une histoire qui le précède et le contient. Nous n'avons accès
à l’éternel que dans le temps ; tel est le jugement, quand il est vrai.

\section{Juger}
%JUGER
C’est relier un fait à une valeur, ou une idée à une autre. C’est pourquoi
« penser, c’est juger », comme disait Kant: parce qu'on ne
commence vraiment à penser qu’en reliant deux idées (au moins deux !) différentes.
Cela suppose l’unité de l'esprit ou du {\it je pense} (« l'unité originairement
synthétique de l’aperception »), comme pouvoir de liaison. Reste à savoir si
cette unité elle-même est première ou seconde, autrement dit si elle est donnée
{\it (a priori)} ou construite (dans le cerveau, dans l'expérience). Est-ce l'unité du
sujet qui rend le jugement possible, ou bien l’unité du jugement, même progressivement
constituée, qui rend le sujet nécessaire ? Est-ce parce que je suis
un sujet que je juge, ou est-ce à force de juger que je deviens sujet ? Est-ce le
transcendantal qui permet l’expérience, ou l'expérience qui constitue
l’immanental ? On remarquera que juger, dans les deux cas, reste le fait d’un
sujet : si le réel se jugeait soi, il serait Dieu ; si Dieu ne juge pas (Spinoza), il est
le réel même.

\section{Juste}
%JUSTE
Celui qui respecte la justice — la légalité et l'égalité, le droit {\it (jus)} et
les droits (des individus) — et qui se bat pour elle, autrement dit pour
que ces deux justices aillent ensemble : pour que la loi soit la même pour tous
(pour que la légalité respecte l'égalité), pour qu’elle soit appliquée avec équité
(voir ce mot), enfin pour que le droit (de la Cité) protège les droits (des individus).
C’est le plus haut devoir ; non, pourtant, la plus haute vertu. « Amis,
disait Aristote, on n’a que faire de la justice ; justes, on a encore besoin de
l'amitié » ({\it Éthique à Nicomaque}, VIII, 1). Ainsi l'amour, qui n’est pas un
devoir, vaut mieux que la justice, qui en est un. Les justes ne l’ignorent pas ;
mais ils n’attendent pas d’aimer pour être justes.

\section{Justice}
%JUSTICE
L'une des quatre vertus cardinales : celle qui respecte l'égalité et la
légalité, les droits (des individus) et le droit (de la Cité). Cela suppose
%— 328 —
que la loi soit la même pour tous, que le droit respecte les droits, enfin que
la justice (au sens juridique) soit juste (au sens moral). Comment le garantir ?
On ne le peut absolument ; c’est pourquoi la politique, même lorsqu’elle y
tend, reste conflictuelle et faillible. C’est pourtant la seule voie. Aucun pouvoir
n’est la justice ; mais il n’y a pas de justice sans pouvoir.

« Sans doute l'égalité des biens est juste, écrit Pascal, mais... » Mais quoi ?
Mais le droit en a décidé autrement, qui protège la propriété privée et par à
l’inégalité des biens. Faut-il le regretter ? Ce n’est pas sûr (il se peut qu’une
société inégalitaire soit plus prospère, même pour les plus pauvres, qu’une
société égalitaire). Ce n’est pas impossible (notamment si on met la justice plus
haut que la prospérité). Qui en décidera ? Le droit positif ({\it jus}, en latin, d’où
vient justice), ou plutôt ceux qui le font. Les plus justes ? Non pas. Les plus
forts — donc presque toujours, dans nos sociétés démocratiques, les plus nombreux.
La propriété privée fait-elle partie du droit naturel ? Fait-elle partie des
droits de l’homme ? Ce sont deux questions différentes, mais toutes deux insolubles
par le droit seul. Questions philosophiques plutôt que juridiques, et politiques
plutôt que morales. « Ne pouvant faire qu’il soit force d’obéir à la justice,
continue Pascal, on a fait qu’il soit juste d’obéir à la force. Ne pouvant fortifier
la justice, on a justifié la force, afin que le juste et le fort fussent ensemble, et
que la paix fût, qui est le souverain bien » ({\it Pensées}, 81-299 ; voir aussi le
fr. 103-298).

C’est où l’on rencontre la fiction, mais éclairante, du contrat social. « La
justice, écrivait Épicure, n’est pas un quelque chose en soi ; elle est seulement,
dans les rassemblements des hommes entre eux, quels qu’en soient le volume et
le lieu, un certain contrat en vue de ne pas faire de tort et de ne pas en subir »
({\it Maximes capitales}, 33 ; voir aussi les maximes 31 à 38). Peu importe que ce
contrat ait existé en fait ; il suffit à la justice qu’il puisse exister en droit : il est
« la règle, souligne Kant, et non pas l’origine de la constitution de l’État, non
le principe de sa fondation mais celui de son administration » ({\it Réfl.}, Ak. XVIII,
n° 7734 ; voir aussi {\it Théorie et pratique}, II, corollaire). Une décision est juste
quand elle pourrait être approuvée, en droit, par tous (par tout un peuple, dit
Kant) et par chacun (s’il fait abstraction de ses intérêts égoïstes ou contingents :
c’est ce que Rawls appelle la « position originelle » ou le « voile d’ignorance »).
Cela vaut pour l’État, mais tout autant pour les individus. « Le moi est injuste
en soi, écrivait Pascal, en ce qu’il se fait le centre de tout » ({\it Pensées}, 597-455).
Contre quoi toute justice est universelle, au moins dans son principe, et n’agit,
en chacun, que contre l’égoïsme ou par décentrement. Cela donne à peu près
la règle, telle qu’Alain la formule, et qui ne vaut pour tous que parce qu’elle
vaut d’abord pour chacun : « Dans tout contrat et dans tout échange, mets-toi
à la place de l’autre, mais avec tout ce que tu sais, et, te supposant aussi libre
%— 329 —
des nécessités qu’un homme peut l'être, vois si, à sa place, tu approuverais cet
échange ou ce contrat » ({\it 81 chapitres...}, VI, 4, « De la justice »). Cela vaut pour
les individus, mais donc aussi pour les citoyens. Pour la morale, mais donc aussi
— si les citoyens font leur devoir — pour la politique. « Est juste, écrivait Kant,
toute action ou toute maxime qui permet à la libre volonté de chacun de
coexister avec la liberté de tout autre suivant une loi universelle » ({\it Doctrine du
droit}, Introd., \S C). Cette coexistence des libertés sous une même loi suppose
leur égalité, au moins en droit, ou plutôt elle la réalise (malgré les inégalités de
fait, qui sont innombrables), et elle seule : c’est la justice même, toujours à faire
ou à refaire, toujours à défendre ou à conquérir.

\section{{\it Kairos}}
%{\it KAIROS}
C'est un mot grec, qu’on peut sans trop de cuistrerie utiliser en
français, faute d’équivalent tout à fait satisfaisant. Le {\it kairos}, c’est
l’occasion propice, le moment opportun ou favorable, autrement dit {\it le bien},
comme on voit chez Aristote, mais {\it dans le temps} ({\it Éthique à Nicomaque}, I, 4).
Le thème, chez Aristote, est antiplatonicien. Le bien n’est pas une essence éternelle
ou absolue. Il se dit en plusieurs sens, tout comme l’être, et nul ne peut le
faire, dans le monde sublunaire, sans tenir compte du devenir : faire le bien,
c’est le faire {\it quand il faut} (le faire trop tôt ou trop tard ce n’est pas le faire, ou
c’est le faire moins bien qu’il ne faudrait ou qu’il n’aurait fallu : ainsi en médecine,
en politique ou en morale). Le {\it kairos} est comme la juste mesure, si l’on
veut, mais appliquée au temps: c’est « la juste mesure de l’irréversible »
(Francis Wolff).

\section{Kinesthésie}
%KINESTHÉSIE
Sensation par un individu des mouvements de son propre
corps (de {\it kinein}, se mouvoir, et {\it aisthèsis}, la sensation). Sans
doute l’une des origines, avec la sensation d’effort, de la conscience de soi. Je
bouge, donc je suis.

% 330
\section{Lâcheté}
%LÂCHETÉ
Le manque de courage : non le fait d’avoir peur, mais l’incapacité
à surmonter la peur qu’on a, ou qu’on est, et même à lui
résister. C’est une complaisance pour sa propre frayeur, comme une soumission
à soi : son geste est de fuir ou de fermer les yeux.

Alain, dans un Propos du 27 mai 1909, notait que « le mot “lâche” est la plus
grave des injures ». Cela, qui semble à peu près vrai, au moins entre hommes, n’en
est pas moins curieux, puisqu'on peut être lâche sans être méchant ou cruel,
méchant et cruel sans être lâche, et que ces deux vices sont pires, assurément, que la
simple lâcheté. Combien de salauds sont capables de courage ? Combien de braves
gens, de lâcheté ? Et qui ne voit que ces millions de Français, qui se contentèrent
lâchement de ne rien faire, pendant l'Occupation, furent pourtant moins coupables
que les plus courageux des nazis ou des collabos ? Mais alors, pourquoi cette injure
fait-elle si mal ? Parce qu'il n’y a pas de vertu sans courage, répond Alain : ainsi le
mot lâche est l'injure la plus grave parce qu’elle est la plus globale, qui ne laisse rien
à estimer. Qu'un brave homme puisse être lâche, à l’occasion, c’est entendu. Mais
s’il l'était toujours, il ne serait plus « brave », en aucun sens du mot : toujours la
peur l'empêcherait d’être généreux ou juste, et même d’être sincère ou aimant ; car
enfin rien de cela ne va sans risque — rien de cela ne va sans courage. Qui s’abandonne
à sa peur, il est même incapable d’être prudent : fuir toujours, fermer les
yeux toujours, c'est manquer de prudence presque autant que de courage.

Cela ne prouve pas que le courage soit la plus grande vertu, mais seulement
qu’il est la plus nécessaire.

\section{Laïc}
%LAÏC
Qui fait partie du peuple {\it (laos)} et non du clergé. Par extension, le mot
désigne tout ce qui est indépendant de la religion, ou doit l'être.

%— 331 —
\section{Laïcité}
%LAÏCITÉ
Ce n’est pas l’athéisme. Ce n’est pas l’irréligion. Encore moins
une religion de plus. La laïcité ne porte pas sur Dieu, mais sur la
société, Ce n’est pas une conception du monde ; c’est une organisation de la
Cité. Ce n’est pas une croyance ; c’est un principe, ou plusieurs : la neutralité
de l’État vis-à-vis de toute religion comme de toute métaphysique, son indépendance
par rapport aux Églises comme l'indépendance des Églises par rapport
à lui, la liberté de conscience et de culte, d'examen et de critique, l’absence
de toute religion officielle, de toute philosophie officielle, le droit en conséquence,
pour chaque individu, de pratiquer la religion de son choix ou de n’en
pratiquer aucune, enfin, mais ce n’est pas le moins important, l’aspect non
confessionnel et non clérical — mais point non plus anticlérical — de l’école
publique. L'essentiel tient en trois mots : {\it neutralité} (de l’État et de l’école),
{\it indépendance} (de l'État vis-à-vis des Églises, et réciproquement), {\it liberté} (de
conscience et de culte). C’est en ce sens que Mgr Lustiger peut se dire laïque,
et je lui en donne bien volontiers acte. Il ne veut pas que l'État régente l’Église,
ni que l’Église régente l'État. Il a évidemment raison, même de son propre
point de vue : il rend « à César ce qui est à César, et à Dieu ce qui est à Dieu »
(Mt 22, 21). Les athées auraient tort de faire la fine bouche. Que l'Église catholique
ait mis tant de temps pour accepter la laïcité, cela ne rend sa conversion,
si l’on peut dire, que plus spectaculaire. Mais cette victoire, pour les laïcs, n’est
pas pour autant une défaite de l’Église : c’est la victoire commune des esprits
libres et tolérants. La laïcité nous permet de vivre ensemble, malgré nos différences
d'opinions et de croyances. C’est pourquoi elle est bonne. C’est pourquoi
elle est nécessaire. Ce n’est pas le contraire de la religion. C’est le
contraire, indissociablement, du cléricalisme (qui voudrait soumettre l’État à
l'Église) et du totalitarisme (qui voudrait soumettre les Églises à l'État).

On comprend qu'Israël, l'Iran ou le Vatican ne sont pas des États laïques,
puisqu'ils se réclament d’une religion officielle ou privilégiée. Mais PAlbanie
d’Enver Hoxha ne l'était pas davantage, qui professait un athéisme d’État. Cela
dit assez ce qu’est vraiment la laïcité : non une idéologie d’État, mais le refus,
par l’État, de se soumettre à quelque idéologie que ce soit.

Et les droits de l’homme ? demandera-t-on. Et la morale ? Ce n’est pas à
eux que l’État se soumet, mais à ses propres lois et à sa propre constitution — ou
aux droits de l’homme pour autant seulement que la constitution les énonce ou
les garantit. Pourquoi, dans nos démocraties, le fait-elle ? Parce que le peuple
souverain en a décidé ainsi, et ce n’est pas moi qui le lui reprocherai. C'est
mettre l’État au service des humains, comme il doit être, plutôt que les
humains au sien. Mais la même raison interdit d’ériger les droits de l’homme
en religion d’État. Distinction des ordres : l’État ne doit régner ni sur les esprits
ni sur les cœurs. Il ne dit ni le vrai ni le bien, mais seulement le permis et le
%— 332 —
défendu. Il n’a pas de religion. Il n’a pas de morale. Il n’a pas de doctrine. Aux
citoyens d’en avoir une, s'ils le veulent. Non pourtant que l’État doive tout
tolérer, ni qu’il le puisse. Mais il n’interdit que des actions, point des pensées,
et pour autant seulement qu’elles enfreignent la loi. Dans un État vraiment
laïque, il n’y a pas de délit d'opinion. Chacun pense ce qu’il veut, croit ce qu’il
veut. Il doit rendre compte de ses actes, non de ses idées. De ce qu’il fait, non
de ce qu’il croit. Les droits de l’homme, pour un État laïque, ne sont pas une
idéologie, encore moins une religion. Ce n’est pas une croyance, c’est une
volonté. Pas une opinion, une loi. On a le droit d’être contre. Pas de les violer.

\section{Laideur}
%LAIDEUR
Ce n’est pas l’absence de beauté, mais son contraire : non ce qui
ne plaît pas, mais ce qui déplaît ; non ce qui ne séduit pas, mais
ce qui repousse.

« La beauté, écrivait Spinoza, n’est pas tant une qualité de l’objet considéré
qu'un effet se produisant en celui qui le considère » ; si nous avions d’autres
yeux ou un autre cerveau, « les choses qui nous semblent belles nous paraitraient
laides et celles qui nous semblent laides deviendraient belles », de même
que « la plus belle main, vue au microscope, paraîtra horrible » ({\it Lettre 54}, à
Hugo Boxel). Ainsi toute laideur est relative, comme toute beauté. Il n’y a pas
de laideur en soi, ni de laideur objective : être laid, c’est déplaire, disais-je, et
l’on ne peut déplaire qu’à un sujet. Cela ne rend pas la laideur moins injuste,
ou plutôt c’est ce qui rend son injustice plus cruelle : parce qu’elle semble
repousser l’amour, et même la sympathie, et les repousse en effet, au moins un
temps — puisqu'elle n’est pas autre chose que cette {\it répulsion} qu’elle suscite et à
quoi on la reconnaît. On peut, en art, jouer avec elle, jusqu’à la beauté (les
« peintures noires » de Goya, la {\it Raie} de Chardin, les portraits de Bacon). Mais
dans la vie ? Il y faut de l’art aussi, et une part de talent — y compris chez le
spectateur.

\section{Langage}
%LANGAGE
Au sens large : toute communication par signes (on parlera par
exemple du « langage des abeilles »). Au sens strict, ou spécifiquement
humain : la faculté de parler (la parole en puissance), ou la totalité des
langues humaines. On remarquera que le langage ne parle pas, ne pense pas, ne
veut rien dire, et qu’il n’est pas une langue ; c’est pourquoi nous pouvons parler
et penser. Le langage n’est qu’une abstraction : seules les paroles, mais en actes,
sont réelles, qui ne s’actualisent que dans une langue particulière. Ainsi le langage
est à peu près aux langues et aux paroles ce que la vie est aux espèces et aux
individus : leur somme, ou leur reste.

%— 335 —
\section{Langue}
%LANGUE
« La langue, disait De Saussure, c’est le langage moins la parole » :
ce qui reste, quand on se tait. C’est ce qui donne tort aux bavards,
et raison aux linguistes.

Mais qu'est-ce que la parole ? La mise en œuvre, par un individu singulier,
en un moment singulier, d’une langue quelconque. Ainsi la langue est ce dans
quoi nous parlons : c’est un ensemble de signes conventionnels, articulés (et
même doublement articulés : en monèmes et en phonèmes) et soumis à un certain
nombre de structures aussi bien sémantiques que grammaticales.

On remarquera que la pluralité des langues, qui est une donnée de fait,
n'exclut ni l’unité du langage (puisque tout discours dans une langue naturelle
peut être traduit dans une autre) ni celle de la raison. Il me semble même
qu’elle les suppose. S'il n’y avait une raison avant le langage, et une fonction
symbolique avant les langues, aucune parole jamais n’eût été possible. De ce
point de vue, l’aporie bien connue de l’origine des langues (il faut une langue
pour raisonner, et de la raison pour inventer une langue) n’en est pas tout à fait
une : d’abord parce qu’une langue n’est pas {\it inventée} (elle est le résultat d’un
processus historique, non d’un acte individuel), ensuite parce que l'intelligence
et la fonction symbolique existent {\it avant} les langues : cela même qui permet aux
nouveau-nés d'apprendre à parler dut permettre aux humains, au fil des millénaires,
de passer d’une communication seulement sensori-motrice (cris, gestes,
mimiques, comme on voit chez les animaux) à une communication linguistique.

Il faut souligner pour finir l’extrême efficacité — en termes de puissance et
d'économie — de ce que Martinet appelle la double articulation. Toute langue
se divisant en unités minimales de signification (les monèmes), dont chacune
peut à son tour se diviser en unités sonores minimales (les phonèmes), on
aboutit à cette espèce de miracle objectif qu’est la communication humaine :
l’ensemble de nos expériences, de nos idées et de nos sentiments — tous les livres
écrits et possibles, toutes les paroles prononcées et prononçables — peut
s’énoncer au moyen de quelques dizaines de petits cris brefs, ou plutôt de sons
minimaux, de pures différences vocales, toujours les mêmes dans chaque langue
(le français, par exemple, compte une quarantaine de phonèmes), à la fois
dépourvus de signification et les permettant toutes. C’est le plus simple toujours
qui permet le plus complexe : on ne pense que grâce à des atomes qui ne
pensent pas, on ne parle que grâce à des sons qui ne veulent rien dire. C’est où
la linguistique, qui semble si peu matérielle, peut mener au matérialisme.

\section{Lapsus}
%LAPSUS
«Ayant entendu quelqu'un crier naguère que sa maison s'était
envolée sur la poule du voisin, je n’ai pas cru qu’il se trompait,
— 334 —
écrit Spinoza, parce que sa pensée me semblait assez claire » ({\it Éthique}, II, scolie
de la prop. 47). C'était en effet ce que nous appelons un lapsus : moins une
erreur, dans la pensée, qu’un acte manqué dans le discours. C’est dire involontairement
un mot à la place d’un autre {\it (lapsus linguae)}, ou l'écrire {\it (lapsus
calami)}. Freud nous a habitués à en chercher le sens inconscient, et l’on aurait
tort de se l’interdire : c’est souvent amusant, parfois éclairant. Il reste qu’un
acte manqué n’a valeur que d’exception, et que son interprétation elle-même
n'est légitime qu’à la condition de n’en être pas un. L’inconscient parle, sans
doute. Mais la conscience parle aussi, et ce qu’elle a à dire, sauf bêtise ou bavardage,
est plus intéressant. Les textes de Freud donnent davantage à penser que
les lapsus de ses patients.

\section{Larmes}
%LARMES
«Ne me secouez pas; je suis plein de larmes. » Cette formule
d'Henri Calet m'a toujours touché. Les larmes ressemblent à la
mer, dont nous sommes issus. C’est de l’eau salée, qui vient humidifier la
cornée, qui la protège par là. Mais pourquoi coulent-elles quand on a du
chagrin ? On dirait que le cœur déborde. Qu'il retrouve en lui, intarissable,
inconsolable, l'océan primordial du malheur. Ou est-ce celui de vivre, qui vient
tout emporter, tout nettoyer, jusqu’à l’envie de pleurer ?

\section{Lassitude}
%LASSITUDE
C’est une fatigue ressentie, et qui se porte à l’âme. Elle doit
moins à l’intensité de l’effort qu’à sa durée, moins au travail
qu’à l'ennui, moins à l'excès qu’à la répétition. C’est comme la fatigue d’être
fatigué. Son remède, les mots l’indiquent assez, est moins le repos que le délassement.
Mais l’homme absolument las n’en a cure, et préférerait n'être pas né.

\section{Latent}
%LATENT
Ce qui existe sans se manifester, ou, s’il se manifeste, sans être
perçu ou compris. Se dit spécialement, chez Freud, du sens des
rêves, qui n'apparaît jamais immédiatement : le travail du rêve transforme —
notamment par déplacement et condensation — leur contenu latent (inconscient)
en contenu manifeste (le rêve tel que le vit le rêveur, ou tel qu’il s’en souvient au
réveil). L'interprétation essaie, à l'inverse, de remonter de celui-ci à celui-là. Le
plus sage, en dehors de la cure, est de ne se préoccuper ni de l’un ni de l’autre.

\section{Légalité}
%LÉGALITÉ
La conformité factuelle à la loi. À ne pas confondre avec la {\it légitimité},
qui suppose un jugement de valeur, ni avec la {\it moralité},
%— 335 —
qui peut parfois pousser à violer une loi juridique et qui ne saurait se
contenter d’être simplement conforme à la loi morale. Par exemple, explique
Kant, le commerçant qui n’est honnête que pour garder ses clients : il agit
bien {\it conformément au devoir}, mais point {\it par devoir} (il agit conformément au
devoir, mais par intérêt) ; son action, pour {\it légale} qu’elle soit (en l’occurrence
au double sens, juridique et moral, du terme), n’en est pas moins dépourvue
de toute valeur proprement {\it morale} (puisqu’une conduite n’a de valeur morale
qu’à la condition d’être désintéressée). Ainsi la légalité n’est qu’un fait, qui ne
dit rien sur la légitimité d’une action ni, encore moins, sur sa moralité. Les
lois antijuives de Vichy, à les supposer même juridiquement inattaquables,
n’en étaient pas moins illégitimes : il était immoral de les voter, de les appliquer,
et même (sauf pour les victimes, quand elles ne pouvaient faire autrement)
de leur obéir.

\section{Légèreté}
%LÉGÈRETÉ
C’est une manière de ne peser sur rien, qui est le propre de
l'esprit, des dieux et, parfois, des musiciens.
« Tout ce qui est bon est léger, écrivait Nietzsche, tout ce qui est divin
court sur des pieds délicats » ({\it Le cas Wagner}, I). Cela, qui fut écrit à propos de
Bizet et contre Wagner, pourrait presque tenir lieu de définition : toute personne
connaissant la musique de ces deux compositeurs peut comprendre ce
qu'est cette {\it divine légèreté}, comme dit ailleurs Nietzsche, qui résonne ou danse
chez le premier, quand on n’entend chez le second, par différence et sauf exception
(comme dans la {\it Siegfried idylle}), que le poids écrasant du sérieux, de la prétention,
de la sublimité feinte ou recherchée. Toutefois cela ne suffit pas à
prouver que Bizet soit un plus grand musicien que Wagner, ni que la légèreté,
en toute chose, soit bonne. Il ne s’agit que d’une catégorie esthétique, qui ne
saurait valoir universellement. La légèreté est le contraire de la lourdeur, du
sérieux, de la gravité. Elle n’exclut pas le tragique ; elle l’ignore ou le surmonte.
C’est comme une grâce, mais qui serait purement immanente, comme une élégance,
mais qui serait de l’âme, comme une insouciance, mais qui serait sans
petitesse. Elle est bouleversante chez Mozart ; cela ne retire rien à Bach ou
Beerhoven, dont la légèreté n’est pas le fort. Enfin elle est simplement agaçante
chez les esprits frivoles, dans les situations qui requièrent du sérieux ou de la
gravité. « C’est une manière qui ne me va guère, disait Colette, que cette affectation
de légèreté envers l’amour. » Celle-là pourtant savait ce que c’est que la
légèreté des mœurs et de la plume. Mais elle n’oubliait pas pour cela la gravité
d'aimer, ou d’être aimée. La légèreté vaut mieux que lourdeur ; la gravité,
mieux que la frivolité.

%— 336 —
\section{Légitimité}
%LÉGITIMITÉ
La notion se situe à l'interface entre le droit et la morale,
mais aussi entre le droit et la politique. Est légitime ce qui est
dans son {\it bon droit}, ce qui suppose qu’un droit ne l’est pas toujours. La légitimité,
c’est la conformité non seulement à la loi (légalité) mais à la justice ou à
un intérêt supérieur. Voler pour ne pas mourir de faim, comploter contre un
tyran, désobéir à un pouvoir totalitaire, résister, les armes à la main, contre un
occupant, voilà autant de comportements qui, pour être le plus souvent illégaux,
n'en seront pas moins, sauf situation très particulière, parfaitement légitimes.
On demandera qui en décide. Un tribunal le peut, s’il est assez fort et
assez juste, mais il ne le fera le plus souvent qu’après coup — quand il sera trop
tard — et non sans risque, bien souvent, de se tromper. Seuls les vainqueurs,
sauf exception, disposent de tribunaux, et rien n’interdit que la justice, parfois,
soit du côté des vaincus. C’est pourquoi la seule instance de légitimation reste
la conscience individuelle. Cela fait peu ? Sans doute, pour ceux qui en manquent
ou qui voudraient une garantie. Mais ce peu, pourtant, doit suffire —
puisqu'il n’y a rien d’autre. Quel était le représentant légitime de la France,
pendant la dernière guerre mondiale : le maréchal Pétain ou le général de
Gaulle ? La réponse est facile aujourd’hui. Mais c’est alors qu’elle était importante.

\section{Lettrés, gens de lettres}
%LETTRÉS, GENS DE LETTRES
Un lettré, ce n’est pas seulement quelqu'un
qui sait lire et écrire: c’est un
homme de culture, surtout littéraire, et de pensée, surtout philosophique —
l'équivalent, chez Voltaire, de ce que nous appelons aujourd’hui un {\it intellectuel}.
Quant aux gens de lettres, ce sont ceux des lettrés qui en font un
métier, qui écrivent des livres, qui les publient, qui essaient d’en vivre... La
dénomination, dans ce petit milieu, est souvent péjorative. C’est qu’elle
désigne d’abord les collègues, qui sont aussi des rivaux et qu’il est dès lors
naturel de détester. Voltaire, qui parlait d'expérience, notait pourtant que
«le plus grand malheur d’un homme de lettres n’est peut-être pas d’être
l’objet de la jalousie de ses confrères [...] ; c’est d’être jugé par des sots ». Il
faut être détesté ou méprisé, mal compris ou mal aimé, et souvent les deux.
Dur métier. « L'homme de lettres est sans secours, continue Voltaire ; il ressemble
aux poissons volants : s’il s'élève un peu, les oiseaux le dévorent ; s’il
plonge, les poissons le mangent.» Bien peu, pourtant, regrettent l’anonymat.
Ils sont descendus pour leur plaisir dans l'arène, comme dit Voltaire,
ils se sont donnés eux-mêmes aux bêtes. Cela leur donne le droit d’être lus,
pas celui de se plaindre.

%— 337 —
\section{Libéral}
%LIBÉRAL
Qui respecte la liberté, et d’abord celle des autres. Un État libéral
est donc celui qui respecte les libertés individuelles, dût-il pour
cela limiter la sienne propre. On ne le confondra pas avec la démocratie (il peut
exister des démocraties autoritaires et des monarchies libérales), encore moins
avec le laisser-faire : les libertés individuelles n’existent que par la loi, qui suppose
la contrainte.

\section{Libéralisme}
%LIBÉRALISME
La doctrine des libéraux, quand ils en ont une. En français,
se dit surtout de la doctrine économique : celle qui veut que
l'État intervienne le moins possible dans la production et les échanges, si ce
n’est pour garantir, quand nécessaire, le libre jeu du marché. Doctrine respectable,
mais qui ne saurait valoir, me semble-t-il, que pour ce qui relève du
marché, autrement dit que pour les marchandises. Or la justice n’en est pas
une, ni la liberté, ni l'égalité, ni la fraternité... Elles sont donc à la charge, légitimement,
de l’État et des citoyens. C’est ce qui permet de distinguer le {\it libéralisme},
pour lequel la politique garde ses droits et ses ambitions, de l’{\it ultra-libéralisme},
qui voudrait cantonner l’État dans ses fonctions régaliennes
d'administration, de police, de justice et de diplomatie. Ce serait renoncer à
agir sur la société elle-même, voire renoncer à la République. Quand le général
de Gaulle, dans les années 60, disait que « la politique de la France ne se fait pas
à la corbeille », il ne manifestait pas seulement un trait de tempérament personnel,
ni ne voulait cantonner la politique dans son pré carré diplomatique,
administratif et judiciaire. Il rappelait un principe essentiel à toute démocratie
véritable. Si le peuple est souverain, comment le marché le serait-il ?

\section{Libéralité}
%LIBÉRALITÉ
Le juste milieu dans les affaires d’argent : c’est n'être ni avare
ni prodigue (Aristote, {\it Éthique à Nicomaque}, IV, 1119 b-1122 a).
Il est improbable que l’homme libéral devienne riche, constate Aristote, et
encore plus qu'il le reste : car «il n’apprécie pas l'argent en lui-même, mais
comme moyen de donner ». Aussi est-il improbable, pour la même raison, que
les riches sachent faire preuve de libéralité : c’est qu’il n’est pas possible
« d’avoir de l'argent si on ne se donne pas de peine pour l’acquérir » et pour le
garder. C’est dire qu’on a peu de chance de devenir riche sans cupidité, comme
de le rester sans avarice. Le juste milieu est plus exigeant qu’on ne le croit.

\section{Libération}
%LIBÉRATION
Le fait de devenir libre, et le processus qui y mène. C’est la
liberté en acte et en travail. S’oppose par là au libre arbitre,
%— 338 —
qui serait une liberté originelle et absolue (une liberté toujours déjà donnée :
une liberté en puissance et en repos !). « Les hommes se trompent en ce qu’ils
se croient libres », disait Spinoza ({\it Éthique}, II, 35, scolie), et cette illusion est
l’une des principales causes qui les empêchent de le devenir. Le mixte singulier
de conscience et d’inconscience qui les constitue (ils ont conscience de leurs
désirs et de leurs actes, mais non des causes qui les font désirer et vouloir) les
{\it assujettit}, comme dira Althusser, en les faisant {\it sujets}. Leur soi-disant liberté
n'est qu'une causalité qui s’ignore. C’est au contraire parce que le libre arbitre
n'existe pas qu'il faut se libérer toujours, et d’abord de soi. La vérité seule le
permet, où toute subjectivité se brise. Liberté, nécessité comprise (Spinoza,
Hegel, Marx, Freud), ou compréhension, plutôt, de la nécessité, Non que la
compréhension échappe à la nécessité (comment le pourrait-elle, puisqu’elle en
fait partie ?), mais parce que la raison n’y obéit qu’à soi ({\it Éthique}, I, déf. 7).
Seule la connaissance est libre, et libère. C’est où l'éthique, qui tend à la liberté,
se distingue de la morale, qui la suppose.

\section{Liberté}
%LIBERTÉ
Être libre, c’est faire ce que l’on veut. De là trois sens principaux
du mot, selon le {\it faire} dont il s’agit : liberté d’action (si faire c’est
agir), liberté de la volonté (si faire c’est vouloir : on verra que ce sens-là se subdivise
en deux), enfin liberté de l’esprit ou de la raison (quand faire c’est
penser).

La liberté d’action ne pose guère de problèmes théoriques. Ce n’est autre
chose, disait Hobbes, que « l’absence de tous les empêchements qui s’opposent
à quelque mouvement : ainsi l’eau qui est enfermée dans un vase n’est pas libre,
à cause que le vase l'empêche de se répandre, et lorsqu'il se rompt, elle recouvre
sa liberté » ({\it Le Citoyen}, IX, 9). S'agissant des humains, c’est ce qu’on appelle
souvent la {\it liberté au sens politique} : parce que l’État est la principale force qui la
limite et la seule qui puisse la garantir à peu près. Je suis libre d’agir quand rien
ni personne ne m'en empêche : c’est pourquoi je le suis davantage dans une
démocratie libérale que dans un État totalitaire, et c’est pourquoi je ne saurais
jamais l’être absolument (il y a toujours des empêchements, qui tiennent spécialement,
dans un État de droit, à la loi : ma liberté s’arrête où commence celle
des autres). C’est la liberté au sens de Hobbes, de Locke, de Voltaire. Elle existe
évidemment, mais {\it plus ou moins} : liberté toujours relative, toujours limitée,
pour cela toujours à défendre ou à conquérir.

La liberté de la volonté ne semble pas poser beaucoup plus de problèmes.
Puis-je vouloir ce que je veux ? Oui, sans doute, puisque personne, sauf manipulation
mentale ou neurologique, ne peut m'empêcher de vouloir ni vouloir à
ma place. Au reste, comment pourrais-je vouloir ce que je ne veux pas ou ne
%— 339 —
pas vouloir ce que je veux ? La liberté de la volonté, en ce sens, est moins un
problème qu’une espèce de pléonasme : vouloir, c’est par définition vouloir ce
que l’on veut (puisque la volonté ne saurait échapper au principe d’identité) et
c’est en quoi c’est être libre. C’est ce que j'appelle la spontanéité du vouloir, qui
n’est pas autre chose que la volonté en acte : au présent, « libre, spontané et
volontaire ne sont qu’une même chose » (comme le reconnaissait Descartes de
l'acte en train de s'accomplir) ; c'est pourquoi toute volonté est libre, et elle
seule (le reste n’est que passion ou passivité). C’est la liberté au sens d’Épicure
et d’Épictète, mais aussi, pour l'essentiel, au sens d’Aristote, de Leibniz ou de
Bergson. Je veux ce que je veux : je veux donc librement.

Soit. Mais peut-on vouloir aussi autre chose ? Autre chose que ce que l’on
veut ? Cela semble violer le principe d’identité. Mais comment, sans ce pouvoir-là,
aurait-on le choix ? Il semble que la volonté ne soit vraiment libre que
si elle peut se choisir elle-même, ce qui suppose — puisqu'on ne choisit que le
futur — qu’elle n’existe pas encore. Il faut donc, pour que la volonté soit absolument
libre, que le sujet préexiste paradoxalement à ce qu’il est (puisqu'il doit
le choisir) : de là le mythe d’Er, chez Platon, le caractère intelligible, chez Kant,
ou l’existence-qui-précède-l’essence chez Sartre. Cette liberté-là reste bien une
liberté de la volonté, si l’on veut, mais antérieure, au moins en droit, à toute
volition effective. Elle est absolue ou elle n’est pas. C’est ce qu’on appelle parfois
la liberté au sens métaphysique du terme, et plus souvent le libre arbitre :
ce n’est plus spontanéité mais création, plus un être mais un néant, comme dit
Sartre, plus un sujet qui choisit mais le choix du sujet par lui-même. C’est la
liberté selon Descartes (peut-être déjà selon Platon, du moins dans certains
textes), selon Kant, selon Sartre : le pouvoir indéterminé de se déterminer soi-même,
autrement dit de se choisir (Sartre: «toute personne est un choix
absolu de soi ») ou de se créer (Sartre encore : « liberté et création ne font
qu’un »). Mais comment, puisque nul ne peut se choisir qu’à la condition
d’être déjà ? Cette liberté-là n’est possible que comme néant : elle n’est possible
qu’à la condition de n’être pas ! J'aurais tendance à y voir une réfutation ; c’est
sans doute que le néant n’est pas mon fort. « Je suis en plein exercice de ma
liberté, écrit Sartre, lorsque, vide et néant moi-même, je {\it néantis} tout ce qui
existe » (« La liberté cartésienne », {\it Situations}, I). C’est une chose dont je n’ai
aucune expérience. Je ne connais que l'être. Je ne connais que l’histoire en train
de se faire, toujours simultanée à soi, toujours déterminée en même temps que
déterminante. Je ne connais, comme liberté de la volonté, que sa spontanéité :
que le pouvoir {\it déterminé} de se déterminer soi-même. Est-ce moi qui manque
d'imagination, ou Sartre, de réalisme ?

« Les hommes se trompent en ce qu’ils se croient libres, écrivait Spinoza, et
cette opinion consiste en cela seul qu’ils ont conscience de leurs actions et sont
%— 340 —
ignorants des causes par où ils sont déterminés » ({\it Éthique}, II, scolie de la
prop. 35). Ils ont conscience de leurs désirs et volitions, mais point des causes
qui les font désirer et vouloir ({\it Éthique} I, Appendice ; voir aussi la {\it Lettre 58}, à
Schuller). Comment ne croiraient-ils pas être libres de vouloir, puisqu'ils veulent
ce qu’ils veulent ? Et certes Spinoza ne nie pas qu’il y ait là une spontanéité
effective (voir par exemple {\it Éthique} III, scolie de la prop. 2), qui est celle du
{\it conatus}. Leur erreur est de l’absolutiser, comme si elle était indépendante de la
nature et de l’histoire. Comment le serait-elle, puisqu’elle n’aurait alors aucune
raison d'exister ni d’agir ? La volonté n’est pas un empire dans un empire. Je
veux ce que je veux ? Certes, mais point de façon indéterminée ! « Il n’y a dans
l’âme aucune volonté absolue ou libre ; mais l’âme est déterminée à vouloir ceci
ou cela par une cause qui est aussi déterminée par une autre, et cette autre l’est
à son tour par une autre, et ainsi à l'infini » ({\it Éthique}, II, prop. 48 ; voir aussi I,
prop. 32 avec sa démonstration). On ne sort pas du réel. On ne sort pas de la
nécessité. Est-ce à dire que chacun reste prisonnier de ce qu’il est ? Non pas,
puisque la raison, qui est en tous, n’appartient à personne. Comment pourrait-elle
nous obéir ? « L'esprit ne doit jamais obéissance, écrivait Alain. Une preuve
de géométrie suffit à le faire voir ; car si vous la croyez sur parole, vous êtes un
sot ; vous trahissez l'esprit » (Propos du 12 juillet 1930). C’est pourquoi aucun
tyran n'aime la vérité. Parce qu’elle n’obéit pas. C’est pourquoi aucun tyran
n'aime la raison. Parce qu’elle n’obéit qu’à elle-même : parce qu’elle est libre.
Non, certes, qu’elle ait le choix, si l’on entend par là qu’elle pourrait penser
n'importe quoi. Mais parce que sa nécessité propre est le gage de son indépendance.
Non que la vérité soit à {\it choisir} ; mais au contraire parce qu’elle ne l’est
pas : parce qu’elle s’impose nécessairement à toute personne qui la connaît au
moins en partie, et parce qu'il suffit de la connaître pour être libéré, au moins
partiellement, de soi (puisque la vérité est la même en tout esprit qui la
perçoit : quand un névrosé fait des mathématiques, la vérité des mathématiques
n’en devient pas névrosée pour autant). C’est ce qu’on peut appeler la liberté
de l’esprit ou de la raison, qui n’est pas autre chose que la libre nécessité du
vrai. C’est la liberté selon Spinoza, selon Hegel, sans doute aussi selon Marx et
Freud : la liberté comme nécessité comprise, ou comme compréhension,
plutôt, de la nécessité. La vérité n’obéit à personne, pas même au sujet qui la
pense : c’est en quoi elle est libre, et libère.

Trois sens, donc, dont le deuxième se subdivise en deux : la liberté d’action,
la liberté de la volonté (qu’on peut penser comme spontanéité ou bien comme
libre arbitre), enfin la liberté de l’esprit ou de la raison. Seul le libre arbitre me
paraît douteux, et à la vérité impensable. Les trois autres libertés n’en existent
pas moins, qui se complètent. À quoi bon vouloir, si l’on ne pouvait agir
librement ? Et au nom de quoi, si toute pensée était esclave ? Mais cela n’est
%— 341 —
pas. Nous sommes libres d’agir, de vouloir, de penser, du moins nous pouvons
l'être, et il dépend de nous — par la raison, par l’action — de le devenir davantage.
Quant à pouvoir faire, vouloir ou penser autre chose que ce que nous faisons,
voulons ou pensons (ce que suppose le libre arbitre), je n’en ai aucune
expérience, je le répète, ni ne vois comment la chose serait possible. On
m'objectera qu’à ce compte notre liberté n’est que relative, toujours dépendante
(du corps ou de la raison, de l’histoire ou du vrai), toujours déterminée,
et j'en suis d’accord. C’est dire, contre Sartre, que la liberté n’est jamais infinie
ni absolue. Mais comment le serait-elle, en des êtres relatifs et finis, comme
nous sommes tous ? Nul n’est libre absolument, ni totalement. On est {\it plus ou
moins libre} : c’est pourquoi on peut philosopher (parce qu’on est un peu libre),
et c’est pourquoi on le doit (pour le devenir davantage). La liberté n’est pas
donnée, elle est à conquérir. Nous ne sommes pas « condamnés à la liberté »,
comme le voulait Sartre, mais point non plus à l'esclavage. Ce n’est pas la
liberté qui est « le fondement du vrai », comme disait encore Sartre (si c'était
vrai, il n’y aurait plus de vérité du tout) ; c’est la vérité qui libère. Ainsi la
liberté est moins un mystère qu’une illusion ou un travail. Les ignorants sont
d'autant moins libres qu'ils se figurent davantage l’être. Au lieu que le sage le
devient, en comprenant qu’il ne l’est pas.

Encore faut-il rappeler que nul n’est sage en entier — que la liberté est
moins une faculté qu’un processus. On ne naît pas libre ; on le devient, et l’on
n’en a jamais fini. C’est parce que le libre arbitre n’existe pas qu’il faut se libérer
toujours, et d’abord de soi. C’est parce que la liberté n’est jamais absolue que
la libération reste toujours possible, et toujours nécessaire.

\section{Liberté de penser}
%LIBERTÉ DE PENSER
C’est moins une liberté de plus, qu’un cas particulier
de toutes : le droit de penser ce qu’on veut,
sans autre contrainte que soi ou la raison. C’est la pensée même, en tant qu’elle
échappe aux préjugés, aux dogmes, aux idéologies, aux inquisitions. Elle n’est
jamais donnée, toujours à conquérir. Sa formule est énoncée par Voltaire, après
Horace, après Montaigne, avant Kant : {\it « Osez penser par vous-même. »}

\section{Libido}
%LIBIDO
{\it Libido}, en latin, c’est le désir, souvent avec un sens péjoratif (l'envie
égoïste, la convoitise, la sensualité, la débauche...). Pascal, dans le
prolongement de saint Jean ({\it 1 Jn}, 21, 16) et de saint Augustin ({\it Confessions}, X,
30-39), y voit un autre nom pour la concupiscence : « Tout ce qui est au
monde est concupiscence de la chair, ou concupiscence des yeux, ou orgueil de
la vie: {\it libido sentiendi, libido sciendi, libido dominandi} » ({\it Pensées}, 545-458 ;
%— 342 —
voir aussi les fr. 145-461 et 933-460). Mais le mot doit son usage moderne à
Freud, qui nomme ainsi l'énergie sexuelle, telle qu’elle se manifeste dans la vie
psychique et quelles que soient les formes — y compris sublimées ou apparemment
désexualisées — qu’elle peut prendre. Plus rien ici de péjoratif : la libido
est « la manifestation dynamique, dans la vie psychique, de la pulsion sexuelle »
({\it Psychoanalyse und Libidotheorie}, 1922, cité par Laplanche et Pontalis). Freud la
distingue parfois de l'instinct de conservation ({\it Introduction à la psychanalyse},
chap. 26), avant de les fondre l’une et l’autre dans la pulsion de vie (les pulsions
d’auto-conservation relevant désormais de la libido, qui s’oppose elle-même à
la pulsion de mort : {\it Essais}, « Au-delà du principe de plaisir », 6). On ne confondra
pas la libido avec le désir sexuel, qui n’est qu’une de ses expressions,
encore moins avec la sexualité génitale, qui n’est qu’une de ses formes. La
libido peut se porter sur le moi aussi bien que sur un objet extérieur, s'investir
dans la sexualité, au sens étroit du terme, aussi bien que dans les affaires, l’art,
la politique ou la philosophie. Son retrait, tel qu’il apparaît dans la mélancolie,
aboutit à «la perte de la capacité d’aimer» ({\it Métapsychologie}, « Deuil et
mélancolie »), voire au suicide. Cela dit peut-être l'essentiel. Qu'est-ce que la
libido ? C’est la puissance de vivre, d’aimer et de jouir, en tant qu’elle est d’origine
sexuelle et susceptible de prendre des formes différentes. C’est le nom
freudien et sexuel du {\it conatus}.

\section{Libre arbitre}
%LIBRE ARBITRE
La liberté de la volonté, en tant qu’elle serait absolue ou
indéterminée : c’est «le pouvoir de se déterminer soi-même
sans être déterminé par rien » (Marcel Conche, {\it L'aléatoire}, V, 7). Pouvoir
mystérieux, et métaphysique strictement : si on pouvait l’expliquer (par
des causes) ou le connaître (par une science), il ne serait plus libre. On ne peut
y croire qu’en renonçant à le comprendre, ou le comprendre (comme illusion)
qu'en cessant d’y croire. C’est où il faut choisir entre Descartes et Spinoza,
entre Alain et Freud, entre l’existentialisme, spécialement sartrien, et ce que j’ai
appelé, faute de mieux, l’{\it insistantialisme} — entre le libre arbitre et la libération.
On ne confondra pas le libre arbitre avec l’indétermination : un électron,
même à le supposer absolument indéterminé, n’est pas doué pour autant de
libre arbitre (qui suppose une volonté), pas plus qu’un cerveau qui dépendrait
de particules indéterminées ne serait libre pour cela (puisqu'il dépend d’autre
chose que de lui-même). Le libre arbitre n’est pas non plus la spontanéité du
vouloir, qui serait plutôt — comme on voit chez Lucrèce ou les stoïciens — le
pouvoir {\it déterminé} de se déterminer soi-même. Mais il emprunte quelque chose
à l’une et l’autre : il est une spontanéité indéterminée, qui aurait, c’est son mystère
propre, la faculté de se choisir ou de se créer soi, ce qui suppose —
%— 343 —
puisqu'on ne peut choisir que l'avenir — qu’elle se précède inexplicablement
elle-même (le mythe d’Er chez Platon, le caractère intelligible chez Kant, le
projet originel ou l’existence-qui-précède-l’essence chez Sartre). Ce n’est pas ce
que je suis qui expliquerait mes choix ; ce sont mes choix, ou un choix originel,
qui expliqueraient ce que je suis. Je ne serais d’abord rien ; et c’est ce néant qui
choisirait librement ce que j'ai à être et que j’aurai été. « Chaque personne, écrit
Sartre dans {\it L'être et le néant}, est un choix absolu de soi. » Le libre arbitre est ce
choix, ou plutôt (puisqu'il faut être d’abord pour se choisir) cette impossibilité.

\section{Lieu}
%LIEU
La situation dans l’espace, ou l’espace qu’un corps occupe : c’est le {\it là}
d’un être, comme l’espace est le {\it là} de tous (la somme de tous les
lieux). Les deux notions d’{\it espace} et de {\it lieu} sont solidaires, voire se présupposent
mutuellement, au point qu’on ne puisse les définir, peut-être, sans tomber dans
un cercle. Ce sont deux façons de penser l’extension des corps — qui est une
donnée de l'expérience -, en l’inscrivant dans une limite (le lieu) ou dans 'illimité
(l’espace). Le lieu, disait Aristote, est « la limite immobile et immédiate du
contenant » ({\it Physique}, IV, 4). L'espace serait plutôt le contenant sans limites.

\section{Logique}
%LOGIQUE
Ce serait la science de la raison ({\it logos}), si une telle science était
possible. À défaut, c’est l'étude des raisonnements, et spécialement
de leurs conditions formelles de validité. Elle apparaît de plus en plus
comme une partie des mathématiques ; cela n’autorise pas les philosophes à
s’en passer.

\section{{\it Logos}}
%{\it LOGOS}
Le mot, en grec, pouvait signifier à la fois la raison et le discours.
Par exemple chez Héraclite : « Il est sage que ceux qui ont écouté,
non moi, mais le {\it logos}, conviennent que tout est un. » Ou chez saint Jean : « Au
commencement était le {\it Logos}, et le {\it Logos} était avec Dieu, et le {\it Logos} était
Dieu... » Une parole, donc, mais qui serait celle de la vérité : le discours vrai,
ou la vérité comme discours. Langage ? Raison ? Plutôt l’unité indissoluble des
deux. C’est ce qui fait la pérennité, même en français, du mot : parler de {\it logos},
aujourd’hui, ce serait suggérer qu’il n’y a ni langage sans raison ni raison sans
langage. Ces deux propositions me semblent douteuses, et la seconde, même,
impensable. Si la raison n’existait avant le langage et indépendamment de lui,
comment le langage serait-il advenu ? Dieu, dirait Spinoza, ne parle ni ne raisonne
(ce n’est pas un {\it Logos} : pas un Verbe). Rien de plus rationnel pourtant
que ce Dieu-là. La vraie raison — le vrai rapport du vrai à lui-même — est en
% 344 —
deçà du langage : l’idée vraie ne consiste « ni dans l’image de quelque chose ni
dans des mots, écrit Spinoza ; car l’essence des mots et des images est constituée
seulement par des mouvements corporels, qui n’enveloppent en aucune façon
le concept de la pensée » ({\it Éthique}, II, scolie de la prop. 49). La vraie logique est
muette : non {\it logos}, mais {\it alogos}. Non un Verbe mais un acte. Non un discours
mais un silence. Logique de l’être : onto-logique. Au commencement était
l'action.

\section{Loi}
%LOI
Un énoncé universel et impératif. En ce sens, il est clair qu’il n’y a pas
de « lois de la nature » : on n’en parle que par analogie, pour évoquer
ou expliquer certaines régularités observables. La rationalité de l’univers est au
contraire toute silencieuse (pas d’énoncés) et simple (pas d’impératifs). L’identité
est ordre qui lui suffit. Et sa nécessité, s’il fallait la formuler, ne se dirait
qu’à l’indicatif. Les lois humaines tendent vers ce modèle (« tout condamné à
mort aura la tête tranchée. »), sans pouvoir jamais complètement y atteindre.
Faute de pouvoir, justement, ou abus de volonté. D’où l’idée de Dieu, qui
serait une volonté toute-puissante : un indicatif-impératif. L'idée de silence,
poussée à sa limite, nous débarrasse de ces deux anthropomorphismes.

Une loi, c’est ce qui s'impose (la nécessité), ou devrait s'imposer (la règle,
l'obligation). On parle dans le premier cas de lois de la nature ; dans le second,
de lois morales ou juridiques. Les premières, qui ne sont voulues par personne,
s'imposent à tous. Les secondes, qui sont voulues par la plupart, ne s'imposent
à personne : elles n’existent, comme loi, que par la puissance que nous gardons,
malgré elles, de les violer. Si le meurtre ne restait possible, aucune loi n’aurait
besoin de l’interdire. Si la gravitation universelle pouvait être violée, ce ne serait
plus une loi.

C’est le sens juridique qui est premier : la loi est d’abord une obligation
imposée par le souverain. On ne parle de lois de la nature que secondairement,
parce qu’on imagine que la nature obéit elle aussi à quelqu'un, qui serait Dieu.
Toutefois ce n’est qu’une métaphore. La nature n’est pas assez libre pour pouvoir
obéir. Dieu le serait trop pour pouvoir commander.

On s’en voudrait de ne pas citer la définition fameuse que donnait
Montesquieu : « Les lois, dans la signification la plus étendue, sont les rapports
nécessaires qui dérivent de la nature des choses » ({\it L'esprit des lois}, I, 1). On ne
peut pourtant s’en contenter : car comment, si cette définition était bonne, y
aurait-il de {\it mauvaises} lois ? C’est que Montesquieu, comme Auguste Comte
après lui, pense d’abord aux lois de la nature, qui ne sont ni bonnes ni mauvaises,
qui ne sont que «des relations constantes entre les phénomènes
observés » ({\it Discours sur l'esprit positif}, \S 12). Les lois humaines sont d’une autre
%— 345 —
sorte : elles ne s'imposent que pour autant qu’on les impose, ce qui ne garantit
ni qu’elles soient justes ni qu’il faille leur obéir. Ainsi cette notion de loi reste
irréductiblement hétérogène : elle sert surtout à masquer cette dualité même
qui la constitue, afin que le fait semble juste ou que la justice semble faite. C’est
une façon d’échapper au désordre et au désespoir. La vérité est qu’il n’y a que
des faits. Mais qui pourrait la supporter ?

\section{Loisir}
%LOISIR
Au singulier, c’est la traduction de l’{\it otium} des Anciens : le temps
libre, celui qui ne sert qu’à vivre, celui qui n’est pas dévoré par le
travail. Non la paresse ni le repos, mais la disponibilité, comme une ouverture
au monde et à soi, au présent et à l'éternité : l’espace offert à l’action, à la
contemplation, à la citoyenneté, à l'humanité.
Au pluriel, c’est l’ensemble des divertissements qui permettent de supporter
ce temps libre, le plus souvent en payant pour qu’il cesse de l’être.

\section{Loterie}
%LOTERIE
Un jeu de hasard, ou le hasard comme jeu. Le mot vient des lots
qu’on y gagne. « C’est le moyen de faire jouer la fortune sans
aucune injustice », écrit Alain. Mais sans aucune justice non plus. Par quoi c’est
une image de la vie, davantage que de la société. « Toute la mécanique de la
loterie va à égaliser les chances, écrit encore Alain. Ainsi la chance se trouve
purifiée. Cent mille pauvres font riche un d’entre eux, sans choix. C’est le
contraire de l'assurance. » C’est que l’assurance est une mutualisation des
risques. La loterie serait plutôt une mutualisation des chances. Qu’on en ait fait
un impôt volontaire montre l'intelligence de nos grands argentiers. Seul l’État
gagne à tout coup. C’est mettre le hasard au service de la comptabilité nationale.

\section{Lucidité}
%LUCIDITÉ
C'est voir ce qui est comme cela est, plutôt que comme on voudrait
que cela soit. Par quoi la lucidité ressemble beaucoup au
pessimisme : non que les choses aillent toujours de pire en pire (pourquoi
serait-ce le cas ?), mais parce qu’il n’est pas d’usage qu’elles aillent comme nous
voudrions, ni habituel que nous voulions qu’elles aillent comme elles vont en
effet. Ainsi la lucidité marque d’abord la distance entre l’ordre du monde et
celui de nos désirs, tout en refusant de renoncer — car alors il n’y aurait plus distance
— à l’un ou à l’autre. C’est l'amour de la vérité, quand elle n’est pas
aimable.

%— 346 —
Cela vaut aussi pour soi. Car enfin se connaître comme on est, c’est presque
toujours se décevoir. Lucidité bien ordonnée commence par soi-même : tel est
le secret de l'humilité.

\section{Lumières}
%LUMIÈRES
Le mot désigne une période en même temps qu’un idéal. La
période, c’est le {\footnotesize XVIII$^\text{e}$} siècle européen. L'idéal, c’est celui de la
raison, que Descartes appelait déjà « la lumière naturelle », mais délivrée de
toute théologie, voire de toute métaphysique, c’est celui de la connaissance,
celui du progrès, de la tolérance, de la laïcité, de l’humanité lucide et libre. Être
un homme des Lumières, explique Kant, c’est penser par soi-même, c’est se
servir librement de sa raison, c’est se libérer des préjugés et de la superstition.
C'est ce qui justifie la formule fameuse : « {\it Sapere aude} ! aie le courage de te
servir de ton propre entendement! voilà la devise des Lumières » (Kant,
{\it Réponse à la question « Qu'est-ce que les Lumières ? »}, 1784). La maxime est
d’Horace, elle pourrait être de Lucrèce, et Montaigne déjà l’avait faite sienne
({\it Essais}, I, 26, 159), comme Voltaire après lui et avant Kant ({\it Dictionnaire philosophique},
article « Liberté de penser »). On voit que les Lumières, en tant
qu’idéal, sont de tous les temps. C’est que la superstition et le dogmatisme toujours
nous précèdent, et nous accompagnent.

\section{Luxe}
%LUXE
C'est jouir de l’inutile. Par exemple une cuillère : qu’elle soit en or,
cela ne sert à rien, mais c’est un plaisir supplémentaire. Notion par
nature relative, mais qui suppose toujours une part d’excès, de surabondance,
d’{\it exagération}, comme dit Kant, dans le confort, la joliesse ou la dépense. C’est
le contraire des plaisirs naturels et nécessaires d’Épicure : le luxe est fait de plaisirs
culturels et superflus. C’est pourquoi c’est un piège, s’il devient nécessaire,
et un luxe, s’il reste superflu.

LUXURE L'usage immodéré des plaisirs sexuels. Ce n’est le plus souvent
qu’une petite faute, mais c’en est une : non parce que le sexe serait
coupable, il l’est rarement, mais parce que l’intempérance l’est toujours. Faute
vénielle, du moins entre partenaires consentants. Mais l’obligation d’obtenir ce
consentement interdit de s’abandonner tout à fait à la luxure. Au reste le corps,
presque toujours, fait une limite suffisante. Si Sade n’avait passé tant d’années
en prison, s’il avait fait davantage l'amour, et plus heureusement, le plaisir
même lui aurait enseigné — contre le faux infini du manque ou de l’imaginaire
%— 347
— son très positif et très voluptueux pouvoir de {\it modération}. La luxure n’est pas
un luxe ; l'érotisme en est un.

\section{Lycée}
%LYCÉE
C'est le nom, en grec, de l’école d’Aristote, parce qu’il enseignait
dans un gymnase proche d’un temple ou d’une statue d’Apollon
Lycien ({\it lykeios}, loup dieu). Aristote y délivrait son enseignement, aussi bien
acroamatique, le matin, qu’exotérique, le soir, dans le promenoir ({\it peripatos} :
d’où le nom d’école péripatéticienne qu’on donne aussi au Lycée) du gymnase.
L'école, à la mort d’Aristote (ou plutôt dès son départ en quasi exil, dans l’île
d’Eubée, où il mourra quelques mois plus tard), sera dirigée par Théophraste,
puis, à la mort de celui-ci, par Straton de Lampsaque. Après quoi elle connut
un déclin progressif, qui n’empêchera pas Andronicos de Rhodes, son dixième
et dernier scholarque, de publier, au I“ siècle avant Jésus-Christ, les œuvres ésotériques
du Maître, lesquelles reprennent pour l'essentiel son enseignement du
matin et sont parvenues, grâce à cette édition, jusqu’à nous. Quant aux textes
exotériques, que les Anciens admiraient fort (Cicéron compare le style d’Ariscote
à un « fleuve d’or », Quintilien en loue la grâce et la douceur), il n’en reste
à peu près rien. De là l’image d’un Aristote exclusivement professoral ou technicien,
qui aurait omis d’être artiste : philosophe professeur, pour les professeurs
de philosophie. L’injustice n’est pas trop grave : même amputé de la
moitié de son œuvre, ce philosophe reste le plus grand de tous — et le Lycée le
modèle, à jamais, de l’exigence intellectuelle. De là un peu de nostalgie, parfois,
quand on pense à nos lycées d’aujourd’hui. Nouvelle injustice : l’éducation de
masse est évidemment un progrès, et l’on ne saurait pas davantage demander à
nos enseignants d’égaler Aristote qu’à leurs élèves de mettre le savoir, dans une
société qui n’y croit plus guère, plus haut que tout. Ce n’est pas une raison
pour remplacer la lecture des grands auteurs par celle des journaux, ni le travail
par le débat, ni l’amour de la vérité par celui de la communication. Le Lycée
n'était pas l’agora : c'était un lieu d’étude, d’enseignement, de réflexion, bien
plus que d’échanges ou de « spontanéité ». Nos lycées ne méritent leur nom
qu’en restant fidèles, au moins de ce point de vue, au grand modèle auquel ils
doivent leur appellation et une partie, malgré tout, de ce qu’on y enseigne.
Mieux vaut rivaliser avec Aristote, même de très loin, qu’avec la télévision.

\section{Lymphatique}
%LYMPHATIQUE
L’un des quatre tempéraments de la tradition hippocratique.
Mollesse, lenteur, inattention.
%{\footnotesize XIX$^\text{e}$} siècle — {\it }

