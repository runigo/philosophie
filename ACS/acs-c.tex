%{\footnotesize XIX$^\text{e}$} siècle {\bf --} {\it }

\chapter{C}

\section{Ça}
%ÇA
L’une des trois instances (avec le moi et le surmoi) de la seconde topique
de Freud. C’est le pôle pulsionnel de l'individu : le ça touche par là à la
biologie et à l’hérédité (à ce que Freud appelle « le passé de l’espèce ») autant
que le surmoi à la culture ou à l'éducation (« le passé de la société »). Il est bien
sûr antérieur à la conscience, et reste indépendant d’elle. Ce n’est pas un sujet,
ni une personne. C’est notre ancrage dans la nature, ou la nature en nous.
Avant la conscience, avant même l’inconscient, il y a le corps.

On pense à L'{\it anti-Œdipe} de Deleuze et Guattari : « Ça respire, ça chauffe,
ça mange. Ça chie, ça baise. On a bien tort d’appeler ça le ça! » Tort ? Pas
forcément. C’était marquer la distance entre ce que nous sommes, ou voulons
être, et ce qui nous fait ou nous traverse. Le ça n’est pas une chose, mais encore
moins un individu. « Là où {\it ça} était, écrit Freud, {\it je} dois advenir. » L’imparfait
ne doit pas tromper. Le ça est originaire, sans doute, mais tout aussi ultime. Il
nous accompagnera jusqu’à notre dernier souffle, qui sera le sien. Le sujet toujours
reste une conquête ou une exigence.

\section{Cadavre}
%CADAVRE
Un corps qui a été vivant et qui ne l’est plus. Ne se dit en pratique
et au sens propre que des animaux (humanité comprise).
C’est la mort réelle, ou la forme réelle de la mort {\bf --} ce qui la sépare, malgré Épicure,
du pur néant. Cela ne change pas l'essentiel, qui est la vie. La mort me
délivrera de mon cadavre, bien avant que les vers ne s’en chargent.

\section{Calcul}
%CALCUL
{\it Calculus}, en latin, c’est un caillou (d’où le sens médical du mot, par
exemple quand on parle d’un calcul dans les reins ou la vessie), une
% 95
boule ou un jeton, bref tout ce qui peut servir à compter (d’où le sens
ordinaire : on est passé de l'instrument à son utilisation). Calculer, c’est
d’abord faire une opération avec ou sur des nombres. Par extension, le mot
désigne toute opération de l'esprit susceptible d’un traitement purement arithmétique,
logique (par l'application de règles ou d’algorithmes), voire mécanique :
calculer, c’est penser comme une machine pourrait le faire, au moins en
droit, voire comme elles semblent le faire (l'intelligence artificielle). Si penser
c’est calculer, comme le voulait Hobbes, cela signifie que esprit est une
machine ; c’est ce qu’on appelle le cerveau. Mais il n’est esprit que par la
conscience ; c’est ce qui le distingue des machines à calculer les plus performantes.
L'ordinateur qui effectue presque instantanément les opérations les
plus complexes ne {\it sait} pas compter, puisqu'il ne sait pas qu’il compte. Cela
permet une autre définition : le calcul, c’est la pensée qui ne se pense pas, ou ce
qui reste de la pensée quand on met la conscience {\bf --} par mécanisme ou par abstraction {\bf --}
hors jeu ou entre parenthèses. C’est économiser l'esprit, qu’on
réserve pour des tâches plus hautes.

« La fonction de penser ne se délègue pas », disait Alain. La fonction de calculer,
si, et c’est en quoi le calcul n’est pas le tout de la pensée.

\section{Calomnie}
%CALOMNIE
C'est dire le mal qui n’est pas : la calomnie est un mensonge
malveillant, ou une méchanceté mensongère. Double faute.
Plus coupable par là que la médisance (dire le mal qui est), et moins agréable.

\section{Candeur}
%CANDEUR
Une forme de crédulité, mais appliquée aux choses humaines. Le
crédule ne sait pas douter ; le candide ne sait pas soupçonner.
Défaut plus sympathique, mais tout aussi dangereux.

\section{Canon}
%CANON
Toute sorte de règle ({\it kanôn} en grec) qui peut servir de norme, de
modèle ou de référence.

\section{Canonique}
%CANONIQUE
C’est un ensemble de règles pour la pensée ou la connaissance.
Spécialement, chez Épicure, c’est ce qui tient lieu à la
fois de logique, de méthode et de théorie de la connaissance : l’exposé des critères
de la vérité (sensation, anticipations, affections) et des moyens (par la
démonstration) d’accroître la connaissance que l’on en a.

% 96
\section{Capital}
%CAPITAL
La richesse, considérée comme moyen de s'enrichir. Ce cercle dit
l'essentiel. Un capital, c’est de l’argent qui rapporte, par opposition
à celui qu'on gagne (les revenus) ou qu’on dépense : c’est une richesse
créatrice de richesse.
On oppose le capital au travail. On à raison : le capital sert surtout à faire
travailler l’argent et les autres.

\section{Capitalisme}
%CAPITALISME
On peut le définir structurellement ou fonctionnellement.

Structurellement, c’est un système économique fondé sur la
propriété privée des moyens de production et d'échange, et réglé, si l’on peut dire,
ou déréglé, par la liberté du marché (y compris du marché du travail : c’est ce qu’on
appelle le salariat). Cette définition débouche sur une tautologie, qui n’est pas sans
conséquences : une entreprise appartient à celui ou ceux qui la possèdent, autrement
dit aux actionnaires. Qu'elle soit au service de ses clients et de ses salariés,
comme on le dit presque partout, n’est qu’un aimable raccourci, qui tait l'essentiel :
elle n’est au service approximatif de ses salariés (disons qu’elle tend à les satisfaire à
peu près, ne serait-ce qu’en leur versant un salaire) que parce que c’est la seule façon
de satisfaire ses clients ; et elle n’est au service de ses clients que parce que c’est la
seule façon de satisfaire durablement l'actionnaire. Ainsi le fameux triangle est
orienté. Les clients et les salariés sont la base ; les actionnaires, le sommet.

Le capitalisme peut aussi se définir d’un autre point de vue, non plus structural
mais fonctionnel : c’est un système qui sert, avec de l'argent, à faire plus d’argent.
Si vous avez un million d’euros sous votre matelas, en billets ou en lingots, vous
êtes assurément un riche, un imprudent, un imbécile, mais vous n’êtes pas pour
autant un capitaliste : votre argent ne vous rapporte rien, votre richesse ne crée pas
de richesse. Si vous avez 1 000 euros en SICAV à votre banque, vous n'êtes certes
pas riche, mais vous êtes déjà, à votre petit niveau, un capitaliste : votre argent vous
enrichit. Ainsi l’accumulation du capital fait partie de sa définition.

Que la richesse aille surtout aux riches, dans une telle société, est à peu près
inévitable. Son but est le profit, non la justice. C’est ce qui rend le capitalisme
moralement insatisfaisant et économiquement efficace. La politique, entre cette
insatisfaction et cette efficacité, essaie d’imposer une espèce d'équilibre. Ne
comptons pas sur le marché pour être juste à notre place. Ni sur la justice pour
créer des richesses.

\section{Caprice}
%CAPRICE
Volonté sans raison, sans durée ou sans force. L'enfance est l’âge
des caprices ; et le caprice, chez un adulte, une marque d’infantilisme.

% 97
\section{Caractère}
% CARACTÈRE
C’est d’abord une empreinte ({\it kharaktêr}, en grec, c’est le graveur
de médaille ou de monnaie), une marque indélébile, un
signe permanent ou distinctif. Le mot désigne pour cela l’ensemble des dispositions
permanentes ou habituelles d’un individu (son {\it êthos}), autrement
dit sa manière propre de sentir et de ressentir, d’agir et de réagir {\bf --} sa façon
particulière d’être soi. « C’est ce que la nature a gravé dans nous », disait
Voltaire. Je n’irais pas jusque là. Le caractère me semble plus individualisé
et évolutif que le tempérament, moins que la personnalité. Je dirais
volontiers : le tempérament d’un individu, c’est ce que la nature a fait de
lui ; son caractère, c’est ce que l’histoire a fait de son tempérament ; sa personnalité,
c’est ce qu’il a fait, et qu’il ne cesse de faire, de ce que l’histoire et
la nature ont fait de lui. Le caractère renvoie au passé, donc à tout ce qui, en
nous, ne dépend plus de nous. Il faut faire avec, comme on dit, et c’est en
quoi, selon une formule célèbre d'Héraclite, « le caractère d’un homme est
son démon » ou son destin : parce qu’il est ce qui choisit en lui, qu’il ne
choisit pas.

Kant distingue, pour tout être humain, un {\it caractère empirique} et un {\it caractère
intelligible} ({\it Critique de la raison pratique}, Examen critique de l’analytique).
Le caractère empirique est soumis aux conditions du temps, donc de la
causalité ; il est totalement déterminé (par le passé) et déterminant (pour
l'avenir), au point qu’on pourrait, si on le connaissait suffisamment, « calculer
la conduite future d’un homme avec autant de certitude qu’une éclipse de lune
ou de soleil ». Le caractère intelligible, au contraire, échappe aux conditions du
temps et donc à l’enchaînement causal des phénomènes : c’est le choix absolument
libre que chacun fait de soi («le caractère qu’il se donne à lui-même »,
écrit Kant). Il en résulte que le même individu est à la fois totalement déterminé
(en tant que phénomène) et totalement libre (en tant que chose en soi).
La conjonction des deux est bien sûr impossible à comprendre ou à expliquer.
Mais quand bien même on l’accorderait à Kant, on ne voit pas comment un
individu pourrait se choisir soi {\bf --} puisqu'il faut, pour ce faire, qu’il existe
d’abord. Le caractère intelligible, même considéré en lui-même, reste inintelligible.

\section{Cardinales (vertus {\bf --})}
%CARDINALES (VERTUS {\bf --})
Ce sont les quatre vertus principales de la tradition,
celles qui portent ou soutiennent ({\it cardo},
en latin, c’est le gond) toutes les autres : la prudence ou sagesse pratique ({\it phronèsis}),
la tempérance, le courage ou force d’âme, la justice. Non pourtant
qu’elles suffisent ; encore faut-il ouvrir la porte.

% 98
\section{Carême}
%CARÊME
Un temps de jeûne ou de pénitence (spécialement, dans le christianisme,
les quarante jours qui précèdent le vendredi saint, en
souvenir des quarante jours que le Christ aurait passés dans le désert). Voltaire
se demande si l'institution doit plus aux médecins, pour soigner des indigestions,
ou à la tristesse, qui coupe l'appétit. Puis s’emporte : « Le riche papiste
qui aura eu sur sa table pour cinq cents francs de poisson sera sauvé ; et le
pauvre, mourant de faim, qui aura mangé pour quatre sous de petit salé, sera
damné ! [...] Prêtres idiots et cruels ! À qui ordonnez-vous le carême ? Est-ce
aux riches ? Ils se gardent bien de l’observer. Est-ce aux pauvres ? Ils font
carême toute l’année. » J'aime cette colère. Mais j'aime aussi que certains soient
capables de jeûner, encore aujourd’hui, pour d’autres raisons que d’hygiène ou
de coquetterie.

\section{\it Carpe diem}
%{\it CARPE DIEM}
C’est une formule d'Horace, en latin, qu’on peut traduire
par « Cueille le jour ». Notre époque hédoniste et velléitaire
y voit volontiers le summum de la sagesse. Il faudrait vivre dans l’instant, profiter
du moment présent, prendre les plaisirs comme ils viennent. Et certes je
ne conteste pas qu'il y ait là comme une sagesse minimale. De là à croire que le
{\it farniente} et la gastronomie pourraient tenir lieu de philosophie, il y a malgré
tout un pas qu’on évitera de franchir. Épicurisme ? On en trouve en effet des
échos chez Horace, pas toujours aussi souriants qu’on le croit, mais tournés,
presque inévitablement, vers les plaisirs les plus proches, les plus faciles ou les
plus matériels. Par exemple dans la formule fameuse : {\it Edite, bibite, post mortem
nulla voluptas} (Mangez, buvez, après la mort il n’y a plus de plaisir). Tout cela
est très vrai, mais un peu court. Vivre dans l’instant ? On ne le peut. Vivre au
présent ? C’est le seul chemin, puisqu’il n’y a rien d’autre. Mais le présent n’est
pas un instant : c’est une durée, qu’on ne peut habiter, montrait Épicure, sans
un rapport délibéré au passé et à l'avenir. Jouir ? Le plus possible. Mais cela ne
nous dit pas quoi faire de notre vie quand elle n’est pas agréable, quand la douleur
ou l'angoisse nous emportent, quand le plaisir est différé ou impossible.
Cueille le jour, donc, mais ne renonce pas pour autant à l’action, ni à la durée,
ni à ces plaisirs spirituels qu’Épicure, à la fin de la {\it Lettre à Ménécée}, appelait
« des biens immortels ». C’est qu'ils n’ont affaire qu’au vrai, qui ne meurt pas.
{\it Carpe aeternitatem}.

\section{Cartésien}
%CARTÉSIEN
De Descartes, ou digne de l’être. Désigne communément un
goût ou un talent particuliers pour l’ordre méthodique, la
rigueur, la clarté. Le mot, de nos jours, se prend parfois en mauvaise part,
% 99
comme si la rigueur excluait l'intuition, comme si l’ordre et la clarté supposaient
qu’on s’aveuglât sur l’obscure complexité de tout. C’est oublier qu’il n’y
a d’obscurité que pour un regard, de complexité que pour une intelligence,
enfin qu’une intuition ne vaut que par l’usage qu’on en fait. Ce qu’il y a de plus
bouleversant, chez Descartes, ce n’est pas la méthode, c’est le courage.

\section{Casuistique}
%CASUISTIQUE
L'étude des cas, et spécialement des cas de conscience. Le
mot, depuis Pascal, vaut le plus souvent comme condamnation,
à cause de l’abus que les jésuites firent de la chose : ce ne serait que subtilités
complaisantes ou hypocrites, sophismes commodes, artifices avantageux
ou lâches {\bf --} bref, l’art de s’innocenter à bon compte. Le fait est qu’on a rarement
besoin, pour faire son devoir, d’une telle débauche d’analyses et de distinguos.
Mais il n’est pas vrai non plus que la loi morale suffise à tout ; il faut
encore l’appliquer aux cas particuliers, ce qui ne va pas sans un peu de réflexion
et d'expérience : c’est à quoi sert la casuistique, au bon sens du terme, comme
morale appliquée et pédagogie du jugement.

\section{Catéchisme}
%CATÉCHISME
Exposé, à visée à la fois dogmatique et pédagogique, d’une
religion ou d’une doctrine. La chose se fait souvent par questions
et réponses : ainsi dans les pittoresques catéchismes à travers lesquels Voltaire,
dans son {\it Dictionnaire}, expose sa propre pensée (catéchisme chinois, catéchismes
du curé, du japonais, du jardinier...), ou surtout dans le {\it Catéchisme
positiviste} d’Auguste Comte. Mais on sent bien que les questions sont là pour
les réponses, non l'inverse. On dirait le squelette d’une pensée. C’est qu’elle est
morte.

\section{Catégories}
%CATÉGORIES
Ce sont les prédicats les plus généraux qu’on peut attribuer,
dans un jugement, à un sujet quelconque : les concepts fondamentaux
qui servent à penser l’être (Aristote) ou à structurer la pensée
(Kant). Ces deux utilisations se rejoignent davantage qu’elles ne s’opposent ; la
seconde permet la première, qui la justifie. Si notre pensée ne pensait que soi,
nous serions Dieu ou fous. Mais alors nous n’aurions plus besoin de penser.
Aristote distinguait dix catégories, qui sont autant de façons de dire l’être
(puisque « l'être se dit en plusieurs sens ») : selon la substance, la quantité, la
qualité, la relation, le lieu, le temps, la position, la possession, l’action, la passion
(voir ces mots). Kant, tout en se réclamant expressément d’Aristote, en
comptait douze, correspondant aux fonctions logiques du jugement, qu’il rangeait
% 100
trois par trois : les catégories de la quantité (unité, pluralité, totalité), de
la qualité (réalité, négation, limitation), de la relation (inhérence et subsistance,
causalité et dépendance, communauté ou action réciproque), enfin de la modalité
(possibilité ou impossibilité, existence ou non-existence, nécessité ou
contingence). Ce sont les genres de l’être (Aristote) ou les concepts purs de
l’entendement (Kant).

\section{Catégorique}
%CATÉGORIQUE
Qui affirme ou nie sans condition ni alternative. Un jugement
catégorique est un jugement qui n’est ni hypothétique
ni disjonctif. Par exemple : « Socrate est un homme. » Ou bien : « Cet
homme n’est pas Socrate. »

De là, chez Kant, la notion d’{\it impératif catégorique} : c'est celui qui commande
absolument, de façon inconditionnelle et sans échappatoire (« Ne mens
pas »). Tous se ramènent à un seul, qui est celui-ci : « Agis uniquement d’après
la maxime qui fait que tu peux vouloir en même temps qu’elle devienne une loi
universelle » ({\it Fondements...}, II). C’est la loi que la volonté s’impose à elle-même,
ou doit s'imposer, en tant qu’elle est législatrice. Tout homme est libre
de la violer, mais n’est autonome qu’en la respectant.

\section{\it Catharsis}
%{\it CATHARSIS}
En grec, c’est la purification, la purgation, la délivrance par
évacuation de ce qui encombre ou perturbe. Ainsi la tragédie
serait une {\it catharsis} des passions, selon Aristote, comme la comédie une {\it catharsis}
de nos ridicules, selon Molière, ou comme certaines psychothérapies se veulent
des {\it catharsis} de nos émotions ou de nos traumatismes. Il est douteux, dans les
trois cas, que cela suffise.

\section{Causalité}
%CAUSALITÉ
C’est une relation entre deux êtres ou deux événements, telle
que l'existence de l’un entraîne celle de l’autre et l'explique.

La causalité se déduit ordinairement de la succession : si à chaque fois
qu’un phénomène {\it a} apparaît, le phénomène {\it b} le suit, on en conclura que {\it a} est
la cause de {\it b}, qui serait son effet. C’est passer du constat empirique d’une
conjonction constante (à notre échelle) à l’idée d’une connexion nécessaire.
Hume n’aura pas de mal à montrer que ce passage reste intellectuellement mal
fondé. Car une succession, aussi répétée qu’elle puisse être, ne saurait en toute
rigueur prouver quoi que ce soit : l’idée d’une connexion nécessaire, entre la
cause et l’effet, n’est que le résultat en nous d’une accoutumance très forte, qui
nous pousse à passer, presque inévitablement, de l’idée d’un objet à celle d’un
% 101 
autre objet qui le précède ou le suit. La causalité n’apparaît pas dans le monde
(entre les choses) mais dans l'esprit (entre les idées). Ou elle n’apparaît dans le
monde, préciserait Kant, que parce qu’elle est d’abord dans l'esprit : c’est une
catégorie de l’entendement, qui ne saurait venir de l'expérience puisqu'elle la
rend possible. On n’échappe à l’empirisme que par le transcendantal ; au transcendantal,
que par l’empirisme ou le matérialisme.

Quoi qu'il en soit du statut de l’idée de causalité ({\it a priori} ou {\it a posteriori}),
on accordera à ces deux auteurs que la causalité en tant que telle n’est jamais
perçue {\bf --} on ne perçoit que des successions ou des simultanéités {\bf --} ni démontrée.
C’est peut-être ce qui explique que les sciences modernes, comme l'avait
remarqué Auguste Comte, s'intéressent moins aux {\it causes} qu'aux {\it lois}. C’est
renoncer au {\it pourquoi}, pour ne plus dire que le {\it comment}, et préférer la {\it prévision}
à {\it l'explication}. L'action y trouve son compte ; mais l'esprit, non. Car quelle est
la cause des lois ?

\section{Causalité (principe de {\bf --})}
%CAUSALITÉ (PRINCIPE DE {\bf --})
Le principe de causalité stipule que tout
fait a une cause et que, dans les mêmes
conditions, la même cause produit les mêmes effets. C’est parier sur la rationalité
du réel et sur la constance de ses lois. Pari sans preuve, comme ils sont tous,
et jusqu'ici sans échec.

On ne confondra pas le principe de causalité avec le déterminisme absolu,
qui suppose non seulement la constance des lois de la nature mais aussi l’unité
et la continuité des séries causales dans le temps, de telle sorte qu’un état donné
de l'univers découle de ses états antérieurs et entraîne nécessairement la totalité
de ses états ultérieurs (ce pour quoi le déterminisme, en ce sens fort, est en
vérité un prédéterminisme : tout est joué ou écrit à l'avance). L’indéterminisme,
de même, ne suppose pas une violation du principe de causalité, mais
simplement la pluralité et la discontinuité des chaînes causales. Ainsi, chez
Lucrèce, le {\it clinamen} n’est pas sans cause (sa cause est l’atome) ; mais il est sans
cause antécédente : c’est un pur présent, qu'aucun passé n’explique ni ne contenait.
Il se produit {\it incerto tempore, incertisque locis} (il est indéterminé dans
l’espace et le temps). Comme le remarque Marcel Conche, « le principe de causalité
n’en est pas pour autant contredit, car, comme tel, il n’implique pas que
toute cause doive produire son effet sous des conditions de lieu et de temps ».
C’est en quoi le clinamen vient briser « le destin des physiciens » (le prédéterminisme),
non la rationalité du réel (la causalité).

On remarquera que toute cause, étant elle-même un fait, doit avoir une
cause, qui doit à son tour en avoir une, et ainsi à l’infini. C’est ce qu’on appelle,
à propos de Spinoza, la chaîne infinie des causes finies ({\it Éthique}, I, prop. 28).

% 102
Le déterminisme suppose la continuité de cette chaîne ; l’indéterminisme, sa
discontinuité. Mais il y a plus. D’un point de vue métaphysique, l’application
indéfiniment réitérable du principe de causalité semble exiger {\bf --} si l’on veut
échapper à la régression à l’infini {\bf --} une cause première, qui serait sans cause ou
cause de soi. Le principe de causalité, pris absolument, aboutit ainsi à sa propre
violation (une cause sans cause) ou au cercle (une cause qui se causerait elle-même).
Si Dieu est cause de tout, quelle est la cause de Dieu ? S’il n’y a pas de
Dieu, quelle est la cause de tout ?

\section{\it Causa sui}
%{\it CAUSA SUI}
Cause de soi. La notion est évidemment paradoxale : elle ne
s'applique légitimement qu’au libre arbitre et à Dieu, s’ils existent,
ou au Tout, s’il est nécessaire. « J'entends par {\it cause de soi}, écrit Spinoza,
ce dont l’essence enveloppe l'existence, autrement dit ce dont la nature ne peut
être conçue sinon comme existante. » Ce sont les toutes premières lignes de
l’{\it Éthique}, qui commence ainsi par un abime. Comment se causer soi, puisqu'il
faudrait, pour en être capable, exister déjà, donc n’avoir plus besoin de cause ?
Mais c’est passer à côté de l'essentiel, qui est l’éternité, qui est la nécessité (être
cause de soi, c’est exister « par la seule nécessité de sa nature » : {\it Éthique}, I, 24,
dém.), qui est l’immanence, qui est « l'affirmation absolue de l’existence d’une
nature quelconque » (premier scolie de l’Éthique), qui est la puissance d’exister
de tout, comme le {\it conatus} de la nature ou « l’auto-productivité même du Réel »
(cette dernière expression est de Laurent Bove, à propos de Spinoza). Un
abîme ? Si l’on veut, mais absolument plein : abîme de l'être, non du néant, et
qui fait comme un sommet indépassable. « La {\it causa sui} n’est pas un principe
abstrait, souligne Laurent Bove : elle est la position du réel (en son essence
identique à sa puissance) comme “affirmation absolue” ou comme autonomie »
({\it La stratégie du conatus}, Vrin, 1996, p. 7). C’est la puissance d’exister de tout,
ou du Tout, sans quoi aucune cause jamais ne serait possible.

\section{Cause}
%CAUSE
Ce qui produit, entraîne ou conditionne autre chose, autrement dit
ce qui permet de l'expliquer : sa condition nécessaire et suffisante,
s’il en est une, ou l’ensemble de ses conditions.

Une cause est ce qui répond à la question {\it « Pourquoi ? »}. Comme on peut
répondre de plusieurs points de vue différents, il y a différents types de cause.
Aristote en distinguait quatre : la cause formelle, la cause matérielle, la cause
efficiente, la cause finale ({\it Métaphysique}, A, 3, {\it Physique}, II, 3 et 7). Soit par
exemple cette statue d’Apollon. Pourquoi existe-t-elle ? Bien sûr parce qu’un
sculpteur l’a taillée ou modelée (c’est sa cause efficiente : par exemple Phidias).
%{\bf --} 103 
Mais elle n’existerait pas, du moins telle qu’elle est, sans la matière dont elle est
faite (c’est sa cause matérielle : par exemple le marbre), ni sans la forme qu’elle
a ou qu’elle est (non la forme supposée d’Apollon, que personne ne connaît,
mais la forme réelle de la statue elle-même : c’est sa cause formelle, son essence
ou quiddité), ni enfin sans le but en vue duquel on l’a sculptée (qui est donc sa
cause finale : par exemple la gloire, la dévotion ou l’argent). Les Modernes, de
ces quatre causes, ne retiennent guère que la cause efficiente. Ils ne croient plus
qu’en l’action, qui n’a pas besoin de croire.

\section{Caverne (mythe de la {\bf --})}
%CAVERNE (MYTHE DE LA-)
Sans doute le mythe le plus célèbre de
Platon. Il se trouve au livre VII de la {\it République}.
Ce qu’il décrit ? Des prisonniers enchaînés dans une caverne, dos à la
lumière, incapables même de tourner la tête : ils ne peuvent voir que la paroi
rocheuse devant eux, sur laquelle un feu qu’ils ne voient pas projette leurs
propres ombres, ainsi que celles d’objets fabriqués ou factices qu’on fait passer
derrière eux... Comme ils n’ont jamais vu autre chose, ils prennent ces ombres
pour des êtres réels, dont ils parlent très sérieusement. Mais voilà qu’on force
l'un de ces prisonniers à sortir : il est ébloui, au point d’abord de ne rien
distinguer ; doit-il retourner dans la caverne, c’est l’obscurité cette fois qui
l’aveugle... Ainsi sommes-nous : nous ne connaissons que les ombres du réel,
notre soleil est comme ce feu, nous ignorons tout du vrai monde (le monde
intelligible) et du vrai soleil qui l’illumine (l'Idée du Bien). Les rares parmi
nous qui s’aventurent à contempler le monde intelligible passent d’un éblouissement,
quand ils montent vers les Idées, à un obscurcissement, quand ils
redescendent dans la caverne. Aussi sont-ils jugés ridicules : il n’est pas exclu,
s'ils veulent inciter les autres à sortir, qu’on les mette à mort.

L'enjeu de ce mythe ? Nous faire comprendre que l'essentiel {\bf --} le Vrai, le
Bien : l’unité des deux {\bf --} est ailleurs, qui ne se donne qu’à la pensée et à condition
seulement qu’on s’arrache au monde sensible. C’est le mythe idéaliste par
excellence, celui qui donne tort au réel, celui qui dévalue le corps et la sensation,
celui qui ne croit qu’à l’autre monde, qu’à la transcendance, qu’aux Idées
{\bf --} qu'à la mort. Le succès de cette fable, chez les philosophes, en dit long sur
leur dégoût ordinaire du réel.

\section{Cercle}
%CERCLE
La figure géométrique relève des mathématiques. Quand les philosophes
parlent de cercle, c’est le plus souvent par métaphore,
pour désigner non une figure mais une faute logique (c’est alors un synonyme
de {\it diallèle}). Une pensée est circulaire lorsqu’elle suppose cela même qu’elle prétend
% 104
démontrer. Un bon exemple en est ce qu’on appelle, depuis Arnauld, le
{\it cercle cartésien}. Descartes, pour garantir la validité de nos idées claires et distinctes,
a besoin d’un Dieu tout-puissant non trompeur, dont il entreprend de
démontrer l'existence et la véracité. Mais ses démonstrations ne valent que si
nos idées claires et distinctes sont fiables, autrement dit que si Dieu existe et
n’est pas trompeur : elles supposent donc (pour être valides) cela même qu’elles
prétendent démontrer (l'existence d’un Dieu vérace). C’est en quoi le système,
dans son ensemble, reste indémontrable.

Ce n’est qu’un exemple parmi cent autres. Toute pensée dogmatique est
circulaire, puisqu’elle a besoin, pour démontrer quoi que ce soit, de supposer
d’abord la validité de la raison (autrement dit qu’il y a de vraies démonstrations),
ce qui est par nature indémontrable. Ce n’est pas une raison pour
renoncer à penser. Mais c’en est une pour renoncer au dogmatisme.

\section{Certitude}
%CERTITUDE
Une pensée est certaine quand elle ne laisse aucune place au
doute.

On distinguera la certitude subjective, qui n’est qu’un état de fait (que je
sois incapable, en fait, de douter de telle ou telle proposition, cela ne suffit pas
à prouver qu’elle est vraie), de la certitude objective, qui serait une nécessité
logique ou de droit (si telle proposition, ou telle démonstration, est en elle-même
ou objectivement indubitable). Mais ce second type de certitude, qui
seule serait absolument satisfaisante, suppose en vérité le premier, et c’est pourquoi
elle ne l’est pas. Marcel Conche, en une phrase et à propos de Montaigne,
a dit l’essentiel : « La certitude qu’il y a des certitudes de droit n’est jamais
qu’une certitude de fait. » À la gloire du pyrrhonisme.

\section{Chaîne}
%CHAÎNE
Une métaphore traditionnelle, pour dire une série ininterrompue.
Par exemple lorsque Descartes, dans le {\it Discours de la
méthode}, évoque « ces longues chaînes de raisons toutes simples et faciles, dont
les géomètres ont coutume de se servir ». Ou lorsque Voltaire, dans son {\it Dictionnaire},
s'interroge sur la « chaîne des êtres créés » (dont il conteste la continuité)
ou sur la « chaîne, des événements », qui interdit qu'aucun puisse se produire
sans cause ou ne se produire pas : ce n’est qu’un autre nom pour dire le
destin, la fatalité ou le déterminisme (« une suite de faits qui paraissent ne tenir
à rien, et qui tiennent à tout »). On parle aussi de la « chaîne infinie des causes
finies » pour désigner et résumer à la fois interminable proposition 28 du
livre I de l'{\it Éthique} de Spinoza (qui n'utilise pas l'expression) : « Une chose singulière
quelconque, autrement dit toute chose qui est finie et a une existence
% 105
déterminée, ne peut exister et être déterminée à produire quelque effet si elle
n’est déterminée à exister et à produire cet effet par une autre cause qui est elle-même
finie et a une existence déterminée ; et à son tour cette cause ne peut non
plus exister et être déterminée à produire quelque effet si elle n’est déterminée
à exister et à produire cet effet par une autre qui est aussi finie et a une existence
déterminée, et ainsi à l’infini. » Cela suppose une cause première, mais non pas
dans le temps (comment la cause de la chaîne serait-elle un chaînon ?) : cela
suppose la puissance éternelle et infinie de la nature. Non un chaînon de plus,
mais la force immanente et incréée {\bf --} causa sui {\bf --} qui les produit, les anime, les
unit. Non la {\it Nature naturée} (la chaîne infinie des causes finies : celles qui sont
aussi des effets), mais la {\it Nature naturante} (non les modes finis, mais les attributs
infinis de la substance : voir {\it Éthique} I, 29, scolie).

\section{Chance}
%CHANCE
C'est le tout-venant du destin, quand il est favorable. Vivre en est
une, et la première.

L'erreur est de croire qu’on puisse posséder la chance (quand c’est elle qui
nous possède), ou la mériter (quand tout mérite la suppose). Inutile même de
remercier. La chance n’est qu’un hasard qui réussit, et qui réussit par hasard.

Mais aucun bonheur, sans elle, ne serait possible : le destin est Le plus fort,
toujours, et seule la chance nous donne parfois le sentiment du contraire. Aie
donc le bonheur modeste, ou le malheur serein. Ni l’un ni l’autre ne sont
mérités.

\section{Changemement}
%CHANGEMENT
Le devenir ou la puissance en acte : le passage d’un lieu à
un autre (le mouvement local d’Aristote), d’un état à un
autre, d’une forme ou d’une grandeur à une autre... Dire que « tout passe et
que rien ne demeure », comme fait Héraclite (fr. A 6), c’est dire aussi que tout
change ({\it panta rhei} : tout s’écoule) et par là constater l’impermanence de tout.
Seul un Dieu pourrait faire exception. Mais s’il ne changeait jamais, ce ne serait
qu’un Dieu mort. Autant prier un morceau de bois.

Nihilisme ? Non pas. Car il faut tenir bon, ici, sur l’unité des contraires : ce
qui change, c’est ce qui demeure. Qui dit changement dit en effet succession
d’au moins deux états différents, pour un même objet {\bf --} ce qui suppose que
l’objet, lui, continue d’exister. Ou s’il disparaît totalement, ce n’est plus lui qui
change (puisqu'il n’existe plus) mais ses éléments ou le monde (qui existent
toujours). Ainsi le changement suppose l'identité, la durée, le maintien dans
l’être de cela même qui se transforme. Une substance ? Pas forcément, si l’on
entend par là quelque chose d’immuable (il se pourrait que {\it tout} change). Mais
% 106
assurément si l’on entend par substance {\it ce} qui change (le sujet ou le support du
changement). Considérons par exemple ce bateau dont on remplace progressivement
toutes les pièces : pas un atome ne demeure de l’ensemble initial, mais
le bateau lui-même n’a changé que pour autant qu’on le suppose subsister (par
sa structure, sa fonction, son nom, sa présence continuée...). Cela vaut pour
tout être, pour tout ensemble, pour tout processus. Un pays, un parti politique,
une entreprise ou un individu ne peuvent changer qu’à la condition de subsister,
au moins partiellement. Je ne peux changer qu’à la condition de rester
moi. Et si tout change dans l’univers, c’est que l’univers, lui, continue d’exister.
À la gloire de Parménide.

Donc il faut durer pour changer. Mais la réciproque est vraie aussi : il faut
changer pour durer. Dans un monde où tout change, l’immuabilité est impossible
ou serait mortifère. Un pays, un parti ou une entreprise ne peuvent se
maintenir qu’à la condition toujours de s’adapter. Un individu ne peut rester
lui-même qu’à la condition d’évoluer, fût-ce à contrecœur ou le moins possible.
Vivre c’est grandir ou vieillir {\bf --} deux façons de changer. À la gloire
d’Héraclite : tout change, tout coule, rien ne demeure que l’universel devenir.

Le changement est plus ou moins perceptible, pour nous, en fonction de
son ampleur et de sa vitesse. Mais le moindre ou le plus lent n’en est pas moins
changement pour autant. Montaigne, magnifiquement : « Le monde n’est
qu’une branloire pérenne. Toutes choses y branlent [changent] sans cesse : la
terre, les rochers du Caucase, les pyramides d'Égypte, et du branle public et du
leur. La constance même n’est autre chose qu’un branle plus languissant »
({\it Essais}, III, 2, p. 804-805). Ainsi le changement est la loi de l’être (par quoi il
ne fait qu’un avec le devenir), et c’est la seule chose peut-être qui ne change
pas : que tout change, c’est une vérité éternelle.

\section{Chaos}
%CHAOS
Désigne un état de complet désordre, qu’on suppose souvent originel
et qui pourrait aussi bien être ultime. Pourquoi l’ordre aurait-il le dernier mot ?

Dans le langage scientifique contemporain, on appelle {\it chaotique} tout système
dont une modification infime des conditions initiales suffit à modifier
considérablement l’évolution, de telle sorte que celle-ci échappe, en pratique, à
toute prévision à long terme. C’est donc un système à la fois déterministe (en
théorie) et imprévisible (en pratique) : la connaissance que nous pouvons avoir
de son état actuel ne sera jamais assez précise, en fait, pour permettre de prévoir
ses états un peu éloignés. C’est l'{\it effet papillon} cher à nos météorologues : le battement
d’une aile de papillon au Mexique peut déclencher de loin en loin une
tempête en Europe. Ou l'effet {\it nez de Cléopâtre}, cher à Pascal : « S’il eût été plus
% 107
court, toute la face de la terre aurait changé » ({\it Pensées}, 413-162). On remarquera
que les phénomènes chaotiques, pour imprévisibles qu’ils soient à une
certaine échelle, ne sont pas pour autant irrationnels. Les théories du chaos sont
une victoire, bien plus qu’une défaite, de la raison humaine : elles nous permettent
de comprendre qu’on ne peut pas tout expliquer, ni tout prévoir.

\section{Charité}
%CHARITÉ
L'amour désintéressé du prochain. Cela tombe bien : le prochain
n’est pas toujours intéressant.

Comme le prochain, par définition, c’est n’importe qui, la charité, dans son
principe, est universelle. C’est ce qui la distingue de l’amitié, qui ne va pas sans
choix ou préférence (Aristote : « Ce n’est pas un ami, celui qui est l’ami de
tous »). On choisit ses amis ; on ne choisit pas son prochain. Aimer ses amis, ce
n'est pas aimer n'importe qui, ni les aimer n’importe comment : c’est les préférer.
La charité serait plutôt une {\it dilection} sans {\it prédilection}. On ne la confondra
pas avec la philanthropie, qui est l’amour de l'humanité, autrement dit d’une
abstraction. La charité ne porte que sur des individus, dans leur singularité,
dans leur concrétude, dans leur fragilité essentielle. C’est aimer n’importe qui,
mais en tant qu’il est quelqu'un ; c’est se réjouir de l’existence de l’autre, quoi
qu’il soit mais tel qu’il est.

Ce qui nous en sépare est le {\it moi}, qui ne sait aimer que soi (égoïsme) ou
pour soi (concupiscence). Cela indique le chemin. « Aimer un étranger comme
soi-même, écrit Simone Weil, implique comme contrepartie : s’aimer soi-même
comme un étranger. » On a raison de dire que {\it Charité bien ordonnée
commence par soi-même}, mais on le prend ordinairement à contresens. La charité
commence quand on cesse, si c’est possible, de {\it se préférer}.

\section{Chasteté}
%CHASTETÉ
On ne la confondra pas avec la continence, qui n’est qu’un
état de fait. La chasteté serait plutôt une vertu : celle qui
triomphe de la concupiscence, spécialement dans le domaine sexuel. Mais faut-il
en triompher ? Et le peut-on autrement ou mieux que par la satisfaction réitérée
du désir ? Voyez comme nous sommes chastes après l’amour. C’est apprivoiser
la bête plutôt que la dompter, et cela vaut mieux.

\section{Choix}
%CHOIX
C'est un acte de la volonté, qui se porte sur tel objet plutôt que sur
tel ou tel autre. Ce choix est-il libre ? Oui, en tant qu’il dépend de
nous. Non, en tant, précisément, qu’il en {\it dépend}. Tout choix suppose un sujet
qui choisit, et qu’on ne choisit pas. Essayez, pour voir, d’être quelqu'un
% 108
d’autre... Ainsi aucun choix n’est absolument libre ; s’il l'était, on ne pourrait
plus choisir.

\section{Chose}
%CHOSE
Un morceau quelconque du réel, mais considéré dans sa durée,
dans sa stabilité au moins relative (c’est ce qui distingue la {\it chose} du
{\it processus} où de l'{\it événement}), et dépourvu, du moins en principe, de quelque
personnalité que ce soit (c’est ce qui distingue la {\it chose} du {\it sujet}).

{\it Chose} dit moins que {\it substance} (qui suppose chez la plupart des auteurs la
permanence et l'indépendance : la substance serait une chose absolue, la chose
une substance relative), moins qu’{\it objet} (qui n’est objet que pour un sujet),
moins même qu'être (où l’on entend davantage l’idée d’unité : « ce qui n'est
pas véritablement {\it un} être, disait Leibniz, n’est pas non plus véritablement un
{\it être} »), enfin, et malgré l’étymologie, moins que {\it cause} (qui serait une chose agissante
ou produisant quelque effet). {\it Chose} ne dit presque rien, et c’est pourquoi
le mot est aussi commode qu’insatisfaisant. Le silence, presque toujours, vaudrait
mieux.

On ne parle ordinairement de {\it choses} qu’à propos d’êtres inanimés. Cet
usage (même s’il est, philosophiquement, quelque peu hésitant : le {\it Cogito}, chez
Descartes, se saisit comme « une chose qui pense ») s'impose spécialement dans
l’ordre éthique et juridique. C’est ce qui autorise, avec Kant, à distinguer la
{\it chose} de la {\it personne}. La chose, qui n’a ni droits ni devoirs, peut être possédée
par telle ou telle personne : ce n’est qu’un moyen, pour qui veut ou peut s’en
servir. Une personne, au contraire, ne saurait être légitimement réduite au seul
rang de moyen : c’est une fin en soi, qui a des droits et des devoirs, et que nul
ne peut posséder. Une chose peut avoir une {\it valeur}, qui est l’objet possible d’un
échange. Seule une personne a une {\it dignité}, qui est l’objet nécessaire d’un respect.
Une chose peut avoir un prix. Une personne n’en a pas {\bf --} sinon pour
autant qu'on la considère, ou qu’elle se considère, et toujours indûment,
comme une chose.

Cela pose la question des bêtes, qui ne sont ni des personnes (elles ne sont
sujets ni du droit ni de la morale) ni pourtant tout à fait des choses, au sens
ordinaire du terme (puisqu'elles sont douées de sensibilité, voire de conscience
ou de personnalité). Une {\it chose}, en ce sens strict, serait alors ce qui n’est ni bête
ni esprit : un morceau inanimé du réel.

\section{Chose en soi}
%CHOSE EN SOI
Une chose, telle qu’elle est en elle-même, indépendamment
de la perception ou de la connaissance que nous
pouvons en avoir, et spécialement, chez Kant, indépendamment des formes
% 109
{\it a priori} de la sensibilité (l’espace et le temps) et de l’entendement (les catégories).
C’est une réalité absolue, non telle qu’elle apparaît (ce n’est pas un phénomène),
mais telle qu’elle est. On peut penser aux monades de Leibniz ou aux
Idées de Platon, mais ce ne sont, sauf à retomber dans le dogmatisme, que des
analogies. La chose en soi est par définition inconnaissable : dès qu’on la
connaît, elle n’est plus {\it en soi} mais {\it pour nous}. On peut pourtant la penser, et
même on le doit ({\it C. R. Pure}, Préface de la 2$^\text{e}$ éd.). S’il n’y avait pas de choses
en soi, comment y aurait-il des choses pour nous ?

Non, pourtant, que la chose en soi, chez Kant, ne soit que le simple corrélat
objectif et indéterminé de nos représentations ({\it l'objet transcendantal} = x), ni
l’objet d’une éventuelle, et pour nous impossible, intuition intellectuelle (le
{\it noumène}). Elle serait plutôt ce qu’il faut supposer pour que ces deux notions
puissent se rejoindre, au moins pour la pensée : ce serait la cause intelligible
(non phénoménale) du phénomène, ou plutôt {\bf --} puisque la notion de {\it cause} ne
peut elle-même s'appliquer légitimement qu’aux objets d’une expérience possible {\bf --}
ce serait « la même réalité que le phénomène, mais en tant qu’elle
n'apparaît pas aux sens et n’est pas modifiée par l’espace et le temps » (Jacques
Rivelaygue, {\it Leçons de métaphysique allemande}, II, p. 142). La notion est par
nature mystérieuse. Une chose en soi, nous dit Kant, n’est ni spatiale ni temporelle.
Mais puisqu'on ne peut absolument pas la connaître, c’est là une affirmation
sans preuve. Pourquoi l’espace et le temps, qui sont les formes de la
sensibilité, ne seraient-ils pas, aussi, des formes de l'être ? Le kantisme reste un
dogmatisme, aussi douteux que tous les autres.

\section{Chrétien}
%CHRÉTIEN
Ce n’est pas seulement un disciple du Christ (Spinoza, sinon,
serait chrétien). Être chrétien, c’est croire en la {\it divinité} du
Christ. Croyance improbable, presque inconcevable (comment un homme
serait-il Dieu ?), et que rien, même dans les Évangiles, n’atteste absolument.
Jésus était un Juif pieux. J'ai quelque peine à imaginer qu’il ait pu se prendre
pour Dieu, et au reste, dans les Évangiles, il n’y prétend nulle part expressément.
Mais quand bien même cela serait, que prouve une croyance ?
Jésus n’était pas chrétien (il était juif ou Dieu). Pourquoi devrions-nous
l'être ? Parce que nous ne sommes pas juifs ? Parce que nous ne sommes pas
Dieu ? Je préfère, dans les deux cas, m’en consoler autrement.

\section{Christianisme}
%CHRISTIANISME
L’une des trois religions du livre : c’est la foi d'Abraham,
quand elle croit avoir trouvé le Messie. Le christianisme
est un judaïsme satisfait, et c’est ce qui le rend insatisfaisant.

% 110
C’est aussi la religion de l’amour. C’est ce qui rend le christianisme aimable
et suspect. Est-ce Dieu qui est amour, ou l'amour qui est Dieu ? Transcendance,
ou sublimation ? « Les hommes aiment tellement la vérité, écrivait saint
Augustin, qu’ils voudraient que ce qu’ils aiment soit vrai. » Et qu’aimons-nous
davantage que l'amour ?

Ainsi le succès du christianisme ne s’explique que trop bien. Heureusement
qu’il y a l’Église. La tentation, autrement, serait trop forte.

\section{Ciel}
%CIEL
Le monde visible, mais au-dessus de nous (en grec, {\it kosmos} et {\it ouranos},
le monde et le ciel, sont souvent synonymes). Les Anciens l'imaginaient
peuplé de dieux, qui seraient les astres. L’usage est resté, mais n’est plus
qu'une métaphore. Cela ne signifie pas qu’il soit toujours vide de sens. Commentant
le {\it Notre-Père}, Simone Weil attache au contraire une grande importance
au fait que Dieu, selon la prière, soit au ciel : « C’est le Père qui est dans
les cieux. Non ailleurs. Si nous croyons avoir un Père ici-bas, ce n’est pas lui,
c'est un faux Dieu » (« À propos du {\it Pater} », {\it Attente de Dieu}, p. 215). Le ciel,
c’est ce qu’on ne peut que regarder, non posséder ou toucher : ainsi les étoiles,
la mort ou Dieu.

\section{Cité}
%CITÉ
L'ensemble des individus soumis à une même loi, qui est celle du
souverain. C’est donc le pouvoir qui définit la cité, non l’inverse. De
là les guerres, les conquêtes, et la résistance qu’elles suscitent. Il s’agit de savoir
qui commande et qui obéit : quels sont ceux qui {\it font la loi}, comme on dit, et
quels sont ceux qui doivent s’y soumettre. La cité est la communauté unifiée de
ceux-ci. Le souverain, de ceux-là. La République, leur identité.

\section{Citoyen}
%CITOYEN
Le membre d’une Cité, en tant qu’il participe au pouvoir souverain
(il ne serait autrement que sujet) et lui est soumis (sans quoi
il serait roi).
Il peut exister des Cités sans démocratie. Mais ce sont des Cités sans
citoyens.

\section{Citoyenneté}
%CITOYENNETÉ
Le propre du citoyen, et spécialement l’ensemble des
droits dont il jouit et des devoirs qui lui incombent. Le
premier devoir est d’obéir à la loi (accepter d’être citoyen, non souverain). Le
premier droit, de participer à son élaboration ou aux rapports de forces qui y
% 111
tendent (être citoyen, non sujet). Deux façons d’être libre, au sens politique du
terme, et on ne peut l'être, dans une Cité, autrement.

\section{Civil}
%CIVIL
Qui relève de la cité ou du citoyen, non de la nature ou de l’État.
Ainsi parle-t-on de l’{\it état civil} (par opposition à l’état de nature), ou
de la {\it société civile} (par opposition à l’État, à l'administration ou aux élus). C’est
la dimension politique, mais non politicienne, de l’humanité.

\section{Civilisation}
%CIVILISATION
Le sens du mot change en fonction de l’article.
{\it La} civilisation est l’ensemble {\bf --} à la fois normatif, évolutif
et hiérarchisé {\bf --} des créations humaines. C’est l’autre de la nature (dont elle fait
pourtant partie) et le contraire de la barbarie.

{\it Une} civilisation est un sous-ensemble de cet ensemble : c’est l’ensemble des
créations humaines (œuvres, techniques, institutions, règles, normes, croyances,
savoirs et savoir-faire.) propre à une société donnée, par quoi elle se distingue
de la nature et des autres sociétés.

Sous l'influence des sciences humaines et spécialement de l’ethnologie, les
deux mots de {\it civilisation} et de {\it culture} sont devenus, de nos jours, à peu près
interchangeables. Si l’on veut continuer à les distinguer, il paraît sage de
réserver {\it culture} à la partie la plus intellectuelle de la civilisation : on est d’autant
plus cultivé qu’on connaît mieux la civilisation dont on fait partie, et celles des
autres.

\section{Civilité}
%CIVILITÉ
La politesse, en tant qu’elle manifeste l’appartenance à une cité
ou à une civilisation. C’est le savoir-vivre-ensemble, y compris,
et peut-être surtout, quand on ne se connaît pas.

\section{Clarté}
%CLARTÉ
Est clair ce qui ne fait pas obstacle au regard (une eau claire) ou à
la pensée (une idée claire). En philosophie, ce qui se comprend
bien, sans autre difficulté à vaincre que la complexité même de la chose (le clair
n’est pas toujours simple) ou la subtilité de la pensée (le clair n’est pas forcément trivial).

La clarté, quand on écrit, est toujours un risque. Mieux on vous comprend,
mieux on peut vous critiquer. C’est pourquoi aussi c’est une vertu : être obscur,
ce serait manquer de politesse, vis-à-vis des lecteurs, ou de courage, vis-à-vis des
adversaires.

% 112
\section{Classe}
%CLASSE
Un ensemble d’éléments dans un ensemble plus vaste, et spécialement
d’individus {\bf --} le plus souvent considérés selon leur métier ou
leurs revenus {\bf --} dans une société. Les classes, en ce sens, sont une abstraction
(seuls les individus existent), mais légitime (l'individu isolé est aussi une
abstraction : il n’existe qu'avec d’autres) et utile. C’est une façon d’y voir plus
clair dans l’infinie complexité du corps social. L'analyse théorique rejoint ici les
exigences de la lutte syndicale ou politique. Diviser pour comprendre ; rassembler
pour agir. Analyse et solidarité.

Le concept doit beaucoup à Marx, qui voyait dans la lutte des classes le
« moteur » de l’histoire et sa réalité {\bf --} depuis la division du travail et jusqu’à
l'avènement du communisme {\bf --} essentielle. Que ce moteur soit le seul est
douteux ; qu’il puisse disparaître l’est encore plus. Pourquoi tous les membres
d’une société auraient-ils les mêmes intérêts ? Comment ceux-ci ne seraient-ils
pas déterminés, au moins en partie, par leur place dans la société ? Comment
échapperaient-ils aux conflits, aux rapports de forces, aux affrontements, aux
compromis ? Si les bourgeois votent si souvent à droite, et d’autant plus qu’ils
sont plus riches, si les ouvriers votent si souvent à gauche, et d’autant plus
qu’ils sont plus organisés et moins xénophobes, on ne me fera pas croire que
c’est seulement par hasard, par habitude ou par aveuglement. Et si les « couches
moyennes » assurent actuellement le triomphe du centre (centre droit, centre
gauche : qu’avons-nous vu d’autre au pouvoir, en France, depuis un quart de
siècle ?), on ne me fera pas croire que c’est seulement par amour de la moyenne
ou du juste milieu.

L'erreur de Marx, ce ne fut pas de parler de lutte des classes ; ce fut plutôt
de vouloir y mettre fin, de rêver d’une société enfin homogène et pacifiée (le
communisme, société sans classes et sans État). Cette utopie fit plus de morts
que la lutte de classe, qui donne des raisons de vivre plutôt que de tuer, et d’agir
plutôt que de rêver.

L'erreur de bien des libéraux, aujourd’hui, est de prétendre que cette lutte
de classes est {\it déjà} terminée, qu’il n’y a plus que des individus {\bf --} tous rivaux,
tous solidaires {\bf --} dans la grande main anonyme et bienfaisante du marché.
Cette anti-utopie tue aussi, mais plus loin (la misère, de nos jours, tue surtout
dans le Tiers-Monde) et plus discrètement. Ce n’est pas une raison pour s’y
résigner.

Être fidèle à Marx {\bf --} y compris contre Marx, et plus encore contre ceux qui
voudraient l’enterrer {\bf --} c’est plutôt penser que la lutte des classes n’aura pas de
fin : qu’il s’agit non de la supprimer mais de l’organiser, mais de la réguler,
mais de l'utiliser. C’est à quoi servent l’État, les syndicats, les partis. Il n’y aura
pas de progrès autrement. À quoi bon un moteur, si ce n’est pour avancer ?

% 113
\section{Classicisme}
%CLASSICISME
L’esthétique des classiques, ou qui s’en inspire. Elle est
caractérisée par un certain idéal d’ordre et de clarté, mais
aussi d'unité, de rationalité, d'équilibre, de discipline, d'harmonie, de simplicité,
de stabilité... C’est l’art {\it apollinien} par excellence : celui qui refuse de
s’abandonner à l'ivresse, aux passions, aux instincts {\bf --} à la démesure des formes
(le baroque) ou des sentiments (le romantisme). Il s’agit de contraindre la
bêtise et l’excès, qui sont naturels à l’homme, par une discipline qui ne l’est pas.
Le classicisme est cette ascèse. C’est un art d’humilité : l’artiste disparaît dans
son œuvre, comme l’œuvre dans le vrai. À la fin il n’y a plus que Dieu, qui disparaît
à son tour.

\section{Classique}
%CLASSIQUE
Étymologiquement, désigne d’abord un auteur de premier ordre
(latin {\it classicus}, de première classe), puis, par une espèce de
contresens pédagogique, un auteur qu’on étudie dans les classes : Victor Hugo
ou Boris Vian sont des {\it classiques}, en ce sens, au même titre que Corneille ou
Racine.

Historiquement, le mot désigne parfois l’Antiquité grecque et latine, considérée
dans sa perfection, mais plus souvent ceux, spécialement en français et au
{\footnotesize XVII$^\text{e}$} siècle, qui s’en réclament.

Esthétiquement, tout ce qui relève du classicisme (voir ce mot).

\section{\it Clinamen}
%{\it CLINAMEN}
Mot latin, qu’on peut traduire par {\it inclinaison}, {\it déclinaison} où
{\it déviation}. Dans l’épicurisme, et spécialement chez Lucrèce,
c’est une infime déviation des atomes, qui les écarte, « en un temps et en un
lieu indéterminés », de la ligne droite ({\it De rerum natura}, II, 216-293). C’est
l'une des trois causes motrices {\bf --} avec le poids et les chocs {\bf --} qui font que les
atomes sont en mouvement. Le {\it clinamen}, contrairement à ce qu’on a souvent
écrit, n’abolit donc pas le principe de causalité. C’est au contraire parce que
« rien ne naît de rien », comme dit Lucrèce, qu’on est obligé d'admettre son
existence : le monde et la liberté, sans lui, seraient impossibles (le monde, parce
que les atomes tomberaient de toute éternité en ligne droite, sans se rencontrer
jamais ; la liberté, parce que chacun serait prisonnier d’un enchaînement
continu de causes). Le {\it clinamen} lui-même n'est-il pas sans cause ? Non plus : il
naît de l'atome, qui ne naît pas (et n’a donc pas besoin de cause). L’indétermination
du clinamen ne porte pas sur son effectuation, mais sur le lieu et le
moment où elle s’effectue. Comme le souligne excellemment Marcel Conche,
« le principe de causalité n’en est pas pour autant contredit, car, comme tel, il
n'implique pas que toute cause doive produire son effet sous des conditions de
% 114
lieu et de temps ». Ce que le {\it clinamen} vient interrompre, ce n’est donc pas la
causalité, mais l’enchaînement {\it continu} des causes, par quoi le présent comme
l'avenir resteraient prisonniers du passé. Chaque déviation appartient à
« l'éternel présent de l'atome », comme dit encore Marcel Conche, sans
qu'aucun passé ne la détermine. C’est en quoi le {\it clinamen} vient briser ce
qu’Épicure appelait « le destin des physiciens »: c’est un pouvoir de commencer
absolument une nouvelle série causale, ce qui est hasard pour les
atomes, et liberté pour les vivants. On remarquera pourtant que cette liberté
reste soumise aux mouvements en nous des atomes, qui ne pensent pas. C’est
ce qui la distingue du libre arbitre, qui serait une liberté absolue.

\section{C{\oe}ur}
%CŒUR
Le siège symbolique de la vie, et spécialement de la vie affective.
L’équivalent en français du {\it thumos} des Grecs (voir notamment
Platon, {\it République}, IV). S’oppose à la tête, lieu de l’intelligence, et au ventre,
lieu des instincts. C’est la métaphore, indissolublement, de l’amour et du courage.
Chez Pascal, l’ordre du cœur ou de la charité s'oppose aux ordres de la chair
et de l'esprit : le cœur a ses raisons que la raison ni le ventre ne connaissent.

\section{Cogito}
%COGITO
{\it Je pense}, en latin. Dans le discours philosophique, c’est presque
toujours une référence au moins implicite à Descartes, et spécialement
au {\it « Cogito, ergo sum »} des {\it Principes de la philosophie} (I, 7), qui traduisait
exactement le fameux « Je pense, donc je suis » du {\it Discours de la méthode},
écrit directement en français. Désigne ordinairement la personne humaine, en
tant que sujet de sa pensée. C’est pourquoi on peut refuser le {\it cogito}, non parce
qu’on nierait l’existence de la pensée (ce serait contradictoire: nier, c’est
penser), mais parce qu’on conteste qu’elle ait nécessairement un sujet. Pour ma
part, d’accord en cela avec Nietzsche, je préférerais la formule {\it Cogitatur} : ça
pense, il y a de la pensée. Du {\it Je pense}, en effet, on ne peut pas plus conclure à
l'existence du {\it Je}, que du {\it Il pleut} on ne peut conclure à l’existence d’un {\it Il}. C’est
où s'opposent les philosophies du sujet (Descartes, Kant, Husserl, Sartre...),
qui prennent le cogito au sérieux, et les autres, dont je suis, qui refusent d’y
croire tout à fait.

\section{Cognitives (sciences {\bf --})}
%COGNITIVES (SCIENCES {\bf --})
Ce sont les sciences ou les disciplines qui
ont la connaissance, et les moyens de la
connaissance, pour objet : ainsi la neurobiologie, la logique, la linguistique,
% 115
l'informatique ou intelligence artificielle, la psychologie, voire la philosophie de
l'esprit, en tant qu’elles s'associent pour essayer de comprendre ce qu'est la
pensée ou comment elle fonctionne. On leur reproche souvent de faire abstraction
du sujet qui connaît, ou de le traiter comme une machine. Mais c’est que
le sujet n’explique rien : il doit être expliqué, et ne peut l’être que par autre
chose que lui-même. Que nos ordinateurs pensent, ce n’est bien sûr qu’une
métaphore. Mais que le cerveau soit une espèce d’ordinateur, c’est davantage
qu’une métaphore : un modèle, qui nous en apprend plus sur la pensée, peut-être
bien, que toute la phénoménologie du monde.

\section{Cognitivisme}
%COGNITIVISME
La philosophie des sciences cognitives, ou ce qui en tient
lieu. C’est considérer la pensée comme un calcul formalisé
(un « processus computationnel »), qui traite l’information, à l’intérieur de
cette espèce d'ordinateur qu'est le cerveau, de façon rationnelle et efficace.
Matérialisme ? Sans doute, mais pensé sur le modèle de l'intelligence
artificielle : la pensée serait le {\it software} (le logiciel et sa mise en œuvre) ; le cerveau,
le {\it hardware} (la machine). C’est moins une mode qu’un modèle.

\section{Cohérence}
%COHÉRENCE
Le fait de se tenir ensemble ({\it co-haerens}), mais en un sens
logique plutôt que physique: est cohérent ce qui est
dépourvu de contradiction. On remarquera que la cohérence ne fait pas preuve,
ou ne prouve qu’elle-même. Des erreurs bien liées et non contradictoires ne
cessent pas pour cela d’être fausses.

\section{Cohésion}
%COHÉSION
C'est un doublet, mais pas tout à fait un synonyme, du précédent.
La cohésion, c’est le fait de se tenir ensemble, mais en un
sens physique plutôt que logique : c’est moins l’absence de contradiction que
l'absence de failles ou d’affrontements. Ainsi parle-t-on de la cohérence d’une
théorie, et de la cohésion d’une société. Cela ne prouve pas que la première soit
vraie, ni que la seconde soit juste. Mais une théorie incohérente est nécessairement
fausse. Et une société sans cohésion, nécessairement injuste.

\section{Colère}
%COLÈRE
Indignation violente et passagère. C’est moins une passion qu’une
émotion : la colère nous emporte, et finit par s’emporter elle-même.
Mieux vaut, pour la surmonter, l’accepter d’abord. Le temps joue pour
nous, et contre elle.

% 116
La colère naît le plus souvent d’un dommage injuste, ou qu’on juge être tel.
Aussi y a-t-il de justes colères, quand elles viennent au secours de la justice.
Mais la plupart ne viennent au secours, hélas, que du narcissisme blessé : désir,
non de justice, mais de vengeance.

Au reste, s’il est de justes et nécessaires colères, il n’en est guère d’intelligentes.
C’est leur faiblesse et leur danger : la plus justifiée peut déboucher
sur injustice. Aurait-on autrement besoin des tribunaux, qui jugent lentement
et sereinement {\bf --} au moins en principe {\bf --} de la colère des hommes ?
Un procès, même médiocre, vaut mieux qu’un bon lynchage. Cela dit à
peu près ce qu’il faut penser de la colère : parfois nécessaire, jamais suffisante.

\section{Collectivisme}
%COLLECTIVISME
Le règne en tout de la collectivité, et spécialement de la
propriété collective. C’est vouloir triompher de l’égoïsme
par la loi. Cela explique que le collectivisme, historiquement, ait mené au totalitarisme.
Il faut bien imposer par la contrainte ce que la morale échoue à
obtenir.

\section{Comédie}
%COMÉDIE
Tout spectacle qui amuse ou fait rire. C’est en quoi la vie est
une comédie, tant qu’on ne la prend ni au sérieux ni au tragique.
Mais s’en amuser ne dispense pas de la vivre.

\section{Comique}
%COMIQUE
L'art de faire rire. On distingue plusieurs types de comique : la
farce, qui fait rire de la bêtise ou bêtement, les jeux de mots,
qui font rire du langage, le comique de caractère, qui fait rire de l'humanité, le
comique de situation, qui fait rire de ce qu’on comprend, le comique de
l'absurde, qui fait rire de l’incompréhensible, le comique de répétition, qui fait
rire du même, l'ironie, qui fait rire des autres, l'humour, qui fait rire de soi et
de tout. Il y a aussi un comique involontaire, qui n’est plus un art mais un
ridicule. Encore n'est-il drôle que par l’art de l’observateur : c’est retrouver
l’humour (si l’on se reconnaît dans le ridicule qu’on perçoit), l'ironie (si l’on ne
s’y reconnaît pas) et Molière.

\section{Communauté}
%COMMUNAUTÉ
Ce qui est commun, et spécialement, chez Kant, l’action
réciproque entre l’agent et le patient. C’est l’une des trois
catégories de la relation (avec l’inhérence et la causalité), en l’occurrence celle
% 117
qui débouche sur la troisième « analogie de l’expérience », dite aussi « principe
de la communauté » : « Toutes les substances, en tant que simultanées, sont
dans une communauté universelle (c’est-à-dire dans un état d’action réciproque) ».

En son sens le plus ordinaire et le plus fort, le mot peut désigner un groupe
quelconque, mais considéré dans ce que ses membres ont en commun : c’est un
ensemble d’individus qui communient au moins en quelque chose.

\section{Communication}
%COMMUNICATION
Échange de signes, de messages, d'informations, entre
deux ou plusieurs individus. Elle ne vaut jamais par
elle-même, mais seulement par son contenu ou son résultat. Une sottise
répandue à des milliers d'exemplaires reste une sottise. Et une idée vraie ou
forte, dans la tête d’un seul, ne cesse pas pour cela de l’être. C’est en quoi l’idée,
si rebattue, de « société de la communication » est inquiétante : c’est accorder
trop d’importance aux médias, pas assez aux messages.

\section{Communion}
%COMMUNION
Un partage sans division. On se répartit un gâteau (chacun
en a d’autant moins que les autres sont plus nombreux ou
en reçoivent davantage). On communie dans le plaisir qu’on prend à le
manger (le plaisir des autres, loin d’amputer le mien, le redoublerait plutôt :
quelle tristesse ce serait que de le manger seul !). On partage un trésor, un
pouvoir, un bureau. On communie dans une connaissance ou un amour (que
nous soyons plusieurs à connaître une même vérité ou à aimer un même individu,
cela ne diminue en rien cette connaissance ni cet amour). C’est pourquoi
l’on parle de la communion des esprits : parce que seul l'esprit sait
partager sans diviser.

\section{Communisme}
%COMMUNISME
Une société sans classes, sans État, sans propriété privée
(au moins pour les moyens de production et d'échange),
qui serait en même temps une société d’abondance et de liberté... Que
demander de plus ? Rien, sinon l’humanité qui va avec.

Le communisme n’a guère le choix qu’entre l’utopie et le totalitarisme : il
faut rêver l’homme ou transformer l'humanité. Illusion, ou bourrage de crâne.
Niaiserie, ou dictature.

Le mot désigne aussi le mouvement politique qui voulait imposer cette
utopie. On lui doit des millions de morts et des milliers de héros.

% 118
\section{Comparaison}
%COMPARAISON
Comparer, c’est associer dans le langage, et par le langage,
deux objets différents : soit pour en souligner les ressemblances
ou dissemblances, soit pour évoquer l’un, poétiquement, par l’invocation
de l’autre. Si cette comparaison reste implicite, il s’agit alors d’une métaphore.
Si l’objet évoqué est abstrait, d’un symbole.

\section{Compassion}
%COMPASSION
C'est souffrir de la souffrance de l’autre. Très proche en
cela de la pitié, mais sans l’espèce de condescendance que
cette dernière, presque inévitablement, comporte ou suggère. La compassion,
dirais-je volontiers, c’est la pitié entre égaux. Très proche aussi, et par là même,
de la {\it commiseratio} spinoziste (que l’on traduit ordinairement par pitié : « une
tristesse qu’accompagne l’idée d’un mal arrivé à un autre, que nous imaginons
être semblable à nous », {\it Éthique}, III, déf. 18 des affects), mais plus encore, à ce
que je crois, de la {\it misericordia}, qu’il vaut mieux traduire par {\it compassion} ou par
{\it sympathie} que par {\it miséricorde} (« un amour, en tant qu’il affecte l’homme de
telle sorte qu’il se réjouisse du bien d’autrui et soit affligé par le mal d’autrui »,
{\it ibid.}, déf. 24 des affects). Entre ces deux notions, Spinoza, curieusement, ne
perçoit guère de différence, « sinon peut-être que la pitié concerne un affect
particulier, et la compassion une disposition habituelle à l’éprouver » ({\it ibid.},
déf. 18, explic.). Je dirais plutôt que la pitié est un sentiment, qui ne connaît
que la tristesse, quand la compassion est une vertu, qui sait aussi se réjouir.
Voyez les larmes du Bouddha, quand il découvre la souffrance, et son sourire,
quand il enseigne à s’en libérer. C’est la grande vertu de l'Orient bouddhiste ;
elle ressemble à la charité des chrétiens, en plus réaliste et en mieux attestée.

\section{Complexe}
%COMPLEXE
Comme adjectif, qualifie tout ensemble qui résiste, du fait de
sa composition, à une compréhension immédiate (avec une
nuance plutôt emphatique, qui le distingue de « compliqué »). Comme nom,
peut désigner cet ensemble lui-même. Le mot sert surtout, en psychologie ou
en psychanalyse, pour désigner un groupe de représentations conscientes ou
inconscientes (désirs, fantasmes, traumatismes), « liées entre elles, comme dit
Freud, et chargées d’affect ». Ainsi le complexe d'Œdipe ou celui de castration.
Ce n’est pas une maladie ; c’est une structure de la personnalité.

Au pluriel et dans le langage courant («avoir des complexes »), désigne le
plus souvent un sentiment d’infériorité ou d’insatisfaction.

En tous ces sens, le mot s’oppose à la simplicité, qui est un état de fait, une
conquête ou une vertu.

% 119
\section{Comportement}
%COMPORTEMENT
Façon d’agir ou de réagir, en tant qu’elle peut être
saisie de l’extérieur. S’oppose à {\it motif} ou à {\it motivation},
et plus généralement à tout ce qui ne peut être saisi que subjectivement ou de
l’intérieur. Ainsi, chez Pascal : « Pour connaître si c’est Dieu qui nous fait agir,
il vaut bien mieux s’examiner par nos comportements au-dehors que par nos
motifs au-dedans ». Depuis Watson et Piéron, on appelle « psychologie du
comportement » (ou parfois {\it béhaviorisme}, de l'américain {\it behavior}, conduite,
comportement) celle qui, se voulant objective, ne se fonde que sur des données
observables de l'extérieur. Aussi s’interdit-elle tout recours à l’introspection,
aux sentiments, voire à la conscience elle-même : elle ne connaît que des {\it stimuli}
et des réactions. C’est vouloir saisir l’esprit du dehors {\bf --} par le corps.

\section{Compossible}
%COMPOSSIBLE
Ce qui est possible avec ou ensemble. Deux événements
sont {\it compossibles} s'ils peuvent se produire, fût-ce à des
moments différents, dans un même monde. La notion est importante, spécialement
chez Leibniz, pour comprendre l’existence du mal. Dieu, quoique tout-puissant
et souverainement bon, ne choisit pas toujours les événements qui
seraient, à les considérer séparément, les meilleurs possibles (par exemple Hitler
mourant à la naissance, ou ne s’occupant que de peinture, ou se convertissant
au judaïsme...). C’est qu’il est soumis à sa propre raison, qui est la raison
même, donc au principe de non-contradiction : il ne peut choisir que les
meilleurs {\it compossibles}, autrement dit que des événements qui peuvent exister
dans un même monde et former ensemble (le monde, étant la totalité des
choses contingentes, ne peut être qu’unique) le meilleur de tous. Un monde
sans Auschwitz aurait sans doute été possible, comme un monde sans cancers,
sans guerres, sans imbéciles. Mais tous ces mondes auraient été pires que le
nôtre. Qu’en savons-nous ? C’est que Dieu, autrement, aurait créé l’un de
ceux-là, plutôt que celui-ci. Argument imparable, puisque {\it a priori}. Mais aussi
sans portée: puisqu'il expliquerait aussi bien n’importe quel monde (par
exemple un monde où Hitler aurait gagné la guerre). Et obscène, puisqu'il
amène à justifier les horreurs de détail (les maladies, les massacres, les tremblements
de terre...) au nom de l’indépassable harmonie, ou supposée telle, de
l’ensemble. Un enfant qui meurt ? Ce n’est qu’un peu de noir, dans un coin du
tableau, pour rehausser, par contraste, les couleurs et la parfaite luminosité de
l’ensemble. Une famine ? Ce n’est qu’une dissonance placée où il faut, qui
donne du relief à l’harmonie ({\it Théodicée}, I, \S 12). Pourquoi Dieu a-t-il choisi ce
monde-là ? Parce que c'était, de toute éternité, le meilleur arrangement d’événements
compossibles {\bf --} le meilleur monde. Comment savons-nous qu’il est le
meilleur ? Parce que Dieu l’a choisi ({\it ibid.}, \S 8 à 10). C’est une surprise toujours
% 120
renouvelée que de voir cet immense génie qu’est Leibniz {\bf --} peut-être le plus
grand qui fût jamais {\bf --} tourner en rond dans son petit cercle optimiste. C’est
que le génie ne peut rien contre la foi, alors que la foi, contre le génie, peut
encore quelque chose {\bf --} le bercer doucement, jusqu’à ce qu’il s’endorme.

\section{Compréhension}
%COMPRÉHENSION
Le fait de comprendre ou de contenir. Spécialement,
en logique ou en linguistique, l’ensemble des caractères
communs aux individus d’une même classe, qui vont servir à en définir le
concept. S’oppose alors à {\it extension}. Par exemple, définir le concept de
« mammifère » {\it en compréhension} (les linguistes disent parfois {\it en intension}), c’est
énumérer les caractéristiques qui justifient l’appartenance à ce groupe : animal
vertébré (c’est le genre prochain), pourvu de mamelles, d’une peau souple, d’un
cœur à quatre cavités, respirant par des poumons, à reproduction vivipare (sauf
les monotrèmes), à température constante... Définir le même concept {\it en extension},
ce serait donner la liste de toutes les espèces de mammifères (qu’il faudrait
alors définir en compréhension), voire de tous les mammifères (si l’on voulait
s’en tenir à une stricte extension). La compréhension fait gagner du temps : elle
seule, presque toujours, permet de définir.
On remarquera que plus la compréhension d’un concept est riche, plus son
extension est pauvre, et inversement : il y a plus dans le concept de mammifère
que dans celui de vertébré, mais plus de vertébrés que de mammifères.

\section{Comprendre}
%COMPRENDRE
Saisir intellectuellement, autrement dit par la pensée :
c’est connaître comme de l’intérieur, par sa structure ou
par son sens, l’objet qu’on considère. C’est donc savoir comment c’est fait,
comment cela fonctionne, ce que cela veut dire, et être capable d’en rendre
raison.

On distingue parfois l'{\it explication}, qui donnerait les causes et connaîtrait du
dehors, de la {\it compréhension}, qui donnerait le sens et connaîtrait du dedans
(« par interpénétration psychologique », disait Jaspers). Les sciences de la
nature relèveraient alors de l’explication ; les sciences humaines, du moins certaines
d’entre elles, de la compréhension. Je ne suis pas sûr qu’une telle distinction
puisse être pensée jusqu’au bout, mais elle indique au moins une direction,
ou plutôt deux : comprendre un texte, c’est savoir ce que son auteur voulait
dire (son sens). L’expliquer, ce serait savoir pourquoi (par quelles causes) il a été
amené à l'écrire, et de cette façon plutôt que d’une autre. Comprendre un
délire, ce serait saisir son sens. L’expliquer, ce serait connaître ses causes. Les
deux démarches sont légitimes, mais point interchangeables. Le sens n’est
% 121
jamais cause, toujours effet. Expliquer peut permettre de comprendre ; comprendre
n’a jamais suffi à expliquer.

Appliquée aux comportements humains, la compréhension s'accompagne
ordinairement de miséricorde : « Juger, c’est de toute évidence ne pas comprendre,
disait Malraux, puisque si l'on comprenait, on ne pourrait plus
juger. » C’est l'esprit de Spinoza, et peut-être l'esprit tout court : {\it « Non ridere,
non lugere, neque detestari, sed intelligere »} (ne pas railler, ne pas déplorer, ne pas
détester, mais comprendre).

Cela toutefois ne dispense pas de juger, pour agir. Mais interdit de réduire
le jugement à la compréhension. Distinction des ordres : la compréhension
relève de la vérité ; le jugement, de la valeur. Comprendre un fou, ce n’est pas
une raison pour déraisonner avec lui, ni pour renoncer à la santé, ni pour
cesser, le cas échéant, de s’en protéger ou de l’empêcher de nuire. Comprendre
un raciste, cela ne dispense pas de le combattre ; cela permet de le combattre
plus intelligemment, plus efficacement, et avec moins de haine.

\section{Compulsion}
%COMPULSION
Une impulsion irrésistible, presque toujours pathologique.

\section{\it Conatus}
%{\it CONATUS}
La puissance de vivre ou d’exister. Ce n’est pas l’être en puissance,
mais la puissance de l'être, en tant qu’elle est toujours en acte.

Le mot, en latin, signifie {\it effort}, {\it tendance}, {\it poussée} ou {\it pulsion}. Dans la langue
philosophique (et même si on le trouve aussi chez Hobbes, Descartes ou Leibniz),
il s’est surtout imposé dans l’acception que lui donne Spinoza : le {\it conatus}
d’un être quelconque, c’est son effort pour persévérer dans son être, autrement
dit sa puissance d’exister, de résister et d’agir ({\it Éthique}, III, prop. 6 et dém.).
C’est son essence actuelle ({\it ibid}., prop. 7), qui prend la forme, pour nous, du
désir ({\it ibid}., prop. 9 et scolie).

S’il fallait traduire {\it conatus} en grec, on hésiterait entre {\it hormè} (la tendance)
et {\it energeia} (la force en action, la puissance en acte). En français, entre {\it effort},
{\it pulsion} (voire « pulsion de vie»), {\it puissance} où {\it énergie}. C’est pourquoi on
renonce ordinairement à le traduire, et l’on a {\bf --} pour une fois {\bf --} raison. Le mot,
dans son usage philosophique, a cessé d’être latin ; il n’est plus que spinoziste.

\section{Concept}
%CONCEPT
C'est Simone de Beauvoir, si mes souvenirs sont exacts, qui
raconte comment Sartre et Merleau-Ponty, étudiants, s’amusaient
à inventer des sujets de dissertation impossibles ou caricaturaux. L'un
d'eux surtout, j'étais en terminale, m'avait à la fois amusé et effrayé : {\it « Le
% 122
concept de notion et la notion de concept »}. C’est que les deux concepts sont tellement
proches qu’on pourrait y voir une même notion : ce seraient deux synonymes,
pour désigner une idée abstraite ou générale.

S’il faut les distinguer, comme la langue y invite, on considérera que la
notion est ordinairement plus vague ou plus vaste, le concept plus précis ou
plus strict. On parlera par exemple de la notion d’animal, et du concept de
mammifère. Ou de la notion de liberté, et du concept de libre arbitre. Le
concept, au sens où je le prends, a une compréhension plus riche {\bf --} donc une
extension moindre {\bf --} que la notion. C’est qu’il est souvent une notion précisée
ou rectifiée.

D’autres différences en découlent. La notion est déjà donnée ; le concept,
produit. La notion est le résultat d’une certaine expérience ou d’une certaine
éducation (la {\it prolèpsis} des Grecs) ; le concept, d’un certain travail. Toute
notion est commune (elle n'appartient qu’à la langue ou à l'humanité) ; tout
concept, singulier (il n’a de sens qu’à l’intérieur d’une certaine théorie). Une
notion est un fait; un concept, une œuvre. On parlera par exemple de la
notion de justice ; et du concept de justice chez Platon. De la notion de force ;
et du concept de force dans la mécanique classique ; de la notion de concept ;
et du concept de notion chez Kant.

Ainsi le concept {\bf --} qu’il soit scientifique ou philosophique {\bf --} est une idée
abstraite, définie et construite avec précision : c’est le résultat d’une pratique et
l'élément d’une théorie.

\section{Conceptualisme}
%CONCEPTUALISME
L’une des trois façons traditionnelles de résoudre le
problème des universaux : le conceptualisme affirme
que les idées générales n'existent que dans l’esprit qui les conçoit (contre le réalisme,
qui voudrait qu’elles existent en elles-mêmes ou dans l'absolu), mais y
existent bien, à titre d’entités mentales, et ne sauraient se réduire aux mots qui
servent à les désigner (contrairement à ce que prétend le nominalisme). C'était
la doctrine, par exemple, d’Abélard ou de Locke. Mais beaucoup de nominalistes,
et dès Guillaume d’Occam, lui empruntent aussi quelque chose. Dès
qu’un concept est {\it pensé}, n’est-ce pas déjà davantage qu’un mot ?

\section{Concile}
%CONCILE
Une assemblée générale des évêques, sous la direction du premier
d’entre eux. Voltaire, dans l’article qu’il leur consacre,
évoque ironiquement les plus fameux : les conciles de Nicée, d’Éphèse, de
Constantinople, de Latran, de Trente... Tous décident souverainement de
l’indécidable (la consubstantialité du Père et du Fils, la divinité du Saint-Esprit,
% 123
la transsubstantiation...). C’est qu’ils sont infaillibles. Qu’en sait-on ? C’est
qu’ils l'ont décidé, avant d’attribuer l’infaillibilité aussi au Pape. Le dogmatisme
fait système. Même Vatican II, si ouvert, si courageux à bien des égards,
n’y changera rien. « Tous les conciles sont infaillibles, disait Voltaire, car ils
sont composés d’hommes. » Que cette explication vaille comme réfutation,
c'est ce que chacun comprend. Mais les conciles préfèrent s'occuper de ce
qu’on ne comprend pas.

\section{Concorde}
%CONCORDE
Une paix librement acceptée, partagée, et comme intériorisée :
non la simple absence de guerre, mais la volonté commune de l'empêcher.

C’est comme une paix qui serait une vertu, comme une vertu qui serait
collective : c’est la vertu des pacifiques, quand ils sont entre eux, ou leur victoire.

On peut imposer la paix ; la concorde, non : on ne peut que la préparer,
l’entretenir, la préserver, et c’est pourquoi il le faut.

\section{Concret}
%CONCRET
Tout ce qui n’est pas séparé du réel par l’abstraction : ce peut
être le réel lui-même (un corps est toujours concret) ou une
certaine façon de l’appréhender, soit par les sens (le concret c’est alors ce qui se
touche, se voit, se sent...), soit même par la pensée, si elle semble éviter tout
recours à quelque théorie ou idée générale que ce soit. En ce dernier sens, c’est
presque inévitablement une illusion : ce serait penser sans mots, sans concepts,
sans opérateurs logiques {\bf --} sans penser. Il n’y a pas de pensée concrète : il n'y a
que de bonnes ou de mauvaises abstractions, selon qu’elles permettent ou non
de comprendre et d’agir.

\section{Concupiscence}
%CONCUPISCENCE
C’est à tort qu’on la confond aujourd’hui avec le
désir sexuel, qui n’en est qu’un cas particulier. La tradition
appelle {\it concupiscence}, en un sens à la fois plus vaste et plus précis, tout
amour égoïste ou intéressé : c’est n’aimer l’autre que pour son bien à soi. La
concupiscence est donc la règle (si j'aime le poulet, ce n’est pas pour le bien du
poulet), aussi sûrement que l’amour de bienveillance {\bf --} aimer l’autre pour son
bien à lui {\bf --} est l’exception. Il arrive pourtant que les deux se mêlent, spécialement
dans la famille et le couple. Si j’aime mes enfants, ce n’est pas seulement
pour mon bien à moi. Et comment ne pas vouloir du bien à celui ou celle qui
nous en fait ?

% 124
La concupiscence est première : elle est l’amour qui prend, et nul n’apprendrait
autrement à donner. Encore faut-il apprendre. Tout commence par la
concupiscence, mais ce n’est qu'un commencement.

\section{Condition}
%CONDITION
Moins qu’une cause, plus qu’une circonstance. C’est une
circonstance nécessaire ou une cause non suffisante : ce sans
quoi le phénomène considéré ne se produirait pas, mais qui ne suffit pas pour
autant à l'expliquer. Ainsi l’existence de nos parents est condition de la nôtre
(nous n’aurions pu exister sans eux), mais point sa cause (ils auraient pu exister
sans nous). On remarquera qu'aucune cause n'étant jamais strictement suffisante,
il n’y a en vérité que des conditions, qui sont toutes conditionnées avant
d’être conditionnantes.

C’est ce qui autorise à parler de {\it condition humaine}, pour désigner
l’ensemble des circonstances qui s'imposent à tout être humain, sans lesquelles
il ne serait pas ce qu’il est : Le corps, la finitude et la mortalité font partie de la
condition humaine.

\section{Conditionnel}
%CONDITIONNEL
Qui dépend d’une condition ou en énonce. Par exemple
une proposition conditionnelle ou hypothétique : « Si
Socrate est un homme, il est mortel. » On remarquera que ce type de proposition
est vraie, ou peut l’être, que la condition soit ou non avérée (vous pouvez
remplacer « Socrate », dans l'exemple précédent, par tout ce que vous voudrez,
homme ou pas, la proposition n’en perdra pas sa vérité).

\section{Confession}
%CONFESSION
C’est avouer une faute, dans l'espoir d’un pardon. Aveu
intéressé, donc, et pour cela toujours suspect.

\section{Confiance}
%CONFIANCE
C’est comme une espérance bien fondée ou volontaire, qui
porte moins sur l’avenir que sur le présent, moins sur ce
qu’on ignore que sur ce qu’on connaît, moins sur ce qui ne dépend pas de nous
que sur ce qui en dépend (puisqu'on est maître au moins de sa confiance : on
choisit ses amis et ses combats). Cela n'empêche ni l'erreur ni la déception,
mais vaut mieux pourtant que l’espérance aveugle ou la suspicion généralisée.
C’est aussi comme une foi, mais en acte, et qui porterait moins sur Dieu
que sur autrui ou sur soi. Foi en l'Homme ? Ce serait bêtise ou religion. Foi,
% 125
plutôt, en tel ou tel que l’on connaît, et d’autant plus qu’on le connaît
davantage : ce n’est plus foi mais confiance. Son lieu naturel est l'amitié.

\section{Confidence}
%CONFIDENCE
C'est dire à quelqu'un, sur soi-même (sans quoi ce n’est plus
confidence mais indiscrétion), ce qu’on ne dirait pas à
n'importe qui : marque de confiance, d’amour ou d'intimité. Se distingue de
l’aveu, parce qu’elle ne suppose aucune culpabilité. De la confession, parce
qu’elle n’attend aucun pardon. C’est la parole privilégiée des amis, qui s’aiment
trop pour se juger.

\section{Confus}
%CONFUS
Ce qui manque d’ordre. À ne pas confondre avec l’obscur, qui
manque de clarté, ni avec le flou, qui manque de précision. Toutefois
les trois, en philosophie, vont souvent ensemble. C’est ce qui rend certaines
dissertations de nos étudiants plus difficiles à comprendre, et plus fatigantes,
que bien des pages d’Aristote ou de Kant.

\section{Conjonction}
%CONJONCTION
Une rencontre ou une liaison. Se dit spécialement, en
logique, d’une proposition composée de deux ou plusieurs
propositions reliées par le connecteur « et » : {\it « p et q »} est une conjonction.
Elle n’est vraie que si toutes les propositions qui la composent le sont.

\section{Connaissance}
%CONNAISSANCE
Connaître, c’est penser ce qui est comme cela est : la
connaissance est un certain rapport d’adéquation entre
le sujet et l’objet, entre l'esprit et le monde, bref entre la {\it veritas intellectus} (la
vérité de l’entendement) et la {\it veritas rei} (la vérité de la chose). Qu'il s'agisse de
deux {\it vérités} est ce qui distingue la connaissance de l’erreur ou de l'ignorance.
Mais qu'il s'agisse de {\it deux} vérités, non d’une seule, est ce qui distingue la
connaissance de la vérité même : la connaissance est un rapport extrinsèque
(c’est l’adéquation de soi à l’autre) ; la vérité, intrinsèque (c’est l’adéquation de
soi à soi). Ainsi tout est vrai, même une erreur (elle est vraiment ce qu’elle est :
vraiment fausse). Mais tout n’est pas connu, ni connaissable.

Parce qu’elle est une relation, toute connaissance est toujours relative : elle
suppose un certain point de vue, certains instruments (les sens, les outils, les
concepts...), certaines limites (celles du sujet qui connaît). Se connaître soi, par
exemple, n’est pas la même chose qu'être soi : nul ne se connaît totalement ;
nul n'existe en partie.

% 126
Une connaissance absolue ne serait plus une connaissance : elle serait la
vérité même, dans l'identité de l’être et de la pensée. C’est ce qu’on peut
appeler Dieu, et qui le rend inconnaissable.

\section{Conscience}
%CONSCIENCE
L’un des mots les plus difficiles à définir {\bf --} peut-être parce
que toute définition s'adresse à une conscience et la suppose.
La conscience est un certain rapport de soi à soi, mais qui n’est ni d’adéquation
(toute conscience n’est pas connaissance : il y a des consciences fausses), ni
d'identité (avoir conscience de soi n’est pas la même chose qu'être soi), ni pourtant
de pure altérité (puisqu'il n’y a de conscience que pour soi). Disons que la
conscience est présence à soi de l'esprit ou de l’âme, comme une pensée qui se
pense : un savoir qui se sait, une croyance qui se croit, une sensation ou un sentiment
qui se sentent. Toute conscience suppose en cela une certaine dualité :
« {\it conscience} veut dire {\it science avec...}, remarquait Biran, science de soi {\it avec celle}
de quelque chose ». C’est aussi ce que suggère, chez les phénoménologues,
l’idée d’intentionnalité, laquelle débouche sur la très fameuse définition
sartrienne : {\it « La conscience est un être pour lequel il est dans son être question de
son être en tant que cet être implique un être autre que lui »} ({\it L'être et le néant},
Introduction). Je ne peux pas avoir conscience de cet arbre ou de cette idée sans
avoir conscience aussi, fût-ce obscurément, de la conscience que j’en ai. Cela ne
veut pas dire que toute conscience soit réflexive, si l’on entend par là qu’elle se
prendrait elle-même, nécessairement, explicitement, pour objet. Mais plutôt
qu'aucun objet n’existe pour elle qu’à la condition qu’elle existe elle-même
pour soi. C’est comme une fenêtre qui ne s’ouvrirait sur le monde qu’en se faisant
d’abord regard. C’est pourquoi il n’y a pas de conscience absolue : parce
que toute conscience est médiation. Quand je regarde cet arbre, est-ce l’arbre
que je vois, ou la vision que j’en ai ?

\section{Constitutif}
%CONSTITUTIF
Est {\it constitutif}, chez Kant, ce qui détermine l’expérience
objective, et peut donc être affirmé de tout objet de l’expérience.
S’oppose à {\it régulateur} (voir ce mot).

\section{Contemplation}
%CONTEMPLATION
Le regard attentif et désintéressé. Par métaphore, désigne
l'attitude de la conscience quand elle se contente de
connaître ce qui est, sans vouloir le posséder, utiliser ou le juger. C’est le
sommet de la vie spirituelle : la pure joie de connaître (la {\it vie théorétique} d’Aristote,
l’{\it amor intellectualis} de Spinoza), ou l'amour vrai du vrai. Le moi semble se
% 127
dissoudre dans la contemplation de son objet : il n’y a plus personne à sauver,
et c’est le salut même.

\section{Continence}
%CONTINENCE
L'absence volontaire de tout plaisir sexuel. Cela suppose
presque toujours qu'on en exagère l'importance, et la
redouble.

\section{Contingence}
%CONTINGENCE
On la définit ordinairement comme le contraire de la nécessité :
est contingent, explique Leibniz, tout ce dont le
contraire est possible, autrement dit tout ce qui {\it pourrait} ou {\it aurait pu} ne pas
être. Ces conditionnels sont à prendre en considération. Car quelle {\it condition} supposent-ils ?
Que le réel ne soit pas ce qu’il est. C’est en quoi tout, dans le temps,
est contingent (le néant était possible aussi, ou un autre réel), aussi sûrement que
tout, au présent, est nécessaire (ce qui est ne peut pas ne pas être pendant qu’il
est). Si le temps et le présent sont une seule et même chose, comme je le crois, il
faut en conclure que {\it contingence} et {\it nécessité} ne s'opposent que pour l’imagination :
quand on compare ce qui est, fut ou sera (le réel), à autre chose, qui
pourrait ou aurait pu être (le possible, en tant qu’il {\it n'est pas} réel). Au présent, ou
{\it sub specie aeternitatis}, seul le réel est possible : tout le contingent est nécessaire,
tout le nécessaire est contingent. C’est où Spinoza et Lucrèce se rejoignent.

\section{Contradiction}
%CONTRADICTION
Le fait de contredire, et spécialement, en philosophie,
de {\it se} contredire (de dire à la fois une chose et sa
négation : {\it p} et {\it non-p}). Cela suppose un {\it dire} : il n’y a de contradiction, en toute
rigueur, que dans le discours, jamais dans le réel (si le réel était contradictoire,
on ne pourrait plus le penser). Tel est en tout cas le sens logique du mot : une
contradiction, c’est la présence, dans le même énoncé, de deux éléments
incompatibles ; « cercle carré » est une contradiction.

Au sens ontologique, ce serait la présence, dans le même être, de deux propriétés
incompatibles (auquel cas l'être en question ne saurait subsister) ou
opposées. En ce dernier sens, qui est un sens vague, mieux vaut parler d’ambivalence,
de discordance ou de conflit. Cela évitera de prendre la dialectique
pour une nouvelle logique, quand elle n’est qu’une nouvelle grille de lecture,
voire une nouvelle rhétorique.

Par exemple, que l’histoire de toute société jusqu’à nos jours soit l’histoire
de luttes de classes, comme dit Marx, cela n’a rien de contradictoire ; il serait
plutôt contradictoire, aux yeux de Marx, qu’il en aille autrement.

% 128
Et que l'inconscient ne soit pas soumis au principe de non-contradiction,
comme dit Freud, cela n’est pas davantage contradictoire, ni ne saurait dispenser
la psychanalyse (en tant que théorie) de s’y soumettre.

Que toute société soit conflictuelle, que tout affect soit ambivalent, cela
peut et doit se penser sans contradiction. Par quoi la contradiction reste critère
de fausseté ; c’est ce qui exclut qu’on la rencontre dans le réel, et qui la rend
indispensable, dans la pensée, à toute recherche de la vérité.

\section{Contradiction (principe de non {\bf --})}
%CONTRADICTION (PRINCIPE DE NON {\bf --})
Le principe de non-contradiction stipule que
deux propositions contradictoires ne peuvent être vraies simultanément : la conjonction
{\it « p et non-p »} est une contradiction, comme telle nécessairement fausse. Il en
résulte que la vérité d’une proposition suffit à prouver la fausseté de sa contradictoire.
On peut ajouter : « et réciproquement ». C’est ce que postule le principe
du tiers exclu ({\it p ou non-p} : deux propositions contradictoires ne peuvent
êtres fausses toutes les deux), qu’on distingue du principe de non-contradiction
mais qui lui est logiquement équivalent.

Le principe de non-contradiction est évidemment indémontrable, puisque
toute démonstration le suppose. Mais la même raison le rend irréfutable (on ne
pourrait le réfuter qu’à la condition de le supposer d’abord : si sa réfutation
l'annule, elle s’annule elle-même). La vérité, qui n’est pas une preuve mais une
justification forte, c’est qu’on ne peut pas penser valablement sans accepter, au
moins implicitement, ce principe : toute discussion intellectuelle le suppose,
montre Aristote ({\it Métaphysique}, $\Gamma$, 3-4), et ne peut avancer qu’à la condition de
s’y soumettre.

\section{Contradictoire}
%CONTRADICTOIRE
Qui contredit, ou {\it se} contredit. Spécialement, en
logique, deux propositions sont contradictoires quand
l'une est la négation de l’autre ({\it p} et {\it non-p} sont deux propositions contradictoires),
ou quand elle implique cette négation (si {\it p} implique {\it non-q}, {\it p} et {\it q} peuvent
être dites contradictoires). Par exemple « Tous les hommes sont mortels »
a pour contradictoire « Tous les hommes ne sont pas mortels », ou plutôt (car
cette dernière formulation est équivoque) « Quelque homme n’est pas mortel ».
Deux propositions contradictoires ne peuvent ni être vraies (principe de non-contradiction)
ni être fausses (principe du tiers exclu) ensemble. Elles constituent
donc une alternative : la fausseté de l’une suffit à prouver la vérité de
l’autre, et réciproquement. De là, dans nos raisonnements, cette démarche en
zigzag (nous serions autrement prisonniers de l’évidence), qui prouve une thèse
% 129
par la fausseté de sa contradictoire. Nous nous heurtons au faux, presque à
chaque pas : la raison nous guide, comme l’aveugle sa canne.

\section{Contraire}
%CONTRAIRE
L’extrême opposé. En logique, le mot désigne deux propositions
universelles comportant les mêmes termes, dont l’une
est affirmative {\it (Tous S est P)} et l’autre négative {\it (aucun S n'est P)}. On évitera
donc, malgré l’usage courant, de confondre la relation de {\it contrariété} avec celle
de {\it contradiction}. La contradictoire de « Tous les hommes sont mortels » est
« Tous les hommes ne sont pas mortels » (il existe au moins un homme qui
n’est pas mortel). Son contraire : « Aucun homme n’est mortel. » Le contradictoire
de « blanc » est « non blanc ». Son contraire, selon le contexte, peut être
noir, rouge, bleu, enfin n'importe quelle couleur qui se trouve, de tel ou tel
point de vue et dans tel ou tel champ, ou qu’on croit se trouver, à l'extrême
opposé.

On remarquera que deux propositions contraires ne peuvent être vraies
ensemble (la vérité de l’une implique donc la fausseté de l’autre), mais peuvent
être fausses toutes les deux (de la fausseté de l’une, on ne peut donc
conclure à la vérité de l’autre). Soient par exemple les deux propositions
contraires : « Tous les cygnes sont blancs », « Aucun cygne n’est blanc ». Un
cygne noir suffit à prouver la fausseté de la première, non la vérité de la
seconde.

Le mot, je lai signalé en passant, a aussi un usage plus large. Il désigne
l'opposition absolue ou, comme disait Aristote, la {\it différence maximale}, dans un
même genre. Le contraire de {\it grand} est {\it petit}, le contraire de {\it beau} est {\it laid}... La
contrariété joue ainsi un rôle important dans les définitions, en ce qu’elle précise,
pour chaque mot, l’ampleur de la différence qu’il autorise en l’excluant
(son empan de sens). Les mots, comme les humains, ne se posent qu’en s’opposant.
Cela n’autorise pas à oublier l’entre-deux.

\section{Contrat}
%CONTRAT
C'est un engagement mutuel, ayant, pour les parties contractantes,
force de loi. On y voit parfois l’origine du droit : le
contrat social serait un contrat de chaque citoyen avec tous. Mais ce genre de
contrat ne vaut vraiment que dans un État de droit. Comment pourrait-il
expliquer ce qui le permet ?
Le contrat social n’est qu’une fiction utile : il indique non l’origine de
l’État de droit, mais son fondement ou sa règle ; non comment il est né, mais
comment on doit le penser pour que chacun y soit libre, ou puisse l'être.

% 130
\section{Conversation}
%CONVERSATION
C’est parler ensemble, sans chercher à convaincre l’autre,
ni à le vaincre : le but est de se comprendre, non de se
mettre d’accord. Se distingue par là de la {\it discussion} (qui suppose un désaccord et le
désir d’y mettre fin) et du {\it dialogue} (qui tend vers une vérité commune). La conversation
ne tend vers rien, ou ne tend que vers elle-même. Sa gratuité fait partie de
son charme. C’est l’un des plaisirs de l'existence, spécialement entre amis : leurs différences
mêmes les réjouissent ; pourquoi chercheraient-ils à les supprimer ?

\section{Convoitise}
%CONVOITISE
C’est désirer ce qu’on n’a pas, qu’on voudrait posséder, et
d’autant plus fortement que quelqu'un d’autre en jouit ou
en dispose. La convoitise commence dans le manque et culmine dans l'envie.
C’est où l’amour de soi mène à la haine de l’autre.

\section{Copernicienne (révolution {\bf --})}
%COPERNICIENNE (RÉVOLUTION {\bf --})
On appelle « révolution copernicienne » celle que Kant prétend
avoir effectuée, qu’il juge analogue, dans la métaphysique, à celle de Copernic
en astronomie : là où l’on croyait traditionnellement que la connaissance devait
se régler sur les objets, supposons plutôt que ce sont les objets qui doivent se
régler sur notre connaissance. Ce qu’il y a de commun avec Copernic, c’est que
ce dernier, selon Kant, a recherché « l'explication des mouvements observés,
non dans les objets du ciel, mais dans leur spectateur » ({\it Critique de la raison
pure}, Préface de la deuxième édition). On remarquera pourtant qu’on pourrait
parler aussi bien de {\it contre-révolution copernicienne} : Kant remet l’homme au
centre de la connaissance, quand Copernic l’avait chassé du centre de l’univers.

\section{Copule}
%COPULE
Dans la logique classique, c’est le mot (le plus souvent le verbe
être) qui relie le sujet et le prédicat.

\section{Corps}
%CORPS
Un peu de matière organisée, et spécialement celle qui nous
constitue : ce serait l’objet dont je suis le sujet. Mais si « l’âme et le
corps sont une seule et même chose », comme dit Spinoza, le corps est à lui-même
son propre sujet ; le {\it moi} ne le dirige qu’autant qu’il en résulte.

\section{Cosmologie}
%COSMOLOGIE
La cosmologie, qui étudie le tout (le {\it cosmos} : le monde,
l'univers), n’est qu’une partie de la physique, qui étudie
% 131
plutôt les éléments {\bf --} comme si le plus grand, pour la connaissance, était inclus
dans le plus petit. Cette espèce de paradoxe est un argument en faveur du
matérialisme : le supérieur résulte de l’inférieur. De fait, tant qu’on connaissait
l'univers mieux que les atomes, il était difficile de ne pas y voir un {\it ordre} (tel
était le premier sens, en grec, du mot {\it kosmos}), qui supposait presque inévitablement
une intention. C’est pourquoi l’astronomie, si souvent, se fit théologie
astrale, chez les Anciens, ou justification astronomique de la théologie, chez les
Modernes. C’était l’univers-horloge de Voltaire, qui supposait un Dieu-horloger.
La connaissance du détail des choses à fragilisé cette évidence : le chaos
semble désormais l’emporter sur le cosmos, le désordre sur l’ordre, la physique
de l’infiniment petit sur celle de l’infiniment grand (même la cosmologie
aujourd’hui est quantique), enfin une espèce de lassitude, comme en fin d’analyse,
sur l’enthousiasme des commencements. Ce n’était donc que cela ?

Sagesse désabusée de Cioran : « Après avoir entendu un astronome parler
de milliards de galaxies, j’ai renoncé à faire ma toilette. À quoi bon se laver
encore ? »

On se lave pourtant. Cela ne prouve rien contre la cosmologie contemporaine, ni
contre Cioran.

\section{Cosmologie rationnelle}
%COSMOLOGIE RATIONNELLE
C’est cette partie de la philosophie classique
qui traitait du monde et de son
origine. Kant a eu raison d’en montrer l'illusion, qui la voue à d’inévitables
antinomies. Le monde, comme ensemble de tous les phénomènes, {\it n'est pas} un
phénomène. Ce n’est pas l’objet d’une expérience possible : ce n’est qu’une idée
de la raison, d’où nul savoir, jamais, ne peut être tiré. La balle, désormais, est
dans le camp des physiciens.

\section{Cosmologique (preuve {\bf --})}
%COSMOLOGIQUE (PREUVE {\bf --})
L'une des trois preuves traditionnelles de
l'existence de Dieu, laquelle prétend
démontrer son existence (comme être nécessaire) à partir de celle du monde
(comme être contingent). Cette preuve, que Leibniz appelait {\it a contingentia
mundi}, peut se résumer de la manière suivante : le monde existe, mais pourrait
ou aurait pu ne pas être (il est contingent) ; il faut donc, pour rendre raison de
son existence, une cause; mais si celle-ci était elle-même contingente, elle
devrait avoir à son tour une cause, et ainsi à l'infini ; or, pour satisfaire le principe
de raison suffisante, il faut s’arrêter quelque part : on ne peut échapper à
la régression à l'infini qu’en supposant, comme raison suffisante du monde, un
être qui n’ait plus besoin lui-même d’une autre raison, autrement dit « un être
% 132
nécessaire, comme dit Leibniz, portant la raison de son existence avec soi ».
C’est cette « dernière raison des choses » qui serait Dieu ({\it Principes de la nature
et de la grâce}, \S 7 et 8).

Que vaut cette preuve ? Elle ne vaut que ce que vaut le principe de raison
suffisante, qui ne saurait valoir absolument. Pourquoi l’être aurait-il besoin
d’une raison, puisque toute raison le suppose ? Au reste, quand bien même on
accorderait à Leibniz ce qu’il prétend avoir démontré (l’existence d’un être
absolument nécessaire), on n’en serait guère plus avancé : qu’est-ce qui nous
prouve que cet être nécessaire soit un Dieu, c’est-à-dire un sujet ou une
personne ? Ce pourrait être aussi bien la nature, comme le voulait Spinoza,
autrement dit un être nécessaire, certes, mais sans conscience, sans volonté, sans
amour : non la cause de tout, mais le tout comme cause. À quoi bon le prier,
s’il ne nous entend pas ? À quoi bon lui obéir, s’il ne commande pas ? À quoi
bon y croire, si lui-même n’y croit pas ?

Ainsi cette preuve cosmologique n’en est pas une ; cela laisse de beaux jours
à la foi, comme l’a vu Kant, et à l’athéisme.

\section{Cosmopolitisme}
%COSMOPOLITISME
« On demandait à Socrate d’où il était. Il ne répondit
pas : d'Athènes ; mais : du monde.» Tel est,
dans les mots de Montaigne (I, 26, p. 157), l'énoncé du cosmopolitisme socratique,
qu’on trouvait déjà évoqué, c’est sans doute l’une des sources de Montaigne,
par Cicéron : Socrate se voulait « habitant et citoyen du monde entier »
({\it Tusculanes}, V, 37). On voit que le mot cosmopolitisme (de {\it kosmos}, monde, et
{\it politês}, citoyen) n’a rien d’abord de péjoratif ; c’est plutôt l’énoncé d’une vertu
ou d’une exigence. Seul le nationalisme y verra un défaut : cela condamne le
nationalisme plus que cela n’atteint le cosmopolitisme, qui est au-dessus de ces
bassesses. Se vouloir citoyen du monde, c’est simplement assumer son humanité
et la faire passer, comme il faut en effet, avant l’appartenance à quelque
nation que ce soit. Il faut ajouter toutefois qu'être citoyen du monde n’a jamais
dispensé d’assumer aussi les devoirs qu’impose la citoyenneté ordinaire.
Socrate, tout citoyen du monde qu’il se voulût, obéit aux lois d'Athènes,
jusqu’à la mort.

\section{Cosmos}
%COSMOS
Pour les Anciens, c'était d’abord l’{\it ordre} ({\it kosmos}), tel qu'il se
donne à voir et à admirer, spécialement, dans le ciel : l’ordre et
la beauté (le bel ordre, la beauté ordonnée) qui nous entourent ou nous surplombent.
C’est pourquoi le mot, très vite, en vint à désigner le {\it monde} même,
qu’on supposait en effet ordonné : c'était le contraire du chaos (chez Hésiode)
% 135
ou du tohu-bohu (dans la Genèse). C’est le monde d’Aristote {\bf --} fini, finalisé,
hiérarchisé {\bf --} plutôt que l’univers d’Épicure, mais qui dominera d’autant plus,
pendant deux mille ans, que le christianisme l’avait entre-temps adopté. Si c’est
Dieu qui crée le monde, comment ne serait-il pas ordonné ? La révolution
scientifique des {\footnotesize XVI$^\text{e}$} et {\footnotesize XVII$^\text{e}$}
siècles (Copernic, Galilée, Newton...) mettra fin,
Koyré en a retracé les étapes, à ce bel édifice : la destruction du cosmos, « conçu
comme un tout fini et bien ordonné », et la géométrisation de l’espace (qui
suppose au contraire son « extension homogène et nécessairement infinie »)
nous ont fait passer {\it du monde clos}, comme dit encore Koyré, {\it à l'univers infini}.
C'était donner raison à Épicure, et redonner sens à la plus difficile sagesse : non
pas occuper la place que le monde nous assigne {\bf --} dans un univers infini, toutes
les places se valent {\bf --}, mais accepter sereinement d’être perdu dans le tout
immense, sans autre justification que de jouir, d’agir et d’aimer. Il n’y a pas
d'ordre du monde (pas de {\it cosmos}), et c’est tant mieux : la nature est libre,
comme dit Lucrèce, et nous en elle. Il s’agit non de se tenir à sa place (ce qui
ne serait que politesse ou religion), mais d’habiter l'infini.

\section{Couple}
%COUPLE
Deux individus qui s’aiment ou qui vivent ensemble (la disjonction
est ici inclusive : amour et cohabitation ne sont pas incompatibles),
qui partagent le plus souvent la même demeure, le même lit, la même
intimité, les mêmes occasions de plaisir et de peine, les mêmes soucis, les
mêmes espérances, enfin l'essentiel de ce qu’on peut partager quand on s’aime
et même, parfois, quand on ne s'aime plus. C’est la forme la plus ordinaire de
l'amour. Aussi est-ce presque toujours une amitié, tant que l’amour demeure,
davantage qu’une passion, un plaisir davantage qu’une souffrance, une joie
davantage qu’un manque, une douceur, enfin, davantage qu’un emportement.
Il faut être bien jeune ou bien niais pour le regretter. Être amoureux est à la
portée de n'importe qui ; aimer est plus difficile, et c’est pourquoi le couple est
difficile.

Qu'il y ait des couples malheureux, lourds, étouffants, des couples sans
amour, d’autres pleins de haine et de mépris, c’est ce que nul n’ignore. Cela fait
comme une prison d’un type particulier, où l’on apporte son manger, où
l'autre fournit le lit et les barreaux... Mais l'échec, aussi fréquent qu’il puisse
être ou qu’il soit, ne saurait suffire à le définir. L’essence du couple se lit mieux
dans ses réussites : c’est l’amour le plus intime, le plus quotidien, le plus lucide,
et le seul peut-être qui puisse nous consoler de ses échecs.

Le couple suppose une intimité sexuelle vécue dans la durée, qu’il est à peu
près seul à permettre et qui le rend spécialement précieux et troublant. Comment
connaître vraiment celui ou celle avec qui on n’a jamais fait l'amour, ou
% 134
quelques fois seulement, comme en passant, comme par hasard ? Et quoi de
plus fort, quoi de plus délicieux, que de faire l’amour avec son ou sa meilleur(e)
ami(e) ? C’est ce qu’on appelle un couple, quand il est heureux.

« On ne peut me connaître mieux que tu ne me connais. » Ce vers d’Eluard
dit la vérité réussie du couple : qu’il est le lieu de rencontre du plaisir, de
l'amour et de la vérité. Si vous n’aimez pas ça, n’en dégoûtez pas les autres.

C’est peut-être Rilke qui a trouvé le mots les plus justes : « deux solitudes
se protégeant, se complétant, se limitant, et s’inclinant l’une devant l’autre ».
Auguste Comte y voyait le début de la vie sociale, et c’est aussi le début de la
vie tout court. Il s’agit de permettre « l’action de la femme sur l’homme »
(comme dit Comte encore, mais la même idée se trouvait déjà chez Lucrèce),
qui nous sauve à peu près de la barbarie. Le couple est la civilisation minimale
{\bf --} le contraire de la guerre, l’antidote de la mort. Alain, qui fut célibataire longtemps,
l’avait compris : « Finalement, c’est le couple qui sauvera l’esprit. »

\section{Courage}
%COURAGE
La vertu qui affronte le danger, la souffrance, la fatigue, qui
surmonte la peur, la plainte ou la paresse. C’est la vertu la plus
universellement admirée, sans doute depuis le plus longtemps, et d’ailleurs
(avec la prudence) l’une des plus nécessaires. Toutes les autres, sans courage,
seraient impuissantes ou incomplètes. Vertu cardinale, donc, proprement, ce
que l’étymologie à sa façon confirme (dans courage il y a cœur, dans vertu il y
a courage) et que l'expérience ne cesse de nous rappeler. On évitera pourtant de
le louer trop aveuglément. D’abord parce que le courage peut servir aussi au
pire ; ensuite parce qu’il ne tient lieu d’aucune des vertus qu’il sert. L'amour
peut donner du courage, non le courage suffire à l'amour.

\section{Courtoisie}
%COURTOISIE
C'est la politesse de la cour, comme l’urbanité est celle de
la ville. On comprend que la courtoisie est plus raffinée,
plus recherchée, plus élégante. Trop ? Ce ne serait plus courtoisie mais snobisme
ou préciosité.

\section{Coutume}
%COUTUME
Une habitude, mais sociale plutôt qu’individuelle : c’est une
habitude qui nous précède, nous constitue ou nous accompagne.
On prend une habitude ; on intériorise une coutume, au point parfois
de ne plus la voir. « Les lois de la conscience que nous disons naître de la
nature, écrit Montaigne, naissent de la coutume. [...] Par où il advient que ce
% 135
qui est hors des gonds de la coutume, on le croit hors des gonds de la raison »
({\it Essais}, I, 23 ; comparer avec Pascal, {\it Pensées}, 125-92 et 126-93).

C’est aussi l’une des sources du droit, ou l’un de ses principes : coutume
fait loi, mais dans la mesure seulement où les lois n’y contredisent pas.

\section{Crainte}
%CRAINTE
La peur présente d’un mal à venir (j'ai peur du chien qui me
menace, je crains qu’il me morde), quand elle porte sur un objet
réel (par différence avec l'angoisse). C’est une peur justifiée, ou qu’on croit
telle, et comme en avance sur le danger qui la suscite. Par quoi elle a sa fonction
vitale, qui est de précaution.

On dit souvent que la crainte est le contraire de l’espérance. Ce n’est pas
faux, si l’on n’oublie pas que c’est l’un des cas les plus évidents de l’unité des
contraires. « Il n’y a pas d’espoir sans crainte, disait Spinoza, ni de crainte sans
espoir » ({\it Éthique}, III, 50, scolie, et explication de la déf. 13 des affects).

Comment se libérer de la crainte, demandera-t-on, sans renoncer à la
prudence ? Par connaissance et volonté. Le sage, face au danger à venir, n’est
pas craintif, disaient les stoïciens. Mais il est vigilant et circonspect. Il ne le
craint pas ; il le limite ou s’y prépare.

\section{Création}
%CRÉATION
Créer, au sens strict ou absolu, ce serait produire quelque chose
à partir de rien, ou plutôt à partir de soi seul : ainsi Dieu,
créant le monde. En un sens plus large, on parle de {\it création} pour toute production
qui semble absolument neuve et singulière, ou dans laquelle nouveauté et
singularité l’emportent sur le simple progrès technique ou la transformation
d'éléments préexistants : ainsi parle-t-on de création artistique, parce que ni les
matériaux utilisés (le marbre, les couleurs, les notes, la langue...) ni les règles
ou procédés ordinaires ne suffisent à l’expliquer. C’est une œuvre sans précédent,
sans modèle ou sans pareil.

On ne confondra pas la {\it création} avec la {\it découverte}, qui suppose un objet
préexistant, ni avec l'{\it invention}, qu’un autre aurait pu faire. Christophe Colomb a
découvert l'Amérique ; il ne l’a ni inventée ni créée (elle existait avant lui). Edison
a inventé la lampe à incandescence : serait-il mort à la naissance, celle-ci n’en existerait
pas moins aujourd’hui. Mais Beethoven a bien {\it créé} ses symphonies : elles
n’existaient pas avant lui ni ne pouvaient être créées par un autre.

\section{Création du monde}
%CRÉATION DU MONDE
Le passage de l’infinie perfection à l’imparfaite
finitude, par quoi Dieu, disent les théologiens,
% 136
{\it condescend} à n'être plus tout. C’est ce que Valéry appelait la {\it diminution divine} :
on passe du plus au moins (« Dieu et toutes les créatures, écrit Simone Weil,
cela est moins que Dieu seul »), du bien absolu au mal relatif. Créer, pour
Dieu, c’est se retirer. C’est la seule solution au problème du mal qui me
paraisse théologiquement satisfaisante (même si, philosophiquement, elle ne
l’est pas tout à fait). Dieu, étant tout le Bien possible, ne pouvait créer que
moins bien que soi {\bf --} il ne pouvait créer que le mal. Pourquoi l’a-t-il fait ? Par
amour, répond Simone Weil : pour nous laisser exister. Le monde n’est que le
vide qui en résulte, comme la trace d’un Dieu absent.
Je vois bien le vide, point la trace.

\section{Credo}
%CREDO
{\it Je crois}, en latin. C’est le nom d’une prière, qui énonce les principaux
dogmes de la foi chrétienne. Par extension, peut désigner tout
ensemble de croyances fondamentales. Mais qu’elles soient fondamentales ne
saurait bien sûr garantir qu’elles soient vraies. Si on les savait telles, aurait-on
besoin d’un {\it credo} ?

\section{Crédule}
%CRÉDULE
Celui qui croit trop facilement. Ce n’est pas qu’il soit plus doué
qu’un autre pour la croyance ; c’est qu’il l’est moins pour le
doute.

\section{Crime}
%CRIME
Plus qu’une faute ou qu’un délit : d’abord parce qu’il est l’une et
l’autre (alors que toute faute n’est pas délit, que tout délit n’est pas
faute), ensuite par sa gravité. Le crime est une violation et du droit et de la
morale, mais sur un point particulièrement important. Aussi le meurtre est-il le
crime par excellence {\bf --} et le crime contre l’humanité, le crime maximum.

\section{Crise}
%CRISE
Un changement rapide et involontaire, qui peut s’avérer favorable
ou défavorable, mais qui est toujours difficile et presque toujours
douloureux. Étymologiquement, c’est le moment de la décision ou du jugement,
disons le moment décisif: non qu’on décide d’une crise, mais parce
qu’elle nous oblige à nous décider, ou décide à notre place. Ainsi l'adolescence
ou l’agonie. On parle aussi de crise cardiaque, de crise économique, de crise
politique, de crise de nerfs... Moments de déséquilibre ou de rupture. Quelque
chose est en train de se décider sans nous; il est urgent, si on le peut, de
prendre une décision.

% 137
Husserl, dans les années 1930, parlait d’une crise des sciences et de l’humanité
européennes. Mais ce sont moins les sciences qui sont en crise que la civilisation,
et moins l’Europe que le monde. Comment survivre à la mort de
Dieu, à la disparition des fondements, à l’entropie généralisée du sens {\bf --} au
nihilisme ? Je ne partage guère les solutions de Husserl, qui ne voyait d’issue
que dans la phénoménologie. Mais je me répète souvent la juste formulation de
son inquiétude, qui est encore la nôtre : « Le principal danger qui menace
l’Europe, disait-il en 1935, c’est la lassitude » ({\it }{\it La crise de l'humanité européenne
et la philosophie}, III).

\section{Criticisme}
%CRITICISME
La philosophie de Kant, ou plutôt la solution kantienne au
problème critique. C’est d’abord une méthode : « La méthode
critique de philosopher, qui consiste à enquêter sur le procédé de la raison elle-même,
à analyser l’ensemble de la faculté humaine de connaissance, et à examiner
jusqu'où peuvent s'étendre ses limites » ({\it Logique}, introd., IV). Mais c’est
aussi un ensemble de réponses aux principales questions auxquelles se ramène
la philosophie. {\it Que puis-je connaître ?} Tout ce qui entre dans le cadre d’une
expérience possible, et cela seul : nous ne connaissons que les phénomènes,
jamais les noumènes ou les choses en soi ; l'absolu, par définition, est hors
d'atteinte. {\it Que dois-je faire ?} Mon devoir, autrement dit ce que la raison (en
tant qu’elle est pratique) ou la liberté (en tant qu’elle est raisonnable) suffisent
à m'ordonner de façon universelle et inconditionnelle. {\it Que m'est-il permis
d'espérer ?} Qu’en me rendant digne du bonheur {\bf --} en faisant mon devoir {\bf --} je
m'en rapprocherai en effet. Cela suppose une conjonction nécessaire du bonheur
et de la vertu (le souverain bien), laquelle n’est pas attestée sur terre, c’est
le moins qu’on puisse dire, mais dont on doit croire {\bf --} si l’on veut échapper au
désespoir {\bf --} qu’elle le sera dans une autre vie, ce qui suppose que Dieu existe et
que l’âme est immortelle. Ainsi le criticisme ramène à la religion (« abolir le
savoir, afin d'obtenir une place pour la foi »), et la « Révolution
copernicienne » aboutit à ce résultat grandiose : « Tout reste dans le même état
avantageux qu'auparavant » ({\it Critique de la raison pure}, Préface de la seconde
édition).

\section{Critique}
%CRITIQUE
Tout ce qui décide ou juge. Spécialement, en philosophie, ce
qui juge le jugement même. C’est soumettre nos connaissances,
nos valeurs et nos croyances au tribunal de la raison. La raison s’y juge
donc elle-même ; c’est ce qui rend la critique nécessaire (une raison qui ne
% 138
s’examine pas pèche contre la raison) et interminable (puisque circulaire) : on
ne peut éviter d’y entrer, ni en sortir.

\section{Croyance}
%CROYANCE
Dit moins que savoir, moins que foi, et les enveloppe pour
cela l’un et l’autre. Croire, c’est penser comme vrai, sans pouvoir
absolument le prouver. Par exemple je crois qu’il fera beau demain, que la
Terre est ronde, que deux plus deux font quatre, que Dieu n’existe pas... On
dira que deux au moins de ces propositions font l’objet d’une démonstration
possible : ce ne serait plus croyance mais savoir. En quoi je ne conteste que le
{\it « ne... plus »}, qui suppose que savoir et croyance soient incompatibles. Il n’en
est rien. Car ces démonstrations (que la Terre est ronde, que deux plus deux
font quatre...) ne valent absolument que s’il y a des démonstrations absolument
valides, ce qui ne se démontre pas. Comment démontrer rationnellement
la validité de la raison, que toute démonstration suppose ? Comment vérifier
empiriquement la validité de l’expérience ? Ainsi le scepticisme revient toujours,
pour lequel toute pensée est croyance. Lisez Hume, qui vous réveillera de
votre sommeil dogmatique. Et évitez, lisant Kant, de vous rendormir trop
vite.

\section{Cruauté}
%CRUAUTÉ
Le goût ou la volonté de faire souffrir. Proche en cela du
sadisme, en plus coupable. Le sadisme est une perversion. La
cruauté, un vice. C’est à mes yeux l’un des péchés capitaux, et le pire de tous.

\section{Culpabilité}
%CULPABILITÉ
Être coupable, c’est être responsable d’une faute qu’on a
accomplie non seulement volontairement mais délibérément,
c’est-à-dire en sachant qu’elle était une faute. C’est en quoi la culpabilité
suppose la liberté (on n’est responsable que de ce qu’on a fait librement) et
semble l’attester. On remarquera pourtant qu’on ne choisit que ce qu’on fait,
non ce qu’on est. Ainsi chacun est coupable de ses actions, et innocent de soi.

\section{Culture}
%CULTURE
Au sens strict, le mot désigne l’ensemble des connaissances
qu’une société transmet et valorise, et spécialement celles qui
portent sur le passé de l'humanité (son histoire, ses croyances, ses œuvres).
C’est le contraire de l’inculture.

Au sens large, qui domine aujourd’hui dans les sciences humaines (sans
doute par influence de l'allemand {\it Kultur}), le mot est devenu un quasi-synonyme
% 139
de civilisation : il désigne tout ce qui est produit ou transformé par
l’humanité. C’est le contraire de la nature.

Le premier sens donne l'adjectif {\it cultivé}, qui s'applique à des individus et
vaut comme éloge, sinon toujours comme approbation.

Le second donne l’adjectif {\it culturel}, qui s'applique plutôt à des produits ou
à des pratiques, et reste le plus souvent dépourvu de toute visée normative.

Une robe, une moissonneuse-batteuse ou un morceau de rap sont aussi {\it culturels},
en ce sens, qu’une symphonie de Mahler. Mais les gens cultivés ne les
mettent pas sur le même plan.

\section{Cupidité}
%CUPIDITÉ
L'amour exagéré de l’argent, spécialement de celui qu’on n’a
pas encore. C’est ce qui distingue la cupidité de l’avarice : le
cupide veut acquérir, l’avare veut conserver. Un psychanalyste pourrait voir
dans le premier le triomphe de l’oralité ; dans le second, celui de l’analité. Ce
sont deux façons infantiles de désirer (deux régressions), qui menacent
particulièrement les vieillards.
En pratique, ces deux passions vont souvent ensemble, voire n’en font
qu’une, qui est la passion d’amasser.

\section{Curiosité}
%CURIOSITÉ
Le désir de savoir, quand on n’en a pas le droit ou l’usage.
C’est aimer la vérité qu’on ignore, et d’autant plus qu’elle se
cache {\bf --} ou qu’on nous la cache {\bf --} davantage. De là les sciences et le voyeurisme.

Un défaut ? Une vertu ? Ce peut être l’un et l’autre, et souvent c’est les
deux. C’est pourquoi c’est si bon et si troublant...

\section{Cynisme}
%CYNISME
Le refus des conventions, des grands principes ou des bons sentiments.
Être cynique, au sens philosophique, c’est refuser de
confondre le réel et le bien, l'être et la valeur, autrement dit « ce qui se fait »,
comme disait Machiavel, et « cela qui se devrait faire » ({\it Le Prince}, XV). Au sens
trivial, c’est ne viser qu’à l'efficacité, sans s’encombrer de morale ni d’idéologie.
On y voit ordinairement une forme d’impudence ; ce peut être aussi une forme
de lucidité : le refus de faire semblant ou de se raconter des histoires.

Avant d’être un défaut, ou d’être considéré comme tel, le cynisme fut
d’abord une école de vertu, peut-être la plus exigeante qui fut jamais.
Philosophiquement, c’est l’un des deux chaînons {\bf --} avec l’école Mégarique {\bf --} qui mène
de Socrate au stoïcisme, et à ce titre l’un des courants les plus importants de
toute la philosophie antique. L'école, inspirée par Antisthène (qui fut l'élève de
% 140
Socrate), est dominée par la belle figure de Diogène, qui fut l’élève d’Antisthène.
C’est un nominalisme radical. Aucune abstraction n’existe, aucune loi
ne vaut, aucune convention n'importe : il n’y a que des individus et des actes.
De là une indépendance farouche, qui ne reconnaît de vertu que libre, et de
liberté que vertueuse. C’est dominer en soi la bête ou la bêtise. Ascétisme ?
Certes ; mais sans pudibonderie. Le même Diogène, qui enlaçait en hiver des
statues gelées, pour s’endurcir, n’hésitait pas à se masturber en public. Il jugeait
l’hypocrisie plus coupable que le plaisir, le plaisir plus sain que la frustration.
C’est le même aussi qui n’eut pour Alexandre que dédain. Liberté, toujours.
Vérité, toujours. Pourquoi cacher ce qui n’est pas un mal ? Pourquoi adorer ce
qui n’est pas un bien ?

Le cynisme ancien (celui de Diogène ou Cratès) et le cynisme moderne
(celui de Machiavel) ont en commun une même disjonction des ordres : le vrai
n’est pas le bien, ce qui se fait ne saurait fonder ce qui se devrait faire, pas plus
que le devoir ne saurait tenir lieu d’efficacité. Peu importe dès lors qu’on privilégie
la vertu morale, comme Diogène, ou l'efficacité politique, comme
Machiavel. L'essentiel est de ne pas les confondre, et de ne renoncer ni à l’une
ni à l’autre.

Le cynique refuse de prendre ses désirs pour la réalité, comme de céder sur
la réalité de ses désirs. Il refuse d’adorer le réel : l’action lui tient lieu de prière.
Il ne croit en rien, mais n’a pas besoin de croire pour vouloir. La volonté lui
suffit. L'action lui suffit.

On les appela des chiens, parce qu’ils enseignaient sur la place du {\it cynosarges}
(le chien agile) et faisaient fi de toute pudeur. Ils en firent leur emblème. « Je
suis en effet un chien, disait Diogène, mais un chien de race, de ceux qui
veillent sur leurs amis. » Vertu cynique : vertu sans foi ni loi, mais non sans
fidélité ni courage.

\section{Cyrénaïque}
%CYRÉNAÏQUE
Aristippe de Cyrène, le fondateur de l’école cyrénaïque, fut
l'élève de Socrate. Il ne reconnaissait qu’une seule certitude,
la sensation, qu’un seul bien, le plaisir, qu’un seul temps, le présent. De
là une sagesse sympathique et un peu courte, qui est un art de jouir plutôt que
d’être heureux. Le cyrénaïsme est comme un épicurisme avant la lettre, mais
qui s’enfermerait dans le corps et le présent : c’est un sensualisme subjectiviste
(nous ne connaissons pas les choses, mais seulement les impressions qu’elles
nous procurent) et un hédonisme de l’instant {\bf --} une esthétique (de {\it aisthèsis}, la
sensation) plutôt qu’une éthique. Sa maxime pourrait être celle d’Horace :
{\it Carpe diem} (cueille le jour). Ou celle d’Oscar Wilde : « Pas le bonheur, le
plaisir ! » C’est renoncer un peu vite à l'éternité et à l’esprit.

% 141
L'école débouchera, avec Hégésias, sur un pessimisme absolu. C’est que le
plaisir est rare ou faible, et que le présent dure longtemps. À quoi l’on
n'échappe que par le suicide, que prônait en effet Hégésias, ou la pensée, qui
ne saurait s'épanouir dans l'instant. L'âme jouit aussi, et davantage : tel sera
l’enseignement d’Épicure, qui l’obligera à dépasser le cyrénaïsme.

