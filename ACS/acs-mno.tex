
\chapter{MNO}
%{\footnotesize XIX$^\text{e}$} siècle — {\it }

\section{Machiavélisme}
%MACHIAVÉLISME
Une forme de cynisme, mais qui sacrifie la morale à la
politique : l'opposé par là, ou le symétrique, du cynisme
de Diogène.

Le mot, qui se prend ordinairement en mauvaise part, vise surtout une certaine
façon, qu’on trouve en effet chez Machiavel, de juger une action à ses
résultats plutôt qu’à sa moralité intrinsèque (« Si le fait l’accuse, le résultat
l’excuse », {\it Discours}, I, 9), et de s’autoriser pour cela, dans l’ordre politique, plusieurs
actions qui seraient, du point de vue de la morale ordinaire, répréhensibles.
C’est considérer que la fin justifie les moyens, et que la ruse, quand elle
est efficace, vaut mieux qu’une droiture qui ne le serait pas. Le machiavélisme
dit ainsi la vérité de la politique : « Il y a si loin de la sorte qu’on vit à celle selon
laquelle on devrait vivre, écrit Machiavel, que celui qui laissera ce qui se fait
pour cela qui se devrait faire, il apprend plutôt à se perdre qu’à se conserver ;
car qui veut faire entièrement profession d'homme de bien, il ne peut éviter sa
perte, parmi tant d’autres qui ne sont pas bons ; aussi est-il nécessaire au Prince
qui se veut conserver, qu’il apprenne à pouvoir n'être pas bon, et d’en user ou
pas selon la nécessité » ({\it Le Prince}, XV). Les médiocres y voient une justification
de limmoralité, de la perfidie, de l’arrivisme sans vergogne, de ce que
Machiavel, qui n’écrivait pas pour les médiocres, appelle la {\it scélératesse}. C’est
bien sûr se méprendre. Un scélérat au pouvoir reste un scélérat.

\section{Machine}
%MACHINE
«Si les navettes tissaient d’elles-mêmes, écrivit un jour Aristote,
alors les artisans n'auraient pas besoin d’ouvriers, ni les
maîtres d'esclaves » (Politique, 1, 4). Cela dit à peu près ce que c’est qu’une
machine : un objet animé, mais sans âme (un automate), capable de fournir un
%— 349 —
certain travail, autrement dit d’utiliser efficacement l'énergie qu’il reçoit ou
dont il dispose. Ainsi un métier à tisser, un lave-linge, un ordinateur. C’est en
ce sens que les animaux, pour Descartes, et l’homme, pour La Mettrie, sont des
machines. Non parce qu’ils seraient dépourvus d’intelligence et de sensibilité,
comme le premier l’a cru bêtement des bêtes, encore moins parce qu’ils seraient
faits de vis et de boulons (pourquoi une machine ignorerait-elle les cellules, les
organes, les échanges biologiquement organisés d'énergie et d’informations ?),
mais parce qu'ils sont {\it sans âme}, autrement dit sans autre réalité substantielle
que matérielle. En ce sens {\it L'homme-machine}, de La Mettrie, énonce l’une des
thèses les plus radicales du matérialisme : nous ne sommes que « des animaux
et des machines perpendiculairement rampantes », comme il dit étonnamment,
mais vivantes (La Mettrie était médecin), conscientes (grâce au cerveau, qui est
comme une machine particulière dans la machine globale de l'organisme), et
capables pour cela de souffrir et de jouir, de connaître et de vouloir, enfin d’agir
et d'aimer. « Nous ne pensons, et même nous ne sommes honnêtes gens, que
comme nous sommes gais ou braves ; tout dépend de la manière dont notre
machine est montée » ({\it L'homme-machine}, Éd. Fayard, p. 70-71).

\section{Magie}
%MAGIE
Une action qui excéderait les lois de la nature ou de la raison

ordinaires : du surnaturel efficace, ou une efficacité surnaturelle,
mais qui obéirait à notre volonté (par différence avec la grâce et le miracle, qui
n’obéissent qu’à Dieu) ou serait instrumentalisé par elle. Cette efficacité, même
lorsqu’elle semble avérée (par exemple dans le chamanisme : une parole qui tue,
un rite qui guérit), suppose toutefois la croyance, ce qui est très naturel et très
raisonnable : ce n’est pas magie, mais suggestion. « L'efficacité de la magie,
écrit Lévi-Strauss, implique la croyance en la magie » ({\it Anthropologie structurale},
IX). Raison de plus pour ne pas y croire.

\section{Magnanimité}
%MAGNANIMITÉ
La grandeur d'âme : se juger digne de grandes choses,
explique Aristote, et l’être en effet ({\it Éthique à Nicomaque},
IV, 7-9). C’est la vertu des héros, comme l’humilité est celle des saints. Vertu
grecque, contre vertu chrétienne.

La magnanimité s’oppose à la fois à la bassesse (ou pusillanimité : se juger
incapable d’une grande action, et l’être par là même) et à la vanité (avoir les
yeux ou le discours plus grands que l’âme : se croire ou se prétendre capable de
ce qu'on est en vérité incapable de réussir). Elle correspond assez bien à
l’{\it acquiescentia in se ipso} de Spinoza (la satisfaction intime, le contentement
lucide, l'amour heureux de soi-même) : « une joie née de ce qu’un homme se
%— 350 —
considère lui-même et sa puissance d’agir » ({\it Éthiqu}e, VII, déf. 25 des affects ;
voir aussi IV, 52, dém. et scolie). Toutefois la magnanimité peut aller sans la
joie, comme on voit chez Athos : ce n’est plus sagesse, c’est toujours vertu.

\section{Maïeutique}
%MAÏEUTIQUE
{\it Maîa}, en grec, c’est la sage femme. C’est à elle que Socrate,
dans le {\it Théétète}, se compare : la maïeutique est l’art
d’accoucher les esprits, autrement dit d’en faire sortir — par le questionnement
et le dialogue — une vérité qu’ils contiennent sans la connaître. L'exemple classique
est celui du jeune esclave du {\it Ménon} : Socrate l’amène à découvrir comment
obtenir un carré double d’un autre (en le construisant sur la diagonale du
premier), alors même qu’il l’ignorait et sans avoir besoin pour cela de lui
apprendre quoi que ce soit. C’est supposer que la vérité est déjà en nous, ou
nous en elle : réminiscence, ou éternité.
En pratique, la maïeutique atteint vite ses limites. Interroger un ignorant,
cela ne saurait suffire à l’éduquer. Le modèle socratique, dans nos classes, n’est
souvent qu’une utopie de plus.

\section{Maître}
%MAÎTRE
Celui qui enseigne, guide ou commande. Les trois à la fois ? Pas
nécessairement. Cela dépend en partie de lui, de ce qu’il sait ou
peut, mais aussi de ceux dont il est le maître : sont-ce des élèves, des disciples
ou des esclaves ?

\section{Majesté}
%MAJESTÉ
Une grandeur visible, qui justifierait à la fois le respect et l’obéissance.
C’est ce qui faisait dire à Alain qu'il était « l’ennemi de
toute espèce de Majesté ». Et d’en donner cette définition parfaite: « La
Majesté, c'est tout ce qui, ayant le pouvoir, veut encore être respecté. » C’est
vouloir régner sur les esprits aussi. Toute majesté est ridicule ou tyrannique.

\section{Majeure}
%MAJEURE
Dans un syllogisme, celle des deux prémisses qui contient le
grand terme. On la place traditionnellement en premier ; mais
c'est son contenu qui importe, non sa place.

\section{Mal}
%MAL
On ne disculpera pas Dieu à si bon compte. Le mal n’est pas seulement
l’absence d’un bien (absence que Dieu ne tolérerait qu’autant
que nécessaire, pour créer autre chose que soi), mais son contraire. Ainsi la
%— 351 —
souffrance est un mal (qui ne se réduit pas à l’absence de plaisir), et le modèle
de tous : le mal, c’est d’abord ce qui {\it fait} mal. Et Dieu n’est innocent qu’à la
condition stricte de n’exister pas.

Le mal existe positivement : non certes parce qu’il serait une réalité objective
ou absolue (il n’y a de mal que pour un sujet), mais parce qu’il constitue,
pour tout sujet, une expérience première. Point besoin, pour souffrir, d’avoir
connu le plaisir. Il est vraisemblable au contraire que le bien ne vienne
qu'après, et secondairement, de ce que l’expérience même du mal fait désirer
sa disparition, et la rend agréable. Épicure est sublime ici de simplicité. Ce
n'est pas le mal qui est l'absence du bien ; c’est le bien qui est l’absence du
mal.

Cela dut être vrai pour l'humanité comme pour l'individu. Le mal est premier.
Et la peur — notre mère la peur — engendre en nous l’espérance, mais aussi
le courage.

Le mal, disais-je, c'est d’abord ce qui fait mal : la souffrance est le mal premier,
et le pire. Toutefois, ce n’est pas le seul : une bassesse indolore, et même qui
serait agréable à tous, ne cesse pas pour cela, moralement, d’être mauvaise. C’est
donc qu’il y a autre chose que la souffrance. Quoi ? « Une certaine idée de
l’homme, comme dit Spinoza, qui soit comme un modèle de la nature humaine
placé devant nos yeux » ; le mal, ou le mauvais ({\it malum}), c’est ce qui nous éloigne
de ce modèle ou nous empêche de le reproduire ({\it Éthique}, IV, Préface).

« On peut prendre le mal métaphysiquement, physiquement et moralement »,
écrivait Leibniz : « Le {\it mal métaphysique} consiste dans la simple imperfection,
le {\it mal physique} dans la souffrance, et le {\it mal moral} dans le péché »
({\it Théodicée}, Y, 21). Qu'en reste-t-il pour l’athée ? Les deux premiers restent à
peu près inentamés. L’imperfection du monde et l’ampleur de la souffrance
font même partie de nos plus fortes raisons de ne pas croire en Dieu. {\it « Si
Deus est, unde malum ? si non est, unde bonum ? »}, demandait Leibniz (I, 20 :
Si Dieu existe, d’où vient le mal ? s’il n’existe pas, d’où vient le bien ?). La
première des deux interrogations semble la plus redoutable. D'abord parce
que le mal l'emporte, en force et en fréquence, ensuite parce que la puissance
indéfinie et imparfaite de la nature est mieux à même d’expliquer le bien
qu'elle comporte que l’infinie et toute-puissante bonté de Dieu ne le serait de
justifier le mal qu’elle tolère. La souffrance serait un châtiment ? Le prix à
payer de notre liberté, de nos fautes ? Comment l’accepter, puisque le mal est
antérieur à la culpabilité et même — les bêtes souffrent aussi — à l'humanité ?
Mieux vaut la révolte, ou plutôt mieux vaut pardonner à Dieu de n’exister
pas.

Qu'en est-il alors, pour finir, du mal moral ? S'il n’est plus {\it péché}, au sens
religieux du terme, c’est-à-dire offense faite à Dieu ou violation d’un de ses
%— 352 —
commandements, il reste à le penser, conformément à la lettre et à l'esprit du
spinozisme, comme ce qui nous éloigne de notre idéal d'humanité ou nous
empêche de le reproduire ({\it Éthique}, IV, Préface ; voir aussi les {\it Lettres} 19, 21 et
23, à Blyenbergh). C’est pécher encore, mais contre l'humanité ou contre soi.
Le mal, c’est ce qui nous empêche d’être pleinement {\it humains}, au sens normatif
du terme, c’est-à-dire accessibles à la raison, quand nous en sommes capables,
ou à la compassion, quand la raison ne suffit pas. « Quant à celui qui n’est
poussé ni par la raison ni par la compassion à être secourable aux autres, on
l'appelle justement inhumain, car il ne paraît pas ressembler à un homme »
({\it Éthique}, IV, scolie de la prop. 50).

\section{Malédiction}
%MALÉDICTION
C’est vouer au mal par des mots. On a bien tort d’y voir
de la magie, quand il n’y a là que haine et superstition. Le
mieux serait d’en rire. Un bras d'honneur, si l’on n’est pas capable d’indifférence,
fait un exorcisme suffisant.
Quant à maudire les méchants, cela ne sert à rien. L'action vaut mieux.

\section{Malheur}
%MALHEUR
Je l'ai vécu juste assez pour savoir ce qu’il est : le malheur, c’est

quand toute joie paraît impossible, quand il n’y a plus que
l'horreur et l'angoisse, que la douleur, que le chagrin, quand on voudrait être
mort, quand vivre n’est plus que survivre et endurer, que souffrir et pleurer.
Se rappeler, dans ces moments-là, que tout est impermanent : ce malheur passera
aussi. Et que sa réalité suffit à prouver, au moins par différence, au moins
pour les autres, la possibilité du bonheur. Ce n’est pas une consolation ? Dans
les pires moments, il m’a semblé que si. Que le malheur tombe sur moi ou sur
un autre, qu'est-ce que cela change d’essentiel ? Consolation insuffisante ? Si
elle ne l'était pas, ce ne serait pas un malheur.

\section{Malveillance}
%MALVEILLANCE
C'est vouloir du mal à quelqu'un, soit par pure méchanceté,
si nous en sommes capables, soit, plus vraisembla-
blement, par haine ou par intérêt (par égoïsme). C’est vouloir le mal, sinon
pour le mal, mais en le sachant tel. Nul n’est méchant volontairement, ni malveillant
involontairement.

\section{Manichéisme}
%MANICHÉISME
C'est d’abord une religion, apparue entre la Mésopotamie
et la Perse, au Hi siècle de notre ère, sous la dynastie des
%— 353 —
Sassanides. Mani, son fondateur, voulut inventer ou transmettre une religion
universelle. Tout en s’inspirant de ses propres visions ou révélations, il tenta
pour cela une espèce de synthèse entre trois religions déjà existantes, qui lui
semblaient aller dans le même sens: l’antique religion persane, celle de
Zoroastre, le christianisme (Mani prétend être le Paraclet annoncé par Jésus)
et le bouddhisme. Sa doctrine, telle qu’on peut à peu près la reconstituer,
était un dualisme gnostique et sotériologique. Le manichéisme oppose en
effet deux principes coéternels — la Lumière et les Ténèbres, le Bien et le Mal,
l'Esprit et la Matière —, qui ne cessent ici-bas de se mêler et de se combattre :
l’âme est le lieu et l’enjeu, en l’homme, de cet affrontement. Cette nouvelle
religion, qui avait ses Écritures, sa liturgie, son Église, fut bientôt combattue
par la force (Mani, d’abord protégé par Shâpur I‘, mourra en prison, sous le
roi Bahrâm I$^\text{er}$, qui voulait restaurer le mazdéisme comme religion d’État).
Elle se répandit pourtant pendant quelques siècles, aussi bien vers l'Afrique
et l’Europe que vers la Chine et l’Inde, avant de disparaître ou de se dissoudre,
sans qu’on sache trop pourquoi ni comment, dans les religions plus
anciennes dont elle s’inspirait ou dans d’autres, plus neuves (spécialement
l'Islam), qui finirent par la recouvrir. Il en reste une tentation gnostique ou
dualiste, repérable dans la plupart des grandes religions, dès lors qu’elles diabolisent
— parfois officiellement, plus souvent sous forme d’hérésies — le
monde ou le corps. Saint Augustin lui-même, qui combattit si vigoureusement
les manichéens de son temps, n’en fut pas toujours exempt : le jansé-
nisme, qui se réclamera à juste titre de l’Évêque d’Hippone, doit sans doute
quelque chose de sa belle intransigeance à ce qu’on pourrait interpréter, au
moins à certains égards, comme un retour du refoulé manichéen.. Mais ce
sont les Cathares, en Occident, qui porteront le plus haut le flambeau dualiste
et gnostique. Ils seront éliminés, on sait avec quelle sauvagerie. Sans
doute pensèrent-ils, sur le bûcher, que leur défaite même leur donnait
raison.

En un sens second, on parle de manichéisme pour qualifier une pensée
qui oppose le Bien et le Mal de façon absolue, comme si tout le bien était
d’un côté (par exemple dans tel camp politique) et tout le mal d’un autre (par
exemple dans le camp opposé). Ce dernier sens est toujours péjoratif. Qu'un
camp soit absolument mauvais, cela peut arriver (le nazisme en est un
exemple commode), mais n’autorise pas à penser que l’autre soit bon absolument.
Quand bien même Hitler serait le diable, cela ne saurait faire de Staline
ou de Roosevelt des anges. Par quoi tout manichéisme, appliqué à la
politique, est bête et dangereux : c’est adorer son propre camp, quand il ne
faudrait que le soutenir.

%— 354 —
\section{Maniérisme}
%MANIÉRISME
C'est une exagération du style, qui mène ordinairement au
baroque. Formes graciles ou allongées, compositions recherchées
voire alambiquées, sentiments délicats ou rares, parfois évanescents, parfois
exacerbés, comme un raffinement qui hésiterait entre la grâce et l’outrance,
entre la poésie et l’affectation, entre la préciosité et l’expressionnisme. C’est
vouloir imiter la manière des maîtres, tout en voulant les dépasser (en allant
plus loin qu’eux, sinon plus haut). Cela vaut mieux que l’académisme, qui
renonce à dépasser ; mais moins que le classicisme, qui n’imite que la nature ou
les Anciens.

L'époque maniériste proprement dite est le {\footnotesize XVI$^\text{e}$} siècle, d’abord en Italie (le
Pontormo, Jules Romain, Giambologna, le Parmesan, le Tintoret....) puis dans
le reste de l'Europe (spécialement avec le Greco, en Espagne, mais aussi avec un
certain nombre d'artistes, souvent d’origine italienne, de l’école de
Fontainebleau : le Rosso, le Primatice, Jean Cousin...). On peut pourtant
parler de maniérisme pour d’autres artistes (il y a du maniérisme, à certains
égards, chez Botticelli ou Dürer) ou pour d’autres époques, comme le {\footnotesize IV$^\text{e}$} siècle
avant Jésus-Christ, en Grèce, ou le début du {\footnotesize XX$^\text{e}$}, en Europe. Le maniérisme est
la tentation des tard venus, qui doivent rivaliser avec plus fort qu'eux — par
exemple avec Phidias ou Michel-Ange. Ils s’en sortent par un excès de
recherche, de virtuosité, de sophistication, au service d’une sensibilité « artiste »
ou mondaine. C’est préférer la grâce à la beauté, le style à la vérité, enfin l’art à
la nature. Esthétique {\it maniérée et somptueuse}. C’est une décadence recherchée.

\section{Marché}
%MARCHÉ
— Quand tu achètes une baguette chez ta boulangère, me
demande un ami économiste, pourquoi te la vend-elle ?

— C’est son métier...

— C’est surtout son intérêt! Elle préfère avoir 4 F 20 plutôt qu’une
baguette...

— C’est normal : la baguette lui a coûté beaucoup moins cher.

— Exactement. Et pourquoi est-ce que tu lui achètes sa baguette ?

— Parce que j’ai besoin de pain...

— Sans doute. Mais tu pourrais faire ton pain toi-même. La vraie raison,
c’est que tu préfères avoir une baguette plutôt que 4 F 20.

— Bien sûr! La baguette, si je devais la fabriquer moi-même, me reviendrait,
temps de travail compris, beaucoup plus cher.

— Tu commences à comprendre ce que c’est que le marché. Elle te vend
une baguette par intérêt, tu l’achètes par intérêt, et chacun de vous y trouve son
compte. C’est le triomphe de l’égoisme...

%— 355 —
— C’est surtout le triomphe de l'intelligence ! Faire son pain, passe encore.
Mais qui pourrait se fabriquer une voiture ou une machine à laver ? C’est ce
que Marx appelle la division du travail...

— Adam Smith en avait parlé avant lui. Or Smith, ici, est plus éclairant que
Marx.

— Je te vois venir : éloge du libéralisme...

— Essayons plutôt de comprendre. Je reviens à ta boulangère. Tu pourrais
aussi bien aller chez un de ses concurrents. Pourquoi vas-tu chez elle ?

— Parce que son pain est meilleur.

— Elle a donc intérêt à faire le meilleur pain possible. Mais l’achèterais-tu à
n'importe quel prix ?

— Sans doute pas.

— Pour te garder comme client, elle a donc intérêt, dans une économie con-
currentielle, à t'offrir le meilleur rapport qualité-prix possible. C’est aussi ce
que tu souhaites. Vos intérêts ne sont pas seulement complémentaires, ils sont
convergents !

— Pas étonnant qu’elle me sourie si gentiment...

— Ni que tu sois si poli avec elle! Chacun de vous deux n’agit que par
égoïsme, mais cela, loin de vous opposer, vous rapproche. Pourquoi serait-on
désagréable avec celui dont on a besoin ? Mais que le pain soit moins bon ou
plus cher qu’à la boulangerie voisine, ou que tu ne puisses plus payer, c'en est
terminé de votre relation : tu ne lui dois rien, ni elle à toi, qu’autant que vous
y trouvez l’un et l’autre votre compte. C’est ce qu’on appelle le marché : la rencontre
de l’offre et de la demande, autrement dit la libre convergence — par la
médiation de l'échange et sous réserve de la concurrence — des égoïsmes.
Chacun est utile à l’autre, sans qu’on ait besoin de le forcer. Tous ne cherchent
que leur propre intérêt, mais ne peuvent le trouver qu’ensemble. C’est pourquoi
le pain est meilleur et plus abondant dans une économie libérale que dans
une économie collectiviste. La convergence des égoïsmes est plus efficace que
les contrôles et la planification !

— Tu enfonces une porte ouverte...

— Elle ne la pas toujours été !

— Elle l’est désormais, depuis des décennies. Qui voudrait fixer un prix par
force ou par décret ? Ce ne serait plus marché, mais racket ou police. La misère,
dans les deux cas, est au bout, et les queues immenses devant des magasins
presque vides...

— Je ne te le fais pas dire ! Mais alors, il faut en tirer les conséquences. Ce
que tu appelais le triomphe de l'intelligence, c’est le triomphe du marché.

— C’est surtout le triomphe de la solidarité.

— Revoilà ta morale et tes idées de gauche...

%— 356 —
— Qui te parle de morale ? Si je devais compter sur la générosité de ma boulangère
pour avoir du pain, je serais mort de faim depuis longtemps ! Comme
elle, si elle devait compter sur ma générosité pour avoir de l'argent. Au
contraire, si nous comptons chacun sur l’égoïsme de l’autre, nous ne serons
jamais déçus !

— C’est ce que j'appelle le marché...

— C’est ce que j'appelle la solidarité : non le contraire de l’égoïsme, comme
la générosité, mais sa socialisation bien réglée. Non le désintéressement, mais la
convergence des intérêts. C’est pourquoi la générosité, moralement, vaut mieux
(elle est désintéressée). Et c’est pourquoi la solidarité, socialement, économiquement,
est beaucoup plus efficace.

— Alors il faut dire que le marché est un formidable créateur de solidarité.
Tes amis de gauche ne vont pas être contents !

— À moins qu’ils ne soient en train de le comprendre. Tu en connais encore
beaucoup qui veulent étatiser l’économie ?

— Peut-être pas. Mais ils croient davantage aux lois et aux impôts qu’au
marché et à la concurrence...

— C’est que le marché ne vaut que pour les marchandises !

— Pour les marchandises et les services.

— Disons pour tout ce qui se vend et qui s’achète. Un service, dès lors qu’il
est à vendre, n’est qu’une marchandise comme une autre. Mais la santé ? Mais
la justice ? Mais l'éducation ? Si tu penses qu’elles sont à vendre, soumets-les au
marché ! Que restera-il de notre société, de nos idéaux et du droit des plus
faibles ? Si au contraire, comme je le crois, la justice n’est pas à vendre, ni la
liberté, ni la santé, ni l'éducation, ni la dignité..., il faut en conclure qu’elles ne
sont pas des marchandises. Le marché, sur elles, est donc sans pertinence, sans
légitimité, sans valeur. On peut aller plus loin. Le monde non plus n’est pas à
vendre (c’est ce qui donne raison aux écologistes : « Le monde n’est pas une
marchandise »). Le marché, même, n’est pas à vendre : c’est ce que signifie le
droit du commerce. C’est pourquoi nous avons besoin de politique : parce que
le marché est nécessaire (or il ne peut se développer vraiment que dans un État
de droit) et parce qu’il ne suffit pas. Quelle folie ce serait que d’abandonner au
marché ce qui n’est pas à vendre ! Autant abandonner à l’État, autre folie, la
fabrication du pain et le sourire de ma boulangère.

— Les médicaments, cela s’achète !

— Mais on ne peut accepter qu’ils soient réservés à ceux qui ont les moyens
de les payer. C’est pourquoi on a inventé la Sécurité sociale.

— ... Et les impôts !

— Tu préférerais qu’on compte sur la charité des plus riches pour que les
plus pauvres puissent se soigner ? Autant compter sur la générosité de ta boulangère
%— 357 —
 pour avoir du pain ! Personne ne cotise à la Sécu ou ne paie ses impôts
par générosité. Nous le faisons tous par intérêt, et il faut bien que quelques
contrôles nous y poussent... Moyennant quoi la fiscalité et la Sécurité sociale
ont fait beaucoup plus, pour la justice, que le marché et la générosité réunis. Ce
n'est pas de la morale, mais de la politique : pas de la charité, mais de la
solidarité !

— Peut-être. Mais s’il n’y avait pas le marché pour créer de la richesse, l’État
n'aurait rien à redistribuer...

— Et s’il n’y avait pas l’État pour garantir le droit de propriété et la liberté
des échanges, il n’y aurait pas de marché du tout.

— Alors ne demandons pas à l’État de produire de la richesse : le marché le
fait plus et mieux !

— Ni au marché de produire de la justice : seul l’État a une chance d'y
parvenir !

— Soyons donc libéraux en économie...

— Et solidaires en politique !

\section{Martyr}
%MARTYR
« Je ne crois que les témoins qui se feraient égorger », dit à peu
près Pascal ({\it Pensées}, 822-593). Cela vaut presque comme définition :
un martyr — c’est-à-dire, étymologiquement, un témoin -, c’est
quelqu’un qui se fait tuer pour qu’on le croie. Mais qu'est-ce que cela prouve ?
Plusieurs de ses assassins se feraient volontiers égorger aussi. Tant d’enthousiasme
ou de fanatisme me rendrait plutôt son témoignage suspect : s’il met sa
foi plus haut que sa vie, il y a lieu de craindre qu’il la mette aussi plus haut que
le bon sens et la lucidité. Galilée, sauvant sa peau contre l’Inquisition, m’inspire
autrement confiance : il eût été bien bête de mourir ; la Terre n’en aurait
pas tourné davantage.

En un autre sens, le martyr est seulement celui qu’on assassine ou qu’on
torture. Ce n’est plus un témoin, c’est une victime. Plus besoin d’être d’accord
avec lui ; l’urgent est de lui porter secours. Logique de humanitaire : logique
de la compassion, non de la foi.

\section{Marxisme}
%MARXISME
La doctrine de Marx et d’Engels, puis le courant de pensée,
passablement hétérogène, qui s’en réclame. C’est un matérialisme
dialectique, appliqué surtout à l’histoire : celle-ci serait soumise à des
forces exclusivement matérielles (principalement économiques, mais aussi
sociales, politiques, idéologiques...) et mue par un certain nombre de contradictions
(entre les forces productives et les rapports de production, entre les
%— 358 —
classes, entre les individus...). La lutte des classes est le moteur de l’histoire, qui
mène nécessairement — on reconnaîtra là une {\it aufhebung} très hégélienne — à une
société sans classe et sans État, le communisme, dont nous ne sommes plus
guère séparés que par une dernière révolution et une dernière dictature (celle
du prolétariat)... Science ? Philosophie ? Ce serait l’une et l’autre, qu’on distingue
parfois sous les deux appellations de {\it matérialisme historique} et de {\it matérialisme
dialectique}, dont la conjonction serait le marxisme même. Cela aboutira
à des dizaines de milliers d'ouvrages, aujourd’hui presque tous illisibles,
mais qui forment pourtant, c’est la moindre des choses, un massif théorique
qui reste impressionnant. Quant au corps de la doctrine, Marx en avait donné
lui-même un résumé fameux, qui mérite d’être cité largement :

\vspace{0.5cm}

{\footnotesize « Le mode de production de la vie matérielle conditionne le processus de vie social,
politique et intellectuel en général. Ce n’est pas la conscience des hommes qui détermine
leur être ; c’est inversement leur être social qui détermine leur conscience. À un
certain stade de leur développement, les forces productives matérielles de la société
entrent en contradiction avec les rapports de production existants, ou, ce qui n’en est
que l’expression juridique, avec les rapports de propriété au sein desquels elles s'étaient
mues jusqu'alors. De formes de développement des forces productives qu’ils étaient, ces
rapports en deviennent des entraves. Alors s'ouvre une époque de révolution sociale.
[...] À grands traits, les modes de production asiatique, antique, féodal et bourgeois
moderne peuvent être qualifiés d’époques progressives de la formation sociale économique.
Les rapports de production bourgeois sont la dernière forme contradictoire du
processus de production sociale [...]. Les forces productives qui se développent au sein
de la société bourgeoise créent en même temps les conditions matérielles pour résoudre
cette contradiction. Avec cette formation sociale s’achève donc la préhistoire de la
société humaine » ({\it Critique de l'économie politique}, Préface).
}

\vspace{0.5cm}

J'ai beau ressentir pour Marx beaucoup d’admiration et de sympathie, cette
dernière phrase me fait froid dans le dos. Cette façon d’annuler tout le passé,
qui ne serait que préhistoire, au nom de l’avenir, au nom d’une histoire enfin
véritable mais qui n’aurait pas encore véritablement commenté, j'y reconnais
trop la structure mortifère de l’utopie, cette volonté de donner tort au réel, de
l’invalider, de le réfuter (utopie comme forclusion du réel : comme psychose
historique), avant de fusiller, au nom des lendemains qui chantent, le triste et
pleurnichard aujourd’hui... On me dira qu’on a le droit de rêver, et même
qu’il le faut. Sans doute. Mais faut-il pour cela prétendre que toute veille,
jusque-là, ne fut qu’un long, qu’un très long et très mensonger sommeil ? Et de
quel droit ériger ce rêve en certitude prétendument démontrée ? Que Marx ait
rêvé une autre politique, qu’il Pait désirée, voulue, préparée, ce n’est pas moi
qui le lui reprocherai. Son erreur fut d’y voir une science, sans renoncer pour
%— 359 —
cela à sa vertu prescriptive : le marxisme dirait à la fois la vérité de ce qui est (le
capitalisme) et de ce qui {\it doit} être (le communisme). De là un penchant originellement
dogmatique et virtuellement totalitaire. Staline y fera son lit, ou son
trône. La vérité ne se vote pas, et ne se discute valablement qu'entre esprits
compétents. S’il existe une politique scientifique (or le marxisme, spécialement
dans sa version léniniste, prétendra être cette science), à quoi bon la
démocratie ? Autant voter pour savoir s’il fera beau demain. Et quel scientifique
voudrait respecter les opinions, dans une science donnée, de ceux, même
sincères, qui n’y connaissent rien ? On a le droit de se tromper, mais cela
n’appelle que correction et travail, qu’explications ou redressement — que pédagogie
ou thérapie. Tout désaccord devient l’indice d’un conflit d’intérêts ou
d’une incompréhension : les positions des adversaires sont toujours idéologiquement
suspectes (encore un valet de la bourgeoisie) et scientifiquement
inconsistantes (encore un idéaliste ou un ignorant). Nul n’est réactionnaire
volontairement, ou bien seulement les riches : éliminons ceux-ci, éduquons ou
rééduquons les autres, et plus rien ne séparera l’humanité de la justice et du
bonheur. C’est ainsi qu’une utopie sympathique doublée d’une pensée forte
dériva, dès ses commencements, vers une conception bureaucratique de la politique
(le Parti communiste comme avant-garde scientifique et révolutionnaire
du prolétariat), avant de s’enfoncer, dès qu’elle parvint au pouvoir, dans les
horreurs totalitaires que l’on sait. Cela était-il évitable ? On ne le saura jamais,
sauf à recommencer l'expérience, ce qui ne paraît guère raisonnable. Cela ne
dispense pas de lire Marx et Engels, de les méditer, de les utiliser, parfois, pour
leur vertu explicative ou critique ; mais devrait dissuader de se dire marxiste.
Cette pensée, qui a échoué partout où elle parvint au pouvoir, et presque toujours
criminellement, du moins dans sa version révolutionnaire, a fait trop de
mal pour qu’on puisse s’en réclamer en bloc. Ce n’est qu’une fausse science, qui
n'aura abouti qu’à de vraies dictatures. Il en reste, chez les lecteurs de Marx, un
peu de nostalgie et d’effroi, qui ne sauraient pourtant tenir lieu d’analyse.
Qu’une si belle intelligence ait pu mener, même indirectement, à tant d’horreurs,
c’est une raison pour se méfier de l'intelligence, certainement pas pour
s’en passer.

\section{Masculinité}
%MASCULINITÉ
Voir « Féminité ».

\section{Matérialisme}
%MATÉRIALISME
Toute doctrine ou attitude qui privilégie, d’une façon ou
d’une autre, la matière. Le mot se prend principalement
%— 360 —
en deux sens, l’un trivial, l’autre philosophique. Il s'oppose dans les deux cas à
l’idéalisme, mais considéré lui aussi en deux acceptions différentes.

Au sens trivial, le matérialisme est un certain type de comportement ou
d’état d’esprit, caractérisé par des préoccupations « matérielles », c’est-à-dire ici
sensibles ou basses. Le mot, dans cette acception, est presque toujours péjoratif.
Le matérialiste, c’est alors celui qui n’a pas d’idéal, qui ne se soucie ni de morale
ni de spiritualité, et qui, ne cherchant que la satisfaction de ses pulsions, penche
toujours vers son corps, pourrait-on dire, plutôt que vers son âme. Au mieux :
un bon vivant. Au pire : un jouisseur, égoïste et grossier.

Mais le mot {\it matérialisme} appartient aussi au langage philosophique : il y
désigne l’un des deux courants antagonistes dont l’opposition, depuis Platon et
Démocrite, traverse et structure l’histoire de la philosophie. Le matérialisme,
c’est alors la conception du monde ou de l’être qui affirme le rôle primordial,
voire l'existence exclusive, de la matière. Être matérialiste, en ce sens philosophique,
c’est affirmer que tout est matière ou produit de la matière, et qu’il
n'existe en conséquence aucune réalité spirituelle ou idéelle autonome — ni
Dieu créateur, ni âme immatérielle, ni valeurs absolues ou en soi. Le matérialisme
s'oppose pour cela au spiritualisme ou à l’idéalisme. Il est incompatible,
sinon avec toute religion (Épicure n’était pas athée, les stoïciens étaient panthéistes),
du moins avec toute croyance en un Dieu immatériel ou transcendant.
C’est un monisme physique, un immanentisme absolu et un naturalisme
radical. « Le matérialisme, écrivait Engels, considère la nature comme la seule
réalité » ; il n’est rien d’autre « qu’une simple intelligence de la nature telle
qu’elle se présente, sans adjonction étrangère » ({\it Ludwig Feuerbach et la fin de la
philosophie classique allemande}, I).

On objectera que cette {\it intelligence}, pour la nature, est {\it déjà} une adjonction
étrangère : si la nature ne pense pas, comment pourrait-on la penser sans en
sortir ? Mais Lucrèce avait déjà répondu : de même qu’on peut rire sans être
formé d’atomes rieurs, on peut philosopher sans être formé d’atomes philosophes.
Ainsi la compréhension matérialiste de la nature est produite — comme
toute pensée, vraie ou fausse —, par une matière qui ne pense pas. C’est ce qui
sépare les matérialistes de Spinoza: la nature, pour eux, n’est pas «chose
pensante » (contrairement à ce que suppose la première proposition du livre II
de l'{\it Éthique}), et c’est pourquoi elle n’est pas Dieu. Il n’est de pensée — par
exemple humaine — que {\it dans} la nature, qui ne pense pas.

Être matérialiste, ce n’est donc pas nier l’existence de la pensée — car alors
le matérialisme se nierait soi. C’est nier son absoluité, son indépendance ontologique
ou sa réalité substantielle : c’est considérer que les phénomènes intellectuels,
moraux ou spirituels (ou supposés tels) n’ont de réalité que seconde et
déterminée. C’est où le matérialisme contemporain rencontre la biologie, et

%— 361 —
spécialement la neurobiologie. Être matérialiste, pour les Modernes, c’est
d’abord reconnaître que c’est le cerveau qui pense, que « l’âme » ou « l'esprit »
ne sont que des illusions ou des métaphores, enfin que l'existence de la pensée
(comme Hobbes, contre Descartes, l’avait fortement marqué) suppose assurément
celle d’un être qui pense, mais nullement que cet être soit lui-même une
pensée ou un esprit : autant dire, parce que je me promène, que je suis une promenade
(Hobbes, Deuxième objection aux {\it Méditations} de Descartes). « Je
pense, donc je suis » ? Sans doute. Mais que suis-je ? Une « chose pensante » ?
Soit. Mais quelle chose ? Les matérialistes répondent : {\it un corps}. C’est peut-être
le point, entre les deux camps, où l'opposition est la plus nette. Là où l’idéaliste
dirait : « J’ai un corps », ce qui suppose qu'il soit autre chose, le matérialiste
dira plutôt : « Je {\it suis} mon corps ». Il y a là une part d’humilité, mais aussi de
défi et d’exigence. Les matérialistes ne prétendent pas être autre chose qu’un
organisme vivant et pensant. C’est pourquoi ils mettent la vie et la pensée si
haut : parce qu’ils n’y voient qu’une exception, d’autant plus précieuse qu’elle
est plus rare et qu’elle les constitue. Ils expliquent donc bien, comme Auguste
Comte l'avait vu, {\it le supérieur} (la vie, la conscience, l'esprit) par {\it l’inférieur} (la
matière inorganique, biologiquement puis culturellement organisée), mais ne
renoncent pas pour autant, d’un point de vue normatif, à sa supériorité. Ils
défendent le {\it primat de la matière}, comme disait Marx, mais n’en sont que plus
attachés à ce que j'appelle la {\it primauté de l'esprit}. Que ce soit le cerveau qui
pense, ce n’est pas une raison pour renoncer à penser ; c’en est une au contraire,
bien forte, pour penser le mieux qu’on peut (puisque toute pensée en dépend).
Et de même : que la conscience soit gouvernée par des processus inconscients
(Freud) ou que l'idéologie soit déterminée en dernière instance par l’économie
(Marx), ce n’est pas une raison pour renoncer à la conscience ou aux idées : c’en
est une, au contraire, pour les défendre (puisqu’elles n’existent qu’à cette condition)
et pour essayer de les rendre — par la raison, par la connaissance — plus
lucides et plus libres. Pourquoi autrement faire une psychanalyse, de la politique
ou des livres ?

L'esprit, loin d’être immortel, est cela même qui va mourir. Il n’est pas
principe mais résultat, non sujet mais effet, non substance mais acte, non
essence mais histoire. Il n’est pas absolu mais relatif (à un corps, à une société,
à une époque...) ; il n’est pas être ou vérité, mais valeur ou sens, et fragile toujours.
La mort aura le dernier mot, ou plutôt le dernier silence, puisqu’elle
seule, comme disait Lucrèce, est immortelle. Raison de plus pour profiter de
cette vie unique et passagère. Le pire seul — ou plutôt le rien — nous attend ; le
mieux, toujours, est à inventer. De là cette constante du matérialisme philosophique,
de déboucher sur une éthique de l’action ou du bonheur. C’est ce
qu’Épicure résumait en quatre propositions, qui constituaient son {\it tetrapharmakon}
%— 362 —
(voir ce mot), dont j’adopterais volontiers cette version légèrement
modifiée :

Il n’y a rien à attendre des dieux ;

Il n’y a rien à attendre de la mort ;

On peut combattre la souffrance ;

On peut atteindre le bonheur.

Ou pour le dire plus simplement : cette vie est ton unique chance ; ne la
gâche pas.

\section{Matérialiste}
%MATÉRIALISTE
« J'ai souvent remarqué le contraste, écrivait Alain, entre
les matérialistes, qui sont des esprits résolus, et les spiritualistes,
qui sont des esprits fatigués » (Propos du 29 juin 1929). Celui-là pourtant
n'était pas matérialiste, mais il avait compris ce qu'est, en son fond, le
matérialisme philosophique : une tentative pour {\it sauver l'esprit en niant
l'esprit}, comme il disait à propos de Lucrèce ({\it ibid.}), autrement dit pour
penser l’esprit comme acte, non comme substance, comme valeur, non
comme être, enfin comme création plutôt que comme créateur ou créature.
Et qui pourrait agir, évaluer ou créer, sinon un corps ? Être matérialiste, ce
n’est pas affirmer que l'esprit n’existe pas (il existe, puisque nous pensons) ;
c’est affirmer qu’il n’existe que de façon seconde et déterminée. « Qu'est-ce
donc que je suis ? Une chose qui pense, répondait Descartes, c’est-à-dire un
esprit » ({\it Méditations}, II). C’est ce que le matérialiste refuse. Il dirait plutôt :
Qu'est-ce donc que je suis ? Une chose qui pense, c’est-à-dire un corps pensant.
C’est en quoi Épicure, Hobbes, Diderot, Marx, Freud ou Althusser sont
des matérialistes. Cela ne les empêchait pas d’avoir des idées, ni des idéaux,
mais leur interdisait (ou aurait dû leur interdire) de les ériger en absolus. Le
matérialisme n’est pas une théorie de la matière ; c’est une théorie de l'esprit,
mais comme effet ou comme acte. Ce n’est pas parce que nous avons un
esprit que nous pensons ; c’est parce que nous pensons que nous avons un
esprit.

\section{Matérielle (cause —)}
%MATÉRIELLE (CAUSE -)
L'une des quatre causes selon Aristote et la scolastique :
celle qui explique un être quelconque
(par exemple une statue) par la matière qui le constitue (par exemple le
marbre). Explication toujours insuffisante et toujours nécessaire. Nulle cause
n’agit qu’en transformant une matière ; mais dès qu’elle agit, et toute matérielle
qu’elle puisse être, elle est déjà efficiente.

%— 363 
\section{Mathématique}
%MATHÉMATIQUE
C’est d’abord la science des grandeurs, des figures et
des nombres (voir Aristote, {\it Métaphysique}, M, 3). Puis,
de plus en plus, la science qui sert à penser ou à calculer, de façon hypothético-déductive,
les ensembles, les structures, les fonctions, les relations. Que le réel
lui obéisse, comme le prouve la mathématisation si spectaculaire de la physique,
ne laisse pas de surprendre. Mais c’est qu’il ne lui obéit pas. La feuille qui
tombe d’un arbre, son mouvement peut assurément être calculé de façon
mathématique. Mais ce ne sont pas les mathématiques qui la font tomber, ni
tournoyer. C’est la gravitation, c’est le vent, c’est la résistance de l’air — qui se
calculent, mais ne calculent pas.
Ce n’est pas l’univers qui est écrit en langage mathématique, comme le
voulait Galilée ; c’est le cerveau humain qui écrit dans le langage de l’univers,
qui est sa langue maternelle.

\section{Matière}
%MATIÈRE
On ne confondra pas le concept scientifique, qui relève de la
physique et évolue en même temps qu’elle, avec la notion ou la
catégorie philosophique de matière, qui peut bien sûr évoluer aussi, en fonction
des théories mises en œuvre, mais dont le contenu essentiel, spécialement chez
les matérialistes, reste à peu près constant. La matière, pour la plupart des philosophes,
c’est tout ce qui existe (ou semble exister) en dehors de l’esprit et
indépendamment de la pensée : c’est la partie non spirituelle et non idéelle du
réel. Définition purement négative ? Sans doute. Mais non pas vide. Car de
l'esprit ou de la pensée nous avons une expérience intérieure, laquelle, fût-elle
illusoire, nous permet, par différence, de donner un contenu aussi à la notion
de matière. Si l’on admet — conformément à cette expérience et d’ailleurs
d'accord en cela avec Bergson et la plupart des spiritualistes — que l'esprit et la
pensée vont ensemble, qu’ils se caractérisent par la conscience, la mémoire,
l’anticipation de l’avenir et la volonté (à quoi j’ajouterais volontiers l’intelligence
et l’affectivité), il faut en conclure que la matière, à l'inverse, est sans
conscience ni mémoire, sans projet ni volonté, sans intelligence ni sentiments.
Cela ne nous dit pas ce qu’elle est (c’est aux physiciens de nous l’apprendre)
mais ce que signifie le mot qui la désigne et comment nous pouvons, philosophiquement,
la penser. Qu'est-ce que la matière ? Tout ce qui existe, disais-je,
ou semble exister, en dehors de l'esprit et indépendamment de la pensée : c’est
tout ce qui est sans conscience, tout ce qui ne pense pas (et qui n’a pas besoin
d’être pensé pour exister), tout ce qui est dépourvu de mémoire, d’intelligence,
de volonté et d’affectivité — tout ce qui n’est pas {\it comme nous}, donc, ou du
moins pas comme nous avons le sentiment, intérieurement, d’être. C’est une
définition qui n’est que nominale (la définition réelle relève des sciences), mais
%— 364 —
c’est la seule qui soit philosophiquement nécessaire et d’ailleurs suffisante.
Ondes ou particules ? Masse ou énergie ? Peu importe ici : les ondes, les particules,
la masse ou l’énergie, sauf à les supposer spirituelles (douées de conscience,
de pensée, d’affectivité....), ne sont que des formes, philosophiquement,
de la matière. Il en va de même, notons-le en passant, de ce que les physiciens
ont maladroitement (de leur propre aveu) appelé l’{\it antimatière} : dès lors qu’on
la suppose non spirituelle, elle est aussi matérielle que le reste.

On se trompe donc quand on prétend définir la matière, au sens philosophique
du terme, par des caractéristiques physiques (la matière, ce serait ce qui
se conserve, ce qu’on peut toucher, ce qui est solide, ce qui a une forme, ce qui
a une masse), et l’on a beau jeu, dès lors, de reprocher au matérialiste d’être
dépassé par l’évolution récente de la physique ! Il n’en est bien sûr rien, et
d’ailleurs bien des physiciens se réclament, aujourd’hui encore et peut-être plus
que jamais, de ce courant de pensée que Bernard d’Espagnat et d’autres prétendent
obsolète. La vérité, c’est que l’idée philosophique de matière fait moins
référence à ce qu’est cette dernière (problème scientifique, répétons-le, bien
plus que philosophique) qu’à ce qu’elle n’est pas (l'esprit, la pensée). Problème
de définition, si l’on veut, non d’essence, de consistance ou de structure : de
même qu’un courant d’air n’est pas moins matériel qu’un rocher, une onde
n’est pas moins matérielle qu’une particule, ni l’énergie moins matérielle que la
masse. Et ni la pensée, dans le cerveau humain, moins matérielle que ce cerveau
lui-même. C’est où la boucle se referme : la matière, c’est tout ce qui existe
indépendamment de l'esprit ou de la pensée, y compris (pour le matérialiste)
l'esprit et la pensée. Contradiction ? Nullement, puisque nous savons que la
pensée peut exister sans se penser soi, et même, en chacun, contre sa propre
volonté (essayez un peu d’arrêter de penser). C’est dire que l’esprit n’est pas une
substance mais un acte, que toute pensée suppose un corps (par exemple un
cerveau) qui la pense, enfin que ce dernier dépend à son tour d’une matière qui
le constitue, et qui ne pense pas.

\section{Mauvais}
%MAUVAIS
Ce qui est mal, fait le mal, ou fait du mal. Se prend le plus souvent
en un sens relatif: « Il n’y a pas de {\it mal} (en soi), écrivait
Deleuze à propos de Spinoza, mais il y a du {\it mauvais} (pour moi) » ({\it Spinoza,
Philosophie pratique}, III). Cette distinction, que le latin de Spinoza n’exprime
pas (il écrirait dans les deux cas {\it malum}), est pourtant fidèle à sa pensée. « Bon
et mauvais se disent en un sens purement relatif, une seule et même chose pouvant
être appelée bonne et mauvaise suivant l’aspect sous lequel on la
considère » (TRE, 5) ou la personne qui en use : par exemple, précise l’{\it Éthique},
« la musique est bonne pour le mélancolique, mauvaise pour l’affligé ; pour le
%— 365 —
sourd, elle n’est ni bonne ni mauvaise » (IV, Préface). Le mauvais, en ce sens,
est la vérité du mal, comme le mal n’est que l’hypostase du mauvais.

\section{Mauvaiseté}
%MAUVAISETÉ
Néologisme introduit en français par certains traducteurs
de Kant, pour désigner le fait d’être mauvais, non méchant :
de faire le mal {\it pour son propre bien} (par égoïsme) plutôt que {\it pour le mal} (par
méchanceté). En ce sens, les hommes ne sont jamais méchants ; mais ils sont tous
mauvais, ou peuvent l'être ({\it La religion dans les limites de la simple raison}, I, 3).

\section{Maxime}
%MAXIME
Une formule singulière, pour énoncer une règle ou une vérité générale.
Plus personnelle qu’un proverbe, moins qu’un aphorisme :
c'est comme un proverbe qui aurait un auteur, comme un aphorisme qui aurait
fait oublier le sien.

Chez Kant, le mot désigne le principe subjectif du vouloir ou de l’action.
C'est ce qui distingue la maxime (qui reste singulière) de la loi (qui est
universelle) : la maxime est « le principe d’après lequel le sujet {\it agit} ; tandis que
la loi est le principe objectif, valable pour tout être raisonnable, d’après lequel
il {\it doit} agir » ({\it Fondements...}, II). C’est ce qui justifie la fameuse formulation de
l'impératif catégorique : « Agis uniquement d’après la maxime qui fait que tu
peux vouloir en même temps qu’elle devienne une loi universelle » ({\it ibid.}). C’est
vouloir singulièrement l’universel.

\section{Mécanisme}
%MÉCANISME
Le mot peut désigner un objet ou une doctrine. Comme objet,
c'est un assemblage mobile ou moteur, capable de transformer
ou de transmettre efficacement un mouvement ou une énergie : c’est
une machine élémentaire, ou l’un des éléments d’une machine, de même
qu'une machine est un mécanisme complexe.

Comme doctrine, c’est considérer la nature et tout ce qui s’y trouve comme
un mécanisme, au sens précédent, ou comme un ensemble de mécanismes, au
point que tout puisse s’y expliquer, comme le voulait Descartes, par « grandeurs,
figures et mouvements ». En ce sens restreint, le mécanisme s’oppose traditionnellement
au dynamisme, qui affirme, avec Leibniz et à juste titre, que
figures et mouvements ne suffisent pas, qu’il faut encore prendre en compte un
certain nombre de {\it forces}. Mais rien n’empêche de considérer ces forces comme
faisant partie des grandeurs ci-dessus évoquées : de là un mécanisme au sens
large, qui s'oppose moins au dynamisme qu’il ne l’inclut. Le mécanisme est
alors la doctrine qui veut tout expliquer — au moins s’agissant de la nature — par
%— 366 —
la seule mécanique, au sens scientifique du terme, c’est-à-dire par l’étude des
forces et des mouvements (au sens où l’on parle, par exemple, de mécanique
quantique). En ce sens large, le mécanisme est très proche du matérialisme, ou
le matérialisme, pour mieux dire, n’est qu’un mécanisme généralisé.

\section{Méchanceté}
%MÉCHANCETÉ
Le fait d’être méchant, ou d’agir comme si on l'était. Le
plus souvent, ce n’est qu’égoïsme : celui qu’on croit méchant
n’est que mauvais. Il ne fait pas le mal pour le mal, ni même pour le seul plaisir
de le faire : il ne fait du mal (à l’autre) que pour son bien (à lui), dont le mal
qu’il fait est moins la cause ou l’objet que la condition. Ce n’est qu’un salaud
ordinaire. Si tous les tortionnaires n'étaient que des sadiques, la torture serait
moins répandue, et moins difficile à combattre. Si seuls les méchants faisaient
le mal, le bien aurait tôt fait de l'emporter.

\section{Méchant}
%MÉCHANT
Le méchant est un être paradoxal. Il semble, c’est la définition
traditionnelle du mot, qu’il fasse le mal pour le mal ; mais cela
suppose en lui une perversité déjà réalisée (une nature mauvaise ou diabolique),
qui lexcuse. S'il est méchant par essence, et non par choix, ce n’est pas sa
faute ; aussi n'est-il pas vraiment méchant, mais victime lui aussi (de sa nature
ou de son histoire, peu importe) et par là innocent. Inversement, comment
expliquer qu’il ait pu {\it choisir} d’être méchant, sinon par une méchanceté antécédente,
qui devrait à son tour être expliquée ? Il faut être bien méchant pour
vouloir le devenir. Et l’on retombe dans le premier cas de figure (la méchanceté
perverse et innocente), où la méchanceté s’annule dans sa factualité. Nul n’est
méchant volontairement (puisqu'il faut l’être déjà pour vouloir le devenir) ni
involontairement (puisqu’une méchanceté involontaire n’en serait plus une).
Le méchant, en ce sens fort, est un être paradoxal et impossible. Le diable, pour
parler comme Kant, n'existe pas : il n’y a pas de méchants ; il n’y a que des
mauvais ou des salauds.

De là un sens affaibli, qui est le seul courant : le méchant est celui qui fait
le mal volontairement, non certes {\it pour le mal}, mais {\it pour son plaisir} (qui est un
bien). Ce n’est pas forcément un sadique (la souffrance d’autrui est plus souvent
le moyen que l’objet de son plaisir), mais toujours un égoïste.

Non, pourtant, que tout égoïste soit méchant (nous le serions tous). Est
égoïste qui ne fait pas, pour autrui, tout le bien qu’il devrait ; est méchant qui
lui fait plus de mal qu’il ne pourrait. L’égoïste manque de générosité ; le
méchant, de douceur et de compassion. Les méchants, en ce dernier sens, existent
bien. Mais ils restent l’exception : il y a moins de salauds que de lâches.

%— 367 —
\section{Médiété}
%MÉDIÉTÉ
Un autre mot pour dire le juste milieu ({\it mésotès}) chez Aristote.
Ainsi la vertu est-elle « une médiété entre deux vices, l’un par
excès, l’autre par défaut ». C’est le contraire d’une médiocrité : une perfection
et un sommet, comme une ligne de crête entre deux abîmes, ou entre un abîme
et un marais ({\it Éthique à Nicomaque}, II, 5-6, 1106 a — 1107 a).

\section{Médiocrité}
%MÉDIOCRITÉ
La moyenne, mais considérée dans son insuffisance. C’est
notre état normal, mais ce n’est pas la norme. L’exception
seule, pour l'esprit, mérite de faire règle.

La médiocrité est l'opposé du {\it juste milieu} aristotélicien : non une ligne de
crête, entre deux abîmes, mais un caniveau, comme on faisait dans les rues au
Moyen-Âge, entre deux pentes. Il suffit de se laisser aller pour y glisser.

\section{Médisance}
%MÉDISANCE
Dire le mal qui est, mais pour le plaisir de le dire plutôt que
par devoir de le dénoncer. C’est une sincérité mauvaise (par
différence avec la calomnie, qui serait plutôt une méchanceté mensongère), et
l'un des plaisirs de l'existence.

\section{Méfiance}
%MÉFIANCE
Défiance généralisée et excessive. Le méfiant est incapable de se
fier à quiconque, y compris à ceux qui le mériteraient. Ce n’est
plus prudence ; c’est petitesse.

\section{Meilleur (principe du —)}
%MEILLEUR (PRINCIPE DU —)
C'est un principe leibnizien, selon lequel
Dieu, étant à la fois tout-puissant, omniscient et parfaitement bon,
agit toujours de façon optimale : il voit tous les possibles,
peut réaliser tous les compossibles (voir ce mot), et choisit toujours,
entre eux, le meilleur arrangement. Le monde étant par définition unique
(puisqu'il est la totalité des choses contingentes), il faut en conclure qué notre
monde, même imparfait (s’il était parfait, il ne serait plus le monde : il serait
Dieu), est le meilleur des mondes possibles : Dieu en aurait autrement créé un
autre ({\it Discours de métaphysique}, \S 3 et 4 ; {\it Théodicée}, I, \S 8 à 19, II, \S 193 à
240, IT, \S 413 à 416...). C’est le fondement de l’optimisme leibnizien, dont
Voltaire s’est moqué dans {\it Candide} et dans son {\it Dictionnaire} : voir l’article
« Bien (tout est —) ». On dira que l'ironie ne tient pas lieu de réfutation. Sans
doute, Mais une foi irréfutable n’est pas davantage prouvée par là.
%— 368 

\section{Mélancolie}
%MÉLANCOLIE
L’humeur noire (ou la bile noire) des Anciens. Aujourd’hui,
le mot se prend surtout en deux acceptions. Dans le
langage courant, c’est une tristesse légère et diffuse, sans objet particulier et
pour cela à peu près inconsolable. Dans le vocabulaire psychiatrique, à
l'inverse, c’est un dérèglement pathologique de l'humeur, caractérisé par une
tristesse extrême, souvent mêlée d’anxiété, d’auto-dépréciation, de ralentissement
psychomoteur et d’idées suicidaires. Inconsolable dans les deux cas, donc,
mais pour des raisons plutôt opposées : parce qu’elle est trop légère ou trop
lourde, trop vague ou trop grave, trop « normale » (la mélancolie ordinaire est
moins un trouble qu’un tempérament) ou pas assez. La première peut être
presque agréable («la mélancolie, disait Victor Hugo, c’est le bonheur d’être
triste ») ; la seconde ne l’est jamais : elle relève de la médecine, et peut tuer si
on ne la soigne pas. Toutefois la distinction, entre ces deux états, n’est pas toujours
aussi nette : les tempéraments mélancoliques ne sont pas à l'abri d’une
psychose ou d’une dépression.

\section{Même}
%MÊME
L'expression de l'identité, qu’elle soit numérique (« nous habitons
dans la même rue») ou spécifique (« nous portons la même
cravate »). S’oppose traditionnellement à l’{\it autre}, spécialement depuis Platon
(voir par exemple le {\it Sophiste}, 254-258, et le {\it Timée}, 34-36). Tout être est réputé
le même que lui-même (principe d’identité) et autre que tous les autres (principe
des indiscernables). Mais cela ne l'empêche pas de devenir autre que soi
(impermanence) : le même, dans le temps, n’est jamais qu’une abstraction. À la
gloire d'Héraclite.

\section{Mémoire}
%MÉMOIRE
La conscience présente du passé, que ce soit en puissance (comme
faculté) ou en acte (comme mémoration ou remémoration).
Cette conscience est actuelle, comme toute conscience, mais elle n’est mémoire
qu’en tant qu’elle perçoit, ou peut percevoir, le passé {\it en tant que passé} — sans
quoi ce ne serait plus mémoire mais hallucination. C’est la conscience actuelle
de ce qui ne l’est plus, en tant que cela l’a été.

On évitera de dire que la mémoire est la {\it trace} du passé : d’abord parce
qu’une tache ou un pli, qui sont assurément de telles traces, ne sont pas des
actes de mémoire ; ensuite parce qu’une trace n’est qu’un morceau du présent,
qui n’évoque le passé que pour une conscience. Qu'il y ait des traces du passé
dans le cerveau, et qu’elles contribuent à la mémoire, c’est vraisemblable. Mais
cela n’est un fait de mémoire que pour autant que le cerveau, grâce à elles, produit
%— 369 —
 — ou peut produire — autre chose que des traces : la conscience présente de
ce qui ne l’est plus.

On évitera aussi de dire que la mémoire est une dimension de la conscience.
Elle est bien plutôt la conscience elle-même, laquelle n’est consciente
qu'à la condition de se souvenir continâment de soi ou de ses objets.
Anticiper ? C’est se souvenir qu’on anticipe. Imaginer ? C’est se souvenir qu’on
imagine. Être attentif ? C’est se souvenir qu’on l’est, ou de l'être, ou de ce à
quoi l’on fait attention. Ainsi toute conscience est mémoire : la mémoire n’est
pas seulement « coextensive à la conscience », comme disait Bergson, elle est la
conscience même.

On parle d’un devoir de mémoire. À ce niveau de généralité, cela n’a pas
grand sens. La mémoire est une faculté, point une vertu : le tout est de s’en
servir au mieux, ce qui ne va pas sans sélection, ni donc sans oubli. Comment
la mémoire pourrait-elle y suffire, puisqu’elle en a besoin ? Ce n’est pas une
faute que d’oublier ce qui ne mérite pas d’être retenu, ni même d’oublier ce qui
mériterait d’être mémorisé mais pour des raisons qui ne touchent pas à la
morale (par exemple son numéro de carte bancaire). Le vrai devoir, ce n’est pas
de se souvenir, c’est de {\it vouloir} se souvenir. Et non de tout ou de n’importe
quoi, mais de ce que l’on doit {\it à d'autres} : à cause du bien qu’ils nous ont fait
(gratitude), du mal qu’ils ont subi ou subissent (compassion, justice) ou qu’on
leur a fait soi-même (repentir). Devoirs, non de mémoire, mais de fidélité.
C’est aussi la seule façon de préparer valablement l’avenir. Du passé, ne faisons
pas table rase.

\section{Mensonge}
%MENSONGE
C'est dire, dans l'intention de tromper (et non par antiphrase
ou par ironie), ce qu’on sait être faux. Tout mensonge suppose
un savoir, et au moins l’idée de vérité. C’est en quoi le mensonge récuse
la sophistique, qui l’excuse. Le paradoxe du Menteur (voir ce mot) montre suffisamment
que le mensonge n’est possible qu’à titre d'exception : par où il confirme
la règle même qu’il viole («la norme, dirait Spinoza, de l’idée vraie
donnée ») et qui le rend possible. Tant pis pour les menteurs et les sophistes.

\section{Menteur (paradoxe du —)}
%MENTEUR (PARADOXE DU-)
Épiménide, qui est crétois, dit : « {\it Tous les
crétois sont menteurs.} » Ce qu'il dit est
donc faux, si c’est vrai (puisque alors il ne ment pas), et vrai, si c’est faux
(puisqu'il ment en effet). C’est l’un des paradoxes traditionnels, depuis les
mégariques, de l’autoréférence. Ce n’est vraiment un paradoxe, et non un
simple sophisme, que si l’on donne à l’expression « être menteur » le sens de
%— 370 
« mentir toujours ». C’est pourquoi, en pratique, ce n’en est pas un. La vérité
est que tous sont menteurs, crétois Où non, et qu'aucun ne ment toujours — car
alors on ne pourrait plus mentir. La formule vraiment paradoxale ou aporétique
serait : « {\it Je suis en train de mentir} » ; ou bien : « {\it La phrase que vous lisez en
ce moment est fausse} ». Car chacune de ces deux propositions serait vraie si elle
est fausse, et fausse si elle est vraie : ce serait une violation du principe de
non-contradiction. On remarquera que ce ne serait pas le cas de ces autres
propositions : « {\it Je mentais} » (qui n’est pas autoréférentielle), ou bien : « {\it La
phrase que vous lisez en ce moment est vraie} », qui sont banalement vraies si elles
sont vraies et fausses si elles sont fausses. Cela semble indiquer que l’autoréférence
n’est logiquement valide que sous réserve, c’est la moindre des choses, de
ne pas nier sa propre vérité. Mais c’est qu’aussi ces propositions autoréférentielles,
même correctement formées, restent en l’occurrence à peu près vides.
Dire « Je dis la vérité », c’est ne rien dire. Dis-la plutôt, au lieu de te contenter
de dire que tu la dis!

\section{Mépris}
%MÉPRIS
C'est refuser le respect ou l’attention. Ainsi peut-on mépriser le
danger ou les convenances. Se dit plus souvent vis-à-vis d’un être
humain : c’est alors refuser à quelqu'un le respect qu’on doit ordinairement à
son prochain, soit parce qu’il en semble en l’occurrence indigne, soit parce
qu’on est incapable, à tort ou à raison, de le considérer comme son égal. On
remarquera que si tous les hommes sont égaux en droits {\it et en dignité}, le mépris
est toujours injuste, et méprisable par là.

\section{Mère}
%MÈRE
« Dieu ne pouvant être partout, dit un proverbe yiddish, il inventa
les mères. » Cela m’éclaire sur l’idée de Dieu, comme sur celle de
maternité.

Qu'est-ce qu’une mère ? C’est la femme qui a porté et enfanté. Celle aussi,
presque toujours, qui a aimé son enfant, qui l’a protégé (y compris contre le
père), nourri, bercé, éduqué, caressé, consolé. Il n’y aurait pas d’amour autrement,
ni d'humanité.

On parle depuis longtemps de mères adoptives et de mères biologiques, et
l’on a raison, depuis peu de mères porteuses, et l’on n’a pas tout à fait tort
(quoique l’expression soit atroce). C’est que les deux fonctions d’enfantement
et d'éducation, ordinairement conjointes, ne le sont pas nécessairement. Et
bien sûr l’amour donné importe davantage ici que les gènes transmis. On dira
que voilà, pour un matérialiste, une curieuse idée. Mais c’est que l’amour n’est
pas moins matériel que le reste.
%— 371 —
On discute beaucoup pour savoir si l’amour maternel est un instinct ou un
fait de culture. Un instinct, certainement pas (puisqu’il connaît des exceptions
et n’entraîne guère de savoir-faire). Un fait de culture ? Il le faut bien, même si
celui-ci vient se greffer, selon toute vraisemblance, sur des données biologiques.
La langue non plus n’est pas un instinct ; cela n'empêche pas que le langage soit
une faculté biologiquement déterminée — et qu’on parle aussi, légitimement, de
langue {\it maternelle}. La parole est un avantage sélectif évident. L'amour parental,
spécialement maternel, aussi. Dans les conditions qui furent celles, pendant des
dizaines de milliers d’années, de nos ancêtres préhistoriques, on n’ose imaginer
ce qu'il fallut d'amour, d’intelligence et de douceur, chez les mères, pour que
l'humanité puisse simplement survivre. Il m'est arrivé de dire que l’amour était
une invention des femmes. C’est une boutade, mais qui n’est pas sans rencontrer
quelque chose d’important, sur quoi Freud, à sa façon, a insisté aussi.
Notre première histoire d’amour, pour la quasi-totalité d’entre nous, hommes
ou femmes, a commencé dans les bras de notre mère : la femme qui nous aima
d’abord, sauf exception, et nous apprit à aimer.

Cela ne veut pas dire que les pères n’ont pas d'importance, ce qui serait une
évidente absurdité (quoiqu’ils n’en aient, s’agissant de l’éducation des enfants
et dans certaines cultures, que fort peu), ni qu’ils soient incapables d’aimer, ce
qui serait une évidente injustice (mais en seraient-ils capables s’ils n'avaient été
aimés d’abord ?), mais que leur rôle et leur amour, aussi considérables qu’ils
puissent être, restent en quelque sorte seconds, au moins chronologiquement,
et comme greffés sur une histoire qui les précède et les prépare. Cela vaut pour
l'espèce autant que pour les individus. Romain Gary, en une phrase, a dit
l'essentiel : « L'homme — c’est-à-dire la civilisation —, ça commence dans les
rapports de l'enfant avec sa mère. »

\section{Mérite}
%MÉRITE
Ce qui rend digne d’éloge ou de récompense. On croit souvent
que cela suppose le libre arbitre, mais à tort. Personne ne décide
librement d’avoir du talent ou du génie : faut-il pour cela réserver nos éloges
aux médiocres besogneux ? Il n’est pas sûr que le courage doive davantage au
libre arbitre : faut-il pour cela refuser de l’admirer ou de le récompenser ? Ce
serait une curieuse conception du mérite, qui rendrait Mozart moins admirable
que Salieri (qui se donna peut-être davantage de mal), et la sainteté ou
l’héroïsme moins méritoires que nos efforts, parfois, pour n'être pas tout à fait
méprisables...
Celui qui donne sans plaisir n’est pas généreux, expliquait Aristote : ce
n’est qu’un avare qui se force. Faut-il pour cela — parce qu’il aurait plus de
mérite ! — l’admirer davantage que celui qui donne sans efforts, facilement,
%— 372 —
spontanément, presque sans y penser, parce que l’amour ou la générosité sont
devenus en lui comme une seconde nature ?

L'amour ne se commande pas, remarquait Kant. On ne m'ôtera pourtant
pas de l’idée que l’amour (en tout cas celui qui donne : {\it philia, agapè}) est bien
un mérite, et le plus grand de tous. Pourquoi, autrement, les chrétiens
loueraient-ils leur Dieu ?

\section{Messianisme}
%MESSIANISME
C'est attendre son salut d’un sauveur, au lieu de s’en
occuper soi-même. Le contraire, donc, de la philosophie.

\section{Messie}
%MESSIE
Un sauveur, qui serait envoyé par Dieu. C’est pourquoi on l'attend,
y compris quand on croit qu’il est déjà venu (on attend alors son
retour). De là le messianisme, qui est une utopie religieuse ou une religion de
l’histoire.

\section{Mesure}
%MESURE
Repas de famille. La mère apporte le dessert. Elle demande à son
petit garçon : « Tu en veux beaucoup ? » Et l'enfant de répondre,
les yeux brillants de convoitise : « J’en veux {\it trop}!»

C'était poser le problème de la mesure, par la démesure. De la règle, par sa
transgression. Comment la démesure pourrait-elle annuler ce qu’elle suppose ?
C’est où échoue, peut-être, le romantisme. Mais n’allons pas trop vite. Ce mot
d'enfant, lu il y a longtemps, je ne sais si les enfants d’aujourd’hui peuvent
encore l’apprécier, voire le comprendre. « Trop », dans leur langage, s’est banalisé
au point de se vider à peu près de son sens : ce n’est souvent qu’un synonyme
de « très » ou de « beaucoup ». Ainsi, au sortir d’un cinéma : « Ce film,
c'est {\it trop bien} !» Ou devant un plat dont ils raffolent : « {\it C'est trop bon} !»
Comme si seul l'excès pouvait suffire. Comme si la démesure était la seule
mesure acceptable. Ce n’est qu’une mode, qui passera comme les autres. Elle
dit pourtant quelque chose sur l'enfance et sur l’époque. Le sens de la mesure
s’acquiert peu à peu, et plus ou moins. Les enfants et les modernes y sont peu
portés. Ils préfèrent l'infini. Ils préfèrent la démesure. Il faudra donc qu’ils
changent, puisque la mesure seule — fât-ce pour habiter l'infini — est à notre
portée. Les Grecs le savaient. L’infini c’est l’inaccessible, inachevé, imparfait.
Toute perfection, à l'inverse, suppose un équilibre, une harmonie, une proportion.
« Ni trop ni trop peu », comme dit souvent Aristote. C’est la seule perfection
qui nous soit accessible. Cela vaut aussi en esthétique : « Ce n’est pas assez
qu’une chose soit belle, expliquait Pascal, il faut qu’elle soit propre au sujet,
%— 373 —
qu'il n’y ait rien de trop ni rien de manque. » Et Poussin : « La mesure nous
astreint à ne passer pas outre, nous faisant opérer en toutes choses avec une certaine
médiocrité et modération... » Cette {\it médiocrité}, pas plus que le {\it juste
milieu} d’Aristote, n’a rien de médiocre, au sens moderne du terme. C’est plutôt
le refus de tous les excès, de tous les défauts, comme un archer vise le centre
({\it medium}) de la cible, non sa périphérie ou son dehors. Ce qu'il faut
comprendre ? Que la mesure est à la fois l’exception et la règle. C’est où commence,
peut-être, le classicisme. « Entre deux mots, écrit Paul Valéry, il faut
choisir le moindre. » Esthétique de la mesure, de la litote, comme disait Gide,
de la finitude heureuse. C’est le contraire de l’exagération, de l’emphase, de la
grandiloquence. Apollon contre Dionysos. Socrate contre Calliclès. C’est victoire
sur soi, sur la démesure de ses désirs, de ses colères, de ses peurs. Par quoi
la mesure touche à l’éthique et devient une vertu.

Le mot « mesure », en français, se prend en deux sens. Il désigne d’abord le
fait de mesurer, c’est-à-dire l'évaluation ou la détermination d’une grandeur,
qu'elle soit intensive (par degrés) ou extensive (par quantités). La mesure, qui
peut être objective ou chiffrée, s'oppose alors au non-mesurable, à l’incommensurable,
à l’indéterminable, à tout ce qui est trop petit, trop grand ou trop fluctuant
pour être mesuré. C’est en ce sens qu’on parlera de la mesure d’une distance,
d’une température, d’une vitesse... Mais le mot désigne aussi — sans
doute par abréviation de l’expression « juste mesure », autrement dit par dérivation
du sens premier — une certaine qualité, ou un certain idéal, de modération,
d'équilibre, de proportion, vers lesquels doivent tendre nos œuvres (d’un
point de vue esthétique) comme nos actions (d’un point de vue éthique). La
mesure s'oppose alors à la démesure, et c’est en ce sens qu’on parlera d’un
homme {\it mesuré} : c'est celui qui refuse tous les excès, spécialement ceux de
l’emportement ou du fanatisme. On voit que la mesure, en ce second sens, est
une donnée plutôt subjective, qu’aucune évaluation chiffrée ne saurait suffire à
définir ou à caractériser. C’est ce qu’on peut appeler aussi la modération, que
les Grecs appelaient sophrosunè, et le contraire de leur {\it hubris} (la démesure,
l'excès). La tempérance ? Disons plutôt que la tempérance est une espèce particulière
de mesure ou de modération : c’est la modération dans les plaisirs sensuels,
le contraire autrement dit de ces excès particuliers que sont la goinfrerie,
l’ivrognerie ou la débauche. L'homme mesuré se doit d’être tempérant. Mais il
ne suffit pas d’être tempérant, hélas, pour être mesuré. Savonarole était tempérant.
Robespierre était tempérant. Quelle démesure pourtant dans l’action, la
pensée, le caractère ! La mesure est comme une modération de l’âme ou de tout
l'être, non seulement face au corps et à ses plaisirs, mais face au monde, à la
pensée, à soi. C’est le contraire du fanatisme, de l’extrémisme, de l’emportement
par les passions. C’est pourquoi la mesure séduit peu. On préfère les passionnés,
%— 374 —
les enthousiastes, tous ceux qui se laissent entraîner par leur foi ou
leurs affects. On préfère les prophètes, les démagogues, les tyrans bien souvent,
aux arpenteurs du réel, aux comptables sourcilleux du possible. L'histoire est
pleine de ces enthousiastes massacreurs, qui ont triomphé, sous les acclamations
de la foule, d’esprits plus mesurés. Mais un triomphe ne prouve rien, ou
moins, en tout cas, qu’un massacre. Quelle paix sans mesure ? Quelle justice
sans mesure ? Quel bonheur sans mesure ?

Épicure, disait Lucrèce, « fixa des bornes au désir comme à la crainte ».
C’est que la démesure voue les humains au malheur, à l’insatisfaction, à
l'angoisse, à la violence. Ils en veulent toujours plus ; comment en auraient-ils
jamais assez ? Ils veulent tout ; comment sauraient-ils partager ou se contenter
de ce qu’ils ont ? Le sage épicurien, au contraire, est un homme mesuré. Il sait
limiter ses désirs aux plaisirs effectivement accessibles, ceux qui peuvent le combler,
ceux qui ont en eux-mêmes leur propre mesure, comme sont les plaisirs
du corps (quand ils sont naturels et nécessaires), ou ceux qu'aucune démesure
ne menace (comme sont les plaisirs de l’amitié ou de la philosophie). Sur ce
dernier point, on peut discuter. Inventer un système, comme fit Épicure, prétendre
dire la vérité sur le tout, n’est-ce pas déjà démesure ? Peut-être bien.
Montaigne sera plus mesuré et plus sage. C’est pourquoi sans doute il est plus
actuel. Les systèmes sont tous morts, tous faux, tous abandonnés. La démesure
vieillit mal, même dans la pensée. Dans l’art ? Cela dépend des goûts. Il y a
ceux qui préfèrent Rabelais, et ceux qui préfèrent Montaigne. Mais même la
belle démesure, celle de Rabelais ou Shakespeare, n’est artistique que par la
mesure qui la domine ou la surmonte. « Tout ce en quoi il y a de Part, disait
Platon, participe de quelque manière à la pratique de la mesure » ({\it Le politique},
285 a). Il n’y a pas de livre infini, il ne peut y en avoir, pas de peinture infinie,
pas de sculpture infinie. Une musique infinie ? On peut la concevoir ou la programmer,
par l'ordinateur. Mais l'écouter, non. Mais la jouer, non. L'homme
n’est pas Dieu, c’est par quoi l’humanisme a à voir, toujours, avec la mesure.

C’est ce que Camus dut rappeler, contre « la démesure contemporaine », et
qu’il faut rappeler avec lui : « Toute pensée, toute action qui dépasse un certain
point se nie elle-même ; il y a en effet une mesure des choses et de l’homme »
({\it L'homme révolté}, V). C’est cette mesure que les révolutionnaires, presque toujours,
ont oubliée, ce qui les vouait au terrorisme (tant qu’ils furent dans
l’opposition) ou au totalitarisme (là où ils sont parvenus au pouvoir). Romantisme
de Marx. Romantisme, presque toujours, des révolutionnaires. C’est ce
qui les rend sympathiques et dangereux. Car la démesure séduit, exalte, fascine.
La mesure ennuie. C’est du moins le préjugé romantique ou moderne, qu’il
faut comprendre et vaincre. « Quoi que nous fassions, écrit encore Camus, la
démesure gardera toujours sa place dans le cœur de l’homme, à l’endroit de la
%— 375 —
solitude. Nous portons tous en nous nos bagnes, nos crimes, nos ravages. Mais
notre tâche n’est pas de les déchaîner à travers le monde ; elle est de les combattre
en nous-mêmes et dans les autres. » Contre la barbarie, quoi ? L’action
prudente, réfléchie, déterminée — la mesure, mais résolue.

On m'objectera, et l’on aura raison, qu’il y a aussi des choses qui ne se
mesurent pas, ou mal. C’est vrai dans les sciences : il y a des mesures impossibles,
incertaines ou paradoxales, soit parce qu’elles modifient ce qu’elles doivent
mesurer (relations d’incertitude de Heisenberg, réduction du paquet
d’ondes en mécanique quantique), soit parce qu’elles varient en fonction de
l'échelle utilisée (par exemple si l’on veut mesurer les côtes de la Bretagne).
C’est vrai dans la vie des sociétés : comment mesurer la liberté ou le bonheur
d’un peuple, sa cohésion, sa civilisation ? C’est vrai enfin, et peut-être surtout,
dans la vie des individus. La souffrance ne se mesure pas. Le plaisir ne se
mesure pas. L'amour ne se mesure pas. L'essentiel ne se mesure pas, et c’est
pourquoi la mesure n’est pas l’essentiel.

Ne confondons pas, toutefois, ce qui ne peut être mesuré exactement ou
absolument avec ce qui n’existerait pas ou récuserait toute approche quantitative.
La longueur des côtes bretonnes varie en fonction de l'échelle utilisée
(selon qu’elle prendra ou non en compte telle ou telle crique, tel ou tel rocher,
telle ou telle anfractuosité, telle ou telle dentelure de tel ou tel rocher ou anfractuosité...). Cela ne signifie pas que la Bretagne n’existe pas, ni qu’elle n’ait pas
de côtes, ni que ces côtes ne soient plus longues que celles de la Vendée ou du
Cotentin... Même chose pour les peuples : leur liberté ou leur paix ne sauraient
se mesurer exactement ; cela ne veut pas dire qu’elles n’existent pas, ni
qu’elles soient constantes ou égales... Même chose, enfin, pour les individus.
Le plaisir ne se mesure pas ; mais tous les plaisirs ne se valent pas. La souffrance
ne se mesure pas ; mais il en est de plus grandes que d’autres. L'amour ne se
mesure pas ; mais il peut être plus ou moins fort, plus ou moins grand, plus ou
moins profond... C’est pourquoi, même quand toute mesure objective ou chiffrée
est impossible, la mesure comme vertu reste nécessaire. Il s’agit de proportionner
sa conduite à ce qu’on ressent effectivement ou à ce que le réel requiert.
Ne pas faire l’acteur tragique, comme disait Marc Aurèle, ne pas s’arracher les
cheveux pour des broutilles, ne pas en rajouter, ne pas se laisser emporter ou
dépasser. Résister, donc. Mais pas non plus vivre par défaut ou {\it a minima}, ne
pas s’enfermer dans le déni ou l’insensibilité, ne pas s’interdire d’aimer, de souffrir,
de jouir. Il y a là un art difficile, qu’on n’a jamais fini d’apprendre, qui
est la mesure même. Un art, mais sans artifice (ou avec le moins d’artifices possible)
et d’ailleurs sans œuvre. C’est ce que Pascal appelle « le simple naturel »,
que j’appellerais plus volontiers la justesse : « Ne pas faire grand ce qui est petit,
ni petit ce qui est grand. » C’est comme une justice à la première personne, ou
%— 376 —
de soi à soi. La balance pourrait lui servir, à elle aussi, de symbole. Mais cette
balance c’est l’âme ou le cœur, comme dirait Pascal, qui mesure ce qui ne se
mesure pas. C’est donc le corps (le seul instrument de mesure dont on ne
puisse se passer, celui que tous les autres supposent), mais éduqué, à la fois sensible
et raisonnable, mesureur et mesuré, et c’est ce qu’on appelle l'esprit.

\section{Métamorphose}
%MÉTAMORPHOSE
Changement complet de forme, quand il est assez
rapide pour surprendre. On parle de métamorphose
pour la chenille qui devient papillon, pas pour le nouveau-né qui devient un
vieillard.

\section{Métaphore}
%MÉTAPHORE
Figure de style. C’est une comparaison implicite, qui fait utiliser
un mot à la place d’un autre, en raison de certaines analogies
ou ressemblances entre les objets comparés. Par exemple, mais ils seraient
bien sûr innombrables, lorsque Homère évoque « l'aurore aux doigts de rose »
(ou Baudelaire, homme du nord, « l’aurore grelottante en robe rose et verte »),
ou lorsque Eschyle nous donne à voir, je ne connais pas de plus suggestive évocation
de la Méditerranée, « le sourire innombrable de la mer ». En français, il
est difficile de ne pas songer à la fin de {\it Booz endormi}, de Victor Hugo. C’est la
nuit. Une jeune fille, allongée sur le dos, contemple la lune et les étoiles. Cela
fait comme un bouquet de métaphores :

{\footnotesize
\begin{center}
\begin{tabular}{l}
Tout reposait dans Ur et dans Jérimadeth ; \\
Les astres émaillaient le ciel profond et sombre ; \\
Le croissant fin et clair parmi ces fleurs de l'ombre \\
Brillait à l'occident, et Ruth se demandait, \\
 \\
Immobile, ouvrant l'œil à moitié sous ses voiles, \\
Quel dieu, quel moissonneur de l'éternel été, \\
Avait, en s’en allant, négligemment jeté \\
Cette faucille d’or dans le champ des étoiles. \\
\end{tabular}
\end{center}
}

Lacan a cru reconnaître la métaphore dans le processus freudien de {\it condensation},
tel qu’il apparaît ou se masque dans les rêves et les symptômes. Il y a
substitution, dans les deux cas, d’un signifiant à un autre : « La {\it Verdichtung},
condensation, c’est la structure de surimposition des signifiants où prend son
champ la métaphore » (« L’instance de la lettre dans l'inconscient », {\it Écrits},
p. 511 ; voir aussi p. 506 à 509). Cela ne suffit pas à faire de l’inconscient un
poète, mais peut expliquer, au moins en partie, l'impact sur nous de la poésie
%— 377 —
en général et de la métaphore en particulier. Toutefois il convient de ne pas en
abuser : désigner une chose par ce qu’elle n’est pas ne saurait suffire à exprimer
ce qu’elle est. C’est où la prose et la veille retrouvent leurs droits, ou plutôt
leurs exigences.

\section{Métaphysique}
%MÉTAPHYSIQUE
C’est une partie de la philosophie, celle qui porte sur les

questions les plus fondamentales, disons sur les ques-
tions premières ou ultimes : l’être, Dieu, l’âme ou la mort sont des problèmes
métaphysiques.

Le mot a une origine curieuse, qui fait comme un jeu de mots objectif.
Lorsque Andronicos de Rhodes, au {\footnotesize I$^\text{er}$} siècle avant Jésus-Christ, voulut éditer les
œuvres ésotériques d’Aristote, il regroupa les textes ou traités dont il disposait
en un certain nombre de recueils, qu’il agença comme il put. Pour plusieurs
d’entre eux, le titre semblait s'imposer, en fonction de leur contenu : la physique,
la politique, l'éthique, la connaissance du vivant et des animaux... Dans
lun de ces recueils, il rassembla un certain nombre de textes majeurs, qui portaient
sur la science de l’être en tant qu'être, sur les premiers principes et les
premières causes, sur la substance et sur Dieu, bref sur ce qu’Aristote aurait
appelé plutôt, sil avait dû donner lui-même un titre, des textes de
« philosophie première » (de même que ce que nous appelons en français les
{\it Méditations métaphysiques} de Descartes s’appelait en latin {\it Meditationes de prima
philosophia}). Il se trouve que ce recueil, dans le classement d’Andronicos, venait
après la physique. On prit l'habitude de l’appeler, d’un mot qu’on ne trouve
jamais chez Aristote, {\it Meta ta phusika} : le livre qui vient {\it après la physique}, peut-être
aussi ({\it meta}, en grec, peut avoir ces deux sens) celui qui va {\it au-delà}. Aussi
l'usage s’imposa-t-il, au fil des siècles, d'appeler {\it métaphysique} tout ce qui allait
au-delà de la physique, c’est-à-dire, et plus généralement, au-delà de l’expérience,
donc de la connaissance scientifique ou empirique. C’est le sens qu’elle
a gardé chez Kant, qui la récuse (comme métaphysique dogmatique : comme
connaissance de l’absolu ou des choses en soi) et veut la sauver (comme métaphysique
critique : comme « l’inventaire, systématiquement ordonné, de tout
ce que nous possédons par la raison pure »). C’est le sens qu’elle a toujours,
même si certains, sottement, y mettent des accents d’ironie ou de dédain. Faire
de la métaphysique, c’est penser plus loin qu’on ne sait et qu’on ne peut savoir.
C’est donc penser aussi loin qu’on peut, et qu’on doit. Celui qui voudrait rester
dans les limites strictes de l'expérience ou des sciences, il ne pourrait répondre
à aucune des questions principales que nous nous posons (sur la vie et la mort,
l'être et le néant, Dieu ou l’homme), ni même à celles que nous posent l’expérience
et les sciences elles-mêmes, ou plutôt que nous nous posons à leur
%— 378 —
propos (sont-elles vraies, à quelles conditions et dans quelles limites ?). C’est en
quoi, comme Schopenhauer l’a vu, « l’homme est un animal métaphysique » :
parce qu’il s'étonne de sa propre existence, comme de celle du monde ou de
tout ({\it Le Monde...}, suppl. au livre I, chap. XVII ; le thème de l’{\it étonnement} est
expressément repris d’Aristote : {\it Métaphysique}, A, 2). La plus grande question
métaphysique, de ce point de vue, est sans doute la question de l’être, telle
qu’elle est posée, par exemple, par Leibniz: {\it Pourquoi y a-t-il quelque chose
plutôt que rien ?} Qu’aucun savoir n’y réponde n’interdit pas de la poser, ni n’en
dispense.

\section{Métempsychose}
%MÉTEMPSYCOSE
Le passage de l’âme ({\it psukhè}) d’un corps à un autre.
Croyance traditionnelle en Orient, plus rare en Occi-
dent (quoiqu’on la trouve dans l’orphisme, chez Pythagore ou chez Platon). Il
faut tenir beaucoup à la vie, et bien peu à ses souvenirs, pour y voir une consolation.

\section{Méthode}
%MÉTHODE
Un ensemble, rationnellement ordonné, de règles ou de principes,
en vue d'obtenir un certain résultat. En philosophie, je
n’en connais pas qui soit vraiment convaincante, sinon la marche même de la
pensée, qui est sans règle, ou sans autre règle que soi. Le {\it Traité de la réforme de
l'entendement} de Spinoza, si difficile, si décevant à certains égards, me paraît
pourtant plus utile et plus vrai que les {\it Règles pour la direction de l'esprit} de Descartes,
ou même que le {\it Discours de la méthode}, évident chef-d'œuvre mais qui
ne le doit guère aux quatre préceptes (de l'évidence, de l’analyse, de la synthèse
et du dénombrement) qu’il propose dans sa deuxième partie. S’il y avait une
méthode pour trouver la vérité, cela se saurait et ne serait plus de la philosophie.
Ainsi parle-t-on de {\it méthode expérimentale}, dans les sciences, mais qui se
ramène à quelques banalités sur les rôles respectifs de la théorie et de l’expérience,
des hypothèses et de la falsification. Cela ne tient lieu, même dans les
sciences, ni de génie ni de créativité. Comment cela pourrait-il suffire à la
vérité ? La vraie méthode, explique Spinoza, est plutôt la vérité même, mais
réfléchie et ordonnée :

{\footnotesize
« La vraie méthode ne consiste pas à chercher la marque à laquelle se reconnaît la
vérité après l’acquisition des idées ; la vraie méthode est la voie par laquelle la vérité
elle-même, ou les essences objectives des choses, ou leurs idées (tous ces termes ont même
signification) sont cherchées dans l’ordre dû. [...] De là il ressort que la méthode n’est
pas autre chose que la connaissance réflexive ou l’idée de l’idée ; et, n’y ayant pas d’idée
%— 579 —
d’une idée si l’idée n’est donnée d’abord, il n’y aura donc point de méthode si une idée
n’est donnée d’abord. La bonne méthode est donc celle qui montre comment l'esprit
doit être dirigé selon la norme de l’idée vraie donnée » ({\it T.R.E.} 27).
}\\

Il s’agit moins d’appliquer des règles que d’apprendre à s’en passer : la
vérité suffit et vaut mieux.

\section{Métonymie}
%MÉTONYMIE
C’est une figure de style par laquelle un mot est utilisé à la
place d’un autre, non pas en vertu d’une comparaison implicite,
comme dans la métaphore, mais en raison d’un rapport, plus ou moins
nécessaire ou constant, de voisinage ou d’interdépendance : par exemple quand
l'effet est désigné par sa cause, ou inversement («la pâle mort méêlait les
sombres bataillons »), le contenu par le contenant («La rue assourdissante
autour de moi hurlait »), ou le tout par la partie (si le rapport est purement
quantitatif, du moins au plus ou du plus au moins, c’est alors une synecdoque :
«trente voiles », dans {\it Le Cid}, pour désigner trente navires). Lacan y voit le
principe du {\it déplacement}, tel qu’il s'effectue dans le travail du rêve ou les
symptômes : « La {\it Verschiebung} ou déplacement, c’est plus près du terme allemand
ce virement de la signification que la métonymie démontre et qui, dès
son apparition dans Freud, est présenté comme le moyen le plus propre à
déjouer la censure » («L’instance de la lettre dans l'inconscient », {\it Écrits},
p. 511 ; voir aussi p. 505-506).

\section{Milieu (juste —)}
%MILIEU (JUSTE —)
Voir les articles « Médiété », « Vertu », et « Vice ».

\section{Mimétique (fonction —)}
%MIMÉTIQUE (FONCTION -)
Ce qui nous pousse à imiter ({\it mimeisthai})
ou passe par limitation. C’est une dimension
essentielle du désir. Le rapport entre le sujet désirant et l’objet désiré n’est
pas duel, montre René Girard : c’est un rapport triangulaire, car médiatisé par
le désir de l’autre (je ne désire un objet que parce qu’un autre le désire, que
j'imite ou auquel je m'identifie). C’est ce que Spinoza appelait « limitation des
affects » : « Si nous imaginons qu’une chose semblable à nous et à l'égard de
laquelle nous n’éprouvons aucun affect éprouve de son côté quelque affect,
nous éprouvons par cela même un affect semblable » ({\it Éthique}, III, prop. 27 et
scolie). De là la commisération, qui est limitation d’une tristesse, et l’émulation,
qui est limitation d’un désir, ou plutôt qui « n’est rien autre que le désir
d’une chose engendré en nous de ce que nous imaginons que d’autres êtres
% 380 
semblables à nous en ont le désir » ({\it ibid.}). De là aussi l'envie, qui est limitation
d’un amour et pousse à la haine : « Si nous imaginons que quelqu'un tire de la
joie d’une chose qu’un seul peut posséder, nous nous efforcerons de faire qu’il
n’en ait plus la possession » (prop. 32). C’est vrai spécialement des enfants
({\it ibid.}, scolie), mais les adultes n’y échappent pas : « Les hommes sont généralement
prêts à avoir de la commisération pour ceux qui sont malheureux, à
envier ceux qui sont heureux, et leur haine pour ces derniers est d’autant plus
grande qu’ils aiment davantage ce qu’ils imaginent dans la possession d’un
autre » ({\it ibid.}). Reste à aimer ce que tous peuvent posséder : l'amour de la vérité
(voir {\it Éthique}, IV, prop. 36 et 37, avec les dém. et scolies) nous fait sortir, sinon
de limitation, du moins de l’envie et de la haine.

\section{Mimétisme}
%MIMÉTISME
Le devenir même de l’autre : c’est devenir semblable à ce
qu’on n’est pas, mais par une imitation involontaire, qui
relève davantage de la physiologie ou de l’imprégnation que d’un apprentissage
délibéré. Ainsi le caméléon, se confondant avec son milieu, ou l’enfant, intériorisant
le sien.

\section{Mineure}
%MINEURE
Dans un syllogisme, celle des deux prémisses qui contient le
petit terme. On la place ordinairement en second ; mais ce n’est
que convention : Socrate, si l’on commence par lui, n’en mourra pas moins...

\section{Miracle}
%MIRACLE
«Pour en être réduit à marcher sur les eaux, me dit un jour
Marcel Conche, il faut vraiment n'avoir que de bien piètres
arguments ! » Cela dit à peu près ce qu’il faut penser des miracles. Ce sont des
événements qui paraissent incompréhensibles, qu’on prétend expliquer par une
intervention surnaturelle qui l’est encore davantage. Mais que prouve une
double incompréhension ?

Hume a bien montré ({\it Enquête sur l'entendement humain}, chap. X) qu’un
miracle est par définition plus incroyable que son absence. On vous dit qu’un
homme a ressuscité. La fausseté du témoignage, fût-elle à vos yeux très improbable
(une chance sur cent ? une sur mille ?), l’est de toute façon beaucoup
moins que la résurrection d’un mort (on n’en connaît pas un seul cas vérifié sur
des milliards). Pourquoi est-ce à celle-ci que vous croyez ?

Et quand il y a non pas témoignage mais expérience directe ? Le même
argument vaut aussi. Vous voyez quelqu'un marcher sur les eaux : il est plus
probable que vous ayez une hallucination, ou que ce soit un illusionniste, ou
%— 381 —
qu'il y ait au phénomène une explication naturelle que vous ignorez, qu’il ne
l’est qu’il y ait en effet un miracle, c’est-à-dire une violation, supposée surnaturelle,
de la causalité ordinaire. Qu'est-ce que j’en sais ? C’est que ce ne serait pas
un {\it miracle} autrement, c’est-à-dire un événement par définition tout à fait
improbable ou exceptionnel. La bêtise ou l’aveuglement le sont moins.

Un miracle est un événement incroyable, auquel on croit pourtant, qu’on
juge inexplicable, et qu’on explique pourtant. Croire aux miracles, c’est non
seulement croire sans comprendre, ce qui est le lot ordinaire, mais croire {\it parce
que} Von ne comprend pas ; ce n’est plus foi mais crédulité.

\section{Mirage}
%MIRAGE
Une apparence trompeuse, due aux contrastes de température
entre différentes couches d’air superposées. En un sens plus large,
et par métaphore, c’est « une erreur chérie, comme dit Alain, qui concerne
principalement les événements extérieurs ». Toutefois on ne parle de mirage
que lorsqu'on cesse d’en être dupe.

\section{Misanthrope}
%MISANTHROPIE
La haine ou le mépris de l'humanité, en tant qu’on en fait
partie. Moins grave par là qu’une haine dont on s’exempte
(ainsi la misogynie, chez un homme, ou le racisme, chez celui qui se croit d’une
race supérieure). Molière, qui fit là-dessus l’un de ses chefs-d’œuvre, montre
bien ce qu’il peut y avoir, dans la misanthropie, d’exigence estimable. Toutefois
ce n'est qu'un leurre : aucune exigence n’est vraiment estimable qui porte sur
autrui. Alceste a beau jeu de mépriser ; ce ne sont pas les occasions qui manquent.
Mais à quoi bon ? Que n’exige-t-il plutôt de lui-même la compassion et
la miséricorde ?

\section{Miséricorde}
%MISÉRICORDE
La vertu du pardon : non en annulant la faute, ce qu’on ne
peut ni ne doit, mais en cessant de haïr. On y parvient par
la connaissance des causes, et de soi. « Ne pas railler, disait Spinoza, ne pas
pleurer, ne pas détester, mais comprendre » ({\it Traité politique}, I, 4). La miséricorde
est ainsi le contraire de la misanthropie, ou plutôt son remède.

\section{Misologue}
%MISOLOGUE
Celui qui déteste la raison. C’est ordinairement qu’elle l’a
déçu, remarque Platon ({\it Phédon}, 89 d — 91 a) : il s’en sert
mal, puis lui reproche de ne pas le servir ; il se trompe, puis lui reproche d’être
trompeuse, C’est le défaut commun des sophistes et des imbéciles.

%— 382 —
\section{Modalité}
%MODALITÉ
Ce jour-là je réunissais dans un restaurant, à propos d’un
numéro de revue que nous faisions ensemble, cinq ou six amis.
Parmi eux, A. et F., tous deux connus en khâgne, tous deux, vingt ans plus
tard, brillants historiens de la philosophie, universitaires réputés, penseurs véritables.
Ils ne se sont pas vus depuis plusieurs années, mais je sais qu’ils ont
gardé, l’un pour l’autre, beaucoup d’estime et d’amitié. Ils parlent d’abord de
broutilles, puis, très vite : « Je voudrais te poser une question, annonce F. : est-ce
que tu crois qu’on peut se forger une représentation cohérente du monde,
sans les catégories de la modalité ? » Silence de plusieurs secondes. A. tire sur sa
pipe. Réfléchit. Puis répond simplement : « Non. » Comme retrouvailles de
vieux copains, j'ai connu plus sentimental, et j'aurais apprécié, je crois, davantage
d'intimité, de confidences, d'émotion... Mais j’admirais aussi, j'admire
toujours, cette façon d’aller droit à l'essentiel, au moins celui de la pensée, cette
intellectualité vraie, dont je savais bien qu’elle n’excluait pas les sentiments,
mais qui ne consentait pas à différer pour cela le débat philosophique...
Quelques années plus tard, je reparlerai à l’un des deux de sa question : il Pavait
oubliée, comme la réponse de l’autre. {\it Sic transit gloria mentis}.

Qu'est-ce que la modalité ? Une modification du jugement, ou plutôt de
son statut. « La modalité des jugements, soulignait Kant, en est une fonction
tout à fait spéciale qui a ce caractère de ne contribuer en rien au contenu du
jugement, [...] mais de ne concerner que la valeur de la copule par rapport à la
pensée en général. Les jugements sont {\it problématiques} lorsqu'on admet l’affirmation
ou la négation comme simplement {\it possibles} ; {\it assertoriques} quand on les
considère comme {\it réelles} (vraies) ; {\it apodictiques} quand on les regarde comme
{\it nécessaires} » ({\it C. R. Pure}, Analytique des concepts, chap. I). Cela débouche, chez
Kant, sur les trois catégories, ou plutôt sur les trois paires de catégories, de la
modalité : {\it possibilité et impossibilité, existence et non-existence, nécessité et contingence}.
Pourquoi peut-on envisager de s’en passer ? Parce qu’elles ne portent pas
sur l’objet (comme les catégories de la quantité ou de la qualité), ni sur les rapports
entre les objets (comme les catégories de la relation), mais simplement sur
le rapport de notre entendement à ces objets. À ne penser que le monde même,
si cela se peut, il semble que le réel y soit tout — que l'existence, comme dirait
Kant, en soit la seule modalité envisageable. Mais alors tout le possible serait
réel, tout le réel serait nécessaire, l'impossible et le contingent ne seraient rien
(ou n’auraient d’existence qu’imaginaire). C’est le monde, à peu près, de Spinoza.
C’est le monde, à peu près, des stoïciens. Est-il cohérent ? Je le crois. Se
passe-t-il des catégories de la modalité ? Pas tout à fait. Mais il les met à leur
place : les unes du côté de l’être ou de Dieu (réalité, possibilité, nécessité : seul
le réel est possible, et il est toujours nécessaire, par quoi ces trois catégories, à la
limite, n’en font qu’une), les autres du côté des êtres de raison ou d’imagination
%— 383 —
(l’impossible, l’inexistant, le contingent : ce ne sont que des façons de
penser ce qui n’est pas). C’est ce qui lui permet d’être cohérent. Non qu’un
monde ne puisse exister sans ces catégories, mais parce que nous ne pourrions,
sans elles, le penser. Ainsi toute représentation cohérente du monde doit intégrer
les catégories de la modalité (puisque notre pensée fait partie du monde),
sans avoir besoin pour cela d’en faire des formes de l'être. Je ne peux penser le
monde sans distinguer le possible du réel ou de l’impossible ; mais cela ne
signifie pas que le monde, lui, les distingue. Ma pensée fait partie du monde,
non le monde, de ma pensée.

\section{Mode}
%MODE
Au masculin : une manière, une façon d’être ou une modification,
mais inessentielles (par différence avec lattribut). « J'entends par
{\it attribut}, écrit Spinoza, ce que l’entendement perçoit d’une substance comme
constituant son essence. J'entends par {\it mode} les affections d’une substance,
autrement dit ce qui est dans une autre chose, par le moyen de laquelle il est
aussi conçu » ({\it Éthique}, I, déf. 4 et 5). Un mode, chez Spinoza, est donc un être
quelconque, en tant qu’il est la modification, dans un attribut donné, de la
substance unique. Il y a des modes finis (cet arbre, cette chaise, vous, moi...) et
des modes infinis (l’entendement de Dieu, le mouvement et le repos, l'univers
entier....). Les premiers, pour finis qu’ils soient, n’en sont pas moins réels : ce
sont des êtres finis mais ce sont des êtres vrais, comme autant de fragments de
l'infini ou de labsolu.

Au féminin, le mot désigne une manière collective et provisoire de se comporter,
par exemple de voir les choses, de parler, de penser, de s’habiller...
D'où la terrible formule de Pascal : « Comme la mode fait l’agrément, aussi
fait-elle la justice» ({\it Pensées}, 61-309). C’est que toute justice, au moins
humaine, est collective et provisoire. En pratique, on réservera pourtant le mot
à ce qui change spécialement vite, sans autre justification apparente que ce
changement même. C’est pourquoi on dit que «la mode, c’est ce qui se
démode » : la fugacité fait partie de sa définition. De là cette surprise, quand on
voit des photos ou des magazines d'il y a vingt ans. Toutefois Mozart et
Molière, qui furent fort à la mode, ne se démodent pas.

Toute mode est normative : elle exprime ce qui se fait, mais est vécue (par
ceux qui la suivent) comme indiquant ce qui {\it doit} se faire. C’est une normativité
fugace, ou une fugacité normative. L'enjeu principal — outre la dimension
purement commerciale — est de distinction : « La mode, écrit excellemment
Edgar Morin, est ce qui permet à l'élite de se différencier du commun, d’où son
mouvement perpétuel, [mais aussi] ce qui permet au commun de ressembler à
l'élite, d’où sa diffusion incessante. »

%— 384 —
\section{Modération}
%MODÉRATION
La mesure (voir ce mot) dans la pensée ou l’action. On ne
la confondra pas avec la petitesse. Un républicain modéré
n’est pas moins républicain que les extrémistes qui le combattent ou le méprisent.
Il le sera souvent davantage. On dira qu’un « révolutionnaire modéré » est
une contradiction dans les termes. Je n’en suis pas sûr (voyez Condorcet ou
Desmoulins) ; mais si c'était vrai, il faudrait en conclure que toute révolution
est excessive — ce qui donnerait raison aux conservateurs ou aux réformistes.
La modération n’est pas le contraire de la force, de la grandeur ou de la
radicalité ; c’est le contraire de l'excès ou de labus. C’est pourquoi elle est
bonne en tout. « La sagesse a ses excès, disait Montaigne, et n’a pas moins
besoin de modération que la folie. »

\section{{\it Modus ponens}}
%{\it MODUS PONENS}
C'est une inférence valide, qui fait passer de la vérité
d’une proposition à la vérité de sa conséquence néces-
saire, sous la forme : Si {\it p} alors {\it q} ; or {\it p} ; donc {\it q} (par exemple : Si Socrate est un
homme, il est mortel ; or Socrate est un homme ; donc Socrate est mortel).

\section{{\it Modus Tollens}}
%MODUS TOLLENS
Une inférence valide, qui conclut à la fausseté d’une
proposition par celle de l’une au moins de ses consé-
quences, sous la forme : Si {\it p} alors {\it q} ; or {\it non-q} ; donc {\it non-p} (par exemple : Si
Socrate est un dieu, il est immortel ; or il n’est pas immortel ; donc il n’est
pas un dieu). C’est cette inférence déductive qui est au cœur, selon Popper,
de la {\it falsification} et donc de la démarche des sciences expérimentales. Si la
prévision {\it q} est une conséquence nécessaire de la théorie (ou de l’hypothèse)
{\it p}, il suffit en effet d’un seul fait attestant la fausseté de {\it q} pour entraîner la
fausseté de {\it p}.

\section{M{\oe}urs}
%MŒURS
Les actions humaines, surtout les plus répandues, considérées comme
objets de connaissance ou de jugement. Et bien sûr la connais-
sance vaut mieux, qui dissuade de juger. C’est pourquoi nos plus grands moralistes
sont si peu moralisateurs.

\section{Moi}
%MOI
Le sujet, mais en tant qu’objet : c’est le nom français ou commun de
l'{\it ego}. C’est aussi, et par là même, un objet (ou un processus) qui se
prend pour un sujet. Le moi n’est pas une substance, ni un être : il n’est « ni
%— 385 —
dans le corps ni dans l’âme » (Pascal, {\it Pensées}, 688-323) ; il n’est que l’ensemble
des qualités qu’on lui prête ou des illusions qu’il se fait sur lui-même.

Dans le vocabulaire psychanalytique, le même pronom substantivé désigne
l’une des trois instances de la seconde topique de Freud : le pôle conscient
(quoique incomplètement) de l’appareil psychique, qui fluctue et se bat comme
il peut entre les pulsions du ça, les exigences du surmoi et les contraintes de la
réalité. Instance d'équilibre, mais instable : quelque chose comme le fléau de la
balance ou le dindon de la farce.

La construction du moi n’en est pas moins un processus nécessaire, toujours inachevé.
C’est que le moi est à lui-même son propre but : « {\it Wo es war},
écrivait Freud, {\it soll ich werden} » ; là ou {\it ça} était, {\it moi} je dois advenir. Non que le
moi doive « déloger le ça », comme on l’a longtemps traduit (le ça garde bien
sûr sa place, qui est fondamentale), mais en ceci que le moi ne saurait pourtant
la lui abandonner toute : car il n'existe, comme instance spécifique, que pour
autant qu'il se construit, et il ne peut le faire que contre le ça et le réel — tout
contre. {\it Devenir ce que l'on est}, comme disait Nietzsche ? Devenir, plutôt, ce que
l’on veut être, mais que l’on est déjà, en effet, par cette volonté même, sans
qu’on ait pourtant jamais fini de le devenir et sans qu’on puisse davantage y
renoncer. Non déloger le ça, donc, mais lui résister et le surmonter, au moins
par endroits. « C’est là une tâche qui incombe à la civilisation, concluait Freud,
tout comme l’assèchement du Zuyderzee » ({\it Nouvelles conférences}, III).

\section{Monade}
%MONADE
C'est une unité ({\it monas}) spirituelle. Le mot n’est plus guère
employé que dans un registre leibnizien. La monade est « une
substance simple, qui entre dans les composés; simple, c’est-à-dire sans
parties » ({\it Monadologie}, \S 1). Un atome ? Non pas, si l’on entend par là un être
matériel. Les monades sont des substances spirituelles, et uniquement
spirituelles : des âmes absolument simples, donc impérissables, qui peuvent être
douées ou non de conscience, mais dont chacune exprime à sa façon l’univers
que leur ensemble constitue. Le leibnizianisme est un panpsychisme pluraliste :
«tout vit, tout est plein d’âme », comme dira Hugo, mais dans l’irréductible
multiplicité de substances individuelles séparées et tout intérieures (« sans porte
ni fenêtres »). C’est en quoi une monadologie est l’homologue spiritualiste de
l’atomisme, donc aussi son contraire.

\section{Monarchie}
%MONARCHIE
Le pouvoir d’un seul, mais conformément à des lois (par différence
avec le despotisme, qui ne se soumet à aucune règle).
On parlera de monarchie absolue quand ces lois dépendent elles-mêmes de la
%— 386 —
volonté du monarque (quand le monarque, autrement dit, est souverain) ; et de
monarchie limitée ou constitutionnelle lorsque c’est au contraire le monarque
qui est soumis aux lois (et spécialement quand c’est le peuple qui est souverain).
On voit que la monarchie absolue est très proche du despotisme (c’est un
despotisme bien réglé), comme la monarchie constitutionnelle peut n'être
qu’une forme de démocratie. En Angleterre ou en Espagne, aujourd’hui, c’est
évidemment le peuple qui est souverain : le monarque règne, dans ces pays ; il
ne gouverne ni ne légifère. Le roi, dans une monarchie constitutionnelle, n’est
pas souverain ; il n’est que le symbole, tant que le peuple y consent, de la
nation ou de la souveraineté.

\section{Monde}
%MONDE
Dans la langue philosophique, c’est souvent un synonyme d’uni-
vers : le monde est « l’assemblage entier des choses contingentes »
(Leibniz), l’ensemble de « tous les phénomènes » (Kant) ou de « tout ce qui
arrive» (Wittgenstein). Mais alors l’idée d’une pluralité des mondes, bien
attestée dans l’histoire de la philosophie, devient inintelligible : comment y
aurait-il plusieurs {\it tout} ? Il convient donc de distinguer le monde (le {\it kosmos} des
Grecs) du Tout ({\it to pan}). Pour les Anciens, le monde est un tout, mais pas {\it le}
Tout. C’est l’ensemble ordonné qui nous contient, tel que nous pouvons
l’observer, depuis la Terre jusqu’au ciel et aux astres. Il n’est pas inenvisageable
qu’il en existe d’autres, voire un nombre infini (c’est ce que pensait Épicure).
Mais nous ne pouvons les connaître, faute d’en avoir la moindre expérience.

Quand on parle du monde, sans plus de précision, il est entendu que c’est
le nôtre. C’est l’ensemble, qui nous contient, de tout ce avec quoi nous sommes
en relation, de tout ce que nous pouvons constater ou expérimenter —
l’ensemble des faits, plutôt que des choses ou des événements. C’est le réel qui
nous est accessible : une petite portion de l’être, valorisée (pour nous) par notre
présence. C’est notre lieu de coïncidence, ou le paquet-cadeau du destin. Après
tout, nous aurions pu tomber plus mal.

Les scientifiques l’appellent parfois l’univers, qui serait le tout du réel. Mais
comme nous ne le connaissons jamais qu’en partie, et comme nous ne connaissons
rien d’autre, comment savoir s’il est le tout ?

\section{Monème}
%MONÈME
Une unité minimale de signification. Par exemple le mot {\it monde}
n’en comporte qu’un (si on le divise, le sens se perd) ; le mot
{\it monisme}, deux : {\it mon} (qui différencie par exemple {\it monisme} de {\it dualisme}) et {\it isme}
(qui le différencie par exemple de {\it monarchie}) ; la phrase « Vous êtes embarqués »,
%— 387 —
cinq. C’est l'élément de la première articulation (voir ce mot), comme
le phonème est celui de la seconde.

\section{Monisme}
%MONISME
Toute doctrine pour laquelle il n’existe qu’une seule substance,
ou qu’un seul type de substances. Un monisme peut être maté-
rialiste, s’il affirme que toute substance est matérielle (ainsi chez les stoïciens,
Diderot ou Marx), spiritualiste, s’il les juge toutes spirituelles (ainsi chez Leibniz
ou Berkeley), ou ni l’un ni l’autre, si matière et pensée ne sont pour lui que
des modes ou des attributs d’une substance unique, qui ne saurait dès lors se
réduire à l’une ou l’autre (c’est le cas, spécialement, chez Spinoza). S'oppose en
tous les cas au dualisme, qui pose l’existence de deux types de substances (Descartes)
ou de deux mondes (Platon, Kant). Devrait pouvoir s’opposer aussi au
pluralisme, qui supposerait, pris en son sens fort, l’existence d’un nombre indéfini
de substances de natures différentes. Mais l'imagination des philosophes ne
s’est guère étendue jusque-là. Nous savons, au moins vaguement, ce que c’est
qu'un corps et qu’un esprit. Mais de substances qui ne seraient ni l’un ni
l’autre, ni l'unité indissoluble des deux, nous n’avons aucune expérience. Comment
pourrions-nous les penser ?

\section{Monnaie}
%MONNAIE
Instrument de paiement : un petit morceau du réel, qui peut
être échangé contre la plupart des autres — à condition toutefois
que quelqu'un les possède et soit prêt à les vendre. C’est « l'équivalent
universel », comme disait Marx, qui libère le commerce du troc et la richesse de
l'encombrement.

\section{Monothéisme}
%MONOTHÉISME
La croyance en un Dieu unique. Les Modernes ont le sentiment
qu’il serait autrement moins divin, son pouvoir se
trouvant inévitablement limité par celui, dans le polythéisme, des autres dieux.
Le fait est que les plus hautes pensées du divin, dès l’Antiquité (chez Platon,
chez Aristote, chez Plotin...), ont eu tendance à le penser comme unité, au
moins en son sommet, et comme unicité : le Bien en soi, le Premier Moteur
immobile ou l’'Un ne sont guère susceptibles d’exister au pluriel. J'y vois plutôt
un progrès. Moins il y a de dieux, mieux ça vaut.

On a pourtant beaucoup reproché au monothéisme, ces dernières décennies,
de mener directement au monolithisme, au totalitarisme, à l'exclusion de
l’autre, de la pluralité, de la différence, du multiple... Que ce thème ait suscité
tellement d’enthousiasme à l’extrême droite me le rendrait déjà suspect. Qu'il
%— 388 —
ait été à ce point démenti par l’histoire (car enfin qui ne voit que les deux
grands totalitarismes de ce siècle n’avaient, pour le monothéisme, que haine ou
mépris ?) n’est pas non plus à son avantage. Mais il y a autre chose, qui est
l’universel. S’il n’existe qu’un seul Dieu, c’est donc le même pour tous : nous
voilà tous frères, en tout cas susceptibles de le devenir, tous ouverts à la même
vérité, tous soumis, au moins en droit, à la même loi. Totalitarisme ? Mais alors
il faudrait parler aussi de totalitarisme pour les sciences, qui sont les mêmes
pour tous, pour la morale, qui tend à l’être, enfin pour les droits de l’homme,
qui n’ont de sens qu’universels. Et au nom de quoi? De Zeus, d’Arès ou
d’Aphrodite, d’Odin ou de Thor, de tout ce grand guignol de Olympe ou du
Walhalla ? Mieux vaut l'immense absence, comme disait Alain, partout présente.
Voilà le monde vidé de tous ses dieux, et bientôt rendu à lui-même.

\section{Morale}
%MORALE
L'ensemble de nos devoirs, autrement dit des obligations ou des
interdits que nous nous imposons à nous-mêmes, indépendamment
de toute récompense ou sanction attendue, et même de toute espérance.
Imaginons qu’on nous annonce la fin du monde, certaine, inévitable, pour
demain matin. La politique n’y survivrait pas, qui a besoin d’un avenir. Mais la
morale ? Elle demeurerait pour l'essentiel inchangée. La fin du monde, même
inévitable à très court terme, n’autoriserait en rien à se moquer des infirmes, à
calomnier, à violer, à torturer, à assassiner, bref à être égoïste ou méchant. C’est
que la morale n’a pas besoin d’avenir. Le présent lui suffit. Elle n’a pas besoin
d'espérance. La volonté lui suffit. « Une action accomplie par devoir tire sa
valeur {\it non pas du but} qui doit être atteint par elle, souligne Kant, mais de la
maxime d’après laquelle elle est décidée. » Sa valeur ne dépend pas de ses effets
attendus, mais seulement de la règle à laquelle elle se soumet, indépendamment
de tout penchant, de toute inclination, de tout calcul égoïste, enfin « sans égard
à aucun des objets de la faculté de désirer » et « abstraction faite des fins qui
peuvent être réalisées par une telle action » ({\it Fondements}..., 1). Si tu agis pour la
gloire, pour ton bonheur, pour ton salut, et quand bien même tu agirais en
tout conformément à la morale, tu n’agis pas encore moralement. Une action
n’a de valeur morale véritable, explique Kant, que dans la mesure où elle est
désintéressée. Cela suppose qu’elle n’est pas seulement accomplie {\it conformément
au devoir} (à quoi l'intérêt peut suffire : ainsi le commerçant qui n’est honnête
que pour garder ses clients), mais bien {\it par devoir}, autrement dit par pur respect
de la loi morale ou, cela revient au même, de l’humanité. La proximité de la fin
du monde n’y changerait rien d’essentiel : nous serions toujours tenus, et
jusqu’au dernier instant, de nous soumettre à ce qui nous paraît universellement
valable, universellement exigible, et spécialement (mais à nouveau cela
%— 389 —
revient au même) de respecter l'humanité en nous et en l’autre. C’est en quoi
la morale est désespérée, au moins en un certain sens, et désespérante peut-être.
« Elle n’a aucunement besoin de la religion », insiste Kant, ni de quelque fin ou
but que ce soit : «elle se suffit à elle-même » ({\it La religion dans les limites de la
simple raison}, Préface). C’est en quoi elle est laïque, même chez les croyants, et
commande, c'est du moins le sentiment que nous avons, absolument. Que
Dieu existe ou pas, qu'est-ce que cela change au devoir de protéger les plus
faibles ? Rien, bien sûr, et c’est pourquoi on n’a pas besoin de savoir ce qu’il en
est de cette existence pour agir humainement.

Imaginons à l'inverse, c’est un exemple qu’on trouve chez Kant, que Dieu
existe et soit connu de tous. Que se passerait-il ? « Dieu et l’éternité, avec leur
majesté redoutable, seraient sans cesse devant nos yeux. » Nul n’oserait plus
désobéir. La peur de l’enfer et l’espérance du paradis donneraient aux commandements
divins une force sans pareille. Ce serait le règne en tout de la soumis-
sion intéressée ou craintive, comme un ordre moral absolu : « La transgression
de la loi serait bien sûr évitée, ce qui est ordonné serait accompli. » Mais nulle
{\it moralité} n’y survivrait : « La plupart des actions conformes à la loi seraient produites
par la crainte, quelques-unes seulement par l’espérance, et aucune par
devoir, de sorte que la valeur morale des actions, sur laquelle seule reposent la
valeur de la personne et même celle du monde aux yeux de la suprême sagesse,
n'existerait plus » ({\it C. R. Pratique}, Dialectique, II, 9). Non seulement il n’est
pas besoin d’espérer pour faire son devoir, mais on ne le fait vraiment qu’à la
condition que ce ne soit pas par espérance.

Où veux-je en venir ? Simplement à ceci : la morale, contrairement à ce
qu'on croit souvent, n’a rien à voir avec la religion, encore moins avec la peur
du gendarme ou du scandale. Ou si elle fut liée, historiquement, aux Églises,
aux États et à l’opinion publique, elle ne devient vraiment elle-même — c’est
l’un des apports décisifs des Lumières — que dans la mesure où elle s’en libère.
C’est ce qu'ont montré, chacun à sa façon, Spinoza, Bayle ou Kant. C’est ce
que Brassens, quand j'avais quinze ans, suffit à me faire comprendre. Ramenée
à son essence, la morale est le contraire du conformisme, de l’intégrisme, de
l’ordre moral, y compris sous leurs formes molles qu’on appelle aujourd’hui le
« politiquement correct ». Elle n’est pas la loi de la société, du pouvoir ou de
Dieu, encore moins celle des médias ou des Églises. Elle est la loi que l'individu
se prescrit à lui-même : c’est en quoi elle est libre, comme dirait Rousseau
(« l’obéissance à la loi qu’on s’est prescrite est liberté »), ou autonome, comme
dirait Kant (parce que l'individu n’y est soumis qu’à « sa législation propre, et
néanmoins universelle »). Que cette liberté ou cette autonomie ne soient elles-mêmes
que relatives, c’est ce que je crois, contre Kant et Rousseau, mais qui ne
change pas le sentiment que nous avons d’un absolu pratique (qui relève de la
%— 390 —
volonté, non de la connaissance) et d’une exigence, pour nous, inconditionnelle.
Que toute morale soit historique, j’en suis convaincu. Mais, loin que cela
supprime la morale, c’est au contraire ce qui la fait exister et nous y soumet :
puisque nous sommes {\it dans} l’histoire et produits par elle. Autonomie relative,
donc, mais qui vaut mieux que l’esclavage des penchants ou des peurs. Qu’est-ce
que la morale ? C’est l’ensemble des règles que je m'impose à moi-même, ou
que je devrais m’imposer, non dans l’espoir d’une récompense ou la crainte
d’un châtiment, ce qui ne serait qu’égoïsme, non en fonction du regard
d’autrui, ce qui ne serait qu'hypocrisie, mais au contraire de façon désintéressée
et libre : parce qu’elles me paraissent s’imposer universellement (pour tout être
raisonnable) et sans qu’on ait besoin pour cela d’espérer ou de craindre quoi
que ce soit. « {\it Tout seul}, disait Alain, {\it universellement}. » C’est la morale même.

Cette morale est-elle {\it vraiment} universelle ? Jamais complètement sans
doute : chacun sait bien qu’il existe des morales différentes, qui varient selon les
lieux et les époques. Mais elle est universalisable sans contradiction, et d’ailleurs
de plus en plus universelle en fait. Si on laisse de côté quelques archaïsmes douloureux,
qui doivent plus à des pesanteurs religieuses ou historiques qu’à des
jugements proprement moraux (je pense spécialement à la question sexuelle et
au statut des femmes), force est de reconnaître que ce qu’on entend par « {\it un
type bien} », en France, n’est pas très différent — et le sera sans doute de moins
en moins — de ce que les expressions équivalentes peuvent désigner en Amérique
ou en Inde, en Norvège ou en Afrique du Sud, au Japon ou dans les pays
du Maghreb. C’est quelqu'un qui est sincère plutôt que menteur, généreux
plutôt qu'égoïste, courageux plutôt que lâche, honnête plutôt que malhonnête,
doux ou compatissant plutôt que violent ou cruel... Cela ne date pas d’hier.
Rousseau déjà, contre le relativisme montanien, ou plutôt contre la vision qu'il
en avait, en appelait à une forme de convergence morale, à travers les différentes
civilisations : « O Montaigne ! toi qui te piques de franchise et de vérité,
sois sincère et vrai, si un philosophe peut l’être, et dis-moi s’il est quelque pays
sur la terre où ce soit un crime de garder sa foi, d’être clément, bienfaisant,
généreux ; où l’homme de bien soit méprisé, et le perfide honoré ? » Montaigne
n’en eût guère trouvé, ni cherché : relisez ce qu’il écrit sur les indiens d'Amérique,
que nous avons si atrocement traités, sur leur courage, sur leur constance, sur
leur « bonté, libéralité, loyauté, franchise » ({\it Essais}, III, 6). L’humanité n’appartient
à personne : le relativisme montanien est aussi un universalisme, et ce
n’est en rien contradictoire (puisque la morale est relative à toute l'humanité :
« chaque homme porte la forme entière de l’humaine condition », III, 2). Au
reste l’histoire, sur tous les continents, parle assez clair. Nul ne sait quand la
morale a commencé ; mais cela fait deux ou trois mille ans, selon les différentes
régions du globe, que l'essentiel a été dit : par les prêtres égyptiens ou assyriens,
%— 391 —
par les prophètes hébreux, par les sages hindous, enfin, c’est l’étonnante floraison
des sixième et cinquième siècles avant notre ère, par Zarathoustra (en
Iran), Lao-tseu et Confucius (en Chine), le Bouddha (en Inde), et en Europe
par les premiers philosophes grecs, ceux qu’on appelle les présocratiques.. Qui
ne voit que leurs messages moraux, par-delà d'innombrables oppositions philosophiques
ou théologiques, sont fondamentalement convergents ? Qui ne voit
que c’est encore plus vrai aujourd’hui ? Prenez l'Abbé Pierre et le Dalaï-Lama. Ils
n'ont pas la même origine, pas la même culture, pas la même religion... Mais il
suffit de les écouter quelques minutes pour constater que les morales qu’ils professent
vont dans la même direction. La mondialisation n’a pas que des mauvais
côtés, et elle a commencé bien plus tôt qu’on ne le croit. Nous bénéficions
aujourd’hui d’un lent processus historique, qui s’est poursuivi, avec des hauts et
des bas, durant quelque vingt-cinq siècles, dont nous sommes à la fois le résultat
et les débiteurs. Ce processus, à ne le considérer ici que d’un point de vue moral,
et malgré les formes violentes qu’il prit souvent, est un processus de convergence
des plus grandes civilisations autour d’un certain nombre de valeurs communes
ou voisines, celles qui nous permettent de vivre ensemble sans trop nous nuire ou
nous haïr. C’est ce qu’on appelle aujourd’hui les droits de l’homme, qui sont surtout,
moralement, ses devoirs.

Cette morale, d’où vient-elle ? De Dieu ? Ce n’est pas impossible : il a pu
mettre en nous, comme le voulait Rousseau, « l’immortelle et céleste voix » de
la conscience, qui prendrait le pas, ou qui devrait le prendre, sur toute autre
considération, fût-elle celle de notre salut ou de sa propre gloire. Mais s’il n’y
a pas de Dieu ? Alors il faut penser que la morale n’est qu’humaine, qu’elle
n’est qu'un produit de l’histoire, que l’ensemble des normes que humanité, au
fil des siècles, a retenues, sélectionnées, valorisées. Pourquoi celles-là ? Sans
doute parce qu’elles étaient favorables à la survie et au développement de
l'espèce (c’est ce que j'appelle la morale selon Darwin), aux intérêts de la société
(c’est la morale selon Durkheim), aux exigences de la raison (c’est la morale
selon Kant), enfin aux recommandations de l’amour (c’est la morale selon Jésus
ou Spinoza).

Imaginez une société qui prônerait le mensonge, l’égoïsme, le vol, le
meurtre, la violence, la cruauté, la haine. Elle n'aurait guère de chances de
subsister, encore moins de se répandre à l’échelle de la planète : parce que les
hommes ne cesseraient de s’y affronter, de s’y nuire, de s’y détruire... Aussi
n'est-ce pas une coïncidence si toutes les civilisations qui se sont répandues
dans le monde s'accordent au contraire pour valoriser la sincérité, la générosité,
le respect de la propriété et de la vie d’autrui, enfin la douceur, la compassion
ou la miséricorde. Quelle humanité autrement ? Quelle civilisation autrement ?
Cela dit quelque chose d’important sur la morale : qu’elle est ce par quoi
%— 392 —
l'humanité devient humaine, au sens normatif du terme (au sens où l'humain
est le contraire de l’inhumain), en refusant la veulerie et la barbarie qui ne cessent,
ensemble, de la menacer, de l'accompagner, et qui la tentent. Seuls les
humains, sur cette terre, ont des devoirs. Cela indique clairement la direction,
vers quoi il faut tendre : le seul devoir, ou celui qui résume tous les autres, c’est
d’agir humainement.

Que cela ne tienne pas lieu de bonheur, ni de sagesse, ni d’amour, c’est une
évidence ; c’est pourquoi nous avons besoin aussi d’une éthique (voir ce mot).
Mais sauf à être sage absolument, ou inhumain absolument, qui pourrait se
passer de morale ?

\section{Mort}
%MORT
Le néant ultime. Ce n’est donc rien ? Pas tout à fait pourtant, puisque
ce {\it rien} nous attend, ou puisque nous l’attendons. Disons que la
mort n’est rien, mais que nous mourrons : cette vérité au moins n’est pas rien.
Épicure et Lucrèce, sur cette question, me paraissent plus judicieux que Spinoza.
« Un homme libre ne pense à aucune chose moins qu’à la mort, dit une
fameuse proposition de l {\it Éthique}, et sa sagesse est une méditation non de la mort
mais de la vie » (IV, 67). À la seconde affirmation, j’adhère absolument. Mais à
la première, non, ni ne vois comment les deux peuvent être compatibles. Comment
méditer la vie sans penser la mort, qui l’achève ? C’est au contraire parce
que nous pensons que la mort n’est rien, dirait Épicure (rien pour les vivants,
puisqu'ils sont vivants, rien pour les morts, puisqu'ils ne sont plus), que nous
pouvons profiter de la vie sereinement. À quoi bon autrement philosopher ? Et
comment le faire en laissant la mort de côté ? Celui qui a peur de la mort, il a
peur, exactement, {\it de rien}. Comment n’aurait-il pas peur de tout ? Alors qu’il n’y
a rien à craindre dans la vie, expliquait encore Épicure, pour celui qui a compris
que le mal le plus redouté, la mort, n’est rien pour nous ({\it Lettre à Ménécée}, 125).
Encore faut-il la penser strictement — comme néant — pour cesser de l’imaginer
(comme enfer ou comme manque) et de la craindre. Cela suffira-t-il ? Ce n’est
pas sûr. Et même ce n’est pas, lorsque la mort sera toute proche, le plus probable.
Mais pourquoi la pensée devrait-elle suffire ? Comment le pourrait-elle ? Et
qu'importe qu’elle ne suffise pas, si cette idée vraie, ou qui nous paraît telle, nous
aide, ici et maintenant, à vivre mieux ? Une philosophie, même insuffisante, vaut
mieux que pas de philosophie du tout.

Apprendre à mourir ? Ce n’est qu’une partie, non la plus importante ni la
plus difficile, du général apprentissage de vivre. Au reste, et comme l’a dit plaisamment
Montaigne, quand bien même nous ne saurions mourir, nous aurions
tort de nous en inquiéter : « Nature nous en informera sur-le-champ, pleinement
et suffisamment » ({\it Essais}, III, 12). S’il faut penser la mort, ce n’est pas
%— 393 —
pour apprendre à mourir — nous y parviendrons de toute façon — mais pour
apprendre à vivre. Penser la mort, donc, pour l’apprivoiser, pour l’accepter,
puis pour penser à autre chose. « Je veux qu’on agisse, écrit merveilleusement
Montaigne, et qu’on allonge les offices de la vie tant qu’on peut ; et que la mort
me trouve plantant mes choux, mais nonchalant d’elle, et encore plus de mon
jardin imparfait » ({\it Essais}, I, 20).

\section{Mot}
%MOT
L'élément d’une langue : élément non pas minimal (ce n’est ni un
phonème ni un monème) mais qui fait, dans une langue donnée,
comme une unité signifiante, empiriquement repérable et reconnaissable.
L'erreur serait d’y voir une copie des choses, quand ce n’est une copie — ou une
matrice — que de notre pensée. Que le mot « néant » existe, par exemple, cela
ne prouve pas que le néant soit.

Les mots sont des outils : morceaux de sens et d’irréel (en tant qu’ils sont
réels, comme bruits, ils ne signifient rien), pour dire l’insignifiante ou insensée
réalité. Il s’agit, par un jeu construit d’unités discrètes, de découper le réel — de
{\it briser le silence} —, puis, comme on peut, d’en recoller les morceaux. De là ces
petits bruits, ces petites idées, et le grand bavardage de l'esprit. De là, aussi,
la tentation du silence. « Qu’y a-t-il dans un mot ? demandait Shakespeare. Ce
que nous appelons rose, sous un autre nom, sentirait aussi bon. »

\section{Mourir}
%MOURIR
C'est le passage ultime, où rien ne passe. C’est pourquoi on ne
meurt pas : on agonise (mais les mourants sont vivants, hélas),
puis on est mort (mais les morts ne sont plus). Mourir est un acte sans sujet, et
sans acte : un rond dans l’eau du destin, une imagination, une fantasmagorie,
cette fois bien douloureuse, de l’'amour-propre. Le corps lâche son âme comme
un pet, voilà ce qu'il faut dire, et le pet seul, à l’avance, se rebiffe. Qu'’as-tu,
mon corps, à te soucier de tes vents ?

\section{Mouvement}
%MOUVEMENT
C'est changer de lieu ou d’état, de position ou de disposition.
Aristote ({\it Physique}, IT, 1 et VIII, 7) distinguait quatre mouvements — mais
on dirait mieux, en français, quatre changements — principaux, correspondant
à autant de catégories : selon le lieu (le mouvement local), selon la substance
(génération et destruction), selon la quantité (accroissement et diminution),
selon la qualité (altération). Il y voyait le passage de la puissance à l’acte, passage
toujours inachevé (puisqu'il y a mouvement) et pour cela indissolublement
%— 394 —
en puissance et en acte : c’est « l’acte de ce qui est en puissance, en tant
que tel », c’est-à-dire, précise Aubenque, en tant qu’il est en puissance — l’acte
de la puissance, ou la puissance en acte (III, 1 et 2 ; voir aussi P. Aubenque, {\it Le
problème de l'être...}, p.454). C’est ce que nous appelons le changement ou le
devenir, dont le mouvement local n’est qu’une espèce, mais sans doute aussi,
dans l’espace, la condition de toutes les autres.

\section{Multitude}
%MULTITUDE
Un grand nombre. Quand on ne précise pas de quoi, il s’agit
le plus souvent d’êtres humains, considérés dans leur rassemblement
purement factuel, sans ordre et sans unité. S’oppose par là à l'État, qui
suppose l’ordre, et au peuple, qui suppose lunité. La multitude est « comme
une hydre à cent têtes, disait Hobbes, qui ne doit prétendre dans la république
qu’à la gloire de l’obéissance » ({\it Le Citoyen}, VI, 1).

\section{Mystère}
%MYSTÈRE
Quelque chose qu’on ne peut comprendre, mais à quoi l’on
croit. Se distingue par là du {\it problème} (quelque chose qu’on ne
comprend pas encore) et de l’{\it aporie} (à laquelle il n’est pas besoin de croire). Par
exemple l’origine de la vie est un problème. L'origine de l’être (« Pourquoi y a-t-il
quelque chose plutôt que rien ? »), une aporie. Dieu, un mystère.

\section{Mystique}
%MYSTIQUE
L’étymologie rattache le mot aux mystères. Mais les mystiques,
dans toutes les religions, nous parlent plutôt d’une espèce
d’évidence. C’est eux qu’il faut croire, plutôt que le passé de la langue ou de la
superstition. Le mystique, c’est celui qui voit la vérité face à face : il n’est plus
séparé du réel par le discours (c’est ce que j'appelle le silence), ni par le manque (ce
que j'appelle la plénitude), ni par le temps (ce que j'appelle l'éternité), ni enfin par
lui-même (ce que j'appelle la simplicité : l’{\it anatta} des bouddhistes). Dieu même a
cessé de lui manquer. Il fait expérience de l'absolu, ici et maintenant. Cet absolu,
est-ce encore un Dieu ? Plusieurs mystiques, spécialement en Orient, ont répondu
que non. De là un « mysticisme pur », comme disait le père Henri de Lubac, qui
est «la forme la plus profonde de l’athéisme» ({\it La mystique et les mystiques},
A. Ravier {\it et all.}, Préface). Ceux-là ne croient en rien : l'expérience leur suffit.

Ce mysticisme-là, qui est le maximum d’évidence, est ainsi le contraire de
la religion, qui est le maximum de mystère.

\section{Mythe}
%MYTHE
Une fable que l’on prend au sérieux.

%  395

%  N

\section{Naïveté}
%NAÏVETÉ
On ne la confondra pas avec la niaiserie. Le niais manque d’intelligence
ou de lucidité; le naïf, de ruse et d'artifice. Vertu
d’enfance ou de nature, qui ne saurait toutefois justifier le manque de maturité,
de culture ou de politesse.

\section{Narcissisme}
%NARCISSISME
L’amour, non de soi, mais de son image : Narcisse, incapable
de la posséder, incapable d’aimer autre chose, finit
par en mourir. C’est la version auto-érotique de l’amour-propre, et un autre
piège. On n’en sort que par l'amour vrai, qui n’a que faire des images.

\section{Nation}
%NATION
Un peuple, mais considéré d’un point de vue politique plutôt que
biologique ou culturel (ce n’est ni une race ni une ethnie), et
comme ensemble d'individus plutôt que comme institution (ce n’est pas, ou
pas nécessairement, un État). Renan a bien vu que l'existence et la pérennité
d’une nation doit moins à la race, à la langue ou à la religion qu’à la mémoire
et à la volonté. Deux choses surtout la constituent : « L’une est la possession en
commun d’un riche legs de souvenirs ; l’autre est le consentement actuel, le
désir de vivre ensemble, la volonté de continuer à faire valoir l’héritage qu’on a
reçu indivis. [...] Avoir des gloires communes dans le passé, une volonté commune
dans le présent ; avoir fait de grandes choses ensemble, vouloir en faire
encore, voilà les conditions essentielles pour être un peuple » ou une nation
({\it Qu'est-ce qu'une nation ?}, III). C’est dire qu’il n’est de nation que fidèle, et tel
est le vrai sens du patriotisme.

%— 396 —
\section{Nationalisme}
%NATIONALISME
C'est ériger la nation en absolu, à quoi tout — le droit, la
morale, la politique — devrait se soumettre. Toujours virtuellement
antidémocratique (si la nation est vraiment un absolu, elle ne
dépend plus du peuple : c’est lui au contraire qui dépend d’elle), presque toujours
xénophobe (ceux qui ne font pas partie de la nation sont comme exclus
de labsolu). C’est un patriotisme exagéré et ridicule : il érige la politique en
religion ou en morale. Aussi est-il volontiers païen et presque inévitablement
immoral.

\section{Naturalisme}
%NATURALISME
Toute doctrine qui considère la nature, prise en son sens
large, comme la seule réalité : c’est considérer que le surnaturel
n’existe pas ou n’est qu’une illusion. Un synonyme, donc, du
matérialisme ? Pas tout à fait. Tout matérialisme est un naturalisme, mais tout
naturalisme n’est pas matérialiste (ainsi, Spinoza). Disons que le naturalisme
est le genre prochain, dont le matérialisme ne serait qu’une espèce : c'est un
naturalisme moniste, comme était aussi le spinozisme, mais qui considère que
la nature est intégralement et exclusivement matérielle.

\section{Nature}
%NATURE
La {\it phusis}, chez les Grecs, comme la {\it natura} chez Lucrèce ou Spinoza,
c’est le réel lui-même, considéré dans son indépendance,
dans sa spontanéité, dans son pouvoir d’auto-production ou d’auto-développement.
S’oppose en cela à l’art ou à la technique (comme ce qui se fait tout seul
à ce qui est fait par l’homme) et au divin (comme ce qui se développe ou
change à ce qui est immuable). Peut se dire en un sens général (la nature est
l’ensemble des êtres naturels) ou en un sens particulier (la nature d’un être,
qu’on appelle parfois son essence, étant alors ce qu’il y a en lui de naturel : son
« principe, comme dit Aristote, de mouvement et de fixité »). S’oppose dans les
deux cas au surnaturel ou au culturel : la nature, c’est tout ce qui existe, ou qui
semble exister, indépendamment de Dieu — sauf bien sûr à définir Dieu comme
la nature elle-même — ou des hommes.

\section{Nature Humaine}
%NATURE HUMAINE
Il était de bon ton, dans les années 1960, de dire
qu’elle n’existe pas : l’homme ne serait que culture et
liberté. Si c'était si simple pourtant, pourquoi aurions-nous tellement peur des
manipulations génétiques sur les cellules germinales, celles qui transmettent le
patrimoine héréditaire de l'humanité ? La vérité, à ce que je crois, c’est qu’il y
a bien une nature humaine, ou en tout cas une nature de l’homme, qui est le
%— 397 —
résultat, en chacun, du processus naturel d’hominisation : notre nature, c’est
tout ce que nous recevons à la naissance et transmettons génétiquement, le cas
échéant et au moins en partie, à nos descendants. On sait de plus en plus à quel
point c’est considérable. Reste à l’humaniser, ce qui ne se fait que par éducation
et apprentissage. C’est pourquoi on peut dire, mais en un tout autre sens, qu’il
n'y a pas de nature humaine : non parce qu’il n’y aurait rien de naturel en
l’homme, mais parce que ce qui est naturel en lui n’est pas humain, au sens
normatif du terme, de même que ce qui est humain n’est pas naturel. On naît
homme, ou femme : telle est notre nature. Puis on devient humain : telle est
notre culture et notre tâche.

\section{Nature naturante / nature naturée}
%NATURE NATURANTE / NATURE NATURÉE
C’est une expression d’origine scolastique : la {\it natura
naturans} serait Dieu, la {\it natura naturata} l'ensemble des choses créées. Toutefois
ces deux expressions, aujourd’hui, sont plus souvent prises dans un sens spinoziste,
donc panthéiste, qui fait référence à un scolie fameux de l’{\it Éthique}. La
{\it Nature naturante}, c’est la nature en tant qu’elle est Dieu, c’est-à-dire cause de
soi et de tout (non les modes, mais les attributs éternels et infinis de la substance).
La {\it Nature naturée} c’est l’ensemble des effets — dont chacun est aussi une
cause — qui en découlent nécessairement : la chaîne infinie des causes finies
(l'ensemble non des attributs mais des modes). Disons que la Nature naturante,
c’est la nature comme cause de soi et de tout ; et la Nature naturée,
l’ensemble des effets, mais en elle, de cette causalité immanente (I, 29, scolie).

\section{Naturel}
%NATUREL
Au sens large ou classique : tout ce qui n’est pas surnaturel. Au
sens étroit et moderne : tout ce qui n’est pas culturel. Ce dernier
sens reste problématique. Si l’homme fait partie de la nature, comme je le crois,
comment ne serait-ce pas vrai aussi de la culture ?

Il reste que s’agissant spécialement du monde humain, il est commode de
distinguer ce qui est {\it naturel} (qui est transmis par les gènes et se reconnaît à
l’universalité) de ce qui est {\it culturel} (qui est transmis par l'éducation et se reconnaît
à des règles particulières). Par exemple la pulsion sexuelle est naturelle. La
façon de vivre cette pulsion, et spécialement de la satisfaire ou pas, est culturelle.
La procréation est naturelle ; la façon de faire des enfants (et {\it a fortiori} de
les élever), culturelle. La faim est naturelle. La gastronomie et la façon de
manger, culturelles. C’est en quoi la prohibition de l'inceste, remarque Lévi-Strauss,
fait problème : c’est qu’elle semble vérifier à la fois l’universalité de la
nature et la particularité réglée de la culture (on ne connaît pas de société
%— 398 —
humaine où l'inceste ne soit prohibé, mais toutes ne le prohibent pas de la
même façon ni dans les mêmes limites). La solution du problème, selon Lévi-Strauss,
c’est que la prohibition de l'inceste relève bien de ces deux ordres :
parce qu’elle assure le {\it passage} de la nature (la procréation) à la culture (la
parenté), de la filiation à l'alliance, de la famille à la société. Par quoi la nature
a toujours le premier mot, comme elle a aussi, par la mort, le dernier.

\section{Néant}
%NÉANT
Le non-être, ou le non-étant, mais considéré plutôt positivement.
Ce n’est pas tout à fait un ensemble vide, ni un pur rien : c’est un
ensemble, plutôt, dont le vide serait l’unique élément, ou un rien qui serait
aussi réel, à sa façon, que le quelque chose. Le néant, remarquait Hegel, « est
égalité simple avec lui-même, vacuité parfaite, absence de détermination et de
contenu, état de non-différenciation en lui-même ». C’est en quoi il n’est rien,
sans cesser pour autant d’être (puisqu'il {\it est} ce rien) : il est « la même détermination,
ou plutôt la même absence de détermination, et partant absolument la
même chose que l'être pur » ({\it Logique}, 1, 1). Bergson, peut-être moins dupe du
langage ou de la dialectique, ne voyait dans le néant qu’un mot, qu’une
pseudo-idée, qui ne serait obtenue que par négation de celle d’être, laquelle
peut seule être pensée positivement. Il avait sans doute raison, mais cela ne
prouve pas que, de cette pseudo-idée, on puisse tout à fait se passer.

Chez Heidegger et Sartre, le néant se révèle dans l’expérience de l’angoisse :
soit parce que l’être ne se donne que sur fond de néant (puisque aucun étant
n’est l’être, puisque l’être n’est rien d’étant) et de facticité (tout ce qui est {\it aurait
pu} ne pas être), soit parce que la conscience — n’étant pas ce qu’elle est, étant ce
qu’elle n’est pas — est pouvoir de néantisation. C’est constater une nouvelle fois
que le néant n’existe que pour l’homme : l’homme est l'être par qui le néant
vient au monde. En ce sens, le néant est moins ce qui n’est pas que ce qui n’est
plus, ou pas encore, ou pas tout à fait. C’est le corrélat vide de la conscience,
par quoi elle n’est jamais prisonnière de ses objets ou de son être : le résultat,
qu’on aurait bien tort d’hypostasier, de son pouvoir de néantisation. « Dans la
nuit claire du Néant de l’angoisse, écrit Heidegger, se montre enfin la manifestation
originelle de l’étant comme tel : à savoir {\it qu'il y ait de l'étant — et non pas
Rien}... Le néant est la condition qui rend possible la révélation de l’étant
comme tel pour la réalité humaine ({\it Dasein}). Le Néant ne forme pas simplement
le concept antithétique de l’étant, mais l'essence de l’Être même comporte
dès l’origine le Néant. C’est dans l’{\it être} de l’étant que se produit le {\it néantir}
du Néant » ({\it Qu'est-ce que la métaphysique ?}, trad. H. Corbin). Par quoi le {\it berger
de l'être}, comme dit ailleurs Heidegger de l’homme, devient {\it « la sentinelle du
Néant »}... Bergson, reviens : ils sont devenus fous !

%— 399 —
\section{Nécessitarisme}
%NÉCESSITARISME
Croyance en la nécessité de tout. À ne pas confondre
avec le fatalisme.

\section{Nécessité}
%NÉCESSITÉ
Le contraire de la contingence : est nécessaire ce qui ne peut
pas ne pas être, autrement dit ce dont la négation est impossible. 
Par exemple la somme des angles d’un triangle, dans un espace euclidien,
fait {\it nécessairement} 180°, ce qui signifie qu’il est impossible, dans cet espace,
qu’elle soit autre. Nécessité absolue ? Non pas, puisqu’un autre espace est possible
ou concevable (celui des géométries non euclidiennes). Mais qui n’en est
pas moins nécessaire pour autant, dans cet espace-là. Toute nécessité est ainsi
hypothétique, comme disait Alain : elle est soumise à la condition d’un principe
ou d’un réel. Si rien n’existait, rien ne serait nécessaire. C’est en quoi tout
le nécessaire, à le considérer globalement ou en détail, reste contingent. La
nécessité de ma mort, par exemple, est soumise à la contingence de ma naissance,
comme la nécessité de ma naissance (dès lors que les conditions en sont
posées) reste soumise à la contingence de ma conception, ou de celle de mes
parents ou grands-parents, et comme la nécessité de l’univers, dès lors qu’il
existe, reste soumise à la contingence de sa propre existence (puisqu'il aurait pu
ne pas exister). Cela pose la question du temps, je veux dire de celui qui passe
ou qui est passé. Prenons par exemple le temps qu’il fait. Est-il nécessaire ou
contingent ? Cela dépend du point de vue chronologique adopté. Le temps
qu’il fait, ici et maintenant, est assurément nécessaire : ce qui est ne peut pas ne
pas être, soulignait à juste titre Aristote, tant qu’il est. Le temps qu’il fera dans
six mois est très vraisemblablement contingent : non seulement imprévisible en
fait, mais sans doute imprévisible en droit, en raison de la complexité et de
l’aléatoire des phénomènes météorologiques, à l’échelle macroscopique, voire
de ce qu’il y a d’indéterminé en eux à l’échelle microscopique ou corpusculaire.
Pourtant, dans six mois, le temps qu’il fera sera nécessaire, comme tout présent.
C’est dire que la nécessité n’est pas un prédéterminisme, qui voudrait que le
temps qu’il fait aujourd’hui ait été inscrit depuis toujours dans le passé de lunivers,
comme celui qu’il fera dans dix mille ans le serait dans son état actuel. La
vérité, c'est que tout présent est nécessaire (sa négation, au présent, est
impossible : s’il pleut, ici et maintenant, il est impossible, ici et maintenant,
qu'il ne pleuve pas), donc aussi toute vérité (puisqu'il n’est de vérité
qu'éternelle : que toujours présente), mais {\it eux seuls et au présent seulement}. Le
temps qu’il fait, ici et maintenant, est nécessaire ; il ne l’était pas il y a six mois.
Il était pourtant vrai, il y a six mois, qu’il ferait ce temps aujourd’hui ? Sans
doute. Mais ce n’est pas parce que c'était vrai il y a six mois qu’il fait ce temps
maintenant ; c’est parce qu'il fait ce temps maintenant que c'était vrai il y a six
%— 400 —
mois ou cent mille ans. L’éternité du vrai dépend de la nécessité du réel, non
l'inverse, de même que le tracé d’un fleuve, sur nos cartes réelles ou possibles,
dépend du cours effectif de ce fleuve et ne saurait évidemment le gouverner.
Ainsi tout est nécessaire, au présent, et c’est pourquoi tout l’est (puisque seul le
présent existe). On n’en conclura pas que ce qui est {\it n'aurait pas pu} ne pas être
(à l’irréel du passé : tant que cela n’était pas), ce qui n’est guère vraisemblable.
Mais seulement que ce qui est {\it ne peut pas} ne pas être (au présent : il est nécessairement
ce qu'il est, tant qu'il est). Dès lors que je suis vivant, il est impossible
que je ne le sois pas. Mais cela ne me rend ni immortel ni incréé.

Un être absolument nécessaire ? Ce serait Dieu, et c’est pourquoi rien, dans
le monde, ne saurait l'être.

\section{Négligence}
%NÉGLIGENCE
Une faute qu’on aurait pu facilement éviter : il aurait suffi
d’un peu d’attention ou d’exigence. Petite faute ? Le plus
souvent, oui. Mais qui mène aux grands abandons, à force de s’habituer aux
petits. On omet d’abord de bien faire, puis on fait mal, ou le mal. C’est où la
négligence mène à la veulerie, comme la veulerie à la scélératesse.

\section{Népotisme}
%NÉPOTISME
Une forme de favoritisme, donc d’injustice, à destination de
ses parents : c’est privilégier les membres de sa famille, dans
des domaines où les liens de sang ou de cœur sont sans pertinence, par exemple
en matière d'emploi public ou d'administration. Quand Le Pen nous explique
qu’il préfère sa fille à sa voisine, et sa voisine à une étrangère, il ne ferait
qu'énoncer une platitude, s’il s'agissait d’un registre purement affectif ou privé.
Mais s’il prétend fonder là-dessus quelque politique que ce soit, ce n’est que la
justification du népotisme. Non plus une platitude, donc, mais une ignominie.

\section{Nerveux}
%NERVEUX
L'un des quatre tempéraments de la tradition hippocratique et
classique : teint pâle, vivacité des réactions, mobilité de l’intelligence
et du visage... Vous vous reconnaissez ? Inutile d’en parler à votre
médecin. Ce n’est pas une maladie, et votre médecin jugerait cette typologie
dépassée. Il a bien sûr raison. Cela ne prouve pas que celle qu’il utilise soit indépassable.

\section{Névrose/psychose}
%NÉVROSE/PSYCHOSE
Deux mots pour dire des troubles du psychisme ou
de la vie mentale. Étymologiquement, le névrosé
%— 401 —
serait malade des nerfs ; le psychotique, de l’esprit. Cela ne dit rien de leur
pathologie respective, ni de son étiologie. La distinction, entre ces deux
concepts, reste d’ailleurs difficile à cerner, au point que certains psychiatres,
notamment américains, renoncent aujourd’hui à les utiliser. Elle garde pourtant
une espèce de puissance classificatrice ou d’usage régulateur : ce sont des
catégories qu’il est bon de connaître, sans y croire tout à fait. Les psychoses,
quoiqu'il y ait des exceptions et des états intermédiaires, sont ordinairement
plus graves : elles perturbent la totalité de la vie psychique, sont souvent accompagnées
de délire ou d’hallucinations, coupent du monde et de l’autre, enfin
relèvent de la psychiatrie davantage que de la psychothérapie. Les névroses sont
moins graves, du moins dans la plupart des cas, et de pronostic habituellement
plus favorable : les troubles y restent parcellaires ou localisés (ils n’atteignent
qu'un pan ou qu’un niveau de la vie psychique), socialement moins invalidants,
dépourvus de délire, susceptibles d’un traitement psychothérapeutique
ou analytique efficace, enfin « relativement superficiels, plastiques et réversibles »
(Henri Ey, {\it Manuel de psychiatrie}). Les principales psychoses sont la
paranoïa, la schizophrénie et la psychose maniaco-dépressive. Les principales
névroses sont la névrose d’angoisse, la névrose obsessionnelle, la névrose phobique
et l’hystérie. On dit parfois que le névrosé bâtit des châteaux en Espagne,
que le psychotique les habite, enfin que le psychanalyste encaisse le loyer.
Disons, plus sérieusement, que c’est sans doute le rapport au réel et aux autres
qui permet le mieux de distinguer ces deux entités nosologiques. Le psychotique
est prisonnier de son monde ou de sa folie, au point d’ignorer souvent
qu’il est malade. Le névrosé habite le monde commun : ses troubles, dont il a
habituellement conscience, ne l’empêchent ni d’agir ni d’entrer en relation avec
autrui {\it à peu près} normalement. Toutefois ce ne sont que des abstractions. Pour
ceux qui ne sont ni malades ni psychiatres, elles servent surtout à s’observer soi.
Il n’est pas inutile, même dans la santé, de savoir de quel côté on penche.

\section{Nihilisme}
%NIHILISME
Le nihiliste, c’est celui qui ne croit à rien ({\it nihil}), même pas à
ce qui est. Le nihilisme est comme une religion négative :
Dieu est mort, emportant avec lui tout ce qu’il prétendait fonder, l’être et la
valeur, le vrai et le bien, le monde et l’homme. Il n’y a plus rien que le rien, en
tout cas rien qui vaille, rien qui mérite d’être aimé ou défendu : tout se vaut, et
ne vaut rien.

Le mot semble avoir été inventé par Jacobi, pour désigner l’incapacité de la
raison à saisir l’existence concrète, qui ne se donnerait qu’à l'intuition sensible
ou mystique. La raison, coupée de la croyance, est incapable de passer du
concept à l’être (comme le prouve la réfutation kantienne de la preuve ontologique) ;
%— 402 —
elle ne peut donc penser que des essences sans existence (sujet et
objet se dissolvant alors dans une pure représentation), et c’est en ce sens que
tout rationalisme, pour Jacobi, est un nihilisme. En français, et dans un usage
moins technique, le mot a été popularisé par Paul Bourget, qui le définissait
comme « une mortelle fatigue de vivre, une morne perception de la vanité de
tout effort ». Mais c’est bien sûr Nietzsche, dans le double prolongement de
Jacobi et de Bourget, qui lui donnera ses lettres de noblesse philosophiques. La
raison ne donne aucune raison de vivre : elle ne débouche que sur des abstractions
mortifères. Le rationalisme, pour Nietzsche aussi, est un nihilisme. Mais
ce n’est pas un courant de pensée parmi d’autres : c’est l’univers spirituel qui
nous attend. « Ce que je raconte, écrit Nietzsche, c’est l’histoire des deux prochains
siècles. Je décris ce qui viendra, ce qui ne peut manquer de venir : {\it l'avènement
du nihilisme} » ({\it La volonté de puissance}, III, 1, 25). Nous y sommes. Le
problème est d’en sortir.

« Que signifie le nihilisme ? Que les valeurs supérieures se déprécient,
répond Nietzsche. Les fins manquent. Il n’est pas de réponse à cette question :
“À quoi bon ?” » ({\it ibid.}, aph. 100). Les sciences, qui voulaient remplacer la religion,
ne donnent aucune raison de vivre : leur culte de la vérité n’est qu’un
culte de la mort. De |à cette « doctrine de la grande lassitude : “À quoi bon ?
Rien n’en vaut la peine !” » ({\it ibid.}, aph. 99). Nietzsche voulut y échapper par
l’esthétisme, autrement dit par le culte du beau mensonge, de l’erreur utile à la
vie, de l'illusion créatrice (« l’art au service de l'illusion, voilà notre culte », II,
5, 582). Cela ne fait qu’un néant de plus, qui règne désormais dans nos musées.
« Tout est faux, tout est permis », disait encore Nietzsche (II, 1, 108), et c’est
le nihilisme d’aujourd’hui. On ne peut en sortir qu’en revenant à la vérité de
l'être, comme dira Heidegger, et de la vie, comme le voulait Nietzsche, mais
qui n’est pas le mensonge et l'illusion : qui est puissance et fragilité, puissance
et résistance ({\it conatus}) — désir, en l’homme, et vérité. C’est choisir Spinoza
plutôt que Nietzsche, la lucidité plutôt que l'illusion, la fidélité plutôt que le
«renversement de toutes les valeurs», enfin l’humanité plutôt que le
surhomme. « Qu'est-ce que le nihilisme, demande Nietzsche, si ce n’est cette
lassitude-là ? Nous sommes fatigués de l’homme...» ({\it Généalogie...}, I, 12).
Parle pour toi. Le nihilisme est une philosophie de peine-à-jouir, de peine-à-aimer,
de peine-à-vouloir. C’est la philosophie de la fatigue, ou la fatigue
comme philosophie. Ils ont perdu la capacité d’aimer, comme dit Freud des
dépressifs, et en concluent que rien n’est aimable. C’est bien sûr se méprendre.
Ce n’est pas parce que le monde et la vie sont aimables qu’il faudrait les aimer ;
c’est parce que nous les aimons qu’ils sont, pour nous, aimables. Les valeurs ne
se déprécient que pour ceux qui ont besoin d’un Dieu pour aimer. Pour les
autres, les valeurs continuent de valoir, ou plutôt n’en valent que de façon plus
%— 403 —
urgente : parce que aucun Dieu ne les fonde ni ne les garantit, parce qu’elles ne
valent qu’à proportion de l'amour que nous leur portons, parce qu’elles ne
valent que pour nous, que par nous, qui avons besoin d’elles. Raison de plus
pour les servir. Le relativisme, loin d’être une forme de nihilisme, est son
remède : que toutes nos valeurs soient relatives (à nos désirs, à nos intérêts, à
notre histoire), c’est une raison forte pour ne pas renoncer aux {\it relations} qui les
font être. Ce n’est pas parce que la justice existe qu’il faut s’y soumettre (dogmatisme),
ni parce qu’elle n’existe pas qu’il faudrait y renoncer (nihilisme) ;
c'est parce qu’elle n’existe pas (sinon en nous, qui la pensons et voulons : relativisme)
qu’elle est à faire.

Contre le dogmatisme, quoi ? La lucidité, le relativisme, la tolérance.

Contre le nihilisme ? L’amour et le courage.

\section{Nirvâna}
%NIRVÂNA
Le nom bouddhiste de l’absolu ou du salut : c’est le relatif même
(le {\it samsâra}), l'impermanence même ({\it anicca}), dès lors qu’on n’en
est plus séparé par le manque, le mental ou l'attente. L’ego s'éteint ({\it nirvâna}, en
sanskrit, signifie extinction) : il n’y a plus que tout. C’est l'équivalent à peu
près, mais dans une tout autre problématique, de l’ataraxie chez Épicure et de
la béatitude chez Spinoza : l'expérience, ici et maintenant, de l’éternité.

\section{Nom}
%NOM
Un mot, mais désignant ordinairement quelque chose d’à peu près
stable : une chose, un individu, une substance (d’où le mot, qu’on utilise
aussi, de {\it substantif}), un état, une abstraction. Les actions, les mouvements
ou les processus sont mieux désignés par des verbes. On objectera qu’« actions »,
« mouvements » ou « processus» sont des noms. Mais c’est qu'il s’agit ici
d’abstractions : le mot « action » n’en est pas une (c’est un nom, qui ne désigne
qu’une idée), mais c’en est une de le prononcer ou d’agir (qui sont des verbes).

Francis Wolff, dans un livre magistral ({\it Dire le monde}, PUF, 1997), a imaginé
un langage-monde qui ne serait fait que de noms : monde d’essences séparées et
immuables, d'individus sans changements, de substances sans accidents, de
choses sans événements, d’êtres sans devenir. Ce serait le monde de Parménide,
s’il était possible, ou le monde intelligible de Platon, s’il était réel. Mais ce n’est
rien, ou presque rien : monde de l’abstraction, ou l’abstraction d’un monde.

\section{Nominalisme}
%NOMINALISME
Toute doctrine qui considère que les idées générales n’ont
d’autre réalité que les mots qui servent à les désigner.
C’est affirmer que seuls les individus existent, et qu’il n’est de généralités (ou
%— 404 —
d’universaux, comme on disait au Moyen Âge) que par et dans le langage. Le
nominalisme s'oppose ainsi au {\it réalisme} (au sens du réalisme des idées, tel qu’on
le voit chez Platon ou Guillaume de Champeaux), et ce dès l’Antiquité : « Je
vois bien le cheval, disait Antisthène, mais non la chevalité ». Il se distingue
aussi du {\it conceptualisme}, pour lequel les idées générales n'existent que dans
notre esprit, mais sans se réduire purement et simplement aux signes qui servent
à les désigner. Cette dernière distinction est pourtant moins fondamentale :
nominalisme et conceptualisme ont en commun d’affirmer l’exis-
tence exclusive des individus, du moins en dehors de lesprit humain, ce qui
explique que la frontière, entre ces deux courants, soit souvent floue (les spécialistes
discutent encore, par exemple, pour savoir si Guillaume d’Occam était
nominaliste ou conceptualiste), alors qu’ils s'opposent frontalement, lun et
l’autre, au réalisme des idées.

Le nominalisme sera la doctrine de Roscelin, au {\footnotesize XI$^\text{e}$} siècle, ou de
Guillaume d’Occam, au {\footnotesize XIV$^\text{e}$}, mais aussi de Hobbes, Hume et Condillac,
comme en général, et jusqu’à aujourd’hui, de la plupart des empiristes. Si les
idées n'existent pas en elles-mêmes, il en découle qu'on ne peut rien
connaître que par l'expérience, et que la logique elle-même n’est qu’une
langue bien faite (elle n’est agencement d’idées que parce qu’elle est, d’abord,
agencement de signes). Le matérialisme, dans une autre problématique, s’y
reconnaît également. Si tout est matière, les idées ne sauraient exister en elles-mêmes :
elles n’existent que dans un cerveau et par la médiation des signes
qui permettent de les désigner. Le matérialisme, de ce point de vue, est un
nominalisme radical et moniste : il n’existe que des individus, et ils sont tous
matériels.

\section{Normal}
%NORMAL
Qui est conforme à la norme, mais à une norme purement factuelle,
en général la moyenne («une taille normale ») ou la
santé (le normal opposé au pathologique). C’est ériger le fait en valeur, la statistique
en jugement, la moyenne en idéal. C’est ce qui rend la notion désagréable,
sans permettre pour autant de toujours s’en passer. « S’il existe des
normes biologiques, écrit Georges Canguilhem, c’est parce que la vie, étant
non pas seulement soumission au milieu mais institution de son milieu
propre, pose par là même des valeurs, non seulement dans le milieu mais
aussi dans l'organisme même. C’est ce que nous appelons la normativité
biologique » ({\it Le normal et le pathologique}, PUF, Conclusion). C'est ce qui
explique que la santé, qui est un fait, ou un rapport entre des faits, soit aussi
un idéal.

%— 405 —
\section{Normatif}
%NORMATIF
Qui établit une norme, en relève ou la suppose ; qui énonce
un jugement de valeur ou en dépend. Le point de vue normatif
s'oppose traditionnellement au point de vue {\it descriptif}, qui se contente d'établir
des faits, ou {\it explicatif}, qui donne les causes ou les raisons. La différence, entre
ces trois points de vue, peut pourtant être floue. Quand je dis de quelqu'un
{\it « C'est un imbécile »}, il est clair que j’émets un jugement de valeur. Mais je peux
aussi constater un fait (si l’imbécillité en question fait partie du réel) et expliquer
un certain type de comportement. Le politiquement correct voudrait nous
interdire ce type de jugement négatif (« On ne dit plus {\it aveugle} mais {\it mal-voyant},
remarquait un humoriste, bientôt on ne dira plus {\it con} mais {\it mal-comprenant} »).
Mais c'est au nom d’un point de vue qui reste lui-même normatif, tout en se
prétendant abusivement descriptif : ils voudraient, absurdement, que l'égalité
des hommes en droits et en dignité entraîne leur égalité en fait et en valeur.
Aussi s’interdisent-ils de juger, sinon pour condamner ceux qui s’y risquent.
C’est qu'ils ont jugé à l’avance et une fois pour toutes. Le politiquement correct
n'est qu'un préjugé normatif.

\section{Norme}
%NORME
{\it Norma}, en latin, c’est l’{\it équerre} : une norme, commente Canguilhem,
« c’est ce qui sert à faire droit, à dresser, à redresser » ({\it Le
normal et le pathologique}, p. 177). Elle dit ce qui doit être, ou permet de juger
ce qui est.

Le mot peut valoir comme synonyme de {\it règle}, d’{\it idéal}, de {\it valeur}... Si on
veut lui donner un sens plus précis, c’est sans doute sur sa généralité qu’il faut
insister : la {\it norme} est le genre commun, dont {\it règles}, {\it idéaux} et {\it valeurs} sont différentes
espèces. C’est pourquoi le mot a quelque chose de flou, qui le rend à la
fois commode et embarrassant. Il perd en compréhension, nécessairement, ce
qu’il gagne en extension.

\section{Nostalgie}
%NOSTALGIE
C’est le manque du passé, en tant qu’il fut. Se distingue par là
du regret (le manque de ce qui ne fut pas) ; s'oppose à la gratitude
(le souvenir reconnaissant de ce qui a eu lieu : la joie présente de ce qui
fut) et à l'espérance (le manque de l’avenir : de ce qui sera peut-être).

J'ai tendance à penser que la nostalgie, de ces quatre sentiments, est le premier,
et que toute espérance, spécialement, n’est que l’expression — ou le
remède imaginaire — d’une nostalgie préalable. Il faudrait relire Platon et saint
Augustin, de ce point de vue, à la lumière de Freud. Et relire aussi Épicure : on
y verrait que la gratitude, contre la nostalgie, est le seul remède vrai.

%— 406 —
\section{Notion}
%NOTION
Une idée abstraite ou générale, le plus souvent considérée comme
déjà donnée dans la langue ou dans l’esprit. C’est ce qui distingue
la notion (qui n’a besoin, étymologiquement, que d’être connue ou reconnue)
du concept (qu’il faut d’abord concevoir). Un concept est le résultat d’un travail
de pensée ; la notion serait plutôt sa condition. Un concept est une œuvre
avant d’être un outil. La notion serait plutôt un matériau ou un point de
départ. Un concept relève d’une science ou d’une philosophie particulière ; une
notion, de la pensée commune.

Chez Kant, « le concept pur, en tant qu’il a uniquement son origine dans
l’entendement (et non dans une image pure de la sensibilité), s'appelle notion ».
Mais cet usage ne s’est pas imposé. C’est que le concept était trop étroit et trop
dépendant d’une doctrine particulière pour s'imposer comme notion.

On appelle {\it notions communes} les idées ou principes qu’on trouve chez
«tous les hommes » (Spinoza), sans lesquelles nous ne pourrions ni raisonner
ni nous comprendre. Les empiristes veulent qu’elles résultent de l’expérience ;
les rationalistes, qu’elles soient innées et la rendent possible. C’est en quoi
l’empirisme m’a toujours paru plus rationaliste (au sens large) que le rationalisme
(au sens étroit) : il ne renonce pas à expliquer les notions dont il se sert.

\section{Noumène}
%NOUMÈNE
Même s’il ne l'a pas inventé (on trouve {\it noumena}, chez Platon,
pour désigner les Idées), le mot, aujourd’hui, fait presque toujours
référence à la philosophie de Kant. Qu'est-ce qu’un noumène ? Un objet
qui ne serait objet que pour l'esprit ({\it noûs}, en grec), un objet qui n'apparaît pas
(ce n’est pas un phénomène), dont nous n’avons aucune expérience, aucune
intuition (puisque nous n’avons d’intuitions que sensibles), que nous ne pouvons
pour cela {\it connaître}, mais tout au plus {\it penser}. La chose en soi ? Une façon,
plutôt, de l’envisager : en tant qu’elle serait un être purement intelligible (ce
que le concept de chose en soi n’impose pas), ou l'objet, si nous en étions
capables, d’une intuition intellectuelle. Concept {\it problématique}, reconnaît
Kant, puisqu'il excède par nature notre connaissance. Mais qui prétend toutefois
résoudre ce problème (quoique de façon non dogmatique) dans un sens
idéaliste : c’est où Kant rejoint Platon.
% 407

%  O
\section{Obéissance}
%OBEISSANCE
La soumission à un pouvoir légitime, ou qu’on juge tel. Il
n’en est pas moins nécessaire, parfois, de désobéir. La légitimité
n’est ni l’infaillibilité ni la justice.

\section{Objectif}
%OBJECTIF
Comme substantif, c’est un but ou un instrument d'optique :
l’objet que l’on vise ou celui qui vise l’objet.

Comme adjectif, le mot peut qualifier tout ce qui doit à l’objet davantage
qu'au sujet : tout ce qui existe indépendamment de quelque sujet que ce soit
ou, si un sujet intervient (par exemple dans un récit ou un jugement), tout ce
qui fait preuve d’objectivité.

Tout cela irait sans dire. Mais il faut signaler surtout, c’est ce qui justifie cet
article, un usage très particulier, qu’on trouve chez les scolastiques et les philosophes
du {\footnotesize XVII$^\text{e}$} siècle, qui peut aujourd’hui prêter à contresens. Est {\it objectif}, en
ce sens, tout ce qui est un objet de la pensée ou de l’entendement, que cet objet
ait ou non une réalité extérieure qui lui corresponde (qu’il existe ou non objectivement,
au sens moderne du terme). L’être objectif d’une idée s’oppose alors
à son être formel : son être formel, c’est ce qu’elle est en soi ; son être objectif,
ce qu’elle est en nous ou pour nous (en tant qu’elle est un objet de notre
pensée). C’est ce qui faisait dire à Descartes, par exemple, que « l’idée du soleil
est le soleil même existant dans l’entendement, non pas à la vérité formellement,
comme il est au ciel, mais objectivement, c’est-à-dire en la manière que
les objets ont coutume d’exister dans l’entendement : laquelle façon d’être est
de vrai bien plus imparfaite que celle par laquelle les choses existent hors de
l’entendement ; mais pourtant ce n’est pas un pur rien » ({\it Réponses aux premières
objections}, AT, p. 82). Ou à Spinoza que « l’idée, en tant qu’elle a une essence
%— 408 —
formelle, peut être l’objet à son tour d’une autre essence objective », qui est
l'idée de l’idée ({\it T.R.E.}, 27).

Enfin, chez Hegel, l'esprit objectif est celui qui dépasse la conscience individuelle et
s’incarne dans des institutions juridiques, sociales ou politiques (le
droit, les mœurs, l’État), où il devient comme un objet pour lui-même. Ce
moment, encore prisonnier de ses limites territoriales, sera lui-même dépassé
dans l’esprit absolu (Part, la religion, la philosophie).

\section{Objectivité}
%OBJECTIVITÉ
C’est voir ou connaître les choses comme elles sont ou comme
elles apparaissent, indépendamment, si c’est possible, de
notre subjectivité, et en tout cas de ce que notre subjectivité peut avoir de particulier
ou de partial. En pratique, c’est voir les choses comme peut les voir tout
observateur de bonne foi, quand il est sans passion et sans parti pris. Que
l’objectivité ne soit jamais absolue, ce qui est bien clair (puisqu'il n’y a de
connaissance que pour un sujet), n'autorise pas à dire qu’elle soit impossible ;
car alors les sciences et la justice le seraient aussi.

\section{Objet}
%OBJET
Étymologiquement, c’est ce qui est placé devant. Devant quoi ?
Devant un sujet. C’est en quoi les deux notions sont indissociables.
Là où il n’y a pas de sujet, il peut bien y avoir des êtres, des événements ou des
choses, mais il n’y a pas d'objet. C’est que tout objet est construit : soit par les
conditions (à la fois subjectives et historiques, sensibles et intellectuelles) de sa
perception, soit par celles (aussi bien expérimentales que théoriques) de sa
connaissance scientifique. Qu’est-ce qu’un objet ? C’est le corrélat objectif, ou
supposé tel, d’un sujet percevant ou connaissant. C’est pourquoi « l'objet n’est
pas l'être », comme Alquié aimait à le répéter, et c’est pourquoi on ne peut
connaître, par définition, que des objets. À la gloire du pyrrhonisme.

\section{Obscur}
%OBSCUR
Ce qui n’est ni clair ni lumineux. À ne pas confondre avec la profondeur.
Une idée qu’on ne comprend pas, comment savoir si elle
est profonde ? Mais pas non plus avec la fausseté. Pourquoi la vérité serait-elle
toujours claire ?

\section{Observation}
%OBSERVATION
C’est une expérience, mais volontaire et attentive. Par
exemple on fait l’expérience du deuil, et l’on observe, si
%— 409 —
l’on veut, si l’on peut, ce qui se passe en soi. Ou bien on fait l’expérience d’une
nuit étoilée, et l’on observe, si l’on veut, les étoiles.

Claude Bernard distinguait légitimement l’observation simplement empirique,
qui est faite « sans idée préconçue », de l'observation scientifique, qui
suppose une hypothèse préalable, qu’il s’agit de vérifier. Mais c’est surtout de
l’expérimentation que l'observation, même scientifique, se distingue : l’expérimentation
est une « observation provoquée », comme disait encore Claude
Bernard ; l'observation, à l'inverse, tient lieu d’expérimentation lorsqu'on ne
peut ni provoquer ni modifier le phénomène observé. C’est souvent le cas en
astronomie, mais aussi dans les sciences humaines : on ne provoque pas une
éclipse, mais pas davantage — en tout cas dans un but scientifique — une révolution,
une névrose ou un suicide. L'observation ne sert pas seulement pour le
plus lointain, mais aussi, souvent, pour le plus proche. Le difficile est qu’elle
risque alors de modifier involontairement cela même qu’elle observe : c’est ce
qui rend l’ethnographie et l’introspection si difficiles et si incertaines. Les relations
d’incertitude, comme dit Heisenberg, ne valent pas seulement en mécanique quantique.

\section{Obstacle épismétologique}
%OBSTACLE ÉPISTÉMOLOGIQUE
« Quand on cherche les conditions psychologiques
des progrès de la science,
écrit Bachelard, on arrive bientôt à cette conviction que c’est en termes d’obstacles
qu’il faut poser le problème de la connaissance scientifique » ({\it La formation
de l'esprit scientifique}, I). La connaissance n’est pas une table rase : on n’y
part jamais de zéro ; on ne connaît que « contre une connaissance antérieure »,
qu'elle soit purement empirique ou déjà scientifique. « Quand il se présente à
la culture scientifique, l'esprit n’est jamais jeune. Il est même très vieux, car il a
l’âge de ses préjugés. Accéder à la science c’est, spirituellement, rajeunir, c’est
accepter une mutation brusque qui doit contredire un passé » ({\it ibid.}). Ce passé,
mais actuel et actif, c’est l’obstacle épistémologique. C’est une opinion, une
représentation ou une habitude intellectuelle, héritée du passé, qui entrave la
connaissance scientifique ou s'oppose, de l’intérieur, à son développement. Par
exemple l’obstacle substantialiste, qui veut tout expliquer par la substance, ou
l'obstacle animiste, qui projette sur la nature ce qu’on croit savoir de la vie.
Autant d’idées faussement claires, qu’il faut comprendre pour s’en libérer.
« Déceler les obstacles épistémologiques, écrit encore Bachelard, c’est contribuer
à fonder les rudiments d’une psychanalyse de la raison » ({\it ibid.}). C’est où
les deux versants de la pensée bachelardienne — philosophie des sciences, philosophie
de l’imaginaire — se rejoignent.

%— 410 —
\section{Occultisme}
%OCCULTISME
Ce n’est pas croire en des vérités cachées (elles le sont presque
toutes : « les yeux, disait Lucrèce, ne peuvent connaître la
nature des choses »). C’est croire en des vérités qui se cachent, ou que l’on
cache, parce qu’elles seraient d’une nature absolument différente des autres :
surnaturelles, surréelles, suprasensibles, d’outre-tombe ou d’outre-monde..
Elles ne seraient accessibles qu’à des {\it sciences occultes}, ce qui fait un bien étrange
oxymore. Faire tourner des tables pour faire parler des esprits, croire aux fantômes
ou aux devins, pratiquer lalchimie ou la magie... Quoi de
«scientifique » là-dedans ? Ce n’est qu’une superstition de l’invisible.

\section{{\OE}uvre}
%ŒUVRE
Le produit d’une activité ou d’un travail. Le mot suppose presque
toujours une normativité au moins implicite : œuvre dit plus que
production, ouvrage ou résultat. C’est le fruit d’un travail, mais considéré dans
sa valeur intrinsèque, et qui vaut plus, presque toujours, que le travail lui-même.

C’est pourquoi on parle souvent, notamment à propos des œuvres d’art, de
création : quelque chose de neuf est apparu, qui semble excéder les moyens utilisés.
Disons que c’est une création humaine, comme la Création (univers)
serait l’œuvre de Dieu.

\section{Oligarchie}
%OLIGARCHIE
C'est le pouvoir d’une petite minorité ({\it oligos}, peu nombreux),
qui prétendent souvent être les meilleurs et qui ne sont en
vérité que les plus puissants — c’est-à-dire, presque toujours, les plus riches. Ils
voudraient constituer une aristocratie. Ce n’est ordinairement qu’une ploutocratie déguisée.

\section{Ontique}
%ONTIQUE
Qui concerne les étants ({\it ta onta}) plutôt que l’être. Se distingue
en cela d’{\it ontologique} (qui concerne l’être de l’étant plutôt que
l’étant lui-même). La notion, on l’a compris, relève du vocabulaire heideggérien.

\section{Ontologie}
%ONTOLOGIE
Le discours sur « l’être en tant qu'être », comme disait Aristote,
ou sur l’être de ce qui est (les étants en général, {\it ta onta},
et non tel étant en particulier). C’est une partie, sauf pour les heideggériens, de
la métaphysique. Mais sur l’être en tant qu'être, que dire, sinon qu'il est ? Les
sciences nous en apprennent davantage. L’être pur n’est qu’un rêve de philosophe.
Mieux vaut l’impureté du réel.

%— 411 
\section{Ontologique (preuve —)}
%ONTOLOGIQUE (PREUVE -)
L’une des trois preuves traditionnelles de
l'existence de Dieu : celle qui veut conclure
son existence de sa seule essence ou définition. Qu'est-ce, en effet, que Dieu ?
Un être suprême (« un être, disait saint Anselme, tel que rien de plus grand ne
puisse être pensé »), un être souverainement parfait (Descartes), un être absolument
infini (Spinoza, Hegel). Or, s’il n'existait pas, il ne serait ni le plus grand,
ni parfait, ni réellement infini. Il existe donc par définition : penser Dieu (le
concevoir comme suprême, parfait, infini.....), c’est le penser comme existant.
Le concept de Dieu, dira Hegel, « inclut en lui l’être » : Dieu est le seul être qui
existe par essence.

Belle preuve, par la simplicité, mais trop belle ou trop abstraite pour être
tout à fait convaincante : c’est vouloir passer de la pensée à l’être, ce qui n’est
possible que par expérience. Mais si nous avions une expérience de Dieu, nous
n’aurions plus besoin de le prouver. Et si nous n’en avons pas, toute preuve de
son existence est définitivement impossible. L’être n’est pas un prédicat,
explique Kant, qu’on pourrait ajouter à un concept ou l’en déduire. C’est pourquoi
il ne suffit pas de définir Dieu pour le prouver, pas plus qu’il ne suffit de
définir la richesse pour s'enrichir. Il n’y a pourtant rien de plus dans cent euros
réels, dirait aujourd’hui Kant, que dans cent euros possibles : le concept, dans
les deux cas, est le même. Mais on est plus riche avec cent euros réels qu'avec
leur simple concept ou définition. Même chose s'agissant de Dieu: son
concept reste le même, qu’il existe ou pas, et ne saurait donc prouver qu’il
existe.

\section{Onto-théologie}
%ONTO-THÉOLOGIE
Le discours, non sur l’être, mais sur l’étant, et spécialement
sur l’étant suprême : Dieu. Ce serait la forme
métaphysique de l’oubli de l'être, ou plutôt la métaphysique même, en tant
qu’elle n’existerait, selon Heidegger, que par cet oubli.

\section{Opinion}
%OPINION
Toute pensée qui n’est pas un savoir. S’oppose pour cela, spécialement,
aux sciences. C’est ce qui faisait écrire à Bachelard, en un
texte fameux : « L'opinion pense mal ; elle ne pense pas : elle traduit des besoins
en connaissances » ({\it La formation de l'esprit scientifique}, X). Toutefois c’est forcer
trop l’opposition. D'abord parce que les opinions jouent un rôle aussi dans les
sciences en train de se faire, et qui n’est pas seulement celui d’obstacle épistémologique
(mais aussi d’idée régulatrice, d’hypothèse vague, d’orientation provisoire
et tâtonnante.....). Ensuite parce qu’il y a des opinions droites, comme
Platon le soulignait déjà, lesquelles, pour insuffisantes qu’elles demeurent, sont
%— 412 —
légitimement tenues pour vraies. Enfin, et surtout, parce qu’une opinion
pensée, réfléchie, théorisée, n’en reste pas moins {\it opinion} pour autant : la philosophie
en est pleine. Par exemple quand Descartes affirme que la volonté est
libre ou quand Spinoza assure qu’elle ne l’est pas : ce sont des opinions, ni plus
ni moins, et pourtant des pièces essentielles, et hautement argumentées, de
leurs systèmes. Et même chose, bien sûr, des « preuves » de l’existence de Dieu,
de la démonstration de l’immortalité de l’âme, ou de sa mortalité, de la
croyance en l’infinité ou en la finitude de l’univers, du statut de la vérité, du
fondement de la morale ou de la définition philosophique de l'opinion. À la
gloire du pyrrhonisme. Il n’y a pas de savoir philosophique (il n’y a de savoir
que sur l’{\it histoire} de la philosophie) ; la philosophie n’est pas une science, et
c’est en quoi toute philosophie, même la plus sophistiquée, est d'opinion.

Qu'est-ce qu’une opinion? Kant en donnait une définition presque
parfaite : « L'opinion est une croyance qui a conscience d’être insuffisante aussi
bien subjectivement qu’objectivement » ({\it C. R. Pure}, « De l'opinion, de la
science et de la foi»; voir aussi {\it Logique}, Introd., IX). Pourquoi {\it presque}
parfaite ? Parce que c’est définir l’opinion lucide, celle qui se sait opinion, non
l'opinion dogmatique, si fréquente, celle qui se prend pour le savoir qu’elle
n'est pas, bien plus que pour la foi qu’elle refuse d’être. Que Spinoza ou Descartes
aient cru à leurs démonstrations, j’en suis convaincu ; mais cela ne nous
dit pas lequel des deux, lorsqu'ils s'opposent (or ils s'opposent presque toujours),
a raison, ni ne nous autorise à prêter à leurs philosophies, comme ils le
voulaient, la certitude — d’ailleurs elle-même relative — des mathématiques. De
là cette définition rectifiée que je propose : l’opinion est le fait de tenir quelque
chose pour vrai, mais en vertu d’un jugement objectivement insuffisant, et
qu’on ait ou pas conscience de cette insuffisance. C’est une croyance incertaine,
c’est-à-dire une croyance, mais désignée (fût-ce par un autre) comme telle : une
croyance dont on refuse de se satisfaire.

\section{Optimisme}
%OPTIMISME
Un optimiste rencontre un pessimiste. « Tout va mal, s’exclame
ce dernier. Ça ne pourrait pas être pire ! » Et l’optimiste
de lui répondre : « Mais si, mais si... » Quel optimisme qui ne donne raison,
pour finir, au pessimisme ?

{\it Optimus}, en latin, est le superlatif de {\it bonus}. Le mot signifie « le meilleur »,
et cette étymologie pourrait presque faire une définition suffisante. Être optimiste,
au sens philosophique du terme, c’est penser, avec Leibniz, que tout va
pour le mieux dans le meilleur des mondes possibles ({\it Théodicée}, I; voir aussi
II, 413 sq.). Doctrine irréfutable (puisque ce monde est le seul que lon
connaisse) et pourtant incroyable (tant le mal y est évident). Voltaire, dans
%— 413 —
{\it Candide}, a dit là-dessus à peu près ce qu’il fallait. On n’en reste pas moins surpris
qu'un génie comme Leibniz, peut-être le plus grand qui fut jamais, ait pu
tomber dans cette niaiserie. C’est qu’il prenait la religion au sérieux, et que la
religion, inévitablement, est optimiste. Si Dieu existe, le meilleur existe : toute
religion est un optimisme métaphysique.

Au sens courant, le mot optimisme désigne moins une doctrine qu’une attitude
ou un penchant : être optimiste, c’est prendre les choses du bon côté, ou
penser, lorsqu’elles sont décidément douloureuses, qu’elles vont s'arranger. Et
après tout, pourquoi pas ? Toutefois la mort et la vieillesse font des raisons
fortes de n’y point croire tout à fait.

« Le pessimisme est d’humeur, disait Alain, l’optimisme est de volonté :
tout homme qui se laisse aller est triste. » Je ne sais. Qu'il faille remonter la
pente plutôt que la descendre, viser la joie plutôt que la tristesse, enfin se gouverner
plutôt que s’abandonner, j’en suis bien sûr d’accord. Mais à condition
de ne pas sacrifier pour cela une once de lucidité. La vérité, pour un philosophe,
vaut plus que le bonheur.

J'aime mieux la formule de Gramsci : {\it « Pessimisme de l'intelligence, optimisme
de la volonté. »} Voir les choses comme elles sont, puis se donner les
moyens de les transformer. Envisager le pire, puis agir pour l’éviter. On n’en
mourra pas moins ? On n’en vieillira pas moins ? Sans doute. Mais on aura
vécu davantage.

\section{Ordre}
%ORDRE
Un désordre facile à mémoriser, à reconnaître ou à utiliser. Ainsi
l’ordre des lettres, dans un mot, ou l’ordre alphabétique, dans un
dictionnaire. « L'ordre, écrit Marcel Conche à propos de Lucrèce, n’est qu’un
cas particulier du désordre » : c’est un désordre commode, efficace ou significatif.
C’est dire qu’il n’y a d’ordre — donc aussi de désordre — que pour nous.
C’est ce qu’indique fort clairement Spinoza, dans l’Appendice du livre I de
l'{\it Éthique} : « Quand les chose sont disposées de façon que, nous les représentant
par les sens, nous puissions facilement les imaginer et, par suite, nous les rappeler
facilement, nous disons qu’elles sont bien ordonnées; dans le cas
contraire, qu’elles sont mal ordonnées ou confuses. Et comme nous trouvons
plus d’agrément qu’aux autres aux choses que nous pouvons imaginer avec facilité,
les hommes préfèrent l’ordre à la confusion ; comme si, sauf par rapport à
notre imagination, l’ordre était quelque chose dans la nature. » Mais ce n’est
qu'une illusion : l’ordre n’est qu’un désordre qui nous arrange ; le désordre,
qu'un ordre qui nous déçoit. C’est pourquoi le second principe de la thermodynamique,
par la notion d’{\it entropie} (voir ce mot), est essentiellement
décevant : parce que le désordre est toujours plus probable que l’ordre, et ne
%— 414 
peut donc, dans un système isolé, que s’accroître. Cela ne retire rien à la thermodynamique,
mais beaucoup à nos espérances : elles ne sont ultimement crédibles
que si univers {\it n'est pas} un système isolé (que s’il existe autre chose que
l'univers, qui peut agir sur lui : par exemple un Dieu et une providence).

La notion d’ordre, sans parler de son acception impérative (« donner un
ordre »), a aussi un autre sens, par exemple dans les sciences naturelles ou chez
Pascal : elle peut désigner un sous-ensemble ou un domaine, ayant ses caractéristiques
ou sa logique propres (lordre des primates, l’ordre du cœur.....).
L'essentiel, philosophiquement, est alors de ne pas confondre des ordres
différents : je m’en explique ci-après.

\section{Ordres (Distinction des —)}
%ORDRES (DISTINCTION DES —)
Cette notion, en philosophie, doit surtout
à Pascal. On sait qu’il distingue
trois ordres différents : l’ordre des corps ou de la chair, l’ordre de l’esprit ou de
la raison, enfin l’ordre du cœur ou de la charité. Chacun de ces ordres a sa
cohérence propre, ses valeurs propres, son efficace propre, mais qui ne peuvent
rien dans un autre ordre. C’est ce qu’indique la fin du décisif fragment 308-793 :

\vspace{0.5cm}

{\footnotesize 
« Tous les corps, le firmament, les étoiles, la terre et ses royaumes ne valent pas le
moindre des esprits. Car il connaît tout cela, et soi ; et les corps, rien.

Tous les corps ensemble et tous les esprit ensemble et toutes leurs productions ne
valent pas le moindre mouvement de charité. Cela est d’un ordre infiniment plus élevé.

De tous les corps ensemble on ne saurait en faire réussir une petite pensée. Cela est
impossible et d’un autre ordre. De tous les corps et esprits on n’en saurait tirer un mouvement
de vraie charité ; cela est impossible, et d’un autre ordre, surnaturel. »
}

\vspace{0.5cm}

Soit par exemple le théorème de Pythagore ou un fait historique quelconque,
bien avéré. Combien d’armées faudrait-il pour les rendre faux ? Un
nombre infini n’y suffirait pas : toutes les armées de l’univers ne peuvent rien
contre une vérité, ni l’univers lui-même.

Et combien de théorèmes faudrait-il pour susciter un vrai mouvement de
charité ? L’infini n’y suffirait pas : tous les théorèmes de l’univers, même joints
à toutes les armées du monde, ne sauraient suppléer aux défaillances du cœur
ou à l’absence de la grâce.

C’est en pensant à Pascal que j’ai repris cette idée de distinction des ordres,
mais en l’appliquant à une classification différente. S'agissant de la société, j'ai
pris l’habitude de distinguer quatre ordres différents : {\it l'ordre techno-scientifique},
structuré intérieurement par l'opposition du possible et de l'impossible (ou,
d’un point de vue scientifique, du possiblement vrai et du certainement faux),
%— 415 —
mais incapable de se limiter lui-même ; limité donc, de l’extérieur, par un
second ordre : {\it l'ordre juridico-politique}, lequel est structuré intérieurement par
l'opposition du légal et de l’illégal, mais tout aussi incapable que le précédent
de se limiter soi ; limité donc à son tour, de l’extérieur, par un troisième ordre :
{\it l'ordre de la morale}, lequel est structuré intérieurement par l'opposition du
devoir et de l’interdit, et complété, plutôt que limité, ou ouvert par en haut,
vers un quatrième ordre : {\it l'ordre éthique, ou ordre de l'amour}. Mon idée est que
chacun de ces ordres a sa cohérence propre, ses contraintes ou ses exigences
propres, enfin son autonomie : on ne vote pas sur le vrai et le faux, ni sur le
bien et le mal; mais la morale ou les sciences ne sauraient pas davantage
régenter la politique et le droit. C’est pourquoi chacun de ces ordres est nécessaire,
sans pouvoir pour autant fonctionner seul : il a besoin de l’ordre immédiatement
inférieur pour exister (c’est ce que j'appelle {\it l'enchaînement descendant
des primats}, sans lequel rien n’est possible), mais ne peut être limité et jugé
que par un ordre supérieur (ce que j'appelle {\it la hiérarchie ascendante des primautés},
sans laquelle rien n’a de sens). Quand on oublie cette distinction des
ordres, ou quand on prétend qu’un seul de ces ordres peut suffire, on est voué
au ridicule ou à la tyrannie, sous deux formes opposées : l’angélisme, si c’est au
bénéfice d’un ordre supérieur, ou la barbarie, si c’est au bénéfice d’un ordre
inférieur (sur tout cela, voir {\it Valeur et vérité}, p. 207 à 226).

À cette quadripartition, qui dessine une espèce de topique, on pourrait
ajouter un ordre zéro, qui serait celui du réel ou de la nature, et un ordre cinquième
et ultime, qui serait, pour ceux qui y croient, l’ordre surnaturel ou
divin. Dans mon esprit, l’ordre zéro contient tous les autres : c’est moins un
ordre de plus que le lieu et la condition de tous. C’est ce qui m’empêche de
considérer un éventuel cinquième ordre autrement que comme le prolongement
fantasmatique des quatre autres (le Dieu tout-puissant et omniscient,
celui qui commande et juge, enfin le Dieu d’amour). Mais, même pour un
croyant, il me semble que l'esprit de la laïcité interdit de soumettre purement
et simplement l’un quelconque de ces quatre ordres au cinquième. Ce n’est pas
parce que Dieu m’ordonne quelque chose que c’est moralement bon, expliquait
déjà Kant, c’est parce que c’est moralement bon que je peux envisager
que cela vienne de Dieu : « Même le Saint de l'Évangile doit être d’abord comparé
avec notre idéal de perfection morale avant qu’on le reconnaisse pour tel »
({\it Fondements...}, II ; voir aussi {\it C. R. Pratique}, Dialectique, II, 9, et {\it La religion
dans les limites de la simple raison}, Préface). Même chose, bien sûr, pour les
ordres 2 et 4 : celui qui voudrait soumettre le droit ou l’amour à la supposée
volonté de Dieu devrait renoncer à la souveraineté du peuple autant qu’à sa
propre autonomie d’être humain : c’est ce qu’on appelle l’intégrisme. Contre
%— 416 —
quoi la distinction des ordres, au sens où je prends l’expression, n’est pas autre
chose qu’une tentative pour penser la laïcité jusqu’au bout.

\section{Orgueil}
%ORGUEIL
Adolescent, je m'étais soumis, à la demande d’une amie, au
fameux « Questionnaire de Proust ». De mes réponses d’alors,
j'ai tout oublié, sauf ceci, qui m’avait paru finement paradoxal :

« — Votre principal défaut ?

— L’orgueil.

— Votre principale qualité ?

— L’orgueil. »

J'en suis bien revenu. Non que j'aie cessé d’être orgueilleux (quoique je le
sois sans doute moins) ; mais j’ai cessé d’y voir une vertu.

Tout orgueil est illusoire : c’est se prêter plus de mérites qu’on n’en a, ou se
flatter, bien sottement, de ceux qu’on peut avoir. Quoi de plus ridicule que
d’être fier de sa taille, de sa beauté, de sa santé ? Et pourquoi le serait-on davantage
de son intelligence ou de sa force ? As-tu choisi d’être qui tu es ? Es-tu
maître de le rester ? Un petit caillot mal placé, te voilà stupide et grabataire. II
n’y aura pas lieu alors d’avoir honte, ni donc aujourd’hui d’être fier.

« L’orgueil, écrivait Spinoza, consiste à avoir de soi, par amour, une
meilleure opinion qu’il n’est juste » ({\it Éthique}, IV, déf. 28 des affects). Tout
orgueil, par définition, est donc injuste : sans justice vis-à-vis des autres, sans
justesse vis-à-vis de soi. Ce n’est qu’un piège de l’amour-propre.

\section{Origine}
%ORIGINE
C’est moins le commencement que ce qui le permet, le précède
ou le prépare. Par exemple le {\it big bang}, pour l’univers, fait un
commencement plausible. Mais assurément pas une origine : car pourquoi le
{\it big bang} ?

L'origine serait donc la cause ? Pas tout à fait ou pas seulement, puisque les
causes sont ordinairement multiples, dont chacune, à son tour, doit avoir sa
propre cause. Une cause explique un fait ou un événement ; l’origine rendrait
plutôt raison d’un être ou d’un devenir. Si c'était une cause, ce serait plutôt la
cause première ou ultime : celle qui rend raison de toutes les autres, ou leur
série complète. Par quoi l’origine nécessairement nous échappe : c’est la ligne
d’horizon de la causalité.

\section{Oubli}
%OUBLI
C’est le contraire non de la mémoire, que l'oubli suppose, mais du
souvenir : il y a oubli lorsque je ne me souviens plus de quelque
%— 417 —
chose que j'ai eu, au moins un temps, en mémoire. Loin que cela soit toujours
fautif ou pathologique, il faut y voir bien souvent une forme de santé, voire de
générosité. C’est ce qu’a vu Nietzsche, dans la deuxième de ses {\it Considérations
intempestives} :

\vspace{0.5cm}

{\footnotesize 

« L'homme qui est incapable de s'asseoir au seuil de l'instant, en oubliant tous les
événements passés, celui qui ne peut pas, sans vertige et sans peur, se dresser un instant
tout debout, comme une victoire, ne saura jamais ce qu'est un bonheur, et, ce qui est
pire, il ne fera rien pour donner du bonheur aux autres. [...] Tout acte exige l’oubli.
[...] Il est possible de vivre presque sans se souvenir, et de vivre heureux, comme le
démontre l’animal, mais il est impossible de vivre sans oublier. Il y a un degré
d’insomnie, de rumination, de sens historique qui nuit au vivant et qui finit par le
détruire, qu’il s'agisse d’un homme, d’une nation ou d’une civilisation. »

}

\vspace{0.5cm}


Toutefois il n’est pas possible non plus d'oublier tout, ou il faudrait pour
cela renoncer à son humanité. Par quoi l’oubli fait partie, exactement, du travail
de la mémoire : c’est son pôle négatif, comme une sélection obligée, qui ne
retiendrait que ce qui est utile, plaisant ou dû (qui ne se souviendrait que par
intérêt, par gratitude ou par fidélité).

\section{\it Ousia}
%{\it OUSIA}
Le mot, qui est un substantif dérivé du verbe eînai (être), peut se traduire,
selon les auteurs et les contextes, par {\it être, essence, réalité} ou
{\it substance}. C’est le réel même, ou la réalité vraie ({\it ousia ontôs ousa}, écrit Platon
dans le {\it Phèdre}, 247 c : la réalité vraiment existante). Mais il ne suffit pas de la
nommer en grec pour savoir ce qu’elle est.

\section{Outil}
%OUTIL
Un objet fabriqué et utile ? Sans doute. Mais cela serait vrai aussi
d’un fauteuil ou d’un lit, qui ne sont pas {\it outils} pour autant. L'outil
est utile, mais à un certain travail : c’est un objet fabriqué, qui sert à en fabriquer
— ou à en transformer — d’autres. On y a longtemps vu le propre de
l’homme, défini alors comme {\it homo faber} (voir par exemple Bergson, {\it L'évolution
créatrice}, II, p. 138 à 140). Paléontologues et éthologues, aujourd’hui, sont
plus réservés : il peut arriver à un grand singe (et il est arrivé à des hominiens,
bien avant {\it homo sapiens}) de fabriquer un outil. C’est qu’il y a là une preuve
d'intelligence, plutôt que d'humanité. Or rien ne prouve que l'intelligence
commence à l'humanité, ni donc l’humanité à l’outil. Ce n’est qu’un moyen ;
seules les fins sont humaines.
%{\footnotesize XIX$^\text{e}$} siècle — {\it }
% 418
