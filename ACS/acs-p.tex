%{\it }
% —
%{\footnotesize XIX$^\text{e}$} siècle — {\it }
PACIFIQUE Toute guerre est atroce, c’est une banalité qu’on ne répétera
jamais assez. Être pacifique, ce n’est pas une opinion : c’est une
vertu, et qui voudrait en manquer ? Toutefois cela n’implique pas que toute
paix soit bonne, ni même acceptable. C’est ce qui distingue le {\it pacifique} du {\it pacifiste}.
Être pacifique, c’est désirer la paix, c’est chercher à l'obtenir ou à la
défendre, mais pas à n’importe quel prix et sans s’interdire absolument la violence
ou la guerre. C’est la position de Spinoza : la guerre ne doit être entreprise
qu’en vue de la paix, et d’une paix qui soit celle non de la servitude mais
d’une population libre. C’est la position de Simone Weil. Toute violence est
mauvaise, mais non pour cela condamnable : la non-violence n’est bonne que
si elle est efficace, et elle ne l’est pas dans toutes les situations (« cela dépend
aussi de l’adversaire »). Être pacifique, pour le dire d’un mot, c’est faire de la
paix son but. Cela ne prouve pas, hélas, qu’elle suffise toujours comme moyen.

PACIFISTE Être pacifiste, ce n’est pas une vertu, encore moins un vice ; c’est
une opinion, une doctrine ou une idéologie, qui juge que toute
guerre est non seulement mauvaise, ce qui est bien clair, mais encore nuisible
ou condamnable, qu’elle ne saurait être justifiée par rien, enfin que la paix, en
toutes circonstances, vaut mieux. C’était à peu près la position d’Alain (encore
acceptait-il, sur le territoire national, des guerres purement défensives), c’est à
peu près celle, aujourd’hui, de Marcel Conche, et ces deux noms suffraient à
me la rendre respectable. Toutefois cela fit aussi d’Alain un Munichois, certes
pour d’estimables raisons (par pacifisme, par antimilitarisme, l’un et l’autre
renforcés par le traumatisme de la première guerre mondiale), mais sans que je
puisse, sur ce terrain-là, tout à fait le suivre. Qu'il n’y ait pas de guerre juste,
%— 419 —
%{\footnotesize XIX$^\text{e}$} siècle — {\it }
comme me l’a répété souvent Marcel Conche, j’en suis bien sûr d’accord, si
l’on entend par là une guerre qui ne tuerait que des coupables. Mais n’est-ce
pas confondre la guerre, précisément, avec la justice ? Il ne s’agit pas de punir,
mais d'empêcher ou de vaincre. Que toute guerre soit injuste, dans son atroce
déroulement, cela ne prouve pas que toute paix soit supportable, ni même
admissible.

PAILLARDISE C'est « le désir gai », disait Alain, spécialement en matière de
sexualité, et comme «une précaution du rire contre les
passions ». Cela ne va pas sans un peu de vulgarité délibérée, qui fait comme
une précaution supplémentaire contre le sérieux ou l’hypocrisie. Toutefois, cela
ne dure qu’un temps — comme le désir, comme la gaieté —, sans quoi ce n’est
plus paillardise mais obsession (non plus une précaution contre les passions,
mais une passion de plus).

PAIX  L’absence, non de conflits, mais de guerre. Ce n’est pas encore la
concorde, mais cela vaut mieux, presque toujours, que la violence
armée ou militaire. Ce {\it presque} ne va pas de soi : c’est ce qui distingue les pacifiques
des pacifistes (voir ces mots). « S’il faut appeler paix l’esclavage, la barbarie
ou l'isolement, disait Spinoza, il n’est rien pour les hommes de si lamentable
que la paix » ({\it Traité politique}, VI, 4 ; voir aussi V, 4). Et rien de meilleur,
si elle va avec la justice et la liberté.

PANENTHÉISME Doctrine pour laquelle tout est en Dieu, sans être Dieu
pour autant. Se distingue par là du panthéisme (voir ce
mot). En ce sens, il y a un panenthéisme chrétien, qui peut se réclamer de saint
Paul (« c’est en Dieu que nous avons la vie, le mouvement et l’être », {\it Actes}...,
17, 28) et dont il arrive que Spinoza se réclame (voir par exemple la {\it Lettre} 73,
à Oldenburg).

PANTHÉISME C’est croire en un Dieu qui serait tout, ou en un tout qui
serait Dieu. Dieu serait donc le monde, comme on voit
chez les stoïciens, ou la nature, comme on voit chez Spinoza (« {\it Deus sive
Natura} »), et il n’y en aurait pas d’autre. C’est ce qui explique que le panthéisme,
bien souvent, fut accusé d’athéisme. Mais ce peut être aussi bien une
religion de l’immanence.

%— 420 —
%{\footnotesize XIX$^\text{e}$} siècle — {\it }
Les historiens de la philosophie distinguent parfois le {\it panthéisme}, qui
affirme que tout est Dieu, du {\it panenthéisme}, qui affirme que tout est {\it en} Dieu.
Ainsi Gueroult, à propos de Spinoza (t. 1, p. 223). Cela maintient une distance
entre Dieu ou la substance, d’une part, et ses modes d’autre part — entre la
Nature naturante, comme on peut dire aussi, et la Nature naturée. Et sans
doute cela, dans l’économie du système, est légitime : Spinoza n’a jamais cru
que les fleurs ou les oiseaux étaient Dieu. Pourtant, dès lors que cette distance
n’existe elle-même qu’en Dieu, dès lors qu’il n’y a, de la Nature naturante à la
Nature naturée, aucune transcendance d’aucune sorte, je ne suis pas sûr que
cette distinction soit vraiment éclairante. « Plus nous connaissons les choses
singulières, écrit Spinoza, plus nous connaissons Dieu » ({\it Éthique}, V, 24). C’est
plus, me semble-t-il, que du panenthéisme. Tout n’est pas Dieu ? Sans doute,
puisque seul le tout est Dieu. Mais il n’en reste pas moins que Dieu et la
Nature sont une seule et même chose : non seulement tout est en Dieu, mais
Dieu est tout en tout (puisqu'il n’y a rien d’autre). Si ce n’est pas du panthéisme,
qu'est-ce ?

PAPISME C’est un autre nom pour le catholicisme, parce qu’il reconnaît
l'autorité et l’infaillibilité du Pape. Le mot, inventé par les protestants,
est bien sûr péjoratif. Toutefois il ne suffit pas de ne pas avoir de pape
pour échapper au fanatisme.

PÂQUES Au singulier, c’est une fête juive, qui commémore la sortie d'Égypte.
Au pluriel, une fête chrétienne, qui commémore la résurrection du
Christ. Fête païenne, disait Alain, puisqu'elle ne fait que célébrer le triomphe
de la vie sur la mort. C’est la fête du printemps, qu’on trouve chez tous les
peuples, et de la résurrection, qu’on trouve chez la plupart. La vraie fête chrétienne,
pour Alain, c’est Noël : parce qu’elle célèbre la faiblesse plutôt que la
force, et l'amour plutôt que la vie ou la victoire. L'enfant nu et pourchassé,
entre le bœuf et l’âne, sans autre protection qu’une jeune mère, qui prie et
tremble, et qu’un père, qui s'interroge. Et la vraie fête de l'esprit, ajouterais-je,
c'est le Vendredi saint : parce quelle ne célèbre rien que la justice offensée,
l'amour déchiré et déchirant, enfin le courage pur, sans haine, sans violence,
sans espérance. « Mon Dieu, mon Dieu, pourquoi m’as-tu abandonné ? » Fête,
non de la foi, mais de la fidélité. Qu'il y ait eu ou pas résurrection, qu'est-ce
que cela change à la grandeur du Christ et de son message ? Cela dit par différence,
sur Pâques, l'essentiel : c’est la fête de la foi plus que de la fidélité, et de
l'espérance plus encore que de la foi. C’est la vraie fête religieuse, donc la vraie
%— 421 —
%{\footnotesize XIX$^\text{e}$} siècle — {\it }
fête chrétienne, malgré Alain, en tant que le christianisme est une religion. On
voudrait y croire, comme on croit au printemps. Mais le printemps n’est pas
dieu. Mais la vie n’est pas Dieu. Les athées, ce jour-là, se sentent plus athées
que jamais : ils ne croient qu’en l'esprit vivant et mortel. Ils ne l’en aiment que
davantage.

PARADIGME Un exemple privilégié ou un modèle, qui sert à penser. Le mot,
qu’on trouve chez Platon ou Aristote ({\it paradeigma}), sert surtout,
aujourd’hui, en épistémologie ou en histoire des sciences. C’est l’un des
concepts majeurs de Thomas Kuhn, dans {\it La structure des révolutions scientifiques}.
Un paradigme, c’est l’ensemble des théories, des techniques, des
valeurs, des problèmes, des métaphores, etc., que partagent, à telle ou telle
époque, les scientifiques d’une discipline donnée: c’est la « matrice
disciplinaire » qui leur permet de se comprendre et d’avancer. C’est aussi, et
par là même, ce qui est transmis aux étudiants, à la même époque, qui leur
permet de comprendre la science de leur temps, de s’y reconnaître et d’y travailler.
L'état normal des sciences (la « science normale », dit Kuhn) est celui
où un paradigme règne. Le terrain de la recherche est alors balisé par les
découvertes antérieures, et cela fait, entre les chercheurs, comme un
consensus efficace : ils sont d’accord non seulement sur les découvertes déjà
faites, mais sur ce qui reste à découvrir et sur les méthodes, pour ce faire, à
mettre en œuvre. Les révolutions scientifiques, à l’inverse, sont les périodes
où un nouveau paradigme apparaît, qui s’oppose à l’ancien, résolvant certains
problèmes jusqu’alors insolubles, en faisant disparaître d’autres, en soulevant
de nouveaux... Ainsi quand on passe de la mécanique classique (celle de
Newton) à la physique relativiste (celle d’Einstein et de ses successeurs) : ce
ne sont pas seulement les solutions qui sont nouvelles, mais aussi les problèmes,
les difficultés, les procédures. Deux paradigmes en concurrence sont
pour cela incommensurables, explique Kuhn : on ne passe de l’un à l’autre
que par une espèce de conversion globale, qui ne saurait se réduire à un progrès
purement rationnel et qui interdit de juger une théorie selon les valeurs
paradigmatiques de l’autre. Cela n'empêche pas qu’il y ait progrès, mais
interdit de le penser comme un processus linéaire et continu. Le progrès,
dans les sciences non plus, n’est pas un long fleuve tranquille.

PARADIS Le lieu de la félicité, qui n’a jamais lieu. Ainsi le paradis n’est rien
de réel : ce n’est qu’un mythe ou une niaiserie.

%— 472 —
%{\footnotesize XIX$^\text{e}$} siècle — {\it }
PARADOXE Une pensée qui va contre l’opinion, ou contre la pensée.
Cela fait deux sens différents. Aller contre l'opinion ({\it doxa}), ce
n’est en rien condamnable. Cela ne prouve bien sûr pas qu’on ait raison (un
paradoxe peut être vrai ou faux), mais suggère au moins qu’on ne se contente
pas de répéter ce qui se dit. Par exemple lorsque Oscar Wilde écrit que « la
nature imite l’art » : c’est un paradoxe, puisque la plupart des gens croient que
l’art imite la nature, mais qui peut être éclairant (il nous laisse entendre que
notre vision de la nature est influencée par celle des artistes : « Avez-vous
remarqué, demandait Oscar Wilde, comme la nature, depuis quelque temps,
ressemble à un tableau impressionniste ? »). Ou quand Talleyrand conseillait :
« Méfiez-vous du premier mouvement ; c’est le bon. » C’est un paradoxe (pourquoi
se méfier de ce qui est bon ?), mais qui donne à réfléchir : si le premier
mouvement est le bon, au sens moral du terme, il peut s’avérer très mauvais
dans un autre registre (par exemple politique ou diplomatique). On remarquera
que la plupart des paradoxes viennent d’un double sens attribué à l’un au
moins des mots utilisés : la formule, qui semble absurde selon l’un de ces sens,
peut s'avérer profonde selon un autre. Toutefois, il y a de vrais paradoxes, qui
vont vraiment contre l'opinion dominante, et sans jouer sur quelque double
sens que ce soit. Par exemple lorsque Spinoza écrit que ce n’est pas parce
qu’une chose est bonne que nous la désirons, mais inversement parce que nous
la désirons que nous la jugeons bonne ({\it Éthique}, III, 9, scolie). Nous avons tous
le sentiment du contraire. Cela ne prouve pas que Spinoza ait tort, ni qu’il ait
raison.

Mais le mot {\it paradoxe} à aussi un sens purement logique : c’est une pensée
qui va contre la pensée, disais-je, autrement dit une contradiction ou une antinomie.
Par exemple le paradoxe de Russel : l’idée de l’ensemble de tous les
ensembles qui ne se contiennent pas eux-mêmes fait un paradoxe, dans la
théorie classique des ensembles (puisque cet ensemble se contient lui-même s’il
ne se contient pas lui-même). On considère ordinairement qu’un paradoxe,
décelé dans une théorie donnée, vaut comme réfutation ou suppose, à tout le
moins, quelque aménagement : c’est ce qui s’est passé, après le paradoxe de
Russel, pour la théorie des ensembles (dont l’axiomatique exclut désormais
qu’un ensemble puisse se définir par la propriété de ne pas se contenir lui-même
comme élément). Par quoi les paradoxes, quand ce ne sont pas des sottises,
aident la pensée à avancer.

PARALOGISME Une faute dans un raisonnement, mais involontaire. C’est
ce qui distingue le paralogisme du sophisme : le sophisme
veut tromper ; le paralogisme se trompe.

%— 423 —
%{\footnotesize XIX$^\text{e}$} siècle — {\it }
Les « paralogismes de la raison pure », chez Kant, sont les raisonnements
dialectiques qui portent sur la première des trois « Idées de la raison » (l’âme, le
monde, Dieu). Ce sont des illusions dans lesquelles tombe inévitablement la
psychologie rationnelle, dès lors qu’elle prétend connaître l’âme (comme noumène)
alors que nous n’en avons aucune expérience. Le paralogisme est en
l'occurrence de prétendre conclure de l’unité purement formelle de l’aperception
transcendantale (Punité du «je pense») à son existence substantielle
comme sujet (âme), à sa simplicité, à sa personnalité ou à son immortalité.
C’est traiter une idée comme un objet, et vouloir passer (comme la preuve
ontologique le fait à propos de Dieu, et aussi illusoirement) de la pensée à
l'existence.

PARANOÏA On ne la confondra pas avec le délire de persécution, qui n’est
qu’une de ses formes. Le paranoïaque, s’il se croit parfois persécuté,
est souvent persécuteur. Mais cela même n’est qu’un symptôme. La paranoïa
n'est pas un vice ; c’est une psychose ou un type de personnalité. Une
folie ? Elle peut aller jusque-là, sans que le paranoïaque ait pour autant perdu
la raison. Il en ferait plutôt un usage exagéré, obsessionnel, agressif. La paranoïa,
disait Kraepelin, est caractérisée par « le développement lent et insidieux
d’un système délirant durable et impossible à ébranler, et par la conservation
absolue de la clarté et de l’ordre dans la pensée, le vouloir et l’action ». Hypertrophie
du moi, survalorisation de la logique, délire d’interprétation ou de persécution,
méfiance, rigidité, inadaptabilité..... Elle est plus fréquente chez les
hommes. C’est l’un des points qui l’opposent à l’hystérie, plus fréquente chez
les femmes, mais ce n’est pas le seul. À les considérer comme types de personnalités
davantage que comme pathologies, l’hystérie et la paranoïa constituent
comme deux pôles opposés : l’hystérique ne vit que pour les autres, ou plutôt
que pour soi et par eux ; le paranoïaque que pour soi et contre les autres. L’hystérique
est influençable, séducteur, peu soucieux de logique, avide d’amour ; le
paranoïaque est inébranlable, soupçonneux, raisonneur, avide de pouvoir. L’un
vit pour plaire : c’est un comédien ou un histrion. L'autre, pour dominer : c’est
un petit chef ou un tyran. L’un multiplie les signes ; l’autre, les interprétations.
L’un voudrait faire de sa vie une œuvre d’art ; l’autre, un système philosophique.
« On pourrait presque dire, remarquait Freud, qu’une hystérie est une
œuvre d’art déformée, qu’une névrose obsessionnelle est une religion déformée, et
un délire paranoïaque, un système philosophique déformé » ({\it Totem et tabou}, II).
Cela ne prouve rien contre l’art, ni contre la philosophie ; mais devrait pousser
à une certaine vigilance, contre l’esthétisme et les systèmes.

%— 424 —
%{\footnotesize XIX$^\text{e}$} siècle — {\it }
PARDON Ce n’est pas l’absolution, qui supprimerait ou effacerait la faute,
ce que nul ne peut ni ne doit. Ce n’est pas l’oubli, qui serait infidèle
ou imprudent. Pardonner, ce n’est ni oublier ni effacer ; c’est renoncer,
selon les cas, à punir ou à haïr, et même, parfois, à juger. Vertu de justice
(puisqu'il faut juger sans haine) et de miséricorde.

PARESSEUX (ARGUMENT -) Cest un argument traditionnellement opposé
au fatalisme, notamment stoïcien. Si tout
est déterminé ou soumis au destin, il n’y a plus lieu d’agir ni de se donner de la
peine pour quoi que ce soit : le fatalisme ne pourrait aboutir qu’à la paresse ou
à l’inaction. Par exemple, s’il est écrit que je serai reçu à mon examen, à quoi
bon le préparer ? Et à quoi bon, s’il est écrit que je serai recalé ? Il n’y aurait
donc lieu de le préparer dans aucun des deux cas, qui sont les seuls possibles.
C'est bien sûr un contresens : ce qui est fatal, pour les stoïciens, ce n’est pas tel
ou tel événement isolé (par exemple le résultat d’un examen), mais l’enchaînement
des causes et des événements (chaque événement étant ainsi « confatal »,
comme disait Chrysippe, avec d’autres : par exemple le travail de l'élève avec ses
résultats à l’examen). On trouve chez Cicéron un autre exemple : « Que tu aies
appelé ou non un médecin, tu guériras » : inutile donc, suggère l'argument
paresseux, de l'appeler et de suivre ses conseils... «C’est là un sophisme,
explique Cicéron ; car il est autant dans ton destin d’appeler un médecin que
de guérir ; ce sont choses que Chrysippe appelle confatales » ({\it De fato}, XIII, 30).
Le réel est à prendre en bloc, ou à laisser ; mais que tu le prennes ou pas, cela
fait partie du réel.

PARFAIT Ce à quoi rien ne manque, ni quantitativement ({\it parfait}, en ce sens,
signifie achevé), ni qualitativement (est parfait, en ce sens, ce qui
ne peut être ni amélioré ni surpassé).

Les deux sens se rejoignent : est parfait ce qui est sans défaut. Mais qu’est-ce
qu’un défaut ? Un manque, c’est-à-dire un néant, qui ne devient réel que par
l'imagination d’autre chose. Par quoi tout est parfait, dès qu’on cesse d’imaginer.
C’est le vrai secret, le plus difficile, le plus simple, que Spinoza — après
saint Thomas et Descartes, mais en en transformant le sens — a génialement
résumé en une phrase : « Par réalité et par perfection j’entends la même chose »
({\it Éthique}, II, déf. 6). Ce qui signifie que le réel est tout ce qu’il est (donc aussi,
au présent, tout ce qu’il peut être), sans aucune faute.

À quoi l’on objecte habituellement l’évidence du mal et la vanité de nos
efforts, si tout est parfait, pour changer ce qui est. C’est doublement se
%— 425 —
%{\footnotesize XIX$^\text{e}$} siècle — {\it }
méprendre. Ce que nous appelons le mal (la douleur, l'injustice, l’égoïsme...)
est aussi réel que le reste, aussi vrai, aussi parfait en ce sens, de même que nos
efforts pour le combattre ou lui résister. La tumeur qui te tue, ce n’est pas parce
qu’elle serait imparfaite ; c’est parce qu’elle est parfaitement tumeur et parfaitement
mortelle, Et même chose, bien sûr, pour les médicaments qu’on lui
oppose : qu’ils soient parfaitement efficaces ou parfaitement insuffisants, ils
n'en sont pas moins parfaitement ce qu’ils sont. Cela signifie que tout jugement
de valeur est subjectif, donc aussi nécessaire (pour les sujets que nous
sommes) qu’illusoire (si nous prétendons y voir autre chose qu’un reflet de
notre subjectivité). C’est ce que Deleuze, lisant Spinoza, sut formuler
exactement : « Si le mal n’est rien, selon Spinoza, ce n’est pas parce que seul le
Bien est et fait être, mais au contraire parce que le bien n’est pas plus que le
mal, et que l’Être est par-delà le bien et le mal » (Spinoza, {\it Philosophie pratique},
III). Par-delà le bien et le mal ? On pourrait dire aussi bien {\it en deçà} : C'est le
point de vue de Dieu (comme dirait Spinoza) ou du vrai (comme je préférerais
dire), qui contient tous les autres. Rien ne manque au réel, voilà le point,
puisque tout est là. C’est la sagesse de Prajnânpad. C’est la sagesse d’Etty
Hillesum, et c’est la seule. Un optimisme ? Nullement. Un pessimisme ? Pas
davantage. L'essentiel tient dans ces quelques phrases, qu’Etty Hillesum écrivit
dans un camp de transit, avant de partir pour Auschwitz : « On me dit parfois :
“Oui, tu vois toujours le bon côté de tout.” Quelle platitude ! Tout est parfaitement
bon. Et en même temps parfaitement mauvais. Les deux faces des
choses s’équilibrent, partout et toujours. Je n’ai jamais eu l’impression de
devoir me forcer à voir le bon côté des choses : tout est toujours parfaitement
bon, tel quel. Toute situation, si déplorable soit-elle, est un absolu et réunit en
soi le bon et le mauvais » ({\it Lettre de Westerbork}, du 11 août 1943). Sagesse
tragique : nous sommes déjà dans le Royaume, mais l’on se trompe, assurément,
si l’on y voit un paradis.

PARI Un engagement qu’on prend sur l’incertain, par exemple une course
de chevaux, qui sera sanctionné, selon le résultat, par un gain ou
une perte. En philosophie, le plus fameux est bien sûr celui de Pascal, qui
veut convaincre l’incroyant — puisque nous sommes « embarqués » — de
parier que Dieu existe : « Si vous gagnez, vous gagnez tout ; si vous perdez,
vous ne perdez rien. Gagez donc qu’il est sans hésiter » ({\it Pensées}, 418-233).
Cela se calcule. L'écart entre la mise et le gain doit être proportionné à la
probabilité de celui-ci. C’est ce qu’on appelle l’espérance mathématique : le
rapport entre le gain et la mise, multiplié par la probabilité de gagner (le pari
est raisonnable si ce rapport est au moins égal à 1). À pile ou face, il n’est
%— 426 —
%{\footnotesize XIX$^\text{e}$} siècle — {\it }
pas raisonnable de parier si le gain n’est pas au moins le double de la mise.
Ni de ne pas parier, s’il est supérieur au double. Avec un seul dé, il n’est pas
raisonnable de parier si le gain n’est pas au moins égal à six fois la mise
(puisqu'on n’a qu’une chance sur six de gagner), ni de ne pas parier, s’il est
supérieur à cette somme. Dès lors que le gain est réputé infini (« une infinité
de vie infiniment heureuse ») pour une mise finie (puisqu'il ne s’agit que de
notre vie terrestre, que d’ailleurs nous n’en vivrons pas moins, ou plutôt que
nous n’en vivrons que mieux) et avec un risque lui-même fini («un hasard
de gain contre un nombre fini de hasards de perte »), il est en effet raisonnable
de parier : « Partout où est l’infini, et où il n’y a pas infinité de hasards
de perte contre celui de gain, il n’y a point à balancer, il faut tout donner »
({\it ibid.}).

On remarquera que ce pari n’est en aucun cas une preuve de l'existence de
Dieu, mais seulement de l’intérêt que nous avons à y croire, ou à essayer d’y
croire (la vraie foi n’est donnée que par la grâce : le pari ne s’adresse, dans
l'esprit de Pascal, qu'aux incroyants). Reste à savoir si la pensée doit se soumettre
à l'intérêt ; c’est ce que je ne crois pas. Combien faudrait-il vous payer
pour être raciste, pour penser que l'injustice est bonne, que la Terre est immobile
ou que deux plus deux font cinq ? Pour un esprit libre, une infinité de
gain, même sans aucun risque, n’y suffirait pas. Ainsi argument du pari, si
fameux, si intelligent, ne vaut que pour ceux qui sont prêts à jouer leur vie, leur
esprit ou leur liberté aux dés, que pour ceux, pour mieux dire, qui soumettent
leur pensée à un calcul d'intérêt. Ils ne sont pas si nombreux qu’on le croit.
Aussi ce pari, pour génial qu'il soit à sa façon, n’a-t-il pas convaincu grand
monde. Les vrais croyants n’en ont pas besoin et le jugeraient indigne. Les
incroyants, s'ils n’ont pas l’esprit vénal, ne peuvent l’accepter. Autant vendre sa
voix, lors d’une élection, au plus offrant. Autant soumettre sa pensée, lors d’un
colloque, à l’espérance d’une place ou d’un prix. Pascal méprisait trop les
humains. Son pari ne peut convaincre qu’un croupier, s’il est vénal, ou un
tiroir-caisse.

PAROLE L'acte, plutôt que la faculté, de parler. Se distingue du langage
comme l'actuel du virtuel; du discours, comme l’acte de son
résultat ; de la langue, comme le singulier du général, ou comme l’individuel
du collectif (Saussure, {\it Cours de linguistique générale}, Payot, chap. III et note 63).
Toute parole est création d’un discours par l’actualisation du langage (comme
faculté) au moyen d’une langue (comme système conventionnel et historique).
C’est le présent du sens.

%— 427 —
%{\footnotesize XIX$^\text{e}$} siècle — {\it }
PARTICULIER Ce qui vaut pour une partie d’un ensemble considéré, autrement
dit pour un ou plusieurs de ses éléments. S’oppose à
{\it universel} (qui vaut pour tous les éléments d’un ensemble) et se distingue de {\it singulier}
(qui ne vaut que pour un seul) tout en pouvant l’inclure. Par exemple
une proposition particulière porte sur quelques individus d’un ensemble
(« Quelques cygnes sont noirs», ou, comme diraient plutôt les logiciens,
« Quelque signe est noir »), voire sur un seul, s’il reste indéterminé («un cygne
est noir »). En revanche, une proposition universelle prenant le sujet dans toute
son extension, on peut dire aussi bien qu’une proposition singulière est universelle
si son sujet est déterminé, par exemple par un nom propre : « Aristote est
l’auteur de l'{\it Éthique à Nicomaque} » est une proposition à la fois singulière
(puisqu'elle ne porte que sur un seul individu) et universelle (puisqu’elle le
désigne tout entier). C’est ce qui explique que le mot {\it particulier} puisse désigner
aussi un individu, mais indéterminé (« un simple particulier ») : c’est n'importe
qui, en tant qu'il n’est pas tout le monde.

PASSÉ Ce qui fut et n’est plus. Tout passé reste éternellement vrai (même
Dieu, reconnaissait Descartes, ne peut faire que ce qui fut n’ait pas
été), mais sans puissance aucune, et sans acte. C’est le ne-plus-être-réel du vrai,
ou plutôt l’être-toujours-vrai de ce qui n’est plus réel. La vérité ne passe pas, et
c'est ce qu'on appelle le passé.

Contrairement à ce qu’on croit souvent, le passé n’agit jamais : ce qui agit
ou peut agir, ce sont ses traces ou ses effets actuels (qui ne sont pas du passé
mais du présent). Ainsi les séquelles d’un accident, les traumatismes, les souvenirs,
les rancunes, les promesses, les documents, et jusqu'aux causes
mêmes, qu'on croit expliquer le présent. Je ne ferais pas ce que je fais si je
n'avais vécu ce que j'ai vécu ; mais pas davantage s’il n’en restait rien de présent.
La première guerre mondiale, pareillement, ne peut contribuer à expliquer
la seconde que par ce qu’il en restait, en 1939, de réel. Enfin les étoiles
que nous contemplons, la nuit, ce n’est pas leur lumière passée qui agit sur
nos yeux (là-bas, il y a plusieurs milliers d'années !), mais ce qui en arrive, ici
et maintenant, jusqu’à nous. C’est où le réel et le vrai, pour la pensée, se séparent.
Tout passé est vrai (un mensonge ou une erreur sur le passé, ce n’est pas
du passé : c’est du présent) ; aucun n’est réel (s’il l'était, il ne serait pas du
passé). Mais comme toute vérité, par définition, est présente, on peut dire
aussi que le passé n’est rien : parce qu’il est passé, et parce que la vérité ne
passe pas. Ainsi il n’y a que le présent, et la vérité en lui de ce qui fut : il n’y
a que l'éternité.

%— 428 —
%{\footnotesize XIX$^\text{e}$} siècle — {\it }
PASSION Ce qu’on subit, mais en soi, sans pouvoir l'empêcher ni tout à fait
le surmonter. C’est le contraire ou le symétrique de l’action :
l’âme se soumet au corps, diraient les classiques, c’est-à-dire à la partie de soi
qui ne pense pas, ou qui pense mal. La folie est ainsi l'extrême de la passion,
comme le penchant ou l’inclination sont sa forme bénigne. Mais on utilise le
terme, plutôt, pour l’entre-deux.

La passion est un état d'âme, souvent vigoureux, mais hétéronome : c’est
un mouvement de l’âme, dirait Descartes, qui résulte en elle d’une action du
corps, qu’elle subit et ressent ({\it Traité des passions}, I, \S 27-29). C’est un affect,
dirait Spinoza, dont je ne suis pas la cause adéquate ({\it Éthique}, III, déf. 3 ; voir
aussi, {\it ibid.}, la Définition générale des affects, à comparer avec le texte latin des
{\it Principes de la philosophie} de Descartes, IV, 190). De là cette passivité qui lui
ressemble, qui n’est pas inaction (ce que l’expérience infirmerait) mais action
imposée ou subie. La passion c’est ce qui, en moi, est plus fort que moi. Une
passion libre ou volontaire, tout passionné le pressent, n’en serait plus une. On
ne décide pas d’aimer à la folie, ni de n’aimer plus, ni d’être avare ou ambitieux....
C’est en quoi la passion est une circonstance atténuante, selon les
juges, et ridicule, selon les philosophes. Un crime passionnel ne mérite ni sévérité
ni respect.

On dit souvent que les classiques blâmaient les passions, que les romantiques,
au contraire, allaient exalter.... C’est moins simple que cela. Descartes
jugeait au contraire qu’elles sont « toutes bonnes de leur nature, et que nous
n'avons rien à éviter que leur mauvais usage ou leurs excès », au point que
« c’est d’elles seules que dépend tout le bien et le mal de cette vie » : les hommes
qu’elles peuvent le plus émouvoir sont capables d’y goûter le plus de douceur
({\it Traité des passions}, III, \S 211 et 212, qu'on nuancera pourtant par les \S 147 et
148). Encore faut-il les contrôler, autant que faire se peut ou se doit, les maîtriser,
quand il le faut, les utiliser, quand c’est possible, et c’est à quoi se reconnaît
l’homme d’action.

On cite souvent le mot de Hegel, dans les {\it Leçons sur la philosophie de l'histoire},
selon lequel «rien de grand ne s’est accompli dans le monde sans
passion ». C’est en effet vraisemblable. Mais rien non plus sans action, et c’est
d’ailleurs ce que Hegel, dans les lignes qui suivent, s’empresse de préciser :
« Passion n’est d’ailleurs pas le mot tout à fait exact pour ce que je veux désigner
ici, qui est l’activité de l’homme dérivant d’intérêts particuliers, de fins
spéciales ou d’intentions égoïstes, en tant que dans ces fins il met toute
l'énergie de son vouloir et de son caractère, en leur sacrifiant autre chose qui
pourrait aussi être une fin ou plutôt en leur sacrifiant tout le reste » (Introduction,
IL, b). Il y a de la passivité dans la passion, et tel est le sens classique du
%— 429 —
%{\footnotesize XIX$^\text{e}$} siècle — {\it }
mot. Mais une passion qui reste passive n’en est plus tout à fait une, au sens
moderne : ce n’est que veulerie ou fascination.

On voit que l’on a tort de réduire la passion à l’état amoureux, qui n’est
qu’une de ses formes. Alain, un jour où il faisait cours sur la passion, rappela à
ses étudiants qu’on distinguait traditionnellement trois passions principales :
l'amour, l'ambition, l’avarice. Puis il ajouta simplement : « vingt ans, quarante,
soixante, » Ce n’était qu’une boutade, mais qui dit quelque chose d’important :
que chaque passion a ses âges, ou plutôt que chaque âge a ses passions, qui
l’emportent davantage que d’autres. Être avare à vingt ans, c’est aussi rare que
d’être amoureux à soixante, et plus grave. Toujours est-il que la passion est
plurielle : toute passion n’est pas amoureuse. Mais toute passion, à ce que je
crois, est aimante. Qu'est-ce que l’ambition, sinon une certaine façon — passionnée,
passionnelle — d’aimer le pouvoir que l’on n’a pas encore ? Qu'est-ce
que l’avarice, sinon l’amour de l’argent qu’on a déjà ? La passion, en ce sens
général, c’est la polarisation du désir sur un seul objet (Tristan), ou sur un seul
type d’objet (Don Juan), que l’on n’a pas ou que l’on craint de perdre. C’est le
triomphe d’Éros, ou plutôt son exacerbation. Le passionné reste prisonnier du
manque, mais sous deux formes différentes : l'amour de ce qu’il n’a pas encore
(Pambitieux, le cupide, le don juan), la peur de perdre ce qu’il a déjà (le puissant
qui s'accroche à son pouvoir, l’avare, le jaloux). Les passionnés, sous leurs
grands airs, sont de petits enfants, qui n’ont pas encore accepté le sevrage : ils
cherchent un sein, ou bien ils ont peur qu’on le leur retire. C’est dire qu’ils
n'aiment qu’eux-mêmes (ils ne savent que prendre ou garder), et cela indique
assez le chemin. Sortir de la passion, c’est se libérer du petit enfant qui pleure
en chacun. C’est apprendre à donner, à agir — à grandir. On n’en a jamais fini.
Raison de plus pour s’y mettre sans tarder.

PATHOLOGIQUE {\it Pathos}, en grec, c’est la passion, le trouble, la douleur, la
maladie, bref tout ce qu’on subit ou endure. C’est en ce
sens que Kant dira {\it pathologique} tout ce qui n’est pas libre ou autonome, et spécialement
tout ce qui est déterminé par la sensibilité. Le sens moderne est beaucoup
plus étroit : est pathologique tout ce qui relève d’une maladie, et cela seul.
Le contraire du normal ? Pas tout à fait, puisqu'il est normal d’avoir des maladies,
et puisque l’état pathologique, comme disait Canguilhem, continue
d'exprimer un rapport à la « normativité biologique », qu’il modifie sans
labolir ({\it Le normal et le pathologique}, Conclusion). Disons que le pathologique
est l’exception qui confirme, le plus souvent douloureusement, la règle de la
santé, qui est d’être fragile et provisoire.

%— 430 —
%{\footnotesize XIX$^\text{e}$} siècle — {\it }
PATIENCE Cest la vertu de l’attente, ou l’attente comme vertu. La chose
paraît mystérieuse, puisque l'attente, portant sur l'avenir, semble
nous vouer à l'impuissance et au manque. Comment serait-ce une vertu ? Mais
c'est que l'attente, même dirigée vers l'avenir, est présente ; la patience l’est donc
aussi : C’est faire ce qui dépend de nous pour attendre au mieux ce qui n'en
dépend pas. C’est une disponibilité au présent, et à la lenteur du présent, bien
plus qu’à l'avenir. Le patient, selon la formule bien connue, laisse du temps au
temps : il habite tranquillement le présent, quand l’impatient voudrait être déjà
demain ou plus tard. C’est pourquoi « {\it patience est tout} », comme disait Rilke :
parce que rien d’important ne naît qui ne prenne du temps, parce qu'il faut
croître lentement, « comme l'arbre qui ne presse pas sa sève, qui résiste, confiant,
aux grands vents du printemps, sans craindre que l’été puisse ne pas venir. L'été
vient. Mais il ne vient que pour ceux qui savent attendre, aussi tranquilles et
ouverts que s’ils avaient l'éternité devant eux... » Ils l'ont en effet, et c'est ce
qu’on appelle le présent. La patience est l’art de l’accueillir à son rythme.

PATRIE Le pays dont on est originaire, où l’on est né, où l’on vit, du moins
pour la plupart des gens, ou dont on se sait, plus que d’aucun
autre, débiteur. Ce n’est pas toujours le même, ce qui explique qu’on puisse
avoir plusieurs patries, ou aucune. Disons que notre patrie, en règle générale,
c'est notre pays d’origine ou d’adoption, celui qui nous a accueilli, à la naissance
ou plus tard, le pays de nos pères ou de nos maîtres, enfin celui que nous
reconnaissons nôtre, non parce qu’il nous appartiendrait mais parce que nous
lui appartenons, au moins pour une part, au moins par le cœur et la fidélité.
C’est l'endroit d’où l’on vient ou que l’on a choisi, celui où l’on se sent chez soi,
enfin que l’on aime plus intimement que les autres pays, quand bien même ils
seraient, et ils le sont souvent, plus intéressants ou plus admirables. Notion non
pas objective, comme est davantage la nation, mais subjective et affective. J'ai
cru longtemps n’en pas avoir : la France m'était à peu près indifférente et je
professais que les intellectuels n’avaient pas davantage de patrie que les prolétaires...
J'ai changé: la France m’est de plus en plus chère, et surtout j'ai
découvert il y a bien des années — en Castille, en Toscane, à Amsterdam, à
Venise, à Prague... —, que j'avais évidemment une patrie, et qu’elle s'appelait
l'Europe. L'idée ne me viendrait pas de m’en vanter. Mais je n'aime pas trop
non plus qu’on me le reproche.

PATRIOTISME L'amour de la patrie, mais sans aveuglement ni xénophobie.
Se distingue par là du nationalisme (voir ce mot), ou sert à
%— 431 —
%{\footnotesize XIX$^\text{e}$} siècle — {\it }
le masquer. Le nationalisme, en règle générale, c’est le patriotisme des autres ;
et le patriotisme, bien souvent, un nationalisme à la première personne. Le
propre de l’aveuglement est de ne pas se voir soi. Aussi le patriotisme ne vaut-il
que soumis à la raison, qui est universelle, et à la justice, qui tend à l’être. Tel
est le sens, aujourd’hui, des droits de l’homme et de nos tribunaux internationaux.

PÉCHÉ C’est le nom religieux de la faute : une offense faite à Dieu, parce
qu’on a violé tel ou tel de ses commandements. Si Dieu n’existe pas,
il n’y a donc plus de péchés à proprement parler. Restent les fautes, qui sont
innombrables, et que rien n’interdit d’appeler des péchés, quoiqu’en un sens
laïcisé : c’est offenser l'humanité en soi ou en autrui.

PÉCHÉ ORIGINEL Ce serait une faute, commise par Adam et Êve, qui nous
vouerait à la culpabilité. L’idée, à peu près inacceptable
pour les Modernes, est fortement exprimée par Pascal : « Il faut que nous naissions
coupables, ou Dieu serait injuste » ({\it Pensées}, 205-489). Il y a pourtant une
autre possibilité : que Dieu n'existe pas.

PÉCHÉS CAPITAUX Les péchés capitaux font partie de notre tradition morale
et spirituelle. Chacun sait qu’il y en a sept. Mais la
plupart d’entre nous auraient bien du mal à en citer la liste complète. La
voici, telle que l’a fixée le pape Grégoire le Grand, à la fin du vr siècle, et telle
que nos catéchismes n’ont cessé, depuis, de la rappeler : {\it l'orgueil, l'avarice, la
luxure, l'envie, la gourmandise, la colère, la paresse}. Cette liste a mal vieilli : il y a
belle lurette que nous n’y reconnaissons plus nos fautes les plus graves, ni nos
dégoûts les plus résolus ! Comme me le disait plaisamment un ami, « il y a dans
ces péchés capitaux un côté doigts dans le pot de confiture, qui les rend comme
enfantins et presque ridicules ». Oui : nous avons désormais d’autres diables à
fouetter.

Qu'est-ce qu’un péché capital ? Pas forcément un péché plus grave que les
autres, mais un péché d’où les autres dérivent. C’est un péché qui vient en tête
de liste ({\it capital} vient du latin {\it caput}, la tête), un péché principiel, si l’on veut,
comme une des sources du mal. C’est où la notion de péché capital, ou de faute
capitale, pourrait retrouver son sens et son utilité, qui serait de nous aider à y
voir plus clair. Mais il faudrait en actualiser résolument la liste. Essayons.

%— 432 —
%{\footnotesize XIX$^\text{e}$} siècle — {\it }
Le premier est tout trouvé. Pourquoi faisons-nous du mal? Par pure
méchanceté ? Je n’y crois pas trop. Le plus souvent nous ne faisons du mal que
pour un bien. C’est un des points, il n’y en a pas tant, où je me sens d’accord
avec Kant : les hommes ne sont pas {\it méchants} (ils ne font pas le mal pour le
mal), ils sont {\it mauvais} (ils font du mal aux autres, pour leur bien à eux). C’est
en quoi l’égoïsme est le fondement de tout mal, comme disait encore Kant, et
le premier, selon moi, des péchés capitaux. C’est l’injustice à la première personne,
Car « le moi est injuste, expliquait Pascal, en ce qu’il se fait centre de
tout : chaque moi est l’ennemi et voudrait être le tyran de tous les autres ». On
ne fait du mal que pour son propre bien. On n’est mauvais que parce qu’on est
égoïste.

« Et le sadique ?, me demandent parfois mes étudiants. Est-ce qu’il ne fait
pas le mal pour le mal ? » Non pas : il fait du mal aux autres, pour son plaisir à
lui ; or son plaisir, pour lui, c’est un bien... Il n’en reste pas moins que la
cruauté existe, et qu’elle est sans doute la faute la plus grave, qui pourra à son
tour en entraîner plusieurs autres. C’est pourquoi il est juste de la considérer
comme un péché capital. Comment la définir ? Comme le goût ou la volonté
de faire souffrir : c’est pécher contre la compassion, contre la douceur, contre
l'humanité, au sens où l'humanité est une vertu. C’est le péché du tortionnaire,
mais aussi du petit chef pervers, du sadique ou du salaud, qui prend plaisir à
martyriser ses victimes.

Troisième péché capital : la lâcheté. Parce que aucune vertu n’est possible
sans courage, ni aucun bien. Parce que la lâcheté est une forme d’égoïsme, face
au danger. Enfin parce que la cruauté reste l'exception : la plupart des mauvaises
actions, même parmi les plus abominables, s’expliquent par la peur de
souffrir davantage que par le désir de faire souffrir autrui. Combien de gardiens,
à Auschwitz, auraient préféré rester tranquillement chez eux, plutôt que
faire ce travail atroce ? Mais ils n’avaient pas le courage de déserter, ni de désobéir,
ni de se révolter... Aussi firent-ils le mal lâchement, consciencieusement,
efficacement. Cela ne les excuse pas. Aucun péché n’est une excuse. Mais cela
explique qu'ils aient été si nombreux. Les vrais salauds sont rares. La plupart ne
sont que des lâches et des égoïstes, qui n’ont pas su résister, dans telle ou telle
situation particulière, à la pente de l’espèce ou de l’époque. Banalité du mal,
disait Hannah Arendt. La cruauté est l'exception ; l’égoïsme et la lâcheté, la
règle.

Encore faut-il pouvoir se supporter, être capable de se regarder, comme on
dit, dans une glace. À un certain degré d’ignominie ou simplement de médiocrité,
cela devient difficile sans se mentir à soi-même. C’est ce qui fait de la
mauvaise foi un péché capital : parce qu’elle rend possibles, en les masquant ou
en leur inventant de fausses justifications, la plupart de nos filouteries. Par
%— 433 —
%{\footnotesize XIX$^\text{e}$} siècle — {\it }
exemple Eichmann, zélé fonctionnaire de la Shoah, expliquant à ses juges,
après la guerre, qu’il n’a fait qu’obéir aux ordres. Ou le violeur, expliquant qu’il
n’a fait qu’obéir à ses pulsions. Ou la crapule ordinaire, expliquant que ce n’est
pas sa faute mais celle de son enfance, de son inconscient, de sa névrose. Bien
commode. Trop commode. Être de mauvaise foi, montrait Sartre, c’est faire
comme si on n'était pas libre, comme si on n’était pas responsable, alors qu’on
l’est, au moins de ses actes. C’est aussi, en un sens plus banal, mentir à autrui.
Mais le principe, bien souvent, est le même : on ment pour cacher sa faute, ou
pour la justifier, ou pour s’attribuer une valeur que l’on n’a pas. Celui qui
renoncerait à mentir — à soi et aux autres —, celui qui aurait cessé de faire semblant,
il n’aurait guère le choix qu’entre la vertu et la honte. Choix douloureux,
choix exigeant, dont la mauvaise foi vise à nous dispenser : c’est s’autoriser le
mal en s’autorisant à le dissimuler.

Je n’ai encore repris aucun des sept péchés capitaux de la tradition. Celui
que je voudrais à présent aborder, sans faire partie de la liste canonique, en est
peut-être le moins éloigné : ce que j'appelle la {\it suffisance} n’est pas très loin de ce
que les pères de l’Église appelaient l’orgueil. Mais c’est un défaut plus général,
plus profond, sans doute aussi moins tonique. Faire preuve de suffisance, ce
n’est pas seulement être orgueilleux ; c’est aussi être fat, présomptueux, vaniteux,
plein de sérieux et d’autosatisfaction, plein de soi ou de la haute idée que
l’on s’en fait... C’est le péché de l’imbécile prétentieux, et je ne connais guère
d'espèce, même chez les gens intelligents, plus désagréable. Mais c’est aussi le
péché qui est à l’origine, bien souvent, de l’abus de pouvoir, de l'exploitation
d'autrui, de la bonne conscience haineuse ou mébprisante, sans parler du
racisme et du sexisme. Le Blanc qui croit appartenir à une race supérieure ou le
macho fier de ce qu’il prend pour sa virilité ne sont pas seulement ridicules : ils
sont dangereux, et c’est pourquoi il convient de les combattre. Un misanthrope
est moins à craindre ; c’est qu’il ne prétend pas faire exception et se sait, lui
aussi, insuffisant.

S'agissant des idées, la suffisance devient fanatisme. C’est un dogmatisme
haineux ou violent, trop sûr de sa vérité pour tolérer celle des autres. C’est plus
que de l'intolérance : c’est vouloir interdire ou supprimer par la force ce qu’on
désapprouve ou qui nous donne tort. Disons que c’est une intolérance exacerbée
et virtuellement criminelle. On en connaît les effets, en tous temps et en
tous pays: massacres, guerres de religions, inquisition, terrorisme, totalitarisme...
On ne fait le mal que pour un bien, disais-je, et l’on s’autorisera
d’autant plus de mal que le bien paraît plus grand. La foi a fait plus de victimes
que la cupidité. L’enthousiasme, plus que l'intérêt. C’est qu’on massacre plus
volontiers pour Dieu que pour soi, pour le bonheur de l'humanité plutôt que
pour le sien propre. « Tuez-les tous, Dieu ou l'Histoire reconnaîtra les siens. »

%— 434 —
%{\footnotesize XIX$^\text{e}$} siècle — {\it }
Fanatisme, crime de masse. C’est le péché qui emplit les camps et allume les
bûchers.

Le dernier péché capital, puisque j'ai choisi de m’en tenir, moi aussi, à une
liste de sept, n’est pas sans évoquer l’un de ceux que retient la tradition : ce que
j'appelle la veulerie est comme une paresse généralisée, de même que la paresse
est une forme de veulerie face au travail.

Qu'est-ce que la veulerie ? Un mélange de mollesse et de complaisance, de
faiblesse et de narcissisme : c’est l'incapacité à s’imposer quoi que ce soit, à faire
un effort un peu durable, à se contraindre, à se dépasser, à se surmonter. Être
veule, ce n’est pas seulement manquer d'énergie ; c’est manquer de volonté et
d’exigence. En quoi est-ce un péché capital ? En ceci, que la veulerie en entraîne
plusieurs autres : la vulgarité, qui est veulerie dans les manières, l’irresponsabilité,
qui est veulerie face à autrui ou à ses devoirs, la négligence, qui est veulerie dans
la conduite ou le métier, la servilité, qui est veulerie face aux puissants, la démagogie,
qui est veulerie face au peuple ou à la foule... «Il faut suivre sa pente,
disait Gide, mais en la remontant. » Le veule est celui qui préfère la descendre.

Sept péchés capitaux, donc : l’égoïsme, la cruauté, la lâcheté, la mauvaise
foi, la suffisance, le fanatisme, la veulerie. Non parce qu’ils seraient forcément
les plus graves, répétons-le, mais parce qu’ils gouvernent ou expliquent tous les
autres. Ce sont les sources du mal, disais-je, et sans doute aussi celles du bien,
au moins pour une part, au moins par l’horreur ou le dégoût qu’ils nous inspirent,
par le désir d’y échapper, enfin par l'effort qu’il faut faire, presque toujours,
pour les surmonter. Pauvres immoralistes, qui ont cru qu’il suffisait de
ne plus croire en Dieu pour être délivré du mal!

PENCHANT Synonyme à peu près de tendance, en plus singulier, ou d’indclination,
en moins plaisant. C’est une orientation durable du
désir, qui doit moins à l'espèce qu’à l'individu, mais plus sans doute à sa nature
qu’à sa culture ou à ses choix. Disons que c’est la pente naturelle d’un être
humain, sur laquelle il peut, ou non, se laisser glisser...

PENSÉE On trouve une définition en extension, bien sûr incomplète, chez
Descartes : « Qu'est-ce donc que je suis ? Une chose qui pense.
Qu'est-ce qu’une chose qui pense ? C’est une chose qui doute, qui conçoit, qui
affirme, qui nie, qui veut, qui ne veut pas, qui imagine aussi, et qui sent »
({\it Méditations}, II). C’est définir la pensée sinon par la conscience, du moins à
partir d’elle, comme une expérience ou une dimension du sujet (« la pensée est
un attribut qui m’appartient », {\it ibid.}), et sans doute on ne peut la définir autrement,
%— 435 —
%{\footnotesize XIX$^\text{e}$} siècle — {\it }
puisque toute définition la suppose et ne s'adresse qu’à un sujet. Celui
qui ne penserait pas, comment lui faire comprendre ce que c’est que penser ?
« Penser, dira Kant, c’est unifier des représentations dans une conscience »
({\it Prolégomènes}, II). C’est en quoi aucun ordinateur ne pense: le mien, par
exemple, pourtant doté d’un logiciel de traitement de texte particulièrement
performant, est d’une bêtise crasse, qui ne cesse de me surprendre. Mais ce n’est
pas qu’il pense mal ; c’est qu’il ne pense pas.

Faut-il dire alors que toute conscience est pensée ? En un sens large, oui :
tel est le sens de Descartes. En un sens plus restreint, on ne parlera de pensée
que pour la dimension intellectuelle ou rationnelle de la conscience, disons
que pour des représentations logiquement liées, fût-ce de façon imparfaite, et
soumises ensemble à l’idée d’une vérité au moins possible. Penser, étymologiquement,
c'est peser : cela suppose l’unité d’une balance ou d’un rapport.
La pensée, c’est ce qui pèse ou soupèse les arguments, les expériences, les
informations, et jusqu’à la pesée elle-même... J'en donnerais volontiers,
complétant Kant par Spinoza et Spinoza par Montaigne, la définition
suivante : {\it Penser, c'est unifier des représentations dans une conscience, sous la
norme de l'idée vraie donnée ou possible}. La pensée est donc bien ce « dialogue
intérieur et silencieux de l’âme avec elle-même » qu’évoquait Platon, mais en
tant qu’elle cherche le vrai (puisqu'il faut « aller au vrai avec toute son âme »)
et, d'avance, s’y soumet.

PERCEPTION Toute expérience, en tant qu’elle est consciente ; toute conscience,
en tant qu'elle est empirique. Se distingue de la sen-
sation comme le plus du moins, comme l’ensemble de ses éléments (une perception
suppose plusieurs sensations liées et organisées), et c’est en quoi la
sensation, à vouloir la penser isolément, n’est qu’une abstraction. Vous voyez
des taches de couleurs ; vous percevez un paysage. « L'esprit met tout en
ordre », comme disait Anaxagore, ou du moins il s’y essaie. Il ne se contente
pas de sentir : il unifie ses sensations dans une conscience, dans une expérience,
dans une forme, non après coup mais dès le départ, et c’est la perception
même. Il transforme des taches lumineuses en distances ou en spectacle,
des bruits en informations, des odeurs en promesses... Percevoir, c’est se
représenter ce qui se présente : la perception est notre ouverture au monde et
à tout.

PERFECTIBILITÉ Ce n’est pas le pouvoir de devenir parfait, mais celui de
se perfectionner. Seul l’imparfait est donc perfectible,
%— 436 —
%{\footnotesize XIX$^\text{e}$} siècle — {\it }
mais il ne l’est qu’à la condition de pouvoir changer, et se changer. Rousseau y
voyait le propre de l'humanité : outre la liberté, explique-t-il, « il y a une autre
qualité très spécifique qui distingue l’homme de l’animal, et sur laquelle il ne
peut y avoir de contestation ; c’est la faculté de se perfectionner, faculté qui, à
l’aide des circonstances, développe successivement toutes les autres et réside
parmi nous tant dans l’espèce que dans l'individu ; au lieu qu’un animal est au
bout de quelques mois ce qu’il sera toute sa vie, et son espèce au bout de mille
ans ce qu’elle était la première année de ces mille ans » ({\it Discours sur l'origine de
l'inégalité}, I ; même idée chez Pascal, mais sans le mot de perfectibilité, dans sa
{\it Préface au traité du vide}). La perfectibilité serait donc une évolution, mais historique
et culturelle, plutôt que naturelle. C’est ce qui rend la notion utile en
même temps que relative : si l’histoire et la culture font partie de la nature,
comme je le crois, la perfectibilité n’est qu’une forme parmi d’autres de l’universel
devenir (non une exception au darwinisme, mais une de ses occurrences).
À la gloire d'Héraclite et des enseignants.

PERFECTION Le fait d’être parfait (voir ce mot). On dit souvent que la perfection
n’est pas de ce monde ; c’est qu'on en imagine un
autre, idéal, auquel on compare celui-ci. La notion de perfection n’a de sens
que relatif (Spinoza, {\it Éthique}, IV, Préface ; voir aussi I, Appendice). Une perfection
absolue est un non-sens, ou n’est que l’absolu lui-même — non parce qu’il
serait sans défaut, de notre point de vue, mais parce qu’il est sans manque (il
est tout ce qu’il est et qu’il peut être). C’est l’être de Parménide et des mystiques.

PERFORMATIF  « {\it Je déclare la réunion ouverte} » ; si je suis président de séance,
elle l’est par là même. Tel est le discours performatif : celui
qui fait être ce qu’il dit, parce que dire et faire, en l’occurrence, sont un. Quand
je dis « {\it Je le jure} », je jure en effet : c'est une expression performative. Si je dis
« {\it Il le jure} », en revanche, je ne jure rien : l'expression n’est pas performative.
Un énoncé performatif se distingue en cela d’un énoncé descriptif ou normatif.
Il est moins soumis à l’exigence de vérité, comme le premier, ou de justesse
comme le second, qu’à celles de possibilité, de cohérence, de réussite..., qui
dépendent du contexte et des individus. Si vous dites « Je déclare la réunion
ouverte » seul dans votre chambre, ou même dans un congrès mais sans avoir
légitimité à le faire, il est probable qu'aucune réunion ne sera ouverte par là. Le
discours performatif est un acte : il a moins à être vrai ou faux qu’à être efficace
ou pas.

%— 437 
%{\footnotesize XIX$^\text{e}$} siècle — {\it }
PERFORMATIVE (CONTRADICTION —) Une contradiction qui oppose
non pas deux énoncés l’un à
l’autre, mais un énoncé (comme proposition) à lui-même (comme acte). On en
donne souvent l'exemple suivant : « {\it J'étais dans un bateau qui a fait naufrage ;
il n'y a pas eu de survivant.} » La phrase n’est pas contradictoire en elle-même (il
n’est pas impossible que je meure dans un naufrage), mais elle l’est avec le fait
que je puisse la prononcer à la première personne. Mon ami Luc Ferry m’a souvent
reproché de tomber dans une contradiction de ce genre, comme tout
matérialiste, dès lors que je considère comme illusoire une subjectivité libre
que, par ailleurs, je serais obligé de supposer pour pouvoir prétendre à quelque
vérité que ce soit (par exemple à la vérité du matérialisme) : il y aurait une
contradiction non entre telle et telle de mes thèses, mais entre ce que je {\it fais}
(mon activité de sujet pensant) et ce que je {\it dis} (que le sujet pensant n’est
qu'une illusion ou que le résultat passif de déterminismes extérieurs). Je n’en
crois bien sûr rien. D'abord parce que l’idée de vérité n’a pas besoin de celle de
liberté, au sens du libre arbitre, voire l’exclut ({\it la vérité est exactement ce qu'on
ne choisit pas}). Ensuite parce que le sujet, de mon point de vue, même illusoire
(en tant qu’il se croit absolument libre ou transparent à lui-même), est assurément
actif : dire « je suis mon corps », ou « Je suis mon histoire », ce n’est pas
dire « Je suis passif » (parce que je serais déterminé par mon corps, par mon histoire,
par mon inconscient, etc.) ; c’est dire exactement l’inverse : si je suis mon
corps, il est exclu que je sois déterminé passivement par lui ; je suis actif, bien
plutôt, quand mon corps est actif, quand mon histoire est action, et c’est pourquoi
je le suis toujours partiellement et jamais totalement. Je n’ai pas convaincu
Luc Ferry davantage qu’il ne m’a convaincu, au moins sur ce point, mais cela
nous a aidés, ce n’est pas rien, à mieux nous comprendre (voir {\it La sagesse des
Modernes}, chap. 1 et conclusion).

PERSÉCUTION Une oppression violente et ciblée. On peut opprimer tout
un peuple ; on ne peut guère persécuter qu’une minorité.
Par exemple les Protestants, dans la France catholique. Ou les Juifs, dans
l’Europe chrétienne. C’est le bras armé du fanatisme ou du racisme.

PERSÉVÉRANCE La patience et la continuité dans l'effort. C’est une forme
de courage, non contre le danger ou la peur, mais contre
la fatigue et le renoncement. Il y faut ordinairement une grande passion, ou un
grand ennui.


%— 438 —
%{\footnotesize XIX$^\text{e}$} siècle — {\it }
On pense à la formule fameuse de Guillaume d'Orange : «Il n’est pas
besoin d’espérer pour entreprendre, ni de réussir pour persévérer. » En effet : il
n’est besoin que de courage et de volonté. Mais il en faut aussi pour changer
d'orientation, quand cela paraît nécessaire ou souhaitable. C’est ce qui distingue
la persévérance de l’obstination.

PERSONNALITÉ Ce qui fait qu’une personne est différente d’une autre, et
de toutes les autres, non seulement numériquement mais
qualitativement. C’est pourquoi une personne peut manquer de personnalité :
quand elle n’est différente des autres que numériquement ou physiquement, et
ressemble pour le reste (les sentiments, les pensées, les comportements.) à
n'importe qui, spécialement, par mimétisme ou mollesse, à ceux qui l’entourent.

PERSONNE  Unindividu, mais considéré comme sujet pensant, à la fois unique
(différent de tous les autres) et un (à travers ses modifications).
C’est le sujet de l’action, qui peut donc lui être imputée : la notion touche à la
morale, spécialement chez Kant, davantage qu’à la métaphysique ou à la
théorie de la connaissance.

PESSIMISME  - Sais-tu quelle différence il y a entre un optimiste et un pessimiste ?

—?...

— Le pessimiste est un optimiste bien informé.

Cette devinette, qui nous vient d'Europe centrale, est elle-même pessimiste.
C’est pourquoi peut-être elle nous amuse : parce que nous y voyons une espèce
de cercle, sans que cela suffise pourtant à la réfuter.

Qu'est-ce que le pessimisme ? C’est mettre les choses au pire ({\it pessimus}), soit
parce qu’on juge qu’il y a plus de maux que de biens, soit parce qu'on pense
que les maux vont s’aggraver. Au sens philosophique, le pessimisme rentre
plutôt dans la première catégorie : c’est un pessimisme actuel plutôt que prospectif
(par quoi Schopenhauer est le grand penseur du pessimisme, comme
Leibniz de l’optimisme). Au sens courant, le pessimisme a plutôt à voir avec la
seconde catégorie, autrement dit avec l’avenir, qu’il imagine pire que le présent.
La vieillesse et la mort semblent lui donner raison, au moins pour l'individu,
comme le progrès et la religion, de façon différente, à l’optimisme. Il n’y avait
plus qu’à faire du progrès une religion pour que le pessimisme, c’est du moins
%— 439 —
%{\footnotesize XIX$^\text{e}$} siècle — {\it }
ce qu'on pouvait croire, fût définitivement vaincu. De là les utopies et les différents
messianismes qui n’ont cessé, depuis le xx: siècle, de nous offrir de
nouvelles raisons d’espérer.... Hélas ! cela ne nous a donné que de nouvelles raisons
de nous méfier. des optimistes.

PETITESSE  L’incapacité à concevoir rien de grand, donc aussi à le faire ou
à l’admirer. Le petit voit tout à son échelle — petit, mesquin,
médiocre. Il appelle cela : « n’être pas dupe ».

PÉTITION DE PRINCIPE Faute logique, qui consiste à poser au départ,
fût-ce sous une autre forme, ce qu’on prétend
démontrer. C’est comme un diallèle élémentaire, de même que le diallèle est
comme une pétition de principe indirecte.

PEUPLE L'ensemble des sujets d’un même souverain, ou des citoyens d’un
même État. Dans une République, c’est donc le souverain lui-même.

On dira que le peuple n’est qu’une abstraction — qu’il n’existe que des individus.
Sans doute. Mais le contrat social ou le suffrage universel réalisent cette
abstraction, donnant au peuple, comme Hobbes l'avait vu, l'unité, certes artificielle
mais effective, d’une personne. C’est ce qui distingue le {\it peuple} de la
{\it multitude} : « Le peuple est un certain corps et une certaine personne, à laquelle
on peut attribuer une seule volonté, et une action propre ; alors qu’il ne se peut
rien dire de semblable de la multitude » (Hobbes, {\it Le Citoyen}, XII, 8 ; voir aussi
VI, 1). Reste à savoir, c'était la question de Rousseau, ce qui fait qu’un peuple
est un peuple. Il faut répondre : le contrat social, autrement dit l’unité de la
volonté générale, quand elle règne. Un peuple n’est vraiment {\it un} — donc n’est
vraiment un peuple — que par la souveraineté qu’il se donne, qu’il exerce ou
qu’il défend. C’est dire qu’un peuple n’est vraiment lui-même que dans une
démocratie, et par elle. Les despotes ne règnent que sur une multitude.

PEUR  L’émotion qui naît en nous à la perception, ou même à l’imagination,
d’un danger. Se distingue de l’angoisse par l’aspect déterminé
de ce dernier. L’angoisse est comme une peur indéterminée ou sans objet ; la
peur, comme une angoisse déterminée, voire objectivement justifiée. Cela ne
%— 440 —
%{\footnotesize XIX$^\text{e}$} siècle — {\it }
dispense pas de l’affronter, ni de la surmonter quand on peut : tel est le courage,
toujours nécessaire, jamais suffisant.

PHÉNOMÈNE Ce qui apparaît. Se distingue pourtant de l'apparence, spécialement
chez Kant et ses successeurs, par son poids de
réalité : ce n’est pas une illusion ; c’est la réalité sensible (par opposition au
noumène, réalité intelligible), la réalité pour nous (par opposition à la chose en
soi), et la seule qui soit connaissable.

Dans la philosophie contemporaine, et spécialement chez les phénoménologues,
le mot ne s’oppose plus guère au noumène ou à la chose en soi. Le phénomène,
écrit Sartre, n'indique plus, « par-dessus son épaule, un être véritable
qui serait, lui, l’absolu ; ce qu’il est, il l’est absolument, car il se dévoile comme
il est » ; il est « le relatif-absolu », qui peut être « étudié et décrit en tant que tel,
car il est absolument indicatif de lui-même » ({\it L'Étre et le néant}, Introduction).
On n’est pas très loin de ce que Marcel Conche appellera l'{\it apparence pure}, et
que Clément Rosset, plus simplement, appelle {\it le réel}.

PHÉNOMÉNOLOGIE L'étude des phénomènes, autrement dit de ce qui
apparaît à la conscience — et, grands dieux, que
pourrait-on étudier d’autre ? La phénoménologie ne serait donc qu’un pléonasme
pour dire la pensée ? Pas tout à fait. Il s’agissait, en décrivant ce qui
apparaît à la conscience (les phénomènes), de découvrir, comme disait Sartre,
« quelque chose de solide, quelque chose enfin qui ne fût pas l'esprit ». Décrire
la conscience, donc, ou ce qui lui apparaît, {\it pour en sortir} (puisque « toute conscience
est conscience de quelque chose » : voir l’article « Intentionnalité »). Ils
appelaient cela : {\it retour aux choses mêmes}. Mais cela nous en a plus appris sur
nous-mêmes que sur le monde. La phénoménologie, écrit sans rire Merleau-Ponty,
est « un inventaire de la conscience comme milieu de l'univers ». C’est
donc à peu près l'inverse de la physique, sans être pour autant une métaphysique
ni une morale. Ce n’est qu’un commencement de la pensée, mais qui
s’épuiserait dans la répétition — à la fois inlassable et lassante — de son premier
pas. Il en reste quelques chefs-d’œuvre absolus, et aussi plusieurs des livres les
plus difficiles et les plus ennuyeux que je connaisse.

PHILOSOPHE Je ne sais plus si c’est Guitton ou Thibon qui raconte l’anecdote,
comme lui étant personnellement arrivée. La scène se

passe au début du {\footnotesize XX$^\text{e}$} siècle, dans une campagne un peu reculée. Un jeune professeur
% — 441 — 
%{\footnotesize XIX$^\text{e}$} siècle — {\it }
de philosophie, se promenant avec un ami, rencontre un paysan, que
son ami connaît, qu’il lui présente, et avec lequel notre philosophe échange
quelques mots.

— Qu'est-ce que vous faites dans la vie ?, lui demande le paysan.

— Je suis professeur de philosophie.

— C’est un métier ?

— Pourquoi non ? Ça vous étonne ?

— Un peu, oui!

— Pourquoi ça ?

— Un philosophe, c’est quelqu'un qui s’en fout. Je ne savais pas que cela
s’apprenait à l’école !

Ce paysan prenait « philosophe » au sens courant, où il signifie à peu près,
sinon quelqu'un qui s’en fout, du moins quelqu'un qui sait faire preuve de
sérénité, de tranquillité, de recul, de décontraction... Un sage ? Pas forcément.
Pas totalement. Mais quelqu’un qui tend à l'être, et tel est aussi, depuis les
Grecs, l’étymologie du mot ({\it philosophos} : celui qui aime la sagesse) et son sens
proprement philosophique. On me dit parfois que cela n’est vrai que des
Anciens. Ce serait déjà beaucoup. Mais c’est oublier Montaigne. Mais c’est
oublier Spinoza. Mais c’est oublier Kant (« La philosophie est la doctrine et
l'exercice de la sagesse, écrivait-il dans son {\it Opus postumum}, non simple
science ; la philosophie est pour l’homme {\it effort vers la sagesse}, qui est toujours
inaccompli »). Mais c’est oublier Schopenhauer, Nietzsche, Alain. Le philosophe,
pour tous ceux-là, ce n’est pas quelqu'un de plus savant ou de plus
érudit que les autres, ni forcément l’auteur d’un système ; c’est quelqu’un qui
vit mieux parce qu'il pense mieux, en tout cas qui essaye (« Bien juger pour
bien faire », disait Descartes : c’est la philosophie même), et c’est en quoi le
philosophe reste cet amant de la sagesse, ou cet apprenti en sagesse, que l’étymologie
désigne et dont la tradition, depuis vingt-cinq siècles, n’a cessé de préserver
le modèle ou l'exigence. Si vous n’aimez pas ça, n’en dégoûtez pas les
autres.

Qu'est-ce qu’un philosophe ? C’est quelqu'un qui pratique la philosophie,
autrement dit qui se sert de la raison pour essayer de penser le monde et sa
propre vie, afin de se rapprocher de la sagesse ou du bonheur. Cela s’apprend-il
à l'école ? Cela doit s’apprendre, puisque nul ne naît philosophe, et puisque
la philosophie est d’abord un travail. Tant mieux si cela commence à l’école.
L'important est que cela commence, et ne s'arrête pas. Il n’est jamais ni trop tôt
ni trop tard pour philosopher, disait à peu près Épicure, puisqu'il n’est jamais
ni trop tôt ni trop tard pour être heureux. Disons qu’il n’est trop tard que
lorsqu'on ne peut plus {\it penser} du tout. Cela peut venir. Raison de plus pour
philosopher sans attendre.

%— 442 —
%{\footnotesize XIX$^\text{e}$} siècle — {\it }
PHILOSOPHIE Une pratique théorique (mais non scientifique), qui a le
tout pour objet, la raison pour moyen, et la sagesse pour
but. Il s’agit de penser mieux, pour vivre mieux.

La philosophie n’est pas une science, ni même une connaissance (ce n’est
pas un savoir de plus, c’est une réflexion sur les savoirs disponibles), et c’est
pourquoi, comme disait Kant, on ne peut apprendre {\it la philosophie} : on ne
peut apprendre qu’à philosopher. Le même, dans un texte fameux, ramenait
le domaine de la philosophie à quatre questions : {\it Que puis-je savoir ? Que
dois-je faire ? Que m'est-il permis d'espérer ? Qu'est-ce que l'homme ?} Les trois
premières « se rapportent à la dernière », remarquait-il ({\it Logique}, Introd.,
III). Mais elles débouchent toutes les quatre, ajouterai-je, sur une cinquième,
qui est donc la question principale de la philosophie, au point
qu’elle pourrait presque suffire à la définir: {\it Comment vivre ?} Dès qu'on
essaie de répondre intelligemment à cette question, on fait de la philosophie,
peu ou prou, bien ou mal. Et comme on ne peut éviter de se la poser, il faut
en conclure qu’on n'échappe à la philosophie que par la bêtise ou l’obscurantisme.

Il m'est arrivé de définir la philosophie, ou l’acte de philosopher, encore
plus simplement : {\it Philosopher, c'est penser sa vie et vivre sa pensée}. Non, bien sûr,
qu’il faille se contenter de l’introspection ou de l’égocentrisme. Penser sa vie,
c’est la penser où elle est : ici et maintenant, certes, mais aussi dans la société,
dans l’histoire, dans le monde, dont elle n’est pas le centre mais l'effet. Et vivre
sa pensée c’est agir, autant qu’on peut, autant qu’on doit, puisqu'on ne pourrait
autrement que subir ou rêver. Ainsi la philosophie est une activité dans la
pensée, qui débouche sur une vie plus active, plus heureuse, plus lucide, plus
libre — plus sage.

PHOBIE Une peur pathologique, déclenchée par la présence d’un objet ou
d’une situation sans autre danger que cette peur même. Relève
moins du courage que de la médecine.
En un sens plus large, toute aversion incontrôlable : c’est moins une peur
qu’un dégoût ou une répulsion. Relève moins de la médecine que de l’évitement,
du courage, ou de l’habitude.

PHONÈME C’est une unité phonique minimale : l'élément de la seconde
articulation (voir ce mot, ainsi que l’article « Langue »).

%— 443 —
%{\footnotesize XIX$^\text{e}$} siècle — {\it }
PHRONÈSIS Le nom grec de la prudence (voir ce mot) ou de la sagesse pratique.
Se distingue par là de la {\it sophia}, sagesse théorique ou
contemplative.

PHYSIQUE Tout ce qui relève de la nature ({\it phusis} en grec), et spécialement
la science qui l’étudie ({\it ta phusika}).

Si la nature est tout, comme je le crois, la physique a vocation à absorber
toutes les sciences. Toutefois ce n’est qu’une vocation, sans doute impossible à
réaliser absolument. Par exemple la matière obéit aux mêmes lois, selon toute
vraisemblance, dans les corps vivants et dans les autres. La biologie, pour comprendre
un organisme quelconque, n’en reste pas moins nécessaire : parce que
la vie a sa rationalité propre, certes incluse dans celle de la matière inorganique
(les atomes sont les mêmes, et soumis aux mêmes lois), mais qui ne saurait pour
autant, si ce n’est par abstraction, être considérée comme nulle et non avenue.
Quelque chose de neuf émerge, qui fait qu’un corps vivant ne se réduit pas à la
simple somme de ses parties. Même chose pour la pensée. Quand je suis triste,
cela correspond assurément à des phénomènes neurobiologiques dans mon cerveau,
donc, en dernière analyse, à des phénomènes physiques dans le monde.
Toutefois ma tristesse s’explique plus simplement par la psychologie (j'ai appris
une mauvaise nouvelle, j'ai perdu un être cher, je fais une dépression...) que
par la physique. Et qui voudrait expliquer les résultats d’une élection par les lois
de la mécanique quantique ? Tout est physique, et c’est pourquoi la physique
est la science du tout. Mais avec des degrés différents de complexité, qui font
que la physique n’est pas tout, ni la science de tout.

PIÉTÉ Mélange d’amour et de respect, à l’égard d’un être qui nous dépasse.
Se dit le plus souvent vis-à-vis de Dieu, parfois vis-à-vis d’un parent
(piété filiale), d’un héros ou d’un maître.

On remarquera que l’étymologie est la même que pour {\it pitié} (les deux mots
furent un temps synonymes : cela s'entend encore dans notre {\it Mont-de-piété}),
que l’on aimerait définir comme un mélange d'amour et de respect, à l'égard
d’un être que l’on dépasse ou que l’on plaint. Le mot de {\it pietas}, en latin, n’a
guère ce sens : il se dit presque exclusivement de ce qu’on doit aux dieux, à ses
parents ou à la patrie. Mais le mot {\it pietà}, en italien ou en histoire de l’art, réunit
les deux idées, comme son équivalent français de « Vierge de pitié ». Voyez par
exemple la sublime {\it Pietà} de Michel-Ange. Quelle plus belle image du divin que
ce jeune homme mort, dans les bras de sa mère ?

% — 444 — 
%{\footnotesize XIX$^\text{e}$} siècle — {\it }
PITIÉ Une forme de compassion, mais qui serait plutôt un sentiment
qu’une vertu (la compassion est les deux), avec je ne sais quoi de
condescendant qui la rend désagréable. L’inverse par là de la piété : c'est une
compassion qui s'exerce, ou qui croit s'exercer, de haut en bas.

PLAISIR L'un des affects fondamentaux, comme tel à peu près indéfinissable.
Disons que c’est l’affect qui s'oppose à la douleur, celui qui
nous plaît, qui nous réjouit ou nous fait du bien : c’est la satisfaction agréable
d’un désir.

On notera que ce désir n’est pas forcément un manque (par exemple dans
le plaisir esthétique) et ne précède pas nécessairement sa satisfaction : une
odeur agréable, un beau paysage ou une bonne nouvelle peuvent me faire
plaisir quand bien même je ne les désirais, avant de les rencontrer, en rien.
Spinoza dirait qu’ils ne s’en accordent pas moins avec ma puissance d'exister
(avec mon {\it conatus}), disons avec ma puissance de jouir, d’agir et de me
réjouir, autrement dit avec mon désir, en effet, mais en tant que puissance
indéterminée. J'en suis d’accord, et c’est en quoi tout plaisir est relatif : ce
n’est pas parce qu’une chose est agréable que nous la désirons, c’est parce que
nous la désirons, ou parce qu’elle s’accorde à nos désirs, qu’elle est, pour
nous, agréable. Pourquoi alors ne pas définir le plaisir, purement et simplement,
par la satisfaction d’un désir ? Parce qu’on peut satisfaire un désir sans
pour autant éprouver de plaisir : les fumeurs savent bien que le plaisir n’est
pas toujours égal, ni même toujours présent, à chaque cigarette ou à chaque
bouffée. Et chacun sait qu’il ne suffit pas de désirer vivre, hélas, pour que la
vie soit agréable.

« Tout plaisir, du fait qu’il a une nature appropriée à la nôtre, est un bien »,
remarquait Épicure. Puis il ajoutait : « tout plaisir, cependant, ne doit pas être
choisi » ({\it Lettre à Ménécée}, 129). C’est que certains apportent plus de maux,
pour soi ou pour autrui, que de biens. C’est où l’hédonisme atteint sa limite.
Le plaisir est « le bien premier et conforme à notre nature », certes, « Le principe
de tout choix et de tout refus», enfin «le principe et la fin de la vie
bienheureuse » ({\it ibid.}). Mais pas tout plaisir, ni toujours les plaisirs les plus
forts. Il faut donc choisir : c’est à quoi servent la prudence, pour ce qui est de
soi, et la morale, pour ce qui est des autres. Non qu’on renonce pour cela au
plaisir, mais parce qu’il n’est pas possible « de vivre avec plaisir sans vivre avec
prudence, honnêteté et justice » ({\it ibid.}, 132). Le plaisir est le but, pas toujours
le chemin.

%— 445 —
%{\footnotesize XIX$^\text{e}$} siècle — {\it }
PLAISIR (PRINCIPE DE -) Il est formulé à peu près par Aristote : « On
choisit ce qui est agréable, on évite ce qui est
pénible » ({\it Éthique à Nicomaque}, X, 1).

Il est repris par Épicure : « C’est pour cela que nous faisons tout : afin de
ne pas souffrir et de n’être pas troublés. [...] C’est pourquoi nous disons que le
plaisir est le commencement et la fin de la vie bienheureuse : c’est en lui que
nous trouvons le principe de tout choix et de tout refus » ({\it Lettre à Ménécée},
128-129).

Il est chanté par Virgile, peut-être influencé par Lucrèce : « {\it Trahit sua
quemque voluptas} » (chacun est entraîné par son propre plaisir, {\it Bucoliques}, II,
65 ; comparer avec le {\it De rerum natura}, II, 172 et 258). Il sera repris par saint
Augustin et Pascal (c’est ce qu’on a appelé le panhédonisme de Port-Royal :
« L'homme est esclave de la délectation ; ce qui le délecte davantage l’attire
infailliblement : on fait toujours ce qui plaît le mieux, c’est-à-dire que l’on
veut toujours ce qui plaît », Pascal, {\it Écrits sur la grâce}, p. 332 a), mais aussi par
Montaigne (« Le plaisir est notre but..., en la vertu même, le dernier but de
notre visée c’est la volupté», I, 20) et la plupart des matérialistes du
XVII siècle... Mais c’est bien sûr Freud, dans la dernière période, qui lui
donne son nom et sa formulation canonique. « L'ensemble de notre activité
psychique a pour but de nous procurer du plaisir et de nous faire éviter le
déplaisir » (Introduction... 22 ; voir aussi « Au-delà du principe de plaisir,
1). {\it Le principe de plaisir} est l’un des deux grands principes qui régissent, selon
Freud, la totalité de notre vie psychique, ou plutôt c’est le seul : tout être
humain (peut-être même tout animal) tend à jouir le plus possible et à souffrir
le moins possible. Le {\it principe de réalité} s'y oppose moins qu’il ne le complète,
et ne le complète qu’en le modifiant. Il s’agit toujours de jouir le plus
possible, de souffrir le moins possible, mais en tenant compte pour ce faire
des contraintes du réel, ce qui suppose qu’on accepte de ne jouir parfois que
plus tard ou moins, voire de souffrir un certain temps, pour augmenter la
jouissance à venir ou éviter un désagrément plus grand. « Le principe de réalité
a également pour but le plaisir, écrit Freud, mais un plaisir qui, s’il est
différé et atténué, a l'avantage d’offrir la certitude que procurent le contact
avec la réalité et la conformité à ses exigences » ({\it ibid.}). Ce n’est guère qu’un
commentaire, mais psychanalytiquement informé, des paragraphes 129
et 130 de la {\it Lettre à Ménécée}.

PLATONICIEN De Platon, ou s’en réclamant. Pas besoin pour cela de rester
platonique.

%— 446 —
%{\footnotesize XIX$^\text{e}$} siècle — {\it }
PLATONIQUE Le mot est à peu près au précédent ce que {\it stoïque} est à
{\it stoïcien} : une banalisation, une popularisation, comme une
pensée d’abord méconnaissable, à force de s’immerger dans la foule, mais où
l’on discerne pourtant, dessous les grimaces ou les contresens, comme une
étrange ressemblance. {\it Platonique}, en l'occurrence, se dit surtout de l’amour : ce
serait un amour purement sentimental ou intellectuel, sans rien de sensuel, de
charnel, de sexuel. Ceux qui ont lu le {\it Banquet} et le {\it Phèdre} s'étonneront d’abord
de cette acception. Ceux qui les reliront s’étonneront moins.

PLATONISME Le système de Platon, qu’il n’est bien sûr pas question d’exposer
ici (est-ce seulement un système ?), et toute pensée
qui en partage l'inspiration principale. À savoir ? Ceci : l'existence d’un monde
purement intelligible, plus vrai que le nôtre, où les idées existeraient par elles-mêmes,
où les valeurs (le Bien, le Beau, le Juste...) seraient autant d’absolus,
qu’il faudrait d’abord connaître, ou reconnaître, pour bien agir. Le monde sensible
ne serait qu’une copie imparfaite, qu’il faudrait toujours corriger d’après
l’Idée. Le réel ne serait qu’un moindre être, qui ne vaudrait que par l'Être
absolu, toujours absent, toujours ailleurs. Qu'un devenir, qui ne vaudrait que
par l'éternité, ici-bas hors d’atteinte. De là cette fascination pour les mathématiques
(« Que nul n’entre ici s’il n’est géomètre »), pour la dictature du philosophe-roi
(voyez la {\it République} et les {\it Lois}), enfin pour la mort (« les philosophes
authentiques sont avides de mourir », {\it Phédon}, 64 b). De là ce dédain pour l’histoire
ou la vie. C’est toujours adorer la pensée, mépriser le corps ; adorer le
savoir, mépriser le désir ; adorer l'absolu ou l’immuable, mébpriser Le relatif ou
le changeant ; adorer la vérité, mépriser le réel. Le platonisme est le modèle des
idéalismes, des dogmatismes, des utopies — et des totalitarismes, quand ils prétendent
à la science. Celui-là aimait la vérité à en mourir (quoique ce fût
Socrate qui mourut). D’autres l’aimeront à tuer... Heureusement qu'il y a
Aristote, pour nous ramener sur terre et nous remettre à notre place.

PLOUTOCRATIE Le pouvoir, direct ou indirect, des plus riches. Le mot n'a
pas de contraire (les plus pauvres n’ont jamais le pouvoir),
mais un remède, qui est la démocratie. Remède toujours nécessaire, rarement suffisant.
Les pauvres, presque inévitablement, votent pour plus riches qu'eux.

POÉSIE L'unité indissociable et presque toujours mystérieuse, dans un discours
donné, de la musique, du sens et du vrai, d’où naît l'émotion.
%— 447 —
%{\footnotesize XIX$^\text{e}$} siècle — {\it }
C’est une vérité qui chante, et qui touche. À ne pas confondre avec la versification,
même pas avec le poème : il est rare qu’un poème soit tout du long poétique,
et il peut arriver qu’une prose, par moments, le soit.

POIÈSIS Le nom grec de la production, de la fabrication, de la création. Se
reconnaît au fait qu’elle vise toujours un résultat extérieur, qui lui
donne son sens et sa valeur (c’est l’œuvre qui juge et justifie l’ouvrier). S'oppose
à ce titre à la {\it praxis} (voir ce mot), qui ne produit qu’elle-même.

POLÉMIQUE Un combat avec des mots : un discours en état de guerre. Ce
n'est pas toujours condamnable, puisque le conflit est essentiel
à la Cité, puisqu'il y a de bons combats, et puisque les mots, quand ils peuvent
suffire, valent mieux que les armes ou les coups. Toutefois le niveau intellectuel,
presque inévitablement, s’en ressent. Le débat, dans la polémique, vise
plus à la victoire qu’à la vérité ou à la justice. De là, même pour celui qui finit
par triompher, un peu de mauvaise conscience, et comme un goût de sang dans
la bouche.

POLÉMOLOGIE La science de la guerre {\it (polemos)}. N’a jamais dispensé de
la faire. N'a jamais suffi à la gagner, ni à l’éviter. La stratégie
est moins son application que l’un de ses objets. La paix, moins son
dehors (puisque toute paix suppose une guerre au moins possible) que l’un de
ses enjeux.

POLICE Les forces de l’ordre de la Cité {\it (polis)}. L'ordre républicain, sans la
police, serait donc faible, ou plutôt, et très vite, il n’y aurait plus
d'ordre du tout, ni donc de République. Cela ne signifie pas que toute
police soit bonne, mais qu’une police, en toute cité, est nécessaire. Ses
agents, selon la belle et vieillotte appellation qu’on leur donne parfois, sont
les gardiens de la paix. Tant pis pour nous si celle-ci ne peut être gardée que
par la force. On préférerait que l'amour, la justice où même la politesse y
suffisent. Mais ce n’est pas le cas. C’est pourquoi il faut une police : pour
que force reste à la loi, sans laquelle la justice, l’amour et la politesse
devraient s’incliner, avant de disparaître tout à fait, devant les voyous ou les
puissants.

%— 448 —
%{\footnotesize XIX$^\text{e}$} siècle — {\it }
POLITESSE  {\it « Après vous. »} Dans cette formule de politesse, Levinas voyait
l'essentiel de la morale. On comprend pourquoi : c’est refuser
l’égoïsme et court-circuiter la violence par le respect. Toutefois ce n’est que
politesse : l’égoïsme reste inentamé, le respect, presque toujours, n’est que feint.
Peu importe. La violence n’en est pas moins évitée, ou plutôt elle ne l’est que
mieux (s’il fallait respecter vraiment pour la faire disparaître, quelle violence
presque partout !). C’est dire, sur la politesse, l'essentiel : qu’elle n’est pas une
vertu mais qu’elle en a l’apparence, et qu’elle est pour cela aussi socialement
nécessaire qu'individuellement insuffisante. Efficacité de l'apparence. Être poli,
c’est agir {\it comme} si l'on était vertueux : c’est faire semblant de respecter (« pardon»,
«s’il vous plaît », « je vous en prie»...), de s’intéresser (« Comment
allez-vous ? »), de ressentir de la gratitude (« merci »), de la compassion (« mes
condoléances »), de la miséricorde (« ce n’est rien »), voire d’être généreux ou
désintéressé (« après vous »).... Ce n’est pas inutile. Ce n’est pas rien. C’est ainsi
que les enfants ont une chance de devenir vertueux, en imitant les vertus qu’ils
n’ont pas. Et que les adultes peuvent se faire pardonner de l'être si peu.
L’étymologie rapproche la politesse de la politique. Non sans raison : c’est
l’art de vivre ensemble, mais en soignant les apparences plutôt que les rapports
de forces, en multipliant les parades plutôt que les compromis, enfin en surmontant
l’égoïsme par les manières plutôt que par le droit ou la justice. C'est
« l’art des signes », disait Alain, et comme une grammaire de la vie intersubjective.
L'intention n’y fait rien ; l’usage y est tout. On aurait tort d’en être dupe,
mais plus encore de prétendre s’en passer. Ce n’est qu’un semblant de vertu,
moralement sans valeur, socialement sans prix.

POLITIQUE Tout ce qui concerne la vie de la Cité {\it (polis)}, et spécialement
la gestion des conflits, des rapports de forces et du pouvoir. La
politique serait donc la guerre ? Plutôt ce qui vise à l'empêcher, à l'éviter, à la
surmonter : c’est la gestion non guerrière des antagonismes, des alliances, des
rapports de domination, de soumission ou d’obéissance. C’est ce qui la rend
nécessaire : nous vivons ensemble, dans un même pays (politique intérieure),
sur une même planète (politique internationale), sans avoir toujours les mêmes
intérêts, ni les mêmes opinions, ni la même histoire. L'égoïsme est la règle. La
peur est la règle. L’incompréhension est la règle. Comment ne serions-nous pas
ennemis ou rivaux plus souvent qu’amis ou solidaires ? De là les conflits — entre
les individus, entre les classes, entre les États —, et la menace toujours de la
guerre. « Les hommes sont conduits plutôt par le désir aveugle que par la
raison », disait Spinoza, aussi sont-ils « par nature ennemis les uns des autres »
({\it Traité politique}, H, 5 et 14). Et Épicure, vingt siècles plus tôt : « L'homme
% — 449 — 
%{\footnotesize XIX$^\text{e}$} siècle — {\it }
n'est de nature ni sociable ni en possession de mœurs douces » (cité par Themistius,
{\it Discours}, XXVI). L'histoire, entre-temps, ne les a guère démentis, et la
politique n’est jamais que l’histoire au présent. Que d’injustices partout ! Que
d’horreurs, presque partout ! Pourtant il faut, c’est notre intérêt à tous, que la
paix advienne ou se perpétue, que la solidarité s’organise ou se développe : elles
ne sont pas données d’abord, mais toujours à faire, à refaire, à préserver, à renforcer.
C’est à quoi servent les partis, les syndicats, les élections. C’est à quoi
servent les États. C’est à quoi sert la politique. Il s’agit de créer des convergences
d’intérêts — ce qui ne va pas sans compromis — afin que la paix soit, afin
que la justice et la liberté puissent être. Qu'est-ce que la politique ? C’est la vie
commune et conflictuelle, sous la domination de l’État et pour son contrôle
(politique intérieure), entre États et sous leur protection (politique internationale) :
c’est l’art de prendre, de garder et d’utiliser le pouvoir. C’est aussi
l’art de le partager ; mais c’est qu’il n’y a pas d’autre façon en vérité de le
prendre, ni de le garder.

POLYTHÉISME La croyance en plusieurs dieux, et après tout pourquoi pas ?
La difficulté commence quand on veut les nommer, en
dresser une liste (finie ? infinie ?), et y croire. Un seul Dieu, il est désigné suffisamment
par son unicité. Mais plusieurs ? Comment les distinguer ? Comment
les reconnaître ? Pourquoi y croire ? Ce ne sont le plus souvent que des forces
naturelles, des passions ou des abstractions personnifiées, comme un animisme
hypostasié. Par exemple : un dieu du vent, un dieu de l'amour, un dieu de la
guerre, un autre pour le ciel ou l’océan, d’autres encore pour le vin, la fécondité
ou la colère, sans parler des dieux professionnels, nationaux ou ethniques.
Pourquoi pas un dieu pour la pesanteur, les banquiers ou les habitants du quatorzième
arrondissement ?

Il est de bon ton, aujourd’hui et dans certains milieux, de voir dans le polythéisme
une école de pluralisme et de tolérance. Le fait est que les Romains, par
exemple, firent aux dieux étrangers un accueil plutôt sympathique, à la seule
exception, mais ce n’est pas rien, de celui qui se prétendait l'unique, dont les
sectateurs furent donnés aux lions ou aux flammes. Bonjour la tolérance, qui ne
tolère que le semblable ! Puis le polythéisme n’empêcha pas les Grecs d’assassiner
Iphigénie et Socrate, ou plutôt il les y poussa efficacement. La religion
n'attendit pas le monothéisme pour être, comme disait Lucrèce, pourvoyeuse
de crimes.

On me dira que l’athéisme fit des morts aussi, et peut-être davantage. C’est
hélas vrai. Mais dans la mesure seulement où il se prit pour une religion (de
l'Histoire ou de l’État) ou un messianisme (du prolétariat). Non par manque
%— 450 —
%{\footnotesize XIX$^\text{e}$} siècle — {\it }
de religion, donc, mais par excès de foi. C’est toujours l'enthousiasme qui
allume les bûchers. Le polythéisme n’en est préservé, quand il l’est, que par
l'impossibilité de prendre tout à fait au sérieux ces dieux trop nombreux et trop
humains. C’est la petite monnaie de l’absolu. Videz vos poches.

POSITIF En philosophie, s'oppose moins à {\it négatif} qu’à {\it naturel}, {\it métaphysique}
ou {\it chimérique}. Est positif ce qui existe en fait (par exemple le {\it droit
positif}, opposé au {\it droit naturel}) ou s'appuie sur les faits (les sciences positives).
Chez Auguste Comte, dans sa fameuse loi des trois états, « l’état positif ou
réel », qui est le troisième, s'oppose à « l’état métaphysique ou abstrait », auquel
il succède, comme celui-ci s'oppose à « l’état théologique ou fictif » qui l'avait
précédé.

POSITION L'une des dix catégories d’Aristote. Ce n’est pas le lieu où l’on
se trouve (la localisation), mais une certaine façon de l’occuper :
par exemple assis ou debout. Le mot {\it situation}, qu’on utilise parfois en ce sens,
prête davantage à confusion : mieux vaut le réserver pour un autre usage.
Leucippe et Démocrite, se souvient Aristote, pensaient que les atomes ne se
distinguaient entre eux que par « la forme, l’ordre et la position », comme font
les lettres dans un mot : « Ainsi {\it A} diffère de {\it N} par la forme, {\it AN} de {\it NA} par
l’ordre, et {\it Z} de {\it N} par la position » (le {\it Z} étant comme un {\it N} couché, et
réciproquement : voir {\it Métaphysique}, A, 4). Pourtant {\it Z} et {\it N} sont deux lettres
différentes, quand Socrate debout ou couché reste un seul et même homme.
C’est que Socrate n’est pas une lettre : il peut changer de position sans que sa
position le change.

POSITIVISME C’est d’abord le système d’Auguste Comte, qui ne voulait
s'appuyer que sur les faits et les sciences : il renonce pour
cela à chercher l'absolu et même les causes (le {\it pourquoi}), pour ne s’en tenir
qu’au relatif et aux lois (le {\it comment}). Il en a fait une puissante synthèse, qui est
le positivisme même. Système impressionnant, aussi bien dans sa masse que
dans son détail, aujourd’hui injustement méprisé. Il faut dire qu’il est desservi
par la personnalité de son auteur, dont la santé mentale laissait à désirer, et par
son style, étonnamment indigeste. Qu’on en juge à ce petit passage, qui dit
pourtant, sur l'esprit du positivisme, l'essentiel :

%— 451 — — {\it }
{\footnotesize 
« Constatant l’inanité radicale des explications vagues et arbitraires propres à la philosophie
initiale, soit théologique, soit métaphysique, l'esprit humain renonce désormais
aux recherches absolues qui ne convenaient qu’à son enfance, et circonscrit ses
efforts dans le domaine, dès lors rapidement progressif, de la véritable observation,
seule base possible des connaissances vraiment accessibles, sagement adaptées à nos
besoins réels. La logique [...] reconnaît désormais, comme {\it règle fondamentale}, que
toute proposition qui n’est pas strictement réductible à la simple énonciation d’un fait,
ou particulier ou général, ne peut offrir aucun sens réel et intelligible. [...] En un mot,
la révolution fondamentale qui caractérise la virilité de notre intelligence consiste essentiellement
à substituer partout, à l’inaccessible détermination des causes proprement
dites, la simple recherche des {\it lois}, c’est-à-dire des relations constantes entre les phénomènes
observés » ({\it Discours sur l'esprit positif}, VX, 12).
}

Le mot, après Auguste Comte, s’est banalisé. Il désigne toute pensée qui
prétend s’en tenir aux faits ou aux sciences, à l’exclusion de toute interprétation
métaphysique ou religieuse, voire de toute spéculation proprement philosophique.
C’est ainsi qu’on parle de {\it positivisme juridique} (une conception du
droit qui ne reconnaît ou n’étudie que le droit positif) ou de {\it positivisme logique}
(la doctrine de Carnap et son école), non que leurs partisans se réclament en
rien d’Auguste Comte, qu’ils n’ont guère lu, mais parce qu’ils s’opposent eux
aussi à la métaphysique et voudraient ne s’en tenir qu’à ce qui peut être positivement
établi (par exemple dans des textes de lois ou des énoncés scientifiques).

En dehors de ces acceptions historiques précises, le mot positivisme est
habituellement pris, du moins aujourd’hui, en un sens péjoratif. Ce serait une
pensée courte, et comme une négation de la philosophie. Cet usage relève surtout
de la polémique. Encore faut-il utiliser le mot à bon escient. On évitera
notamment de confondre le {\it positivisme}, qui renonce à la métaphysique, avec le
{\it scientisme}, qui voudrait que la science en soit une.

POSSESSION Le fait de posséder quelque chose, c’est-à-dire d’en avoir la
jouissance ou l’usage. Mais ce n’est qu’un fait : c’est ce qui
distingue la {\it possession} de la {\it propriété}, qui est un droit.

POSSIBLE Ce qui peut être ou arriver. Donc ce qui n’est pas ? Point forcément,
ni en toute rigueur : ce qui est peut être, puisqu'il est, et
ce serait plutôt ce qui n’est pas qui s’avère présentement impossible (puisqu'il
n'est pas). Par exemple il est assurément possible que je sois assis, puisque je le
suis, et présentement impossible, tant que je le suis, que je ne le sois pas. Mais
alors seul le réel serait possible, qui serait aussi nécessaire, et tout le reste serait
%— 452 —
%{\footnotesize XIX$^\text{e}$} siècle — {\it }
impossible. 11 n’y aurait que ce qu’il y a : les catégories de la modalité (voir ce
mot) s’aboliraient dans une espèce de monisme ontologique. C’est le monde
même. Mais comment le penser au futur, sans distinguer ce qui peut arriver (le
possible) de ce qui ne le peut pas (l’impossible) ou de ce qui arrivera inévitablement
(le nécessaire) ? Il faut se donner une autre définition, proprement
modale, qui ne définit plus le possible par rapport à l'être mais par rapport à
son contraire : est possible, au sens large, tout ce qui n’est pas impossible. C’est
donc la modalité la plus vaste, qui inclut tout ce qui est réel, tout ce qui peut
le devenir, tout ce qui le deviendra nécessairement. En un sens restreint, en
revanche, on entend par possible tout ce qui n’est ni réel, ni nécessaire, ni
impossible : tout ce qui peut être ou ne pas être, donc ce qui n’est pas encore
et ne sera peut-être jamais. Cela n'existe que pour la pensée : c’est un être de
raison, comme dit Spinoza, mais aucune raison, si elle se fait prospective, ne
peut s’en passer.

POSTULAT Un principe qu’on pose, sans pouvoir le démontrer. Ne se distingue
de l’axiome que par une évidence moindre. Les mathématiciens
modernes ont d’ailleurs renoncé à cette distinction. C’est qu'ils ont
renoncé à l'évidence des principes, pour ne plus reconnaître que la nécessité des
inférences.

POSTULATS DE LA RAISON PRATIQUE Chez Kant, ce sont des propositions
théoriques dont la vérité
est affirmée — mais en vertu d’une nécessité seulement pratique et subjective,
donc sans qu’on puisse y voir une démonstration — à partir des exigences
de la moralité. Ces postulats sont au nombre de trois : la liberté de la volonté,
l’immortalité de l’âme, l'existence de Dieu ({\it C.R. Pratique}, Dialectique, IV-VI).
Il faut y croire, selon Kant, pour que l’expérience morale ait un sens. Cela ne
prouve pas que Dieu existe, que nous soyons libres ou que l'âme soit immortelle
(puisque rien ne prouve que la morale ait un sens), ni n’entraîne que ce
soit un devoir d’y croire (puisque « ce ne peut être un devoir d’admettre l’existence
d’une chose »), mais qu’il est moralement nécessaire d’accepter ces trois
postulats : on ne saurait autrement échapper à l'absurde et au désespoir. Mais
pourquoi faudrait-il y échapper ? Pour faire son devoir ? Non pas, puisque
celui-ci n’a pas besoin d’espérance. Les postulats de la raison pratique ne répondent
pas à la question {\it « Que dois-je faire ? »}, mais à la question {\it « Que m'est-il
permis d'espérer ? »}. C’est la dimension d’espérance de la morale, par où elle
conduit à la religion.

%— 453 —
%{\footnotesize XIX$^\text{e}$} siècle — {\it }
POUR-SOI  Exister {\it pour soi}, c’est être en relation avec soi sur un autre mode
que la seule identité. Se distingue par là de l'en soi, spécialement
chez Hegel et Sartre. L’en soi est ce qu’il est ; le pour-soi a à l’être,
explique Sartre, ce qui suppose qu’il n’est pas ce qu’il est et qu’il est ce qu’il
n’est pas. C’est le mode d’être de la conscience, qui lui interdit de coïncider
exactement avec elle-même (d’exister en soi) : « Le pour-soi est un être pour qui
son être est en question dans son être en tant que cet être est essentiellement
une certaine manière de {\it ne pas être} un être qu’il pose du même coup comme
autre que lui » ({\it L'être et le néant}, p. 222). Vouloir être en soi, pour un homme,
ce serait faire semblant de n’être pas libre (mauvaise foi) ; vouloir exister en-soi-pour-soi,
ce serait vouloir être Dieu : « Ainsi la passion de l’homme est-elle
inverse de celle du Christ, car l’homme se perd en tant qu'homme pour que
Dieu naisse. Mais l’idée de Dieu est contradictoire et nous nous perdons en
vain : l’homme est une passion inutile ({\it ibid.}, p. 708).

POUVOIR Définition parfaite chez Hobbes : « Le pouvoir d’un homme
consiste dans ses moyens présents d’obtenir quelque bien apparent
futur » ({\it Léviathan}, chap. 10). C’est donc du réel (puisque c’est du présent),
mais tout entier tourné vers l'avenir. Pouvoir, c’est pouvoir faire. Encore
faut-il distinguer le {\it pouvoir de}, qu’on appellerait mieux {\it puissance} (pouvoir de
marcher, de parler, d’acheter, de faire l'amour), et le {\it pouvoir sur}, qui est une
forme du précédent (c’est le pouvoir de commander et de se faire obéir), mais
qui porte sur des êtres humains et qui est le pouvoir au sens strict. Non la
simple action possible, mais l’ordre possible, la contrainte possible, le contrôle
possible, la sanction possible... Dès que l’action possible est action possible (et
reconnue telle de part et d’autre) sur la volonté de quelqu'un d’autre, on passe
du {\it pouvoir de} au {\it pouvoir sur} — et l’action possible est alors immédiatement
action réelle. Pouvoir punir ou récompenser, pouvoir autoriser ou interdire,
cette simple possibilité (comme {\it pouvoir de}) est déjà une réalité (comme {\it pouvoir
sur}). C’est le secret du pouvoir : il s'exerce même quand il n’agit pas ; il gouverne
même quand il n’ordonne pas. La simple possibilité d’agir (quand c’est
agir sur quelqu'un) est déjà une action. Pouvoir commander, c’est déjà commander
en effet.

Deux sens, donc : {\it pouvoir de}, et {\it pouvoir sur}. L'action possible, ou la domination
réelle. On peut, pour les distinguer, appeler le premier {\it puissance}
({\it potentia}, en latin), et garder pour le second le mot de {\it pouvoir}, en un sens strict
({\it potestas}). Mais à condition de ne pas oublier que la puissance est première : la
{\it potestas} n'est qu'une {\it potentia} particulière ; le pouvoir n’est que la puissance
d'un homme ou d’un groupe sur d’autres hommes ou d’autres groupes. Le
%— 454 —
%{\footnotesize XIX$^\text{e}$} siècle — {\it }
pouvoir, c’est la puissance humaine que l’on subit ou, plus rarement, que l'on
exerce. La puissance, nous la partageons avec la nature. Il n’est de pouvoir
qu’humain. C’est pourquoi le pouvoir est tellement agaçant, quand c’est celui
des autres, et tellement délicieux, quand c’est le sien. Hobbes encore : « Je
mets au premier rang, à titre d’inclination générale de toute l'humanité, un
désir perpétuel et sans trêve d’acquérir pouvoir après pouvoir, désir qui ne cesse
qu’à la mort » ({\it op. cit.}, chap. 11). Au premier rang ? Pour ce qui me concerne,
je n’irais pas jusque-là. Plusieurs inclinations, qui ne sont pas toutes estimables,
m'importent davantage.

PRAGMATIQUE Qui relève de l’action ({\it pragma}) et ne reconnaît d’autre
critère que sa réussite ou son efficacité. En philosophie,
et contrairement à l’usage politique ou journalistique, le mot exprime plutôt
une réserve qu’une approbation. Une injustice efficace n’en est pas moins
injuste pour autant.

PRAGMATISME Une attitude ou une doctrine qui privilégie l’action, et la
réussite de l’action, jusqu’à en faire le seul critère légitime
d'évaluation. Le bien ? C’est ce qui réussit. Le vrai ? C’est ce qui est utile ou
efficace («ce qui marche »). On peut en donner une version courte, qui ne
serait qu’une forme de sophistique : le nazisme serait vrai si Hitler avait gagné
la guerre. Mais on peut aussi, avec Charles Sanders Peirce et William James, y
voir une philosophie de la science et de la démocratie. Le fait qu’il s'agisse de
deux philosophes américains ne saurait tenir lieu de réfutation.

Qu'est-ce que le pragmatisme ? Une doctrine, répond Peirce, qui identifie
la conception d’un objet à celle de ses effets possibles. Savoir ce que sont le feu
ou la gravitation, c’est savoir quels effets ils peuvent produire. Aussi une idée
n'est-elle qu’une hypothèse, qu’il faut soumettre à l'expérimentation pour
déterminer sa valeur ; il n’y a aucun sens à la tenir pour vraie si aucun effet ne
la valide. La vérité, pour le pragmatiste, c’est donc bien ce qui réussit, mais pas
au sens mercantile du terme (« ce qui paye ») : c’est ce qui résiste efficacement
à sa mise à l'épreuve expérimentale. La vérité n’est pas un absolu ; c'est une
hypothèse qui a fait ses preuves.

Le même genre d’idées peut s'appliquer à la politique. Une injustice efficace,
disais-je dans l’article précédent, n’en est pas moins injuste pour autant.
Mais une justice sans effet, répondrait un pragmatiste, comment serait-elle
juste ? Aussi faut-il soumettre nos idées à l’épreuve du réel, plutôt que le réel à
une idée préconçue, comme fait le totalitarisme. Non qu’il faille pour cela se
%— 455 —
%{\footnotesize XIX$^\text{e}$} siècle — {\it }
passer d’idéaux, ni même qu’on le puisse. Mais parce qu’un idéal n’est que
l’ensemble des conséquences prévues d’une activité, qui la motivent, certes,
mais doivent à leur tour être soumises à l’expérience (un idéal qui ne peut
réussir est un mauvais idéal). La démocratie est la mise en œuvre de cette expérience
commune, qui la valide dans la mesure même où elle s’y soumet.

Ce pragmatisme-là n’est pas une sophistique ; c’est un empirisme radical
(l'expression est de William James) et une philosophie de l’action.

PRATIQUE La définition élaborée par Althusser me paraît trop étroite :
« Par {\it pratique} en général, écrivait-il, nous entendons tout processus
de transformation d’une matière première donnée déterminée, en un
produit déterminé, transformation effectuée par un travail humain déterminé,
utilisant des moyens (de “production”) déterminés » ({\it Pour Marx}, p. 167).
C'était accorder trop à la production et au travail. Je dirais plutôt : J'entends
par pratique une activité ({\it praxis}, en grec, ou {\it energeia}) qui transforme quelque
chose ou quelqu'un, soit en produisant une œuvre extérieure à cette activité
(Aristote parlait alors de {\it poièsis}), soit en ne produisant que cette activité même
({\it praxis} au sens restreint). C’est « l’activité humaine concrète », comme disait
Marx ({\it Thèses sur Feuerbach}, 1), dont le travail n’est qu’un cas particulier.

PRATIQUE THÉORIQUE Une activité dans la pensée, et qui la transforme.
Sa matière première est faite de représentations,
de concepts, de faits, de théories, de valeurs, de connaissances (voir
Althusser, {\it Pour Marx}, p. 168), qu’elle travaille ou critique jusqu’à en obtenir
d’autres représentations, d’autres concepts, d’autres faits, d’autres théories,
d’autres valeurs ou d’autres connaissances. Les pratiques théoriques sont bien
sûr multiples, voire innombrables : chaque science est l’une d’elles, ou plutôt
chaque activité scientifique ; la philosophie en est une autre, ou plutôt toute
activité philosophante.

PRAXIS Le nom grec de l’action ; le nom snob ou marxiste de la pratique.
Le mot ne me semble guère utile que par son opposition, d’origine
aristotélicienne, à la {\it poièsis}. Ce sont deux types d’action, mais qui se distinguent
par la présence ou non d’un but extérieur. La {\it praxis} est alors une action
qui ne vise rien d’autre que son bon déroulement (son {\it eupraxia}) : elle ne tend
à aucune fin extérieure à elle-même ni à aucune œuvre extérieure à celui qui
agit. Ce n’est pas qu’elle soit stérile ; c’est quelle se suffit à elle-même. La posèsis,
%— 456 —
%{\footnotesize XIX$^\text{e}$} siècle — {\it }
au contraire, est une production ou une création : elle n’a jamais sa fin en elle-même,
mais toujours dans son résultat, qui lui reste extérieur (le produit ou
l’œuvre : {\it ergon}). La vie, par exemple, est une {\it praxis} : vivre, c’est créer sans
œuvre. Et le travail ou l’art, une {\it poièsis}. Celle-ci n’a de sens qu’au service de
celle-là.

PRÉCAUTION (PRINCIPE DE-) Prendre des précautions, c’est agir pour
éviter un mal, ou ce qu’on juge être tel.
Prudence appliquée, face à un risque réel ou supposé. Ainsi en matière de
contraception ou d’alpinisme : la prudence n’impose pas qu’on renonce à faire
l'amour lorsqu'on ne veut pas d’enfant, ni à escalader la montagne lorsqu'on ne
veut pas mourir, mais assurément qu'on prenne, pour le faire, un certain
nombre de précautions (un moyen contraceptif efficace dans le premier cas, un
équipement et un entraînement adaptés dans le second..). Ces exemples, on
pourrait en prendre beaucoup d’autres, justifient deux remarques.

La première, c’est que la précaution suppose une évaluation préalable, et ne
saurait en tenir lieu. Faire un enfant, est-ce un bien ou un mal ? Les précautions
éventuelles en dépendent ; elles n’en décident pas.

La seconde remarque, c’est que la précaution est ordinairement tout autre
chose qu’un évitement. Que penseriez-vous de celui qui vous expliquerait :
« En matière d’alpinisme et de sexualité, j'ai pris mes précautions : j'ai choisi la
plaine et la chasteté » ? Que ce n’est plus précaution mais fuite. Prendre des
précautions, c’est agir ; non pour supprimer tout risque, ce qu’on ne peut, mais
pour le réduire le plus possible, dans une situation donnée — y compris
lorsqu'elle reste, comme dans le cas de l’alpinisme, inévitablement périlleuse. Il
ne s’agit pas de renoncer, mais de préparer, d’anticiper, d’assurer — de faire
attention. Prudence appliquée, disais-je, et c’est la prudence même.

Qu'en est-il alors de ce fameux {\it « principe de précaution »}, dont on nous
rebat depuis quelques années les oreilles, le plus souvent sans prendre la peine
de le définir ni même de l’énoncer ? En quoi se distingue-t-il de la simple
prudence ?

D'abord en ceci, me semble-t-il, qu’il concerne surtout les pouvoirs publics
ou, à tout le moins, les collectivités : un gouvernement ou une entreprise peuvent
appliquer le principe de précaution ; un individu se contentera, dans sa vie
privée, de prendre les siennes.

Ensuite en ceci que le principe de précaution suppose que les risques soient
impossibles à mesurer exactement, voire à attester absolument. Quand un pays
décide d’une limitation de vitesse sur ses routes, il n’applique pas le principe de
précaution : les risques de la vitesse, en matière de sécurité routière, sont tristement
%— 457 —
%{\footnotesize XIX$^\text{e}$} siècle — {\it }
avérés et d’ailleurs faciles, fât-ce par voie statistique, à mesurer avec un
degré satisfaisant de précision et de certitude. Aussi est-ce moins {\it précaution} que
{\it prévention}. Mais en matière d'organismes génétiquement modifiés (les fameux
OGM) ? Mais en matière de transfusion sanguine ? Mais en matière d’énergie
nucléaire? Qu'il y ait des risques dans les trois cas, c’est plus que
vraisemblable ; mais ils ne relèvent du principe de précaution — et non de la
seule prudence — que dans la mesure où ces risques ne peuvent être déterminés
exactement, ni même d’une façon qui permette de les comparer précisément
avec les avantages attendus des pratiques qui les font naître (les manipulations
génétiques, les perfusions, les centrales nucléaires...). C’est ce qui distingue la
{\it précaution} de la {\it prévention}. « La prévention, remarque Catherine Larrère, a rapport
aux risques avérés, dont l'existence est certaine et la probabilité plus ou
moins bien établie. La précaution a affaire aux risques potentiels, non encore
avérés » ({\it Dictionnaire d'éthique et de philosophie morale}, PUF, article « Principe
de précaution »). Qu'il y ait, dans chaque centrale nucléaire, une prévention
des risques, c’est la moindre des choses. Mais faut-il, ou pas, construire de telles
centrales ? Cela ne relève plus de la prévention, mais du principe de
précaution : parce qu'il s’agit de confronter des avantages déterminés (de coût,
d'indépendance énergétique, de sûreté des approvisionnements, de réduction
de l'effet de serre.) à des risques qui restent pour une bonne part indéterminables
(ceux d’un accident ou d’une guerre, ceux qui concernent le stockage
des déchets pendant plusieurs milliers d’années....). C’est aussi ce qui distingue
le principe de précaution de la simple prudence. « Le principe de précaution,
me dit un jour Jean-Pierre Dupuis, c’est la prudence en situation d’incertitude »,
non au sens ordinaire du terme (car c’est le lot presque toujours de la
prudence, que d’être confrontée à l’incertain), mais au sens où l’on parle
d'incertitude en mécanique quantique : quand la détermination des risques
bute sur une limite infranchissable, qui ne permet ni de vérifier ni de quantifier
leur réalité.

Le principe de précaution a donc bien à voir avec la prudence, dont il n’est
qu’une occurrence particulière : c’est la prudence en situation d’incertitude et
de responsabilité collective. Le législateur en a donné la formulation suivante :
« L'absence de certitudes, compte tenu des connaissances scientifiques et techniques
du moment, ne doit pas retarder l’adoption de mesures effectives et proportionnées
visant à prévenir un risque de dommages graves et irréversibles à
l’environnement à un coût économiquement acceptable » (loi du 2 février
1995, dite « loi Barbier »). Pour ceux qui ne sont pas juristes, je proposerais
volontiers une formulation plus simple, qui pourrait s'adresser à tout responsable
d’une collectivité quelconque, qu’elle soit publique ou privée : {\it N'attends
pas qu'un risque soit démontré ou mesuré pour essayer de le prévenir ou d'en limiter
%— 458 —
%{\footnotesize XIX$^\text{e}$} siècle — {\it }
les effets}. Et je ne connais guère de principe plus incontestable, ni plus incontesté.

Dans la pratique, toutefois, il me semble que ce principe tend communément
à prendre une autre forme, souvent implicite mais qui ressort de l'usage
qui est en fait. Tel qu’il fonctionne dans nos journaux ou dans les discours de
nos hommes politiques, il pourrait plutôt s’énoncer sous cette forme : {\it « Ne faisons
rien qui puisse présenter un risque que nous ne serions pas capables d'évaluer
précisément ou que nous ne serions pas certains de pouvoir surmonter. »} Et quoi, en
apparence, de plus raisonnable ? Le problème, c’est que si l’on adopte cette dernière
formulation, il faut en conclure que ce serait violer le principe de précaution
que de se lever le matin : qui sait quels risques possiblement mortels cela
nous fait courir ? Mais rester au lit toute la journée et tous les jours n’est pas
non plus sans danger : voilà que le principe de précaution nous enferme dans
une contradiction insurmontable... Je plaisante, puisque le principe de précaution
ne vaut guère, je l’ai signalé en passant, que dans les situations de responsabilité
publique ou collective, et puisque les risques, ici, pourraient être statistiquement
mesurés (les assureurs y parviennent fort bien). Mais même à
considérer le principe de précaution dans son champ légitime d’application, il
n’est pas difficile de trouver des apories comparables. Lorsque l’automobile fut
inventée, qui pouvait en évaluer précisément les risques, aussi bien en matière
d’accidents que de pollution ? Et qui le peut aujourd’hui ? « Il va y avoir des
milliers de morts ! », pouvait dire l’un. Il y en eut plutôt des millions. Mais fallait-il
pour autant renoncer à l'automobile? Question légitime, encore
aujourd’hui. Je ne vois pas que le principe de précaution suffise à y répondre.

On m’objectera que l'invention de l’automobile n’entraïnait pas un changement
irréversible : il n’est pas impossible, au moins en théorie, de revenir en
arrière. Cela est vrai ; et même si cette possibilité reste en effet purement théorique
(quel gouvernement pourrait aujourd’hui interdire l'automobile ?), cela
dit quelque chose d’important sur le principe de précaution : qu’il doit s’appliquer
d’autant plus rigoureusement que les risques encourus peuvent être irréversibles.
C’est le cas, par exemple, du débat sur les OGM. Une fois que des
gènes modifiés se seront répandus dans la nature, il sera sans doute impossible
de les supprimer : le changement sera irréversible, et les risques encourus, dès
lors, le seront tout autant. Raison de plus pour être vigilant. Mais est-ce une
raison suffisante pour renoncer aux OGM, et aux avantages éventuels (en
matière de rentabilité, mais aussi en matière de protection de l’environnement,
de lutte contre la faim, de recherche médicale...) qu’on peut en attendre ? Je ne
sais. J'ai participé à plusieurs tables rondes sur la question : j’ai pu constater que
les experts eux-mêmes, sur la décision à prendre, s’opposaient vigoureusement.
Je doute que le principe de précaution suffise à les mettre d’accord.

%— 459 —
%{\footnotesize XIX$^\text{e}$} siècle — {\it }
Il m'arrive d'imaginer un débat, il y a plusieurs centaines de milliers
d'années, lorsque les premiers hominiens entreprirent de maîtriser le feu. D’un
côté un apprenti sorcier, qui joue avec des silex et des bouts de bois. De l’autre,
un sage écologiste, qui se soucie de la nature et de l’avenir : « Attention, s’écrie
ce dernier : avec le feu, on ne sait pas où l’on va ! On ne peut mesurer exactement
les risques : il y aura forcément des accidents, des incendies, peut-être des
milliers de morts. » Il y en eut bien davantage. Mais l’humanité a maîtrisé le
feu.

Ou un autre débat, il y a trois siècles, autour de la machine à vapeur. D’un
côté un apprenti sorcier, qui bricole sa marmite et ses pistons. De l’autre, un
sage soucieux d'environnement et de tradition : « Attention, s’alarme-t-il : avec
la machine à vapeur, nous entrons dans un domaine inconnu, avec des risques
que nous ne pouvons évaluer ! Cette nouvelle technologie peut bouleverser
toute notre économie, remettre en cause l'équilibre de nos campagnes et de nos
villes, menacer la forêt, épuiser nos réserves de charbon, modifier le climat. Il
peut y avoir des milliers de morts ! » Il y en eut bien davantage. Mais l’humanité
a fait la révolution industrielle.

Je ne dis pas cela contre les écologistes, pour qui j’ai souvent voté, pour qui
je voterai encore, mais contre un certain usage du principe de précaution, ou
plutôt de sa caricature, qui me paraît nous enfermer dans l’inaction ou le
conservatisme. Toute nouveauté présente un risque, qu’il est presque toujours
impossible de mesurer exactement. Le principe de précaution, s’il est bien compris,
n’impose pas qu’on renonce pour cela au progrès, mais simplement qu’on
y tende en essayant de prévenir ou de limiter les risques, fussent-ils seulement
possibles, que telle ou telle nouveauté peut entraîner. Il est toujours coupable
de ne rien faire contre un danger possible, voilà ce qu’indique bien clairement
le principe de précaution. Mais il n’en résulte pas qu’il soit toujours coupable
de faire quelque chose quand cela peut présenter un certain risque. Car alors on
ne ferait plus rien du tout, en tout cas plus rien de neuf : ce ne serait plus précaution
mais immobilisme.

Bref, le principe de précaution est un principe positif, qui {\it impose} quelque
chose (une action, contre un danger possible : « Dans le doute, fais quelque
chose pour limiter les risques »), non un principe négatif, qui {\it interdirait} une
action dès lors qu’elle peut entraîner un certain risque (sur le mode : « Dans le
doute, abstiens-toi »). Car autrement, le risque zéro n’existant pas, comme on
ne cesse à juste titre de le rappeler, le principe de précaution, sous sa forme abstentionniste,
nous vouerait à l’inaction — il y a toujours un doute : il faudrait
s'abstenir toujours ! — et donnerait tort à toute l’histoire humaine, qui n’a cessé
de prendre des risques et de les surmonter.

%— 460 —
%{\footnotesize XIX$^\text{e}$} siècle — {\it }
On m'objectera que la formulation du principe de précaution, telle que je
l’énonce, ne nous dit pas ce qu’il faut décider en matière d'OGM, d'énergie
nucléaire ou de transfusion sanguine (s’agissant aujourd’hui, par exemple, de la
maladie de Creutzfeldt-Jakob, pour laquelle les risques transfusionnels, en cette
année 2000, sont possibles mais non avérés). J’en suis d’accord, mais je n’y vois
pas une objection contre ce principe, ni contre ma formulation. Aucun principe
ne peut décider à notre place, et c’est heureux : le principe de précaution
peut éclairer une décision politique, il ne saurait en tenir lieu. C’est où l'on
retrouve la prudence, qui n’est pas un principe mais une vertu. Et la démocratie,
qui n’est pas une garantie mais une exigence.

Tout ce qu’on fait est dangereux, mais inégalement : il serait aussi imprudent
de ne rien faire que de faire n’importe quoi.

PRÉCONSCIENT  L’une des trois instances de la première topique de
Freud.

Le préconscient n’est pas une espèce d’intermédiaire ou de sas, comme on
le croit parfois, entre le conscient et l'inconscient (ce qu’on appellerait mieux le
subconscient, concept non freudien). Il ne fait, avec le conscient, qu’un seul et
même système (le système Pcs-Cs), lequel est séparé de l'inconscient (le système
Ics) par le refoulement et la résistance. Qu'’est-ce que le préconscient ? C’est
l’ensemble de tout ce qui peut être conscient (sans avoir besoin pour cela de
vaincre les défenses de l'inconscient), mais qui ne l’est pas actuellement. Le
conscient n’est que sa pointe extrême et infime. Par exemple votre date de naissance,
le prénom de votre conjoint ou la couleur de vos yeux : selon toute vraisemblance,
aucune de ces trois données ne faisait partie, il y a vingt secondes,
du champ de votre conscience ; elles y sont maintenant (elles sont passées du
préconscient au conscient), sans avoir eu besoin pour cela de déjouer quelque
censure que ce soit. Tel est le préconscient : une immense, du moins à notre
échelle, et morne salle d’attente, avant et après le petit train de la conscience.
C’est l'inconscient, si l’on en croit Freud, qui a posé les rails. Mais c’est le
monde qui les porte, et qu’on voit par les fenêtres. {\it E pericoloso ma buono sporgersi.}


PRÉDESTINATION Un destin écrit à l’avance.

La prédestination est à peu près au destin, ou même à
la providence, ce que le prédéterminisme est au déterminisme : son anticipation
rétrospective et superstitieuse. Toutefois il est difficile, si l’on croit en un
Dieu omniscient et tout-puissant, d’y échapper. Dieu me donne l'être et la vie.
%— 461 —
%{\footnotesize XIX$^\text{e}$} siècle — {\it }
Il sait de toute éternité si je serai sauvé ou damné, et même (par la grâce) il en
décide : comment pourrais-je y échapper ? C’est la lumière noire de la foi, celle
de saint Augustin, de Calvin, de Pascal. Elle effraie nos croyants modernes,
presque tous pélagiens ou jésuites. Cela même, peut-être, était écrit.

PRÉDÉTERMINISME Un déterminisme faraliste (ce qu’Épicure appelait
« le destin des physiciens ») : le passé serait cause du
présent, comme le présent de l’avenir, de sorte que tout, toujours, serait écrit à
l'avance. C’est un déterminisme dilaté dans le temps — un déterminisme obèse.
C'est aussi, à ce que je crois, un contresens sur la causalité. Si seul le présent
existe, lui seul (dont nous faisons partie) est cause et effet : comment le passé,
qui n'est plus, pourrait-il gouverner l'avenir, qui n’est pas ? Comment l’un ou
l’autre pourraient-ils commander le présent, qui est tout ? Le démon de Laplace
n’était qu’un mauvais rêve.

PRÉDICAT Tout ce qui est affirmé d’un sujet quelconque. Par exemple
dans les propositions {\it « Socrate est un homme »} où {\it « Socrate se
promène »} : «est un homme » et « se promène » sont des prédicats.

PRÉDICTION Cest dire à l'avance ce qui sera, quand on croit le savoir par
des voies mystérieuses ou surnaturelles (prédiction n’est pas
prévision). Le propre des prophètes et des imbéciles.

PRÉJUGÉ Ce qui a été jugé avant. Avant quoi ? Avant d’y avoir réfléchi
vraiment et comme il faut. C’est le nom classique et péjoratif de
l'opinion, spécialement chez Descartes, en tant qu’elle est préconçue.

La force des préjugés tient au fait que « nous avons tous été enfants avant
que d’être hommes », et avons commencé à penser bien avant de savoir raisonner
(si tant est que nous sachions). Le remède, selon Descartes, est dans le
doute et la méthode. Force est pourtant de constater que le cartésianisme donnera
raison, pour finir, à la plupart des préjugés de son époque. Pour être
homme, et même grand homme, on n’en est pas moins enfant.

PRÉMÉDITATION Une volonté anticipée : c’est vouloir à l'avance, ou
lutôt (car il n’est de volonté qu’actuelle) c’est projeter
%— 462 —
%{\footnotesize XIX$^\text{e}$} siècle — {\it }
et continuer de vouloir. On considère d'ordinaire, spécialement dans les tribunaux,
qu’une volonté ainsi préparée est plus grave, lorsqu'elle est coupable,
qu’une autre. C’est qu’elle ôte l’excuse possible de la colère ou de l’irréflexion.

PRÉMISSE Une proposition considérée comme première (par rapport à ses
conséquences), et spécialement les deux premières propositions
— la majeure et la mineure — d’un syllogisme.

PRÉNOTION L'une des deux traductions usuelles (avec « anticipation ») du
grec {\it prolèpsis}. C’est un autre nom pour désigner les idées
générales qui résultent de la répétition d’expériences à peu près identiques, ou
dont on ne retient que ce qui l’est. Par exemple l’idée d’arbre, pour Épicure, est
une prénotion : elle n’existe, comme idée, que parce que j'ai perçu plusieurs
arbres différents, dont je n’ai retenu que ce qu'ils avaient de commun. Elle est
donc postérieure à l'expérience. Pourquoi l’appelle-t-on {\it pré}notion ? Parce qu'il
faut en disposer d’abord pour pouvoir reconnaître, voyant un arbre, que c'en
est un. C’est le contraire d’une idée {\it a priori} : elle ne précède une expérience
donnée que pour autant qu’elle résulte d’autres, qui la précèdent. Ainsi c’est
l'expérience qui est à la base de tout, y compris des idées qui la pensent.

Le mot prendra plus tard, spécialement chez Bacon et Durkheim, un sens
péjoratif : la prénotion serait une idée préconçue, antérieure à toute réflexion
ou à toute enquête scientifique, et risquant par là d’en détourner. S’oppose en
ce sens à concept. Mais les concepts d’une époque deviennent très vite les prénotions
d’une autre. On n’en a jamais fini de penser, ni de se libérer de ses
idées.

Ainsi la prénotion est une idée avec laquelle (chez Épicure et les stoïciens)
ou contre laquelle (chez Bacon ou Durkheim) on pense : c’est un outil ou un
obstacle, et parfois les deux.

PRÉSAGE Le signe présent de quelque événement à venir. Ce n’est souvent
que superstition. Toutefois il arrive qu’une observation un peu
attentive et régulière débouche sur un répertoire plus ou moins fiable de signes,
associés à autant de prévisions. Ainsi en médecine, en météorologie, en économie...
Que ces gros nuages noirs, là-bas, laissent présager quelque orage, ce
n’est ni magie ni même prédiction. Ce n’est pas l’avenir que l’on voit, c’est le
présent, dont on a appris à prévoir, parfois, les effets ou les suites. Mais on ne
parle de présage, en règle générale, que pour les signes mystérieux ou irrationnels.
%— 463 —
%{\footnotesize XIX$^\text{e}$} siècle — {\it }
Ceux-là mentent presque toujours. Même quand il arrive qu’ils tombent
juste, le mieux est de les oublier aussitôt.

PRÉSENT Ce qui sépare le passé de l’avenir. Mais si le passé et l’avenir ne
sont rien, rien ne les sépare. Il n’y a plus que l'éternité, qui est le
présent même. Entre rien et rien : tout.

C’est le lieu de coïncidence du réel et du vrai, qui aussitôt se séparent (le
passé reste vrai, qui n'est plus réel) sans se perdre (puisque la vérité reste présente).
C’est peut-être l’espace lui-même, où l'univers éternellement {\it se présente.}

Être présent, c’est être ou devenir. Le présent dure, c’est-à-dire continue
d’être présent, sans cesser de changer. C’est pourquoi il y a du temps, que nous
pouvons, par la pensée, indéfiniment diviser entre passé et avenir. Le présent,
pour la pensée, est ce qui les sépare. Mais ce présent abstrait n’est alors qu’un
instant sans épaisseur : non une durée, disait Aristote, mais la limite entre deux
durées. Le réel n’en continue pas moins sans limites, et c’est le présent même :
la continuation indivisible et illimitée de tout.

On remarquera que la mémoire et l’imagination en font partie. Vivre au
présent, comme disaient les stoïciens, comme disent tous les sages, ce n’est
donc pas vivre dans l'instant. Qui peut aimer sans se souvenir de ceux qu'il
aime ? Penser, sans se souvenir de ses idées ? Agir, sans se souvenir de ses désirs,
de ses projets, de ses rêves ? Non qu’il y ait pour cela autre chose que le présent.
Qui peut aimer, penser ou agir au passé ou au futur ? Vivre au présent, c’est
simplement vivre en vérité : c’est la vie éternelle, et il n’y en a pas d’autre.
Seules nos illusions nous en séparent, ou plutôt seules nos illusions (qui en font
partie) nous donnent le sentiment d’en être séparés. « Tant que tu fais une différence
entre le nirvâna et le samsâra, disait Nâgârjuna, tu es dans le samsâra. »
Tant que tu fais une différence entre le temps et l'éternité, tu es dans le temps.
Le présent, qui est leur vérité conjointe, ou leur conjonction vraie, est donc
l'unique lieu du salut. Nous sommes déjà dans le Royaume : l'éternité, c’est
maintenant.

PRÊTRE C'est une espèce de fonctionnaire, qui servirait l’Église plutôt que
l'État, ou Dieu plutôt que la nation. Ministre du culte, donc,
plutôt que de la Cité. Il serait absurde de les juger en bloc. Même Voltaire, qui
les combat, s’y refuse. Un prêtre doit être « le médecin des âmes », écrit-il, mais
tous ne se valent pas. « Quand un prêtre dit : “Adorez Dieu, soyez juste, indulgent,
compatissant”, c’est alors un très bon médecin. Quand il dit : “Croyez-moi,
ou vous serez brûlé”, c’est un assassin. »

%— 464 —
%{\footnotesize XIX$^\text{e}$} siècle — {\it }
PREUVE Un fait ou une pensée, qui suffit à attester la vérité d’un autre fait
ou d’une autre pensée. Toutefois la preuve la plus solide ne vaut
que ce que vaut l’esprit qui s’en sert. Il faudrait donc prouver d’abord la valeur
de l'esprit, et c’est ce qu’on ne peut sans tomber dans un cercle. Ainsi il n’y a
pas de preuve absolue. Il n’y a que des expériences ou des démonstrations qui
mettent fin au doute. C’est ce qu’on appelle une preuve, ou une autre façon de
la définir : une preuve est une pensée ou un fait qui rend le doute, sur une
question donnée, impossible, sauf à douter de tout. Par quoi la logique est sans
force contre le scepticisme, comme le scepticisme contre la logique.

PRÉVISION C’est voir avant. Parce qu’on verrait l'avenir ? Bien sûr que non
(comment voir ce qui n’est pas ?) ; mais parce qu'on voit ses
signes ou ses causes, qui font partie du présent, et qu’on interprète. À ne pas
confondre avec l’espérance. Le météorologue qui annonce une tempête, cela ne
veut pas dire qu’il espère. Le touriste qui espère le beau temps, cela ne veut pas
dire qu’il le prévoit. L’espérance est fondée sur un désir ; la prévision, sur une
connaissance. Celle-ci est-elle fiable ? Cela dépend des domaines considérés. La
meilleure prévision n’est pas forcément la plus certaine ; c’est celle qui évalue sa
propre marge d'incertitude, jusqu’à prévoir les moyens, si nécessaire, de surmonter
son propre échec. C’est où connaissance et volonté se rejoignent : ce
n’est plus prévision mais stratégie.

PRIÈRE C’est parler à Dieu, le plus souvent pour lui demander quelque
chose. Mais à quoi bon lui parler, s’il sait déjà tout ? Et pourquoi
demander, s’il sait mieux que nous ce qu’il nous faut ? Le silence serait plus
digne, et aussi efficace.
On dira qu’on parle aussi, en amour, pour dire à quelqu'un ce qu’il sait
déjà et qu’il se plaît à entendre... Mais c’est qu’il a besoin d’être rassuré, conforté,
cajolé.... Ce n’est pas prière mais caresse. Et qui oserait caresser Dieu ?

PRIMAT/PRIMAUTÉ Les deux mots sont souvent synonymes : ils indiquent
ce qui vient au premier rang, ce qui est le plus
important ou a le plus de valeur. C’est pourquoi j'ai pris habitude de les distinguer.
La notion de « premier rang » n’a de sens qu’en fonction d’un ordre
que l’on suit : s’il arrive que les premiers soient les derniers, comme il est dit
dans les Écritures, c’est moins souvent du fait d’un bouleversement interne de
la série (un pauvre qui fait fortune, un riche qui se ruine) qu’en raison d’un
%— 465 —
%{\footnotesize XIX$^\text{e}$} siècle — {\it }
changement, entre les deux hiérarchies, de point de vue. Cela vaut spécialement
quand on passe de la théorie à la pratique, de la vérité à la valeur, de
l’ordre des causes à celui des fins. Ce qui est le plus important, du point de vue
de l’être ou de la connaissance, n’est pas forcément — et même, me semble-t-il,
n'est jamais — ce qui vaut le plus, du point de vue du sujet ou du jugement.
Être matérialiste, par exemple, c’est affirmer le {\it primat de la matière}. Mais en
tant que le matérialisme est une philosophie, il ne peut renoncer à la {\it primauté
de la pensée ou de l'esprit}. Marx, pour avoir affirmé le primat de l’économie, ne
s’est pas cru tenu de soumettre sa vie à l’argent : il n’était, que je sache, ni banquier
ni vénal. Et Freud, pour avoir soutenu le primat de la sexualité et de
l'inconscient, n’a pas davantage décidé de leur soumettre son existence : il
n'était ni idiot ni débauché, et mettait l’art et la lucidité plus haut que le sexe
ou les actes manqués. Au reste, c’est ce qu’expriment assez les notions d’idéologie,
chez le premier, et de sublimation, chez le second. Que l'idéologie soit,
comme le disait Marx, une {\it camera obscura}, qui donne une image inversée de la
réalité (ce qui est en haut paraît en bas, ce qui est en bas paraît en haut, comme
dans la chambre noire des anciens appareils photographiques), ce n’est bien sûr
pas une raison pour prétendre se passer d’idéologie. Et si toute valeur résulte
d’une sublimation, comme le veut Freud, on aurait assurément tort de
renoncer pour cela à l’art, à la morale ou à la politique. La notion de primat ou
de primauté doit donc être scindée en deux concepts différents, et même
opposés, qui peuvent être définis de la façon suivante : j’entends par {\it primat} ce
qui est objectivement le plus important, dans un enchaînement descendant de
déterminations ; par {\it primauté}, ce qui vaut le plus, subjectivement, dans une
hiérarchie ascendante d'évaluations. Le concept de {\it prima}t est ontologique ou
explicatif : c’est l’ordre des causes et de la connaissance, qui tend au plus profond
ou au plus fondamental ; celui de {\it primauté} est normatif ou pratique : c’est
l’ordre des valeurs et des fins, qui tend au meilleur ou au plus élevé. Le premier
sert à comprendre ; le second, à juger et à agir.

La différence, entre ces deux points de vue, est essentielle au matérialisme
philosophique. Être matérialiste, Comte a raison sur ce point, c’est expliquer le
supérieur par l’inférieur. Mais ce n’est pas pour autant renoncer à la supériorité
de celui-là, ni se mettre à adorer celui-ci. Que la pensée, par exemple,
s'explique par le cerveau, cela ne saurait justifier qu’on renonce à penser, ni
qu’on soumette toute vérité à la neurobiologie (car alors la neurobiologie elle-même
deviendrait impossible ou impensable : une idée fausse n’est pas moins
réelle, dans le cerveau, qu’une idée vraie). Que la vie s'explique par la matière
inanimée, ce n’est pas davantage une raison pour renoncer à vivre, ni pour soumettre
toute vie à la physique (car alors la notion de bioéthique perdrait son
sens, et la physique elle-même serait sans valeur : un imbécile n’est pas moins
%— 466 —
%{\footnotesize XIX$^\text{e}$} siècle — {\it }
matériel qu’un physicien). Il faut donc distinguer ce qui relève de la connaissance
ou des causes, d’une part, et ce qui relève du jugement de valeur ou de
l’action d’autre part. Cela définit deux points de vue différents : l’un théorique,
qui tend à l’objectivité ; l’autre pratique, qui doit s’assumer comme subjectif.
Ce qui est le plus important dans le premier (la matière, l’enchaînement des
causes ou des déterminations) n’est jamais ce qui vaut le plus dans le second
(l'esprit, la hiérarchie des finalités ou des valeurs), et inversement. C’est ce qui
m'a permis, dans {\it Le mythe d'Icare}, de penser le matérialisme comme essentiellement
{\it ascendant} : du {\it primat} (de la matière, de la nature, de l’économie, de la
force, de la sexualité, de l'inconscient, du corps, du réel, du monde...) à la {\it primauté}
(de la pensée, de la culture, de la politique, du droit, de l'amour, de la
conscience, de l’âme, de l’idéal, du sens....). Si l’on renonce au primat, on n’est
plus matérialiste ; si l’on renonce à la primauté, on n’est plus philosophe : on
ne défend plus qu’un matérialisme vulgaire ou avachi. Autant se mettre à
genoux, dans un cas, ou se coucher, dans l’autre.

Parce que ces deux logiques, celle du primat et celle de la primauté, s’opposent,
on a affaire à une dialectique. Reste à penser — pour qu’il y ait dialectique
et non incohérence — leur articulation. Comment passe-t-on de l’un de ces
deux points de vue à l’autre ? Entre le primat et la primauté, quoi ? Le mouvement
ascendant du désir, qui nous fait passer de l’un (primat du primat : tout
part du corps, y compris le désir lui-même) à l’autre (primauté de la primauté :
rien ne vaut, y compris le corps, que pour l'esprit). Par exemple chez Épicure :
«le plaisir du ventre, disait-il, est le fondement de tout bien » (c’est à lui que se
ramènent, selon l’ordre des causes, «les biens spirituels et les valeurs
supérieures ») ; mais les plaisirs de l'âme (l'amitié, la philosophie, la sagesse) lui
sont pourtant supérieurs : c’est d’eux, bien plus que des plaisirs corporels, que
dépendent notre dignité, notre liberté, notre bonheur. Par exemple chez Marx :
l’économie est déterminante « en dernière instance » ; mais les hommes n’agissent
qu’en fonction de la représentation idéologique qu’ils ont de la société et
d'eux-mêmes (un militant qui se bat pour le communisme, ce n’est pas la
même chose qu’un nervi qui se vend au plus offrant). Par exemple chez Freud :
la psychanalyse, comme théorie, enseigne le primat de l'inconscient et de la
sexualité, mais apprend, en pratique, à s’en libérer, au moins partiellement, au
nom de valeurs supérieures (qui résultent elles-mêmes de la sexualité, par la
sublimation, mais la jugent, par le surmoi, et tendent à s’en affranchir : aucune
cure ne peut réussir, insiste Freud, sans « l’amour de la vérité », qui peut certes
s'expliquer par l'inconscient mais vise à augmenter, autant que faire se peut,
notre part de conscience et de liberté). Bref, tout part du corps, de l’économie
ou du {\it ça}, mais c’est l'esprit, l'idéologie ou l'éducation qui fixent les valeurs
supérieures, vers lesquelles nous tendons, au moins consciemment, et qui peuvent
%— 467 —
%{\footnotesize XIX$^\text{e}$} siècle — {\it }
seules donner un sens, fât-il relatif et provisoire, à notre vie. C’est cette
{\it ascension} que le mythe d’Icare, dans mon premier livre, m’a paru pouvoir
symboliser : Icare, prisonnier d’un labyrinthe horizontal (objectivement, tout
se vaut et ne vaut rien), mais créant lui-même (primat de l’action, primauté de
l'œuvre), guidé par son père (tout désir est biographiquement déterminé :
primat de l’histoire, primauté de la fidélité), les ailes de son désir et de son vol.
« À l'encontre de la philosophie idéaliste allemande, qui descend du ciel sur la
terre, disaient Marx et Engels, {\it c'est de la terre au ciel que l'on monte ici} » ({\it L'idéologie
allemande}, 1). Encore faut-il {\it monter}, en effet : passer de la terre au ciel,
disons du {\it primat} à la {\it primauté} — du désir (comme puissance) au désirable
(comme valeur).

Ce désir est un effort ({\it conatus}), une tension, un {\it acte}. On ne passe du primat
à la primauté qu’à la condition de le vouloir. Qu'un instant l'effort se relâche,
que le désir se fatigue ou s’oublie, on n’a plus qu’un matérialisme plat ou veule
— qu'un matérialisme qui redescend, et qui ne saurait par conséquent (la sagesse
est un idéal) être philosophique.

Cette dialectique, qui vaut à l’échelle du monde, vaut aussi, plus spécialement,
à celle de la société. On sait, je m’en suis expliqué ailleurs, que j’y distingue
quatre ordres différents : l’ordre techno-scientifique (qui inclut l’économie),
l’ordre juridico-politique, l’ordre de la morale, enfin l’ordre de
l'éthique ou de l'amour. Chacun de ces ordres a sa logique et sa cohérence
propres ; mais il reste conditionné par l’ordre qui le précède, et il n’est limité et
jugé, de l'extérieur, que par l’ordre qui le suit. Aussi peut-on les disposer, sous
forme d’une topique, de bas en haut. On y retrouvera alors la dialectique du
primat et de la primauté, non suivant une simple hiérarchie ascendante, ce
serait trop simple, mais selon un double mouvement, montant et descendant,
qui forme comme deux hiérarchies croisées : ce qui vaut le plus, subjectivement,
pour l'individu, n’est jamais ce qui est le plus important, objectivement,
pour le groupe; et inversement. La hiérarchie ascendante des primautés
s'ajoute ainsi, mais sans l’annuler, à l’enchaînement descendant des primats.
Chacun de ces ordres est en effet déterminant pour l’ordre immédiatement
supérieur, dont il crée les conditions de possibilité, mais aussi régulateur pour
l’ordre immédiatement inférieur, dont il fixe les limites et auquel il essaie de
donner une orientation ou un sens. On est donc contraint, pour expliquer un
ordre, de prendre en compte les ordres inférieurs ; mais on ne peut le juger
qu’en faisant référence aux ordres supérieurs. C’est ainsi qu’on doit énoncer à
la fois (mais de deux points de vue différents) le primat de l’économie et la primauté
de la politique, entre les ordres 1 et 2 ; le primat de la politique et la primauté
de la morale, entre les ordres 2 et 3 ; enfin le primat de la morale et la
primauté de l’amour, entre les ordres 3 et 4. Toute confusion, entre ces ordres,
%— 468 —
%{\footnotesize XIX$^\text{e}$} siècle — {\it }
est ridicule, comme dirait Pascal, ou tyrannique. Mais elle peut l’être de deux
façons différentes : en soumettant un ordre donné, avec ses valeurs propres, à
un ordre inférieur (c’est ce que j'appelle la barbarie, qui renonce à la primauté),
ou en prétendant annuler ou déstructurer un ordre donné, avec ses contraintes
propres, au nom d’un ordre supérieur (c’est ce que j'appelle l’angélisme, qui
oublie le primat).

Contre l’angélisme, quoi ? La lucidité.

Contre la barbarie ? L'amour et le courage.

Par quoi la dialectique du primat et de la primauté débouche sur une
éthique, qui est à la fois intellectuelle (primauté, pour la pensée, de la lucidité :
l'amour de la vérité doit l'emporter, intellectuellement, sur tout) et pratique
(primauté, pour l’action, de ce qu’on aime et veut : l’amour et la liberté, pour
presque tous, seront les valeurs suprêmes). Cela n’autorise pas à soumettre la
pensée à l’amour ou à la liberté (ce ne serait plus philosophie mais sophistique,
fût-elle généreuse et démocratique: c’est ce qu’on appelle aujourd’hui le
« politiquement correct »), ni l’amour ou la volonté à la connaissance (ce ne
serait plus morale ou éthique mais scientisme ou dogmatisme). On ne vote pas
sur une vérité : ce serait de l’angélisme démocratique. Ni sur le bien et le mal :
ce serait de la barbarie démocratique. Mais la vérité ou la morale ne sauraient
pas davantage dire la loi, qui ne relève, dans une démocratie, que du peuple
souverain. C’est ce qu’on appelle la laïcité : la démocratie ne tient pas lieu de
conscience, ni la conscience, de démocratie.

La distinction des ordres débouche ainsi, philosophiquement, sur ce que
j'ai appelé le {\it cynisme}. De quoi s'agit-il ? D’une certaine façon de penser le rapport
entre la valeur et la vérité, sans les confondre (la valeur est sans vérité
objective, la vérité sans valeur intrinsèque), mais sans non plus renoncer à l’une
ou à l’autre. Le cynique, au sens philosophique du terme, c’est celui qui refuse
de prendre ses désirs pour la réalité, voyez Machiavel, mais aussi de céder sur la
réalité de ses désirs, voyez Diogène (et aussi Machiavel : c’est ce qu’il appelle la
{\it virt\'{u}}). Reste à penser, entre ces deux versants du cynisme, une articulation :
c’est ce que fait Spinoza, ou du moins ce qu’il autorise. Ce n’est pas parce que
la vérité est bonne qu’elle est connaissable ou aimable ; c’est parce qu'elle est
vraie que nous pouvons la connaître et, si nous en sommes capables, l'aimer. À
l'inverse, ce n’est pas parce que le bien est vrai qu’il faut l’aimer ou le faire, c’est
parce que nous l’aimons ou le voulons qu’il devient, pour nous, une valeur
(laquelle ne fonctionne comme vérité que pour et par un sujet). Par exemple
l’égoïsme est au moins aussi vrai que la générosité, et même, quant aux faits, il
l’est bien davantage (il explique un plus grand nombre de comportements, et
peut-être la générosité elle-même) ; aussi faut-il le connaître et le comprendre.
Mais cela n'empêche pas que la générosité, moralement, lui soit supérieure.

%— 469 —
%{\footnotesize XIX$^\text{e}$} siècle — {\it }
Distinction des ordres : la logique des primats explique celle des primautés,
pour la connaissance, mais ne saurait l’annuler, pour l’action, pas plus que
celle-ci ne saurait, pour la pensée, tenir lieu de celle-là. Il en résulte qu’on ne
peut jamais tout à fait vivre ce qu’on sait, ni savoir ce qu'on vit. « Nous
sommes, je ne sais comment, doubles en nous-mêmes », disait Montaigne
({\it Essais}, II, 16; voir aussi mon article sur « Montaigne cynique », {\it Valeur et
vérité}, p.55 à 104). Mais c’est Lévi-Strauss, à propos de Montaigne (et cette
rencontre n’est ni de hasard ni de faible poids), qui a su désigner le plus nettement
cet écart en nous à la fois nécessaire et tragique :

{\footnotesize « La connaissance et l’action sont à jamais placées dans une situation fausse : prises
entre deux systèmes de référence mutuellement exclusifs et qui s'imposent à elles, bien
que la confiance même temporaire faite à l’un détruise la validité de l’autre. Il nous faut
pourtant les apprivoiser pour qu’ils cohabitent en chacun de nous sans trop de drames.
La vie est courte : c’est l'affaire d’un peu de patience. Le sage trouve son hygiène intellectuelle
et morale dans la gestion lucide de cette schizophrénie » ({\it Histoire de Lynx},
chap. XVIII, « En relisant Montaigne », p. 288).
}

Schizophrénie ? Ce n’est pas le mot que j’utiliserais, puisque cet écart
résulte non d’une quelconque pathologie mais d’une dualité constitutive de
l'être humain, qui est celle du désir et de la raison, et puisque cette dualité, qui
est en l'occurrence structurante bien plus que destructrice, peut être et pensée
et voulue (elle est à la fois rationnelle et raisonnable) : on peut expliquer le désir
(@ quoi bon, autrement, les sciences humaines ?) et désirer la raison (à quoi
bon, autrement, la philosophie ?).

L'essentiel, pour résumer, tient donc bien dans la formule que j’utilisais en
commençant : primat de la matière et primauté de l'esprit. Mais cela, s’agissant
de l'être humain, doit être précisé en une autre, plus lourde mais plus explicite :
primat du corps et du désir, primauté (théorique) de la raison et (pratique) de
l’amour et de la liberté. C’est le désir, disait Spinoza, non la raison ou l’amour,
qui est l'essence de l’homme ({\it Éthique}, III). Mais c’est la raison qui libère, et
l'amour qui sauve ({\it Éthique}, IV et V). Les sciences de la nature et de l’homme,
pour le dire avec nos mots d’aujourd’hui, nous en apprennent davantage sur
nous-mêmes que la morale, qu’elles expliquent et qui ne les explique pas. Mais
connaître la vie ou l’humanité n’a jamais suffi à les aimer, ni même à aimer la
connaissance. À la gloire du spinozisme. Ce n’est pas parce qu’une chose est
bonne que nous l’aimons ou la désirons, c’est inversement parce que nous
l’aimons ou la désirons que nous la jugeons bonne ({\it Éthique}, III, 9, scolie : voir
aussi, chez Freud, les deux dernières pages de {\it Malaise dans la civilisation}). Si
vous n’aimez pas l’amour et la vérité, n’en dégoûtez pas les autres. Mais si vous
%— 470 —
%{\footnotesize XIX$^\text{e}$} siècle — {\it }
aimez les deux, comme il convient, ne vous croyez pas autorisé par là à les
confondre.

PRINCIPE Un commencement théorique : le point de départ d’un raisonnement.
Il est de la nature d’un principe d’être indémontrable
(sans quoi ce ne serait plus un principe mais un théorème ou une loi), comme
il est de la nature de la démonstration de requérir quelque principe indémontré.
La différence avec un axiome ou un postulat ? C’est que ceux-ci ne se
disent guère que de systèmes formels ou hypothético-déductifs. Un principe se
dit aussi bien dans les sciences expérimentales, en morale ou en politique. Reste
à savoir pourquoi poser tel principe plutôt que tel autre. C’est parfois qu’on ne
peut faire autrement (le principe de non-contradiction) ou qu’on y voit une
espèce d’évidence. D’autres fois qu’on en a besoin pour agir ou vivre d’une
façon qui nous paraisse humainement acceptable. Les principes de la morale
sont de ce type : nullement évidents ni logiquement nécessaires, mais subjectivement
indispensables.

PRIVATION L'absence de quelque chose qu’on devrait avoir : l’aveugle,
non la pierre, est privé de la vue.

PROBABILITÉ Un degré de possibilité, en tant qu’il peut faire l’objet d’un
calcul ou d’une prévision. Se dit surtout, dans le langage
courant, quand ce degré est élevé. Mais parler de {\it probabilité infime} n’est pas
pour autant contradictoire. Par exemple si vous jouez au loto : il est très improbable
que vous gagniez, mais cette probabilité, qui peut se calculer très précisément,
n’est pas nulle. Elle est simplement trop faible, même par rapport à
l'enjeu, pour que le seul calcul des probabilités puisse justifier votre mise. Heureusement,
pour le Trésor public, que le désir s’en mêle.

PROBLÉMATIQUE Comme adjectif, qualifie ce qui n’est ni réel (ou,
s'agissant d’un jugement, assertorique) ni nécessaire
(ou apodictique). « Les jugements sont {\it problématiques}, écrit Kant, lorsqu'on
admet l’affirmation ou la négation comme simplement {\it possibles} (il y a choix),
{\it assertoriques} quand on les y considère comme {\it réelles} (vraies), {\it apodictiques} quand
on les y regarde comme {\it nécessaires} » ({\it C. R. Pure}, Analytique des concepts, I, 2,
\S 9).

%— 471 —
%{\footnotesize XIX$^\text{e}$} siècle — {\it }
Comme substantif, le mot désigne l'élaboration d’un problème. Construire
une problématique, c’est expliquer {\it comment} un problème se pose, ou comment
on a décidé de le poser, afin d’avoir une chance, peut-être, de le résoudre. Dans
une dissertation philosophique, la problématique doit idéalement apparaître
dès la fin de l’introduction ; elle prend communément la forme d’un système
ordonné de questions.

PROBLÈME Une difficulté à résoudre. Une question ? C’est ordinairement
la forme que prend pour nous un problème, ou plutôt que
nous lui donnons. D’où ce passage fameux de Bachelard : « Avant tout, il faut
savoir poser des problèmes. Et, quoi qu’on dise, dans la vie scientifique, les problèmes
ne se posent pas d’eux-mêmes. C’est précisément ce {\it sens du problème}
qui donne la marque du véritable esprit scientifique. Pour un esprit scientifique,
toute connaissance est une réponse à une question. S’il n’y a pas eu de
question, il ne peut y avoir connaissance scientifique. Rien ne va de soi. Rien
n'est donné. Tout est construit » ({\it La formation de l'esprit scientifique}, I). Cela
vaut aussi en philosophie, et sans doute pour toute pensée digne de ce nom.
Mais toute question n’est pas un problème. Par exemple quelqu'un vous
demande l'heure : c’est une question, pas un problème. Si vous lui demandez :
« Pourquoi voulez-vous savoir l’heure qu’il est ?», vous transformez — en
l'occurrence indûment — sa question en problème. Il pourrait vous le
reprocher : « Pourquoi faites-vous un problème de ma question ? » En philosophie,
c’est différent. Seuls les problèmes comptent, qu’il faut poser avant de les
résoudre. Qu'est-ce que poser un problème ? C’est expliquer {\it pourquoi} une
question se pose, et {\it doit} se poser, non à tel ou tel individu, mais pour tout
esprit raisonnable fini, doué d’une culture au moins minimum. Tel est le but
de l'introduction, dans une dissertation philosophique : il s’agit de passer de la
contingence d’une question à la nécessité d’un problème, avant d’élaborer, si
possible, une problématique.

PROCHAIN  Autrui, tel qu’il se donne dans la rencontre. Il a droit à plus
qu’à du respect. À quoi bon, autrement, la rencontre ?

PROFONDEUR Ia distance entre la surface et le fond. En philosophie,
c’est surtout une métaphore, pour indiquer la quantité de
pensée qu’un discours peut contenir ou susciter. Même Nietzsche, si amoureux
de la surface et de la belle apparence, y voyait légitimement une vertu intellectuelle.
%— 471 —
%{\footnotesize XIX$^\text{e}$} siècle — {\it }
C’est que la superficialité elle-même n’a de sens qu’au service de la profondeur.
Ainsi, chez les Grecs : « Ah ! ces Grecs, comme ils savaient vivre ! Cela
demande la résolution de rester bravement à la surface, de s’en tenir à la draperie,
à l’épiderme, d’adorer l'apparence et de croire à la forme, aux sons, aux
mots, à tout l’Olympe de l'apparence ! Ces Grecs étaient superficiels... par
profondeur ! » ({\it Le gai savoir}, Avant-Propos). Combien, aujourd’hui, faisant le
chemin inverse, voudraient sembler profonds à force de superficialité ?

D’autres voudraient obtenir le même résultat à force d’obscurité. Nietzsche,
dont ils se réclament parfois, leur a pourtant donné tort. Il préférait « la belle
clarté française », celle de Pascal ou de Voltaire. {\it Être profond}, explique-t-il, n’est
pas la même chose que {\it sembler profond} : « Celui qui se sait profond s’efforce
être clair; celui qui aimerait sembler profond à la foule s'efforce d’être
obscur. Car la foule croit profond tout ce dont elle ne peut voir le fond. Elle a
si peur ! elle aime si peu aller dans l’eau ! » ({\it op. cit}, III, 173). Ainsi la superficialité
n’est bonne qu’à condition d’être profonde ; et la profondeur, qu'à
condition d’être claire.

PROGRÈS Dans mon premier livre, {\it Le mythe d'Icare}, j'insistais sur la relativité
de l’idée de progrès. Marcel Conche, qui avait bien voulu
lire le manuscrit, inscrivit simplement en marge : « Quel progrès, pourtant, que
la Sécurité sociale ! » Il avait évidemment raison. Un progrès relatif reste un
progrès, et il n’y en a pas d’autre.

Qu'est-ce que le progrès ? Un changement vers le mieux. Notion normative,
donc subjective. Il n’y a guère que dans les sciences que le progrès, pour
relatif qu’il demeure, soit incontestable : parce que la science d’aujourd’hui
peut rendre compte de celle des siècles passés, quand la réciproque n'est pas
vraie. C’est ce qui fait que l’histoire des sciences est «une histoire jugée »,
comme disait Bachelard, et jugée par son progrès même, lequel est « démontrable
et démontré » : c’est « une histoire récurrente, une histoire qu’on éclaire
par la finalité du présent, une histoire qui part des certitudes du présent et
découvre, dans le passé, les formations progressives de la vérité » ({\it L'activité
rationaliste de la physique contemporaine}, chap. I). En politique, il en va
autrement : juger le passé au nom du présent n’est pas plus légitime (ce qui ne
veut pas dire que ce soit évitable) qu’il le serait de juger le présent au nom du
passé. Relativisme sans appel, donc, comme dit Lévi-Strauss, et c’est aussi vrai
dans le temps que dans l’espace: non qu’on ne puisse juger (on le peut,
puisqu'on le fait, et qu’on le doit), mais parce qu’on ne peut le faire objectivement
ou absolument. Un réactionnaire, par exemple, n’est pas quelqu'un qui
est contre le progrès, comme le croient naïvement les progressistes, mais
%— 473 —
%{\footnotesize XIX$^\text{e}$} siècle — {\it }
quelqu'un qui juge que c’en serait un, voire le seul possible, que de revenir à
telle ou telle situation antérieure. Et comment lui démontrer qu’il a tort ? On
connaît des cas, notamment en médecine, où une {\it involution} serait seule favorable.
Guérir, c’est le plus souvent revenir à la situation antérieure, ou s’en rapprocher.
Et qui ne souhaiterait rajeunir ? Ce n’est pas une raison pour souhaiter
retomber en enfance, ni, encore moins, pour revenir à l’Ancien régime. Le progrès
(social, politique, économique...) n’est ni linéaire ni absolu. Il n’est progrès,
même, que relativement à certains désirs qui sont les nôtres (de bien-être,
de justice, de liberté...). Cela, qui lui interdit de prétendre à l'absolu, ne
l’'annule pas ; c’est au contraire ce qui fait sa réalité, pour ceux — presque tous
— qui partagent ces désirs et constatent, malgré tant d’horreurs qui demeurent,
quelques avancées. Le progrès n’est pas une providence ; c’est une histoire, et
un point de vue sur cette histoire. C’est où l’on retrouve la Sécurité sociale, les
Lumières, les droits de l’homme, et même l'enthousiasme, mais guéri d’utopie,
de notre jeunesse : {\it Ce n'est qu'un début, continuons le combat !}

PROGRESSISTE Ce n’est pas quelqu'un qui est pour le progrès (personne
n’est contre), mais quelqu'un qui pense que le progrès —
social, politique, économique — est la tendance normale de l’histoire : que le
présent est globalement supérieur au passé, comme l'avenir, sauf catastrophe,
sera supérieur au présent. Aussi veut-il aller de l’avant : c’est ce qu’on appelle
« avoir des idées avancées ». Le progressisme est un optimisme, sans doute le
plus légitime qui soit : le progrès des sociétés est plus probable, et mieux avéré,
que celui des individus ou des civilisations.

PROJET Un désir présent, portant sur l’avenir en tant qu’il dépend de
nous. Ce n’est pas encore une volonté (vouloir c’est faire), ou
plutôt ce n’est que la volonté (actuelle) de vouloir (plus tard). Y voir la source
d’une liberté absolue, comme le fait Sartre, c’est oublier qu’un projet, en tant
qu'il est actuel, est aussi réel — donc aussi nécessaire — que le reste.

PROPHÈTE Celui qui parle {\it (phanai)} à la place de ou en avance {\it (pro)}. Cela
ressemble à une maladie. Les croyants y voient un miracle.
«Il faut convenir que c’est un méchant métier que celui de prophète », écrit
Voltaire. Le succès est rien moins qu’assuré. Cet homme qui parle au nom de
Dieu, est-ce un prophète ou un fou ? Un visionnaire ou un fanatique ? Et comment
savoir s’il dit vrai? « La prophétie est un art difficile, dira plus tard
%— 474 —
%{\footnotesize XIX$^\text{e}$} siècle — {\it }
Woody Allen, surtout lorsqu'elle porte sur le futur. » Mieux vaut s'occuper du
présent, et préparer l'avenir plutôt que le prophétiser.

PROPOSITION Un énoncé élémentaire, en tant qu’il peut être vrai ou
faux. « Tout discours n’est donc pas une proposition, souligne
Aristote, mais seulement le discours dans lequel réside le vrai ou le faux,
ce qui n'arrive pas dans tous les cas : ainsi la prière est un discours, mais elle
n'est ni vraie ni fausse » ({\it De l'interprétation}, 4).

PROPRIÉTÉ Ce qui est propre à un individu ou à un groupe, autrement dit
ce qui leur appartient. Se dit spécialement, en droit, d’une
possession légitime, en principe garantie par la loi. Se distingue par là de la {\it possession},
qui n’est qu'un état de fait. Voyez Rousseau, {\it Contrat social}, X, 8-9.

PROTOCOLE La mise en scène codifiée d’une hiérarchie. On peut le respecter,
et même c’est ordinairement le plus simple, à condition
de ne pas y croire. L'essentiel, par définition, est ailleurs.

PROVIDENCE C'est le nom religieux du destin : l’espérance comme ordre
du monde.

PRUDENCE On évitera de la réduire au simple évitement des dangers,
{\it a fortiori} à je ne sais quelle lâcheté intelligente ou calculatrice.
Et on ne la confondra pas davantage, malgré Kant, avec la simple habileté
égoïste. La prudence, au sens philosophique du terme, est l’une des quatre
vertus cardinales de l'Antiquité et du Moyen Âge, sans laquelle les trois autres
(le courage, la tempérance, la justice) resteraient aveugles ou indéterminées :
c’est l’art de choisir les meilleurs moyens, en vue d’une fin supposée bonne. Il
ne suffit pas de vouloir la justice pour agir justement, ni d’être courageux, tempérant
et juste pour bien agir (puisqu'on peut se tromper dans le choix des
moyens). Voyez la politique. La plupart de nos gouvernants veulent le bien du
pays et le nôtre. Mais {\it comment} le faire ? C’est ce qui les oppose, et nous oppose.
Voyez les parents. Presque tous veulent le bien de leurs enfants. Cela n’a jamais
suffi, hélas, pour être de bons parents ! Encore faut-il savoir comment élever ses
enfants, comment faire en effet leur bien, ou comment les aider à faire le leur.

%— 475 —
%{\footnotesize XIX$^\text{e}$} siècle — {\it }
Qu’on le veuille, c’est la moindre des choses. Mais par quel chemin y parvenir ?
La vraie question, presque toujours, porte sur les moyens, non sur la fin. Que
faire, et comment ? C’est ce que l’amour voudrait savoir et ne suffit pas à déterminer.
Aimer, cela ne dispense pas d’être intelligent. C’est ce qui rend la prudence
nécessaire. Vertu intellectuelle, disait Aristote : c’est l’art de vivre et
d’agir le plus {\it intelligemment} possible.

C’est où l’on rencontre le sens ordinaire du mot. La bêtise, presque toujours,
est dangereuse. Nos politiques le savent bien. Nos militaires le savent
bien. Tous désirent la victoire ; mais cela ne tient lieu ni de tactique ni de stratégie.
Même chose pour nos industriels ou nos commerçants : tous désirent le
profit; cela ne leur dit pas comment y parvenir. Même chose pour nos
médecins : tous désirent la guérison ; cela ne leur dit pas comment l’atteindre
ou y contribuer. La prudence ne délibère pas sur les fins, remarquait Aristote,
mais sur les moyens. Elle ne choisit pas le but ; elle indique le chemin, quand
aucune science ou technique n’y suffit. C’est une espèce de sagesse pratique
{\it (phronèsis)}, sans laquelle aucune sagesse vraie {\it (sophia)} ne serait possible. « La
prudence, soulignait Épicure, est plus précieuse même que la philosophie : c’est
d’elle que proviennent toutes les autres vertus » ({\it Lettre à Ménécée}, 132) et la
philosophie elle-même. D’où provient la prudence ? De la raison (qui choisit
les moyens), lorsqu'elle se met au service du désir (qui fixe les fins). La prudence
ne règne pas (elle n’a de sens qu’au service d’autre chose), mais elle gouverne.
Elle ne remplace aucune autre vertu ; mais elle les dirige toutes, dans le
choix des moyens (voir saint Thomas, {\it Somme théologique}, Ia-IIæ, quest. 57,
art. 5, et 61, art. 2). Ce n’est pas la vertu la plus haute ; mais c’est (avec le courage)
l’une des plus nécessaires.

PSYCHANALYSE C’est à la fois une technique et une théorie. La technique
est fondée sur un certain usage de la parole (les associations
libres) et de la relation duelle (le transfert), dans un dispositif spatio-temporel
particulier (le cabinet, le divan, les séances). La théorie porte sur
l’ensemble de la vie psychique, en tant qu’elle est dominée par l’inconscient
et la sexualité. Le but est moins le bonheur que la santé ou la liberté : il s’agit
de rendre à l'individu son histoire, pour l’en libérer, au moins en partie, ou
en tout cas pour qu'il cesse d’en être aveuglément prisonnier. De là une thérapeutique,
pour les névrosés; une tentation, pour les curieux ou les
narcissiques ; et un métier, pour les psychanalystes. Il faut bien que tout le
monde vive.
Freud, qui fonda la chose et inventa le mot, dut être déçu, il le laisse parfois
entendre, par la répétitivité et la platitude de ce que la psychanalyse découvrait,
%— 476 —
%{\footnotesize XIX$^\text{e}$} siècle — {\it }
grâce à lui, en l’être humain. On comprend que les psychanalystes s’endorment,
parfois, pendant les séances. La psychanalyse est une blessure narcissique.
On se serait cru plus intéressant.

Mais c’est là son prix, et la grande leçon qu’elle nous donne. Nous sommes
le résultat d’une histoire sans intérêt. Qui s’en rend compte et l’accepte, il passe
à autre chose. Et la cure est finie.

PSYCHOLOGIE L'étude du psychisme, mais considérée plutôt comme discipline
objective et expérimentale («à la troisième per-
sonne », comme on dit parfois, par différence avec l’introspection, qui se fait à
la première, et avec la psychanalyse, qui suppose la deuxième). En son sommet,
c’est une science humaine, certes plurielle (il y a plusieurs écoles, plusieurs
méthodes, plusieurs doctrines, parfois incompatibles), mais pas plus peut-être
que l’histoire ou la sociologie. Reste à savoir à quoi elle sert. À connaître, ou à
manipuler ? À libérer, ou à instrumentaliser ? De là ce conseil d’orientation,
que Canguilhem, dans un texte fameux, donnait aux psychologues : « Quand
on sort de la Sorbonne par la rue Saint-Jacques, on peut monter ou descendre ;
si l’on va en montant, on se rapproche du Panthéon, qui est le Conservatoire
de quelques grands hommes ; mais si l’on va en descendant, on se dirige sûrement
vers la Préfecture de Police » (« Qu'est-ce que la psychologie ? », in {\it Études
d'histoire et de philosophie des sciences}, Vrin 1970, p. 381). Il y a une troisième
solution, qui est de ne pas quitter la Sorbonne. C’est la plus confortable, et la
plus ennuyeuse.

PSYCHOLOGISME C'est vouloir tout expliquer — y compris la logique ou
la raison — par la psychologie. Mais alors la psycho-
logie elle-même ne serait qu’un effet parmi d’autres du psychisme : non une
science mais un symptôme, voire une maladie, comme disait Karl Kraus de la
psychanalyse, qui se prend pour son propre remède. On n’y échappe que par
un rationalisme strict, qui refuse de soumettre la vérité à quelque causalité
externe que ce soit. Tout mensonge relève de la psychologie. Toute erreur peut
en relever. Mais quand un individu énonce une vérité, il entre, au moins de ce
point de vue, dans un autre ordre : on peut demander à la psychologie d’expliquer
comment il la connaît ou pourquoi il éprouve le besoin de la dire, mais
assurément pas pourquoi elle est vraie. Expliquer une idée (comme fait psychique)
ne saurait suffire à la juger (comme vérité). Ou bien il n’y a plus de
vérité du tout, ni donc de psychologie.

%— 477 —
%{\footnotesize XIX$^\text{e}$} siècle — {\it }
PSYCHOSE Voir « Névrose/psychose ».

PSYCHOSOMATIQUE Ce qui concerne à la fois l'esprit {\it (psuchè)} et le
corps {\it (soma)}, ou ce qui relève de leur interaction.
La notion n’a de sens que si l'esprit et le corps sont deux choses différentes. Ce
n’est donc pas le contraire du dualisme, comme on le croit parfois, mais sa version
naïvement médicale et moderniste : c’est un dualisme mou.

PUDEUR La vertu qui cache. Cela suppose une envie de tout montrer (sans
quoi la pudeur ne serait pas une vertu), et c’est ce qui rend la
pudeur particulièrement troublante — par le trouble qu’elle manifeste en voulant
l’éviter. Littré la définit comme une « honte honnête » ; ce paradoxe (la
honte de ce qui n’est pas honteux) fait partie de son charme. La pudeur va plus
loin que la décence, ou plus profond : elle relève moins des convenances que de
la délicatesse, moins de la société que de l’individu, moins de la politesse que
de la morale. C’est une façon de se protéger — et de protéger l’autre — contre le
désir qu’on suscite ou qu’on ressent. Seuls les amants peuvent s’en passer.

PUISSANCE Une force qui s’exerce (puissance en acte : {\it energeia}) ou qui
peut s'exercer (puissance en puissance : {\it dunamis}). Les deux, au
présent, sont une seule et même chose : « toute puissance est acte, active, et en
acte », disait Deleuze à propos de Spinoza, et il n’y a rien d’autre que la puissance.
C’est l’être même, en tant qu’il est puissance d’être ({\it conatus}, force,
énergie).

Nietzsche voit dans la {\it volonté de puissance} « l'essence la plus intime de
l'être ». C’est une force d’affirmation, de création, de différenciation, qui fait
du plaisir et de la douleur « comme des faits cardinaux » (« tout accroissement
de puissance est plaisir, tout sentiment de ne pouvoir résister, de ne pouvoir
dominer est douleur », {\it La volonté de puissance}, éd. Wurzbach-Bianquis, I, 54).
Un vouloir-vivre, comme chez Schopenhauer ? Non pas. « Il n’y a de volonté
que dans la vie, reconnaît Nietzsche, mais cette volonté n’est pas vouloir vivre ;
en vérité, elle est volonté de dominer », et d’abord de se dominer soi : « Tout
travaille à se surpasser sans cesse » ({\it Zarathoustra}, II, « De la maîtrise de soi »).
Telle est la volonté de puissance. C’est une espèce de {\it conatus} (on sait que
Nietzsche s’est reconnu en Spinoza «un prédécesseur »), mais qui tendrait
moins à la conservation de soi-même qu’à son dépassement, qu’à l'extension,
dût-elle être fatale, de sa propre puissance : la tendance de tout être, non à persévérer
%— 478 —
%{\footnotesize XIX$^\text{e}$} siècle — {\it }
dans son être, comme le voulait Spinoza, mais à le « surmonter », mais
à « manifester sa puissance » et à l’accroître ({\it op. cit.}, II, \S 42-50 ; {\it Le gai savoir},
V, 349). On évitera de n’y voir qu’une apologie de la violence ou de
l’expansionnisme : la {\it puissance} qu’évoque Nietzsche est celle du créateur davantage
que du conquérant. Ce n’est pas un but, c’est une force (la puissance n’est
pas ce que veut la volonté, disait Deleuze, mais {\it ce qui} veut en elle, {\it Nietzsche et
la philosophie}, III, 6).

La proximité avec Spinoza est plus grande qu’il n’y paraît, et que Nietzsche
ne l’a cru. Ce dernier ne voyait dans le {\it conatus} spinoziste qu’une tendance
purement conservatrice et défensive, contre quoi lui-même se singulariserait en
pensant la volonté de puissance comme positive, affirmative et créatrice. C’était
méconnaître que la puissance, chez Spinoza aussi, est « affirmation absolue de
l'existence » ({\it Éthique}, I, 8, scolie) ; tel est le sens de la {\it causa sui} (dont nous relevons,
puisque nous faisons partie de la nature : « la puissance de l’homme est
une partie de la puissance infinie », IV, 4, dém.) et de la vie. La « puissance
d’exister et d’agir », pour Spinoza, est bien autre chose qu’un simple instinct de
conservation. Il ne s’agit pas seulement de résister à la mort, mais d’exister,
d’agir et de se réjouir le plus possible. La joie est une {\it augmentation} de puissance,
et c’est la joie qui est bonne. Il reste, c’est une vraie différence entre nos
deux penseurs, que « l’effort pour se conserver » est bien indissociable, pour
Spinoza, du désir « d’être heureux, de bien agir et de bien vivre » ({\it Éthique}, IV,
21, 22 et dém.). Spinoza, lui, n'aurait pas conseillé de « vivre dangereusement »...

PULSION Une force vitale ou innée, mais sans le savoir qui va avec. Se distingue
par là de l'instinct, qui est comme un mode d'emploi
génétiquement programmé. La pulsion sexuelle, par exemple, ne suffit pas à
l'érotisme. Ni l'érotisme, à la pulsion.

PURETÉ Ce qui est sans tache ou sans mélange. Par exemple une eau
pure : c’est une eau qui n’est mêlée à rien, souillée par rien, qui
ne comporte rien d’autre qu’elle-même, une eau qui n’est que de l’eau. C’est
donc une eau morte et improbable : la pureté n’est ni naturelle ni humaine.
Par métaphore, on entend aussi par pureté une certaine disposition de
l'individu, quand il fait preuve de désintéressement. Par exemple un artiste, un
savant où un militant : dire que ce sont des purs, c’est dire qu’ils mettent leur
art, leur science ou leur cause plus haut que leur carrière ou que leurs intérêts
égoïstes. Cela culmine dans le {\it pur amour}, tel que Fénelon l’a pensé: c’est
% — 479 — 
%{\footnotesize XIX$^\text{e}$} siècle — {\it }
l'amour désintéressé, celui qui n’espère rien, en tout cas pour soi, celui dans
lequel « on s’oublie et se compte pour rien ». C’est en quoi le plaisir peut être
pur parfois (la {\it pura voluptas} de Lucrèce : quand il n’y a plus que le plaisir), ce
que la frustration n’est jamais.

Il y a un lac radioactif, au nord de l’Oural, stérilisé par des déchets
nucléaires. Les eaux y sont d’un bleu très pur, et pourtant sans vie aucune. Mais
ce « pourtant » est de trop : la propreté et la mort vont ensemble, et toute vie
est impure. Stériliser, c’est tuer ; cela en dit long sur la vie.

J'ai passé cet été quelques jours chez un ami, dans un coin charmant et
perdu des Alpes. Il me montre sa piscine, remplie à l’eau de pluie : l’eau y est
d’un vert glauque, opaque, avec d’inquiétantes suspensions. Je fais la moue,
et mon ami sent bien que, malgré la chaleur, j'hésite à plonger. « Ce n’est rien,
me dit-il, ce sont des algues, des micro-organismes. Attends un peu, tu vas
voir ! » Et de verser dans la piscine une bonne ration d’eau de Javel.. Quelques
minutes plus tard, de fait, l’eau s’était éclaircie. Le lendemain, elle était comme
neuve : nous y primes quelques bains joyeux et confiants... La vie avait reculé,
et cela nous parut un progrès décisif vers la propreté. Pourquoi non ? N'est-ce
pas ainsi qu’on nettoie les chambres d’hôpital et, en effet, les piscines ? Mais
chacun en sent bien aussi les limites et les dangers. Tout ce qui vit salit ; tout
ce qui nettoie tue. Demandez un peu aux microbes ce qu’ils pensent du savon.
Et à la ménagère maniaque, ce qu’elle pense des enfants.

Toute vie est impure, disais-je, et l’on ne saurait, sans tomber dans une
idéologie mortifère, lui préférer la pureté. Une chambre d’hôpital, ce n’est pas
un modèle de société, ni même un modèle de chambre. D'ailleurs les germes,
de plus en plus résistants, finissent par s’y glisser quand même, qui produisent
alors, plusieurs malades en sont morts, d’étonnants ravages. D'où je tirerais
volontiers une conclusion politique. La santé d’un peuple n’a jamais tenu à sa
pureté, qu'elle soit ethnique ou morale, mais seulement à sa capacité d’absorber
les mélanges, de maintenir, entre toutes ses composantes, un équilibre instable
mais vivant (vivant donc instable), enfin de gérer, dans l’à-peu-près, leurs différences
ou leurs conflits. Sans donner à cette métaphore biologisante plus de
valeur qu’elle n’en mérite (un peuple n’est pas un organisme, un individu n’est
pas un germe), on peut du moins réfléchir à ce lac de l'Oural, limpide et
mort comme un rêve d'ingénieur ou de tyran. On a parlé de « purification
ethnique », dans l’ex-Yougoslavie : qu’était-ce d’autre qu’une justification des
déportations ou des massacres ? Plusieurs rêvent d’une France propre, stérile et
pure comme un lac atomique, d’ailleurs artificielle comme lui ({\it pure}, la France
ne l’a jamais été) et comme lui promise à la mort immaculée.. Puissent-ils
songer de temps en temps au petit lac de l'Oural, d’un bleu si pur et si
transparent !

%— 480 —
%{\footnotesize XIX$^\text{e}$} siècle — {\it }
De quoi l’on pourrait tirer aussi bien, et peut-être mieux, une conclusion
morale, qui serait de vigilance contre la morale. La voilà de retour, dit-on, et
c’est tant mieux. J'ai assez bataillé contre le nihilisme et la veulerie pour ne pas
m'en plaindre. Mais la morale est comme l'hygiène : elle est au service de la vie
ou ce n’est qu’une manie dangereuse. C’est ce qui distingue la morale du moralisme,
et les braves gens des censeurs. Qu'est-ce que {\it l'ordre moral}, si ce n’est la
volonté d’inverser cette hiérarchie, de mettre la vie au service de la morale, de
{\it telle} morale, et d’en chasser l’impur ? Rêve fou : rêve de mort. S’il y a une
pureté de l’âme, elle est à l'opposé, et c’est ce que Simone Weil avait vu : « {\it La
pureté}, disait-elle, {\it est le pouvoir de contempler la souillure.} » Je dirais plus : de
l’accepter, de l’'habiter, de la vivre. L'âme est ce qui accueille le corps, et s’y
recueille. Sans honte. Sans frayeur. Sans mépris.

Ainsi, devant l’obscène du désir, la pureté de l'amour.

PUSILLANIMITÉ Du latin {\it pusilla anima}, où {\it pusillus animus}, petite âme,
esprit étroit ou étriqué. C’est le contraire de la
magnanimité : un nom savant pour dire un mélange de bassesse et de petitesse.
Se reconnaît surtout au manque de courage. C’est qu’il n’y a pas de grandeur
sans risque.

PYRRHONISME La doctrine de Pyrrhon, pour autant qu’on puisse la
reconstituer et que ce soit une doctrine (il n’a rien écrit,
ni rien affirmé absolument). Il tenait toutes choses pour « également indifférentes,
immesurables, indécidables », nous dit Aristoclès : aussi convient-il
selon lui « d’être sans jugement, sans inclination d’aucun côté, inébranlable, en
disant de chaque chose qu’elle n’est pas plus qu’elle n’est pas, ou qu’elle est et
n’est pas, ou qu’elle n’est ni n’est pas. Pour ceux qui se trouvent dans ces dispositions,
ce qui en résultera c’est d’abord l’aphasie, puis l’ataraxie » (Aristoclès,
cité par Eusèbe, {\it Prép. Évang.} XIX, 18 ; pour l'interprétation, voir Marcel
Conche, {\it Pyrrhon ou l'apparence}, PUF, 1994). Philosophie du silence, qui ne
peut s’énoncer sans se détruire. C’est peut-être le nihilisme le plus radical qui
ait jamais été pensé. Mais peut-on l’habiter ?

Dans les textes modernes, le mot a souvent un sens beaucoup plus général :
c’est un autre nom, spécialement chez Montaigne et Pascal, du scepticisme.
« La profession des pyrrhoniens est de branler, douter et enquérir, ne s'assurer
de rien, de rien ne se répondre » ({\it Essais}, II, 12) ; c’est « le plus sage parti des
philosophes » (IE, 15). La plupart, pourtant, ne s’en réclament pas. Cela même
donne raison aux pyrrhoniens, par l'impossibilité où ils se savent de le prouver :

%— 481 —
%{\footnotesize XIX$^\text{e}$} siècle — {\it }
« Rien ne fortifie plus le pyrrhonisme que ce qu’il y en a qui ne sont point pyrrhoniens.
Si tous l’étaient, ils auraient tort » ({\it Pensées}, 33-374 ; voir aussi les
fragments 131-434 et 521-387).

Le problème est alors d’assumer ce scepticisme sans tomber pour autant
dans le nihilisme ou la sophistique : Montaigne, Hume, Marcel Conche.
%
%{\footnotesize XIX$^\text{e}$} siècle — {\it }

