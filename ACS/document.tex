\documentclass[12pt, a4paper]{report}
%\documentclass[11pt, a4paper]{article}

%====================== PACKAGES ======================


\usepackage[french]{babel}

\frenchbsetup{StandardLists=true}
\usepackage{enumitem}
\usepackage{pifont}

\usepackage[utf8x]{inputenc}
%\usepackage[latin1]{inputenc} % Feynman ?

%pour gérer les positionnement d'images
\usepackage{float}
\usepackage{amsmath}
\DeclareMathOperator{\dt}{dt}
\usepackage{graphicx}
\usepackage{tabularx}
\usepackage[colorinlistoftodos]{todonotes}
\usepackage{url}
%pour les informations sur un document compilé en PDF et les liens externes / internes
\usepackage[pdfborder=0]{hyperref}
\hypersetup{
	colorlinks = true
	}
%pour la mise en page des tableaux
\usepackage{array}
\usepackage{tabularx}
\usepackage{multirow}
\usepackage{multicol}
\setlength{\columnsep}{50pt}
%pour utiliser \floatbarrier
%\usepackage{placeins}
%\usepackage{floatrow}
%espacement entre les lignes
\usepackage{setspace}
%modifier la mise en page de l'abstract
\usepackage{abstract}
%police et mise en page (marges) du document
\usepackage[T1]{fontenc}
\usepackage[top=2cm, bottom=2cm, left=2cm, right=2cm]{geometry}
%Pour les galerie d'images
\usepackage{subfig}

\usepackage{pdfpages}
\usepackage{tikz}
%\usepackage{tikz}
\usetikzlibrary{trees}
\usetikzlibrary{decorations.pathmorphing}
\usetikzlibrary{decorations.markings}
\usetikzlibrary{decorations.pathreplacing,calligraphy}
%\usetikzlibrary{decorations}
\usetikzlibrary{angles, quotes}
\usepackage{verbatim}

\usepackage{appendix}

\usepackage{comment}

\usepackage{xcolor}


%\PreviewEnvironment{tikzpicture}
%\setlength\PreviewBorder{0pt}%

%====================== INFORMATION ET REGLES ======================

%rajouter les numérotation pour les \paragraphe et \subparagraphe
\setcounter{secnumdepth}{4}
\setcounter{tocdepth}{4}

\hypersetup{							% Information sur le document
pdfauthor = {Stephan Runigo},			% Auteurs
pdftitle = {Documentation},			% Titre du document
pdfsubject = {Documentation},		% Sujet
pdfkeywords = {Document},	% Mots-clefs
pdfstartview={FitH}}	% ajuste la page à la largeur de l'écran
%pdfcreator = {MikTeX},% Logiciel qui a crée le document
%pdfproducer = {} % Société avec produit le logiciel
%======================== DEBUT DU DOCUMENT ========================
%
\begin{document}
%
%régler l'espacement entre les lignes
\newcommand{\HRule}{\rule{\linewidth}{0.5mm}}
%
% Titre, résumé, ... %
%%
%
\begin{titlepage}
%
~\\[1cm]

\begin{center}
%\includegraphics[scale=0.5]{./presentation/chambreABulle}
\end{center}

\textsc{\Large }\\[0.5cm]

% Title \\[0.4cm]
\HRule

\begin{center}
{\huge \bfseries Spiritualisme et\\
matérialisme\\[0.4cm] }
\end{center}

\HRule \\[1.5cm]


\vfill

\hfill
\begin{minipage}{0.4\textwidth}
\begin{flushright} \large
%\emph{Auteur:}\\
%Stephan \textsc{Runigo}
Extraits de dictionnaires et d'encyclopédies
\end{flushright}
\end{minipage}

\vfill
{\sf \footnotesize
\begin{itemize}[leftmargin=1cm, label=\ding{32}, itemsep=1pt]
\item {\bf Objet : } Étudier les concepts liés au spiritualisme et au matérialisme.
\item {\bf Contenu : } Définitions et philosophie encyclopédique.
\item {\bf Public concerné : } Néophyte.
\end{itemize}
}

\vfill

% Bottom of the page
{\large \today}

\end{titlepage}

\newpage
%\begin{center}
\Large
Résumé
\normalsize
\end{center}
\vspace{3cm}
\begin{itemize}[leftmargin=1cm, label=\ding{32}, itemsep=21pt]
\item {\bf Objet : éclairer } .
\item {\bf Contenu : définitions} .
\item {\bf Public concerné : néophyte} .
\end{itemize}

\vspace{3cm}

résumé.

\vspace{3cm}

Ce document contient des définitions provenant de divers dictionnaires.




\thispagestyle{empty}

\begin{center}
\Large
%Introduction
Préambule
\normalsize
\end{center}
\vspace{3cm}

Ce document est une compilation d'articles provenant de quatre ouvrages : un dictionnaire encyclopédique de poche, {\it La pratique de la philosophie} destiné aux lycéens, une encyclopédie de la philosophie destinée aux néophytes, et le dictionnaire philosophique d'André Comte-Sponville.
les chapitres contiennent les articles des trois premiers ouvrages, les articles du dictionnaire de philosophie d'André Comte-Sponville sont reproduits en annexe.

\vspace{1.3cm}

Chaque chapitre contient les articles correspondant à une notion particulière. Ces notions ont été choisies en raison de leurs liens avec la question du hasard. Ces choix ont été guidés : 1. Par les renvois vers d'autres articles présent dans les ouvrages, 2. Mes propres choix, liés à mes connaissances, 3. La volonté d'obtenir une quantité raisonnable d'information.

\vspace{1.3cm}

Dans {\it La pratique de la philosophie}, l'article concernant la nécessité renvoit à des textes de Spinoza dont le choix reste subjectif à l'ouvrage. J'ai néanmoins reproduit en annexe ces textes ainsi que l'article concernant Spinoza. (les autres annexes sont d'autres renvois de cet ouvrage)

Les articles compilés dans ce document comportent donc les choix "discutables" réalisés dans les trois ouvrages utilisés. Il s'agit donc d'un document de "travail" destiné à apporter quelques points de vues philosophiques de manière relativement élémentaire.

\vspace{1.3cm}

Les trois premiers chapitres abordent les thèmes du hasard, de la nécessité puis du vitalisme. Les chapitres suivants élargissent le champ de vision philosophique en abordant les thèmes du déterminisme, de la contingence et de la providence.

\vspace{2.3cm}

\hfill Stephan Runigo

%%%%%%%%%%%%%%%%%%%%%%%%%%%%%%%%%%%%%%%%%%%%%%%%

%
\newpage
%
%

\thispagestyle{empty}
\begin{center}
\Large
%Introduction
Les ouvrages utilisés
\normalsize
\end{center}

\vspace{1.3cm}

Premières de couverture

\begin{center}
\includegraphics[scale=0.43]{./presentation/dictionnaire-1}
\hfill
\includegraphics[scale=0.43]{./presentation/pratique-1}
\hfill
\includegraphics[scale=0.43]{./presentation/encyclopedie-1}
\end{center}

\vspace{1.3cm}

Quatrièmes de couverture

\begin{center}
\includegraphics[scale=0.43]{./presentation/dictionnaire-2}
\hfill
\includegraphics[scale=0.43]{./presentation/pratique-2}
\hfill
\includegraphics[scale=0.43]{./presentation/encyclopedie-2}
\end{center}



%\newpage
%

%
% Table des matières
\tableofcontents
\thispagestyle{empty}
\setcounter{page}{0}
%
%espacement entre les lignes des tableaux
\renewcommand{\arraystretch}{1.5}
%
%====================== INCLUSION DES PARTIES ======================
%
~
\thispagestyle{empty}
%recommencer la numérotation des pages à "1"
\setcounter{page}{0}
\newpage
%
%{\footnotesize XIX$^\text{e}$} siècle — {\it }

\chapter{E}
%
%{\footnotesize XIX$^\text{e}$} siècle — {\it }

%ÉCHANGE
\section{Échange}
Changement simultané de propriétaire et de propriété, le plus
souvent sous la forme d’une cession mutuelle : ce qui appartenait
à l’un appartient désormais à l’autre, et réciproquement.

Les ethnologues, prenant le mot en un sens élargi, distinguent trois
échanges fondamentaux : l’échange des biens, qui est troc ou commerce ;
l'échange des signes, qui est langage ; l’échange des femmes, qui est alliance. Ce
dernier échange, qui nous choque (parce qu’on y échange des sujets, non des
objets), tend à disparaître : on n’échange plus les femmes ; hommes et femmes
se donnent ou se prêtent, sans que nul pourtant puisse jamais les posséder.
C'est libérer le couple des rapports marchands, et la seule forme réalisable,
peut-être bien, du communisme — non par la propriété commune, qui ne serait
que collectivisme, mais par l'abolition de la propriété. Par quoi l'amour et le
respect, tant qu'ils durent, sont la seule utopie vraie.

%ÉCHEC
\section{Échec}
L'écart entre le résultat qu’on visait et celui qu’on obtient. C’est
pourquoi l’histoire de toute vie, comme disait Sartre, est l’histoire
d’un échec : le réel résiste et nous emporte.

On n'échappe à l'échec qu’en cessant de viser un résultat. Non parce qu’on
cesserait d'agir, ce qui ne serait qu’un échec de plus, mais parce qu’on ne vise plus
que l’action même. C’est ce qu’on appelle la sagesse, qui serait la seule vie réussie.
On n’a une chance de l’atteindre qu’à condition de cesser de la poursuivre.

%ÉCLECTISME
\section{Éclectisme}
Une pensée qui se constituerait par emprunts et collages : un
système de bric et de broc, une doctrine qui serait un

%— 191 —
%{\footnotesize XIX$^\text{e}$} siècle — {\it }
mélange de doctrines, comme un patchwork théorique. L’éclectisme se doit
pourtant d’être cohérent : c’est ce qui le distingue du syncrétisme, et qui en fait
une philosophie possible (par exemple chez Leibniz, qui n’utilise pas le mot, ou
chez Victor Cousin, qui baptise ainsi son école). Dans la philosophie contemporaine,
le mot tend toutefois à valoir comme condamnation. C’est que les
grands philosophes n’empruntent pas : ils absorbent et recréent, transforment
et dépassent. « Les abeilles pillotent [pillent, butinent] deçà delà les fleurs,
remarquait Montaigne après tant d’autres, mais elles en font après le miel, qui
est tout leur ; ce n’est plus thym, ni marjolaine ; ainsi les pièces empruntées
d’autrui, il les transformera et confondra pour en faire un ouvrage tout sien, à
savoir son jugement » (I, 26, 152). Cette métaphore, quoique rebattue, dit à
peu près l'essentiel. Le miel n’est pas un mélange de fleurs. Mais cela n’a jamais
dispensé les abeilles de butiner.

%ÉCOLE
\section{École}
Le lieu où l’on enseigne et où l’on apprend. Cela suppose un maître,
celui qui sait et enseigne, et des écoliers, qui ne savent pas et sont là
pour apprendre. La définition même de l’école semble rétrograde et antidémocratique.
C’est bien ainsi. Toute école représente le passé, qu’elle doit transmettre
à ceux, plus tard, qui inventeront l'avenir. Et aucune ne saurait se
soumettre, sans y perdre son âme, à l’exigence démocratique, qui est celle du
nombre et de l'égalité. On ne va pas voter, dans les classes, pour savoir comment
s'écrit un mot, combien font trois fois huit, ou quelles sont les causes de
la première guerre mondiale. Ni pour savoir s’il faut étudier l'orthographe,
l’arithmétique et l’histoire. Le maître ne peut transmettre son savoir que si son
pouvoir est à peu près reconnu par tous. Par quoi il ny a pas d’école sans discipline,
ni de discipline sans sanctions. L'école démocratique ? C’est celle qui
est soumise à la démocratie, c’est-à-dire au peuple souverain, qui décide des
budgets, des programmes, des objectifs. Non celle qui serait soumise, absurdement,
aux suffrages des élèves ou des parents. Ouvrir l’école sur la vie ? Ce
serait l'ouvrir au marché, à la violence, aux fanatismes de toutes obédiences.
Mieux vaut la fermer sur elle-même — lieu d’accueil et de recueillement — pour
l'ouvrir aux savoirs et à tous.

%ÉCOLOGIE
\section{Écologie}
L'étude des milieux ou des habitats — {\it oikos}, en grec, c’est la maison —,
spécialement des biotopes (les milieux vivants) et en
général de la biosphère (Pensemble des biotopes). On ne la confondra pas avec
l’écologisme, qui est l’idéologie qui s’en réclame, ni même avec la protection de
l’environnement, qui l'utilise. Rien n’interdirait, en théorie, d’utiliser l'écologie

%— 192 —
%{\footnotesize XIX$^\text{e}$} siècle — {\it }
pour polluer le plus efficacement possible. En pratique, toutefois, personne
ne le souhaite. C’est ce qui fait que le mot, presque toujours, prend un sens
normatif, voire prescriptif. Reste à savoir si on veut mettre l’écologie au service
de l'humanité (écologie humaniste), ou mettre l’humanité à son service (écologie
radicale). Les deux positions sont respectables. Cela ne nous dispensera
pas, entre les deux, de choisir ou de chercher un compromis.

%ÉCONOMIE
\section{Économie}
Étymologiquement, c’est la loi ou l’administration ({\it nomos}) de
la maison ({\it oikos}). La première économie est domestique : c’est
la gestion des biens d’une famille, de ses ressources, de ses dépenses, ce que
Montaigne appelait le {\it mesnage}, où il voyait une tâche « plus empêchante [plus
absorbante] que difficile ». L'essentiel est ici de ne pas dépenser plus qu’on ne
possède ou qu’on ne gagne : de là le sens courant du mot, spécialement au pluriel
(« faire des économies »), qui vise surtout une restriction des dépenses. Au
sens moderne, le mot désigne à la fois une science humaine et l’objet qu’elle
étudie : l’économie, c’est tout ce qui concerne la production, la consommation
et l’échange des biens matériels, marchandises ou services, aussi bien à l’échelle
des individus et des entreprises (micro-économie) qu’à l'échelle de la société ou
de la planète (macro-économie). C’est moins l’art de réduire les dépenses que
celui d’augmenter les richesses. Son lieu de prédilection est le marché, où règne
la loi de l'offre et de la demande. Tout ce qui est rare est cher, mais à condition
seulement d’être désiré par plusieurs (par quoi toute valeur, même économique,
reste subjective) et de pouvoir être échangé (par quoi la subjectivité de
la valeur, dans un marché donné, fonctionne objectivement). Si tout était à
vendre, l’économie régnerait seule. On n’y échappe que par ce qui ne vaut rien
(la gratuité) et par ce qui n’a pas de prix (la dignité, la justice). On n’y échappe
donc, et jamais totalement, que par exception ou par devoir. C’est ce que nous
rappellent la misère, chez les pauvres, et l’avidité, chez les riches.

%ÉCRITURE
\section{Écriture}
C’est une technique, inventée il y a quelque cinq mille ans, qui
permet de fixer la parole ou la pensée sur un support durable,
au moyen de signes plus ou moins symboliques (pictogrammes, idéogrammes)
ou conventionnels (les lettres d’un alphabet). C’est inscrire la pensée dans
l’espace, où elle se fige et se conserve, et la libérer par là, au moins partiellement,
du temps. Les livres remplacent la mémoire, ou plutôt la soutiennent, la
démultiplient, la sauvent. Dans une société sans écriture, dit-on souvent, «un
vieillard qui meurt, c’est une bibliothèque qui brûle ». Or les vieillards meurent
tous, toujours. Les bibliothèques ne brûlent que par exception. L’accumulation

%— 193 —
%{\footnotesize XIX$^\text{e}$} siècle — {\it }
des souvenirs, des idées, des savoirs, devient ainsi, grâce à l’écriture, indéfinie.
On ne se contente plus de résister individuellement à l'oubli ; on ajoute des
traces aux traces, des œuvres aux œuvres. C’est passer d’une logique de la répétition,
qui est celle du mythe, à une logique de l’accumulation et du progrès,
qui est celle de l’histoire. Ce n’est donc pas un hasard, ni pure convention, si
l'invention de l'écriture marque la fin de la préhistoire. La conservation du
passé bouleverse notre rapport à l’avenir : le présent, qui les sépare et les relie,
est entré dans l’histoire.

%ÉDUCATION
\section{Éducation}
C’est transformer un petit d'homme — le même à la naissance,
à très peu près, que son ancêtre d’il y a dix mille ans — en être
humain civilisé. Cela suppose qu’on lui transmette, dans la mesure du possible,
ce que l’humanité à fait de meilleur ou de plus utile, ou qu’elle juge être tel :
certains savoirs et savoir-faire (à commencer par la parole), certaines règles, certaines
valeurs, certains idéaux, enfin l’accès à certaines œuvres et la capacité
d’en jouir. C’est reconnaître qu’il n’y a pas de transmission héréditaire des
caractères acquis, et que l’humanité, en chacun, est aussi une acquisition : on
naît homme, ou femme, on {\it devient} humain. C’est reconnaître que la liberté
n'est pas donnée d’abord, qu’elle ne va pas sans raison ni la raison sans
apprentissage : on ne naît pas libre, on le devient. Cela ne va pas sans amour,
dans la famille, mais pas non plus sans contraintes. Et encore moins, à l’école,
sans travail, sans efforts, sans discipline. Le plaisir ? On n’en a jamais trop. Mais
telle n’est pas la principale fonction de l’école, ni même de la famille. L’éducation
est presque toute du côté du principe de réalité. Il s’agit non de remplacer
l'effort par le plaisir, mais d’aider l'enfant à trouver du plaisir, peu à peu, dans
l'effort accepté et maîtrisé. Jouer ? On ne le fait jamais trop. Mais c’est le travail
qui est grand, et qui fait grandir. D'ailleurs les enfants jouent à travailler, très
vite, et cela indique assez la direction. L'éducation n’est pas au service des
enfants, comme on le dit toujours, mais au service des adultes qu’ils veulent et
doivent devenir.
On se trompe pourtant si l’on croit que l’éducation doit inventer l’avenir.
De quel droit parents et pédagogues, qui sont en charge de l'éducation, choisiraient-ils
l'avenir des enfants à leur place ? La vraie fonction de l'éducation, et
spécialement de l’école, ce n’est pas d’inventer l’avenir, c’est de transmettre le
passé. C’est ce qu'avait vu Hannah Arendt, dès les années 50 : « Le conservatisme,
pris au sens de conservation [je préférerais dire la transmission], est
l'essence même de l’éducation », disait-elle. Non, certes, parce qu’il faudrait
renoncer à transformer le monde, mais au contraire pour permettre aux enfants
de le faire, s’ils le veulent, quand ils seront grands : « C’est justement pour préserver
% 194 —
%{\footnotesize XIX$^\text{e}$} siècle — {\it }
ce qui est neuf et révolutionnaire dans chaque enfant que l’éducation
doit être conservatrice » ({\it La crise de la culture}, V). C’est ce qu'avait vu Alain,
dès les années 20 : « L'enseignement doit être résolument retardataire. Non pas
rétrograde, tout au contraire. C’est pour marcher dans le sens direct qu’il prend
du recul ; car, si l’on ne se place point dans le moment dépassé, comment le
dépasser ? ({\it Propos sur l'éducation}, XVII). Vous pouvez bien mettre des ordinateurs
et des journaux dans toutes les classes. Cela ne remplacera jamais les
chefs-d’œuvre — littéraires, artistiques, scientifiques — qui ont fait de l’humanité
ce qu’elle est. D’ailleurs ordinateurs et journaux sont du passé aussi (dès qu’ils
se répandent, ils sont obsolètes) et vieilliront plus vite que Pascal ou Newton,
Hugo ou Rembrandt. Le progrès ? Il suppose la transmission, et ne saurait par
conséquent autoriser qu’on y renonce. L'avenir ? Ce n’est pas une valeur en soi
(sans quoi la mort, pour chacun, en serait une). Il ne vaut, ou plutôt il ne
vaudra, que par fidélité d’abord à ce que nous avons reçu, que nous avons à
charge de transmettre. Du passé, ne faisons pas table rase.

%EFFET
\section{Effet}
Un fait quelconque, en tant qu’il résulte d’une ou plusieurs causes.
Ainsi tout fait est effet, et c’est ce que signifie le principe de causalité.

%EFFICIENTE (CAUSE -)
\section{Efficiente (cause —)}
L’une des quatre sortes de causes selon Aristote, et
la seule que la modernité ait véritablement
retenue. Une cause efficiente, c’est une cause qui n’est ni finale ni formelle
(voir ces mots), et qui ne se réduit pas non plus à la simple matière dont l'effet
est constitué (ce qu’Aristote appelait la cause matérielle). C’est dire qu’elle produit
ses effets par son action propre : c’est le « moteur prochain », comme disait
Aristoté, autrement dit ce qui meut ou transforme une matière première. Par
exemple le sculpteur est la cause efficiente de la statue, comme les intempéries,
la pollution et les touristes sont les causes efficientes de son inexorable dégradation.

On dit parfois que la cause efficiente est celle qui précède son effet (par
opposition à la cause finale, qui le suivrait). Mais elle ne le produirait pas si elle
n'avait avec lui au moins un point de tangence dans le temps, que ce soit directement
ou par telle ou telle de ses suites. Par exemple mes parents décédés ne
sont la cause efficiente de mon existence présente que parce qu’ils ont été celle
de ma conception : preuve en est que je peux désormais exister sans eux comme
eux sans moi. Ce n’est pas le passé qui agit sur le présent ou qui le cause ; c’est
le présent qui continue le passé en agissant sur lui-même. La cause efficiente est
le nom qu’on donne traditionnellement à cette action et à cette continuation.

%— 195 —
%{\footnotesize XIX$^\text{e}$} siècle — {\it }
%EFFORT
\section{Effort}
Une force volontaire ou instinctive, quand elle s'oppose à une
résistance. Maine de Biran voyait dans l’effort « le fait primitif du
sens intime » : celui par lequel le moi se découvre ou se constitue « par le seul
fait de la distinction qui s'établit entre le sujet de cet effort libre et le terme qui
résiste immédiatement par son inertie propre ». Mais comment savoir si le moi
est la cause (comme le veut Biran) ou l'effet (comme je préférerais dire) de cet
effort ? C’est où il faut choisir entre le {\it conatus} spinoziste et l'{\it effort} biranien, qui
est sa version française et spiritualiste.

%ÉGALITÉ
\section{Égalité}
Deux êtres sont égaux lorsqu'ils sont de même grandeur ou possèdent
la même quantité de quelque chose. La notion n’a donc de
sens que relatif: elle suppose une grandeur de référence. On parlera par
exemple de l'égalité de deux distances, de deux poids, de deux fortunes, de
deux intelligences... Mais une distance n’est pas égale à un poids, ni une fortune
à une intelligence. Une égalité absolue ? Ce serait une identité, et nul ne
serait légal, en ce sens, que de soi. Prise absolument, la notion perd son sens ou
devient autre. On ne peut lui être absolument fidèle qu’à la condition
d’accepter qu’elle soit relative.

Les hommes sont-ils égaux ? Tout dépend de quoi l’on parle : question de
fait, ou question de droit ? En fait, c’est bien sûr l'inégalité qui est la règle : les
hommes ne sont ni aussi forts, ni aussi intelligents, ni aussi généreux les uns
que les autres. Ces différences parfois s’équilibrent ou se compensent : tel sera
plus fort que tel autre, qui sera plus intelligent ou moins égoïste... Mais il
arrive aussi, et peut-être plus souvent, qu’elles s'ajoutent : certains semblent
avoir toutes les chances, tous les talents, toutes les vertus, quand d’autres n’ont
que des faiblesses, que des tares ou du malheur. Si toutes les très belles femmes
étaient stupides et méchantes, cela serait moins dur pour les laides. Si tous les
champions étaient idiots et impuissants, ce serait moins blessant pour les
autres. Mais ce n’est pas le cas, et les voilà qui deviennent, cela fait une inégalité
de plus, riches à millions. En fait, donc, il n’y a pas de doute : quelle que soit
la grandeur considérée, et même en essayant d’unifier ces grandeurs en une
moyenne, les êtres humains, à les considérer comme individus, sont manifestement
inégaux. Un Dieu juste, peut-être, aurait pu ou dû l’éviter. Mais s’il n’y
a pas de Dieu ? Ou s’il en a jugé autrement ?

La démocratie ? Les droits de l’homme ? J'y suis attaché autant qu’un autre.
Mais pourquoi faudrait-il, pour être démocrate, prétendre qu’Eichmann est
l’égal, en fait, d’Einstein ou de Cavaillès ?

On a insuffisamment répondu à Le Pen, quand on lui objecte, sans autre
précision, que tous les hommes sont égaux. Tous les individus ? Ce serait
%— 196 —
%{\footnotesize XIX$^\text{e}$} siècle — {\it }
opposer un pieux mensonge à un mensonge impie. Toutes les races ? Si le
concept est sans pertinence, comme semblent le penser nos généticiens, ce problème
n’en a pas davantage. Si l’on continue pourtant à se servir de cette
notion équivoque et désagréable, comme font beaucoup d’antiracistes (« toutes
les races sont égales »), attention de ne pas tomber dans la même confusion que
font ou qu’entretiennent ceux que nous combattons. Que les hommes soient
égaux {\it en droit et en dignité}, cela seul dépend de nous, cela seul, moralement,
politiquement, importe. Qu'ils soient égaux en fait, cela dépend de la nature, et
rien ne garantit qu’elle soit démocrate, progressiste et humaniste. Elle ignore
nos lois. Pourquoi faudrait-il accepter aveuglément la sienne ? « La biologie a
depuis longtemps réfuté le racisme », disent de braves gens. Je m’en réjouis fort.
Mais fallait-il être raciste, tant que ce n’était pas le cas ? Faudrait-il le devenir, si
la biologie changeait d’orientation ? Faut-il soumettre nos principes au diktat des
laboratoires ? N’être antiraciste que sous réserve d’inventaire génétique ? Ce serait
confondre le fait et le droit, et c’est en quoi il est essentiel de les distinguer.
Quand bien même les généticiens nous expliqueraient demain que les Noirs sont
en effet plus doués pour la course à pied que les Blancs, quand bien même ils
découvriraient que ce n’est pas absolument par hasard (ni pour des raisons seulement
culturelles) que les Juifs ont proportionnellement davantage de prix Nobel
que les soi-disant aryens, cela ne donnerait évidemment aucun droit ni aucune
dignité supplémentaires aux uns ou aux autres. La question n’est pas de savoir si
les hommes sont égaux en fait — cela ne dépend pas de nous, et les individus ne
le sont pas —, mais si nous voulons qu’ils Le soient (et il suffit que nous le voulions
pour qu’ils le soient en effet) en droit et en dignité.

La réponse n’est pas dans les laboratoires des généticiens, ni dans les tests
des psychologues. Elle est dans nos cœurs et dans nos parlements — dans nos
principes et dans nos lois. Ne comptons pas sur la biologie pour être égalitaire
à notre place. Ne renonçons pas à l’être sous prétexte que la nature, s'agissant
des individus, ne l’est pas.

Ce n’est pas parce que les hommes sont égaux qu’ils ont les mêmes droits.
C’est parce qu’ils ont les mêmes droits qu’ils sont égaux.

%ÉGLISE
\section{Église}
La communauté des croyants, pour une religion donnée et lorsqu’elle
a atteint une certaine dimension : c’est une secte trop nombreuse
pour accepter d’en être une.

Quand il est utilisé sans autre précision, le mot désigne presque toujours
l’une des Églises chrétiennes, et spécialement l'Église catholique. C’est qu’elles
ont réussi, en tant qu’institutions, plus solidement que les autres. Cela ne suffit
pas à leur donner raison, ni tort.

%— 197 —
%{\footnotesize XIX$^\text{e}$} siècle — {\it }
« Jésus annonçait le Royaume, écrit Alfred Loisy, et c’est l'Église qui est
venue. » Cela pourrait presque valoir comme définition. Une Église, c’est ce
qui vient à la place du Royaume qu’elle annonce. Seuls les vrais mystiques et les
vrais athées peuvent s’en passer.

%EGO
\section{Ego}
Le moi, le plus souvent considéré comme objet de la conscience. C’est
moins ce que je suis que ce que je crois être, moins le {\it je} que le {\it me} (par
exemple quand on dit « je {\it me} connais », « je {\it me} sens triste »..), moins le sujet
transcendantal, malgré Husserl, que l’objet transcendant, comme dit Sartre,
d’une conscience impersonnelle ({\it La transcendance de l'Ego}, Conclusion :
« L’Ego n’est pas propriétaire de la conscience, il en est l’objet » ; voir aussi
{\it L'être et le néant}, p. 147). Ce n’est le sujet de la pensée qu’autant qu’elle croit
en avoir un, et c’est pourquoi ce n’est rien.

%ÉGOCENTRISME
\section{Égocentrisme}
C'est se mettre au centre de tout. Se distingue de
l’égoïsme par une dimension plus intellectuelle.
L’égoïsme est une faute ; l’égocentrisme serait plutôt une illusion ou une erreur
de perspective. C’est le point de vue spontané de l’enfant et de l’imbécile. Son
remède est le décentrement ; son contraire, l’universel.

%ÉGOÏSME
\section{Égoïsme}
Ce n’est pas l'amour de soi ; c’est l’incapacité à aimer quelqu'un
d'autre, ou à l’aimer autrement que pour son bien à soi. C’est
pourquoi j'y vois un péché capital (l’amour de soi serait plutôt une vertu) et le
principe de tous.
C’est aussi une tendance constitutive de la nature humaine. On ne la surmonte
que par effort ou par amour — par vertu ou grâce.

%ÉGOTISME
\section{Égotisme}
Ce n’est pas le culte du moi. Chez Stendhal, qui popularisa le
mot, c'est bien plutôt son étude bienveillante et lucide, son
analyse, son approfondissement, son perfectionnement — non son culte, mais sa
culture.

%ÉIDÉTIQUE
\section{Éidétique}
Qui porte sur l’essence (l’{\it eîdos}) ou les essences. Le mot n’est
guère utilisé que par les phénoménologues.

%— 198 —
%{\footnotesize XIX$^\text{e}$} siècle — { }
%EÎDOS
\section{\it Eîdos}
Ce qu’on voit, par les yeux ou par l'esprit : la forme, l’idée, l'essence.
Mais pourquoi le dire en grec ?

%ÉLÉATES
\section{Éléates}
Élée était une colonie grecque, au sud de l'Italie. C’est là que sont
nés Parménide et Zénon d’Élée (qu’on ne confondra pas avec
Zénon de Cittium, le fondateur du stoïcisme), qui dominent l’école dite des
éléates. Si l’on se fie à ce qu’en a retenu la tradition, ils niaient l'existence du
mouvement, du devenir, de la multiplicité, pour ne plus célébrer que
l'unicité et l’immuabilité de l’être. C’était donner tort aux apparences et à
l'opinion : de là les paradoxes de Zénon (Achille incapable de rattraper une
tortue, la flèche immobile en plein vol.) et l’Être éternel et absolument plein
de Parménide. L’être est, le non-être n’est pas. Comment le devenir (qui suppose
le passage de l'être au non-être, du non-être à l’être) pourrait-il exister ?
L’éléatisme, en ce sens, serait le contraire de l’héraclitéisme.

Pour ma part, guidé par Marcel Conche (Parménide, {\it Le poème : fragments},
PUF, 1996), j'y vois plutôt une pensée de l’éternelle présence : non le contraire
de l’héraclitéisme, mais sa saisie {\it sub specie aeternitatis}. Au présent, l'être et le
devenir sont un. Tout change (c’est ce qui donne raison à Héraclite), mais rien
ne change qu’au présent, qui reste toujours identique à soi (c’est ce qui donne
raison à Parménide : « Ni il n’était, ni il ne sera, puisqu'il est maintenant »).
Ainsi l'éternité et le temps sont une seule et même chose. Quelle chose ? Le
présent, qui passe et demeure — comme une flèche, en effet, à la fois mobile
(elle n’est plus où elle était, pas encore où elle sera) et immobile (elle est ici et
maintenant) en plein vol.

%ÉLITE
\section{Élite}
Étymologiquement, c’est l’ensemble de ceux qui ont été élus ; mais
ils l'ont été par le hasard ou par eux-mêmes plutôt que par Dieu ou
par le peuple. C’est une espèce d’aristocratie laïque et méritocratique : les
meilleurs, quand ils ne le doivent qu’à leur talent ou à leur travail. L'erreur
serait d'y croire tout à fait ou en bloc. Il y a des élites différentes (celle du sport
n’est pas celle du savoir, qui n’est pas celle des affaires ou de la politique), dont
la médiocrité, quand on y pénètre, ne cesse de surprendre. Une salle de profs,
dans la plus prestigieuse des universités, ne donne pas une très haute idée de
l'humanité. J'imagine qu’il en va de même dans d’autres milieux, que j'ai
moins fréquentés. Nul n’est le meilleur en tout et absolument. Un sujet d'élite,
c'est un individu dont la médiocrité générale, qui est celle de l’espèce, comporte
au moins une exception, qui est celle du talent ou du savoir-faire. C'est
mieux que rien. Ce n’est pas grand-chose.

%— 199 —
%{\footnotesize XIX$^\text{e}$} siècle — {\it }
On s'étonne qu’un grand philosophe ait été nazi, que tel autre ait été mesquin
ou lâche... C’est qu’ils appartenaient à une élite, non à toutes. Autant
s'étonner qu’un ministre fasse des fautes de français.

%ÉLITISME
\section{Élitisme}
Toute pensée qui reconnaît l’existence d’une élite ou favorise son
émergence. L’élitisme républicain, qui vise à produire les élites
dont la République a besoin, s’oppose donc à l’aristocratisme (qui voudrait que
les élites se reproduisent par le sang) comme à l’égalitarisme (qui voudrait qu’il
n'y ait pas d’élites du tout). C’est ce qui le rend doublement nécessaire. Pourquoi
l'égalité des chances devrait-elle aboutir à un nivellement par le bas ?

%ÉLOQUENCE
\section{Éloquence}
L’art de la parole (c’est ce qui distingue l’éloquence de la rhétorique,
qui serait plutôt l’art du discours) ou le talent qui
permet d’y exceller. Art mineur, talent dangereux.

%ÉMOTION
\section{Émotion}
C'est un affect momentané, qui nous meut plus qu’il ne nous
structure (comme ferait un sentiment) ou qu’il ne nous
emporte (comme ferait une passion). Par exemple la colère, la peur ou le coup
de foudre sont des émotions. Mais qui peuvent déboucher sur des passions ou
des sentiments, comme sont par exemple la haine, l’anxiété ou l’amour. « Une
suite d'émotions vives et liées au même objet produit la passion, écrivait Alain ;
et l’état de passion surmonté se nomme sentiment. » Mais l’enchaînement peut
se prendre aussi dans l’autre sens : toute passion est source d'émotions, et l’on
ne surmonte guère que les passions fatiguées. Les frontières, entre ces différents
affects ou concepts, sont floues, et ce flou est essentiel à l’émotion : si l’on y
voyait absolument clair, on ne serait pas ému. Par exemple ce trouble, entre
deux regards qui se croisent, cette attirance, cette sensualité légère et joyeuse,
cette inquiétude aussi, cette accélération du cœur et de la pensée, est-ce un
amour qui commence ou un désir qui passe ? On ne le saura vraiment que plus
tard, quand l'émotion sera retombée ou installée.

%EMPIRIQUE
\section{Empirique}
Qui vient de l’expérience ou en dépend. Le mot, dans la philosophie
continentale et depuis Kant, se prend plutôt en
mauvaise part. Toute expérience étant particulière et contingente, une
connaissance empirique ne saurait être ni universelle ni nécessaire. Que les
choses, telles que nous avons pu les observer, se soient jusqu’à présent passées

%— 200 —
%{\footnotesize XIX$^\text{e}$} siècle — {\it }
de telle et telle façons, cela ne prouve pas qu’elles se passent et se passeront
{\it toujours} comme cela. Toute connaissance empirique reste donc approxima-
tive et provisoire. Mais une connaissance qui n'aurait absolument rien
d’empirique ne serait plus une connaissance du tout. Si nous n’avions l’expé-
rience des signes et des figures, de l’évidence et de l’absurdité, que resterait-il
des mathématiques ?

%EMPIRISME
\section{Empirisme}
Toute théorie de la connaissance qui accorde la première place
à l’expérience. C’est refuser les idées innées d’un Descartes
autant que les formes {\it a priori} d’un Kant. La raison, pour l’empiriste, n'est pas
une donnée première ou absolue : elle est elle-même issue de l'expérience, aussi
bien extérieure (sensations) qu’intérieure (réflexion), et en dépend autant
qu’elle la met en forme (spécialement par l’usage des signes). L'empirisme
s'oppose en cela au rationalisme en son sens étroit et gnoséologique. Cela ne
l'empêche évidemment pas d’être rationaliste au sens large ou normatif : la plupart
des grands empiristes (Épicure, Bacon, Hobbes, Locke, Hume...) se sont
battus pour que la raison l’emporte, non certes contre l’expérience, ce qu’elle ne
peut ni ne doit, mais contre l’obscurantisme et la barbarie.

C’est ce qui amènera l’empirisme logique, au {\footnotesize XX$^\text{e}$} siècle, à s'intéresser surtout
aux sciences, au point de rejeter toute métaphysique : « L'analyse logique
rend un verdict de non-sens, écrit Carnap, contre toute prétendue connaissance
qui veut avoir prise par-delà ou par-derrière l’expérience. » La logique ne
connaît que soi (elle est analytique, non synthétique). Seule l'expérience — et
spécialement l’expérimentation scientifique — nous permet de connaître le
monde.

Reste toutefois à penser cela même que l’on connaît, ce qui suppose toujours
autre chose, que l’on ne connaît pas. C’est où la philosophie revient, y
compris dans sa dimension spéculative, et qui lui interdit de se prendre pour
une science. L’empirisme n’est pas nécessairement antimétaphysique. Mais il
est antidogmatique, ou doit l'être. Si toute connaissance vient de l’expérience,
comment pourrions-nous tout connaître ou connaître quoi que ce soit
absolument ? Comment être certain que l'expérience dit vrai, puisqu'on ne
pourrait en décider que. par une autre expérience ! ? Comment être certain que
P inexpérimentable n'existe pas, puisqu'il serait par nature hors de notre
portée ? L’empirisme, qui fut d’abord dogmatique (spécialement chez Épi-
cure et Lucrèce), a surtout à voir, dans les temps modernes, avec le scepti-
cisme. Plus l’expérience progresse, mieux on en comprend les limites, qui
sont les nôtres.

%— 201 —
%{\footnotesize XIX$^\text{e}$} siècle — {\it }
%EMPORTEMENT
\section{Emportement}
Une colère qui passe à l’acte. C’est suivre son corps, au
lieu de le gouverner.

%ÉNERGIE
\section{Énergie}
Une force en puissance ({\it dunamis}) ou en acte ({\it energeia}) : c’est la
capacité de produire un effort ou un travail.

Nos physiciens nous apprennent que l'énergie peut prendre des formes différentes
(cinétique, thermique, électrique), qu’elle ne se perd ni ne se crée
(elle se conserve toujours, même si elle se dégrade en chaleur), enfin qu’elle est
équivalente à la masse ({\it E = mc$^2$}). Philosophiquement, le mot ferait une assez
bonne traduction pour le {\it conatus} spinoziste. Tout être tend à persévérer dans
son être, et toute énergie se conserve : les deux idées, pour différentes qu’elles
soient évidemment, sont au moins compatibles. Cela n’empêche ni la mort ni
la fatigue.

%ENFANCE
\section{Enfance}
Le premier âge de la vie : les années qui séparent la naissance de
l’adolescence ou de la puberté. C’est l’âge de la plus grande fragilité —
l'enfant est à peu près sans défense contre le mal et le malheur — et des
plus grandes promesses. C’est ce qui nous impose, vis-à-vis des enfants, les plus
grands devoirs (devoir de protection, de respect, d’éducation...), sans aucun
droit jamais sur eux. « Cette faiblesse est Dieu », disait Alain. C’est qu’elle commande
absolument, par l'incapacité où elle est de punir comme de récompenser.
L’enfant-roi est sa caricature : s’il gouverne, il ne règne plus.

Les enfants veulent grandir. Notre devoir est de les y aider, et pour cela de
grandir nous-mêmes. C’est la seule façon d’être fidèle à l'enfant que nous fûmes
et que nous sommes. « Nous poussons notre enfance devant nous, écrit encore
Alain, et tel est notre avenir réel. »

%ENFER
\section{Enfer}
Le lieu du plus grand malheur. Les religions y voient souvent un
châtiment, qui viendrait, après leur mort, punir les méchants. Les
matérialistes, pour qui la mort n’est rien, y voient plutôt une métaphore :
« C’est ici-bas, écrit Lucrèce, que la vie des sots devient un véritable enfer. »
Hélas ! pas des sots seulement. La mort en délivre, plus sûrement que l’intelligence.

%ENGAGEMENT
\section{Engagement}
C'est mettre son action ou sa personne au service d’un
combat que l’on croit juste. Le mot sert surtout pour les
%— 202 —
%{\footnotesize XIX$^\text{e}$} siècle — {\it }
intellectuels, jusqu’à en désigner un certain type (« l’intellectuel engagé »). Le
risque, pour eux, est de soumettre aussi leur pensée aux nécessités du combat —
quand elle devrait ne se soumettre qu’au vrai, ou à ce qui semble tel. Mieux
vaudraient, me semble-t-il, des intellectuels citoyens. Participer au débat
public, dans la limite de ses compétences, cela fait assurément partie de la responsabilité
d’un intellectuel. Mais cela ne l’oblige pas à soumettre sa pensée,
comme firent certains, à une cause déjà constituée par ailleurs. La bonne foi est
plus importante que la foi. La liberté de l'esprit plus importante que l’engagement.

On pense aux Romains d’Astérix: {\it « Engagez-vous, engagez-vous, qu'ils
disaient ! »} Le mot est d’abord militaire, et l’idée, dans son usage intellectuel, en
a gardé quelque chose. Tout engagement suppose l’obéissance. Toute pensée la
récuse. C’est assez, chers collègues, d’agir avec les autres. Cela ne saurait nous
autoriser à penser pour leur faire plaisir ou pour leur donner raison.

%ÉNIGME
\section{Énigme}
Un problème qu’on ne peut résoudre, non parce qu’il excède nos
moyens de connaissance (ce n’est pas un mystère), ni pour des raisons
seulement logiques (ce n’est pas une aporie), mais parce qu’il est mal posé.
C’est pourquoi «l’énigme n'existe pas», comme disait Wittgenstein, ou
n'existe que pour ceux qui en sont dupes : ce n’est qu’un jeu ou une illusion.

%ENNUI
\section{Ennui}
«Le temps, c’est ce qui passe quand rien ne se passe. » Cette for-

mule, dont j'ignore l’auteur, dit la vérité de l'ennui : c’est une expé-
rience du temps, mais réduit absurdement à lui-même, comme s'il était
quelque chose en dehors de ce qui dure et change. On s’ennuie quand le temps
semble vide : parce que rien n'arrive, parce qu’on n’a rien à faire, ou parce
qu’on échoue à s’y intéresser. Souvent c’est parce qu’on attend un avenir qui ne
vient pas, ou qui vient trop lentement pour notre goût, bref qui nous empêche
de désirer le présent : on s'ennuie quand on est séparé du bonheur par son
attente, sans pouvoir agir pour accélérer sa venue. Mais on s’ennuie aussi, bien
souvent, quand on n'attend plus rien, quand on n’est plus séparé du bonheur
par aucun manque, sans pouvoir pour autant être heureux. C’est l’ennui selon
Schopenhauer : l'absence du bonheur au lieu même de sa présence attendue.
On n’y échappe que par la souffrance, comme on n’échappe à la souffrance que
par l'ennui :

\vspace{0.5cm}
{\footnotesize
« Vouloir, s’efforcer, voilà tout leur être ; c’est comme une soif inextinguible. Or tout
vouloir a pour principe un besoin, un manque, donc une douleur... Mais que la
%— 203 —
%{\footnotesize XIX$^\text{e}$} siècle — {\it }
volonté vienne à manquer d’objet, qu’une prompte satisfaction vienne à lui enlever
tout motif de désirer, et les voilà tombés dans un vide épouvantable, dans l'ennui ; leur
nature, leur existence leur pèse d’un poids intolérable. La vie donc oscille, comme un
pendule, de droite à gauche, de la souffrance à l'ennui ; ce sont là les deux éléments
dont elle est faite, en somme. De là ce fait bien significatif par son étrangeté même : les
hommes ayant placé toutes les douleurs, toutes les souffrances dans l’enfer, pour remplir
le ciel n’ont plus trouvé que l'ennui. » ({\it Le monde...}, IV, 57.)
}

\vspace{0.5cm}
L’ennui a pourtant son utilité, qui est de désillusion. Quelqu'un qui ne
s’ennuierait jamais, que saurait-il de soi et de sa vie ? Souvenez-vous de Pascal :

\vspace{0.5cm}
{\footnotesize
« Ennui.

Rien n’est si insupportable à l’homme que d’être dans un plein repos, sans passions,
sans affaires, sans divertissement, sans application.

Il sent alors son néant, son abandon, son insuffisance, sa dépendance, son impuissance,
son vide.

Incontinent il sortira du fond de son âme l'ennui, la noirceur, la tristesse, le chagrin,
le dépit, le désespoir. »
}

\vspace{0.5cm}
C’est toucher le fond, ou constater qu’il n’y en a pas. Bonne occasion pour
s'occuper enfin d’autre chose que de soi.

%EN SOI
\section{En soi}
Ce qui existe en soi, c’est ce qui existe indépendamment d’autre
chose (la substance) et de nous (la chose en soi chez Kant). Mais
c'est aussi ce qui existe sans se penser, indépendamment de toute réflexion et
de toute conscience : c’est l’être qui est ce qu’il est, explique Sartre, de façon
opaque ou massive, sans autre rapport à soi que d'identité ({\it L'être et le néant},
p. 33-34). Se distingue par là du pour soi, comme la matière se distingue de la
conscience ou de l’esprit.

%ENSTASE
\section{Enstase}
Néologisme forgé sur le modèle d’{\it extase}, et qui lui sert de contraire.
C’est entrer en soi, pour se fondre en tout — comme un plongeon
dans l’immanence (dans l’absolu où nous sommes).

Le mot sert surtout pour décrire certaines expériences mystiques, spécialement
orientales. Si l’atman et le Brahman sont un, comme dans l’hindouisme,
ou s’ils n'existent pas, comme dans le bouddhisme, comment pourrait-on
passer de l’un à l’autre? Mystiques non de la rencontre, mais de
l'unité ou de l’immersion.

%— 204 —
%{\footnotesize XIX$^\text{e}$} siècle — {\it }
%ENTÉLÉCHIE
\section{Entéléchie}
L'être en acte, par opposition avec l'être en puissance. Synonyme
en ce sens d’{\it energeia}. Mais le mot, qui appartient au
vocabulaire aristotélicien, désigne l’acte accompli (celui qui a sa fin, {\it telos}, en
lui-même), plutôt que celui en train de se faire (qui tend vers sa fin, et ne l’a
donc pas atteinte). C’est l’acte parfaitement achevé, ou la perfection en acte.

Chez Leibniz, le mot désigne les monades, en tant qu’elles ont en elles
«une certaine perfection, qui les rend sources de leurs actions internes »
({\it Monadologie}, 18).

%ENTENDEMENT
\section{Entendement}
La raison modeste et laborieuse, celle qui refuse les facilités
de l’intuition et de la dialectique autant que les tentations
de l’absolu, et qui se donne par là les moyens de connaître. C’est la puissance
de comprendre, en tant qu’elle est toujours finie et déterminée — notre
accès particulier (puisque humain) à l’universel. Son défaut est d'imaginer partout
un ordre, pour ne se perdre pas.

%ENTHOUSIASME
\section{Enthousiasme}
« Le mot grec, écrit Voltaire, signifie {\it émotion d'entrailles,
agitation intérieure}. » Ce n’est pas ce qu’on lit
dans nos dictionnaires, qui rattachent le mot au verbe {\it enthousiazein}, « être inspiré
par la divinité », lui-même dérivé du substantif {\it theos}. L’enthousiasme est
un transport divin, ou se croit tel, ou y ressemble. Il se pourrait pourtant que
Voltaire n’ait pas tout à fait tort : que les entrailles jouent un plus grand rôle,
dans l’enthousiasme, que la divinité.

En un sens plus large, le mot désigne une exaltation joyeuse ou admirative. C’est
une espèce d'ivresse, mais tonique, mais généreuse, que la raison doit pourtant
apprendre à contrôler. Le danger est d’y perdre tout sens critique, toute indépendance,
toute lucidité, tout recul. C’est vrai surtout des enthousiasmes collectifs, qui
font la mode et les fanatismes. « L'esprit de parti dispose merveilleusement à
l'enthousiasme, notait Voltaire ; il n’est point de faction qui n’ait ses énergumènes. »

%ENTITÉ
\section{Entité}
Un étant ({\it ens, entis}), qu’on peut penser, mais qu’on ne peut prendre
tout à fait pour une chose ou un individu : c’est un être abstrait, ou
l’abstraction d’un être.

%ENTROPIE
\section{Entropie}
Qualifie l’état d’un système physique isolé (ou considéré comme
tel) par la quantité de transformation spontanée dont il est
%— 205 —
%{\footnotesize XIX$^\text{e}$} siècle — {\it }
capable : l’entropie est à son maximum quand le système est devenu incapable
de se modifier lui-même — parce qu’il a atteint son état d'équilibre, qui est
aussi, d’un point de vue statistique et au niveau des particules qui le composent,
son état le moins ordonné ou le plus probable. Ainsi, c’est l’exemple traditionnel,
dans une tasse de café : il est exclu que le café se réchauffe ou se
sépare de lui-même du sucre qu’on y a mis (il ne peut que refroidir et rester
sucré). Le second principe de la thermodynamique stipule que l’entropie, dans
un système clos, ne peut que croître, ce qui suppose que le désordre y tend vers
un maximum : c’est ce que confirment l’histoire de l'univers (à l’exception de
la vie) et la chambre de nos enfants (à l'exception du ménage). Le soleil ou les
parents paient la facture.

%ENVIE
\section{Envie}
Le désir de ce qu’on n’a pas et qu’un autre possède, joint au désir
d’être cet autre ou de prendre sa place. Il y a de la haine dans l’envie,
presque toujours. Et de l’envie dans la haine, bien souvent. Comme on nous
pardonne mieux nos échecs que nos réussites ! Cela vaut spécialement dans la
vie intellectuelle. La haine s’accroît en proportion du succès.

%ÉPICURISME
\section{Épicurisme}
La doctrine d’Épicure et de ses disciples. C’est un matérialisme
radical, qui prolonge l’atomisme de Démocrite : rien
n'existe que les atomes en nombre infini dans le vide infini ; rien n’advient que
leurs mouvements ou leurs rencontres. C’est aussi un sensualisme paradoxal,
puisque les atomes et le vide, qui font toute la réalité, sont insensibles. C’est
enfin, et surtout, un hédonisme exigeant : le plaisir, qui est le seul bien, culmine
dans ces plaisirs de l’âme que sont la philosophie, la sagesse et l'amitié.
Cette âme n’est bien sûr qu’une partie du corps, composée d’atomes simplement
plus mobiles que les autres : elle mourra avec lui. Pas d’autre vie que
celle-ci. Pas d’autre récompense que le plaisir de bien la vivre. Nulle providence.
Nul destin. Nulle finalité. Notre monde? Ce n’est qu’un agrégat
d’atomes, qui est né par hasard, qui aura nécessairement une fin. Les dieux ? Ils
sont aussi matériels que le reste, et incapables d’ordonner une nature qu’ils
n'ont pas créée et qui les contient. Au reste, ils ne s’occupent pas des humains :
leur propre bonheur leur suffit. À nous, qui ne sommes pas des dieux, de
prendre modèle sur eux. Cela passe par un «quadruple remède» (le
{\it tetrapharmakon}) : comprendre que la mort n’est rien, qu’il n’y a rien à craindre
des dieux, qu’on peut supporter la douleur, qu’on peut atteindre le bonheur.
Le remède est simple. Le chemin, toutefois, ne va pas de soi : il suppose que
nous renoncions aux désirs vains, ceux qui ne peuvent être rassasiés (désirs de
%— 206 —
%{\footnotesize XIX$^\text{e}$} siècle — {\it }
gloire, de pouvoir, de richesse), pour nous consacrer aux désirs naturels et
nécessaires (manger, boire, dormir, philosopher...), qui sont bornés et faciles à
satisfaire. Beaucoup ont cru que l’hédonisme épicurien débouchait ici sur une
espèce d’ascétisme. C’est se méprendre. S’il faut renoncer à jouir toujours plus,
ce n’est pas par dédain du plaisir ou fascination pour l'effort : c’est pour jouir au
mieux. « Épicure, nous dit Lucrèce, fixa des bornes au désir comme à la
crainte. » Les deux vont ensemble. Si tu désires n’importe quoi, tu auras peur de
tout. Si tu ne désires que ce qui est à portée de main ou d'âme, tu n’auras peur
de rien. L’épicurisme n’est pas un ascétisme ; c’est un hédonisme {\it a minima}.
Encore ne l’est-il que relativement aux objets de la jouissance. Car la jouissance
elle-même, libérée du manque et de la peur, est une jouissance maximale : « Du
pain d’orge et de l’eau donnent le plaisir extrême, écrivait Épicure, lorsqu'on les
porte à sa bouche en ayant faim ou soif. » Ainsi le plaisir est « le commencement
et la fin de la vie heureuse », mais pour celui seulement qui sait choisir entre ses
désirs. C’est la sagesse la plus simple et la plus difficile : l’art de jouir (plaisirs du
corps) et de se réjouir (plaisirs de l’âme) sereinement — « comme un dieu parmi
les hommes ». C’est où l’hédonisme mène à l’eudémonisme.

%ÉPISTÉMOLOGIE
\section{Épistémologie}
C’est la partie de la philosophie qui porte non sur le
savoir en général (théorie de la connaissance, gnoséologie),
mais sur une ou plusieurs sciences en particulier. Une théorie de la
connaissance se situe plutôt en amont du savoir : elle se demande à quelles
conditions les sciences sont possibles. Une épistémologie, en aval : elle s’interroge
moins sur les conditions des sciences que sur leur histoire, leurs méthodes,
leurs concepts, leurs paradigmes. Elle sera le plus souvent régionale ou plurielle
(l’épistémologie des mathématiques n’est pas celle de la physique, qui n’est pas
celle de la biologie...). C’est philosophie appliquée, comme elle est presque
toujours, mais sur le terrain des sciences plutôt que sur le sien propre. L’épistémologue
campe en terre étrangère (c’est ordinairement un scientifique qui s’est
lancé dans la philosophie, ou un philosophe qui s'intéresse aux sciences). Il
méprise un peu les autochtones, ceux qui ne parlent qu’une seule langue, qui
ne connaissent qu’un seul continent, où ils se croient chez eux. Il voudrait leur
faire la leçon. Eux l’écoutent, quand ils en ont le temps, avec la politesse
quelque peu condescendante qu’on réserve aux touristes étrangers.

%{\it ÉPOCHÉ}
\section{Époché}
Le mot, en grec, signifie arrêt ou interruption. Dans le langage
philosophique, où l’on omet souvent de le traduire, il désigne la
suspension du jugement (spécialement chez les sceptiques) ou la mise entre
%— 207 —
%{\footnotesize XIX$^\text{e}$} siècle — {\it }
parenthèses du monde objectif (spécialement chez les phénoménologues) : c’est
alors s'abstenir de toute position portant sur le monde, pour ne plus laisser
paraître que l’absolu de la conscience ou de la vie. Synonyme, en ce dernier
sens, de réduction phénoménologique (Husserl, {\it Méditations cartésiennes}, I, 8).

%ÉQUIPE
\section{Équipe}
Un petit groupe organisé, en vue d’une fin commune. Le contraire
d’une foule, et le moyen parfois de la contrôler ou de lui plaire.

%ÉQUITÉ
\section{Équité}
La vertu qui permet d’appliquer la généralité de la loi à la singularité
des situations concrètes : c’est « un correctif de la loi », écrit
Aristote ({\it Éthique à Nicomaque}, V, 14), qui permet d’en sauver l'esprit quand la
lettre n’y suffit pas. C’est justice appliquée, justice en situation, justice vivante,
et la seule qui soit vraiment juste.

En un second sens, plus vague, le mot finit par désigner la justice elle-même,
en tant qu'elle ne peut se réduire ni à l'{\it égalité} (donner ou demander à
tous les mêmes choses, ce ne serait pas juste : ils n’ont ni les mêmes besoins ni
les mêmes capacités) ni à la {\it légalité} (puisqu’une loi peut être injuste). Disons
que c’est la justice au sens moral du terme, qui permet seule de juger l’autre.
On remarquera qu’elle porte pourtant l'égalité dans son nom ({\it aequus}, égal).
Cela dit sans doute l'essentiel. Elle est la vertu qui restaure légalité de droit,
non seulement contre les inégalités de fait, qui demeurent, mais en en tenant
compte. Par exemple en matière de fiscalité : un impôt progressif, qui taxe plus
lourdement les plus riches, est plus équitable qu’un impôt simplement proportionnel,
qui demanderait à chacun la même portion de son revenu. C’est considérer
que les hommes sont égaux en droits et en dignité, même quand ils sont
inégaux, comme presque toujours, en talents, pouvoirs, richesses.

%ÉQUIVOQUE
\section{Équivoque}
Synonyme à peu près d’ambiguïté, avec quelque chose de
plus inquiétant : l’équivoque est comme une ambiguïté
volontaire, néfaste ou menaçante. Une situation ambiguë, par exemple entre
un homme et une femme, peut être charmante. Une situation équivoque sera
plutôt gênante ou suspecte. C’est une ambiguïté mal intentionnée ou mal
interprétée.

%ÉRISTIQUE
\section{Éristique}
L’art de la controverse ({\it erizein}, disputer), ou ce qui en relève.
Mais c'est moins un art qu’un artifice. « Est éristique, écrit
%— 208 —
%{\footnotesize XIX$^\text{e}$} siècle — {\it }
Aristote, le syllogisme qui part d'opinions qui, tout en paraissant probables, en
réalité ne le sont pas » ({\it Topiques}, I, 1). Le substantif peut devenir un synonyme
de sophistique. Mais il désigne plus spécifiquement un type d’argumentation
critique, qui tend moins à établir une vérité qu’à réfuter la position d’un adversaire.
Les mégariques s’en étaient fait une espèce de spécialité : de là le nom
d’école éristique qu’on leur donne parfois.

%ÉROTISME
\section{Érotisme}
L'art de jouir ? Plutôt l’art de désirer, et de faire désirer, jusqu’à
jouir du désir même (le sien, celui de l’autre) pour en obtenir
une satisfaction plus raffinée ou plus durable. L’orgasme est à la portée de
n'importe qui: chacun, pour soi-même, y suffit. Mais qui voudrait s’en
contenter ?

%ERREUR
\section{Erreur}
Le propre de l’erreur est qu’on la prend pour une vérité. C’est ce
qui distingue l'erreur de la simple fausseté (on peut savoir qu’on
dit faux, non savoir qu’on se trompe) et qui lui interdit d’être volontaire. Ainsi
l'erreur n’est pas seulement une idée fausse : c’est une idée fausse qu’on croit
vraie. En tant qu’elle est fausse, elle n’a d’être que négatif (voir {\it fausseté}). Mais
en tant qu’elle est idée, elle fait partie du réel ou du vrai (on se trompe
réellement : l'erreur est vraiment fausse). Par exemple, explique Spinoza, « les
hommes se trompent en ce qu’ils se croient libres ; et cette opinion consiste en
cela seul qu’ils ont conscience de leurs actions et sont ignorants des causes par
où ils sont déterminés » ({\it Éthique}, II, 35, scolie). Ce n’est pas parce que nous
sommes libres que nous nous trompons, comme le voulait Descartes ; c’est
parce que nous nous trompons que nous nous croyons libres, et cette erreur
n’est elle-même qu’une vérité incomplète (puisqu'il est vrai que nous agissons).
On ne se trompe que par ignorance ou impuissance. L'erreur n’est rien de
positif : il n’y a que des connaissances partielles ou inachevées. Par quoi la
pensée est un travail, et l'erreur, un moment nécessaire.

%ESCHATOLOGIE
\section{Eschatologie}
La doctrine des fins dernières de l’humanité ou du
monde. Le mot {\it fin} se prenant en deux sens (comme
{\it finitude} ou comme {\it finalité}), il semble que l’eschatologie doive aussi être
double : la mort et la fin du monde en relèveraient, autant que le jugement dernier
ou la résurrection. En pratique, toutefois, on ne parle guère d’eschatologie
que pour des pensées finalistes ou religieuses. C’est que le néant ne fait pas sens.
Une eschatologie matérialiste, comme on voit chez Lucrèce, serait plutôt une
%— 209 —
%{\footnotesize XIX$^\text{e}$} siècle — {\it }
anti-eschatologie : d’abord parce que l’univers est sans fin ({\it De rerum}, I, 1001),
ensuite parce que aucun monde, dans l’univers, n’échappe à la mort (II, 1144-1174,
V, 91-125 et 235-415), ni aucune vie dans le monde (III, 417-1094). Le
salut n’est pas à espérer, il est à faire.

%ÉSOTÉRIQUE
\section{Ésotérique}
Ce qui est réservé aux initiés ou aux spécialistes. Le mot,
pris en lui-même, n’est pas péjoratif. Mais il le devient, légitimement,
si l'initiation est elle-même réservée à certains, et spécialement si elle
suppose une foi préalable : c’est soumettre l’universel au particulier, l’école à la
secte, et l'esprit au gourou.

%ÉSOTERISME
\section{Ésotérisme}
Toute doctrine qui réserve la vérité aux initiés. C’est le
contraire des Lumières, et un proche parent, presque toujours,
de l’occultisme. C’est un obscurantisme savant, ou qui voudrait l’être.

%ESPACE
\section{Espace}
Ce qui reste quand on a tout ôté : le vide, mais à trois dimensions.
On voit que ce n’est qu’une abstraction (si l’on ôtait vraiment tout,
il n’y aurait plus rien : ce ne serait pas l’espace mais le néant), qu’on conçoit
davantage qu’on ne l’expérimente. C’est l'étendue, mais considérée indépendamment
des corps qui l’occupent ou la délimitent. C’est l’univers, mais considéré
indépendamment de son contenu (indépendamment de lui-même !).
L'espace est à l’étendue ce que le temps est à la durée : son abstraction,
qu'on finit par prendre pour son lieu ou sa condition. S'il n’y avait pas le
temps, se demande-t-on, comment les corps pourraient-ils durer ? S’il n’y avait
pas l’espace, comment pourraient-ils s'étendre ? C’est soumettre le réel à la
pensée, quand c’est le contraire qu’il faut faire. Ce n’est pas parce qu’il y a du
temps que l'être dure ; c’est parce qu’il dure qu’il y a du temps. Ce n’est pas
parce qu'il y a de l’espace que l’être est étendu ; c’est parce qu’il est étendu — ou
parce qu'il s’étend — qu’il y a de l’espace. L'espace n’est pas un être ; c’est le lieu
de tous, considéré en faisant abstraction de leur existence ou de leur localisation.
Non un être, donc, mais une pensée : c’est le lieu universel et vide. Comment
ne serait-il pas infini, continu, homogène, isotrope et indéfiniment divisible,
puisque aucun corps, par définition, ne peut le limiter, le rompre ou
l’orienter ? Mais cela nous en apprend plus sur notre pensée que sur l'étendue
du réel ou de l’univers.
L'espace est-il seulement une forme de la sensibilité, comme le voulait
Kant ? Ce n’est guère vraisemblable : s’il n’était que cela, {\it où} la sensibilité aurait-elle
%— 210 —
%{\footnotesize XIX$^\text{e}$} siècle — {\it }
pu apparaître ? Et comment penser l’extension de l’univers, des milliards
d’années avant l'existence de toute vie et de toute sensibilité ? On dira que cela
ne prouve rien, puisque cette extension et ces milliards d’années n’existent pour
nous qu’à partir de notre sensibilité. J’en suis d’accord : l’idéalité transcendantale
de l’espace n’est pas réfutable. Mais son objectivité ne l’est pas davantage
(que l’espace soit une forme de la sensibilité, cela n’empêche pas qu’il soit aussi
une forme de l’être), tout en étant plus économe : elle ne suppose pas l'existence
d’un être non spatial, comme Kant est obligé de faire, pour rendre
l’espace pensable.

L'espace, comme le temps, contient tout, mais dans la simultanéité d’un
même présent : c’est « l’ordre des coexistences, écrivait Leibniz, comme le
temps est l’ordre des successions ». On remarquera qu’au présent les deux ne
font qu’un, qu’on peut appeler (comme font les physiciens, mais pour d’autres
raisons) l’espace-temps. C’est le lieu des présences, ou le présent comme lieu.

%ESPÈCE
\section{Espèce}
Un ensemble, le plus souvent défini, à l’intérieur d’un ensemble
plus vaste (le genre prochain), par une ou plusieurs caractéristiques
communes (les différences spécifiques). Par exemple l’espérance et la volonté
sont deux espèces de désir, comme les tigres et les chats deux espèces de félins.
Pour le biologiste, une espèce se reconnaît ordinairement à l’interfécondité :
deux individus de sexes différents appartiennent à une même espèce s’ils peuvent
se reproduire et engendrer un être lui-même fécond (par quoi l’âne et le
cheval sont deux espèces différentes : mulets et bardots sont stériles). C’est
pourquoi il vaut mieux parler de l'{\it espèce humaine} que du {\it genre humain}. Que
l'unité de l'humanité soit une valeur morale, cela n'empêche pas qu’elle soit
aussi et d’abord un fait biologique.

%ESPÉRANCE
\section{Espérance}
Une certaine espèce de désir : c’est un désir qui porte sur ce
qu'on n’a pas, ou qui n’est pas (espérer c’est désirer sans
jouir), dont on ignore s’il est ou s’il sera satisfait (espérer c’est désirer sans
savoir), enfin dont la satisfaction ne dépend pas de nous (espérer, c’est désirer
sans pouvoir). S’oppose pour cela à la volonté (un désir dont la satisfaction
dépend de nous), à la prévision rationnelle (lorsque l’avenir peut faire l’objet
d’un savoir ou d’un calcul de probabilités), enfin à l’amour (lorsqu’on désire ce
qui est ou ce dont on jouit). Cela indique assez le chemin : ne t’interdis pas
d’espérer ; apprends plutôt à vouloir, à connaître, à aimer.
L’espérance porte le plus souvent sur l’avenir : c’est qu’il est, de tous nos
objets de désirs, celui qui est le plus souvent soustrait à toute jouissance, à toute
%— 211 —
%{\footnotesize XIX$^\text{e}$} siècle — {\it }
connaissance et à toute action possibles. Le passé est mieux connu. Le présent,
plus disponible. Cela n'empêche pas d’espérer ce qui fut (« j’espère que je ne
l'ai pas blessé ») ou ce qui est (« j'espère qu’il est guéri ») : il suffit pour cela
qu'on le désire, que cela ne dépende pas de nous et qu’on ignore ce qu’il en est.
L'orientation temporelle est moins essentielle à l’espérance que l'impuissance et
l'ignorance : nul n’espère ce qu’il sait ni ce qu’il peut. L’espérance, marque de
notre faiblesse. Comment serait-ce une vertu ? C’est le désir le plus facile et le
plus faible.

%ESPOIR
\section{Espoir}
Souvent synonyme d’espérance. Lorsqu'on veut les distinguer, c’est
presque toujours au bénéfice de cette dernière : l’espérance serait
une vertu, l'espoir ne serait qu’une passion. C’est le cas spécialement dans la
théologie chrétienne, où l’espérance est l’une des trois vertus théologales, parce
qu'elle a Dieu même pour objet. Qu’en conclure ? Qu’à chaque fois que
j'espère autre chose que Dieu, ou autrement qu’en Dieu, ce n’est pas une espérance,
ce n’est qu’un espoir, passionnel et vain comme ils sont tous. Et que
cette distinction n’a guère de sens pour un philosophe non religieux : les Grecs
ne la faisaient pas, et je ne vois nulle raison de la faire.

Spinoza ne la faisait pas davantage. Qu'est-ce que l'espoir ? « Une joie
inconstante, répondait-il, née de l’idée d’une chose future ou passée, de l’issue
de laquelle nous doutons en quelque mesure. » C’est pourquoi, selon une formule
fameuse de l’{\it Éthique}, « il n’y a pas d’espoir sans crainte ni de crainte sans
espoir » (III, 50, scolie, et déf. 13 des affects, explication). Le même doute,
nécessaire à l’un et l’autre, fait qu’ils ne peuvent exister qu’ensemble. Espérer,
c'est craindre d’être déçu ; craindre, c’est espérer d’être rassuré. La sérénité, si
elle exclut la crainte, exclut donc aussi tout espoir : ce que j’ai appelé le {\it gai
désespoir}, et que Spinoza, plus sage que moi, appelle la sagesse ou la béatitude.
Le thème est stoïcien avant d’être spinoziste : « Tu cesseras de craindre, disait
Hécaton, si tu as cessé d’espérer. » Et cynique avant d’être stoïcien : « Seul est
libre, disait Démonax, celui qui n’a ni espoir ni crainte. » Le sage n’espère rien :
il a cessé d’avoir peur. Il ne craint rien : il a cessé d’espérer quoi que ce soit.
Parce qu’il serait sans désirs ? Au contraire : parce qu’il ne désire que ce qui est
(ce n’est plus espérance mais amour) ou que ce qu’il peut (ce n’est plus espérance
mais volonté).

On dira que cette sagesse est pour nous hors d’atteinte : l'espoir est là, toujours,
puisque la faiblesse est là, puisque l'ignorance est là, puisque l’angoisse
est là. Sans doute. Aussi ne sert-il à rien, je l’ai dit bien souvent, de vouloir
s’'amputer vivant de toute espérance : ce serait faire de la sagesse un nouvel
espoir, qui nous en séparerait aussitôt. Développons plutôt notre part de puissance,
%— 212 —
%{\footnotesize XIX$^\text{e}$} siècle — {\it }
de liberté, de joie : apprenons à connaître, à agir, à aimer. La sagesse
n’est pas un idéal ; c’est un processus. « Plus nous nous efforçons de vivre sous
la conduite de la raison, écrit Spinoza, plus nous faisons effort pour nous
rendre moins dépendants de l'espoir, nous affranchir de la crainte, commander
à la fortune autant que nous le pouvons, et diriger nos actions suivant le sûr
conseil de la raison » ({\it Éthique}, IV, scolie de la prop. 47).

%ESPRIT
\section{Esprit}
La puissance de penser, en tant qu’elle a accès au vrai, à l’universel
ou au rire.

Le mot, en ce sens, ne s’utilise guère qu’au singulier (parler {\it des esprits}, c'est
superstition). C’est que la vérité, pour autant qu’on y accède, est la même en
tous. C’est en quoi elle est libre (elle n’obéit à personne, pas même au cerveau
qui la pense), et libère. Cette liberté en nous, qui n’est pas celle d’un sujet mais
de la raison, c’est l’esprit même.

On se trompe si l’on y voit une substance, mais pas moins si l’on n’y voit
qu’un pur néant. L'esprit n’est pas une hypothèse, disait Alain, puisqu'il est
incontestable que nous pensons. Ni une substance, puisqu'il ne peut exister
seul. Disons que c’est le corps en acte, en tant qu'il a la vérité en puissance.

En puissance, non en acte. Aucun esprit n’est la vérité ; aucune vérité n'est
esprit (ce serait Dieu). C’est pourquoi l'esprit doute de lui-même et de tout. Il
sait qu’il ne sait pas, ou qu’il ne sait que peu. Il s’en inquiète ou s’en amuse.
Deux façons (par la réflexion, par le rire) d’être fidèle à soi, sans se croire.
L'esprit, sous ces deux formes, semble le propre de l’homme. C’est aussi une
vertu : celle qui surmonte le fanatisme et la bêtise.

%ESPRIT FAUX
\section{Esprit faux}
Un don particulier pour l’erreur. C’est l'incapacité à raisonner
juste, au moins sur certaines questions, et quelque intelligent
qu’on puisse être par ailleurs. C’est comme un manque de bon sens, qui
autoriserait À penser n'importe quoi. Qui disait de Sartre qu’il était « un grand
esprit faux » ? Cela ne l’empêchait pas d’avoir du talent, et sans doute davantage
que l'inventeur de la formule. «Les plus grands génies peuvent avoir
l'esprit faux sur un principe qu’ils ont reçu sans examen, notait Voltaire.
Newton avait l'esprit faux quand il commentait l'Apocalypse. » Et Voltaire,
pour d’autres raisons, quand il parle de l'Ancien Testament.

%ESSENCE
\section{Essence}
Le mot, qui semble mystérieux, a pourtant une étymologie transparente :
il est forgé à partir de l’infinitif du verbe {\it être} en latin
%— 213 —
%{\footnotesize XIX$^\text{e}$} siècle — {\it }
{\it (esse)}. En l'occurrence, l’étymologie trompe moins que le mystère : l’essence
d’une chose, c’est son être vrai ou profond (par opposition aux apparences, qui
peuvent être superficielles ou trompeuses), autrement dit ce qu’elle est (par
opposition au simple fait qu’elle soit : son existence ; mais aussi à ce qui lui
arrive : ses accidents). Synonyme à peu près de nature (l'essence d’une chose,
c'est sa nature véritable), mais préférable : parce que le mot peut s’appliquer
aussi à des objets culturels. L’essence d’un homme, par exemple, c’est ce qu’il
est. Qui peut croire que la nature suffise à l'expliquer ?

L’essence, c’est donc ce qui répond aux questions « Quoi ? » ou « Qu'est-ce
que c’est ? » ({\it Quid ?}, en latin, ce pourquoi les scolastiques parlaient plutôt de
{\it quiddité}). Reste à savoir si ce qui répond à cette question c’est une définition
ou un être, et si cet être est individuel ou générique. Soit cette table sur laquelle
j'écris. Quelle est son essence ? D’être une table, ou d’être cette table-ci ? Des
mots, ou du réel ? Une idée, ou un processus ? Il se pourrait qu’il n’y ait pas
d’essences du tout, mais seulement des accidents, des rencontres, de l’histoire —
non des êtres, mais des événements. On dira que pour que quelque chose
arrive, il faut déjà que quelque chose soit. Sans doute. Mais pourquoi serait-ce
autre chose que cela même qui arrive ?

C'est où l’on retrouve Spinoza. Qu'est-ce que l’essence d’une chose
singulière ? Non, du tout, une abstraction ou une virtualité, mais son être
même, considéré dans sa dimension affirmative, autrement dit dans sa puissance
d'exister : ce qui la fait être (ce qui la « pose », écrit Spinoza), mais de
l’intérieur (à la différence des causes qui la font être de l'extérieur). Par exemple
cette table : qu’elle ait des causes externes, c’est une évidence ; mais elle n’existerait
pas si elle n’avait en elle-même une essence affirmative — une puissance
d’être — qui ne peut pas plus exister ou être conçue sans la table que la table ne
peut exister ou être conçue sans elle ({\it Éthique}, II, déf. 2). Il en résulte que
« l'effort par lequel chaque chose s’efforce de persévérer dans son être n’est rien
en dehors de l’essence actuelle de cette chose » ({\it Éth.}, III, prop. 7) : l'essence
d’un être, c’est sa puissance d’exister ; son existence, c’est son essence en acte.

Toute la difficulté est de penser les deux ensemble, dans leur simultanéité
nécessaire — ce qui récuse l’essentialisme aussi bien que l’existentialisme. « L’essence,
écrit Sartre après Hegel, c’est ce qui a été» ({\it L'être et le néant}, p.72
et 577). Mais comment, si le passé n’est plus ? L’être et l'événement au présent
ne font qu’un : ainsi l'essence et l’existence.

%ESSENTIALISME
\section{Essentialisme}
Le contraire de l’existentialisme et du nominalisme : c’est
croire que l'essence précède l'existence, ou qu’un être est
contenu dans sa définition. Confiance exagérée dans le langage ou la pensée.
%— 214 —
%{\footnotesize XIX$^\text{e}$} siècle — {\it }
C’est le péché mignon des philosophes. La critique de l’essentialisme, bien
avant Popper ou Sartre, a été énoncée au plus court par Shakespeare : «Il y a
plus de choses dans le ciel et sur la terre, Horatio, que n’en rêve ta
philosophie. » Ou encore : « Qu’y a-t-il donc en un nom ? Ce que nous appelons
une rose, sous un autre nom, sentirait aussi bon. » Non qu’on doive se
passer de définitions, ni même qu’on le puisse ; mais parce que aucune définition
ne tient lieu d’expérience, ni d’existence.

%ESTHÈTE
\section{Esthète}
Celui qui aime le beau — et spécialement le beau artistique — plus
que tout, au point de lui sacrifier ou de lui soumettre tout le
reste. Le vrai ? Le bien ? Ils ne valent, pour l’esthète, que s’ils sont beaux :
mieux vaut un beau mensonge qu’une vérité laide ; mieux vaut un beau crime
qu’une faute de goût. L’esthète n’est pas toujours un artiste, ni souvent (la plupart
des grands créateurs mettent le vrai ou le bien plus haut que le beau). C’est
un croyant. Il a fait de l’art sa religion : l'esthétique lui tient lieu de logique, de
morale, de métaphysique. Philosophiquement, cela culmine chez Nietzsche :
« Pour nous, seul le jugement esthétique fait loi », écrit-il. Et d’ajouter : « L'art
et rien que l’art ! C’est lui qui nous permet de vivre, qui nous persuade de vivre,
qui nous stimule à vivre. L'art a {\it plus de valeur} que la vérité. L'art au service
de l'illusion — voilà notre culte » ({\it Volonté de puissance}, IV, 8, et III, 582). Ce
qui résume au fond l'essentiel, par quoi Nietzsche est un esthète, et qui
m’empêche d’être nietzschéen.

%ESTHÉTIQUE
\section{Esthétique}
L’étude ou la théorie du beau. On considère habituellement
que c’est une partie de la philosophie, plutôt qu'un des
Beaux-Arts. C’est justice. Le concept de beau n’est pas beau. Le concept
d'œuvre d’art n’en est pas une. C’est pourquoi les artistes se méfient des esthéticiens,
qui prennent le beau pour une pensée.

%ESTHÉTIQUE TRANSCENDANTALE
\section{Esthétique transcendentale}
La première partie de la {\it Critique
de la raison pure} de Kant. Elle ne
porte pas sur le beau, qui sera étudié dans la {\it Critique de la faculté de juger}, mais
sur la sensation ({\it aisthèsis}) ou la sensibilité. Elle est transcendantale en tant
qu’elle fait ressortir les conditions de possibilité de toute expérience (l’espace et
le temps comme formes {\it a priori} de la sensibilité). Se distingue de la Logique
transcendantale, comme les formes de la sensibilité se distinguent des formes de
la pensée (les catégories et principes de l’entendement).

%— 215 —
%{\footnotesize XIX$^\text{e}$} siècle — {\it }
%ESTIME
\section{Estime}
C’est un respect particulier : non celui qu’on doit à tout être
humain, mais celui qu’on réserve à ceux qu’on juge les meilleurs,
tant que leur valeur ne passe pas la norme commune ou la nôtre (auquel cas ce
n'est plus estime mais admiration). L’estime manifeste une sorte d'égalité positive,
qui en fait le prix. Ce n’est pas encore l’amitié, mais presque toujours une
de ses conditions. Je peux estimer sans aimer. Mais comment aimer celui ou
celle que je méprise ?

%ÉTANT
\section{Étant}
L'être en train d’être : l'être au présent, et le seul. L’usage substantivé
de ce participe présent, quoique heurtant nos oreilles, permet
d'éviter l'ambiguïté du mot {\it être}, qui en français désigne à la fois {\it ce qui est} (le {\it to
on} des Grecs, l’{\it ens} des Latins ou des scolastiques : l’étant) et le fait que ce qui
est {\it soit} (acte d’être : {\it to einai} ou {\it esse}). Dans son grand livre sur saint Thomas,
Étienne Gilson notait que ce dernier disposait de « deux vocables distincts,
pour désigner un étant, {\it ens}, et pour désigner l’acte même d’être, {\it esse} », ce qui
n'est pas le cas du français et rend un certain nombre de traductions, faute de
faire cette différence, inintelligibles. « La seule solution satisfaisante du problème,
ajoutait Gilson en note, serait d’avoir le courage de reprendre la terminologie
essayée au {\footnotesize XVII$^\text{e}$} siècle par quelques scolastiques français, qui traduisaient
{\it ens} par {\it étant}, et {\it esse} par {\it être} » ({\it Le thomisme}, Vrin, 1979, p. 170). Cet
usage tend aujourd’hui à se répandre, mais davantage par l'influence du {\it Seiend}
allemand que par celle de saint Thomas ou de Gilson. Chez Heidegger et les
heideggériens, l’étant (ce qui est : cette table, cette chaise, vous, moi...) se distingue
en effet de l’être — c’est la fameuse « différence ontologique » — sans être
pourtant autre chose. Que l'arbre soit un arbre, et cet arbre-ci, c’est sa banalité
d’étant. Mais qu’il {\it soit}, c’est l'événement de l'être. Ainsi l’étant est l’être même,
quand on s'interroge sur ce qu’il est au lieu de s'étonner de ce qu’il soit. Et
réciproquement : l’être est l’étant, quand on s’étonne qu’il {\it soit} au lieu de chercher
seulement {\it ce} qu’il est ou à quoi il peut servir.

%ÉTAT
\section{État}
Une façon d’être. Pris absolument, et avec une majuscule, c’est un
corps politique, qui rassemble un certain nombre d'individus (le
peuple) sous un même pouvoir (le souverain). Quand le peuple et le souverain
sont un, l’État est une république.

%ÉTAT CIVIL
\section{État civil}
Le contraire de l’état de nature : c’est la vie en société, en tant
qu’elle suppose un pouvoir et des lois.

%— 216 —
%{\footnotesize XIX$^\text{e}$} siècle — {\it }
%ÉTAT DE NATURE
\section{État de nature}
L'état sans État : situation des êtres humains avant
l'instauration d’un pouvoir commun, de règles communes,
voire avant toute vie en société. État purement hypothétique, vraisemblablement
insatisfaisant. « La vie de l’homme, disait Hobbes, est alors solitaire,
besogneuse, pénible, quasi animale, et brève » ({\it Léviathan}, I, 13).

%ÉTENDUE
\section{Étendue}
L‘étendue est à l’espace ce que la durée est au temps: son
contenu, sa condition, sa réalité. L’étendue d’un corps, c’est la
portion d’espace que ce corps occupe ; l’espace n’est que l’abstraction d’une
étendue qui existerait indépendamment des corps qui l’occupent ou la traversent.
C’est donc l’étendue qui est première : ce n’est pas parce que les corps
sont dans l’espace qu’ils sont étendus ; c’est parce qu’ils sont étendus, ou parce
qu’ils s'étendent, qu’il y a de l’espace. Cette étendue, on pourrait l’appeler aussi
bien, et peut-être mieux, l'{\it extension} (comme on pourrait appeler la durée
{\it duration}) : c’est le fait de s’étendre et d’occuper ainsi un certain espace.

%ÉTERNITÉ
\section{Éternité}
Si c'était un temps infini, quel ennui ! Cela donnerait raison à
Woody Allen : « L’éternité c’est long, surtout vers la fin... »
C’est qu’il n’y aurait pas de fin : on n’en aurait jamais fini d’attendre, et aucune
raison pour commencer quoi que ce soit. Cela ferait comme un dimanche
infini. Quelle plus belle image de l’enfer ?

Mais l'éternité, au sens où la prennent la plupart des philosophes, c’est tout
autre chose. Ce n’est pas un temps infini (car alors il ne serait composé que de
passé et d’avenir, qui ne sont pas), ni pourtant l’absence de temps (car alors ce
ne serait rien): c’est un présent qui reste présent, comme {\it un perpétuel
aujourd'hui}, disait saint Augustin, et c’est le présent même. Qui a jamais vécu
un seul {\it hier} ? un seul {\it demain} ? Qui a jamais vu le présent cesser ou disparaître ?
C’est toujours aujourd’hui, c’est toujours maintenant : c’est toujours l'éternité,
et c’est en quoi, en effet, elle est éternelle.

On ne confondra pas l'éternité avec l’immuabilité. Que tout change, c’est
une vérité éternelle. Mais rien ne change qu’au présent, et c’est l’éternité vraie.
On ne se baigne jamais deux fois dans le même fleuve ? Sans doute. Mais
encore moins dans un fleuve passé ou futur. Ainsi il n’y a que le présent : il n’y
a que l'éternité du {\it il y a}. Parménide et Héraclite, même combat !

L’éternité peut se penser de deux façons, qu’on peut formuler, par commodité,
selon les deux attributs de Spinoza : selon l’étendue ou selon la pensée.
Selon l'étendue, l'éternité ne fait qu’un avec le devenir : c’est le toujours-présent
du réel (être, c’est être maintenant). Selon la pensée, elle ne fait qu’un avec
%— 217 —
%{\footnotesize XIX$^\text{e}$} siècle — {\it }
la vérité : c’est le toujours-présent du vrai (une vérité n’est jamais future ou
passée : ce qui était vrai l’est encore, ce qui le sera l’est déjà). C’est où le réel et
le vrai, pour la pensée, se séparent : ce qui était réel ne l’est plus, ce qui était vrai
l’est toujours. Par exemple la promenade que je fis hier : ce n’est plus réel, c’est
toujours vrai. Ou celle que je ferai demain, si j’en fais une : ce n’est pas encore
réel, c’est déjà vrai. On évitera pourtant d’absolutiser cette différence. Le réel et
le vrai ne coïncident qu’au présent, certes ; mais ils coïncident donc {\it toujours},
pour tout réel donné, et {\it nécessairement}. Ainsi ces deux éternités n’en font
qu'une (le présent est le lieu de leur conjonction : le point de tangence du réel
et du vrai). C’est en quoi je suis libre de me promener ou non aujourd’hui : ce
n'est pas parce que c'était déjà vrai de toute éternité que je le ferai au présent ;
c'est parce que je le fais au présent, si je le fais, que c’est vrai de toute éternité.
Le réel commande : le présent commande, puisqu'il n’y a rien d’autre, et c’est
en quoi les deux attributs, au présent, ne font qu’un. Pluralité des attributs,
dirait Spinoza, unité de la substance ou de la nature. L’éternité n’est pas un
autre monde ; c’est la vérité de celui-ci.

%ÉTHIQUE
\section{Éthique}
C’est souvent un synonyme de morale, en plus chic. Mieux vaut
donc, quand on ne les distingue pas, parler plutôt de morale.
Mais si on veut les distinguer ? L’étymologie ne nous aide guère. « Morale » et
« éthique » viennent de deux mots — {\it ethos} en grec, {\it mos} ou {\it mores} en latin — qui
signifiaient à peu près la même chose (les mœurs, les caractères, les façons de
vivre et d’agir) et que les Anciens considéraient comme la traduction l’un de
l’autre. Aussi est-ce une distinction qu’ils ne faisaient pas : {\it morale} et {\it éthique} ne
seraient pour eux, si nous les interrogions en français, que deux façons différentes —
l’une d’origine grecque, l’autre d’origine latine — de dire la même
chose. Si l’on veut pourtant se servir de ces deux mots pour penser deux réalités
différentes, comme un usage récent nous y pousse, le plus opératoire est sans
doute de prendre au sérieux ce que l’histoire de la philosophie nous propose de
plus clair : Kant, parmi les Modernes, est le grand philosophe de la morale ; et
Spinoza, de l'éthique. Sans reprendre en détail ce que j'ai montré ailleurs
({\it Valeur et vérité}, chap. 8), cela amène à opposer la morale et l’éthique comme
l'absolu (ou prétendu tel) et le relatif, comme l’universel (ou prétendu tel) et le
particulier, enfin comme l’inconditionnel (impératif catégorique de Kant) et
le conditionné (qui n’admet d’impératifs qu'hypothétiques). En deux mots : la
morale commande, l'éthique recommande. Ces oppositions débouchent sur
deux définitions différentes, que je ne fais ici que rappeler :

Par {\it morale}, j'entends le discours normatif et impératif qui résulte de
l’opposition du Bien et du Mal, considérés comme valeurs absolues ou transcendantes.
%— 218 —
%{\footnotesize XIX$^\text{e}$} siècle — {\it }
Elle est faite de commandements et d’interdits : c’est l’ensemble de
nos devoirs. La morale répond à la question « Que dois-je faire ? ». Elle se veut
une et universelle. Elle tend vers la vertu et culmine dans la sainteté (au sens de
Kant : au sens où une volonté sainte est une volonté conforme en tout à la loi
morale).

Et j'entends par {\it éthique} un discours normatif mais non impératif (ou sans
autres impératifs qu'hypothétiques), qui résulte de l'opposition du {\it bon} et du
{\it mauvais}, considérés comme valeurs simplement relatives. Elle est faite de
connaissances et de choix : c’est l’ensemble réfléchi et hiérarchisé de nos désirs.
Une éthique répond à la question « Comment vivre ? ». Elle est toujours particulière
à un individu ou à un groupe. C’est un art de vivre : elle tend le plus
souvent vers le bonheur et culmine dans la sagesse.

L'erreur, entre l’une et l’autre, serait de vouloir choisir. Nul ne peut se
passer d'éthique, puisque la morale ne répond que très incomplètement à la
question « Comment vivre ? », puisqu'elle ne suffit ni au bonheur ni à la
sagesse. Et seul un sage pourrait se passer de morale : parce que la connaissance
et l’amour lui suffiraient. Nous en sommes loin, et c’est pourquoi nous avons
besoin de morale (voir ce mot).

L’éthique est pourtant la notion la plus vaste. Elle inclut la morale, alors
que la réciproque n’est pas vraie (répondre à la question « Comment vivre ? »,
c’est entre autres choses déterminer la place de ses devoirs ; répondre à la question
« Que dois-je faire ? », cela ne suffit pas à dire comment vivre). Elle est
aussi la plus fondamentale : elle dit la vérité de la morale (qu’elle n’est qu’un
désir qui se prend pour un absolu), et la sienne propre (qu’elle est comme une
morale désillusionnée et libre). Ce serait la morale de Dieu, s’il existait. Nous
ne pouvons ni tout à fait l’atteindre, ni tout à fait y renoncer.

Ainsi l'éthique est un travail, un processus, un cheminement : c’est le
chemin réfléchi de vivre, en tant qu’il tend vers la vie bonne, comme disaient
les Grecs, ou la moins mauvaise possible, et c’est la seule sagesse en vérité.

%ETHNIE
\section{Ethnie}
Un peuple, mais considéré d’un point de vue culturel plutôt que
biologique (ce n’est pas une race) ou politique (ce n’est ni une
nation ni un État).

%ETHNOCENTRISME
\section{Ethnocentrisme}
C'est juger les cultures des autres à partir de la sienne
propre, érigée (le plus souvent inconsciemment) en
absolu. Tendance spontanée de tout être humain, dont on ne sort, toujours
incomplètement, que par l’étude patiente et généreuse des autres cultures, ce
%— 219 —
%{\footnotesize XIX$^\text{e}$} siècle — {\it }
qui amène à relativiser celle dans laquelle on a été élevé. La difficulté est alors
de ne pas renoncer pour autant à toute exigence d’universalité, ni à toute normativité.
Si tous les points de vue se valaient, au nom de quoi combattre
l’ethnocentrisme ?

%ETHNOCIDE
\section{Ethnocide}
La destruction délibérée d’une culture. À ne pas confondre
avec le génocide, qui veut supprimer une race ou un peuple.
Par exemple au Tibet, occupé par la Chine : c’est l’ethnocide qui menace, non,
semble-t-il, le génocide.

%ETHNOGRAPHIE
\section{Ethnographie}
L’étude descriptive d’une ethnie ou en général d’un
groupe humain, considéré dans ses spécificités culturelles
ou comportementales. Se distingue de l’ethnologie par son aspect surtout
empirique : l’ethnographe observe et décrit ; l’ethnologue compare, classe,
interprète, théorise. L’ethnographie requiert une présence sur le terrain, souvent
pendant de très longues périodes ; l’ethnologie, s'appuyant sur les travaux
ethnographiques disponibles, peut se faire en chambre ou en amphithéâtre, ce
qui est tout de même plus confortable. En pratique, un ethnologue est souvent
un ethnographe qui a réussi ou qui s’est lassé des voyages.

On parle surtout d’ethnographie à propos de populations dites primitives.
Mais rien n'empêche, et cela se fait de plus en plus, d’appliquer les mêmes exigences
à l’étude d’une cité ouvrière, d’une entreprise ou d’un parti politique.
C’est alors une partie de la sociologie, davantage que de l’anthropologie : elle
nous en apprend moins sur l’homme que sur la société.

%ETHNOLOGIE
\section{Ethnologie}
L'étude comparative des ethnies et en général des groupes
humains. Correspond à peu près à ce que les Anglo-Saxons,
qui ne parlent plus guère d’ethnologie, appellent plutôt l’anthropologie sociale
et culturelle. L’ethnologie fait partie des sciences humaines : elle contribue à
nous faire mieux connaître l'humanité, en faisant ressortir un certain nombre
de différences mais aussi d’invariants structurels ou comportementaux.
Comme le remarque Lévi-Strauss, l’ethnologie prolonge une remarque de
Rousseau : « Quand on veut étudier les hommes, écrivait ce dernier, il faut
regarder près de soi ; mais pour étudier l’homme, il faut apprendre à porter la
vue au loin; il faut d’abord observer les différences pour découvrir les
propriétés » ({\it Essai sur l'origine des langues}, VIII, cité dans {\it La pensée sauvage}, IX).

%— 220 —
%{\footnotesize XIX$^\text{e}$} siècle — {\it }
L’un des apports décisifs de l’ethnologie est d’opérer un décentrement, qui
met l’ethnocentrisme à distance (quitte à en faire un objet d’étude). «Il faut
beaucoup d’égocentrisme et de naïveté, souligne Lévi-Strauss, pour croire que
l’homme est tout entier réfugié dans un seul des modes historiques ou géographiques
de son être, alors que la vérité de l’homme réside dans le système de
leurs différences et de leurs communes propriétés » ({\it La pensée sauvage}, XX ; voir
aussi {\it Anthropologie structurale deux}, XVIII, 3). L’ethnologie tend à l’universel,
comme toute science, mais par l'étude du particulier.

%ÉTHOLOGIE
\section{Éthologie}
Étude objective des mœurs ou des comportements, chez les
hommes comme chez les bêtes, sans aucune visée normative.
Ce dernier point est ce qui distingue l’éthologie de l'éthique, à peu près comme
l’objectivité de la biologie (pour laquelle la vie est un fait, non une valeur) la
distingue de la médecine (qui suppose la vie et la santé comme normes). Disons
que l’éthologie est une science, ou tend à en être une ; l'éthique serait plutôt un
art : c’est l’art de vivre le mieux qu’on peut.

C’est en quoi l'éthique de Spinoza, contrairement à ce qu’on en a dit, ne se
réduit nullement à une éthologie. Qu'il faille connaître et comprendre avant de
juger, c’est une évidence. Que connaissance et valeur soient irréductibles l’une
à l’autre, c’est un point essentiel du spinozisme (toute vérité est objective : elle
n’a que faire de nos désirs ; toute valeur est subjective : elle n’existe que pour
autant que nous la désirons). Mais dès lors que nous sommes essentiellement
des êtres de désirs, non de purs sujets connaissants, nous ne pouvons ni ne
devons renoncer à juger : ce serait nous prendre pour Dieu, et nous vouer par
là à l'illusion, ou bien renoncer à l'humanité, et nous vouer par là au malheur
ou à la barbarie. Cela n'empêche pas qu’il y ait dans l’{\it Éthique} un moment éthologique,
ou un point de vue éthologique, qui est fortement marqué : « Je considérerai
les actions et les appétits humains, écrit Spinoza au début du livre III,
comme s’il était question de lignes, de surfaces et de solides. » Mais les livres IV
et V montrent clairement que cela ne suffit pas : que nous avons besoin aussi
de « former une idée de l’homme qui soit comme un modèle de la nature
humaine placé devant nos yeux», en référence auquel nous jugerons les
hommes « plus ou moins parfaits », et les actions plus ou moins bonnes ou
mauvaises (IV, Préface). L’éthologie est nécessaire, non suffisante. Le but n’est
pas seulement de connaître les hommes, mais d’en devenir un point trop
imparfait, autrement dit de nous rapprocher le plus que nous pouvons de « la
liberté de l’âme ou béatitude » (V, Préface). À quoi l’éthologie peut et doit
contribuer, mais au service d’une visée normative (« un bien véritable », dit Spinoza)
qu’elle peut connaître, comme fait, mais qu’elle ne saurait à elle seule justifier
%— 221 —
%{\footnotesize XIX$^\text{e}$} siècle — {\it }
comme valeur. La sagesse n’est ni un absolu ni une science : elle ne vaut
que pour qui la désire ou s’efforce vers elle (III, 9, scolie). Ne compte pas sur la
vérité pour être sage à ta place.

%ÉTIOLOGIE
\section{Étiologie}
L'étude des causes. Se dit surtout en médecine, par différence
avec la sémiologie ou symptomatologie (l'étude des symptômes).

%ÉTONNEMENT
\section{Étonnement}
Au sens fort et classique : une surprise qui foudroie ou
frappe de stupeur. Au sens moderne : toute surprise qui
ne s'explique pas seulement par la soudaineté, mais bien par l'aspect étrange ou
mystérieux du phénomène considéré, C’est en ce sens que l’étonnement est
essentiel à la philosophie, qui s'étonne moins de ce qui est nouveau ou inattendu
que de ce qui résiste à l’évidence ou à la familiarité. Le philosophe
s'étonne de ce qui n’étonne pas, ou plus, la plupart de ses contemporains.
« C’est l’étonnement qui poussa, comme aujourd’hui, les premiers penseurs
aux spéculations philosophiques », remarquait déjà Aristote ({\it Métaphysique}, A,
2), et c’est ce que Jeanne Hersch, revenant sur vingt-cinq siècles de philosophie,
a brillamment confirmé ({\it L'étonnement philosophique, Une histoire de la
philosophie}, Gallimard, Folio-Essais, rééd. 1993). Par exemple l’existence du
monde est étonnante : non qu’elle soit soudaine ou imprévue, mais en ceci
qu'elle plonge l'esprit, pour peu qu’il s'interroge, dans une perplexité qui peut
aller, en effet, jusqu’à la stupeur. Pourquoi y a-t-il quelque chose plutôt que
rien ? Il en va de même de notre propre existence dans le monde. « Quand je
considère, écrit Pascal, la petite durée de ma vie, absorbée dans l'éternité précédente
et suivante, le petit espace que je remplis et même que je vois, abimé dans
l'infinie immensité des espaces que j'ignore et qui m’ignorent, je m’effraye et
m'étonne de me voir ici plutôt que là, car il n’y a point de raison pourquoi ici
plutôt que là, pourquoi à présent plutôt que lors. Qui m’y a mis ? Par l’ordre et la
conduite de qui ce lieu et ce temps ont-ils été destinés à moi ? » De cet étonnement,
on ne sort que par l'explication rationnelle, lorsqu’elle est possible, ou par
l'habitude. C’est pourquoi la philosophie n’en sort guère, et y ramène.

%ÊTRE
\section{Être}
«On ne peut pas entreprendre de définir l’être, observait Pascal, sans
tomber dans cette absurdité [d’expliquer un mot par ce mot même] :
car on ne peut définir un mot sans commencer par celui-ci, {\it c'est}, soit qu’on
l’exprime ou qu’on le sous-entende. Donc pour définir l’être, il faudrait dire
%— 222 —
%{\footnotesize XIX$^\text{e}$} siècle — {\it }
{\it c'est}, et ainsi employer le mot défini dans la définition » ({\it De l'esprit géométrique},
I). Ce que le {\it Vocabulaire} de Lalande, sans citer Pascal, confirmera : {\it être} est « un
terme simple, impossible à définir ». Non que nous ne sachions ce que signifie
le mot, mais en ceci plutôt que nous ne pouvons le définir sans présupposer ce
savoir, même vague, que nous en avons. Si « l’être se dit en plusieurs sens »,
comme le remarquait Aristote (chez lequel chacun de ces sens débouche sur
une {\it catégorie} : l'être se dit comme substance, comme quantité, comme qualité,
comme relation...), cela ne nous dit pas encore ce que c’est qu'être, ni ce que
ces différents sens peuvent avoir en commun.

Pour essayer d’y voir plus clair, on remarquera d’abord qu’{\it être}, en français,
est à la fois un verbe et un substantif, et que c’est le verbe qui est premier (le
substantif, une fois le verbe supposé défini, poserait moins de problème : l'{\it être},
mais on pourrait dire aussi bien {\it étant}, c’est {\it ce qui est}). S'agissant du verbe, on
distingue traditionnellement deux usages principaux : un usage absolu (« cette
table {\it est} »), et un usage relatif, logique ou copulatif (qui lie un sujet et un
prédicat : « cette table {\it est} rectangulaire »). « Le verbe {\it être} est employé en deux
sens, observe par exemple saint Thomas : d’une part il désigne l’acte d’exister,
d’autre part il marque la structure d’une proposition que l'esprit forme en joignant
un prédicat à un sujet. » Ces deux sens sont-ils vraiment différents ? Ne
pourrait-on pas, par exemple, donner à la proposition {\it « Cette table est»}, la
forme copulative {\it « Cette table est un être »} ? Sans doute, mais cela ne nous
apprendrait rien de plus sur la table. Au premier sens, remarque Kant, « {\it Être}
n'est évidemment pas un prédicat réel, c’est-à-dire un concept de quelque
chose qui puisse s’ajouter au concept d’une chose. C’est simplement la position
d’une chose ou de certaines déterminations en soi ». Au second sens, autrement
dit « dans l’usage logique, ce n’est que la copule d’un jugement » ({\it Critique de
la raison pure}, « L'idéal de la raison pure », 4). Métaphysiquement, c’est bien
sûr le premier sens surtout qui fait problème. Pourquoi l'être n’est-il pas un
prédicat réel ? Parce qu’il n’ajoute rien au sujet supposé. Par exemple, explique
Kant, que Dieu soit ou ne soit pas, le concept de Dieu n’en est pas pour cela
changé : c’est pourquoi on ne peut pas passer du concept à l’existence, ni donc
démontrer (comme le voudrait la preuve ontologique) l'existence de Dieu à
partir de sa simple définition.

On remarquera que dans cet usage absolu, et sauf distinction particulière à
tel ou tel philosophe, {\it être} signifie à peu près {\it exister} : c’est le contraire de n’être
pas, comme l'être est le contraire du néant. C’est où l’on retrouve Parménide.
« L’être est » : il y a de l’être, et non pas rien. Voilà ce que toute expérience et
toute pensée nous apprennent ou supposent. Être, c’est faire partie de cet {\it il y
a} : c'est être présent dans l’espace et le temps (ce que j'appelle {\it exister}), c’est persévérer
dans la présence (ce que j’appelle {\it insister}), ou simplement être présent
%— 223 —
%{\footnotesize XIX$^\text{e}$} siècle — {\it }
(ce que j'appelle {\it être}, proprement). Que cela ne vaille pas comme définition,
c'est une évidence — puisque toutes ces expressions supposent l'être —, mais qui
nous renvoie à nouveau à Pascal. On ne peut définir que ce qui est (les étants),
point l'être même, que tout discours suppose. Spinoza, qui dans les {\it Pensées
métaphysiques} risquait pourtant une définition (il entend par {\it être} « tout ce que,
quand nous en avons une perception claire et distincte, nous trouvons qui
existe nécessairement où au moins peut exister»), se gardera bien, dans
l’{\it Éthique}, de la reprendre ou d’en proposer une autre. Bel exemple, qu’on peut
suivre. L’être n’est pas d’abord un concept, qu’on pourrait définir ; il est une
expérience, une présence, un acte, que toute définition suppose et qu'aucune
ne saurait contenir. Par quoi l'être est silence, et condition du discours.

%ÊTRE-LÀ
\section{Être-là}
Voir {\it Dasein}.

%EUDÉMONISME
\section{Eudémonisme}
Toute éthique qui fait du bonheur ({\it eudaimonia}) le souverain
bien. C’est le cas, depuis Socrate, de la quasi-totalité
des écoles antiques, qui s’accordaient à penser que tout homme veut être
heureux et que tel est le but aussi de la philosophie. Cela n’empêchait pas les
philosophes de s'opposer résolument les uns aux autres — non sur ce but, qui
leur est commun, mais sur son contenu ou ses conditions. Qu'est-ce qui fait le
bonheur ? Le savoir (Socrate), la justice (Platon, dans la {\it République}), un mixte
de plaisir et de sagesse (Platon, dans le {\it Philèbe}), la raison ou la contemplation
(Aristote), l'indifférence (Pyrrhon), le plaisir (Épicure), la vertu (les stoïciens) ?
Ces différents eudémonismes s'opposent davantage qu’ils ne se complètent. Ils
cherchent la même chose — le bonheur —, mais ce n’est pas le même bonheur
qu'ils trouvent. L’eudémonisme est un lieu commun de la sagesse grecque.
Mais ce lieu est une arène, où les philosophes s’affrontaient. Les Modernes préfèrent
parler d’autre chose. Non qu’ils aient forcément renoncé au bonheur.
Mais parce qu’ils ont renoncé au souverain bien (voir ce mot).

%EUGÉNISME
\section{Eugénisme}
C’est vouloir améliorer l'espèce humaine, non par l’éducation
des individus mais par la sélection ou la manipulation
des gènes — en transformant le patrimoine héréditaire de l’humanité plutôt
qu'en développant son patrimoine culturel. L'idée, aujourd’hui disqualifiée par
l'usage qu’en firent les nazis, pouvait paraître belle. Agir sur les gènes ? On le
fait bien pour différentes espèces animales, ou pour tel ou tel être humain (les
thérapies géniques). Pourquoi ne pas améliorer l'humanité elle-même ? La
%— 224 —
%{\footnotesize XIX$^\text{e}$} siècle — {\it }
réponse, très difficile à argumenter dans le détail, me paraît tenir pour l’essentiel
en une phrase, qui n’a rien à voir avec la biologie : {\it Parce que tous les êtres
humains sont égaux en droits et en dignité}. Cela, qui vaut spécialement pour le
droit de vivre et de faire des enfants, rend toute idée d’un {\it tri}, au sein de
l'humanité, inacceptable : parce qu’elle est attentatoire à l’égale dignité de tous.
On a le droit de faire ou pas des enfants, mais pas celui de choisir les enfants
que l’on fait. On objectera qu’un tel choix existe pourtant, dans les avortements
thérapeutiques. Sans doute. Mais pour combattre une souffrance,
point pour fabriquer un surhomme. Pour épargner un individu, point pour
améliorer l'espèce. Par compassion, point par eugénisme. Cela indique à peu
près la voie, qui requiert d’autant plus de vigilance qu’elle est étroite et tortueuse.

%EUROPE
\section{Europe}
L'Europe n’est pas vraiment un continent : ce n’est qu’un cap de
l'Asie. Ce n’est pas un État: ce n’est qu’une communauté, et
encore, d’États indépendants. Combien de guerres entre eux dans le passé !
Combien, encore aujourd’hui, de conflits d’intérêts ou de sensibilités ! Ni la
géographie ni l’histoire ne suffisent à faire de l’Europe autre chose qu’une abstraction
ou qu’un idéal. Il faut donc qu’elle soit un idéal ou qu’elle ne soit rien,
en tout cas rien qui vaille, rien qui mérite d’être défendu. L'Europe n'existe
pas ; elle est à faire. Autant dire qu’elle n’existe que par les défis qu’elle affronte,
dont le premier sans doute est celui de sa propre existence. L'Europe ne vaut
qu’autant que nous le voulons, qu’autant que nous {\it la} voulons. Ce n’est ni un
continent ni un État: c’est un effort, c’est un combat, c’est une exigence.
L'Europe est devant nous, au moins autant que derrière. Mais elle ne vaut — et
elle ne vaudra — que par fidélité à ce qu’elle fut. Fidélité critique, cela va sans
dire, et d’ailleurs la critique (y compris réflexive) fait partie de son passé. Fidélité
à Socrate, à Montaigne, à Hume, à Kant — et à nous-mêmes. L'Europe est
notre origine et notre but, notre lieu et notre destin : l’Europe est notre tâche.
La vraie question reste celle de Rousseau : Qu'est-ce qui fait qu’un peuple
est un peuple ? Ou pour l’Europe en construction : qu'est-ce qui fait que plusieurs
peuples, tout en restant différents, peuvent tendre, et dans quelles
limites, à n’en faire qu’un ? Cela suppose des institutions, et qu’on choisisse
entre les deux modèles, fédéral ou confédéral, qui s’offrent à nous. Rassemblement
de Républiques (confédération), ou République rassemblée (fédération) ?
Souveraineté nationale, pour chaque pays, ou supranationale, pour l’ensemble ?
Aucune de ces deux voies n’est indigne, et aucune n’est facile. Mais refuser de
choisir entre l’une et l’autre serait une façon sûre de les fermer toutes deux.

%— 225 —
%{\footnotesize XIX$^\text{e}$} siècle — {\it }
Toutes les institutions resteront vaines, pourtant, si l’Europe ne sait
affronter le principal défi qui s'offre à elle, qui est celui de son esprit ou, cela
revient au même, de sa civilisation. L'Europe n’est pas une race ; c’est un espace
économique, politique et culturel. Mais il faut ajouter : culturel d’abord et surtout.
L'économie n’est qu’un moyen. La politique n’est qu’un moyen. Au service
de quoi ? De certaines valeurs, de certaines traditions, de certains idéaux —
au service d’une civilisation. Celle-ci est un fait de l’histoire. L'Europe, c’est
d’abord lempire romain : le mariage obligé d’Athènes et de Jérusalem, sur
l'autel de leur vainqueur et le civilisant peu à peu... C’est d’où nous sommes
issus, que nous ne pourrons continuer qu'à la condition de ne pas le trahir.
C’est ce que Rémy Brague appelle {\it la voie romaine} : être européen, c’est n’exister
que par cette tension en soi «entre un classicisme à assimiler et une barbarie
intérieure à dominer ». L'Europe, la vérité de l’Europe, c’est la Renaissance, ou
plutôt c’est « cette série ininterrompue de “Renaissances” qui constitue l’histoire
de la culture européenne », comme dit encore Rémy Brague ({\it Europe, la
voie romaine}, p. 165), ou, mieux encore, c’est cette hésitation toujours, cette
oscillation toujours, cette tension toujours entre la Renaissance et la décadence,
entre les Lumières et l’obscurantisme, entre la fidélité et la barbarie. Fidélité
critique, là encore : être européen, en ce sens, c’est être fidèle à la meilleure part
de l’Europe, telle qu’elle se donne dans les sommets indépassés de son histoire.
« {\it Notre patrie sacrée, l'Europe}... », disait Stefan Zweig. Mais à la condition seulement
de choisir ce qui mérite, dans cette patrie, d’être défendu.

On dira que la civilisation européenne est devenue mondiale, en tout cas
occidentale, et qu’elle ne se distingue plus guère, ou de moins en moins, de sa
filleule américaine. Sans doute, et c’est un danger encore qui la menace que
cette dissolution dans ce qu’elle croit son triomphe, qui pourrait être sa défaite
ultime. Le développement sans précédent des moyens de communication et des
échanges ne peut qu’entraîner, à l’échelle de la planète, une réduction des différences.
Sommes-nous pour autant condamnés à l’uniformité ? À l’expansion
irrésistible d’une sous-culture {\it « made in USA »}, avec son esthétique de {\it fast-food}
et de {\it sit-com} ? Le {\it show-biz} est-il l'avenir de l’homme ? L’américanisation, celui
de l'Europe ? Ce n’est pas sûr, mais ce n’est pas impossible. C’est ce qui donne
aux Européens des raisons de s'inquiéter, et de se battre. Contre quoi ? Contre
la barbarie qu’ils portent en eux, ou qu’ils importent, qui risque de les
emporter. Pour quoi ? Pour une Renaissance de l’Europe, et c’est l’Europe
même.

%EUTHANASIE
\section{Euthanasie}
Étymologiquement : une bonne mort. En pratique, le mot
ne sert guère que pour désigner une mort délibérément
%— 226 —
%{\footnotesize XIX$^\text{e}$} siècle — {\it }
acceptée ou provoquée, avec l’aide de la médecine, pour abréger les souffrances
d’un malade incurable : c’est une mort médicalement assistée. Le mot, qui fut
lui aussi compromis dans l’abjection nazie, s’en sort pourtant mieux que celui
d’eugénisme. C’est sans doute que l'euthanasie, à condition qu’elle soit strictement
contrôlée, pose moins de problèmes : d’abord parce qu’elle ne concerne
que des individus, point l'espèce elle-même ; ensuite, et surtout, parce qu’elle
ne vaut que pour des malades incurables, qui l’ont expressément demandée
(euthanasie volontaire) ou dont les proches, si les malades ne peuvent
s'exprimer, l’ont demandé à leur place (euthanasie non volontaire). J'y vois un
progrès plus qu’un danger. Quand la médecine ne peut nous guérir, pourquoi
ne nous aiderait-elle pas à mourir ? Le danger n’en existe pas moins, qui est
celui d’une élimination systématique des malades les plus lourds. Raison de
plus pour qu’ une loi, comme il convient dans un État de droit, vienne fixer des
limites et imposer des contrôles.

%ÉVANGILE
\section{Évangile}
Du grec {\it euangelos}, le bon messager, celui qui apporte une
bonne nouvelle. Avec une majuscule, et souvent au pluriel, c’est
le nom donné aux quatre livres qui retracent la vie et l’enseignement de Jésus-Christ.
Voltaire rappelle qu’ils ont été « fabriqués environ un siècle après Jésus-Christ »,
et qu’il en existe plusieurs autres, dits apocryphes, qui mériteraient
autant d'intérêt. Cela confirme la singularité de cette histoire. Quand bien
même ce ne serait qu’un roman, ce que je ne crois pas, et quoiqu'il soit parfois
ennuyeux, ce {\it roman}-là resterait, de tous les livres de l'humanité, lun des plus
éclairants. Pour ce qu’il nous dit sur Dieu ? Guère. Mais pour ce qu’il nous dit
sur nous-mêmes. Pour la résurrection de son personnage principal ? Non plus.
Mais pour sa vie.

Si on laisse de côté l’invraisemblable exploitation théologique qui en sera
faite, les Évangiles sont le récit d’une existence, et le portrait, même approximatif,
d’un individu. On aurait bien tort de les abandonner aux Églises. Jésus,
pour moi, n’est pas un prophète — je ne crois pas aux prophètes —, encore
moins le Messie ou Dieu. C'était un homme, et d’ailleurs il n’a jamais prétendu
être autre chose. C’est pourquoi il m'intéresse. C’est pourquoi il me touche.
Par la simplicité. Par la fragilité. Par l'humanité nue. Qui peut imaginer, lisant
les Évangiles, que cet homme ait pu se prendre pour Dieu ? Pour son fils ?
Nous le sommes tous, puisque ce Dieu-là, selon la prière même que Jésus nous
laissa, serait Notre Père. Bref, Jésus, tel que je le vois, tel que je crois le comprendre
en lisant les Évangiles, n’a jamais été chrétien. Pourquoi le serions-nous ?
C'était un Juif pieux. C’était un homme plein de sagesse et d’amour. La
seule façon de lui être vraiment fidèle, pour ceux qui ne sont ni juifs ni
%— 227 —
%{\footnotesize XIX$^\text{e}$} siècle — {\it }
croyants, c’est d’être un peu plus sages, un peu plus aimants, un peu plus
humains, et pour cela d’abord de respecter la justice et la charité, qui sont toute
la loi. C’est ce que Spinoza appelait « l'esprit du Christ », qui est l'esprit tout
court et le principal message des Évangiles.

%ÉVÉNEMENT
\section{Événement}
Ce qui advient, plutôt que ce qui est ou dure : « non ce qui
subsiste, mais ce qui survient» (Francis Wolff, {\it Dire Le
monde}, 1). L'événement s'oppose en cela à la substance, à l'être, à la chose — à
tout ce qui demeure. Au monde ? Seulement si l’on suppose un monde qui
serait fait de choses, d’essences ou de substances. Mais il se pourrait qu’il soit
fait plutôt d’événements : que le monde soit l’ensemble de tout ce qui arrive,
comme disait Wittgenstein, plutôt que de tout ce qui est (la totalité des événements,
non des choses), ou du moins que cette distinction n’ait de sens que
pour nous, qui pensons et vivons dans le temps, non pour le réel, qui n’existe
qu’au présent. Tout événement occupe une certaine durée, fût-elle infiniment
brève : rien ne se passe que dans un présent qui passe. Toute durée est faite
d'événements, qu’ils soient très lents ou très rapides : l'expansion de l’univers,
la dérive des continents, un enfant qui grandit ou qui tombe, un oiseau qui
s'envole. Advenir et durer au présent sont un : ainsi l'être et l'événement.
(On notera que la notion d’événement, en philosophie et contrairement à
l'usage historique ou journalistique du mot, est le plus souvent dépourvue de
toute visée normative. Là où le langage courant ne parle d'événement que pour
un fait d’une certaine importance — un train qui arrive à l'heure n’est pas un
événement, un train qui déraille en est un —, les philosophes prennent ordinairement
le mot dans son sens neutre et son extension maximale : tout ce qui
arrive ou a lieu est un événement, quand bien même cela n’aurait d’importance
pour personne. Le seul {\it non-événement}, pour le philosophe, c’est celui qui
n’advient pas.)

%ÉVIDENCE
\section{Évidence}
Ce qui s’impose à la pensée, ce qui ne peut être contesté ou nié,
ce dont la vérité paraît immédiatement et ne peut être mise en
doute. Il n’y aurait pas autrement de certitude, et c’est pourquoi il n’y en a
jamais d’absolue. Par exemple le {\it cogito}, le postulat d’Euclide ou l’immobilité de
la Terre ont longtemps été tenus pour des évidences, ce qu’ils ne sont plus.
C’est dire que l’évidence dépend de l’état des connaissances. Comment pourrait-elle
les fonder ou les garantir ?
Si l’on se fie à l’étymologie ({\it evidens}, en latin, vient de {\it videre}), le modèle de
l'évidence est visuel : {\it « Je l'ai vu, te dis-je, vu de mes yeux vu, ce qui s'appelle vu »},
%— 228 —
%{\footnotesize XIX$^\text{e}$} siècle — {\it }
tel est, dans les mots de Molière, le type même de l’évidence. Pour la vie courante,
c’est un critère très fiable, et d’autant plus qu’il est attesté par un plus
grand nombre d'individus : si plusieurs témoins vous ont vu assassiner
quelqu'un, vous aurez quelque peine à faire croire que vous n'êtes pour rien
dans sa mort. Sous réserve toutefois de la vraisemblance, de la confrontation
et de la critique des témoignages. Les milliers d'individus qui ont vu fort distinctement
la sainte Vierge, que ce soit ensemble ou séparément, ne convainquent
que ceux, sauf exception, qui y croyaient déjà. Quoi de plus évident, tant
qu’on est dedans, qu’un rêve ou qu’un délire ?

« Les dieux existent, disait Épicure : la connaissance que nous en avons est
évidente » ({\it Lettre à Ménécée}, 123). Je ne connais pas de phrase qui m’ait davantage
poussé vers l’athéisme.

%ÉVOLUTION
\section{Évolution}
La transformation, souvent lente et en tout cas progressive,
d’un être ou d’un système : s’oppose à {\it permanence} (l'absence
de changement) et à {\it révolution} (un changement brusque et global).

Le vocable doit beaucoup de son succès, à partir du {\footnotesize XIX$^\text{e}$} siècle, aux différentes
{\it théories de l'évolution} (spécialement celle de Darwin, même si ce dernier
n’utilisa le mot qu'avec réticence) visant à expliquer l’origine et le développement
des espèces vivantes. Cet exemple privilégié montre qu’une évolution
peut se faire de façon discontinue et hasardeuse (les mutations) ; elle suppose
toutefois la continuité, au moins relative et fût-elle reconstruite après coup,
d’un processus. « Nul n’appellera stades évolutifs, remarque le {\it Lalande}, les
transformations qu’on observe dans un kaléidoscope. » Non, pourtant, que
chacun de ces mouvements soit irrationnel ou sans cause ; mais parce que leur
série paraît sans logique, sans continuité, sans orientation. C’est dire que les
mutations, à elles seules, ne suffraient pas à parler d'évolution des espèces : il y
faut encore la sélection naturelle et l’apparente finalité qu’elle entraîne. Par
quoi l’{\it évolution}, qui avance vers des stades de plus en plus complexes ou différenciés,
s'oppose à l’{\it involution}, qui régresse vers le plus simple, le plus homogène,
ou le plus pauvre. La croissance, pour l'individu, est une évolution ; le
vieillissement, une involution.

%EXACTITUDE
\section{Exactitude}
Une vérité modeste, qui tiendrait tout entière dans la précision
des mesures, des descriptions, des constats, sans prétendre
pour cela atteindre l’être ou l'absolu : c’est une adéquation de surface, la
seule peut-être qui permette d’avancer en profondeur.

%— 229 —
%{\footnotesize XIX$^\text{e}$} siècle — {\it }
L’exactitude dépend bien sûr de l’échelle considérée. Une erreur d’un
micron, en biologie ou en physique des particules, peut être plus inexacte
qu’une erreur de plusieurs kilomètres, en astronomie. Il n’est d’exactitude que
relative : c’est une erreur minimale.

%EXCEPTION
\section{Exception}
Un cas singulier, qui semble violer une loi et par là la suppose.
On dit que l’exception confirme la règle ; le vrai est
qu’elle l’enfreint sans l’abolir. Par exemple quand le Comité national consultatif
d'éthique suggère une {\it « exception d'euthanasie »} : c’est reconnaître que la
règle, pour les médecins comme pour tous, reste le respect de la vie humaine ;
mais que ce respect peut justifier parfois qu’on l’interrompe, quand elle ne
pourrait continuer que dans l'horreur. Respecter la vie humaine, c’est aussi lui
permettre de rester humaine jusqu’au bout.

%EXEMPLE
\section{Exemple}
Un cas particulier, qui sert à illustrer une loi ou une vérité générale.
Un exemple ne prouve jamais rien (alors qu’un contre-exemple
peut être une réfutation suffisante), mais il aide à comprendre, et à
faire comprendre. C’est comme une expérience de pensée, à visée surtout pédagogique
ou persuasive. En philosophie on ne peut guère s’en passer, ni s’en
contenter.

%EXERCICE
\section{Exercice}
Une action, le plus souvent répétitive, qui ne se justifie que par
d’autres, qu’elle prépare ou facilite. C’est s’habituer au difficile,
pour qu’il le soit moins.

Les Anciens parlaient d’exercices ({\it askèsis}) de sagesse : parce qu’il est difficile
d’être simple ou libre, et qu’on n’y parvient qu’à la condition de s’y entraîner.
Diogène, en hiver, étreignant une statue gelée : c’est s’exercer à vouloir, pour
apprendre à agir.

Il reste qu'aucun exercice ne vaut par soi. C’est ce que les ascètes
oublient, parfois, et que les sages leur rappellent. Vas-tu passer ta vie à faire
des gammes ?

%EXHIBITIONNISME
\section{Exhibitionnisme}
C'est jouir du spectacle que l’on donne ou que l’on
est, et d'autant plus qu’il est plus intime ou plus obscène.
L’exhibitionnisme attente à la pudeur, ou (entre amants) s’en libère.

%— 230 —
%{\footnotesize XIX$^\text{e}$} siècle — {\it }
%EXIGENCE
\section{Exigence}
Un désir confiant et résolu, qui ne se résigne pas au médiocre
ou au pire. C’est le contraire de la veulerie (s'agissant de soi) ou
de la complaisance (s’agissant d’autrui).

%EXISTENCE
\section{Existence}
Souvent synonyme d’être. L’étymologie suggère pourtant une
différence. Exister, c’est naître ou se trouver {\it (sistere) dehors
(ex)}, autrement dit — il n’y a pas de {\it dehors} absolu — dans autre chose : c’est être
dans le monde, dans l’univers, dans l’espace et le temps. Par exemple, on hésitera
à dire que les êtres mathématiques {\it existent}. Et quand bien même Dieu
{\it serait}, il ne pourrait pas pour autant, remarquait Lagneau, être dit {\it exister} en ce
sens. « Exister, c’est dépendre, écrira Alain, c’est être battu du flot extérieur. »
Si Dieu existait, il ne serait pas Dieu, puisqu'il serait une partie de l’univers,
puisqu'il serait dans un dehors, puisqu'il en dépendrait, et c’est pourquoi il
n'existe pas. « L'existence toujours suppose l'existence, hors d’elle toujours et
autre ; et voilà l'existence. » Son essence est de n’en pas avoir : « La nature,
même intérieure, de toute chose est hors delle. La relation est la loi de
l'existence » ({\it Entretiens au bord de la mer}, VI). Alain, précurseur de l’existentialisme ?
C’est ce que Jean Hyppolite s’amusait à suggérer, qui n’est pas
tout à fait faux ({\it Figures de la pensée philosophique}, XX et X). Mais à ceci près que
l'existence, pour Alain, ne saurait se dire de l’homme seul. L'existence est la loi
du monde, ou le monde même comme loi. L'homme ne s’en distingue que par
la conscience qu’il en prend, qui le sépare de ce {\it dehors} qui le fait exister, et de
lui-même. C’est en quoi il {\it ex-siste}, au sens cette fois heideggérien ou existentialiste
du terme, toujours hors de soi, toujours en avant de soi et de tout, toujours
jeté (dans le monde) et se projetant (dans l'avenir), toujours autre qu’il
n’est, toujours libre, toujours voué au souci ou à l'angoisse, toujours tourné
vers la mort ou le néant. On voit que ces deux sens, pour différents qu’ils
demeurent, peuvent être pris ensemble. Exister, c’est être dehors: c'est
dépendre ou se séparer. S’oppose par là à un Être absolu, qui ne serait qu’indépendance
et intériorité. Exister, c’est n’être pas Dieu : c’est être au monde, toujours
pris dans un dehors (toujours dedans, donc, mais {\it un dedans qui n'est pas
soi}), toujours dépendant, toujours luttant ou résistant. « Ce qui n'existe pas
c’est l’inhérence, écrit Alain, c’est l'indépendance, c’est le changement [seulement]
interne, c’est le dieu ; et ce n’est rien. »

%EXISTENTIALISME
\section{Existencialisme}
Toute philosophie qui part de l’existence individuelle
plutôt que de l’être ou du concept (c’est en ce sens que
Pascal et Kierkegaard sont souvent considérés comme les précurseurs de l’existentialisme),
%— 231 —
%{\footnotesize XIX$^\text{e}$} siècle — {\it }
et spécialement, selon une formule fameuse de Jean-Paul Sartre,
toute doctrine pour laquelle {\it « l'existence précède l'essence »}. Qu'est-ce à dire ?
Que l’homme n’a pas d’abord une essence, qui lui préexisterait et dont il resterait
prisonnier, mais qu'il existe « avant de pouvoir être défini par aucun
concept » et ne {\it sera} (quand on pourra parler de son essence au passé) que ce
qu’il aura {\it choisi} d’être. C’est dire qu’il est libre absolument : « Qu'est-ce que
signifie ici que l’existence précède l’essence ? Cela signifie que l’homme existe
d’abord, se rencontre, surgit dans le monde, et qu’il se définit après. L'homme,
tel que le conçoit l’existentialisme, s’il n’est pas définissable, c’est qu’il n’est
d’abord rien. Il ne sera qu’ensuite, et il sera tel qu’il se sera fait. Ainsi il n’y a
pas de nature humaine, puisqu'il n’y a pas de Dieu pour la concevoir. [...]
L'homme n’est rien d’autre que ce qu’il se fait » {\it (L'existentialisme est un humanisme)}.
Par quoi l’existentialisme est une philosophie de la liberté, au sens
métaphysique du terme, et l’une des plus radicales qui fut jamais.

Reste à savoir si on peut la faire sienne. Comment {\it exister}, comment faire ou
choisir quoi que ce soit, comment s’inventer ou se projeter, avant d’{\it être} d’abord
quelque chose ou quelqu'un ? Qui dirait d’un nouveau-né qu’il n’est {\it rien} ? Et
comment considérer qu'être ce qu’on est soit seulement une {\it situation}, qu’il
nous appartiendrait de transcender, et point, au moins en partie, une {\it détermination},
dont il est exclu que nous sortions jamais (puisque changer, c’est toujours
{\it se} changer) ? « Chaque personne est un choix absolu de soi », écrit Sartre
dans {\it L'Être et le Néant} ; c'est ce que je n’ai jamais pu croire ni penser. Comment
choisir sans être d’abord ? Ou plutôt quel sens y a-t-il, au présent, à distinguer
ce que je {\it fais} ou {\it veux} de ce que je {\it suis} ? Exister, c’est être en acte et en
situation : l'essence et l’existence, au présent, sont une seule et même chose.
Certes ce n’est pas ainsi que nous le vivons ou l’imaginons ; nous avons le sentiment
d’être ce que le passé a fait de nous, et de choisir ce que nous ferons de
l'avenir. « L’essence, c’est {\it ce qui a été} », écrit Sartre dans {\it L'Être et le Néant};
l'existence, à l’inverse, c’est ce qui n’est pas encore, ce qui se jette vers l’avenir,
ce qui {\it sera}, si je le veux ou le fais. « La liberté s'échappe vers le futur, elle se
définit par la fin qu’elle pro-jette, c’est-à-dire par le futur qu’elle a à être »
({\it L'Étre et le Néant}, p. 577). Mais cette distinction, entre l’essence et l’existence,
n’a dès lors de sens que pour la conscience, qui se donne un passé et un avenir,
point pour le réel lui-même, qui n’existe qu’au présent — et dont la conscience,
qu’elle le veuille ou pas, fait partie. Un souvenir n’existe qu’au présent. Un
projet n'existe qu’au présent. Et comment le présent pourrait-il n’être pas ce
qu’il est ou être autre ? Sartre, en toute cohérence, explique que la liberté n’est
possible que comme néant, point comme être, ce qui m’a toujours paru une
réfutation suffisante : la liberté, en ce sens absolu, n’est possible qu’à la condition
de n’être pas. L’existentialisme n’est qu’un humanisme imaginaire.

%— 232 —
%{\footnotesize XIX$^\text{e}$} siècle — {\it }
Faut-il alors retomber dans un essentialisme qui nous enfermerait à jamais
dans ce que nous sommes, pour lequel l’existence ne serait qu’un {\it effet} de
l'essence ? Nullement. Au présent, l’essence et l'existence ne font qu’un, et ne
sauraient se {\it précéder} mutuellement. Ni existentialisme, donc, ni essentialisme :
l'existence ne précède pas plus l’essence que l’essence ne précède l’existence.
Elles n'existent qu’ensemble, dans un même monde, dans un même présent, et
c’est ce que signifie exister.

%EXOTÉRIQUE
\section{Exotérique}
Qui s'adresse à tous, y compris à ceux qui sont en dehors
{\it (exô)} de l’école ou du groupe. S’oppose à l’enseignement
ésotérique ou acroamatique, qui ne s'adresse qu’aux initiés ou aux spécialistes,
On remarquera que l’école publique, dès lors qu’elle est laïque, gratuite et obligatoire,
tend à relativiser cette opposition : c’est qu’elle forme des élèves, non
des disciples ; des citoyens, non des initiés.

EXPÉRIENCE
\section{Expérience}
Notre voie d’accès au réel: tout ce qui vient en nous du
dehors (expérience externe), et même du dedans (expérience
interne), en tant que cela nous apprend quelque chose. S’oppose à la raison,
mais aussi la suppose et l’inclut. Pour un être tout à fait dépourvu d’intelligence,
aucun fait ne ferait expérience, puisqu'il ne lui apprendrait rien. Et un
raisonnement, pour nous, n’est qu’un fait comme un autre. Ainsi on ne sort
pas de l'expérience ; c’est ce qui donne raison à l’empirisme et qui lui interdit
d’être dogmatique.

%EXPÉRIMENTATION
\section{Expérimentation}
Une expérience active et délibérée : c’est interroger
le réel, au lieu de se contenter de l'entendre (expérience)
et même de l’écouter (observation). Se dit spécialement de l’expérimentation
scientifique, qui vise ordinairement à tester une hypothèse en la soumettant
à des conditions inédites, artificiellement obtenues (le plus souvent en
laboratoire) et reproductibles. Cela suppose qu’on cherche quelque chose, et
même, presque toujours, qu’on sache ce qu’on cherche : il n’y a pas d’expérimentation
sans une hypothèse préalable et une théorie — fût-elle fausse ou provisoire —
de référence. « Pour un esprit scientifique, écrit Bachelard, toute
connaissance est une réponse à une question. S’il n’y a pas eu de question, il ne
peut y avoir connaissance scientifique. Rien ne va de soi. Rien n’est donné.
Tout est construit. » ({\it La formation de l'esprit scientifique}, X). L’expérimentation
est une expérience qui ne va pas de soi — une expérience {\it construite}.

%— 233 —
%{\footnotesize XIX$^\text{e}$} siècle — {\it }
Aucune expérimentation ne suffit jamais à prouver la vérité d’une hypothèse,
encore moins d’une théorie. On cherche une preuve ; on ne trouve
qu'un exemple, ou un contre-exemple : ce dernier seul est probant. Vous
pouvez vérifier dix mille fois que la nature a horreur du vide ou que les corps
les plus lourds tombent plus vite que les autres, vous n’aurez pas prouvé par là
que c’est vrai ; une seule expérimentation, si elle est bien conduite et reproductible,
peut suffire à montrer que c’est faux. Ainsi l’expérimentation n’est décisive
ou cruciale que par les théories qu’elle permet d’éliminer. Les sciences
expérimentales avancent par conjectures et réfutations, montre Popper, non
par induction et vérification : « c’est la falsifiabilité d’un système, et non sa vérifiabilité,
qu’il faut prendre comme critère de démarcation » d’une démarche
expérimentale ({\it La logique de la découverte scientifique}, I). C’est ce qui permet
aux sciences d’avancer, sans jamais les autoriser à s’arrêter.

%EXPLICATION
\section{Explication}
Le fait d'expliquer, c’est-à-dire de donner la cause, le sens
ou la raison. Le principe de raison et le principe de causalité
entraînent que tout fait, quel qu’il soit, a une explication : l’inexplicable
n'existe pas. On remarquera que cette explication n’a en elle-même aucune
visée normative, et ne saurait donc valoir comme approbation ou comme justification.
Qu’une maladie puisse s'expliquer, cela ne la rend ni moins pathologique
ni moins grave, Que le nazisme puisse s’expliquer, cela ne le rend ni
moins ignoble ni moins lourd de conséquences. J'ai lu plusieurs fois que la
Shoah était par nature inexplicable, qu’il fallait la laisser telle, qu’on ne pourrait
entreprendre de l'expliquer, d’ailleurs en pure perte, qu’à la condition de nier
d’abord son irréductible et atroce singularité. C’est donner raison à l’irrationalisme
nazi, et à la nuit contre les Lumières. Pourquoi le racisme serait-il
inexplicable ? Et quoi de plus explicable, quand le racisme atteint ce degré de
fanatisme et de haine, qu’il devienne assassin ? Racisme de masse : crime de
masse. Mieux vaut essayer de le comprendre, pour le combattre. Mais si l’on
comprend, dira-t-on, on ne peut plus juger ! C’est se méprendre. Ce n’est pas
la cancérologie qui nous dit que le cancer est mauvais ; mais elle nous aide à le
combattre. L’explication ne tient jamais lieu de jugement de valeur, ni le jugement,
d'explication.

%EXTASE
\section{Extase}
C’est sortir de soi et de tout, pour se fondre en autre chose (spécialement
en Dieu) — comme un saut dans la transcendance ou
dans l’absolu. S’oppose par là à l’{\it enstase} (voir ce mot).

%— 234 —
%{\footnotesize XIX$^\text{e}$} siècle — {\it }
%EXTENSION
\section{Extension}
L'ensemble des objets désignés par un même signe ou compris
dans un même concept. Définir ce concept {\it en extension},
ce sera dresser la liste, quand c’est possible, de tous les objets auquel il
s'applique. S’oppose à {\it compréhension} (voir ce mot). L'extension du concept
« homme » est l’ensemble de tous les hommes. Les femmes en font-elles partie ?
Cela dépend de la {\it compréhension} du concept.

%EXTRÉMISME
\section{Extrémisme}
Propension à aller jusqu’au bout, dans une direction donnée,
en oubliant ce que les autres directions peuvent avoir
aussi de légitime ou de sensé. Si la droite n’est qu’une erreur, l'extrême gauche
a raison, contre la gauche, comme l'extrême droite contre la droite si la gauche
est le mal. Et à quoi bon autrement être de droite ou de gauche ? Ainsi l’extrémisme
est la tentation des plus convaincus ou des plus haineux : double
danger, double force.

« On ne pense bien qu'aux extrêmes », disait Louis Althusser, et cela sans
doute n’est pas tout à fait faux. Un marxiste ou un ultra-libéral, parlant d’économie,
seront presque toujours plus intéressants, intellectuellement, qu’un centriste
ou un social-démocrate. Mais le réel résiste, qui n’est pas une pensée. « Le
peuple se trompe, observait Montaigne : on va bien plus facilement par les
bouts, où l’extrémité sert de borne d’arrêt et de guide, que par la voie du
milieu, large et ouverte, et selon l’art que selon la nature, mais bien moins
noblement aussi, et moins recommandablement » ({\it Essais}, III, 13). Quel penseur
plus radical pourtant que celui-ci ? Et quel vivant plus modéré ? On ne
pense bien qu'aux extrêmes. On ne vit bien que dans l’entre-deux.

« La sagesse est l’extrême de vivre », ai-je pourtant écrit quelque part. Mais
c'est qu’elle n’est qu’une idée de philosophe. « La sagesse a ses excès, écrit
encore Montaigne, et n’a pas moins besoin de modération que la folie » (III, 5).
Celui-ci était sage véritablement, qui ne crut jamais à la sagesse.
%{\footnotesize XIX$^\text{e}$} siècle — {\it }


%
%{\footnotesize XIX$^\text{e}$} siècle — {\it }

\chapter{FGH}
\section{Fable}
%FABLE
Une histoire inventée, qu’on ne cherche pas à faire passer pour vraie,
ou dont on ne peut envisager qu’elle le soit : c’est un mythe qui
donne à penser ou à rire, plutôt qu’à croire.

\section{Factice}
%FACTICE
Ce qui est faux ou fabriqué.

\section{Facticité}
%FACTICITÉ
Dans la langue philosophique contemporaine, ici très influencée
par l’allemand, le substantif a rarement le sens que l’adjectif
semble annoncer : la {\it facticité} caractérise non ce qui est faux ou fabriqué, mais ce
qui est un fait, à la fois nécessaire (puisqu'il est là) et contingent (il aurait pu ne
pas y être), comme ils sont tous. Mieux vaudrait, en ce sens, parler de factualité.

\section{Faculté}
%FACULTÉ
Une puissance innée ou {\it a priori} : par exemple la puissance de
sentir (la sensibilité), de penser (l'intelligence, l’entendement), de
désirer, de vouloir, d'imaginer, de se souvenir. La difficulté est de rattacher la pluralité
de ces facultés, qui semble un fait d'expérience, à l’unité de l'esprit ou du cerveau,
sans laquelle il n’y aurait pas d’expérience du tout. La neurobiologie, dans ce
domaine, a sans doute davantage à nous apprendre que la philosophie. La « doctrine
des facultés », comme on disait jadis, a laissé place aux sciences cognitives.

\section{Fait}
%FAIT
Un événement quelconque, dès lors qu’il est constaté ou établi — ce
qui ne peut se faire que par expérience. On parle de « fait scientifique »
%— 236 —
quand il a été l’objet d’une expérimentation, ou à tout le moins d’une observation
rigoureuse, ce qui suppose presque toujours une théorie préalable et une
technologie adaptée : c’est un fait « bien fait », comme dit Bachelard, plutôt
que tout fait.

En philosophie, on oppose traditionnellement la question de fait {\it (quid
facti)} à la question de droit {\it (quid juris)}, comme ce qui est à ce qui doit ou
devrait être. Par exemple, qu’il y ait des riches et des pauvres : c’est là un fait
incontestable ; mais qui ne dit rien sur sa légitimité. « L'égalité des biens serait
juste, écrit Pascal, mais. » Mais quoi ? Mais cela n’est pas en fait, et même le
droit en a décidé autrement. Car le droit lui-même, pris au sens juridique, n’est
qu’un fait comme un autre.

Cela vaut aussi d’un point de vue gnoséologique ou pratique. Que nous
ayons, en fait, des sciences et une morale, cela ne nous dit pas ce qu’elles valent,
ni à quelles conditions. En faire la critique ? C’est toujours légitime, mais cela
ne fera qu’un fait de plus, qui viendra s'ajouter aux autres sans pouvoir les
fonder.

Ainsi il n’y a que des faits, et c’est ce qu’on appelle le monde.

\section{Falsifiabilité}
%FALSIFIABILITÉ
Néologisme proposé par Karl Popper, qui y voit la ligne
de démarcation entre les sciences empiriques, d’un côté,
et de l’autre les énoncés métaphysiques, pseudo-scientifiques, ou encore relevant
de la seule logique formelle. Un énoncé n’est {\it falsifiable} que s’il peut être
contredit, au moins en principe, par l’expérience, autrement dit que si l’on peut
concevoir au moins un fait susceptible, le cas échéant, de le réfuter. Par
exemple les énoncés « Il pleuvra ici demain » ou « Tous les cygnes sont blancs »
sont falsifiables : on peut imaginer un fait qui les réfute (qu’il ne pleuve pas ici
demain, ou qu’on voie un cygne qui ne soit pas blanc). Ce sont des énoncés
empiriques. En revanche, les énoncés «IL pleuvra ou il ne pleuvra pas ici
demain », « Dieu existe » ou « le communisme est l’avenir de l’humanité » ne
sont pas falsifiables : on ne peut concevoir aucun fait qui suffirait à prouver
qu'ils sont faux. Ce ne sont donc pas des énoncés empiriques. On remarquera
que ces trois énoncés ont pourtant un sens (puisqu’on peut les comprendre, les
approuver ou les critiquer), et que le premier est assurément vrai (c'est une
tautologie). La falsifiabilité n’est ni un critère de signification ni un critère de
vérité, mais seulement un critère d’empiricité et donc, s'agissant des sciences
expérimentales, de scientificité possible. « Un système n’est empirique ou scientifique
que s’il est susceptible d’être soumis à des tests expérimentaux », reconnaît
classiquement Popper ; mais aucun test, ajoute-t-il, ne suffit jamais à
prouver la vérité d’une théorie : quand bien même j'aurais vu cent mille cygnes
%— 257 —
blancs ou comparé cent mille fois la vitesse de corps en chute libre, cela ne suffira
pas à prouver que {\it tous} les cygnes sont blancs ni que {\it tous} les corps, dans le
vide, tombent à la même vitesse. « Les théories ne sont donc {\it jamais} vérifiables
empiriquement », conclut Popper ; elles ne peuvent être testées que négativement :
« c’est la falsifiabilité et non la vérifiabilité d’un système qu’il faut
prendre comme critère de démarcation » ({\it La logique de la découverte scientifique},
I, 6 ; voir aussi le chap. IV).

J'ai longtemps regretté que les traducteurs français se soient résignés au
néologisme {\it falsifiabilité} (alors que « falsifier », en français, a un tout autre sens,
et que le mot {\it réfutabilité} était disponible). Mais, outre que les deux mots existent
chez Popper, et qu’il était légitime de les traduire en français par deux
mots différents, on remarquera avec Alain Boyer qu’ils ne sont pas tout à fait
synonymes : une théorie peut être {\it réfutée} par un argument simplement logique
(par exemple en mathématiques) ; elle ne peut être {\it falsifiée} que par un fait
empirique. Ainsi {\it réfutabilité} est le genre prochain ; la {\it falsifiabilité} est une espèce
particulière de réfutabilité, dont l’empiricité est la différence spécifique. Ce qui
autorise une définition simplifiée : est falsifiable tout énoncé possiblement
réfutable par l’expérience.

\section{Famille}
%FAMILLE
Ensemble d'individus, liés par le sang, le mariage ou l'amour.
Où finit la famille ? Cela dépend des époques, des régions, des
contextes. De nos jours, et dans nos pays, on peut distinguer la famille au sens
restreint (le père, la mère, leurs enfants), et la famille au sens large (il faut alors
ajouter les grands-parents, les oncles et tantes, cousins et neveux, sans parler de
la belle-famille ni des familles recomposées.....). Où commence-t-elle ? Cela
peut se discuter, notamment d’un point de vue juridique. Pour ma part, et
d’un point de vue philosophique, je répondrai simplement : la famille commence
à l'enfant. Un ménage sans enfant, ce n’est pas une famille : c’est un
couple. Alors qu’une mère célibataire, qui élève seule son ou ses enfants, c’est
évidemment une famille. On m'objectera qu’un orphelinat, où il y a tant
d’enfants, n’est pas une famille pour autant. Certes ; mais c’est que les enfants
y sont considérés simplement comme enfants, non comme fils ou filles. Cela
fait toute la différence, qui distingue aussi la famille de l’école. La famille, c’est
la filiation acceptée, assumée, {\it cultivée}. Car la famille est un fait de culture,
autant ou davantage qu’un fait biologique. Deux adultes qui adoptent un
enfant, c’est une famille ; un couple qui abandonne le sien, ce n’en est pas une.
La famille est la filiation selon l'esprit, ou le devenir-esprit de la filiation.
La famille, sous des formes bien sûr différentes, semble avoir existé à toutes
les époques, et en tous lieux : « le fait de famille, reconnaît Claude Lévi-Strauss,
%— 238 —
est universel » ({\it Le regard éloigné}, p. 80). Ce n’est pas sans poser un problème.
Si l’on admet avec Lévi-Strauss que l’universel est le critère de la nature, et la
règle particulière le critère de la culture, comment expliquer l'existence {\it universelle}
d’une institution évidemment {\it réglée}, et donc culturelle ? On reconnaît ici
la problématique qui est celle de Lévi-Strauss quand il s'interroge sur la prohibition
de l'inceste, et ce n’est pas un hasard : si la famille, comme la prohibition
de l'inceste, présente les deux caractères en principe opposés de deux ordres
exclusifs (luniversalité de la nature, la particularité réglée de la culture), c’est
que la famille réalise concrètement — non une fois pour toutes mais à chaque
génération, et pour chaque individu de chaque génération — ce que la prohibition
de l’inceste ne fait qu’instituer formellement : le {\it passage} de la nature à la
culture, de humanité biologique à l’humanité culturelle — de la filiation selon
la chair à la filiation selon l'esprit, de l'humanité comme espèce à l’humanité
comme valeur.

On sait que la prohibition de l'inceste, pour les ethnologues, vaut moins
par ce qu’elle interdit que par ce qu’elle impose : l'échange sexuel avec un
membre d’une autre famille, d’où résulte l'alliance entre les familles et donc la
société. Ce que je ne peux trouver chez les miens — la jouissance sexuelle du
corps de l’autre —, il faut que je le cherche à l'extérieur, dans une autre famille,
et c’est ce qui permet, ou impose, d’en fonder une troisième... « Dans tous les
cas, remarque encore Lévi-Strauss, la parole de l’Écriture : {\it “Tu quitteras ton père
et ta mère”}, fournit sa règle d’or (ou, si l’on préfère, sa loi d’airain) à l’état de
société. » La famille n’est pas seulement l’élément premier de la société, comme
le voulait Auguste Comte (plusieurs familles dispersées ne font pas encore une
société), mais bien sa condition : elle représente la nature dans la culture, par la
filiation, et la culture dans la nature, par la prohibition de l'inceste. Elle est le
creuset où animalité et humanité ne cessent de se fondre : elle réalise le passage
de la nature à la culture, en imposant le passage de la famille à la société.

La famille, qui donne tout à l’enfant, finit ainsi par donner son enfant
même. À qui ? À un autre homme, à une autre femme, certes, mais aussi — et
d’abord, et surtout — à lui-même. C’est ce dernier don, le plus beau, le plus difficile
qu’on appelle {\it liberté}. La famille donne et perd : elle donne {\it pour perdre},
même, pour que l'enfant s’en aille, pour qu’il puisse quitter sa famille, et c’est
ce qu’on appelle élever un enfant.

\section{Fanatisme}
%FANATISME
«Le fanatisme, disait Alain, ce redoutable amour de la
vérité. » Mais il n’aime que la sienne. C’est un dogmatisme
haineux ou violent, trop sûr de sa bonne foi pour tolérer celle des autres. Le terrorisme
est au bout.

%— 239 —
On remarquera qu’il n’y a pas de fanatisme dans les domaines où une
preuve est possible (pas de fanatisme en mathématiques, en physique, presque
pas en histoire, quand les faits sont un peu anciens et bien établis), et c’est ce
qui énerve les fanatiques : parce qu’ils ne peuvent ni faire partager leur certitude,
ni accepter qu’elle n’en soit pas une. Le fanatisme touche à la foi, mais
l’exacerbe. À l'enthousiasme, mais le pervertit. Il est à la superstition, disait
Voltaire, « ce que le transport est à la fièvre, ce que la rage est à la colère. Celui
qui a des extases, des visions, qui prend des songes pour des réalités, et ses imaginations
pour des prophéties, est un enthousiaste ; celui qui soutient sa folie
par le meurtre est un fanatique. » C’est se laisser emporter par sa faiblesse, au
point de la prendre pour une force.

\section{Fantaisie}
%FANTAISIE
L’imagination, mais joueuse plutôt que visionnaire, plaisante
plutôt que fascinante, enfin sans trop d'illusions sur elle-même.
C’est l'imagination la plus libre et la plus sympathique : celle qui n’est pas dupe
de ses rêves, ni de soi.

\section{Fantasme}
%FANTASME
Une image ou un scénario suscités par le désir, mais dont on
voit clairement qu’ils sont imaginaires (c’est ce qui distingue le
fantasme de l'illusion), voire, parfois, qu’ils doivent le rester.

\section{Fatalisme}
%FATALISME
Croyance en la fatalité de tout. Cela revient à décourager
l’action : tout fatalisme est paresseux ou devrait l'être.

\section{Fatalité}
%FATALITÉ
Le nom superstitieux du destin : tout serait écrit à l'avance, de
sorte que l'avenir serait aussi impossible à changer que le passé.
Et certes il était vrai, il y a cent mille ans, que tu lirais ces lignes aujourd’hui.
Mais ce n’est pas parce que c'était vrai que tu les lis ; c’est parce que tu les lis
que c'était vrai. La fatalité n’est qu’un contresens sur l’éternité : c’est soumettre
le réel au vrai, quand toute action fait l'inverse.

\section{Fatigue}
%FATIGUE
C’est un affaiblissement, durable ou passager, de la puissance
d’exister et d’agir, suite à un effort trop intense ou trop long. On
dirait une usure ou un épuisement du {\it conatus}, mais qui affecterait le corps ou
le cerveau plutôt que l’âme. C’est ce qui distingue la fatigue de la tristesse, et
%— 240 —
qui explique qu’elles aillent si souvent ensemble. Toute tristesse fatigue, et il
n'est guère de fatigue, si elle dure, qui ne rende un peu triste. L'usage et l’expérience
interdisent pourtant de les confondre absolument : aucun repos ne suffit
à la joie, aucune joie au repos.

Vivre fatigue, et la fatigue ne se dit au sens propre que des êtres vivants. Elle
est l’entropie de vivre. Cela fait comme une lourdeur de tout l’être : les jambes,
la tête, les paupières, la pensée... Le corps n’est plus qu’un poids, et l'esprit
n’est plus rien. C’est le triomphe des imbéciles et des physiciens. Une force obscure
— la vie même — nous pousse vers la mort ou le repos. Vivre fatigue et tue.
Le sommeil est une homéopathie de la mort.

\section{Fausseté}
%FAUSSETÉ
C’est une pensée qui ne correspond pas au réel ou au vrai. Elle
en fait pourtant partie (elle existe réellement : elle est vraiment
fausse), et c’est en quoi sa fausseté reste une détermination extrinsèque ou négative.
« Il n’y a dans les idées rien de positif à cause de quoi elles sont dites
fausses », écrit Spinoza, ce qui veut dire qu'aucune idée n’est fausse en elle-même
ou par ce qu’elle est, mais seulement par ce qu’elle n’est pas ou qui — si
on la compare à une idée vraie — lui fait défaut : « La fausseté consiste dans une
privation de connaissance qu’enveloppent les idées inadéquates, c’est-à-dire
mutilées et confuses » ({\it Éthique}, II, prop. 33 et 35). Ainsi tout est vrai en Dieu
ou en soi, sans que cela nous empêche de mentir ou de nous tromper. C’est que
nous ne sommes pas Dieu. La fausseté est la marque en nous de la finitude : ce
n’est qu’un premier pas vers le vrai. L'erreur est de vouloir s’arrêter.

\section{Fausseté des vertus humaines}
%FAUSSETÉ DES VERTUS HUMAINES
C’est le titre d’un livre de Jacques
Esprit (1611-1678), comparable,
avec moins de talent, aux {\it Maximes} de La Rochefoucauld. Toutes nos vertus ne
seraient que des vices déguisés, que des ruses de l'intérêt ou des mensonges de
l'amour-propre. Voltaire lui consacre un des articles de son {\it Dictionnaire}. Il lui
reproche de ne critiquer la morale que pour faire le lit de la religion, en l’occurrence
catholique, et surtout de mettre Marc Aurèle ou Épictète sur le même
plan que le premier coquin venu. Car enfin si toute vertu est fausse, pourquoi
admirer ces deux-là ou s’interdire de ressembler à celui-ci ? Non, pourtant, que
l'amour-propre n’ait en effet ses pièges, ses leurres, ses illusions. Comment la
vertu serait-elle transparente, puisqu’elle est humaine ? Mais elle n’en continue
pas moins, même opaque, de valoir mieux que son absence. J'imagine, chers
immoralistes, que vous faites une différence entre Cavaillès et ses bourreaux.
Que vous mettez un homme courageux, généreux et droit plus haut qu’un
%— 241 —
salaud égoïste et lâche. Mais alors pourquoi ce mot de {\it vertu} vous agace-t-il si
fort ? Parce que vous doutez qu’elle soit complètement désintéressée ? La belle
affaire ! Qu’un héros puisse trouver du plaisir à en être un, cela en fait-il un
méchant homme ? Parce que vous ne savez pas ce que c’est ? Je vous renvoie à
la définition que j’en donne, qui reprend celles d’Aristote et de Spinoza. Mais
pour aller au plus court, je vous répondrais volontiers ce que Voltaire lançaïit à
Jacques Esprit : « Qu'est-ce que la vertu, mon ami ? C’est de faire du bien :
fais-nous-en, et cela suffit. Alors nous te ferons grâce du motif. »

\section{Faute}
%FAUTE
Une erreur pratique, qui s’écarte moins du vrai que du bien ou du
juste. Par exemple une faute d'orthographe : ce n’est pas qu’elle soit
{\it moins vraie} que l'écriture correcte, mais qu’elle n’est pas {\it la bonne}. Toute faute
suppose une norme de référence, qu’elle reconnaît (dans son principe) et
méconnaît ou transgresse (dans son détail).

Une faute est souvent une erreur, mais pas toujours. Il arrive que je fasse ce
que je crois, à tort, être bien, mais aussi que je fasse ce que je sais pertinemment
être mal. C’est à peu près la différence qu’il y a entre une faute intellectuelle
(une erreur de jugement) et une faute morale. Je suis responsable de la
première ; coupable de la seconde.

\section{Favoritisme}
%FAVORITISME
C'est manquer à la justice par amour ou par solidarité.
Comment est-ce possible ? C’est qu’il s’agit d’un amour
toujours particulier, d’une solidarité toujours partielle, contre la justice universelle.
L'amour n’excuse pas tout ; la solidarité non plus. C’est ce que le mot de
favoritisme, qui vaut universellement comme condamnation, nous rappelle.

\section{Félicité}
%FÉLICITÉ
Ce serait un bonheur absolu : une joie permanente, et qui persévérerait
dans son intensité. Mais la notion en est contradictoire :
ce serait un {\it passage} (Spinoza, {\it Éthique}, III, déf. 2 des affects, et explic. de la
déf. 3) qui ne {\it passerait} pas. Son impossibilité nous distingue des dieux ; son
rêve, des animaux. Elle est au paradis terrestre ce que la béatitude serait à
l'autre. Double mensonge.

\section{Féminité}
%FÉMINITÉ
C'était à la rue d’Ulm, dans les soixante-dix. Je bavardais avec
un ami, dans un couloir de l’École normale supérieure. Soudain,
je vois arriver trois jeunes femmes, bottées, casquées, la cigarette au bec,
%— 241 —
qui me demandent d’un ton rogue : {\it « C'est où, les chiottes ? »} Elles devaient descendre
de moto, et je n’ai évidemment rien contre. Elles avaient bien sûr le
droit de fumer et de dire des gros mots. Mais elles étaient étonnamment masculines,
au pire sens du terme : sans douceur, sans finesse, sans poésie. Cela
suggère, au moins par différence, ce qu'est la féminité. Non une essence ou un
absolu, cela va de soi (les trois motardes, aussi peu féminines qu’elles m’aient
paru, n’en étaient pas moins femmes pour autant), mais un certain nombre de
traits ou de caractères qu’on trouve plus souvent chez les femmes, sans lesquels
l'humanité se réduirait à la masculinité, avec tout ce qu’elle comporte de violence
et de lourdeur, de prosaïsme et d’ambition — ce que Rilke appelait « le
mâle prétentieux et impatient ».

Les deux notions de féminité et de masculinité ne peuvent se définir que
lune par l’autre. C’est ce qui les rend toujours relatives et insatisfaisantes, mais
aussi nécessaires. Freud, à la fin de sa vie, s’interrogeait encore sur ce que veulent
les femmes. Peut-être savait-il mieux, parce qu’il en était un, ce que veulent
les hommes : le pouvoir, le sexe, l’argent, l'efficacité, la gloire. On m’objectera
que les femmes n’y sont pas non plus indifférentes. Je le sais bien. Mais il
me semble qu’elles auront tendance, plus souvent que les hommes, à privilégier
un certain nombre d’enjeux qui relèvent davantage de la vie privée et affective :
la parole, l'amour, les enfants, le bonheur, la durée, la paix, la vie... Il faut
certes se méfier de ces catégories, toujours trop massives, toujours trop vagues,
et qui risquent d’enfermer chaque être humain dans un rôle convenu, qu’il
n'aurait pas choisi. Mais comment — sauf à refuser à la différence sexuelle toute
autre réalité que physiologique — s’en passer tout à fait ? Il m’est arrivé de dire,
par boutade, que l'amour était une invention des femmes, qu’une humanité
exclusivement masculine n’en aurait jamais eu l’idée — que le sexe et la guerre
lui auraient suffi, toujours. Disons, plus sérieusement, qu’hommes et femmes
ont tendance à vivre différemment l'articulation de l’amour et de la sexualité.
La plupart des hommes mettent l'amour au service du sexe, quand les femmes,
du moins la plupart d’entre elles, mettraient plutôt le sexe au service de
l'amour. Ce n’est qu’une tendance, dont il se peut qu’elle soit davantage culturelle
que naturelle. Elle n’en fait pas moins partie, me semble-t-il, de notre
expérience. Cela laisse une chance à la séduction et au couple (puisque nous
désirons les uns et les autres et l’amour et le sexe), mais n’est pas non plus sans
entraîner parfois, dans nos relations, un certain nombre de difficultés ou de
malentendus.

On pourrait faire des remarques du même genre sur la violence et la douceur,
sur le rapport au temps ou à l’action. Quelques semaines de guerre suffisent
à tout détruire : les hommes s’en chargent fort bien. Après quoi il faut des
%— 243 —
années de patience et d’efforts pour que la vie reprenne ses droits : je ne suis pas
sûr que nous en serions capables sans les femmes.

Mais revenons à nos trois motardes. C’était la première fois, je crois bien,
que la féminité, alors peu en vogue, m’apparaissait comme une valeur. J'avais
vingt ans. Je n'avais pas encore lu Rilke (« la femme est sans doute plus mûre,
plus près de l’humain que l’homme... »), ni Colette, ni Simone Weil, ni Etty
Hillesum... Certes, j'avais fait toutes mes études secondaires dans un lycée
mixte : sortant de l’école communale, alors exclusivement masculine, cela
m'avait paru une espèce de paradis, comme quand on quitte un wagon de
bidasses pour entrer dans un lieu civilisé.. Mais je disais « les filles » alors,
plutôt que les femmes ou la féminité, et n’aurais guère osé, surtout, en tirer
quelque idée générale que ce soit. Le féminisme, ces années-là, était une espèce
d’évidence, pour tout intellectuel progressiste. Mais la féminité, non : beaucoup,
chez les hommes comme chez les femmes, n’y voyaient qu’un dernier
piège ou une dernière illusion, dont il était urgent — au nom de l’universel ou
de la Révolution, voire au nom du féminisme lui-même — de se libérer. Regar-
dant les trois motardes s'éloigner, je réalisais que ce n’était pas si simple, et que
nous risquions, sur cette pente-là, de perdre quelque chose d’important. Elles
ne m'ont pas rendu le féminisme moins sympathique. Mais la féminité, plus
précieuse.

\section{Femme}
%FEMME
Un être humain, de sexe féminin. On dira que c’est l’humanité qui
importe, non le sexe. Peut-être. Mais que l’humanité soit sexuée
n’est pas non plus anecdotique.

La différence sexuelle est sans doute l’une des plus fortes, des plus constantes,
des plus structurantes qui soient. Chacun d’entre nous ne cesse de s’y
confronter. Et pourtant toute tentative de caractériser positivement cette
moitié-là de l’humanité (donc aussi de caractériser l’autre) ne débouche que sur
des approximations ou des platitudes. Que les femmes soient ordinairement
moins violentes que les hommes, qu’elles aient davantage le sens du concret, de
la durée, du quotidien (une certaine façon, chez les meilleures, d’être de plain-pied
avec la vie ou le réel), qu’elles soient plus douées pour l’amour et l’intimité,
moins portées vers la pornographie et le pouvoir, c’est ce qui semble souvent
vrai, mais qui ne va pas, chez les hommes comme chez les femmes, sans de
nombreux contre-exemples, qui interdisent d’en faire une loi ou une essence.
La différence, entre les deux sexes, reste floue, et doit autant ou davantage à la
culture, selon toute vraisemblance, qu’à la nature. Toutefois cela ne prouve pas
que les femmes n’existent pas, ni qu’elles ne soient tendanciellement différentes
des hommes. Pourquoi le flou existerait-il moins que le net, ou le culturel
%— 244 —
moins que le naturel ? « On ne naît pas femme, disait Simone de Beauvoir, on
le devient. » C’était mettre curieusement le corps entre parenthèses. La biologie
m'éclaire davantage (on naît femme, ou homme, puis on devient ce que l’on
est, de façon plus ou moins féminine ou masculine), mais peu importe : ce
{\it devenir}-là, quand bien même il devrait tout à la culture, est l’un des plus beaux
cadeaux que l'humanité se soit faits à elle-même.

\section{Fête}
%FÊTE
Un moment privilégié, souvent inscrit d’avance dans le calendrier en
souvenir d’un autre, qu’il commémore: occasion d’abord de
recueillement, puis de réjouissances. De nos jours, les réjouissances tendent de
plus en plus à l’emporter. C’est pourquoi elles sont souvent un peu tristes ou
contraintes — ou le seraient, sans l'alcool. « Une fête est un excès permis, écrit
Freud, voire ordonné » ({\it Totem et tabou}, IV, 5). Mais quoi de plus oppressant
qu’un excès obligatoire ? Quoi de plus démoralisant qu’une joie programmée ?
Heureusement que la fête nous le fait oublier ! Puis c’est aussi l’occasion de
revoir ses amis, à quoi le hasard et l’amitié ne suffisent pas toujours.

\section{Fétichisme}
%FÉTICHISME
C’est aimer un objet plutôt qu’un sujet, un symbole plutôt
que ce qu’il symbolise, ou la partie plutôt que le tout : par
exemple une statuette plutôt que le dieu, la valeur d'échange ou l'argent plutôt
que la valeur d’usage ou le travail (Marx), une partie du corps ou un vêtement
plutôt que le corps entier et sexué (Freud). C’est avouer et dénier à la fois une
absence : celle du dieu, celle du travail vivant, celle (selon Freud) du phallus de
la mère.

En un sens plus large, on pourrait parler de fétichisme pour tout amour qui
reste prisonnier de son objet, dont il croit dépendre. Si je l'aime parce qu'il est
seul aimable, comment pourrais-je jouir ou me réjouir sans lui ? À quoi Spinoza
objecte que le désir est premier, qui donne à l’objet sa valeur : ce n’est pas
parce qu’il est aimable que je l’aime, c’est parce que je l’aime qu’il est aimable
(pour moi), et c’est en quoi tout l’est ou peut l’être. Ainsi tout amour est fétichiste,
tant qu’il n’est pas universel. C’est avouer et dénier à la fois l’absence en
nous de l’amour pour tout le reste. Seule la charité y échappe. Mais en sommes-nous
capables ?

\section{Fidéisme}
%FIDÉISME
Toute doctrine, spécialement religieuse, qui ne se fonde que sur
la foi {\it (fides)}, à l'exclusion de toute connaissance rationnelle.
C’est le contraire du rationalisme en matière de religion. L'Église y voit traditionnellement
%— 245 —
deux hérésies : ni foi seule ni raison seule ne suffisent ; la vraie
religion a besoin des deux. Sans doute. Mais quelle est la vraie religion ? Seule
la foi répond, et c’est ce qui donne raison, malgré tout, au fidéisme — ou à
l’athéisme.

\section{Fidélité}
%FIDÉLITÉ
On ne la confondra pas avec l'exclusivité. Être fidèle à ses amis,
ce n’est pas n’en avoir qu’un. Être fidèle à ses idées, ce n’est pas
se contenter d’une seule. Même en matière amoureuse ou sexuelle, et malgré
l'usage ordinaire du mot, la fidélité ne se réduit pas plus à l'exclusivité qu’elle
ne la suppose nécessairement. Rien n'empêche, au moins en théorie, deux
amants de se rester fidèles, quand bien même ils pratiqueraient l’échangisme ou
s’autoriseraient mutuellement des aventures avec d’autres. Combien d’époux, à
l'inverse, sans jamais se tromper mutuellement, au sens sexuel du terme, ne cessent
pourtant de se mentir, de se mépriser, de se haïr parfois, qui sont pour cela
plus infidèles que les plus libérés des amants ?

La fidélité n’est pas l’exclusivité ; c’est la constance, c’est la loyauté, c’est la
gratitude, mais tournées toutes les trois vers l’avenir au moins autant que vers
le passé. Vertu de mémoire, certes, mais aussi d’engagement : c’est le souvenir
reconnaissant de ce qui a eu lieu, joint à la volonté de l’entretenir, de le protéger,
de le faire durer, tant que c’est possible, bref de résister à l'oubli, à la trahison,
à l’inconstance, à la frivolité, et même à la lassitude. C’est pourquoi la
fidélité amoureuse a souvent à voir, en pratique, avec l'exclusivité sexuelle : dès
lors qu'on s’est promis celle-ci, elle fait partie de celle-à. Faut-il se la
promettre ? Affaire de goût ou de convenance, à ce que je crois, davantage que
de morale, La fidélité, sans cette exclusivité, n’en vaut pas moins. Mais elle me
semble plus inconfortable, plus exposée, plus aléatoire, enfin plus difficile et
trop pour moi.

On parle aussi de fidélité en matière de religion : c’est la vertu des croyants,
qui restent fidèles à leur foi ou à leur Église. On aurait tort de croire que les
athées en seraient pour cela dispensés. C’est le contraire qui me paraît vrai.
Tant que la foi est là, elle pousse à la fidélité. C’est dire que la fidélité n’est
jamais aussi nécessaire que lorsque la foi fait défaut. Que les deux mots aient la
même origine (le latin {\it fides}), cela ne signifie pas qu’ils soient synonymes. La foi
est une croyance ; la fidélité, une volonté. La foi est une grâce ou une illusion ;
la fidélité, un effort. La foi est une espérance ; la fidélité, un engagement.
Allons-nous, sous prétexte que nous ne croyons plus en Dieu, oublier toutes ces
valeurs que nous avons reçues, dont la plupart sont d’origine religieuse, certes,
mais dont rien ne prouve qu’elles aient besoin d’un Dieu pour subsister, et
dont tout prouve au contraire que nous avons besoin d’elles pour survivre
%— 246 —
d’une façon qui nous paraisse humainement acceptable ? Allons-nous, sous
prétexte que Dieu est mort, laisser son héritage en déshérence ? Que Dieu
existe ou pas, qu'est-ce que cela change à la valeur de la sincérité, de la générosité,
de la justice, de la miséricorde, de la compassion, de l'amour ? Vertu de
mémoire, disais-je, qui vaut aussi pour la mémoire des civilisations. S'agissant
de la nôtre, puisqu'il faut bien en dire un mot, la vraie question me paraît être
la suivante : que reste-t-il de l'Occident chrétien, quand il n’est plus chrétien ?
Je ne vois guère que deux réponses possibles. Ou bien vous pensez qu’il n’en
reste rien, et alors nous sommes une civilisation morte, ou mourante : nous
n'avons plus rien à opposer ni au fanatisme, à l’extérieur, ni au nihilisme, à
l'intérieur (et le nihilisme est de très loin le danger principal). Autant aller se
coucher et attendre la fin, qui ne tardera pas... Ou bien, deuxième possibilité,
vous pensez qu’il en reste quelque chose ; et si ce qu’il en reste ce n’est plus une
{\it foi} commune (puisqu'elle a cessé, de fait, d’être commune : un Français sur
deux se dit athée ou agnostique, un sur quatorze est musulman...), ce ne peut
être qu’une {\it fidélité} commune : le refus d’oublier ce qui nous a faits, de trahir ce
que nous avons reçu, de le laisser finir ou dépérir avec nous. La fidélité, c’est ce
qui reste de la foi quand on l’a perdue : un attachement partagé à ces valeurs
que nous avons reçues et que nous avons à charge de transmettre (la seule façon
d’être vraiment fidèle à ce qu’on a reçu, c’est d’assurer sa pérennité, pour autant
qu’elle dépend de nous, autrement dit sa transmission). Du passé ne faisons pas
table rase : ce serait vouer l’avenir à la barbarie.

« Quand on ne sait où l’on va, dit un proverbe africain, il faut se souvenir
d’où l’on vient. » C’est la seule façon de savoir où l’on {\it veut} aller. Cela vaut pour
notre civilisation comme cela vaut pour les autres, et pour leur ensemble qui est
la civilisation même. Fidélité à l'humanité, je veux dire (car la fidélité n’est ni
la complaisance ni l’aveuglement) à sa meilleure part: celle qui la rend
humaine, et nous avec.

« La fidélité, disait Alain, est la principale vertu de l'esprit. » C’est qu’il n’y
a pas d’esprit sans mémoire, et que la mémoire pourtant n’y suffit pas : encore
faut-il vouloir ne pas oublier, ne pas trahir, ne pas abandonner, ne pas
renoncer, et c’est la fidélité même.

\section{Fierté}
%FIERTÉ
Contentement de soi (fût-ce dans son rapport à autrui : « Je suis
fier de toi »), mais qui ne va guère sans un peu de mépris pour les
autres. C’est le sentiment, qui se croit légitime, de sa propre supériorité, ou en
tout cas de sa propre valeur, en tant qu’elle excède la moyenne ou qu’elle serait
(mais par qui ?) {\it méritée}. Si la fierté se situe quelque part entre la dignité et
%— 247 —
l’orgueil, elle est plus proche de celui-ci : c’est un défaut qui se prend pour une
vertu, une petitesse qui se croit grande. Manque d’humilité, donc de lucidité.

La fierté ne vaut que comme défense, contre le mépris dont on est l’objet.
Qu’on manifeste pour la « fierté homosexuelle » (la {\it gay pride}), soit. Une manifestation
pour la fierté hétéro ne serait qu’un rassemblement de beaufs ou
d’homophobes.

\section{Fin}
%FIN
Le mot a deux sens très différents, qu’il importe de ne pas confondre :
il peut désigner la limite ou le but, le terme ou la destination, la {\it finitude}
ou la {\it finalité}. Par exemple que la mort soit la {\it fin} de la vie, cela fait partie
de sa définition ; mais ne nous dit pas si elle est son {\it but}, comme le voulait
Platon, ou simplement, et comme disait Montaigne, son {\it bout} ({\it Essais}, III, 12,
1051). On dira que les deux peuvent ne faire qu’un : tel serait le cas de la ligne
d’arrivée dans une course. Je ne suis pas sûr que l’exemple soit pertinent (le but,
dans une course, est moins d’arriver que de vaincre), ni, encore moins, qu’on
puisse le généraliser. Ce n’est pas pour son dernier mot qu’on écrit un livre, pas
pour son dernier jour qu’on vit tous les autres, pas pour sa clôture qu’on cultive
un jardin.

Ce qui reste vrai, en revanche, et que les Grecs savaient mieux que nous,
c’est que ces deux sens sont à la fois liés et asymétriques : la finalité suppose la
finitude, non la finitude la finalité. L’infini, par définition, ne peut aller nulle
part, ni tendre vers quoi que ce soit. Imaginez une autoroute infinie : {\it où} pourrait-elle
aller ? Un univers infini : {\it vers quoi} pourrait-il tendre ? Un Dieu infini :
{\it à quoi} pourrait-il servir ? Alors que le moindre de nos actes, pour fini qu’il soit,
n’en a pas moins une finalité (le but que nous visons à travers lui). L’infini n’est
pas un but plausible, ni n’en peut avoir. Seul le fini vaut la peine qu’il se donne.

\section{Finale (cause —)}
%FINALE (CAUSE —)
Une cause, c’est ce qui répond à la question {\it « Pourquoi ? »} ;
une cause finale, ce qui y répond par l’énoncé
d’un {\it but}. Par exemple, explique Aristote, la santé est la {\it cause finale} de la promenade,
si, à la question : « Pourquoi se promène-t-il ? », on peut répondre
légitimement : « Pour sa santé » ({\it Physique}, II, 3). Ainsi la cause finale est {\it ce en
vue de quoi} quelque chose existe, qui n’existerait pas sans cela.

Mais alors pourquoi ne se promènent-ils pas tous, ni toujours ? La santé
n'est-elle une fin que pour quelques-uns, et à quelques moments ? Non pas ;
mais elle n’agit, comme fin, qu’à condition qu’un désir actuel et actif la vise.
C’est où l’on échappe au finalisme — et à Aristote — par Spinoza : la santé n’est
%— 248 
une cause finale que pour autant que le désir de santé est une cause efficiente
({\it Éthique}, IV, Préface).

\section{Finalisme}
%FINALISME
Toute doctrine qui accorde aux causes finales un rôle effectif.
On l’illustre souvent par les exemples délicieusement outrés de
Bernardin de Saint-Pierre. Pourquoi les melons sont-ils divisés en côtes ? Pour
qu’on puisse plus aisément les manger en famille. Pourquoi Dieu nous a-t-il
donné des fesses ? Pour que nous puissions plus confortablement nous asseoir
et méditer sur les merveilles de sa création. Mais cela ne doit pas faire oublier
que la plupart des grands philosophes — depuis Platon et Aristote jusqu’à
Bergson et Teilhard de Chardin — ont été finalistes. Au reste, comment y
échapper, si l’on croit en un Dieu créateur ou ordonnateur ? Même Descartes,
qui fit tant pour que les scientifiques cessent de chercher des causes finales, ne
contestait pas qu’elles puissent exister en Dieu, mais seulement que nous puissions
les connaître ({\it Principes}, III, 1-3). Quant à Leibniz, il y voyait « le principe
de toutes les existences et des lois de la nature, parce que Dieu se propose toujours
le meilleur et le plus parfait » ({\it Discours de métaphysique}, 19). Pourquoi
aurions-nous des yeux, si ce n’était {\it pour voir} ?

Ce dernier exemple, repris par Leibniz et tant d’autres, suggère bien l’essentiel.
Pourquoi voyons-nous ? Parce que nous avons des yeux. Pourquoi avons-nous
des yeux ? Pour voir. Ainsi les yeux sont la cause efficiente de la vue ; la
vue, la cause finale des yeux. Mais laquelle de ces deux causes existe d’abord ?
Est-ce la fonction qui crée l’organe (ce qui est une forme de finalisme), ou
l'organe qui crée la fonction ? Les matérialistes, presque tous, presque seuls,
choisissent résolument le deuxième terme de l'alternative. Penser que nous
avons des yeux pour voir, explique Lucrèce, « c’est faire un raisonnement qui
renverse le rapport des choses, c’est mettre partout la cause après l’effet » (IV,
823 sq.). Mais pourquoi, alors, avons-nous des yeux ? Par hasard ? Non pas ;
mais seulement par des causes efficientes, qui renvoient au passé de l’espèce
(par l’hérédité et la sélection naturelle), non à l’avenir de l’individu.

Spinoza, sur cette question comme sur tant d’autres, est du côté des matérialistes.
Le finalisme, pour lui aussi, « renverse totalement la nature : il considère
comme effet ce qui en réalité est cause, et met après ce qui de nature est
avant » ({\it Éthique}, I, Appendice). C’est là le préjugé fondamental, dont tous les
autres dérivent : « Les hommes supposent communément que toutes les choses
de la nature agissent, comme eux-mêmes, en vue d’une fin » ({\it ibid.}). S'agissant
de la nature, c’est clairement une illusion. Mais s'agissant d’eux-mêmes ? C'en
est une aussi, du moins si l’on y voit la cause de leurs actes : cette illusion
s'appelle alors le libre arbitre. Soit par exemple cette maison que je construis.
%— 249 —
Pourquoi le fais-je ? Si on répond {\it « pour l'habiter »}, comme beaucoup le feront
spontanément, cela signifie que ce qui n’est pas encore (l’habitation) explique
et produit ce qui est (le travail de construction), que c’est la fin, comme dira
Sartre, qui « éclaire ce qui est » : tel est le paradoxe de la liberté ({\it L'Étre et le
Néant}, p. 519-520, 530, 577-578...). Mais comment ce qui n’est pas encore
pourrait-il produire ou expliquer quoi que ce soit ? Que les hommes agissent
toujours « en vue d’une fin », comme le reconnaît Spinoza, cela ne prouve pas
que cette fin soit la cause de l’action : ce qu’on prend pour une cause finale
n’est rien d’autre qu’un désir déterminé (ici le désir d’habiter cette maison),
et ce désir «est en réalité une cause efficiente » ({\it Éthique}, IV, Préface). Les
hommes se trompent en ce qu’ils se figurent être libres ; cette illusion n’est
pas autre chose qu’un finalisme à la première personne : ils ont conscience de
viser une fin, point des causes efficientes qui les déterminent à le faire
(I, Appendice).

Ainsi il n’y a pas de finalité du tout : il n’y a que la puissance aveugle de la
nature, et celle, qui peut être éclairée, du désir.

\section{Finalité}
%FINALITÉ
Le fait de tendre vers une fin ou un but. C’est le cas par exemple
de la plupart de nos actions, et même, selon Freud, de nos actes
manqués. Cela ne prouve pas que cette fin soit la {\it cause} de l’acte. C’est où la
finalité, qui est une donnée de la conscience ou un fait, se distingue du finalisme,
qui est une doctrine et un contresens. On agit {\it pour} une fin, mais {\it par} un
désir : la finalité n’est elle-même qu’un effet de l'efficience du désir.

On pourrait en dire autant de la finalité du vivant, qui est un fait d’expérience,
mais dont rien ne prouve qu’elle soit une {\it cause}. C’est pourquoi nos biologistes
parlent plutôt, pour la désigner, de téléonomie, qui est une finalité sans
finalisme (une finalité pensée comme effet, point comme cause).

\section{Finesse (esprit de —)}
%FINESSE (ESPRIT DE —)
Le sens des nuances, du flou, de la complexité
inépuisable du réel, bref de tout ce qui se sent
davantage que cela ne se voit ou ne se démontre. S’oppose traditionnellement,
surtout depuis Pascal, à l'esprit de géométrie ({\it Pensées}, 512-1 et 513-4).

\section{Fini}
%FINI
En philosophie, c’est moins ce qui est terminé que ce qui peut l'être :
ce qui n’est pas infini. La {\it Symphonie inachevée} est aussi {\it finie} que toutes
les autres. Et nous sommes voués à la finitude bien avant d’être morts.

%— 250 
\section{Finitude}
%FINITUDE
Le fait d’avoir une limite, un terme, une borne {\it (finis)} — de n’être
pas infini. Les Anciens y voyaient plutôt un bonheur, pour qui
savait s’en contenter. « Épicure fixa des bornes {\it (finem statuit)} au désir comme à
la crainte », se réjouit Lucrèce, et il n’y aurait pas de sagesse autrement. Le faux
infini des désirs nous voue à l’insatisfaction, au malheur, à la démesure. Seul
celui qui s’accepte fini peut échapper à l'angoisse : la sagesse est une finitude
heureuse, dans l'infini qui nous contient. Quel bonheur d’être un homme,
quand on ne prétend pas à autre chose!

Chez les Modernes, et spécialement chez les existentialistes, la finitude
prend des couleurs plus sombres : c’est comme une amputation de l'infini, qui
resterait, telle un membre fantôme, à jamais douloureuse. C’est le malheur de
n'être pas Dieu. La finitude, en ce sens, est le propre de l’homme, en tant qu’il
est voué à la mort. Non qu’il soit seul fini, ni qu’il le soit par la mort seule.
Mais parce qu’il est seul à savoir clairement qu’il l’est (les animaux, selon toute
vraisemblance, n’ont aucune notion de l'infini, ni donc de la finitude), et qu’il
va mourir. Attention, toutefois, de ne pas accorder trop à la mort. Sur la finitude,
il me semble que le sexe et la fatigue nous en apprennent davantage.

\section{Flatterie}
%FLATTERIE
C'est dire à quelqu'un, pour attirer ses bonnes grâces, plus de
bien qu’on n’en pense. Quand c’est fait un peu habilement,
cela ne rate presque jamais, ce qui en dit long sur l'humanité. L’amour-propre,
presque inévitablement, l'emporte sur l'amour de la vérité. « Ainsi, écrit Pascal,
la vie humaine n’est qu’une illusion perpétuelle ; on ne fait que s’entre-tromper
et s’entre-flatter. Personne ne parle de nous en notre présence comme il en
parle en notre absence. L’union qui est entre les hommes n’est fondée que sur
cette mutuelle tromperie ; et peu d’amitiés subsisteraient, si chacun savait ce
que son ami dit de lui lorsqu'il n’y est pas, quoiqu'il en parle alors sincèrement
et sans passion » ({\it Pensées}, 978-100).

\section{Foi}
%FOI
Croyance sans preuve, comme toute croyance, mais qui s’en passe avantageusement,
par volonté, confiance ou grâce. Avantage équivoque,
voire suspect. C’est se croire, se fier, ou se soumettre. Toute foi pêche par suffisance,
ou par insuffisance.

« La foi, écrit Kant, est une croyance qui n’est suffisante que subjectivement »
({\it C. R. Pure}, théorie de la méthode, II, 3). Elle ne l’est donc que
pour les sujets qui se suffisent de leur subjectivité. Pour les autres, le doute
l'accompagne et la sauve.

%— 251 —
Au sens le plus ordinaire, le mot désigne une croyance religieuse, et tout ce
qui lui ressemble. C’est croire en une vérité qui serait une valeur, en une valeur
qui serait une vérité. Avoir foi en la justice, par exemple, c’est non seulement
aimer la justice mais croire qu’elle existe. Avoir foi en l’amour, c’est non seulement
l’aimer mais en faire un absolu, qui existerait indépendamment de nos
amours très relatives. C’est pourquoi la foi porte spécialement sur Dieu : parce
qu’il serait la conjonction absolue de la valeur (qu’on doit aimer) et de la vérité
(qu’on peut connaître ou reconnaître). C’est aussi sa limite : si l’on connaissait
Dieu, on n’aurait plus besoin d’y croire.

La foi porte également sur l'avenir. C’est comme une utopie métaphysique :
l’espérance s’invente un objet, qui la transforme en vérité. Il s’agit de
croire, comme disait Kant, que « quelque chose est. puisque quelque chose
doit arriver ». Ce mensonge, dans sa sincérité, est la religion même.

La foi ne se nourrit que de l’ignorance de son objet. « Je dus donc mettre
de côté le {\it savoir}, reconnaît encore Kant, afin d’obtenir une place pour la {\it foi} »
({\it C. R. Pure}, Préface de la 2$^\text{e}$ éd.). Les savants, depuis vingt-cinq siècles, font
l'inverse.

\section{Folie}
%FOLIE
« Le fou a tout perdu, disait un psychiatre, sauf la raison. » Mais elle
tourne à vide : elle a perdu les {\it rails} du réel. Le délire paranoïaque,
par exemple, peut être d’une formidable cohérence (ce n’est pas un hasard si
Freud compare les systèmes philosophiques à des paranoïas réussies) ; mais il
est fermé sur lui-même au lieu d’ouvrir sur le monde. C’est une leçon pour le
philosophe : la pensée n’échappe à la folie que par son dehors, qui est le réel ou
la pensée des autres. Ce qui n’est vrai que pour toi ne l’est pas.

\section{Fondement}
%FONDEMENT
Disons d’abord, avec Marcel Conche, ce que ce n’est pas :
un fondement n’est ni un principe, ni une cause, ni une origine.
La cause explique un fait ; le fondement établit un droit ou un devoir.
L'origine rend raison d’un devenir ; le fondement, d’une valeur. Enfin un principe
n’est que le point de départ — qui peut être hypothétique ou douteux —
d’un raisonnement ; le fondement serait « la justification radicale du principe
lui-même» ({\it Le fondement de la morale}, Introduction). Qu'est-ce qu’un
fondement ? La justification nécessaire et suffisante d’un droit, d’un devoir,
d’une valeur ou d’un principe, de telle sorte que l'esprit puisse {\it et doive} leur
donner son assentiment. Un fondement, c’est donc ce qui garantit la valeur ou
la vérité de ce qu’il fonde : ce qui nous permettrait d’être certain (non seulement
en fait mais en droit) d’avoir raison.

%— 252 —
C’est pourquoi il n’y a pas de fondement, me semble-t-il, ni ne peut y en
avoir : parce qu’il faudrait qu’il soit d’abord lui-même rationnellement
démontré ou établi, ce qui n’est possible qu’à la condition de fonder d’abord la
valeur de notre raison, laquelle ne peut l'être ni sur elle-même (car il y aurait là
un cercle) ni sur autre chose (car il y aurait là une régression à l’infini, cette
autre chose devant à son tour être fondée, et ne pouvant l’être que sur la raison
ou sur autre chose). Non, certes, que la raison ne vaille rien, ce qui n’est ni
démontrable ni vraisemblable, mais parce que sa valeur, qui permet nos
démonstrations, ne peut elle-même être rationnellement démontrée. La proposition
«Il y a de vraies démonstrations » est indémontrable — puisque toute
démonstration le suppose.

Un fondement des mathématiques ? Non seulement on a fait des mathématiques
bien avant de disposer de quelque fondement que ce soit, mais les
mathématiciens d’aujourd’hui, si brillants, si performants, ont à peu près
renoncé à en chercher un. Au reste, dès lors que le théorème de Güdel a établi
que, dans un système formel contenant au moins l’arithmétique, on ne peut ni
tout démontrer (il y a des énoncés indécidables) ni démontrer que le système
n’est pas contradictoire (la cohérence du système est elle-même indécidable à
l'intérieur de ce système), on ne voit guère, d’un point de vue philosophique,
quel sens il y aurait à prétendre {\it fonder} les mathématiques : comment garantir
une cohérence que l’on ne peut démontrer ? La proposition « Les mathématiques
sont vraies » (ou « Les mathématiques sont cohérentes ») n’est pas susceptible
d’une démonstration mathématique, ni de quelque démonstration que
ce soit. Cela, qui interdit de fonder les mathématiques, n’empêche pas d’en
faire, et ne leur retire rien d’autre que l'illusion de l’absoluité.

Un fondement de la morale ? Ce ne pourrait être que la conjonction nécessaire
et absolue (ni contingente ni dépendante) du vrai et du bien, de la valeur
et de la vérité : ce ne pourrait être que Dieu, et c’est pourquoi ce n’est pas. Que
vaudrait une morale qui aurait besoin d’un Dieu pour valoir ? Ce serait une
morale dépendante d’une religion, qu’il faudrait à son tour fonder : démontrez-moi
quelle est la vraie religion, je vous dirai quelle est la vraie morale ! Si on
laisse de côté ce fondement théologique, qui n’en serait pas un, tout fondement
de la morale doit lui-même être démontré (ce qui nous renvoie aux apories précédentes),
et peut d’autant moins l’être que sa vérité même, à supposer qu’on
puisse l’établir, n’y suffirait pas : car pourquoi devrais-je me soumettre au vrai ?
Pourquoi ne pas préférer le faux, l'erreur, l'illusion ? Cet individu, par exemple,
qui n’hésite pas à assassiner, à violer, à torturer, pourquoi devrait-il se soumettre
au principe de non-contradiction ? Et pourquoi aurions-nous besoin
d’un fondement pour le combattre ou pour lui résister ? L’horreur suffit. La
compassion suffit, et vaut mieux.

%— 253 —
\section{Force}
%FORCE
Une puissance en acte ({\it energeia}, en grec, plutôt que {\it dunamis}). Se
dit spécialement en mécanique : on appelle {\it force} ce qui modifie le
mouvement (ou le repos) d’un corps, qui resterait autrement — par le principe
d’inertie — rectiligne et uniforme.

On oppose souvent la force au droit, comme les lois de la nature aux lois
des hommes. On a raison. Le droit du plus fort n’est pas un droit ; le droit du
plus faible, pas une force. C’est pourquoi on a besoin d’un État, pour que la
force et le droit aillent ensemble.

On parle parfois de {\it force d'âme} pour désigner le courage. C’est encore,
quoiqu'en un sens métaphorique, le contraire de l’inertie : la puissance de
modifier son propre mouvement, ou son propre repos. Le corps voudrait fuir,
et l'on ne fuit pas. Céder, et l’on ne cède pas. Frapper, et l’on ne frappe pas.
C'est ce qui fait croire à l'âme, et l’on a à nouveau bien raison. Mais elle
n'existe que par courage et volonté : « Ce beau mot ne désigne nullement un
être, disait Alain, mais toujours une action. » Ainsi toute Âme est force d'âme,
mais non toute force.

\section{Forclore}
%FORCLORE
C’est enfermer dehors. Se dit par exemple d’un droit, lorsque
son délai d'application a été dépassé, ou d’une représentation,
quand on refuse de la prendre en compte. Le mot, qui n’appartenait guère qu’au
vocabulaire des juristes, a trouvé une nouvelle jeunesse et un nouvel emploi dans
la psychanalyse, spécialement lacanienne, pour traduire le {\it Verwerfung} de Freud :
c'est rejeter une représentation ou un signifiant hors du sujet ou de son univers
symbolique (et non pas, comme dans le refoulement, à l’intérieur de l’inconscient),
de telle sorte qu’ils feront retour du dehors, spécialement sous forme
d’hallucinations : « Ce qui a été aboli à l’intérieur revient de l'extérieur », écrit
Freud, ce qui veut dire, précise Lacan, que « ce qui a été forclos du symbolique
réapparaît dans le réel ». Ce mécanisme serait à l’origine des psychoses, et les distinguerait
des névroses (qui doivent davantage au refoulement et au retour du
refoulé). Le névrosé est prisonnier du passé, qu’il a refoulé ; le psychotique, du
présent, qu'il a forclos : il s’enferme hors du réel (délire, hallucination) en voulant
l’enfermer hors de lui (forclusion). Cela me fait penser à cette histoire d’un fou
aveugle, qui se heurte à une colonne, qui en fait le tour plusieurs fois, qui la
palpe, comme à la recherche d’une issue, puis qui finit par s’écrier : « Je suis
enfermé ! » Ainsi le psychotique, enfermé dans le réel même qu’il rejette.

\section{Formalisme}
%FORMALISME
Juger sur la forme, plutôt que sur le contenu matériel ou
affectif. Ainsi en logique formelle ou en mathématiques :
%— 25 —
on raisonne sur des x et des y, à l’intérieur d’un système de signes lui-même
réglé par une axiomatique et sans se soucier de ce que ces signes peuvent signifier.
C’est remplacer la représentation par le calcul, et il n’y aurait pas de
science autrement. Cela, toutefois, ne prouve pas que le monde soit fait d’x et
d'y.

En philosophie morale, on parle de formalisme, spécialement chez Kant,
pour désigner une doctrine morale qui fait de « la pure forme d’une loi » (donc
de l’exigence d’universalisation possible) l'essentiel de la moralité, indépendamment
des affects mis en jeu comme des effets de l’action. C’est mettre le devoir
plus haut que les sentiments, et l’intention plus haut que la réussite.

\section{Forme}
%FORME
Soit, par exemple, cette statue d’Apollon. On peut, avec Aristote,
distinguer sa {\it matière} (le marbre dont elle est faite) et sa {\it forme} (celle
que le sculpteur lui a donnée). On comprend que la forme est la fin, vers
laquelle tend le travail du sculpteur (la matière n’est qu’un matériau : un point
de départ), et où il s’arrête. Mais la forme est aussi l’essence ou la quiddité.
Qu'est-ce que c’est ? Une statue d’Apollon. On aurait pu faire, avec le même
bloc de marbre, une tout autre statue, comme on aurait pu faire la même, pour
l'essentiel, avec un autre bloc, voire en bois ou en bronze. Ainsi la forme est à
la fois la définition et le définitif: ce qu’est cette statue, une fois qu’elle est
achevée. Non, bien sûr, qu’on ne puisse la briser (le définitif n’est pas éternel),
mais parce qu’elle a atteint sa perfection ou son entéléchie : elle n’était qu’en
puissance dans la matière ; elle est en acte dans sa forme.

Une forme sans matière ? Ce ne serait plus une forme mais une {\it idée} (en grec
c’est le même mot : {\it eîdos}) ou Dieu. C’est où l’on remonte d’Aristote à Platon —
ou du moins où l’on remonterait, si une forme sans matière était autre chose
qu’une abstraction.

Chez Kant, la forme est ce qui met en forme, autrement dit ce qui, venant
du sujet, organise la matière de la sensation. Ainsi les formes de la sensibilité
(l'espace et le temps) ou de l’entendement (les catégories). Reste à savoir si ces
formes existent elles-mêmes indépendamment de la matière, comme le veut
Kant, ou si elles ne sont que l’effet en nous de sa puissance auto-organisatrice.
C’est où il faut choisir entre l’idéalisme et le matérialisme : entre la transcendance
de la forme et l’immanence de la structure.

\section{Formelle (cause —)}
%FORMELLE (CAUSE)
L’une des quatre causes chez Aristote: celle qui
répond à la question « Pourquoi ? » par l’énoncé
d’une {\it forme}. Par exemple : Pourquoi cette maison ? À cause des briques (cause
%— 255 
matérielle), du plaisir de l’habiter (cause finale), du maçon, de l’architecte, du
maître d'œuvre (causes efficientes) ? Sans doute. Mais aucune de ces causes
n'aurait fait une maison sans le plan qui la rend possible et qu’elle actualise —
sans la forme ou idée ({\it eîdos}) de la maison, non de façon séparée, comme le voudrait
Platon, mais telle qu’elle existe d’abord dans l'esprit des bâtisseurs puis
dans la maison elle-même (sa forme immanente). On dira qu’à ce compte, la
cause de la maison... c’est la maison. Pourquoi non ? En ce sens, écrit Aristote,
«nous entendons par cause l’essence ({\it ousia}) de la chose, ce qui fait qu’elle est
ce qu'elle est » ({\it Métaphysique}, A, 3 ; Tricot traduit : « la substance formelle ou
quiddité »). Aucune des trois autres causes, sans celle-ci, ne rendra jamais
compte du réel. Mais celle-ci fait-elle autre chose que le présupposer ?

\section{Fortune}
%FORTUNE
Le hasard ou la richesse. Comme le remarque Alain, la rencontre
de ces deux sens est bien instructive : « c’est ramener
l’origine des richesses au pur hasard ; ce qui va au fond ; car le travail n’enrichit
pas sans quelque rencontre de fortune. Ainsi demander si la fortune est juste,
c'est demander si la loterie est juste » ({\it Définitions}, art. « Fortune »). Heureusement
qu'il y a les impôts et la sécurité sociale.

\section{Foule}
%FOULE
Une multitude d’individus rassemblés, considérée d’un point de
vue seulement quantitatif. Il y manque la qualité : les corps
s’additionnent ; les esprits, non. De là cette force collective des passions, des
émotions, des pulsions. Le plus bas, presque inévitablement, prend le dessus.
C’est parfois plaisant, parfois exaltant, parfois effrayant, jamais admirable.
Toute foule est dérisoire ou dangereuse.

\section{Franchise}
%FRANCHISE
Une sincérité simple et directe. C’est s’interdire non seulement
le mensonge, mais la dissimulation et le calcul. La franchise
va souvent contre la politesse, parfois contre la compassion. À réserver à
ses amis et aux puissants.

\section{Fraude}
%FRAUDE
Une tromperie intéressée. Voltaire se demande « s’il faut user de
fraudes pieuses avec le peuple », pour le maintenir dans le droit
chemin. Platon répondait que oui. Voltaire s’y refuse : la vraie religion, celle
qui est « sans superstition », n’a pas besoin de ces mensonges ; et « la vertu doit
%— 256 —
être embrassée par amour, non par crainte ». Mais en quoi, alors, a-t-elle besoin
de religion ?

\section{Frivolité}
%FRIVOLITÉ
Le contraire de la gravité, À ne pas confondre avec la légèreté.
La légèreté est sans lourdeur ; la frivolité, sans profondeur.
C’est moins un goût pour les petites choses qu’une incapacité à s'intéresser aux
grandes. On n’est pas frivole parce qu’on aime la bonne chère, les jeux de mots
ou les bals. Mais on l’est, assurément, si l’on est incapable d’apprécier autre
chose. Célimène est frivole, non parce qu’elle est coquette, mais parce qu’elle
ne sait pas aimer.

\section{Frustration}
%FRUSTRATION
Un manque, quand on est incapable de le satisfaire ou d’y
renoncer. Se distingue par là de l’espérance (qui peut être
satisfaite), du deuil (qui renonce à l'être) et du plaisir (qui est la satisfaction
même).
La frustration aboutit presque toujours à exagérer les plaisirs qu’on n’a pas
(obsession), et d’autant plus que d’autres en jouissent (envie). Contre quoi le
plaisir est un premier pas vers la sagesse.

\section{Futur}
%FUTUR
Un autre mot pour l'avenir. Si on veut les distinguer, on peut dire
que le futur désigne davantage une dimension du temps que son
contenu. L'avenir, c’est ce qui viendra ; le futur, le {\it temps} à venir. L'avenir est
fait d'événements, dont nous ignorons la plupart. Le futur n’est fait que de lui-même :
c’est un temps vide, bien sûr imaginaire, que l’avenir viendra remplir.
Mais alors ce ne sera plus du futur, ni de l’avenir : ce sera du présent, qui est le
seul temps réel.
% 257

\section{Gaieté}
%GAIETÉ
J'aime cet {\it e} central, facultatif et muet, comme un rayon de
lumière ou de silence. J’y reconnais quelque chose de la gaieté : sa
transparence, sa fragilité, sa fraîcheur, sa légèreté, sa délicieuse inutilité...
Qu’est-elle ? Une disposition à la joie, qui la rend facile, naturelle, spontanée,
comme déjà là avant même qu’on ait quelque raison de se réjouir. Vertu
d’insouciance, qui serait d’humeur plus que de volonté. Sa force est dans sa
superficialité : les grand malheurs comme les grandes joies sont trop profonds
pour elle ; ils la traversent davantage qu’ils ne l’atteignent. Être gai, c’est avoir
la joie facile ou à fleur de peau. Quel talent plus enviable ? Quel charme plus
séduisant ?

\section{Généalogie}
%GÉNÉALOGIE
L'étude des origines, de la filiation, de la genèse. Se dit surtout
des familles et, depuis Nietzsche, des valeurs : c’est rattacher
un individu à ses ancêtres ou une valeur à un type de vie, pour les faire
valoir ou au contraire pour les dévaluer. Dans {\it La généalogie de la morale}, spécialement,
Nietzsche s'interroge sur « l’origine de nos préjugés moraux» (y
compris sur la valeur de la vérité), afin de se diriger « vers une véritable histoire
de la morale ». Travail d’historien ? Si l’on veut. Mais c’est une histoire normative
et critique, qui met la santé plus haut que le vrai, et qui doit déboucher, en
retour, sur une nouvelle évaluation. « Nous avons besoin d’une {\it critique} des
valeurs morales, écrit Nietzsche, et {\it la valeur de ces valeurs} doit tout d’abord être
mise en question. » Comment ? Par l’étude des « conditions et des milieux qui
leur ont donné naissance, au sein desquels elles se sont développées et déformées
(la morale en tant que conséquence, symptôme, masque, tartuferie,
maladie ou malentendu ; mais aussi la morale en tant que cause, remède, stimulant,
%— 258 —
entrave ou poison). » Philosophie à coups de marteaux. Mais c’est le
marteau d’un archéologue, voire d’un médecin (pour tester les réflexes), avant
d’être celui d’un iconoclaste.

\section{Général}
%GÉNÉRAL
Qui concerne un vaste ensemble (un genre) ou la plupart de ses
éléments. S'oppose à {\it spécifique} (qui concerne un ensemble moins
vaste: une espèce), à {\it particulier} (qui ne vaut que pour une partie d’un
ensemble), enfin et surtout à {\it singulier} (qui ne vaut que pour un seul individu
ou un seul groupe). À ne pas confondre avec {\it universel}, qui concerne tous les
genres ou tous les individus d’un même genre. Par exemple la parole est un
attribut {\it général} de l'humanité (il y a des individus qui ne parlent pas), mais un
trait {\it universel} des peuples. Et la prohibition de l'inceste, qui est une règle universelle,
est généralement respectée.

\section{Génération}
%GÉNÉRATION
Le fait d’engendrer {\it (generare)}, le temps qu’il y faut (de la
naissance d’un individu à celle de ses enfants : environ un
quart de siècle), ou bien l’ensemble des individus qui ont été engendrés à peu
près à la même époque, qui ont donc à peu près le même âge et souvent un certain
nombre d’expériences ou de préoccupations communes ou proches. Ce
dernier sens pose bien sûr la question des limites, ici toujours floues et qui doivent
plus à l’histoire qu’à la génétique. La génération à laquelle on appartient,
c’est moins celle de sa naissance que de sa jeunesse : la génération des soixante-
huitards est née dans les années 40 ou 50, mais ce ne sont pas ces années-là qui
la définissent. Et la « génération Mitterrand », si elle existe, est l’ensemble de
ceux qui ont été jeunes — et non qui sont nés ou qui ont vécu — sous sa
présidence : ni mes amis, s’ils ont mon âge, ni mes enfants (qui sont nés dans
les années 80) n’en font partie. Une génération se forge pendant l’adolescence :
c’est l’ensemble de ceux qui ont été jeunes dans la même période historique. Ils
trimbaleront ce poids ou cette légèreté toute leur vie, comme une patrie com-
mune, comme un accent, comme un terroir, jusqu'à ne plus pouvoir com-
prendre tout à fait ceux qui les suivent, qui ne les comprennent pas davantage.
Ils sont {\it « pays »}, comme on disait autrefois, mais dans le temps plutôt que dans
l’espace : ils sont contemporains, depuis leur jeunesse et par elle, de la même
histoire ou de la même éternité. C’est parfois une chance, parfois un handicap,
plus souvent un mixte des deux. C’est où le hasard se mue en destin ; c’est où
le destin, qu’on le fasse ou qu’on le subisse, reste essentiellement hasardeux. Si
nous étions nés vingt ans plus tôt ou plus tard, que serions-nous ? Nous serions
quelqu’un d’autre, et c’est pourquoi nous ne serions pas.

%— 259 —
\section{Générosité}
%GÉNÉROSITÉ
C’est la vertu du don. On dira qu’on donne aussi par
amour. Sans doute, et c’est pourquoi l’amour est généreux.
Mais toute générosité n’est pas aimante, et même elle n’est une vertu spécifique
que vis-à-vis de ceux que l’on n'aime pas. Qui se jugerait généreux parce qu’il
couvre ses enfants de cadeaux ? Il sait bien que c’est amour, non générosité. La
générosité est la vertu du don, en tant qu’elle excède l'amour dont on est
capable. Vertu classique plutôt que chrétienne. Morale, plutôt qu’éthique. Cela
fixe sa limite en même temps que sa grandeur. L'amour ne se commande pas ;
la générosité, si. L'amour ne dépend pas de nous (c’est nous qui en
dépendons) ; la générosité, si. Pour être généreux, il suffit de le vouloir ; pour
aimer, non. La générosité touche à la liberté, comme l’a vu Descartes : c’est la
conscience d’être libre, expliquait-il, jointe à la résolution d’en bien user ({\it Traité
des passions}, III, 153). Cela suppose qu’on vainque en soi tout ce qui n’est pas
libre : sa propre petitesse, sa propre avidité, sa propre peur, enfin la plupart de
ses passions, jusqu’à mépriser son propre intérêt, pour ne plus s’occuper que du
bien qu’on peut faire à autrui (III, 156). Pas étonnant que la générosité soit si
rare ! Donner c’est perdre, et l’on voudrait garder. C’est prendre un risque, et
l’on a peur. Nul n’est libre qu’à la condition d’abord de se surmonter. Nul n’est
généreux par naissance, malgré l’étymologie, mais par éducation, mais par
choix, mais par volonté. La générosité est vertu du don, mais n’est pas elle-même
un don ; c’est une conquête, c’est une victoire, qui ne va pas sans courage
(les deux mots, chez Corneille, sont à peu près synonymes) et qui peut
aller jusqu’à l’héroïsme. C’est donner ce qu’on possède, plutôt qu’en être possédé.
C’est la liberté à l'égard de soi et de sa peur : le contraire de l’égoïsme et
de la lâcheté.

\section{Génèse}
%GENÈSE
Un devenir primordial, comme en amont de la naissance ou du
réel. C’est moins une origine que ce qui en résulte. Moins un commencement,
que le processus qui y mène ou le constitue. Toute genèse prend
du temps : elle ne peut être qu’historique ou mythique.

\section{Génétique}
%GÉNÉTIQUE
Comme substantif, c’est désormais une science, celle de
l’hérédité. Comme adjectif, c’est plutôt un point de vue, qui
peut certes concerner les gènes ou ce qui en dépend (une maladie génétique),
mais qui porte plus souvent, en philosophie, sur la genèse ou le devenir d’un
être quelconque. Une {\it définition génétique} est celle qui comporte en elle, comme
le voulait Spinoza, l’origine ou la cause prochaine de ce qu’elle définit (par
exemple cette définition du cercle, dans le {\it Traité de la réforme de l'entendement} :
%— 260 —
« Une figure qui est décrite par une ligne quelconque dont une extrémité est
fixe et l’autre mobile »). Et l’{\it épistémologie génétique} est celle qui étudie la
connaissance scientifique, comme le voulait Piaget, dans son développement,
aussi bien individuel (chez l’enfant) que collectif (dans l’histoire des sciences) :
c’est s'interroger sur le processus de la connaissance plutôt que sur son origine
ou son fondement.

\section{Génie}
%GÉNIE
L'abbé Dubos, au début du {\footnotesize XVIII$^\text{e}$} siècle, en donnait la définition suivante :
« On appelle génie l'aptitude qu’un homme a reçue de la
nature pour faire bien et facilement certaines choses que les autres ne sauraient
faire que très mal, même en prenant beaucoup de peine. » En ce sens général,
c’est un synonyme de {\it talent}. L'usage s’est pourtant généralisé de distinguer les
deux notions. D’abord par une différence de degré : le génie est comme un
talent extrême ; le talent, comme un génie limité. Mais aussi par une différence
plus mystérieuse, qui semble de statut ou d’essence. « Le talent fait ce qu'il
veut ; le génie, ce qu’il peut. » Cette formule, dont je ne sais plus l’auteur,
indique au moins une direction. Le génie est une puissance créatrice, qui
excède non seulement la puissance commune (ce que fait déjà le talent) mais
celle même du créateur, au point d'échapper, au moins en partie, à son contrôle
ou à sa volonté. On ne choisit pas d’avoir du génie, ni lequel, ni même toujours
ce qu’on en fait. Le génie est un « don naturel », écrit Kant, autrement dit « une
disposition innée de l'esprit, par laquelle la nature donne à l’art ses règles »
({\it Critique de la faculté de juger}, X, \S 46). Cela ne signifie pas que le génie n’ait
pas besoin d’être cultivé, mais qu'aucune culture ne saurait en donner ou le
remplacer. Mozart, si son père n'avait pas été le pédagogue que l’on sait,
n’aurait peut-être jamais été musicien. Mais aucun pédagogue ne fera un
Mozart d’un enfant sans génie. Le génie est comme un dieu personnel (el était
le sens, en latin, de {\it genius}), qu’on ne choisit pas, mais qui nous choisit. C’est
marquer assez ce qu’il doit au hasard ou à l'injustice. Comment se consoler de
n'être pas Mozart ?

La différence avec le talent ne doit pas être exagérée, ni pourtant, me
semble-t-il, tout à fait abolie. Si on laisse de côté l’exaltation romantique, j'y
vois plus une différence de degré que de nature, de point de vue que d’orientation.
Toutefois quelque chose résiste, dans certaines œuvres, qui interdit de n’y
voir que talent et travail. Voyez Bach ou Michel-Ange, Rembrandt ou Shakespeare,
Newton ou Einstein, Spinoza ou Leibniz. Illusion rétrospective ? Sans
doute, pour une part. Si l’on fait du génie une exception absolue, il est clair
qu’il est toujours mythique, et qu'il convient pour cela de n’en parler qu'à
propos des morts. Tout vivant est médiocre par tel ou tel côté. Seuls le temps
%— 261 
et l’absence donneront à certains cette stature démesurée. Mais enfin l’œuvre
reste, qui maintient ou rétablit les proportions. « Un livre n’est jamais un chef-d'œuvre,
remarquaient finement les Goncourt, il le devient : le génie est le
talent d’un homme mort. » Ces deux-là toutefois sont morts, qui n’ont toujours
que du talent.

\section{Génie (malin —)}
%GÉNIE (MALIN —)
Chez Descartes, c’est un petit dieu ou démon, bien sûr
imaginaire, qui nous tromperait toujours ({\it Méditations}, 1).
Le but de cette fiction est d’exagérer le doute (puisqu’on va considérer comme
faux ce qui n’est qu’incertain), afin de nous désaccoutumer de nos préjugés, de
nos anciennes opinions, enfin de toute croyance. C’est tordre le bâton dans
l’autre sens, pour le redresser. Le but est d’atteindre une certitude absolue —
celle qui résisterait à l'hypothèse du malin génie. Ce sera le {\it cogito}, qui n’est
peut-être qu’un génie un peu plus {\it malin} que les autres.

\section{Génocide}
%GÉNOCIDE
L’extermination d’un peuple. Ce n’est pas seulement un crime
de masse : c’est un crime contre l'humanité, en tant qu’elle est
une et plurielle.

\section{Genre}
%GENRE
Un vaste ensemble, qui ne se définit pourtant que par rapport à
d’autres : plus vaste que l’espèce (un genre en comporte plusieurs),
plus réduit que l’ordre (au sens biologique du terme : le genre {\it Homo}, dont
{\it Homo sapiens} est la seule espèce survivante, fait partie de l’ordre des primates).
Notion par nature relative. Ce qui est {\it genre} pour ses espèces peut être
{\it espèce} pour un autre genre, qui l’inclut. Par exemple le quadrilatère, qui est
un genre pour ses différentes espèces (trapèze, losange, rectangle...) est lui-même
une espèce du genre polygone, qui peut à son tour être considéré
comme une espèce du genre figure géométrique. Tout dépend de l'échelle et
du point de vue adoptés. C’est pourquoi on parle de {\it genre prochain}, depuis
Aristote, pour désigner l’ensemble immédiatement supérieur (en extension) à
celui qu’on veut définir (en compréhension) : il suffira pour cela d’indiquer
la ou les différences spécifiques de ce dernier. Par exemple un quadrilatère est
un polygone (genre prochain) à quatre côtés (différence spécifique), comme
un trapèze est un quadrilatère (genre prochain) dont deux des côtés sont
parallèles (différence spécifique). C’est une façon d’ordonner le réel, pour
pouvoir le dire.

%— 262 
\section{Géométrie (esprit de —)}
%GÉOMÉTRIE (ESPRIT DE —)
L’art de raisonner juste, explique Pascal, sur
des principes « palpables mais éloignés de
l'usage commun » : une fois qu’on les voit, « il faudrait avoir l'esprit tout à fait
faux pour mal raisonner sur des principes si gros qu’il est presque impossible
qu’ils échappent » ({\it Pensées}, 512-1). S’oppose à l'esprit de finesse (voir ce mot).

\section{Gloire}
%GLOIRE 
Ce serait la confirmation de notre valeur par le grand nombre de
ceux qui en ont moins et le reconnaissent : comble de lélitisme (la
gloire ne va qu’à quelques-uns) en même temps que de la démagogie (elle
dépend de tous), et qui trouve ses limites dans cette contradiction. Qu'elle
nous tente, c’est ce que Descartes, Pascal et Spinoza ont assez expliqué : c’est
l'amour de soi, mais comblé par les louanges dont on imagine être l’objet. C’est
mettre l'humanité plus haut que tout, comme il convient, et soi plus haut que
tout autre. C’est un humanisme narcissique.

Grandeur de l’homme : de pouvoir admirer. Misère de l’homme : d’avoir
besoin de l'être. « La plus grande bassesse de l’homme est la recherche de la
gloire, écrit Pascal, mais c’est cela même qui est la plus grande marque de son
excellence ; car, quelque possession qu’il ait sur la terre, quelque santé et commodité
essentielle qu’il ait, il n’est pas satisfait s’il n’est dans l'estime des
hommes » ({\it Pensées}, 470-404). La gloire, tant qu’on y rêve, fait comme un
simulacre de salut. Son seul avantage, pour qui l’atteindrait de son vivant, serait
qu’elle nous guérirait de la désirer, peut-être. On dit aussi qu’elle sauve de la
mort ; mais c’est l’inverse qui est vrai.

\section{Gnose}
%GNOSE
Doctrine religieuse des premiers siècles de l’ère chrétienne, peut-être
d'inspiration platonicienne (voire manichéenne : ce monde est
le mal, tout le bien vient d’ailleurs), qui voudrait assurer le salut par la connaissance
{\it (gnosis)} de Dieu, telle qu’elle est transmise aux initiés par une tradition
primordiale et secrète. Le gnosticisme est un ésotérisme. C’est une superstition
de la connaissance.

L'Église considère le gnosticisme comme une hérésie. De fait, les commentateurs
y voient souvent une contamination du christianisme par lhellénisme.
Ce peut être aussi l'inverse, et c’est en quoi Simone Weil, à bien des égards,
relève de ce courant. Le gnosticisme, qu’on retrouve dans d’autres religions et
à d’autres époques, se reconnaît presque toujours à la haine du monde, du
corps ou de soi. Le gnostique ne veut sauver que son esprit, et par l'esprit seul.
D'où le paradoxe de la gnose, qui est d’être une sotériologie pessimiste. Le
monde est une prison ; le gnostique ne trouve de salut que dans la fuite.

%— 263 —
\section{Gnoséologie}
%GNOSÉOLOGIE
L'étude ou la philosophie de la connaissance {\it (gnôsis)}. Plus
abstraite que l’épistémologie (qui porte moins sur la
connaissance en général que sur les sciences en particulier). Le mot vaut surtout
par l'adjectif {\it gnoséologique}, qui est commode et n’a guère de synonyme. Le
substantif, en français, reste rare : les philosophes parleront plus volontiers de
théorie de la connaissance.

\section{Goût}
%GOÛT
C'est la faculté de juger du beau et du laid, du bon et du mauvais,
comme un plaisir qui serait critère de vérité. Le goût touche au
corps, par la sensation, et à l'esprit, par la culture. Il s’'éduque ; il ne se crée pas.
Le goût prétend à l’universel (jai le sentiment que tout le monde devrait trouver
beau, en droit, ce que je juge être tel), tout en restant subjectif (je n’ai aucun moyen
d'obtenir, en fait, l'accord de tous). C’est ce qui le voue presque inévitablement au
conflit et à la polémique. Il ne s’agit pas de tout aimer, de tout admirer, encore
moins de faire semblant. « Le vrai goût, disait Auguste Comte, suppose toujours un
vif dégoût. » Et Kant, plus profondément : « Une obligation de jouir est une évidente
absurdité. » Le goût ne se commande pas, puisque c’est lui qui commande.

Ainsi le plaisir a toujours raison, mais ne prouve rien. On peut {\it discuter} du
goût (prétendre à l’assentiment nécessaire d’autrui), observe Kant, non {\it disputer}
à son sujet (décider par des preuves). C’est ce qu’on oublie presque toujours, et
qui nous voue, en un autre sens, aux disputes.

\section{Gouvernment}
%GOUVERNEMENT
Ce n'est pas le souverain (qui fait la loi), mais un
« corps intermédiaire », comme disait Rousseau, « chargé
de l'exécution des lois et du maintien de la liberté » ({\it Contrat social}, III, 1). C’est
donc le pouvoir exécutif, ou plutôt son sommet. Il ne gouverne légitimement
qu’à la condition d’obéir. « Le gouvernement reçoit du souverain, écrit excellemment
Rousseau, les ordres qu’il donne au peuple. » Dans une démocratie,
cela signifie qu’il doit être soumis, d’une manière ou d’une autre, au suffrage
universel et au contrôle du parlement.

\section{Grâce}
%GRÂCE
Un don sans raison, sans condition, sans mérite. Le réel en est une,
tant qu’il nous épargne ou nous comble, et la seule.

\section{Grandeur}
%GRANDEUR
Une quantité quelconque, mais perçue positivement, voire
emphatiquement (c’est le con\-traire de la petitesse). C’est
%— 264 —
pourquoi on se sert de l’expression {\it grandeur d'âme} pour traduire la {\it megalopsuchia}
où {\it magnanimitas} des Anciens. C’est que la quantité semble ici valoir
comme qualité. « Il n’y a point d'âme vile, disait Alain, mais seulement un
manque d'âme. »

\section{Gratitude}
%GRATITUDE
C'est le souvenir reconnaissant de ce qui a eu lieu : souvenir
d’un bonheur ou d’une grâce, et bonheur lui-même, et grâce
renouvelée. C’est par quoi c’est une vertu : parce qu’elle se réjouit de ce qu’elle
doit, quand l’amour-propre préférerait l'oublier.

La gratitude porte sur ce qui fut, en tant que ce qui fut demeure. C’est la
joie de la mémoire, et le contraire de la nostalgie : il s’agit d’aimer le passé, non
en tant qu’il manque (nostalgie), mais dans sa vérité toujours présente, qui ne
manque jamais. C’est Le temps retrouvé, et Proust a bien montré ce qui s’y joue
de joie ou d’éternité. La mémoire fait comme un port, dans la tempête de vivre.
Au contraire, disait Épicure, « la vie de l’insensé est ingrate et inquiète : elle se
porte tout entière vers l'avenir ». La gratitude est le contraire de la nostalgie, et
l'inverse de l’espérance.

\section{Gratuit}
%GRATUIT
Ce n’est pas ce qui n’a pas de prix (gratuité n’est pas dignité),
mais ce pour quoi on n’exige pas d’être payé ou récompensé — ce
qui est disponible sans échange ou offert sans contrepartie. On parle aussi
d’{\it acte gratuit}, depuis Gide, pour désigner un acte sans motif. Ce serait bien sûr
se tromper que d’y voir un acte libre : à supposer qu’il soit en effet sans motif,
il n’est pas pour autant sans causes (car alors il n’existerait pas). Et il n’est pas
besoin non plus de travailler {\it gratuitement} pour travailler {\it librement}... Ainsi le
gratuit n’est pas le libre, mais le désintéressé : marque, selon les cas, d’indifférence,
de surabondance, de générosité, ou de folie.

\section{Gravité}
%GRAVITÉ
Je n’aime pas trop qu’on en dise du mal. Tout le monde ne peut
pas être Mozart, et la légèreté n’y a jamais suffi. Au reste, la gravité,
malgré l’étymologie, n’est pas la lourdeur. Elle est plutôt une sensibilité à
ce qui pèse, au poignant des choses et de la vie. « Le vice, la mort, la pauvreté,
les maladies, sont sujets graves, écrit Montaigne, et qui grèvent » ({\it Essais}, III, 5,
p. 841). C’est comme un tragique quotidien, mais sans emphase, sans grandiloquence,
sans {\it affolement}, comme dit encore Montaigne — un tragique qui ne
se prend pas au tragique, mais qui ne parvient pas non plus à en rire. On ne la
%— 265 —
confondra pas avec l’esprit de sérieux : son contraire n’est pas l'humour, mais
la frivolité.

\section{Groupe}
%GROUPE
Un ensemble d'individus en interaction, de telle sorte qu'il y a
plus dans le groupe que la simple addition des comportements
individuels. Notion par nature relative, et même floue, qu’on ne peut guère
préciser que par différence. C’est à la fois moins (d’un point de vue quantitatif)
et plus (d’un point de vue qualitatif) qu’une foule : c’est comme une foule
limitée, unifiée, presque toujours hiérarchisée. Par exemple dans un stade, lors
d’un match de football : parmi la {\it foule} des spectateurs, il y a des {\it groupes} de supporteurs,
et deux groupes de joueurs (les équipes).

\section{Guerre}
%GUERRE
«La guerre, écrivait Hobbes, ne consiste pas dans un combat
effectif, mais dans une disposition avérée, allant dans ce sens,
aussi longtemps qu’il n’y a pas d’assurance du contraire. Tout autre temps se
nomme paix » ({\it Léviathan}, I, 13). Cela, qui distingue la guerre de la bataille,
suggère assez que la guerre, entre les États, est la disposition première : la guerre
est donnée ; la paix, il faut la {\it faire}. C’est ce qui donne raison aux pacifiques,
sans donner tort aux militaires.

On remarquera que le but d’une guerre est ordinairement la victoire, qui
est une paix avantageuse. Que le droit y trouve aussi son compte n’est jamais
garanti, mais peut seul la justifier. Une guerre juste ? Elle peut l’être par ses
buts, jamais totalement par ses moyens. Le mieux, presque toujours, est de
“éviter : le rapport violent des forces (la guerre) n’est légitime que lorsque leur
rapport non violent (la politique) serait suicidaire ou indigne.
% 266

\section{Habitude}
%HABITUDE
La facilité qui naît de la répétition. Un acte qu’on accomplit
souvent devient ainsi presque instinctif, remarquait déjà Aristote
({\it Rhétorique}, I, 11), et c’est pourquoi on dit souvent que l’habitude est une
seconde nature : c’est comme une nature acquise, qui viendrait corriger la première
ou prendre sa place. Reste à savoir, observe Pascal, si cette nature elle-même
n’est pas une première habitude ({\it Pensées}, 126-93).

L'habitude, en diminuant la difficulté, rend la conscience moins nécessaire.
Elle peut même s’en passer tout à fait, abandonnant le corps, pour ainsi dire, à
lui-même. C’est une « spontanéité irréfléchie » (Ravaisson, {\it De l'habitude}), qui
permet de penser à autre chose. Ainsi le virtuose, libéré des notes, n’a plus à
s’occuper que de la musique.

Cabanis et Destutt de Tracy ont souligné que les effets de l’habitude sont
contrastés, qui peuvent aussi bien développer une faculté que l’engourdir : une
oreille exercée entendra ce qu’une autre n’entendrait pas, comme un nez exercé
percevra des arômes pour d’autres imperceptibles ; mais on finit par ne plus
entendre un bruit trop habituel, par ne plus sentir une odeur trop constante.
« L’accoutumance accroît toutes les habitudes actives, remarquait déjà Hume,
et affaiblit les habitudes passives » ({\it Traité...}, II, 3, 5 ; il semble que l’idée
vienne de Joseph Butler). C’est ce qui justifiera la distinction, chez Maine de
Biran, entre les impressions passives (les sensations : voir, entendre, sentir, toucher...)
et les impressions actives (les perceptions : regarder, écouter, humer,
palper...). L'habitude affaiblit ou obscurcit les premières, avive ou précise les
secondes. Elle fait qu’on entend moins (par exemple le tic-tac du réveil ou le
bruit d’une autoroute) et qu’on écoute mieux (par exemple une musique ou un
souffle). Aussi peut-elle servir de réactif pour dissocier en nous ce qui est passif
(que Maine de Biran rattachera au corps) de ce qui est actif (qu’il rattachera au
%— 267 —
moi ou à la volonté: {\it De l'influence de l'habitude}, chap. I et II). C’est où le
thème de l’habitude, d’abord empiriste, chez Condillac et Hume, devient spiritualiste.
Cela continuera jusqu’à Ravaisson : « La réceptivité diminue, la
spontanéité augmente ; telle est la loi générale de l'habitude » (Ravaisson,
{\it op. cit.}, II). C’est l’esprit qui redescend dans la nature, ou le devenir esprit de la
spontanéité naturelle. Dualisme ? Ce n’est pas si sûr. Spiritualisme ? Ce n’est
qu’une possibilité parmi d’autres. Il se pourrait aussi bien, et peut-être mieux,
que l’âme ne soit qu’une habitude du corps.

\section{{\it Habitus}}
%{\it HABITUS}
Une manière d’être et d’agir (une disposition), mais acquise et
durable. Le mot, réactualisé par Bourdieu, sert surtout aux
sociologues. Un {\it habitus}, en ce sens, c’est comme une idéologie incarnée et
génératrice de pratiques : c’est notre façon d’être nous-mêmes et d’agir comme
nous agissons, mais en tant qu’elle résulte de notre insertion dans une société
donnée, dont nous incorporons inconsciemment les structures, les clivages, les
valeurs, les hiérarchies.. Par quoi chacun fait librement, ou en tout cas volontairement,
ce qu’il est socialement déterminé à vouloir.

\section{Haine}
%HAINE
« La seule chose universelle, me dit un jour Bernard Kouchner,
c’est la haine ! » Il revenait d’une de ses expéditions humanitaires,
au fin fond de l’horreur et du monde. La seule ? Je n’irais pas jusque-là. Mais
que la haine soit universelle, en effet, partout présente, partout agissante, c’est
ce que trop de massacres ne cessent de nous confirmer. Reste à la penser, pour
essayer d’en sortir ou de s’en protéger. Qu'est-ce que la haine ? « Une tristesse,
répondait Spinoza, qu’accompagne l’idée d’une cause extérieure » ({\it Éthique}, III,
13, scolie, et déf. 7 des Affects). Haïr, c’est {\it s'attrister de}. Or c’est la joie qui est
bonne : toute haine, par définition, est mauvaise. C’est aussi ce qui la rend
mortifère. Celui qui hait, ajoute Spinoza, « s'efforce d’écarter et de détruire la
chose qu’il a en haine » — parce qu’il préfère la joie, comme tout le monde,
autrement dit par amour. Mais c’est un amour malheureux, qui en veut à
l’autre de son propre échec. Ainsi toute haine, même justifiée, est injuste.

\section{Hallucination}
%HALLUCINATION
C'est percevoir ce qui n’est pas. Mais comme nous
n'avons aucun moyen de savoir ce qui est qu’à la
condition de le percevoir, directement ou indirectement, nous n’avons non
plus aucun moyen de distinguer absolument la perception de l’hallucination,
sinon en confrontant nos perceptions à celles d’autrui ou au souvenir de nos
%— 268 —
perceptions passées. Encore cela ne nous dit-il pas si c’est l’hallucination qui est
une perception pathologique, ou si c’est la perception qui est une hallucination
collective et durable... On ne peut en décider par des preuves, et cela n’a pas
tant d'importance. Ce que tout le monde perçoit fait partie du réel commun,
quand bien même il n’aurait d’autre réalité que cette perception (Berkeley). Ce
que je perçois seul, quand les autres devraient le percevoir, est réputé
hallucinatoire : c’est un réel privé, mais qu’on ne sait pas tel, comme un monde
intérieur qu’on prendrait abusivement pour l’autre. « Il y a pour les éveillés un
monde unique et commun, disait Héraclite, mais chacun des endormis se
détourne dans un monde particulier. » L’hallucination est comme un rêve
éveillé ; le rêve, comme une hallucination endormie.

\section{Harmonie}
%HARMONIE
C'est un accord heureux ou agréable, entre plusieurs éléments
simultanés mais indépendants les uns des autres : par exemple
entre plusieurs sons (l'harmonie s’oppose alors à la mélodie, qui unit des sons
successifs), entre plusieurs couleurs, entre plusieurs individus. Leibniz parlait
d'{\it harmonie préétablie} entre l'âme et le corps ; c’est qu’il refusait que ces deux
substances puissent agir l’une sur l’autre et constatait pourtant, comme tout le
monde, leur singulier accord (je veux lever le bras : mon bras se lève). L'âme
et le corps seraient donc comme deux horloges, l’image est de Leibniz, tellement
bien fabriquées et réglées au départ qu’elles seront toujours d’accord par
la suite, sans qu’on ait besoin pour cela de supposer entre elles quelque relation
causale que ce soit. Cette théorie, aussi difficile à accepter qu’à réfuter, explique
que l'expression ait souvent pris, par la suite, un sens péjoratif : une harmonie
préétablie serait une espèce de miracle originel, trop étonnant pour qu’on
puisse y croire. C’est reconnaître que l’harmonie n’est jamais le plus probable,
ce qui explique qu’elle soit rarement donnée au départ. Elle résulte d’un travail
ou d’une adaptation, plus souvent que de la chance.

\section{Hasard}
%HASARD
Ce n’est ni l’indétermination ni l'absence de cause. Quoi de plus
déterminé qu’un dé qui roule sur une table ? Le six sort ? C’est là
un effet, qui résulte de causes très nombreuses (le geste de la main, l'attraction
terrestre, la résistance de l'air, la forme du dé, sa masse, son angle de contact
avec la nappe, ses frottements contre elle, ses rebonds, son inertie...). Si l'on
juge pourtant légitimement que le {\it six} est sorti {\it par hasard}, c’est que ces causes
sont trop nombreuses et trop indépendantes de notre volonté pour qu’on
puisse, lorsqu'on jette le dé, choisir ou prévoir le résultat qu’on obtiendra. Ainsi
le hasard est une détermination imprévisible et involontaire, qui résulte de la
%— 269 —
rencontre de plusieurs séries causales indépendantes les unes des autres, comme
disait Cournot, rencontre qui échappe pour cela à tout contrôle comme à toute
intention. Ce n’est pas le contraire du déterminisme : c’est le contraire de la
liberté, de la finalité ou de la providence.

Un autre exemple ? On peut reprendre celui de Spinoza, dans l’Appendice
de la première partie de l'{\it Éthique}. Une tuile tombe d’un toit. Il y a à cela des
causes (le poids de la tuile, la pente du toit, le vent qui soufflait, un clou rongé
par la rouille, qui finit par céder), dont chacune s'explique à son tour par une
ou plusieurs autres, et ainsi à l'infini. Vous étiez, à ce moment précis, sur le
trottoir, juste à la verticale du toit. Cela s'explique aussi, ou peut s'expliquer,
par un certain nombre de causes : vous alliez à un rendez-vous, vous aviez
choisi l'itinéraire le plus simple, vous pensiez que la marche à pied vous ferait
du bien. Ni la chute de la tuile ni votre présence sur le trottoir ne sont donc
sans causes. Mais les deux séries causales (celle qui fait tomber la tuile, celle qui
vous amène où vous êtes), outre leur complexité propre, qui suffirait à les
rendre hasardeuses, sont indépendantes l’une de l’autre : ce n’est pas parce que
la tuile tombe que vous êtes là, ni parce que vous êtes là qu’elle tombe. Si elle
vous brise le crâne, vous serez donc bien mort par hasard : non parce qu’il y
aurait (à une exception au principe de causalité, mais parce que celui-ci s’est
exercé de façon irréductiblement multiple, imprévisible et aveugle. Ou bien il
faut imaginer un Dieu qui aurait voulu ou prévu la rencontre de la tuile et de
votre crâne. La providence est un anti-hasard, et le hasard une anti-religion.

Le hasard se calcule, mais dans sa masse plutôt que dans son détail. C’est ce
qui permet aux assureurs de mesurer les risques que nous prenons, et qu’ils
prennent : un accident de voiture, aussi imprévisible qu’il puisse être, fait partie
d’une série (le nombre d’accidents dans une période donnée) qui peut se prévoir
à peu près. C’est vrai aussi pour les jeux de hasard. S’il est impossible, sauf
trucage, de prévoir le résultat d’un seul coup de dé, il est facile de calculer la
répartition statistique de coups très nombreux : chacune des six possibilités se
vérifiera, si vous jouez assez longtemps, environ une fois sur six, et s’approchera
d’autant plus de cette moyenne que la série sera plus longue. C’est pourquoi la
chance ne dure pas toujours, ni la malchance, du moins dans les phénomènes
qui ne dépendent que du hasard. Simplement la vie ne dure pas assez longtemps,
et est soumise à des causes trop lourdes et trop constantes, pour que le
hasard vienne toujours corriger l'injustice.

Toute vie n’en est pas moins hasardeuse, dans son détail comme dans son
principe. La naissance de chacun d’entre nous, quelques années avant notre
conception, était d’une probabilité extrêmement faible ; comme c’est vrai aussi
des naissances de nos parents, de nos grands-parents, etc., qui conditionnent la
nôtre, il en résulte que notre existence, il y a quelques siècles, était d’une probabilité
%— 270 —
quasi nulle, comme celle, si l’on prend assez de recul, de tout événement
contingent. C’est en quoi tout réel, aussi banal qu’il soit, a quelque chose
de rétrospectivement surprenant, qui tient au fait qu’il était non seulement
imprévisible à l’avance mais hautement improbable : c’est l’exception du possible.
L'univers fait une espèce de loterie, dont le présent serait le gros lot.
D’aucuns s’étonnent que ce soit justement ce numéro-là qui soit sorti, alors
que c'était tellement improbable... Mais qu'aucun numéro ne sorte, une fois la
loterie lancée, l'était bien davantage.

\section{Hédonisme}
%HÉDONISME
Toute doctrine qui fait du plaisir ({\it hèdonè}) le souverain bien
ou le principe de la morale : ainsi chez Aristippe, Épicure
(quoique son hédonisme se double d’un eudémonisme), ou aujourd’hui chez
Michel Onfray. Ce n’est pas forcément un égoïsme, puisqu'on peut prendre en
compte le plaisir des autres ; ni un matérialisme, puisqu'il peut exister des plaisirs
spirituels. C’est même le point faible de l’hédonisme : la doctrine n’est
acceptable qu’à la condition de donner au mot {\it plaisir} une extension tellement
vaste qu’il ne veut plus dire grand-chose. Je veux bien que celui qui meurt sous
la torture, plutôt que de dénoncer ses camarades, agisse {\it pour le plaisir} (ou pour
éviter une souffrance plus grande : celle d’avoir trahi, celle de ses camarades qui
seraient autrement torturés à leur tour, celle de la défaite....). Mais alors l’hédonisme
n’est qu’une espèce de théorie passe-partout, qui perd sa vertu discriminante.
Si tout le monde en relève, à quoi bon s’en réclamer ?

La maxime de l’hédonisme est bien énoncée par Chamfort : « Jouis et fais
jouir, sans faire de mal ni à toi ni à personne, voilà, je crois, toute la morale »
({\it Maximes}..., 319). Formule sympathique, et même vraie pour une bonne part,
mais courte. C’est ériger le principe de plaisir (qui ne se veut que descriptif) en
éthique (qui serait normative). Mais comment ce principe, dans son universalité
simple, pourrait-il suffire ? Reste à savoir quels plaisirs, et pour qui, peuvent
justifier quelles souffrances. Il faut donc choisir {\it entre les plaisirs}, comme disait
Épicure, et il est douteux que le plaisir, moralement, y suffise. Combien de
salauds jouisseurs ? Combien de souffrances admirables ? Et un mensonge qui
ne fait de mal à personne, fût-il même agréable pour tout le monde (vous vous
vantez d’un exploit que vous n’avez jamais accompli : vos auditeurs sont
presque aussi contents que vous), en quoi en est-il moins méprisable ? On me
répondra que le mépris est une espèce de déplaisir, et que mon exemple
confirme par là l’hédonisme qu’il prétend réfuter.. J'y consens, mais cela
redouble plutôt mes réticences. L’hédonisme est aussi irréfutable qu’insatisfaisant :
il n'échappe au paradoxe que pour tomber dans la tautologie.

%— 271 —
\section{Héraclitéisme}
%HÉRACLITÉISME
La pensée d'Héraclite, et toute pensée qui en reprend la
thèse centrale : qu’il n’y a pas d'êtres immuables, que
tout change, que tout coule {\it (« Panta rhei »)}, que tout est devenir. Ainsi peut-on
parler de l’héraclitéisme de Montaigne. « Je ne peins pas l'être, disait-il, je
peins le passage. » C’est le seul être qui nous soit accessible. « Le monde n’est
qu'une branloire pérenne. Toutes choses y branlent sans cesse : la terre, les
rochers du Caucase, les pyramides d'Égypte, et du branle public et du leur. La
constance même n'est autre chose qu’un branle plus languissant » ({\it Essais}, III,
2). C’est le contraire de l’éléatisme, ou sa vérité {\it sub specie temporis}.

\section{Herméneute}
%HERMÉNEUTE
Au sens général : celui qui interprète, c’est-à-dire qui
cherche le sens de quelque chose (d’un signe, d’un discours,
d’un événement). En un sens plus restreint, j’appelle {\it herméneute} toute
personne qui prend le sens absolument au sérieux : qui veut en rendre raison
par lui-même ou par un autre sens, au lieu de l’expliquer par ses causes, qui ne
signifient rien. C’est supposer un sens ultime ou infini (un sens du sens), qui
serait la vérité même. Mais si la vérité voulait dire quelque chose, elle serait
Dieu. Toute herméneutique, en ce sens restreint, est religieuse, ou tend à le
devenir : ce n’est qu’une superstition du sens.

\section{Héroïsme}
%HÉROÏSME
C’est un courage extrême et désintéressé, face à tous les maux
réels ou possibles.
Ce courage-là ne résiste pas seulement à la peur, mais aussi à la souffrance,
à la fatigue, à l'abattement, au dégoût, à la tentation. Vertu d’exception, pour
des individus d’exception. Nul n’est tenu d’être un héros, et c’est ce qui rend
les héros admirables.

\section{Hétéronomie}
%HÉTÉRONOMIE
C’est être soumis à une autre loi que la sienne propre, et
le contraire par là de l’autonomie (voir ce mot). S’applique
spécialement, chez Kant, à toute détermination de la volonté par autre
chose que la loi qu’elle se fixe à elle-même (la loi morale), par exemple par tel
ou tel objet de la faculté de désirer (le plaisir, le bonheur...) ou par tel ou tel
commandement extérieur, fût-il par ailleurs légitime. Obéir à ses penchants,
c’est être esclave. Obéir à l’État ou à Dieu ? On ne le peut de façon autonome
qu’à la condition de leur obéir par devoir, non par crainte ou par espérance. On
n'a le droit d’obéir (hétéronomie) qu’à la condition de se gouverner (autonomie).

%— 272 —
\section{Heureux}
%HEUREUX
Être heureux, étymologiquement, c’est avoir de la chance. Cela
en dit long sur le bonheur : non que la chance y suffise, mais
parce que aucun bonheur, sans elle, n’est possible. Tu es heureux ? C’est
d’abord que rien ne t’en empêche, qui serait plus fort que toi. Et comment
serais-tu plus fort que tout ?

Être heureux, c’est aussi avoir le sentiment de l’être. C’est ce que Marcel
Conche appelle le {\it cogito eudémonique} de Montaigne : « Je pense être heureux,
donc je le suis. » Cela laisse une marge à l’évaluation, au travail sur soi, à la philosophie.
Plutôt que de regretter toujours ce qui n’est pas (« Quel malheur de
n'être pas heureux ! »), apprends plutôt, quand c’est possible, à jouir de ce qui
est (« Quel bonheur de n être pas malheureux ! »).

C’est aussi une question de tempérament. Heureux ceux qui sont doués
pour le bonheur !

\section{Heuristique}
%HEURISTIQUE
Qui concerne la recherche ou la découverte ({\it heuriskein} :
trouver, de {\it heuris}, le flair). Par exemple une hypothèse
heuristique est une hypothèse qui prétend moins résoudre un problème que
permettre de le poser autrement, et mieux : elle ne propose pas une solution,
elle aide à la chercher.

\section{{\it Hic et nunc}}
%{\it HIC ET NUNC}
Ici et maintenant, en latin. C’est la situation de tout être, de
tout sujet, de tout événement : son ancrage singulier dans
l’'universel devenir. Même la mémoire et l'imagination n’y échappent pas (se
souvenir d’un passé, imaginer un ailleurs ou un avenir, c’est toujours s’en souvenir
ou les imaginer {\it ici et maintenant}). Nos utopies sont datées, autant que
nos émotions, et vieillissent plus mal.

\section{Hiérarchie}
%HIÉRARCHIE
C’est un classement normatif, qui instaure des liens de
domination, de dépendance ou de subordination. La norme
est le plus souvent de puissance ou d’argent. C’est ce qu’on appelle la hiérarchie
sociale, qui rend l’idée même de hiérarchie suspecte. Si tous les hommes sont
égaux en droit et en dignité, comment pourrait-on les classer par ordre
d'importance ou de valeur ? Mais c’est que la hiérarchie, quand elle est légitime,
porte moins sur les êtres que sur les fonctions ou les œuvres. Ainsi dans
l'État ou dans un parti, dans une entreprise ou une Église, en art ou en sport.
Que tous les hommes soient égaux en droit et en dignité, cela ne signifie pas
que tous les pouvoirs le soient, ni toutes les responsabilités, ni tous les talents.

%— 273 —
L'erreur, à quoi pousse l’étymologie, est d’y voir du sacré {\it (hiéros)}, quand il ne
s’agit que d'organisation ou d'efficacité. Le protocole, qui met en scène cette
hiérarchie, le montre bien : ce n’est qu’une apparence réglée, qui ne dit rien sur
la valeur des individus mais qui manifeste quelque chose de l'institution ou des
pouvoirs. Pascal, comme toujours, va au plus court : « M. N... est un plus
grand géomètre que moi ; en cette qualité il veut passer devant moi. Je lui dirai
qu'il n’y entend rien. La géométrie est une grandeur naturelle ; elle demande
une préférence d’estime ; mais les hommes n’y ont attaché aucune préférence
extérieure. Je passerai donc devant lui, et l’estimerai plus que moi en qualité de
géomètre » ({\it Trois discours...}, 2). La hiérarchie des pouvoirs n’est pas celle des
talents, comme celle de la naissance n’est pas celle des vertus. Pascal encore :
« Il n’est pas nécessaire, parce que vous êtes duc, que je vous estime ; mais il est
nécessaire que je vous salue. » Il n’y a pas de hiérarchie absolue. Cela, loin de
les annuler toutes, justifie leur pluralité. C’est vrai spécialement en démocratie :
que tous les citoyens soient égaux, cela ne dit rien sur leurs fonctions ou leurs
mérites respectifs, ni ne dispense d’admirer les uns et d’obéir à d’autres. Ainsi
l’idée de hiérarchie revient toujours, mais au pluriel et sans qu’on puisse jamais
en absolutiser une.

\section{Histoire}
%HISTOIRE
L'ensemble non seulement de tout ce qui arrive (le monde),
mais aussi de tout ce qui est arrivé et arrivera : la totalité diachronique
des événements. C’est en ce sens qu’on parle d’une histoire de l’univers,
qui serait la seule histoire universelle. En pratique, le mot a pourtant rarement
une extension aussi large : sauf précision particulière, il ne désigne que le
passé humain, et la connaissance de ce passé. De là deux sens différents, que le
français ne distingue pas : il y a l’histoire réelle ({\it Geschichte} en allemand : ce
qu'on appelait en latin les {\it res gestas}, les faits accomplis, le passé tel qu’il fut,
l’histoire des hommes historiques) et l’histoire comme discipline (qu’on appelle
parfois {\it Historie} en allemand : {\it l’historia rerum gestarum}, la connaissance du
passé, l’histoire des historiens). On ne connaît la première que par la seconde ;
mais la seconde n’existe que par et dans la première.

L’histoire a-t-elle un sens ? La science historique a celui qu’on lui prête ou
qu'on y trouve : faire de l’histoire, cela peut viser un certain but ou signifier
quelque chose, qui variera en fonction des historiens. Mais l’histoire réelle ?
Quel sens pourrait-elle avoir ? Qu’on le prenne comme but ou comme signification,
ce sens ne pourrait être qu'autre chose que l’histoire (comment aller
vers soi ? comment se signifier soi ?), qui serait son message ou sa fin. Mais
l’idée d’une fin de l’histoire est contradictoire ou absurde : elle ne peut exister
ni si l’histoire continue (car sa fin alors n’en serait pas une) ni si elle s’arrête (car
%— 274 —
sa fin alors ne serait pas sienne, et ne ferait pas sens). Quant à penser une signification
de l’histoire, c’est lui supposer un sujet, qui veuille dire, à travers elle,
quelque chose. Mais si ce sujet est {\it dans} l'histoire, comment ce qu’il veut dire
serait-il le sens de l’histoire, puisqu'il en fait partie ? Et comment, s’il est au-dehors,
pourrait-il s'y exprimer ? On dira qu’il en est ainsi de tout sens. Mais
non, puisque le sens d’une phrase, par exemple, n’est ni son point final ni une
partie de cette phrase : son sens est un dehors, qui est visé de l’extérieur par
celui qui l’énonce (lequel ne fait pas partie de la phrase). Mais hors de l’histoire,
quoi, et pour quel locuteur ? « Le sens du monde, disait Wittgenstein, doit se
trouver hors du monde. » Le sens de l’histoire, pareillement, ne peut exister
qu’en dehors d’elle. C’est ce qu’on appelle Dieu, lorsqu'on y croit : ce n’est plus
histoire mais théodicée. Pour les autres, ceux qui n’adorent aucun Dieu, on
peut dire de l’histoire ce que Shakespeare disait de toute vie : « C’est une histoire
pleine de bruit et de fureur, racontée par un idiot, et qui ne signifie rien. »
Cela ne nous empêche pas, en elle, de viser tel ou tel but, ni de tenir tel ou tel
discours. Mais nous interdit de croire que c’est elle qui parle ou vise à travers
nous. Que signifie la Guerre de 1914 ? Quel but visait-elle ? Aucun, bien sûr,
puisque les individus qui la firent poursuivaient eux-mêmes des buts différents,
qui donnaient à leur existence, bien souvent, des significations opposées. Il en
va de même pour tout événement. La Révolution française ? La Révolution
russe ? Ceux qui les firent visaient sans doute un but ou un sens, mais pas
davantage que ceux qui les combattirent. C’est ce qu'avait vu Engels :
« L'histoire se fait de telle façon que le résultat final se dégage toujours des
conflits d’un grand nombre de volontés individuelles, dont chacune à son tour
est faite telle qu’elle est par une foule de conditions particulières d’existence ; il
y a donc là d'innombrables forces qui se contrecarrent mutuellement, un
groupe infini de parallélogrammes de forces, d’où ressort une résultante — l’événement
historique — qui peut être regardée elle-même, à son tour, comme le
produit d’une force agissant comme un tout, de façon inconsciente et aveugle.
Car ce que veut chaque individu est empêché par chaque autre, et ce qui s’en
dégage est quelque chose que personne n’a voulu » ({\it Lettre à Joseph Bloch}, du
21 septembre 1890). C’est précisément parce que tout individu, dans l’histoire,
poursuit un but, ou plusieurs, qu’il est exclu que l’histoire elle-même veuille
aller quelque part ou signifier quelque chose. Si tout sujet est historique, comment
l’histoire serait-elle sujet ? Si tout sens est dans l’histoire, comme l’histoire
elle-même pourrait-elle en avoir un ? Cela ne nous empêche pas d’y poursuivre
des buts, répétons-le, ni même de les atteindre parfois. Mais le sens alors qui
s’en dégage n’est pas celui de l’histoire ; c’est celui de notre action. Mieux vaut
un militant qu’un prophète.

%— 275 —
\section{Historicisme}
%HISTORICISME
C'est vouloir tout expliquer par l’histoire. Mais si l’histoire
produit tout, y compris les explications qu’on en donne,
que valent ces explications, et que vaut l’historicisme ?

Il faut que l’histoire soit rationnelle, ou bien que la raison soit historique :
rationalisme ou historicisme. Les deux ne peuvent être vrais ensemble et totalement.
Mais ils peuvent l’être l’un et l’autre dans des ordres différents : rationalisme
dans l’ordre théorique (contre la sophistique), historicisme dans l’ordre
pratique (contre le dogmatisme pratique). Toute valeur est historique ; toute
vérité, éternelle. Qu’on ne puisse connaître totalement celle-ci, ce n’est pas une
objection contre elle — puisque toute objection la suppose. Qu’on ne puisse
totalement se défaire de celle-là, et qu’on ne le doive, ce n’est pas davantage une
objection contre l’historicisme : l’histoire, en effet, peut l’expliquer, par
l'impossibilité où nous sommes d’en sortir. L’historicisme et le rationalisme
peuvent ainsi s’articuler l’un à l’autre : l’histoire explique tout, sauf ce qu’il y a
de vrai dans nos explications.

\section{Holisme}
%HOLISME
Une pensée qui accorde davantage d’importance au tout ({\it holos},
entier) qu’à ses parties, ou qui s’interdit de réduire un ensemble
aux éléments qui le composent. Appliqué à la société, le holisme s’oppose à
l’individualisme.

\section{Hominisation}
%HOMINISATION
L’humanité n’est pas une essence, c’est une histoire, et
cette histoire est d’abord naturelle : l’hominisation est ce
processus biologique par lequel {\it homo sapiens} se distingue progressivement — par
mutations et sélection naturelle — des espèces dont il descend. Reste, ensuite, à
devenir humain, au sens normatif du terme : ce n’est plus {\it hominisation}, mais
{\it humanisation}. La seconde, sans la première, serait impossible. Mais la première,
sans la seconde, serait vaine : cela ne ferait qu’un grand singe de plus.

\section{Homme}
%HOMME
Au sens étroit : un membre de l'espèce humaine, de sexe masculin
et d’âge adulte. Au sens large ou générique : tout être humain,
quel que soit son âge ou son sexe. C’est en ce sens que tous les hommes sont
égaux en droits et en dignité. Cela ne supprime bien sûr pas la différence
sexuelle, pas plus que celle-ci ne porte atteinte à l’unité de l’espèce. Que
l'humanité soit sexuée, c’est au contraire ce qui lui permet d’exister et d’être
humaine. Qu'est-ce, en effet, qu’un être humain ? Un animal qui parle, qui raisonne,
qui vit en société, qui travaille, qui rit, qui crée ? Rien de tout cela,
%— 276 —
puisque la découverte, sur Terre ou sur une autre planète, d’une espèce vivante
douée de langage, de raison, etc., ne changerait pas les limites de l’espèce
humaine ; et puisqu’un débile profond, même incapable de parler, de raisonner,
de travailler, de créer, de rire, et fût-il, tel l'Enfant sauvage d’Itard,
dépourvu de toute socialité, n’en est pas moins {\it homme} pour autant. On peut
faire le même reproche à la fameuse définition de Linné : {\it Animal rationale,
loquens, erectum, bimane}. Elle ne vaut pas pour tous les hommes, et rien ne
prouve qu’elle vaille pour eux seuls. Aucune définition fonctionnelle ou normative
n’est ici acceptable, puisqu’elle reviendrait à exclure de l’espèce humaine
ceux qui sont hors d’état de {\it fonctionner} normalement. Il faut donc un autre critère,
non plus fonctionnel mais générique, non plus normatif mais spécifique.
La biologie en propose un, pour toute espèce animale, qui est l’interfécondité :
un individu appartient à une espèce s’il peut se reproduire par croisement avec
un autre membre de cette espèce et engendrer un être lui-même fécond. Cela
toutefois ne vaut que pour l'espèce, point pour l’individu : un homme stérile
ne cesse pas pour cela d’être un homme. Nous devons donc, si nous voulons un
critère qui soit individuellement opératoire, prendre le problème par l’autre
côté : celui non de l’engendrement, mais de la filiation. On aboutit alors à la
définition suivante, qui me paraît seule valoir pour tout le défini et pour aucun
autre : {\it Est un être humain tout être né de deux êtres humains}. On m’objectera
que cette définition est circulaire, puisqu’elle suppose humanité. Mais c’est
moins une faiblesse définitionnelle qu’un fait de l'espèce, que tout individu
suppose en effet pour pouvoir exister et être défini. Quant à définir l'espèce
elle-même, c’est aux naturalistes de le faire, qui nous apprennent qu’{\it homo
sapiens} fait partie de la classe des mammifères et de l’ordre des primates, dont il
se distingue spécifiquement (quoiqu'il puisse y avoir des exceptions individuelles)
par un certain nombre de caractères génétiques bien connus : un cerveau
plus gros, un pouce opposable aux autres doigts, un larynx apte à la
parole... L’humanité, sans ces caractères biologiques, ne serait pas ce qu’elle
est. Mais ce n’est pas parce qu’il a ces caractères qu’un individu est humain :
c’est parce qu’il est né de deux êtres humains qu’il peut les avoir — et qu’il restera
humain, d’ailleurs, quand bien même tel ou tel de ces caractères lui ferait
défaut. Si c’est la filiation qui fait l’homme, l’humanité ne peut être définie, en
chacun, par ses performances. Comment un handicap, aussi grave soit-il, pourrait-il
annuler ce qu’il suppose ?

On m'objectera aussi le clonage reproductif, qui permettrait d’engendrer
un être humain à partir d’un seul individu. J’y vois moins une objection qu’une
raison forte de refuser le clonage : non parce qu’il invaliderait ma définition
(laquelle pourrait s’en trouver simplement modifiée : Est un être humain,
pourrait-on dire sans difficulté, tout être né d’au moins un être humain), mais
%— 277 —
parce qu’il mettrait en cause l’un des traits les plus précieux de l'humanité, qui
est l’engendrement, par deux individus différents, et parce qu’ils sont différents,
d’un troisième individu, qui ne saurait pour cela être identique à aucun
des deux premiers. Se reproduire, pour un homme ou une femme, ce n’est
jamais se reproduire à l'identique, et c’est très bien ainsi. Un être humain
engendré par clonage, à partir d’un seul individu, appartiendrait assurément à
l'espèce (d’où la définition modifiée que je propose par anticipation) ; mais
l’humanité, si la chose se généralisait, en serait moins riche de différences, et
par là moins humaine : ce n’est pas une définition, qu’il faut sauver, mais ce
qu'il y a d’infiniment diversifié et imprévisible dans l'humanité. Procréer, c’est
créer, non répéter. Le droit d’être différent de ses parents fait partie des droits
de l’homme.

\section{Honnêteté}
%HONNÊTETÉ
C’est la justice à la première personne, telle qu’elle s'impose
surtout dans les rapports de propriété, les échanges, les
contrats : le respect non seulement de la {\it légalité}, dans un pays donné, mais de
l'{\it égalité}, au moins en droit, entre tous les individus concernés. Par exemple si je
vends un appartement sans indiquer à l’acheteur tel vice caché qui s’y trouve,
ou même tel inconvénient du voisinage. Il se peut que je n’aie violé aucune loi
(cela dépend de l’état de la législation : nul vendeur n’est légalement obligé, que
je sache, d’indiquer que son voisin est bruyant ou grossier) ; mais je n’aurai pas
été tout à fait {\it honnête} avec lui, puisque j'aurai profité d’une inégalité (de
connaissance, d’information) entre nous pour en tirer parti à son désavantage.
La règle de l'honnêteté est donc une règle de justice, en tant qu’elle s'impose à
chacun dans ses rapports, spécialement marchands ou contractuels, avec autrui.
Elle est bien énoncée par Alain : « Dans tout contrat et dans tout échange,
mets-toi à la place de l’autre, mais avec tout ce que tu sais, et, te supposant aussi
libre des nécessités qu’un homme peut l'être, vois si, à sa place, tu approuverais
cet échange ou ce contrat » ({\it 81 chapitres...}, VI, 4). L’échange ou le contrat sont
honnêtes quand je peux répondre oui; malhonnête dans le cas contraire.
L’honnêteté est moins fréquente qu’on ne le croit.

En pratique, la notion sert surtout par rapport à la propriété : être honnête,
ce serait respecter la propriété d’autrui ; être malhonnête, s’en emparer indûment.
C’est bien commode pour les propriétaires. Et certes le voleur, même le
plus pauvre, manque à l'honnêteté. Toute personne qui a été cambriolée le sait
bien : qu’on prenne ce qui lui appartient, sans son accord et sans contrepartie,
cela n’est pas juste. Mais combien de riches, même n’ayant jamais volé personne,
même honnêtes en ce sens, attentent — par trop de richesse, par trop
d’égoïsme et de bonne conscience — à la justice ? C’est que l'égalité des hommes
%— 278 —
entre eux ne saurait se cantonner au monde du droit, des échanges, des
contrats. L’honnêteté est une justice de propriétaires, ou vis-à-vis d'eux. Mais
c’est la justice, moralement, qui vaut, non la propriété.

\section{Honneur}
%HONNEUR
La dignité, quand elle passe par le regard des autres. Ou
l’amour-propre, quand il se prend au sérieux. Peut pousser à
l’héroïsme autant qu’à la guerre ou à l'assassinat (les « crimes d’honneur »).
C’est un sentiment foncièrement équivoque, qu’on ne saurait ni admirer ni
mépriser tout à fait. C’est une passion noble ; mais ce n’est qu’une passion,
point une vertu. Qu’on ne puisse socialement s’en passer, jen suis d’accord.
Raison de plus, individuellement, pour s’en méfier. « L’honneur national,
disait Alain, est comme un fusil chargé. » Et que dire de l’honneur de ces adolescents
qui s’entre-tuent, à la porte de nos collèges, pour un regard ou une
injure ? L’honneur a fait plus de morts que la honte, et plus d’assassins que de
héros.

\section{Honte}
%HONTE
Ce n’est pas le sentiment de culpabilité, puisqu’on peut avoir honte
en se sachant innocent. Par exemple parce qu’on souffre, sous le
regard d’autrui, de se sentir ridicule ou pitoyable. Un faux pas, une disgrâce
physique, une tache sur le visage ou un vêtement, peuvent y suffire. Et combien
de femmes, après avoir été violées, disent la honte qu’elles ont ressentie,
non assurément qu’elles se jugeassent coupables, mais parce qu’elles s'étaient
senties humiliées, méprisées, avilies, quoiqu’elles n’y soient pour rien, enfin en
quelque chose atteintes dans leur honneur ou leur dignité. Le jugement compte
moins ici que la sensibilité, la morale moins que la souffrance, la culpabilité
moins que l’amour-propre. On ne peut donc accepter tout à fait, du moins en
français, la définition que proposait Spinoza : « La honte est une tristesse
qu’accompagne l’idée d’une action, dont nous imaginons qu’elle est blâmée par
d’autres » ({\it Éthique}, III, déf. 31 des Affects). Que ce puisse être le cas, dans telle
ou telle honte particulière, n'autorise pas à penser que ce soit le cas de toutes.
On peut avoir honte là où l’on n’a pas agi, là où aucun blâme n’est envisa-
geable, simplement parce que l’image qu’on donne de soi, ou que les autres en
ont, ne correspond pas à celle que l’on voudrait offrir. Certains peuvent avoir
honte de leur corps, de leur misère, de leur inculture, de leurs parents parfois,
non parce qu'ils s’en sentent responsables, mais parce qu’ils se sentent amoin-
dris ou humiliés, sous le regard des autres, d’être dans ce corps, cet état ou cette
famille... Descartes a bien vu que la honte avait à voir avec l’amour de soi, et
que cela, loin de la condamner, pouvait faire, parfois, son utilité ({\it Passions}, III,
%— 279 —
205 et 206). Reste à n’en être pas prisonnier. C’est un amour malheureux ou
blessé, qu’il importe de guérir.

On remarquera qu’on n’a pas honte devant les animaux, ni tout à fait dans
la solitude (ce ne serait plus honte mais remords ou repentir). La honte est un
sentiment de soi à soi, mais par la médiation d’un ou de plusieurs autres. La
honte, souligne Jean-Paul Sartre, est «une appréhension unitaire de trois
dimensions ». J'ai honte lorsque le sujet que je suis se sent objet, pour un autre
sujet : « J'ai honte de {\it moi} devant {\it autrui} » ({\it L'être et le néant}, p. 350). On n’y
échappe qu’en échappant aux regards, par la solitude, ou au statut d’objet, par
l’amour ou le respect. Nietzsche, en trois aphorismes, a peut-être dit l'essentiel :

« {\it Qui appelles-tu mauvais ?} — Celui qui veut toujours faire honte.

{\it Que considères-tu comme ce qu'il y a de plus humain ?} — Épargner la honte à
quelqu'un.

{\it Quel est le sceau de la liberté conquise ?} — Ne plus avoir honte de soi-même »
({\it Le gai savoir}, III, 273-275).

\section{Humanisation}
%HUMANISATION
On naît homme, ou femme ; on devient humain. Ce
processus, qui vaut pour l’espèce autant que pour l’individu,
c’est ce qu’on peut appeler l’humanisation : c’est le devenir humain de
l’homme — le prolongement culturel de l’hominisation.

\section{Humanisme}
%HUMANISME
Historiquement, c’est d’abord un courant intellectuel de la
Renaissance (ceux qu’on appelle les humanistes : Pétrarque,
Pic de La Mirandole, Érasme, Budé.....), fondé sur l'étude des humanités
grecques et latines et débouchant sur une certaine valorisation de l’individu.
Mais le mot, en philosophie, a un sens beaucoup plus large : être humaniste,
c'est considérer l’humanité comme une valeur, voire comme la valeur suprême.
Reste à savoir si cette valeur est elle-même un absolu, qui se donne à connaître,
à reconnaître, à contempler, ou bien si elle reste relative à notre histoire, à nos
désirs, à une certaine société ou civilisation. On parlera dans le premier cas
d’{\it humanisme théorique}, lequel peut être métaphysique ou transcendantal, mais
tend toujours à devenir une religion de l’homme (voyez {\it L'homme-Dieu}, de Luc
Ferry) ; dans le second, d’{\it humanisme pratique}, qui ne prétend à aucun absolu,
à aucune religion, à aucune transcendance : ce n’est qu’une morale ou un guide
pour l’action. Le premier est une foi ; le second, une fidélité. Le premier fait de
l’humanité un principe, une essence ou un absolu ; le second n’y voit qu’un
résultat, qu’une histoire, qu’une exigence. La vraie question est de savoir s’il
faut croire en l’homme (humanisme théorique) pour vouloir le bien des individus,
%— 280 —
ou si l’on peut vouloir leur bien (humanisme pratique) quand bien
même on aurait toutes les raisons de ne pas s’illusionner sur ce qu’ils sont. Tel
était l’humanisme de Montaigne, dont j’ai parlé ailleurs ({\it Valeur et vérité}, p. 94-95
et 238-240). Tel aussi celui de La Mettrie. « Je déplore le sort de l'humanité,
écrivait-il, d’être, pour ainsi dire, en d’aussi mauvaises mains que les siennes. »
Ce n’est pas une raison pour l’abandonner à son sort, puisque ces mains, précisément,
sont les nôtres. En bon matérialiste, La Mettrie ne voit dans les êtres
humains que de purs produits de la matière et de l’histoire (c’est la thèse
fameuse de l’{\it homme machine}, qui fonde un anti-humanisme théorique). Mais
le médecin qu’il était aussi n’a pas renoncé pour cela à les soigner, pas plus que
le philosophe à les comprendre et à leur pardonner. « Savez-vous pourquoi je
fais encore quelque cas des hommes ?, demandait-il. C’est que je les crois
sérieusement des machines. Dans l’hypothèse contraire, j’en connais peu dont
la société fût estimable. Le matérialisme est l’antidote de la misanthropie. »
Humanisme sans illusions, et de sauvegarde. Ce n’est pas la valeur des hommes
qui fonde le respect que nous leur devons ; c’est ce respect qui leur donne de la
valeur. Ce n’est pas parce que les hommes sont bons qu’il faut les aimer ; c'est
parce qu’il n’y a pas de bonté sans amour. Enfin, ce n’est pas parce qu’ils sont
libres qu’il faut les éduquer ; c’est pour qu’ils aient une chance, peut-être, de le
devenir. C’est ce que j'appelle l’humanisme pratique, qui ne vaut que par les
actions qu’il suscite. Ce n’est pas une croyance ; c’est une volonté. Pas une
religion ; üne morale. Croire en l’homme ? Je ne vois pas ce que cela signifie,
puisque son existence est avérée, ni pourquoi ce serait nécessaire. Pas besoin de
croire en l’homme pour vouloir le bien des individus et le progrès de l’humanité.
Au reste, nous partons de si bas qu’il doit bien être possible de nous élever
quelque peu. C’est où l’on retrouve le premier sens du mot {\it humanisme}, qui
renvoie aux études, à la culture, à l’étude attentive et fidèle du passé humain.
C’est la seule voie pour l'avenir, si l’on veut qu’il soit acceptable. L'homme
n’est pas Dieu. Faisons au moins en sorte, et l’on n’en a jamais fini, qu’il soit à
peu près humain.

\section{Humanité}
%HUMANITÉ
Le mot se prend en deux sens, l’un descriptif l’autre normatif :
humanité, le cas est assez singulier, est à la fois une espèce
animale et une vertu. Et non par métaphore (comme on dit d’un homme très
courageux : « C’est un lion»), mais par métonymie: c’est passer du tout
(l'espèce humaine) à la partie (puisque certains membres de l’espèce seront dits
inhumains), du donné au résultat (humanité, c’est ce que l'espèce humaine a
fait de soi, qu’elle doit préserver), de la nature à la culture, du fait à la valeur,
de l’existence à l’exigence, de l’appartenance à la fidélité. C’est pourquoi on
%— 281 —
parle aussi des {\it Humanités}, pour désigner la culture, spécialement littéraire :
parce que le passé de l'humanité s’y reflète, qui nous apprend ce que nous
sommes par ce que nous pouvons et devons devenir. « Il n’y a point d’Humanités
modernes, disait Alain. Il faut que le passé éclaire le présent, sans quoi nos
contemporains ne sont à nos yeux que des animaux énigmatiques. » L’humanité
n’est devant nous, comme idéal, que parce qu’elle est d’abord derrière
nous, comme histoire, comme mémoire, comme fidélité, Que nous devions
nous soucier de nos descendants, c’est une évidence. Mais qui ne vient pas
d’eux.

\section{Humilité}
%HUMILITÉ
Être humble, c’est avoir le sentiment de sa propre insuffisance.
C’est pourquoi c’est une vertu : être {\it suffisant}, c'est manquer à
la fois de lucidité et d’exigence. Voyez par exemple la prétention de Greuze,
Boucher, Fragonard, et l'humilité de Chardin. On ne confondra donc pas
l’humilité avec la haine de soi, encore moins avec la servilité ou la bassesse.
L’homme humble ne se croit pas inférieur aux autres : il a cessé de se croire
supérieur. Il n’ignore pas ce qu’il vaut, ou peut valoir : il refuse de s’en
contenter. Vertu humble (qui se vanterait de la sienne prouverait par à qu’il en
manque), mais nécessaire. C’est le contraire de l’orgueil, de la vanité, de
l’amour aveugle de soi-même, de la complaisance, de la suffisance, j'y reviens,
et c'est pourquoi elle est si précieuse : il ne lui manque qu’un peu de simplicité
et d’amour pour être bonne absolument.

\section{Humour}
%HUMOUR
C'est une forme de comique, mais qui fait rire surtout de ce qui
n'est pas drôle. Par exemple ce condamné à mort qu’évoque
Freud, qu’on mène un lundi à l’échafaud : « Voilà une semaine qui commence
bien ! », murmure-t-il. Ou Woody Allen : « Non seulement Dieu n'existe pas,
mais essayez de trouver un plombier pendant le week-end ! » Ou encore Pierre
Desproges annonçant sa maladie au public : « Plus cancéreux que moi, tu
meurs ! » Cela suppose un travail, une élaboration, une création. Ce n’est pas le
réel qui est drôle, mais ce qu’on en dit. Non son sens, mais son interprétation
— ou son non-sens. Non le plaisir qu’il nous offre, mais celui que nous prenons
à constater qu’il n’en propose aucun qui puisse nous satisfaire. Conduite de
deuil : nous cherchons un sens ; nous constatons qu’il fait défaut ou se détruit ;
nous rions de notre propre déconfiture. Et cela fait comme un triomphe pourtant
de l'esprit.

L'humour se distingue de l'ironie par la réflexivité ou l’universalité. L’ironiste
rit des autres. L’humoriste, de soi ou de tout. Il s’inclut dans le rire qu’il
%— 282 —
suscite. C’est pourquoi il nous fait du bien, en mettant l’{\it ego} à distance. L’ironie
méprise, exclut, condamne ; l’humour pardonne ou comprend. L’ironie blesse ;
humour soigne ou apaise.

Il y a du tragique dans l’humour ; mais c’est un tragique qui refuse de se
prendre au sérieux. Il travaille sur nos espérances, pour en marquer la limite,
sur nos déceptions, pour en rire, sur nos angoisses, pour les surmonter. « Ce
n’est pas que j'aie peur de la mort, explique par exemple Woody Allen, mais je
préférerais être ailleurs quand cela se produira. » Défense dérisoire ? Sans doute.
Mais qui s’avoue telle, et qui indique assez, contre la mort, qu’elles le sont
toutes. Si les fidèles avaient le sens de l'humour, que resterait-il de la religion ?

\section{Hypocrisie}
%HYPOCRISIE
C’est vouloir passer pour ce qu’on n’est pas, afin d’en tirer
avantage — non par vanité, comme dans le snobisme, mais
par calcul ou intérêt ; non pour imiter ceux qu’on admire ou qu’on envie, mais
pour duper ceux qu’on méprise ou qu’on veut utiliser. Le snob est un simulateur
sincère, qui se dupe lui-même ; l’hypocrite, un simulateur mensonger, qui
dupe les autres. Ainsi, Tartuffe: sil admirait les dévots, il ne serait pas
hypocrite ; il serait snob. Mais il n’a que faire d’admirer : il ne veut paraître ce
qu’il n’est pas que pour tromper et utiliser. L’hypocrisie est une mauvaise foi
lucide et intéressée. C’est pourquoi elle est rare (la lucidité l’est toujours) et
habituellement efficace (elle peut compter sur le snobisme des autres : si Orgon
n'avait voulu passer pour pieux, à ses propres yeux autant qu’à ceux des autres,
il ne se serait pas entiché à ce point de Tartuffe). Contre elle, la lucidité et
l'humour : se défier des autres, et de soi.

\section{Hypostase}
%HYPOSTASE
Ce qui se trouve sous (c’est l'équivalent grec du latin {\it substantia}),
autrement dit ce qui porte ou fonde. Le mot désigne
une réalité existant en soi, mais considérée, surtout depuis le néoplatonisme,
dans son rapport à d’autres hypostases, dont elle procède ou qu’elle engendre :
ainsi, chez Plotin, l’Un (première hypostase) engendre l’Intellect (deuxième
hypostase, qui émane de l’Un), lequel engendre à son tour l'Âme du monde
(troisième hypostase, qui émane de l’Intellect). Le mot sera repris par la tradi-
tion chrétienne, pour désigner les trois Personnes de la Trinité — le Père, le Fils,
le Saint Esprit —, considérées comme trois hypostases d’un seul et même être
({\it ousia}), qui est Dieu. Ces deux usages, néoplatonicien puis chrétien, fortement
marqués de mysticisme, expliquent peut-être que le mot ait fini par se distin-
guer assez fortement de celui de substance. On peut dire d’une pierre qu’elle est
une substance (si l’on considère qu’elle existe en soi) ; on évitera de dire qu’elle
%— 283 —
est une hypostase. C’est qu’il y a du mystère dans l’hypostase : c’est une substance
qu'on ne comprend pas, qui nous dépasse, qu’on ne peut expérimenter,
si on le peut, que de façon surnaturelle ou mystique. De là le sens péjoratif que
le mot, dans la période moderne, finit par prendre : une hypostase serait une
substance supposée ou fictive, une entité à laquelle on accorderait, à tort, une
réalité indépendante. Ainsi Platon hypostasie ses idées, comme Descartes son
{\it cogito}. Un matérialiste leur objectera que les Idées ou l’âme ne sont que des
fictions : une façon d’ériger la pensée, qui n’est qu’un acte du corps, au rang de
réalité indépendante ou substantielle. L’hypostase, en ce dernier sens, n’est
qu'une abstraction hypostasiée : on sépare d’abord la pensée de ce qui la produit
(le corps, le cerveau), puis on en fait une réalité existant en soi. Reste à
savoir, toutefois, si la matière n’est pas, elle aussi, une hypostase.

\section{Hypostasier}
%HYPOSTASIER
Considérer comme hypostase ou comme substance. Le
mot a le plus souvent un sens péjoratif : c’est prêter une
réalité absolue ou indépendante à ce qui n’est qu’un processus, qu’un accident
ou qu’une abstraction.

\section{Hypothèse}
%HYPOTHÈSE
C’est une supposition, qui prend ordinairement place dans
une démarche démonstrative ou expérimentale : une idée
qu’on admet provisoirement comme vraie, afin d’en déduire les conséquences
et, le cas échéant, d’en confirmer ou d’en infirmer la vérité. Dans les sciences
expérimentales, c’est une «explication anticipée», disait Claude Bernard,
qu’on soumet à l'expérience afin d’en tester la validité. Ces sciences, qu’on a dit
longtemps inductives (parce qu’elles iraient du fait à la loi), sont plutôt
hypothético-expérimentales : leurs hypothèses ne sont scientifiques, montre Popper,
que dans la mesure où elles peuvent être soumises à l’expérience et, le cas
échéant, réfutées par elle (voir l’article « Falsifiabilité »). En mathématiques, les
hypothèses sont plutôt des conventions, qui valent moins par elles-mêmes que
par le système des conséquences nécessaires qu'on en peut déduire (les
théorèmes) : elles forment une axiomatique, qui sert de base à un système
hypothético-déductif.

\section{Hypothético-déductive (méthode —)}
%HYPOTHÉTICO-DÉDUCTIVE (MÉTHODE —)
Toute méthode qui part
d’hypothèses pour en
déduire des conséquences, que celles-ci soient falsifiables (dans les sciences
expérimentales) ou non. Se dit spécialement des mathématiques, qui visent
% 284 
moins à vérifier leurs hypothèses (comment une convention pourrait-elle être
démontrée ?) qu’à produire, à partir d’elles, un système cohérent : la vérité est
moins dans les théorèmes que dans le lien nécessaire qui les unit aux hypothèses
de départ (principes, axiomes, postulats....) ou à d’autres théorèmes. C’est l’une
des conséquences épistémologiques de l'invention des géométries non euclidiennes :
le postulat d’Euclide n’est plus une évidence ni une proposition à
démontrer, mais une simple convention, qu’on peut poser ou pas, et qui ne
débouche que sur un système géométrique particulier (la géométrie euclidienne),
parmi d’autres possibles. Il en va de même des autres axiomes ou postulats,
de telle sorte que le statut des théorèmes eux-mêmes s’en trouve transformé.
Il n’y a plus pour eux de vérité séparée : « leur vérité, c’est seulement
leur intégration au système » (R. Blanché, {\it L'axiomatique}, PUF, p. 7). Ce système
est-il vrai ? Ce n’est plus la question : il lui suffit d’être cohérent. C’est
pourquoi les mathématiques ne suffisent pas.

\section{Hypothétique (jugement —)}
%HYPOTHÉTIQUE (JUGEMENT —)
Ce n’est pas un jugement qui ne prétendrait
qu’au statut d’hypothèse ou
de simple possibilité (on parle alors de jugement {\it problématique}). Un jugement
est {\it hypothétique} lorsqu'il énonce une relation entre une hypothèse et l’une au
moins de ses conséquences. Par exemple : « Si Médor est un homme, il est
mortel. » Ou bien : « Si Médor est un triangle, ses trois angles sont égaux à
deux droits. » On voit que ce jugement, même valide, ne prouve rien sur la
nature de mon chien, ni sur sa forme ou sa mortalité. Le jugement, considéré
en tant que tel, n’en est pas moins assertorique : la relation entre l’hypothèse et
sa conséquence est énoncée comme un fait, non comme une hypothèse. Rien
n'empêche pour autant qu’un jugement hypothétique, du point de vue de la
relation, soit par ailleurs problématique, quant à sa modalité (voir ce mot). Par
exemple : « Si Dieu existe, il est possible que l’âme soit immortelle. » Cela ne
prouve pas davantage que Dieu existe, ni que l’âme soit immortelle. Mais
reconnaît que l'existence de celui-là ne saurait garantir l’immortalité de celle-ci.

\section{Hystérie}
%HYSTÉRIE
C’est une névrose, qui enferme l’hystérique dans l'apparence
qu’il veut prendre : le voilà comme prisonnier à la surface de
soi. L’utérus, malgré l’étymologie, n’y est pour rien. Il y a des hommes hystériques,
et certains psychiatres me disent qu’il y en a de plus en plus. C’est que
la notion relève de la psychopathologie, non de la physiologie. Et dépend de
l’évolution de la société, selon toute vraisemblance, presque autant que des histoires
individuelles. Les grandes crises d’hystérie, telles que les décrivait
%— 285 —
Charcot, se font rares (il m’est pourtant arrivé d’en voir une). C’est peut-être
que notre société, où tout est spectacle, n’a plus besoin de tels débordements —
que l’hystérie s’est banalisée en se répandant. Société du spectacle : hystérisation
de la société. Plus besoin de courir les hôpitaux. La rue et la télévision suffisent.

Freud voyait dans l’hystérie un effet du refoulement. D’un point de vue
philosophique, j’y verrais plutôt, mais ce n’est pas contradictoire, une incapacité
à supporter la vérité, une fuite dans l’apparence, un enfermement dans le
simulacre. C’est une maladie du mensonge, mais d’abord à soi-même. L’hystérique
est un simulateur sincère, comme un comédien qui se prendrait pour son
personnage. Il veut faire illusion, et y parvient en effet, jusqu’à y croire lui-même.
Il veut séduire, et y parvient souvent. Mais cela ne fait que le couper un
peu plus du réel, que l’enfermer davantage dans le semblant, dans le factice,
dans la superficialité. Hyperexpressivité, mais à vide ; émotivité à fleur de peau,
mais sans chair ou sans cœur. Volubilité, suggestibilité, mythomanie. Beaucoup
de charme au-dehors, beaucoup de vide au-dedans. L’hystérique en fait
trop ; mais c’est pour masquer (et {\it se} masquer) un manque d’être. Chatoiement
de surface, absence de profondeur. Multiplication des signes, fuite du sens.
Somatisation, théâtralisation, donjuanisme. Besoin éperdu de séduire, incapacité
d’aimer et de jouir. C’est comme un narcissisme extraverti, qui ne saurait
s'aimer que dans le regard de l’autre. Avec l’âge, cela devient de plus en plus
difficile : la dépression ou l’hypocondrie menacent. Tristesse du comédien,
quand le public se détourne.
%{\footnotesize XIX$^\text{e}$} siècle — {\it }


%
%{\footnotesize XIX$^\text{e}$} siècle — {\it }
ICÔNE Une image signifiante, ou un signe imagé. Proche en ce sens de {\it symbole},
mais avec quelque chose de plus immédiatement figuratif.
C’est que les symboles portent le plus souvent sur des abstractions ; les icônes,
sur des objets ou des individus.

IDÉAL C’est quelque chose qui n’existe qu’en idée, donc qui n’existe pas.

Ainsi l’homme idéal, ou la femme idéale, ou la société idéale.
Comme nos idées correspondent plus facilement à nos désirs que ne le fait la
réalité, qui n’en a cure, le mot indique aussi une quasi-perfection. L’idéal n’a
qu’un seul défaut ; c’est qu’il n'existe pas.

«Il faut croire au bien, disait Alain, car il n’est pas ; par exemple à la justice, car
elle n’est pas » ({\it 81 chapitres}, IV, 7). C’est dire que le bien et la justice ne sont que
des idéaux. On n’en condlura pas qu’il n’y a pas lieu de s’en occuper, mais au
contraire qu'ils n'existent que dans la mesure où nous nous en occupons. Rien n'est
réel, dans l'idéal, que la valeur que nous lui prêtons — que le désir, qui nous fait agir.

IDÉALISME Le mot se prend principalement en trois sens, l’un trivial, les
deux autres philosophiques.

Au sens trivial, c’est le fait d’avoir des idéaux, autrement dit de ne pas se résigner
à la médiocrité ambiante, aux plaisirs matériels, à la réalité telle qu’elle est.
S’oppose alors au matérialisme ou au cynisme, pris eux aussi en un sens trivial.

Dans le langage philosophique, le mot peut désigner une certaine conception
de l’être (une ontologie) ou une certaine théorie de la connaissance (une
gnoséologie).

%— 287 —
D'un point de vue ontologique, il désigne l’un des deux grands camps dont
l'opposition, au moins depuis Démocrite et Platon, traverse et structure la
philosophie : est {\it idéaliste} toute doctrine pour laquelle la pensée existe indépendamment
de la matière, voire existe seule, que ce soit sous la forme d’idées
(idéalisme au sens strict) ou sous la forme d’êtres spirituels (auquel cas on parlera
plutôt de spiritualisme). C’est le contraire du matérialisme au sens philosophique.

D'un point de vue gnoséologique, l’idéalisme désigne plutôt une limite de
la connaissance : est {\it idéaliste} tout penseur pour lequel nous ne pouvons rien
connaître de la réalité en soi, soit parce qu’elle n’existe pas, soit parce que nous
ne pouvons connaître que nos représentations. C’est le contraire du réalisme,
au sens gnoséologique du terme. C’est cette dernière acception qui explique
que Kant ait pu caractériser son propre système à la fois comme {\it idéalisme transcendantal}
(nous ne connaissons que des phénomènes, jamais les choses en soi)
et comme {\it réalisme empirique} (nous connaissons effectivement les phénomènes,
qui ne sont pas de pures illusions).

On remarquera qu’on peut être idéaliste au sens ontologique sans l’être au
sens gnoséologique (c’est le cas par exemple de Descartes) ; mais qu’il est difficile
de l’être au sens gnoséologique sans l'être aussi au sens ontologique (si nous
ne connaissons que nos représentations ou notre esprit, pourquoi penser qu'il
existe autre chose, qui serait d’un autre ordre ?). Enfin, qu’on peut être idéaliste
au sens trivial et matérialiste au sens philosophique. Ainsi, Marx : le communisme
était son idéal ; le matérialisme, sa philosophie.

IDÉE C’est une représentation : les idées ne sont visibles ({\it idein}, en grec,
signifie voir) que pour l’esprit, et tout ce que l’esprit se représente
peut être appelé {\it idée}. La forme de cet arbre, devant moi, est son {\it eîdos} (son
aspect, sa forme visible). Mais en tant que je la perçois intérieurement, c’est
une idée. « J’appelle généralement du nom d’idée, écrit Descartes, tout ce qui
est dans notre esprit, lorsque nous concevons une chose, de quelque manière
que nous la concevions » ({\it Lettre à Mersenne}, juillet 1641). En pratique, toutefois,
le mot ne sert guère que pour les représentations les plus abstraites ou
les plus élaborées, à l'exclusion des simples images ou perceptions : on parlera
de l’idée d’arbre, plutôt que de l’idée de cet arbre-ci, et cette dernière ne sera
appelée {\it idée} qu’à la condition de comporter quelque chose de plus que la
simple sensation. L’idée, en ce sens, ce n’est pas seulement ce qui est « dans
la pensée », comme disait aussi Descartes, mais ce qui en résulte, ce que la
pensée produit ou élabore, qui est moins son objet que son effet. Penser, c’est
avoir des idées, mais on ne peut les avoir qu’à la condition de les produire ou
%— 288 —
de les reproduire — qu’à la condition de les penser -, ce qui ne va pas sans
effort ou travail. « Par idée, écrivait Spinoza, j'entends un concept de l'esprit,
que l'esprit forme parce qu’il est une chose pensante » ({\it Éth.} II, déf. 3). L'idée
n’est pas une copie des choses, mais le résultat d’un acte de penser : non
« quelque chose de muet, comme une peinture sur un panneau, souligne
encore Spinoza, mais un mode de penser, savoir l’acte même de connaître »
({\it ibid.}, scolie de la prop. 43). C’est dire, contre Platon, qu’il n’y a pas d’idées
séparées ou en soi : il n’y a que le travail de la pensée. Comment existerait-il
des idées innées ou absolues ? Ce serait pensée sans travail — pensée sans
pensée. Une idée qui n’est pensée par personne n’est pas une idée et n’est
rien.

Ce travail a ses exigences propres, qui sont de vérité plutôt que de ressemblance.
Certes, « une idée vraie doit s’accorder avec l’objet dont elle est l’idée »
({\it Éth.} I, axiome 6). Mais cet accord ne saurait prendre la forme d’une reproduction.
La pensée n’est pas un art figuratif : une idée n’est ni une peinture ni une
image ({\it Éth.} II., scolie de la prop. 48). Il s’agit de penser vrai, pas de faire ressemblant.
L'idée de cercle n’est pas ronde, l’idée de chien n’aboie pas ({\it T.R.E},
27), et aucune idée n’a d’idée.

IDENTITÉ Le fait d’être le même. Mais le même que quoi ? Le même que
le même : il n’y aurait pas autrement identité. Ainsi l’identité
est d’abord une relation de soi à soi (mon identité, c’est mon être moi-même),
ou, lorsqu'il ne s’agit pas de sujets, une relation entre deux objets qui n’en font
qu'un. « Pris au sens strict, ce terme est on ne peut plus précis, remarque
Quine : une chose est identique à elle-même et à rien d’autre, pas même à un
double gémellaire » ({\it Quiddités}, art. « Identité »). Deux jumeaux monozygotes,
même à les supposer parfaitement semblables, ne sont jumeaux qu’en tant qu’il
sont deux individus différents : s’ils étaient absolument le même (au sens où
l’auteur de {\it La Chartreuse de Parme} et celui de {\it Lucien Leuwen} sont le même), ils
ne seraient qu’un et il n’y aurait pas de jumeaux. Ainsi l'identité, prise en ce
sens strict, suppose l’unicité : c’est {\it être un et le même}, et nul n’est le même que
de soi.

En un sens plus large, mais bien avéré dans la tradition, il arrive pourtant
qu’on parle d'identité à propos de deux objets différents, pour marquer qu’ils
sont semblables : par exemple lorsqu'on constate, entre amis, une identité de
points de vue ou de goûts.

Ces deux sens peuvent être légitimes l’un et l’autre ; encore convient-il
de ne pas les confondre. C’est pourquoi on parle souvent, pour désigner le
premier, d'{\it identité numérique} (être un et le même : « Nous habitons dans le
%— 289 —
même immeuble »), alors qu’on parlera d'{\it identité spécifique} où qualitative pour
désigner la parfaite similitude entre plusieurs objets différents (« Nous avons la
même voiture », c’est-à-dire ici deux véhicules de la même marque, du même
modèle et de la même couleur).

Cette dernière identité n’est jamais absolue (deux voitures identiques ne
sont jamais absolument indiscernables). Mais l'identité numérique l’est-elle ?
Au présent, sans doute ; mais au présent seulement. À la considérer dans le
temps, elle est aussi relative — et peut-être plus illusoire — que l’autre. Le
Stendhal qui commence {\it Lucien Leuwen}, en 1834, a quatre ans de moins que
celui qui écrira {\it La Chartreuse de Parme}. Comment lui serait-il identique ? Et
pourquoi, s’il l'était, n’écrivit-il pas le même livre ?

L'erreur serait de croire que cette notion, qui reste toute formelle, puisse
nous apprendre quoi que ce soit sur le réel. Que Stendhal, Henri Beyle et
l’auteur de {\it La vie de Henri Brulard} ne fassent qu’un, cela ne nous apprend
quelque chose que pour autant que nous savons ce que ces mots désignent, ou
plutôt c’est parce que nous le savons que nous pouvons affirmer que ces trois
personnages ne font qu’un. L'identité, pas plus que la carte du même nom, ne
se prononce sur le contenu de ce qu’elle désigne (ce n’est pas la quiddité), mais
seulement sur l'égalité de ce contenu à lui-même. A = A. L'identité n’est pas
l'essence, mais l'essence suppose l'identité.

Il se pourrait, et c’est ce que je crois, que rien, dans le temps, ne reste identique
à soi: que tout soit impermanent, comme disent les bouddhistes, et
qu'on ne se baigne jamais deux fois dans le même fleuve. Le réel ne cesserait pas
pour autant, au présent, d’être identique à soi. C’est où Parménide triomphe
d’'Héraclite, mais en vain : puisqu'il triomphe même si Héraclite a raison. Le
même est à penser, qui est ; mais ce qu'il est, la pensée ne peut l’apprendre que
de l'être, point du même. Il n’y a pas d’ontologie {\it a priori}. L'identité est un
concept nécessaire, mais vide. Elle est le nom qu’on donne à la pure présence à
soi du réel, qui n’est pas un nom.

C’est une dimension du silence, par quoi le discours est possible.

IDENTITÉ (PRINCIPE D’) C’est le principe qui fonde l'adéquation de la

vérité à elle-même. Tout être est ce qu’il est :
a = a, p = p ; où mieux, et comme disaient déjà les stoïciens : {\it Si a, alors a} ; {\it Si p,
alors p}. Si je vis, je vis ; si je fais ce que je fais, je le fais. D’où il ne suit rien, que
la nécessité absolue du présent, qui est tout.
Le principe d’identité est ce qui rend la pensée possible, et la vérité nécessaire.
%— 290 —
IDÉOLOGIE Pour ceux des disciples de Condillac qu’on appelait, au tout
début du {\footnotesize XIX$^\text{e}$} siècle, les {\it idéologues} — et spécialement chez Destutt
de Tracy, qui inventa le mot —, c'était la science des idées, qui serait par là
même la science des sciences. Mais une telle science n’existe pas : on ne peut
connaître que le cerveau, qui pense les idées, ou une théorie particulière, qui
s’en sert ou les sert. C’est pourquoi sans doute ce sens est aujourd’hui tombé en
désuétude. Le mot, depuis des décennies, ne se prend plus guère que dans son
acception marxiste : l'idéologie est un ensemble d’idées ou de représentations
(valeurs, principes, croyances...) qui ne s'expliquent pas par un processus de
connaissance — l’idéologie n’est pas une science — mais par les conditions historiques
de leur production, dans une société donnée, et spécialement par le jeu
conflictuel des intérêts, des alliances et des rapports de forces. C’est comme une
pensée sociale, qui ne serait pensée par personne mais qui penserait en tous, ou
plutôt à l’intérieur de laquelle tous, nécessairement, penseraient. L’idéologie est
inconsciente : elle est le lieu, socialement et historiquement déterminé, de toute
conscience possible. C’est « le langage de la vie réelle » (Marx et Engels, {\it L'idéologie
allemande}, I). Elle est par nature hétéronome : son histoire est soumise à
celle de la société matérielle, elle-même dominée « en dernière instance » par
l'infrastructure économique (forces productives et rapports de production). On
n’a pas les mêmes idées à l’âge de la pierre taillée et à celui de la pierre polie,
dans une société féodale et dans une société capitaliste, à l’époque de la révolution
industrielle et à celle de la révolution informatique.

« L’idéologie n’a pas d’histoire », écrivaient les mêmes auteurs ({\it ibid.}). Il
faut entendre : pas d’histoire autonome, pas d’autre histoire que celle de la
société dont elle fait partie, qui la détermine et sur laquelle elle agit en retour.
Car l'idéologie n’est pas un simple reflet, encore moins un épiphénomène.
C’est une force agissante : la fonction pratico-sociale, soulignait Althusser,
l'emporte en elle sur la fonction théorique. Elle fait de nous des sujets, en nous
assujettissant à elle. Elle constitue « le rapport imaginaire des individus à leurs
conditions d’existence » ({\it Positions}, p. 101 ; voir aussi {\it Pour Marx}, p. 240). Elle
vise moins un effet de connaissance qu’un effet de pouvoir ou de sens.

L’idéologie dominante, disait Marx, est l'idéologie de la classe dominante :
celle-ci fait passer pour des exigences universelles — bien sûr en y croyant elle-même
— des opinions qui ne font qu’exprimer ses intérêts particuliers, tels
qu’ils résultent de sa position dans les rapports sociaux. Seule la vérité lui
échappe, qui n’a que faire de nos intérêts. C’est dire que tout ce qui n’est pas
vrai, dans une pensée donnée, est idéologique. De là l’usage souvent péjoratif
du mot, qui assimile l’idéologie à une conscience fausse. Cet usage est lui-même
idéologique. Qu’une pensée ne soit pas vraie n’implique pas, en effet,
qu’elle soit fausse. Soit par exemple la proposition : {\it « Tous les hommes sont
%— 291 —
égaux en droit et en dignité. »} Que cette affirmation ne relève pas d’une connaissance,
c’est bien clair. Mais cela même, qui lui interdit en effet d’être vraie, lui
interdit aussi d’être fausse. Cela ne signifie pas qu’elle soit sans portée ou sans
valeur. Une thèse idéologique n’est ni vraie ni fausse, explique Althusser, mais
elle peut être juste ou injuste, dans un combat donné. Et je n’en connais guère
de plus juste, dans le combat qui est aujourd’hui le nôtre, que celle-ci. Ainsi
rien n’est faux, dans l'idéologie, que sa prétention à la vérité : ce n’est pas une
conscience fausse, c’est une conscience illusoire, et nécessairement illusoire —
non un ensemble d’erreurs, mais un ensemble d'illusions nécessaires. Que la
morale en fasse partie, par exemple, n'implique pas qu’on doive ou qu’on
puisse se passer de morale : c’est ce qui le rend au contraire impossible. Le
scientisme, qui voudrait se passer de toute idéologie, n’est qu’une idéologie
parmi d’autres. «Seule une conception idéologique de la société, écrit
Althusser, a pu imaginer des sociétés sans idéologies ({\it Pour Marx}, Maspero,
1965, p. 238). Et seule une conception idéologique du marxisme, ajouterai-je,
a pu imaginer qu’il échappe à l'idéologie.

On demandera si la philosophie fait partie de l'idéologie. Il faut répondre :
oui, sauf pour ce qui, dans une philosophie donnée, relève de la vérité (vérité et
science, il faut le rappeler, ne sont pas synonymes), et quand bien même il
serait impossible, entre ces deux parts, de fixer quelque limite assurée que ce
soit. C’est ce qui explique que les pensées d’Aristote ou de Montaigne, qui
vivaient dans des sociétés si différentes de la nôtre, nous semblent encore si
vivantes, si éclairantes, si actuelles. Cela ne signifie pas qu’on puisse, au
{\footnotesize XXI$^\text{e}$} siècle, être aristotélicien ou réécrire les {\it Essais}, mais qu’on peut lire Montaigne
ou Âristote pour un intérêt autre qu’historique : parce qu'ils nous
aident, aujourd’hui, à penser. Ils avaient le même corps que nous, le même cerveau
que nous. Comment n’auraient-ils pas, pour une part, le même esprit ?
L'économie n’est pas tout. L’infrastructure biologique compte aussi, et sans
doute davantage. Au reste, si nous étions incapables de vérité (si tout était idéologie),
le marxisme n’aurait aucun sens. Il faut donc que quelque chose échappe
à l'idéologie pour que la notion d’idéologie puisse prétendre à la vérité. Parce
qu’elle est scientifique ? Je n’en suis pas sûr. Mais parce qu’elle est rationnelle.
On pourrait dire la même chose, chez Montaigne ou Aristote, de toute argumentation
rigoureuse, et on leur en doit d'innombrables. Ainsi toute philosophie
est dans l'idéologie ; mais tout, dans une philosophie, n’est pas nécessairement
idéologique.

IDÉOLOGUE C'est d’abord un praticien de l’idéologie, au sens premier du
terme, c’est-à-dire de ce qui passait, au début du {\footnotesize XIX$^\text{e}$} siècle,
%— 292 —
pour la science des idées : Cabanis et Destutt de Tracy sont les plus fameux ;
Stendhal et le jeune Maine de Biran s’en réclameront. Ce sens n’a plus d’usage
qu’historique. Aujourd’hui, on appelle plutôt {\it idéologue} toute personne qui développe
ou représente une idéologie, au sens actuel et plus ou moins marxiste du
terme, Le mot, dans cette dernière acception, est presque toujours péjoratif : un
idéologue, c’est quelqu'un qui travaille dans l'illusion, mais sans le savoir, et qui
prétend pour cela ériger en vérité universelle son propre point de vue, lequel ne
fait qu’exprimer des intérêts ou des partis pris banalement particuliers.

IDIOSYNCRASIE C’est le {\it mélange (sunkrasis) propre (idios)} à un individu
donné, autrement dit ce qu’il a de singulier, qui résulte
de la rencontre en lui d'éléments qui ne le sont pas. Ce n’est qu’un mot savant,
pour dire la banalité hétérogène d’être soi.

IDIOTIE Manque extrême d'intelligence. Dans la psychopathologie traditionnelle,
l’idiot est l'équivalent de ce qu’on appellerait aujourd’hui
un débile profond (par différence avec l’imbécile, qui correspond plutôt
au débile léger). L’idiot est incapable de parler; limbécile, de parler
intelligemment. Mais le mot a fait son entrée dans la langue proprement philosophique,
il y a une vingtaine d’années, en un sens tout à fait différent, que l’on
doit à Clément Rosset et qui renvoie à l’étymologie. {\it Idiôtès}, en grec, c’est le
simple particulier (le mot est dérivé d’{\it idios}, propre), par opposition aux magistrats
ou aux savants, qui sont supposés parler du point de vue de l’universel.
L’idiotie, en ce sens, est le propre de tout être singulier, en tant qu’il n’est que
soi : c’est la singularité brute, sans phrases, sans double, sans alternative. C’est
comme un idiotisme ontologique : la pure singularité d’exister. C’est donc le
propre de tout être (la singularité est une caractéristique universelle), et c’est ce
qu’indique bien clairement l’un des plus beaux titres de Clément Rosset et de
l’histoire de la philosophie : {\it Le réel, Traité de l'idiotie} (Éditions de Minuit,
1977 ; sur le sens du mot, voir spécialement les p. 7 et 40-51).

IDOLÂTRIE C’est adorer une idole, c’est-à-dire une image de la divinité
plutôt que Dieu même, ou un faux dieu plutôt que le vrai.
L'idolâtrie, en ce sens, est la religion {\it des autres}. On ne pourrait y échapper
qu’en adorant un Dieu qu’on ne puisse aucunement imaginer. Mais comment
savoir, alors, s’il est Dieu ?

%— 293 —
On peut parler d’idolâtrie, en un sens plus général, pour toute adoration
d’un objet visible ou sensible, et même d’une entité quelconque, dès lors qu’elle
est supposée exister ici-bas. Adorer la Nature, la Force, l’État, la Société,
l’Argent, la Science, l'Histoire ou l'Homme, c’est idolâtrie toujours. Simone
Weil, commentant le début du Notre Père, en apparence si dérisoire (« Notre
Père, qui êtes aux cieux. »), en a fait la remarque : « Le Père est dans les cieux.
Non ailleurs. Si nous croyons avoir un Père ici-bas, ce n’est pas lui, c’est un
faux Dieu » ({\it Attente de Dieu}, p. 215). Dieu n’est Dieu que par son absence, et
tel est le secret peut-être de la transcendance : tant que nous adorons quelque
chose de présent, nous adorons un faux Dieu ; même monothéistes ou athées,
nous sommes idolâtres. On n’y échappe qu’en adorant l'absence même, ou en
cessant d’adorer.

IDOLE Une image ({\it eidolon}) divine, ou un Dieu imaginaire. Par métaphore,
toute personne que l’on adore comme un dieu. Reste à
savoir si Dieu lui-même n’est pas une première métaphore.

ILLUSION Ce n’est pas la même chose qu’une erreur. C’est une représentation
prisonnière de son point de vue, et qui résiste même à la
connaissance de sa propre fausseté : j’ai beau savoir que la Terre tourne autour
du Soleil, je n’en vois pas moins le Soleil se mouvoir d’est en ouest. « Est illusion,
écrit Kant, le leurre qui subsiste, même quand on sait que l’objet supposé
n'existe pas » ou est autre ({\it Anthropologie...}, \S 13). Il y a donc une positivité de
l'illusion. Si l'erreur n’est qu’une privation de connaissance (ce en quoi elle
n'est rien et s’abolit dans le vrai), l'illusion serait plutôt un excès de croyance,
d'imagination ou de subjectivité : c’est une pensée qui s'explique moins par le
réel que je connais que par le réel que je suis.

Cette subjectivité peut être purement sensorielle (les illusions des sens) ou
transcendantale (s’il existe, comme le veut Kant, des illusions de la raison).
Mais elle s'exprime plus souvent comme subjectivité désirante : se faire des illusions,
c’est prendre ses désirs pour la réalité. Tel est le sens du mot chez Freud :
« Ce qui caractérise l'illusion, écrit-il, c’est d’être dérivée des désirs humains »
({\it L'avenir d'une illusion}, VI). Toute erreur n’est donc pas une illusion, ni toute
illusion une erreur. Je peux me tromper sans que ce soit du fait de mes désirs
(c'est alors une erreur, non une illusion), et ne pas me tromper bien que ma
pensée doive plus à mes désirs qu’à une connaissance (c’est alors une illusion,
non une erreur : par exemple la jeune fille pauvre qui croit qu’un prince va
venir l’épouser ; quelques cas de ce genre, observe Freud, se sont réellement
%— 294 —
présentés). L’illusion, bien qu’elle puisse être fausse, et bien qu’elle le soit le
plus souvent, n’est donc pas un certain type d’erreur. C’est un certain type de
croyance : « Nous appelons illusion une croyance, continue Freud, quand, dans
la motivation de celle-ci, la réalisation d’un désir est prévalante », et quoi qu’il
en soit par ailleurs de son rapport à la réalité. C’est une croyance désirante, ou
un désir crédule.

Si l’on admet, avec Spinoza, que tout jugement de valeur suppose un désir
et s’y ramène ({\it Éthique}, III, 9, scolie), il en résulte que toutes nos valeurs sont
des illusions. On n’en conclura pas qu’il faudrait s’en passer, mais au contraire
qu'on ne le peut (puisque nous sommes des êtres de désirs) et qu’on ne le doit
(l'humanité n’y survivrait pas). Illusions nécessaires : on ne pourrait y échapper
que pour tomber aussitôt dans d’autres. « Seule une conception idéologique de
la société a pu imaginer des sociétés sans idéologies », écrivait Althusser. Seule
une conception illusoire de l'humanité a pu imaginer une humanité sans illusions.

IMAGE Reproduction ou figuration sensible — qu’elle soit matérielle ou
mentale — d’un objet quelconque. L'important n’est pas que cet
objet existe ou pas réellement (on peut imaginer ou peindre une chimère aussi
bien que son voisin de palier), mais qu’il soit figurable. De à les métaphores,
symboles, allégories et autres images (mais en un sens dérivé), qui visent à
représenter ce qui n’est pas immédiatement présentable : par exemple une
balance pour la justice, une colombe pour la paix, un vieillard ou un jeune
homme pour Dieu.

IMAGINATION La faculté d’imaginer, autrement dit de se représenter
intérieurement des images, y compris et surtout quand ce
qu’elles représentent est absent. Ces images sont des actes, remarquait Sartre,
non des choses : l’imagination est « une certaine façon qu’a la conscience de se
donner un objet », mais de se le donner, paradoxalement, comme absent. C’est
ce qui la rend utile et dangereuse : elle libère du réel, dont elle fait pourtant
partie, mais aussi nous en sépare. Se distingue par là de la connaissance, qui
libère sans séparer, et de la folie, qui sépare sans libérer.
On oppose couramment les classiques, qui se méfiaient de limagination
(« la folle du logis »), aux romantiques et aux modernes, qui en font la faculté
créatrice par excellence. C’est bien sûr moins simple que cela. L'imagination,
écrivait par exemple Pascal, « est cette partie dominante dans l’homme, cette
maîtresse d’erreur et de fausseté, et d’autant plus fourbe qu’elle ne l’est pas
%— 295 —
toujours ; car elle serait une règle infaillible de vérité, si elle l'était infaillible du
mensonge, » C'est ce qui permet à certains romans d’être vrais, et à tant
d’autres d’être faux. « Je ne parle pas des fous, ajoute Pascal, je parle des plus
sages, et c’est parmi eux que l'imagination a le grand droit de persuader les
hommes. La raison a beau crier, elle ne peut mettre le prix aux choses. [...]
L’imagination dispose de tout : elle fait la beauté, la justice et le bonheur, qui
est le tout du monde » ({\it Pensées}, 44-82). Maîtresse d’erreur, créatrice de valeur.
Seule la vérité lui échappe, qui ne vaut, toutefois, que pour autant qu’on l’imagine.

IMMANENCE C'est la présence de tout dans tout (immanence absolue), ou
dans autre chose (immanence relative). Le contraire donc de
la transcendance. Est transcendant ce qui s'élève ({\it scandere}) au-delà ({\it trans}) ;
immanent, ce qui reste ({\it manere}) dans ({\it in}). Se dit spécialement de ce qui est
dans la nature et en dépend. Si tout est matériel, comme je le crois, s’il n'existe
rien d’autre que l'univers ou la nature (rien d’autre que tout !), il faut en
conclure que tout est immanent : la transcendance n’est qu’une extériorité imaginaire,
comme telle immanente (l'imagination fait partie de l’univers).

IMMANENT Est immanent, au sens classique, ce qui est intérieur, ce qui
demeure dans ({\it in-manere}) quelque chose ou quelqu'un. On
parlera par exemple de « justice immanente » pour désigner une récompense ou
une punition incluses dans l’acte même qu’elles sanctionnent (la satisfaction du
devoir accompli, chez l’homme vertueux, la solitude du méchant, l’indigestion
du goinfre...), par opposition à une justice transcendante, qui suppose une intervention
extérieure, qu’elle soit divine ou humaine. On remarquera que l’une et
l’autre sont douteuses. Mais cela en dit plus sur la justice que sur l’immanence.
Chez Kant, tout ce qui fait partie de l’expérience et ne s'applique qu’à elle.
Chez Husserl et les phénoménologues, tout ce qui est intérieur à la conscience.
Au sens absolu, est immanent tout ce qui est intérieur au tout, ou du moins
(si l’on veut que la notion de transcendance garde un contenu) tout ce qui fait
partie de l’univers matériel, c’est-à-dire de l’univers. Le matérialisme, en ce
sens, est un immanentisme absolu : seul Dieu serait transcendant, qui n’est pas.

IMMANENTAL Il m'est arrivé d’appeler {\it immanental} tout ce qui, à l’intérieur
de l'expérience, la rend possible : les conditions
%— 296 —
empiriques de l’empiricité, autrement dit son pouvoir ou son processus d’autoconstitution
historique. C’est l’équivalent, pour le matérialisme, du {\it transcendantal}
pour l’idéalisme. Est immanental tout ce qui constitue une condition de
possibilité, mais {\it a posteriori}, de la connaissance : le corps (spécialement le cerveau),
le langage et l'expérience sont des immanentaux.

On dira qu’à ce compte il y a cercle (puisque les conditions de l’expérience
résultent de l'expérience). Mais ce cercle, qui est plutôt une spirale, est celui-là
même de la pensée. « Les choses qu’il faut avoir apprises pour les faire, disait
Aristote, c’est en les faisant que nous les apprenons » ({\it Éthique à Nicomaque}, II,
1 : c'est en forgeant que l’on devient forgeron). Ainsi faut-il apprendre à
penser, et nul ne le peut qu’en pensant. L’immanental indique que l’origine de
ce processus, qui rend la pensée possible, n’est pas elle-même une pensée, mais
une expérience. À la gloire de l’empirisme.

IMMANENTISME Doctrine pour laquelle tout est immanent, au sens
absolu du terme, ce qui suppose qu’il n'existe aucune
transcendance. Synonyme parfois de matérialisme, mais avec une extension
plus large : le spinozisme — qui n’est ni matérialiste ni idéaliste — est un immanentisme,
et le modèle, depuis trois cents ans, de tous.

IMMATÉRIALISME Une forme extrême et rare d’idéalisme, qui va jusqu’à
nier l'existence de la matière. La philosophie de Berkeley
est sans doute l’exemple le plus radical qu’on en puisse proposer. « Être,
c’est percevoir ou être perçu »; il n'existe que des esprits et des idées. La
«matière» n’est qu’un mot, qui ne correspond à aucune expérience réelle
(puisque nous ne pouvons expérimenter, par définition, que nos perceptions,
qui sont en nous). Qu’une pensée aussi éloignée du sens commun soit irréfutable
en dit long sur la pensée, et sur le sens commun.

IMMORAL Qui s’oppose à la morale, tant que celle-ci est supposée légitime.
ne pas confondre avec l’immoralisme, qui conteste cette légitimité,
ni avec l’amoralité, qui n’en relève pas.

IMMORALISME C’est s’opposer à la morale, le plus souvent parce qu’elle
ne serait qu’une illusion néfaste. Ainsi, chez Nietzsche :
« La morale est le danger par excellence, écrivait-il, l'instinct négateur de la vie :
%— 297 —
il faut détruire la morale pour libérer la vie. » Aussi faut-il s’efforcer de vivre
« par-delà le bien et le mal », se faire « plus fort, plus méchant, plus profond ».
C’est combattre la morale, mais au nom d’une certaine éthique : « Par-delà le
Bien et le Mal, disait encore Nietzsche, cela du moins ne veut pas dire : par-delà
le bon et le mauvais. » On ne sort pas des jugements de valeur : l’immoralisme
est le contraire d’un amoralisme. C’est ce qui autorise la plupart des
immoralistes à être de très braves gens: ce qu’ils reprochent à la morale,
presque toujours, c’est d’être immorale.

IMMORTALITÉ Est immortel ce qui ne peut mourir : ainsi l’âme selon
Platon, ou Dieu selon les croyants. On remarquera que
l’immortalité de l’âme n’est pas une idée chrétienne (si Jésus nous sauve de la
mort, et si nous pouvons ressusciter, c’est que nous pouvons mourir), ni une
idée juive. Une idée grecque ? En partie. Épicure, si elle n’avait été si répandue,
n'aurait pas consacré autant d'énergie à la combattre. Il y voyait moins une
espérance qu’une source intarissable de craintes. Être immortel, ce serait être
exposé à jamais au malheur, aux châtiments, à la répétition — à l’enfer. La mortalité
vaut mieux, qui ne nous expose qu’au néant.

IMPÉRATIF Un commandement, mais qui s’énoncerait à la première personne :
non le contraire de la liberté, mais ce qu’elle s'impose
à elle-même. Ce n’est pas la même chose d’obéir à un souverain ou à un Dieu
(commandement), ou de n’obéir qu’à soi (impératif). Obéir à un commandement,
c’est se soumettre, et sans doute il le faut souvent. Obéir à un impératif,
c’est se gouverner, et il le faut toujours.

On distingue depuis Kant deux types d’impératifs : l'impératif hypothétique
et l’impératif catégorique.

Le premier reste soumis à une condition, qui est ordinairement la fin poursuivie.
Par exemple : « Si tu veux que tes amis soient loyaux avec toi, sois loyal
avec eux. » Ou encore : « Si tu veux éviter la prison, sois honnête. » Ce ne sont
que des règles de la prudence ou de l’habileté. Il s’agit de choisir des moyens
adaptés à la fin qu’on poursuit, et ils ne valent que pour autant qu’on la poursuit
effectivement.

L’impératif catégorique, au contraire, est inconditionnel. C’est qu’il n’a
que faire de quelque fin que ce soit. Par exemple : « Sois loyal avec tes amis. »
Ou bien : « Ne mens pas. » Impératifs moraux, qui commandent absolument :
ils n’ont pas à voir avec la réussite ou l'efficacité, comme la prudence ou l’habileté,
mais avec le devoir. Ainsi, explique Kant, lorsqu'il s’agit de témoigner
%— 298 —
devant un tribunal : celui qui se demande dans quel but il devrait dire la vérité
est déjà un misérable.

Les impératifs hypothétiques restent particuliers : ils ne valent que pour
ceux qui en vérifient la condition, autrement dit qui visent tel ou tel but (des
amis loyaux, la confiance, la réussite...). L’impératif catégorique, parce qu’il est
inconditionnel et ne vise aucun but, est universel : il vaut pour tout être raisonnable
fini, comme dit Kant, y compris pour ceux qui ne le respectent pas. Il est
l’'universel même, en tant que la raison l’énonce et se Le prescrit à elle-même —
non pour la pensée seule (raison théorique), mais pour laction (raison
pratique). C’est ce qui détermine sa formule bien connue et bien exigeante :
« Agis uniquement d’après la maxime qui fait que tu peux vouloir en même
temps qu'elle devienne une loi universelle » ({\it Fondements...}, II). C’est n’obéir
qu’à la raison en soi, autrement dit qu’à la partie de soi qui est libre (parce
qu’elle n’est pas soumise aux penchants ou aux instincts du « cher petit moi »).
C’est n’obéir qu’à soi (autonomie) en se libérant de soi (universalité). Ainsi la
morale ne vaut pour tous que parce qu’elle vaut pour chacun (« tout seul, disait
Alain, universellement »), et le seul devoir est d’être libre.

IMPLICATION C'est une relation entre deux propositions, telle que la

seconde soit une conséquence nécessaire de la première : {\it si
p, alors q}. Si la première est vraie, la seconde l’est aussi. Si la seconde est fausse,
la première également. En revanche, si la première est fausse, la seconde peut
être vraie ou fausse. À la considérer en bloc et d’un point de vue strictement
logique, une implication n’est donc fausse que si et seulement si elle relie un
antécédent vrai à un conséquent faux : « Si Paris est la capitale de la France,
alors les poules ont des dents » est une proposition fausse. Une implication
commençant par une proposition fausse, à l’inverse, est nécessairement valide.
«Si les poules ont des dents, alors je suis roi de France » est une proposition
vraie, qu’elle soit prononcée par Louis XIV ou par votre serviteur.

IMPRESSION C’est une espèce de perception, mais qui renvoie davantage
à l’état du sujet percevant qu’à celui de l’objet perçu. Toute
impression est subjective ; c’est sa façon à elle d’être vraie, ou de pouvoir l'être.
Ainsi dans l’impressionnisme (qui doit son nom, d’abord péjoratif, à un
tableau fameux de Monet, intitulé {\it Impression, soleil levant}) : il s’agit de peindre
non ce qu’on sait ou croit être, mais ce qu’on voit. De là une objectivité nouvelle,
plutôt phénoménologique qu’ontologique. C’est un réalisme à la première personne,
qui cherche moins la vérité des choses que leur apparence fugitive,
%— 299 —
moins l’éternité que l’instant, moins l’absolu que le mouvement ou la
lumière. Par quoi les plus grands, parfois, retrouveront — comme faisait déjà
Corot, comme fera Cézanne, dans ses meilleurs toiles — et la vérité, et l’éternité,
et l’absolu, qui sont le devenir même, dans son impermanence dévoilée ou
retrouvée.

En philosophie, toutefois, le mot renvoie moins à l'esthétique qu’à la
théorie de la connaissance, spécialement dans sa version empiriste et sceptique.
Les impressions, écrit Hume, sont « les perceptions qui pénètrent en nous avec
le plus de force et de violence » (par différence avec les idées, qui sont comme
les images effacées ou affaiblies des impressions dans nos pensées) ; « et sous ce
nom, ajoute-t-il, je comprends toutes nos sensations, passions et émotions,
telles qu’elles font leur première apparition dans l’âme » ({\it Traité...}, I, I, 1). Il en
résulte qu’on ne connaît que des impressions ou des idées, sans pouvoir jamais
les comparer à quelque modèle original qui serait l’objet même (puisqu'on ne
pourrait connaître celui-ci que par l'intermédiaire d’une impression). C’est où
l’empirisme mène au scepticisme.

IMPULSIF Celui qui ne peut résister à ses impulsions : elles sont trop fortes
pour lui ; il est trop faible pour elles.

IMPULSION Un mouvement irraisonné. À la raison de le comprendre (il
est irraisonné, point irrationnel : il a des causes) ; à la volonté
de le contrôler, s’il le faut, ou de l’utiliser, si elle le peut.

INCERTITUDE Est incertain tout ce dont on peut ou doit douter. C’est-à-dire
tout ? Oui, en un sens, puisqu'il se peut que je rêve,
que nous soyons tous fous, ou qu’un Dieu tout-puissant prenne un malin
plaisir à nous tromper toujours. Il n’en serait pas moins vrai que j’existe ?
Cette évidence suppose la validité de notre raison, qui est sans preuve (puisque
toute preuve la suppose), et n’en est donc pas une. Au reste, quand bien même
on admettrait qu’une erreur suppose quelque chose qui se trompe, cela ne saurait
prouver, tout au plus, que l’existence de... quelque chose. De [à à prétendre
que cette chose est moi. Qui sait si je ne suis pas le rêve d’un autre, ou
bien un fou qui se prend pour André Comte-Sponville, ou encore un cerveau
dans une cuve, qu’un expérimentateur de génie — ou un technicien médiocre
dans dix mille ans — programmerait en permanence, à l’aide d’électrodes et
d’ordinateurs, pour qu’il se croie philosophe et en train, par exemple, d’écrire
%— 300 —
une définition de l’incertitude.. Cela est improbable ? Sans doute. Par quoi il
n’est pas certain, comme disait Pascal, que tout soit incertain. Mais cela ne fait
qu’une incertitude de plus.

Toutefois ce doute, pour légitime qu’il soit en toute rigueur, reste
métaphysique : nous serons presque tous portés, comme le Descartes de la
sixième Méditation, à le trouver quelque peu « hyperbolique et ridicule ». Aussi
ne parlera-t-on d’incertitude, en un sens plus restreint, que pour ce qui peut
être faux, quand bien même nos sens et notre raison seraient supposés à peu
près fiables. L’incertain, c’est alors ce dont on peut ou doit douter, non en
toute rigueur ou dans l’absolu, mais dans les conditions ordinaires de notre vie
et de notre pensée — ce qui est {\it particulièrement} douteux. Par exemple que
Napoléon ait été assassiné est incertain ; qu’il soit mort ne l’est pas. L'existence
d’une vie extra-terrestre est incertaine ; celle d’une vie sur Terre ne l’est pas.
Que nous ayons une Âme immatérielle est incertain ; que nous ayons un corps
matériel ne l’est pas. Cela ne prouve pas que les sceptiques aient tort, mais simplement
qu’on n’a pas besoin d’être dogmatique pour faire une différence entre
ce qui est incertain, en ce sens restreint, et ce qui ne l’est pas. Hume, quand il
jouait au trictrac, ne doutait pas de son jeu.

INCERTITUDE (RELATIONS D’—) C’est une espèce de principe, qu’on
appelle parfois « principe d’indétermination »,
et qu’on doit à Heisenberg. Ce dernier a montré que les conditions
de l'observation modifiant, à l'échelle quantique, ce qu’on veut observer (en
éclairant une particule, on modifie sa trajectoire), il n’est pas possible de déterminer
à la fois la position et la vitesse d’une particule, voire qu’il n’est pas possible
de leur attribuer à la fois l’une et l’autre de ces deux caractéristiques. On
en conclut parfois que l'esprit humain est voué à l’échec, qu’il n’y a pas de
vérité, que l’idée même d’une connaissance scientifique s’écroule... C’est bien
sûr un contresens. La physique quantique est au contraire l’une des plus formidables
victoires de l'esprit humain, l’un des principaux progrès scientifiques de
tous les temps, enfin l’une des plus certaines (au sens restreint ci-dessus défini)
de nos théories. Tant pis pour les sophistes. Tant mieux pour les physiciens et
les rationalistes.

INCLINATION C’est un penchant durable et plaisant, qui séduit davantage
qu’il ne contraint. On peut résister à ses inclinations (c’est
ce qui les distingue des compulsions) ; mais il est plus sage, quand elles ne sont
%— 301 —
pas déshonorantes, de s’y abandonner de temps en temps : cela évitera de les
transformer en obsessions ou en regrets.

INCONDITIONNÉ Le mot parle de lui-même : est inconditionné ce qui ne
dépend d’aucune condition. C’est un autre nom pour
l'absolu théorique. Il est par nature inconnaissable. On ne pourrait en effet le
connaître, montre Kant (après Montaigne et Hume), qu’en le soumettant aux
{\it conditions} de nos sens et de notre esprit. Mais alors ce ne serait plus
linconditionné : ce ne serait que le réel, qui est l’ensemble indéfini de toutes les
conditions. On remarquera pourtant que cet ensemble de toutes les conditions
est lui-même nécessairement inconditionné : on ne peut pas davantage
renoncer à le penser que parvenir à le connaître.

INCONDITIONNEL Un de mes fils, il devait avoir sept ou huit ans, me
posa un jour la question suivante : « Qu'est-ce que je
pourrais faire, qui ferait que tu ne m'aimerais plus ? » Je n’ai pas trouvé la
réponse, ou plutôt je n’ai pu répondre que {\it « rien »}. Cela m’a surpris moi-même.
C’est la première fois que je comprenais ce qu’est un amour inconditionnel.
Je ne l'ai vécu, faut-il le préciser, qu'avec mes enfants. Mais cela m’en
a plus appris sur l’amour que tous les livres.

Est {\it inconditionnel} ce qui ne dépend d’aucune condition, mais dans l’ordre
pratique ou affectif plutôt que théorique : ce qui s'impose absolument au cœur
ou à la volonté, non parce que cela existerait de façon inconditionnée (hélas,
nos enfants dépendent de tellement de choses !), mais parce que nous ne saurions
vivre autrement. C’est ce que j'appelle l’absolu pratique : ce que nous
voulons ou aimons sans réserve et sans condition (de façon, dirait-on
aujourd’hui, « non négociable »), au point de lui sacrifier, si nécessaire, tout le
reste (du moins tout ce qui, dans le reste, n’est pas inconditionnel).

L’inconditionnel n’est donc pas nécessairement inconditionné, et même,
pour le matérialiste que je suis, il ne l’est jamais. Par exemple le refus du
racisme : qu'il y ait là une valeur inconditionnelle, cela n'empêche pas qu’elle
n’apparaisse que sous certaines conditions. C’est ce qui distingue la morale de
la religion : l’absolu pratique, pour l’athée, n'existe que {\it relativement à nous}.

INCONSCIENT Comme adjectif, c’est tout ce qui n’est pas conscient : par
exemple la circulation du sang ou les échanges électriques
entre les neurones sont des processus inconscients, comme la quasi-totalité de
%— 302 —
notre fonctionnement organique. « On ne sait pas ce que peut le corps », disait
Spinoza. C’est que l’essentiel de ce qu’il peut est inconscient.

Comme substantif, c’est tout ce qui {\it pourrait} être conscient, en droit, mais
qui ne {\it peut} l'être, en fait : le refoulement et la résistance s’y opposent. L’inconscient
est alors un inconscient {\it psychique}, comme dit Freud, et c’est ce paradoxe
qui le définit : c’est comme un esprit sans esprit, une pensée sans pensée, un
sujet sans sujet. Impossible ? Ce n’est pas sûr. Il se pourrait que l’inconscient
soit la vérité de l'esprit, dont la conscience ne serait que la pointe ultime, toujours
menacée, ou le sommet, toujours à conquérir. Si la pensée se pensait soi,
elle serait Dieu. L’inconscient est ce qui nous en sépare.

On évitera donc de l’adorer, et même d’y croire tout à fait. L’inconscient
non plus n’est pas Dieu. La psychanalyse, quand on veut en faire une religion,
n’est qu’une superstition comme une autre.

INDÉFINI Ce qui n’a pas de définition ou de fin déterminées. C’est le cas,
spécialement, des termes qui ne sont que la négation d’un autre.
Par exemple, précise Aristote, « non-homme est seulement un nom indéfini,
car il appartient pareillement à n'importe quoi, à ce qui est et à ce qui n'est
pas » ({\it De l'interprétation}, 2). Un chat, une racine carrée, un dahu ou Dieu font
partie de l’ensemble de tout ce qui n’est pas un homme ; mais cela ne nous dit
pas ce qu'est cet ensemble (sinon négativement) ni ne permet de lui fixer une
limite (cet ensemble, incluant par exemple la suite des nombres, dont aucun
n’est un homme, est bien sûr infini).

On évitera pourtant de confondre l’{\it indéfini} et l'{\it infini}. Si on laisse de côté
la question des termes négatifs, l’indéfini occupe une espèce d’entre-deux entre
le fini et l'infini. L’infini est ce qui n’a pas de limite. L’indéfini, ce dont la
limite est indéterminée ou indéterminable. Par exemple la suite des nombres
entiers est infinie ; l’histoire de l'humanité, indéfinie. L'ensemble des vérités
possibles est infini ; le progrès des connaissances, indéfini.

On notera que Descartes appelle parfois {\it indéfini} ce qui n’est infini que
d’un certain point de vue ou dans un certain ordre, et n’appelle {\it infini} que ce
qui n’a aucune limite, de quelque point de vue ou en quelque ordre que ce soit.
En ce sens, Dieu seul est infini, explique-t-il ; l’étendue des espaces imaginaires
ou la multitude des nombres ne sont qu’indéfinis. Ce sens est à connaître, mais
point, me semble-t-il, à utiliser.

INDÉTERMINISME Toute pensée qui nie la validité universelle du déterminisme.
Les partisans de l’indéterminisme soutiennent
%— 303 —
qu’il existe des phénomènes absolument indéterminés, autrement dit sans
causes nécessaires et suffisantes : par exemple un acte libre, chez Sartre, ou
l’effectuation spatio-temporelle du clinamen chez Lucrèce (le clinamen a bien
une cause, qui est l’atome, mais cette cause agit en un temps et en un lieu que
rien ne détermine). On y voit souvent une condition de la liberté. Mais cette
condition n’est pas suffisante. À supposer que les particules qui constituent
mon cerveau soient indéterminées, au niveau quantique, elles n’en seraient pas
moins déterminantes, au niveau neurobiologique, où plutôt elles le seraient
encore plus (si elles sont indéterminées, il est exclu que je puisse les gouverner,
non qu’elles me gouvernent). Et cette condition n’est pas non plus nécessaire.
Une autre liberté est possible, montre Spinoza, qui ne serait pas l’absence de
détermination, mais une détermination propre et par soi ({\it Éthique}, I, déf. 7:
« Cette chose est dite libre qui existe par la seule nécessité de sa nature et est
{\it déterminée par soi seule à agir} »). Il n’y a que Dieu, en ce sens, qui soit absolument
libre. Mais la raison l’est davantage en nous que la folie, nos actions
davantage que nos passions, nos vertus davantage que nos vices. Ce n’est plus
indéterminisme, mais indépendance ou autonomie. Qu'elle ne soit jamais
complète, c’est une raison pour l’accroître, non pour y renoncer.

Ainsi l’indéterminisme relève de la physique, non de la morale. Reste à
savoir s’il exprime une dimension du réel (des événements absolument indéterminés)
ou seulement une limite de nos connaissances (des événements indéterminables).
Il me semble que le deuxième terme de l’alternative, qui est avéré,
interdit d’exclure le premier tout autant que de l’ériger en certitude.

INDICE Un signe fondé sur un rapport de causalité : c’est un fait perceptible
qui renvoie à un autre, ordinairement imperceptible, qu’il
implique ou annonce, au point que nous utilisons celui-là comme le signe de
celui-ci. Un symptôme est l'indice d’une maladie, comme ces gros nuages noirs
peuvent être l’indice d’un orage prochain. Pourtant ni la fièvre ni les nuages ne
veulent rien dire : c’est nous qui les faisons parler en les interprétant. Un indice
n’est donc un signe que pour nous : il ne veut rien dire ; c’est nous qui le faisons
parler. Disons que c’est un fait susceptible d’une interprétation : un fait
significatif, mais sans volonté de signification.

INDICIBLE Ce qui ne peut être dit, parce qu’il excéderait tout discours
possible. On pense à la dernière formule, si fameuse, du {\it Tractatus}
de Wittgenstein : « Ce dont on ne peut parler, il faut le taire. » Mais pourquoi
ce {\it « il faut »}, si l’on ne peut ? À quoi bon interdire ce dont nul n’est
%— 304 —
capable ? C’est qu’en vérité il n’y a pas d’indicible : tout peut être dit, bien ou
mal, mais il arrive que le silence, en effet, vaille mieux.

Dieu, par exemple, serait indicible. Les mystiques pourtant n’ont cessé d’en
parler, souvent fort bien, comme aussi, avec d’autres mots, les philosophes et
les théologiens. Leur discours reste inadéquat à son objet, qui l’excède de toute
part ? Sans doute. Mais c’est le cas aussi de l’univers, et de tout ce qui s’y
trouve. Essayez de dire adéquatement un caillou : l’essentiel nécessairement
échappe, qui est la différence entre le caillou et ce qui en est dit — qui est donc,
très exactement, le caillou lui-même. Est-ce à dire que le caillou soit indicible ?
Bien sûr que non, puisque vous pouvez en parler avec pertinence. Simplement
ce caillou, même dicible, même dit, est autre chose qu’un discours. C’est ce que
j'appelle non l’indicible, mais le silence. La différence entre les deux ? Le
silence, lui, peut être dit. C’est même le cas dans la plupart de nos discours. Le
métalangage, malgré nos bavards et nos sophistes, reste l’exception : le plus
souvent, et c’est heureux, on parle d’autre chose que du langage. Par exemple
ces deux vers d’Angelus Silesius : « La rose est sans pourquoi, fleurit parce
qu’elle fleurit, / N’a souci d’elle-même, ne désire être vue. » Cela parle, mais de
quelque chose qui ne parle pas. Non que la rose soit indicible, mais parce
qu’elle est silencieuse. Non qu’elle soit ineffable, mais parce qu’elle est {\it ineffante}
(si l’on m’autorise ce décalque de l’{\it infans} latin : celui qui ne parle pas). Royauté
du silence : royauté d’un enfant.

C’est donc Hegel, sur ce point, qui a raison : « Ce qu’on appelle l’indicible
n’est pas autre chose que le non-vrai, le non-rationnel, le seulement visé »
({\it Phénoménologie de l'esprit}, « La certitude sensible », III). Non que le vrai soit un
discours, mais parce que toute vérité peut être dite. Elle n’en restera pas moins
silencieuse pour autant : si le réel n’est pas un discours, comment un discours,
même vrai, pourrait-il le contenir tout entier ou le dissoudre ? Toute vérité
peut être dite, mais aucun discours n’est la vérité. Ce dont on peut parler n’en
continue pas moins à se taire. Royauté d’un enfant : royauté du silence.

INDIFFÉRENCE Ce n’est pas l’absence de différences (l'identité), mais le
refus ou l'incapacité d’en faire qui soient affectivement
significatives : l'absence non de différences, mais de préférences, de hiérarchie
ou même de normativité. Pour l’indifférent, tout n’est pas le même (tout n’est
pas identique), mais tout {\it revient au même}, comme on dit, tout est égal, ce qui
signifie que les différences, même effectives, ne sont jamais des différences de
valeur. C’est la version pyrrhonienne de l’ataraxie ({\it adiaphoria} : l'indifférence),
telle qu’elle résulte du fameux {\it ou mallon} : une chose n’est pas plus ({\it ou mallon})
qu’elle n’est pas, ni plutôt ceci que cela, ni ne vaut davantage ou moins qu’une
%— 305 —
autre. Non, répétons-le, que toutes les apparences soient identiques (Pyrrhon,
qui en vendait parfois sur les marchés, devait bien faire la différence entre un
cochon et un poulet), mais parce que aucune n’est fondée en vérité ni en valeur :
parce que tout se vaut, qui ne vaut rien. On objectera que Pyrrhon, sur les marchés,
devait bien évaluer différemment ce qu’il vendait (un cochon n’a pas le
même prix qu’un poulet). Mais c'était le problème des clients, non le sien. Être
indifférent, ce n’est pas être aveugle ou stupide. C’est être neutre et serein.

Le peut-on? Et pourquoi le faudrait-il, si la sérénité elle-même est
indifférente ? Pyrrhon est l’un des rares philosophes qui ait été vraiment nihiliste,
et qui donne envie de l’être. Mais cette envie le réfute ou nous empêche
de le suivre. Si rien ne vaut, le nihilisme ne vaut rien.

Il y a pourtant de bonnes indifférences, mais partielles ou ciblées : celles qui
refusent de considérer ce qui ne doit pas l’être ou d’attacher de l'importance à
ce qui n’en a pas. La justice, par exemple, ne va guère sans impartialité, qui est
comme une indifférence de principe. Et la charité se distingue de l’amitié en ce
qu’elle est un amour indifférencié (ce que Fénelon appelait la « sainte
indifférence »). Mais ni la charité ni la justice ne supposent pour cela qu’on soit
indifférent à tout ou à elles-mêmes : elles cesseraient autrement de valoir. Être
impartial, ce n’est pas être indifférent à la justice. C’est être indifférent à tout le
reste. Ainsi l’indifférence ne vaut, c’est son paradoxe, qu’à la condition d’être
différenciée. Il arrive même qu’elle soit une vertu: rester indifférent au
médiocre ou au dérisoire, ce n’est pas nihilisme, c’est grandeur d’âme.

INDIFFÉRENCE (LIBERTÉ D’) Le libre arbitre, mais en tant qu’il ne
serait soumis à aucune inclination ou
préférence : c’est la liberté de l’âne de Buridan, sil en a une. Descartes n’y
voyait que « le plus bas degré de la liberté, qui fait plutôt paraître un défaut
dans la connaissance qu’une perfection dans la volonté ; car si je connaissais
toujours clairement ce qui est vrai et ce qui est bon, je ne serais jamais en peine
de délibérer quel jugement et quel choix je devrais faire, et ainsi je serais entièrement
libre sans jamais être indifférent ({\it Méditations}, IV ; voir aussi la {\it Lettre au
Père Mesland} du 9 février 1645). Celui, à l'inverse, qui serait parfaitement
indifférent, que lui importerait d’être libre ? Et quel usage pourrait-il faire de sa
liberté ?

INDISCERNABLES (PRINCIPE DES —) C’est un principe leibnizien, qui

stipule que tout être réel est
intrinsèquement différent de tous les autres, autrement dit qu’il n’existe pas
%— 306 —
d’êtres absolument identiques ou indiscernables (qui ne se distingueraient que
numériquement ou par des données extrinsèques, comme leur position dans
l’espace et le temps). Deux gouttes d’eau, deux feuilles du même arbre ou deux
cachets d’aspirine ne semblent indiscernables que parce que nous les observons
mal : mettez-les sous un microscope, il ne sera plus possible de les confondre.
Le principe, selon Leibniz, est sans exception : tout être est unique ; l’infinie
multiplicité du réel n’est constituée que de singularités absolues (les monades).

Et si l’on n’est pas leibnizien ? Alors il reste que tout être est différent de
tous les autres — y compris de lui-même à un autre moment. C’est pourquoi il
n’y a pas d'êtres : il n’y a que des événements. On ne se baigne jamais deux fois
dans le même fleuve, ni même une seule fois. On n’est plus chez Leibniz, mais
chez Héraclite ou Montaigne.

INDIVIDU Un être vivant quelconque, dans une espèce quelconque, mais
en tant qu'il est différent de tous les autres. Rien de plus banal
qu’un individu, et rien de plus singulier : c’est la banalité d’être soi.

Se dit spécialement d’un être humain, mais considéré plutôt comme objet
que comme sujet, plutôt comme résultat que comme principe, plutôt comme
élément (dans un ensemble donné : une espèce, une société, une classe.) que
comme personne. N'importe qui, donc, en tant qu’il est quelqu'un.

Indivisible ? C’est ce que suggère l’étymologie : {\it individuum}, en latin, traduit
le grec {\it atomon}. Cela ne prouve pas plus dans un cas que dans l’autre (nous
savons aujourd’hui que les atomes sont sécables), mais rencontre pourtant
l'expérience commune. Non qu’on ne puisse diviser un être vivant ; mais parce
que ce qu’il y a d’individuel en lui n’est pas divisé par là. Un cul-de-jatte n’est
pas une moitié d’individu.

INDIVIDUALISME C’est mettre l'individu plus haut que l’espèce ou que la
société, voire plus haut que tout (par exemple plus
haut que Dieu ou la justice). Mais quel individu ? S'il ne s’agit que du moi,
l’individualisme n’est qu’un autre nom, moins péjoratif, pour dire l’égoïsme.
S’il s’agit de tout individu, ou plutôt de tout être humain, ce n’est qu’un autre
nom, moins emphatique, pour dire l’humanisme. Entre ces deux pôles, le mot
ne cesse de fluctuer ; il doit l'essentiel de son succès au flou qui en résulte.

INDUCTION C'est un type de raisonnement, qu’on définit classiquement
par le passage du particulier au général, ou des faits à la loi.
%— 307 —
S’oppose en cela à la déduction, qui va du général au particulier ou d’un principe
à ses conséquences.

On comprend que l’induction, qui est amplifiante, pose davantage de problèmes
que la déduction, qui est plutôt réductrice. Une fois admis que tous les
hommes sont mortels, il n’est pas douteux que cet homme-ci le soit : le singulier
n’est qu’un sous-ensemble de l’universel. Mais combien faut-il avoir vu
d'hommes mourir pour savoir qu’aucun n’est immortel ? En pratique ou psychologiquement,
beaucoup moins qu’il n’en meurt en effet. Mais d’un point de
vue logique ? Comment passer d’énoncés singuliers en nombre toujours fini
(« Tel homme est mort, et tel autre, et tel autre, et tel autre... ») à un énoncé
universel (« {\it Tous} les hommes sont mortels ») ? C’est ce qu’on appelle depuis
Hume le problème de l’induction. Combien faut-il avoir vu de cygnes blancs
pour savoir qu'ils le sont tous ? Combien de corps en chute libre faut-il avoir
observés pour savoir que, dans le vide, ils tombent tous à la même vitesse ? Il
faudrait avoir vu tous les cygnes et mesuré toutes les chutes, ce qui est bien sûr
impossible, ou bien supposer, au bout d’une certain nombre d’observations,
que les cas à venir ressembleront à ceux déjà observés. Mais cette dernière supposition
— que le futur ressemblera au passé — ne va pas de soi et ne peut être
démontrée ni par déduction (puisqu'il s’agit d’une question de fait) ni par
induction (puisque toute induction le suppose). Toute induction aboutit donc
à une conclusion qui excède ses capacités logiques : elle est formellement non
valide et empiriquement douteuse. La confiance que nous accordons à ce type
de raisonnement doit beaucoup plus à l’habitude, conclut Hume, qu’à la
logique ({\it Traité}, I, 3 ; {\it Enquête}, IV). Or, s'agissant de la connaissance du monde,
c’est ordinairement l'induction qui fournit à la déduction les principes généraux
dont celle-ci déduit les conséquences : si l'induction est douteuse, toute
déduction appliquée à l’expérience l’est aussi. À la gloire de Hume et du scepticisme.

À ce problème de l'induction, je ne connais qu’une seule solution satisfaisante.
C’est celle de Popper, qui résout le problème, de façon à la fois négative
et radicale, en montrant qu’{\it il n'y à pas d'induction logiquement valide}. Comment
les sciences expérimentales sont-elles pourtant possibles ? Par déduction :
on pose un principe (une hypothèse, une loi, une théorie...), dont on déduit
des conséquences (par exemple sous la forme de prévisions). Si ces conséquences
sont réfutées par l’expérience, le principe est faux. Si les conséquences
ne sont pas réfutées par l'expérience, ou tant qu’elles ne le sont pas, le principe
est considéré comme possiblement vrai : il a survécu, au moins provisoirement,
à l'épreuve du réel. Il en résulte que « seule la fausseté d’une théorie est susceptible
d’être inférée des données empiriques, et que cette sorte d’inférence est
%— 308 —
purement déductive » ({\it Conjectures et réfutations}, I, 9 ; voir aussi {\it La logique de la
découverte scientifique}, I).

L’argumentation de Popper est fondée, comme il le remarque lui-même,
sur « une {\it asymétrie} entre la vérifiabilité et la falsifiabilité, asymétrie qui résulte
de la forme logique des énoncés universels » : on ne peut conclure de la vérité
d’énoncés singuliers à celle d’un énoncé universel (dix mille cygnes blancs ne
prouvent pas qu’ils le soient tous), mais on peut conclure de leur vérité à la
{\it fausseté} d’énoncés universels (il suffit d’un cygne noir pour prouver qu’ils ne
sont pas tous blancs). « Cette manière de prouver la fausseté d’énoncés universels,
conclut Popper, constitue la seule espèce d’inférence strictement
déductive qui procède, pour ainsi dire, dans la “direction inductive”, c’est-à-dire
qui va des énoncés singuliers aux énoncés universels. » Ainsi il n’y a pas
de logique inductive, ni d’induction logiquement probante, mais il y a ce
qu’on pourrait appeler un {\it effet d'induction} (on passe bien du particulier au
général ou à l’universel) qui permet d’énoncer des lois scientifiques — par
exemple celle de la chute des corps — qui sont à la fois possiblement vraies et
empiriquement testables. Les sciences et l’action n’en demandent pas davantage.

INEFFABLE Synonyme à peu près d’indicible, en plus suave ou plus mystérieux.
L’indicible le serait plutôt par excès de force, de plénitude
ou de simplicité ; l’ineffable, par excès de délicatesse, de finesse, de subtilité...
Enfin, l’ineffable ne se dit guère que positivement (alors qu'on peut
parler d’une souffrance indicible, d’un malheur indicible). Ces nuances, sans
être tout à fait indicibles, restent toutefois quelque peu ineffables.

INERTIE C’est d’abord et paradoxalement une force : la force qu’a un corps
de persévérer dans son mouvement ou son repos. Le principe
d'inertie stipule en effet qu’un objet matériel conserve de lui-même son état de
repos ou de mouvement rectiligne uniforme : qu’il ne peut être mû (s’il est en
repos), dévié ou freiné (s’il est en mouvement) que par une force extérieure.
L’inertie n’est donc pas l’immobilité (un corps en mouvement rectiligne uniforme
ne manifeste pas moins d’inertie qu’un corps immobile), ni même l’inaction
(un corps inerte peut produire quelque effet : par exemple s’il me tombe
sur le pied). L’inertie, c’est l'incapacité à changer de soi-même son mouvement,
ou à se changer soi. C’est pourquoi le mot, appliqué à un être humain, est toujours
péjoratif : se subir, c’est déchoir.
%— 309 —
INESPOIR L'absence d’espoir, mais considérée comme un état neutre et
originel : « non le deuil de l'espoir, disait Mounier, mais son
constat de défaut ». C’est ce qui le distingue du désespoir, ou qui l’en distinguerait
s’il était possible de se passer d’espoir sans en faire le deuil. Mais qui le
peut ? L'espoir est premier : il faut le perdre, d’abord, pour apprendre à s’en
passer. Ainsi l’inespoir n’est qu’un désespoir qui serait parfaitement accompli.
Non un point de départ, mais un point d’arrivée. Non un travail, mais un
repos. C’est le propre des sages et des dieux. Pour tous les autres, ce n’est qu’un
mensonge de plus.

INFÉRENCE C’est passer d’une proposition tenue pour vraie à une autre

qu'on juge en conséquence l'être aussi, en vertu d’un lien
nécessaire ou supposé tel. Ce passage peut être inductif (si on passe de faits particuliers
à une conclusion plus générale) ou déductif (si on passe d’une proposition
à l’une de ses conséquences). On considère généralement que l’inférence
inductive ne peut faire passer que du vrai au probable, quand la déduction conclut
du vrai au vrai. On n’en tirera pas trop vite un argument contre les faits. Il
suffit d’un seul pour réfuter — par inférence déductive, en l’occurrence sous la
forme du {\it modus tollens} — la théorie la plus ambitieuse. C’est ce que Karl Popper
appelle la falsification ({\it La logique de la découverte scientifique}, chap. X, III et IV).

INFINI  L’étymologie est transparente : l'infini, c’est ce qui est sans limite,
sans borne ({\it finis}), sans fin. On ne le confondra pas avec l’indéfini,
qui est sans limite connue ou connaissable.

Les illustrations les plus commodes sont mathématiques. Chacun comprend
que la suite des nombres est infinie, puisqu'on peut toujours ajouter un
nombre quelconque à celui qui serait supposé être le plus grand. On remarquera
qu’une partie d’un ensemble infini n’est pas nécessairement infinie (par
exemple les entiers situés entre 3 et 12 sont en nombre fini), mais peut l’être (la
suite des nombres pairs est aussi infinie que celle des entiers, dont elle constitue
pourtant une partie). Ainsi un ensemble infini a cette caractéristique exceptionnelle
— qui peut servir, mathématiquement, à le définir — qu’il peut être mis en
bijection (en correspondance biunivoque) avec l’un au moins de ses sous-ensembles
stricts : tout entier peut être mis en relation biunivoque avec son
carré, c'était l’exemple pris par Galilée, quand bien même la série infinie des
carrés parfaits n’est qu’un sous-ensemble de la série des entiers. Il en résulte que
le tout, s'agissant de l'infini, n’est pas nécessairement plus grand que telle de ses
parties (puisque celle-ci peut être aussi infinie). Ce qui permet de définir, par
%— 310 —
différence, la finitude : est fini tout ensemble qui est nécessairement plus grand
que l’un quelconque de ses sous-ensembles stricts (c’est-à-dire autres que lui-même).

Un exemple non mathématique ? On pense bien sûr à Dieu, dont Descartes
disait qu'il était seul infini proprement, n’ayant aucune borne d'aucune
sorte. Si on lui applique l'observation qui précède, on peut en conclure qu’il ne
serait pas nécessairement plus grand que telle de ses parties, et par exemple
qu’un Dieu trinitaire ne serait pas plus grand que chacune des trois Personnes
qui constituent l’unité de son essence (du moins si chacune est supposée
infinie). Cela toutefois ne prouve pas qu’il existe, ni qu’il soit trois.

Quant à trouver un exemple empirique, c’est ce qu’on ne peut : l’expérience
n’a accès qu’au fini, ou à l’indéfini. On pensera pourtant à la fameuse
formule de Pascal, dont on croit parfois qu’elle s’applique à Dieu, ce n’est pas
un hasard, mais que Pascal, qui reprend d’ailleurs une métaphore traditionnelle,
n'utilise qu’à propos de l’univers : « une sphère infinie dont le centre est
partout, la circonférence nulle part » ({\it Pensées}, 199-72). Mais de cette infinité,
d’ailleurs douteuse, nous n’avons qu’une idée, point une expérience.

INJURE C'est une dénonciation haineuse, qui s'adresse, c’est son paradoxe,
à celui-là même qu’elle dénonce. Dans quel but ? D'abord le plaisir
qu’on y trouve, qui est parfois d’hygiène. Cela vaut mieux qu’un meurtre ou
qu’un ulcère de l’estomac. Aussi une certaine exigence de vérité, voire de justice,
qu’on imposerait par la parole à celui qu’on insulte, comme s’il était
besoin de lui apprendre ce qu’il est, ou ce qu’on en pense, de le démasquer à
ses propres yeux, enfin de l’obliger à se voir, au moins une fois, dans le miroir
de notre mépris. Regarde-toi dans mon regard, juge-toi dans mon jugement :
tu es ce que je dis ! C’est comme une vérité performative, et l’on se doute que
la logique n’y trouve pas son compte. L’injure peut être fondée ou non (elle
peut être médisance ou calomnie) ; elle est toujours injuste, comme l'indique
l’étymologie, par le refus de comprendre et la volonté de blesser. Mais cette
injustice en vient corriger une autre, qui nous semble, au moins dans l'instant,
plus grave ou plus insupportable. La logique est celle d’un châtiment ; injurier,
c’est se faire juge, procureur et bourreau. La conjonction de ces trois rôles suffit
à mettre l’injure à sa place : ce n’est pas justice, mais colère.

INNÉ Ce qui est donné ou programmé dès la naissance. On ne confondra
pas l’inné (qui s’oppose à l’acquis) avec l’{\it a priori} (qui s’oppose à l'empirique).
L’inné relève moins du transcendantal que de l’immanental (voir ces mots).
%— 311 —
Et l’on évitera de dire trop vite que rien n’est inné en l’homme : ce serait faire
comme si nous n'avions pas de corps (lequel est par définition inné) ou comme s’il
était quantité négligeable. L'expérience et la génétique prouvent le contraire.

INNÉISME Ce n’est pas croire qu’il y a de l’inné en l’homme, ce qui n’est
guère contestable, mais que l’innéité ne se réduit pas au corps :
que certaines idées ou conduites sont inscrites en nous dès la naissance. C’est
croire en l’innéité non du corps seulement mais de l'esprit.

Pour un matérialiste, un certain innéisme va de soi : si le corps et l'esprit
sont une seule et même chose, l’innéité du corps entraîne celle de l’esprit. Le
cerveau, par définition, est inné. Cela n'empêche pas qu’il se développe et se
construise aussi {\it après} la naissance. L’inné ne s’oppose à l’acquis que pour autant
qu'il le permet, comme l’acquis ne s'oppose à l’inné que pour autant qu’il le
suppose. Par exemple le langage, comme faculté, est inné ; mais toute langue
est acquise. L'erreur de Descartes est d’avoir cru que l’innéité de l'esprit était
celle d’idées toutes faites, alors qu’il s’agit plutôt de fonctions (indissociablement
neuronales et logiques) et d’idées à faire.

INQUIÉTUDE C’est un rapport présent à l’avenir, en tant que l’avenir
vient troubler le présent. Oscille entre le souci ({\it « Que faire ? »})
et la crainte ({\it « Comment y échapper ? »}). C’est pourquoi on n’échappe pas à l’inquiétude,
ou c’est pourquoi, plutôt, on n’y échappe que dans les rares moments où l’on
vit pleinement au présent : dans la quiétude de la contemplation ou de l’action.

INQUISITION Le mot signifie d’abord recherche ou enquête. Mais avec
une majuscule, c’est une enquête très particulière, plutôt
policière que théorique : les inquisiteurs ne cherchent plus la vérité, puisqu'ils
sont censés la connaître, mais des coupables. Qu’on ait pu torturer et massacrer
au nom des Évangiles en dit long sur la bêtise humaine, et sur les dangers du
fanatisme. C’est vouloir éclairer la splendeur de la vérité, comme dit Jean-Paul II,
par des bûchers. {\it Veritatis terror !}

INSENSÉ Qui est contraire au bon sens ou à la raison.
On remarquera que ce qui est insensé est rarement insignifiant.
La folie est un état grave, qui peut être riche de sens ; et le bavardage le plus
insignifiant est rarement insensé.

%— 312 —
INSIGNIFIANT Qui n’a pas de sens, pas de valeur ou pas d'importance.
Cette polysémie en dit long sur les humains, qui n’accordent
ordinairement d'intérêt qu’à ce qui signifie. Le doigt montre la lune. Ils
regardent le doigt.

INSISTANCE C’est un mot que j'ai pris l’habitude d’opposer à l'existence,
au sens existentialiste du terme, et qui permet de la penser
autrement. Exister, c’est être dehors (hors de soi, hors de tout). Insister, c’est
être {\it dans}, et s’efforcer de s’y maintenir. Dans quoi ? Dans l’être, dans le présent,
dans tout. Comment pourrait-on {\it exister} autrement ? L’insistance, au sens
où je prends le mot, est un autre nom pour dire le {\it conatus}, en tant qu’il n'existe
que dans le temps et l’espace, en tant qu’il est toujours confronté à autre chose,
à quoi il résiste. C’est moins le contraire de l'existence que sa vérité. Nul ne
s'oppose à la nature qu’à la condition d’être dedans ; nul ne se projette dans
l'avenir qu’au présent : exister, c’est insister et résister. L’homme n’est pas un
empire dans un empire, ni un néant dans l’être. Il n’y a pas de transcendance :
il n’y a que l’immanence, que l’impermanence, et le {\it dur désir de durer}. n’y a
que des forces et des rapports de forces. Il n’y a que l’histoire. Il n’y a que l'être,
et l’insistance de l’être.

INSISTANTIALISME Il m'est arrivé d’utiliser ce mot, par jeu et par opposition
à l’existentialisme, pour caractériser ma propre
position ({\it L'être-temps}, VIII). Exister, C’est être {\it dehors} — hors de soi, hors de
tout. Insister, c’est être {\it dans}, et s’efforcer d’y rester : philosophie de l’immanence,
de la puissance, de la résistance (l’{\it energeia} chez Épicure, la {\it tendance} où
la {\it tension} chez les stoïciens, le {\it conatus} chez Spinoza, la volonté de puissance
chez Nietzsche, l'intérêt chez Marx, la pulsion de vie chez Freud...). C’est qu’il
n’y a pas de {\it dehors} absolu, pas de transcendance, pas d’au-delà : il n’y a que le
monde, il n’y a que tout. L’insistantialisme n’est pas un humanisme : c’est une
pensée de l’être, non de l’homme. Il enseigne que l’essence précède l'existence,
ou plutôt que rien n'existe que ce qui est (essence et existence, dans le présent
de l'être, sont bien sûr confondues ; c’est ce que Spinoza appelle {\it essence
actuelle}, qui ne fait qu’un avec la puissance en acte : {\it Éthique}, III, prop. 7 et
dém.) et continue d’être : rien n'{\it existe} que ce qui {\it insiste}. Philosophie non de
l'homme mais de la nature. Non de la transcendance mais de l’immanence.
Non du néant mais de l’être. Non du sujet mais du devenir. Non de la conscience
mais de l’acte. Non du libre arbitre, mais de la nécessité et de la libération.

%— 313 —
INSTANT Ce serait un point de temps : une portion de durée qui ne durerait
pas — non une durée, disait Aristote, mais une limite entre
deux durées. Ce n’est donc qu’une abstraction. Le seul instant réel, c’est le présent,
qui ne cesse de continuer. En quoi est-ce un instant ? En ceci qu'il est
indivisible (que serait un demi-présent ?) et sans durée (combien dure le
présent ? et comment pourrait-il durer sans être composé de passé et
d’avenir ?). C’est l'instant vrai : non une portion de durée qui ne durerait pas,
mais l'acte même de durer, en tant qu’il est indivisible et sans durée. C’est
l'éternité en acte.

INSTINCT Un savoir-faire transmis biologiquement. L'homme en est à
peu près dépourvu : il n’a guère que des pulsions, qu’il faut
éduquer.

INTELLECTUEL C'est celui qui vit de sa pensée, ou pour sa pensée. Il n’a
guère le choix qu’entre une petitesse (penser pour vivre)
et une illusion (vivre pour penser). Il n’y a pas de sot métier, mais pas non plus
de vanité intelligente.

INTELLIGENCE La capacité, plus ou moins grande, de résoudre un problème,
autrement dit de comprendre le complexe ou le
nouveau.

INTELLIGIBLE Au sens large : ce qui peut être compris par l'intelligence
(c’est le contraire d’inintelligible).

Au sens étroit : ce qui ne peut être compris ou visé {\it que} par l’intelligence,
jamais senti par le corps. Ainsi le {\it monde intelligible} chez Platon. C’est le
contraire du matériel ou du sensible.

INTENTION Une volonté présente, mais qui porte sur l’avenir ou sur la fin
qu’on poursuit. C’est le projet de vouloir, ou la visée de la
volonté.

C’est pourquoi on parle d’une {\it morale de l'intention}, pour désigner une
morale, comme celle de Kant, qui mesure la moralité d’une action non à ses
effets mais à la disposition de la volonté qui l’accomplit. L'expression, bizarrement,
%— 314 —
a souvent un sens péjoratif. J'y vois un contresens sur l’intention ou sur
Kant. Une morale de l'intention n’est pas une morale qui se contenterait
d’avoir de bonnes intentions, comme on dit: c’est une morale qui juge la
volonté en acte, non ses suites éventuelles, que personne, quand il agit, ne
connaît tout à fait. Un homme tombe à la mer. L'un des matelots lui jette un
rondin, pour l’assommer ; Le rondin flotte et le sauve. Un autre matelot lui jette
une bouée, pour qu’il s’y agrippe : la bouée lui tombe sur la tête et l’assomme ;
il se noie. Une morale de l'intention est celle qui juge l'attitude du second
marin, aussi néfaste qu’elle s’avère finalement, plus estimable que celle du
premier, fût-elle salutaire. Qui ne voit que c’est la morale même ? Nul n’est
tenu de réussir, ni dispensé d’essayer.

INTENTIONNALITÉ Rien à voir avec le fait d’avoir une intention, au sens
courant du terme. L’intentionnalité, en philosophie,
relève du vocabulaire technique (l’{\it intentio}, chez les scolastiques, est l’application
de l'esprit à son objet), spécialement phénoménologique. « Le mot {\it intentionnalité},
écrit Husserl, ne signifie rien d’autre que cette particularité foncière
et générale qu’a la conscience d’être conscience de quelque chose, de porter, en
sa qualité de {\it cogito}, son {\it cogitatum} en elle-même » ({\it Méditations cartésiennes},
\S 14). La conscience n’est pas une chose ou une substance, qui se suffirait de
soi ; elle est visée, relation à, éclatement vers. « Toute conscience, écrit Husserl,
est conscience {\it de} quelque chose. » Telle est l’intentionnalité. « La conscience,
écrira Sartre, n’a pas de “dedans” ; elle n’est rien que le dehors d’elle-même »
(« Une idée fondamentale de la phénoménologie de Husserl : l’intentionnalité »,
{\it Situations}, I). L’intentionnalité est cette ouverture de la conscience vers
son dehors (y compris vers l’{\it ego}, en tant qu’il est objet pour la conscience).
C’est la seule intériorité vraie : « tout est dehors, écrit Sartre, tout, jusqu’à
nous-mêmes » ({\it ibid.}).

INTÉRÊT Subjectivement, c’est une forme de désir ou de curiosité, et souvent
la conjonction des deux. Mais on peut aussi avoir objectivement
intérêt à quelque chose pour lequel on n’éprouve ni désir ni curiosité. On
peut dire, par exemple, que c’est l'intérêt de l’enfant que d’apprendre sa leçon ;
ou, avec Marx, que les travailleurs ont intérêt à la révolution : cela ne prouve
pas que tous les travailleurs soient révolutionnaires, ni que tous les élèves soient
studieux.
Qu'est-ce alors que cet intérêt prétendument objectif? Non ce qu'on
désire, mais ce qu’on {\it devrait} désirer, si l’on connaissait vraiment son bien et le
%— 315 —
chemin qui y mène. C’est le désir réfléchi ou intelligent : celui qu’on éprouve à
juste titre ou qu’un autre juge qu’on devrait éprouver. Intérêt objectif ? Seulement
pour celui qui le juge tel. Intérêt subjectif, donc, ou projectif. Car enfin
nul n’est tenu d’être marxiste, ni studieux.

La plupart des matérialistes enseignent que c’est l'intérêt qui meut les
hommes : ainsi chez Épicure, chez Hobbes, chez Helvétius (« on obéit toujours
à son intérêt », {\it De l'esprit}, I, 4 et II, 3), chez d’Holbach (« l'intérêt est l’unique
mobile des actions humaines », {\it Système de la nature}, I, 15), enfin chez Freud
(par les principes de plaisir et de réalité) ou chez Marx (« Les individus ne cherchent
que leur intérêt particulier », {{\it Idéologie allemande}, I). C’est vrai aussi (sous
l'appellation de « l’utile propre » : voir {\it Éthique}, IV, prop. 19 à 25) chez Spinoza,
qui n'est pas matérialiste, comme chez Hegel, qui ne l’est pas davantage
ou plutôt qui l’est encore moins : « C’est leur bien propre que peuples et individus
cherchent et obtiennent dans leur agissante vitalité » ; toute la « ruse de la
raison » est de mettre ces intérêts particuliers au service d’une fin générale
({\it Leçons sur la philosophie de l'histoire}, Introd., II). On se trompe évidemment si
l’on y voit une apologie de l’égoïsme. C’est tout le contraire : une tentative
pour le dépasser ou le sublimer. Il est de l'intérêt de chacun, pour tous ces
auteurs, de tendre à l'intérêt général. Qui peut être heureux tout seul ? Qui
peut s’aimer sans s’estimer ? C’est où l’amour de soi mène à l’amour de la justice
ou du prochain. « Soyez égoïstes, dit un jour le dalaï-lama : Aimez-vous les
uns les autres ! »

INTERPRÉTATION C'est chercher ou révéler le sens de quelque chose
(d’un signe, d’un discours, d’une œuvre, d’un événement...).
S'oppose par là à l'{\it explication}, qui donne non le sens mais la cause.
Les deux démarches peuvent bien sûr être légitimes ; mais il ne l’est jamais de
les confondre. Tout a une cause, et certains faits ont un sens. Mais comment
un fait pourrait-il s’expliquer par ce qu’il signifie ? On pense aux actes manqués
ou aux symptômes selon Freud : leur sens (par exemple un désir refoulé) n’est-il
pas aussi leur cause ? Sans doute, mais de deux points de vue différents. Ce
n’est pas parce qu’il signifie quelque chose qu’un lapsus se produit (car bien des
lapsus seraient tout aussi signifiants, qui ne se produisent pas) ; c’est parce qu’il
est produit par autre chose (un désir, une résistance, tel ou tel processus psychique
ou neuronal...) qu’il a un sens. Ainsi l’ordre des signes est soumis à
celui des causes, qui ne signifie rien. C’est en quoi la sexualité est bien, comme
Freud l’a dit lui-même, «le bloc de granit» de la psychanalyse. Tout sens
inconscient y renvoie. Mais elle-même ne signifie rien.

%— 316 —
INTERSUBJECTIVITÉ L'ensemble des relations entre les sujets : leurs
échanges, leurs sentiments mutuels, leurs ébats et
leurs débats, leurs conflits, leurs rapports de forces ou de séduction... Il n’y
aurait pas de sujet autrement. Chacun ne se constitue que dans son rapport aux
autres : on ne se pose qu’en s’opposant, comme disait Hegel, on n’apprend à
aimer qu’en étant aimé d’abord, à penser qu’en comprenant la pensée d'autrui,
etc. Le solipsisme est une idée de philosophe, et c’est une idée creuse. Cela, toutefois,
ne supprime pas la solitude. On n'existe qu’avec les autres, mais ils ne
sauraient exister à notre place.

INTROSPECTION L'observation de soi par soi, mais tournée vers l’intérieur,
comme une autocontemplation de l'esprit. Elle
est impossible à effectuer en toute rigueur (autant vouloir, disait Auguste
Comte, du haut de son balcon, se regarder passer dans la rue) et pourtant
nécessaire : il faut bien apprendre à se connaître ou à se reconnaître. Le {\it je} se
mire dans la conscience ; c’est ce qu’on appelle le {\it moi}. Mais ce n’est jamais
qu’un reflet, qu'une image sans consistance ni véritable profondeur. La
mémoire, le dialogue et l’action nous en apprennent davantage.

INTUITION  {\it Intueri}, en latin, c’est voir ou regarder : l’intuition serait une
vue de l'esprit, avec tout ce que cela suppose d’immédiat,
d’instantané, de simple... et de douteux. Avoir une intuition, c’est sentir ou
pressentir, sans pouvoir démontrer ni prouver. L’intuition se situe en amont du
raisonnement. Mais un esprit totalement dépourvu d’intuition serait aveugle.
Comment pourrait-il raisonner ?

Dans la langue philosophique, la polysémie du mot, qui est à peu près inépuisable,
s'organise autour de trois sens principaux : ceux de Descartes, Kant et
Bergson.

Pour Descartes, l'intuition est « une représentation qui est le fait de l’intelligence
pure et attentive, représentation si facile et si distincte qu’il ne subsiste
aucun doute sur ce que l’on comprend » ({\it Règles pour la direction de l'esprit}, III).
C’est la simple vision intellectuelle du simple : vision évidente de l'évidence.
« Ainsi, chacun peut voir par intuition qu’il existe, qu’il pense, que le triangle
est délimité par trois lignes seulement et la sphère par une seule surface. »
S’oppose en ce sens à la déduction ou au raisonnement, mais aussi les permet :
ce sont comme des chaînes de pensées, dont chaque maillon doit être vu par
intuition pour que la chaîne elle-même soit saisie ou suivie par « un mouvement
continu et ininterrompu de la pensée qui prend de chaque terme une
%— 317 —
intuition claire ». Pour comprendre que 2 et 2 font la même chose que 3 et 1,
explique Descartes, « il faut voir intuitivement, non seulement que 2 et 2 font
4, et que 3 et 1 font 4 également, mais en outre que, de ces deux propositions,
cette troisième-là se conclut nécessairement ». Ainsi l'intuition peut se passer de
raisonnement, non le raisonnement se passer d’intuition.

Chez Kant, l'intuition est aussi une façon immédiate, pour la connaissance,
de se rapporter à un objet quelconque : elle est ce par quoi un objet nous est
donné ({\it C. R. Pure}, « Esthétique transcendantale », \S 1). Mais l’homme ne dispose
d'aucune intuition intellectuelle ou créatrice. Il n’a d’intuition que sensible
et passive : l'intuition est la réceptivité de la sensibilité (elle ne contient
que « la manière dont nous sommes affectés par des objets »), dont les formes
pures sont l’espace et Le temps. S’oppose à la connaissance par concepts, qui
n'est pas intuitive mais discursive : « Des pensées sans contenu sont vides ; des
intuitions sans concepts sont aveugles » ({\it C. R. Pure}, « Logique transcendantale »,
Introd., I).

Chez Bergson, l'intuition est surtout une méthode : c’est retrouver les vrais
problèmes et les vraies différences (celles qui sont de nature, non de degré) en
se transportant «à l’intérieur d’un objet, pour coïncider avec ce qu'il a
d’unique et par conséquent d’inexprimable », spécialement avec sa durée
propre et son essentielle mobilité. Penser intuitivement, c’est penser en durée
et en mouvement. L’intuition s'oppose en ce sens à l’analyse (qui exprime une
chose «en fonction de ce qui n’est pas elle ») ou à l'intelligence (qui part de
l'immobile), comme la métaphysique s'oppose à la science (qui ne connaît
guère que l’espace) et comme l'absolu s'oppose au relatif ({\it La pensée et le mouvant},
II et VI ; voir aussi Deleuze, {\it Le bergsonisme}, PUF, 1966, chap. I).

On voit que l'intuition, dans les trois cas, est définie positivement, mais de
trois façons différentes : elle est la condition de toute pensée (Descartes), de
toute connaissance (Kant), enfin de toute saisie de l'absolu, qui est esprit,
durée, changement pur (Bergson). Elle donne à la pensée son évidence (chez
Descartes), son objet (chez Kant) ou son absoluité (chez Bergson). Elle n’a en
commun, chez ces trois auteurs, qu’une forme d’immédiateté, qui peut servir
pour cela de définition générale : on peut appeler {\it intuition} toute connaissance
immédiate, s’il en est une.

INVENTION Plus qu’une découverte, moins qu’une création. Inventer,
c'est faire être ce qui n’existait pas (c'est ce qui distingue
l'invention de la découverte), mais qui aurait existé tôt ou tard (c’est ce qui distingue
l'invention de la création). Ainsi, Christophe Colomb a {\it découvert} l'Amérique
(il ne l’a pas inventée : elle existait avant lui) ; Newton a {\it découvert} la gravitation
%— 318 —
universelle (même remarque) ; alors que Denis Papin a {\it inventé} la
machine à vapeur, avec d’autres, comme Edison le télégraphe, le phonographe
et la lampe à incandescence. Ces quatre instruments, avant d’être inventés,
n’existaient nulle part. Mais sans Papin ou Edison, nul doute qu’ils auraient
fini, fût-ce un peu plus tard et sous une autre forme, par exister : notre univers
technique, si ces deux inventeurs étaient morts à la naissance, serait peut-être
différent de ce qu’il est, mais plutôt dans ses détails ou le cheminement anecdotique
de son développement que dans son contenu essentiel (la révolution
industrielle et celle de la communication n’en auraient pas moins eu lieu).
Alors que si Mozart ou Michel-Ange étaient morts à la naissance, nous
n’aurions jamais eu ni les {\it Esclaves du Louvre} ni le {\it Concerto} pour clarinette : ce
n'est plus {\it invention} mais {\it création}.

INVOLUTION Le contraire d’une évolution (voir ce mot), ou sa forme
régressive : c’est évoluer à l'envers.

IPSÉITÉ Du latin {\it ipse} (même : au sens de {\it lui-même}, {\it moi-même}, etc.). Le fait
d’être soi. Suppose l’unité, l’unicité, l'identité, enfin (spécialement
chez les phénoménologues : voir {\it L'être et le néant}, p. 147 à 149) la conscience.
C’est l’idiotie (voir ce mot) en personne.

IRONIE C'est se moquer des autres, ou de soi (dans l’autodérision) comme
d’un autre. L’ironie met à distance, éloigne, repousse, rabaisse.
Elle vise moins à rire qu’à faire rire. Moins à amuser qu’à désabuser. Ainsi
Socrate, vis-à-vis de tous les savoirs, et du sien propre. Il interroge ({\it eirôneia}, en
grec, c’est l'interrogation), quitte, parfois, à feindre d’ignorer ce qu’il sait, pour
essayer de découvrir ce qu’il ignore ou qu’on ne peut savoir. L’ironie est le
contraire d’un jeu : elle relève moins du principe de plaisir, dirait Freud, que
du principe de réalité, moins du loisir que du travail, moins de la paix que du
combat. Elle est utile : c’est là sa force en même temps que sa limite. C’est une
arme, c’est un outil, et ce n’est que cela. Un moyen, jamais une fin. Parfois
indispensable, jamais suffisante. C’est un moyen de se faire valoir, fût-ce à ses
dépens. C’est le moment du négatif, qui n’est supportable que comme
moment. L'esprit toujours nie, mais évite soigneusement, dans l'ironie, de se
nier lui-même : c’est un rire qui se prend au sérieux. Comment atteindrait-il
l'essentiel, puisqu'il nous en sépare ? « Gagnez les profondeurs, conseillait Rilke :
%— 319 —
l'ironie n’y descend pas. » Ce ne serait pas vrai de l’humour, et suffit à les distinguer.

IRRATIONNEL Ce qui n’est pas conforme à la raison théorique : ce qu’elle
ne peut, en droit, ni connaître ni comprendre. Si la raison
a toujours raison, comme le veut le rationalisme et comme je le crois, l’irrationnel
n’est qu’une illusion ou un passage à la limite : on ne juge irrationnel
(c’est-à-dire incompréhensible en droit) que ce qu’on n'arrive pas, en fait, à
comprendre. Ainsi l’irrationnel n’existe pas. Cela suffit à le distinguer du déraisonnable,
qui n’existe que trop.

IRRÉVERSIBILITÉ Le fait de ne pouvoir revenir en arrière : ainsi le temps
est irréversible, et lui seul peut-être. Mais tout ce qui se
passe dans le temps — c’est-à-dire tout — l’est aussi par là même. Projeter un film
à l’envers, quand on le peut sans absurdité (parce que les phénomènes filmés
sont réversibles), n’est possible que parce que le temps, lui, continue de se
dérouler à l'endroit : la projection inversée reste pour cela tout aussi irréversible
que l’autre. C’est en quoi toute action, même physiquement réversible (encore
qu'elle ne le soit jamais absolument : il y faut une nouvelle énergie), est ontologiquement
irréversible. Ce qu’on a fait, on peut ordinairement le défaire,
mais point faire que cela n’ait pas eu lieu. Pénélope, amoureuse irréversible.

ISOLEMENT L'absence de relations avec autrui. À ne pas confondre avec la
solitude, qui est une relation — à la fois singulière et inaliénable
— à soi et à tout.

ISONOMIE Une loi égale, ou légalité devant la loi. Le mot, en Grec, vient
de la terminologie juridique ou politique. Mais il est susceptible
d’un usage plus large, par exemple médical (pour dire l’équilibre des
humeurs) ou cosmologique (pour dire l’égale distribution des êtres dans l’univers).
Dans l’atomisme antique, par exemple, la notion semble avoir joué un
rôle important pour penser la répartition des atomes dans le vide (donc la pluralité
des mondes) ainsi que l’équilibre entre les forces destructrices et les forces
de conservation. C'était parier non sur la justice de la nature, mais sur la neutralité
du hasard.

% 321
JALOUSIE Parfois un synonyme d’{\it envie} ; plus souvent une de ses formes ou
de ses variantes. L’envieux voudrait posséder ce qu’il n’a pas et
qu'un autre possède ; le jaloux veut posséder seul ce qu’il croit être à lui. L'un
souffre du manque ; l’autre, du partage. Se dit surtout en matière amoureuse
ou sexuelle. C’est que toute possession vraie y est impossible ou illusoire.
Cela rend la jalousie plus cruelle. Il arrive que l'envie s’apaise, soit parce
qu'on possède enfin ce qu’on désirait, soit parce que celui qu’on enviait ne le
possède plus. La jalousie, non : tant que l’amour demeure, elle s’entretient
d'elle-même, par le soupçon ou l’interprétation interminable des signes.
L'envie est un rapport imaginaire au réel (« Qu'est-ce que je serais heureux
si... »). La jalousie serait plutôt un rapport réel à l'imaginaire (« Qu’est-ce
que je suis malheureux de. »). L’envie a davantage à voir avec l'espérance.
La jalousie, avec la crainte. C’est pourquoi elles vont ensemble, sans tout à
fait se confondre.

JE Le sujet à la première personne et en tant que tel. Se distingue par là du
{\it moi}, qui serait le sujet en tant qu’objet (y compris pour le {\it je} : quand je
{\it me} connais, ou crois me connaître). Le {\it je} est par nature insaisissable (on ne
peut saisir, et encore, que le {\it me}), rarement saisissant. Le monde ou la vérité
sont plus intéressants ; le {\it tu} ou le {\it nous}, plus aimables. À quoi l’on objectera que
c'est toujours un {\it je} qui connaît ou aime... Sans doute. C’est ce qui le sauve en
l’abolissant — quand il n’y a plus que la vérité ou l'amour. « Le péché en moi dit
“je” », écrit Simone Weil dans {\it La pesanteur et la grâce}. Puis elle ajoute, très
proche ici du Vedânta : « Je suis tout. Mais ce “je” là est Dieu. Et ce n’est pas
un je. »

%— 321 —
JEU Une activité sans autre but qu’elle-même ou que le plaisir qu'on y
trouve, sans autre contrainte que ses propres règles, enfin sans effet irréversible
(ce qu’une partie a fait, une autre partie peut toujours l’ignorer ou le
refaire). C’est pourquoi la vie, même visée pour elle-même ou pour le plaisir,
n’est pas un jeu : parce que les contraintes y sont innombrables, qui sont celles
du réel, parce qu’on n’y a pas le choix des règles ni du jeu, parce qu’on y vit et
qu’on y meurt {\it pour de vrai}, sans pouvoir jamais recommencer à neuf ni jouer à
autre chose. La vie, pourtant, vaut mieux que tous les jeux. Les enfants le savent
bien, qui jouent à être grands. Et les joueurs, lorsqu'ils jouent de l'argent. C'est
que le jeu ne leur suffit pas. Ils ont besoin d’enjeu, de sérieux, de gravité. Ils ont
besoin de la morsure du réel — non du jeu, mais du tragique. « Le contraire du
jeu n’est pas le sérieux, disait Freud, mais la réalité. » Cela n'empêche pas le jeu
d’être réel aussi, mais interdit de s’en contenter.

JEUNE Celui qui n’a pas encore commencé à décliner, qui a encore des progrès
à faire, qui les fera sans doute... Notion par nature relative. Un
sportif de trente-cinq ans est un vieux sportif ; un philosophe de trente-cinq
ans, un jeune philosophe. Être jeune, c’est avoir, dans tel ou tel domaine, plus
d'avenir, du moins en principe, que de passé. Quant au présent, tous en ont
autant. C’est ce qui rend la jeunesse impatiente, et les vieillards nostalgiques. Le
présent ne leur suffit pas.

On pense à la formule fameuse de Nizan, dans {\it Aden Arabie} : « J'avais vingt
ans. Je ne laisserai personne dire que c’est le plus bel âge de la vie. » C’est qu'il
n’y a pas de plus bel âge. Les uns auront préféré l'enfance, d’autres l’adolescence,
d’autres la quarantaine. Mais enfin je n’en connais pas qui préfèrent la
vieillesse. La jeunesse, même difficile, est plus désirable. Le corps ne s’y trompe
pas. Non pourtant que les vieux soient à plaindre. Ils ont été jeunes aussi. Les
jeunes ne sont pas sûrs d’être vieux un jour, ni même de pouvoir disposer d’une
jeunesse entière. Pour moi, je ne me reconnais jamais mieux que dans mes
dix-sept ans. Mais si j'étais mort à vingt, je n’en aurais pas moins raté l’essentiel.
Mieux vaut vivre vieux que mourir jeune. Cette évidence met la jeunesse à
sa place, qui n’est la première que dans le temps. Ce n’est pas une valeur ; c'est
une étape.

JEUNISME C’est faire de la jeunesse une valeur, voire l’ériger en valeur
suprême, y compris dans des domaines (politiques, artistiques,
intellectuels, culturels.) où elle n’a aucune légitimité particulière. Le premier
imbécile venu, à vingt ans, est ordinairement plus beau, plus fort, plus souple,
%— 322 —
plus désirable que ses parents ou ses grands-parents. Il n’en est pas moins bête
pour autant. Quant aux avantages de la jeunesse, ils sont surtout indirects. Ce
n'est pas l’âge en lui-même qui vaut ; c’est la beauté, la force, la souplesse, la
santé, l’éroticité... Qu'on s'efforce de les conserver le plus longtemps possible,
je n’ai rien contre. Ce n’est pas refuser de vieillir ; c’est vouloir vieillir le mieux
qu'on peut. Tant mieux si nos médecins peuvent nous y aider. Mais ne leur
demandons pas davantage. Si la jeunesse était le souverain bien, nos jeunes
n'auraient plus rien à désirer. Idée de vieux.

JOIE L'un des affects fondamentaux, en tant que tel impossible à définir
absolument. Qui ne l’aurait jamais ressentie, comment lui faire comprendre
ce qu’elle est ? Mais nous en avons tous une expérience. La joie naît
lorsqu’un désir intense est satisfait (la joie du bachelier, le jour des résultats),
lorsqu'un malheur est évité (la joie du miraculé ou du convalescent), lorsqu'un
bonheur arrive, ou semble arriver (la joie de l’amoureux, lorsqu'il se sait
aimé)... Jouissance, mais spirituelle ou spiritualisée (ré-jouissance). Elle est
l'élément du bonheur, à la fois minimum (dans le temps) et maximum (en
intensité). Élément singulier pourtant : on n’imagine pas le bonheur sans elle
(du moins sans sa possibilité), mais elle peut exister sans lui. C’est comme une
satisfaction momentanée de tout l’être — un acquiescement à soi et au monde.
Épicure dirait : plaisir en mouvement de l’âme ; et Spinoza : accroissement de
puissance ({\it passage} à une perfection supérieure). De fait, il y a dans la joie une
mobilité spécifique, qui est sa force en même temps que sa faiblesse. Quelque
chose en elle — ou en nous — lui interdit de durer. De à le désir de béatitude
(désir d’éternité) et le rêve du bonheur (dont la joie est peut-être le seul
contenu psychologique observable). La joie est ainsi notre guide et notre règle :
c’est l'étoile du berger de l’âme. Elle est l’origine, pour nous, de l’idée de salut.
« La joie, écrit Spinoza, est le passage de l’homme d’une moindre à une plus
grande perfection » ({\it Éthique}, III, déf. 2 des affects ; voir aussi le scolie de la
prop. 11). Comme la perfection elle-même n’est pas autre chose que la réalité,
cela signifie que la joie est passage à une réalité supérieure, ou plutôt à un degré
supérieur de réalité. Se réjouir, c’est exister davantage : la joie est le sentiment
qui accompagne en nous une expansion, ou une intensification, de notre puissance
d'exister et d’agir. C’est le plaisir — en mouvement et en acte — d’exister
plus et mieux.

JUDAÏSME C'était au début des années 80. Je rencontre un ancien
condisciple de khâgne et de la rue d’Ulm, que j'avais perdu de
%— 323 —
vue depuis nos années d’études. Nous prenons un verre, nous faisons en vitesse
le bilan de nos vies. Le métier, le mariage, les enfants, les livres projetés ou en
cours. Puis mon ami ajoute :

— «Il y a autre chose. Maintenant, je retourne à la synagogue.

— Tu étais juif ?

— Je le suis toujours ! Tu ne le savais pas ?

— Comment l’aurais-je su ? Tu n’en parlais jamais.

— Avec le nom que je porte !

— Tu sais, quand on n’est ni juif ni antisémite, un nom, sauf à s'appeler
Lévy ou Cohen, cela ne dit pas grand-chose. J'ai gardé de toi le souvenir d’un
kantien athée. Ce n’est pas une appartenance ethnique ou religieuse ! »

De fait, cet ami faisait partie de cette génération de jeunes Juifs si parfaitement
intégrés que leur judéité, pour qui en était informé, semblait comme
irréelle ou purement réactive. Ils donnaient raison à Sartre : ils ne se sentaient
juifs que pour autant qu’il y avait des antisémites. Beaucoup d’entre eux, plus
tard, feront ce chemin d’une réappropriation spirituelle, qui donnera un sens
positif — celui d’une appartenance, celui d’une fidélité — au fait, d’abord contingent,
d’être juif. L’ami dont je parle fut le premier pour moi d’une longue série,
qui me donnera beaucoup à réfléchir. Peut-être avions-nous tort de dénigrer
systématiquement le passé, la tradition, la transmission ? Mais je n’en étais pas
encore là. En l'occurrence, c’est surtout la question religieuse qui me turlupinait.
Je lui demande :

— « Mais alors, maintenant. tu crois en Dieu ?

— Tu sais, me répond-il en souriant, pour un Juif, l'existence de Dieu, ce
n’est pas vraiment la question importante ! »

Pour quelqu'un qui a été élevé dans le catholicisme, comme c’est mon
cas, la réponse était étonnante : croire ou non en Dieu, c'était la seule chose,
s'agissant de religion, qui me paraissait compter vraiment ! Naïveté de goy.
Ce que je lisais, dans le sourire de mon ami, c’était tout autre chose : qu’il est
vain de centrer une existence sur ce qu’on ignore, que la question de l’appartenance
— à une communauté, à une tradition, à une histoire — est plus
importante que celle de la croyance, enfin que l'étude, l’observance et la
mémoire — ce que j’appellerai plus tard la fidélité — importent davantage que
la foi.

Le judaïsme est religion du Livre. Je sais bien qu’on peut le dire aussi du
christianisme et de l’islam. Mais pas, me semble-t-il, avec la même pertinence.
« Le judaïsme, ajoute mon ami, c’est la seule religion où le premier devoir des
parents est d’apprendre à lire à leurs enfants... » C’est que la Bible est là, qui
les attend, qui les définit. Pour un chrétien, sans doute aussi pour un
musulman, c’est Dieu d’abord qui compte et qui sauve : le Livre n’est que le
%— 324 —
chemin qui en vient et y mène, que sa trace, que sa parole, qui ne vaut absolument
que par Celui qui l’énonce ou l’inspire. Pour un Juif, me semble-t-il, c’est
différent. Le Livre vaut pour lui-même, par lui-même, et continuerait de valoir
si Dieu n'existait pas ou était autre. D’ailleurs, qu’est-il ? Nul prophète juif n’a
prétendu le savoir, mais seulement ce qu’il voulait ou ordonnait. Le judaïsme
est religion du Livre, et ce Livre est une Loi (une {\it Thora}) bien davantage qu’un
{\it Credo} : c’est ce qu’il faut faire qu’il énonce, bien plus que ce qu’il faudrait
croire ou penser ! D'ailleurs on peut croire ce qu’on veut, penser ce qu’on veut,
c'est pourquoi l'esprit est libre. Mais point faire ce qu’on veut, puisque nous
sommes en charge, moralement, les uns des autres.

Si le Christ n’est pas Dieu, s’il n’est pas ressuscité, que reste-t-il du
christianisme ? Rien de spécifique, rien de proprement religieux, et pourtant, à
mes yeux d’athée, l’essentiel : une certaine fidélité, une certaine morale — une
certaine façon, parmi cent autres possibles, d’être juif... Il m’est arrivé, quand
on m'interrogeait sur ma religion, de me définir comme {\it goy assimilé}. C’est que
je suis judéo-chrétien, que je le veuille ou pas, et d’autant plus {\it assimilé}, en effet,
que j'ai perdu la foi. Il ne me reste que la fidélité pour échapper au nihilisme
ou à la barbarie.

Il y a quelques années, lors d’une conférence à Reims ou à Strasbourg, je ne
sais plus, j’eus l’occasion de m'expliquer sur ces deux notions de {\it foi} et de {\it fidélité}.
Après la conférence, qui se passait dans une faculté ou une grande école, se
tient une espèce de cocktail. On me présente un certain nombre de collègues et
de personnalités. Parmi celles-ci, un rabbin.

«— Pendant votre conférence, me dit-il, il s’est passé quelque chose d’amusant...

— Quoi donc ?

— Vous étiez en train de parler de fidélité. Je dis à l’oreille de l'ami qui
m'accompagnait : “Cela me fait penser à une histoire juive. Je te la raconterai
tout à l’heure...”

— Et alors ?

— Alors, c’est l’histoire que vous avez racontée vous-même, quelques
secondes plus tard ! »

Voici donc cette histoire, qui me paraît résumer l’esprit du judaïsme, ou du
moins la part de lui qui me touche le plus, et qu’il me plaît de voir ainsi, en
quelque sorte, authentifiée.

C’est l’histoire de deux rabbins, qui dînent ensemble. Ils discutent de l’existence
de Dieu, et concluent d’un commun accord que Dieu, finalement,
n'existe pas. Puis ils vont se coucher. Le jour se lève. L'un de nos deux rabbins
se réveille, cherche son ami, ne le trouve pas dans la maison, va le chercher
%— 325 —
dehors, et le trouve en effet dans le jardin, en train de faire sa prière rituelle du
matin. Il va le voir, quelque peu interloqué :

— « Qu'est-ce que tu fais ?

— Tu le vois bien : je fais ma prière rituelle du matin.

— Mais pourquoi ? Nous en avons discuté toute une partie de la nuit, nous
avons conclu que Dieu n'existait pas, et toi, maintenant, tu fais ta prière rituelle
du matin?!»

L'autre lui répond simplement :

— « Qu'est-ce que Dieu vient faire là-dedans ? »

Humour juif : sagesse juive. Qu’a-t-on besoin de croire en Dieu pour faire
ce que l’on doit ? Qu’a-t-on besoin d’avoir la foi pour rester fidèle ?

Dostoïevski, à côté, est un petit enfant. Que Dieu existe ou pas, tout n’est
pas permis : puisque la Loi demeure, aussi longtemps que les hommes s’en souviennent,
l’étudient et la transmettent.

L'esprit du judaïsme, c’est l’esprit tout court, qui est humour, connaissance
et fidélité.

Comment les barbares ne seraient-ils pas antisémites ?

JUGEMENT Une pensée qui vaut, ou qui prétend valoir. C’est en quoi tout
jugement est un jugement de valeur, quand bien même la
valeur en question n’est pas autre chose que la vérité (et quand bien même la
vérité, à la considérer en elle-même, n’est pas une valeur). Un jugement de réalité,
comme « La Terre est ronde », peut toujours être formulé, sans en changer
le contenu, sous la forme « Il est vrai que la Terre est ronde », où l’idée de vérité
fonctionne, pour nous, de façon normative. De même qu’un jugement normatif
peut prendre la forme d’un jugement de réalité : {\it « Cet homme est un
salaud. »} Ainsi la frontière entre les jugements normatifs et les jugements descriptifs
reste floue. Cela ne signifie pas que les normes soient réelles, ni que la
réalité soit la norme. Mais que tout jugement est humain, et subjectif par là.
Seule la vérité, qui ne juge pas, est objective. Mais nul ne la connaît que par ses
jugements, qui restent subjectifs. C’est pourquoi Dieu ne juge pas, dirait
Spinoza : parce qu’il est la vérité même. Et c’est pourquoi nous jugeons : parce
que nous ne sommes pas Dieu.

Un jugement, dans sa forme élémentaire, unit classiquement un sujet à un
prédicat, ou un prédicat à un sujet, par la médiation d’une copule : {\it « A est B »}
(dans les jugements affirmatifs) ou {\it « A n'est pas B »} (dans les jugements négatifs).
Par exemple : {\it « Socrate est mortel »} où {\it « Socrate n'est pas immortel »} sont
des jugements. On parle de jugement analytique, spécialement depuis Kant,
lorsque «le prédicat B appartient au sujet À comme quelque chose qui est
%— 326 —
contenu implicitement dans ce concept»; et de {\it jugement synthétique},
« lorsque B est entièrement en dehors du concept A, quoiqu'il soit, à la vérité,
en connexion avec lui » ({\it C. R. Pure}, Introd., IV). Par exemple, précise Kant,
{\it « Tous les corps sont étendus »} est un jugement analytique (il suffit de décomposer,
ou même de comprendre, l’idée de corps pour y trouver l'étendue) ;
alors que « Tous les corps sont pesants » est un jugement synthétique : l’idée de
poids n’est pas contenue dans celle de corps (l’idée d’un corps sans poids n’est
pas contradictoire), elle n’est unie à lui que de l’extérieur, en fonction d’autre
chose (en l'occurrence de l’expérience).

Il en suit, selon Kant :

1 — Que les jugements analytiques n’étendent pas du tout nos connaissances
(ils ne nous apprennent rien de nouveau), mais les développent, les précisent
ou les explicitent ;

2 — Que les jugements empiriques sont tous synthétiques ;

3 — Que les jugements synthétiques 4 priori, comme on voit dans les
sciences en général {\it (« Tout ce qui arrive a une cause »)} et dans les mathématiques
en particulier {\it (« 7 + 5 = 12 »)}, sont profondément mystérieux : sur quoi
peut-on bien s’appuyer pour sortir ainsi du concept et lui rattacher, de façon à
la fois universelle et nécessaire, un prédicat qu’il ne contient pas ? Tel est le problème
de la {\it Critique de la raison pure} : « Comment des jugements synthétiques
{\it a priori} sont-ils possibles ? » Kant répondra qu’un tel jugement n’est possible
que pour autant que nous nous appuyons, pour l’énoncer, sur les formes pures
de l'intuition (l’espace et le temps) ou de la pensée (les catégories de l’entendement).
Il ne vaut donc que pour nous, point en soi, et dans les limites d’une
expérience possible, point dans l’absolu. C’est résoudre le problème des jugements
synthétiques {\it a priori}, mais par l'{\it a priori} lui-même, considéré comme
antérieur au jugement et le rendant possible, autrement dit par le transcendantal.
Ce n’est pas l'expérience qui permet la connaissance ; ce sont au
contraire les formes {\it a priori} du sujet qui rendent l’expérience possible et la
connaissance nécessaire. Telle est la révolution copernicienne : c’est faire
tourner l’objet autour du jugement (ou du sujet qui juge), non le jugement
autour de l’objet.

Une autre solution, à laquelle je me range plus volontiers, serait de dire
qu'il n’y a pas de jugement à priori : c’est sortir de Kant pour revenir à Hume,
à l’empirisme et à l’histoire des sciences. Un pas en arrière, deux pas en avant.
C’est où il faut choisir entre le sujet transcendantal et le processus immanental
(voir ce mot), entre l’anhistoricité de la conscience et l’historicité des connaissances.
Ainsi, Cavaillès : «Je crois malhonnête tout recours à un {\it a priori}
quelconque », écrivait-il en 1938 à son ami Paul Labérenne, avant d’en
conclure qu’il fallait donc opérer une « rupture complète avec l’idéalisme,
%— 327 —
même brunschvicgien », considérer la logique « comme une première technique
naturelle », bref affirmer la « subordination complète » de la connaissance,
y compris des mathématiques, « à une expérience, qui n’est sans doute
pas l'expérience historique, puisqu’elle permet d’obtenir des résultats dont la
validité est hors du temps, mais qui {\it naît} de l’expérience historique ». Ainsi il
n’y a pas d’{\it a priori}, et tout jugement, même éternellement vrai, n’est rendu
possible que par une histoire qui le précède et le contient. Nous n'avons accès
à l’éternel que dans le temps ; tel est le jugement, quand il est vrai.

JUGER C’est relier un fait à une valeur, ou une idée à une autre. C’est pourquoi
« penser, c’est juger », comme disait Kant: parce qu'on ne
commence vraiment à penser qu’en reliant deux idées (au moins deux !) différentes.
Cela suppose l’unité de l'esprit ou du {\it je pense} (« l'unité originairement
synthétique de l’aperception »), comme pouvoir de liaison. Reste à savoir si
cette unité elle-même est première ou seconde, autrement dit si elle est donnée
{\it (a priori)} ou construite (dans le cerveau, dans l'expérience). Est-ce l'unité du
sujet qui rend le jugement possible, ou bien l’unité du jugement, même progressivement
constituée, qui rend le sujet nécessaire ? Est-ce parce que je suis
un sujet que je juge, ou est-ce à force de juger que je deviens sujet ? Est-ce le
transcendantal qui permet l’expérience, ou l'expérience qui constitue
l’immanental ? On remarquera que juger, dans les deux cas, reste le fait d’un
sujet : si le réel se jugeait soi, il serait Dieu ; si Dieu ne juge pas (Spinoza), il est
le réel même.

JUSTE Celui qui respecte la justice — la légalité et l'égalité, le droit {\it (jus)} et
les droits (des individus) — et qui se bat pour elle, autrement dit pour
que ces deux justices aillent ensemble : pour que la loi soit la même pour tous
(pour que la légalité respecte l'égalité), pour qu’elle soit appliquée avec équité
(voir ce mot), enfin pour que le droit (de la Cité) protège les droits (des individus).
C’est le plus haut devoir ; non, pourtant, la plus haute vertu. « Amis,
disait Aristote, on n’a que faire de la justice ; justes, on a encore besoin de
l'amitié » ({\it Éthique à Nicomaque}, VIII, 1). Ainsi l'amour, qui n’est pas un
devoir, vaut mieux que la justice, qui en est un. Les justes ne l’ignorent pas ;
mais ils n’attendent pas d’aimer pour être justes.

JUSTICE L'une des quatre vertus cardinales : celle qui respecte l'égalité et la
légalité, les droits (des individus) et le droit (de la Cité). Cela suppose
%— 328 —
que la loi soit la même pour tous, que le droit respecte les droits, enfin que
la justice (au sens juridique) soit juste (au sens moral). Comment le garantir ?
On ne le peut absolument ; c’est pourquoi la politique, même lorsqu’elle y
tend, reste conflictuelle et faillible. C’est pourtant la seule voie. Aucun pouvoir
n’est la justice ; mais il n’y a pas de justice sans pouvoir.

« Sans doute l'égalité des biens est juste, écrit Pascal, mais... » Mais quoi ?
Mais le droit en a décidé autrement, qui protège la propriété privée et par à
l’inégalité des biens. Faut-il le regretter ? Ce n’est pas sûr (il se peut qu’une
société inégalitaire soit plus prospère, même pour les plus pauvres, qu’une
société égalitaire). Ce n’est pas impossible (notamment si on met la justice plus
haut que la prospérité). Qui en décidera ? Le droit positif ({\it jus}, en latin, d’où
vient justice), ou plutôt ceux qui le font. Les plus justes ? Non pas. Les plus
forts — donc presque toujours, dans nos sociétés démocratiques, les plus nombreux.
La propriété privée fait-elle partie du droit naturel ? Fait-elle partie des
droits de l’homme ? Ce sont deux questions différentes, mais toutes deux insolubles
par le droit seul. Questions philosophiques plutôt que juridiques, et politiques
plutôt que morales. « Ne pouvant faire qu’il soit force d’obéir à la justice,
continue Pascal, on a fait qu’il soit juste d’obéir à la force. Ne pouvant fortifier
la justice, on a justifié la force, afin que le juste et le fort fussent ensemble, et
que la paix fût, qui est le souverain bien » ({\it Pensées}, 81-299 ; voir aussi le
fr. 103-298).

C’est où l’on rencontre la fiction, mais éclairante, du contrat social. « La
justice, écrivait Épicure, n’est pas un quelque chose en soi ; elle est seulement,
dans les rassemblements des hommes entre eux, quels qu’en soient le volume et
le lieu, un certain contrat en vue de ne pas faire de tort et de ne pas en subir »
({\it Maximes capitales}, 33 ; voir aussi les maximes 31 à 38). Peu importe que ce
contrat ait existé en fait ; il suffit à la justice qu’il puisse exister en droit : il est
« la règle, souligne Kant, et non pas l’origine de la constitution de l’État, non
le principe de sa fondation mais celui de son administration » ({\it Réfl.}, Ak. XVIII,
n° 7734 ; voir aussi {\it Théorie et pratique}, II, corollaire). Une décision est juste
quand elle pourrait être approuvée, en droit, par tous (par tout un peuple, dit
Kant) et par chacun (s’il fait abstraction de ses intérêts égoïstes ou contingents :
c’est ce que Rawls appelle la « position originelle » ou le « voile d’ignorance »).
Cela vaut pour l’État, mais tout autant pour les individus. « Le moi est injuste
en soi, écrivait Pascal, en ce qu’il se fait le centre de tout » ({\it Pensées}, 597-455).
Contre quoi toute justice est universelle, au moins dans son principe, et n’agit,
en chacun, que contre l’égoïsme ou par décentrement. Cela donne à peu près
la règle, telle qu’Alain la formule, et qui ne vaut pour tous que parce qu’elle
vaut d’abord pour chacun : « Dans tout contrat et dans tout échange, mets-toi
à la place de l’autre, mais avec tout ce que tu sais, et, te supposant aussi libre
%— 329 —
des nécessités qu’un homme peut l'être, vois si, à sa place, tu approuverais cet
échange ou ce contrat » ({\it 81 chapitres...}, VI, 4, « De la justice »). Cela vaut pour
les individus, mais donc aussi pour les citoyens. Pour la morale, mais donc aussi
— si les citoyens font leur devoir — pour la politique. « Est juste, écrivait Kant,
toute action ou toute maxime qui permet à la libre volonté de chacun de
coexister avec la liberté de tout autre suivant une loi universelle » ({\it Doctrine du
droit}, Introd., \S C). Cette coexistence des libertés sous une même loi suppose
leur égalité, au moins en droit, ou plutôt elle la réalise (malgré les inégalités de
fait, qui sont innombrables), et elle seule : c’est la justice même, toujours à faire
ou à refaire, toujours à défendre ou à conquérir.

{\it KAIROS} C'est un mot grec, qu’on peut sans trop de cuistrerie utiliser en
français, faute d’équivalent tout à fait satisfaisant. Le {\it kairos}, c’est
l’occasion propice, le moment opportun ou favorable, autrement dit {\it le bien},
comme on voit chez Aristote, mais {\it dans le temps} ({\it Éthique à Nicomaque}, I, 4).
Le thème, chez Aristote, est antiplatonicien. Le bien n’est pas une essence éternelle
ou absolue. Il se dit en plusieurs sens, tout comme l’être, et nul ne peut le
faire, dans le monde sublunaire, sans tenir compte du devenir : faire le bien,
c’est le faire {\it quand il faut} (le faire trop tôt ou trop tard ce n’est pas le faire, ou
c’est le faire moins bien qu’il ne faudrait ou qu’il n’aurait fallu : ainsi en médecine,
en politique ou en morale). Le {\it kairos} est comme la juste mesure, si l’on
veut, mais appliquée au temps: c’est « la juste mesure de l’irréversible »
(Francis Wolff).

KINESTHÉSIE Sensation par un individu des mouvements de son propre
corps (de {\it kinein}, se mouvoir, et {\it aisthèsis}, la sensation). Sans
doute l’une des origines, avec la sensation d’effort, de la conscience de soi. Je
bouge, donc je suis.

% 330
LÂCHETÉ Le manque de courage : non le fait d’avoir peur, mais l’incapacité
à surmonter la peur qu’on a, ou qu’on est, et même à lui
résister. C’est une complaisance pour sa propre frayeur, comme une soumission
à soi : son geste est de fuir ou de fermer les yeux.

Alain, dans un Propos du 27 mai 1909, notait que « le mot “lâche” est la plus
grave des injures ». Cela, qui semble à peu près vrai, au moins entre hommes, n’en
est pas moins curieux, puisqu'on peut être lâche sans être méchant ou cruel,
méchant et cruel sans être lâche, et que ces deux vices sont pires, assurément, que la
simple lâcheté. Combien de salauds sont capables de courage ? Combien de braves
gens, de lâcheté ? Et qui ne voit que ces millions de Français, qui se contentèrent
lâchement de ne rien faire, pendant l'Occupation, furent pourtant moins coupables
que les plus courageux des nazis ou des collabos ? Mais alors, pourquoi cette injure
fait-elle si mal ? Parce qu'il n’y a pas de vertu sans courage, répond Alain : ainsi le
mot lâche est l'injure la plus grave parce qu’elle est la plus globale, qui ne laisse rien
à estimer. Qu'un brave homme puisse être lâche, à l’occasion, c’est entendu. Mais
s’il l'était toujours, il ne serait plus « brave », en aucun sens du mot : toujours la
peur l'empêcherait d’être généreux ou juste, et même d’être sincère ou aimant ; car
enfin rien de cela ne va sans risque — rien de cela ne va sans courage. Qui s’abandonne
à sa peur, il est même incapable d’être prudent : fuir toujours, fermer les
yeux toujours, c'est manquer de prudence presque autant que de courage.

Cela ne prouve pas que le courage soit la plus grande vertu, mais seulement
qu’il est la plus nécessaire.

LAÏC Qui fait partie du peuple {\it (laos)} et non du clergé. Par extension, le mot
désigne tout ce qui est indépendant de la religion, ou doit l'être.

%— 331 —
LAÏCITÉ Ce n’est pas l’athéisme. Ce n’est pas l’irréligion. Encore moins
une religion de plus. La laïcité ne porte pas sur Dieu, mais sur la
société, Ce n’est pas une conception du monde ; c’est une organisation de la
Cité. Ce n’est pas une croyance ; c’est un principe, ou plusieurs : la neutralité
de l’État vis-à-vis de toute religion comme de toute métaphysique, son indépendance
par rapport aux Églises comme l'indépendance des Églises par rapport
à lui, la liberté de conscience et de culte, d'examen et de critique, l’absence
de toute religion officielle, de toute philosophie officielle, le droit en conséquence,
pour chaque individu, de pratiquer la religion de son choix ou de n’en
pratiquer aucune, enfin, mais ce n’est pas le moins important, l’aspect non
confessionnel et non clérical — mais point non plus anticlérical — de l’école
publique. L'essentiel tient en trois mots : {\it neutralité} (de l’État et de l’école),
{\it indépendance} (de l'État vis-à-vis des Églises, et réciproquement), {\it liberté} (de
conscience et de culte). C’est en ce sens que Mgr Lustiger peut se dire laïque,
et je lui en donne bien volontiers acte. Il ne veut pas que l'État régente l’Église,
ni que l’Église régente l'État. Il a évidemment raison, même de son propre
point de vue : il rend « à César ce qui est à César, et à Dieu ce qui est à Dieu »
(Mt 22, 21). Les athées auraient tort de faire la fine bouche. Que l'Église catholique
ait mis tant de temps pour accepter la laïcité, cela ne rend sa conversion,
si l’on peut dire, que plus spectaculaire. Mais cette victoire, pour les laïcs, n’est
pas pour autant une défaite de l’Église : c’est la victoire commune des esprits
libres et tolérants. La laïcité nous permet de vivre ensemble, malgré nos différences
d'opinions et de croyances. C’est pourquoi elle est bonne. C’est pourquoi
elle est nécessaire. Ce n’est pas le contraire de la religion. C’est le
contraire, indissociablement, du cléricalisme (qui voudrait soumettre l’État à
l'Église) et du totalitarisme (qui voudrait soumettre les Églises à l'État).

On comprend qu'Israël, l'Iran ou le Vatican ne sont pas des États laïques,
puisqu'ils se réclament d’une religion officielle ou privilégiée. Mais PAlbanie
d’Enver Hoxha ne l'était pas davantage, qui professait un athéisme d’État. Cela
dit assez ce qu’est vraiment la laïcité : non une idéologie d’État, mais le refus,
par l’État, de se soumettre à quelque idéologie que ce soit.

Et les droits de l’homme ? demandera-t-on. Et la morale ? Ce n’est pas à
eux que l’État se soumet, mais à ses propres lois et à sa propre constitution — ou
aux droits de l’homme pour autant seulement que la constitution les énonce ou
les garantit. Pourquoi, dans nos démocraties, le fait-elle ? Parce que le peuple
souverain en a décidé ainsi, et ce n’est pas moi qui le lui reprocherai. C'est
mettre l’État au service des humains, comme il doit être, plutôt que les
humains au sien. Mais la même raison interdit d’ériger les droits de l’homme
en religion d’État. Distinction des ordres : l’État ne doit régner ni sur les esprits
ni sur les cœurs. Il ne dit ni le vrai ni le bien, mais seulement le permis et le
%— 332 —
défendu. Il n’a pas de religion. Il n’a pas de morale. Il n’a pas de doctrine. Aux
citoyens d’en avoir une, s'ils le veulent. Non pourtant que l’État doive tout
tolérer, ni qu’il le puisse. Mais il n’interdit que des actions, point des pensées,
et pour autant seulement qu’elles enfreignent la loi. Dans un État vraiment
laïque, il n’y a pas de délit d'opinion. Chacun pense ce qu’il veut, croit ce qu’il
veut. Il doit rendre compte de ses actes, non de ses idées. De ce qu’il fait, non
de ce qu’il croit. Les droits de l’homme, pour un État laïque, ne sont pas une
idéologie, encore moins une religion. Ce n’est pas une croyance, c’est une
volonté. Pas une opinion, une loi. On a le droit d’être contre. Pas de les violer.

LAIDEUR Ce n’est pas l’absence de beauté, mais son contraire : non ce qui
ne plaît pas, mais ce qui déplaît ; non ce qui ne séduit pas, mais
ce qui repousse.

« La beauté, écrivait Spinoza, n’est pas tant une qualité de l’objet considéré
qu'un effet se produisant en celui qui le considère » ; si nous avions d’autres
yeux ou un autre cerveau, « les choses qui nous semblent belles nous paraitraient
laides et celles qui nous semblent laides deviendraient belles », de même
que « la plus belle main, vue au microscope, paraîtra horrible » ({\it Lettre 54}, à
Hugo Boxel). Ainsi toute laideur est relative, comme toute beauté. Il n’y a pas
de laideur en soi, ni de laideur objective : être laid, c’est déplaire, disais-je, et
l’on ne peut déplaire qu’à un sujet. Cela ne rend pas la laideur moins injuste,
ou plutôt c’est ce qui rend son injustice plus cruelle : parce qu’elle semble
repousser l’amour, et même la sympathie, et les repousse en effet, au moins un
temps — puisqu'elle n’est pas autre chose que cette {\it répulsion} qu’elle suscite et à
quoi on la reconnaît. On peut, en art, jouer avec elle, jusqu’à la beauté (les
« peintures noires » de Goya, la {\it Raie} de Chardin, les portraits de Bacon). Mais
dans la vie ? Il y faut de l’art aussi, et une part de talent — y compris chez le
spectateur.

LANGAGE Au sens large : toute communication par signes (on parlera par
exemple du « langage des abeilles »). Au sens strict, ou spécifiquement
humain : la faculté de parler (la parole en puissance), ou la totalité des
langues humaines. On remarquera que le langage ne parle pas, ne pense pas, ne
veut rien dire, et qu’il n’est pas une langue ; c’est pourquoi nous pouvons parler
et penser. Le langage n’est qu’une abstraction : seules les paroles, mais en actes,
sont réelles, qui ne s’actualisent que dans une langue particulière. Ainsi le langage
est à peu près aux langues et aux paroles ce que la vie est aux espèces et aux
individus : leur somme, ou leur reste.

%— 335 —
LANGUE  « La langue, disait De Saussure, c’est le langage moins la parole » :
ce qui reste, quand on se tait. C’est ce qui donne tort aux bavards,
et raison aux linguistes.

Mais qu'est-ce que la parole ? La mise en œuvre, par un individu singulier,
en un moment singulier, d’une langue quelconque. Ainsi la langue est ce dans
quoi nous parlons : c’est un ensemble de signes conventionnels, articulés (et
même doublement articulés : en monèmes et en phonèmes) et soumis à un certain
nombre de structures aussi bien sémantiques que grammaticales.

On remarquera que la pluralité des langues, qui est une donnée de fait,
n'exclut ni l’unité du langage (puisque tout discours dans une langue naturelle
peut être traduit dans une autre) ni celle de la raison. Il me semble même
qu’elle les suppose. S'il n’y avait une raison avant le langage, et une fonction
symbolique avant les langues, aucune parole jamais n’eût été possible. De ce
point de vue, l’aporie bien connue de l’origine des langues (il faut une langue
pour raisonner, et de la raison pour inventer une langue) n’en est pas tout à fait
une : d’abord parce qu’une langue n’est pas {\it inventée} (elle est le résultat d’un
processus historique, non d’un acte individuel), ensuite parce que l'intelligence
et la fonction symbolique existent {\it avant} les langues : cela même qui permet aux
nouveau-nés d'apprendre à parler dut permettre aux humains, au fil des millénaires,
de passer d’une communication seulement sensori-motrice (cris, gestes,
mimiques, comme on voit chez les animaux) à une communication linguistique.

Il faut souligner pour finir l’extrême efficacité — en termes de puissance et
d'économie — de ce que Martinet appelle la double articulation. Toute langue
se divisant en unités minimales de signification (les monèmes), dont chacune
peut à son tour se diviser en unités sonores minimales (les phonèmes), on
aboutit à cette espèce de miracle objectif qu’est la communication humaine :
l’ensemble de nos expériences, de nos idées et de nos sentiments — tous les livres
écrits et possibles, toutes les paroles prononcées et prononçables — peut
s’énoncer au moyen de quelques dizaines de petits cris brefs, ou plutôt de sons
minimaux, de pures différences vocales, toujours les mêmes dans chaque langue
(le français, par exemple, compte une quarantaine de phonèmes), à la fois
dépourvus de signification et les permettant toutes. C’est le plus simple toujours
qui permet le plus complexe : on ne pense que grâce à des atomes qui ne
pensent pas, on ne parle que grâce à des sons qui ne veulent rien dire. C’est où
la linguistique, qui semble si peu matérielle, peut mener au matérialisme.

LAPSUS «Ayant entendu quelqu'un crier naguère que sa maison s'était
envolée sur la poule du voisin, je n’ai pas cru qu’il se trompait,
— 334 —
écrit Spinoza, parce que sa pensée me semblait assez claire » ({\it Éthique}, II, scolie
de la prop. 47). C'était en effet ce que nous appelons un lapsus : moins une
erreur, dans la pensée, qu’un acte manqué dans le discours. C’est dire involontairement
un mot à la place d’un autre {\it (lapsus linguae)}, ou l'écrire {\it (lapsus
calami)}. Freud nous a habitués à en chercher le sens inconscient, et l’on aurait
tort de se l’interdire : c’est souvent amusant, parfois éclairant. Il reste qu’un
acte manqué n’a valeur que d’exception, et que son interprétation elle-même
n'est légitime qu’à la condition de n’en être pas un. L’inconscient parle, sans
doute. Mais la conscience parle aussi, et ce qu’elle a à dire, sauf bêtise ou bavardage,
est plus intéressant. Les textes de Freud donnent davantage à penser que
les lapsus de ses patients.

LARMES «Ne me secouez pas; je suis plein de larmes. » Cette formule
d'Henri Calet m'a toujours touché. Les larmes ressemblent à la
mer, dont nous sommes issus. C’est de l’eau salée, qui vient humidifier la
cornée, qui la protège par là. Mais pourquoi coulent-elles quand on a du
chagrin ? On dirait que le cœur déborde. Qu'il retrouve en lui, intarissable,
inconsolable, l'océan primordial du malheur. Ou est-ce celui de vivre, qui vient
tout emporter, tout nettoyer, jusqu’à l’envie de pleurer ?

LASSITUDE C’est une fatigue ressentie, et qui se porte à l’âme. Elle doit
moins à l’intensité de l’effort qu’à sa durée, moins au travail
qu’à l'ennui, moins à l'excès qu’à la répétition. C’est comme la fatigue d’être
fatigué. Son remède, les mots l’indiquent assez, est moins le repos que le délassement.
Mais l’homme absolument las n’en a cure, et préférerait n'être pas né.

LATENT Ce qui existe sans se manifester, ou, s’il se manifeste, sans être
perçu ou compris. Se dit spécialement, chez Freud, du sens des
rêves, qui n'apparaît jamais immédiatement : le travail du rêve transforme —
notamment par déplacement et condensation — leur contenu latent (inconscient)
en contenu manifeste (le rêve tel que le vit le rêveur, ou tel qu’il s’en souvient au
réveil). L'interprétation essaie, à l'inverse, de remonter de celui-ci à celui-là. Le
plus sage, en dehors de la cure, est de ne se préoccuper ni de l’un ni de l’autre.

LÉGALITÉ La conformité factuelle à la loi. À ne pas confondre avec la {\it légitimité},
qui suppose un jugement de valeur, ni avec la {\it moralité},
%— 335 —
qui peut parfois pousser à violer une loi juridique et qui ne saurait se
contenter d’être simplement conforme à la loi morale. Par exemple, explique
Kant, le commerçant qui n’est honnête que pour garder ses clients : il agit
bien {\it conformément au devoir}, mais point {\it par devoir} (il agit conformément au
devoir, mais par intérêt) ; son action, pour {\it légale} qu’elle soit (en l’occurrence
au double sens, juridique et moral, du terme), n’en est pas moins dépourvue
de toute valeur proprement {\it morale} (puisqu’une conduite n’a de valeur morale
qu’à la condition d’être désintéressée). Ainsi la légalité n’est qu’un fait, qui ne
dit rien sur la légitimité d’une action ni, encore moins, sur sa moralité. Les
lois antijuives de Vichy, à les supposer même juridiquement inattaquables,
n’en étaient pas moins illégitimes : il était immoral de les voter, de les appliquer,
et même (sauf pour les victimes, quand elles ne pouvaient faire autrement)
de leur obéir.

LÉGÈRETÉ C’est une manière de ne peser sur rien, qui est le propre de
l'esprit, des dieux et, parfois, des musiciens.
« Tout ce qui est bon est léger, écrivait Nietzsche, tout ce qui est divin
court sur des pieds délicats » ({\it Le cas Wagner}, I). Cela, qui fut écrit à propos de
Bizet et contre Wagner, pourrait presque tenir lieu de définition : toute personne
connaissant la musique de ces deux compositeurs peut comprendre ce
qu'est cette {\it divine légèreté}, comme dit ailleurs Nietzsche, qui résonne ou danse
chez le premier, quand on n’entend chez le second, par différence et sauf exception
(comme dans la {\it Siegfried idylle}), que le poids écrasant du sérieux, de la prétention,
de la sublimité feinte ou recherchée. Toutefois cela ne suffit pas à
prouver que Bizet soit un plus grand musicien que Wagner, ni que la légèreté,
en toute chose, soit bonne. Il ne s’agit que d’une catégorie esthétique, qui ne
saurait valoir universellement. La légèreté est le contraire de la lourdeur, du
sérieux, de la gravité. Elle n’exclut pas le tragique ; elle l’ignore ou le surmonte.
C’est comme une grâce, mais qui serait purement immanente, comme une élégance,
mais qui serait de l’âme, comme une insouciance, mais qui serait sans
petitesse. Elle est bouleversante chez Mozart ; cela ne retire rien à Bach ou
Beerhoven, dont la légèreté n’est pas le fort. Enfin elle est simplement agaçante
chez les esprits frivoles, dans les situations qui requièrent du sérieux ou de la
gravité. « C’est une manière qui ne me va guère, disait Colette, que cette affectation
de légèreté envers l’amour. » Celle-là pourtant savait ce que c’est que la
légèreté des mœurs et de la plume. Mais elle n’oubliait pas pour cela la gravité
d'aimer, ou d’être aimée. La légèreté vaut mieux que lourdeur ; la gravité,
mieux que la frivolité.

%— 336 —
LÉGITIMITÉ La notion se situe à l'interface entre le droit et la morale,
mais aussi entre le droit et la politique. Est légitime ce qui est
dans son {\it bon droit}, ce qui suppose qu’un droit ne l’est pas toujours. La légitimité,
c’est la conformité non seulement à la loi (légalité) mais à la justice ou à
un intérêt supérieur. Voler pour ne pas mourir de faim, comploter contre un
tyran, désobéir à un pouvoir totalitaire, résister, les armes à la main, contre un
occupant, voilà autant de comportements qui, pour être le plus souvent illégaux,
n'en seront pas moins, sauf situation très particulière, parfaitement légitimes.
On demandera qui en décide. Un tribunal le peut, s’il est assez fort et
assez juste, mais il ne le fera le plus souvent qu’après coup — quand il sera trop
tard — et non sans risque, bien souvent, de se tromper. Seuls les vainqueurs,
sauf exception, disposent de tribunaux, et rien n’interdit que la justice, parfois,
soit du côté des vaincus. C’est pourquoi la seule instance de légitimation reste
la conscience individuelle. Cela fait peu ? Sans doute, pour ceux qui en manquent
ou qui voudraient une garantie. Mais ce peu, pourtant, doit suffire —
puisqu'il n’y a rien d’autre. Quel était le représentant légitime de la France,
pendant la dernière guerre mondiale : le maréchal Pétain ou le général de
Gaulle ? La réponse est facile aujourd’hui. Mais c’est alors qu’elle était importante.

LETTRÉS, GENS DE LETTRES Un lettré, ce n’est pas seulement quelqu'un
qui sait lire et écrire: c’est un
homme de culture, surtout littéraire, et de pensée, surtout philosophique —
l'équivalent, chez Voltaire, de ce que nous appelons aujourd’hui un {\it intellectuel}.
Quant aux gens de lettres, ce sont ceux des lettrés qui en font un
métier, qui écrivent des livres, qui les publient, qui essaient d’en vivre... La
dénomination, dans ce petit milieu, est souvent péjorative. C’est qu’elle
désigne d’abord les collègues, qui sont aussi des rivaux et qu’il est dès lors
naturel de détester. Voltaire, qui parlait d'expérience, notait pourtant que
«le plus grand malheur d’un homme de lettres n’est peut-être pas d’être
l’objet de la jalousie de ses confrères [...] ; c’est d’être jugé par des sots ». Il
faut être détesté ou méprisé, mal compris ou mal aimé, et souvent les deux.
Dur métier. « L'homme de lettres est sans secours, continue Voltaire ; il ressemble
aux poissons volants : s’il s'élève un peu, les oiseaux le dévorent ; s’il
plonge, les poissons le mangent.» Bien peu, pourtant, regrettent l’anonymat.
Ils sont descendus pour leur plaisir dans l'arène, comme dit Voltaire,
ils se sont donnés eux-mêmes aux bêtes. Cela leur donne le droit d’être lus,
pas celui de se plaindre.

%— 337 —
LIBÉRAL Qui respecte la liberté, et d’abord celle des autres. Un État libéral
est donc celui qui respecte les libertés individuelles, dût-il pour
cela limiter la sienne propre. On ne le confondra pas avec la démocratie (il peut
exister des démocraties autoritaires et des monarchies libérales), encore moins
avec le laisser-faire : les libertés individuelles n’existent que par la loi, qui suppose
la contrainte.

LIBÉRALISME La doctrine des libéraux, quand ils en ont une. En français,
se dit surtout de la doctrine économique : celle qui veut que
l'État intervienne le moins possible dans la production et les échanges, si ce
n’est pour garantir, quand nécessaire, le libre jeu du marché. Doctrine respectable,
mais qui ne saurait valoir, me semble-t-il, que pour ce qui relève du
marché, autrement dit que pour les marchandises. Or la justice n’en est pas
une, ni la liberté, ni l'égalité, ni la fraternité... Elles sont donc à la charge, légitimement,
de l’État et des citoyens. C’est ce qui permet de distinguer le {\it libéralisme},
pour lequel la politique garde ses droits et ses ambitions, de l’{\it ultra-libéralisme},
qui voudrait cantonner l’État dans ses fonctions régaliennes
d'administration, de police, de justice et de diplomatie. Ce serait renoncer à
agir sur la société elle-même, voire renoncer à la République. Quand le général
de Gaulle, dans les années 60, disait que « la politique de la France ne se fait pas
à la corbeille », il ne manifestait pas seulement un trait de tempérament personnel,
ni ne voulait cantonner la politique dans son pré carré diplomatique,
administratif et judiciaire. Il rappelait un principe essentiel à toute démocratie
véritable. Si le peuple est souverain, comment le marché le serait-il ?

LIBÉRALITÉ Le juste milieu dans les affaires d’argent : c’est n'être ni avare
ni prodigue (Aristote, {\it Éthique à Nicomaque}, IV, 1119 b-1122 a).
Il est improbable que l’homme libéral devienne riche, constate Aristote, et
encore plus qu'il le reste : car «il n’apprécie pas l'argent en lui-même, mais
comme moyen de donner ». Aussi est-il improbable, pour la même raison, que
les riches sachent faire preuve de libéralité : c’est qu’il n’est pas possible
« d’avoir de l'argent si on ne se donne pas de peine pour l’acquérir » et pour le
garder. C’est dire qu’on a peu de chance de devenir riche sans cupidité, comme
de le rester sans avarice. Le juste milieu est plus exigeant qu’on ne le croit.

LIBÉRATION Le fait de devenir libre, et le processus qui y mène. C’est la
liberté en acte et en travail. S’oppose par là au libre arbitre,
%— 338 —
qui serait une liberté originelle et absolue (une liberté toujours déjà donnée :
une liberté en puissance et en repos !). « Les hommes se trompent en ce qu’ils
se croient libres », disait Spinoza ({\it Éthique}, II, 35, scolie), et cette illusion est
l’une des principales causes qui les empêchent de le devenir. Le mixte singulier
de conscience et d’inconscience qui les constitue (ils ont conscience de leurs
désirs et de leurs actes, mais non des causes qui les font désirer et vouloir) les
{\it assujettit}, comme dira Althusser, en les faisant {\it sujets}. Leur soi-disant liberté
n'est qu'une causalité qui s’ignore. C’est au contraire parce que le libre arbitre
n'existe pas qu'il faut se libérer toujours, et d’abord de soi. La vérité seule le
permet, où toute subjectivité se brise. Liberté, nécessité comprise (Spinoza,
Hegel, Marx, Freud), ou compréhension, plutôt, de la nécessité, Non que la
compréhension échappe à la nécessité (comment le pourrait-elle, puisqu’elle en
fait partie ?), mais parce que la raison n’y obéit qu’à soi ({\it Éthique}, I, déf. 7).
Seule la connaissance est libre, et libère. C’est où l'éthique, qui tend à la liberté,
se distingue de la morale, qui la suppose.

LIBERTÉ Être libre, c’est faire ce que l’on veut. De là trois sens principaux
du mot, selon le {\it faire} dont il s’agit : liberté d’action (si faire c’est
agir), liberté de la volonté (si faire c’est vouloir : on verra que ce sens-là se subdivise
en deux), enfin liberté de l’esprit ou de la raison (quand faire c’est
penser).

La liberté d’action ne pose guère de problèmes théoriques. Ce n’est autre
chose, disait Hobbes, que « l’absence de tous les empêchements qui s’opposent
à quelque mouvement : ainsi l’eau qui est enfermée dans un vase n’est pas libre,
à cause que le vase l'empêche de se répandre, et lorsqu'il se rompt, elle recouvre
sa liberté » ({\it Le Citoyen}, IX, 9). S'agissant des humains, c’est ce qu’on appelle
souvent la {\it liberté au sens politique} : parce que l’État est la principale force qui la
limite et la seule qui puisse la garantir à peu près. Je suis libre d’agir quand rien
ni personne ne m'en empêche : c’est pourquoi je le suis davantage dans une
démocratie libérale que dans un État totalitaire, et c’est pourquoi je ne saurais
jamais l’être absolument (il y a toujours des empêchements, qui tiennent spécialement,
dans un État de droit, à la loi : ma liberté s’arrête où commence celle
des autres). C’est la liberté au sens de Hobbes, de Locke, de Voltaire. Elle existe
évidemment, mais {\it plus ou moins} : liberté toujours relative, toujours limitée,
pour cela toujours à défendre ou à conquérir.

La liberté de la volonté ne semble pas poser beaucoup plus de problèmes.
Puis-je vouloir ce que je veux ? Oui, sans doute, puisque personne, sauf manipulation
mentale ou neurologique, ne peut m'empêcher de vouloir ni vouloir à
ma place. Au reste, comment pourrais-je vouloir ce que je ne veux pas ou ne
%— 339 —
pas vouloir ce que je veux ? La liberté de la volonté, en ce sens, est moins un
problème qu’une espèce de pléonasme : vouloir, c’est par définition vouloir ce
que l’on veut (puisque la volonté ne saurait échapper au principe d’identité) et
c’est en quoi c’est être libre. C’est ce que j'appelle la spontanéité du vouloir, qui
n’est pas autre chose que la volonté en acte : au présent, « libre, spontané et
volontaire ne sont qu’une même chose » (comme le reconnaissait Descartes de
l'acte en train de s'accomplir) ; c'est pourquoi toute volonté est libre, et elle
seule (le reste n’est que passion ou passivité). C’est la liberté au sens d’Épicure
et d’Épictète, mais aussi, pour l'essentiel, au sens d’Aristote, de Leibniz ou de
Bergson. Je veux ce que je veux : je veux donc librement.

Soit. Mais peut-on vouloir aussi autre chose ? Autre chose que ce que l’on
veut ? Cela semble violer le principe d’identité. Mais comment, sans ce pouvoir-là,
aurait-on le choix ? Il semble que la volonté ne soit vraiment libre que
si elle peut se choisir elle-même, ce qui suppose — puisqu'on ne choisit que le
futur — qu’elle n’existe pas encore. Il faut donc, pour que la volonté soit absolument
libre, que le sujet préexiste paradoxalement à ce qu’il est (puisqu'il doit
le choisir) : de là le mythe d’Er, chez Platon, le caractère intelligible, chez Kant,
ou l’existence-qui-précède-l’essence chez Sartre. Cette liberté-là reste bien une
liberté de la volonté, si l’on veut, mais antérieure, au moins en droit, à toute
volition effective. Elle est absolue ou elle n’est pas. C’est ce qu’on appelle parfois
la liberté au sens métaphysique du terme, et plus souvent le libre arbitre :
ce n’est plus spontanéité mais création, plus un être mais un néant, comme dit
Sartre, plus un sujet qui choisit mais le choix du sujet par lui-même. C’est la
liberté selon Descartes (peut-être déjà selon Platon, du moins dans certains
textes), selon Kant, selon Sartre : le pouvoir indéterminé de se déterminer soi-même,
autrement dit de se choisir (Sartre: «toute personne est un choix
absolu de soi ») ou de se créer (Sartre encore : « liberté et création ne font
qu’un »). Mais comment, puisque nul ne peut se choisir qu’à la condition
d’être déjà ? Cette liberté-là n’est possible que comme néant : elle n’est possible
qu’à la condition de n’être pas ! J'aurais tendance à y voir une réfutation ; c’est
sans doute que le néant n’est pas mon fort. « Je suis en plein exercice de ma
liberté, écrit Sartre, lorsque, vide et néant moi-même, je {\it néantis} tout ce qui
existe » (« La liberté cartésienne », {\it Situations}, I). C’est une chose dont je n’ai
aucune expérience. Je ne connais que l'être. Je ne connais que l’histoire en train
de se faire, toujours simultanée à soi, toujours déterminée en même temps que
déterminante. Je ne connais, comme liberté de la volonté, que sa spontanéité :
que le pouvoir {\it déterminé} de se déterminer soi-même. Est-ce moi qui manque
d'imagination, ou Sartre, de réalisme ?

« Les hommes se trompent en ce qu’ils se croient libres, écrivait Spinoza, et
cette opinion consiste en cela seul qu’ils ont conscience de leurs actions et sont
%— 340 —
ignorants des causes par où ils sont déterminés » ({\it Éthique}, II, scolie de la
prop. 35). Ils ont conscience de leurs désirs et volitions, mais point des causes
qui les font désirer et vouloir ({\it Éthique} I, Appendice ; voir aussi la {\it Lettre 58}, à
Schuller). Comment ne croiraient-ils pas être libres de vouloir, puisqu'ils veulent
ce qu’ils veulent ? Et certes Spinoza ne nie pas qu’il y ait là une spontanéité
effective (voir par exemple {\it Éthique} III, scolie de la prop. 2), qui est celle du
{\it conatus}. Leur erreur est de l’absolutiser, comme si elle était indépendante de la
nature et de l’histoire. Comment le serait-elle, puisqu’elle n’aurait alors aucune
raison d'exister ni d’agir ? La volonté n’est pas un empire dans un empire. Je
veux ce que je veux ? Certes, mais point de façon indéterminée ! « Il n’y a dans
l’âme aucune volonté absolue ou libre ; mais l’âme est déterminée à vouloir ceci
ou cela par une cause qui est aussi déterminée par une autre, et cette autre l’est
à son tour par une autre, et ainsi à l'infini » ({\it Éthique}, II, prop. 48 ; voir aussi I,
prop. 32 avec sa démonstration). On ne sort pas du réel. On ne sort pas de la
nécessité. Est-ce à dire que chacun reste prisonnier de ce qu’il est ? Non pas,
puisque la raison, qui est en tous, n’appartient à personne. Comment pourrait-elle
nous obéir ? « L'esprit ne doit jamais obéissance, écrivait Alain. Une preuve
de géométrie suffit à le faire voir ; car si vous la croyez sur parole, vous êtes un
sot ; vous trahissez l'esprit » (Propos du 12 juillet 1930). C’est pourquoi aucun
tyran n'aime la vérité. Parce qu’elle n’obéit pas. C’est pourquoi aucun tyran
n'aime la raison. Parce qu’elle n’obéit qu’à elle-même : parce qu’elle est libre.
Non, certes, qu’elle ait le choix, si l’on entend par là qu’elle pourrait penser
n'importe quoi. Mais parce que sa nécessité propre est le gage de son indépendance.
Non que la vérité soit à {\it choisir} ; mais au contraire parce qu’elle ne l’est
pas : parce qu’elle s’impose nécessairement à toute personne qui la connaît au
moins en partie, et parce qu'il suffit de la connaître pour être libéré, au moins
partiellement, de soi (puisque la vérité est la même en tout esprit qui la
perçoit : quand un névrosé fait des mathématiques, la vérité des mathématiques
n’en devient pas névrosée pour autant). C’est ce qu’on peut appeler la liberté
de l’esprit ou de la raison, qui n’est pas autre chose que la libre nécessité du
vrai. C’est la liberté selon Spinoza, selon Hegel, sans doute aussi selon Marx et
Freud : la liberté comme nécessité comprise, ou comme compréhension,
plutôt, de la nécessité. La vérité n’obéit à personne, pas même au sujet qui la
pense : c’est en quoi elle est libre, et libère.

Trois sens, donc, dont le deuxième se subdivise en deux : la liberté d’action,
la liberté de la volonté (qu’on peut penser comme spontanéité ou bien comme
libre arbitre), enfin la liberté de l’esprit ou de la raison. Seul le libre arbitre me
paraît douteux, et à la vérité impensable. Les trois autres libertés n’en existent
pas moins, qui se complètent. À quoi bon vouloir, si l’on ne pouvait agir
librement ? Et au nom de quoi, si toute pensée était esclave ? Mais cela n’est
%— 341 —
pas. Nous sommes libres d’agir, de vouloir, de penser, du moins nous pouvons
l'être, et il dépend de nous — par la raison, par l’action — de le devenir davantage.
Quant à pouvoir faire, vouloir ou penser autre chose que ce que nous faisons,
voulons ou pensons (ce que suppose le libre arbitre), je n’en ai aucune
expérience, je le répète, ni ne vois comment la chose serait possible. On
m'objectera qu’à ce compte notre liberté n’est que relative, toujours dépendante
(du corps ou de la raison, de l’histoire ou du vrai), toujours déterminée,
et j'en suis d’accord. C’est dire, contre Sartre, que la liberté n’est jamais infinie
ni absolue. Mais comment le serait-elle, en des êtres relatifs et finis, comme
nous sommes tous ? Nul n’est libre absolument, ni totalement. On est {\it plus ou
moins libre} : c’est pourquoi on peut philosopher (parce qu’on est un peu libre),
et c’est pourquoi on le doit (pour le devenir davantage). La liberté n’est pas
donnée, elle est à conquérir. Nous ne sommes pas « condamnés à la liberté »,
comme le voulait Sartre, mais point non plus à l'esclavage. Ce n’est pas la
liberté qui est « le fondement du vrai », comme disait encore Sartre (si c'était
vrai, il n’y aurait plus de vérité du tout) ; c’est la vérité qui libère. Ainsi la
liberté est moins un mystère qu’une illusion ou un travail. Les ignorants sont
d'autant moins libres qu'ils se figurent davantage l’être. Au lieu que le sage le
devient, en comprenant qu’il ne l’est pas.

Encore faut-il rappeler que nul n’est sage en entier — que la liberté est
moins une faculté qu’un processus. On ne naît pas libre ; on le devient, et l’on
n’en a jamais fini. C’est parce que le libre arbitre n’existe pas qu’il faut se libérer
toujours, et d’abord de soi. C’est parce que la liberté n’est jamais absolue que
la libération reste toujours possible, et toujours nécessaire.

LIBERTÉ DE PENSER C’est moins une liberté de plus, qu’un cas particulier
de toutes : le droit de penser ce qu’on veut,
sans autre contrainte que soi ou la raison. C’est la pensée même, en tant qu’elle
échappe aux préjugés, aux dogmes, aux idéologies, aux inquisitions. Elle n’est
jamais donnée, toujours à conquérir. Sa formule est énoncée par Voltaire, après
Horace, après Montaigne, avant Kant : {\it « Osez penser par vous-même. »}

LIBIDO  {\it Libido}, en latin, c’est le désir, souvent avec un sens péjoratif (l'envie
égoïste, la convoitise, la sensualité, la débauche...). Pascal, dans le
prolongement de saint Jean ({\it 1 Jn}, 21, 16) et de saint Augustin ({\it Confessions}, X,
30-39), y voit un autre nom pour la concupiscence : « Tout ce qui est au
monde est concupiscence de la chair, ou concupiscence des yeux, ou orgueil de

la vie: {\it libido sentiendi, libido sciendi, libido dominandi} » ({\it Pensées}, 545-458 ;
%— 342 —
voir aussi les fr. 145-461 et 933-460). Mais le mot doit son usage moderne à
Freud, qui nomme ainsi l'énergie sexuelle, telle qu’elle se manifeste dans la vie
psychique et quelles que soient les formes — y compris sublimées ou apparemment
désexualisées — qu’elle peut prendre. Plus rien ici de péjoratif : la libido
est « la manifestation dynamique, dans la vie psychique, de la pulsion sexuelle »
({\it Psychoanalyse und Libidotheorie}, 1922, cité par Laplanche et Pontalis). Freud la
distingue parfois de l'instinct de conservation ({\it Introduction à la psychanalyse},
chap. 26), avant de les fondre l’une et l’autre dans la pulsion de vie (les pulsions
d’auto-conservation relevant désormais de la libido, qui s’oppose elle-même à
la pulsion de mort : {\it Essais}, « Au-delà du principe de plaisir », 6). On ne confondra
pas la libido avec le désir sexuel, qui n’est qu’une de ses expressions,
encore moins avec la sexualité génitale, qui n’est qu’une de ses formes. La
libido peut se porter sur le moi aussi bien que sur un objet extérieur, s'investir
dans la sexualité, au sens étroit du terme, aussi bien que dans les affaires, l’art,
la politique ou la philosophie. Son retrait, tel qu’il apparaît dans la mélancolie,
aboutit à «la perte de la capacité d’aimer» ({\it Métapsychologie}, « Deuil et
mélancolie »), voire au suicide. Cela dit peut-être l'essentiel. Qu'est-ce que la
libido ? C’est la puissance de vivre, d’aimer et de jouir, en tant qu’elle est d’origine
sexuelle et susceptible de prendre des formes différentes. C’est le nom
freudien et sexuel du {\it conatus}.

LIBRE ARBITRE La liberté de la volonté, en tant qu’elle serait absolue ou
indéterminée : c’est «le pouvoir de se déterminer soi-même
sans être déterminé par rien » (Marcel Conche, {\it L'aléatoire}, V, 7). Pouvoir
mystérieux, et métaphysique strictement : si on pouvait l’expliquer (par
des causes) ou le connaître (par une science), il ne serait plus libre. On ne peut
y croire qu’en renonçant à le comprendre, ou le comprendre (comme illusion)
qu'en cessant d’y croire. C’est où il faut choisir entre Descartes et Spinoza,
entre Alain et Freud, entre l’existentialisme, spécialement sartrien, et ce que j’ai
appelé, faute de mieux, l’{\it insistantialisme} — entre le libre arbitre et la libération.
On ne confondra pas le libre arbitre avec l’indétermination : un électron,
même à le supposer absolument indéterminé, n’est pas doué pour autant de
libre arbitre (qui suppose une volonté), pas plus qu’un cerveau qui dépendrait
de particules indéterminées ne serait libre pour cela (puisqu'il dépend d’autre
chose que de lui-même). Le libre arbitre n’est pas non plus la spontanéité du
vouloir, qui serait plutôt — comme on voit chez Lucrèce ou les stoïciens — le
pouvoir {\it déterminé} de se déterminer soi-même. Mais il emprunte quelque chose
à l’une et l’autre : il est une spontanéité indéterminée, qui aurait, c’est son mystère
propre, la faculté de se choisir ou de se créer soi, ce qui suppose —
%— 343 —
puisqu'on ne peut choisir que l'avenir — qu’elle se précède inexplicablement
elle-même (le mythe d’Er chez Platon, le caractère intelligible chez Kant, le
projet originel ou l’existence-qui-précède-l’essence chez Sartre). Ce n’est pas ce
que je suis qui expliquerait mes choix ; ce sont mes choix, ou un choix originel,
qui expliqueraient ce que je suis. Je ne serais d’abord rien ; et c’est ce néant qui
choisirait librement ce que j'ai à être et que j’aurai été. « Chaque personne, écrit
Sartre dans {\it L'être et le néant}, est un choix absolu de soi. » Le libre arbitre est ce
choix, ou plutôt (puisqu'il faut être d’abord pour se choisir) cette impossibilité.

LIEU La situation dans l’espace, ou l’espace qu’un corps occupe : c’est le {\it là}
d’un être, comme l’espace est le {\it là} de tous (la somme de tous les
lieux). Les deux notions d’{\it espace} et de {\it lieu} sont solidaires, voire se présupposent
mutuellement, au point qu’on ne puisse les définir, peut-être, sans tomber dans
un cercle. Ce sont deux façons de penser l’extension des corps — qui est une
donnée de l'expérience -, en l’inscrivant dans une limite (le lieu) ou dans 'illimité
(l’espace). Le lieu, disait Aristote, est « la limite immobile et immédiate du
contenant » ({\it Physique}, IV, 4). L'espace serait plutôt le contenant sans limites.

LOGIQUE Ce serait la science de la raison ({\it logos}), si une telle science était
possible. À défaut, c’est l'étude des raisonnements, et spécialement
de leurs conditions formelles de validité. Elle apparaît de plus en plus
comme une partie des mathématiques ; cela n’autorise pas les philosophes à
s’en passer.

{\it LOGOS} Le mot, en grec, pouvait signifier à la fois la raison et le discours.
Par exemple chez Héraclite : « Il est sage que ceux qui ont écouté,
non moi, mais le {\it logos}, conviennent que tout est un. » Ou chez saint Jean : « Au
commencement était le {\it Logos}, et le {\it Logos} était avec Dieu, et le {\it Logos} était
Dieu... » Une parole, donc, mais qui serait celle de la vérité : le discours vrai,
ou la vérité comme discours. Langage ? Raison ? Plutôt l’unité indissoluble des
deux. C’est ce qui fait la pérennité, même en français, du mot : parler de {\it logos},
aujourd’hui, ce serait suggérer qu’il n’y a ni langage sans raison ni raison sans
langage. Ces deux propositions me semblent douteuses, et la seconde, même,
impensable. Si la raison n’existait avant le langage et indépendamment de lui,
comment le langage serait-il advenu ? Dieu, dirait Spinoza, ne parle ni ne raisonne
(ce n’est pas un {\it Logos} : pas un Verbe). Rien de plus rationnel pourtant
que ce Dieu-là. La vraie raison — le vrai rapport du vrai à lui-même — est en
% 344 —
deçà du langage : l’idée vraie ne consiste « ni dans l’image de quelque chose ni
dans des mots, écrit Spinoza ; car l’essence des mots et des images est constituée
seulement par des mouvements corporels, qui n’enveloppent en aucune façon
le concept de la pensée » ({\it Éthique}, II, scolie de la prop. 49). La vraie logique est
muette : non {\it logos}, mais {\it alogos}. Non un Verbe mais un acte. Non un discours
mais un silence. Logique de l’être : onto-logique. Au commencement était
l'action.

LOI Un énoncé universel et impératif. En ce sens, il est clair qu’il n’y a pas
de « lois de la nature » : on n’en parle que par analogie, pour évoquer
ou expliquer certaines régularités observables. La rationalité de l’univers est au
contraire toute silencieuse (pas d’énoncés) et simple (pas d’impératifs). L’identité
est ordre qui lui suffit. Et sa nécessité, s’il fallait la formuler, ne se dirait
qu’à l’indicatif. Les lois humaines tendent vers ce modèle (« tout condamné à
mort aura la tête tranchée. »), sans pouvoir jamais complètement y atteindre.
Faute de pouvoir, justement, ou abus de volonté. D’où l’idée de Dieu, qui
serait une volonté toute-puissante : un indicatif-impératif. L'idée de silence,
poussée à sa limite, nous débarrasse de ces deux anthropomorphismes.

Une loi, c’est ce qui s'impose (la nécessité), ou devrait s'imposer (la règle,
l'obligation). On parle dans le premier cas de lois de la nature ; dans le second,
de lois morales ou juridiques. Les premières, qui ne sont voulues par personne,
s'imposent à tous. Les secondes, qui sont voulues par la plupart, ne s'imposent
à personne : elles n’existent, comme loi, que par la puissance que nous gardons,
malgré elles, de les violer. Si le meurtre ne restait possible, aucune loi n’aurait
besoin de l’interdire. Si la gravitation universelle pouvait être violée, ce ne serait
plus une loi.

C’est le sens juridique qui est premier : la loi est d’abord une obligation
imposée par le souverain. On ne parle de lois de la nature que secondairement,
parce qu’on imagine que la nature obéit elle aussi à quelqu'un, qui serait Dieu.
Toutefois ce n’est qu’une métaphore. La nature n’est pas assez libre pour pouvoir
obéir. Dieu le serait trop pour pouvoir commander.

On s’en voudrait de ne pas citer la définition fameuse que donnait
Montesquieu : « Les lois, dans la signification la plus étendue, sont les rapports
nécessaires qui dérivent de la nature des choses » ({\it L'esprit des lois}, I, 1). On ne
peut pourtant s’en contenter : car comment, si cette définition était bonne, y
aurait-il de {\it mauvaises} lois ? C’est que Montesquieu, comme Auguste Comte
après lui, pense d’abord aux lois de la nature, qui ne sont ni bonnes ni mauvaises,
qui ne sont que «des relations constantes entre les phénomènes
observés » ({\it Discours sur l'esprit positif}, \S 12). Les lois humaines sont d’une autre
%— 345 —
sorte : elles ne s'imposent que pour autant qu’on les impose, ce qui ne garantit
ni qu’elles soient justes ni qu’il faille leur obéir. Ainsi cette notion de loi reste
irréductiblement hétérogène : elle sert surtout à masquer cette dualité même
qui la constitue, afin que le fait semble juste ou que la justice semble faite. C’est
une façon d’échapper au désordre et au désespoir. La vérité est qu’il n’y a que
des faits. Mais qui pourrait la supporter ?

LOISIR Au singulier, c’est la traduction de l’{\it otium} des Anciens : le temps
libre, celui qui ne sert qu’à vivre, celui qui n’est pas dévoré par le
travail. Non la paresse ni le repos, mais la disponibilité, comme une ouverture
au monde et à soi, au présent et à l'éternité : l’espace offert à l’action, à la
contemplation, à la citoyenneté, à l'humanité.
Au pluriel, c’est l’ensemble des divertissements qui permettent de supporter
ce temps libre, le plus souvent en payant pour qu’il cesse de l’être.

LOTERIE Un jeu de hasard, ou le hasard comme jeu. Le mot vient des lots
qu’on y gagne. « C’est le moyen de faire jouer la fortune sans
aucune injustice », écrit Alain. Mais sans aucune justice non plus. Par quoi c’est
une image de la vie, davantage que de la société. « Toute la mécanique de la
loterie va à égaliser les chances, écrit encore Alain. Ainsi la chance se trouve
purifiée. Cent mille pauvres font riche un d’entre eux, sans choix. C’est le
contraire de l'assurance. » C’est que l’assurance est une mutualisation des
risques. La loterie serait plutôt une mutualisation des chances. Qu’on en ait fait
un impôt volontaire montre l'intelligence de nos grands argentiers. Seul l’État
gagne à tout coup. C’est mettre le hasard au service de la comptabilité nationale.

LUCIDITÉ C'est voir ce qui est comme cela est, plutôt que comme on voudrait
que cela soit. Par quoi la lucidité ressemble beaucoup au
pessimisme : non que les choses aillent toujours de pire en pire (pourquoi
serait-ce le cas ?), mais parce qu’il n’est pas d’usage qu’elles aillent comme nous
voudrions, ni habituel que nous voulions qu’elles aillent comme elles vont en
effet. Ainsi la lucidité marque d’abord la distance entre l’ordre du monde et
celui de nos désirs, tout en refusant de renoncer — car alors il n’y aurait plus distance
— à l’un ou à l’autre. C’est l'amour de la vérité, quand elle n’est pas
aimable.

%— 346 —
Cela vaut aussi pour soi. Car enfin se connaître comme on est, c’est presque
toujours se décevoir. Lucidité bien ordonnée commence par soi-même : tel est
le secret de l'humilité.

LUMIÈRES Le mot désigne une période en même temps qu’un idéal. La
période, c’est le {\footnotesize XVIII$^\text{e}$} siècle européen. L'idéal, c’est celui de la
raison, que Descartes appelait déjà « la lumière naturelle », mais délivrée de
toute théologie, voire de toute métaphysique, c’est celui de la connaissance,
celui du progrès, de la tolérance, de la laïcité, de l’humanité lucide et libre. Être
un homme des Lumières, explique Kant, c’est penser par soi-même, c’est se
servir librement de sa raison, c’est se libérer des préjugés et de la superstition.
C'est ce qui justifie la formule fameuse : « {\it Sapere aude} ! aie le courage de te
servir de ton propre entendement! voilà la devise des Lumières » (Kant,
{\it Réponse à la question « Qu'est-ce que les Lumières ? »}, 1784). La maxime est
d’Horace, elle pourrait être de Lucrèce, et Montaigne déjà l’avait faite sienne
({\it Essais}, I, 26, 159), comme Voltaire après lui et avant Kant ({\it Dictionnaire philosophique},
article « Liberté de penser »). On voit que les Lumières, en tant
qu’idéal, sont de tous les temps. C’est que la superstition et le dogmatisme toujours
nous précèdent, et nous accompagnent.

LUXE C'est jouir de l’inutile. Par exemple une cuillère : qu’elle soit en or,
cela ne sert à rien, mais c’est un plaisir supplémentaire. Notion par
nature relative, mais qui suppose toujours une part d’excès, de surabondance,
d’{\it exagération}, comme dit Kant, dans le confort, la joliesse ou la dépense. C’est
le contraire des plaisirs naturels et nécessaires d’Épicure : le luxe est fait de plaisirs
culturels et superflus. C’est pourquoi c’est un piège, s’il devient nécessaire,
et un luxe, s’il reste superflu.

LUXURE L'usage immodéré des plaisirs sexuels. Ce n’est le plus souvent
qu’une petite faute, mais c’en est une : non parce que le sexe serait
coupable, il l’est rarement, mais parce que l’intempérance l’est toujours. Faute
vénielle, du moins entre partenaires consentants. Mais l’obligation d’obtenir ce
consentement interdit de s’abandonner tout à fait à la luxure. Au reste le corps,
presque toujours, fait une limite suffisante. Si Sade n’avait passé tant d’années
en prison, s’il avait fait davantage l'amour, et plus heureusement, le plaisir
même lui aurait enseigné — contre le faux infini du manque ou de l’imaginaire
%— 347
— son très positif et très voluptueux pouvoir de {\it modération}. La luxure n’est pas
un luxe ; l'érotisme en est un.

LYCÉE C'est le nom, en grec, de l’école d’Aristote, parce qu’il enseignait
dans un gymnase proche d’un temple ou d’une statue d’Apollon
Lycien ({\it lykeios}, loup dieu). Aristote y délivrait son enseignement, aussi bien
acroamatique, le matin, qu’exotérique, le soir, dans le promenoir ({\it peripatos} :
d’où le nom d’école péripatéticienne qu’on donne aussi au Lycée) du gymnase.
L'école, à la mort d’Aristote (ou plutôt dès son départ en quasi exil, dans l’île
d’Eubée, où il mourra quelques mois plus tard), sera dirigée par Théophraste,
puis, à la mort de celui-ci, par Straton de Lampsaque. Après quoi elle connut
un déclin progressif, qui n’empêchera pas Andronicos de Rhodes, son dixième
et dernier scholarque, de publier, au I“ siècle avant Jésus-Christ, les œuvres ésotériques
du Maître, lesquelles reprennent pour l'essentiel son enseignement du
matin et sont parvenues, grâce à cette édition, jusqu’à nous. Quant aux textes
exotériques, que les Anciens admiraient fort (Cicéron compare le style d’Ariscote
à un « fleuve d’or », Quintilien en loue la grâce et la douceur), il n’en reste
à peu près rien. De là l’image d’un Aristote exclusivement professoral ou technicien,
qui aurait omis d’être artiste : philosophe professeur, pour les professeurs
de philosophie. L’injustice n’est pas trop grave : même amputé de la
moitié de son œuvre, ce philosophe reste le plus grand de tous — et le Lycée le
modèle, à jamais, de l’exigence intellectuelle. De là un peu de nostalgie, parfois,
quand on pense à nos lycées d’aujourd’hui. Nouvelle injustice : l’éducation de
masse est évidemment un progrès, et l’on ne saurait pas davantage demander à
nos enseignants d’égaler Aristote qu’à leurs élèves de mettre le savoir, dans une
société qui n’y croit plus guère, plus haut que tout. Ce n’est pas une raison
pour remplacer la lecture des grands auteurs par celle des journaux, ni le travail
par le débat, ni l’amour de la vérité par celui de la communication. Le Lycée
n'était pas l’agora : c'était un lieu d’étude, d’enseignement, de réflexion, bien
plus que d’échanges ou de « spontanéité ». Nos lycées ne méritent leur nom
qu’en restant fidèles, au moins de ce point de vue, au grand modèle auquel ils
doivent leur appellation et une partie, malgré tout, de ce qu’on y enseigne.
Mieux vaut rivaliser avec Aristote, même de très loin, qu’avec la télévision.

LYMPHATIQUE L’un des quatre tempéraments de la tradition hippocratique.
Mollesse, lenteur, inattention.
%{\footnotesize XIX$^\text{e}$} siècle — {\it }


%

%{\footnotesize XIX$^\text{e}$} siècle — {\it }

MACHIAVÉLISME Une forme de cynisme, mais qui sacrifie la morale à la
politique : l'opposé par là, ou le symétrique, du cynisme
de Diogène.

Le mot, qui se prend ordinairement en mauvaise part, vise surtout une certaine
façon, qu’on trouve en effet chez Machiavel, de juger une action à ses
résultats plutôt qu’à sa moralité intrinsèque (« Si le fait l’accuse, le résultat
l’excuse », {\it Discours}, I, 9), et de s’autoriser pour cela, dans l’ordre politique, plusieurs
actions qui seraient, du point de vue de la morale ordinaire, répréhensibles.
C’est considérer que la fin justifie les moyens, et que la ruse, quand elle
est efficace, vaut mieux qu’une droiture qui ne le serait pas. Le machiavélisme
dit ainsi la vérité de la politique : « Il y a si loin de la sorte qu’on vit à celle selon
laquelle on devrait vivre, écrit Machiavel, que celui qui laissera ce qui se fait
pour cela qui se devrait faire, il apprend plutôt à se perdre qu’à se conserver ;
car qui veut faire entièrement profession d'homme de bien, il ne peut éviter sa
perte, parmi tant d’autres qui ne sont pas bons ; aussi est-il nécessaire au Prince
qui se veut conserver, qu’il apprenne à pouvoir n'être pas bon, et d’en user ou
pas selon la nécessité » ({\it Le Prince}, XV). Les médiocres y voient une justification
de limmoralité, de la perfidie, de l’arrivisme sans vergogne, de ce que
Machiavel, qui n’écrivait pas pour les médiocres, appelle la {\it scélératesse}. C’est
bien sûr se méprendre. Un scélérat au pouvoir reste un scélérat.

MACHINE «Si les navettes tissaient d’elles-mêmes, écrivit un jour Aristote,
alors les artisans n'auraient pas besoin d’ouvriers, ni les
maîtres d'esclaves » (Politique, 1, 4). Cela dit à peu près ce que c’est qu’une
machine : un objet animé, mais sans âme (un automate), capable de fournir un
%— 349 —
certain travail, autrement dit d’utiliser efficacement l'énergie qu’il reçoit ou
dont il dispose. Ainsi un métier à tisser, un lave-linge, un ordinateur. C’est en
ce sens que les animaux, pour Descartes, et l’homme, pour La Mettrie, sont des
machines. Non parce qu’ils seraient dépourvus d’intelligence et de sensibilité,
comme le premier l’a cru bêtement des bêtes, encore moins parce qu’ils seraient
faits de vis et de boulons (pourquoi une machine ignorerait-elle les cellules, les
organes, les échanges biologiquement organisés d'énergie et d’informations ?),
mais parce qu'ils sont {\it sans âme}, autrement dit sans autre réalité substantielle
que matérielle. En ce sens {\it L'homme-machine}, de La Mettrie, énonce l’une des
thèses les plus radicales du matérialisme : nous ne sommes que « des animaux
et des machines perpendiculairement rampantes », comme il dit étonnamment,
mais vivantes (La Mettrie était médecin), conscientes (grâce au cerveau, qui est
comme une machine particulière dans la machine globale de l'organisme), et
capables pour cela de souffrir et de jouir, de connaître et de vouloir, enfin d’agir
et d'aimer. « Nous ne pensons, et même nous ne sommes honnêtes gens, que
comme nous sommes gais ou braves ; tout dépend de la manière dont notre
machine est montée » ({\it L'homme-machine}, Éd. Fayard, p. 70-71).

MAGIE Une action qui excéderait les lois de la nature ou de la raison

ordinaires : du surnaturel efficace, ou une efficacité surnaturelle,
mais qui obéirait à notre volonté (par différence avec la grâce et le miracle, qui
n’obéissent qu’à Dieu) ou serait instrumentalisé par elle. Cette efficacité, même
lorsqu’elle semble avérée (par exemple dans le chamanisme : une parole qui tue,
un rite qui guérit), suppose toutefois la croyance, ce qui est très naturel et très
raisonnable : ce n’est pas magie, mais suggestion. « L'efficacité de la magie,
écrit Lévi-Strauss, implique la croyance en la magie » ({\it Anthropologie structurale},
IX). Raison de plus pour ne pas y croire.

MAGNANIMITÉ La grandeur d'âme : se juger digne de grandes choses,
explique Aristote, et l’être en effet ({\it Éthique à Nicomaque},
IV, 7-9). C’est la vertu des héros, comme l’humilité est celle des saints. Vertu
grecque, contre vertu chrétienne.

La magnanimité s’oppose à la fois à la bassesse (ou pusillanimité : se juger
incapable d’une grande action, et l’être par là même) et à la vanité (avoir les
yeux ou le discours plus grands que l’âme : se croire ou se prétendre capable de
ce qu'on est en vérité incapable de réussir). Elle correspond assez bien à
l’{\it acquiescentia in se ipso} de Spinoza (la satisfaction intime, le contentement
lucide, l'amour heureux de soi-même) : « une joie née de ce qu’un homme se
%— 350 —
considère lui-même et sa puissance d’agir » ({\it Éthiqu}e, VII, déf. 25 des affects ;
voir aussi IV, 52, dém. et scolie). Toutefois la magnanimité peut aller sans la
joie, comme on voit chez Athos : ce n’est plus sagesse, c’est toujours vertu.

MAÏEUTIQUE {\it Maîa}, en grec, c’est la sage femme. C’est à elle que Socrate,
dans le {\it Théétète}, se compare : la maïeutique est l’art
d’accoucher les esprits, autrement dit d’en faire sortir — par le questionnement
et le dialogue — une vérité qu’ils contiennent sans la connaître. L'exemple classique
est celui du jeune esclave du {\it Ménon} : Socrate l’amène à découvrir comment
obtenir un carré double d’un autre (en le construisant sur la diagonale du
premier), alors même qu’il l’ignorait et sans avoir besoin pour cela de lui
apprendre quoi que ce soit. C’est supposer que la vérité est déjà en nous, ou
nous en elle : réminiscence, ou éternité.
En pratique, la maïeutique atteint vite ses limites. Interroger un ignorant,
cela ne saurait suffire à l’éduquer. Le modèle socratique, dans nos classes, n’est
souvent qu’une utopie de plus.

MAÎTRE Celui qui enseigne, guide ou commande. Les trois à la fois ? Pas
nécessairement. Cela dépend en partie de lui, de ce qu’il sait ou
peut, mais aussi de ceux dont il est le maître : sont-ce des élèves, des disciples
ou des esclaves ?

MAJESTÉ Une grandeur visible, qui justifierait à la fois le respect et l’obéissance.
C’est ce qui faisait dire à Alain qu'il était « l’ennemi de
toute espèce de Majesté ». Et d’en donner cette définition parfaite: « La
Majesté, c'est tout ce qui, ayant le pouvoir, veut encore être respecté. » C’est
vouloir régner sur les esprits aussi. Toute majesté est ridicule ou tyrannique.

MAJEURE Dans un syllogisme, celle des deux prémisses qui contient le
grand terme. On la place traditionnellement en premier ; mais
c'est son contenu qui importe, non sa place.

MAL On ne disculpera pas Dieu à si bon compte. Le mal n’est pas seulement
l’absence d’un bien (absence que Dieu ne tolérerait qu’autant
que nécessaire, pour créer autre chose que soi), mais son contraire. Ainsi la
%— 351 —
souffrance est un mal (qui ne se réduit pas à l’absence de plaisir), et le modèle
de tous : le mal, c’est d’abord ce qui {\it fait} mal. Et Dieu n’est innocent qu’à la
condition stricte de n’exister pas.

Le mal existe positivement : non certes parce qu’il serait une réalité objective
ou absolue (il n’y a de mal que pour un sujet), mais parce qu’il constitue,
pour tout sujet, une expérience première. Point besoin, pour souffrir, d’avoir
connu le plaisir. Il est vraisemblable au contraire que le bien ne vienne
qu'après, et secondairement, de ce que l’expérience même du mal fait désirer
sa disparition, et la rend agréable. Épicure est sublime ici de simplicité. Ce
n'est pas le mal qui est l'absence du bien ; c’est le bien qui est l’absence du
mal.

Cela dut être vrai pour l'humanité comme pour l'individu. Le mal est premier.
Et la peur — notre mère la peur — engendre en nous l’espérance, mais aussi
le courage.

Le mal, disais-je, c'est d’abord ce qui fait mal : la souffrance est le mal premier,
et le pire. Toutefois, ce n’est pas le seul : une bassesse indolore, et même qui
serait agréable à tous, ne cesse pas pour cela, moralement, d’être mauvaise. C’est
donc qu’il y a autre chose que la souffrance. Quoi ? « Une certaine idée de
l’homme, comme dit Spinoza, qui soit comme un modèle de la nature humaine
placé devant nos yeux » ; le mal, ou le mauvais ({\it malum}), c’est ce qui nous éloigne
de ce modèle ou nous empêche de le reproduire ({\it Éthique}, IV, Préface).

« On peut prendre le mal métaphysiquement, physiquement et moralement »,
écrivait Leibniz : « Le {\it mal métaphysique} consiste dans la simple imperfection,
le {\it mal physique} dans la souffrance, et le {\it mal moral} dans le péché »
({\it Théodicée}, Y, 21). Qu'en reste-t-il pour l’athée ? Les deux premiers restent à
peu près inentamés. L’imperfection du monde et l’ampleur de la souffrance
font même partie de nos plus fortes raisons de ne pas croire en Dieu. {\it « Si
Deus est, unde malum ? si non est, unde bonum ? »}, demandait Leibniz (I, 20 :
Si Dieu existe, d’où vient le mal ? s’il n’existe pas, d’où vient le bien ?). La
première des deux interrogations semble la plus redoutable. D'abord parce
que le mal l'emporte, en force et en fréquence, ensuite parce que la puissance
indéfinie et imparfaite de la nature est mieux à même d’expliquer le bien
qu'elle comporte que l’infinie et toute-puissante bonté de Dieu ne le serait de
justifier le mal qu’elle tolère. La souffrance serait un châtiment ? Le prix à
payer de notre liberté, de nos fautes ? Comment l’accepter, puisque le mal est
antérieur à la culpabilité et même — les bêtes souffrent aussi — à l'humanité ?
Mieux vaut la révolte, ou plutôt mieux vaut pardonner à Dieu de n’exister
pas.

Qu'en est-il alors, pour finir, du mal moral ? S'il n’est plus {\it péché}, au sens
religieux du terme, c’est-à-dire offense faite à Dieu ou violation d’un de ses
%— 352 —
commandements, il reste à le penser, conformément à la lettre et à l'esprit du
spinozisme, comme ce qui nous éloigne de notre idéal d'humanité ou nous
empêche de le reproduire ({\it Éthique}, IV, Préface ; voir aussi les {\it Lettres} 19, 21 et
23, à Blyenbergh). C’est pécher encore, mais contre l'humanité ou contre soi.
Le mal, c’est ce qui nous empêche d’être pleinement {\it humains}, au sens normatif
du terme, c’est-à-dire accessibles à la raison, quand nous en sommes capables,
ou à la compassion, quand la raison ne suffit pas. « Quant à celui qui n’est
poussé ni par la raison ni par la compassion à être secourable aux autres, on
l'appelle justement inhumain, car il ne paraît pas ressembler à un homme »
({\it Éthique}, IV, scolie de la prop. 50).

MALÉDICTION C’est vouer au mal par des mots. On a bien tort d’y voir
de la magie, quand il n’y a là que haine et superstition. Le
mieux serait d’en rire. Un bras d'honneur, si l’on n’est pas capable d’indifférence,
fait un exorcisme suffisant.
Quant à maudire les méchants, cela ne sert à rien. L'action vaut mieux.

MALHEUR Je l'ai vécu juste assez pour savoir ce qu’il est : le malheur, c’est

quand toute joie paraît impossible, quand il n’y a plus que
l'horreur et l'angoisse, que la douleur, que le chagrin, quand on voudrait être
mort, quand vivre n’est plus que survivre et endurer, que souffrir et pleurer.
Se rappeler, dans ces moments-là, que tout est impermanent : ce malheur passera
aussi. Et que sa réalité suffit à prouver, au moins par différence, au moins
pour les autres, la possibilité du bonheur. Ce n’est pas une consolation ? Dans
les pires moments, il m’a semblé que si. Que le malheur tombe sur moi ou sur
un autre, qu'est-ce que cela change d’essentiel ? Consolation insuffisante ? Si
elle ne l'était pas, ce ne serait pas un malheur.

MALVEILLANCE C'est vouloir du mal à quelqu'un, soit par pure méchanceté,
si nous en sommes capables, soit, plus vraisembla-
blement, par haine ou par intérêt (par égoïsme). C’est vouloir le mal, sinon
pour le mal, mais en le sachant tel. Nul n’est méchant volontairement, ni malveillant
involontairement.

MANICHÉISME C'est d’abord une religion, apparue entre la Mésopotamie
et la Perse, au Hi siècle de notre ère, sous la dynastie des
%— 353 —
Sassanides. Mani, son fondateur, voulut inventer ou transmettre une religion
universelle. Tout en s’inspirant de ses propres visions ou révélations, il tenta
pour cela une espèce de synthèse entre trois religions déjà existantes, qui lui
semblaient aller dans le même sens: l’antique religion persane, celle de
Zoroastre, le christianisme (Mani prétend être le Paraclet annoncé par Jésus)
et le bouddhisme. Sa doctrine, telle qu’on peut à peu près la reconstituer,
était un dualisme gnostique et sotériologique. Le manichéisme oppose en
effet deux principes coéternels — la Lumière et les Ténèbres, le Bien et le Mal,
l'Esprit et la Matière —, qui ne cessent ici-bas de se mêler et de se combattre :
l’âme est le lieu et l’enjeu, en l’homme, de cet affrontement. Cette nouvelle
religion, qui avait ses Écritures, sa liturgie, son Église, fut bientôt combattue
par la force (Mani, d’abord protégé par Shâpur I‘, mourra en prison, sous le
roi Bahrâm I$^\text{er}$, qui voulait restaurer le mazdéisme comme religion d’État).
Elle se répandit pourtant pendant quelques siècles, aussi bien vers l'Afrique
et l’Europe que vers la Chine et l’Inde, avant de disparaître ou de se dissoudre,
sans qu’on sache trop pourquoi ni comment, dans les religions plus
anciennes dont elle s’inspirait ou dans d’autres, plus neuves (spécialement
l'Islam), qui finirent par la recouvrir. Il en reste une tentation gnostique ou
dualiste, repérable dans la plupart des grandes religions, dès lors qu’elles diabolisent
— parfois officiellement, plus souvent sous forme d’hérésies — le
monde ou le corps. Saint Augustin lui-même, qui combattit si vigoureusement
les manichéens de son temps, n’en fut pas toujours exempt : le jansé-
nisme, qui se réclamera à juste titre de l’Évêque d’Hippone, doit sans doute
quelque chose de sa belle intransigeance à ce qu’on pourrait interpréter, au
moins à certains égards, comme un retour du refoulé manichéen.. Mais ce
sont les Cathares, en Occident, qui porteront le plus haut le flambeau dualiste
et gnostique. Ils seront éliminés, on sait avec quelle sauvagerie. Sans
doute pensèrent-ils, sur le bûcher, que leur défaite même leur donnait
raison.

En un sens second, on parle de manichéisme pour qualifier une pensée
qui oppose le Bien et le Mal de façon absolue, comme si tout le bien était
d’un côté (par exemple dans tel camp politique) et tout le mal d’un autre (par
exemple dans le camp opposé). Ce dernier sens est toujours péjoratif. Qu'un
camp soit absolument mauvais, cela peut arriver (le nazisme en est un
exemple commode), mais n’autorise pas à penser que l’autre soit bon absolument.
Quand bien même Hitler serait le diable, cela ne saurait faire de Staline
ou de Roosevelt des anges. Par quoi tout manichéisme, appliqué à la
politique, est bête et dangereux : c’est adorer son propre camp, quand il ne
faudrait que le soutenir.

%— 354 —
MANIÉRISME C'est une exagération du style, qui mène ordinairement au
baroque. Formes graciles ou allongées, compositions recherchées
voire alambiquées, sentiments délicats ou rares, parfois évanescents, parfois
exacerbés, comme un raffinement qui hésiterait entre la grâce et l’outrance,
entre la poésie et l’affectation, entre la préciosité et l’expressionnisme. C’est
vouloir imiter la manière des maîtres, tout en voulant les dépasser (en allant
plus loin qu’eux, sinon plus haut). Cela vaut mieux que l’académisme, qui
renonce à dépasser ; mais moins que le classicisme, qui n’imite que la nature ou
les Anciens.

L'époque maniériste proprement dite est le {\footnotesize XVI$^\text{e}$} siècle, d’abord en Italie (le
Pontormo, Jules Romain, Giambologna, le Parmesan, le Tintoret....) puis dans
le reste de l'Europe (spécialement avec le Greco, en Espagne, mais aussi avec un
certain nombre d'artistes, souvent d’origine italienne, de l’école de
Fontainebleau : le Rosso, le Primatice, Jean Cousin...). On peut pourtant
parler de maniérisme pour d’autres artistes (il y a du maniérisme, à certains
égards, chez Botticelli ou Dürer) ou pour d’autres époques, comme le {\footnotesize IV$^\text{e}$} siècle
avant Jésus-Christ, en Grèce, ou le début du {\footnotesize XX$^\text{e}$}, en Europe. Le maniérisme est
la tentation des tard venus, qui doivent rivaliser avec plus fort qu'eux — par
exemple avec Phidias ou Michel-Ange. Ils s’en sortent par un excès de
recherche, de virtuosité, de sophistication, au service d’une sensibilité « artiste »
ou mondaine. C’est préférer la grâce à la beauté, le style à la vérité, enfin l’art à
la nature. Esthétique {\it maniérée et somptueuse}. C’est une décadence recherchée.

MARCHÉ - Quand tu achètes une baguette chez ta boulangère, me
demande un ami économiste, pourquoi te la vend-elle ?

— C’est son métier...

— C’est surtout son intérêt! Elle préfère avoir 4 F 20 plutôt qu’une
baguette...

— C’est normal : la baguette lui a coûté beaucoup moins cher.

— Exactement. Et pourquoi est-ce que tu lui achètes sa baguette ?

— Parce que j’ai besoin de pain...

— Sans doute. Mais tu pourrais faire ton pain toi-même. La vraie raison,
c’est que tu préfères avoir une baguette plutôt que 4 F 20.

— Bien sûr! La baguette, si je devais la fabriquer moi-même, me reviendrait,
temps de travail compris, beaucoup plus cher.

— Tu commences à comprendre ce que c’est que le marché. Elle te vend
une baguette par intérêt, tu l’achètes par intérêt, et chacun de vous y trouve son
compte. C’est le triomphe de l’égoisme...

%— 355 —
— C’est surtout le triomphe de l'intelligence ! Faire son pain, passe encore.
Mais qui pourrait se fabriquer une voiture ou une machine à laver ? C’est ce
que Marx appelle la division du travail...

— Adam Smith en avait parlé avant lui. Or Smith, ici, est plus éclairant que
Marx.

— Je te vois venir : éloge du libéralisme...

— Essayons plutôt de comprendre. Je reviens à ta boulangère. Tu pourrais
aussi bien aller chez un de ses concurrents. Pourquoi vas-tu chez elle ?

— Parce que son pain est meilleur.

— Elle a donc intérêt à faire le meilleur pain possible. Mais l’achèterais-tu à
n'importe quel prix ?

— Sans doute pas.

— Pour te garder comme client, elle a donc intérêt, dans une économie con-
currentielle, à t'offrir le meilleur rapport qualité-prix possible. C’est aussi ce
que tu souhaites. Vos intérêts ne sont pas seulement complémentaires, ils sont
convergents !

— Pas étonnant qu’elle me sourie si gentiment...

— Ni que tu sois si poli avec elle! Chacun de vous deux n’agit que par
égoïsme, mais cela, loin de vous opposer, vous rapproche. Pourquoi serait-on
désagréable avec celui dont on a besoin ? Mais que le pain soit moins bon ou
plus cher qu’à la boulangerie voisine, ou que tu ne puisses plus payer, c'en est
terminé de votre relation : tu ne lui dois rien, ni elle à toi, qu’autant que vous
y trouvez l’un et l’autre votre compte. C’est ce qu’on appelle le marché : la rencontre
de l’offre et de la demande, autrement dit la libre convergence — par la
médiation de l'échange et sous réserve de la concurrence — des égoïsmes.
Chacun est utile à l’autre, sans qu’on ait besoin de le forcer. Tous ne cherchent
que leur propre intérêt, mais ne peuvent le trouver qu’ensemble. C’est pourquoi
le pain est meilleur et plus abondant dans une économie libérale que dans
une économie collectiviste. La convergence des égoïsmes est plus efficace que
les contrôles et la planification !

— Tu enfonces une porte ouverte...

— Elle ne la pas toujours été !

— Elle l’est désormais, depuis des décennies. Qui voudrait fixer un prix par
force ou par décret ? Ce ne serait plus marché, mais racket ou police. La misère,
dans les deux cas, est au bout, et les queues immenses devant des magasins
presque vides...

— Je ne te le fais pas dire ! Mais alors, il faut en tirer les conséquences. Ce
que tu appelais le triomphe de l'intelligence, c’est le triomphe du marché.

— C’est surtout le triomphe de la solidarité.

— Revoilà ta morale et tes idées de gauche...

%— 356 —
— Qui te parle de morale ? Si je devais compter sur la générosité de ma boulangère
pour avoir du pain, je serais mort de faim depuis longtemps ! Comme
elle, si elle devait compter sur ma générosité pour avoir de l'argent. Au
contraire, si nous comptons chacun sur l’égoïsme de l’autre, nous ne serons
jamais déçus !

— C’est ce que j'appelle le marché...

— C’est ce que j'appelle la solidarité : non le contraire de l’égoïsme, comme
la générosité, mais sa socialisation bien réglée. Non le désintéressement, mais la
convergence des intérêts. C’est pourquoi la générosité, moralement, vaut mieux
(elle est désintéressée). Et c’est pourquoi la solidarité, socialement, économiquement,
est beaucoup plus efficace.

— Alors il faut dire que le marché est un formidable créateur de solidarité.
Tes amis de gauche ne vont pas être contents !

— À moins qu’ils ne soient en train de le comprendre. Tu en connais encore
beaucoup qui veulent étatiser l’économie ?

— Peut-être pas. Mais ils croient davantage aux lois et aux impôts qu’au
marché et à la concurrence...

— C’est que le marché ne vaut que pour les marchandises !

— Pour les marchandises et les services.

— Disons pour tout ce qui se vend et qui s’achète. Un service, dès lors qu’il
est à vendre, n’est qu’une marchandise comme une autre. Mais la santé ? Mais
la justice ? Mais l'éducation ? Si tu penses qu’elles sont à vendre, soumets-les au
marché ! Que restera-il de notre société, de nos idéaux et du droit des plus
faibles ? Si au contraire, comme je le crois, la justice n’est pas à vendre, ni la
liberté, ni la santé, ni l'éducation, ni la dignité..., il faut en conclure qu’elles ne
sont pas des marchandises. Le marché, sur elles, est donc sans pertinence, sans
légitimité, sans valeur. On peut aller plus loin. Le monde non plus n’est pas à
vendre (c’est ce qui donne raison aux écologistes : « Le monde n’est pas une
marchandise »). Le marché, même, n’est pas à vendre : c’est ce que signifie le
droit du commerce. C’est pourquoi nous avons besoin de politique : parce que
le marché est nécessaire (or il ne peut se développer vraiment que dans un État
de droit) et parce qu’il ne suffit pas. Quelle folie ce serait que d’abandonner au
marché ce qui n’est pas à vendre ! Autant abandonner à l’État, autre folie, la
fabrication du pain et le sourire de ma boulangère.

— Les médicaments, cela s’achète !

— Mais on ne peut accepter qu’ils soient réservés à ceux qui ont les moyens
de les payer. C’est pourquoi on a inventé la Sécurité sociale.

— ... Et les impôts !

— Tu préférerais qu’on compte sur la charité des plus riches pour que les
plus pauvres puissent se soigner ? Autant compter sur la générosité de ta boulangère
%— 357 —
 pour avoir du pain ! Personne ne cotise à la Sécu ou ne paie ses impôts
par générosité. Nous le faisons tous par intérêt, et il faut bien que quelques
contrôles nous y poussent... Moyennant quoi la fiscalité et la Sécurité sociale
ont fait beaucoup plus, pour la justice, que le marché et la générosité réunis. Ce
n'est pas de la morale, mais de la politique : pas de la charité, mais de la
solidarité !

— Peut-être. Mais s’il n’y avait pas le marché pour créer de la richesse, l’État
n'aurait rien à redistribuer...

— Et s’il n’y avait pas l’État pour garantir le droit de propriété et la liberté
des échanges, il n’y aurait pas de marché du tout.

— Alors ne demandons pas à l’État de produire de la richesse : le marché le
fait plus et mieux !

— Ni au marché de produire de la justice : seul l’État a une chance d'y
parvenir !

— Soyons donc libéraux en économie...

— Et solidaires en politique !

MARTYR « Je ne crois que les témoins qui se feraient égorger », dit à peu
près Pascal ({\it Pensées}, 822-593). Cela vaut presque comme définition :
un martyr — c’est-à-dire, étymologiquement, un témoin -, c’est
quelqu’un qui se fait tuer pour qu’on le croie. Mais qu'est-ce que cela prouve ?
Plusieurs de ses assassins se feraient volontiers égorger aussi. Tant d’enthousiasme
ou de fanatisme me rendrait plutôt son témoignage suspect : s’il met sa
foi plus haut que sa vie, il y a lieu de craindre qu’il la mette aussi plus haut que
le bon sens et la lucidité. Galilée, sauvant sa peau contre l’Inquisition, m’inspire
autrement confiance : il eût été bien bête de mourir ; la Terre n’en aurait
pas tourné davantage.

En un autre sens, le martyr est seulement celui qu’on assassine ou qu’on
torture. Ce n’est plus un témoin, c’est une victime. Plus besoin d’être d’accord
avec lui ; l’urgent est de lui porter secours. Logique de humanitaire : logique
de la compassion, non de la foi.

MARXISME La doctrine de Marx et d’Engels, puis le courant de pensée,
passablement hétérogène, qui s’en réclame. C’est un matérialisme
dialectique, appliqué surtout à l’histoire : celle-ci serait soumise à des
forces exclusivement matérielles (principalement économiques, mais aussi
sociales, politiques, idéologiques...) et mue par un certain nombre de contradictions
(entre les forces productives et les rapports de production, entre les
%— 358 —
classes, entre les individus...). La lutte des classes est le moteur de l’histoire, qui
mène nécessairement — on reconnaîtra là une {\it aufhebung} très hégélienne — à une
société sans classe et sans État, le communisme, dont nous ne sommes plus
guère séparés que par une dernière révolution et une dernière dictature (celle
du prolétariat)... Science ? Philosophie ? Ce serait l’une et l’autre, qu’on distingue
parfois sous les deux appellations de {\it matérialisme historique} et de {\it matérialisme
dialectique}, dont la conjonction serait le marxisme même. Cela aboutira
à des dizaines de milliers d'ouvrages, aujourd’hui presque tous illisibles,
mais qui forment pourtant, c’est la moindre des choses, un massif théorique
qui reste impressionnant. Quant au corps de la doctrine, Marx en avait donné
lui-même un résumé fameux, qui mérite d’être cité largement :

\vspace{0.5cm}

{\footnotesize « Le mode de production de la vie matérielle conditionne le processus de vie social,
politique et intellectuel en général. Ce n’est pas la conscience des hommes qui détermine
leur être ; c’est inversement leur être social qui détermine leur conscience. À un
certain stade de leur développement, les forces productives matérielles de la société
entrent en contradiction avec les rapports de production existants, ou, ce qui n’en est
que l’expression juridique, avec les rapports de propriété au sein desquels elles s'étaient
mues jusqu'alors. De formes de développement des forces productives qu’ils étaient, ces
rapports en deviennent des entraves. Alors s'ouvre une époque de révolution sociale.
[...] À grands traits, les modes de production asiatique, antique, féodal et bourgeois
moderne peuvent être qualifiés d’époques progressives de la formation sociale économique.
Les rapports de production bourgeois sont la dernière forme contradictoire du
processus de production sociale [...]. Les forces productives qui se développent au sein
de la société bourgeoise créent en même temps les conditions matérielles pour résoudre
cette contradiction. Avec cette formation sociale s’achève donc la préhistoire de la
société humaine » ({\it Critique de l'économie politique}, Préface).
}

\vspace{0.5cm}

J'ai beau ressentir pour Marx beaucoup d’admiration et de sympathie, cette
dernière phrase me fait froid dans le dos. Cette façon d’annuler tout le passé,
qui ne serait que préhistoire, au nom de l’avenir, au nom d’une histoire enfin
véritable mais qui n’aurait pas encore véritablement commenté, j'y reconnais
trop la structure mortifère de l’utopie, cette volonté de donner tort au réel, de
l’invalider, de le réfuter (utopie comme forclusion du réel : comme psychose
historique), avant de fusiller, au nom des lendemains qui chantent, le triste et
pleurnichard aujourd’hui... On me dira qu’on a le droit de rêver, et même
qu’il le faut. Sans doute. Mais faut-il pour cela prétendre que toute veille,
jusque-là, ne fut qu’un long, qu’un très long et très mensonger sommeil ? Et de
quel droit ériger ce rêve en certitude prétendument démontrée ? Que Marx ait
rêvé une autre politique, qu’il Pait désirée, voulue, préparée, ce n’est pas moi
qui le lui reprocherai. Son erreur fut d’y voir une science, sans renoncer pour
%— 359 —
cela à sa vertu prescriptive : le marxisme dirait à la fois la vérité de ce qui est (le
capitalisme) et de ce qui {\it doit} être (le communisme). De là un penchant originellement
dogmatique et virtuellement totalitaire. Staline y fera son lit, ou son
trône. La vérité ne se vote pas, et ne se discute valablement qu'entre esprits
compétents. S’il existe une politique scientifique (or le marxisme, spécialement
dans sa version léniniste, prétendra être cette science), à quoi bon la
démocratie ? Autant voter pour savoir s’il fera beau demain. Et quel scientifique
voudrait respecter les opinions, dans une science donnée, de ceux, même
sincères, qui n’y connaissent rien ? On a le droit de se tromper, mais cela
n’appelle que correction et travail, qu’explications ou redressement — que pédagogie
ou thérapie. Tout désaccord devient l’indice d’un conflit d’intérêts ou
d’une incompréhension : les positions des adversaires sont toujours idéologiquement
suspectes (encore un valet de la bourgeoisie) et scientifiquement
inconsistantes (encore un idéaliste ou un ignorant). Nul n’est réactionnaire
volontairement, ou bien seulement les riches : éliminons ceux-ci, éduquons ou
rééduquons les autres, et plus rien ne séparera l’humanité de la justice et du
bonheur. C’est ainsi qu’une utopie sympathique doublée d’une pensée forte
dériva, dès ses commencements, vers une conception bureaucratique de la politique
(le Parti communiste comme avant-garde scientifique et révolutionnaire
du prolétariat), avant de s’enfoncer, dès qu’elle parvint au pouvoir, dans les
horreurs totalitaires que l’on sait. Cela était-il évitable ? On ne le saura jamais,
sauf à recommencer l'expérience, ce qui ne paraît guère raisonnable. Cela ne
dispense pas de lire Marx et Engels, de les méditer, de les utiliser, parfois, pour
leur vertu explicative ou critique ; mais devrait dissuader de se dire marxiste.
Cette pensée, qui a échoué partout où elle parvint au pouvoir, et presque toujours
criminellement, du moins dans sa version révolutionnaire, a fait trop de
mal pour qu’on puisse s’en réclamer en bloc. Ce n’est qu’une fausse science, qui
n'aura abouti qu’à de vraies dictatures. Il en reste, chez les lecteurs de Marx, un
peu de nostalgie et d’effroi, qui ne sauraient pourtant tenir lieu d’analyse.
Qu’une si belle intelligence ait pu mener, même indirectement, à tant d’horreurs,
c’est une raison pour se méfier de l'intelligence, certainement pas pour
s’en passer.

MASCULINITÉ Voir « Féminité ».

MATÉRIALISME Toute doctrine ou attitude qui privilégie, d’une façon ou
d’une autre, la matière. Le mot se prend principalement
%— 360 —
en deux sens, l’un trivial, l’autre philosophique. Il s'oppose dans les deux cas à
l’idéalisme, mais considéré lui aussi en deux acceptions différentes.

Au sens trivial, le matérialisme est un certain type de comportement ou
d’état d’esprit, caractérisé par des préoccupations « matérielles », c’est-à-dire ici
sensibles ou basses. Le mot, dans cette acception, est presque toujours péjoratif.
Le matérialiste, c’est alors celui qui n’a pas d’idéal, qui ne se soucie ni de morale
ni de spiritualité, et qui, ne cherchant que la satisfaction de ses pulsions, penche
toujours vers son corps, pourrait-on dire, plutôt que vers son âme. Au mieux :
un bon vivant. Au pire : un jouisseur, égoïste et grossier.

Mais le mot {\it matérialisme} appartient aussi au langage philosophique : il y
désigne l’un des deux courants antagonistes dont l’opposition, depuis Platon et
Démocrite, traverse et structure l’histoire de la philosophie. Le matérialisme,
c’est alors la conception du monde ou de l’être qui affirme le rôle primordial,
voire l'existence exclusive, de la matière. Être matérialiste, en ce sens philosophique,
c’est affirmer que tout est matière ou produit de la matière, et qu’il
n'existe en conséquence aucune réalité spirituelle ou idéelle autonome — ni
Dieu créateur, ni âme immatérielle, ni valeurs absolues ou en soi. Le matérialisme
s'oppose pour cela au spiritualisme ou à l’idéalisme. Il est incompatible,
sinon avec toute religion (Épicure n’était pas athée, les stoïciens étaient panthéistes),
du moins avec toute croyance en un Dieu immatériel ou transcendant.
C’est un monisme physique, un immanentisme absolu et un naturalisme
radical. « Le matérialisme, écrivait Engels, considère la nature comme la seule
réalité » ; il n’est rien d’autre « qu’une simple intelligence de la nature telle
qu’elle se présente, sans adjonction étrangère » ({\it Ludwig Feuerbach et la fin de la
philosophie classique allemande}, I).

On objectera que cette {\it intelligence}, pour la nature, est {\it déjà} une adjonction
étrangère : si la nature ne pense pas, comment pourrait-on la penser sans en
sortir ? Mais Lucrèce avait déjà répondu : de même qu’on peut rire sans être
formé d’atomes rieurs, on peut philosopher sans être formé d’atomes philosophes.
Ainsi la compréhension matérialiste de la nature est produite — comme
toute pensée, vraie ou fausse —, par une matière qui ne pense pas. C’est ce qui
sépare les matérialistes de Spinoza: la nature, pour eux, n’est pas «chose
pensante » (contrairement à ce que suppose la première proposition du livre II
de l'{\it Éthique}), et c’est pourquoi elle n’est pas Dieu. Il n’est de pensée — par
exemple humaine — que {\it dans} la nature, qui ne pense pas.

Être matérialiste, ce n’est donc pas nier l’existence de la pensée — car alors
le matérialisme se nierait soi. C’est nier son absoluité, son indépendance ontologique
ou sa réalité substantielle : c’est considérer que les phénomènes intellectuels,
moraux ou spirituels (ou supposés tels) n’ont de réalité que seconde et
déterminée. C’est où le matérialisme contemporain rencontre la biologie, et

%— 361 —
spécialement la neurobiologie. Être matérialiste, pour les Modernes, c’est
d’abord reconnaître que c’est le cerveau qui pense, que « l’âme » ou « l'esprit »
ne sont que des illusions ou des métaphores, enfin que l'existence de la pensée
(comme Hobbes, contre Descartes, l’avait fortement marqué) suppose assurément
celle d’un être qui pense, mais nullement que cet être soit lui-même une
pensée ou un esprit : autant dire, parce que je me promène, que je suis une promenade
(Hobbes, Deuxième objection aux {\it Méditations} de Descartes). « Je
pense, donc je suis » ? Sans doute. Mais que suis-je ? Une « chose pensante » ?
Soit. Mais quelle chose ? Les matérialistes répondent : {\it un corps}. C’est peut-être
le point, entre les deux camps, où l'opposition est la plus nette. Là où l’idéaliste
dirait : « J’ai un corps », ce qui suppose qu'il soit autre chose, le matérialiste
dira plutôt : « Je {\it suis} mon corps ». Il y a là une part d’humilité, mais aussi de
défi et d’exigence. Les matérialistes ne prétendent pas être autre chose qu’un
organisme vivant et pensant. C’est pourquoi ils mettent la vie et la pensée si
haut : parce qu’ils n’y voient qu’une exception, d’autant plus précieuse qu’elle
est plus rare et qu’elle les constitue. Ils expliquent donc bien, comme Auguste
Comte l'avait vu, {\it le supérieur} (la vie, la conscience, l'esprit) par {\it l’inférieur} (la
matière inorganique, biologiquement puis culturellement organisée), mais ne
renoncent pas pour autant, d’un point de vue normatif, à sa supériorité. Ils
défendent le {\it primat de la matière}, comme disait Marx, mais n’en sont que plus
attachés à ce que j'appelle la {\it primauté de l'esprit}. Que ce soit le cerveau qui
pense, ce n’est pas une raison pour renoncer à penser ; c’en est une au contraire,
bien forte, pour penser le mieux qu’on peut (puisque toute pensée en dépend).
Et de même : que la conscience soit gouvernée par des processus inconscients
(Freud) ou que l'idéologie soit déterminée en dernière instance par l’économie
(Marx), ce n’est pas une raison pour renoncer à la conscience ou aux idées : c’en
est une, au contraire, pour les défendre (puisqu’elles n’existent qu’à cette condition)
et pour essayer de les rendre — par la raison, par la connaissance — plus
lucides et plus libres. Pourquoi autrement faire une psychanalyse, de la politique
ou des livres ?

L'esprit, loin d’être immortel, est cela même qui va mourir. Il n’est pas
principe mais résultat, non sujet mais effet, non substance mais acte, non
essence mais histoire. Il n’est pas absolu mais relatif (à un corps, à une société,
à une époque...) ; il n’est pas être ou vérité, mais valeur ou sens, et fragile toujours.
La mort aura le dernier mot, ou plutôt le dernier silence, puisqu’elle
seule, comme disait Lucrèce, est immortelle. Raison de plus pour profiter de
cette vie unique et passagère. Le pire seul — ou plutôt le rien — nous attend ; le
mieux, toujours, est à inventer. De là cette constante du matérialisme philosophique,
de déboucher sur une éthique de l’action ou du bonheur. C’est ce
qu’Épicure résumait en quatre propositions, qui constituaient son {\it tetrapharmakon}
%— 362 —
(voir ce mot), dont j’adopterais volontiers cette version légèrement
modifiée :

Il n’y a rien à attendre des dieux ;

Il n’y a rien à attendre de la mort ;

On peut combattre la souffrance ;

On peut atteindre le bonheur.

Ou pour le dire plus simplement : cette vie est ton unique chance ; ne la
gâche pas.

MATÉRIALISTE « J'ai souvent remarqué le contraste, écrivait Alain, entre
les matérialistes, qui sont des esprits résolus, et les spiritualistes,
qui sont des esprits fatigués » (Propos du 29 juin 1929). Celui-là pourtant
n'était pas matérialiste, mais il avait compris ce qu'est, en son fond, le
matérialisme philosophique : une tentative pour {\it sauver l'esprit en niant
l'esprit}, comme il disait à propos de Lucrèce ({\it ibid.}), autrement dit pour
penser l’esprit comme acte, non comme substance, comme valeur, non
comme être, enfin comme création plutôt que comme créateur ou créature.
Et qui pourrait agir, évaluer ou créer, sinon un corps ? Être matérialiste, ce
n’est pas affirmer que l'esprit n’existe pas (il existe, puisque nous pensons) ;
c’est affirmer qu’il n’existe que de façon seconde et déterminée. « Qu'est-ce
donc que je suis ? Une chose qui pense, répondait Descartes, c’est-à-dire un
esprit » ({\it Méditations}, II). C’est ce que le matérialiste refuse. Il dirait plutôt :
Qu'est-ce donc que je suis ? Une chose qui pense, c’est-à-dire un corps pensant.
C’est en quoi Épicure, Hobbes, Diderot, Marx, Freud ou Althusser sont
des matérialistes. Cela ne les empêchait pas d’avoir des idées, ni des idéaux,
mais leur interdisait (ou aurait dû leur interdire) de les ériger en absolus. Le
matérialisme n’est pas une théorie de la matière ; c’est une théorie de l'esprit,
mais comme effet ou comme acte. Ce n’est pas parce que nous avons un
esprit que nous pensons ; c’est parce que nous pensons que nous avons un
esprit.

MATÉRIELLE (CAUSE -) L'une des quatre causes selon Aristote et la scolastique :
celle qui explique un être quelconque
(par exemple une statue) par la matière qui le constitue (par exemple le
marbre). Explication toujours insuffisante et toujours nécessaire. Nulle cause
n’agit qu’en transformant une matière ; mais dès qu’elle agit, et toute matérielle
qu’elle puisse être, elle est déjà efficiente.

%— 363 —
MATHÉMATIQUE C’est d’abord la science des grandeurs, des figures et
des nombres (voir Aristote, {\it Métaphysique}, M, 3). Puis,
de plus en plus, la science qui sert à penser ou à calculer, de façon hypothético-déductive,
les ensembles, les structures, les fonctions, les relations. Que le réel
lui obéisse, comme le prouve la mathématisation si spectaculaire de la physique,
ne laisse pas de surprendre. Mais c’est qu’il ne lui obéit pas. La feuille qui
tombe d’un arbre, son mouvement peut assurément être calculé de façon
mathématique. Mais ce ne sont pas les mathématiques qui la font tomber, ni
tournoyer. C’est la gravitation, c’est le vent, c’est la résistance de l’air — qui se
calculent, mais ne calculent pas.
Ce n’est pas l’univers qui est écrit en langage mathématique, comme le
voulait Galilée ; c’est le cerveau humain qui écrit dans le langage de l’univers,
qui est sa langue maternelle.

MATIÈRE On ne confondra pas le concept scientifique, qui relève de la
physique et évolue en même temps qu’elle, avec la notion ou la
catégorie philosophique de matière, qui peut bien sûr évoluer aussi, en fonction
des théories mises en œuvre, mais dont le contenu essentiel, spécialement chez
les matérialistes, reste à peu près constant. La matière, pour la plupart des philosophes,
c’est tout ce qui existe (ou semble exister) en dehors de l’esprit et
indépendamment de la pensée : c’est la partie non spirituelle et non idéelle du
réel. Définition purement négative ? Sans doute. Mais non pas vide. Car de
l'esprit ou de la pensée nous avons une expérience intérieure, laquelle, fût-elle
illusoire, nous permet, par différence, de donner un contenu aussi à la notion
de matière. Si l’on admet — conformément à cette expérience et d’ailleurs
d'accord en cela avec Bergson et la plupart des spiritualistes — que l'esprit et la
pensée vont ensemble, qu’ils se caractérisent par la conscience, la mémoire,
l’anticipation de l’avenir et la volonté (à quoi j’ajouterais volontiers l’intelligence
et l’affectivité), il faut en conclure que la matière, à l'inverse, est sans
conscience ni mémoire, sans projet ni volonté, sans intelligence ni sentiments.
Cela ne nous dit pas ce qu’elle est (c’est aux physiciens de nous l’apprendre)
mais ce que signifie le mot qui la désigne et comment nous pouvons, philosophiquement,
la penser. Qu'est-ce que la matière ? Tout ce qui existe, disais-je,
ou semble exister, en dehors de l'esprit et indépendamment de la pensée : c’est
tout ce qui est sans conscience, tout ce qui ne pense pas (et qui n’a pas besoin
d’être pensé pour exister), tout ce qui est dépourvu de mémoire, d’intelligence,
de volonté et d’affectivité — tout ce qui n’est pas {\it comme nous}, donc, ou du
moins pas comme nous avons le sentiment, intérieurement, d’être. C’est une
définition qui n’est que nominale (la définition réelle relève des sciences), mais
%— 364 —
c’est la seule qui soit philosophiquement nécessaire et d’ailleurs suffisante.
Ondes ou particules ? Masse ou énergie ? Peu importe ici : les ondes, les particules,
la masse ou l’énergie, sauf à les supposer spirituelles (douées de conscience,
de pensée, d’affectivité....), ne sont que des formes, philosophiquement,
de la matière. Il en va de même, notons-le en passant, de ce que les physiciens
ont maladroitement (de leur propre aveu) appelé l’{\it antimatière} : dès lors qu’on
la suppose non spirituelle, elle est aussi matérielle que le reste.

On se trompe donc quand on prétend définir la matière, au sens philosophique
du terme, par des caractéristiques physiques (la matière, ce serait ce qui
se conserve, ce qu’on peut toucher, ce qui est solide, ce qui a une forme, ce qui
a une masse), et l’on a beau jeu, dès lors, de reprocher au matérialiste d’être
dépassé par l’évolution récente de la physique ! Il n’en est bien sûr rien, et
d’ailleurs bien des physiciens se réclament, aujourd’hui encore et peut-être plus
que jamais, de ce courant de pensée que Bernard d’Espagnat et d’autres prétendent
obsolète. La vérité, c’est que l’idée philosophique de matière fait moins
référence à ce qu’est cette dernière (problème scientifique, répétons-le, bien
plus que philosophique) qu’à ce qu’elle n’est pas (l'esprit, la pensée). Problème
de définition, si l’on veut, non d’essence, de consistance ou de structure : de
même qu’un courant d’air n’est pas moins matériel qu’un rocher, une onde
n’est pas moins matérielle qu’une particule, ni l’énergie moins matérielle que la
masse. Et ni la pensée, dans le cerveau humain, moins matérielle que ce cerveau
lui-même. C’est où la boucle se referme : la matière, c’est tout ce qui existe
indépendamment de l'esprit ou de la pensée, y compris (pour le matérialiste)
l'esprit et la pensée. Contradiction ? Nullement, puisque nous savons que la
pensée peut exister sans se penser soi, et même, en chacun, contre sa propre
volonté (essayez un peu d’arrêter de penser). C’est dire que l’esprit n’est pas une
substance mais un acte, que toute pensée suppose un corps (par exemple un
cerveau) qui la pense, enfin que ce dernier dépend à son tour d’une matière qui
le constitue, et qui ne pense pas.

MAUVAIS Ce qui est mal, fait le mal, ou fait du mal. Se prend le plus souvent
en un sens relatif: « Il n’y a pas de {\it mal} (en soi), écrivait
Deleuze à propos de Spinoza, mais il y a du {\it mauvais} (pour moi) » ({\it Spinoza,
Philosophie pratique}, III). Cette distinction, que le latin de Spinoza n’exprime
pas (il écrirait dans les deux cas {\it malum}), est pourtant fidèle à sa pensée. « Bon
et mauvais se disent en un sens purement relatif, une seule et même chose pouvant
être appelée bonne et mauvaise suivant l’aspect sous lequel on la
considère » (TRE, 5) ou la personne qui en use : par exemple, précise l’{\it Éthique},
« la musique est bonne pour le mélancolique, mauvaise pour l’affligé ; pour le
%— 365 —
sourd, elle n’est ni bonne ni mauvaise » (IV, Préface). Le mauvais, en ce sens,
est la vérité du mal, comme le mal n’est que l’hypostase du mauvais.

MAUVAISETÉ Néologisme introduit en français par certains traducteurs
de Kant, pour désigner le fait d’être mauvais, non méchant :
de faire le mal {\it pour son propre bien} (par égoïsme) plutôt que {\it pour le mal} (par
méchanceté). En ce sens, les hommes ne sont jamais méchants ; mais ils sont tous
mauvais, ou peuvent l'être ({\it La religion dans les limites de la simple raison}, I, 3).

MAXIME Une formule singulière, pour énoncer une règle ou une vérité générale.
Plus personnelle qu’un proverbe, moins qu’un aphorisme :
c'est comme un proverbe qui aurait un auteur, comme un aphorisme qui aurait
fait oublier le sien.

Chez Kant, le mot désigne le principe subjectif du vouloir ou de l’action.
C'est ce qui distingue la maxime (qui reste singulière) de la loi (qui est
universelle) : la maxime est « le principe d’après lequel le sujet {\it agit} ; tandis que
la loi est le principe objectif, valable pour tout être raisonnable, d’après lequel
il {\it doit} agir » ({\it Fondements...}, II). C’est ce qui justifie la fameuse formulation de
l'impératif catégorique : « Agis uniquement d’après la maxime qui fait que tu
peux vouloir en même temps qu’elle devienne une loi universelle » ({\it ibid.}). C’est
vouloir singulièrement l’universel.

MÉCANISME Le mot peut désigner un objet ou une doctrine. Comme objet,
c'est un assemblage mobile ou moteur, capable de transformer
ou de transmettre efficacement un mouvement ou une énergie : c’est
une machine élémentaire, ou l’un des éléments d’une machine, de même
qu'une machine est un mécanisme complexe.

Comme doctrine, c’est considérer la nature et tout ce qui s’y trouve comme
un mécanisme, au sens précédent, ou comme un ensemble de mécanismes, au
point que tout puisse s’y expliquer, comme le voulait Descartes, par « grandeurs,
figures et mouvements ». En ce sens restreint, le mécanisme s’oppose traditionnellement
au dynamisme, qui affirme, avec Leibniz et à juste titre, que
figures et mouvements ne suffisent pas, qu’il faut encore prendre en compte un
certain nombre de {\it forces}. Mais rien n’empêche de considérer ces forces comme
faisant partie des grandeurs ci-dessus évoquées : de là un mécanisme au sens
large, qui s'oppose moins au dynamisme qu’il ne l’inclut. Le mécanisme est
alors la doctrine qui veut tout expliquer — au moins s’agissant de la nature — par
%— 366 —
la seule mécanique, au sens scientifique du terme, c’est-à-dire par l’étude des
forces et des mouvements (au sens où l’on parle, par exemple, de mécanique
quantique). En ce sens large, le mécanisme est très proche du matérialisme, ou
le matérialisme, pour mieux dire, n’est qu’un mécanisme généralisé.

MÉCHANCETÉ Le fait d’être méchant, ou d’agir comme si on l'était. Le
plus souvent, ce n’est qu’égoïsme : celui qu’on croit méchant
n’est que mauvais. Il ne fait pas le mal pour le mal, ni même pour le seul plaisir
de le faire : il ne fait du mal (à l’autre) que pour son bien (à lui), dont le mal
qu’il fait est moins la cause ou l’objet que la condition. Ce n’est qu’un salaud
ordinaire. Si tous les tortionnaires n'étaient que des sadiques, la torture serait
moins répandue, et moins difficile à combattre. Si seuls les méchants faisaient
le mal, le bien aurait tôt fait de l'emporter.

MÉCHANT Le méchant est un être paradoxal. Il semble, c’est la définition
traditionnelle du mot, qu’il fasse le mal pour le mal ; mais cela
suppose en lui une perversité déjà réalisée (une nature mauvaise ou diabolique),
qui lexcuse. S'il est méchant par essence, et non par choix, ce n’est pas sa
faute ; aussi n'est-il pas vraiment méchant, mais victime lui aussi (de sa nature
ou de son histoire, peu importe) et par là innocent. Inversement, comment
expliquer qu’il ait pu {\it choisir} d’être méchant, sinon par une méchanceté antécédente,
qui devrait à son tour être expliquée ? Il faut être bien méchant pour
vouloir le devenir. Et l’on retombe dans le premier cas de figure (la méchanceté
perverse et innocente), où la méchanceté s’annule dans sa factualité. Nul n’est
méchant volontairement (puisqu'il faut l’être déjà pour vouloir le devenir) ni
involontairement (puisqu’une méchanceté involontaire n’en serait plus une).
Le méchant, en ce sens fort, est un être paradoxal et impossible. Le diable, pour
parler comme Kant, n'existe pas : il n’y a pas de méchants ; il n’y a que des
mauvais ou des salauds.

De là un sens affaibli, qui est le seul courant : le méchant est celui qui fait
le mal volontairement, non certes {\it pour le mal}, mais {\it pour son plaisir} (qui est un
bien). Ce n’est pas forcément un sadique (la souffrance d’autrui est plus souvent
le moyen que l’objet de son plaisir), mais toujours un égoïste.

Non, pourtant, que tout égoïste soit méchant (nous le serions tous). Est
égoïste qui ne fait pas, pour autrui, tout le bien qu’il devrait ; est méchant qui
lui fait plus de mal qu’il ne pourrait. L’égoïste manque de générosité ; le
méchant, de douceur et de compassion. Les méchants, en ce dernier sens, existent
bien. Mais ils restent l’exception : il y a moins de salauds que de lâches.

%— 367 —
MÉDIÉTÉ Un autre mot pour dire le juste milieu ({\it mésotès}) chez Aristote.
Ainsi la vertu est-elle « une médiété entre deux vices, l’un par
excès, l’autre par défaut ». C’est le contraire d’une médiocrité : une perfection
et un sommet, comme une ligne de crête entre deux abîmes, ou entre un abîme
et un marais ({\it Éthique à Nicomaque}, II, 5-6, 1106 a — 1107 a).

MÉDIOCRITÉ La moyenne, mais considérée dans son insuffisance. C’est
notre état normal, mais ce n’est pas la norme. L’exception
seule, pour l'esprit, mérite de faire règle.

La médiocrité est l'opposé du {\it juste milieu} aristotélicien : non une ligne de
crête, entre deux abîmes, mais un caniveau, comme on faisait dans les rues au
Moyen-Âge, entre deux pentes. Il suffit de se laisser aller pour y glisser.

MÉDISANCE Dire le mal qui est, mais pour le plaisir de le dire plutôt que
par devoir de le dénoncer. C’est une sincérité mauvaise (par
différence avec la calomnie, qui serait plutôt une méchanceté mensongère), et
l'un des plaisirs de l'existence.

MÉFIANCE  Défiance généralisée et excessive. Le méfiant est incapable de se
fier à quiconque, y compris à ceux qui le mériteraient. Ce n’est
plus prudence ; c’est petitesse.

MEILLEUR (PRINCIPE DU —) C'est un principe leibnizien, selon lequel
Dieu, étant à la fois tout-puissant, omniscient et parfaitement bon,
agit toujours de façon optimale : il voit tous les possibles,
peut réaliser tous les compossibles (voir ce mot), et choisit toujours,
entre eux, le meilleur arrangement. Le monde étant par définition unique
(puisqu'il est la totalité des choses contingentes), il faut en conclure qué notre
monde, même imparfait (s’il était parfait, il ne serait plus le monde : il serait
Dieu), est le meilleur des mondes possibles : Dieu en aurait autrement créé un
autre ({\it Discours de métaphysique}, \S 3 et 4 ; {\it Théodicée}, I, \S 8 à 19, II, \S 193 à
240, IT, \S 413 à 416...). C’est le fondement de l’optimisme leibnizien, dont
Voltaire s’est moqué dans {\it Candide} et dans son {\it Dictionnaire} : voir l’article
« Bien (tout est —) ». On dira que l'ironie ne tient pas lieu de réfutation. Sans
doute, Mais une foi irréfutable n’est pas davantage prouvée par là.
%— 368 —
MÉLANCOLIE L’humeur noire (ou la bile noire) des Anciens. Aujourd’hui,
le mot se prend surtout en deux acceptions. Dans le
langage courant, c’est une tristesse légère et diffuse, sans objet particulier et
pour cela à peu près inconsolable. Dans le vocabulaire psychiatrique, à
l'inverse, c’est un dérèglement pathologique de l'humeur, caractérisé par une
tristesse extrême, souvent mêlée d’anxiété, d’auto-dépréciation, de ralentissement
psychomoteur et d’idées suicidaires. Inconsolable dans les deux cas, donc,
mais pour des raisons plutôt opposées : parce qu’elle est trop légère ou trop
lourde, trop vague ou trop grave, trop « normale » (la mélancolie ordinaire est
moins un trouble qu’un tempérament) ou pas assez. La première peut être
presque agréable («la mélancolie, disait Victor Hugo, c’est le bonheur d’être
triste ») ; la seconde ne l’est jamais : elle relève de la médecine, et peut tuer si
on ne la soigne pas. Toutefois la distinction, entre ces deux états, n’est pas toujours
aussi nette : les tempéraments mélancoliques ne sont pas à l'abri d’une
psychose ou d’une dépression.

MÊME L'expression de l'identité, qu’elle soit numérique (« nous habitons
dans la même rue») ou spécifique (« nous portons la même
cravate »). S’oppose traditionnellement à l’{\it autre}, spécialement depuis Platon
(voir par exemple le {\it Sophiste}, 254-258, et le {\it Timée}, 34-36). Tout être est réputé
le même que lui-même (principe d’identité) et autre que tous les autres (principe
des indiscernables). Mais cela ne l'empêche pas de devenir autre que soi
(impermanence) : le même, dans le temps, n’est jamais qu’une abstraction. À la
gloire d'Héraclite.

MÉMOIRE La conscience présente du passé, que ce soit en puissance (comme
faculté) ou en acte (comme mémoration ou remémoration).
Cette conscience est actuelle, comme toute conscience, mais elle n’est mémoire
qu’en tant qu’elle perçoit, ou peut percevoir, le passé {\it en tant que passé} — sans
quoi ce ne serait plus mémoire mais hallucination. C’est la conscience actuelle
de ce qui ne l’est plus, en tant que cela l’a été.

On évitera de dire que la mémoire est la {\it trace} du passé : d’abord parce
qu’une tache ou un pli, qui sont assurément de telles traces, ne sont pas des
actes de mémoire ; ensuite parce qu’une trace n’est qu’un morceau du présent,
qui n’évoque le passé que pour une conscience. Qu'il y ait des traces du passé
dans le cerveau, et qu’elles contribuent à la mémoire, c’est vraisemblable. Mais
cela n’est un fait de mémoire que pour autant que le cerveau, grâce à elles, produit
%— 369 —
 — ou peut produire — autre chose que des traces : la conscience présente de
ce qui ne l’est plus.

On évitera aussi de dire que la mémoire est une dimension de la conscience.
Elle est bien plutôt la conscience elle-même, laquelle n’est consciente
qu'à la condition de se souvenir continâment de soi ou de ses objets.
Anticiper ? C’est se souvenir qu’on anticipe. Imaginer ? C’est se souvenir qu’on
imagine. Être attentif ? C’est se souvenir qu’on l’est, ou de l'être, ou de ce à
quoi l’on fait attention. Ainsi toute conscience est mémoire : la mémoire n’est
pas seulement « coextensive à la conscience », comme disait Bergson, elle est la
conscience même.

On parle d’un devoir de mémoire. À ce niveau de généralité, cela n’a pas
grand sens. La mémoire est une faculté, point une vertu : le tout est de s’en
servir au mieux, ce qui ne va pas sans sélection, ni donc sans oubli. Comment
la mémoire pourrait-elle y suffire, puisqu’elle en a besoin ? Ce n’est pas une
faute que d’oublier ce qui ne mérite pas d’être retenu, ni même d’oublier ce qui
mériterait d’être mémorisé mais pour des raisons qui ne touchent pas à la
morale (par exemple son numéro de carte bancaire). Le vrai devoir, ce n’est pas
de se souvenir, c’est de {\it vouloir} se souvenir. Et non de tout ou de n’importe
quoi, mais de ce que l’on doit {\it à d'autres} : à cause du bien qu’ils nous ont fait
(gratitude), du mal qu’ils ont subi ou subissent (compassion, justice) ou qu’on
leur a fait soi-même (repentir). Devoirs, non de mémoire, mais de fidélité.
C’est aussi la seule façon de préparer valablement l’avenir. Du passé, ne faisons
pas table rase.

MENSONGE C'est dire, dans l'intention de tromper (et non par antiphrase
ou par ironie), ce qu’on sait être faux. Tout mensonge suppose
un savoir, et au moins l’idée de vérité. C’est en quoi le mensonge récuse
la sophistique, qui l’excuse. Le paradoxe du Menteur (voir ce mot) montre suffisamment
que le mensonge n’est possible qu’à titre d'exception : par où il confirme
la règle même qu’il viole («la norme, dirait Spinoza, de l’idée vraie
donnée ») et qui le rend possible. Tant pis pour les menteurs et les sophistes.

MENTEUR (PARADOXE DU-)  Épiménide, qui est crétois, dit : « {\it Tous les
crétois sont menteurs.} » Ce qu'il dit est
donc faux, si c’est vrai (puisque alors il ne ment pas), et vrai, si c’est faux
(puisqu'il ment en effet). C’est l’un des paradoxes traditionnels, depuis les
mégariques, de l’autoréférence. Ce n’est vraiment un paradoxe, et non un
simple sophisme, que si l’on donne à l’expression « être menteur » le sens de
%— 370 —
« mentir toujours ». C’est pourquoi, en pratique, ce n’en est pas un. La vérité
est que tous sont menteurs, crétois Où non, et qu'aucun ne ment toujours — car
alors on ne pourrait plus mentir. La formule vraiment paradoxale ou aporétique
serait : « {\it Je suis en train de mentir} » ; ou bien : « {\it La phrase que vous lisez en
ce moment est fausse} ». Car chacune de ces deux propositions serait vraie si elle
est fausse, et fausse si elle est vraie : ce serait une violation du principe de
non-contradiction. On remarquera que ce ne serait pas le cas de ces autres
propositions : « {\it Je mentais} » (qui n’est pas autoréférentielle), ou bien : « {\it La
phrase que vous lisez en ce moment est vraie} », qui sont banalement vraies si elles
sont vraies et fausses si elles sont fausses. Cela semble indiquer que l’autoréférence
n’est logiquement valide que sous réserve, c’est la moindre des choses, de
ne pas nier sa propre vérité. Mais c’est qu’aussi ces propositions autoréférentielles,
même correctement formées, restent en l’occurrence à peu près vides.
Dire « Je dis la vérité », c’est ne rien dire. Dis-la plutôt, au lieu de te contenter
de dire que tu la dis!

MÉPRIS C'est refuser le respect ou l’attention. Ainsi peut-on mépriser le
danger ou les convenances. Se dit plus souvent vis-à-vis d’un être
humain : c’est alors refuser à quelqu'un le respect qu’on doit ordinairement à
son prochain, soit parce qu’il en semble en l’occurrence indigne, soit parce
qu’on est incapable, à tort ou à raison, de le considérer comme son égal. On
remarquera que si tous les hommes sont égaux en droits {\it et en dignité}, le mépris
est toujours injuste, et méprisable par là.

MÈRE  « Dieu ne pouvant être partout, dit un proverbe yiddish, il inventa
les mères. » Cela m’éclaire sur l’idée de Dieu, comme sur celle de
maternité.

Qu'est-ce qu’une mère ? C’est la femme qui a porté et enfanté. Celle aussi,
presque toujours, qui a aimé son enfant, qui l’a protégé (y compris contre le
père), nourri, bercé, éduqué, caressé, consolé. Il n’y aurait pas d’amour autrement,
ni d'humanité.

On parle depuis longtemps de mères adoptives et de mères biologiques, et
l’on a raison, depuis peu de mères porteuses, et l’on n’a pas tout à fait tort
(quoique l’expression soit atroce). C’est que les deux fonctions d’enfantement
et d'éducation, ordinairement conjointes, ne le sont pas nécessairement. Et
bien sûr l’amour donné importe davantage ici que les gènes transmis. On dira
que voilà, pour un matérialiste, une curieuse idée. Mais c’est que l’amour n’est
pas moins matériel que le reste.
%— 371 —
On discute beaucoup pour savoir si l’amour maternel est un instinct ou un
fait de culture. Un instinct, certainement pas (puisqu’il connaît des exceptions
et n’entraîne guère de savoir-faire). Un fait de culture ? Il le faut bien, même si
celui-ci vient se greffer, selon toute vraisemblance, sur des données biologiques.
La langue non plus n’est pas un instinct ; cela n'empêche pas que le langage soit
une faculté biologiquement déterminée — et qu’on parle aussi, légitimement, de
langue {\it maternelle}. La parole est un avantage sélectif évident. L'amour parental,
spécialement maternel, aussi. Dans les conditions qui furent celles, pendant des
dizaines de milliers d’années, de nos ancêtres préhistoriques, on n’ose imaginer
ce qu'il fallut d'amour, d’intelligence et de douceur, chez les mères, pour que
l'humanité puisse simplement survivre. Il m'est arrivé de dire que l’amour était
une invention des femmes. C’est une boutade, mais qui n’est pas sans rencontrer
quelque chose d’important, sur quoi Freud, à sa façon, a insisté aussi.
Notre première histoire d’amour, pour la quasi-totalité d’entre nous, hommes
ou femmes, a commencé dans les bras de notre mère : la femme qui nous aima
d’abord, sauf exception, et nous apprit à aimer.

Cela ne veut pas dire que les pères n’ont pas d'importance, ce qui serait une
évidente absurdité (quoiqu’ils n’en aient, s’agissant de l’éducation des enfants
et dans certaines cultures, que fort peu), ni qu’ils soient incapables d’aimer, ce
qui serait une évidente injustice (mais en seraient-ils capables s’ils n'avaient été
aimés d’abord ?), mais que leur rôle et leur amour, aussi considérables qu’ils
puissent être, restent en quelque sorte seconds, au moins chronologiquement,
et comme greffés sur une histoire qui les précède et les prépare. Cela vaut pour
l'espèce autant que pour les individus. Romain Gary, en une phrase, a dit
l'essentiel : « L'homme — c’est-à-dire la civilisation —, ça commence dans les
rapports de l'enfant avec sa mère. »

MÉRITE Ce qui rend digne d’éloge ou de récompense. On croit souvent
que cela suppose le libre arbitre, mais à tort. Personne ne décide
librement d’avoir du talent ou du génie : faut-il pour cela réserver nos éloges
aux médiocres besogneux ? Il n’est pas sûr que le courage doive davantage au
libre arbitre : faut-il pour cela refuser de l’admirer ou de le récompenser ? Ce
serait une curieuse conception du mérite, qui rendrait Mozart moins admirable
que Salieri (qui se donna peut-être davantage de mal), et la sainteté ou
l’héroïsme moins méritoires que nos efforts, parfois, pour n'être pas tout à fait
méprisables...
Celui qui donne sans plaisir n’est pas généreux, expliquait Aristote : ce
n’est qu’un avare qui se force. Faut-il pour cela — parce qu’il aurait plus de
mérite ! — l’admirer davantage que celui qui donne sans efforts, facilement,
%— 372 —
spontanément, presque sans y penser, parce que l’amour ou la générosité sont
devenus en lui comme une seconde nature ?

L'amour ne se commande pas, remarquait Kant. On ne m'ôtera pourtant
pas de l’idée que l’amour (en tout cas celui qui donne : {\it philia, agapè}) est bien
un mérite, et le plus grand de tous. Pourquoi, autrement, les chrétiens
loueraient-ils leur Dieu ?

MESSIANISME C'est attendre son salut d’un sauveur, au lieu de s’en
occuper soi-même. Le contraire, donc, de la philosophie.

MESSIE Un sauveur, qui serait envoyé par Dieu. C’est pourquoi on l'attend,
y compris quand on croit qu’il est déjà venu (on attend alors son
retour). De là le messianisme, qui est une utopie religieuse ou une religion de
l’histoire.

MESURE Repas de famille. La mère apporte le dessert. Elle demande à son
petit garçon : « Tu en veux beaucoup ? » Et l'enfant de répondre,
les yeux brillants de convoitise : « J’en veux {\it trop}!»

C'était poser le problème de la mesure, par la démesure. De la règle, par sa
transgression. Comment la démesure pourrait-elle annuler ce qu’elle suppose ?
C’est où échoue, peut-être, le romantisme. Mais n’allons pas trop vite. Ce mot
d'enfant, lu il y a longtemps, je ne sais si les enfants d’aujourd’hui peuvent
encore l’apprécier, voire le comprendre. « Trop », dans leur langage, s’est banalisé
au point de se vider à peu près de son sens : ce n’est souvent qu’un synonyme
de « très » ou de « beaucoup ». Ainsi, au sortir d’un cinéma : « Ce film,
c'est {\it trop bien} !» Ou devant un plat dont ils raffolent : « {\it C'est trop bon} !»
Comme si seul l'excès pouvait suffire. Comme si la démesure était la seule
mesure acceptable. Ce n’est qu’une mode, qui passera comme les autres. Elle
dit pourtant quelque chose sur l'enfance et sur l’époque. Le sens de la mesure
s’acquiert peu à peu, et plus ou moins. Les enfants et les modernes y sont peu
portés. Ils préfèrent l'infini. Ils préfèrent la démesure. Il faudra donc qu’ils
changent, puisque la mesure seule — fât-ce pour habiter l'infini — est à notre
portée. Les Grecs le savaient. L’infini c’est l’inaccessible, inachevé, imparfait.
Toute perfection, à l'inverse, suppose un équilibre, une harmonie, une proportion.
« Ni trop ni trop peu », comme dit souvent Aristote. C’est la seule perfection
qui nous soit accessible. Cela vaut aussi en esthétique : « Ce n’est pas assez
qu’une chose soit belle, expliquait Pascal, il faut qu’elle soit propre au sujet,
%— 373 —
qu'il n’y ait rien de trop ni rien de manque. » Et Poussin : « La mesure nous
astreint à ne passer pas outre, nous faisant opérer en toutes choses avec une certaine
médiocrité et modération... » Cette {\it médiocrité}, pas plus que le {\it juste
milieu} d’Aristote, n’a rien de médiocre, au sens moderne du terme. C’est plutôt
le refus de tous les excès, de tous les défauts, comme un archer vise le centre
({\it medium}) de la cible, non sa périphérie ou son dehors. Ce qu'il faut
comprendre ? Que la mesure est à la fois l’exception et la règle. C’est où commence,
peut-être, le classicisme. « Entre deux mots, écrit Paul Valéry, il faut
choisir le moindre. » Esthétique de la mesure, de la litote, comme disait Gide,
de la finitude heureuse. C’est le contraire de l’exagération, de l’emphase, de la
grandiloquence. Apollon contre Dionysos. Socrate contre Calliclès. C’est victoire
sur soi, sur la démesure de ses désirs, de ses colères, de ses peurs. Par quoi
la mesure touche à l’éthique et devient une vertu.

Le mot « mesure », en français, se prend en deux sens. Il désigne d’abord le
fait de mesurer, c’est-à-dire l'évaluation ou la détermination d’une grandeur,
qu'elle soit intensive (par degrés) ou extensive (par quantités). La mesure, qui
peut être objective ou chiffrée, s'oppose alors au non-mesurable, à l’incommensurable,
à l’indéterminable, à tout ce qui est trop petit, trop grand ou trop fluctuant
pour être mesuré. C’est en ce sens qu’on parlera de la mesure d’une distance,
d’une température, d’une vitesse... Mais le mot désigne aussi — sans
doute par abréviation de l’expression « juste mesure », autrement dit par dérivation
du sens premier — une certaine qualité, ou un certain idéal, de modération,
d'équilibre, de proportion, vers lesquels doivent tendre nos œuvres (d’un
point de vue esthétique) comme nos actions (d’un point de vue éthique). La
mesure s'oppose alors à la démesure, et c’est en ce sens qu’on parlera d’un
homme {\it mesuré} : c'est celui qui refuse tous les excès, spécialement ceux de
l’emportement ou du fanatisme. On voit que la mesure, en ce second sens, est
une donnée plutôt subjective, qu’aucune évaluation chiffrée ne saurait suffire à
définir ou à caractériser. C’est ce qu’on peut appeler aussi la modération, que
les Grecs appelaient sophrosunè, et le contraire de leur {\it hubris} (la démesure,
l'excès). La tempérance ? Disons plutôt que la tempérance est une espèce particulière
de mesure ou de modération : c’est la modération dans les plaisirs sensuels,
le contraire autrement dit de ces excès particuliers que sont la goinfrerie,
l’ivrognerie ou la débauche. L'homme mesuré se doit d’être tempérant. Mais il
ne suffit pas d’être tempérant, hélas, pour être mesuré. Savonarole était tempérant.
Robespierre était tempérant. Quelle démesure pourtant dans l’action, la
pensée, le caractère ! La mesure est comme une modération de l’âme ou de tout
l'être, non seulement face au corps et à ses plaisirs, mais face au monde, à la
pensée, à soi. C’est le contraire du fanatisme, de l’extrémisme, de l’emportement
par les passions. C’est pourquoi la mesure séduit peu. On préfère les passionnés,
%— 374 —
les enthousiastes, tous ceux qui se laissent entraîner par leur foi ou
leurs affects. On préfère les prophètes, les démagogues, les tyrans bien souvent,
aux arpenteurs du réel, aux comptables sourcilleux du possible. L'histoire est
pleine de ces enthousiastes massacreurs, qui ont triomphé, sous les acclamations
de la foule, d’esprits plus mesurés. Mais un triomphe ne prouve rien, ou
moins, en tout cas, qu’un massacre. Quelle paix sans mesure ? Quelle justice
sans mesure ? Quel bonheur sans mesure ?

Épicure, disait Lucrèce, « fixa des bornes au désir comme à la crainte ».
C’est que la démesure voue les humains au malheur, à l’insatisfaction, à
l'angoisse, à la violence. Ils en veulent toujours plus ; comment en auraient-ils
jamais assez ? Ils veulent tout ; comment sauraient-ils partager ou se contenter
de ce qu’ils ont ? Le sage épicurien, au contraire, est un homme mesuré. Il sait
limiter ses désirs aux plaisirs effectivement accessibles, ceux qui peuvent le combler,
ceux qui ont en eux-mêmes leur propre mesure, comme sont les plaisirs
du corps (quand ils sont naturels et nécessaires), ou ceux qu'aucune démesure
ne menace (comme sont les plaisirs de l’amitié ou de la philosophie). Sur ce
dernier point, on peut discuter. Inventer un système, comme fit Épicure, prétendre
dire la vérité sur le tout, n’est-ce pas déjà démesure ? Peut-être bien.
Montaigne sera plus mesuré et plus sage. C’est pourquoi sans doute il est plus
actuel. Les systèmes sont tous morts, tous faux, tous abandonnés. La démesure
vieillit mal, même dans la pensée. Dans l’art ? Cela dépend des goûts. Il y a
ceux qui préfèrent Rabelais, et ceux qui préfèrent Montaigne. Mais même la
belle démesure, celle de Rabelais ou Shakespeare, n’est artistique que par la
mesure qui la domine ou la surmonte. « Tout ce en quoi il y a de Part, disait
Platon, participe de quelque manière à la pratique de la mesure » ({\it Le politique},
285 a). Il n’y a pas de livre infini, il ne peut y en avoir, pas de peinture infinie,
pas de sculpture infinie. Une musique infinie ? On peut la concevoir ou la programmer,
par l'ordinateur. Mais l'écouter, non. Mais la jouer, non. L'homme
n’est pas Dieu, c’est par quoi l’humanisme a à voir, toujours, avec la mesure.

C’est ce que Camus dut rappeler, contre « la démesure contemporaine », et
qu’il faut rappeler avec lui : « Toute pensée, toute action qui dépasse un certain
point se nie elle-même ; il y a en effet une mesure des choses et de l’homme »
({\it L'homme révolté}, V). C’est cette mesure que les révolutionnaires, presque toujours,
ont oubliée, ce qui les vouait au terrorisme (tant qu’ils furent dans
l’opposition) ou au totalitarisme (là où ils sont parvenus au pouvoir). Romantisme
de Marx. Romantisme, presque toujours, des révolutionnaires. C’est ce
qui les rend sympathiques et dangereux. Car la démesure séduit, exalte, fascine.
La mesure ennuie. C’est du moins le préjugé romantique ou moderne, qu’il
faut comprendre et vaincre. « Quoi que nous fassions, écrit encore Camus, la
démesure gardera toujours sa place dans le cœur de l’homme, à l’endroit de la
%— 375 —
solitude. Nous portons tous en nous nos bagnes, nos crimes, nos ravages. Mais
notre tâche n’est pas de les déchaîner à travers le monde ; elle est de les combattre
en nous-mêmes et dans les autres. » Contre la barbarie, quoi ? L’action
prudente, réfléchie, déterminée — la mesure, mais résolue.

On m'’objectera, et l’on aura raison, qu’il y a aussi des choses qui ne se
mesurent pas, ou mal. C’est vrai dans les sciences : il y a des mesures impossibles,
incertaines ou paradoxales, soit parce qu’elles modifient ce qu’elles doivent
mesurer (relations d’incertitude de Heisenberg, réduction du paquet
d’ondes en mécanique quantique), soit parce qu’elles varient en fonction de
l'échelle utilisée (par exemple si l’on veut mesurer les côtes de la Bretagne).
C’est vrai dans la vie des sociétés : comment mesurer la liberté ou le bonheur
d’un peuple, sa cohésion, sa civilisation ? C’est vrai enfin, et peut-être surtout,
dans la vie des individus. La souffrance ne se mesure pas. Le plaisir ne se
mesure pas. L'amour ne se mesure pas. L'essentiel ne se mesure pas, et c’est
pourquoi la mesure n’est pas l’essentiel.

Ne confondons pas, toutefois, ce qui ne peut être mesuré exactement ou
absolument avec ce qui n’existerait pas ou récuserait toute approche quantitative.
La longueur des côtes bretonnes varie en fonction de l'échelle utilisée
(selon qu’elle prendra ou non en compte telle ou telle crique, tel ou tel rocher,
telle ou telle anfractuosité, telle ou telle dentelure de tel ou tel rocher ou anfractuosité...). Cela ne signifie pas que la Bretagne n’existe pas, ni qu’elle n’ait pas
de côtes, ni que ces côtes ne soient plus longues que celles de la Vendée ou du
Cotentin... Même chose pour les peuples : leur liberté ou leur paix ne sauraient
se mesurer exactement ; cela ne veut pas dire qu’elles n’existent pas, ni
qu’elles soient constantes ou égales... Même chose, enfin, pour les individus.
Le plaisir ne se mesure pas ; mais tous les plaisirs ne se valent pas. La souffrance
ne se mesure pas ; mais il en est de plus grandes que d’autres. L'amour ne se
mesure pas ; mais il peut être plus ou moins fort, plus ou moins grand, plus ou
moins profond... C’est pourquoi, même quand toute mesure objective ou chiffrée
est impossible, la mesure comme vertu reste nécessaire. Il s’agit de proportionner
sa conduite à ce qu’on ressent effectivement ou à ce que le réel requiert.
Ne pas faire l’acteur tragique, comme disait Marc Aurèle, ne pas s’arracher les
cheveux pour des broutilles, ne pas en rajouter, ne pas se laisser emporter ou
dépasser. Résister, donc. Mais pas non plus vivre par défaut ou {\it a minima}, ne
pas s’enfermer dans le déni ou l’insensibilité, ne pas s’interdire d’aimer, de souffrir,
de jouir. Il y a là un art difficile, qu’on n’a jamais fini d’apprendre, qui
est la mesure même. Un art, mais sans artifice (ou avec le moins d’artifices possible)
et d’ailleurs sans œuvre. C’est ce que Pascal appelle « le simple naturel »,
que j’appellerais plus volontiers la justesse : « Ne pas faire grand ce qui est petit,
ni petit ce qui est grand. » C’est comme une justice à la première personne, ou
%— 376 —
de soi à soi. La balance pourrait lui servir, à elle aussi, de symbole. Mais cette
balance c’est l’âme ou le cœur, comme dirait Pascal, qui mesure ce qui ne se
mesure pas. C’est donc le corps (le seul instrument de mesure dont on ne
puisse se passer, celui que tous les autres supposent), mais éduqué, à la fois sensible
et raisonnable, mesureur et mesuré, et c’est ce qu’on appelle l'esprit.

MÉTAMORPHOSE Changement complet de forme, quand il est assez
rapide pour surprendre. On parle de métamorphose
pour la chenille qui devient papillon, pas pour le nouveau-né qui devient un
vieillard.

MÉTAPHORE Figure de style. C’est une comparaison implicite, qui fait utiliser
un mot à la place d’un autre, en raison de certaines analogies
ou ressemblances entre les objets comparés. Par exemple, mais ils seraient
bien sûr innombrables, lorsque Homère évoque « l'aurore aux doigts de rose »
(ou Baudelaire, homme du nord, « l’aurore grelottante en robe rose et verte »),
ou lorsque Eschyle nous donne à voir, je ne connais pas de plus suggestive évocation
de la Méditerranée, « le sourire innombrable de la mer ». En français, il
est difficile de ne pas songer à la fin de {\it Booz endormi}, de Victor Hugo. C’est la
nuit. Une jeune fille, allongée sur le dos, contemple la lune et les étoiles. Cela
fait comme un bouquet de métaphores :

{\footnotesize
\begin{center}
\begin{tabular}{l}
Tout reposait dans Ur et dans Jérimadeth ; \\
Les astres émaillaient le ciel profond et sombre ; \\
Le croissant fin et clair parmi ces fleurs de l'ombre \\
Brillait à l'occident, et Ruth se demandait, \\
 \\
Immobile, ouvrant l'œil à moitié sous ses voiles, \\
Quel dieu, quel moissonneur de l'éternel été, \\
Avait, en s’en allant, négligemment jeté \\
Cette faucille d’or dans le champ des étoiles. \\
\end{tabular}
\end{center}
}

Lacan a cru reconnaître la métaphore dans le processus freudien de {\it condensation},
tel qu’il apparaît ou se masque dans les rêves et les symptômes. Il y a
substitution, dans les deux cas, d’un signifiant à un autre : « La {\it Verdichtung},
condensation, c’est la structure de surimposition des signifiants où prend son
champ la métaphore » (« L’instance de la lettre dans l'inconscient », {\it Écrits},
p. 511 ; voir aussi p. 506 à 509). Cela ne suffit pas à faire de l’inconscient un
poète, mais peut expliquer, au moins en partie, l'impact sur nous de la poésie
%— 377 —
en général et de la métaphore en particulier. Toutefois il convient de ne pas en
abuser : désigner une chose par ce qu’elle n’est pas ne saurait suffire à exprimer
ce qu’elle est. C’est où la prose et la veille retrouvent leurs droits, ou plutôt
leurs exigences.

MÉTAPHYSIQUE C’est une partie de la philosophie, celle qui porte sur les

questions les plus fondamentales, disons sur les ques-
tions premières ou ultimes : l’être, Dieu, l’âme ou la mort sont des problèmes
métaphysiques.

Le mot a une origine curieuse, qui fait comme un jeu de mots objectif.
Lorsque Andronicos de Rhodes, au {\footnotesize I$^\text{er}$} siècle avant Jésus-Christ, voulut éditer les
œuvres ésotériques d’Aristote, il regroupa les textes ou traités dont il disposait
en un certain nombre de recueils, qu’il agença comme il put. Pour plusieurs
d’entre eux, le titre semblait s'imposer, en fonction de leur contenu : la physique,
la politique, l'éthique, la connaissance du vivant et des animaux... Dans
lun de ces recueils, il rassembla un certain nombre de textes majeurs, qui portaient
sur la science de l’être en tant qu'être, sur les premiers principes et les
premières causes, sur la substance et sur Dieu, bref sur ce qu’Aristote aurait
appelé plutôt, sil avait dû donner lui-même un titre, des textes de
« philosophie première » (de même que ce que nous appelons en français les
{\it Méditations métaphysiques} de Descartes s’appelait en latin {\it Meditationes de prima
philosophia}). Il se trouve que ce recueil, dans le classement d’Andronicos, venait
après la physique. On prit l'habitude de l’appeler, d’un mot qu’on ne trouve
jamais chez Aristote, {\it Meta ta phusika} : le livre qui vient {\it après la physique}, peut-être
aussi ({\it meta}, en grec, peut avoir ces deux sens) celui qui va {\it au-delà}. Aussi
l'usage s’imposa-t-il, au fil des siècles, d'appeler {\it métaphysique} tout ce qui allait
au-delà de la physique, c’est-à-dire, et plus généralement, au-delà de l’expérience,
donc de la connaissance scientifique ou empirique. C’est le sens qu’elle
a gardé chez Kant, qui la récuse (comme métaphysique dogmatique : comme
connaissance de l’absolu ou des choses en soi) et veut la sauver (comme métaphysique
critique : comme « l’inventaire, systématiquement ordonné, de tout
ce que nous possédons par la raison pure »). C’est le sens qu’elle a toujours,
même si certains, sottement, y mettent des accents d’ironie ou de dédain. Faire
de la métaphysique, c’est penser plus loin qu’on ne sait et qu’on ne peut savoir.
C’est donc penser aussi loin qu’on peut, et qu’on doit. Celui qui voudrait rester
dans les limites strictes de l'expérience ou des sciences, il ne pourrait répondre
à aucune des questions principales que nous nous posons (sur la vie et la mort,
l'être et le néant, Dieu ou l’homme), ni même à celles que nous posent l’expérience
et les sciences elles-mêmes, ou plutôt que nous nous posons à leur
%— 378 —
propos (sont-elles vraies, à quelles conditions et dans quelles limites ?). C’est en
quoi, comme Schopenhauer l’a vu, « l’homme est un animal métaphysique » :
parce qu’il s'étonne de sa propre existence, comme de celle du monde ou de
tout ({\it Le Monde...}, suppl. au livre I, chap. XVII ; le thème de l’{\it étonnement} est
expressément repris d’Aristote : {\it Métaphysique}, A, 2). La plus grande question
métaphysique, de ce point de vue, est sans doute la question de l’être, telle
qu’elle est posée, par exemple, par Leibniz: {\it Pourquoi y a-t-il quelque chose
plutôt que rien ?} Qu’aucun savoir n’y réponde n’interdit pas de la poser, ni n’en
dispense.

MÉTEMPSYCOSE Le passage de l’âme ({\it psukhè}) d’un corps à un autre.
Croyance traditionnelle en Orient, plus rare en Occi-
dent (quoiqu’on la trouve dans l’orphisme, chez Pythagore ou chez Platon). Il
faut tenir beaucoup à la vie, et bien peu à ses souvenirs, pour y voir une consolation.

MÉTHODE Un ensemble, rationnellement ordonné, de règles ou de principes,
en vue d'obtenir un certain résultat. En philosophie, je
n’en connais pas qui soit vraiment convaincante, sinon la marche même de la
pensée, qui est sans règle, ou sans autre règle que soi. Le {\it Traité de la réforme de
l'entendement} de Spinoza, si difficile, si décevant à certains égards, me paraît
pourtant plus utile et plus vrai que les {\it Règles pour la direction de l'esprit} de Descartes,
ou même que le {\it Discours de la méthode}, évident chef-d'œuvre mais qui
ne le doit guère aux quatre préceptes (de l'évidence, de l’analyse, de la synthèse
et du dénombrement) qu’il propose dans sa deuxième partie. S’il y avait une
méthode pour trouver la vérité, cela se saurait et ne serait plus de la philosophie.
Ainsi parle-t-on de {\it méthode expérimentale}, dans les sciences, mais qui se
ramène à quelques banalités sur les rôles respectifs de la théorie et de l’expérience,
des hypothèses et de la falsification. Cela ne tient lieu, même dans les
sciences, ni de génie ni de créativité. Comment cela pourrait-il suffire à la
vérité ? La vraie méthode, explique Spinoza, est plutôt la vérité même, mais
réfléchie et ordonnée :

{\footnotesize
« La vraie méthode ne consiste pas à chercher la marque à laquelle se reconnaît la
vérité après l’acquisition des idées ; la vraie méthode est la voie par laquelle la vérité
elle-même, ou les essences objectives des choses, ou leurs idées (tous ces termes ont même
signification) sont cherchées dans l’ordre dû. [...] De là il ressort que la méthode n’est
pas autre chose que la connaissance réflexive ou l’idée de l’idée ; et, n’y ayant pas d’idée
%— 579 —
d’une idée si l’idée n’est donnée d’abord, il n’y aura donc point de méthode si une idée
n’est donnée d’abord. La bonne méthode est donc celle qui montre comment l'esprit
doit être dirigé selon la norme de l’idée vraie donnée » ({\it T.R.E.} 27).
}\\

Il s’agit moins d’appliquer des règles que d’apprendre à s’en passer : la
vérité suffit et vaut mieux.

MÉTONYMIE C’est une figure de style par laquelle un mot est utilisé à la
place d’un autre, non pas en vertu d’une comparaison implicite,
comme dans la métaphore, mais en raison d’un rapport, plus ou moins
nécessaire ou constant, de voisinage ou d’interdépendance : par exemple quand
l'effet est désigné par sa cause, ou inversement («la pâle mort méêlait les
sombres bataillons »), le contenu par le contenant («La rue assourdissante
autour de moi hurlait »), ou le tout par la partie (si le rapport est purement
quantitatif, du moins au plus ou du plus au moins, c’est alors une synecdoque :
«trente voiles », dans {\it Le Cid}, pour désigner trente navires). Lacan y voit le
principe du {\it déplacement}, tel qu’il s'effectue dans le travail du rêve ou les
symptômes : « La {\it Verschiebung} ou déplacement, c’est plus près du terme allemand
ce virement de la signification que la métonymie démontre et qui, dès
son apparition dans Freud, est présenté comme le moyen le plus propre à
déjouer la censure » («L’instance de la lettre dans l'inconscient », {\it Écrits},
p. 511 ; voir aussi p. 505-506).

MILIEU (JUSTE —) Voir les articles « Médiété », « Vertu », et « Vice ».

MIMÉTIQUE (FONCTION -) Ce qui nous pousse à imiter ({\it mimeisthai})
ou passe par limitation. C’est une dimension
essentielle du désir. Le rapport entre le sujet désirant et l’objet désiré n’est
pas duel, montre René Girard : c’est un rapport triangulaire, car médiatisé par
le désir de l’autre (je ne désire un objet que parce qu’un autre le désire, que
j'imite ou auquel je m'identifie). C’est ce que Spinoza appelait « limitation des
affects » : « Si nous imaginons qu’une chose semblable à nous et à l'égard de
laquelle nous n’éprouvons aucun affect éprouve de son côté quelque affect,
nous éprouvons par cela même un affect semblable » ({\it Éthique}, III, prop. 27 et
scolie). De là la commisération, qui est limitation d’une tristesse, et l’émulation,
qui est limitation d’un désir, ou plutôt qui « n’est rien autre que le désir
d’une chose engendré en nous de ce que nous imaginons que d’autres êtres
%— 380 —
semblables à nous en ont le désir » ({\it ibid.}). De là aussi l'envie, qui est limitation
d’un amour et pousse à la haine : « Si nous imaginons que quelqu'un tire de la
joie d’une chose qu’un seul peut posséder, nous nous efforcerons de faire qu’il
n’en ait plus la possession » (prop. 32). C’est vrai spécialement des enfants
({\it ibid.}, scolie), mais les adultes n’y échappent pas : « Les hommes sont généralement
prêts à avoir de la commisération pour ceux qui sont malheureux, à
envier ceux qui sont heureux, et leur haine pour ces derniers est d’autant plus
grande qu’ils aiment davantage ce qu’ils imaginent dans la possession d’un
autre » ({\it ibid.}). Reste à aimer ce que tous peuvent posséder : l'amour de la vérité
(voir {\it Éthique}, IV, prop. 36 et 37, avec les dém. et scolies) nous fait sortir, sinon
de limitation, du moins de l’envie et de la haine.

MIMÉTISME Le devenir même de l’autre : c’est devenir semblable à ce
qu’on n’est pas, mais par une imitation involontaire, qui
relève davantage de la physiologie ou de l’imprégnation que d’un apprentissage
délibéré. Ainsi le caméléon, se confondant avec son milieu, ou l’enfant, intériorisant
le sien.

MINEURE Dans un syllogisme, celle des deux prémisses qui contient le
petit terme. On la place ordinairement en second ; mais ce n’est
que convention : Socrate, si l’on commence par lui, n’en mourra pas moins...

MIRACLE «Pour en être réduit à marcher sur les eaux, me dit un jour
Marcel Conche, il faut vraiment n'avoir que de bien piètres
arguments ! » Cela dit à peu près ce qu’il faut penser des miracles. Ce sont des
événements qui paraissent incompréhensibles, qu’on prétend expliquer par une
intervention surnaturelle qui l’est encore davantage. Mais que prouve une
double incompréhension ?

Hume a bien montré ({\it Enquête sur l'entendement humain}, chap. X) qu’un
miracle est par définition plus incroyable que son absence. On vous dit qu’un
homme a ressuscité. La fausseté du témoignage, fût-elle à vos yeux très improbable
(une chance sur cent ? une sur mille ?), l’est de toute façon beaucoup
moins que la résurrection d’un mort (on n’en connaît pas un seul cas vérifié sur
des milliards). Pourquoi est-ce à celle-ci que vous croyez ?

Et quand il y a non pas témoignage mais expérience directe ? Le même
argument vaut aussi. Vous voyez quelqu'un marcher sur les eaux : il est plus
probable que vous ayez une hallucination, ou que ce soit un illusionniste, ou
%— 381 —
qu'il y ait au phénomène une explication naturelle que vous ignorez, qu’il ne
l’est qu’il y ait en effet un miracle, c’est-à-dire une violation, supposée surnaturelle,
de la causalité ordinaire. Qu'est-ce que j’en sais ? C’est que ce ne serait pas
un {\it miracle} autrement, c’est-à-dire un événement par définition tout à fait
improbable ou exceptionnel. La bêtise ou l’aveuglement le sont moins.

Un miracle est un événement incroyable, auquel on croit pourtant, qu’on
juge inexplicable, et qu’on explique pourtant. Croire aux miracles, c’est non
seulement croire sans comprendre, ce qui est le lot ordinaire, mais croire {\it parce
que} Von ne comprend pas ; ce n’est plus foi mais crédulité.

MIRAGE Une apparence trompeuse, due aux contrastes de température
entre différentes couches d’air superposées. En un sens plus large,
et par métaphore, c’est « une erreur chérie, comme dit Alain, qui concerne
principalement les événements extérieurs ». Toutefois on ne parle de mirage
que lorsqu'on cesse d’en être dupe.

MISANTHROPIE La haine ou le mépris de l'humanité, en tant qu’on en fait
partie. Moins grave par là qu’une haine dont on s’exempte
(ainsi la misogynie, chez un homme, ou le racisme, chez celui qui se croit d’une
race supérieure). Molière, qui fit là-dessus l’un de ses chefs-d’œuvre, montre
bien ce qu’il peut y avoir, dans la misanthropie, d’exigence estimable. Toutefois
ce n'est qu'un leurre : aucune exigence n’est vraiment estimable qui porte sur
autrui. Alceste a beau jeu de mépriser ; ce ne sont pas les occasions qui manquent.
Mais à quoi bon ? Que n’exige-t-il plutôt de lui-même la compassion et
la miséricorde ?

MISÉRICORDE La vertu du pardon : non en annulant la faute, ce qu’on ne
peut ni ne doit, mais en cessant de haïr. On y parvient par
la connaissance des causes, et de soi. « Ne pas railler, disait Spinoza, ne pas
pleurer, ne pas détester, mais comprendre » ({\it Traité politique}, I, 4). La miséricorde
est ainsi le contraire de la misanthropie, ou plutôt son remède.

MISOLOGUE Celui qui déteste la raison. C’est ordinairement qu’elle l’a
déçu, remarque Platon ({\it Phédon}, 89 d — 91 a) : il s’en sert
mal, puis lui reproche de ne pas le servir ; il se trompe, puis lui reproche d’être
trompeuse, C’est le défaut commun des sophistes et des imbéciles.

%— 382 —
MODALITÉ Ce jour-là je réunissais dans un restaurant, à propos d’un
numéro de revue que nous faisions ensemble, cinq ou six amis.
Parmi eux, A. et F., tous deux connus en khâgne, tous deux, vingt ans plus
tard, brillants historiens de la philosophie, universitaires réputés, penseurs véritables.
Ils ne se sont pas vus depuis plusieurs années, mais je sais qu’ils ont
gardé, l’un pour l’autre, beaucoup d’estime et d’amitié. Ils parlent d’abord de
broutilles, puis, très vite : « Je voudrais te poser une question, annonce F. : est-ce
que tu crois qu’on peut se forger une représentation cohérente du monde,
sans les catégories de la modalité ? » Silence de plusieurs secondes. A. tire sur sa
pipe. Réfléchit. Puis répond simplement : « Non. » Comme retrouvailles de
vieux copains, j'ai connu plus sentimental, et j'aurais apprécié, je crois, davantage
d'intimité, de confidences, d'émotion... Mais j’admirais aussi, j'admire
toujours, cette façon d’aller droit à l'essentiel, au moins celui de la pensée, cette
intellectualité vraie, dont je savais bien qu’elle n’excluait pas les sentiments,
mais qui ne consentait pas à différer pour cela le débat philosophique...
Quelques années plus tard, je reparlerai à l’un des deux de sa question : il Pavait
oubliée, comme la réponse de l’autre. {\it Sic transit gloria mentis}.

Qu'est-ce que la modalité ? Une modification du jugement, ou plutôt de
son statut. « La modalité des jugements, soulignait Kant, en est une fonction
tout à fait spéciale qui a ce caractère de ne contribuer en rien au contenu du
jugement, [...] mais de ne concerner que la valeur de la copule par rapport à la
pensée en général. Les jugements sont {\it problématiques} lorsqu'on admet l’affirmation
ou la négation comme simplement {\it possibles} ; {\it assertoriques} quand on les
considère comme {\it réelles} (vraies) ; {\it apodictiques} quand on les regarde comme
{\it nécessaires} » ({\it C. R. Pure}, Analytique des concepts, chap. I). Cela débouche, chez
Kant, sur les trois catégories, ou plutôt sur les trois paires de catégories, de la
modalité : {\it possibilité et impossibilité, existence et non-existence, nécessité et contingence}.
Pourquoi peut-on envisager de s’en passer ? Parce qu’elles ne portent pas
sur l’objet (comme les catégories de la quantité ou de la qualité), ni sur les rapports
entre les objets (comme les catégories de la relation), mais simplement sur
le rapport de notre entendement à ces objets. À ne penser que le monde même,
si cela se peut, il semble que le réel y soit tout — que l'existence, comme dirait
Kant, en soit la seule modalité envisageable. Mais alors tout le possible serait
réel, tout le réel serait nécessaire, l'impossible et le contingent ne seraient rien
(ou n’auraient d’existence qu’imaginaire). C’est le monde, à peu près, de Spinoza.
C’est le monde, à peu près, des stoïciens. Est-il cohérent ? Je le crois. Se
passe-t-il des catégories de la modalité ? Pas tout à fait. Mais il les met à leur
place : les unes du côté de l’être ou de Dieu (réalité, possibilité, nécessité : seul
le réel est possible, et il est toujours nécessaire, par quoi ces trois catégories, à la
limite, n’en font qu’une), les autres du côté des êtres de raison ou d’imagination
%— 383 —
(l’impossible, l’inexistant, le contingent : ce ne sont que des façons de
penser ce qui n’est pas). C’est ce qui lui permet d’être cohérent. Non qu’un
monde ne puisse exister sans ces catégories, mais parce que nous ne pourrions,
sans elles, le penser. Ainsi toute représentation cohérente du monde doit intégrer
les catégories de la modalité (puisque notre pensée fait partie du monde),
sans avoir besoin pour cela d’en faire des formes de l'être. Je ne peux penser le
monde sans distinguer le possible du réel ou de l’impossible ; mais cela ne
signifie pas que le monde, lui, les distingue. Ma pensée fait partie du monde,
non le monde, de ma pensée.

MODE Au masculin : une manière, une façon d’être ou une modification,
mais inessentielles (par différence avec lattribut). « J'entends par
{\it attribut}, écrit Spinoza, ce que l’entendement perçoit d’une substance comme
constituant son essence. J'entends par {\it mode} les affections d’une substance,
autrement dit ce qui est dans une autre chose, par le moyen de laquelle il est
aussi conçu » ({\it Éthique}, I, déf. 4 et 5). Un mode, chez Spinoza, est donc un être
quelconque, en tant qu’il est la modification, dans un attribut donné, de la
substance unique. Il y a des modes finis (cet arbre, cette chaise, vous, moi...) et
des modes infinis (l’entendement de Dieu, le mouvement et le repos, l'univers
entier....). Les premiers, pour finis qu’ils soient, n’en sont pas moins réels : ce
sont des êtres finis mais ce sont des êtres vrais, comme autant de fragments de
l'infini ou de labsolu.

Au féminin, le mot désigne une manière collective et provisoire de se comporter,
par exemple de voir les choses, de parler, de penser, de s’habiller...
D'où la terrible formule de Pascal : « Comme la mode fait l’agrément, aussi
fait-elle la justice» ({\it Pensées}, 61-309). C’est que toute justice, au moins
humaine, est collective et provisoire. En pratique, on réservera pourtant le mot
à ce qui change spécialement vite, sans autre justification apparente que ce
changement même. C’est pourquoi on dit que «la mode, c’est ce qui se
démode » : la fugacité fait partie de sa définition. De là cette surprise, quand on
voit des photos ou des magazines d'il y a vingt ans. Toutefois Mozart et
Molière, qui furent fort à la mode, ne se démodent pas.

Toute mode est normative : elle exprime ce qui se fait, mais est vécue (par
ceux qui la suivent) comme indiquant ce qui {\it doit} se faire. C’est une normativité
fugace, ou une fugacité normative. L'enjeu principal — outre la dimension
purement commerciale — est de distinction : « La mode, écrit excellemment
Edgar Morin, est ce qui permet à l'élite de se différencier du commun, d’où son
mouvement perpétuel, [mais aussi] ce qui permet au commun de ressembler à
l'élite, d’où sa diffusion incessante. »

%— 384 —
MODÉRATION La mesure (voir ce mot) dans la pensée ou l’action. On ne
la confondra pas avec la petitesse. Un républicain modéré
n’est pas moins républicain que les extrémistes qui le combattent ou le méprisent.
Il le sera souvent davantage. On dira qu’un « révolutionnaire modéré » est
une contradiction dans les termes. Je n’en suis pas sûr (voyez Condorcet ou
Desmoulins) ; mais si c'était vrai, il faudrait en conclure que toute révolution
est excessive — ce qui donnerait raison aux conservateurs ou aux réformistes.
La modération n’est pas le contraire de la force, de la grandeur ou de la
radicalité ; c’est le contraire de l'excès ou de labus. C’est pourquoi elle est
bonne en tout. « La sagesse a ses excès, disait Montaigne, et n’a pas moins
besoin de modération que la folie. »

{\it MODUS PONENS} C'est une inférence valide, qui fait passer de la vérité

d’une proposition à la vérité de sa conséquence néces-
saire, sous la forme : Si {\it p} alors {\it q} ; or {\it p} ; donc {\it q} (par exemple : Si Socrate est un
homme, il est mortel ; or Socrate est un homme ; donc Socrate est mortel).

MODUS TOLLENS Une inférence valide, qui conclut à la fausseté d’une
proposition par celle de l’une au moins de ses consé-
quences, sous la forme : Si {\it p} alors {\it q} ; or {\it non-q} ; donc {\it non-p} (par exemple : Si
Socrate est un dieu, il est immortel ; or il n’est pas immortel ; donc il n’est
pas un dieu). C’est cette inférence déductive qui est au cœur, selon Popper,
de la {\it falsification} et donc de la démarche des sciences expérimentales. Si la
prévision {\it q} est une conséquence nécessaire de la théorie (ou de l’hypothèse)
{\it p}, il suffit en effet d’un seul fait attestant la fausseté de {\it q} pour entraîner la
fausseté de {\it p}.

MŒURS Les actions humaines, surtout les plus répandues, considérées comme

objets de connaissance ou de jugement. Et bien sûr la connais-
sance vaut mieux, qui dissuade de juger. C’est pourquoi nos plus grands moralistes
sont si peu moralisateurs.

MOI Le sujet, mais en tant qu’objet : c’est le nom français ou commun de
l'{\it ego}. C’est aussi, et par là même, un objet (ou un processus) qui se
prend pour un sujet. Le moi n’est pas une substance, ni un être : il n’est « ni
%— 385 —
dans le corps ni dans l’âme » (Pascal, {\it Pensées}, 688-323) ; il n’est que l’ensemble
des qualités qu’on lui prête ou des illusions qu’il se fait sur lui-même.

Dans le vocabulaire psychanalytique, le même pronom substantivé désigne
l’une des trois instances de la seconde topique de Freud : le pôle conscient
(quoique incomplètement) de l’appareil psychique, qui fluctue et se bat comme
il peut entre les pulsions du ça, les exigences du surmoi et les contraintes de la
réalité. Instance d'équilibre, mais instable : quelque chose comme le fléau de la
balance ou le dindon de la farce.

La construction du moi n’en est pas moins un processus nécessaire, toujours inachevé.
C’est que le moi est à lui-même son propre but : « {\it Wo es war},
écrivait Freud, {\it soll ich werden} » ; là ou {\it ça} était, {\it moi} je dois advenir. Non que le
moi doive « déloger le ça », comme on l’a longtemps traduit (le ça garde bien
sûr sa place, qui est fondamentale), mais en ceci que le moi ne saurait pourtant
la lui abandonner toute : car il n'existe, comme instance spécifique, que pour
autant qu'il se construit, et il ne peut le faire que contre le ça et le réel — tout
contre. {\it Devenir ce que l'on est}, comme disait Nietzsche ? Devenir, plutôt, ce que
l’on veut être, mais que l’on est déjà, en effet, par cette volonté même, sans
qu’on ait pourtant jamais fini de le devenir et sans qu’on puisse davantage y
renoncer. Non déloger le ça, donc, mais lui résister et le surmonter, au moins
par endroits. « C’est là une tâche qui incombe à la civilisation, concluait Freud,
tout comme l’assèchement du Zuyderzee » ({\it Nouvelles conférences}, III).

MONADE C'est une unité ({\it monas}) spirituelle. Le mot n’est plus guère
employé que dans un registre leibnizien. La monade est « une
substance simple, qui entre dans les composés; simple, c’est-à-dire sans
parties » ({\it Monadologie}, \S 1). Un atome ? Non pas, si l’on entend par là un être
matériel. Les monades sont des substances spirituelles, et uniquement
spirituelles : des âmes absolument simples, donc impérissables, qui peuvent être
douées ou non de conscience, mais dont chacune exprime à sa façon l’univers
que leur ensemble constitue. Le leibnizianisme est un panpsychisme pluraliste :
«tout vit, tout est plein d’âme », comme dira Hugo, mais dans l’irréductible
multiplicité de substances individuelles séparées et tout intérieures (« sans porte
ni fenêtres »). C’est en quoi une monadologie est l’homologue spiritualiste de
l’atomisme, donc aussi son contraire.

MONARCHIE Le pouvoir d’un seul, mais conformément à des lois (par différence
avec le despotisme, qui ne se soumet à aucune règle).
On parlera de monarchie absolue quand ces lois dépendent elles-mêmes de la
%— 386 —
volonté du monarque (quand le monarque, autrement dit, est souverain) ; et de
monarchie limitée ou constitutionnelle lorsque c’est au contraire le monarque
qui est soumis aux lois (et spécialement quand c’est le peuple qui est souverain).
On voit que la monarchie absolue est très proche du despotisme (c’est un
despotisme bien réglé), comme la monarchie constitutionnelle peut n'être
qu’une forme de démocratie. En Angleterre ou en Espagne, aujourd’hui, c’est
évidemment le peuple qui est souverain : le monarque règne, dans ces pays ; il
ne gouverne ni ne légifère. Le roi, dans une monarchie constitutionnelle, n’est
pas souverain ; il n’est que le symbole, tant que le peuple y consent, de la
nation ou de la souveraineté.

MONDE Dans la langue philosophique, c’est souvent un synonyme d’uni-
vers : le monde est « l’assemblage entier des choses contingentes »
(Leibniz), l’ensemble de « tous les phénomènes » (Kant) ou de « tout ce qui
arrive» (Wittgenstein). Mais alors l’idée d’une pluralité des mondes, bien
attestée dans l’histoire de la philosophie, devient inintelligible : comment y
aurait-il plusieurs {\it tout} ? Il convient donc de distinguer le monde (le {\it kosmos} des
Grecs) du Tout ({\it to pan}). Pour les Anciens, le monde est un tout, mais pas {\it le}
Tout. C’est l’ensemble ordonné qui nous contient, tel que nous pouvons
l’observer, depuis la Terre jusqu’au ciel et aux astres. Il n’est pas inenvisageable
qu’il en existe d’autres, voire un nombre infini (c’est ce que pensait Épicure).
Mais nous ne pouvons les connaître, faute d’en avoir la moindre expérience.

Quand on parle du monde, sans plus de précision, il est entendu que c’est
le nôtre. C’est l’ensemble, qui nous contient, de tout ce avec quoi nous sommes
en relation, de tout ce que nous pouvons constater ou expérimenter —
l’ensemble des faits, plutôt que des choses ou des événements. C’est le réel qui
nous est accessible : une petite portion de l’être, valorisée (pour nous) par notre
présence. C’est notre lieu de coïncidence, ou le paquet-cadeau du destin. Après
tout, nous aurions pu tomber plus mal.

Les scientifiques l’appellent parfois l’univers, qui serait le tout du réel. Mais
comme nous ne le connaissons jamais qu’en partie, et comme nous ne connaissons
rien d’autre, comment savoir s’il est le tout ?

MONÈME Une unité minimale de signification. Par exemple le mot {\it monde}
n’en comporte qu’un (si on le divise, le sens se perd) ; le mot
{\it monisme}, deux : {\it mon} (qui différencie par exemple {\it monisme} de {\it dualisme}) et {\it isme}
(qui le différencie par exemple de {\it monarchie}) ; la phrase « Vous êtes embarqués »,
%— 387 —
cinq. C’est l'élément de la première articulation (voir ce mot), comme
le phonème est celui de la seconde.

MONISME Toute doctrine pour laquelle il n’existe qu’une seule substance,
ou qu’un seul type de substances. Un monisme peut être maté-
rialiste, s’il affirme que toute substance est matérielle (ainsi chez les stoïciens,
Diderot ou Marx), spiritualiste, s’il les juge toutes spirituelles (ainsi chez Leibniz
ou Berkeley), ou ni l’un ni l’autre, si matière et pensée ne sont pour lui que
des modes ou des attributs d’une substance unique, qui ne saurait dès lors se
réduire à l’une ou l’autre (c’est le cas, spécialement, chez Spinoza). S'oppose en
tous les cas au dualisme, qui pose l’existence de deux types de substances (Descartes)
ou de deux mondes (Platon, Kant). Devrait pouvoir s’opposer aussi au
pluralisme, qui supposerait, pris en son sens fort, l’existence d’un nombre indéfini
de substances de natures différentes. Mais l'imagination des philosophes ne
s’est guère étendue jusque-là. Nous savons, au moins vaguement, ce que c’est
qu'un corps et qu’un esprit. Mais de substances qui ne seraient ni l’un ni
l’autre, ni l'unité indissoluble des deux, nous n’avons aucune expérience. Comment
pourrions-nous les penser ?

MONNAIE Instrument de paiement : un petit morceau du réel, qui peut
être échangé contre la plupart des autres — à condition toutefois
que quelqu'un les possède et soit prêt à les vendre. C’est « l'équivalent
universel », comme disait Marx, qui libère le commerce du troc et la richesse de
l'encombrement.

MONOTHÉISME La croyance en un Dieu unique. Les Modernes ont le sentiment
qu’il serait autrement moins divin, son pouvoir se
trouvant inévitablement limité par celui, dans le polythéisme, des autres dieux.
Le fait est que les plus hautes pensées du divin, dès l’Antiquité (chez Platon,
chez Aristote, chez Plotin...), ont eu tendance à le penser comme unité, au
moins en son sommet, et comme unicité : le Bien en soi, le Premier Moteur
immobile ou l’'Un ne sont guère susceptibles d’exister au pluriel. J'y vois plutôt
un progrès. Moins il y a de dieux, mieux ça vaut.

On a pourtant beaucoup reproché au monothéisme, ces dernières décennies,
de mener directement au monolithisme, au totalitarisme, à l'exclusion de
l’autre, de la pluralité, de la différence, du multiple... Que ce thème ait suscité
tellement d’enthousiasme à l’extrême droite me le rendrait déjà suspect. Qu'il
%— 388 —
ait été à ce point démenti par l’histoire (car enfin qui ne voit que les deux
grands totalitarismes de ce siècle n’avaient, pour le monothéisme, que haine ou
mépris ?) n’est pas non plus à son avantage. Mais il y a autre chose, qui est
l’universel. S’il n’existe qu’un seul Dieu, c’est donc le même pour tous : nous
voilà tous frères, en tout cas susceptibles de le devenir, tous ouverts à la même
vérité, tous soumis, au moins en droit, à la même loi. Totalitarisme ? Mais alors
il faudrait parler aussi de totalitarisme pour les sciences, qui sont les mêmes
pour tous, pour la morale, qui tend à l’être, enfin pour les droits de l’homme,
qui n’ont de sens qu’universels. Et au nom de quoi? De Zeus, d’Arès ou
d’Aphrodite, d’Odin ou de Thor, de tout ce grand guignol de Olympe ou du
Walhalla ? Mieux vaut l'immense absence, comme disait Alain, partout présente.
Voilà le monde vidé de tous ses dieux, et bientôt rendu à lui-même.

MORALE L'ensemble de nos devoirs, autrement dit des obligations ou des
interdits que nous nous imposons à nous-mêmes, indépendamment
de toute récompense ou sanction attendue, et même de toute espérance.
Imaginons qu’on nous annonce la fin du monde, certaine, inévitable, pour
demain matin. La politique n’y survivrait pas, qui a besoin d’un avenir. Mais la
morale ? Elle demeurerait pour l'essentiel inchangée. La fin du monde, même
inévitable à très court terme, n’autoriserait en rien à se moquer des infirmes, à
calomnier, à violer, à torturer, à assassiner, bref à être égoïste ou méchant. C’est
que la morale n’a pas besoin d’avenir. Le présent lui suffit. Elle n’a pas besoin
d'espérance. La volonté lui suffit. « Une action accomplie par devoir tire sa
valeur {\it non pas du but} qui doit être atteint par elle, souligne Kant, mais de la
maxime d’après laquelle elle est décidée. » Sa valeur ne dépend pas de ses effets
attendus, mais seulement de la règle à laquelle elle se soumet, indépendamment
de tout penchant, de toute inclination, de tout calcul égoïste, enfin « sans égard
à aucun des objets de la faculté de désirer » et « abstraction faite des fins qui
peuvent être réalisées par une telle action » ({\it Fondements}..., 1). Si tu agis pour la
gloire, pour ton bonheur, pour ton salut, et quand bien même tu agirais en
tout conformément à la morale, tu n’agis pas encore moralement. Une action
n’a de valeur morale véritable, explique Kant, que dans la mesure où elle est
désintéressée. Cela suppose qu’elle n’est pas seulement accomplie {\it conformément
au devoir} (à quoi l'intérêt peut suffire : ainsi le commerçant qui n’est honnête
que pour garder ses clients), mais bien {\it par devoir}, autrement dit par pur respect
de la loi morale ou, cela revient au même, de l’humanité. La proximité de la fin
du monde n’y changerait rien d’essentiel : nous serions toujours tenus, et
jusqu’au dernier instant, de nous soumettre à ce qui nous paraît universellement
valable, universellement exigible, et spécialement (mais à nouveau cela
%— 389 —
revient au même) de respecter l'humanité en nous et en l’autre. C’est en quoi
la morale est désespérée, au moins en un certain sens, et désespérante peut-être.
« Elle n’a aucunement besoin de la religion », insiste Kant, ni de quelque fin ou
but que ce soit : «elle se suffit à elle-même » ({\it La religion dans les limites de la
simple raison}, Préface). C’est en quoi elle est laïque, même chez les croyants, et
commande, c'est du moins le sentiment que nous avons, absolument. Que
Dieu existe ou pas, qu'est-ce que cela change au devoir de protéger les plus
faibles ? Rien, bien sûr, et c’est pourquoi on n’a pas besoin de savoir ce qu’il en
est de cette existence pour agir humainement.

Imaginons à l'inverse, c’est un exemple qu’on trouve chez Kant, que Dieu
existe et soit connu de tous. Que se passerait-il ? « Dieu et l’éternité, avec leur
majesté redoutable, seraient sans cesse devant nos yeux. » Nul n’oserait plus
désobéir. La peur de l’enfer et l’espérance du paradis donneraient aux commandements
divins une force sans pareille. Ce serait le règne en tout de la soumis-
sion intéressée ou craintive, comme un ordre moral absolu : « La transgression
de la loi serait bien sûr évitée, ce qui est ordonné serait accompli. » Mais nulle
{\it moralité} n’y survivrait : « La plupart des actions conformes à la loi seraient produites
par la crainte, quelques-unes seulement par l’espérance, et aucune par
devoir, de sorte que la valeur morale des actions, sur laquelle seule reposent la
valeur de la personne et même celle du monde aux yeux de la suprême sagesse,
n'existerait plus » ({\it C. R. Pratique}, Dialectique, II, 9). Non seulement il n’est
pas besoin d’espérer pour faire son devoir, mais on ne le fait vraiment qu’à la
condition que ce ne soit pas par espérance.

Où veux-je en venir ? Simplement à ceci : la morale, contrairement à ce
qu'on croit souvent, n’a rien à voir avec la religion, encore moins avec la peur
du gendarme ou du scandale. Ou si elle fut liée, historiquement, aux Églises,
aux États et à l’opinion publique, elle ne devient vraiment elle-même — c’est
l’un des apports décisifs des Lumières — que dans la mesure où elle s’en libère.
C’est ce qu'ont montré, chacun à sa façon, Spinoza, Bayle ou Kant. C’est ce
que Brassens, quand j'avais quinze ans, suffit à me faire comprendre. Ramenée
à son essence, la morale est le contraire du conformisme, de l’intégrisme, de
l’ordre moral, y compris sous leurs formes molles qu’on appelle aujourd’hui le
« politiquement correct ». Elle n’est pas la loi de la société, du pouvoir ou de
Dieu, encore moins celle des médias ou des Églises. Elle est la loi que l'individu
se prescrit à lui-même : c’est en quoi elle est libre, comme dirait Rousseau
(« l’obéissance à la loi qu’on s’est prescrite est liberté »), ou autonome, comme
dirait Kant (parce que l'individu n’y est soumis qu’à « sa législation propre, et
néanmoins universelle »). Que cette liberté ou cette autonomie ne soient elles-mêmes
que relatives, c’est ce que je crois, contre Kant et Rousseau, mais qui ne
change pas le sentiment que nous avons d’un absolu pratique (qui relève de la
%— 390 —
volonté, non de la connaissance) et d’une exigence, pour nous, inconditionnelle.
Que toute morale soit historique, j’en suis convaincu. Mais, loin que cela
supprime la morale, c’est au contraire ce qui la fait exister et nous y soumet :
puisque nous sommes {\it dans} l’histoire et produits par elle. Autonomie relative,
donc, mais qui vaut mieux que l’esclavage des penchants ou des peurs. Qu’est-ce
que la morale ? C’est l’ensemble des règles que je m'impose à moi-même, ou
que je devrais m’imposer, non dans l’espoir d’une récompense ou la crainte
d’un châtiment, ce qui ne serait qu’égoïsme, non en fonction du regard
d’autrui, ce qui ne serait qu'hypocrisie, mais au contraire de façon désintéressée
et libre : parce qu’elles me paraissent s’imposer universellement (pour tout être
raisonnable) et sans qu’on ait besoin pour cela d’espérer ou de craindre quoi
que ce soit. « {\it Tout seul}, disait Alain, {\it universellement}. » C’est la morale même.

Cette morale est-elle {\it vraiment} universelle ? Jamais complètement sans
doute : chacun sait bien qu’il existe des morales différentes, qui varient selon les
lieux et les époques. Mais elle est universalisable sans contradiction, et d’ailleurs
de plus en plus universelle en fait. Si on laisse de côté quelques archaïsmes douloureux,
qui doivent plus à des pesanteurs religieuses ou historiques qu’à des
jugements proprement moraux (je pense spécialement à la question sexuelle et
au statut des femmes), force est de reconnaître que ce qu’on entend par « {\it un
type bien} », en France, n’est pas très différent — et le sera sans doute de moins
en moins — de ce que les expressions équivalentes peuvent désigner en Amérique
ou en Inde, en Norvège ou en Afrique du Sud, au Japon ou dans les pays
du Maghreb. C’est quelqu'un qui est sincère plutôt que menteur, généreux
plutôt qu'égoïste, courageux plutôt que lâche, honnête plutôt que malhonnête,
doux ou compatissant plutôt que violent ou cruel... Cela ne date pas d’hier.
Rousseau déjà, contre le relativisme montanien, ou plutôt contre la vision qu'il
en avait, en appelait à une forme de convergence morale, à travers les différentes
civilisations : « O Montaigne ! toi qui te piques de franchise et de vérité,
sois sincère et vrai, si un philosophe peut l’être, et dis-moi s’il est quelque pays
sur la terre où ce soit un crime de garder sa foi, d’être clément, bienfaisant,
généreux ; où l’homme de bien soit méprisé, et le perfide honoré ? » Montaigne
n’en eût guère trouvé, ni cherché : relisez ce qu’il écrit sur les indiens d'Amérique,
que nous avons si atrocement traités, sur leur courage, sur leur constance, sur
leur « bonté, libéralité, loyauté, franchise » ({\it Essais}, III, 6). L’humanité n’appartient
à personne : le relativisme montanien est aussi un universalisme, et ce
n’est en rien contradictoire (puisque la morale est relative à toute l'humanité :
« chaque homme porte la forme entière de l’humaine condition », III, 2). Au
reste l’histoire, sur tous les continents, parle assez clair. Nul ne sait quand la
morale a commencé ; mais cela fait deux ou trois mille ans, selon les différentes
régions du globe, que l'essentiel a été dit : par les prêtres égyptiens ou assyriens,
%— 391 —
par les prophètes hébreux, par les sages hindous, enfin, c’est l’étonnante floraison
des sixième et cinquième siècles avant notre ère, par Zarathoustra (en
Iran), Lao-tseu et Confucius (en Chine), le Bouddha (en Inde), et en Europe
par les premiers philosophes grecs, ceux qu’on appelle les présocratiques.. Qui
ne voit que leurs messages moraux, par-delà d'innombrables oppositions philosophiques
ou théologiques, sont fondamentalement convergents ? Qui ne voit
que c’est encore plus vrai aujourd’hui ? Prenez l'Abbé Pierre et le Dalaï-Lama. Ils
n'ont pas la même origine, pas la même culture, pas la même religion... Mais il
suffit de les écouter quelques minutes pour constater que les morales qu’ils professent
vont dans la même direction. La mondialisation n’a pas que des mauvais
côtés, et elle a commencé bien plus tôt qu’on ne le croit. Nous bénéficions
aujourd’hui d’un lent processus historique, qui s’est poursuivi, avec des hauts et
des bas, durant quelque vingt-cinq siècles, dont nous sommes à la fois le résultat
et les débiteurs. Ce processus, à ne le considérer ici que d’un point de vue moral,
et malgré les formes violentes qu’il prit souvent, est un processus de convergence
des plus grandes civilisations autour d’un certain nombre de valeurs communes
ou voisines, celles qui nous permettent de vivre ensemble sans trop nous nuire ou
nous haïr. C’est ce qu’on appelle aujourd’hui les droits de l’homme, qui sont surtout,
moralement, ses devoirs.

Cette morale, d’où vient-elle ? De Dieu ? Ce n’est pas impossible : il a pu
mettre en nous, comme le voulait Rousseau, « l’immortelle et céleste voix » de
la conscience, qui prendrait le pas, ou qui devrait le prendre, sur toute autre
considération, fût-elle celle de notre salut ou de sa propre gloire. Mais s’il n’y
a pas de Dieu ? Alors il faut penser que la morale n’est qu’humaine, qu’elle
n’est qu'un produit de l’histoire, que l’ensemble des normes que humanité, au
fil des siècles, a retenues, sélectionnées, valorisées. Pourquoi celles-là ? Sans
doute parce qu’elles étaient favorables à la survie et au développement de
l'espèce (c’est ce que j'appelle la morale selon Darwin), aux intérêts de la société
(c’est la morale selon Durkheim), aux exigences de la raison (c’est la morale
selon Kant), enfin aux recommandations de l’amour (c’est la morale selon Jésus
ou Spinoza).

Imaginez une société qui prônerait le mensonge, l’égoïsme, le vol, le
meurtre, la violence, la cruauté, la haine. Elle n'aurait guère de chances de
subsister, encore moins de se répandre à l’échelle de la planète : parce que les
hommes ne cesseraient de s’y affronter, de s’y nuire, de s’y détruire... Aussi
n'est-ce pas une coïncidence si toutes les civilisations qui se sont répandues
dans le monde s'accordent au contraire pour valoriser la sincérité, la générosité,
le respect de la propriété et de la vie d’autrui, enfin la douceur, la compassion
ou la miséricorde. Quelle humanité autrement ? Quelle civilisation autrement ?
Cela dit quelque chose d’important sur la morale : qu’elle est ce par quoi
%— 392 —
l'humanité devient humaine, au sens normatif du terme (au sens où l'humain
est le contraire de l’inhumain), en refusant la veulerie et la barbarie qui ne cessent,
ensemble, de la menacer, de l'accompagner, et qui la tentent. Seuls les
humains, sur cette terre, ont des devoirs. Cela indique clairement la direction,
vers quoi il faut tendre : le seul devoir, ou celui qui résume tous les autres, c’est
d’agir humainement.

Que cela ne tienne pas lieu de bonheur, ni de sagesse, ni d’amour, c’est une
évidence ; c’est pourquoi nous avons besoin aussi d’une éthique (voir ce mot).
Mais sauf à être sage absolument, ou inhumain absolument, qui pourrait se
passer de morale ?

MORT Le néant ultime. Ce n’est donc rien ? Pas tout à fait pourtant, puisque
ce {\it rien} nous attend, ou puisque nous l’attendons. Disons que la
mort n’est rien, mais que nous mourrons : cette vérité au moins n’est pas rien.
Épicure et Lucrèce, sur cette question, me paraissent plus judicieux que Spinoza.
« Un homme libre ne pense à aucune chose moins qu’à la mort, dit une
fameuse proposition de l {\it Éthique}, et sa sagesse est une méditation non de la mort
mais de la vie » (IV, 67). À la seconde affirmation, j’adhère absolument. Mais à
la première, non, ni ne vois comment les deux peuvent être compatibles. Comment
méditer la vie sans penser la mort, qui l’achève ? C’est au contraire parce
que nous pensons que la mort n’est rien, dirait Épicure (rien pour les vivants,
puisqu'ils sont vivants, rien pour les morts, puisqu'ils ne sont plus), que nous
pouvons profiter de la vie sereinement. À quoi bon autrement philosopher ? Et
comment le faire en laissant la mort de côté ? Celui qui a peur de la mort, il a
peur, exactement, {\it de rien}. Comment n’aurait-il pas peur de tout ? Alors qu’il n’y
a rien à craindre dans la vie, expliquait encore Épicure, pour celui qui a compris
que le mal le plus redouté, la mort, n’est rien pour nous ({\it Lettre à Ménécée}, 125).
Encore faut-il la penser strictement — comme néant — pour cesser de l’imaginer
(comme enfer ou comme manque) et de la craindre. Cela suffira-t-il ? Ce n’est
pas sûr. Et même ce n’est pas, lorsque la mort sera toute proche, le plus probable.
Mais pourquoi la pensée devrait-elle suffire ? Comment le pourrait-elle ? Et
qu'importe qu’elle ne suffise pas, si cette idée vraie, ou qui nous paraît telle, nous
aide, ici et maintenant, à vivre mieux ? Une philosophie, même insuffisante, vaut
mieux que pas de philosophie du tout.

Apprendre à mourir ? Ce n’est qu’une partie, non la plus importante ni la
plus difficile, du général apprentissage de vivre. Au reste, et comme l’a dit plaisamment
Montaigne, quand bien même nous ne saurions mourir, nous aurions
tort de nous en inquiéter : « Nature nous en informera sur-le-champ, pleinement
et suffisamment » ({\it Essais}, III, 12). S’il faut penser la mort, ce n’est pas
%— 393 —
pour apprendre à mourir — nous y parviendrons de toute façon — mais pour
apprendre à vivre. Penser la mort, donc, pour l’apprivoiser, pour l’accepter,
puis pour penser à autre chose. « Je veux qu’on agisse, écrit merveilleusement
Montaigne, et qu’on allonge les offices de la vie tant qu’on peut ; et que la mort
me trouve plantant mes choux, mais nonchalant d’elle, et encore plus de mon
jardin imparfait » ({\it Essais}, I, 20).

MOT L'élément d’une langue : élément non pas minimal (ce n’est ni un
phonème ni un monème) mais qui fait, dans une langue donnée,
comme une unité signifiante, empiriquement repérable et reconnaissable.
L'erreur serait d’y voir une copie des choses, quand ce n’est une copie — ou une
matrice — que de notre pensée. Que le mot « néant » existe, par exemple, cela
ne prouve pas que le néant soit.

Les mots sont des outils : morceaux de sens et d’irréel (en tant qu’ils sont
réels, comme bruits, ils ne signifient rien), pour dire l’insignifiante ou insensée
réalité. Il s’agit, par un jeu construit d’unités discrètes, de découper le réel — de
{\it briser le silence} —, puis, comme on peut, d’en recoller les morceaux. De là ces
petits bruits, ces petites idées, et le grand bavardage de l'esprit. De là, aussi,
la tentation du silence. « Qu’y a-t-il dans un mot ? demandait Shakespeare. Ce
que nous appelons rose, sous un autre nom, sentirait aussi bon. »

MOURIR C'est le passage ultime, où rien ne passe. C’est pourquoi on ne
meurt pas : on agonise (mais les mourants sont vivants, hélas),
puis on est mort (mais les morts ne sont plus). Mourir est un acte sans sujet, et
sans acte : un rond dans l’eau du destin, une imagination, une fantasmagorie,
cette fois bien douloureuse, de l’'amour-propre. Le corps lâche son âme comme
un pet, voilà ce qu'il faut dire, et le pet seul, à l’avance, se rebiffe. Qu'’as-tu,
mon corps, à te soucier de tes vents ?

MOUVEMENT C'est changer de lieu ou d’état, de position ou de disposition.
Aristote ({\it Physique}, IT, 1 et VIII, 7) distinguait quatre mouvements — mais
on dirait mieux, en français, quatre changements — principaux, correspondant
à autant de catégories : selon le lieu (le mouvement local), selon la substance
(génération et destruction), selon la quantité (accroissement et diminution),
selon la qualité (altération). Il y voyait le passage de la puissance à l’acte, passage
toujours inachevé (puisqu'il y a mouvement) et pour cela indissolublement
%— 394 —
en puissance et en acte : c’est « l’acte de ce qui est en puissance, en tant
que tel », c’est-à-dire, précise Aubenque, en tant qu’il est en puissance — l’acte
de la puissance, ou la puissance en acte (III, 1 et 2 ; voir aussi P. Aubenque, {\it Le
problème de l'être...}, p.454). C’est ce que nous appelons le changement ou le
devenir, dont le mouvement local n’est qu’une espèce, mais sans doute aussi,
dans l’espace, la condition de toutes les autres.

MULTITUDE Un grand nombre. Quand on ne précise pas de quoi, il s’agit
le plus souvent d’êtres humains, considérés dans leur rassemblement
purement factuel, sans ordre et sans unité. S’oppose par là à l'État, qui
suppose l’ordre, et au peuple, qui suppose lunité. La multitude est « comme
une hydre à cent têtes, disait Hobbes, qui ne doit prétendre dans la république
qu’à la gloire de l’obéissance » ({\it Le Citoyen}, VI, 1).

MYSTÈRE Quelque chose qu’on ne peut comprendre, mais à quoi l’on
croit. Se distingue par là du {\it problème} (quelque chose qu’on ne
comprend pas encore) et de l’{\it aporie} (à laquelle il n’est pas besoin de croire). Par
exemple l’origine de la vie est un problème. L'origine de l’être (« Pourquoi y a-t-il
quelque chose plutôt que rien ? »), une aporie. Dieu, un mystère.

MYSTIQUE L’étymologie rattache le mot aux mystères. Mais les mystiques,
dans toutes les religions, nous parlent plutôt d’une espèce
d’évidence. C’est eux qu’il faut croire, plutôt que le passé de la langue ou de la
superstition. Le mystique, c’est celui qui voit la vérité face à face : il n’est plus
séparé du réel par le discours (c’est ce que j'appelle le silence), ni par le manque (ce
que j'appelle la plénitude), ni par le temps (ce que j'appelle l'éternité), ni enfin par
lui-même (ce que j'appelle la simplicité : l’{\it anatta} des bouddhistes). Dieu même a
cessé de lui manquer. Il fait expérience de l'absolu, ici et maintenant. Cet absolu,
est-ce encore un Dieu ? Plusieurs mystiques, spécialement en Orient, ont répondu
que non. De là un « mysticisme pur », comme disait le père Henri de Lubac, qui
est «la forme la plus profonde de l’athéisme» ({\it La mystique et les mystiques},
A. Ravier {\it et all.}, Préface). Ceux-là ne croient en rien : l'expérience leur suffit.

Ce mysticisme-là, qui est le maximum d’évidence, est ainsi le contraire de
la religion, qui est le maximum de mystère.

MYTHE Une fable que l’on prend au sérieux.

%  395

%  N

NAÏVETÉ On ne la confondra pas avec la niaiserie. Le niais manque d’intelligence
ou de lucidité; le naïf, de ruse et d'artifice. Vertu
d’enfance ou de nature, qui ne saurait toutefois justifier le manque de maturité,
de culture ou de politesse.

NARCISSISME  L’amour, non de soi, mais de son image : Narcisse, incapable
de la posséder, incapable d’aimer autre chose, finit
par en mourir. C’est la version auto-érotique de l’amour-propre, et un autre
piège. On n’en sort que par l'amour vrai, qui n’a que faire des images.

NATION Un peuple, mais considéré d’un point de vue politique plutôt que
biologique ou culturel (ce n’est ni une race ni une ethnie), et
comme ensemble d'individus plutôt que comme institution (ce n’est pas, ou
pas nécessairement, un État). Renan a bien vu que l'existence et la pérennité
d’une nation doit moins à la race, à la langue ou à la religion qu’à la mémoire
et à la volonté. Deux choses surtout la constituent : « L’une est la possession en
commun d’un riche legs de souvenirs ; l’autre est le consentement actuel, le
désir de vivre ensemble, la volonté de continuer à faire valoir l’héritage qu’on a
reçu indivis. [...] Avoir des gloires communes dans le passé, une volonté commune
dans le présent ; avoir fait de grandes choses ensemble, vouloir en faire
encore, voilà les conditions essentielles pour être un peuple » ou une nation
({\it Qu'est-ce qu'une nation ?}, III). C’est dire qu’il n’est de nation que fidèle, et tel
est le vrai sens du patriotisme.

%— 396 —
NATIONALISME C'est ériger la nation en absolu, à quoi tout — le droit, la
morale, la politique — devrait se soumettre. Toujours virtuellement
antidémocratique (si la nation est vraiment un absolu, elle ne
dépend plus du peuple : c’est lui au contraire qui dépend d’elle), presque toujours
xénophobe (ceux qui ne font pas partie de la nation sont comme exclus
de labsolu). C’est un patriotisme exagéré et ridicule : il érige la politique en
religion ou en morale. Aussi est-il volontiers païen et presque inévitablement
immoral.

NATURALISME Toute doctrine qui considère la nature, prise en son sens
large, comme la seule réalité : c’est considérer que le surnaturel
n’existe pas ou n’est qu’une illusion. Un synonyme, donc, du
matérialisme ? Pas tout à fait. Tout matérialisme est un naturalisme, mais tout
naturalisme n’est pas matérialiste (ainsi, Spinoza). Disons que le naturalisme
est le genre prochain, dont le matérialisme ne serait qu’une espèce : c'est un
naturalisme moniste, comme était aussi le spinozisme, mais qui considère que
la nature est intégralement et exclusivement matérielle.

NATURE La {\it phusis}, chez les Grecs, comme la {\it natura} chez Lucrèce ou Spinoza,
c’est le réel lui-même, considéré dans son indépendance,
dans sa spontanéité, dans son pouvoir d’auto-production ou d’auto-développement.
S’oppose en cela à l’art ou à la technique (comme ce qui se fait tout seul
à ce qui est fait par l’homme) et au divin (comme ce qui se développe ou
change à ce qui est immuable). Peut se dire en un sens général (la nature est
l’ensemble des êtres naturels) ou en un sens particulier (la nature d’un être,
qu’on appelle parfois son essence, étant alors ce qu’il y a en lui de naturel : son
« principe, comme dit Aristote, de mouvement et de fixité »). S’oppose dans les
deux cas au surnaturel ou au culturel : la nature, c’est tout ce qui existe, ou qui
semble exister, indépendamment de Dieu — sauf bien sûr à définir Dieu comme
la nature elle-même — ou des hommes.

NATURE HUMAINE Il était de bon ton, dans les années 1960, de dire
qu’elle n’existe pas : l’homme ne serait que culture et
liberté. Si c'était si simple pourtant, pourquoi aurions-nous tellement peur des
manipulations génétiques sur les cellules germinales, celles qui transmettent le
patrimoine héréditaire de l'humanité ? La vérité, à ce que je crois, c’est qu’il y
a bien une nature humaine, ou en tout cas une nature de l’homme, qui est le
%— 397 —
résultat, en chacun, du processus naturel d’hominisation : notre nature, c’est
tout ce que nous recevons à la naissance et transmettons génétiquement, le cas
échéant et au moins en partie, à nos descendants. On sait de plus en plus à quel
point c’est considérable. Reste à l’humaniser, ce qui ne se fait que par éducation
et apprentissage. C’est pourquoi on peut dire, mais en un tout autre sens, qu’il
n'y a pas de nature humaine : non parce qu’il n’y aurait rien de naturel en
l’homme, mais parce que ce qui est naturel en lui n’est pas humain, au sens
normatif du terme, de même que ce qui est humain n’est pas naturel. On naît
homme, ou femme : telle est notre nature. Puis on devient humain : telle est
notre culture et notre tâche.

NATURE NATURANTE / NATURE NATURÉE C’est une expression d’origine scolastique : la {\it natura
naturans} serait Dieu, la {\it natura naturata} l'ensemble des choses créées. Toutefois
ces deux expressions, aujourd’hui, sont plus souvent prises dans un sens spinoziste,
donc panthéiste, qui fait référence à un scolie fameux de l’{\it Éthique}. La
{\it Nature naturante}, c’est la nature en tant qu’elle est Dieu, c’est-à-dire cause de
soi et de tout (non les modes, mais les attributs éternels et infinis de la substance).
La {\it Nature naturée} c’est l’ensemble des effets — dont chacun est aussi une
cause — qui en découlent nécessairement : la chaîne infinie des causes finies
(lensemble non des attributs mais des modes). Disons que la Nature naturante,
c’est la nature comme cause de soi et de tout ; et la Nature naturée,
l’ensemble des effets, mais en elle, de cette causalité immanente (I, 29, scolie).

NATUREL Au sens large ou classique : tout ce qui n’est pas surnaturel. Au
sens étroit et moderne : tout ce qui n’est pas culturel. Ce dernier
sens reste problématique. Si l’homme fait partie de la nature, comme je le crois,
comment ne serait-ce pas vrai aussi de la culture ?

Il reste que s’agissant spécialement du monde humain, il est commode de
distinguer ce qui est {\it naturel} (qui est transmis par les gènes et se reconnaît à
l’'universalité) de ce qui est {\it culturel} (qui est transmis par l'éducation et se reconnaît
à des règles particulières). Par exemple la pulsion sexuelle est naturelle. La
façon de vivre cette pulsion, et spécialement de la satisfaire ou pas, est culturelle.
La procréation est naturelle ; la façon de faire des enfants (et {\it a fortiori} de
les élever), culturelle. La faim est naturelle. La gastronomie et la façon de
manger, culturelles. C’est en quoi la prohibition de l'inceste, remarque Lévi-Strauss,
fait problème : c’est qu’elle semble vérifier à la fois l’universalité de la
nature et la particularité réglée de la culture (on ne connaît pas de société
%— 398 —
humaine où l'inceste ne soit prohibé, mais toutes ne le prohibent pas de la
même façon ni dans les mêmes limites). La solution du problème, selon Lévi-Strauss,
c’est que la prohibition de l'inceste relève bien de ces deux ordres :
parce qu’elle assure le {\it passage} de la nature (la procréation) à la culture (la
parenté), de la filiation à l'alliance, de la famille à la société. Par quoi la nature
a toujours le premier mot, comme elle a aussi, par la mort, le dernier.

NÉANT Le non-être, ou le non-étant, mais considéré plutôt positivement.
Ce n’est pas tout à fait un ensemble vide, ni un pur rien : c’est un
ensemble, plutôt, dont le vide serait l’unique élément, ou un rien qui serait
aussi réel, à sa façon, que le quelque chose. Le néant, remarquait Hegel, « est
égalité simple avec lui-même, vacuité parfaite, absence de détermination et de
contenu, état de non-différenciation en lui-même ». C’est en quoi il n’est rien,
sans cesser pour autant d’être (puisqu'il {\it est} ce rien) : il est « la même détermination,
ou plutôt la même absence de détermination, et partant absolument la
même chose que l'être pur » ({\it Logique}, 1, 1). Bergson, peut-être moins dupe du
langage ou de la dialectique, ne voyait dans le néant qu’un mot, qu’une
pseudo-idée, qui ne serait obtenue que par négation de celle d’être, laquelle
peut seule être pensée positivement. Il avait sans doute raison, mais cela ne
prouve pas que, de cette pseudo-idée, on puisse tout à fait se passer.

Chez Heidegger et Sartre, le néant se révèle dans l’expérience de l’angoisse :
soit parce que l’être ne se donne que sur fond de néant (puisque aucun étant
n’est l’être, puisque l’être n’est rien d’étant) et de facticité (tout ce qui est {\it aurait
pu} ne pas être), soit parce que la conscience — n’étant pas ce qu’elle est, étant ce
qu’elle n’est pas — est pouvoir de néantisation. C’est constater une nouvelle fois
que le néant n’existe que pour l’homme : l’homme est l'être par qui le néant
vient au monde. En ce sens, le néant est moins ce qui n’est pas que ce qui n’est
plus, ou pas encore, ou pas tout à fait. C’est le corrélat vide de la conscience,
par quoi elle n’est jamais prisonnière de ses objets ou de son être : le résultat,
qu’on aurait bien tort d’hypostasier, de son pouvoir de néantisation. « Dans la
nuit claire du Néant de l’angoisse, écrit Heidegger, se montre enfin la manifestation
originelle de l’étant comme tel : à savoir {\it qu'il y ait de l'étant — et non pas
Rien}... Le néant est la condition qui rend possible la révélation de l’étant
comme tel pour la réalité humaine ({\it Dasein}). Le Néant ne forme pas simplement
le concept antithétique de l’étant, mais l'essence de l’Être même comporte
dès l’origine le Néant. C’est dans l’{\it être} de l’étant que se produit le {\it néantir}
du Néant » ({\it Qu'est-ce que la métaphysique ?}, trad. H. Corbin). Par quoi le {\it berger
de l'être}, comme dit ailleurs Heidegger de l’homme, devient {\it « la sentinelle du
Néant »}... Bergson, reviens : ils sont devenus fous !

%— 399 —
NÉCESSITARISME Croyance en la nécessité de tout. À ne pas confondre
avec le fatalisme.

NÉCESSITÉ Le contraire de la contingence : est nécessaire ce qui ne peut
pas ne pas être, autrement dit ce dont la négation est impossible. 
Par exemple la somme des angles d’un triangle, dans un espace euclidien,
fait {\it nécessairement} 180°, ce qui signifie qu’il est impossible, dans cet espace,
qu’elle soit autre. Nécessité absolue ? Non pas, puisqu’un autre espace est possible
ou concevable (celui des géométries non euclidiennes). Mais qui n’en est
pas moins nécessaire pour autant, dans cet espace-là. Toute nécessité est ainsi
hypothétique, comme disait Alain : elle est soumise à la condition d’un principe
ou d’un réel. Si rien n’existait, rien ne serait nécessaire. C’est en quoi tout
le nécessaire, à le considérer globalement ou en détail, reste contingent. La
nécessité de ma mort, par exemple, est soumise à la contingence de ma naissance,
comme la nécessité de ma naissance (dès lors que les conditions en sont
posées) reste soumise à la contingence de ma conception, ou de celle de mes
parents ou grands-parents, et comme la nécessité de l’univers, dès lors qu’il
existe, reste soumise à la contingence de sa propre existence (puisqu'il aurait pu
ne pas exister). Cela pose la question du temps, je veux dire de celui qui passe
ou qui est passé. Prenons par exemple le temps qu’il fait. Est-il nécessaire ou
contingent ? Cela dépend du point de vue chronologique adopté. Le temps
qu’il fait, ici et maintenant, est assurément nécessaire : ce qui est ne peut pas ne
pas être, soulignait à juste titre Aristote, tant qu’il est. Le temps qu’il fera dans
six mois est très vraisemblablement contingent : non seulement imprévisible en
fait, mais sans doute imprévisible en droit, en raison de la complexité et de
l’aléatoire des phénomènes météorologiques, à l’échelle macroscopique, voire
de ce qu’il y a d’indéterminé en eux à l’échelle microscopique ou corpusculaire.
Pourtant, dans six mois, le temps qu’il fera sera nécessaire, comme tout présent.
C’est dire que la nécessité n’est pas un prédéterminisme, qui voudrait que le
temps qu’il fait aujourd’hui ait été inscrit depuis toujours dans le passé de lunivers,
comme celui qu’il fera dans dix mille ans le serait dans son état actuel. La
vérité, c'est que tout présent est nécessaire (sa négation, au présent, est
impossible : s’il pleut, ici et maintenant, il est impossible, ici et maintenant,
qu'il ne pleuve pas), donc aussi toute vérité (puisqu'il n’est de vérité
qu'éternelle : que toujours présente), mais {\it eux seuls et au présent seulement}. Le
temps qu’il fait, ici et maintenant, est nécessaire ; il ne l’était pas il y a six mois.
Il était pourtant vrai, il y a six mois, qu’il ferait ce temps aujourd’hui ? Sans
doute. Mais ce n’est pas parce que c'était vrai il y a six mois qu’il fait ce temps
maintenant ; c’est parce qu'il fait ce temps maintenant que c'était vrai il y a six
%— 400 —
mois ou cent mille ans. L’éternité du vrai dépend de la nécessité du réel, non
l'inverse, de même que le tracé d’un fleuve, sur nos cartes réelles ou possibles,
dépend du cours effectif de ce fleuve et ne saurait évidemment le gouverner.
Ainsi tout est nécessaire, au présent, et c’est pourquoi tout l’est (puisque seul le
présent existe). On n’en conclura pas que ce qui est {\it n'aurait pas pu} ne pas être
(à l’irréel du passé : tant que cela n’était pas), ce qui n’est guère vraisemblable.
Mais seulement que ce qui est {\it ne peut pas} ne pas être (au présent : il est nécessairement
ce qu'il est, tant qu'il est). Dès lors que je suis vivant, il est impossible
que je ne le sois pas. Mais cela ne me rend ni immortel ni incréé.

Un être absolument nécessaire ? Ce serait Dieu, et c’est pourquoi rien, dans
le monde, ne saurait l'être.

NÉGLIGENCE Une faute qu’on aurait pu facilement éviter : il aurait suffi
d’un peu d’attention ou d’exigence. Petite faute ? Le plus
souvent, oui. Mais qui mène aux grands abandons, à force de s’habituer aux
petits. On omet d’abord de bien faire, puis on fait mal, ou le mal. C’est où la
négligence mène à la veulerie, comme la veulerie à la scélératesse.

NÉPOTISME Une forme de favoritisme, donc d’injustice, à destination de
ses parents : c’est privilégier les membres de sa famille, dans
des domaines où les liens de sang ou de cœur sont sans pertinence, par exemple
en matière d'emploi public ou d'administration. Quand Le Pen nous explique
qu’il préfère sa fille à sa voisine, et sa voisine à une étrangère, il ne ferait
qu'énoncer une platitude, s’il s'agissait d’un registre purement affectif ou privé.
Mais s’il prétend fonder là-dessus quelque politique que ce soit, ce n’est que la
justification du népotisme. Non plus une platitude, donc, mais une ignominie.

NERVEUX L'un des quatre tempéraments de la tradition hippocratique et
classique : teint pâle, vivacité des réactions, mobilité de l’intelligence
et du visage... Vous vous reconnaissez ? Inutile d’en parler à votre
médecin. Ce n’est pas une maladie, et votre médecin jugerait cette typologie
dépassée. Il a bien sûr raison. Cela ne prouve pas que celle qu’il utilise soit indépassable.

NÉVROSE/PSYCHOSE Deux mots pour dire des troubles du psychisme ou
de la vie mentale. Étymologiquement, le névrosé
%— 401 —
serait malade des nerfs ; le psychotique, de l’esprit. Cela ne dit rien de leur
pathologie respective, ni de son étiologie. La distinction, entre ces deux
concepts, reste d’ailleurs difficile à cerner, au point que certains psychiatres,
notamment américains, renoncent aujourd’hui à les utiliser. Elle garde pourtant
une espèce de puissance classificatrice ou d’usage régulateur : ce sont des
catégories qu’il est bon de connaître, sans y croire tout à fait. Les psychoses,
quoiqu'il y ait des exceptions et des états intermédiaires, sont ordinairement
plus graves : elles perturbent la totalité de la vie psychique, sont souvent accompagnées
de délire ou d’hallucinations, coupent du monde et de l’autre, enfin
relèvent de la psychiatrie davantage que de la psychothérapie. Les névroses sont
moins graves, du moins dans la plupart des cas, et de pronostic habituellement
plus favorable : les troubles y restent parcellaires ou localisés (ils n’atteignent
qu'un pan ou qu’un niveau de la vie psychique), socialement moins invalidants,
dépourvus de délire, susceptibles d’un traitement psychothérapeutique
ou analytique efficace, enfin « relativement superficiels, plastiques et réversibles »
(Henri Ey, {\it Manuel de psychiatrie}). Les principales psychoses sont la
paranoïa, la schizophrénie et la psychose maniaco-dépressive. Les principales
névroses sont la névrose d’angoisse, la névrose obsessionnelle, la névrose phobique
et l’hystérie. On dit parfois que le névrosé bâtit des châteaux en Espagne,
que le psychotique les habite, enfin que le psychanalyste encaisse le loyer.
Disons, plus sérieusement, que c’est sans doute le rapport au réel et aux autres
qui permet le mieux de distinguer ces deux entités nosologiques. Le psychotique
est prisonnier de son monde ou de sa folie, au point d’ignorer souvent
qu’il est malade. Le névrosé habite le monde commun : ses troubles, dont il a
habituellement conscience, ne l’empêchent ni d’agir ni d’entrer en relation avec
autrui {\it à peu près} normalement. Toutefois ce ne sont que des abstractions. Pour
ceux qui ne sont ni malades ni psychiatres, elles servent surtout à s’observer soi.
Il n’est pas inutile, même dans la santé, de savoir de quel côté on penche.

NIHILISME Le nihiliste, c’est celui qui ne croit à rien ({\it nihil}), même pas à
ce qui est. Le nihilisme est comme une religion négative :
Dieu est mort, emportant avec lui tout ce qu’il prétendait fonder, l’être et la
valeur, le vrai et le bien, le monde et l’homme. Il n’y a plus rien que le rien, en
tout cas rien qui vaille, rien qui mérite d’être aimé ou défendu : tout se vaut, et
ne vaut rien.

Le mot semble avoir été inventé par Jacobi, pour désigner l’incapacité de la
raison à saisir l’existence concrète, qui ne se donnerait qu’à l'intuition sensible
ou mystique. La raison, coupée de la croyance, est incapable de passer du
concept à l’être (comme le prouve la réfutation kantienne de la preuve ontologique) ;
%— 402 —
elle ne peut donc penser que des essences sans existence (sujet et
objet se dissolvant alors dans une pure représentation), et c’est en ce sens que
tout rationalisme, pour Jacobi, est un nihilisme. En français, et dans un usage
moins technique, le mot a été popularisé par Paul Bourget, qui le définissait
comme « une mortelle fatigue de vivre, une morne perception de la vanité de
tout effort ». Mais c’est bien sûr Nietzsche, dans le double prolongement de
Jacobi et de Bourget, qui lui donnera ses lettres de noblesse philosophiques. La
raison ne donne aucune raison de vivre : elle ne débouche que sur des abstractions
mortifères. Le rationalisme, pour Nietzsche aussi, est un nihilisme. Mais
ce n’est pas un courant de pensée parmi d’autres : c’est l’univers spirituel qui
nous attend. « Ce que je raconte, écrit Nietzsche, c’est l’histoire des deux prochains
siècles. Je décris ce qui viendra, ce qui ne peut manquer de venir : {\it l'avènement
du nihilisme} » ({\it La volonté de puissance}, III, 1, 25). Nous y sommes. Le
problème est d’en sortir.

« Que signifie le nihilisme ? Que les valeurs supérieures se déprécient,
répond Nietzsche. Les fins manquent. Il n’est pas de réponse à cette question :
“À quoi bon ?” » ({\it ibid.}, aph. 100). Les sciences, qui voulaient remplacer la religion,
ne donnent aucune raison de vivre : leur culte de la vérité n’est qu’un
culte de la mort. De |à cette « doctrine de la grande lassitude : “À quoi bon ?
Rien n’en vaut la peine !” » ({\it ibid.}, aph. 99). Nietzsche voulut y échapper par
l’esthétisme, autrement dit par le culte du beau mensonge, de l’erreur utile à la
vie, de l'illusion créatrice (« l’art au service de l'illusion, voilà notre culte », II,
5, 582). Cela ne fait qu’un néant de plus, qui règne désormais dans nos musées.
« Tout est faux, tout est permis », disait encore Nietzsche (II, 1, 108), et c’est
le nihilisme d’aujourd’hui. On ne peut en sortir qu’en revenant à la vérité de
l'être, comme dira Heidegger, et de la vie, comme le voulait Nietzsche, mais
qui n’est pas le mensonge et l'illusion : qui est puissance et fragilité, puissance
et résistance ({\it conatus}) — désir, en l’homme, et vérité. C’est choisir Spinoza
plutôt que Nietzsche, la lucidité plutôt que l'illusion, la fidélité plutôt que le
«renversement de toutes les valeurs», enfin l’humanité plutôt que le
surhomme. « Qu'est-ce que le nihilisme, demande Nietzsche, si ce n’est cette
lassitude-là ? Nous sommes fatigués de l’homme...» ({\it Généalogie...}, I, 12).
Parle pour toi. Le nihilisme est une philosophie de peine-à-jouir, de peine-à-aimer,
de peine-à-vouloir. C’est la philosophie de la fatigue, ou la fatigue
comme philosophie. Ils ont perdu la capacité d’aimer, comme dit Freud des
dépressifs, et en concluent que rien n’est aimable. C’est bien sûr se méprendre.
Ce n’est pas parce que le monde et la vie sont aimables qu’il faudrait les aimer ;
c’est parce que nous les aimons qu’ils sont, pour nous, aimables. Les valeurs ne
se déprécient que pour ceux qui ont besoin d’un Dieu pour aimer. Pour les
autres, les valeurs continuent de valoir, ou plutôt n’en valent que de façon plus
%— 403 —
urgente : parce que aucun Dieu ne les fonde ni ne les garantit, parce qu’elles ne
valent qu’à proportion de l'amour que nous leur portons, parce qu’elles ne
valent que pour nous, que par nous, qui avons besoin d’elles. Raison de plus
pour les servir. Le relativisme, loin d’être une forme de nihilisme, est son
remède : que toutes nos valeurs soient relatives (à nos désirs, à nos intérêts, à
notre histoire), c’est une raison forte pour ne pas renoncer aux {\it relations} qui les
font être. Ce n’est pas parce que la justice existe qu’il faut s’y soumettre (dogmatisme),
ni parce qu’elle n’existe pas qu’il faudrait y renoncer (nihilisme) ;
c'est parce qu’elle n’existe pas (sinon en nous, qui la pensons et voulons : relativisme)
qu’elle est à faire.

Contre le dogmatisme, quoi ? La lucidité, le relativisme, la tolérance.

Contre le nihilisme ? L’amour et le courage.

NIRVÂNA Le nom bouddhiste de l’absolu ou du salut : c’est le relatif même
(le {\it samsâra}), l'impermanence même ({\it anicca}), dès lors qu’on n’en
est plus séparé par le manque, le mental ou l'attente. L’ego s'éteint ({\it nirvâna}, en
sanskrit, signifie extinction) : il n’y a plus que tout. C’est l'équivalent à peu
près, mais dans une tout autre problématique, de l’ataraxie chez Épicure et de
la béatitude chez Spinoza : l'expérience, ici et maintenant, de l’éternité.

NOM Un mot, mais désignant ordinairement quelque chose d’à peu près
stable : une chose, un individu, une substance (d’où le mot, qu’on utilise
aussi, de {\it substantif}), un état, une abstraction. Les actions, les mouvements
ou les processus sont mieux désignés par des verbes. On objectera qu’« actions »,
« mouvements » ou « processus» sont des noms. Mais c’est qu'il s’agit ici
d’abstractions : le mot « action » n’en est pas une (c’est un nom, qui ne désigne
qu’une idée), mais c’en est une de le prononcer ou d’agir (qui sont des verbes).

Francis Wolff, dans un livre magistral ({\it Dire le monde}, PUF, 1997), a imaginé
un langage-monde qui ne serait fait que de noms : monde d’essences séparées et
immuables, d'individus sans changements, de substances sans accidents, de
choses sans événements, d’êtres sans devenir. Ce serait le monde de Parménide,
s’il était possible, ou le monde intelligible de Platon, s’il était réel. Mais ce n’est
rien, ou presque rien : monde de l’abstraction, ou l’abstraction d’un monde.

NOMINALISME Toute doctrine qui considère que les idées générales n’ont
d’autre réalité que les mots qui servent à les désigner.
C’est affirmer que seuls les individus existent, et qu’il n’est de généralités (ou
%— 404 —
d’universaux, comme on disait au Moyen Âge) que par et dans le langage. Le
nominalisme s'oppose ainsi au {\it réalisme} (au sens du réalisme des idées, tel qu’on
le voit chez Platon ou Guillaume de Champeaux), et ce dès l’Antiquité : « Je
vois bien le cheval, disait Antisthène, mais non la chevalité ». Il se distingue
aussi du {\it conceptualisme}, pour lequel les idées générales n'existent que dans
notre esprit, mais sans se réduire purement et simplement aux signes qui servent
à les désigner. Cette dernière distinction est pourtant moins fondamentale :
nominalisme et conceptualisme ont en commun d’affirmer l’exis-
tence exclusive des individus, du moins en dehors de lesprit humain, ce qui
explique que la frontière, entre ces deux courants, soit souvent floue (les spécialistes
discutent encore, par exemple, pour savoir si Guillaume d’Occam était
nominaliste ou conceptualiste), alors qu’ils s'opposent frontalement, lun et
l’autre, au réalisme des idées.

Le nominalisme sera la doctrine de Roscelin, au {\footnotesize XI$^\text{e}$} siècle, ou de
Guillaume d’Occam, au {\footnotesize XIV$^\text{e}$}, mais aussi de Hobbes, Hume et Condillac,
comme en général, et jusqu’à aujourd’hui, de la plupart des empiristes. Si les
idées n'existent pas en elles-mêmes, il en découle qu'on ne peut rien
connaître que par l'expérience, et que la logique elle-même n’est qu’une
langue bien faite (elle n’est agencement d’idées que parce qu’elle est, d’abord,
agencement de signes). Le matérialisme, dans une autre problématique, s’y
reconnaît également. Si tout est matière, les idées ne sauraient exister en elles-mêmes :
elles n’existent que dans un cerveau et par la médiation des signes
qui permettent de les désigner. Le matérialisme, de ce point de vue, est un
nominalisme radical et moniste : il n’existe que des individus, et ils sont tous
matériels.

NORMAL Qui est conforme à la norme, mais à une norme purement factuelle,
en général la moyenne («une taille normale ») ou la
santé (le normal opposé au pathologique). C’est ériger le fait en valeur, la statistique
en jugement, la moyenne en idéal. C’est ce qui rend la notion désagréable,
sans permettre pour autant de toujours s’en passer. « S’il existe des
normes biologiques, écrit Georges Canguilhem, c’est parce que la vie, étant
non pas seulement soumission au milieu mais institution de son milieu
propre, pose par là même des valeurs, non seulement dans le milieu mais
aussi dans l'organisme même. C’est ce que nous appelons la normativité
biologique » ({\it Le normal et le pathologique}, PUF, Conclusion). C'est ce qui
explique que la santé, qui est un fait, ou un rapport entre des faits, soit aussi
un idéal.

%— 405 —
NORMATIF Qui établit une norme, en relève ou la suppose ; qui énonce
un jugement de valeur ou en dépend. Le point de vue normatif
s'oppose traditionnellement au point de vue {\it descriptif}, qui se contente d'établir
des faits, ou {\it explicatif}, qui donne les causes ou les raisons. La différence, entre
ces trois points de vue, peut pourtant être floue. Quand je dis de quelqu'un
{\it « C'est un imbécile »}, il est clair que j’émets un jugement de valeur. Mais je peux
aussi constater un fait (si l’imbécillité en question fait partie du réel) et expliquer
un certain type de comportement. Le politiquement correct voudrait nous
interdire ce type de jugement négatif (« On ne dit plus {\it aveugle} mais {\it mal-voyant},
remarquait un humoriste, bientôt on ne dira plus {\it con} mais {\it mal-comprenant} »).
Mais c'est au nom d’un point de vue qui reste lui-même normatif, tout en se
prétendant abusivement descriptif : ils voudraient, absurdement, que l'égalité
des hommes en droits et en dignité entraîne leur égalité en fait et en valeur.
Aussi s’interdisent-ils de juger, sinon pour condamner ceux qui s’y risquent.
C’est qu'ils ont jugé à l’avance et une fois pour toutes. Le politiquement correct
n'est qu'un préjugé normatif.

NORME  {\it Norma}, en latin, c’est l’{\it équerre} : une norme, commente Canguilhem,
« c’est ce qui sert à faire droit, à dresser, à redresser » ({\it Le
normal et le pathologique}, p. 177). Elle dit ce qui doit être, ou permet de juger
ce qui est.

Le mot peut valoir comme synonyme de {\it règle}, d’{\it idéal}, de {\it valeur}... Si on
veut lui donner un sens plus précis, c’est sans doute sur sa généralité qu’il faut
insister : la {\it norme} est le genre commun, dont {\it règles}, {\it idéaux} et {\it valeurs} sont différentes
espèces. C’est pourquoi le mot a quelque chose de flou, qui le rend à la
fois commode et embarrassant. Il perd en compréhension, nécessairement, ce
qu’il gagne en extension.

NOSTALGIE C’est le manque du passé, en tant qu’il fut. Se distingue par là
du regret (le manque de ce qui ne fut pas) ; s'oppose à la gratitude
(le souvenir reconnaissant de ce qui a eu lieu : la joie présente de ce qui
fut) et à l'espérance (le manque de l’avenir : de ce qui sera peut-être).

J'ai tendance à penser que la nostalgie, de ces quatre sentiments, est le premier,
et que toute espérance, spécialement, n’est que l’expression — ou le
remède imaginaire — d’une nostalgie préalable. Il faudrait relire Platon et saint
Augustin, de ce point de vue, à la lumière de Freud. Et relire aussi Épicure : on
y verrait que la gratitude, contre la nostalgie, est le seul remède vrai.

%— 406 —
NOTION Une idée abstraite ou générale, le plus souvent considérée comme
déjà donnée dans la langue ou dans l’esprit. C’est ce qui distingue
la notion (qui n’a besoin, étymologiquement, que d’être connue ou reconnue)
du concept (qu’il faut d’abord concevoir). Un concept est le résultat d’un travail
de pensée ; la notion serait plutôt sa condition. Un concept est une œuvre
avant d’être un outil. La notion serait plutôt un matériau ou un point de
départ. Un concept relève d’une science ou d’une philosophie particulière ; une
notion, de la pensée commune.

Chez Kant, « le concept pur, en tant qu’il a uniquement son origine dans
l’entendement (et non dans une image pure de la sensibilité), s'appelle notion ».
Mais cet usage ne s’est pas imposé. C’est que le concept était trop étroit et trop
dépendant d’une doctrine particulière pour s'imposer comme notion.

On appelle {\it notions communes} les idées ou principes qu’on trouve chez
«tous les hommes » (Spinoza), sans lesquelles nous ne pourrions ni raisonner
ni nous comprendre. Les empiristes veulent qu’elles résultent de l’expérience ;
les rationalistes, qu’elles soient innées et la rendent possible. C’est en quoi
l’empirisme m’a toujours paru plus rationaliste (au sens large) que le rationalisme
(au sens étroit) : il ne renonce pas à expliquer les notions dont il se sert.

NOUMÈNE Même s’il ne l'a pas inventé (on trouve {\it noumena}, chez Platon,
pour désigner les Idées), le mot, aujourd’hui, fait presque toujours
référence à la philosophie de Kant. Qu'est-ce qu’un noumène ? Un objet
qui ne serait objet que pour l'esprit ({\it noûs}, en grec), un objet qui n'apparaît pas
(ce n’est pas un phénomène), dont nous n’avons aucune expérience, aucune
intuition (puisque nous n’avons d’intuitions que sensibles), que nous ne pouvons
pour cela {\it connaître}, mais tout au plus {\it penser}. La chose en soi ? Une façon,
plutôt, de l’envisager : en tant qu’elle serait un être purement intelligible (ce
que le concept de chose en soi n’impose pas), ou l'objet, si nous en étions
capables, d’une intuition intellectuelle. Concept {\it problématique}, reconnaît
Kant, puisqu'il excède par nature notre connaissance. Mais qui prétend toutefois
résoudre ce problème (quoique de façon non dogmatique) dans un sens
idéaliste : c’est où Kant rejoint Platon.
% 407

%  O
OBEISSANCE La soumission à un pouvoir légitime, ou qu’on juge tel. Il
n’en est pas moins nécessaire, parfois, de désobéir. La légitimité
n’est ni l’infaillibilité ni la justice.

OBJECTIF Comme substantif, c’est un but ou un instrument d'optique :
l’objet que l’on vise ou celui qui vise l’objet.

Comme adjectif, le mot peut qualifier tout ce qui doit à l’objet davantage
qu'au sujet : tout ce qui existe indépendamment de quelque sujet que ce soit
ou, si un sujet intervient (par exemple dans un récit ou un jugement), tout ce
qui fait preuve d’objectivité.

Tout cela irait sans dire. Mais il faut signaler surtout, c’est ce qui justifie cet
article, un usage très particulier, qu’on trouve chez les scolastiques et les philosophes
du {\footnotesize XVII$^\text{e}$} siècle, qui peut aujourd’hui prêter à contresens. Est {\it objectif}, en
ce sens, tout ce qui est un objet de la pensée ou de l’entendement, que cet objet
ait ou non une réalité extérieure qui lui corresponde (qu’il existe ou non objectivement,
au sens moderne du terme). L’être objectif d’une idée s’oppose alors
à son être formel : son être formel, c’est ce qu’elle est en soi ; son être objectif,
ce qu’elle est en nous ou pour nous (en tant qu’elle est un objet de notre
pensée). C’est ce qui faisait dire à Descartes, par exemple, que « l’idée du soleil
est le soleil même existant dans l’entendement, non pas à la vérité formellement,
comme il est au ciel, mais objectivement, c’est-à-dire en la manière que
les objets ont coutume d’exister dans l’entendement : laquelle façon d’être est
de vrai bien plus imparfaite que celle par laquelle les choses existent hors de
l’entendement ; mais pourtant ce n’est pas un pur rien » ({\it Réponses aux premières
objections}, AT, p. 82). Ou à Spinoza que « l’idée, en tant qu’elle a une essence
%— 408 —
formelle, peut être l’objet à son tour d’une autre essence objective », qui est
l'idée de l’idée ({\it T.R.E.}, 27).

Enfin, chez Hegel, l'esprit objectif est celui qui dépasse la conscience individuelle et
s’incarne dans des institutions juridiques, sociales ou politiques (le
droit, les mœurs, l’État), où il devient comme un objet pour lui-même. Ce
moment, encore prisonnier de ses limites territoriales, sera lui-même dépassé
dans l’esprit absolu (Part, la religion, la philosophie).

OBJECTIVITÉ C’est voir ou connaître les choses comme elles sont ou comme
elles apparaissent, indépendamment, si c’est possible, de
notre subjectivité, et en tout cas de ce que notre subjectivité peut avoir de particulier
ou de partial. En pratique, c’est voir les choses comme peut les voir tout
observateur de bonne foi, quand il est sans passion et sans parti pris. Que
l’objectivité ne soit jamais absolue, ce qui est bien clair (puisqu'il n’y a de
connaissance que pour un sujet), n'autorise pas à dire qu’elle soit impossible ;
car alors les sciences et la justice le seraient aussi.

OBJET Étymologiquement, c’est ce qui est placé devant. Devant quoi ?
Devant un sujet. C’est en quoi les deux notions sont indissociables.
Là où il n’y a pas de sujet, il peut bien y avoir des êtres, des événements ou des
choses, mais il n’y a pas d'objet. C’est que tout objet est construit : soit par les
conditions (à la fois subjectives et historiques, sensibles et intellectuelles) de sa
perception, soit par celles (aussi bien expérimentales que théoriques) de sa
connaissance scientifique. Qu’est-ce qu’un objet ? C’est le corrélat objectif, ou
supposé tel, d’un sujet percevant ou connaissant. C’est pourquoi « l'objet n’est
pas l'être », comme Alquié aimait à le répéter, et c’est pourquoi on ne peut
connaître, par définition, que des objets. À la gloire du pyrrhonisme.

OBSCUR Ce qui n’est ni clair ni lumineux. À ne pas confondre avec la profondeur.
Une idée qu’on ne comprend pas, comment savoir si elle
est profonde ? Mais pas non plus avec la fausseté. Pourquoi la vérité serait-elle
toujours claire ?

OBSERVATION C’est une expérience, mais volontaire et attentive. Par
exemple on fait l’expérience du deuil, et l’on observe, si
%— 409 —
l’on veut, si l’on peut, ce qui se passe en soi. Ou bien on fait l’expérience d’une
nuit étoilée, et l’on observe, si l’on veut, les étoiles.

Claude Bernard distinguait légitimement l’observation simplement empirique,
qui est faite « sans idée préconçue », de l'observation scientifique, qui
suppose une hypothèse préalable, qu’il s’agit de vérifier. Mais c’est surtout de
l’expérimentation que l'observation, même scientifique, se distingue : l’expérimentation
est une « observation provoquée », comme disait encore Claude
Bernard ; l'observation, à l'inverse, tient lieu d’expérimentation lorsqu'on ne
peut ni provoquer ni modifier le phénomène observé. C’est souvent le cas en
astronomie, mais aussi dans les sciences humaines : on ne provoque pas une
éclipse, mais pas davantage — en tout cas dans un but scientifique — une révolution,
une névrose ou un suicide. L'observation ne sert pas seulement pour le
plus lointain, mais aussi, souvent, pour le plus proche. Le difficile est qu’elle
risque alors de modifier involontairement cela même qu’elle observe : c’est ce
qui rend l’ethnographie et l’introspection si difficiles et si incertaines. Les relations
d’incertitude, comme dit Heisenberg, ne valent pas seulement en mécanique quantique.

OBSTACLE ÉPISTÉMOLOGIQUE  « Quand on cherche les conditions psychologiques
des progrès de la science,
écrit Bachelard, on arrive bientôt à cette conviction que c’est en termes d’obstacles
qu’il faut poser le problème de la connaissance scientifique » ({\it La formation
de l'esprit scientifique}, I). La connaissance n’est pas une table rase : on n’y
part jamais de zéro ; on ne connaît que « contre une connaissance antérieure »,
qu'elle soit purement empirique ou déjà scientifique. « Quand il se présente à
la culture scientifique, l'esprit n’est jamais jeune. Il est même très vieux, car il a
l’âge de ses préjugés. Accéder à la science c’est, spirituellement, rajeunir, c’est
accepter une mutation brusque qui doit contredire un passé » ({\it ibid.}). Ce passé,
mais actuel et actif, c’est l’obstacle épistémologique. C’est une opinion, une
représentation ou une habitude intellectuelle, héritée du passé, qui entrave la
connaissance scientifique ou s'oppose, de l’intérieur, à son développement. Par
exemple l’obstacle substantialiste, qui veut tout expliquer par la substance, ou
l'obstacle animiste, qui projette sur la nature ce qu’on croit savoir de la vie.
Autant d’idées faussement claires, qu’il faut comprendre pour s’en libérer.
« Déceler les obstacles épistémologiques, écrit encore Bachelard, c’est contribuer
à fonder les rudiments d’une psychanalyse de la raison » ({\it ibid.}). C’est où
les deux versants de la pensée bachelardienne — philosophie des sciences, philosophie
de l’imaginaire — se rejoignent.

%— 410 —
OCCULTISME Ce n’est pas croire en des vérités cachées (elles le sont presque
toutes : « les yeux, disait Lucrèce, ne peuvent connaître la
nature des choses »). C’est croire en des vérités qui se cachent, ou que l’on
cache, parce qu’elles seraient d’une nature absolument différente des autres :
surnaturelles, surréelles, suprasensibles, d’outre-tombe ou d’outre-monde..
Elles ne seraient accessibles qu’à des {\it sciences occultes}, ce qui fait un bien étrange
oxymore. Faire tourner des tables pour faire parler des esprits, croire aux fantômes
ou aux devins, pratiquer lalchimie ou la magie... Quoi de
«scientifique » là-dedans ? Ce n’est qu’une superstition de l’invisible.

ŒUVRE Le produit d’une activité ou d’un travail. Le mot suppose presque
toujours une normativité au moins implicite : œuvre dit plus que
production, ouvrage ou résultat. C’est le fruit d’un travail, mais considéré dans
sa valeur intrinsèque, et qui vaut plus, presque toujours, que le travail lui-même.

C’est pourquoi on parle souvent, notamment à propos des œuvres d’art, de
création : quelque chose de neuf est apparu, qui semble excéder les moyens utilisés.
Disons que c’est une création humaine, comme la Création (univers)
serait l’œuvre de Dieu.

OLIGARCHIE C'est le pouvoir d’une petite minorité ({\it oligos}, peu nombreux),
qui prétendent souvent être les meilleurs et qui ne sont en
vérité que les plus puissants — c’est-à-dire, presque toujours, les plus riches. Ils
voudraient constituer une aristocratie. Ce n’est ordinairement qu’une ploutocratie déguisée.

ONTIQUE Qui concerne les étants ({\it ta onta}) plutôt que l’être. Se distingue
en cela d’{\it ontologique} (qui concerne l’être de l’étant plutôt que
l’étant lui-même). La notion, on l’a compris, relève du vocabulaire heideggérien.

ONTOLOGIE Le discours sur « l’être en tant qu'être », comme disait Aristote,
ou sur l’être de ce qui est (les étants en général, {\it ta onta},
et non tel étant en particulier). C’est une partie, sauf pour les heideggériens, de
la métaphysique. Mais sur l’être en tant qu'être, que dire, sinon qu'il est ? Les
sciences nous en apprennent davantage. L’être pur n’est qu’un rêve de philosophe.
Mieux vaut l’impureté du réel.

%— 411 —
ONTOLOGIQUE (PREUVE -) L’une des trois preuves traditionnelles de
l'existence de Dieu : celle qui veut conclure
son existence de sa seule essence ou définition. Qu'est-ce, en effet, que Dieu ?
Un être suprême (« un être, disait saint Anselme, tel que rien de plus grand ne
puisse être pensé »), un être souverainement parfait (Descartes), un être absolument
infini (Spinoza, Hegel). Or, s’il n'existait pas, il ne serait ni le plus grand,
ni parfait, ni réellement infini. Il existe donc par définition : penser Dieu (le
concevoir comme suprême, parfait, infini.....), c’est le penser comme existant.
Le concept de Dieu, dira Hegel, « inclut en lui l’être » : Dieu est le seul être qui
existe par essence.

Belle preuve, par la simplicité, mais trop belle ou trop abstraite pour être
tout à fait convaincante : c’est vouloir passer de la pensée à l’être, ce qui n’est
possible que par expérience. Mais si nous avions une expérience de Dieu, nous
n’aurions plus besoin de le prouver. Et si nous n’en avons pas, toute preuve de
son existence est définitivement impossible. L’être n’est pas un prédicat,
explique Kant, qu’on pourrait ajouter à un concept ou l’en déduire. C’est pourquoi
il ne suffit pas de définir Dieu pour le prouver, pas plus qu’il ne suffit de
définir la richesse pour s'enrichir. Il n’y a pourtant rien de plus dans cent euros
réels, dirait aujourd’hui Kant, que dans cent euros possibles : le concept, dans
les deux cas, est le même. Mais on est plus riche avec cent euros réels qu'avec
leur simple concept ou définition. Même chose s'agissant de Dieu: son
concept reste le même, qu’il existe ou pas, et ne saurait donc prouver qu’il
existe.

ONTO-THÉOLOGIE Le discours, non sur l’être, mais sur l’étant, et spécialement
sur l’étant suprême : Dieu. Ce serait la forme
métaphysique de l’oubli de l'être, ou plutôt la métaphysique même, en tant
qu’elle n’existerait, selon Heidegger, que par cet oubli.

OPINION Toute pensée qui n’est pas un savoir. S’oppose pour cela, spécialement,
aux sciences. C’est ce qui faisait écrire à Bachelard, en un
texte fameux : « L'opinion pense mal ; elle ne pense pas : elle traduit des besoins
en connaissances » ({\it La formation de l'esprit scientifique}, X). Toutefois c’est forcer
trop l’opposition. D'abord parce que les opinions jouent un rôle aussi dans les
sciences en train de se faire, et qui n’est pas seulement celui d’obstacle épistémologique
(mais aussi d’idée régulatrice, d’hypothèse vague, d’orientation provisoire
et tâtonnante.....). Ensuite parce qu’il y a des opinions droites, comme
Platon le soulignait déjà, lesquelles, pour insuffisantes qu’elles demeurent, sont
%— 412 —
légitimement tenues pour vraies. Enfin, et surtout, parce qu’une opinion
pensée, réfléchie, théorisée, n’en reste pas moins {\it opinion} pour autant : la philosophie
en est pleine. Par exemple quand Descartes affirme que la volonté est
libre ou quand Spinoza assure qu’elle ne l’est pas : ce sont des opinions, ni plus
ni moins, et pourtant des pièces essentielles, et hautement argumentées, de
leurs systèmes. Et même chose, bien sûr, des « preuves » de l’existence de Dieu,
de la démonstration de l’immortalité de l’âme, ou de sa mortalité, de la
croyance en l’infinité ou en la finitude de l’univers, du statut de la vérité, du
fondement de la morale ou de la définition philosophique de l'opinion. À la
gloire du pyrrhonisme. Il n’y a pas de savoir philosophique (il n’y a de savoir
que sur l’{\it histoire} de la philosophie) ; la philosophie n’est pas une science, et
c’est en quoi toute philosophie, même la plus sophistiquée, est d'opinion.

Qu'est-ce qu’une opinion? Kant en donnait une définition presque
parfaite : « L'opinion est une croyance qui a conscience d’être insuffisante aussi
bien subjectivement qu’objectivement » ({\it C. R. Pure}, « De l'opinion, de la
science et de la foi»; voir aussi {\it Logique}, Introd., IX). Pourquoi {\it presque}
parfaite ? Parce que c’est définir l’opinion lucide, celle qui se sait opinion, non
l'opinion dogmatique, si fréquente, celle qui se prend pour le savoir qu’elle
n'est pas, bien plus que pour la foi qu’elle refuse d’être. Que Spinoza ou Descartes
aient cru à leurs démonstrations, j’en suis convaincu ; mais cela ne nous
dit pas lequel des deux, lorsqu'ils s'opposent (or ils s'opposent presque toujours),
a raison, ni ne nous autorise à prêter à leurs philosophies, comme ils le
voulaient, la certitude — d’ailleurs elle-même relative — des mathématiques. De
là cette définition rectifiée que je propose : l’opinion est le fait de tenir quelque
chose pour vrai, mais en vertu d’un jugement objectivement insuffisant, et
qu’on ait ou pas conscience de cette insuffisance. C’est une croyance incertaine,
c’est-à-dire une croyance, mais désignée (fût-ce par un autre) comme telle : une
croyance dont on refuse de se satisfaire.

OPTIMISME Un optimiste rencontre un pessimiste. « Tout va mal, s’exclame
ce dernier. Ça ne pourrait pas être pire ! » Et l’optimiste
de lui répondre : « Mais si, mais si... » Quel optimisme qui ne donne raison,
pour finir, au pessimisme ?

{\it Optimus}, en latin, est le superlatif de {\it bonus}. Le mot signifie « le meilleur »,
et cette étymologie pourrait presque faire une définition suffisante. Être optimiste,
au sens philosophique du terme, c’est penser, avec Leibniz, que tout va
pour le mieux dans le meilleur des mondes possibles ({\it Théodicée}, I; voir aussi
II, 413 sq.). Doctrine irréfutable (puisque ce monde est le seul que lon
connaisse) et pourtant incroyable (tant le mal y est évident). Voltaire, dans
%— 413 —
{\it Candide}, a dit là-dessus à peu près ce qu’il fallait. On n’en reste pas moins surpris
qu'un génie comme Leibniz, peut-être le plus grand qui fut jamais, ait pu
tomber dans cette niaiserie. C’est qu’il prenait la religion au sérieux, et que la
religion, inévitablement, est optimiste. Si Dieu existe, le meilleur existe : toute
religion est un optimisme métaphysique.

Au sens courant, le mot optimisme désigne moins une doctrine qu’une attitude
ou un penchant : être optimiste, c’est prendre les choses du bon côté, ou
penser, lorsqu’elles sont décidément douloureuses, qu’elles vont s'arranger. Et
après tout, pourquoi pas ? Toutefois la mort et la vieillesse font des raisons
fortes de n’y point croire tout à fait.

« Le pessimisme est d’humeur, disait Alain, l’optimisme est de volonté :
tout homme qui se laisse aller est triste. » Je ne sais. Qu'il faille remonter la
pente plutôt que la descendre, viser la joie plutôt que la tristesse, enfin se gouverner
plutôt que s’abandonner, j’en suis bien sûr d’accord. Mais à condition
de ne pas sacrifier pour cela une once de lucidité. La vérité, pour un philosophe,
vaut plus que le bonheur.

J'aime mieux la formule de Gramsci : {\it « Pessimisme de l'intelligence, optimisme
de la volonté. »} Voir les choses comme elles sont, puis se donner les
moyens de les transformer. Envisager le pire, puis agir pour l’éviter. On n’en
mourra pas moins ? On n’en vieillira pas moins ? Sans doute. Mais on aura
vécu davantage.

ORDRE Un désordre facile à mémoriser, à reconnaître ou à utiliser. Ainsi
l’ordre des lettres, dans un mot, ou l’ordre alphabétique, dans un
dictionnaire. « L'ordre, écrit Marcel Conche à propos de Lucrèce, n’est qu’un
cas particulier du désordre » : c’est un désordre commode, efficace ou significatif.
C’est dire qu’il n’y a d’ordre — donc aussi de désordre — que pour nous.
C’est ce qu’indique fort clairement Spinoza, dans l’Appendice du livre I de
l'{\it Éthique} : « Quand les chose sont disposées de façon que, nous les représentant
par les sens, nous puissions facilement les imaginer et, par suite, nous les rappeler
facilement, nous disons qu’elles sont bien ordonnées; dans le cas
contraire, qu’elles sont mal ordonnées ou confuses. Et comme nous trouvons
plus d’agrément qu’aux autres aux choses que nous pouvons imaginer avec facilité,
les hommes préfèrent l’ordre à la confusion ; comme si, sauf par rapport à
notre imagination, l’ordre était quelque chose dans la nature. » Mais ce n’est
qu'une illusion : l’ordre n’est qu’un désordre qui nous arrange ; le désordre,
qu'un ordre qui nous déçoit. C’est pourquoi le second principe de la thermodynamique,
par la notion d’{\it entropie} (voir ce mot), est essentiellement
décevant : parce que le désordre est toujours plus probable que l’ordre, et ne
%— 414 —
peut donc, dans un système isolé, que s’accroître. Cela ne retire rien à la thermodynamique,
mais beaucoup à nos espérances : elles ne sont ultimement crédibles
que si univers {\it n'est pas} un système isolé (que s’il existe autre chose que
l'univers, qui peut agir sur lui : par exemple un Dieu et une providence).

La notion d’ordre, sans parler de son acception impérative (« donner un
ordre »), a aussi un autre sens, par exemple dans les sciences naturelles ou chez
Pascal : elle peut désigner un sous-ensemble ou un domaine, ayant ses caractéristiques
ou sa logique propres (lordre des primates, l’ordre du cœur.....).
L'essentiel, philosophiquement, est alors de ne pas confondre des ordres
différents : je m’en explique ci-après.

ORDRES (DISTINCTION DES —) Cette notion, en philosophie, doit surtout
à Pascal. On sait qu’il distingue
trois ordres différents : l’ordre des corps ou de la chair, l’ordre de l’esprit ou de
la raison, enfin l’ordre du cœur ou de la charité. Chacun de ces ordres a sa
cohérence propre, ses valeurs propres, son efficace propre, mais qui ne peuvent
rien dans un autre ordre. C’est ce qu’indique la fin du décisif fragment 308-793 :

\vspace{0.5cm}

{\footnotesize 
« Tous les corps, le firmament, les étoiles, la terre et ses royaumes ne valent pas le
moindre des esprits. Car il connaît tout cela, et soi ; et les corps, rien.

Tous les corps ensemble et tous les esprit ensemble et toutes leurs productions ne
valent pas le moindre mouvement de charité. Cela est d’un ordre infiniment plus élevé.

De tous les corps ensemble on ne saurait en faire réussir une petite pensée. Cela est
impossible et d’un autre ordre. De tous les corps et esprits on n’en saurait tirer un mouvement
de vraie charité ; cela est impossible, et d’un autre ordre, surnaturel. »
}

\vspace{0.5cm}

Soit par exemple le théorème de Pythagore ou un fait historique quelconque,
bien avéré. Combien d’armées faudrait-il pour les rendre faux ? Un
nombre infini n’y suffirait pas : toutes les armées de l’univers ne peuvent rien
contre une vérité, ni l’univers lui-même.

Et combien de théorèmes faudrait-il pour susciter un vrai mouvement de
charité ? L’infini n’y suffirait pas : tous les théorèmes de l’univers, même joints
à toutes les armées du monde, ne sauraient suppléer aux défaillances du cœur
ou à l’absence de la grâce.

C’est en pensant à Pascal que j’ai repris cette idée de distinction des ordres,
mais en l’appliquant à une classification différente. S'agissant de la société, j'ai
pris l’habitude de distinguer quatre ordres différents : {\it l'ordre techno-scientifique},
structuré intérieurement par l'opposition du possible et de l'impossible (ou,
d’un point de vue scientifique, du possiblement vrai et du certainement faux),
%— 415 —
mais incapable de se limiter lui-même ; limité donc, de l’extérieur, par un
second ordre : {\it l'ordre juridico-politique}, lequel est structuré intérieurement par
l'opposition du légal et de l’illégal, mais tout aussi incapable que le précédent
de se limiter soi ; limité donc à son tour, de l’extérieur, par un troisième ordre :
{\it l'ordre de la morale}, lequel est structuré intérieurement par l'opposition du
devoir et de l’interdit, et complété, plutôt que limité, ou ouvert par en haut,
vers un quatrième ordre : {\it l'ordre éthique, ou ordre de l'amour}. Mon idée est que
chacun de ces ordres a sa cohérence propre, ses contraintes ou ses exigences
propres, enfin son autonomie : on ne vote pas sur le vrai et le faux, ni sur le
bien et le mal; mais la morale ou les sciences ne sauraient pas davantage
régenter la politique et le droit. C’est pourquoi chacun de ces ordres est nécessaire,
sans pouvoir pour autant fonctionner seul : il a besoin de l’ordre immédiatement
inférieur pour exister (c’est ce que j'appelle {\it l'enchaînement descendant
des primats}, sans lequel rien n’est possible), mais ne peut être limité et jugé
que par un ordre supérieur (ce que j'appelle {\it la hiérarchie ascendante des primautés},
sans laquelle rien n’a de sens). Quand on oublie cette distinction des
ordres, ou quand on prétend qu’un seul de ces ordres peut suffire, on est voué
au ridicule ou à la tyrannie, sous deux formes opposées : l’angélisme, si c’est au
bénéfice d’un ordre supérieur, ou la barbarie, si c’est au bénéfice d’un ordre
inférieur (sur tout cela, voir {\it Valeur et vérité}, p. 207 à 226).

À cette quadripartition, qui dessine une espèce de topique, on pourrait
ajouter un ordre zéro, qui serait celui du réel ou de la nature, et un ordre cinquième
et ultime, qui serait, pour ceux qui y croient, l’ordre surnaturel ou
divin. Dans mon esprit, l’ordre zéro contient tous les autres : c’est moins un
ordre de plus que le lieu et la condition de tous. C’est ce qui m’empêche de
considérer un éventuel cinquième ordre autrement que comme le prolongement
fantasmatique des quatre autres (le Dieu tout-puissant et omniscient,
celui qui commande et juge, enfin le Dieu d’amour). Mais, même pour un
croyant, il me semble que l'esprit de la laïcité interdit de soumettre purement
et simplement l’un quelconque de ces quatre ordres au cinquième. Ce n’est pas
parce que Dieu m’ordonne quelque chose que c’est moralement bon, expliquait
déjà Kant, c’est parce que c’est moralement bon que je peux envisager
que cela vienne de Dieu : « Même le Saint de l'Évangile doit être d’abord comparé
avec notre idéal de perfection morale avant qu’on le reconnaisse pour tel »
({\it Fondements...}, II ; voir aussi {\it C. R. Pratique}, Dialectique, II, 9, et {\it La religion
dans les limites de la simple raison}, Préface). Même chose, bien sûr, pour les
ordres 2 et 4 : celui qui voudrait soumettre le droit ou l’amour à la supposée
volonté de Dieu devrait renoncer à la souveraineté du peuple autant qu’à sa
propre autonomie d’être humain : c’est ce qu’on appelle l’intégrisme. Contre
%— 416 —
quoi la distinction des ordres, au sens où je prends l’expression, n’est pas autre
chose qu’une tentative pour penser la laïcité jusqu’au bout.

ORGUEIL Adolescent, je m'étais soumis, à la demande d’une amie, au
fameux « Questionnaire de Proust ». De mes réponses d’alors,
j'ai tout oublié, sauf ceci, qui m’avait paru finement paradoxal :

« — Votre principal défaut ?

— L’orgueil.

— Votre principale qualité ?

— L’orgueil. »

J'en suis bien revenu. Non que j'aie cessé d’être orgueilleux (quoique je le
sois sans doute moins) ; mais j’ai cessé d’y voir une vertu.

Tout orgueil est illusoire : c’est se prêter plus de mérites qu’on n’en a, ou se
flatter, bien sottement, de ceux qu’on peut avoir. Quoi de plus ridicule que
d’être fier de sa taille, de sa beauté, de sa santé ? Et pourquoi le serait-on davantage
de son intelligence ou de sa force ? As-tu choisi d’être qui tu es ? Es-tu
maître de le rester ? Un petit caillot mal placé, te voilà stupide et grabataire. II
n’y aura pas lieu alors d’avoir honte, ni donc aujourd’hui d’être fier.

« L’orgueil, écrivait Spinoza, consiste à avoir de soi, par amour, une
meilleure opinion qu’il n’est juste » ({\it Éthique}, IV, déf. 28 des affects). Tout
orgueil, par définition, est donc injuste : sans justice vis-à-vis des autres, sans
justesse vis-à-vis de soi. Ce n’est qu’un piège de l’amour-propre.

ORIGINE C’est moins le commencement que ce qui le permet, le précède
ou le prépare. Par exemple le {\it big bang}, pour l’univers, fait un
commencement plausible. Mais assurément pas une origine : car pourquoi le
{\it big bang} ?

L'origine serait donc la cause ? Pas tout à fait ou pas seulement, puisque les
causes sont ordinairement multiples, dont chacune, à son tour, doit avoir sa
propre cause. Une cause explique un fait ou un événement ; l’origine rendrait
plutôt raison d’un être ou d’un devenir. Si c'était une cause, ce serait plutôt la
cause première ou ultime : celle qui rend raison de toutes les autres, ou leur
série complète. Par quoi l’origine nécessairement nous échappe : c’est la ligne
d’horizon de la causalité.

OUBLI C’est le contraire non de la mémoire, que l'oubli suppose, mais du
souvenir : il y a oubli lorsque je ne me souviens plus de quelque
%— 417 —
chose que j'ai eu, au moins un temps, en mémoire. Loin que cela soit toujours
fautif ou pathologique, il faut y voir bien souvent une forme de santé, voire de
générosité. C’est ce qu’a vu Nietzsche, dans la deuxième de ses {\it Considérations
intempestives} :

\vspace{0.5cm}

{\footnotesize 

« L'homme qui est incapable de s'asseoir au seuil de l'instant, en oubliant tous les
événements passés, celui qui ne peut pas, sans vertige et sans peur, se dresser un instant
tout debout, comme une victoire, ne saura jamais ce qu'est un bonheur, et, ce qui est
pire, il ne fera rien pour donner du bonheur aux autres. [...] Tout acte exige l’oubli.
[...] Il est possible de vivre presque sans se souvenir, et de vivre heureux, comme le
démontre l’animal, mais il est impossible de vivre sans oublier. Il y a un degré
d’insomnie, de rumination, de sens historique qui nuit au vivant et qui finit par le
détruire, qu’il s'agisse d’un homme, d’une nation ou d’une civilisation. »

}

\vspace{0.5cm}


Toutefois il n’est pas possible non plus d'oublier tout, ou il faudrait pour
cela renoncer à son humanité. Par quoi l’oubli fait partie, exactement, du travail
de la mémoire : c’est son pôle négatif, comme une sélection obligée, qui ne
retiendrait que ce qui est utile, plaisant ou dû (qui ne se souviendrait que par
intérêt, par gratitude ou par fidélité).

{\it OUSIA} Le mot, qui est un substantif dérivé du verbe eînai (être), peut se traduire,
selon les auteurs et les contextes, par {\it être, essence, réalité} ou
{\it substance}. C’est le réel même, ou la réalité vraie ({\it ousia ontôs ousa}, écrit Platon
dans le {\it Phèdre}, 247 c : la réalité vraiment existante). Mais il ne suffit pas de la
nommer en grec pour savoir ce qu’elle est.

OUTIL Un objet fabriqué et utile ? Sans doute. Mais cela serait vrai aussi
d’un fauteuil ou d’un lit, qui ne sont pas {\it outils} pour autant. L'outil
est utile, mais à un certain travail : c’est un objet fabriqué, qui sert à en fabriquer
— ou à en transformer — d’autres. On y a longtemps vu le propre de
l’homme, défini alors comme {\it homo faber} (voir par exemple Bergson, {\it L'évolution
créatrice}, II, p. 138 à 140). Paléontologues et éthologues, aujourd’hui, sont
plus réservés : il peut arriver à un grand singe (et il est arrivé à des hominiens,
bien avant {\it homo sapiens}) de fabriquer un outil. C’est qu’il y a là une preuve
d'intelligence, plutôt que d'humanité. Or rien ne prouve que l'intelligence
commence à l'humanité, ni donc l’humanité à l’outil. Ce n’est qu’un moyen ;
seules les fins sont humaines.
%{\footnotesize XIX$^\text{e}$} siècle — {\it }
% 418

%
%{\it }
% —
\chapter{P}
%{\footnotesize XIX$^\text{e}$} siècle — {\it }
\section{Pacifique}
%PACIFIQUE
Toute guerre est atroce, c’est une banalité qu’on ne répétera
jamais assez. Être pacifique, ce n’est pas une opinion : c’est une
vertu, et qui voudrait en manquer ? Toutefois cela n’implique pas que toute
paix soit bonne, ni même acceptable. C’est ce qui distingue le {\it pacifique} du {\it pacifiste}.
Être pacifique, c’est désirer la paix, c’est chercher à l'obtenir ou à la
défendre, mais pas à n’importe quel prix et sans s’interdire absolument la violence
ou la guerre. C’est la position de Spinoza : la guerre ne doit être entreprise
qu’en vue de la paix, et d’une paix qui soit celle non de la servitude mais
d’une population libre. C’est la position de Simone Weil. Toute violence est
mauvaise, mais non pour cela condamnable : la non-violence n’est bonne que
si elle est efficace, et elle ne l’est pas dans toutes les situations (« cela dépend
aussi de l’adversaire »). Être pacifique, pour le dire d’un mot, c’est faire de la
paix son but. Cela ne prouve pas, hélas, qu’elle suffise toujours comme moyen.

\section{Pacifiste}
%PACIFISTE
Être pacifiste, ce n’est pas une vertu, encore moins un vice ; c’est
une opinion, une doctrine ou une idéologie, qui juge que toute
guerre est non seulement mauvaise, ce qui est bien clair, mais encore nuisible
ou condamnable, qu’elle ne saurait être justifiée par rien, enfin que la paix, en
toutes circonstances, vaut mieux. C’était à peu près la position d’Alain (encore
acceptait-il, sur le territoire national, des guerres purement défensives), c’est à
peu près celle, aujourd’hui, de Marcel Conche, et ces deux noms suffraient à
me la rendre respectable. Toutefois cela fit aussi d’Alain un Munichois, certes
pour d’estimables raisons (par pacifisme, par antimilitarisme, l’un et l’autre
renforcés par le traumatisme de la première guerre mondiale), mais sans que je
puisse, sur ce terrain-là, tout à fait le suivre. Qu'il n’y ait pas de guerre juste,
%— 419 —
%{\footnotesize XIX$^\text{e}$} siècle — {\it }
comme me l’a répété souvent Marcel Conche, j’en suis bien sûr d’accord, si
l’on entend par là une guerre qui ne tuerait que des coupables. Mais n’est-ce
pas confondre la guerre, précisément, avec la justice ? Il ne s’agit pas de punir,
mais d'empêcher ou de vaincre. Que toute guerre soit injuste, dans son atroce
déroulement, cela ne prouve pas que toute paix soit supportable, ni même
admissible.

\section{Paillardise}
%PAILLARDISE
C'est « le désir gai », disait Alain, spécialement en matière de
sexualité, et comme «une précaution du rire contre les
passions ». Cela ne va pas sans un peu de vulgarité délibérée, qui fait comme
une précaution supplémentaire contre le sérieux ou l’hypocrisie. Toutefois, cela
ne dure qu’un temps — comme le désir, comme la gaieté —, sans quoi ce n’est
plus paillardise mais obsession (non plus une précaution contre les passions,
mais une passion de plus).

\section{Paix}
%PAIX
L’absence, non de conflits, mais de guerre. Ce n’est pas encore la
concorde, mais cela vaut mieux, presque toujours, que la violence
armée ou militaire. Ce {\it presque} ne va pas de soi : c’est ce qui distingue les pacifiques
des pacifistes (voir ces mots). « S’il faut appeler paix l’esclavage, la barbarie
ou l'isolement, disait Spinoza, il n’est rien pour les hommes de si lamentable
que la paix » ({\it Traité politique}, VI, 4 ; voir aussi V, 4). Et rien de meilleur,
si elle va avec la justice et la liberté.

\section{Panenthéisme}
%PANENTHÉISME
Doctrine pour laquelle tout est en Dieu, sans être Dieu
pour autant. Se distingue par là du panthéisme (voir ce
mot). En ce sens, il y a un panenthéisme chrétien, qui peut se réclamer de saint
Paul (« c’est en Dieu que nous avons la vie, le mouvement et l’être », {\it Actes}...,
17, 28) et dont il arrive que Spinoza se réclame (voir par exemple la {\it Lettre} 73,
à Oldenburg).

\section{Panthéisme}
%PANTHÉISME
C’est croire en un Dieu qui serait tout, ou en un tout qui
serait Dieu. Dieu serait donc le monde, comme on voit
chez les stoïciens, ou la nature, comme on voit chez Spinoza (« {\it Deus sive
Natura} »), et il n’y en aurait pas d’autre. C’est ce qui explique que le panthéisme,
bien souvent, fut accusé d’athéisme. Mais ce peut être aussi bien une
religion de l’immanence.

%— 420 —
%{\footnotesize XIX$^\text{e}$} siècle — {\it }
Les historiens de la philosophie distinguent parfois le {\it panthéisme}, qui
affirme que tout est Dieu, du {\it panenthéisme}, qui affirme que tout est {\it en} Dieu.
Ainsi Gueroult, à propos de Spinoza (t. 1, p. 223). Cela maintient une distance
entre Dieu ou la substance, d’une part, et ses modes d’autre part — entre la
Nature naturante, comme on peut dire aussi, et la Nature naturée. Et sans
doute cela, dans l’économie du système, est légitime : Spinoza n’a jamais cru
que les fleurs ou les oiseaux étaient Dieu. Pourtant, dès lors que cette distance
n’existe elle-même qu’en Dieu, dès lors qu’il n’y a, de la Nature naturante à la
Nature naturée, aucune transcendance d’aucune sorte, je ne suis pas sûr que
cette distinction soit vraiment éclairante. « Plus nous connaissons les choses
singulières, écrit Spinoza, plus nous connaissons Dieu » ({\it Éthique}, V, 24). C’est
plus, me semble-t-il, que du panenthéisme. Tout n’est pas Dieu ? Sans doute,
puisque seul le tout est Dieu. Mais il n’en reste pas moins que Dieu et la
Nature sont une seule et même chose : non seulement tout est en Dieu, mais
Dieu est tout en tout (puisqu'il n’y a rien d’autre). Si ce n’est pas du panthéisme,
qu'est-ce ?

\section{Papisme}
%PAPISME
C’est un autre nom pour le catholicisme, parce qu’il reconnaît
l'autorité et l’infaillibilité du Pape. Le mot, inventé par les protestants,
est bien sûr péjoratif. Toutefois il ne suffit pas de ne pas avoir de pape
pour échapper au fanatisme.

\section{Pâques}
%PÂQUES
Au singulier, c’est une fête juive, qui commémore la sortie d'Égypte.
Au pluriel, une fête chrétienne, qui commémore la résurrection du
Christ. Fête païenne, disait Alain, puisqu'elle ne fait que célébrer le triomphe
de la vie sur la mort. C’est la fête du printemps, qu’on trouve chez tous les
peuples, et de la résurrection, qu’on trouve chez la plupart. La vraie fête chrétienne,
pour Alain, c’est Noël : parce qu’elle célèbre la faiblesse plutôt que la
force, et l'amour plutôt que la vie ou la victoire. L'enfant nu et pourchassé,
entre le bœuf et l’âne, sans autre protection qu’une jeune mère, qui prie et
tremble, et qu’un père, qui s'interroge. Et la vraie fête de l'esprit, ajouterais-je,
c'est le Vendredi saint : parce quelle ne célèbre rien que la justice offensée,
l'amour déchiré et déchirant, enfin le courage pur, sans haine, sans violence,
sans espérance. « Mon Dieu, mon Dieu, pourquoi m’as-tu abandonné ? » Fête,
non de la foi, mais de la fidélité. Qu'il y ait eu ou pas résurrection, qu'est-ce
que cela change à la grandeur du Christ et de son message ? Cela dit par différence,
sur Pâques, l'essentiel : c’est la fête de la foi plus que de la fidélité, et de
l'espérance plus encore que de la foi. C’est la vraie fête religieuse, donc la vraie
%— 421 —
%{\footnotesize XIX$^\text{e}$} siècle — {\it }
fête chrétienne, malgré Alain, en tant que le christianisme est une religion. On
voudrait y croire, comme on croit au printemps. Mais le printemps n’est pas
dieu. Mais la vie n’est pas Dieu. Les athées, ce jour-là, se sentent plus athées
que jamais : ils ne croient qu’en l'esprit vivant et mortel. Ils ne l’en aiment que
davantage.

\section{Paradigme}
%PARADIGME
Un exemple privilégié ou un modèle, qui sert à penser. Le mot,
qu’on trouve chez Platon ou Aristote ({\it paradeigma}), sert surtout,
aujourd’hui, en épistémologie ou en histoire des sciences. C’est l’un des
concepts majeurs de Thomas Kuhn, dans {\it La structure des révolutions scientifiques}.
Un paradigme, c’est l’ensemble des théories, des techniques, des
valeurs, des problèmes, des métaphores, etc., que partagent, à telle ou telle
époque, les scientifiques d’une discipline donnée: c’est la « matrice
disciplinaire » qui leur permet de se comprendre et d’avancer. C’est aussi, et
par là même, ce qui est transmis aux étudiants, à la même époque, qui leur
permet de comprendre la science de leur temps, de s’y reconnaître et d’y travailler.
L'état normal des sciences (la « science normale », dit Kuhn) est celui
où un paradigme règne. Le terrain de la recherche est alors balisé par les
découvertes antérieures, et cela fait, entre les chercheurs, comme un
consensus efficace : ils sont d’accord non seulement sur les découvertes déjà
faites, mais sur ce qui reste à découvrir et sur les méthodes, pour ce faire, à
mettre en œuvre. Les révolutions scientifiques, à l’inverse, sont les périodes
où un nouveau paradigme apparaît, qui s’oppose à l’ancien, résolvant certains
problèmes jusqu’alors insolubles, en faisant disparaître d’autres, en soulevant
de nouveaux... Ainsi quand on passe de la mécanique classique (celle de
Newton) à la physique relativiste (celle d’Einstein et de ses successeurs) : ce
ne sont pas seulement les solutions qui sont nouvelles, mais aussi les problèmes,
les difficultés, les procédures. Deux paradigmes en concurrence sont
pour cela incommensurables, explique Kuhn : on ne passe de l’un à l’autre
que par une espèce de conversion globale, qui ne saurait se réduire à un progrès
purement rationnel et qui interdit de juger une théorie selon les valeurs
paradigmatiques de l’autre. Cela n'empêche pas qu’il y ait progrès, mais
interdit de le penser comme un processus linéaire et continu. Le progrès,
dans les sciences non plus, n’est pas un long fleuve tranquille.

\section{Paradis}
%PARADIS
Le lieu de la félicité, qui n’a jamais lieu. Ainsi le paradis n’est rien
de réel : ce n’est qu’un mythe ou une niaiserie.

%— 472 —
%{\footnotesize XIX$^\text{e}$} siècle — {\it }
\section{Paradoxe}
%PARADOXE
Une pensée qui va contre l’opinion, ou contre la pensée.
Cela fait deux sens différents. Aller contre l'opinion ({\it doxa}), ce
n’est en rien condamnable. Cela ne prouve bien sûr pas qu’on ait raison (un
paradoxe peut être vrai ou faux), mais suggère au moins qu’on ne se contente
pas de répéter ce qui se dit. Par exemple lorsque Oscar Wilde écrit que « la
nature imite l’art » : c’est un paradoxe, puisque la plupart des gens croient que
l’art imite la nature, mais qui peut être éclairant (il nous laisse entendre que
notre vision de la nature est influencée par celle des artistes : « Avez-vous
remarqué, demandait Oscar Wilde, comme la nature, depuis quelque temps,
ressemble à un tableau impressionniste ? »). Ou quand Talleyrand conseillait :
« Méfiez-vous du premier mouvement ; c’est le bon. » C’est un paradoxe (pourquoi
se méfier de ce qui est bon ?), mais qui donne à réfléchir : si le premier
mouvement est le bon, au sens moral du terme, il peut s’avérer très mauvais
dans un autre registre (par exemple politique ou diplomatique). On remarquera
que la plupart des paradoxes viennent d’un double sens attribué à l’un au
moins des mots utilisés : la formule, qui semble absurde selon l’un de ces sens,
peut s'avérer profonde selon un autre. Toutefois, il y a de vrais paradoxes, qui
vont vraiment contre l'opinion dominante, et sans jouer sur quelque double
sens que ce soit. Par exemple lorsque Spinoza écrit que ce n’est pas parce
qu’une chose est bonne que nous la désirons, mais inversement parce que nous
la désirons que nous la jugeons bonne ({\it Éthique}, III, 9, scolie). Nous avons tous
le sentiment du contraire. Cela ne prouve pas que Spinoza ait tort, ni qu’il ait
raison.

Mais le mot {\it paradoxe} à aussi un sens purement logique : c’est une pensée
qui va contre la pensée, disais-je, autrement dit une contradiction ou une antinomie.
Par exemple le paradoxe de Russel : l’idée de l’ensemble de tous les
ensembles qui ne se contiennent pas eux-mêmes fait un paradoxe, dans la
théorie classique des ensembles (puisque cet ensemble se contient lui-même s’il
ne se contient pas lui-même). On considère ordinairement qu’un paradoxe,
décelé dans une théorie donnée, vaut comme réfutation ou suppose, à tout le
moins, quelque aménagement : c’est ce qui s’est passé, après le paradoxe de
Russel, pour la théorie des ensembles (dont l’axiomatique exclut désormais
qu’un ensemble puisse se définir par la propriété de ne pas se contenir lui-même
comme élément). Par quoi les paradoxes, quand ce ne sont pas des sottises,
aident la pensée à avancer.

\section{Paralogisme}
%PARALOGISME
Une faute dans un raisonnement, mais involontaire. C’est
ce qui distingue le paralogisme du sophisme : le sophisme
veut tromper ; le paralogisme se trompe.

%— 423 —
%{\footnotesize XIX$^\text{e}$} siècle — {\it }
Les « paralogismes de la raison pure », chez Kant, sont les raisonnements
dialectiques qui portent sur la première des trois « Idées de la raison » (l’âme, le
monde, Dieu). Ce sont des illusions dans lesquelles tombe inévitablement la
psychologie rationnelle, dès lors qu’elle prétend connaître l’âme (comme noumène)
alors que nous n’en avons aucune expérience. Le paralogisme est en
l'occurrence de prétendre conclure de l’unité purement formelle de l’aperception
transcendantale (Punité du «je pense») à son existence substantielle
comme sujet (âme), à sa simplicité, à sa personnalité ou à son immortalité.
C’est traiter une idée comme un objet, et vouloir passer (comme la preuve
ontologique le fait à propos de Dieu, et aussi illusoirement) de la pensée à
l'existence.

\section{Paranoïa}
%PARANOÏA
On ne la confondra pas avec le délire de persécution, qui n’est
qu’une de ses formes. Le paranoïaque, s’il se croit parfois persécuté,
est souvent persécuteur. Mais cela même n’est qu’un symptôme. La paranoïa
n'est pas un vice ; c’est une psychose ou un type de personnalité. Une
folie ? Elle peut aller jusque-là, sans que le paranoïaque ait pour autant perdu
la raison. Il en ferait plutôt un usage exagéré, obsessionnel, agressif. La paranoïa,
disait Kraepelin, est caractérisée par « le développement lent et insidieux
d’un système délirant durable et impossible à ébranler, et par la conservation
absolue de la clarté et de l’ordre dans la pensée, le vouloir et l’action ». Hypertrophie
du moi, survalorisation de la logique, délire d’interprétation ou de persécution,
méfiance, rigidité, inadaptabilité..... Elle est plus fréquente chez les
hommes. C’est l’un des points qui l’opposent à l’hystérie, plus fréquente chez
les femmes, mais ce n’est pas le seul. À les considérer comme types de personnalités
davantage que comme pathologies, l’hystérie et la paranoïa constituent
comme deux pôles opposés : l’hystérique ne vit que pour les autres, ou plutôt
que pour soi et par eux ; le paranoïaque que pour soi et contre les autres. L’hystérique
est influençable, séducteur, peu soucieux de logique, avide d’amour ; le
paranoïaque est inébranlable, soupçonneux, raisonneur, avide de pouvoir. L’un
vit pour plaire : c’est un comédien ou un histrion. L'autre, pour dominer : c’est
un petit chef ou un tyran. L’un multiplie les signes ; l’autre, les interprétations.
L’un voudrait faire de sa vie une œuvre d’art ; l’autre, un système philosophique.
« On pourrait presque dire, remarquait Freud, qu’une hystérie est une
œuvre d’art déformée, qu’une névrose obsessionnelle est une religion déformée, et
un délire paranoïaque, un système philosophique déformé » ({\it Totem et tabou}, II).
Cela ne prouve rien contre l’art, ni contre la philosophie ; mais devrait pousser
à une certaine vigilance, contre l’esthétisme et les systèmes.

%— 424 
%{\footnotesize XIX$^\text{e}$} siècle — {\it }
\section{Pardon}
%PARDON
Ce n’est pas l’absolution, qui supprimerait ou effacerait la faute,
ce que nul ne peut ni ne doit. Ce n’est pas l’oubli, qui serait infidèle
ou imprudent. Pardonner, ce n’est ni oublier ni effacer ; c’est renoncer,
selon les cas, à punir ou à haïr, et même, parfois, à juger. Vertu de justice
(puisqu'il faut juger sans haine) et de miséricorde.

\section{Paresseux (argument —)}
%PARESSEUX (ARGUMENT -)
Cest un argument traditionnellement opposé
au fatalisme, notamment stoïcien. Si tout
est déterminé ou soumis au destin, il n’y a plus lieu d’agir ni de se donner de la
peine pour quoi que ce soit : le fatalisme ne pourrait aboutir qu’à la paresse ou
à l’inaction. Par exemple, s’il est écrit que je serai reçu à mon examen, à quoi
bon le préparer ? Et à quoi bon, s’il est écrit que je serai recalé ? Il n’y aurait
donc lieu de le préparer dans aucun des deux cas, qui sont les seuls possibles.
C'est bien sûr un contresens : ce qui est fatal, pour les stoïciens, ce n’est pas tel
ou tel événement isolé (par exemple le résultat d’un examen), mais l’enchaînement
des causes et des événements (chaque événement étant ainsi « confatal »,
comme disait Chrysippe, avec d’autres : par exemple le travail de l'élève avec ses
résultats à l’examen). On trouve chez Cicéron un autre exemple : « Que tu aies
appelé ou non un médecin, tu guériras » : inutile donc, suggère l'argument
paresseux, de l'appeler et de suivre ses conseils... «C’est là un sophisme,
explique Cicéron ; car il est autant dans ton destin d’appeler un médecin que
de guérir ; ce sont choses que Chrysippe appelle confatales » ({\it De fato}, XIII, 30).
Le réel est à prendre en bloc, ou à laisser ; mais que tu le prennes ou pas, cela
fait partie du réel.

\section{Parfait}
%PARFAIT
Ce à quoi rien ne manque, ni quantitativement ({\it parfait}, en ce sens,
signifie achevé), ni qualitativement (est parfait, en ce sens, ce qui
ne peut être ni amélioré ni surpassé).

Les deux sens se rejoignent : est parfait ce qui est sans défaut. Mais qu’est-ce
qu’un défaut ? Un manque, c’est-à-dire un néant, qui ne devient réel que par
l'imagination d’autre chose. Par quoi tout est parfait, dès qu’on cesse d’imaginer.
C’est le vrai secret, le plus difficile, le plus simple, que Spinoza — après
saint Thomas et Descartes, mais en en transformant le sens — a génialement
résumé en une phrase : « Par réalité et par perfection j’entends la même chose »
({\it Éthique}, II, déf. 6). Ce qui signifie que le réel est tout ce qu’il est (donc aussi,
au présent, tout ce qu’il peut être), sans aucune faute.

À quoi l’on objecte habituellement l’évidence du mal et la vanité de nos
efforts, si tout est parfait, pour changer ce qui est. C’est doublement se
%— 425 —
%{\footnotesize XIX$^\text{e}$} siècle — {\it }
méprendre. Ce que nous appelons le mal (la douleur, l'injustice, l’égoïsme...)
est aussi réel que le reste, aussi vrai, aussi parfait en ce sens, de même que nos
efforts pour le combattre ou lui résister. La tumeur qui te tue, ce n’est pas parce
qu’elle serait imparfaite ; c’est parce qu’elle est parfaitement tumeur et parfaitement
mortelle, Et même chose, bien sûr, pour les médicaments qu’on lui
oppose : qu’ils soient parfaitement efficaces ou parfaitement insuffisants, ils
n'en sont pas moins parfaitement ce qu’ils sont. Cela signifie que tout jugement
de valeur est subjectif, donc aussi nécessaire (pour les sujets que nous
sommes) qu’illusoire (si nous prétendons y voir autre chose qu’un reflet de
notre subjectivité). C’est ce que Deleuze, lisant Spinoza, sut formuler
exactement : « Si le mal n’est rien, selon Spinoza, ce n’est pas parce que seul le
Bien est et fait être, mais au contraire parce que le bien n’est pas plus que le
mal, et que l’Être est par-delà le bien et le mal » (Spinoza, {\it Philosophie pratique},
III). Par-delà le bien et le mal ? On pourrait dire aussi bien {\it en deçà} : C'est le
point de vue de Dieu (comme dirait Spinoza) ou du vrai (comme je préférerais
dire), qui contient tous les autres. Rien ne manque au réel, voilà le point,
puisque tout est là. C’est la sagesse de Prajnânpad. C’est la sagesse d’Etty
Hillesum, et c’est la seule. Un optimisme ? Nullement. Un pessimisme ? Pas
davantage. L'essentiel tient dans ces quelques phrases, qu’Etty Hillesum écrivit
dans un camp de transit, avant de partir pour Auschwitz : « On me dit parfois :
“Oui, tu vois toujours le bon côté de tout.” Quelle platitude ! Tout est parfaitement
bon. Et en même temps parfaitement mauvais. Les deux faces des
choses s’équilibrent, partout et toujours. Je n’ai jamais eu l’impression de
devoir me forcer à voir le bon côté des choses : tout est toujours parfaitement
bon, tel quel. Toute situation, si déplorable soit-elle, est un absolu et réunit en
soi le bon et le mauvais » ({\it Lettre de Westerbork}, du 11 août 1943). Sagesse
tragique : nous sommes déjà dans le Royaume, mais l’on se trompe, assurément,
si l’on y voit un paradis.

\section{Pari}
%PARI
Un engagement qu’on prend sur l’incertain, par exemple une course
de chevaux, qui sera sanctionné, selon le résultat, par un gain ou
une perte. En philosophie, le plus fameux est bien sûr celui de Pascal, qui
veut convaincre l’incroyant — puisque nous sommes « embarqués » — de
parier que Dieu existe : « Si vous gagnez, vous gagnez tout ; si vous perdez,
vous ne perdez rien. Gagez donc qu’il est sans hésiter » ({\it Pensées}, 418-233).
Cela se calcule. L'écart entre la mise et le gain doit être proportionné à la
probabilité de celui-ci. C’est ce qu’on appelle l’espérance mathématique : le
rapport entre le gain et la mise, multiplié par la probabilité de gagner (le pari
est raisonnable si ce rapport est au moins égal à 1). À pile ou face, il n’est
%— 426 —
%{\footnotesize XIX$^\text{e}$} siècle — {\it }
pas raisonnable de parier si le gain n’est pas au moins le double de la mise.
Ni de ne pas parier, s’il est supérieur au double. Avec un seul dé, il n’est pas
raisonnable de parier si le gain n’est pas au moins égal à six fois la mise
(puisqu'on n’a qu’une chance sur six de gagner), ni de ne pas parier, s’il est
supérieur à cette somme. Dès lors que le gain est réputé infini (« une infinité
de vie infiniment heureuse ») pour une mise finie (puisqu'il ne s’agit que de
notre vie terrestre, que d’ailleurs nous n’en vivrons pas moins, ou plutôt que
nous n’en vivrons que mieux) et avec un risque lui-même fini («un hasard
de gain contre un nombre fini de hasards de perte »), il est en effet raisonnable
de parier : « Partout où est l’infini, et où il n’y a pas infinité de hasards
de perte contre celui de gain, il n’y a point à balancer, il faut tout donner »
({\it ibid.}).

On remarquera que ce pari n’est en aucun cas une preuve de l'existence de
Dieu, mais seulement de l’intérêt que nous avons à y croire, ou à essayer d’y
croire (la vraie foi n’est donnée que par la grâce : le pari ne s’adresse, dans
l'esprit de Pascal, qu'aux incroyants). Reste à savoir si la pensée doit se soumettre
à l'intérêt ; c’est ce que je ne crois pas. Combien faudrait-il vous payer
pour être raciste, pour penser que l'injustice est bonne, que la Terre est immobile
ou que deux plus deux font cinq ? Pour un esprit libre, une infinité de
gain, même sans aucun risque, n’y suffirait pas. Ainsi argument du pari, si
fameux, si intelligent, ne vaut que pour ceux qui sont prêts à jouer leur vie, leur
esprit ou leur liberté aux dés, que pour ceux, pour mieux dire, qui soumettent
leur pensée à un calcul d'intérêt. Ils ne sont pas si nombreux qu’on le croit.
Aussi ce pari, pour génial qu'il soit à sa façon, n’a-t-il pas convaincu grand
monde. Les vrais croyants n’en ont pas besoin et le jugeraient indigne. Les
incroyants, s'ils n’ont pas l’esprit vénal, ne peuvent l’accepter. Autant vendre sa
voix, lors d’une élection, au plus offrant. Autant soumettre sa pensée, lors d’un
colloque, à l’espérance d’une place ou d’un prix. Pascal méprisait trop les
humains. Son pari ne peut convaincre qu’un croupier, s’il est vénal, ou un
tiroir-caisse.

\section{Parole}
%PAROLE
L'acte, plutôt que la faculté, de parler. Se distingue du langage
comme l'actuel du virtuel; du discours, comme l’acte de son
résultat ; de la langue, comme le singulier du général, ou comme l’individuel
du collectif (Saussure, {\it Cours de linguistique générale}, Payot, chap. III et note 63).
Toute parole est création d’un discours par l’actualisation du langage (comme
faculté) au moyen d’une langue (comme système conventionnel et historique).
C’est le présent du sens.

%— 427 —
%{\footnotesize XIX$^\text{e}$} siècle — {\it }
\section{Particulier}
%PARTICULIER
Ce qui vaut pour une partie d’un ensemble considéré, autrement
dit pour un ou plusieurs de ses éléments. S’oppose à
{\it universel} (qui vaut pour tous les éléments d’un ensemble) et se distingue de {\it singulier}
(qui ne vaut que pour un seul) tout en pouvant l’inclure. Par exemple
une proposition particulière porte sur quelques individus d’un ensemble
(« Quelques cygnes sont noirs», ou, comme diraient plutôt les logiciens,
« Quelque signe est noir »), voire sur un seul, s’il reste indéterminé («un cygne
est noir »). En revanche, une proposition universelle prenant le sujet dans toute
son extension, on peut dire aussi bien qu’une proposition singulière est universelle
si son sujet est déterminé, par exemple par un nom propre : « Aristote est
l’auteur de l'{\it Éthique à Nicomaque} » est une proposition à la fois singulière
(puisqu'elle ne porte que sur un seul individu) et universelle (puisqu’elle le
désigne tout entier). C’est ce qui explique que le mot {\it particulier} puisse désigner
aussi un individu, mais indéterminé (« un simple particulier ») : c’est n'importe
qui, en tant qu'il n’est pas tout le monde.

\section{Passé}
%PASSÉ
Ce qui fut et n’est plus. Tout passé reste éternellement vrai (même
Dieu, reconnaissait Descartes, ne peut faire que ce qui fut n’ait pas
été), mais sans puissance aucune, et sans acte. C’est le ne-plus-être-réel du vrai,
ou plutôt l’être-toujours-vrai de ce qui n’est plus réel. La vérité ne passe pas, et
c'est ce qu'on appelle le passé.

Contrairement à ce qu’on croit souvent, le passé n’agit jamais : ce qui agit
ou peut agir, ce sont ses traces ou ses effets actuels (qui ne sont pas du passé
mais du présent). Ainsi les séquelles d’un accident, les traumatismes, les souvenirs,
les rancunes, les promesses, les documents, et jusqu'aux causes
mêmes, qu'on croit expliquer le présent. Je ne ferais pas ce que je fais si je
n'avais vécu ce que j'ai vécu ; mais pas davantage s’il n’en restait rien de présent.
La première guerre mondiale, pareillement, ne peut contribuer à expliquer
la seconde que par ce qu’il en restait, en 1939, de réel. Enfin les étoiles
que nous contemplons, la nuit, ce n’est pas leur lumière passée qui agit sur
nos yeux (là-bas, il y a plusieurs milliers d'années !), mais ce qui en arrive, ici
et maintenant, jusqu’à nous. C’est où le réel et le vrai, pour la pensée, se séparent.
Tout passé est vrai (un mensonge ou une erreur sur le passé, ce n’est pas
du passé : c’est du présent) ; aucun n’est réel (s’il l'était, il ne serait pas du
passé). Mais comme toute vérité, par définition, est présente, on peut dire
aussi que le passé n’est rien : parce qu’il est passé, et parce que la vérité ne
passe pas. Ainsi il n’y a que le présent, et la vérité en lui de ce qui fut : il n’y
a que l'éternité.

%— 428 —
%{\footnotesize XIX$^\text{e}$} siècle — {\it }
\section{Passion}
%PASSION
Ce qu’on subit, mais en soi, sans pouvoir l'empêcher ni tout à fait
le surmonter. C’est le contraire ou le symétrique de l’action :
l’âme se soumet au corps, diraient les classiques, c’est-à-dire à la partie de soi
qui ne pense pas, ou qui pense mal. La folie est ainsi l'extrême de la passion,
comme le penchant ou l’inclination sont sa forme bénigne. Mais on utilise le
terme, plutôt, pour l’entre-deux.

La passion est un état d'âme, souvent vigoureux, mais hétéronome : c’est
un mouvement de l’âme, dirait Descartes, qui résulte en elle d’une action du
corps, qu’elle subit et ressent ({\it Traité des passions}, I, \S 27-29). C’est un affect,
dirait Spinoza, dont je ne suis pas la cause adéquate ({\it Éthique}, III, déf. 3 ; voir
aussi, {\it ibid.}, la Définition générale des affects, à comparer avec le texte latin des
{\it Principes de la philosophie} de Descartes, IV, 190). De là cette passivité qui lui
ressemble, qui n’est pas inaction (ce que l’expérience infirmerait) mais action
imposée ou subie. La passion c’est ce qui, en moi, est plus fort que moi. Une
passion libre ou volontaire, tout passionné le pressent, n’en serait plus une. On
ne décide pas d’aimer à la folie, ni de n’aimer plus, ni d’être avare ou ambitieux....
C’est en quoi la passion est une circonstance atténuante, selon les
juges, et ridicule, selon les philosophes. Un crime passionnel ne mérite ni sévérité
ni respect.

On dit souvent que les classiques blâmaient les passions, que les romantiques,
au contraire, allaient exalter.... C’est moins simple que cela. Descartes
jugeait au contraire qu’elles sont « toutes bonnes de leur nature, et que nous
n'avons rien à éviter que leur mauvais usage ou leurs excès », au point que
« c’est d’elles seules que dépend tout le bien et le mal de cette vie » : les hommes
qu’elles peuvent le plus émouvoir sont capables d’y goûter le plus de douceur
({\it Traité des passions}, III, \S 211 et 212, qu'on nuancera pourtant par les \S 147 et
148). Encore faut-il les contrôler, autant que faire se peut ou se doit, les maîtriser,
quand il le faut, les utiliser, quand c’est possible, et c’est à quoi se reconnaît
l’homme d’action.

On cite souvent le mot de Hegel, dans les {\it Leçons sur la philosophie de l'histoire},
selon lequel «rien de grand ne s’est accompli dans le monde sans
passion ». C’est en effet vraisemblable. Mais rien non plus sans action, et c’est
d’ailleurs ce que Hegel, dans les lignes qui suivent, s’empresse de préciser :
« Passion n’est d’ailleurs pas le mot tout à fait exact pour ce que je veux désigner
ici, qui est l’activité de l’homme dérivant d’intérêts particuliers, de fins
spéciales ou d’intentions égoïstes, en tant que dans ces fins il met toute
l'énergie de son vouloir et de son caractère, en leur sacrifiant autre chose qui
pourrait aussi être une fin ou plutôt en leur sacrifiant tout le reste » (Introduction,
IL, b). Il y a de la passivité dans la passion, et tel est le sens classique du
%— 429 —
%{\footnotesize XIX$^\text{e}$} siècle — {\it }
mot. Mais une passion qui reste passive n’en est plus tout à fait une, au sens
moderne : ce n’est que veulerie ou fascination.

On voit que l’on a tort de réduire la passion à l’état amoureux, qui n’est
qu’une de ses formes. Alain, un jour où il faisait cours sur la passion, rappela à
ses étudiants qu’on distinguait traditionnellement trois passions principales :
l'amour, l'ambition, l’avarice. Puis il ajouta simplement : « vingt ans, quarante,
soixante, » Ce n’était qu’une boutade, mais qui dit quelque chose d’important :
que chaque passion a ses âges, ou plutôt que chaque âge a ses passions, qui
l’emportent davantage que d’autres. Être avare à vingt ans, c’est aussi rare que
d’être amoureux à soixante, et plus grave. Toujours est-il que la passion est
plurielle : toute passion n’est pas amoureuse. Mais toute passion, à ce que je
crois, est aimante. Qu'est-ce que l’ambition, sinon une certaine façon — passionnée,
passionnelle — d’aimer le pouvoir que l’on n’a pas encore ? Qu'est-ce
que l’avarice, sinon l’amour de l’argent qu’on a déjà ? La passion, en ce sens
général, c’est la polarisation du désir sur un seul objet (Tristan), ou sur un seul
type d’objet (Don Juan), que l’on n’a pas ou que l’on craint de perdre. C’est le
triomphe d’Éros, ou plutôt son exacerbation. Le passionné reste prisonnier du
manque, mais sous deux formes différentes : l'amour de ce qu’il n’a pas encore
(Pambitieux, le cupide, le don juan), la peur de perdre ce qu’il a déjà (le puissant
qui s'accroche à son pouvoir, l’avare, le jaloux). Les passionnés, sous leurs
grands airs, sont de petits enfants, qui n’ont pas encore accepté le sevrage : ils
cherchent un sein, ou bien ils ont peur qu’on le leur retire. C’est dire qu’ils
n'aiment qu’eux-mêmes (ils ne savent que prendre ou garder), et cela indique
assez le chemin. Sortir de la passion, c’est se libérer du petit enfant qui pleure
en chacun. C’est apprendre à donner, à agir — à grandir. On n’en a jamais fini.
Raison de plus pour s’y mettre sans tarder.

\section{Pathologique}
%PATHOLOGIQUE
{\it Pathos}, en grec, c’est la passion, le trouble, la douleur, la
maladie, bref tout ce qu’on subit ou endure. C’est en ce
sens que Kant dira {\it pathologique} tout ce qui n’est pas libre ou autonome, et spécialement
tout ce qui est déterminé par la sensibilité. Le sens moderne est beaucoup
plus étroit : est pathologique tout ce qui relève d’une maladie, et cela seul.
Le contraire du normal ? Pas tout à fait, puisqu'il est normal d’avoir des maladies,
et puisque l’état pathologique, comme disait Canguilhem, continue
d'exprimer un rapport à la « normativité biologique », qu’il modifie sans
labolir ({\it Le normal et le pathologique}, Conclusion). Disons que le pathologique
est l’exception qui confirme, le plus souvent douloureusement, la règle de la
santé, qui est d’être fragile et provisoire.

%— 430 —
%{\footnotesize XIX$^\text{e}$} siècle — {\it }
\section{Patience}
%PATIENCE
Cest la vertu de l’attente, ou l’attente comme vertu. La chose
paraît mystérieuse, puisque l'attente, portant sur l'avenir, semble
nous vouer à l'impuissance et au manque. Comment serait-ce une vertu ? Mais
c'est que l'attente, même dirigée vers l'avenir, est présente ; la patience l’est donc
aussi : C’est faire ce qui dépend de nous pour attendre au mieux ce qui n'en
dépend pas. C’est une disponibilité au présent, et à la lenteur du présent, bien
plus qu’à l'avenir. Le patient, selon la formule bien connue, laisse du temps au
temps : il habite tranquillement le présent, quand l’impatient voudrait être déjà
demain ou plus tard. C’est pourquoi « {\it patience est tout} », comme disait Rilke :
parce que rien d’important ne naît qui ne prenne du temps, parce qu'il faut
croître lentement, « comme l'arbre qui ne presse pas sa sève, qui résiste, confiant,
aux grands vents du printemps, sans craindre que l’été puisse ne pas venir. L'été
vient. Mais il ne vient que pour ceux qui savent attendre, aussi tranquilles et
ouverts que s’ils avaient l'éternité devant eux... » Ils l'ont en effet, et c'est ce
qu’on appelle le présent. La patience est l’art de l’accueillir à son rythme.

\section{Patrie}
%PATRIE
Le pays dont on est originaire, où l’on est né, où l’on vit, du moins
pour la plupart des gens, ou dont on se sait, plus que d’aucun
autre, débiteur. Ce n’est pas toujours le même, ce qui explique qu’on puisse
avoir plusieurs patries, ou aucune. Disons que notre patrie, en règle générale,
c'est notre pays d’origine ou d’adoption, celui qui nous a accueilli, à la naissance
ou plus tard, le pays de nos pères ou de nos maîtres, enfin celui que nous
reconnaissons nôtre, non parce qu’il nous appartiendrait mais parce que nous
lui appartenons, au moins pour une part, au moins par le cœur et la fidélité.
C’est l'endroit d’où l’on vient ou que l’on a choisi, celui où l’on se sent chez soi,
enfin que l’on aime plus intimement que les autres pays, quand bien même ils
seraient, et ils le sont souvent, plus intéressants ou plus admirables. Notion non
pas objective, comme est davantage la nation, mais subjective et affective. J'ai
cru longtemps n’en pas avoir : la France m'était à peu près indifférente et je
professais que les intellectuels n’avaient pas davantage de patrie que les prolétaires...
J'ai changé: la France m’est de plus en plus chère, et surtout j'ai
découvert il y a bien des années — en Castille, en Toscane, à Amsterdam, à
Venise, à Prague... —, que j'avais évidemment une patrie, et qu’elle s'appelait
l'Europe. L'idée ne me viendrait pas de m’en vanter. Mais je n'aime pas trop
non plus qu’on me le reproche.

\section{Patriotisme}
%PATRIOTISME
L'amour de la patrie, mais sans aveuglement ni xénophobie.
Se distingue par là du nationalisme (voir ce mot), ou sert à
%— 431 —
%{\footnotesize XIX$^\text{e}$} siècle — {\it }
le masquer. Le nationalisme, en règle générale, c’est le patriotisme des autres ;
et le patriotisme, bien souvent, un nationalisme à la première personne. Le
propre de l’aveuglement est de ne pas se voir soi. Aussi le patriotisme ne vaut-il
que soumis à la raison, qui est universelle, et à la justice, qui tend à l’être. Tel
est le sens, aujourd’hui, des droits de l’homme et de nos tribunaux internationaux.

\section{Péché}
%PÉCHÉ
C’est le nom religieux de la faute : une offense faite à Dieu, parce
qu’on a violé tel ou tel de ses commandements. Si Dieu n’existe pas,
il n’y a donc plus de péchés à proprement parler. Restent les fautes, qui sont
innombrables, et que rien n’interdit d’appeler des péchés, quoiqu’en un sens
laïcisé : c’est offenser l'humanité en soi ou en autrui.

\section{Péché originel}
%PÉCHÉ ORIGINEL
Ce serait une faute, commise par Adam et Êve, qui nous
vouerait à la culpabilité. L’idée, à peu près inacceptable
pour les Modernes, est fortement exprimée par Pascal : « Il faut que nous naissions
coupables, ou Dieu serait injuste » ({\it Pensées}, 205-489). Il y a pourtant une
autre possibilité : que Dieu n'existe pas.

\section{Pechés capitaux}
%PÉCHÉS CAPITAUX
Les péchés capitaux font partie de notre tradition morale
et spirituelle. Chacun sait qu’il y en a sept. Mais la
plupart d’entre nous auraient bien du mal à en citer la liste complète. La
voici, telle que l’a fixée le pape Grégoire le Grand, à la fin du vr siècle, et telle
que nos catéchismes n’ont cessé, depuis, de la rappeler : {\it l'orgueil, l'avarice, la
luxure, l'envie, la gourmandise, la colère, la paresse}. Cette liste a mal vieilli : il y a
belle lurette que nous n’y reconnaissons plus nos fautes les plus graves, ni nos
dégoûts les plus résolus ! Comme me le disait plaisamment un ami, « il y a dans
ces péchés capitaux un côté doigts dans le pot de confiture, qui les rend comme
enfantins et presque ridicules ». Oui : nous avons désormais d’autres diables à
fouetter.

Qu'est-ce qu’un péché capital ? Pas forcément un péché plus grave que les
autres, mais un péché d’où les autres dérivent. C’est un péché qui vient en tête
de liste ({\it capital} vient du latin {\it caput}, la tête), un péché principiel, si l’on veut,
comme une des sources du mal. C’est où la notion de péché capital, ou de faute
capitale, pourrait retrouver son sens et son utilité, qui serait de nous aider à y
voir plus clair. Mais il faudrait en actualiser résolument la liste. Essayons.

%— 432 —
%{\footnotesize XIX$^\text{e}$} siècle — {\it }
Le premier est tout trouvé. Pourquoi faisons-nous du mal? Par pure
méchanceté ? Je n’y crois pas trop. Le plus souvent nous ne faisons du mal que
pour un bien. C’est un des points, il n’y en a pas tant, où je me sens d’accord
avec Kant : les hommes ne sont pas {\it méchants} (ils ne font pas le mal pour le
mal), ils sont {\it mauvais} (ils font du mal aux autres, pour leur bien à eux). C’est
en quoi l’égoïsme est le fondement de tout mal, comme disait encore Kant, et
le premier, selon moi, des péchés capitaux. C’est l’injustice à la première personne,
Car « le moi est injuste, expliquait Pascal, en ce qu’il se fait centre de
tout : chaque moi est l’ennemi et voudrait être le tyran de tous les autres ». On
ne fait du mal que pour son propre bien. On n’est mauvais que parce qu’on est
égoïste.

« Et le sadique ?, me demandent parfois mes étudiants. Est-ce qu’il ne fait
pas le mal pour le mal ? » Non pas : il fait du mal aux autres, pour son plaisir à
lui ; or son plaisir, pour lui, c’est un bien... Il n’en reste pas moins que la
cruauté existe, et qu’elle est sans doute la faute la plus grave, qui pourra à son
tour en entraîner plusieurs autres. C’est pourquoi il est juste de la considérer
comme un péché capital. Comment la définir ? Comme le goût ou la volonté
de faire souffrir : c’est pécher contre la compassion, contre la douceur, contre
l'humanité, au sens où l'humanité est une vertu. C’est le péché du tortionnaire,
mais aussi du petit chef pervers, du sadique ou du salaud, qui prend plaisir à
martyriser ses victimes.

Troisième péché capital : la lâcheté. Parce que aucune vertu n’est possible
sans courage, ni aucun bien. Parce que la lâcheté est une forme d’égoïsme, face
au danger. Enfin parce que la cruauté reste l'exception : la plupart des mauvaises
actions, même parmi les plus abominables, s’expliquent par la peur de
souffrir davantage que par le désir de faire souffrir autrui. Combien de gardiens,
à Auschwitz, auraient préféré rester tranquillement chez eux, plutôt que
faire ce travail atroce ? Mais ils n’avaient pas le courage de déserter, ni de désobéir,
ni de se révolter... Aussi firent-ils le mal lâchement, consciencieusement,
efficacement. Cela ne les excuse pas. Aucun péché n’est une excuse. Mais cela
explique qu'ils aient été si nombreux. Les vrais salauds sont rares. La plupart ne
sont que des lâches et des égoïstes, qui n’ont pas su résister, dans telle ou telle
situation particulière, à la pente de l’espèce ou de l’époque. Banalité du mal,
disait Hannah Arendt. La cruauté est l'exception ; l’égoïsme et la lâcheté, la
règle.

Encore faut-il pouvoir se supporter, être capable de se regarder, comme on
dit, dans une glace. À un certain degré d’ignominie ou simplement de médiocrité,
cela devient difficile sans se mentir à soi-même. C’est ce qui fait de la
mauvaise foi un péché capital : parce qu’elle rend possibles, en les masquant ou
en leur inventant de fausses justifications, la plupart de nos filouteries. Par
%— 433 —
%{\footnotesize XIX$^\text{e}$} siècle — {\it }
exemple Eichmann, zélé fonctionnaire de la Shoah, expliquant à ses juges,
après la guerre, qu’il n’a fait qu’obéir aux ordres. Ou le violeur, expliquant qu’il
n’a fait qu’obéir à ses pulsions. Ou la crapule ordinaire, expliquant que ce n’est
pas sa faute mais celle de son enfance, de son inconscient, de sa névrose. Bien
commode. Trop commode. Être de mauvaise foi, montrait Sartre, c’est faire
comme si on n'était pas libre, comme si on n’était pas responsable, alors qu’on
l’est, au moins de ses actes. C’est aussi, en un sens plus banal, mentir à autrui.
Mais le principe, bien souvent, est le même : on ment pour cacher sa faute, ou
pour la justifier, ou pour s’attribuer une valeur que l’on n’a pas. Celui qui
renoncerait à mentir — à soi et aux autres —, celui qui aurait cessé de faire semblant,
il n’aurait guère le choix qu’entre la vertu et la honte. Choix douloureux,
choix exigeant, dont la mauvaise foi vise à nous dispenser : c’est s’autoriser le
mal en s’autorisant à le dissimuler.

Je n’ai encore repris aucun des sept péchés capitaux de la tradition. Celui
que je voudrais à présent aborder, sans faire partie de la liste canonique, en est
peut-être le moins éloigné : ce que j'appelle la {\it suffisance} n’est pas très loin de ce
que les pères de l’Église appelaient l’orgueil. Mais c’est un défaut plus général,
plus profond, sans doute aussi moins tonique. Faire preuve de suffisance, ce
n’est pas seulement être orgueilleux ; c’est aussi être fat, présomptueux, vaniteux,
plein de sérieux et d’autosatisfaction, plein de soi ou de la haute idée que
l’on s’en fait... C’est le péché de l’imbécile prétentieux, et je ne connais guère
d'espèce, même chez les gens intelligents, plus désagréable. Mais c’est aussi le
péché qui est à l’origine, bien souvent, de l’abus de pouvoir, de l'exploitation
d'autrui, de la bonne conscience haineuse ou mébprisante, sans parler du
racisme et du sexisme. Le Blanc qui croit appartenir à une race supérieure ou le
macho fier de ce qu’il prend pour sa virilité ne sont pas seulement ridicules : ils
sont dangereux, et c’est pourquoi il convient de les combattre. Un misanthrope
est moins à craindre ; c’est qu’il ne prétend pas faire exception et se sait, lui
aussi, insuffisant.

S'agissant des idées, la suffisance devient fanatisme. C’est un dogmatisme
haineux ou violent, trop sûr de sa vérité pour tolérer celle des autres. C’est plus
que de l'intolérance : c’est vouloir interdire ou supprimer par la force ce qu’on
désapprouve ou qui nous donne tort. Disons que c’est une intolérance exacerbée
et virtuellement criminelle. On en connaît les effets, en tous temps et en
tous pays: massacres, guerres de religions, inquisition, terrorisme, totalitarisme...
On ne fait le mal que pour un bien, disais-je, et l’on s’autorisera
d’autant plus de mal que le bien paraît plus grand. La foi a fait plus de victimes
que la cupidité. L’enthousiasme, plus que l'intérêt. C’est qu’on massacre plus
volontiers pour Dieu que pour soi, pour le bonheur de l'humanité plutôt que
pour le sien propre. « Tuez-les tous, Dieu ou l'Histoire reconnaîtra les siens. »

%— 434 —
%{\footnotesize XIX$^\text{e}$} siècle — {\it }
Fanatisme, crime de masse. C’est le péché qui emplit les camps et allume les
bûchers.

Le dernier péché capital, puisque j'ai choisi de m’en tenir, moi aussi, à une
liste de sept, n’est pas sans évoquer l’un de ceux que retient la tradition : ce que
j'appelle la veulerie est comme une paresse généralisée, de même que la paresse
est une forme de veulerie face au travail.

Qu'est-ce que la veulerie ? Un mélange de mollesse et de complaisance, de
faiblesse et de narcissisme : c’est l'incapacité à s’imposer quoi que ce soit, à faire
un effort un peu durable, à se contraindre, à se dépasser, à se surmonter. Être
veule, ce n’est pas seulement manquer d'énergie ; c’est manquer de volonté et
d’exigence. En quoi est-ce un péché capital ? En ceci, que la veulerie en entraîne
plusieurs autres : la vulgarité, qui est veulerie dans les manières, l’irresponsabilité,
qui est veulerie face à autrui ou à ses devoirs, la négligence, qui est veulerie dans
la conduite ou le métier, la servilité, qui est veulerie face aux puissants, la démagogie,
qui est veulerie face au peuple ou à la foule... «Il faut suivre sa pente,
disait Gide, mais en la remontant. » Le veule est celui qui préfère la descendre.

Sept péchés capitaux, donc : l’égoïsme, la cruauté, la lâcheté, la mauvaise
foi, la suffisance, le fanatisme, la veulerie. Non parce qu’ils seraient forcément
les plus graves, répétons-le, mais parce qu’ils gouvernent ou expliquent tous les
autres. Ce sont les sources du mal, disais-je, et sans doute aussi celles du bien,
au moins pour une part, au moins par l’horreur ou le dégoût qu’ils nous inspirent,
par le désir d’y échapper, enfin par l'effort qu’il faut faire, presque toujours,
pour les surmonter. Pauvres immoralistes, qui ont cru qu’il suffisait de
ne plus croire en Dieu pour être délivré du mal!

\section{Penchant}
%PENCHANT
Synonyme à peu près de tendance, en plus singulier, ou d’indclination,
en moins plaisant. C’est une orientation durable du
désir, qui doit moins à l'espèce qu’à l'individu, mais plus sans doute à sa nature
qu’à sa culture ou à ses choix. Disons que c’est la pente naturelle d’un être
humain, sur laquelle il peut, ou non, se laisser glisser...

\section{Pensée}
%PENSÉE
On trouve une définition en extension, bien sûr incomplète, chez
Descartes : « Qu'est-ce donc que je suis ? Une chose qui pense.
Qu'est-ce qu’une chose qui pense ? C’est une chose qui doute, qui conçoit, qui
affirme, qui nie, qui veut, qui ne veut pas, qui imagine aussi, et qui sent »
({\it Méditations}, II). C’est définir la pensée sinon par la conscience, du moins à
partir d’elle, comme une expérience ou une dimension du sujet (« la pensée est
un attribut qui m’appartient », {\it ibid.}), et sans doute on ne peut la définir autrement,
%— 435 —
%{\footnotesize XIX$^\text{e}$} siècle — {\it }
puisque toute définition la suppose et ne s'adresse qu’à un sujet. Celui
qui ne penserait pas, comment lui faire comprendre ce que c’est que penser ?
« Penser, dira Kant, c’est unifier des représentations dans une conscience »
({\it Prolégomènes}, II). C’est en quoi aucun ordinateur ne pense: le mien, par
exemple, pourtant doté d’un logiciel de traitement de texte particulièrement
performant, est d’une bêtise crasse, qui ne cesse de me surprendre. Mais ce n’est
pas qu’il pense mal ; c’est qu’il ne pense pas.

Faut-il dire alors que toute conscience est pensée ? En un sens large, oui :
tel est le sens de Descartes. En un sens plus restreint, on ne parlera de pensée
que pour la dimension intellectuelle ou rationnelle de la conscience, disons
que pour des représentations logiquement liées, fût-ce de façon imparfaite, et
soumises ensemble à l’idée d’une vérité au moins possible. Penser, étymologiquement,
c'est peser : cela suppose l’unité d’une balance ou d’un rapport.
La pensée, c’est ce qui pèse ou soupèse les arguments, les expériences, les
informations, et jusqu’à la pesée elle-même... J'en donnerais volontiers,
complétant Kant par Spinoza et Spinoza par Montaigne, la définition
suivante : {\it Penser, c'est unifier des représentations dans une conscience, sous la
norme de l'idée vraie donnée ou possible}. La pensée est donc bien ce « dialogue
intérieur et silencieux de l’âme avec elle-même » qu’évoquait Platon, mais en
tant qu’elle cherche le vrai (puisqu'il faut « aller au vrai avec toute son âme »)
et, d'avance, s’y soumet.

\section{Perception}
%PERCEPTION
Toute expérience, en tant qu’elle est consciente ; toute conscience,
en tant qu'elle est empirique. Se distingue de la sen-
sation comme le plus du moins, comme l’ensemble de ses éléments (une perception
suppose plusieurs sensations liées et organisées), et c’est en quoi la
sensation, à vouloir la penser isolément, n’est qu’une abstraction. Vous voyez
des taches de couleurs ; vous percevez un paysage. « L'esprit met tout en
ordre », comme disait Anaxagore, ou du moins il s’y essaie. Il ne se contente
pas de sentir : il unifie ses sensations dans une conscience, dans une expérience,
dans une forme, non après coup mais dès le départ, et c’est la perception
même. Il transforme des taches lumineuses en distances ou en spectacle,
des bruits en informations, des odeurs en promesses... Percevoir, c’est se
représenter ce qui se présente : la perception est notre ouverture au monde et
à tout.

\section{Perfectibilité}
%PERFECTIBILITÉ
Ce n’est pas le pouvoir de devenir parfait, mais celui de
se perfectionner. Seul l’imparfait est donc perfectible,
%— 436 —
%{\footnotesize XIX$^\text{e}$} siècle — {\it }
mais il ne l’est qu’à la condition de pouvoir changer, et se changer. Rousseau y
voyait le propre de l'humanité : outre la liberté, explique-t-il, « il y a une autre
qualité très spécifique qui distingue l’homme de l’animal, et sur laquelle il ne
peut y avoir de contestation ; c’est la faculté de se perfectionner, faculté qui, à
l’aide des circonstances, développe successivement toutes les autres et réside
parmi nous tant dans l’espèce que dans l'individu ; au lieu qu’un animal est au
bout de quelques mois ce qu’il sera toute sa vie, et son espèce au bout de mille
ans ce qu’elle était la première année de ces mille ans » ({\it Discours sur l'origine de
l'inégalité}, I ; même idée chez Pascal, mais sans le mot de perfectibilité, dans sa
{\it Préface au traité du vide}). La perfectibilité serait donc une évolution, mais historique
et culturelle, plutôt que naturelle. C’est ce qui rend la notion utile en
même temps que relative : si l’histoire et la culture font partie de la nature,
comme je le crois, la perfectibilité n’est qu’une forme parmi d’autres de l’universel
devenir (non une exception au darwinisme, mais une de ses occurrences).
À la gloire d'Héraclite et des enseignants.

\section{Perfection}
%PERFECTION
Le fait d’être parfait (voir ce mot). On dit souvent que la perfection
n’est pas de ce monde ; c’est qu'on en imagine un
autre, idéal, auquel on compare celui-ci. La notion de perfection n’a de sens
que relatif (Spinoza, {\it Éthique}, IV, Préface ; voir aussi I, Appendice). Une perfection
absolue est un non-sens, ou n’est que l’absolu lui-même — non parce qu’il
serait sans défaut, de notre point de vue, mais parce qu’il est sans manque (il
est tout ce qu’il est et qu’il peut être). C’est l’être de Parménide et des mystiques.

\section{Performatif}
%PERFORMATIF
« {\it Je déclare la réunion ouverte} » ; si je suis président de séance,
elle l’est par là même. Tel est le discours performatif : celui
qui fait être ce qu’il dit, parce que dire et faire, en l’occurrence, sont un. Quand
je dis « {\it Je le jure} », je jure en effet : c'est une expression performative. Si je dis
« {\it Il le jure} », en revanche, je ne jure rien : l'expression n’est pas performative.
Un énoncé performatif se distingue en cela d’un énoncé descriptif ou normatif.
Il est moins soumis à l’exigence de vérité, comme le premier, ou de justesse
comme le second, qu’à celles de possibilité, de cohérence, de réussite..., qui
dépendent du contexte et des individus. Si vous dites « Je déclare la réunion
ouverte » seul dans votre chambre, ou même dans un congrès mais sans avoir
légitimité à le faire, il est probable qu'aucune réunion ne sera ouverte par là. Le
discours performatif est un acte : il a moins à être vrai ou faux qu’à être efficace
ou pas.

%— 437 
%{\footnotesize XIX$^\text{e}$} siècle — {\it }
\section{Performative (contradiction —)}
%PERFORMATIVE (CONTRADICTION —)
Une contradiction qui oppose
non pas deux énoncés l’un à
l’autre, mais un énoncé (comme proposition) à lui-même (comme acte). On en
donne souvent l'exemple suivant : « {\it J'étais dans un bateau qui a fait naufrage ;
il n'y a pas eu de survivant.} » La phrase n’est pas contradictoire en elle-même (il
n’est pas impossible que je meure dans un naufrage), mais elle l’est avec le fait
que je puisse la prononcer à la première personne. Mon ami Luc Ferry m’a souvent
reproché de tomber dans une contradiction de ce genre, comme tout
matérialiste, dès lors que je considère comme illusoire une subjectivité libre
que, par ailleurs, je serais obligé de supposer pour pouvoir prétendre à quelque
vérité que ce soit (par exemple à la vérité du matérialisme) : il y aurait une
contradiction non entre telle et telle de mes thèses, mais entre ce que je {\it fais}
(mon activité de sujet pensant) et ce que je {\it dis} (que le sujet pensant n’est
qu'une illusion ou que le résultat passif de déterminismes extérieurs). Je n’en
crois bien sûr rien. D'abord parce que l’idée de vérité n’a pas besoin de celle de
liberté, au sens du libre arbitre, voire l’exclut ({\it la vérité est exactement ce qu'on
ne choisit pas}). Ensuite parce que le sujet, de mon point de vue, même illusoire
(en tant qu’il se croit absolument libre ou transparent à lui-même), est assurément
actif : dire « je suis mon corps », ou « Je suis mon histoire », ce n’est pas
dire « Je suis passif » (parce que je serais déterminé par mon corps, par mon histoire,
par mon inconscient, etc.) ; c’est dire exactement l’inverse : si je suis mon
corps, il est exclu que je sois déterminé passivement par lui ; je suis actif, bien
plutôt, quand mon corps est actif, quand mon histoire est action, et c’est pourquoi
je le suis toujours partiellement et jamais totalement. Je n’ai pas convaincu
Luc Ferry davantage qu’il ne m’a convaincu, au moins sur ce point, mais cela
nous a aidés, ce n’est pas rien, à mieux nous comprendre (voir {\it La sagesse des
Modernes}, chap. 1 et conclusion).

\section{Persécution}
%PERSÉCUTION
Une oppression violente et ciblée. On peut opprimer tout
un peuple ; on ne peut guère persécuter qu’une minorité.
Par exemple les Protestants, dans la France catholique. Ou les Juifs, dans
l’Europe chrétienne. C’est le bras armé du fanatisme ou du racisme.

\section{Persévérance}
%PERSÉVÉRANCE
La patience et la continuité dans l'effort. C’est une forme
de courage, non contre le danger ou la peur, mais contre
la fatigue et le renoncement. Il y faut ordinairement une grande passion, ou un
grand ennui.

%— 438 —
%{\footnotesize XIX$^\text{e}$} siècle — {\it }
On pense à la formule fameuse de Guillaume d'Orange : «Il n’est pas
besoin d’espérer pour entreprendre, ni de réussir pour persévérer. » En effet : il
n’est besoin que de courage et de volonté. Mais il en faut aussi pour changer
d'orientation, quand cela paraît nécessaire ou souhaitable. C’est ce qui distingue
la persévérance de l’obstination.

\section{Personnalité}
%PERSONNALITÉ
Ce qui fait qu’une personne est différente d’une autre, et
de toutes les autres, non seulement numériquement mais
qualitativement. C’est pourquoi une personne peut manquer de personnalité :
quand elle n’est différente des autres que numériquement ou physiquement, et
ressemble pour le reste (les sentiments, les pensées, les comportements.) à
n'importe qui, spécialement, par mimétisme ou mollesse, à ceux qui l’entourent.

\section{Personne}
%PERSONNE
Un individu, mais considéré comme sujet pensant, à la fois unique
(différent de tous les autres) et un (à travers ses modifications).
C’est le sujet de l’action, qui peut donc lui être imputée : la notion touche à la
morale, spécialement chez Kant, davantage qu’à la métaphysique ou à la
théorie de la connaissance.

\section{Pessimisme}
%PESSIMISME
- Sais-tu quelle différence il y a entre un optimiste et un pessimiste ?

—?...

— Le pessimiste est un optimiste bien informé.

Cette devinette, qui nous vient d'Europe centrale, est elle-même pessimiste.
C’est pourquoi peut-être elle nous amuse : parce que nous y voyons une espèce
de cercle, sans que cela suffise pourtant à la réfuter.

Qu'est-ce que le pessimisme ? C’est mettre les choses au pire ({\it pessimus}), soit
parce qu’on juge qu’il y a plus de maux que de biens, soit parce qu'on pense
que les maux vont s’aggraver. Au sens philosophique, le pessimisme rentre
plutôt dans la première catégorie : c’est un pessimisme actuel plutôt que prospectif
(par quoi Schopenhauer est le grand penseur du pessimisme, comme
Leibniz de l’optimisme). Au sens courant, le pessimisme a plutôt à voir avec la
seconde catégorie, autrement dit avec l’avenir, qu’il imagine pire que le présent.
La vieillesse et la mort semblent lui donner raison, au moins pour l'individu,
comme le progrès et la religion, de façon différente, à l’optimisme. Il n’y avait
plus qu’à faire du progrès une religion pour que le pessimisme, c’est du moins
%— 439 —
%{\footnotesize XIX$^\text{e}$} siècle — {\it }
ce qu'on pouvait croire, fût définitivement vaincu. De là les utopies et les différents
messianismes qui n’ont cessé, depuis le xx: siècle, de nous offrir de
nouvelles raisons d’espérer.... Hélas ! cela ne nous a donné que de nouvelles raisons
de nous méfier. des optimistes.

\section{Petitesse}
%PETITESSE
L’incapacité à concevoir rien de grand, donc aussi à le faire ou
à l’admirer. Le petit voit tout à son échelle — petit, mesquin,
médiocre. Il appelle cela : « n’être pas dupe ».

\section{Pétition de principe}
%PÉTITION DE PRINCIPE
Faute logique, qui consiste à poser au départ,
fût-ce sous une autre forme, ce qu’on prétend
démontrer. C’est comme un diallèle élémentaire, de même que le diallèle est
comme une pétition de principe indirecte.

\section{Peuple}
%PEUPLE
L'ensemble des sujets d’un même souverain, ou des citoyens d’un
même État. Dans une République, c’est donc le souverain lui-même.

On dira que le peuple n’est qu’une abstraction — qu’il n’existe que des individus.
Sans doute. Mais le contrat social ou le suffrage universel réalisent cette
abstraction, donnant au peuple, comme Hobbes l'avait vu, l'unité, certes artificielle
mais effective, d’une personne. C’est ce qui distingue le {\it peuple} de la
{\it multitude} : « Le peuple est un certain corps et une certaine personne, à laquelle
on peut attribuer une seule volonté, et une action propre ; alors qu’il ne se peut
rien dire de semblable de la multitude » (Hobbes, {\it Le Citoyen}, XII, 8 ; voir aussi
VI, 1). Reste à savoir, c'était la question de Rousseau, ce qui fait qu’un peuple
est un peuple. Il faut répondre : le contrat social, autrement dit l’unité de la
volonté générale, quand elle règne. Un peuple n’est vraiment {\it un} — donc n’est
vraiment un peuple — que par la souveraineté qu’il se donne, qu’il exerce ou
qu’il défend. C’est dire qu’un peuple n’est vraiment lui-même que dans une
démocratie, et par elle. Les despotes ne règnent que sur une multitude.

\section{Peur}
%PEUR
L’émotion qui naît en nous à la perception, ou même à l’imagination,
d’un danger. Se distingue de l’angoisse par l’aspect déterminé
de ce dernier. L’angoisse est comme une peur indéterminée ou sans objet ; la
peur, comme une angoisse déterminée, voire objectivement justifiée. Cela ne
%— 440 —
%{\footnotesize XIX$^\text{e}$} siècle — {\it }
dispense pas de l’affronter, ni de la surmonter quand on peut : tel est le courage,
toujours nécessaire, jamais suffisant.

\section{Phénomène}
%PHÉNOMÈNE
Ce qui apparaît. Se distingue pourtant de l'apparence, spécialement
chez Kant et ses successeurs, par son poids de
réalité : ce n’est pas une illusion ; c’est la réalité sensible (par opposition au
noumène, réalité intelligible), la réalité pour nous (par opposition à la chose en
soi), et la seule qui soit connaissable.

Dans la philosophie contemporaine, et spécialement chez les phénoménologues,
le mot ne s’oppose plus guère au noumène ou à la chose en soi. Le phénomène,
écrit Sartre, n'indique plus, « par-dessus son épaule, un être véritable
qui serait, lui, l’absolu ; ce qu’il est, il l’est absolument, car il se dévoile comme
il est » ; il est « le relatif-absolu », qui peut être « étudié et décrit en tant que tel,
car il est absolument indicatif de lui-même » ({\it L'Étre et le néant}, Introduction).
On n’est pas très loin de ce que Marcel Conche appellera l'{\it apparence pure}, et
que Clément Rosset, plus simplement, appelle {\it le réel}.

\section{Phénoménologie}
%PHÉNOMÉNOLOGIE
L'étude des phénomènes, autrement dit de ce qui
apparaît à la conscience — et, grands dieux, que
pourrait-on étudier d’autre ? La phénoménologie ne serait donc qu’un pléonasme
pour dire la pensée ? Pas tout à fait. Il s’agissait, en décrivant ce qui
apparaît à la conscience (les phénomènes), de découvrir, comme disait Sartre,
« quelque chose de solide, quelque chose enfin qui ne fût pas l'esprit ». Décrire
la conscience, donc, ou ce qui lui apparaît, {\it pour en sortir} (puisque « toute conscience
est conscience de quelque chose » : voir l’article « Intentionnalité »). Ils
appelaient cela : {\it retour aux choses mêmes}. Mais cela nous en a plus appris sur
nous-mêmes que sur le monde. La phénoménologie, écrit sans rire Merleau-Ponty,
est « un inventaire de la conscience comme milieu de l'univers ». C’est
donc à peu près l'inverse de la physique, sans être pour autant une métaphysique
ni une morale. Ce n’est qu’un commencement de la pensée, mais qui
s’épuiserait dans la répétition — à la fois inlassable et lassante — de son premier
pas. Il en reste quelques chefs-d’œuvre absolus, et aussi plusieurs des livres les
plus difficiles et les plus ennuyeux que je connaisse.

\section{Philosophe}
%PHILOSOPHE
Je ne sais plus si c’est Guitton ou Thibon qui raconte l’anecdote,
comme lui étant personnellement arrivée. La scène se
passe au début du {\footnotesize XX$^\text{e}$} siècle, dans une campagne un peu reculée. Un jeune professeur
% — 441 — 
%{\footnotesize XIX$^\text{e}$} siècle — {\it }
de philosophie, se promenant avec un ami, rencontre un paysan, que
son ami connaît, qu’il lui présente, et avec lequel notre philosophe échange
quelques mots.

— Qu'est-ce que vous faites dans la vie ?, lui demande le paysan.

— Je suis professeur de philosophie.

— C’est un métier ?

— Pourquoi non ? Ça vous étonne ?

— Un peu, oui!

— Pourquoi ça ?

— Un philosophe, c’est quelqu'un qui s’en fout. Je ne savais pas que cela
s’apprenait à l’école !

Ce paysan prenait « philosophe » au sens courant, où il signifie à peu près,
sinon quelqu'un qui s’en fout, du moins quelqu'un qui sait faire preuve de
sérénité, de tranquillité, de recul, de décontraction... Un sage ? Pas forcément.
Pas totalement. Mais quelqu’un qui tend à l'être, et tel est aussi, depuis les
Grecs, l’étymologie du mot ({\it philosophos} : celui qui aime la sagesse) et son sens
proprement philosophique. On me dit parfois que cela n’est vrai que des
Anciens. Ce serait déjà beaucoup. Mais c’est oublier Montaigne. Mais c’est
oublier Spinoza. Mais c’est oublier Kant (« La philosophie est la doctrine et
l'exercice de la sagesse, écrivait-il dans son {\it Opus postumum}, non simple
science ; la philosophie est pour l’homme {\it effort vers la sagesse}, qui est toujours
inaccompli »). Mais c’est oublier Schopenhauer, Nietzsche, Alain. Le philosophe,
pour tous ceux-là, ce n’est pas quelqu'un de plus savant ou de plus
érudit que les autres, ni forcément l’auteur d’un système ; c’est quelqu’un qui
vit mieux parce qu'il pense mieux, en tout cas qui essaye (« Bien juger pour
bien faire », disait Descartes : c’est la philosophie même), et c’est en quoi le
philosophe reste cet amant de la sagesse, ou cet apprenti en sagesse, que l’étymologie
désigne et dont la tradition, depuis vingt-cinq siècles, n’a cessé de préserver
le modèle ou l'exigence. Si vous n’aimez pas ça, n’en dégoûtez pas les
autres.

Qu'est-ce qu’un philosophe ? C’est quelqu'un qui pratique la philosophie,
autrement dit qui se sert de la raison pour essayer de penser le monde et sa
propre vie, afin de se rapprocher de la sagesse ou du bonheur. Cela s’apprend-il
à l'école ? Cela doit s’apprendre, puisque nul ne naît philosophe, et puisque
la philosophie est d’abord un travail. Tant mieux si cela commence à l’école.
L'important est que cela commence, et ne s'arrête pas. Il n’est jamais ni trop tôt
ni trop tard pour philosopher, disait à peu près Épicure, puisqu'il n’est jamais
ni trop tôt ni trop tard pour être heureux. Disons qu’il n’est trop tard que
lorsqu'on ne peut plus {\it penser} du tout. Cela peut venir. Raison de plus pour
philosopher sans attendre.

%— 442 —
%{\footnotesize XIX$^\text{e}$} siècle — {\it }
\section{Philosophie}
%PHILOSOPHIE
Une pratique théorique (mais non scientifique), qui a le
tout pour objet, la raison pour moyen, et la sagesse pour
but. Il s’agit de penser mieux, pour vivre mieux.

La philosophie n’est pas une science, ni même une connaissance (ce n’est
pas un savoir de plus, c’est une réflexion sur les savoirs disponibles), et c’est
pourquoi, comme disait Kant, on ne peut apprendre {\it la philosophie} : on ne
peut apprendre qu’à philosopher. Le même, dans un texte fameux, ramenait
le domaine de la philosophie à quatre questions : {\it Que puis-je savoir ? Que
dois-je faire ? Que m'est-il permis d'espérer ? Qu'est-ce que l'homme ?} Les trois
premières « se rapportent à la dernière », remarquait-il ({\it Logique}, Introd.,
III). Mais elles débouchent toutes les quatre, ajouterai-je, sur une cinquième,
qui est donc la question principale de la philosophie, au point
qu’elle pourrait presque suffire à la définir: {\it Comment vivre ?} Dès qu'on
essaie de répondre intelligemment à cette question, on fait de la philosophie,
peu ou prou, bien ou mal. Et comme on ne peut éviter de se la poser, il faut
en conclure qu’on n'échappe à la philosophie que par la bêtise ou l’obscurantisme.

Il m'est arrivé de définir la philosophie, ou l’acte de philosopher, encore
plus simplement : {\it Philosopher, c'est penser sa vie et vivre sa pensée}. Non, bien sûr,
qu’il faille se contenter de l’introspection ou de l’égocentrisme. Penser sa vie,
c’est la penser où elle est : ici et maintenant, certes, mais aussi dans la société,
dans l’histoire, dans le monde, dont elle n’est pas le centre mais l'effet. Et vivre
sa pensée c’est agir, autant qu’on peut, autant qu’on doit, puisqu'on ne pourrait
autrement que subir ou rêver. Ainsi la philosophie est une activité dans la
pensée, qui débouche sur une vie plus active, plus heureuse, plus lucide, plus
libre — plus sage.

\section{Phobie}
%PHOBIE
Une peur pathologique, déclenchée par la présence d’un objet ou
d’une situation sans autre danger que cette peur même. Relève
moins du courage que de la médecine.
En un sens plus large, toute aversion incontrôlable : c’est moins une peur
qu’un dégoût ou une répulsion. Relève moins de la médecine que de l’évitement,
du courage, ou de l’habitude.

\section{Phonème}
%PHONÈME
C’est une unité phonique minimale : l'élément de la seconde
articulation (voir ce mot, ainsi que l’article « Langue »).

%— 443 —
%{\footnotesize XIX$^\text{e}$} siècle — 
\section{{\it Phronésis}}
%PHRONÈSIS
Le nom grec de la prudence (voir ce mot) ou de la sagesse pratique.
Se distingue par là de la {\it sophia}, sagesse théorique ou
contemplative.

\section{Physique}
%PHYSIQUE
Tout ce qui relève de la nature ({\it phusis} en grec), et spécialement
la science qui l’étudie ({\it ta phusika}).

Si la nature est tout, comme je le crois, la physique a vocation à absorber
toutes les sciences. Toutefois ce n’est qu’une vocation, sans doute impossible à
réaliser absolument. Par exemple la matière obéit aux mêmes lois, selon toute
vraisemblance, dans les corps vivants et dans les autres. La biologie, pour comprendre
un organisme quelconque, n’en reste pas moins nécessaire : parce que
la vie a sa rationalité propre, certes incluse dans celle de la matière inorganique
(les atomes sont les mêmes, et soumis aux mêmes lois), mais qui ne saurait pour
autant, si ce n’est par abstraction, être considérée comme nulle et non avenue.
Quelque chose de neuf émerge, qui fait qu’un corps vivant ne se réduit pas à la
simple somme de ses parties. Même chose pour la pensée. Quand je suis triste,
cela correspond assurément à des phénomènes neurobiologiques dans mon cerveau,
donc, en dernière analyse, à des phénomènes physiques dans le monde.
Toutefois ma tristesse s’explique plus simplement par la psychologie (j'ai appris
une mauvaise nouvelle, j'ai perdu un être cher, je fais une dépression...) que
par la physique. Et qui voudrait expliquer les résultats d’une élection par les lois
de la mécanique quantique ? Tout est physique, et c’est pourquoi la physique
est la science du tout. Mais avec des degrés différents de complexité, qui font
que la physique n’est pas tout, ni la science de tout.

\section{Piété}
%PIÉTÉ
Mélange d’amour et de respect, à l’égard d’un être qui nous dépasse.
Se dit le plus souvent vis-à-vis de Dieu, parfois vis-à-vis d’un parent
(piété filiale), d’un héros ou d’un maître.

On remarquera que l’étymologie est la même que pour {\it pitié} (les deux mots
furent un temps synonymes : cela s'entend encore dans notre {\it Mont-de-piété}),
que l’on aimerait définir comme un mélange d'amour et de respect, à l'égard
d’un être que l’on dépasse ou que l’on plaint. Le mot de {\it pietas}, en latin, n’a
guère ce sens : il se dit presque exclusivement de ce qu’on doit aux dieux, à ses
parents ou à la patrie. Mais le mot {\it pietà}, en italien ou en histoire de l’art, réunit
les deux idées, comme son équivalent français de « Vierge de pitié ». Voyez par
exemple la sublime {\it Pietà} de Michel-Ange. Quelle plus belle image du divin que
ce jeune homme mort, dans les bras de sa mère ?

% — 444 — 
%{\footnotesize XIX$^\text{e}$} siècle — {\it }
\section{Pitié}
%PITIÉ
Une forme de compassion, mais qui serait plutôt un sentiment
qu’une vertu (la compassion est les deux), avec je ne sais quoi de
condescendant qui la rend désagréable. L’inverse par là de la piété : c'est une
compassion qui s'exerce, ou qui croit s'exercer, de haut en bas.

\section{Plaisir}
%PLAISIR
L'un des affects fondamentaux, comme tel à peu près indéfinissable.
Disons que c’est l’affect qui s'oppose à la douleur, celui qui
nous plaît, qui nous réjouit ou nous fait du bien : c’est la satisfaction agréable
d’un désir.

On notera que ce désir n’est pas forcément un manque (par exemple dans
le plaisir esthétique) et ne précède pas nécessairement sa satisfaction : une
odeur agréable, un beau paysage ou une bonne nouvelle peuvent me faire
plaisir quand bien même je ne les désirais, avant de les rencontrer, en rien.
Spinoza dirait qu’ils ne s’en accordent pas moins avec ma puissance d'exister
(avec mon {\it conatus}), disons avec ma puissance de jouir, d’agir et de me
réjouir, autrement dit avec mon désir, en effet, mais en tant que puissance
indéterminée. J'en suis d’accord, et c’est en quoi tout plaisir est relatif : ce
n’est pas parce qu’une chose est agréable que nous la désirons, c’est parce que
nous la désirons, ou parce qu’elle s’accorde à nos désirs, qu’elle est, pour
nous, agréable. Pourquoi alors ne pas définir le plaisir, purement et simplement,
par la satisfaction d’un désir ? Parce qu’on peut satisfaire un désir sans
pour autant éprouver de plaisir : les fumeurs savent bien que le plaisir n’est
pas toujours égal, ni même toujours présent, à chaque cigarette ou à chaque
bouffée. Et chacun sait qu’il ne suffit pas de désirer vivre, hélas, pour que la
vie soit agréable.

« Tout plaisir, du fait qu’il a une nature appropriée à la nôtre, est un bien »,
remarquait Épicure. Puis il ajoutait : « tout plaisir, cependant, ne doit pas être
choisi » ({\it Lettre à Ménécée}, 129). C’est que certains apportent plus de maux,
pour soi ou pour autrui, que de biens. C’est où l’hédonisme atteint sa limite.
Le plaisir est « le bien premier et conforme à notre nature », certes, « Le principe
de tout choix et de tout refus», enfin «le principe et la fin de la vie
bienheureuse » ({\it ibid.}). Mais pas tout plaisir, ni toujours les plaisirs les plus
forts. Il faut donc choisir : c’est à quoi servent la prudence, pour ce qui est de
soi, et la morale, pour ce qui est des autres. Non qu’on renonce pour cela au
plaisir, mais parce qu’il n’est pas possible « de vivre avec plaisir sans vivre avec
prudence, honnêteté et justice » ({\it ibid.}, 132). Le plaisir est le but, pas toujours
le chemin.

%— 445 
%{\footnotesize XIX$^\text{e}$} siècle — {\it }
\section{Plaisir (principe de —)}
%PLAISIR (PRINCIPE DE -)
Il est formulé à peu près par Aristote : « On
choisit ce qui est agréable, on évite ce qui est
pénible » ({\it Éthique à Nicomaque}, X, 1).

Il est repris par Épicure : « C’est pour cela que nous faisons tout : afin de
ne pas souffrir et de n’être pas troublés. [...] C’est pourquoi nous disons que le
plaisir est le commencement et la fin de la vie bienheureuse : c’est en lui que
nous trouvons le principe de tout choix et de tout refus » ({\it Lettre à Ménécée},
128-129).

Il est chanté par Virgile, peut-être influencé par Lucrèce : « {\it Trahit sua
quemque voluptas} » (chacun est entraîné par son propre plaisir, {\it Bucoliques}, II,
65 ; comparer avec le {\it De rerum natura}, II, 172 et 258). Il sera repris par saint
Augustin et Pascal (c’est ce qu’on a appelé le panhédonisme de Port-Royal :
« L'homme est esclave de la délectation ; ce qui le délecte davantage l’attire
infailliblement : on fait toujours ce qui plaît le mieux, c’est-à-dire que l’on
veut toujours ce qui plaît », Pascal, {\it Écrits sur la grâce}, p. 332 a), mais aussi par
Montaigne (« Le plaisir est notre but..., en la vertu même, le dernier but de
notre visée c’est la volupté», I, 20) et la plupart des matérialistes du
XVII siècle... Mais c’est bien sûr Freud, dans la dernière période, qui lui
donne son nom et sa formulation canonique. « L'ensemble de notre activité
psychique a pour but de nous procurer du plaisir et de nous faire éviter le
déplaisir » (Introduction... 22 ; voir aussi « Au-delà du principe de plaisir,
1). {\it Le principe de plaisir} est l’un des deux grands principes qui régissent, selon
Freud, la totalité de notre vie psychique, ou plutôt c’est le seul : tout être
humain (peut-être même tout animal) tend à jouir le plus possible et à souffrir
le moins possible. Le {\it principe de réalité} s'y oppose moins qu’il ne le complète,
et ne le complète qu’en le modifiant. Il s’agit toujours de jouir le plus
possible, de souffrir le moins possible, mais en tenant compte pour ce faire
des contraintes du réel, ce qui suppose qu’on accepte de ne jouir parfois que
plus tard ou moins, voire de souffrir un certain temps, pour augmenter la
jouissance à venir ou éviter un désagrément plus grand. « Le principe de réalité
a également pour but le plaisir, écrit Freud, mais un plaisir qui, s’il est
différé et atténué, a l'avantage d’offrir la certitude que procurent le contact
avec la réalité et la conformité à ses exigences » ({\it ibid.}). Ce n’est guère qu’un
commentaire, mais psychanalytiquement informé, des paragraphes 129
et 130 de la {\it Lettre à Ménécée}.

\section{Platonicien}
%PLATONICIEN
De Platon, ou s’en réclamant. Pas besoin pour cela de rester
platonique.

%— 446 —
%{\footnotesize XIX$^\text{e}$} siècle — {\it }
\section{Platonique}
%PLATONIQUE
Le mot est à peu près au précédent ce que {\it stoïque} est à
{\it stoïcien} : une banalisation, une popularisation, comme une
pensée d’abord méconnaissable, à force de s’immerger dans la foule, mais où
l’on discerne pourtant, dessous les grimaces ou les contresens, comme une
étrange ressemblance. {\it Platonique}, en l'occurrence, se dit surtout de l’amour : ce
serait un amour purement sentimental ou intellectuel, sans rien de sensuel, de
charnel, de sexuel. Ceux qui ont lu le {\it Banquet} et le {\it Phèdre} s'étonneront d’abord
de cette acception. Ceux qui les reliront s’étonneront moins.

\section{Platonisme}
%PLATONISME
Le système de Platon, qu’il n’est bien sûr pas question d’exposer
ici (est-ce seulement un système ?), et toute pensée
qui en partage l'inspiration principale. À savoir ? Ceci : l'existence d’un monde
purement intelligible, plus vrai que le nôtre, où les idées existeraient par elles-mêmes,
où les valeurs (le Bien, le Beau, le Juste...) seraient autant d’absolus,
qu’il faudrait d’abord connaître, ou reconnaître, pour bien agir. Le monde sensible
ne serait qu’une copie imparfaite, qu’il faudrait toujours corriger d’après
l’Idée. Le réel ne serait qu’un moindre être, qui ne vaudrait que par l'Être
absolu, toujours absent, toujours ailleurs. Qu'un devenir, qui ne vaudrait que
par l'éternité, ici-bas hors d’atteinte. De là cette fascination pour les mathématiques
(« Que nul n’entre ici s’il n’est géomètre »), pour la dictature du philosophe-roi
(voyez la {\it République} et les {\it Lois}), enfin pour la mort (« les philosophes
authentiques sont avides de mourir », {\it Phédon}, 64 b). De là ce dédain pour l’histoire
ou la vie. C’est toujours adorer la pensée, mépriser le corps ; adorer le
savoir, mépriser le désir ; adorer l'absolu ou l’immuable, mébpriser Le relatif ou
le changeant ; adorer la vérité, mépriser le réel. Le platonisme est le modèle des
idéalismes, des dogmatismes, des utopies — et des totalitarismes, quand ils prétendent
à la science. Celui-là aimait la vérité à en mourir (quoique ce fût
Socrate qui mourut). D’autres l’aimeront à tuer... Heureusement qu'il y a
Aristote, pour nous ramener sur terre et nous remettre à notre place.

\section{Ploutocratie}
%PLOUTOCRATIE
Le pouvoir, direct ou indirect, des plus riches. Le mot n'a
pas de contraire (les plus pauvres n’ont jamais le pouvoir),
mais un remède, qui est la démocratie. Remède toujours nécessaire, rarement suffisant.
Les pauvres, presque inévitablement, votent pour plus riches qu'eux.

\section{Poésie}
%POÉSIE
L'unité indissociable et presque toujours mystérieuse, dans un discours
donné, de la musique, du sens et du vrai, d’où naît l'émotion.
%— 447 —
%{\footnotesize XIX$^\text{e}$} siècle — {\it }
C’est une vérité qui chante, et qui touche. À ne pas confondre avec la versification,
même pas avec le poème : il est rare qu’un poème soit tout du long poétique,
et il peut arriver qu’une prose, par moments, le soit.

\section{Poièsis}
%POIÈSIS
Le nom grec de la production, de la fabrication, de la création. Se
reconnaît au fait qu’elle vise toujours un résultat extérieur, qui lui
donne son sens et sa valeur (c’est l’œuvre qui juge et justifie l’ouvrier). S'oppose
à ce titre à la {\it praxis} (voir ce mot), qui ne produit qu’elle-même.

\section{Polémique}
%POLÉMIQUE
Un combat avec des mots : un discours en état de guerre. Ce
n'est pas toujours condamnable, puisque le conflit est essentiel
à la Cité, puisqu'il y a de bons combats, et puisque les mots, quand ils peuvent
suffire, valent mieux que les armes ou les coups. Toutefois le niveau intellectuel,
presque inévitablement, s’en ressent. Le débat, dans la polémique, vise
plus à la victoire qu’à la vérité ou à la justice. De là, même pour celui qui finit
par triompher, un peu de mauvaise conscience, et comme un goût de sang dans
la bouche.

\section{Polémologie}
%POLÉMOLOGIE
La science de la guerre {\it (polemos)}. N’a jamais dispensé de
la faire. N'a jamais suffi à la gagner, ni à l’éviter. La stratégie
est moins son application que l’un de ses objets. La paix, moins son
dehors (puisque toute paix suppose une guerre au moins possible) que l’un de
ses enjeux.

\section{Police}
%POLICE
Les forces de l’ordre de la Cité {\it (polis)}. L'ordre républicain, sans la
police, serait donc faible, ou plutôt, et très vite, il n’y aurait plus
d'ordre du tout, ni donc de République. Cela ne signifie pas que toute
police soit bonne, mais qu’une police, en toute cité, est nécessaire. Ses
agents, selon la belle et vieillotte appellation qu’on leur donne parfois, sont
les gardiens de la paix. Tant pis pour nous si celle-ci ne peut être gardée que
par la force. On préférerait que l'amour, la justice où même la politesse y
suffisent. Mais ce n’est pas le cas. C’est pourquoi il faut une police : pour
que force reste à la loi, sans laquelle la justice, l’amour et la politesse
devraient s’incliner, avant de disparaître tout à fait, devant les voyous ou les
puissants.

%— 448 —
%{\footnotesize XIX$^\text{e}$} siècle — {\it }
\section{Politesse}
%POLITESSE
{\it « Après vous. »} Dans cette formule de politesse, Levinas voyait
l'essentiel de la morale. On comprend pourquoi : c’est refuser
l’égoïsme et court-circuiter la violence par le respect. Toutefois ce n’est que
politesse : l’égoïsme reste inentamé, le respect, presque toujours, n’est que feint.
Peu importe. La violence n’en est pas moins évitée, ou plutôt elle ne l’est que
mieux (s’il fallait respecter vraiment pour la faire disparaître, quelle violence
presque partout !). C’est dire, sur la politesse, l'essentiel : qu’elle n’est pas une
vertu mais qu’elle en a l’apparence, et qu’elle est pour cela aussi socialement
nécessaire qu'individuellement insuffisante. Efficacité de l'apparence. Être poli,
c’est agir {\it comme} si l'on était vertueux : c’est faire semblant de respecter (« pardon»,
«s’il vous plaît », « je vous en prie»...), de s’intéresser (« Comment
allez-vous ? »), de ressentir de la gratitude (« merci »), de la compassion (« mes
condoléances »), de la miséricorde (« ce n’est rien »), voire d’être généreux ou
désintéressé (« après vous »).... Ce n’est pas inutile. Ce n’est pas rien. C’est ainsi
que les enfants ont une chance de devenir vertueux, en imitant les vertus qu’ils
n’ont pas. Et que les adultes peuvent se faire pardonner de l'être si peu.
L’étymologie rapproche la politesse de la politique. Non sans raison : c’est
l’art de vivre ensemble, mais en soignant les apparences plutôt que les rapports
de forces, en multipliant les parades plutôt que les compromis, enfin en surmontant
l’égoïsme par les manières plutôt que par le droit ou la justice. C'est
« l’art des signes », disait Alain, et comme une grammaire de la vie intersubjective.
L'intention n’y fait rien ; l’usage y est tout. On aurait tort d’en être dupe,
mais plus encore de prétendre s’en passer. Ce n’est qu’un semblant de vertu,
moralement sans valeur, socialement sans prix.

\section{Politique}
%POLITIQUE
Tout ce qui concerne la vie de la Cité {\it (polis)}, et spécialement
la gestion des conflits, des rapports de forces et du pouvoir. La
politique serait donc la guerre ? Plutôt ce qui vise à l'empêcher, à l'éviter, à la
surmonter : c’est la gestion non guerrière des antagonismes, des alliances, des
rapports de domination, de soumission ou d’obéissance. C’est ce qui la rend
nécessaire : nous vivons ensemble, dans un même pays (politique intérieure),
sur une même planète (politique internationale), sans avoir toujours les mêmes
intérêts, ni les mêmes opinions, ni la même histoire. L'égoïsme est la règle. La
peur est la règle. L’incompréhension est la règle. Comment ne serions-nous pas
ennemis ou rivaux plus souvent qu’amis ou solidaires ? De là les conflits — entre
les individus, entre les classes, entre les États —, et la menace toujours de la
guerre. « Les hommes sont conduits plutôt par le désir aveugle que par la
raison », disait Spinoza, aussi sont-ils « par nature ennemis les uns des autres »
({\it Traité politique}, H, 5 et 14). Et Épicure, vingt siècles plus tôt : « L'homme
% — 449 — 
%{\footnotesize XIX$^\text{e}$} siècle — {\it }
n'est de nature ni sociable ni en possession de mœurs douces » (cité par Themistius,
{\it Discours}, XXVI). L'histoire, entre-temps, ne les a guère démentis, et la
politique n’est jamais que l’histoire au présent. Que d’injustices partout ! Que
d’horreurs, presque partout ! Pourtant il faut, c’est notre intérêt à tous, que la
paix advienne ou se perpétue, que la solidarité s’organise ou se développe : elles
ne sont pas données d’abord, mais toujours à faire, à refaire, à préserver, à renforcer.
C’est à quoi servent les partis, les syndicats, les élections. C’est à quoi
servent les États. C’est à quoi sert la politique. Il s’agit de créer des convergences
d’intérêts — ce qui ne va pas sans compromis — afin que la paix soit, afin
que la justice et la liberté puissent être. Qu'est-ce que la politique ? C’est la vie
commune et conflictuelle, sous la domination de l’État et pour son contrôle
(politique intérieure), entre États et sous leur protection (politique internationale) :
c’est l’art de prendre, de garder et d’utiliser le pouvoir. C’est aussi
l’art de le partager ; mais c’est qu’il n’y a pas d’autre façon en vérité de le
prendre, ni de le garder.

\section{Polythéisme}
%POLYTHÉISME
La croyance en plusieurs dieux, et après tout pourquoi pas ?
La difficulté commence quand on veut les nommer, en
dresser une liste (finie ? infinie ?), et y croire. Un seul Dieu, il est désigné suffisamment
par son unicité. Mais plusieurs ? Comment les distinguer ? Comment
les reconnaître ? Pourquoi y croire ? Ce ne sont le plus souvent que des forces
naturelles, des passions ou des abstractions personnifiées, comme un animisme
hypostasié. Par exemple : un dieu du vent, un dieu de l'amour, un dieu de la
guerre, un autre pour le ciel ou l’océan, d’autres encore pour le vin, la fécondité
ou la colère, sans parler des dieux professionnels, nationaux ou ethniques.
Pourquoi pas un dieu pour la pesanteur, les banquiers ou les habitants du quatorzième
arrondissement ?

Il est de bon ton, aujourd’hui et dans certains milieux, de voir dans le polythéisme
une école de pluralisme et de tolérance. Le fait est que les Romains, par
exemple, firent aux dieux étrangers un accueil plutôt sympathique, à la seule
exception, mais ce n’est pas rien, de celui qui se prétendait l'unique, dont les
sectateurs furent donnés aux lions ou aux flammes. Bonjour la tolérance, qui ne
tolère que le semblable ! Puis le polythéisme n’empêcha pas les Grecs d’assassiner
Iphigénie et Socrate, ou plutôt il les y poussa efficacement. La religion
n'attendit pas le monothéisme pour être, comme disait Lucrèce, pourvoyeuse
de crimes.

On me dira que l’athéisme fit des morts aussi, et peut-être davantage. C’est
hélas vrai. Mais dans la mesure seulement où il se prit pour une religion (de
l'Histoire ou de l’État) ou un messianisme (du prolétariat). Non par manque
%— 450 —
%{\footnotesize XIX$^\text{e}$} siècle — {\it }
de religion, donc, mais par excès de foi. C’est toujours l'enthousiasme qui
allume les bûchers. Le polythéisme n’en est préservé, quand il l’est, que par
l'impossibilité de prendre tout à fait au sérieux ces dieux trop nombreux et trop
humains. C’est la petite monnaie de l’absolu. Videz vos poches.

\section{Positif}
%POSITIF
En philosophie, s'oppose moins à {\it négatif} qu’à {\it naturel}, {\it métaphysique}
ou {\it chimérique}. Est positif ce qui existe en fait (par exemple le {\it droit
positif}, opposé au {\it droit naturel}) ou s'appuie sur les faits (les sciences positives).
Chez Auguste Comte, dans sa fameuse loi des trois états, « l’état positif ou
réel », qui est le troisième, s'oppose à « l’état métaphysique ou abstrait », auquel
il succède, comme celui-ci s'oppose à « l’état théologique ou fictif » qui l'avait
précédé.

\section{Position}
%POSITION
L'une des dix catégories d’Aristote. Ce n’est pas le lieu où l’on
se trouve (la localisation), mais une certaine façon de l’occuper :
par exemple assis ou debout. Le mot {\it situation}, qu’on utilise parfois en ce sens,
prête davantage à confusion : mieux vaut le réserver pour un autre usage.
Leucippe et Démocrite, se souvient Aristote, pensaient que les atomes ne se
distinguaient entre eux que par « la forme, l’ordre et la position », comme font
les lettres dans un mot : « Ainsi {\it A} diffère de {\it N} par la forme, {\it AN} de {\it NA} par
l’ordre, et {\it Z} de {\it N} par la position » (le {\it Z} étant comme un {\it N} couché, et
réciproquement : voir {\it Métaphysique}, A, 4). Pourtant {\it Z} et {\it N} sont deux lettres
différentes, quand Socrate debout ou couché reste un seul et même homme.
C’est que Socrate n’est pas une lettre : il peut changer de position sans que sa
position le change.

\section{Positivisme}
%POSITIVISME
C’est d’abord le système d’Auguste Comte, qui ne voulait
s'appuyer que sur les faits et les sciences : il renonce pour
cela à chercher l'absolu et même les causes (le {\it pourquoi}), pour ne s’en tenir
qu’au relatif et aux lois (le {\it comment}). Il en a fait une puissante synthèse, qui est
le positivisme même. Système impressionnant, aussi bien dans sa masse que
dans son détail, aujourd’hui injustement méprisé. Il faut dire qu’il est desservi
par la personnalité de son auteur, dont la santé mentale laissait à désirer, et par
son style, étonnamment indigeste. Qu’on en juge à ce petit passage, qui dit
pourtant, sur l'esprit du positivisme, l'essentiel :

\vspace{0.5cm}

%— 451 — — {\it }
{\footnotesize 
« Constatant l’inanité radicale des explications vagues et arbitraires propres à la philosophie
initiale, soit théologique, soit métaphysique, l'esprit humain renonce désormais
aux recherches absolues qui ne convenaient qu’à son enfance, et circonscrit ses
efforts dans le domaine, dès lors rapidement progressif, de la véritable observation,
seule base possible des connaissances vraiment accessibles, sagement adaptées à nos
besoins réels. La logique [...] reconnaît désormais, comme {\it règle fondamentale}, que
toute proposition qui n’est pas strictement réductible à la simple énonciation d’un fait,
ou particulier ou général, ne peut offrir aucun sens réel et intelligible. [...] En un mot,
la révolution fondamentale qui caractérise la virilité de notre intelligence consiste essentiellement
à substituer partout, à l’inaccessible détermination des causes proprement
dites, la simple recherche des {\it lois}, c’est-à-dire des relations constantes entre les phénomènes
observés » ({\it Discours sur l'esprit positif}, VX, 12).
}

\vspace{0.5cm}

Le mot, après Auguste Comte, s’est banalisé. Il désigne toute pensée qui
prétend s’en tenir aux faits ou aux sciences, à l’exclusion de toute interprétation
métaphysique ou religieuse, voire de toute spéculation proprement philosophique.
C’est ainsi qu’on parle de {\it positivisme juridique} (une conception du
droit qui ne reconnaît ou n’étudie que le droit positif) ou de {\it positivisme logique}
(la doctrine de Carnap et son école), non que leurs partisans se réclament en
rien d’Auguste Comte, qu’ils n’ont guère lu, mais parce qu’ils s’opposent eux
aussi à la métaphysique et voudraient ne s’en tenir qu’à ce qui peut être positivement
établi (par exemple dans des textes de lois ou des énoncés scientifiques).

En dehors de ces acceptions historiques précises, le mot positivisme est
habituellement pris, du moins aujourd’hui, en un sens péjoratif. Ce serait une
pensée courte, et comme une négation de la philosophie. Cet usage relève surtout
de la polémique. Encore faut-il utiliser le mot à bon escient. On évitera
notamment de confondre le {\it positivisme}, qui renonce à la métaphysique, avec le
{\it scientisme}, qui voudrait que la science en soit une.

\section{Possession}
%POSSESSION
Le fait de posséder quelque chose, c’est-à-dire d’en avoir la
jouissance ou l’usage. Mais ce n’est qu’un fait : c’est ce qui
distingue la {\it possession} de la {\it propriété}, qui est un droit.

\section{Possible}
%POSSIBLE
Ce qui peut être ou arriver. Donc ce qui n’est pas ? Point forcément,
ni en toute rigueur : ce qui est peut être, puisqu'il est, et
ce serait plutôt ce qui n’est pas qui s’avère présentement impossible (puisqu'il
n'est pas). Par exemple il est assurément possible que je sois assis, puisque je le
suis, et présentement impossible, tant que je le suis, que je ne le sois pas. Mais
alors seul le réel serait possible, qui serait aussi nécessaire, et tout le reste serait
%— 452 —
%{\footnotesize XIX$^\text{e}$} siècle — {\it }
impossible. 11 n’y aurait que ce qu’il y a : les catégories de la modalité (voir ce
mot) s’aboliraient dans une espèce de monisme ontologique. C’est le monde
même. Mais comment le penser au futur, sans distinguer ce qui peut arriver (le
possible) de ce qui ne le peut pas (l’impossible) ou de ce qui arrivera inévitablement
(le nécessaire) ? Il faut se donner une autre définition, proprement
modale, qui ne définit plus le possible par rapport à l'être mais par rapport à
son contraire : est possible, au sens large, tout ce qui n’est pas impossible. C’est
donc la modalité la plus vaste, qui inclut tout ce qui est réel, tout ce qui peut
le devenir, tout ce qui le deviendra nécessairement. En un sens restreint, en
revanche, on entend par possible tout ce qui n’est ni réel, ni nécessaire, ni
impossible : tout ce qui peut être ou ne pas être, donc ce qui n’est pas encore
et ne sera peut-être jamais. Cela n'existe que pour la pensée : c’est un être de
raison, comme dit Spinoza, mais aucune raison, si elle se fait prospective, ne
peut s’en passer.

\section{Postulat}
%POSTULAT
Un principe qu’on pose, sans pouvoir le démontrer. Ne se distingue
de l’axiome que par une évidence moindre. Les mathématiciens
modernes ont d’ailleurs renoncé à cette distinction. C’est qu'ils ont
renoncé à l'évidence des principes, pour ne plus reconnaître que la nécessité des
inférences.

\section{Postulats de la raison pratique}
%POSTULATS DE LA RAISON PRATIQUE
Chez Kant, ce sont des propositions
théoriques dont la vérité
est affirmée — mais en vertu d’une nécessité seulement pratique et subjective,
donc sans qu’on puisse y voir une démonstration — à partir des exigences
de la moralité. Ces postulats sont au nombre de trois : la liberté de la volonté,
l’immortalité de l’âme, l'existence de Dieu ({\it C.R. Pratique}, Dialectique, IV-VI).
Il faut y croire, selon Kant, pour que l’expérience morale ait un sens. Cela ne
prouve pas que Dieu existe, que nous soyons libres ou que l'âme soit immortelle
(puisque rien ne prouve que la morale ait un sens), ni n’entraîne que ce
soit un devoir d’y croire (puisque « ce ne peut être un devoir d’admettre l’existence
d’une chose »), mais qu’il est moralement nécessaire d’accepter ces trois
postulats : on ne saurait autrement échapper à l'absurde et au désespoir. Mais
pourquoi faudrait-il y échapper ? Pour faire son devoir ? Non pas, puisque
celui-ci n’a pas besoin d’espérance. Les postulats de la raison pratique ne répondent
pas à la question {\it « Que dois-je faire ? »}, mais à la question {\it « Que m'est-il
permis d'espérer ? »}. C’est la dimension d’espérance de la morale, par où elle
conduit à la religion.

%— 453 —
%{\footnotesize XIX$^\text{e}$} siècle — {\it }
\section{Pour-soi}
%POUR-SOI  Exister {\it pour soi}, c’est être en relation avec soi sur un autre mode
que la seule identité. Se distingue par là de l'en soi, spécialement
chez Hegel et Sartre. L’en soi est ce qu’il est ; le pour-soi a à l’être,
explique Sartre, ce qui suppose qu’il n’est pas ce qu’il est et qu’il est ce qu’il
n’est pas. C’est le mode d’être de la conscience, qui lui interdit de coïncider
exactement avec elle-même (d’exister en soi) : « Le pour-soi est un être pour qui
son être est en question dans son être en tant que cet être est essentiellement
une certaine manière de {\it ne pas être} un être qu’il pose du même coup comme
autre que lui » ({\it L'être et le néant}, p. 222). Vouloir être en soi, pour un homme,
ce serait faire semblant de n’être pas libre (mauvaise foi) ; vouloir exister en-soi-pour-soi,
ce serait vouloir être Dieu : « Ainsi la passion de l’homme est-elle
inverse de celle du Christ, car l’homme se perd en tant qu'homme pour que
Dieu naisse. Mais l’idée de Dieu est contradictoire et nous nous perdons en
vain : l’homme est une passion inutile ({\it ibid.}, p. 708).

\section{Pouvoir}
%POUVOIR
Définition parfaite chez Hobbes : « Le pouvoir d’un homme
consiste dans ses moyens présents d’obtenir quelque bien apparent
futur » ({\it Léviathan}, chap. 10). C’est donc du réel (puisque c’est du présent),
mais tout entier tourné vers l'avenir. Pouvoir, c’est pouvoir faire. Encore
faut-il distinguer le {\it pouvoir de}, qu’on appellerait mieux {\it puissance} (pouvoir de
marcher, de parler, d’acheter, de faire l'amour), et le {\it pouvoir sur}, qui est une
forme du précédent (c’est le pouvoir de commander et de se faire obéir), mais
qui porte sur des êtres humains et qui est le pouvoir au sens strict. Non la
simple action possible, mais l’ordre possible, la contrainte possible, le contrôle
possible, la sanction possible... Dès que l’action possible est action possible (et
reconnue telle de part et d’autre) sur la volonté de quelqu'un d’autre, on passe
du {\it pouvoir de} au {\it pouvoir sur} — et l’action possible est alors immédiatement
action réelle. Pouvoir punir ou récompenser, pouvoir autoriser ou interdire,
cette simple possibilité (comme {\it pouvoir de}) est déjà une réalité (comme {\it pouvoir
sur}). C’est le secret du pouvoir : il s'exerce même quand il n’agit pas ; il gouverne
même quand il n’ordonne pas. La simple possibilité d’agir (quand c’est
agir sur quelqu'un) est déjà une action. Pouvoir commander, c’est déjà commander
en effet.

Deux sens, donc : {\it pouvoir de}, et {\it pouvoir sur}. L'action possible, ou la domination
réelle. On peut, pour les distinguer, appeler le premier {\it puissance}
({\it potentia}, en latin), et garder pour le second le mot de {\it pouvoir}, en un sens strict
({\it potestas}). Mais à condition de ne pas oublier que la puissance est première : la
{\it potestas} n'est qu'une {\it potentia} particulière ; le pouvoir n’est que la puissance
d'un homme ou d’un groupe sur d’autres hommes ou d’autres groupes. Le
%— 454 —
%{\footnotesize XIX$^\text{e}$} siècle — {\it }
pouvoir, c’est la puissance humaine que l’on subit ou, plus rarement, que l'on
exerce. La puissance, nous la partageons avec la nature. Il n’est de pouvoir
qu’humain. C’est pourquoi le pouvoir est tellement agaçant, quand c’est celui
des autres, et tellement délicieux, quand c’est le sien. Hobbes encore : « Je
mets au premier rang, à titre d’inclination générale de toute l'humanité, un
désir perpétuel et sans trêve d’acquérir pouvoir après pouvoir, désir qui ne cesse
qu’à la mort » ({\it op. cit.}, chap. 11). Au premier rang ? Pour ce qui me concerne,
je n’irais pas jusque-là. Plusieurs inclinations, qui ne sont pas toutes estimables,
m'importent davantage.

\section{Pragmatique}
%PRAGMATIQUE
Qui relève de l’action ({\it pragma}) et ne reconnaît d’autre
critère que sa réussite ou son efficacité. En philosophie,
et contrairement à l’usage politique ou journalistique, le mot exprime plutôt
une réserve qu’une approbation. Une injustice efficace n’en est pas moins
injuste pour autant.

\section{Pragmatisme}
%PRAGMATISME
Une attitude ou une doctrine qui privilégie l’action, et la
réussite de l’action, jusqu’à en faire le seul critère légitime
d'évaluation. Le bien ? C’est ce qui réussit. Le vrai ? C’est ce qui est utile ou
efficace («ce qui marche »). On peut en donner une version courte, qui ne
serait qu’une forme de sophistique : le nazisme serait vrai si Hitler avait gagné
la guerre. Mais on peut aussi, avec Charles Sanders Peirce et William James, y
voir une philosophie de la science et de la démocratie. Le fait qu’il s'agisse de
deux philosophes américains ne saurait tenir lieu de réfutation.

Qu'est-ce que le pragmatisme ? Une doctrine, répond Peirce, qui identifie
la conception d’un objet à celle de ses effets possibles. Savoir ce que sont le feu
ou la gravitation, c’est savoir quels effets ils peuvent produire. Aussi une idée
n'est-elle qu’une hypothèse, qu’il faut soumettre à l'expérimentation pour
déterminer sa valeur ; il n’y a aucun sens à la tenir pour vraie si aucun effet ne
la valide. La vérité, pour le pragmatiste, c’est donc bien ce qui réussit, mais pas
au sens mercantile du terme (« ce qui paye ») : c’est ce qui résiste efficacement
à sa mise à l'épreuve expérimentale. La vérité n’est pas un absolu ; c'est une
hypothèse qui a fait ses preuves.

Le même genre d’idées peut s'appliquer à la politique. Une injustice efficace,
disais-je dans l’article précédent, n’en est pas moins injuste pour autant.
Mais une justice sans effet, répondrait un pragmatiste, comment serait-elle
juste ? Aussi faut-il soumettre nos idées à l’épreuve du réel, plutôt que le réel à
une idée préconçue, comme fait le totalitarisme. Non qu’il faille pour cela se
%— 455 —
%{\footnotesize XIX$^\text{e}$} siècle — {\it }
passer d’idéaux, ni même qu’on le puisse. Mais parce qu’un idéal n’est que
l’ensemble des conséquences prévues d’une activité, qui la motivent, certes,
mais doivent à leur tour être soumises à l’expérience (un idéal qui ne peut
réussir est un mauvais idéal). La démocratie est la mise en œuvre de cette expérience
commune, qui la valide dans la mesure même où elle s’y soumet.

Ce pragmatisme-là n’est pas une sophistique ; c’est un empirisme radical
(l'expression est de William James) et une philosophie de l’action.

\section{Pratique}
%PRATIQUE
La définition élaborée par Althusser me paraît trop étroite :
« Par {\it pratique} en général, écrivait-il, nous entendons tout processus
de transformation d’une matière première donnée déterminée, en un
produit déterminé, transformation effectuée par un travail humain déterminé,
utilisant des moyens (de “production”) déterminés » ({\it Pour Marx}, p. 167).
C'était accorder trop à la production et au travail. Je dirais plutôt : J'entends
par pratique une activité ({\it praxis}, en grec, ou {\it energeia}) qui transforme quelque
chose ou quelqu'un, soit en produisant une œuvre extérieure à cette activité
(Aristote parlait alors de {\it poièsis}), soit en ne produisant que cette activité même
({\it praxis} au sens restreint). C’est « l’activité humaine concrète », comme disait
Marx ({\it Thèses sur Feuerbach}, 1), dont le travail n’est qu’un cas particulier.

\section{Pratique théorique}
%PRATIQUE THÉORIQUE
Une activité dans la pensée, et qui la transforme.
Sa matière première est faite de représentations,
de concepts, de faits, de théories, de valeurs, de connaissances (voir
Althusser, {\it Pour Marx}, p. 168), qu’elle travaille ou critique jusqu’à en obtenir
d’autres représentations, d’autres concepts, d’autres faits, d’autres théories,
d’autres valeurs ou d’autres connaissances. Les pratiques théoriques sont bien
sûr multiples, voire innombrables : chaque science est l’une d’elles, ou plutôt
chaque activité scientifique ; la philosophie en est une autre, ou plutôt toute
activité philosophante.

\section{Praxis}
%PRAXIS
Le nom grec de l’action ; le nom snob ou marxiste de la pratique.
Le mot ne me semble guère utile que par son opposition, d’origine
aristotélicienne, à la {\it poièsis}. Ce sont deux types d’action, mais qui se distinguent
par la présence ou non d’un but extérieur. La {\it praxis} est alors une action
qui ne vise rien d’autre que son bon déroulement (son {\it eupraxia}) : elle ne tend
à aucune fin extérieure à elle-même ni à aucune œuvre extérieure à celui qui
agit. Ce n’est pas qu’elle soit stérile ; c’est quelle se suffit à elle-même. La posèsis,
%— 456 —
%{\footnotesize XIX$^\text{e}$} siècle  {\it }
au contraire, est une production ou une création : elle n’a jamais sa fin en elle-même,
mais toujours dans son résultat, qui lui reste extérieur (le produit ou
l’œuvre : {\it ergon}). La vie, par exemple, est une {\it praxis} : vivre, c’est créer sans
œuvre. Et le travail ou l’art, une {\it poièsis}. Celle-ci n’a de sens qu’au service de
celle-là.

\section{Précaution (principe de —)}
%PRÉCAUTION (PRINCIPE DE-) Prendre des précautions, c’est agir pour
éviter un mal, ou ce qu’on juge être tel.
Prudence appliquée, face à un risque réel ou supposé. Ainsi en matière de
contraception ou d’alpinisme : la prudence n’impose pas qu’on renonce à faire
l'amour lorsqu'on ne veut pas d’enfant, ni à escalader la montagne lorsqu'on ne
veut pas mourir, mais assurément qu'on prenne, pour le faire, un certain
nombre de précautions (un moyen contraceptif efficace dans le premier cas, un
équipement et un entraînement adaptés dans le second..). Ces exemples, on
pourrait en prendre beaucoup d’autres, justifient deux remarques.

La première, c’est que la précaution suppose une évaluation préalable, et ne
saurait en tenir lieu. Faire un enfant, est-ce un bien ou un mal ? Les précautions
éventuelles en dépendent ; elles n’en décident pas.

La seconde remarque, c’est que la précaution est ordinairement tout autre
chose qu’un évitement. Que penseriez-vous de celui qui vous expliquerait :
« En matière d’alpinisme et de sexualité, j'ai pris mes précautions : j'ai choisi la
plaine et la chasteté » ? Que ce n’est plus précaution mais fuite. Prendre des
précautions, c’est agir ; non pour supprimer tout risque, ce qu’on ne peut, mais
pour le réduire le plus possible, dans une situation donnée — y compris
lorsqu'elle reste, comme dans le cas de l’alpinisme, inévitablement périlleuse. Il
ne s’agit pas de renoncer, mais de préparer, d’anticiper, d’assurer — de faire
attention. Prudence appliquée, disais-je, et c’est la prudence même.

Qu'en est-il alors de ce fameux {\it « principe de précaution »}, dont on nous
rebat depuis quelques années les oreilles, le plus souvent sans prendre la peine
de le définir ni même de l’énoncer ? En quoi se distingue-t-il de la simple
prudence ?

D'abord en ceci, me semble-t-il, qu’il concerne surtout les pouvoirs publics
ou, à tout le moins, les collectivités : un gouvernement ou une entreprise peuvent
appliquer le principe de précaution ; un individu se contentera, dans sa vie
privée, de prendre les siennes.

Ensuite en ceci que le principe de précaution suppose que les risques soient
impossibles à mesurer exactement, voire à attester absolument. Quand un pays
décide d’une limitation de vitesse sur ses routes, il n’applique pas le principe de
précaution : les risques de la vitesse, en matière de sécurité routière, sont tristement
%— 457 —
%{\footnotesize XIX$^\text{e}$} siècle — {\it }
avérés et d’ailleurs faciles, fât-ce par voie statistique, à mesurer avec un
degré satisfaisant de précision et de certitude. Aussi est-ce moins {\it précaution} que
{\it prévention}. Mais en matière d'organismes génétiquement modifiés (les fameux
OGM) ? Mais en matière de transfusion sanguine ? Mais en matière d’énergie
nucléaire? Qu'il y ait des risques dans les trois cas, c’est plus que
vraisemblable ; mais ils ne relèvent du principe de précaution — et non de la
seule prudence — que dans la mesure où ces risques ne peuvent être déterminés
exactement, ni même d’une façon qui permette de les comparer précisément
avec les avantages attendus des pratiques qui les font naître (les manipulations
génétiques, les perfusions, les centrales nucléaires...). C’est ce qui distingue la
{\it précaution} de la {\it prévention}. « La prévention, remarque Catherine Larrère, a rapport
aux risques avérés, dont l'existence est certaine et la probabilité plus ou
moins bien établie. La précaution a affaire aux risques potentiels, non encore
avérés » ({\it Dictionnaire d'éthique et de philosophie morale}, PUF, article « Principe
de précaution »). Qu'il y ait, dans chaque centrale nucléaire, une prévention
des risques, c’est la moindre des choses. Mais faut-il, ou pas, construire de telles
centrales ? Cela ne relève plus de la prévention, mais du principe de
précaution : parce qu'il s’agit de confronter des avantages déterminés (de coût,
d'indépendance énergétique, de sûreté des approvisionnements, de réduction
de l'effet de serre.) à des risques qui restent pour une bonne part indéterminables
(ceux d’un accident ou d’une guerre, ceux qui concernent le stockage
des déchets pendant plusieurs milliers d’années....). C’est aussi ce qui distingue
le principe de précaution de la simple prudence. « Le principe de précaution,
me dit un jour Jean-Pierre Dupuis, c’est la prudence en situation d’incertitude »,
non au sens ordinaire du terme (car c’est le lot presque toujours de la
prudence, que d’être confrontée à l’incertain), mais au sens où l’on parle
d'incertitude en mécanique quantique : quand la détermination des risques
bute sur une limite infranchissable, qui ne permet ni de vérifier ni de quantifier
leur réalité.

Le principe de précaution a donc bien à voir avec la prudence, dont il n’est
qu’une occurrence particulière : c’est la prudence en situation d’incertitude et
de responsabilité collective. Le législateur en a donné la formulation suivante :
« L'absence de certitudes, compte tenu des connaissances scientifiques et techniques
du moment, ne doit pas retarder l’adoption de mesures effectives et proportionnées
visant à prévenir un risque de dommages graves et irréversibles à
l’environnement à un coût économiquement acceptable » (loi du 2 février
1995, dite « loi Barbier »). Pour ceux qui ne sont pas juristes, je proposerais
volontiers une formulation plus simple, qui pourrait s'adresser à tout responsable
d’une collectivité quelconque, qu’elle soit publique ou privée : {\it N'attends
pas qu'un risque soit démontré ou mesuré pour essayer de le prévenir ou d'en limiter
%— 458 —
%{\footnotesize XIX$^\text{e}$} siècle — {\it }
les effets}. Et je ne connais guère de principe plus incontestable, ni plus incontesté.

Dans la pratique, toutefois, il me semble que ce principe tend communément
à prendre une autre forme, souvent implicite mais qui ressort de l'usage
qui est en fait. Tel qu’il fonctionne dans nos journaux ou dans les discours de
nos hommes politiques, il pourrait plutôt s’énoncer sous cette forme : {\it « Ne faisons
rien qui puisse présenter un risque que nous ne serions pas capables d'évaluer
précisément ou que nous ne serions pas certains de pouvoir surmonter. »} Et quoi, en
apparence, de plus raisonnable ? Le problème, c’est que si l’on adopte cette dernière
formulation, il faut en conclure que ce serait violer le principe de précaution
que de se lever le matin : qui sait quels risques possiblement mortels cela
nous fait courir ? Mais rester au lit toute la journée et tous les jours n’est pas
non plus sans danger : voilà que le principe de précaution nous enferme dans
une contradiction insurmontable... Je plaisante, puisque le principe de précaution
ne vaut guère, je l’ai signalé en passant, que dans les situations de responsabilité
publique ou collective, et puisque les risques, ici, pourraient être statistiquement
mesurés (les assureurs y parviennent fort bien). Mais même à
considérer le principe de précaution dans son champ légitime d’application, il
n’est pas difficile de trouver des apories comparables. Lorsque l’automobile fut
inventée, qui pouvait en évaluer précisément les risques, aussi bien en matière
d’accidents que de pollution ? Et qui le peut aujourd’hui ? « Il va y avoir des
milliers de morts ! », pouvait dire l’un. Il y en eut plutôt des millions. Mais fallait-il
pour autant renoncer à l'automobile? Question légitime, encore
aujourd’hui. Je ne vois pas que le principe de précaution suffise à y répondre.

On m’objectera que l'invention de l’automobile n’entraïnait pas un changement
irréversible : il n’est pas impossible, au moins en théorie, de revenir en
arrière. Cela est vrai ; et même si cette possibilité reste en effet purement théorique
(quel gouvernement pourrait aujourd’hui interdire l'automobile ?), cela
dit quelque chose d’important sur le principe de précaution : qu’il doit s’appliquer
d’autant plus rigoureusement que les risques encourus peuvent être irréversibles.
C’est le cas, par exemple, du débat sur les OGM. Une fois que des
gènes modifiés se seront répandus dans la nature, il sera sans doute impossible
de les supprimer : le changement sera irréversible, et les risques encourus, dès
lors, le seront tout autant. Raison de plus pour être vigilant. Mais est-ce une
raison suffisante pour renoncer aux OGM, et aux avantages éventuels (en
matière de rentabilité, mais aussi en matière de protection de l’environnement,
de lutte contre la faim, de recherche médicale...) qu’on peut en attendre ? Je ne
sais. J'ai participé à plusieurs tables rondes sur la question : j’ai pu constater que
les experts eux-mêmes, sur la décision à prendre, s’opposaient vigoureusement.
Je doute que le principe de précaution suffise à les mettre d’accord.

%— 459 —
%{\footnotesize XIX$^\text{e}$} siècle — {\it }
Il m'arrive d'imaginer un débat, il y a plusieurs centaines de milliers
d'années, lorsque les premiers hominiens entreprirent de maîtriser le feu. D’un
côté un apprenti sorcier, qui joue avec des silex et des bouts de bois. De l’autre,
un sage écologiste, qui se soucie de la nature et de l’avenir : « Attention, s’écrie
ce dernier : avec le feu, on ne sait pas où l’on va ! On ne peut mesurer exactement
les risques : il y aura forcément des accidents, des incendies, peut-être des
milliers de morts. » Il y en eut bien davantage. Mais l’humanité a maîtrisé le
feu.

Ou un autre débat, il y a trois siècles, autour de la machine à vapeur. D’un
côté un apprenti sorcier, qui bricole sa marmite et ses pistons. De l’autre, un
sage soucieux d'environnement et de tradition : « Attention, s’alarme-t-il : avec
la machine à vapeur, nous entrons dans un domaine inconnu, avec des risques
que nous ne pouvons évaluer ! Cette nouvelle technologie peut bouleverser
toute notre économie, remettre en cause l'équilibre de nos campagnes et de nos
villes, menacer la forêt, épuiser nos réserves de charbon, modifier le climat. Il
peut y avoir des milliers de morts ! » Il y en eut bien davantage. Mais l’humanité
a fait la révolution industrielle.

Je ne dis pas cela contre les écologistes, pour qui j’ai souvent voté, pour qui
je voterai encore, mais contre un certain usage du principe de précaution, ou
plutôt de sa caricature, qui me paraît nous enfermer dans l’inaction ou le
conservatisme. Toute nouveauté présente un risque, qu’il est presque toujours
impossible de mesurer exactement. Le principe de précaution, s’il est bien compris,
n’impose pas qu’on renonce pour cela au progrès, mais simplement qu’on
y tende en essayant de prévenir ou de limiter les risques, fussent-ils seulement
possibles, que telle ou telle nouveauté peut entraîner. Il est toujours coupable
de ne rien faire contre un danger possible, voilà ce qu’indique bien clairement
le principe de précaution. Mais il n’en résulte pas qu’il soit toujours coupable
de faire quelque chose quand cela peut présenter un certain risque. Car alors on
ne ferait plus rien du tout, en tout cas plus rien de neuf : ce ne serait plus précaution
mais immobilisme.

Bref, le principe de précaution est un principe positif, qui {\it impose} quelque
chose (une action, contre un danger possible : « Dans le doute, fais quelque
chose pour limiter les risques »), non un principe négatif, qui {\it interdirait} une
action dès lors qu’elle peut entraîner un certain risque (sur le mode : « Dans le
doute, abstiens-toi »). Car autrement, le risque zéro n’existant pas, comme on
ne cesse à juste titre de le rappeler, le principe de précaution, sous sa forme abstentionniste,
nous vouerait à l’inaction — il y a toujours un doute : il faudrait
s'abstenir toujours ! — et donnerait tort à toute l’histoire humaine, qui n’a cessé
de prendre des risques et de les surmonter.

%— 460 —
%{\footnotesize XIX$^\text{e}$} siècle — {\it }
On m'objectera que la formulation du principe de précaution, telle que je
l’énonce, ne nous dit pas ce qu’il faut décider en matière d'OGM, d'énergie
nucléaire ou de transfusion sanguine (s’agissant aujourd’hui, par exemple, de la
maladie de Creutzfeldt-Jakob, pour laquelle les risques transfusionnels, en cette
année 2000, sont possibles mais non avérés). J’en suis d’accord, mais je n’y vois
pas une objection contre ce principe, ni contre ma formulation. Aucun principe
ne peut décider à notre place, et c’est heureux : le principe de précaution
peut éclairer une décision politique, il ne saurait en tenir lieu. C’est où l'on
retrouve la prudence, qui n’est pas un principe mais une vertu. Et la démocratie,
qui n’est pas une garantie mais une exigence.

Tout ce qu’on fait est dangereux, mais inégalement : il serait aussi imprudent
de ne rien faire que de faire n’importe quoi.

\section{Préconscient}
%PRÉCONSCIENT
L’une des trois instances de la première topique de
Freud.

Le préconscient n’est pas une espèce d’intermédiaire ou de sas, comme on
le croit parfois, entre le conscient et l'inconscient (ce qu’on appellerait mieux le
subconscient, concept non freudien). Il ne fait, avec le conscient, qu’un seul et
même système (le système Pcs-Cs), lequel est séparé de l'inconscient (le système
Ics) par le refoulement et la résistance. Qu'’est-ce que le préconscient ? C’est
l’ensemble de tout ce qui peut être conscient (sans avoir besoin pour cela de
vaincre les défenses de l'inconscient), mais qui ne l’est pas actuellement. Le
conscient n’est que sa pointe extrême et infime. Par exemple votre date de naissance,
le prénom de votre conjoint ou la couleur de vos yeux : selon toute vraisemblance,
aucune de ces trois données ne faisait partie, il y a vingt secondes,
du champ de votre conscience ; elles y sont maintenant (elles sont passées du
préconscient au conscient), sans avoir eu besoin pour cela de déjouer quelque
censure que ce soit. Tel est le préconscient : une immense, du moins à notre
échelle, et morne salle d’attente, avant et après le petit train de la conscience.
C’est l'inconscient, si l’on en croit Freud, qui a posé les rails. Mais c’est le
monde qui les porte, et qu’on voit par les fenêtres. {\it E pericoloso ma buono sporgersi.}

\section{Prédestination}
%PRÉDESTINATION
Un destin écrit à l’avance.

La prédestination est à peu près au destin, ou même à
la providence, ce que le prédéterminisme est au déterminisme : son anticipation
rétrospective et superstitieuse. Toutefois il est difficile, si l’on croit en un
Dieu omniscient et tout-puissant, d’y échapper. Dieu me donne l'être et la vie.
%— 461 —
%{\footnotesize XIX$^\text{e}$} siècle — {\it }
Il sait de toute éternité si je serai sauvé ou damné, et même (par la grâce) il en
décide : comment pourrais-je y échapper ? C’est la lumière noire de la foi, celle
de saint Augustin, de Calvin, de Pascal. Elle effraie nos croyants modernes,
presque tous pélagiens ou jésuites. Cela même, peut-être, était écrit.

\section{Prédéterminisme}
%PRÉDÉTERMINISME
Un déterminisme faraliste (ce qu’Épicure appelait
« le destin des physiciens ») : le passé serait cause du
présent, comme le présent de l’avenir, de sorte que tout, toujours, serait écrit à
l'avance. C’est un déterminisme dilaté dans le temps — un déterminisme obèse.
C'est aussi, à ce que je crois, un contresens sur la causalité. Si seul le présent
existe, lui seul (dont nous faisons partie) est cause et effet : comment le passé,
qui n'est plus, pourrait-il gouverner l'avenir, qui n’est pas ? Comment l’un ou
l’autre pourraient-ils commander le présent, qui est tout ? Le démon de Laplace
n’était qu’un mauvais rêve.

\section{Prédicat}
%PRÉDICAT
Tout ce qui est affirmé d’un sujet quelconque. Par exemple
dans les propositions {\it « Socrate est un homme »} où {\it « Socrate se
promène »} : «est un homme » et « se promène » sont des prédicats.

\section{Prédiction}
%PRÉDICTION
Cest dire à l'avance ce qui sera, quand on croit le savoir par
des voies mystérieuses ou surnaturelles (prédiction n’est pas
prévision). Le propre des prophètes et des imbéciles.

\section{Préjugé}
%PRÉJUGÉ
Ce qui a été jugé avant. Avant quoi ? Avant d’y avoir réfléchi
vraiment et comme il faut. C’est le nom classique et péjoratif de
l'opinion, spécialement chez Descartes, en tant qu’elle est préconçue.

La force des préjugés tient au fait que « nous avons tous été enfants avant
que d’être hommes », et avons commencé à penser bien avant de savoir raisonner
(si tant est que nous sachions). Le remède, selon Descartes, est dans le
doute et la méthode. Force est pourtant de constater que le cartésianisme donnera
raison, pour finir, à la plupart des préjugés de son époque. Pour être
homme, et même grand homme, on n’en est pas moins enfant.

\section{Préméditation}
%PRÉMÉDITATION
Une volonté anticipée : c’est vouloir à l'avance, ou
plutôt (car il n’est de volonté qu’actuelle) c’est projeter
%— 462 —
%{\footnotesize XIX$^\text{e}$} siècle — {\it }
et continuer de vouloir. On considère d'ordinaire, spécialement dans les tribunaux,
qu’une volonté ainsi préparée est plus grave, lorsqu'elle est coupable,
qu’une autre. C’est qu’elle ôte l’excuse possible de la colère ou de l’irréflexion.

\section{Prémisse}
%PRÉMISSE
Une proposition considérée comme première (par rapport à ses
conséquences), et spécialement les deux premières propositions
— la majeure et la mineure — d’un syllogisme.

\section{Prénotion}
%PRÉNOTION
L'une des deux traductions usuelles (avec « anticipation ») du
grec {\it prolèpsis}. C’est un autre nom pour désigner les idées
générales qui résultent de la répétition d’expériences à peu près identiques, ou
dont on ne retient que ce qui l’est. Par exemple l’idée d’arbre, pour Épicure, est
une prénotion : elle n’existe, comme idée, que parce que j'ai perçu plusieurs
arbres différents, dont je n’ai retenu que ce qu'ils avaient de commun. Elle est
donc postérieure à l'expérience. Pourquoi l’appelle-t-on {\it pré}notion ? Parce qu'il
faut en disposer d’abord pour pouvoir reconnaître, voyant un arbre, que c'en
est un. C’est le contraire d’une idée {\it a priori} : elle ne précède une expérience
donnée que pour autant qu’elle résulte d’autres, qui la précèdent. Ainsi c’est
l'expérience qui est à la base de tout, y compris des idées qui la pensent.

Le mot prendra plus tard, spécialement chez Bacon et Durkheim, un sens
péjoratif : la prénotion serait une idée préconçue, antérieure à toute réflexion
ou à toute enquête scientifique, et risquant par là d’en détourner. S’oppose en
ce sens à concept. Mais les concepts d’une époque deviennent très vite les prénotions
d’une autre. On n’en a jamais fini de penser, ni de se libérer de ses
idées.

Ainsi la prénotion est une idée avec laquelle (chez Épicure et les stoïciens)
ou contre laquelle (chez Bacon ou Durkheim) on pense : c’est un outil ou un
obstacle, et parfois les deux.

\section{Présage}
%PRÉSAGE
Le signe présent de quelque événement à venir. Ce n’est souvent
que superstition. Toutefois il arrive qu’une observation un peu
attentive et régulière débouche sur un répertoire plus ou moins fiable de signes,
associés à autant de prévisions. Ainsi en médecine, en météorologie, en économie...
Que ces gros nuages noirs, là-bas, laissent présager quelque orage, ce
n’est ni magie ni même prédiction. Ce n’est pas l’avenir que l’on voit, c’est le
présent, dont on a appris à prévoir, parfois, les effets ou les suites. Mais on ne
parle de présage, en règle générale, que pour les signes mystérieux ou irrationnels.
%— 463 —
%{\footnotesize XIX$^\text{e}$} siècle — {\it }
Ceux-là mentent presque toujours. Même quand il arrive qu’ils tombent
juste, le mieux est de les oublier aussitôt.

\section{Présent}
%PRÉSENT
Ce qui sépare le passé de l’avenir. Mais si le passé et l’avenir ne
sont rien, rien ne les sépare. Il n’y a plus que l'éternité, qui est le
présent même. Entre rien et rien : tout.

C’est le lieu de coïncidence du réel et du vrai, qui aussitôt se séparent (le
passé reste vrai, qui n'est plus réel) sans se perdre (puisque la vérité reste présente).
C’est peut-être l’espace lui-même, où l'univers éternellement {\it se présente.}

Être présent, c’est être ou devenir. Le présent dure, c’est-à-dire continue
d’être présent, sans cesser de changer. C’est pourquoi il y a du temps, que nous
pouvons, par la pensée, indéfiniment diviser entre passé et avenir. Le présent,
pour la pensée, est ce qui les sépare. Mais ce présent abstrait n’est alors qu’un
instant sans épaisseur : non une durée, disait Aristote, mais la limite entre deux
durées. Le réel n’en continue pas moins sans limites, et c’est le présent même :
la continuation indivisible et illimitée de tout.

On remarquera que la mémoire et l’imagination en font partie. Vivre au
présent, comme disaient les stoïciens, comme disent tous les sages, ce n’est
donc pas vivre dans l'instant. Qui peut aimer sans se souvenir de ceux qu'il
aime ? Penser, sans se souvenir de ses idées ? Agir, sans se souvenir de ses désirs,
de ses projets, de ses rêves ? Non qu’il y ait pour cela autre chose que le présent.
Qui peut aimer, penser ou agir au passé ou au futur ? Vivre au présent, c’est
simplement vivre en vérité : c’est la vie éternelle, et il n’y en a pas d’autre.
Seules nos illusions nous en séparent, ou plutôt seules nos illusions (qui en font
partie) nous donnent le sentiment d’en être séparés. « Tant que tu fais une différence
entre le nirvâna et le samsâra, disait Nâgârjuna, tu es dans le samsâra. »
Tant que tu fais une différence entre le temps et l'éternité, tu es dans le temps.
Le présent, qui est leur vérité conjointe, ou leur conjonction vraie, est donc
l'unique lieu du salut. Nous sommes déjà dans le Royaume : l'éternité, c’est
maintenant.

\section{Prêtre}
%PRÊTRE
C'est une espèce de fonctionnaire, qui servirait l’Église plutôt que
l'État, ou Dieu plutôt que la nation. Ministre du culte, donc,
plutôt que de la Cité. Il serait absurde de les juger en bloc. Même Voltaire, qui
les combat, s’y refuse. Un prêtre doit être « le médecin des âmes », écrit-il, mais
tous ne se valent pas. « Quand un prêtre dit : “Adorez Dieu, soyez juste, indulgent,
compatissant”, c’est alors un très bon médecin. Quand il dit : “Croyez-moi,
ou vous serez brûlé”, c’est un assassin. »

%— 464 —
%{\footnotesize XIX$^\text{e}$} siècle — {\it }
\section{Preuve}
%PREUVE
Un fait ou une pensée, qui suffit à attester la vérité d’un autre fait
ou d’une autre pensée. Toutefois la preuve la plus solide ne vaut
que ce que vaut l’esprit qui s’en sert. Il faudrait donc prouver d’abord la valeur
de l'esprit, et c’est ce qu’on ne peut sans tomber dans un cercle. Ainsi il n’y a
pas de preuve absolue. Il n’y a que des expériences ou des démonstrations qui
mettent fin au doute. C’est ce qu’on appelle une preuve, ou une autre façon de
la définir : une preuve est une pensée ou un fait qui rend le doute, sur une
question donnée, impossible, sauf à douter de tout. Par quoi la logique est sans
force contre le scepticisme, comme le scepticisme contre la logique.

\section{Prévision}
%PRÉVISION
C’est voir avant. Parce qu’on verrait l'avenir ? Bien sûr que non
(comment voir ce qui n’est pas ?) ; mais parce qu'on voit ses
signes ou ses causes, qui font partie du présent, et qu’on interprète. À ne pas
confondre avec l’espérance. Le météorologue qui annonce une tempête, cela ne
veut pas dire qu’il espère. Le touriste qui espère le beau temps, cela ne veut pas
dire qu’il le prévoit. L’espérance est fondée sur un désir ; la prévision, sur une
connaissance. Celle-ci est-elle fiable ? Cela dépend des domaines considérés. La
meilleure prévision n’est pas forcément la plus certaine ; c’est celle qui évalue sa
propre marge d'incertitude, jusqu’à prévoir les moyens, si nécessaire, de surmonter
son propre échec. C’est où connaissance et volonté se rejoignent : ce
n’est plus prévision mais stratégie.

\section{Prière}
%PRIÈRE
C’est parler à Dieu, le plus souvent pour lui demander quelque
chose. Mais à quoi bon lui parler, s’il sait déjà tout ? Et pourquoi
demander, s’il sait mieux que nous ce qu’il nous faut ? Le silence serait plus
digne, et aussi efficace.
On dira qu’on parle aussi, en amour, pour dire à quelqu'un ce qu’il sait
déjà et qu’il se plaît à entendre... Mais c’est qu’il a besoin d’être rassuré, conforté,
cajolé.... Ce n’est pas prière mais caresse. Et qui oserait caresser Dieu ?

\section{Primat/primauté}
%PRIMAT/PRIMAUTÉ
Les deux mots sont souvent synonymes : ils indiquent
ce qui vient au premier rang, ce qui est le plus
important ou a le plus de valeur. C’est pourquoi j'ai pris habitude de les distinguer.
La notion de « premier rang » n’a de sens qu’en fonction d’un ordre
que l’on suit : s’il arrive que les premiers soient les derniers, comme il est dit
dans les Écritures, c’est moins souvent du fait d’un bouleversement interne de
la série (un pauvre qui fait fortune, un riche qui se ruine) qu’en raison d’un
%— 465 —
%{\footnotesize XIX$^\text{e}$} siècle — {\it }
changement, entre les deux hiérarchies, de point de vue. Cela vaut spécialement
quand on passe de la théorie à la pratique, de la vérité à la valeur, de
l’ordre des causes à celui des fins. Ce qui est le plus important, du point de vue
de l’être ou de la connaissance, n’est pas forcément — et même, me semble-t-il,
n'est jamais — ce qui vaut le plus, du point de vue du sujet ou du jugement.
Être matérialiste, par exemple, c’est affirmer le {\it primat de la matière}. Mais en
tant que le matérialisme est une philosophie, il ne peut renoncer à la {\it primauté
de la pensée ou de l'esprit}. Marx, pour avoir affirmé le primat de l’économie, ne
s’est pas cru tenu de soumettre sa vie à l’argent : il n’était, que je sache, ni banquier
ni vénal. Et Freud, pour avoir soutenu le primat de la sexualité et de
l'inconscient, n’a pas davantage décidé de leur soumettre son existence : il
n'était ni idiot ni débauché, et mettait l’art et la lucidité plus haut que le sexe
ou les actes manqués. Au reste, c’est ce qu’expriment assez les notions d’idéologie,
chez le premier, et de sublimation, chez le second. Que l'idéologie soit,
comme le disait Marx, une {\it camera obscura}, qui donne une image inversée de la
réalité (ce qui est en haut paraît en bas, ce qui est en bas paraît en haut, comme
dans la chambre noire des anciens appareils photographiques), ce n’est bien sûr
pas une raison pour prétendre se passer d’idéologie. Et si toute valeur résulte
d’une sublimation, comme le veut Freud, on aurait assurément tort de
renoncer pour cela à l’art, à la morale ou à la politique. La notion de primat ou
de primauté doit donc être scindée en deux concepts différents, et même
opposés, qui peuvent être définis de la façon suivante : j’entends par {\it primat} ce
qui est objectivement le plus important, dans un enchaînement descendant de
déterminations ; par {\it primauté}, ce qui vaut le plus, subjectivement, dans une
hiérarchie ascendante d'évaluations. Le concept de {\it prima}t est ontologique ou
explicatif : c’est l’ordre des causes et de la connaissance, qui tend au plus profond
ou au plus fondamental ; celui de {\it primauté} est normatif ou pratique : c’est
l’ordre des valeurs et des fins, qui tend au meilleur ou au plus élevé. Le premier
sert à comprendre ; le second, à juger et à agir.

La différence, entre ces deux points de vue, est essentielle au matérialisme
philosophique. Être matérialiste, Comte a raison sur ce point, c’est expliquer le
supérieur par l’inférieur. Mais ce n’est pas pour autant renoncer à la supériorité
de celui-là, ni se mettre à adorer celui-ci. Que la pensée, par exemple,
s'explique par le cerveau, cela ne saurait justifier qu’on renonce à penser, ni
qu’on soumette toute vérité à la neurobiologie (car alors la neurobiologie elle-même
deviendrait impossible ou impensable : une idée fausse n’est pas moins
réelle, dans le cerveau, qu’une idée vraie). Que la vie s'explique par la matière
inanimée, ce n’est pas davantage une raison pour renoncer à vivre, ni pour soumettre
toute vie à la physique (car alors la notion de bioéthique perdrait son
sens, et la physique elle-même serait sans valeur : un imbécile n’est pas moins
%— 466 —
%{\footnotesize XIX$^\text{e}$} siècle — {\it }
matériel qu’un physicien). Il faut donc distinguer ce qui relève de la connaissance
ou des causes, d’une part, et ce qui relève du jugement de valeur ou de
l’action d’autre part. Cela définit deux points de vue différents : l’un théorique,
qui tend à l’objectivité ; l’autre pratique, qui doit s’assumer comme subjectif.
Ce qui est le plus important dans le premier (la matière, l’enchaînement des
causes ou des déterminations) n’est jamais ce qui vaut le plus dans le second
(l'esprit, la hiérarchie des finalités ou des valeurs), et inversement. C’est ce qui
m'a permis, dans {\it Le mythe d'Icare}, de penser le matérialisme comme essentiellement
{\it ascendant} : du {\it primat} (de la matière, de la nature, de l’économie, de la
force, de la sexualité, de l'inconscient, du corps, du réel, du monde...) à la {\it primauté}
(de la pensée, de la culture, de la politique, du droit, de l'amour, de la
conscience, de l’âme, de l’idéal, du sens....). Si l’on renonce au primat, on n’est
plus matérialiste ; si l’on renonce à la primauté, on n’est plus philosophe : on
ne défend plus qu’un matérialisme vulgaire ou avachi. Autant se mettre à
genoux, dans un cas, ou se coucher, dans l’autre.

Parce que ces deux logiques, celle du primat et celle de la primauté, s’opposent,
on a affaire à une dialectique. Reste à penser — pour qu’il y ait dialectique
et non incohérence — leur articulation. Comment passe-t-on de l’un de ces
deux points de vue à l’autre ? Entre le primat et la primauté, quoi ? Le mouvement
ascendant du désir, qui nous fait passer de l’un (primat du primat : tout
part du corps, y compris le désir lui-même) à l’autre (primauté de la primauté :
rien ne vaut, y compris le corps, que pour l'esprit). Par exemple chez Épicure :
«le plaisir du ventre, disait-il, est le fondement de tout bien » (c’est à lui que se
ramènent, selon l’ordre des causes, «les biens spirituels et les valeurs
supérieures ») ; mais les plaisirs de l'âme (l'amitié, la philosophie, la sagesse) lui
sont pourtant supérieurs : c’est d’eux, bien plus que des plaisirs corporels, que
dépendent notre dignité, notre liberté, notre bonheur. Par exemple chez Marx :
l’économie est déterminante « en dernière instance » ; mais les hommes n’agissent
qu’en fonction de la représentation idéologique qu’ils ont de la société et
d'eux-mêmes (un militant qui se bat pour le communisme, ce n’est pas la
même chose qu’un nervi qui se vend au plus offrant). Par exemple chez Freud :
la psychanalyse, comme théorie, enseigne le primat de l'inconscient et de la
sexualité, mais apprend, en pratique, à s’en libérer, au moins partiellement, au
nom de valeurs supérieures (qui résultent elles-mêmes de la sexualité, par la
sublimation, mais la jugent, par le surmoi, et tendent à s’en affranchir : aucune
cure ne peut réussir, insiste Freud, sans « l’amour de la vérité », qui peut certes
s'expliquer par l'inconscient mais vise à augmenter, autant que faire se peut,
notre part de conscience et de liberté). Bref, tout part du corps, de l’économie
ou du {\it ça}, mais c’est l'esprit, l'idéologie ou l'éducation qui fixent les valeurs
supérieures, vers lesquelles nous tendons, au moins consciemment, et qui peuvent
%— 467 —
%{\footnotesize XIX$^\text{e}$} siècle — {\it }
seules donner un sens, fât-il relatif et provisoire, à notre vie. C’est cette
{\it ascension} que le mythe d’Icare, dans mon premier livre, m’a paru pouvoir
symboliser : Icare, prisonnier d’un labyrinthe horizontal (objectivement, tout
se vaut et ne vaut rien), mais créant lui-même (primat de l’action, primauté de
l'œuvre), guidé par son père (tout désir est biographiquement déterminé :
primat de l’histoire, primauté de la fidélité), les ailes de son désir et de son vol.
« À l'encontre de la philosophie idéaliste allemande, qui descend du ciel sur la
terre, disaient Marx et Engels, {\it c'est de la terre au ciel que l'on monte ici} » ({\it L'idéologie
allemande}, 1). Encore faut-il {\it monter}, en effet : passer de la terre au ciel,
disons du {\it primat} à la {\it primauté} — du désir (comme puissance) au désirable
(comme valeur).

Ce désir est un effort ({\it conatus}), une tension, un {\it acte}. On ne passe du primat
à la primauté qu’à la condition de le vouloir. Qu'un instant l'effort se relâche,
que le désir se fatigue ou s’oublie, on n’a plus qu’un matérialisme plat ou veule
— qu'un matérialisme qui redescend, et qui ne saurait par conséquent (la sagesse
est un idéal) être philosophique.

Cette dialectique, qui vaut à l’échelle du monde, vaut aussi, plus spécialement,
à celle de la société. On sait, je m’en suis expliqué ailleurs, que j’y distingue
quatre ordres différents : l’ordre techno-scientifique (qui inclut l’économie),
l’ordre juridico-politique, l’ordre de la morale, enfin l’ordre de
l'éthique ou de l'amour. Chacun de ces ordres a sa logique et sa cohérence
propres ; mais il reste conditionné par l’ordre qui le précède, et il n’est limité et
jugé, de l'extérieur, que par l’ordre qui le suit. Aussi peut-on les disposer, sous
forme d’une topique, de bas en haut. On y retrouvera alors la dialectique du
primat et de la primauté, non suivant une simple hiérarchie ascendante, ce
serait trop simple, mais selon un double mouvement, montant et descendant,
qui forme comme deux hiérarchies croisées : ce qui vaut le plus, subjectivement,
pour l'individu, n’est jamais ce qui est le plus important, objectivement,
pour le groupe; et inversement. La hiérarchie ascendante des primautés
s'ajoute ainsi, mais sans l’annuler, à l’enchaînement descendant des primats.
Chacun de ces ordres est en effet déterminant pour l’ordre immédiatement
supérieur, dont il crée les conditions de possibilité, mais aussi régulateur pour
l’ordre immédiatement inférieur, dont il fixe les limites et auquel il essaie de
donner une orientation ou un sens. On est donc contraint, pour expliquer un
ordre, de prendre en compte les ordres inférieurs ; mais on ne peut le juger
qu’en faisant référence aux ordres supérieurs. C’est ainsi qu’on doit énoncer à
la fois (mais de deux points de vue différents) le primat de l’économie et la primauté
de la politique, entre les ordres 1 et 2 ; le primat de la politique et la primauté
de la morale, entre les ordres 2 et 3 ; enfin le primat de la morale et la
primauté de l’amour, entre les ordres 3 et 4. Toute confusion, entre ces ordres,
%— 468 —
%{\footnotesize XIX$^\text{e}$} siècle — {\it }
est ridicule, comme dirait Pascal, ou tyrannique. Mais elle peut l’être de deux
façons différentes : en soumettant un ordre donné, avec ses valeurs propres, à
un ordre inférieur (c’est ce que j'appelle la barbarie, qui renonce à la primauté),
ou en prétendant annuler ou déstructurer un ordre donné, avec ses contraintes
propres, au nom d’un ordre supérieur (c’est ce que j'appelle l’angélisme, qui
oublie le primat).

Contre l’angélisme, quoi ? La lucidité.

Contre la barbarie ? L'amour et le courage.

Par quoi la dialectique du primat et de la primauté débouche sur une
éthique, qui est à la fois intellectuelle (primauté, pour la pensée, de la lucidité :
l'amour de la vérité doit l'emporter, intellectuellement, sur tout) et pratique
(primauté, pour l’action, de ce qu’on aime et veut : l’amour et la liberté, pour
presque tous, seront les valeurs suprêmes). Cela n’autorise pas à soumettre la
pensée à l’amour ou à la liberté (ce ne serait plus philosophie mais sophistique,
fût-elle généreuse et démocratique: c’est ce qu’on appelle aujourd’hui le
« politiquement correct »), ni l’amour ou la volonté à la connaissance (ce ne
serait plus morale ou éthique mais scientisme ou dogmatisme). On ne vote pas
sur une vérité : ce serait de l’angélisme démocratique. Ni sur le bien et le mal :
ce serait de la barbarie démocratique. Mais la vérité ou la morale ne sauraient
pas davantage dire la loi, qui ne relève, dans une démocratie, que du peuple
souverain. C’est ce qu’on appelle la laïcité : la démocratie ne tient pas lieu de
conscience, ni la conscience, de démocratie.

La distinction des ordres débouche ainsi, philosophiquement, sur ce que
j'ai appelé le {\it cynisme}. De quoi s'agit-il ? D’une certaine façon de penser le rapport
entre la valeur et la vérité, sans les confondre (la valeur est sans vérité
objective, la vérité sans valeur intrinsèque), mais sans non plus renoncer à l’une
ou à l’autre. Le cynique, au sens philosophique du terme, c’est celui qui refuse
de prendre ses désirs pour la réalité, voyez Machiavel, mais aussi de céder sur la
réalité de ses désirs, voyez Diogène (et aussi Machiavel : c’est ce qu’il appelle la
{\it virt\'{u}}). Reste à penser, entre ces deux versants du cynisme, une articulation :
c’est ce que fait Spinoza, ou du moins ce qu’il autorise. Ce n’est pas parce que
la vérité est bonne qu’elle est connaissable ou aimable ; c’est parce qu'elle est
vraie que nous pouvons la connaître et, si nous en sommes capables, l'aimer. À
l'inverse, ce n’est pas parce que le bien est vrai qu’il faut l’aimer ou le faire, c’est
parce que nous l’aimons ou le voulons qu’il devient, pour nous, une valeur
(laquelle ne fonctionne comme vérité que pour et par un sujet). Par exemple
l’égoïsme est au moins aussi vrai que la générosité, et même, quant aux faits, il
l’est bien davantage (il explique un plus grand nombre de comportements, et
peut-être la générosité elle-même) ; aussi faut-il le connaître et le comprendre.
Mais cela n'empêche pas que la générosité, moralement, lui soit supérieure.

%— 469 —
%{\footnotesize XIX$^\text{e}$} siècle — {\it }
Distinction des ordres : la logique des primats explique celle des primautés,
pour la connaissance, mais ne saurait l’annuler, pour l’action, pas plus que
celle-ci ne saurait, pour la pensée, tenir lieu de celle-là. Il en résulte qu’on ne
peut jamais tout à fait vivre ce qu’on sait, ni savoir ce qu'on vit. « Nous
sommes, je ne sais comment, doubles en nous-mêmes », disait Montaigne
({\it Essais}, II, 16; voir aussi mon article sur « Montaigne cynique », {\it Valeur et
vérité}, p.55 à 104). Mais c’est Lévi-Strauss, à propos de Montaigne (et cette
rencontre n’est ni de hasard ni de faible poids), qui a su désigner le plus nettement
cet écart en nous à la fois nécessaire et tragique :

{\footnotesize « La connaissance et l’action sont à jamais placées dans une situation fausse : prises
entre deux systèmes de référence mutuellement exclusifs et qui s'imposent à elles, bien
que la confiance même temporaire faite à l’un détruise la validité de l’autre. Il nous faut
pourtant les apprivoiser pour qu’ils cohabitent en chacun de nous sans trop de drames.
La vie est courte : c’est l'affaire d’un peu de patience. Le sage trouve son hygiène intellectuelle
et morale dans la gestion lucide de cette schizophrénie » ({\it Histoire de Lynx},
chap. XVIII, « En relisant Montaigne », p. 288).
}

Schizophrénie ? Ce n’est pas le mot que j’utiliserais, puisque cet écart
résulte non d’une quelconque pathologie mais d’une dualité constitutive de
l'être humain, qui est celle du désir et de la raison, et puisque cette dualité, qui
est en l'occurrence structurante bien plus que destructrice, peut être et pensée
et voulue (elle est à la fois rationnelle et raisonnable) : on peut expliquer le désir
(@ quoi bon, autrement, les sciences humaines ?) et désirer la raison (à quoi
bon, autrement, la philosophie ?).

L'essentiel, pour résumer, tient donc bien dans la formule que j’utilisais en
commençant : primat de la matière et primauté de l'esprit. Mais cela, s’agissant
de l'être humain, doit être précisé en une autre, plus lourde mais plus explicite :
primat du corps et du désir, primauté (théorique) de la raison et (pratique) de
l’amour et de la liberté. C’est le désir, disait Spinoza, non la raison ou l’amour,
qui est l'essence de l’homme ({\it Éthique}, III). Mais c’est la raison qui libère, et
l'amour qui sauve ({\it Éthique}, IV et V). Les sciences de la nature et de l’homme,
pour le dire avec nos mots d’aujourd’hui, nous en apprennent davantage sur
nous-mêmes que la morale, qu’elles expliquent et qui ne les explique pas. Mais
connaître la vie ou l’humanité n’a jamais suffi à les aimer, ni même à aimer la
connaissance. À la gloire du spinozisme. Ce n’est pas parce qu’une chose est
bonne que nous l’aimons ou la désirons, c’est inversement parce que nous
l’aimons ou la désirons que nous la jugeons bonne ({\it Éthique}, III, 9, scolie : voir
aussi, chez Freud, les deux dernières pages de {\it Malaise dans la civilisation}). Si
vous n’aimez pas l’amour et la vérité, n’en dégoûtez pas les autres. Mais si vous
%— 470 —
%{\footnotesize XIX$^\text{e}$} siècle — {\it }
aimez les deux, comme il convient, ne vous croyez pas autorisé par là à les
confondre.

\section{Principe}
%PRINCIPE
Un commencement théorique : le point de départ d’un raisonnement.
Il est de la nature d’un principe d’être indémontrable
(sans quoi ce ne serait plus un principe mais un théorème ou une loi), comme
il est de la nature de la démonstration de requérir quelque principe indémontré.
La différence avec un axiome ou un postulat ? C’est que ceux-ci ne se
disent guère que de systèmes formels ou hypothético-déductifs. Un principe se
dit aussi bien dans les sciences expérimentales, en morale ou en politique. Reste
à savoir pourquoi poser tel principe plutôt que tel autre. C’est parfois qu’on ne
peut faire autrement (le principe de non-contradiction) ou qu’on y voit une
espèce d’évidence. D’autres fois qu’on en a besoin pour agir ou vivre d’une
façon qui nous paraisse humainement acceptable. Les principes de la morale
sont de ce type : nullement évidents ni logiquement nécessaires, mais subjectivement
indispensables.

\section{Privation}
%PRIVATION
L'absence de quelque chose qu’on devrait avoir : l’aveugle,
non la pierre, est privé de la vue.

\section{Probabilité}
%PROBABILITÉ
Un degré de possibilité, en tant qu’il peut faire l’objet d’un
calcul ou d’une prévision. Se dit surtout, dans le langage
courant, quand ce degré est élevé. Mais parler de {\it probabilité infime} n’est pas
pour autant contradictoire. Par exemple si vous jouez au loto : il est très improbable
que vous gagniez, mais cette probabilité, qui peut se calculer très précisément,
n’est pas nulle. Elle est simplement trop faible, même par rapport à
l'enjeu, pour que le seul calcul des probabilités puisse justifier votre mise. Heureusement,
pour le Trésor public, que le désir s’en mêle.

\section{Problématique}
%PROBLÉMATIQUE
Comme adjectif, qualifie ce qui n’est ni réel (ou,
s'agissant d’un jugement, assertorique) ni nécessaire
(ou apodictique). « Les jugements sont {\it problématiques}, écrit Kant, lorsqu'on
admet l’affirmation ou la négation comme simplement {\it possibles} (il y a choix),
{\it assertoriques} quand on les y considère comme {\it réelles} (vraies), {\it apodictiques} quand
on les y regarde comme {\it nécessaires} » ({\it C. R. Pure}, Analytique des concepts, I, 2,
\S 9).

%— 471 —
%{\footnotesize XIX$^\text{e}$} siècle — {\it }
Comme substantif, le mot désigne l'élaboration d’un problème. Construire
une problématique, c’est expliquer {\it comment} un problème se pose, ou comment
on a décidé de le poser, afin d’avoir une chance, peut-être, de le résoudre. Dans
une dissertation philosophique, la problématique doit idéalement apparaître
dès la fin de l’introduction ; elle prend communément la forme d’un système
ordonné de questions.

\section{Problème}
%PROBLÈME
Une difficulté à résoudre. Une question ? C’est ordinairement
la forme que prend pour nous un problème, ou plutôt que
nous lui donnons. D’où ce passage fameux de Bachelard : « Avant tout, il faut
savoir poser des problèmes. Et, quoi qu’on dise, dans la vie scientifique, les problèmes
ne se posent pas d’eux-mêmes. C’est précisément ce {\it sens du problème}
qui donne la marque du véritable esprit scientifique. Pour un esprit scientifique,
toute connaissance est une réponse à une question. S’il n’y a pas eu de
question, il ne peut y avoir connaissance scientifique. Rien ne va de soi. Rien
n'est donné. Tout est construit » ({\it La formation de l'esprit scientifique}, I). Cela
vaut aussi en philosophie, et sans doute pour toute pensée digne de ce nom.
Mais toute question n’est pas un problème. Par exemple quelqu'un vous
demande l'heure : c’est une question, pas un problème. Si vous lui demandez :
« Pourquoi voulez-vous savoir l’heure qu’il est ?», vous transformez — en
l'occurrence indûment — sa question en problème. Il pourrait vous le
reprocher : « Pourquoi faites-vous un problème de ma question ? » En philosophie,
c’est différent. Seuls les problèmes comptent, qu’il faut poser avant de les
résoudre. Qu'est-ce que poser un problème ? C’est expliquer {\it pourquoi} une
question se pose, et {\it doit} se poser, non à tel ou tel individu, mais pour tout
esprit raisonnable fini, doué d’une culture au moins minimum. Tel est le but
de l'introduction, dans une dissertation philosophique : il s’agit de passer de la
contingence d’une question à la nécessité d’un problème, avant d’élaborer, si
possible, une problématique.

\section{Prochain}
%PROCHAIN
Autrui, tel qu’il se donne dans la rencontre. Il a droit à plus
qu’à du respect. À quoi bon, autrement, la rencontre ?

\section{Profondeur}
%PROFONDEUR
La distance entre la surface et le fond. En philosophie,
c’est surtout une métaphore, pour indiquer la quantité de
pensée qu’un discours peut contenir ou susciter. Même Nietzsche, si amoureux
de la surface et de la belle apparence, y voyait légitimement une vertu intellectuelle.
%— 471 —
%{\footnotesize XIX$^\text{e}$} siècle — {\it }
C’est que la superficialité elle-même n’a de sens qu’au service de la profondeur.
Ainsi, chez les Grecs : « Ah ! ces Grecs, comme ils savaient vivre ! Cela
demande la résolution de rester bravement à la surface, de s’en tenir à la draperie,
à l’épiderme, d’adorer l'apparence et de croire à la forme, aux sons, aux
mots, à tout l’Olympe de l'apparence ! Ces Grecs étaient superficiels... par
profondeur ! » ({\it Le gai savoir}, Avant-Propos). Combien, aujourd’hui, faisant le
chemin inverse, voudraient sembler profonds à force de superficialité ?

D’autres voudraient obtenir le même résultat à force d’obscurité. Nietzsche,
dont ils se réclament parfois, leur a pourtant donné tort. Il préférait « la belle
clarté française », celle de Pascal ou de Voltaire. {\it Être profond}, explique-t-il, n’est
pas la même chose que {\it sembler profond} : « Celui qui se sait profond s’efforce
être clair; celui qui aimerait sembler profond à la foule s'efforce d’être
obscur. Car la foule croit profond tout ce dont elle ne peut voir le fond. Elle a
si peur ! elle aime si peu aller dans l’eau ! » ({\it op. cit}, III, 173). Ainsi la superficialité
n’est bonne qu’à condition d’être profonde ; et la profondeur, qu'à
condition d’être claire.

\section{Progrès}
%PROGRÈS
Dans mon premier livre, {\it Le mythe d'Icare}, j'insistais sur la relativité
de l’idée de progrès. Marcel Conche, qui avait bien voulu
lire le manuscrit, inscrivit simplement en marge : « Quel progrès, pourtant, que
la Sécurité sociale ! » Il avait évidemment raison. Un progrès relatif reste un
progrès, et il n’y en a pas d’autre.

Qu'est-ce que le progrès ? Un changement vers le mieux. Notion normative,
donc subjective. Il n’y a guère que dans les sciences que le progrès, pour
relatif qu’il demeure, soit incontestable : parce que la science d’aujourd’hui
peut rendre compte de celle des siècles passés, quand la réciproque n'est pas
vraie. C’est ce qui fait que l’histoire des sciences est «une histoire jugée »,
comme disait Bachelard, et jugée par son progrès même, lequel est « démontrable
et démontré » : c’est « une histoire récurrente, une histoire qu’on éclaire
par la finalité du présent, une histoire qui part des certitudes du présent et
découvre, dans le passé, les formations progressives de la vérité » ({\it L'activité
rationaliste de la physique contemporaine}, chap. I). En politique, il en va
autrement : juger le passé au nom du présent n’est pas plus légitime (ce qui ne
veut pas dire que ce soit évitable) qu’il le serait de juger le présent au nom du
passé. Relativisme sans appel, donc, comme dit Lévi-Strauss, et c’est aussi vrai
dans le temps que dans l’espace: non qu’on ne puisse juger (on le peut,
puisqu'on le fait, et qu’on le doit), mais parce qu’on ne peut le faire objectivement
ou absolument. Un réactionnaire, par exemple, n’est pas quelqu'un qui
est contre le progrès, comme le croient naïvement les progressistes, mais
%— 473 —
%{\footnotesize XIX$^\text{e}$} siècle — {\it }
quelqu'un qui juge que c’en serait un, voire le seul possible, que de revenir à
telle ou telle situation antérieure. Et comment lui démontrer qu’il a tort ? On
connaît des cas, notamment en médecine, où une {\it involution} serait seule favorable.
Guérir, c’est le plus souvent revenir à la situation antérieure, ou s’en rapprocher.
Et qui ne souhaiterait rajeunir ? Ce n’est pas une raison pour souhaiter
retomber en enfance, ni, encore moins, pour revenir à l’Ancien régime. Le progrès
(social, politique, économique...) n’est ni linéaire ni absolu. Il n’est progrès,
même, que relativement à certains désirs qui sont les nôtres (de bien-être,
de justice, de liberté...). Cela, qui lui interdit de prétendre à l'absolu, ne
l’'annule pas ; c’est au contraire ce qui fait sa réalité, pour ceux — presque tous
— qui partagent ces désirs et constatent, malgré tant d’horreurs qui demeurent,
quelques avancées. Le progrès n’est pas une providence ; c’est une histoire, et
un point de vue sur cette histoire. C’est où l’on retrouve la Sécurité sociale, les
Lumières, les droits de l’homme, et même l'enthousiasme, mais guéri d’utopie,
de notre jeunesse : {\it Ce n'est qu'un début, continuons le combat !}

\section{Progressiste}
%PROGRESSISTE
Ce n’est pas quelqu'un qui est pour le progrès (personne
n’est contre), mais quelqu'un qui pense que le progrès —
social, politique, économique — est la tendance normale de l’histoire : que le
présent est globalement supérieur au passé, comme l'avenir, sauf catastrophe,
sera supérieur au présent. Aussi veut-il aller de l’avant : c’est ce qu’on appelle
« avoir des idées avancées ». Le progressisme est un optimisme, sans doute le
plus légitime qui soit : le progrès des sociétés est plus probable, et mieux avéré,
que celui des individus ou des civilisations.

\section{Projet}
%PROJET
Un désir présent, portant sur l’avenir en tant qu’il dépend de
nous. Ce n’est pas encore une volonté (vouloir c’est faire), ou
plutôt ce n’est que la volonté (actuelle) de vouloir (plus tard). Y voir la source
d’une liberté absolue, comme le fait Sartre, c’est oublier qu’un projet, en tant
qu'il est actuel, est aussi réel — donc aussi nécessaire — que le reste.

\section{Prophète}
%PROPHÈTE
Celui qui parle {\it (phanai)} à la place de ou en avance {\it (pro)}. Cela
ressemble à une maladie. Les croyants y voient un miracle.
«Il faut convenir que c’est un méchant métier que celui de prophète », écrit
Voltaire. Le succès est rien moins qu’assuré. Cet homme qui parle au nom de
Dieu, est-ce un prophète ou un fou ? Un visionnaire ou un fanatique ? Et comment
savoir s’il dit vrai? « La prophétie est un art difficile, dira plus tard
%— 474 —
%{\footnotesize XIX$^\text{e}$} siècle — {\it }
Woody Allen, surtout lorsqu'elle porte sur le futur. » Mieux vaut s'occuper du
présent, et préparer l'avenir plutôt que le prophétiser.

\section{Proposition}
%PROPOSITION
Un énoncé élémentaire, en tant qu’il peut être vrai ou
faux. « Tout discours n’est donc pas une proposition, souligne
Aristote, mais seulement le discours dans lequel réside le vrai ou le faux,
ce qui n'arrive pas dans tous les cas : ainsi la prière est un discours, mais elle
n'est ni vraie ni fausse » ({\it De l'interprétation}, 4).

\section{Propriété}
%PROPRIÉTÉ
Ce qui est propre à un individu ou à un groupe, autrement dit
ce qui leur appartient. Se dit spécialement, en droit, d’une
possession légitime, en principe garantie par la loi. Se distingue par là de la {\it possession},
qui n’est qu'un état de fait. Voyez Rousseau, {\it Contrat social}, X, 8-9.

\section{Protocole}
%PROTOCOLE
La mise en scène codifiée d’une hiérarchie. On peut le respecter,
et même c’est ordinairement le plus simple, à condition
de ne pas y croire. L'essentiel, par définition, est ailleurs.

\section{Providence}
%PROVIDENCE
C'est le nom religieux du destin : l’espérance comme ordre
du monde.

\section{Prudence}
%PRUDENCE
On évitera de la réduire au simple évitement des dangers,
{\it a fortiori} à je ne sais quelle lâcheté intelligente ou calculatrice.
Et on ne la confondra pas davantage, malgré Kant, avec la simple habileté
égoïste. La prudence, au sens philosophique du terme, est l’une des quatre
vertus cardinales de l'Antiquité et du Moyen Âge, sans laquelle les trois autres
(le courage, la tempérance, la justice) resteraient aveugles ou indéterminées :
c’est l’art de choisir les meilleurs moyens, en vue d’une fin supposée bonne. Il
ne suffit pas de vouloir la justice pour agir justement, ni d’être courageux, tempérant
et juste pour bien agir (puisqu'on peut se tromper dans le choix des
moyens). Voyez la politique. La plupart de nos gouvernants veulent le bien du
pays et le nôtre. Mais {\it comment} le faire ? C’est ce qui les oppose, et nous oppose.
Voyez les parents. Presque tous veulent le bien de leurs enfants. Cela n’a jamais
suffi, hélas, pour être de bons parents ! Encore faut-il savoir comment élever ses
enfants, comment faire en effet leur bien, ou comment les aider à faire le leur.

%— 475 —
%{\footnotesize XIX$^\text{e}$} siècle — {\it }
Qu’on le veuille, c’est la moindre des choses. Mais par quel chemin y parvenir ?
La vraie question, presque toujours, porte sur les moyens, non sur la fin. Que
faire, et comment ? C’est ce que l’amour voudrait savoir et ne suffit pas à déterminer.
Aimer, cela ne dispense pas d’être intelligent. C’est ce qui rend la prudence
nécessaire. Vertu intellectuelle, disait Aristote : c’est l’art de vivre et
d’agir le plus {\it intelligemment} possible.

C’est où l’on rencontre le sens ordinaire du mot. La bêtise, presque toujours,
est dangereuse. Nos politiques le savent bien. Nos militaires le savent
bien. Tous désirent la victoire ; mais cela ne tient lieu ni de tactique ni de stratégie.
Même chose pour nos industriels ou nos commerçants : tous désirent le
profit; cela ne leur dit pas comment y parvenir. Même chose pour nos
médecins : tous désirent la guérison ; cela ne leur dit pas comment l’atteindre
ou y contribuer. La prudence ne délibère pas sur les fins, remarquait Aristote,
mais sur les moyens. Elle ne choisit pas le but ; elle indique le chemin, quand
aucune science ou technique n’y suffit. C’est une espèce de sagesse pratique
{\it (phronèsis)}, sans laquelle aucune sagesse vraie {\it (sophia)} ne serait possible. « La
prudence, soulignait Épicure, est plus précieuse même que la philosophie : c’est
d’elle que proviennent toutes les autres vertus » ({\it Lettre à Ménécée}, 132) et la
philosophie elle-même. D’où provient la prudence ? De la raison (qui choisit
les moyens), lorsqu'elle se met au service du désir (qui fixe les fins). La prudence
ne règne pas (elle n’a de sens qu’au service d’autre chose), mais elle gouverne.
Elle ne remplace aucune autre vertu ; mais elle les dirige toutes, dans le
choix des moyens (voir saint Thomas, {\it Somme théologique}, Ia-IIæ, quest. 57,
art. 5, et 61, art. 2). Ce n’est pas la vertu la plus haute ; mais c’est (avec le courage)
l’une des plus nécessaires.

\section{Psychanalyse}
%PSYCHANALYSE
C’est à la fois une technique et une théorie. La technique
est fondée sur un certain usage de la parole (les associations
libres) et de la relation duelle (le transfert), dans un dispositif spatio-temporel
particulier (le cabinet, le divan, les séances). La théorie porte sur
l’ensemble de la vie psychique, en tant qu’elle est dominée par l’inconscient
et la sexualité. Le but est moins le bonheur que la santé ou la liberté : il s’agit
de rendre à l'individu son histoire, pour l’en libérer, au moins en partie, ou
en tout cas pour qu'il cesse d’en être aveuglément prisonnier. De là une thérapeutique,
pour les névrosés; une tentation, pour les curieux ou les
narcissiques ; et un métier, pour les psychanalystes. Il faut bien que tout le
monde vive.
Freud, qui fonda la chose et inventa le mot, dut être déçu, il le laisse parfois
entendre, par la répétitivité et la platitude de ce que la psychanalyse découvrait,
%— 476 —
%{\footnotesize XIX$^\text{e}$} siècle — {\it }
grâce à lui, en l’être humain. On comprend que les psychanalystes s’endorment,
parfois, pendant les séances. La psychanalyse est une blessure narcissique.
On se serait cru plus intéressant.

Mais c’est là son prix, et la grande leçon qu’elle nous donne. Nous sommes
le résultat d’une histoire sans intérêt. Qui s’en rend compte et l’accepte, il passe
à autre chose. Et la cure est finie.

\section{Psychologie}
%PSYCHOLOGIE
L'étude du psychisme, mais considérée plutôt comme discipline
objective et expérimentale («à la troisième per-
sonne », comme on dit parfois, par différence avec l’introspection, qui se fait à
la première, et avec la psychanalyse, qui suppose la deuxième). En son sommet,
c’est une science humaine, certes plurielle (il y a plusieurs écoles, plusieurs
méthodes, plusieurs doctrines, parfois incompatibles), mais pas plus peut-être
que l’histoire ou la sociologie. Reste à savoir à quoi elle sert. À connaître, ou à
manipuler ? À libérer, ou à instrumentaliser ? De là ce conseil d’orientation,
que Canguilhem, dans un texte fameux, donnait aux psychologues : « Quand
on sort de la Sorbonne par la rue Saint-Jacques, on peut monter ou descendre ;
si l’on va en montant, on se rapproche du Panthéon, qui est le Conservatoire
de quelques grands hommes ; mais si l’on va en descendant, on se dirige sûrement
vers la Préfecture de Police » (« Qu'est-ce que la psychologie ? », in {\it Études
d'histoire et de philosophie des sciences}, Vrin 1970, p. 381). Il y a une troisième
solution, qui est de ne pas quitter la Sorbonne. C’est la plus confortable, et la
plus ennuyeuse.

\section{Psychologisme}
%PSYCHOLOGISME
C'est vouloir tout expliquer — y compris la logique ou
la raison — par la psychologie. Mais alors la psycho-
logie elle-même ne serait qu’un effet parmi d’autres du psychisme : non une
science mais un symptôme, voire une maladie, comme disait Karl Kraus de la
psychanalyse, qui se prend pour son propre remède. On n’y échappe que par
un rationalisme strict, qui refuse de soumettre la vérité à quelque causalité
externe que ce soit. Tout mensonge relève de la psychologie. Toute erreur peut
en relever. Mais quand un individu énonce une vérité, il entre, au moins de ce
point de vue, dans un autre ordre : on peut demander à la psychologie d’expliquer
comment il la connaît ou pourquoi il éprouve le besoin de la dire, mais
assurément pas pourquoi elle est vraie. Expliquer une idée (comme fait psychique)
ne saurait suffire à la juger (comme vérité). Ou bien il n’y a plus de
vérité du tout, ni donc de psychologie.

%— 477 —
%{\footnotesize XIX$^\text{e}$} siècle — {\it }
\section{Psychose}
%PSYCHOSE
Voir « Névrose/psychose ».

\section{Psychosomatique}
%PSYCHOSOMATIQUE
Ce qui concerne à la fois l'esprit {\it (psuchè)} et le
corps {\it (soma)}, ou ce qui relève de leur interaction.
La notion n’a de sens que si l'esprit et le corps sont deux choses différentes. Ce
n’est donc pas le contraire du dualisme, comme on le croit parfois, mais sa version
naïvement médicale et moderniste : c’est un dualisme mou.

\section{Pudeur}
%PUDEUR
La vertu qui cache. Cela suppose une envie de tout montrer (sans
quoi la pudeur ne serait pas une vertu), et c’est ce qui rend la
pudeur particulièrement troublante — par le trouble qu’elle manifeste en voulant
l’éviter. Littré la définit comme une « honte honnête » ; ce paradoxe (la
honte de ce qui n’est pas honteux) fait partie de son charme. La pudeur va plus
loin que la décence, ou plus profond : elle relève moins des convenances que de
la délicatesse, moins de la société que de l’individu, moins de la politesse que
de la morale. C’est une façon de se protéger — et de protéger l’autre — contre le
désir qu’on suscite ou qu’on ressent. Seuls les amants peuvent s’en passer.

\section{Puissance}
%PUISSANCE
Une force qui s’exerce (puissance en acte : {\it energeia}) ou qui
peut s'exercer (puissance en puissance : {\it dunamis}). Les deux, au
présent, sont une seule et même chose : « toute puissance est acte, active, et en
acte », disait Deleuze à propos de Spinoza, et il n’y a rien d’autre que la puissance.
C’est l’être même, en tant qu’il est puissance d’être ({\it conatus}, force,
énergie).

Nietzsche voit dans la {\it volonté de puissance} « l'essence la plus intime de
l'être ». C’est une force d’affirmation, de création, de différenciation, qui fait
du plaisir et de la douleur « comme des faits cardinaux » (« tout accroissement
de puissance est plaisir, tout sentiment de ne pouvoir résister, de ne pouvoir
dominer est douleur », {\it La volonté de puissance}, éd. Wurzbach-Bianquis, I, 54).
Un vouloir-vivre, comme chez Schopenhauer ? Non pas. « Il n’y a de volonté
que dans la vie, reconnaît Nietzsche, mais cette volonté n’est pas vouloir vivre ;
en vérité, elle est volonté de dominer », et d’abord de se dominer soi : « Tout
travaille à se surpasser sans cesse » ({\it Zarathoustra}, II, « De la maîtrise de soi »).
Telle est la volonté de puissance. C’est une espèce de {\it conatus} (on sait que
Nietzsche s’est reconnu en Spinoza «un prédécesseur »), mais qui tendrait
moins à la conservation de soi-même qu’à son dépassement, qu’à l'extension,
dût-elle être fatale, de sa propre puissance : la tendance de tout être, non à persévérer
%— 478 —
%{\footnotesize XIX$^\text{e}$} siècle — {\it }
dans son être, comme le voulait Spinoza, mais à le « surmonter », mais
à « manifester sa puissance » et à l’accroître ({\it op. cit.}, II, \S 42-50 ; {\it Le gai savoir},
V, 349). On évitera de n’y voir qu’une apologie de la violence ou de
l’expansionnisme : la {\it puissance} qu’évoque Nietzsche est celle du créateur davantage
que du conquérant. Ce n’est pas un but, c’est une force (la puissance n’est
pas ce que veut la volonté, disait Deleuze, mais {\it ce qui} veut en elle, {\it Nietzsche et
la philosophie}, III, 6).

La proximité avec Spinoza est plus grande qu’il n’y paraît, et que Nietzsche
ne l’a cru. Ce dernier ne voyait dans le {\it conatus} spinoziste qu’une tendance
purement conservatrice et défensive, contre quoi lui-même se singulariserait en
pensant la volonté de puissance comme positive, affirmative et créatrice. C’était
méconnaître que la puissance, chez Spinoza aussi, est « affirmation absolue de
l'existence » ({\it Éthique}, I, 8, scolie) ; tel est le sens de la {\it causa sui} (dont nous relevons,
puisque nous faisons partie de la nature : « la puissance de l’homme est
une partie de la puissance infinie », IV, 4, dém.) et de la vie. La « puissance
d’exister et d’agir », pour Spinoza, est bien autre chose qu’un simple instinct de
conservation. Il ne s’agit pas seulement de résister à la mort, mais d’exister,
d’agir et de se réjouir le plus possible. La joie est une {\it augmentation} de puissance,
et c’est la joie qui est bonne. Il reste, c’est une vraie différence entre nos
deux penseurs, que « l’effort pour se conserver » est bien indissociable, pour
Spinoza, du désir « d’être heureux, de bien agir et de bien vivre » ({\it Éthique}, IV,
21, 22 et dém.). Spinoza, lui, n'aurait pas conseillé de « vivre dangereusement »...

\section{Pulsion}
%PULSION
Une force vitale ou innée, mais sans le savoir qui va avec. Se distingue
par là de l'instinct, qui est comme un mode d'emploi
génétiquement programmé. La pulsion sexuelle, par exemple, ne suffit pas à
l'érotisme. Ni l'érotisme, à la pulsion.

\section{Pureté}
%PURETÉ
Ce qui est sans tache ou sans mélange. Par exemple une eau
pure : c’est une eau qui n’est mêlée à rien, souillée par rien, qui
ne comporte rien d’autre qu’elle-même, une eau qui n’est que de l’eau. C’est
donc une eau morte et improbable : la pureté n’est ni naturelle ni humaine.
Par métaphore, on entend aussi par pureté une certaine disposition de
l'individu, quand il fait preuve de désintéressement. Par exemple un artiste, un
savant où un militant : dire que ce sont des purs, c’est dire qu’ils mettent leur
art, leur science ou leur cause plus haut que leur carrière ou que leurs intérêts
égoïstes. Cela culmine dans le {\it pur amour}, tel que Fénelon l’a pensé: c’est
% — 479 — 
%{\footnotesize XIX$^\text{e}$} siècle — {\it }
l'amour désintéressé, celui qui n’espère rien, en tout cas pour soi, celui dans
lequel « on s’oublie et se compte pour rien ». C’est en quoi le plaisir peut être
pur parfois (la {\it pura voluptas} de Lucrèce : quand il n’y a plus que le plaisir), ce
que la frustration n’est jamais.

Il y a un lac radioactif, au nord de l’Oural, stérilisé par des déchets
nucléaires. Les eaux y sont d’un bleu très pur, et pourtant sans vie aucune. Mais
ce « pourtant » est de trop : la propreté et la mort vont ensemble, et toute vie
est impure. Stériliser, c’est tuer ; cela en dit long sur la vie.

J'ai passé cet été quelques jours chez un ami, dans un coin charmant et
perdu des Alpes. Il me montre sa piscine, remplie à l’eau de pluie : l’eau y est
d’un vert glauque, opaque, avec d’inquiétantes suspensions. Je fais la moue,
et mon ami sent bien que, malgré la chaleur, j'hésite à plonger. « Ce n’est rien,
me dit-il, ce sont des algues, des micro-organismes. Attends un peu, tu vas
voir ! » Et de verser dans la piscine une bonne ration d’eau de Javel.. Quelques
minutes plus tard, de fait, l’eau s’était éclaircie. Le lendemain, elle était comme
neuve : nous y primes quelques bains joyeux et confiants... La vie avait reculé,
et cela nous parut un progrès décisif vers la propreté. Pourquoi non ? N'est-ce
pas ainsi qu’on nettoie les chambres d’hôpital et, en effet, les piscines ? Mais
chacun en sent bien aussi les limites et les dangers. Tout ce qui vit salit ; tout
ce qui nettoie tue. Demandez un peu aux microbes ce qu’ils pensent du savon.
Et à la ménagère maniaque, ce qu’elle pense des enfants.

Toute vie est impure, disais-je, et l’on ne saurait, sans tomber dans une
idéologie mortifère, lui préférer la pureté. Une chambre d’hôpital, ce n’est pas
un modèle de société, ni même un modèle de chambre. D'ailleurs les germes,
de plus en plus résistants, finissent par s’y glisser quand même, qui produisent
alors, plusieurs malades en sont morts, d’étonnants ravages. D'où je tirerais
volontiers une conclusion politique. La santé d’un peuple n’a jamais tenu à sa
pureté, qu'elle soit ethnique ou morale, mais seulement à sa capacité d’absorber
les mélanges, de maintenir, entre toutes ses composantes, un équilibre instable
mais vivant (vivant donc instable), enfin de gérer, dans l’à-peu-près, leurs différences
ou leurs conflits. Sans donner à cette métaphore biologisante plus de
valeur qu’elle n’en mérite (un peuple n’est pas un organisme, un individu n’est
pas un germe), on peut du moins réfléchir à ce lac de l'Oural, limpide et
mort comme un rêve d'ingénieur ou de tyran. On a parlé de « purification
ethnique », dans l’ex-Yougoslavie : qu’était-ce d’autre qu’une justification des
déportations ou des massacres ? Plusieurs rêvent d’une France propre, stérile et
pure comme un lac atomique, d’ailleurs artificielle comme lui ({\it pure}, la France
ne l’a jamais été) et comme lui promise à la mort immaculée.. Puissent-ils
songer de temps en temps au petit lac de l'Oural, d’un bleu si pur et si
transparent !

%— 480 —
%{\footnotesize XIX$^\text{e}$} siècle — {\it }
De quoi l’on pourrait tirer aussi bien, et peut-être mieux, une conclusion
morale, qui serait de vigilance contre la morale. La voilà de retour, dit-on, et
c’est tant mieux. J'ai assez bataillé contre le nihilisme et la veulerie pour ne pas
m'en plaindre. Mais la morale est comme l'hygiène : elle est au service de la vie
ou ce n’est qu’une manie dangereuse. C’est ce qui distingue la morale du moralisme,
et les braves gens des censeurs. Qu'est-ce que {\it l'ordre moral}, si ce n’est la
volonté d’inverser cette hiérarchie, de mettre la vie au service de la morale, de
{\it telle} morale, et d’en chasser l’impur ? Rêve fou : rêve de mort. S’il y a une
pureté de l’âme, elle est à l'opposé, et c’est ce que Simone Weil avait vu : « {\it La
pureté}, disait-elle, {\it est le pouvoir de contempler la souillure.} » Je dirais plus : de
l’accepter, de l’'habiter, de la vivre. L'âme est ce qui accueille le corps, et s’y
recueille. Sans honte. Sans frayeur. Sans mépris.

Ainsi, devant l’obscène du désir, la pureté de l'amour.

\section{Pusillanimité}
%PUSILLANIMITÉ
Du latin {\it pusilla anima}, où {\it pusillus animus}, petite âme,
esprit étroit ou étriqué. C’est le contraire de la
magnanimité : un nom savant pour dire un mélange de bassesse et de petitesse.
Se reconnaît surtout au manque de courage. C’est qu’il n’y a pas de grandeur
sans risque.

\section{Pyrrhonisme}
%PYRRHONISME
La doctrine de Pyrrhon, pour autant qu’on puisse la
reconstituer et que ce soit une doctrine (il n’a rien écrit,
ni rien affirmé absolument). Il tenait toutes choses pour « également indifférentes,
immesurables, indécidables », nous dit Aristoclès : aussi convient-il
selon lui « d’être sans jugement, sans inclination d’aucun côté, inébranlable, en
disant de chaque chose qu’elle n’est pas plus qu’elle n’est pas, ou qu’elle est et
n’est pas, ou qu’elle n’est ni n’est pas. Pour ceux qui se trouvent dans ces dispositions,
ce qui en résultera c’est d’abord l’aphasie, puis l’ataraxie » (Aristoclès,
cité par Eusèbe, {\it Prép. Évang.} XIX, 18 ; pour l'interprétation, voir Marcel
Conche, {\it Pyrrhon ou l'apparence}, PUF, 1994). Philosophie du silence, qui ne
peut s’énoncer sans se détruire. C’est peut-être le nihilisme le plus radical qui
ait jamais été pensé. Mais peut-on l’habiter ?

Dans les textes modernes, le mot a souvent un sens beaucoup plus général :
c’est un autre nom, spécialement chez Montaigne et Pascal, du scepticisme.
« La profession des pyrrhoniens est de branler, douter et enquérir, ne s'assurer
de rien, de rien ne se répondre » ({\it Essais}, II, 12) ; c’est « le plus sage parti des
philosophes » (IE, 15). La plupart, pourtant, ne s’en réclament pas. Cela même
donne raison aux pyrrhoniens, par l'impossibilité où ils se savent de le prouver :

%— 481 —
%{\footnotesize XIX$^\text{e}$} siècle — {\it }
« Rien ne fortifie plus le pyrrhonisme que ce qu’il y en a qui ne sont point pyrrhoniens.
Si tous l’étaient, ils auraient tort » ({\it Pensées}, 33-374 ; voir aussi les
fragments 131-434 et 521-387).

Le problème est alors d’assumer ce scepticisme sans tomber pour autant
dans le nihilisme ou la sophistique : Montaigne, Hume, Marcel Conche.
%
%{\footnotesize XIX$^\text{e}$} siècle — {\it }


%
%{\it }
%{\footnotesize XIX$^\text{e}$} siècle — {\it }
QUALITÉ Ce qui répond à la question {\it qualis ?} (quel est-il ? comment est-il ?).
Par exemple : « Il est grand et fort ; il est très gentil et un peu
bête. » Ce sont des {\it qualités}, et l’on voit que le mot, en philosophie, n’est
pas forcément élogieux (la qualité s’oppose à l’essence ou à la quantité
beaucoup plus qu’au défaut, qui n’est qu’une qualité négative). La qualité,
c’est ce qui fait qu’un être est comme il est (par différence avec l’essence ou
quiddité, qui fait qu’il est {\it ce} qu’il est), ou le fait qu’il le soit : c’est une
manière d’être ou une propriété, qui vient s'ajouter à la substance ou la
modifier. C’est la troisième catégorie d’Aristote : « J’appelle {\it qualité} ce en
vertu de quoi on est dit être tel » ({\it Catégories}, 8 ; voir aussi {\it Métaphysique}, $\Delta$,
14).

Chez Kant, les catégories de la qualité sont la réalité, la négation et la
limitation. Elles correspondent aux trois qualités du jugement, qui est soit
affirmatif, soit négatif, soit indéfini ({\it C. R. Pure}, Analytique, I).

On distingue parfois, spécialement depuis Locke, les {\it qualités premières},
qui sont indissociables de la matière et comme objectives (la solidité,
‘étendue, la forme, la vitesse...), des {\it qualités secondes}, qui n'existent que
pour et par le sujet qui les perçoit (la couleur, l'odeur, la saveur, le son...).
Celles-ci s’expliquent par celles-là (par exemple la couleur, par l’action
d’une certaine longueur d’onde de la lumière sur les cellules nerveuses de la
rétine), mais n’en sont pas moins réelles à leur façon. Quand je dis que cet
arbre est actuellement couvert de feuillage vert, j’énonce bien une proposition
vraie. Je ne me trompe que si je crois que sa couleur est indépendante
de la lumière et du regard. Mais je me tromperais bien davantage si je le
disais rouge ou bleu. Le sujet, c’est du réel aussi.

%— 483 —
%{\footnotesize XIX$^\text{e}$} siècle — {\it }
QUANTITÉ Ce qui répond à la question « Combien ? ». C’est une certaine
grandeur, qui renvoie à une échelle donnée de numération ou
de mesure. Par exemple : combien sont-ils ? combien ça pèse ? combien ça
coûte ? Quelle longueur, quelle hauteur, quelle superficie ? Chez Kant, et en
général d’un point de vue logique, la quantité des jugements mesure leur
extension : ils peuvent être universels, particuliers ou singuliers (voir ces mots).
De à les catégories de la quantité, qui sont l'unité, la pluralité et la totalité
({\it C. R. Pure}, Analytique, I).

QUESTION Le réel ne parle pas de lui-même. Mais il répond parfois, par
politesse, quand l’homme l’interroge. On appelle {\it question} ce
type de discours qui en sollicite un autre, dont il attend une information.
Questionner, c’est parler pour faire parler : par quoi le sens rebondit, pour ainsi
dire, sur son appel. Attitude proprement humaine, que les animaux ignorent (il
en est qui sont doués de langage, mais point qui soient aptes au dialogue, au jeu
libre des questions et des réponses) et que les dieux nous envient. À force de
connaître toutes les réponses, cela fait une telle torpeur en soi, une telle incuriosité
de tout, une telle lassitude. L’Olympe n’est pas ce qu’on croit. Le sens
ne rebondit plus, voilà, et les dieux s’ennuient. C’est pourquoi ils ont créé les
hommes : pour se distraire en les regardant se poser des questions.

QUIDDITÉ Ce qui répond à la question {\it « Quid? »} (Quoi?) ou {\it « Quid
sit ? »} (Qu'est-ce que c’est ?). On y répond ordinairement par
une définition. Ce livre aurait donc pu s'appeler {\it Quiddités}, si le titre n’était déjà
pris par Quine et s’il n’était à ce point ésotérique ou archaïsant.

La question {\it Quid sit ?} (Qu'est-ce que c’est?) est traditionnellement
opposée à la question {\it An sit ?} (Est-ce que c’est ?) : la {\it quiddité} est un synonyme
scolastique d’{\it essence}, qu’elle soit générique ou individuelle, et s'oppose à ce
titre, comme elle, à {\it l'existence}. Une fois qu’on a défini le bonheur, Socrate ou
Dieu, on n'a pas encore répondu à la question de leur existence. Mais pour
pouvoir répondre à la question de leur existence, il faut d’abord savoir ce qu'ils
sont, ou seraient : telle est la quiddité, toujours nécessaire, jamais suffisante.

QUIÉTISME L'Église catholique y voit une hérésie, qui serait celle, en
France, de Mme Guyon et de Fénelon. J'y verrais plutôt une
tentation et un danger : la tentation du repos, le danger de l’inaction. Faire
preuve de quiétisme, c’est croire que la quiétude suffit à tout, quand elle ne

%— 484 —
%{\footnotesize XIX$^\text{e}$} siècle — {\it }
suffit qu’à elle-même. Autant prétendre que le mysticisme, même le plus pur et
le plus élevé, pourrait tenir lieu de morale, de politique ou de philosophie, ce
qui serait une évidente sottise. N'oublions pas pourtant que la réciproque est
vraie aussi : morale, politique et philosophie ne sauraient tenir lieu de quiétude,
ni même vraiment y mener. Tous les mots du monde, même les mieux choisis,
ne feront jamais un silence. Tous les combats du monde, même les plus nécessaires,
jamais une paix. Ainsi le quiétisme est la vérité du mysticisme, et sa
limite.

QUIÉTUDE Un repos sans trouble, sans crainte, sans espérance, sans fatigue :
c’est le nom chrétien de l’ataraxie.

L’oraison de quiétude est celle qui ne demande rien, qui n’espère rien,
même pas le salut : elle s’abandonne passivement à Dieu ou au silence, jusqu’à
s’y perdre. C’est la pointe extrême de l’acquiescement mystique, de la contemplation,
enfin du {\it pur amour}, celui qui aime Dieu, comme disait Fénelon, « sans
aucun motif d'intérêt personnel ». À la fin, il n’y a plus que Dieu, et la question
du salut ne se pose plus.

RACE Un groupe, biologiquement défini, au sein d’une même espèce animale
(s'agissant des végétaux, on parlera plutôt de différentes
{\it variétés}). Par exemple les différentes races d’ânes ou de chevaux : un baudet du
Poitou ou un pur-sang arabe se distinguent respectivement d’un âne de Nubie
ou d’un percheron par un certain nombre de caractères héréditaires, qui sont à
la fois communs (à la race) et distinctifs (dans l’espèce). Il en va de même, au
moins à première vue, des êtres humains : ce n’est pas être raciste que de constater
que les Scandinaves présentent souvent un certain nombre de caractères
héréditaires communs, qui les distinguent, au moins superficiellement, des
Japonais ou des Pygmées.

Les races se distinguent des espèces, outre leur extension moindre, par
l’interfécondité : un mâle et une femelle de races différentes mais appartenant
à la même espèce sont normalement interféconds (ils peuvent.se reproduire
entre eux et engendrer des descendants eux-mêmes féconds), alors que deux
individus d’espèces différentes ne le sont pas. Il est vrai que, si les deux espèces
sont voisines, deux de leurs représentants respectifs peuvent éventuellement
s’accoupler (quoi que cela ne se produise pratiquement jamais à l’état sauvage),
mais leur descendant, sauf exception, sera stérile : ainsi un baudet et une
jument engendrant un mulet, ou une ânesse et un cheval engendrant un
bardot ; ni le mulet ni le bardot n’auront de descendance.

%— 485 —
%{\footnotesize XIX$^\text{e}$} siècle — {\it }
La notion de race est surtout descriptive ; mais il arrive souvent, s'agissant
des espèces domestiques et dans un milieu donné, qu’elle prenne un sens normatif,
voire prescriptif. Les notions de pur-sang ou de pedigree l’indiquent
assez. Ce sont catégories d’éleveurs et de marchands. C’est dire que la notion,
appliquée à l'humanité, est toujours suspecte. Au reste, on sait aujourd’hui
qu’elle n’a guère de sens : non seulement parce qu’il n’y a pas de races pures (les
différents groupes humains, qui dérivent sans doute d’une souche unique,
n'ont cessé depuis de se mêler), mais aussi parce que les caractéristiques différentielles
(la couleur de la peau, la taille, la forme du nez...) sont en l’occurrence
trop superficielles pour être prises — sauf par les racistes ou les imbéciles
— tout à fait au sérieux. Les progrès de la génétique confirment d’ailleurs et
l'unité de l’espèce humaine (99,99 \% du génome est commun) et la non-pertinence
biologique du concept de race (deux individus de la même race peuvent
être plus éloignés l’un de l’autre, d’un point de vue génétique, que deux individus
de races différentes). Toutefois ce n’est qu'un point de fait, certes heureux,
mais dont on aurait tort d’exagérer l'importance en matière d’antiracisme.
Quand bien même les races humaines se distingueraient davantage
qu'elles ne font, jusqu’à avoir (comme on voit dans certaines espèces animales)
des aptitudes différentes, cela ne changerait rien au respect indifférencié qu’on
doit à tout être humain. Le racisme n’est pas seulement une erreur
intellectuelle ; c’est aussi, et d’abord, une faute morale.

RACISME  « Je ne suis pas raciste », me dit un jour ma grand-mère. Puis elle
ajouta, en guise d’explication : « C’est vrai, quoi : c’est pas de leur
faute s’ils sont noirs ! » Elle avait plus de quatre-vingts ans, elle s'était occupée
de nous, mieux que nos parents, elle nous aimait plus que tout au monde...
J'avoue n’avoir pas eu le courage de lui expliquer, comme il l'aurait fallu, que
sa raison de n’être pas raciste. était raciste.

La même, un autre jour : « Je n’aime pas les Allemands ; ils sont tous
racistes. » Même remarque : c'était du racisme anti-allemand.

Qu'est-ce que le racisme ? Toute doctrine qui fait dépendre la valeur des
individus du groupe biologique, ou prétendu tel, auquel ils appartiennent.
C’est une pensée prisonnière du corps, comme un matérialisme barbare. Sa
logique est de frapper.

Matérialisme ? Voire. Car le corps y est beaucoup moins pensé comme la
{\it cause} de telle ou telle valeur psychique ou spirituelle que comme son {\it signe}.
Ainsi la blancheur ou la noirceur du corps révéleraient celles de l’âme.. Spiritualisme,
bien plutôt, à fleur de peau. C’est une herméneutique de l’épiderme.

%— 486 —
%{\footnotesize XIX$^\text{e}$} siècle — {\it }
C’est aussi, presque toujours, un narcissisme collectif et haineux. Deux raisons
de plus (le narcissisme, la haine) de le combattre.

RAISON C'est le rapport vrai au vrai, ou du vrai à lui-même. Mais qu'est-ce
que le vrai ? Nous n’y avons guère accès, sinon par la mise en évidence
du faux. De là un sens plus étroit ou plus spécifique : la raison est le pouvoir
de penser, en l’homme, conformément aux lois immanentes de la pensée,
en tout. Elle est pour cela toujours nécessaire (toujours soumise à des lois) et
toujours libre (elle n’a d’autres lois que les siennes propres). Un raisonnement
mathématique, dans sa perfection, en donne à peu près l’idée, qui est d’être une
liberté sans sujet : une liberté sans libre arbitre. C’est la liberté de Dieu, dirait
Spinoza (la nécessité de la nature ou du vrai), et la {\it libération} du sage, qui
devient Dieu dans l’exacte mesure où il cesse d’être soi. C’est là le bon usage,
non narcissique, du {\it connais-toi toi-même} : connaître le moi, c’est le dissoudre.
La raison, parce qu’elle est universelle, est comme une catharsis de l’égoïsme.
C’est ce qui explique que les sages, sans se piquer de morale, fassent ordinairement
preuve de la plus grande générosité. Là où le {\it moi} était, {\it tout} (la vérité) doit
advenir.

La raison est impersonnelle, universelle, objective. Aucun atome jamais n’a
violé la moindre de ses lois, ni aucun homme : le réel est rationnel ; le rationnel
est réel. Du moins c’est ce que pensent les rationalistes. Qu'ils soient incapables
de le prouver ne les réfute pas, puisque toute preuve et toute réfutation le supposent.

On distingue parfois, surtout depuis Kant, la {\it raison pratique}, celle qui commande,
et la {\it raison théorique}, celle qui connaît. Je n’ai jamais expérimenté la
première, ni réussi à la penser. Qu’une action puisse être raisonnable ou non,
c’est une affaire entendue. Mais pourquoi ? Parce qu’elle serait conforme ou
non à la raison ? Non pas (la folie l’est, puisqu’elle est rationnelle). Mais parce
qu’elle est conforme ou non à notre {\it désir} de raison (c’est-à-dire ici de cohérence,
de lucidité, d’efficacité.....). Aristote, Spinoza, Hume, plus éclairants que
Kant. Ce n’est pas la raison qui commande ou qui fait agir. C’est le désir : « Il
n’y a qu’un seul principe moteur, la faculté désirante » (Aristote, {\it De anima}, III,
10). La raison ne peut à elle seule réduire aucun affect (Spinoza, {\it Éthique}, IV,
prop. 7 et 14), ni produire aucune action (Hume, {\it Traité}..., livre II, III, 3 : «Il
n’est pas contraire à la raison de préférer la destruction du monde entier à une
égratignure de mon doigt »). Ainsi il n’y a pas de raison pratique ; il n’y a que
des actions raisonnables. Elles n’en sont pas pour cela plus rationnelles ; mais
plus efficaces, plus libres, plus heureuses.

%— 487 —
%{\footnotesize XIX$^\text{e}$} siècle — {\it }
RAISON SUFFISANTE (PRINCIPE DE -) « Nos raisonnements sont fon-
dés sur deux grands principes,
écrit Leibniz: {\it celui de la contradiction} [qu’on appelle plus volontiers
aujourd’hui principe de non-contradiction], en vertu duquel nous jugeons
faux ce qui en enveloppe, et vrai ce qui est opposé ou contradictoire au faux ;
et {\it celui de la raison suffisante}, en vertu duquel nous considérons qu'aucun fait
ne saurait se trouver vrai ou existant, aucune énonciation véritable, sans qu'il
y ait une raison suffisante, pourquoi il en soit ainsi et non pas autrement.
Quoique ces raisons le plus souvent ne puissent point nous être connues »
({\it Monadologie}, \S 31 et 32 ; voir aussi Théodicée, I, \S 44). C’est parier sur la
rationalité du réel, ou plutôt ce n’est pas un pari mais la condition, pour
nous, de toute rationalité possible. La raison suffisante, c’est ce qui répond
suffisamment à la question {\it « Pourquoi ? »}. Le principe stipule qu’il est toujours
possible, au moins en droit, d'y répondre: que rien n’est {\it « sans
pourquoi »}, malgré Angelus Silesius, sinon peut-être la série entière des raisons
ou la raison ultime (il n’est pas impossible, me semble-t-il malgré Leibniz,
que ce soit sans raison qu'il y ait quelque chose, par exemple tout ou
Dieu, plutôt que rien). Une chose ne peut s’expliquer que par une autre : par
exemple la rose par sa graine ou le monde par Dieu. Mais comment expliquer
qu'il y ait quelque chose, puisque toute explication le suppose ? Appelons
{\it Tout}, selon le sens ordinaire du mot, l’ensemble de tout ce qui est ou arrive,
par exemple la somme de Dieu et du monde. Que tout ce qui existe ou arrive
dans le Tout puisse et doive s'expliquer, cela n’entraîne pas nécessairement
que le Tout lui-même soit explicable, et même cela rend cette explicabilité
par avance impensable : puisque la raison qu’on pourrait en donner devrait
en faire partie et ne pourrait dès lors en rendre raison. Certains m’objecteront
que la sommation de Dieu et du monde est un concept impossible, qui
mélange des ordres différents. Admettons-le. Mais le même raisonnement,
appliqué à Dieu seul (quelle est sa raison suffisante ? Aucune ou lui-même :
on ne peut donc l’expliquer, puisque toute explication qu’on pourrait en
donner le suppose), interdit pareillement de le soumettre au principe. C’est
dire que le principe de raison suffisante, qui affirme que tout peut s’expliquer,
ne saurait s’appliquer valablement au Tout lui-même (sauf à supposer
autre chose que tout, ce qui est contradictoire) ni à la raison suffisante de
tout. Il ne peut pas davantage s’appliquer à soi (quelle est la raison suffisante
du principe de raison suffisante ? On ne peut répondre, et c’est en quoi justement
c’est un principe : voir le grand livre de Francis Wolff, {\it Dire le monde},
PUF, 1997, p. 85 à 87). Par quoi le principe de raison suffisante reste métaphysiquement
insuffisant. Tout s'explique, sauf le Tout lui-même et que tout
s'explique.

%— 488 —
%{\footnotesize XIX$^\text{e}$} siècle — {\it }
RAISONNABLE Ce qui est conforme à la raison pratique, comme dirait
Kant, ou plutôt, comme je préférerais dire, à notre désir de
vivre conformément à la raison {\it (homologoumenôs)}. On remarquera que ce désir
suppose toujours autre chose que la raison, qui ne désire pas. C’est ce qui
interdit de confondre le raisonnable et le rationnel. Est rationnel ce que la
raison peut connaître ou expliquer. Est raisonnable ce qu’elle peut justifier, eu
égard à un certain nombre de désirs ou d’idéaux donnés par ailleurs. Cet
{\it ailleurs} est l’histoire ; cet {\it égard} est la pensée.

RAISONNEMENT Une inférence, ou, plus souvent, une suite continue d’inférences.
C’est établir une vérité (si le raisonnement est
valide) par l’enchaînement ordonné de plusieurs autres. À quoi l’on objectera
qu’un raisonnement, même valide, peut n’aboutir qu’à une probabilité, voire à
une impossibilité de trancher. Mais il n’est valide, répondrai-je, que s’il montre
que cette probabilité ou cette impossibilité sont vraies. C’est important, spécialement,
en philosophie. Les propositions « Il est probable que nous puissions
savoir quelque chose » et « Il est impossible que nous sachions tout » sont des
propositions vraies, qu’un raisonnement peut établir.

RANCŒUR C'est une rancune envieuse, pour un bien qu’on ne nous a pas
fait.

RANCUNE Une vengeance rentrée, qui s’est rancie. Nous en voulons à
quelqu'un du mal qu’il nous a fait, dont nous gardons le souvenir
et, faute de pouvoir le lui rendre, le goût. C’est une haine présente, pour
une souffrance passée. Le mal qu’on nous a fait, longtemps nous fait du mal.

RASOIR D'OCKHAM Guillaume d’Ockham, qui mourut au milieu du
{\footnotesize XIV$^\text{e}$} siècle, est l’un des plus grands penseurs du
Moyen-Âge. C’est un nominaliste : il ne reconnaît l'existence que d’êtres singuliers,
considère la distinction entre l’essence et l'existence comme vide de sens,
et ne voit dans le genre et l’espèce, comme dans toute idée générale, qu’une
conception de l’âme ({\it intentio animae}, ce qui peut le rapprocher du conceptualisme)
ou « un signe attribuable à plusieurs sujets » (ce qui en fait un nominaliste).
Hors du cercle des spécialistes, il n’est pourtant guère connu ni cité,
sinon à propos de ce {\it rasoir} auquel son nom est traditionnellement attaché. De

%— 489 —
%{\footnotesize XIX$^\text{e}$} siècle — {\it }
quoi s'agit-il ? D’un principe d’économie : ne pas multiplier inutilement les
entités, couper tout ce qui dépasse du réel ou de l'expérience, c’est-à-dire en
l'occurrence toute idée qui n’est pas indispensable ou qui prétendrait exister en
soi ou de façon séparée. C’est un instrument d’hygiène intellectuelle, en même
temps qu’une arme contre le platonisme.

RATIONALISME J'avais cité, dans l’un de mes livres, la formule fameuse de
Hegel : « Ce qui est rationnel est réel ; ce qui est réel est
rationnel. » Michel Polac, dans un article critique, s’en était énervé : « On ne
peut plus dire ça, protestait-il, depuis Freud et la mécanique quantique ! »
C'était bien sûr se méprendre. La mécanique quantique est l’une des plus étonnantes
victoires de la raison humaine, et je ne connais pas d’auteur plus rationaliste
que Freud. Que nous n’ayons pas accès absolument à l'absolu (que le
réel soit « voilé ») ni ne puissions vivre de façon tout à fait raisonnable (que la
raison en nous ne soit pas tout, ni même l'essentiel), cela même, cher Michel
Polac, est pleinement rationnel. Il serait contradictoire, étant ce que nous
sommes, qu’il en aille autrement. Par exemple l'inconscient ne se soucie pas du
principe de non-contradiction ; c’est sa façon de lui rester soumis (un inconscient
logicien, ce serait contradictoire), et la psychanalyse n’est possible,
comme science où comme discipline rationnelle, qu’à la condition de le respecter.
Une contradiction, découverte dans un texte de Freud, serait une objection
autrement plus forte, contre la psychanalyse, que le plus déraisonnable des
délires, qui n’en est pas une. La folie est aussi rationnelle que la santé mentale
(la psychiatrie serait autrement impossible), le rêve aussi rationnel, quoique de
façon différente, que la conscience éveillée : Freud n’a inventé la psychanalyse
que pour comprendre rationnellement ce qui, {\it avant lui}, semblait irrationnel, ce
qui devrait nous aider, pensait-il, à vivre de façon un peu plus raisonnable.

Mais qu'est-ce que le rationalisme ? Le mot se prend principalement en
deux sens.

Au sens large et courant, qui est celui que je viens d'évoquer, le rationalisme
exprime d’abord une certaine confiance en la raison : c’est penser qu’elle
peut et doit tout comprendre, au moins en droit, autrement dit que le réel est
rationnel, en effet, et que l’irrationnel n’existe pas. Le rationalisme s’oppose
alors à l’irrationalisme, à l’obscurantisme, à la superstition. C’est l'esprit des
Lumières, et la lumière de l'esprit.

Au sens étroit et technique, le mot relève de la théorie de la connaissance :
on appelle {\it rationalisme} toute doctrine pour laquelle la raison en nous est indépendante
de l'expérience (parce qu’elle serait innée ou {\it a priori}) et la rend
possible ; c’est le contraire de l’empirisme.

%— 490 —
%{\footnotesize XIX$^\text{e}$} siècle — {\it }
On remarquera qu’un même philosophe peut donc être rationaliste aux
deux sens du terme (Descartes, Leibniz, Kant), mais aussi au sens large sans
l’être au sens strict (Épicure, Diderot, Marx, Cavaillès : c’est bien sûr le courant
dans lequel je me reconnais), voire au sens strict sans l’être au sens large (Heidegger ?).

RATIONNEL Ce qui est conforme à la raison théorique, autrement dit ce
que la raison peut penser, calculer ({\it ratio}, en latin, c’est
d’abord le calcul), connaître et, au moins en droit, expliquer. La folie n’est pas
moins rationnelle que la santé mentale (la psychiatrie serait autrement impossible).
Mais elle est moins raisonnable (la psychiatrie serait autrement inutile).

RÉALISME Au sens courant : voir les choses comme elles sont, et s’y adapter
efficacement. Le contraire, non de l’idéalisme, mais de la niaiserie,
de l’utopie ou de l’angélisme.

Au sens esthétique : tout courant artistique qui soumet l’art à l'observation
et à l’imitation de la réalité, davantage qu’à l’imagination ou à la morale. Historiquement,
cela peut désigner une époque en particulier: en gros, la
deuxième moitié du {\footnotesize XIX$^\text{e}$} siècle, qui s'oppose au romantisme comme le symbolisme,
au tournant du siècle, s’opposera à son tour au réalisme. Mais le mot
peut bien sûr être utilisé plus largement : Molière et Philippe de Champaigne
sont davantage réalistes que Corneille ou Poussin ; Rembrandt, le Caravage et
Zurbaran, davantage que Botticelli ou Boucher. Ces exemples disent assez que
le mot, pris en ce sens large, n’a rien de péjoratif. Il peut pourtant garder, dans
certains de ses emplois, quelque chose de restrictif. Que Courbet ou Flaubert
soient des artistes réalistes, cela ne retire rien à leur génie. Mais que le mot
s'applique mal à Chardin ou Stendhal, Rembrandt ou Proust (qui en relèvent
pourtant, mais l’excèdent de toutes parts), il me semble que cela ajoute quelque
chose au leur. C’est peut-être que tout {\it isme} enferme, ou que la réalité est plus
riche que le réalisme. C’est aussi que le réalisme, en ce sens restrictif, serait
plutôt un prosaïsme : il souffre moins d’un excès d’observation que d’un
manque de poésie.

Au sens proprement philosophique, enfin, le réalisme est une doctrine qui
affirme l'existence d’une réalité indépendante de l’esprit humain, que celui-ci
peut connaître au moins en partie. On parle par exemple de {\it réalisme moral},
pour désigner une doctrine qui affirme l’objectivité de la morale ou la réalité de
faits moraux irréductibles à quelque illusion ou croyance que ce soit (voir
Ruwen Ogien, {\it Le réalisme moral}, PUF, 1999). Mais le réalisme est surtout
%— 491 —
%{\footnotesize XIX$^\text{e}$} siècle — {\it }
évoqué en un sens plus général ou plus métaphysique. Il n’affirme pas l’existence
de telle ou telle réalité (par exemple morale), mais d’{\it une} réalité, quelle
qu’elle soit, voire de {\it la} réalité. Le mot, en ce sens technique et quoiqu'il soit un,
peut alors désigner deux courants très différents, selon la nature du réel
considéré : réalisme des Idées, des universaux ou de l’intelligible (par exemple
chez Platon, saint Anselme ou Frege), ou bien réalisme du monde sensible ou
matériel (par exemple chez Épicure, Descartes ou Popper). Le premier s'oppose
au nominalisme ou au conceptualisme ; le second, à une forme d’idéalisme ou
d’immatérialisme. On notera que ces deux réalismes s’opposent souvent l’un à
l’autre (Épicure contre Platon), mais pas toujours : Russel, au moins un temps,
et Popper ont soutenu l’un et l’autre.

RÉALITÉ (PRINCIPE DE) Ce n’est pas le contraire du principe de plaisir,
ni même tout à fait son complément. C’est
plutôt sa forme lucide et intelligente, qui s’y soumet en tenant compte — dans
sa recherche du plaisir et pour l’atteindre — de la réalité. Il s’agit toujours de
jouir le plus possible, de souffrir le moins possible (principe de plaisir), mais {\it en
tenant compte des contraintes et des dangers du réel} (principe de réalité). Cela
nous conduit à différer souvent le plaisir, voire à y renoncer ponctuellement ou
à accepter un déplaisir, pour jouir, plus tard, davantage ou plus longtemps.
Ainsi arrête-t-on de fumer, quand on y parvient, pour la même raison qui nous
a fait fumer si longtemps (pour le plaisir), mais appliquée autrement : parce
qu'on juge (le principe de réalité est un principe intellectuel) que le tabac
apporte, au bout du compte et au moins statistiquement, davantage de désagréments
que de jouissances. Ce n’est pas échapper au principe de plaisir ; c’est s’y
soumettre autrement. Ainsi va-t-on chez le dentiste pour le plaisir, comme on
va travailler pour le plaisir, quand bien même ce plaisir, le plus souvent, ne
nous attend ni chez le dentiste ni au travail. Il serait plus agréable de rester au
lit ? Sans doute, à court terme. Mais le principe de réalité est justement ce qui
nous libère de la dictature du court terme : principe de prudence (c’est l’équivalent
à peu près de la {\it phronèsis} des Anciens) et d'imagination.

RÉEL L'ensemble des choses ({\it res}) et des événements, connus ou inconnus,
durables ou éphémères, en tant qu’ils sont présents : c’est l’ensemble
de tout ce qui arrive ou continue. Se distingue par là de la vérité, qui n’arrive ni
ne dure, mais demeure. Par exemple que je sois assis actuellement devant mon
ordinateur, que je m’interroge sur la définition du réel, qu’il y ait un bouquet de
fleurs sur la table, des voitures dans la rue, que la Terre tourne autour du Soleil,
%— 492 —
%{\footnotesize XIX$^\text{e}$} siècle — {\it }
que d’autres étoiles naissent ou meurent, etc., c'est du réel. Quand le bouquet
sera fané, quand je serai mort, quand les voitures, la Terre et le Soleil auront disparu,
quand d’autres étoiles ou rien auront remplacé celles qui naissent ou se consument
actuellement, ce ne sera plus du réel. Mais il restera vrai que tout cela s’est
produit, comme il était vrai, avant que cela n'arrive, que cela se produirait. Éternité
du vrai, impermanence du réel. Les deux ne coïncident qu’au présent ; ils
coïncident donc toujours, pour tout réel donné (non, certes, pour toute vérité),
et c’est le présent même : le point de tangence du réel et du vrai. Non, pourtant,
que l’un et l’autre soient exactement sur le même plan. Ce n’est pas parce que
quelque chose était vrai de toute éternité que cela arrive ; c’est parce que cela
arrive que c'était, que c’est et que ce sera vrai de toute éternité. La vérité est sans
puissance propre, sans force, sans réalité : ce n’est que l’ombre portée, dans la
pensée, du réel, ou plutôt que sa lumière, pour nous, antécédente et rémanente.
Les matérialistes ne peuvent lui accorder tout à fait l’être ; ni les rationalistes lui
refuser pourtant toute consistance (puisque la vérité, même considérée indépendamment
du réel, a ses contraintes propres, qui sont celles de la logique). De là,
pour tout matérialisme rationaliste, une espèce de tension, qui fait sa difficulté et
sa limite. Un Dieu serait plus simple. Un monde d’Idées serait plus simple. Ou
bien la sophistique et la bêtise. Mais pourquoi faudrait-il se tenir au plus simple ?
Que le matérialisme soit difficile et limité, cela ne le réfute pas. Si le réel n’est pas
une pensée, comment la pensée pourrait-elle le saisir sans peine et sans limites ?
Ainsi le réel a toujours le dernier mot, mais ce n’est pas un mot: c’est ce qui
interdit qu'aucun discours, jamais, le dise adéquatement. Ce petit mot de réel, si
commode, si pauvre, n’est qu’une étiquette que nous collons, parce que cela nous
est utile, sur l'infini silence de ce qui dure et passe. Nos discours en font partie,
comme nos rêves et nos erreurs. C’est l’ensemble le plus vaste, le plus complet, le
plus concret : le divers du donné et de ce qui pourrait l’être — l’objet d’une expérience
possible ou impossible. Spinoza l’appelait la nature ({\it infinita infinitis modis,
hoc est omnia...}) ou Dieu, qui est tout. La définition du mot exclut que quoi que
ce soit lui échappe.

RÉFÉRENT L'objet, réel ou imaginaire mais non linguistique (sauf dans le
métalangage), d’un signe linguistique. Soit par exemple le mot
«chien » : ni le signifiant ni le signifié n’aboient ou ne mordent ; leur référent
peut faire l’un et l’autre.

RÉFLEXE Un mouvement involontaire, qui répond à un stimulus extérieur.
Par exemple l’œil qui se ferme, devant le coup, ou la main qui se
%— 493 —
%{\footnotesize XIX$^\text{e}$} siècle — {\it }
retire, devant la douleur. On parle de {\it réflexes conditionnés} pour ceux qui associent
artificiellement deux stimuli (par exemple une sonnerie et un aliment),
jusqu’à ce que l’un des deux suffise à produire un réflexe qui ne lui est pas normalement
associé (par exemple la sonnerie entraînant la salivation). On peut
aussi parler de réflexe, en un sens plus large, pour toute réaction quasi automatique,
résultant d’un apprentissage ou d’une habitude : ainsi chez le marin ou
l’automobiliste. Cela se fait sans que j'aie besoin d’y penser, ce qui est bien
commode et souvent efficace, mais ne saurait suffire : le réflexe ne dispense ni
de volonté ni de réflexion.

RÉFLEXION Au sens large : un effort particulier de pensée. Au sens étroit :
un retour de la pensée sur elle-même, qui se prend alors pour
objet. Ce dernier mouvement serait, avec la sensation, l’un des deux constituants
de l’expérience, donc l’une des deux sources, comme dit Locke, de nos
idées : nous n’aurions sans elle aucune idée de « ce qu’on appelle apercevoir,
penser, douter, croire, raisonner, connaître, vouloir et toutes les différentes
actions de notre âme » ({\it Essai}, II, 1, \S 4). La réflexion est donc une espèce de
sens intérieur, mais intellectuel et délibéré : c’est « la connaissance que l’âme
prend de ses différentes opérations, par où l’entendement vient à s’en former
des idées » ({\it ibid.}). Mouvement nécessaire, mais qui ne saurait épuiser à lui seul
le champ de la pensée. Mieux vaut philosopher à la façon des Grecs, conseille
Marcel Conche ({\it Présence de la nature}, II), plutôt que s’enfermer, comme Descartes
ou Husserl, dans la réflexivité ou le sujet : mieux vaut penser le réel
(réfléchir, au sens large) plutôt que se regarder penser (s’enfermer dans la
réflexivité, au sens étroit). Le moi, certes, fait partie du réel : penser le monde,
c’est donc aussi se penser. Mais il n’en est qu’une partie infime : se penser n’a
jamais suffi à penser ce qui est, ni même ce qu’on est (un vivant). La logique et
la neurobiologie nous en apprennent plus, sur la pensée, que la réflexion (au
sens étroit). Mieux vaut penser la pensée, comme dit Alain, que se penser soi
({\it Cahiers de Lorient}, I, p. 72). Mieux vaut connaître et réfléchir (au sens large) à
ce qu'on sait ou croit savoir, plutôt que s’enfermer dans la réflexion (au sens
étroit). « La pensée, disait encore Alain, ne doit pas avoir d’autre chez soi que
tout l’univers ; c’est là seulement qu’elle est libre et vraie. Hors de soi! Au
dehors ! » ({\it ibid.}). La réflexion mène à tout, à condition d’en sortir.

REFOULEMENT L'un des grands concepts de la psychanalyse. C’est le rejet
d’une représentation dans l’inconscient, où elle reste bloquée.
Il s’agit de protéger le moi, spécialement quand il est écartelé entre les
%— 494 —
%{\footnotesize XIX$^\text{e}$} siècle — {\it }
désirs du ça et les exigences du sur-moi. Mais le remède, parfois, est pire que le
mal: ce qui a été refoulé peut, sous les déformations que la résistance lui
impose, venir perturber la vie consciente (retour du refoulé : actes manqués,
rêves, symptômes). Par quoi le refoulement, sans être en lui-même pathologique,
peut devenir pathogène. Son remède n’est pas le défoulement, mais la
cure analytique. Son contraire, pas le retour du refoulé (qui lui reste soumis),
mais l’acceptation. On n’oubliera pas pourtant qu’accepter une représentation
n’est pas forcément satisfaire le désir qui s’y exprime. C’est la vérité, non la
jouissance, qui libère et guérit.

RÉFUTATION Cest démontrer la fausseté d’une proposition ou d’une
théorie. On y parvient ordinairement en montrant qu’elle
est incohérente (réfutation logique) ou démentie par l'expérience (falsification).
Ces deux voies, en philosophie, restent incertaines. La philosophie n’est pas
une science : les objections qu’on y fait, même rationnellement argumentées,
peuvent toujours être intégrées (« dépassées ») dans le système qu’elles attaquent,
ou faire l’objet elles-mêmes d’un certain nombre d’objections. Personne,
À ma connaissance, n’a jamais réfuté valablement Malebranche ou Berkeley.
C’est sans importance : leur philosophie n’en est pas moins morte pour autant.

RÈGLE Un énoncé normatif, qui sert moins à comprendre qu’à agir. Un
esprit fini ne peut s’en passer (voyez par exemple Spinoza, {\it Ethique},
V, 10, scolie), pas plus qu’un esprit libre s’en contenter.

REGRET Un désir présent qui porte sur le passé, mais comme doublement
en creux : c’est le manque en nous de ce qui ne fut pas. Se distingue
par là de la nostalgie (le manque en nous de ce qui fut) et s'oppose à la
gratitude (la joie présente de ce qui fut). « {\it Le regret}, disait Camus, {\it cette autre
forme de l'espoir...} » Mais le contexte — il s’agit de Don Juan, dans le {\it Mythe de
Sisyphe} — indique qu’il pense plutôt à ce que j'appelle la nostalgie. De fait, ce
sont les deux symétriques de l’espérance : le manque du passé (en tant qu’il fut
ou ne fut pas), comme l’espérance est le manque de l'avenir (qui sera peut-être).
Cette asymétrie rend le regret plus douloureux, et l’espérance plus forte.

RÉGULATEUR Ce qui fournit une règle, un horizon ou un fil conducteur,
sans permettre de dire ce qui est. S’oppose, spécialement
%— 495 —
%{\footnotesize XIX$^\text{e}$} siècle — {\it }
chez Kant, à {\it constitutif}. Par exemple l’idée de finalité dans la nature n’est qu’un
principe régulateur pour la faculté de juger réfléchissante : il est utile de la chercher,
impossible de la prouver ({\it C.F.J.}, \S 67 et 75). Comment expliquer l'œil,
sans supposer qu'il est là {\it pour voir} ? Mais comment prouver que c’est en effet
le cas ? On ne le peut : il faut faire {\it comme si}, disait un de mes professeurs, sans
jamais pouvoir prouver que c’est {\it comme ça}. Un principe régulateur indique une
direction, vers quoi il faut tendre ; il ne saurait déterminer ce qui est (il n’est
pas constitutif). Il aide à penser ; il ne suffit pas à connaître.

RELATIF Je ne sais plus qui soulignait plaisamment la grandeur du peuple
juif, ou l'importance pour nous de ses apports, en cinq noms

propres :

Moïse, qui enseigne que la Loi est tout ;

Jésus, qui enseigne que l’amour est tout ;

Marx, qui enseigne que l’argent est tout ;

Freud, qui enseigne que le sexe est tout ;

Einstein, qui enseigne... que tout est relatif.

La formule est jolie. On évitera pourtant de la prendre trop au sérieux.
Quant au fond, ce n’est qu’une série de contresens. Si la Loi était tout, il n’y
aurait plus besoin de Loi. Si l'amour était tout, il n’y aurait pas de monde (nous
serions déjà au paradis) et Jésus serait venu pour rien. Si l’argent était tout, il
n'y aurait pas de marxisme. Si le sexe était tout, pas de psychanalyse. Enfin, si
tout était relatif, quel sens y aurait-il à affirmer la supériorité de la Théorie de
la Relativité sur l’astronomie ptolémaïque ou la mécanique céleste de Newton ?
Mais essayons d’abord de définir.

Qu'est-ce que le relatif ? Le contraire de l'absolu : est relatif ce qui est non
séparé et non séparable (sinon par abstraction), autrement dit ce qui existe en
autre chose (relativité des modes ou des accidents, absoluité de la substance) ou
en dépend (relativité des effets, absoluité d’une cause qui ne serait causée ou
influencée par rien). Dans les doctrines religieuses, on considère ordinairement
que Dieu seul est absolu : toutes les créatures émanent de lui ou en dépendent
et sont donc relatives ; lui seul ne dépend de rien. Les athées ou les matérialistes
diront plutôt que tout est relatif (toute cause est elle-même effet d’une autre
cause, et ainsi à l'infini, tout événement est influencé par d’autres événements),
à la seule exception peut-être du Tout lui-même, dont on ne voit guère comment
il pourrait résulter ou dépendre d’autre chose — sinon de lui-même ou de
son état antérieur, C’est ainsi que nous sommes au cœur de l'absolu, tout en
étant voués au relatif.

%— 496 —
%{\footnotesize XIX$^\text{e}$} siècle — {\it }
Il faudrait donc donner raison à Einstein ? Sans doute, mais en évitant le
contresens habituel sur la Théorie de la Relativité. Celle-ci n’indique aucunement
que tout est relatif, au sens trivial du terme, c’est-à-dire subjectif ou
variable (relatif à un certain sujet ou à un certain point de vue). La théorie
d’Einstein, posant la relativité respective de l’espace et du temps, débouche au
contraire sur un certain nombre d’invariants, à commencer par la vitesse de la
lumière ou l’équivalence de la masse et de l'énergie, qui ne dépendent, précisément,
ni du sujet ni du point de vue. En ce sens, elle n’est pas {\it plus relative} mais
{\it plus absolue} que celle de Newton, qu’elle explique (comme un cas particulier :
pour des vitesses et des distances point trop grandes) et qui ne l'explique pas.
Que tout soit relatif (dans le tout, seul absolu), cela n'autorise pas à penser
n'importe quoi, ni n'importe comment.

RELATIVISME Toute doctrine qui affirme l'impossibilité d’une doctrine
absolue. En ce sens large, ce n’est qu’un truisme. Comment
un esprit fini pourrait-il avoir accès absolument à l’absolu, si l'absolu est un
esprit infini ou n’est pas {\it esprit} du tout ? Le relativisme ne prend son véritable
sens qu’en se particularisant, sous deux formes principales. Il faut en effet distinguer
un relativisme épistémique ou gnoséologique, d’une part, et un relativisme
éthique ou normatif, d’autre part. Les deux peuvent aller de pair (par
exemple chez Montaigne), mais aussi séparément (par exemple chez Spinoza,
qui ne relève que du second, ou chez Kant, qui ne relève que du premier).

Le relativisme épistémique ou gnoséologique affirme la relativité de toute
connaissance : nous n’avons accès à aucune vérité absolue. C’est le contraire du
dogmatisme théorique. Un scepticisme ? Pas forcément, puisqu’une connaissance
relative n’en est pas moins connaissance pour autant, et peut même, au
moins dans son ordre, être considérée comme certaine. Montaigne ou Hume
sont assurément relativistes, en ce sens ; mais Kant, qui n'était pas sceptique,
l’est également, comme d’ailleurs, aujourd’hui, la plupart de nos savants. C’est
l’un des résultats paradoxaux de la physique quantique. Mieux ils connaissent
le monde, moins ils ont le sentiment de le connaître absolument.

Quant au relativisme éthique ou normatif, il porte sur les valeurs, dont
il affirme la relativité. Nous n’avons accès à aucune valeur absolue ; tout
jugement de valeur est relatif à un certain sujet (subjectivisme), à certains
gènes (biologisme), à une certaine époque (historicisme), à une certaine
société ou culture (sociologisme, culturalisme) — voire, c’est d’ailleurs ce que
je crois, à tout cela à la fois. C’est le contraire du dogmatisme pratique. Un
nihilisme ? Pas forcément. Une valeur relative n’en est pas moins réelle pour
autant, ni ne cesse pour cela de valoir. Que la valeur d’une marchandise, par
%— 497 —
%{\footnotesize XIX$^\text{e}$} siècle — {\it }
exemple, ne soit pas absolue (elle dépend des conditions de sa production, du
marché, de la monnaie......), cela ne signifie pas que cette marchandise ne vaille
rien, ni que son prix soit arbitraire. Que la compassion, de même, soit diversement
appréciée (en fonction des cultures, des époques, des individus...), cela
n'entraîne pas qu’elle soit sans valeur, ni qu’elle ne vaille pas mieux, par
exemple, que l'indifférence ou la cruauté. Je dirais même plus : c’est uniquement
à la condition d’exister comme valeur, pour tel ou tel groupe humain,
qu'une valeur peut être relative, ce qu’un pur néant ne saurait être. Le relativisme,
loin de déboucher nécessairement sur le nihilisme (qui n’est que sa
forme outrancière, un peu comme le scepticisme outré que dénonçait Hume
l'est au relativisme gnoséologique, ou scepticisme modéré, qu’il professe), est
plutôt une raison forte de le refuser (intellectuellement) et de lui résister (moralement),
en même temps, et au fond pour la même raison, qu’au dogmatisme
pratique. Ces deux dernière positions ont en effet en commun de ne vouloir
reconnaître de valeurs qu’absolues : les uns affirment qu’il en existe de telles
(dogmatisme pratique), d’autres le nient (nihilisme), mais ils ne s’opposent en
cela que sur la base d’un accord premier, qu’on peut appeler leur absolutisme
commun. Les relativistes sont moins exigeants et plus lucides. Ce n’est pas
l'absolu qu’ils cherchent, ni le néant qu’ils trouvent. Ils s'intéressent aux conditions
réelles du marché (pour les valeurs économiques), de l’histoire, de la
société et de la vie (pour les valeurs morales, politiques ou spirituelles), et en
trouvent plus qu’assez pour en vivre et même, le cas échéant, pour en mourir.
J'attends qu’on m'explique quelles raisons un nihiliste pourrait avoir de combattre
la barbarie au péril de sa propre vie, et pourquoi il faudrait, pour la combattre,
se réclamer de valeurs absolues. « Parce que sinon, m’a-t-on souvent
répondu, le barbare vous opposera ses valeurs à lui : par exemple, s’il s’agit d’un
nazi, la pureté de la race, le culte du chef, de la nation et de la force, qu’il opposera
à votre respect efféminé ou judaïsé des droits de l’homme... » Je réponds
que c’est en effet ce qui se passe, et que je trouve curieux qu’on m’objecte ainsi
le réel même qui me donne raison. Qu'un nazi soit nazi au nom de certaines
valeurs, et qu’un démocrate le combatte au nom d’autres valeurs, c’est une
donnée de fait, qui prouve que ces valeurs existent, au moins pour nous, au
moins par nous, et suffisent. Si vous avez besoin, pour être antinazi, que
l’absolu le soit aussi, libre à vous. Mais imaginez que Dieu soit nazi et nous le
fasse savoir : deviendriez-vous nazi pour autant ? Ou qu’il n’y ait pas d’absolu
du tout: renonceriez-vous pour cela à respecter les droits de l’homme ?
Curieuse morale, qui dépend d’une métaphysique douteuse, comme elles sont
toutes !

Que toute valeur soit relative, cela ne prouve aucunement que rien ne
vaille. Cela le rend même improbable : comment un néant serait-il relatif ? Le
%— 498 —
%{\footnotesize XIX$^\text{e}$} siècle — {\it }
nihilisme n’est qu’un relativisme outré, ou vautré. Le relativisme, à l'inverse, est
un nihilisme ontologique (s'agissant des valeurs : elles ne sont pas des êtres ni
des Idées en soi) mais doublé d’un réalisme pratique (les valeurs existent réellement,
au moins pour nous, puisqu'elles nous font agir, ou puisque nous agis-
sons pour elles). Une valeur n’est pas une vérité : elle est l’objet d’un désir, non
d’une connaissance ; elle relève de l’action, non de la contemplation. Mais elle
n’est pas non plus un pur néant, ni une simple illusion : elle vaut vraiment, au
moins pour nous, au moins par nous, puisqu'il est vrai que nous la désirons. Ce
qu’il y a d’illusoire, dans nos valeurs, ce n’est pas leur valeur, mais le sentiment
que nous avons, presque inévitablement, de leur absoluité. Ou pour le dire
autrement : il n’y a d’absolu moral que pour et par la volonté. C’est ce que
j'appelle un absolu pratique : ce que je veux absolument, c’est-à-dire de façon
inconditionnelle ou non négociable. Parce que cela existerait en soi ? Nullement.
Mais parce que cela est indissociable de mon désir de vivre et d’agir
humainement. L'essentiel est exprimé par Spinoza, dans le décisif scolie de la
proposition 9 du livre III de l'{\it Éthique} : « Nous ne faisons effort vers rien, ne
voulons, n’appétons ni ne désirons aucune chose parce que nous jugeons
qu’elle est bonne ; mais, au contraire, nous jugeons qu’une chose est bonne
parce que nous faisons effort vers elle, parce que nous la voulons, appétons et
désirons. » Que toute valeur soit relative au désir qui la vise (donc à la vie, à
l’histoire, à l'individu : {\it au désir biologiquement, historiquement et biographiquement
déterminé}), ce n’est pas une raison pour cesser de la désirer, ni pour prétendre
que ce désir (qui peut lui-même être désiré) est sans valeur. Quand tu
bandes, as-tu besoin que Dieu ou la vérité bande aussi ? Pourquoi faudrait-il,
pour aimer la justice, qu’elle existe absolument ? C’est plutôt l’inverse : si elle
existait, elle n’aurait pas besoin de nous et nous serions dès lors moins tenus de
l'aimer. Mais cela n’est point. Ce n’est pas parce que la justice est bonne qu’il
faut l'aimer, ni parce qu’elle existe qu’il faut s’y soumettre. C’est parce que
nous l’aimons qu’elle est bonne (raison de plus pour l'aimer : elle ne vaut qu’à
cette condition !), et parce qu’elle n’existe pas, comme disait Alain, qu’il faut la
faire. Nihilisme, philosophie de la paresse ou du néant. Relativisme, philosophie
du désir et de l’action.

RELIGION Un ensemble de croyances et de pratiques qui ont Dieu, ou des
dieux, pour objet. Cela fait lien (selon une étymologie possible
et douteuse, qui rattache le mot à {\it religare} : la religion relie les croyants entre
eux, en les reliant tous à Dieu) et sens (puisqu'il existe autre chose que ce
monde, qui peut être son but ou sa signification). Qui n’en rêverait ? Toutefois
rien ne prouve que ce soit autre chose qu’un rêve.

%— 499 —
%{\footnotesize XIX$^\text{e}$} siècle — {\it }
« Croire en un Dieu, disait Wittgenstein, c’est voir que la vie a un sens. »
Disons que c’est le croire, et prendre ce sens au sérieux. Par quoi la religion est
le contraire de l'humour et de la connaissance : c’est le sens du sens enfin saisi,
fût-ce obscurément, recueilli (selon une autre étymologie tout aussi possible et
douteuse : {\it religere}, recueillir), perpétuellement relu (troisième étymologie :
{\it relegere}, relire), à la fois hypostasié et adoré. Comme ce sens est toujours absent,
la religion se fait espérance et foi. Cela, qui nous manque (le sens), ne manque
de rien — et nous sera donné, un jour. Reste, d’ici là, à prier, à croire, à obéir.
Toute religion débouche sur une morale dogmatique ou en procède : le bien
érigé en vérité, le devoir en Loi, la vertu en soumission. Bossuet a résumé
l'essentiel en une phrase : « Tout le bien vient de Dieu, tout le mal de nous
seuls. » La religion est la honte de l'esprit. {\it Mea culpa, mea maxima culpa...}
C’est aussi ce qui la sauve, parfois. Mieux vaut la honte que limpudence (Spinoza,
{\it Éthique} IV, 58, scolie). Mieux vaut une vertu soumise que pas de vertu
du tout. Mieux vaut aimer Dieu que n’aimer rien ou que n’aimer que soi. Au
reste cet amour, comme tout amour, est une joie, et source de joies (or « tout
ce qui donne de la joie est bon » : {\it Éthique}, IV, appendice, 30), donc source
d'amour... C’est ce qu’il y a de fort dans la sainteté, et de vrai dans la religion.
J'ai connu quelques vrais croyants dont l’évidente supériorité, au moins par
rapport à moi, devait trop à leur foi pour que je m’autorise à la condamner. La
religion n’est haïssable que lorsqu'elle débouche sur la haine ou la violence : ce
n'est plus religion mais fanatisme.

RÉMINISCENCE Dans le langage courant, c’est un souvenir involontaire,
voire partiellement inconscient ou méconnu comme
souvenir. Se dit surtout d’expériences sensorielles ou affectives (la madeleine de
Proust) ; ce sont des souvenirs qui s'imposent à nous, mais comme venant de
très loin, au point souvent d’en rester mystérieux ou méconnaissables. Par
exemple en art : on parle de {\it réminiscence} quand on croit reconnaître, dans une
œuvre donnée, la trace, mais involontaire et souvent inconsciente, d’un artiste
antérieur. Ce n'est pas un plagiat. Ce n’est pas une citation. Ce n’est même pas
une allusion ou un clin d’œil. C’est un écho, mais inaperçu du créateur, d’une
création autre. Cela fera le bonheur, plus tard, des érudits. Il faut bien que tout
le monde vive. Cela fait, surtout, un peu de l'épaisseur impersonnelle, ou transpersonnelle,
de l’art vrai. Le génie est le contraire d’une table rase.

Dans le langage philosophique, le mot sert surtout à traduire l’{\it anamnèsis}
des Grecs. Il peut alors désigner, à l'opposé du sens précédent, la recherche ou
la mobilisation volontaire d’un souvenir (Aristote, {\it De la mémoire}, 2 ; mieux
vaudrait, me semble-t-il, traduire par {\it remémoration} ou {\it anamnèse}). Mais il fait
%— 500 —
%{\footnotesize XIX$^\text{e}$} siècle — {\it }
plus souvent référence à Platon : la réminiscence est la trace en nous des Idées
éternelles, que notre âme aurait perçues, entre deux incarnations, face à face.
C’est ce qui nous permet, comme le petit esclave du {\it Ménon}, de découvrir, sans
sortir de nous-mêmes, des vérités que nous ignorions. Connaître ne serait que
reconnaître ; penser, que se ressouvenir. Si c'était vrai, l’histoire des sciences
n’avancerait qu’à reculons, vers une vérité qui la précède. C’est en effet le cas,
dira-t-on, puisque toute vérité est éternelle. Mais non. Qu'elle soit éternelle
est au contraire ce qui interdit de l’enfermer dans le passé : elle est tout autant
dans l'avenir, qui n’est rien, et davantage dans le présent, qui la contient toute.
La réminiscence platonicienne n’est qu’une métaphore, comme l’éternel retour
de Nietzsche, pour penser l'éternité du vrai.

REMONTRANCE C’est montrer à quelqu’un le mal qu’il a fait. Cela n’est
guère utile qu'avec les enfants, à condition de le faire
avec délicatesse, et avec les puissants, à condition de le faire sans démagogie.

REMORDS Une tristesse présente, pour une faute passée, comme une honte
de soi à soi. C’est un sentiment, qui peut être déchirant, non
une vertu. Le remords touche pourtant à la morale, par le jugement (la conscience
douloureuse d’avoir mal agi). C’est comme une nostalgie du bien. Se
mue en repentir, quand s’y ajoute la volonté de se reprendre.

RENAISSANCE Le fait de renaître. Mais le mot, en philosophie, est plus
souvent utilisé avec une majuscule. Il désigne alors une
époque, un mouvement ou un concept.L'époque couvre les
{\footnotesize XV$^\text{e}$} et {\footnotesize XVI$^\text{e}$} siècles.
Le mouvement, qui part de l'Italie du Nord, se déploie progressivement dans
toute l’Europe : c’est celui de la redécouverte conjointe de l'Antiquité et de
l'individu. Il en naît un {\it ars nova}, qui est comme la pointe extrême et exquise
de ce mouvement. Mais ce n’est que sa pointe. La Renaissance touche aussi à
l’économie, ou est touchée par elle (c’est l’époque où émerge ce que nous appelons
aujourd’hui le capitalisme), à la politique (par le renforcement des Cités
ou des États), à la pensée (par le progrès des sciences et de l’humanisme), à la
spiritualité (par la Réforme puis la Contre-Réforme), enfin et en général à la
conception du monde (aussi bien par la découverte de l'Amérique que par le
passage, comme dira Koyré, du monde clos, celui des Anciens et du Moyen
Âge, à l'univers infini, celui des Modernes). C’est l’époque de Brunelleschi et
de Gutenberg, de Donatello et de Van Eyck, d’Érasme et de Rabelais, de
%— 501 —
%{\footnotesize XIX$^\text{e}$} siècle — {\it }
Machiavel et de Montaigne, de Copernic et de Christophe Colomb, mais
aussi de Luther et Giordano Bruno, de Van der Weyden et Dürer, de Josquin
Des Prés et Palestrina, de Léonard de Vinci et Michel-Ange, de Raphaël et du
Titien.... Époque admirable, plus qu'aucune autre peut-être, au moins pour
les arts plastiques, et les contemporains ne s’y trompèrent pas. Voici par
exemple ce qu'Alberti écrivait dans la dédicace adressée à Brunelleschi,
excusez du peu, de son traité {\it De la peinture} : « Pour les Anciens, qui avaient
des exemples à imiter et des préceptes à suivre, atteindre dans les arts
suprêmes ces connaissances qui exigent de nous tant d’efforts aujourd’hui
était sans doute moins difficile. Et notre gloire, je l'avoue, ne peut être que
plus grande, nous qui, sans précepteurs et sans exemples, avons créé des arts
et des sciences jamais vus ou entendus. » On voit que la {\it Rinascita}, comme on
disait dès le Quattrocento, ne s’enferme nullement dans la nostalgie de
l'Antiquité : admiration pour les Anciens n'exclut pas une admiration
redoublée pour les contemporains, quand ils se montrent, et dans des conditions
peut-être plus difficiles, à la hauteur de leurs glorieux et antiques prédécesseurs.
Il reste que la Renaissance n’est pas seulement un progrès ou une
éclosion ; elle ne fut elle-même qu’en retrouvant quelques-uns des secrets ou
des idéaux de l’Antiquité. C’est ce qui nous mène au concept : on peut parler
de Renaissance, en un sens plus général mais qui reste analogique, pour tout
mouvement de renouveau qui se fonde sur un retour — au moins partiel, au
moins provisoire — à une époque plus ancienne. Le mot, qui continue de
valoir positivement, peut alors prendre un sens prospectif. Travailler à une
Renaissance, c’est reconnaître une décadence préalable, à quoi l’on essaie
d'échapper. C’est remonter vers la source, mais pour ne point renoncer à
l’océan. Reculer, au moins en apparence, mais pour avancer. C’est donc le
contraire d’une position réactionnaire ou conservatrice : un progressisme
cultivé et fidèle, qui veut éclairer l’avenir par l'étude patiente du passé, et qui
préfère rivaliser avec les maîtres d’autrefois, comme disait Fromentin, plutôt
qu'avec les contemporains ou les journaux.

RENOMMÉE Moins que la gloire, plus que la réputation (qui peut être
mauvaise) ou la notoriété (qui est neutre). On peut parler
d’un criminel notoire. On hésiterait à parler d’un criminel renommé. Non
pourtant que la renommée tienne lieu de jugement de valeur — un écrivain
renommé peut être un médiocre écrivain —, mais en ceci plutôt qu’elle véhicule
les jugements de valeur {\it des autres}. C’est pourquoi ses trompettes, comme disait
Brassens, sont mal embouchées. C’est qu’on n’y souffle pas soi-même.

%— 502 —
%{\footnotesize XIX$^\text{e}$} siècle — {\it }
REPENTIR  « C’est une espèce de tristesse, disait Descartes, qui vient de ce
qu’on croit avoir fait quelque mauvaise action ; et elle est très
amère, parce que sa cause ne vient que de nous. Ce qui n'empêche pas néanmoins
qu’elle soit fort utile » ({\it Passions}, III, 191). Cette définition pourrait
valoir aussi bien, et peut-être mieux, pour le remords. Descartes distingue ces
deux affections par le doute, qui serait présent dans le remords, absent dans
le repentir. Mais cet usage ne s’est pas imposé. Janet, ici, est plus éclairant :
« {\it Remords} se distingue de {\it repentir}, qui désigne un état d’âme plus volontaire,
moins purement passif. Le repentir est déjà presque une vertu ; le remords
est un châtiment » ({\it Traité de philosophie}, p. 655, cité par Lalande). Disons
que le remords n’est qu’un sentiment, quand le repentir est déjà une volonté :
c’est la conscience douloureuse d’une faute passée, jointe à la volonté de
l’éviter désormais et de la réparer si possible. Une vertu ? C’est ce que contestait
Spinoza. D'abord parce que tout repentir suppose la croyance au libre
arbitre ({\it Éthique}, III, déf. 27 des affects), et est illusoire par là. La connaissance
des causes et de soi vaudrait mieux. Ensuite, parce que c’est une tristesse,
quand il n’est de vertu vraie que joyeuse. Enfin parce que le repentir
n’est que le sentiment d’une impuissance, non la connaissance d’une puissance
même limitée : « Le repentir n’est pas une vertu, c’est-à-dire qu'il ne
tire pas son origine de la raison ; celui qui se repent de ce qu’il a fait est deux
fois misérable ou impuissant » ({\it Éthique}, IV, prop. 54 ; voir aussi la démonstration,
qui renvoie à celle de la prop. 53, sur l'humilité). La première misère
est d’avoir mal agi ; la seconde, de mal penser. Pourtant le repentir, comme
la honte, vaut mieux que la bonne conscience du salaud satisfait : « La honte,
quoiqu'’elle ne soit pas une vertu, est bonne cependant, en tant qu’elle dénote
dans l’homme envahi par la honte un désir de vivre honnêtement, tout
comme la douleur, qu’on dit bonne en tant qu’elle montre que la partie
blessée n’est pas encore pourrie. Bien qu’il soit triste donc, en réalité,
l’homme qui a honte de ce qu’il a fait est cependant plus parfait que l'impudent
qui n’a aucun désir de vivre honnêtement » (IV, 58, scolie). On retrouve
l'idée de Janet, selon laquelle le repentir est {\it presque} une vertu. Ce n’en est pas
une, puisque ce n’est pas un acte. Mais cela peut y mener, et même ce n'est
repentir (et non simplement remords) qu’à la condition d’y mener au moins
en partie.


REPRÉSENTATION Tout ce qui se présente à l'esprit, ou que l'esprit se
représente : une image, un souvenir, une idée, un fan-
tasme... sont des représentations.

%— 503 —
%{\footnotesize XIX$^\text{e}$} siècle — {\it }
C’est ce qui faisait dire à Schopenhauer que «le monde est ma repré-
sentation » — car je ne sais rien de lui, hormis ce que j’en perçois ou en pense.
Mais s’il n’y avait que des représentations, que représenteraient-elles ?

RÉPROBATION Un jugement de valeur négatif, sur l’acte d’un autre. Cela
ne touche à la morale que par les conséquences qu’on en
tire pour soi-même. Sans quoi ce n’est que médisance ou bonne conscience. La
miséricorde et le silence, presque toujours, valent mieux.

RÉPUBLIQUE Étymologiquement, c’est la chose publique ({\it res publica}). Mais
le mot désigne bien davantage : une forme d’organisation de
la société et de l’État, dans laquelle le pouvoir appartient à tous, au moins en
droit, et s'exerce, au moins en principe, au bénéfice de tous. Selon une formule
traditionnelle, c’est le pouvoir du peuple, par le peuple, pour le peuple — même
si ce pouvoir s'exerce le plus souvent par l’intermédiaire de représentants élus.
C’est donc une démocratie, mais radicale. Il peut se faire qu’une démocratie ait
un roi, si le peuple le juge bon ou laccepte (lAngleterre et l’Espagne,
aujourd’hui, sont assurément des démocraties : c’est le peuple, non le roi, qui y
décide de la politique suivie, et même du maintien ou non de la monarchie) ;
mais alors ce n’est pas une république (puisqu’une partie du pouvoir, en
l'occurrence le choix du monarque, échappe au peuple). En ce premier sens,
qui est constitutionnel, la république est donc une démocratie où tout le pouvoir
appartient au peuple et ne s’exerce que par ses élus : la France, les USA ou
l’Allemagne sont des républiques ; l'Angleterre et l'Espagne, non.

Il peut se faire aussi, et il se fait souvent, que le pouvoir, dans une démocratie,
ne se mette au service que des plus influents ou des plus nombreux ;
mais alors, même sans roi, ce n’est plus tout à fait une république, qui veut que
le pouvoir vise l'intérêt commun et non la simple sommation ou moyenne des
intérêts particuliers. On voit que le mot, en ce dernier sens, est moins constitutionnel
que normatif : il suppose un jugement de valeur, et comme une volonté
obstinée de résister aux égoïsmes, aux privilèges, aux corporatismes, aux Églises,
et même aux individus. La liberté ? Assurément. Mais pas au prix de l'égalité,
de la laïcité, de la justice. La république, en ce sens, est moins un type de gouvernement
qu’un idéal ou un principe régulateur : être républicain, c’est vouloir
que la démocratie se mette au service du peuple, non à celui, comme c’est
sa pente naturelle, de la majorité ou de l'idéologie dominante. On comprend
que cela ne dispense pas de respecter la démocratie, ni n’autorise à violer, fût-ce
dans l’intérêt du peuple, les libertés individuelles. Qui peut le plus peut le
%— 504 —
%{\footnotesize XIX$^\text{e}$} siècle — {\it }
moins. La démocratie, pour un républicain, est le minimum obligé ; la république,
le maximum souhaitable.

RÉSIGNATION C’est renoncer à la satisfaction d’un désir, qui subsiste
pourtant. Ce n’est plus la révolte, qui dit non, ni tout à fait
Pacceptation, qui dit oui. La résignation dirait plutôt « {\it oui mais} », ou « {\it oui
malgré tout} », ou « {\it tant pis} », mais sans y croire tout à fait. C’est comme un travail
du deuil inachevé, peut-être inachevable, qui s’accepte tel. Ce n’est pas la
sagesse, faute de joie. Ce n’est pas — ou plus — le malheur. C’est une espèce
d’entre-deux morne et confortable. Double piège. Double échec. Trop confortable
pour qu’on veuille en sortir. Trop morne pour qu’on se plaise à y rester.
Cest l’état souvent des vieilles gens, ou de ceux qui ont vieilli avant l'heure.
C’est ce qui la rend peu attirante. « Ce mot de résignation m'irrite, disait
George Sand ; dans l’idée que je m’en fais, à tort ou à raison, c’est une sorte de
paresse qui veut se soustraire à l’inexorable logique du malheur » ({\it Histoire de
ma vie}, X). Mais elle ne le peut, et encore, que par l'habitude et le renoncement.
Ce n’est pas une victoire ; c’est un abandon. C’est ce qui la rend nécessaire,
parfois, et insuffisante toujours. C’est comme une sagesse minimale, pour
ceux qui seraient incapables de la vraie. Sa formule semble se trouver, étonnamment,
dans les {\it Nourritures terrestres} de Gide : « Où tu ne peux pas dire {\it tant
mieux}, dis {\it tant pis}. Il y a là de grandes promesses de bonheur. » C’est trop dire
sans doute, ou cela suppose un bonheur préalable, ou une sagesse ultime, qui
n’est plus résignation mais acceptation pleine et entière. À côté de quoi la résignation
n’est qu’un moment. Elle ne vaut vraiment que pour ceux qui ne s’y
résignent pas, où qui la dépassent. Ce n’est un chemin qu’à la condition d’en
sortir.

RÉSISTANCE Une force, en tant qu’elle s'oppose à une autre. C’est l’état
ordinaire du {\it conatus} : tout être s'efforce de persévérer dans
son être, et s’oppose par là, autant qu’il le peut, à ceux qui le pressent, l’agressent
ou le menacent. Ainsi la résistance d’un corps, contre un autre qui l’écrase.
D'un organisme, contre les microbes. De la vie, contre la mort. D’un homme
libre, contre les tyrans.

Destutt de Tracy et Maine de Biran voyaient dans la résistance des corps
extérieurs à notre action sur eux l’une des sources (avec l'effort, qui en est indissociable)
de notre conscience et de nous-mêmes et de quelque chose (le monde)
qui n’est pas nous. C’est une récusation en acte du solipsisme. Mais c’est sans
doute Spinoza qui, pour penser la résistance, est le plus précieux. La résistance
%— 505 —
%{\footnotesize XIX$^\text{e}$} siècle — {\it }
n’est pas un accident, ni la marque de je ne sais quelle pensée réactive. Elle est
la vérité de l'être, en tant qu'il est puissance d’exister et d’agir, dès lors que cette
puissance est une (dans la substance) et multiple (par les modes). Seul l'infini
est pure affirmation ({\it Éthique}, I, 8, scolie). Toute chose finie peut être limitée
({\it Éth.} I, déf. 2) ou détruite ({\it Éth.} IV, axiome) par une autre de même nature.
C’est ce qui la voue à la résistance. Exister, c’est insister (s’efforcer d’être et de
durer) ; mais c’est aussi, par là même, résister : le {\it conatus} est cette « puissance
singulière d’affirmation et de résistance » (Laurent Bove, {\it La stratégie du conatus,
Affirmation et résistance chez Spinoza}, Vrin, 1996, p. 14) par quoi chaque être
fini tend à persévérer dans son être en résistant à l’écrasement ou à l'oppression.
Cela vaut en particulier pour l'être humain ({\it Éthique}, IV, prop. 3), qui résiste à
la tristesse et à la mort. L’éthique spinoziste est une éthique de la puissance et
de la joie. Mais c’est aussi, par à même, « une éthique de la résistance et de
l'amour » (L. Bove, {\it op. cit.}, p. 139 sq.). La politique de Spinoza est une politique
de la puissance et de la liberté. C’est pourquoi elle débouche sur une stratégie
de la résistance et de la souveraineté : si « c’est l’obéissance qui fait les
sujets » ({\it T. Th.P.}, XVII), « c’est la résistance qui fait les citoyens » (L. Bove, {\it op.
cit.}, p. 301). Pas étonnant qu’Alain ait rêvé de fonder «le parti Spinoza » :
« Obéir en résistant, disait-il, c’est tout le secret. Ce qui détruit l’obéissance est
anarchie ; ce qui détruit la résistance est tyrannie » (Propos du 24 avril 1911 ;
voir aussi mon article sur la philosophie politique d’Alain, « Le philosophe
contre les pouvoirs », {\it Revue internationale de philosophie}, n° 215, 2001, spécialement
aux p. 150 à 160). Pas étonnant que Cavaillès, l’un de nos plus grands
résistants au nazisme, se soit toujours dit spinoziste : il trouvait dans cette
pensée de quoi éclairer son combat. La Résistance, telle que Cavaillès et
d’autres la menaient, était la seule façon, face à la barbarie nazie, de persévérer
— quitte à en mourir — dans leur être de citoyens et d'hommes libres.

On parle aussi de {\it résistance} en psychanalyse. C’est une force, explique
Freud, qui s'oppose à la conscience et à l’analyse : elle empêche les représentations
inconscientes de remonter à la surface ou en déforme les manifestations.
Cette résistance résulte du refoulement, ou plutôt c’est comme un refoulement
continué : elle en maintient l’efficace et par là le confirme. C’est un obstacle,
pendant la cure, en même temps qu’un matériau.

RÉSOLUTION « Rien de plus facile que d’arrêter de fumer : je l’ai fait au
moins cent fois ! » Cette boutade marque à peu près la dis-
tance qu’il y a entre la {\it décision}, qui est une volonté instantanée, et la {\it résolution},
qui serait une volonté continuée. C’est vouloir vouloir, mais dans la durée :
vouloir (aujourd’hui) vouloir encore (demain ou dans dix ans). Par exemple
%— 506 —
%{\footnotesize XIX$^\text{e}$} siècle — {\it }
celui qui arrête de fumer, en effet, ou qui commence des études difficiles : une
décision n’y suffit pas ; encore faudra-t-il maintenir cette décision dans le
temps. Il s’agit moins de vouloir, en l’occurrence, que de vouloir continuer à
vouloir. Mais comment, puisqu'on ne peut vouloir qu’au présent ? La résolution
voudrait s’armer d’avance contre la lassitude, le renoncement, la versatilité.
Ce n’est qu’un leurre. Il faudra vouloir à nouveau chaque jour, et à chaque instant
de chaque jour. La résolution est simplement l’état d’une volonté qui le
sait et s’y prépare. Ce n’est pas un gage suffisant de réussite. Mais son absence,
presque toujours, est gage d’insuccès.

RESPECT Le sentiment en nous de la dignité de quelque chose (spécialement
de la loi morale, chez Kant) ou de quelqu'un (une personne).
On s’est parfois étonné que je n’en fasse pas l’une des grandes vertus de
mon {\it Petit traité}. C’est qu’elle m’a semblé équivoque. Dire de quelqu'un qu’il
est {\it respectueux}, ce n’est pas toujours ni souvent souligner l’une de ses vertus.
On imagine déjà des courbettes, des complaisances, des hiérarchies, toute la
gymnastique de l’intérêt et des grandeurs d’établissement — moins le sentiment
de la dignité de l’autre qu’un oubli de l’égale dignité de tous. Bien des fois, c’est
l’irrespect, spécialement face aux puissants, qui serait nécessaire et méritoire.
Voyez Diogène ou Brassens. Quant au respect qu’on doit aux plus faibles ou à
tous, la politesse, la compassion et la justice en disent l’essentiel. « Le devoir de
respecter mon prochain, écrit Kant, est compris dans la maxime de ne ravaler
aucun autre homme au rang de pur moyen au service de mes fins » ({\it Doctrine de
la vertu}, \S 25). C’est l'antidote de l’égoïsme, et comme le contrepoids de
l'amour (qui incite les humains à se rapprocher les uns des autres, alors que le
respect les conduit à maintenir entre eux une certaine distance : {\it ibid.}, \S 24).
C’est moins une vertu de plus que la conjonction de plusieurs. Le respect n’en
est pas moins nécessaire, ou plutôt il l’est d’autant plus, sans être pourtant
suffisant : il ne dispense ni d’amour ni de générosité. Il est vrai que la réciproque
est vraie aussi. L'amour et la générosité, sans respect, ne sauraient nous
satisfaire : ce ne serait que concupiscence ou condescendance.

RESPONSABILITÉ « Responsable, mais non coupable. » La formule, dans
la bouche d’un ministre, avait choqué. Prise en elle-
même, elle n’était pourtant ni absurde ni contradictoire. Je suis responsable de
tout ce que j'ai fait volontairement, ou de tout ce que j’ai laissé faire et que
j'aurais pu empêcher. Ainsi suis-je responsable de mes erreurs. Aucun élève ne
demandera qu’on relève sa note, ou qu’on l’attribue à quelqu'un d’autre, sous

%— 507 —
%{\footnotesize XIX$^\text{e}$} siècle — {\it }
prétexte qu’il n’a pas fait exprès de se tromper. Aucun homme politique sérieux
ne demandera qu’on tienne ses échecs pour rien. Cela ne signifie pas qu’ils se
sentent coupables, ni qu’ils le soient. Je suis responsable de mes erreurs et de
mes échecs. Je ne suis coupable que des fautes que j’ai accomplies délibérément,
en sachant qu’elles étaient des fautes. C’est la différence qu’il y a, en voiture,
entre griller un stop qu’on n’a pas vu, et foncer délibérément sur quelqu'un.
S’il y a mort d'homme, on se sentira sans doute responsable dans les deux cas.
On ne sera coupable, au moins de cette mort, que dans le second (ce qui
n'exclut pas qu’on soit coupable, dans le premier, d’inattention, d’excès de
vitesse ou d’imprudence). Les tribunaux en tiennent compte, qui ne punissent
pas les ivrognes homicides, sur nos routes, aussi sévèrement que certains le
voudraient : c’est qu’ils sont coupables d’avoir conduit en état d'ivresse, mais
pas plus que tous ceux qui le font en ayant la chance de ne tuer personne. Que
l’on condamne ces derniers, quand ils sont pris, plus sévèrement qu’on ne le
fait, cela me paraît urgent. Mais faut-il pour autant traiter les premiers — qui
n'ont pas bu davantage mais qui ont eu moins de chance — comme des
assassins ? Ce ne serait plus justice mais vengeance. Ce sont des chauffards ?
sans doute, et cela mérite d’être sanctionné. Mais ce ne sont pas des assassins :
ils sont coupables de conduite en état d’ivresse ; ils sont responsables, mais
non coupables, de la mort d’un individu. Je ne dis pas cela, mes exemples
l’indiquent assez, pour exempter nos ministres. La responsabilité, en politique
comme ailleurs, suffit à justifier une sanction politique (la démission, le
renvoi, la non-réélection...). Seule la culpabilité mérite une sanction pénale.
Le ministre en question était-il coupable ? Ce n’est pas à moi d’en décider : je
n'ai pour cela ni compétence ni goût. Mais qu’il ait été responsable, avec
d’autres, de la mort de plusieurs centaines d’hémophiles et de transfusés fait
une charge assez lourde, dont il était légitime de tenir compte. Notre ministre
l’a d’ailleurs fait, au moins en partie. Il y avait quelque injustice, me semble-t-il,
à lui reprocher sa formule comme si elle était intrinsèquement absurde ou
lâche.

Être responsable, c’est pouvoir et devoir répondre de ses actes. C’est donc
assumer le pouvoir qui est le sien, jusque dans ses échecs, et accepter d’en supporter
les conséquences. Seul le très jeune enfant et le dément y échappent, et
cela dit peut-être l'essentiel : la responsabilité est Le prix à payer d’être libre.

RESSENTIMENT La rancune des faibles. Le mot, dans la langue philosophique,
est définitivement marqué par Nietzsche : le ressentiment
est une « vengeance imaginaire », par laquelle les esclaves, incapables
d’agir, essaient de compenser leur infériorité réelle en condamnant fantasmatiquement
%— 508 —
%{\footnotesize XIX$^\text{e}$} siècle — {\it }
— par la morale et la religion — les barbares ou les aristocrates qui les
oppriment, dont ils ne peuvent autrement triompher. Ce mouvement marque
« la révolte des esclaves dans la morale » ({\it Généalogie...}, 1, 10), et c’est en quoi
les Juifs furent « le peuple sacerdotal du ressentiment par excellence » (I, 16).
Le ressentiment opère un renversement des valeurs (le « bon » des maîtres,
c’est-à-dire l’aristocrate, devient le « méchant » des esclaves), que Nietzsche
veut renverser à son tour. Le paradoxe de l’histoire, explique Nietzsche, c’est
que les faibles ont gagné. Parce qu’ils étaient beaucoup plus nombreux, beaucoup
plus sournois, beaucoup plus patients, beaucoup plus prudents... Le
temps, le nombre et la fatigue travaillent pour eux. En Europe, spécialement,
les Juifs n’ont cessé de gagner : en Grèce, avec « ce Juif de Socrate » (et avec
Platon, peut-être formé « chez les Juifs d'Égypte »), dans la « Rome judaïsée »,
celle de l'Église, dans la Réforme (« la Judée triompha de nouveau »), enfin,
« dans un sens plus décisif, plus radical encore, la Judée remporta une nouvelle
victoire sur l’idéal classique avec la Révolution française : c’est alors que la
dernière noblesse politique qui subsistait encore en Europe, celle des {\footnotesize XVII$^\text{e}$} et
{\footnotesize XVII$^\text{e}$} siècles français, s’effondra sous le coup des instincts populaires du ressentiment »
(I, 16). Ces lignes, qui sont désagréables (mais je pourrais citer bien
pire), ne sauraient toutefois suffire à invalider le concept de ressentiment, qui
reste éclairant. Elles doivent pourtant pousser à une certaine vigilance. Le
contraire du ressentiment, ou plutôt son symétrique, c’est le mépris, qui n’est
pas moins désagréable. Le ressentiment est la force des faibles ; le mépris, la faiblesse
des forts. Concepts utiles, affects dangereux. La miséricorde, dans les
deux cas, vaut mieux.

RÉSURRECTION Le fait de ressusciter, autrement dit de vivre à nouveau
alors qu’on était mort (se distingue par là de l’immorta-
lité), tout en restant le même individu, autrement dit le même composé âme-corps
(se distingue par là de la réincarnation). Ainsi Lazare ou Jésus. L’Ancien
Testament, sur le sujet, reste flou. La croyance en la résurrection n’apparaît
dans le judaïsme qu’assez tardivement, et plutôt comme un sujet de discorde :
les saducéens, si l’on en croit saint Paul, refusaient là-dessus de suivre les pharisiens
({\it Acte des Apôtres}, 23). Elle est en revanche, comme chacun sait, l’une des
pierres angulaires du christianisme. Le Christ est mort, il est ressuscité, et tel est
aussi le sort qui nous attend. Sous quelle forme ? On ne sait trop. Le {\it Credo}
annonce « la résurrection des corps », ce qui est bien embarrassant. Un corps,
même spirituel, doit avoir un âge, une forme, un certain aspect... Mais lequel ?
Est-ce le corps du vieillard, qui ressuscite, ou celui de l’adolescent ? Aura-t-il un
sexe et un ventre ? Aura-t-il les désirs qui vont avec ? les plaisirs qui vont avec ?

%— 509 —
%{\footnotesize XIX$^\text{e}$} siècle — {\it }
Sera-t-il beau ou laid, gros ou maigre, grand ou petit ? Comment serait-ce un
{\it corps} autrement ? La plupart des chrétiens jugent ces questions bien niaises : ils
préfèrent croire en l’immortalité de l’âme, comme Platon, et c’est en effet plus
commode. Mais qu’on ne parle plus, alors, de résurrection.

RÊVE C’est comme une hallucination, mais qu’on n’aurait au sens propre
que pendant le sommeil: Cela suffit pourtant à faire peser un doute
sur notre état de veille. C’est l’argument de Descartes, dans la première
Méditation : « Je vois si manifestement qu’il n’y a point d’indices concluants,
ni de marques assez certaines par où l’on puisse distinguer nettement la veille
d’avec le sommeil, que j’en suis tout étonné ; et mon étonnement est tel, qu’il
est presque capable de me persuader que je dors. » C’est l'interrogation de
Pascal ({\it Pensées}, 131-434) : « Qui sait si cette autre moitié de la vie, où nous
pensons veiller, n’est pas un autre sommeil, un peu différent du premier ? »
Pourquoi différent ? Par la continuité, à quoi nous reconnaissons le réel, ou ce
que nous prenons pour tel. « Si nous rêvions toutes les nuits la même chose,
écrit ailleurs Pascal, elle nous affecterait autant que les objets que nous voyons
tous les jours. Et si un artisan était sûr de rêver douze heures durant qu’il est
roi, je crois qu'il serait presque aussi heureux qu’un roi qui rêverait toutes les
nuits douze heures durant qu’il est artisan » (802-122). J'aime beaucoup ce
{\it presque}, comme le {\it un peu} du fragment précédent, qu’on retrouve d’ailleurs
dans l’alexandrin parfait qui clôt celui-ci : « {\it Car la vie est un songe un peu moins
inconstant.} »

RÉVERSIBILITÉ Ce qui peut se retourner sans perdre ses propriétés : par
exemple un manteau, si l’intérieur peut devenir l'extérieur,
ou un film, s’il peut se projeter indifféremment dans les deux sens. La notion sert
surtout en physique : les équations microscopiques sont réversibles ; les macroscopiques
ne le sont pas. C’est que le hasard et l’entropie, pour tout phénomène
complexe, interdisent qu’on revienne au point de départ. La tasse de café ne se
réchauffe pas toute seule ; les fleuves ne remontent pas vers leur source ; le
désordre, dans un système isolé, ne peut que s’accroître. C’est ce qu’on appelle la
flèche du temps, par quoi c’est l’irréversibilité qui est le vrai. « Ni temps passé /
Ni les amours reviennent / Sous le pont Mirabeau coule la Seine... »

RÉVOLTE Une opposition résolue et violente : c’est le refus d’obéir, de se
soumettre, et même d’accepter. Le mot sert surtout, et de plus
%— 510 —
%{\footnotesize XIX$^\text{e}$} siècle — {\it }
en plus, pour désigner une attitude individuelle (pour les révoltes collectives,
on parlera plus volontiers d’émeute, de soulèvement, de révolution......). C’est
que Camus est passé par là. « Qu'est-ce qu’un homme révolté ? Un homme qui
dit non. Mais s’il refuse, il ne renonce pas : c’est aussi un homme qui dit oui,
dès son premier mouvement » ({\it L'homme révolté}, 1). Oui à quoi ? À sa révolte, à
son combat, aux valeurs qui le fondent ou en naissent. « Le révolté, au sens étymologique,
fait volte-face. Il marchait sous le fouet du maître. Le voilà qui fait
face. Il oppose ce qui est préférable à ce qui ne l’est pas. Toute valeur n’entraîne
pas la révolte, mais tout mouvement de révolte invoque tacitement une valeur »
({\it ibid.}). Cette «affirmation passionnée » est ce qui distingue la révolte du
ressentiment : « Apparemment négative, puisqu'elle ne crée rien, la révolte est
profondément positive, puisqu'elle révèle ce qui, en l’homme, est toujours à
défendre » ({\it ibid.}). C’est où l’on sort de la solitude : la révolte est « un lieu
commun qui fonde sur tous les hommes la première valeur. Je me révolte, donc
nous sommes » (ibid.). C’est où l’on sort du nihilisme. C’est où l’on sort,
même, de la révolte, ou plutôt c’est où, sans en sortir, on l’inclut dans un
ensemble plus vaste, qui est la vie, dans une valeur plus haute, qui est l’humanité.
La révolte est « le mouvement même de la vie » ; elle est « amour et fécondité,
ou elle n’est rien » ({\it op. cit.}, V). « L'homme est la seule créature qui refuse
d’être ce qu’elle est» ({\it op. cit.}, Introduction). Mais cela, au moins, il faut
l’accepter. Ainsi la révolte n’est qu’un passage, entre l’absurde et l’amour, entre
le {\it non} et le {\it oui}. C’est pourquoi il faut y entrer, puisque « l'absurde n’est qu’un
point de départ » ({\it ibid.}). C’est pourquoi on n’a pas le droit d’en sortir tout à
fait. Cela fait comme un rythme ternaire. D’abord le {\it non} du monde à l’homme
(l'absurde) ; puis le {\it non} de l’homme au monde (la révolte) ; enfin le grand {\it oui}
de la sagesse ou de l’amour (le «tout est bien » de Sisyphe). Mais ce {\it oui}
n’annule aucun des deux non qui le précèdent et le préparent. Il les prolonge.
Il les accepte. Cela vaut pour l’absurde, qui n’est qu’un point de départ mais
qui demeure inentamé (la sagesse n’est ni une justification ni une herméneutique).
Cela vaut plus encore pour la révolte. Dire {\it oui} à tout, ce qui est l’unique
sagesse, c’est dire {\it oui} aussi à ce {\it non} de la révolte et de l’homme.

RÉVOLUTION Une révolte collective et triomphante : c’est une émeute qui
a réussi, au moins un temps, jusqu’à bouleverser les struc-
tures de la société ou de l’État. Les exemples archétypiques sont la Révolution
française de 1789 et la Révolution soviétique de 1917. Beaucoup de bonnes
raisons dans les deux cas. Beaucoup d’horreurs dans les deux cas. Mais une très
grande différence : on n’est jamais revenu tout à fait sur la première (Napoléon
l'installe au moins autant qu’il la clôt), alors que la seconde n’aura abouti, au

%— 511 —
%{\footnotesize XIX$^\text{e}$} siècle — {\it }
bout du compte, qu’à un capitalisme sous-développé, plus sauvage et plus
mafieux que le nôtre... C’est sans doute qu’il est moins difficile de transformer
l’État que la société (le féodalisme, pour l'essentiel, était déjà mort {\it avant} 1789),
plus facile de faire de nouvelles lois qu’une humanité nouvelle. Les fonctionnaires
finissent toujours par obéir. L'économie et l'humanité, non.

RHÉTORIQUE L’art du discours (par différence avec l’éloquence, qui est
l’art de la parole), en tant qu ‘il vise à la persuasion. C’est
mettre la forme, avec son efficace propre, au service de la pensée. Par exemple
un chiasme, une antithèse ou une métaphore : cela ne prouve rien, cela n’est
même pas un argument, mais peut aider à convaincre. Il convient donc de ne
pas en abuser. Une rhétorique qui se voudrait suffisante ne serait plus rhétorique
mais sophistique. Elle n’en reste pas moins nécessaire, ou il serait bien
prétentieux de prétendre absolument s’en passer. Les meilleurs s’en servent.
Voyez Pascal et Rousseau : qu’ils aient été d’éblouissants rhéteurs ne les a pas
empêchés d’être des écrivains et des philosophes de génie. Il est vrai que Montaigne
finit par séduire davantage, par plus de liberté, d’inventivité, de spontanéité...
C’est qu’il se soucie moins de convaincre. La vérité lui suffit. La liberté
lui suffit. Cela ne signifie pourtant pas qu’il se soit passé entièrement de rhétorique,
mais simplement qu’il sut, mieux que d’autres, s’en libérer. Apprends
d’abord ton métier. Puis oublie-le.

RIDICULE  « On ne prouve pas qu’on doit être aimé, écrit Pascal, en exposant
d’ordre les causes de l’amour ; cela serait ridicule » ({\it Pensées},
298-283). Pascal n’explique jamais. Cela fait une partie de son charme.
Essayons donc de comprendre. Ce qui est ridicule, c’est de confondre des
ordres différents, en l’occurrence l’ordre du cœur et celui de l'esprit ou de la
raison. C'était le début du fragment : « Le cœur a son ordre, l'esprit a le sien,
qui est par principe et démonstration. Le cœur en a un autre. » Essayez un peu
de démontrer rationnellement à quelqu'un qu’il doit vous aimer : son rire ou
son mépris donneront raison à Pascal, et il le citera peut-être : « Le cœur a ses
raisons que la raison ne connaît point » ({\it Pensées}, 423-277 ; voir aussi le fragment
110-282). Même chose pour le roi qui dit : « Je suis fort, donc on doit
m'aimer. » Son discours est faux et tyrannique, note Pascal ({\it Pensées}, 58-332) :
il confond l’ordre de la chair, où le roi règne et où la force l'emporte, avec les
ordres du cœur et de l’esprit, où la royauté ni la force ne sont rien. Même chose
enfin, mais on pourrait multiplier les exemples, pour celui qui s’étonnerait de
la basse extraction de Jésus-Christ : « Il est bien ridicule de se scandaliser de la

%— 512 —
%{\footnotesize XIX$^\text{e}$} siècle — {\it }
bassesse de Jésus-Christ, comme si cette bassesse était du même ordre duquel
est la grandeur qu’il venait faire paraître » ({\it Pensées}, 308-793). C'est toujours
confondre les ordres. Autant s'étonner que nos puissants ne soient pas des
saints.

Le ridicule, ce n’est donc pas seulement ce qui prête à rire (tout comique
n’est pas ridicule) : c’est ce qui prête à rire en confondant des ordres différents,
ou parce qu’on les confond. Cela rejoint le sens ordinaire du mot. Quelqu'un
fait un faux pas et tombe : si je le juge ridicule par là, ou s’il craint de lavoir
été, c’est que lui ou moi confondons l’ordre de la chair, où la pesanteur règne,
avec celui de l’esprit, où elle n’est rien. Par quoi toute tyrannie est ridicule, qui
veut faire adorer la force ou forcer la pensée à obéir ; et tout rire, contre les
tyrans, libérateur.

RIRE Mouvement involontaire et joyeux du visage et du thorax, face au
comique ou au ridicule. C’est une espèce de réflexe, mais qui ne va
guère sans un minimum de réflexion : on ne rit, presque toujours, que pour
autant qu’on a compris quelque chose — füt-ce, dans le comique de l'absurde,
qu'il n’y a rien à comprendre. Bergson voulait y voir « du mécanique plaqué sur
du vivant » : nous rions, disait-il, toutes les fois qu’une personne nous donne
l'impression d’une machine ou d’une chose ({\it Le rire}, I). J'aurais tendance, avec
Clément Rosset, à inverser la formule : à voir plutôt, dans ce qui nous fait rire,
«du vivant plaqué sur du mécanique (...) et se volatilisant à son contact »
({\it Logique du pire}, IV, 4). Un automate imitant un homme n’est presque jamais
drôle ; un homme ressemblant à une machine est presque toujours ridicule ou
comique, et d'autant plus qu’il s’en rend moins compte ou l’a moins décidé.
Ainsi, c'est un exemple qu’on trouvait chez Bergson, de cet orateur « qui
éternue au moment le plus pathétique de son discours ». Le corps se venge,
contre les simagrées de l'esprit. Le réel, contre les prétentions du sens. Par quoi
l'esprit se déprend de lui-même et de tout. C’est l'esprit vrai : du sens plaqué
sur du non-sens, et partant en éclats. C’est pourquoi, comme le notait Bergson,
«il n’y a pas de comique en dehors de ce qui est proprement humain » ou doué
d'intelligence. Parce qu’il n’y a de comique que par le sens, et de sens que pour
l'esprit — mais qui ne s’en amuse que dans la mesure où il cesse d’y croire. Nous
rions lorsque le sens se heurte plaisamment au réel, jusqu’à se pulvériser à son
contact. Rire : explosion de sens. Aussi peut-on rire de tout (tout sens est fragile,
tout sérieux, ridicule) ; c’est ce que prouve l'humour, et qui le rend nécessaire.
Deux formules me reviennent en mémoire, qui ont guidé ma jeunesse,
comme deux balises de l'esprit. Celle d’Épicure, balise souriante : {\it « Il faut rire
%— 513 —
%{\footnotesize XIX$^\text{e}$} siècle — {\it }
tout en philosophant »} (S.V., 41). Et celle de Spinoza, qui semble dire le
contraire : {\it « Non ridere, non lugere, neque detestari, sed intelligere »} (T. P., X, 4 :
« Ne pas rire, ne pas pleurer, ne pas détester, mais comprendre »). L’opposition,
entre les deux, n’est pourtant que de surface. Épicure n’a jamais cru que
le rire puisse suffire (la philosophie, avant comme après, reste nécessaire) ; ni
Spinoza, qu'on doive y renoncer. La célèbre formule du {\it Traité politique} n’est
en rien une condamnation du rire; ce n’est une condamnation que de la
raillerie, de la dérision, du rire haineux ou méprisant. Spinoza s’en explique
dans l’{\it Éthique} : ce n’est pas le rire qui est mauvais dans la raillerie, c’est la haine
(IV, 45, corollaire 1). Quant au rire considéré en lui-même, c’est au contraire
«une pure joie », dont on aurait bien tort de se priver. Pourquoi serait-il plus
légitime d’apaiser la faim et la soif que de combattre la mélancolie ({\it ibid.},
scolie ; voir aussi C. T., II, 11) ? On rit parfois de bonheur, plus souvent pour
surmonter l'angoisse ou la tristesse. Ce n’est pas un hasard si tant de nos humoristes
s’avouent, dans le privé, d'humeur sombre. Rions avant d’être heureux,
conseillait La Bruyère, de peur de mourir sans avoir ri.

ROMAN Un genre littéraire, sans autres contraintes que la narrativité et la
fiction. C’est raconter une histoire inventée comme si elle était
vraie, ou une histoire vraie comme si elle était inventée. Le mensonge est son
principe, mais aussi, bien souvent, son ressort principal. Il s’agit de rendre intéressant
ce qui ne l’est pas, de donner sens à ce qui en est dépourvu, de faire
rêver plutôt que penser, d’émouvoir plutôt que d'éclairer, de passionner plutôt
que d’éduquer. Le roman, presque inévitablement, exagère la vie, comme dirait
mon ami Marc, et ne vaut, sauf chez les plus grands, que par cette exagération.
Cela explique son succès, chez presque tous, et la terrible tentation qu’il exerce
parfois, même chez les meilleurs, quand l'esprit se relâche. C’est le mensonge
érigé en esthétique ou en divertissement.

Il arrive pourtant que la vérité affleure (le {\it mentir-vrai} d'Aragon), et se serve
de ce masque pour s’avouer. Le roman est un détour par la fiction, qui peut
ramener au vrai : nous lui devons quelques-uns des plus grands chefs-d’œuvre de
la littérature universelle. Force m'est pourtant de reconnaître que j'en lis de
moins en moins, et avec de moins en moins de plaisir. Il m’arrive de penser que
le roman est un genre mineur, dont chaque réussite confirme, en les dépassant,
les limites : Proust, Céline ou Joyce sont grands {\it malgré} ou {\it contre} le roman, me
semble-t-il, plutôt que grâce à à lui. Cela pourrait sembler justifier à l’avance le
Nouveau-roman, qui n’est qu’un antiroman (pas de personnages, pas di intrigue
pas d'aventure, pas de romanesque). Mais l’ennui ne justifie rien. C’est ce qui
sauve le roman, ou qui le sauvera. On n’a pas fini d’inventer des histoires.

%— 514 —
%{\footnotesize XIX$^\text{e}$} siècle — {\it }
ROMANTISME  L’opposé du classicisme, mais en aval et quant au fond (par
différence avec le baroque, qui le serait plutôt en amont,
du moins en France, et quant à la forme). Les romantiques ont besoin de règles
pour les violer, de traditions pour s’y opposer, de maîtres pour s’en libérer, de
raison, enfin, pour vouloir s’en affranchir ou lui préférer les sentiments. En
quoi c’est un mouvement second, et secondaire le plus souvent. Mais il exprime
aussi quelque chose d’essentiel à l’âme humaine, qui est le malheur éperdu de
durer, de n’être pas Dieu, de devoir finir... Le temps est son mal et sa raison
être. D’où la nostalgie toujours, qui est le sentiment romantique par
excellence : cendre, pour le poète, du feu qui le consume! La vraie vie est
ailleurs — et nous sommes ici. Le romantisme est pour cela condamné au
double langage ou aux arrière-mondes, à l’idéalisme ou à la déception. Il aspire
à l'infini, ne trouve que le fini. Il cherche l'absolu, ne trouve que le moi. Il voudrait
se fondre dans l’unité, se heurte partout au multiple ou à la dualité. Il
voudrait s’abandonner à l'inspiration, ne réussit que par le travail. Il exalte la
passion, l’imaginaire, la sensibilité. Il ne débouche que sur la lassitude ou
l'ennui. La fuite est sa tentation ; le rêve, son excuse. C’est un art passionnel,
qui n’a guère le choix qu’entre l’onirisme et la religion : esthétique de l'exil ou
de l'évasion, du mystère ou du déchirement. Cela ne retire rien aux plus grands
(Novalis et Hülderlin, Byron et Keats, Delacroix, Berlioz, Nerval...), mais
m’empêche de les suivre tout à fait. Seul Hugo, pour mon goût, fait exception.
Mais c’est qu’il est l’exception absolue : il excède le romantisme autant que
Bach le baroque. Il récuse à lui seul la formule injuste et profonde de Goethe :
{\it « J'appelle classique ce qui est sain, romantique ce qui est malade. »} Ou plutôt il la
récuserait, si une exception, aussi glorieuse fût-elle, pouvait invalider... une
définition.

RUMEUR Un bruit anonyme, mais chargé de sens. C’est donc que quelqu’un
parle. Qui? Personne, tout le monde: le sujet de la
rumeur est le {\it on}, qui est moins un sujet qu’une foule impersonnelle et insaisissable.
La rumeur fait comme un discours sans sujet, dont personne n’a à
répondre. C’est ce qui la rend particulièrement propice aux fausses nouvelles,
aux sottises, aux calomnies. Seul celui qui la lance, s’il y en a un, en est vraiment
responsable. Seuls ceux qui se taisent ou la combattent en sont vraiment
innocents. L'idéal serait de n’y prêter aucune attention. La moindre des choses,
de ne pas en rajouter. Ce ne sont que des on-dit, qu’on ne peut toujours
ignorer mais qu’il faudrait s’interdire de propager.
%
%{\footnotesize XIX$^\text{e}$} siècle — {\it }


%
%S{\it }
%— 604
%{\footnotesize XIX$^\text{e}$} siècle — {\it }

SACRÉ Ce qui vaut absolument, au point de ne pouvoir être touché, sauf
précautions particulières, sans sacrilège. Le sacré est un monde à
part, comme le représentant, dans celui-ci, de l’autre. Il est séparé, ou doit
l'être, du quotidien, du laïque, du simplement humain. C’est en quoi le mot
dit plus que {\it dignité} — et dit sans doute trop. Le sacré mérite davantage que du
respect : il mérite, ou plutôt il exige, vénération, adoration, « crainte et
tremblement », comme un mélange d’effroi et de fascination. Le mot, en ce
sens strict, appartient au vocabulaire religieux : le sacré s'oppose au profane
comme le divin à l’humain ou comme le surnaturel à la nature. S'il n’y a ni
dieux ni surnaturel, comme je le crois, ce sacré-là n’est qu’un mot, que nous
mettons sur des sentiments archaïques ou illusoires.

En un sens plus général et plus vague, on appelle parfois {\it sacré} ce qui
semble avoir une valeur absolue, qui mérite pour cela un respect
inconditionnel : ainsi parle-t-on du caractère sacré de la personne humaine,
voire (dans la déclaration des droits de l’homme de 1789) de la propriété
privée comme étant « un droit inviolable et sacré ».. Le sacré, en ce sens
large, c’est ce qui peut être profané et ne le doit, qui mérite pour cela qu’on
se {\it sacrifie} pour lui. C’est ce qui fait dire à mon ami Luc Ferry que tout être
humain est sacré, et il m'est arrivé, quoique rarement, de le dire aussi. Reste
à savoir s’il l’est au sens strict (auquel cas l’humanisme est une religion :
voyez {\it L'homme-Dieu} de Luc Ferry) ou au sens large (auquel cas l’humanisme
n'est qu’une morale). Je penche trop résolument vers le deuxième terme de
l'alternative pour ne pas juger ce mot de sacré, appliqué à l'humain, quelque
peu excessif. C’est moins un concept, au moins dans ma bouche, qu’une
métaphore. Elle est parfois éclairante ; ce n’est pas une raison pour y croire
tout à fait. Le respect suffit, et vaut mieux.

%— 516 —
%{\footnotesize XIX$^\text{e}$} siècle — {\it }
SACREMENT Un rite qui rend sacré, ou par lequel le sacré opère : c’est
comme un miracle institutionnel. Si le mariage est un sacrement,
par exemple, il devient sacré: nul ne peut s’en libérer, fût-ce d’un
commun accord, sans sacrilège. Le divorce prouve le contraire, ou plutôt ce
n’est qu’à supposer le contraire que le divorce devient admissible. Comment
l'Église pourrait-elle l’accepter ? Raison de plus pour ne pas laisser l'Église
décider de nos amours.

On remarquera que la naissance n’est pas un sacrement, alors que le baptême
en est un. Non bien sûr qu’un membre de l’Église soit plus sacré, même
aux yeux des croyants (au moins aujourd’hui), qu’un autre être humain, mais
en ceci que la naissance ne suppose aucune intervention surnaturelle. Or quoi
de plus émouvant qu’une naissance ? Quoi de plus kitsch, presque toujours,
qu’un baptême ? La vie est plus précieuse que les sacrements, et n'en a pas
besoin.

SACRIFICE Une offrande qu’on fait au sacré, le plus souvent sous la forme
d’un animal ou d’un être humain qu’on immole. La plupart des
religions considèrent aujourd’hui que les sacrifices humains sont sacrilèges, ou
plutôt que nul n’a le droit de sacrifier que soi. C’est soumettre la religion à la
morale, comme il faut faire en effet, et l’une des marques les plus sûres de la
modernité. On dira qu’à ce compte la modernité commencerait à Abraham.
Pourquoi non ? Mais ce n’est que depuis Kant, peut-être bien, qu'on commence
à le comprendre.

En un sens plus général, le sacrifice est un don que l’on fait pour quelque
chose ou quelqu'un que l’on aime ou respecte. Le sacrifice ultime est celui de
sa propre vie — non qu’on la juge sans valeur, mais parce qu’on juge qu’elle ne
vaut qu’au service d’autre chose, qui la dépasse, ou de quelqu'un d’autre, qu'on
ne peut abandonner sans se trahir. Ainsi font les héros, et c’est à quoi, une fois
qu'ils sont morts, on les reconnaît.

SACRILÈGE Une offense faite au sacré. Par exemple, au sens strict, cracher
sur un crucifix (ainsi Giordano Bruno, sur le bûcher). Ou bien,
au sens large : le viol, la torture, l'assassinat (le sacré, c’est ce qui peut être
profané : le corps humain est sacré). On voit que tous les sacrilèges ne se valent
pas. Celui de Giordano Bruno est l’un des actes les plus admirables que je
connaisse.

%— 517 —
%{\footnotesize XIX$^\text{e}$} siècle — {\it }
SADISME Une perversion, qui consiste à jouir, comme on voit chez Sade,
de la souffrance d’autrui. Se distingue de la cruauté par une
charge érotique plus forte. Cela toutefois n’excuse rien, sauf consentement
explicite du partenaire.

SAGE Celui qui n’a pas besoin, pour être heureux, de se mentir, ni de se
raconter des histoires, ni même d’avoir de la chance. On dirait qu’il
se suffit à lui-même ; c’est en quoi il est libre. Maïs la vérité est qu’il se suffit de
tout, ou que tout lui suffit. Cela le distingue assez de l’ignorant, pour lequel
{\it tout}, ce n'est jamais assez. C’est que l’ignorant veut prendre, posséder, garder,
quand le sage se contente de connaître, de goûter ({\it sapere}, d’où vient {\it sapiens},
c'est avoir du goût), de se réjouir. C’est moins un savant qu’un connaisseur.
Moins un expert qu’un amateur (au double sens du terme : celui qui aime,
celui qui ne fait pas profession). Moins un propriétaire qu’un homme libre (le
{\it jivan mukta} des Orientaux : le libéré vivant). Le sage est sans maître, mais aussi
sans maîtrise, sinon sur soi, sans Église, sans appartenance, sans attaches, sans
attachements (ce qu’il aime, il ne le possède pas, ni n’en est possédé). Même
son bonheur ne lui appartient pas : ce n’est qu’un peu de joie dans le grand
vent du monde. Il est dépris de lui-même et de tout. C’est pourquoi peut-être
il est heureux : parce qu’il n’a plus besoin de l'être. Et sage : parce qu'il ne croit
plus à la sagesse.

SAGESSE L'idéal d’une vie réussie — non parce qu’on aurait réussi {\it dans} la vie,
ce qui ne serait que carriérisme, mais parce qu’on aurait réussi sa
vie elle-même. C’est le but, depuis les Grecs, de la philosophie. Toutefois ce
n'est qu’un idéal, dont il importe de se libérer aussi. Le vrai sage n’a que faire
de réussir quoi que ce soit : sa vie ne lui importe pas plus, ni moins, que celle
d'autrui. Il se contente de la vivre, et il y trouve un {\it contentement} suffisant, qui
est la seule sagesse en vérité. « Pour moi j'aime la vie », disait Montaigne. C’est
en quoi il était sage : parce qu’il n’attendait pas que la vie soit aimable (facile,
agréable, réussie...) pour l'aimer. Question de tempérament ? Question de
doctrine ? Sans doute un peu des deux. On est plus ou moins doué pour la vie,
plus ou moins sage ; ceux qui le sont moins ont donc besoin, j’en sais quelque
chose, de philosopher davantage. Mais nul n’est sage absolument, ni tout entier :
tous ont besoin de philosopher, ne serait-ce que pour se déprendre de la philosophie
elle-même. De la sagesse ? Bien sûr : on ne l’atteint qu’à condition de
cesser d’y croire. L'homme le plus sage du monde, un caillot ou un virus suffit
à le rendre fou. Ou un chagrin plus fort que les autres et que sa sagesse. Il le
%— 518 —
%{\footnotesize XIX$^\text{e}$} siècle — {\it }
sait, et d'avance l’accepte. Ses échecs ne sont pas moins vrais que ses succès.
Pourquoi seraient-ils moins sages ? La sagesse, la vraie sagesse, n’est pas une
assurance tous risques, ni une panacée, ni une œuvre d’art. C’est le repos, mais
joyeux et libre, dans la vérité. Un savoir ? Tel est en effet le sens du mot, chez
les Grecs ({\it sophia}) comme chez les Latins ({\it sapientia}). Mais c’est un savoir très
particulier. « La sagesse ne peut être ni une science ni une technique », disait
Aristote : elle porte moins sur ce qui est vrai ou efficace que sur ce qui est bon,
pour soi et pour les autres. Un savoir ? Certes. Mais c’est un savoir-vivre.

Les Grecs distinguaient la sagesse théorique ou contemplative ({\it sophia}) de la
sagesse pratique ({\it phronèsis}). Mais l’une ne va guère sans l’autre, ou la vraie
sagesse, plutôt, serait la conjonction des deux. Elle se reconnaît à une certaine
sérénité, mais plus encore à une certaine joie, à une certaine liberté, à une certaine
éternité (le sage vit au présent : il sent et expérimente, comme disait Spinoza,
qu'il est éternel), à un certain amour... « De tous les biens que la sagesse
nous procure pour le bonheur de la vie tout entière, soulignait Épicure, l'amitié
est de beaucoup le plus grand » ({\it Maximes capitales}, XXVII). C’est que l'amour-propre
a cessé de faire obstacle. Que la peur a cessé de faire obstacle. Que le
manque a cessé de faire obstacle. Que le mensonge a cessé de faire obstacle. Il
n’y a plus que la joie de connaître : il n’y a plus que l'amour et la vérité. C’est
pourquoi nous avons tous nos moments de sagesse, quand l’amour et la vérité
nous suffisent. Et de folie, quand ils nous déchirent ou nous font défaut. La
vraie sagesse n’est pas un idéal ; c’est un état, toujours approximatif, toujours
instable (il n’est éternel, comme l'amour, que tant qu’il dure), c’est une expérience,
c’est un acte. Ce n’est pas un absolu, malgré les stoïciens (on est {\it plus ou
moins sage}), mais un maximum (comme tel relatif) : c’est le maximum de bonheur,
dans le maximum de lucidité. Il dépend de la situation de tel ou tel, des
capacités de tel ou tel (la sagesse n’est pas la même à Auschwitz ou à Paris, pour
Etty Hillesum ou pour Cavaillès), bref de l’état du monde et de soi. Ce n’est
pas un absolu ; c’est la façon, toujours relative, d’habiter le réel, qui est le seul
absolu en vérité. Cette sagesse-là vaut mieux que tous les livres qu’on a écrits
sur elle, qui risquent de nous en séparer. À chacun d'inventer la sienne.
« Quand bien même nous pourrions être savants du savoir d’autrui, disait
Montaigne, au moins sages ne pouvons-nous être que de notre propre sagesse »
({\it Essais}, I, 25).

SAINT Le mot se prend principalement en deux sens, l’un religieux, l'autre
moral.
Pour la religion, Le saint est celui qui est uni à Dieu (qui seul est saint absolument)
par la foi, l’espérance et la charité : il aime Dieu plus que tout, et que

%— 519 —
%{\footnotesize XIX$^\text{e}$} siècle — {\it }
soi. Aussi est-il déjà sauvé, par cet amour même, déjà bienheureux, déjà dans le
Royaume, qu'il ne quittera plus. Il agit bien sûr moralement, mais par amour
et foi plutôt que par devoir.

Du point de vue moral, le saint est celui dont la volonté se conforme en
tout à la loi morale, au point que celle-ci, pour lui, ne vaille plus comme obligation
ou devoir (ce qui suppose la contrainte), mais bien comme liberté
(comme autonomie en acte). Ce n’est vraiment possible, selon Kant, qu’en
Dieu ({\it C. R. Pratique}, T, Analytique, \S 7, scolie) ou qu'après la mort ({\it op. cit.},
Dialectique, IV). Toutefois nul n’est tenu d’être kantien, ni dispensé, dès cette
vie, de faire son devoir : on peut appeler {\it saint}, en un sens plus général, celui qui
le ferait toujours.

Rien n'empêche, en ce dernier sens, qu’un saint soit athée, ou qu’un athée
soit saint. Toutefois ce n’est pas la règle. Non seulement parce que la médiocrité
est toujours le plus probable, chez les croyants comme chez les incroyants,
mais aussi parce que la plupart des saints (pour autant qu’il en existe, ou dans
la mesure où ils le sont) auront tendance à croire vrai ce qu’ils aiment ou qui
les meut. Ils ont raison, puisque cette vérité est en eux, ou puisqu'ils sont cette
vérité-là. Mais tort, peut-être, de croire qu’elle existe aussi hors d’eux-mêmes,
hors du monde et absolument. Que nous fassions parfois le bien, c’est difficilement
contestable, même pour les pécheurs que nous sommes. Mais pourquoi
faudrait-il que le Bien existe indépendamment de nous, et nous fasse ?

Au sens moral comme au sens religieux, le saint se distingue du sage, qui
n’a pas besoin de croire, ni d’espérer, ni d’obéir. Un Dieu ? Le salut ? La Loi ?
Il y a bien longtemps que le sage ne se soucie plus de ces abstractions ! Il les
laisse aux philosophes, qui en ont besoin.

Le sage et le saint voisinent, comme dirait Heidegger, sur des monts
séparés.

Le saint, sommet de la foi ou de la morale.

Le sage, sommet de l'éthique.

De l'extérieur, ils se ressemblent tellement qu’on pourrait les confondre, et
d’ailleurs rien n’interdit, au moins en droit, qu’un même individu soit les deux
à la fois. Mais le saint n’a que faire d’être sage, ni le sage d’être saint.

Qu'ils n'existent ni l’un ni l’autre absolument, c’est une évidence (comment
un sommet serait-il absolu ?), mais qui ne les réfute pas. Cela donnerait
plutôt raison au saint, par l'humilité. Et au sage, par l'humour.

SAINTETÉ La perfection morale ou religieuse. Elle n'appartient absolument
qu'à Dieu, s’il existe, mais peut se dire, par extension, de
ceux qui sont unis à lui ou qui respectent en tout la loi morale. Ce n’est qu’un

%— 520 —
%{\footnotesize XIX$^\text{e}$} siècle — {\it }
idéal, comme la sagesse : nul ne saurait y prétendre sans s’en éloigner. Mais les
deux idéaux sont très différents : idéal de soumission dans un cas (soumission à
Dieu, soumission à la loi morale : les saints sont des {\it muslims}, comme on dirait
en arabe, c’est-à-dire, étymologiquement, des soumis) ; idéal de liberté dans
l’autre (le sage est un {\it jivan mukta}, comme on dit en Inde, un libéré vivant).
Toutefois cette opposition reste abstraite, comme ces idéaux eux-mêmes. Comment
se libérer sans se soumettre à la nécessité ? Comment obéir à la loi morale
(c’est-à-dire à la liberté en soi : autonomie) sans être libre ? Sages et saints,
lorsqu'ils se rencontrent, préfèrent parler d’autre chose ou sourire en silence.

SALAUD Le nom commun du méchant, ou plutôt du mauvais. Il ne fait pas
le mal pour le mal, mais par intérêt, par lâcheté ou par plaisir,
autrement dit par égoïsme : il fait du mal aux autres, pour son bien à soi. Le
salaud, ce serait donc l’égoïste ? Pas seulement, car alors nous le serions tous.
C’est l’égoïste sans frein, sans scrupules, sans douceur, sans compassion. La vulgarité
du mot traduit la bassesse de la chose, et se justifie par là.

Chez Sartre, le salaud est le gros plein d’être, celui qui se prend au sérieux,
celui qui se croit, celui qui oublie sa propre contingence, sa propre responsabilité,
son propre néant, celui qui fait semblant de n’être pas libre (c’est ce que
Sartre appelle la mauvaise foi), enfin qui fait le mal, lorsqu'il y trouve son
intérêt, en étant persuadé de sa propre innocence ou, s’il se sent parfois coupable,
d’innombrables circonstances atténuantes, qui l’excusent.

Ces deux définitions se rejoignent. Qu'est-ce qu’un salaud ? C’est un égoïste
qui a bonne conscience. Aussi est-il persuadé que le salaud, c’est l’autre. Il
s’autorise le pire, au nom du meilleur ou de soi — d’autant plus salaud qu’il se
croit justifié de l’être, et pense donc ne l'être pas. Comment s’imposerait-il
quelque frein que ce soit ? Pourquoi devrait-il se repentir ? Saloperie : égoïsme
de bonne conscience et de mauvaise foi.

SALUT Le fait d’être sauvé : « Il ne dut son salut qu’à la fuite ». En philosophie,
et pris absolument, le mot indique pourtant davantage qu’une
survie, toujours provisoire. Le vrai salut serait complet et définitif. Ce serait une
existence libérée de la souffrance et de la mort : la vie éternelle et parfaite. C’est
donc un mythe, qui relève comme tel du vocabulaire religieux. Ou bien il faut
considérer que l'éternité n’est pas autre chose que le présent, ni la perfection
autre chose que la réalité. C’est ce qui m’a fait dire parfois que nous sommes
déjà sauvés. Non parce que nous cesserions par là d’être perdus, mais parce que
le salut et la perte sont une seule et même chose. C’est ce que j’appelle le tragique,
%— 521 —
%{\footnotesize XIX$^\text{e}$} siècle — {\it }
et la seule sagesse qui ne mente pas. L’éternité, c’est maintenant : le salut
n'est pas une autre vie, mais la vérité de celle-ci. Nous sommes déjà dans le
Royaume. Aussi est-il vain de l’attendre, et même de l’espérer. C’est l'esprit de
Nagarjuna : « Tant que tu fais une différence entre le nirvâna et le samsâra, tu
es dans le samsära. » Tant que tu fais une différence entre ta vie telle qu’elle est
et le salut, tu es dans ta vie telle qu’elle est. C’est l'esprit de Prajnânpad : « La
vérité ne viendra pas ; elle est ici et maintenant. » Ce n’est plus religion, mais
sagesse. Plus promesse, mais don. Plus espérance, mais expérience. « La béatitude
est éternelle, écrit Spinoza, et ne peut être dite commencer que
fictivement » ({\it Éthique}, V, 33, sc.). Le salut est cette fiction, ou cette éternité.

SANGUIN L'un des quatre tempéraments selon Hippocrate et Gallien. Embonpoint,
teint vif, irritabilité, violence. On sait aujourd’hui que le
sang n'y est pour rien. Cela n’a pas suffi à les apaiser, ni à les faire maigrir.

SANTÉ  « La santé est un état précaire, qui ne présage rien de bon. » Le docteur
Knock avait évidemment raison. Celui qui n’est pas malade
peut toujours le devenir, et même, sauf accident mortel, le deviendra inévitablement.
Il n’y a pas de santé absolue, pas de santé définitive : il n’y a que le
combat contre la maladie, contre la mort, contre l’usure, et c’est la santé même.
Elle n’est pas seulement l’absence de maladies (puisqu’on peut être en {\it mauvaise
santé}), mais la force en nous qui leur résiste, autrement dit la vie elle-même,
dans son équilibre fonctionnel et efficace. « La vie est une victoire qui dure »,
disait Jean Barois, et chacun sait bien qu’elle ne durera pas toujours. La santé
n'est pas son triomphe, mais son combat continué.

Il faut citer, parce qu’elle est absurde, la définition qu’en donne l'Organisation
Mondiale de la Santé: « La santé n’est pas seulement l'absence de
maladie ou d’infirmité. C’est un état de complet bien-être physique, psychique
et social. » L'Union soviétique avait donc raison d’interner ses dissidents en
hôpital psychiatrique : ils étaient malades, puisque leur bien-être, spécialement
psychique et social, n’était pas complet... Quant à moi, si j'ai eu, depuis que je
suis né, trois jours de santé, au sens de l’'O.M.S., c’est un maximum. Les
moments de bien-être, cela m'arrive assez souvent. Mais complets, c’est une
autre histoire. « Il y a toujours quelque pointe qui va de travers », comme disait
Montaigne, toujours quelque souci, quelque douleur, quelque angoisse.
« Docteur, ce matin, j'ai pensé à la mort. Cela m'inquiète. Mon état de bien-être
n’est pas complet : vous ne pourriez pas me donner quelque chose ? » C’est
— 522 —
%{\footnotesize XIX$^\text{e}$} siècle — {\it }
confondre la santé et le salut, donc la médecine et la religion. Dieu est mort :
vive la Sécu !

Pourtant il est vrai, je le disais en commençant, que la santé n’est pas seulement
l’absence de maladies. Car alors les morts et les pierres seraient en
bonne santé. Non l’absence de maladies, donc, mais la puissance en nous, toujours
finie, toujours variable, toujours {\it précaire}, en effet, qui leur résiste ou les
surmonte. C’est pourquoi c’est le bien le plus précieux. Plus que la sagesse ?
Bien sûr, puisque aucune sagesse n’est possible sans une santé (notamment
mentale) au moins minimale. La santé n’est pas le souverain bien (elle ne tient
lieu ni de bonheur ni de vertu) ; mais elle est le bien le plus important —
puisqu'elle est la condition de tous. Ce n’est pas un salut ; c’est un combat. Pas
un but, un moyen. Pas une victoire, une force. C’est le {\it conatus} d’un vivant, tant
qu’il réussit à peu près.

SAUVAGERIE C’est comme une barbarie individuelle ou native, qui pour
cela inquiète moins. Il peut y avoir de bons sauvages ; il n’y
a pas de bons barbares.

La sauvagerie est proximité avec la nature («chez moi, en pays sauvage »,
écrit Montaigne : cela veut dire à peu près qu’il vit à la campagne). La barbarie
est distance d’avec la civilisation. Le sauvage n’est pas encore civilisé. Le barbare
ne l’est plus. Le sauvage est derrière nous. Le barbare, devant.

SAVOIR Comme substantif, c’est un synonyme à peu près de connaissance.
Si on veut les distinguer, on peut dire que la connaissance serait
plutôt un acte, dont le savoir serait le résultat. Ou que les connaissances sont
multiples ; le savoir serait plutôt leur somme ou leur synthèse. Ces différences
restent pourtant approximatives et fluctuantes : l’usage ne les impose ni ne les
interdit.

Comme verbe, en revanche, la différence est plus nette : je sais lire et
écrire ; je connais (plus ou moins) le vocabulaire, la grammaire, l'orthographe.
Je sais conduire ; je connais le code de la route. Je connais la musique pour
piano de Schubert ; je ne sais pas la jouer, ni la lire. Je connais plus ou moins
la vie ; je sais plus ou moins vivre. Il me semble que toutes ces expressions vont
à peu près dans le même sens, qui indique au moins une direction. Le {\it connaître}
porte sur un objet ou une discipline ; le {\it savoir} porte plutôt sur une pratique ou
un comportement. Connaître, c’est avoir une idée vraie ; savoir, c’est pouvoir
faire. C’est pourquoi il ne suffit pas de savoir penser pour connaître, ni de
connaître pour savoir penser.

%— 523 —
%{\footnotesize XIX$^\text{e}$} siècle — {\it }
SCÉLÉRATESSE Se conduire comme un scélérat, c’est-à-dire comme un criminel
ou, plus souvent, comme un salaud. (Le mot est
utile, parce que saloperie, en français, a un tout autre sens : une saloperie, c’est
l’acte d’un salaud ; la scélératesse, son caractère ou sa disposition).

SCEPTICISME Le contraire du dogmatisme, au sens technique du terme.
Être sceptique, c’est penser que toute pensée est douteuse —
que nous n’avons accès à aucune certitude absolue. On remarquera que le scepticisme,
sauf à se détruire, doit donc s’inclure dans le doute général qu’il
instaure : tout est incertain, y compris que tout soit incertain. À la gloire du
pyrrhonisme, disait Pascal. Cela n’interdit pas de penser, et même c’est ce qui
oblige à penser toujours. Les sceptiques cherchent la vérité, comme tout philosophe
(c’est ce qui les distingue des sophistes), mais ne sont jamais certains de
l'avoir trouvée, ni même qu’on le puisse (c’est ce qui les distingue des
dogmatiques). Cela ne les chagrine pas. Ce n’est pas la certitude qu’ils aiment,
mais la pensée et la vérité. Aussi aiment-ils la pensée en acte, et la vérité en
puissance : c’est la philosophie même, et c’est en quoi, disait Lagneau, « Le scepticisme
est le vrai ». Il en découle que nul n’est tenu d’être sceptique, ni autorisé
à l’être dogmatiquement.

SCIENCES Mieux vaut en parler au pluriel qu’au singulier. {\it La} science
n'existe pas : il n’y a que {\it des} sciences, et elles sont toutes différentes,
par leur objet ou leur méthode. Toutefois le pluriel, ici comme ailleurs,
suppose le singulier. Nul ne peut savoir ce que sont {\it les} sciences, s’il ne sait pas
ce que c’est qu’{\it une} science.

Disons d’abord ce que ce n’est pas. Ce n’est pas une connaissance certaine,
malgré Descartes, ni toujours une connaissance démontrée (puisqu’une hypothèse
peut être scientifique, puisqu'il n’y a pas de science sans hypothèses), ni
même une connaissance vérifiable (il est plus facile de vérifier la fermeture de
votre braguette que la non-contradiction des mathématiques, ce que nul ne
peut : cela ne retire rien à la scientificité des mathématiques, ni ne rend scientifique
votre comportement vestimentaire). Ce n’est pas non plus un ensemble
d'opinions ou de pensées, fût-il cohérent et rationnel — car alors la philosophie
serait une science, ce qu’elle n’est ni ne peut être.

Toute science, pourtant, relève bien de la pensée rationnelle ; disons que
c'est le genre prochain, dont les sciences sont une certaine espèce. Reste à
trouver leurs différences spécifiques. Qu'est-ce qu’une science? C’est un
ensemble de connaissances, de théories et d’hypothèses portant sur le même

%— 524 —
%{\footnotesize XIX$^\text{e}$} siècle — {\it }
objet ou le même domaine (par exemple la nature, le vivant, la Terre, la
société), qu’elle construit plutôt qu’elle ne le constate, historiquement produites
(toute vérité est éternelle, aucune science ne l’est), logiquement organisées
ou démontrées, autant qu’elles peuvent l'être, collectivement reconnues, au
moins par les esprits compétents (c’est ce qui distingue les sciences de la philosophie,
où les esprits compétents s’opposent), enfin — sauf pour les mathématiques —
empiriquement falsifiables. Si l’on ajoute à cela que les sciences s’opposent
ordinairement à l’opinion (une connaissance scientifique, c'est une
connaissance qui ne va pas de soi), on peut risquer une définition simplifiée :
{\it une science est un ensemble ordonné de paradoxes testables, et d'erreurs rectifiées}. Le
progrès fait partie de son essence ; non que les sciences avancent de certitude en
certitude, comme on le croit parfois, mais parce qu’elles se développent par
« conjectures et réfutations ».

Karl Popper, à qui j'emprunte cette dernière expression, s’est longtemps
battu, en un temps où c'était nécessaire, pour montrer que le marxisme et la
psychanalyse ne sont pas des sciences (aucun fait empirique n’est susceptible de
les réfuter). Il avait évidemment raison. Mais cette irréfutabilité, qui donne tort
à la plupart des marxistes et des psychanalystes, ne saurait valoir elle-même
comme réfutation, ou ne réfute que la prétendue scientificité des deux théories
en question. On évitera d’en conclure que marxisme et psychanalyse seraient
sans intérêt ou sans vérité. Tout ce qui est scientifique n’est pas vrai, tout ce qui
est vrai (ou possiblement vrai) n’est pas scientifique : la notion d’erreur scientifique
n’est pas contradictoire, celle de vérité scientifique n’est pas pléonastique.
C’est pourquoi la philosophie reste possible, et les doctrines, nécessaires.

SCIENTISME La religion de la science, ou la science comme religion. C’est
vouloir que les sciences disent l'absolu, quand elles ne peuvent
atteindre que le relatif, et qu’elles commandent, quand elles ne savent que
décrire ou (parfois) expliquer. C’est ériger la science en dogme, et le dogme en
impératif. Que resterait-il de nos doutes, de notre liberté, de notre
responsabilité ? Les sciences ne sont soumises ni à la volonté individuelle ni au
suffrage universel. Que resterait-il de nos choix? Que resterait-il de nos
démocraties ? Le mathématicien Henri Poincaré, contre cette niaiserie dangereuse,
a dit ce qu’il fallait : « Une science parle toujours à l'indicatif, jamais à
l'impératif. » Elle dit ce qui est, dans le meilleur des cas, plus souvent ce qui
paraît ou peut être, parfois ce qui sera, jamais ce qui {\it doit} être. C’est pourquoi
elle ne tient pas lieu de morale, ni de politique, ni, encore moins, de religion.
C’est ce que le scientisme méconnaît, et qui le condamne. Le positivisme, à
tout prendre, vaudrait mieux.

%— 525 —
%{\footnotesize XIX$^\text{e}$} siècle — {\it }
SCOLASTIQUE La doctrine et les procédés de l’École, c’est-à-dire, selon l’acception
la plus usuelle, des universités européennes au
Moyen-Âge : mélange de théologie chrétienne et de philosophie grecque
(d’abord platonicienne, du fait de l'influence de saint Augustin, puis de plus en
plus aristotélicienne), de logique et d'arguments d’autorité, de rigueur et de
monotonie..…. C’est un beau moment de l’esprit, où l'Occident s’invente, mais
qui finit par paralyser la pensée en l’enfermant dans des querelles aussi érudites
que stériles. Déjà Montaigne ne l’évoque que pour s’en moquer ou s’en
plaindre. Descartes, que pour l’enterrer. Tout indique pourtant qu’il y avait là
des trésors d’intelligence. Mais à quoi bon un trésor, quand on n’en a plus
l'usage ?

En un sens plus général et plus péjoratif, on appelle souvent {\it scolastique} la
doctrine d’une école, quelle qu’elle soit, dès lors qu’elle s’enferme dans une
orthodoxie déjà constituée (quitte à en complexifier indéfiniment les détails),
au point que c’est la pensée du Maître — et non plus l’accord avec le réel ou
l'expérience — qui décide de la vérité possible d’une proposition. Dogmatisme,
psittacisme, maniérisme. C’est ainsi qu’on a pu parler de la scolastique freudienne
(ou, en France, lacanienne), de la scolastique marxiste-léniniste, de la
scolastique heideggérienne.. Beaucoup d'intelligence dans les trois cas. C’est
ce qui rend la scolastique si dangereuse. Rien de tel pour stériliser un bel esprit.
J'en connais plusieurs qui sont passés pour cela à côté d’une œuvre possible.

SECTARISME Un certain type de comportement intellectuel, digne d’une
secte, indigne d’un esprit libre. C’est un mélange d’étroitesse,
d’intolérance et de conviction : certitude d’avoir raison, même contre tous,
mépris ou rejet des autres positions, toujours suspectées d’aveuglement ou de
mauvaise foi, culte du chef, de la doctrine ou de l’organisation. C’est le dogmatisme
des imbéciles.

SECTE  « Toute secte, disait Voltaire, est le ralliement du doute et de l’erreur. »
C’est qu’on ne dispute que sur ce qu’on échoue à connaître.
Il n’y a point de secte en géométrie, continuait Voltaire : « on ne dit point un
euclidien, un archimédien »... Même l'invention des géométries non euclidiennes
n’y a rien changé. Les sciences n’ont pas besoin d’absolu. L’universel
leur suffit. Toute religion, à l'inverse, est particulière : « Vous êtes mahométan,
donc il y a des gens qui ne le sont pas, donc vous pourriez bien avoir tort »
(Voltaire, {\it Dictionnaire philosophique}, article « Secte »). C’est ce qui énerve les
sectaires. Ils sentent bien que la pluralité des sectes, qui fait partie du concept,
%— 526 —
%{\footnotesize XIX$^\text{e}$} siècle — {\it }
est un formidable argument contre chacune d’entre elles. Vous êtes chrétien ;
c’est donc que tous ne le sont pas. Pourquoi auriez-vous raison davantage que
les autres ?

Qu'est-ce qu’une secte ? C’est une Église, vue par ceux qui n’en font pas
partie et qui la jugent sectaire. Le mot, en son sens moderne, vaut donc comme
rejet ou condamnation : la secte, c’est l’Église de l’autre. Ce n’est pas une raison
pour les interdire, tant qu’elles respectent la loi. Autant interdire la bêtise ou la
superstition.

On s'interroge sur la différence entre une secte et une Église. Je reprendrais
volontiers une formule qui n’est pas de moi : « Une Église, c’est une secte qui
a réussi. » Cela dit, par différence, ce qu’est une secte : une Église en gestation
ou en échec. Ses membres sont persuadés que le temps travaille pour eux, s’irritent
que cela aille si lentement, nous en veulent de ne rien faire pour accélérer
le processus. Ils sont pleins d’impatience, de mépris, de colère, de certitude.
C’est ce qui les rend sectaires. Redoutable engeance.

SÉLECTION Un choix par élimination. Par exemple la sélection naturelle
des espèces, selon Darwin, par l'élimination des moins aptes.
Ou la sélection des meilleurs, à l’Université, par l'élimination des plus faibles,
des plus pauvres ou des moins studieux. Nos étudiants sont contre, ils l'ont fait
vertement savoir en de multiples occasions, et nos politiques, qui sont gens
prudents, ont renoncé à en parler. Cela n’a jamais empêché que la sélection se
fasse, bien sûr par l'échec (quelle sélection autrement ?), mais n’aide pas à la
faire dans de bonnes conditions — sur des critères purement scolaires, comme il
faudrait, et non en fonction des moyens financiers des parents. J'aimerais
mieux des examens plus sévères, et des bourses plus généreuses : cela serait
moins injuste et plus efficace.

SENS {\it Sens} se dit principalement en trois sens : comme sensibilité (le sens de
l’odorat), comme direction (le sens d’un fleuve), comme signification
(le sens d’une phrase). Un sens, c’est ce qui sent ou ressent, ce qu’on suit ou
poursuit, enfin ce qu’on comprend.

Le premier de ces sens est défini ailleurs (voir les articles « Sensation » et
« Sensibilité »). Les deux autres sont liés, au moins pour nous : le but d’une
action lui donne aussi une signification (si vous courez pour aller plus vite, cela
signifie vraisemblablement que vous êtes pressés) ; et la signification d’une
phrase, c’est ce qu’elle veut dire ou obtenir, autrement dit le but que poursuit
celui qui l’énonce ou vers lequel, même inconsciemment, il tend. Avoir un

%— 527 —
%{\footnotesize XIX$^\text{e}$} siècle — {\it }
sens, c'est {\it vouloir dire} ou {\it vouloir faire}. Cette volonté peut être explicite ou
implicite, consciente ou inconsciente, elle peut même n’être que l’apparence
d’une volonté ; cela nuance, mais n’annule pas, cette caractéristique générale :
il n’est de sens que là où intervient une volonté ou quelque chose qui lui ressemble
(un désir, une tendance, une pulsion). La sphère du sens et celle de
l’action se recouvrent : toute parole est un acte ; tout acte est un signe ou peut
être interprété comme tel.

Il en résulte qu’il n’est de sens que pour un sujet (que pour un être capable
de désirer ou de vouloir), et par lui. Un sens objectif ? C’est une contradiction
dans les termes. Un sens absolu ? Cela supposerait un Sujet absolu, qui serait
Dieu. On parle pourtant, je le signalais en commençant, du sens d’un fleuve.
Mais ce n’est, précisément, qu’une façon de parler. Si je dis par exemple que la
Loire coule d’Est en Ouest, ou qu’elle se dirige vers l’océan, cela ne suffit pas à
lui donner un sens : non seulement parce que la Loire ne veut rien dire (elle n’a
pas de signification), mais aussi parce qu’elle ne se dirige en vérité vers rien :
elle ne fait que suivre la pente. De même quand on dit qu’une fleur se dirige
vers le Soleil. Cela ne fait sens que pour nous, point pour elle : son phototropisme
doit tout à la nature, rien à la finalité ou à l’herméneutique.

Faut-il dire alors qu’il n’est de sens qu’humain ? Je n’en suis pas sûr. Des
animaux peuvent poursuivre un but et interpréter, même à l’état sauvage, le
comportement de tel ou tel de leurs congénères. Les éthologues, là-dessus, nous
renseignent suffisamment. Faut-il dire qu’il n’est de sens que pour une
conscience ? Pas davantage : la psychanalyse nous a assez éclairés sur la signification
inconsciente de tel ou tel de nos actes, de nos rêves ou de nos symp-
tômes. Je dirais plutôt qu’il n’est de sens que pour un être capable de désirer,
donc sans doute capable de souffrir et de jouir. C’est où l’on retrouve le mot
« sens » en sa première acception : il n’est de sens (comme signification ou
direction) que pour un être doué de sens (comme sensibilité), et proportionnellement
sans doute à cette faculté. La frontière est floue ? Pourquoi ne le serait-elle
pas ? L'homme n’est pas un empire dans un empire. Le sens non plus.

On remarquera que dans ces trois acceptions principales, et spécialement
dans les deux qui nous occupent (comme direction et comme signification), le
sens suppose une extériorité, une altérité, disons une relation à autre chose qu’à
soi. Prendre l’autoroute en direction de Paris n’est possible qu’à condition de
{\it n'être pas} à Paris. Et un signe n’a de sens que dans la mesure où il renvoie à
autre chose qu’à ce signe. Quel mot qui se signifie soi ? Quel acte qui se signifie
soi ? Tout mot signifie autre chose que lui-même (une idée : son signifié ; ou
un objet : son référent). Tout acte signifie autre chose que lui-même (son but,
conscient ou inconscient, ou le désir qui le vise). Pas de sens qui soit purement
intrinsèque : vouloir dire ou vouloir faire, c’est toujours vouloir autre chose que

%— 528 —
%{\footnotesize XIX$^\text{e}$} siècle — {\it }
soi. C’est ce qu'avait vu Merleau-Ponty : « Sous toutes les acceptions du mot
{\it sens}, nous retrouvons la même notion fondamentale d’un être orienté ou polarisé
vers ce qu'il n’est pas » ({\it Phénoménologie de la perception}, III, 2). Le sens
d’un acte n’est pas cet acte. Le sens d’un signe n’est pas ce signe. C’est ce qu’on
peut appeler la structure extatique du sens (il est toujours ailleurs). Nul ne peut
aller où il se trouve, ni se signifier soi. C’est ce qui nous interdit le confort,
l’autoréférence satisfaite, peut-être même le repos. On ne s’installe pas dans le
sens comme dans un fauteuil. On ne le possède pas comme un bibelot ou un
compte en banque. On le cherche, on le poursuit, on le perd, on l’anticipe…
Le sens n’est jamais là, jamais présent, jamais donné. Il n’est pas où je suis, mais
où je vais ; non ce que nous sommes ou faisons, mais ce que nous voulons faire,
ou qui nous fait. Il n’est sens, à jamais, que de l’autre.

Le sens de la vie ? Ce ne pourrait être qu'autre chose que la vie : ce ne pourrait
être qu’une autre vie ou la mort. C’est ce qui nous voue à l’absurde ou à la
religion. Le sens du présent ? Ce ne peut être que le passé ou l'avenir. C’est ce
qui nous voue au temps. Un fait quelconque n’a de sens, ici et maintenant, que
pour autant qu’il annonce un certain avenir (c’est la logique de l’action, tout
entière tendue vers son résultat) ou résulte d’un certain passé (c’est la logique
de l'interprétation, par exemple en archéologie ou en psychanalyse). Le sens de
ce qui est, c’est ce qui n’est plus ou pas encore : le sens de l'être, c’est le temps.
C’est ce qui justifie la belle formule de Claudel, dans {\it L'art poétique} : « Le temps
est le sens de la vie ({\it sens} : comme on dit le sens d’un cours d’eau, le sens d’une
phrase, le sens d’une étoffe, le sens de l’odorat). » Mais c’est aussi pourquoi le
sens, comme le temps, ne cesse de nous fuir, et d’autant plus qu’on le cherche
davantage : le sens du présent n’est jamais présent. Aussi le sens, comme le
temps, ne cesse-t-il de nous séparer de nous-mêmes, du réel, de tout. On le
trouve parfois, mais le sens qu’on trouve, comme dit Lévi-Strauss, « n’est
jamais le bon » ({\it La pensée sauvage}, IX), ou n’a lui-même de sens que par autre
chose, qui n’en a pas ou que l’on cherche. La quête du sens est par nature
infinie. C’est ce qui nous condamne à l’insatisfaction : toujours cherchant autre
chose, qui serait le sens, toujours cherchant le sens, qui ne peut être qu'autre
chose. Mais comment autre chose que le réel (son sens) serait-il réel ? « Le sens
du monde doit se trouver hors du monde », disait à juste titre Wittgenstein.
Mais hors du monde, quoi, sinon Dieu ? Le sens du présent, pareillement, doit
se trouver en dehors du présent. Mais hors du présent, quoi, sinon le passé ou
l'avenir, qui ne sont pas ? Sens, c’est absence : il n’est là (pour nous) qu'en tant
qu’il n’y est pas (en soi). Il y a donc du sens dans ma vie, puisque je me projette
vers l'avenir, puisque je reste marqué par mon passé, puisque j'essaie d'agir et
de comprendre. Mais comment ma vie elle-même aurait elle un sens, si elle ne
peut avoir que celui qu’elle n’a plus ou pas encore ?

%— 529 —
%{\footnotesize XIX$^\text{e}$} siècle — {\it }
« Quand le doigt montre la lune, l’imbécile regarde le doigt. » Ce proverbe
oriental va plus loin qu’il ne paraît. Cet imbécile nous ressemble, ou c’est nous,
bien souvent, qui lui ressemblons. Que fait-il ? Il regarde ce qui a du sens (le
doigt) plutôt que ce que le sens désigne, qui n’en a pas (la lune). Il se trompe
sur le sens, qui le fascine, et méconnaît le réel. Ainsi faisons-nous, à chaque fois
que nous sacrifions ce qui est à ce que cela pourrait signifier ou annoncer. Renversons
plutôt les priorités. Le sens ne vaut qu’au service d’autre chose, qui n’en
a pas. Comment serait-il le Tout (puisque le Tout, par définition, n’a pas
d’autre) ? Comment serait-il l’essentiel ? Rien de ce qui importe vraiment n’a
de sens. Que signifient nos enfants ? Que signifie le monde ? Que signifie
l'humanité ? Que signifie la justice ? Ce n’est pas parce qu’ils ont du sens que
nous les aimons ; c’est parce que nous les aimons que notre vie, pour nous,
prend sens. Une illusion ? Non pas, puisqu'il est vrai que nous les aimons.
L’illusion serait d’hypostasier ce sens, d’en faire un absolu, de croire qu’il existe
hors de nous et de sa quête. Ce serait Dieu — et la question se poserait d’ailleurs
de savoir ce qu’il pourrait bien {\it signifier}. Mais à quoi bon ? L'action suffit. Le
désir suffit. Il n’y a pas de sens du sens, ni de sens absolu, ni de sens en soi.
Tout sens, par nature, est relatif : ce n’est pas une substance, c’est un rapport ;
ce n'est pas un être, c’est une relation. C’est toujours la logique de l’altérité :
tout ce que nous faisons, qui a du sens, ne vaut qu’au service d’autre chose, qui
n'en a pas. Ce n’est pas le sens qu’il faut poursuivre, c’est ce qu’on poursuit qui
fait sens.

Souvenons-nous du Laboureur et de ses enfants. Il n’y a pas de trésor
caché, montre La Fontaine, mais le travail en est un, et le seul. Je dirais volontiers
la même chose du sens : il n’y a pas de sens caché, mais la vie en produit
et elle seule. Le sens n’est pas à chercher, ni à trouver, comme s’il existait déjà
ailleurs, comme s’il nous attendait. Ce n’est pas un trésor ; c’est un travail. Il
n'est pas tout fait: il est à faire (mais toujours en faisant autre chose), à
inventer, à créer. C’est la fonction de l’art. C’est la fonction de la pensée. C’est
la fonction de l'amour. Le sens est moins la source d’une finalité que le résultat
ou la trace d’un désir (or le désir, rappelle Spinoza, est cause efficiente). Moins
l’objet d’une herméneutique que d’une poésie — ou il ne peut y avoir herméneutique,
plutôt, que là où il y a eu d’abord {\it poièsis}, comme on dirait en grec,
c'est-à-dire création : dans nos œuvres, dans nos actes, dans nos discours. Le
sens n'est pas un secret, qu'il faudrait découvrir, ni un Graal, qu’il faudrait
atteindre. Il est un certain rapport, mais en nous, entre ce que nous sommes et
avons été, entre ce que nous sommes et voulons être, entre ce que nous désirons
et faisons. Ce n’est pas parce que la vie a un sens qu’il faut l’aimer ; c’est parce
que nous l’aimons, ou dans la mesure où nous l’aimons, qu’elle prend, pour
nous, du sens.

%— 530 —
%{\footnotesize XIX$^\text{e}$} siècle — {\it }
La vie a-t-elle un sens ? Aucun qui la précède ou la justifie absolument.
« Elle doit être elle-même à soi sa visée », comme dit Montaigne (III, 12,
1052). Elle n’est pas une énigme, qu’il faudrait résoudre. Ni une course, qu’il
faudrait gagner. Ni un symptôme, qu’il faudrait interpréter. Elle est une aventure,
un risque, un combat — qui vaut la peine, si nous l’aimons.

C’est ce qu’il faut rappeler à nos enfants, avant qu’ils ne crèvent d’ennui ou
de violence.

Ce n’est pas le sens qui est aimable ; c’est l'amour qui fait sens.

SENS COMMUN C'est le bon sens installé : moins une puissance de juger
que son résultat socialement disponible et reconnu, autrement
dit un ensemble d’opinions ou d’évidences qu’il serait déraisonnable,
croit-on, de contester. L'expression, qui valait d’abord positivement (voyez le
{\it Lalande}), tend de plus en plus à devenir suspecte, voire péjorative. Nous avons
appris à nous méfier des évidences : le soupçon, face à l’unanimité, est notre
première réaction. C’est notre sens commun à nous.

SENSATION Une perception élémentaire, ou l'élément d’une perception
possible. Il y a sensation lorsqu'une modification physiologique,
d’origine le plus souvent externe, excite l’un quelconque de nos sens. Par
exemple l’action de la lumière sur la rétine, ou des vibrations de l'air sur le
tympan, entraînent des modifications, via le système nerveux, jusqu’au
cerveau : c’est ce qui nous permet de prendre conscience de ce que nous voyons
ou entendons.

La perception est plutôt du côté de la conscience ; la sensation, du côté du
corps : elle fournit la matière que la perception mettra en forme. C’est pourquoi
c’est une abstraction, qui n’existe jamais seule. Nous n’avons affaire qu'à
des sensations plurielles, liées, organisées — qu’à des perceptions. Celles-ci sont
en quelque sorte au-delà du corps. La sensation, en deçà de l'esprit. C’est ce qui
faisait dire à Lagneau que « la sensation n’est pas une donnée de la conscience ».
Mais il ny aurait pas de conscience sans elle, ou ce ne serait qu’une conscience
vide.

La perception suppose la sensation ; elle ne s’y réduit pas. On ne peut percevoir
sans sentir ; mais il est possible de sentir sans percevoir. Par exemple le
contact du sol sous mes pieds : il est vraisemblable que je le sens toujours, du
moins quand je suis debout ou assis ; je ne le perçois qu’assez rarement. Ou le
bruit au loin de la rue : il est vraisemblable que je l’entends toujours ; je ne le
perçois (je ne m'aperçois que je l’entends) que lorsqu'il est spécialement fort ou

%— 531 —
%{\footnotesize XIX$^\text{e}$} siècle — {\it }
lorsque je l'écoute. C’est que la perception suppose une activité ou une attention,
au moins minimale, de l'esprit ; la sensation se suffit d’un esprit passif, ou
de la seule activité du corps. Ainsi quand je dors : j'entends, puisqu’un bruit
peut me réveiller. Mais je ne perçois aucun son. Cela dit, par différence, ce
qu'est la perception : non forcément une sensation active (percevoir un son
n'est pas forcément l’écouter), mais une sensation, ou plus souvent un
ensemble de sensations, dont on prend conscience ou auxquelles on prête
attention. La sensation est la même chose, abstraction faite de cette attention et
même de cette conscience. Mais il est vraisemblable que cette {\it abstraction} existe
d’abord et très concrètement : c’est l’ouverture du corps au monde, comme la
perception est ouverture de l’esprit au corps et à tout.

SENSIBILITÉ La faculté de sentir ou de ressentir. Le mot peut désigner la
condition en nous d’un phénomène physique (la sensation),
affectif (le sentiment), voire intellectuel (le bon sens, comme sensibilité au vrai
ou au réel). Kant nous a habitués à considérer la sensibilité comme purement
réceptive ou passive. « La capacité de recevoir (réceptivité) des représentations
grâce à la manière dont nous sommes affectés par les objets se nomme {\it sensibilité}.
Ainsi, c’est au moyen de la sensibilité que des objets nous sont {\it donnés}, seule
elle nous fournit des {\it intuitions} ; mais C’est l’entendement qui pense ces objets,
et c'est de lui que naissent les {\it concepts} » ({\it C. R. Pure}, Esthétique transcendantale,
\S 1). Mais ce n’est passivité que pour l'esprit ; le corps, lui, fait activement son
travail, qui est de réagir aux excitations extérieures ou intérieures. C’est pourquoi
un bruit, une lumière ou une douleur peuvent nous réveiller. Parce que la
sensibilité, elle, ne dort jamais. C’est le travail du corps, et le repos de l'esprit.

SENSIBLE Qui est doué de sensibilité, ou qui peut être perçu par les sens.
En philosophie, cette seconde acception est plus fréquente : le
monde sensible s'oppose au monde intelligible, depuis Platon, comme ce qui
est perçu par les sens à ce qui est connu par l’esprit. Mais notre monde, le seul
que nous puissions expérimenter et connaître, est l’unité des deux.

SENSUALISME La doctrine qui veut ramener toutes nos connaissances aux
sensations. Le mot est souvent pris péjorativement, mais à
tort. L’épicurisme, par exemple, est un sensualisme : les trois critères de la
vérité — les sensations, les anticipations et les affections — se ramènent au premier
d’entre eux (Diogène Laërce, X, 31-34), si bien que les sens, comme on
%— 532 —
voit chez Lucrèce, sont la source, le fondement et la garantie de toute connaissance
vraie ({\it De rerum}, IV, 479-521). Reste à ne pas transformer ce sensualisme
en sottise. Épicure ni Lucrèce n’ont jamais dit qu’on pouvait sentir la vérité
elle-même, ni qu’il suffisait de regarder pour comprendre. Ils ont même dit fort
clairement le contraire : les yeux ne peuvent connaître la nature des choses (IV,
385), pas plus qu’aucun sens ne peut percevoir les atomes ou le vide, qui sont
pourtant la seule réalité. Sensualisme paradoxal, donc, mais sensualisme : toute
vérité est insensible, mais toute vérité vient des sensations. Il ne suffit pas de
sentir pour connaître : le sensualisme d’Épicure est aussi un rationalisme (au
sens large) ; mais aucune connaissance, sans la sensation, ne serait possible :
c’est en quoi le rationalisme d’Épicure est d’abord un sensualisme. Connaître
est plus que sentir ; mais ce {\it plus} lui-même (la raison, les anticipations.….) est
issu des sensations et en dépend (D.L., X, 32 ; {\it De rerum}, IV, 484). Sensualisme
rationaliste, donc, qui suppose une théorie sensualiste de la raison.

SENTIMENT Ce qu’on ressent, c’est-à-dire la conscience qu’on prend de
quelque chose qui se passe dans le corps, qui modifie notre
puissance d’exister et d’agir, comme dit Spinoza, et spécialement notre joie ou
notre tristesse. C’est le nom ordinaire des affects (voir ce mot), en tant qu'ils
sont durables (par différence avec les émotions) et concernent l'esprit ou le
cœur davantage que le corps ou les sens (par différence avec les sensations).

Le corps sent, l'esprit ressent : c’est à peu près la différence qu'il y a entre
une sensation et un sentiment. La sensation est une modification ({\it affectio}) du
corps ; le sentiment, un affect ({\it affectus}) de l'âme. On évitera pourtant de trop
forcer l’opposition. Si l’âme et le corps sont une seule et même chose, comme
dit Spinoza et comme je le crois, la différence entre les sentiments et les sensations
est plus de point de vue que d’essence : point de vue organique ou physiologique
dans un cas, affectif ou psychologique dans l’autre. Subjectivement,
pourtant, cette différence reste importante : ce n’est pas la même chose qu'avoir
mal et être triste, que se cogner contre un mur ou contre ses angoisses, que de
voir un visage ou d’en tomber amoureux. La sensation est un rapport au corps
et au monde ; le sentiment, un rapport à soi et à autrui.

SÉRIEUX Ce qui mérite attention et application, ou celui qui en fait preuve.
À ne pas confondre avec la dignité, qui mérite respect, ni avec la
gravité, qui se confronte davantage au tragique. Le sérieux, c'est ce avec quoi on
ne doit pas plaisanter, ou celui qui ne plaisante pas. C’est pourquoi le mot est
souvent péjoratif, soit parce qu’on le confond avec l’esprit de sérieux, soit parce

%— 533—
%{\footnotesize XIX$^\text{e}$} siècle — {\it }
qu'on y voit un manque d’humour ou de légèreté. Cela peut arriver. Mais le
sérieux peut aussi marquer les limites de l'humour, qui le séparent de la
frivolité : qu’on puisse rire de tout, comme c’est en effet le cas, cela ne saurait
dispenser quiconque de faire son devoir. Le sérieux, dans certaines circonstances,
est une exigence éthique : celle de la responsabilité, de la constance, de
« l'engagement, comme disait Mounier, sans duperie et sans avarice ». Les
parents le savent bien : comme ils sont devenus sérieux, dès leur premier
enfant ! Cela ne les empêche pas de rire, et de leur sérieux même. Mais ils
savent bien qu’aucun rire désormais ne saurait les exempter de leurs responsabilités.
Cela vaut, plus généralement, dans toute situation qui nous confronte à
nos devoirs. Nul n’est tenu d’être un héros ; nul n’est dispensé d’assumer ses
responsabilités. « Il ne s’agit pas d’être sublime, disait Jankélévitch, il suffit
d’être fidèle et sérieux » ({\it L'imprescriptible}, p. 55).

SÉRIEUX (ESPRIT DE) Se prendre soi-même au sérieux, ou ériger en
absolu les valeurs dont on se réclame. C’est
oublier le néant qu’on est et qui nous attend, mais aussi sa propre liberté (selon
Sartre), sa propre fragilité, sa propre dépendance, sa propre contingence.
Manque de lucidité et d'humour : c’est pécher, doublement, contre l'esprit.

SERVILITÉ L’attitude d’un esclave ({\it servus}) ou d’un inférieur, quand il a
intériorisé sa propre soumission au point de la croire légitime.
« C’est une flatterie en action », disait Alain : « Tout fait signe qu’on exécutera
et qu'on approuvera. La servilité n'attend pas les ordres, elle les espère, elle se
précipite au-devant » {\it (Définitions)}. C’est une obéissance sans résistance, sans
révolte, sans dignité. Mauvaise obéissance. Il est indigne de la manifester, mais
plus encore de la susciter ou de l’encourager. La servilité est une soumission qui
se voudrait flatteuse, et qui est en vérité insultante. De quel droit le traiterais-je
comme un esclave ? De quel droit me traite-t-il comme un esclavagiste ?

SERVITUDE Soumission de fait, sans choix et sans limites, à un pouvoir
extérieur. C’est le contraire de la liberté, de l'indépendance, de
l'autonomie, mais aussi de la citoyenneté (qui est soumission de droit à un souverain
légitime, dont on fait partie) et même de la simple obéissance à une
autorité que lon s’est choisie ou que l’on accepte, dans des limites qui sont
celles de la dignité et de la responsabilité.

%— 534 —
%{\footnotesize XIX$^\text{e}$} siècle — {\it }
La Boétie, s'agissant de politique, parle de {\it servitude volontaire} : non que
personne choisisse d’être esclave, mais parce que nul tyran ne pourrait régner
sans le soutien, ou en tout cas l’acceptation, du plus grand nombre. Toutefois
cette volonté doit moins à un libre choix qu’à un système de croyances et d’illusions.
C’est ce qu’a vu Spinoza : « Le grand secret du régime monarchique et
son intérêt majeur est de tromper les hommes et de colorer du nom de religion
la crainte qui doit les maîtriser, afin qu’ils combattent pour leur servitude
comme s’il s'agissait de leur salut. On ne peut, par contre, rien concevoir ni
tenter de plus ficheux dans une libre république, puisqu'il est entièrement
contraire à la liberté commune que le libre jugement propre soit asservi aux
préjugés ou subisse aucune contrainte » ({\it T. Th. P.}, Préface).

Dans le champ de la philosophie éthique, on parle aussi de servitude pour
désigner la soumission d’un individu à ses passions : c’est obéir à son corps ou
à ses affects, au lieu de les commander ou d’essayer (en les comprenant) de s’en
libérer. C’est ainsi que Spinoza intitule la quatrième partie de son {\it Éthique} « De
la servitude humaine ou de la force des affects » : « J'appelle {\it servitude}, explique-t-il
dans sa préface, l'impuissance de l’homme à gouverner ses affects ; tant qu'il
leur reste soumis, en effet, l’homme ne relève pas de lui-même mais de la fortune,
dont le pouvoir est tel sur lui que souvent il est contraint, voyant le
meilleur, de faire le pire. » On sait que la cinquième partie s’intitulera « De la
puissance de l’entendement ou de la liberté humaine ». Que ce soit politique
ou morale, on ne sort de la servitude que par la raison, qui n’obéit à personne.

SEXE C’est une partie du corps (les organes génitaux) en même temps
qu’une fonction, elle-même multiple (d’excitation, de plaisir, de
copulation, de reproduction…), l’une et l’autre divisant la plupart des espèces
animales en deux genres — on dit aussi en deux sexes —, qui sont les femelles et
les mâles. C’est notre façon d’appartenir à l'espèce (être humain, c’est être
femme ou homme), de pouvoir en jouir (par l’orgasme) et la prolonger (par la
reproduction). Beaucoup de plaisirs et de soucis en perspective.
« Le bas-ventre, disait Nietzsche, est cause que l’homme ait quelque peine
à se prendre pour un Dieu » ({\it Par-delà le bien et le mal}, IV, 141). C’est qu'un
Dieu doit être libre, et que nul ne l’est de son sexe : nous ne choisissons pas
d’en avoir un, ni lequel, pas plus que nous ne décidons de la force ou de la faiblesse
de ses désirs, ni de notre puissance, ou de notre impuissance, à leur
résister ou à les satisfaire... Les philosophes, pour cette raison, ont souvent
parlé du sexe avec suspicion ou dédain, quand ce n’est pas avec sottise ou pudibonderie.
Tant pis pour eux. Je ne sais guère que Montaigne qui en ait parlé
comme il faut, avec plaisir et humour, simplicité et vérité (voir spécialement

%— 535 —
%{\footnotesize XIX$^\text{e}$} siècle — {\it }
l’admirable chapitre 5 du livre III, « Sur des vers de Virgile »). « Chacune de
mes pièces me fait également moi que toute autre, disait-il. Et nulle autre ne
me fait plus proprement homme [ou femme] que cette-ci. » C’est que le désir,
non la liberté ou la raison, est l’essence de l'humanité, et que ce désir, sans être
uniquement sexuel, est toujours et tout entier sexué. Cela même qui nous
empêche de nous prendre pour un Dieu nous oblige à nous reconnaître animaux,
et à {\it devenir} humains. Jouir de l’autre, s’il y consent, ou le faire jouir, si
nous en sommes capables, cela ne saurait nous autoriser à l’asservir. Le désirer,
cela ne saurait nous dispenser de l’aimer et de le respecter.

SEXISME Une forme de racisme, fondée sur la différence sexuelle. Mieux
toléré que le racisme ordinaire. C’est qu’il est plus fréquent.

SEXUALITÉ Tout ce qui concerne le sexe, et spécialement les plaisirs qu’on
y trouve ou qu'on y cherche. C’est moins un instinct qu’une
fonction, moins une fonction qu’une puissance : puissance de jouir, et de faire
jouir. C’est le désir même, en tant qu’il est sexué. L’essence de l’homme, donc,
et de la femme, en tant qu’ils n’ont pas la même.

SIGNAL Définition parfaite chez Prieto : « Un signal est un fait qui a été
produit artificiellement pour servir d'indice ». Il annonce un fait
ou ordonne une action. Se dit surtout, en pratique, des signes non linguistiques.

SIGNE Tout objet susceptible d’en représenter un autre, auquel il est lié par
ressemblance ou par analogie (on parle alors d’icône ou de symbole),
par une relation causale (on parle alors d’indice ou de symptôme), ou encore et
surtout par convention (les Anglo-Saxons parlent alors de {\it symbol} ; mieux vaut
parler en français de signe conventionnel, ou de signe strictement dit). Le signe
linguistique entre bien sûr dans cette dernière catégorie. Il n’unit pas une chose
et un nom, montre Saussure, mais un concept (le signifié) et une image acoustique
(le signifiant). Le signe est l’unité des deux ; et c’est cette unité intra-linguistique
qui peut éventuellement désigner autre chose à l'extérieur du langage
(le référent). Le lien unissant le signifiant au signifié est purement
conventionnel : c’est ce que Saussure appelle « l'arbitraire du signe » ({\it Cours de
linguistique générale}, Y, chap. 1). Et celui unissant le signe à son référent, sauf

— 536 —
%{\footnotesize XIX$^\text{e}$} siècle — {\it }
exception (les onomatopées), l’est tout autant. Cela ne signifie pas qu’on puisse
utiliser n'importe quel signifiant pour signifier n'importe quelle idée, ni
n'importe quel signe pour désigner n’importe quoi, mais que ce rapport est fixé
par une règle, non imposé par la nature ou suggéré par une ressemblance.

SIGNIFIANT/SIGNIFIÉ Les deux faces du signe, spécialement linguistique :
côté son, côté sens. Le {\it signifiant}, c’est la réalité
matérielle, ou plutôt sensorielle, du signe (le son que lon émet quand on le
prononce). Le {\it signifié}, c’est sa réalité intellectuelle ou mentale : le concept ou la
représentation que le signifiant véhicule (ce qu’on veut dire ou qu'on comprend
grâce à lui). Le signe est l'unité indissoluble des deux (De Saussure,
{\it Cours de linguistique générale}, TX, chap. 1). On remarquera que le rapport entre
le signifiant et le signifié, qui est arbitraire, reste interne au signe ; c’est ce qui
distingue le {\it signifié} du {\it référent}, et la {\it signification} de la {\it désignation}.

SIGNIFICATION C'est un rapport, interne au signe, entre un signifiant et
un signifié. À distinguer de la {\it désignation} (ou dénotation),
qui est un rapport entre le signe et ce à quoi il renvoie à l'extérieur de lui-même
(son référent). Par exemple quand je dis : « Il y a un oiseau sur la branche. » Le
rapport entre le signifiant (la réalité sonore et sensorielle du mot « oiseau », ce
que Saussure appelle son image acoustique) et le signifié (le concept d'oiseau)
reste interne au signe, qui est l’unité indissociable des deux : c’est ce rapport
qu’on appelle la {\it signification}. Le rapport de {\it désignation}, au contraire, tout en
restant intralinguistique (il ne vaut qu’à l’intérieur d’une langue donnée), unit
un signe à un objet qui existe en dehors du langage et, le plus souvent, indépendamment
de lui : l'oiseau n’est pas un signe, et n’habite que le silence.

SILENCE Au sens où je prends le mot, c’est l’absence non de sons mais de
sens. Un bruit peut donc être silencieux, comme un silence peut
être sonore. Ainsi le bruit du vent, ou le silence de la mer.

Le silence, c’est ce qui reste quand on se tait — c’est-à-dire tout, moins le
sens que nous lui prêtons (y compris donc ce sens même, quand nous cessons
de lui en chercher un autre). Ce n’est qu’un autre nom pour le réel, en tant
qu’il n’est pas un nom.

C'est aussi l’état ordinaire du vivant. « La santé, c’est le silence des
organes » (Paul Valéry). La sagesse, le silence de l'esprit.

%— 537 —
%{\footnotesize XIX$^\text{e}$} siècle — {\it }
À quoi bon interpréter toujours, parler toujours, signifier toujours ? Écoute
plutôt le silence du vent.

SIMPLE Ce qui est indivisible ou indécomposable (« simple, dit Leibniz,
c'est-à-dire sans parties »). Se dit aussi, mais par abus de langage, de
ce qui est facile à comprendre ou à faire. De là peut-être un troisième sens, qui
désigne une espèce de vertu, comme une facilité à vivre et à être soi. Être
simple, en ce dernier sens, c’est exister tout d’une pièce, sans duplicité, sans
calcul, sans composition : c’est être ce qu’on est, sans se soucier de le paraître,
sans s’efforcer d’être autre chose, c’est ne pas faire semblant, c’est n’être ni snob
ni intéressé, ni hystérique ni manipulateur. Je ne connais pas de vertu plus
agréable, et c’est à quoi peut-être les simples se reconnaissent le mieux : ils sont
faciles à vivre, à comprendre, à aimer.

SINCERITÉ Le fait de ne pas mentir. Ce n’est pas toujours une vertu (il
arrive que le mensonge vaille mieux), mais c’en est une que d’y
tendre.

SINGULIER Qui ne vaut que pour un seul élément d’un ensemble donné.
S’oppose à ce titre à {\it universel} (qui vaut pour tous), à {\it général}
(qui vaut pour la plupart) et à {\it particulier} (qui vaut pour quelques-uns).

Dans le langage courant, le mot est souvent un synonyme de rare ou
d’étrange. Cet usage, en philosophie, est à éviter. L’individu le plus banal n’en
est pas moins {\it singulier} pour autant : la singularité est une caractéristique universelle
des individus.

SITUATION On y voit parfois l’une des dix catégories d’Aristote, d’ailleurs
point toujours la même : mieux vaut, pour éviter cette ambi-
guïté, parler de {\it lieu} ou de {\it position} (voir ces mots).

La situation d’un être, au sens courant du terme, c’est la portion d’espace-temps
qu'il occupe (son ici-et-maintenant propre), donc aussi son environnement
et sa place, le cas échéant, dans une hiérarchie. C’est également, s'agissant
d’un être humain, ce qu’il y fait. Par exemple quand on dit de quelqu’un qu’il
a « une belle situation » : cela désigne moins un lieu qu’un métier, qu’une fonction,
qu’un certain {\it rang} dans une hiérarchie sociale ou professionnelle. Toutefois
l’usage philosophique du mot tend de plus en plus à se concentrer sur son
%— 538 —
%{\footnotesize XIX$^\text{e}$} siècle — {\it }
acception sartrienne : être en {\it situation}, c’est être soumis à un certain nombre de
données et de contraintes que l’on n’a pas choisies (être un homme ou une
femme, grand ou petit, d’origine bourgeoise ou prolétarienne, dans tel ou tel
pays, à telle ou telle époque....), mais que l’on reste libre d’assumer ou non. La
situation, écrit Sartre, est un phénomène ambigu : c’est le « produit commun
de la contingence de l’en-soi et de la liberté ». C’est donc notre lot, définitivement.
Il y a toujours un monde, un environnement, des contraintes, des obstacles.
Toujours la possibilité de les affronter ou de les fuir. C’est ce que Sartre
appelle « le paradoxe de la liberté : il n’y a de liberté qu’en situation, et il n’y a
de situation que par la liberté » ({\it L'être et le néant}, IV, I, 2, p. 568-569). Mais
qu’en est-il alors de la liberté elle-même ? Elle n’est pas l'effet du donné, ni
conditionnée par lui (puisque le donné « ne peut produire que du donné »).
Elle n’est pas un être ; comment serait-elle déterminée par ce qui est ? Elle n’est
que néant et pouvoir de néantisation : elle échappe à l’être par la conscience
qu’elle en prend et la fin qu’elle projette. C’est bien commode. Il suffit
d'appeler « situation » tout ce que les sciences humaines considèrent comme
des déterminismes (le corps, l'inconscient, l'éducation, le milieu social.) pour
que le libre arbitre soit sauvé par là. Reste pourtant à savoir ce qui fait que je
choisis ce que je choisis. Si c’est ce que je suis (puisqu’un autre choisirait autrement),
tout choix est donc déterminé par quelque chose que je n’ai pas choisi.
Aussi faut-il, pour sauver la liberté, que mon choix s'explique non par ce que je
suis, mais {\it par ce que je ne suis pas} : c’est où la liberté retombe sur ses pieds ou
sur son néant. Qu'est-ce qu'être en situation ? C’est être confronté, comme
néant, à un être déterminé mais non déterminant (puisqu’une détermination
ne peut porter que sur l'être, point sur le néant), qu’on reste libre pour cela
d'assumer ou non. La situation est donc le corrélat objectif (déterminé, non
déterminant) de ma subjectivité : c’est l’être propre de mon néant.

La notion ne sauve la liberté, comme on le voit, que pour ceux qui
ne-sont-pas-ce-qu'ils-sont-et-sont-ce-qu'ils-ne-sont-pas, autrement dit qui échappent,
ou croient échapper, au principe d'identité. Pour ceux qui sont ce qu’ils sont et
ne sont pas ce qu'ils ne sont pas (non un néant, donc, mais un être), la situation
n’est qu’une façon de parler : c’est un déterminisme qui n’ose pas dire son
nom.

SNOBISME C’est prendre modèle sur une élite, ou supposée telle, faute de
pouvoir lui appartenir. Le snob mime une distinction qu’il n’a
pas, qu’il ne peut avoir. Il veut moins devenir ce qu'il imite (ce ne serait plus
snobisme mais émulation) qu’en prendre l'apparence. Son art est tout de naïveté
et de faux-semblants : c’est un simulateur sincère, comme l’hystérique

%— 539 —
%{\footnotesize XIX$^\text{e}$} siècle — {\it }
qu'il est parfois, et crédule, comme le superstitieux qu'il est souvent : c’est un
idolâtre de la forme, un adorateur des signes. Très proche ici du dandy,
l’humour en moins, le ridicule en plus. C’est un dandy qui se croit, quand le
dandy serait plutôt un snob qui se sait.

Une étymologie douteuse mais suggestive voudrait que le mot {\it snob}, bien
sûr d’origine anglaise, vienne du latin {\it sine nobilitate}. Le snob est sans noblesse
et cherche à le masquer ; il voudrait passer pour le noble qu’il n’est pas, en affichant
les manières qu’il croit être celles de l'aristocratie. C’est le syndrome de
M. Jourdain : un bourgeois qui veut passer pour gentilhomme. Ce n’est qu’un
snobisme parmi d’autres, aujourd’hui plutôt anachronique. Nos snobs ont
d’autres modèles. Mais ils ne sont snobs, aujourd’hui comme du temps de
Molière, que par incapacité à en prendre autre chose que les apparences. Soit
par exemple la culture : étaler celle qu’on a, c’est être cuistre ou pédant ;
simuler celle qu’on n’a pas, c’est être snob. La richesse ? Exhiber la sienne, c’est
être vaniteux, vulgaire, ostentatoire ; vouloir passer pour riche, c’est être snob.
Les relations, les conquêtes ? En afficher de vraies, c’est être mondain ou
goujat ; les exagérer ou en inventer de fausses, c’est être snob.

C’est où le snobisme touche à la mauvaise foi : être snob, c’est vouloir
passer pour ce qu'on n’est pas, en imitant ceux que l’on admire, que l’on envie
ou à qui l’on voudrait ressembler. Le snob se distingue par là de l’hypocrite, qui
n'imite que par intérêt, point par envie, qui ne veut pas ressembler mais
tromper, qui n'admire pas mais qui utilise. Le snob est ordinairement la première dupe —
et parfois la seule — de son personnage. C’est ce qui le rend plus
sympathique, plus ridicule, et moins rare. Pour un Tartufe, combien de bourgeois
gentilshommes et de précieuses ridicules ? L’hypocrisie est l” exception ; le
snobisme, [a règle. Qui peut être certain de lui échapper toujours ? Écrirait-on,
si l’on ne voulait passer pour écrivain ? Lirait-on, si l’on ne voulait que cela se
sache ? Chacun commence par faire semblant, par imiter ce qui lui manque, et
il n’y aurait pas de culture autrement. Le snobisme n’est qu’un premier pas, qui
voudrait donner l'illusion du parcours entier. L'erreur n’est pas d’imiter, mais
de faire semblant, et c’est le snobisme même: se contenter d’un jeu réglé
d’apparences au lieu de s'imposer un travail effectif, qui permettrait seul
d'avancer véritablement. Que M. Jourdain prenne des cours de musique ou de
philosophie, ce n’est pas moi qui le lui reprocherai. Mais on ne peut pas faire
semblant de penser, ni de chanter. Le snob est un mauvais élève ; il joue au professeur
au lieu de faire ses exercices.

SOCIALISME C'est d’abord une conception de la société et de la politique,
qui vise à mettre celle-ci au service de celle-là. En ce sens,
%— 540 —
%{\footnotesize XIX$^\text{e}$} siècle — {\it }
c'est presque un pléonasme, ce pourquoi tout le monde peut s’en réclamer (y
compris, dans l’Allemagne des années 1930, le parti National-socialiste).
Quelle politique voudrait aller contre l'intérêt de la société ? Pourtant le mot,
lorsqu'il apparaît, au début du xIx siècle, est loin de faire l’unanimité. Pierre
Leroux, qui l’inventa peut-être, du moins en français, l’opposait à l’individualisme.
Il y voyait un danger au moins autant qu’une promesse. Le socialisme
veut le bien de la société. Mais à quel prix ? Ce serait une politique collective,
sinon collectiviste, qui servirait le groupe plutôt que les individus. Comment
ne pas craindre qu’elle les écrase ?

En un sens plus restreint et plus rigoureux, on appelle {\it socialisme} tout système
fondé sur la propriété collective des moyens de production et d’échange :
c’est le contraire du capitalisme. Marx y voyait une période de transition,
devant mener au communisme. Certains de ses disciples, au Xx° siècle, voulurent
l’imposer par la révolution et la dictature du prolétariat. D’autres, par des
réformes et la démocratie. Cette opposition entre les révolutionnaires et les
réformistes portait moins sur le but que sur les moyens : il s’agissait, dans les
deux cas, de rompre avec le capitalisme.

L’échec tragique du marxisme-léninisme, dans tous les pays où il parvint au
pouvoir, mais aussi l’incapacité des sociaux-démocrates à triompher, même graduellement,
du capitalisme, rend ce deuxième sens presque obsolète. Les socialistes
d’aujourd’hui ont renoncé à sortir du capitalisme. Ils veulent seulement le
gérer de façon plus sociale, c’est-à-dire dans l’intérêt de l’ensemble de la société
et, spécialement, des plus pauvres. Ils ont fini par accepter l’économie de
marché, sans renoncer toutefois à l’encadrer : ils croient moins au « libre jeu des
initiatives et des intérêts individuels », comme dit Lalande, qu’à l’organisation
de l’État et de la société. Ce socialisme-là a renoncé à toute visée collectiviste.
Ce n’est plus le contraire du capitalisme : c’est son régulateur, et le contraire de
l'ultra-libéralisme.

SOCIÉTÉ  {\it Socius}, en latin, c’est le compagnon, l'associé, l’allié : vivre en
société, c’est vivre en compagnie, mais aussi dans un système
structuré d'associations et d’alliances.

« Humaine ou animale, une société est une organisation, notait Bergson :
elle implique une coordination et généralement aussi une subordination d’éléments
les uns aux autres » ({\it Les deux sources}, I). C’est le contraire de la solitude,
ou plutôt de l'isolement, de la dispersion, de la guerre, comme disait
Hobbes, de chacun contre chacun. C’est pourquoi les humains ont besoin de
société. Parce qu’ils ne peuvent vivre seuls, ni seulement les uns contre les
autres. Parce qu’ils ne peuvent s’isoler, comme disait Marx, qu’au sein de la
%— 541 —
%{\footnotesize XIX$^\text{e}$} siècle — {\it }
société. Mais leurs sociétés sont autrement fragiles que celles des insectes. C’est
que les règles y sont culturelles. C’est que les individus y sont libres de les violer
ou pas. C'est où commence la politique. C’est où commence la morale. Il peut
y avoir des sociétés sans État, sans pouvoir, sans hiérarchie. Mais il n°y a pas de
société sans solidarité, ni d’ailleurs de solidarité sans société.

SOCIOBIOLOGIE La science, ou prétendue telle, qui veut expliquer les
faits sociaux par des faits biologiques. C’est une espèce
de darwinisme social, mais revisité à la lumière des progrès de la génétique et
de l’éthologie.

Qu’une telle explication soit parfois possible ou nécessaire, spécialement
chez les animaux, nul ne le conteste. La sélection naturelle retient les gènes les
plus efficaces, parce que les mieux à même de se transmettre. Comment cela
n'aurait-il pas de répercussion sur les comportements sociaux, qu’ils soient
individuels ou collectifs ? Mais ce n’est alors qu'une partie de la biologie, bien
sûr légitime, qui nous en apprend davantage sur l’espèce que sur la société. Que
l’égoïsme et l’altruisme, par exemple, soient des comportements génétiquement
déterminés, on peut assurément l’admettre. Mais cela ne saurait expliquer leur
fonctionnement différencié, dans telle ou telle société, ni, encore moins, permettre
de choisir entre l’un et l’autre. La sociobiologie ne saurait donc tenir
lieu ni de sociologie ni de morale.

On a vu dans la sociobiologie une caution possible pour certaines thèses de
l'extrême droite. Si la société est soumise à la sélection naturelle, c’est-à-dire à
l'élimination des plus faibles au bénéfice des plus forts ou des plus aptes, on
aurait tort, nous dit-on, de se plaindre qu’elle soit injuste : l'inégalité serait un
avantage sélectif, qu’il faudrait préserver soigneusement. On n'échappe pas à
cet argument en contestant toute détermination biologique de l’humanité, ce
qui serait faire un bien beau cadeau à l'extrême droite, mais en refusant à cette
détermination toute pertinence politique et morale. Que nous soyons des animaux,
ce n'est pas une grande découverte, ou elle n’est pas récente. Cela ne
nous dispense pas de devenir humains, ni de le rester.

SOCIOLOGIE La science de la société, ou {\it des} sociétés. Le mot est inventé
par Auguste Comte. Mais c’est Durkheim qui fonde vraiment
la discipline. Il s’agit d'étudier les faits sociaux comme des choses,
explique-t-il, autrement dit comme extérieurs à notre intelligence et indépendants
de notre volonté. C’est pourquoi l’introspection ne suffit pas, ni l’observation
des comportements individuels. « Le groupe pense, sent, agit tout autrement
%— 542 —
%{\footnotesize XIX$^\text{e}$} siècle — {\it }
 que ne feraient ses membres, s’ils étaient isolés. Si donc on part de ces
derniers, on ne pourra rien comprendre à ce qui se passe dans le groupe » ({\it Les
règles de la méthode sociologique}, V, 2). C’est dire que la sociologie ne saurait se
réduire à la psychologie, ni même en dépendre. Un fait social ne peut et ne doit
s'expliquer que par un autre fait social : la société est « une réalité {\it sui generis} »,
explique Durkheim, avec ses caractères et ses déterminismes propres, qu’on
« ne retrouve pas sous la même forme dans le reste de l’univers ». La sociologie
est la science qui prend en compte cette réalité, ce qui n’est possible qu’à la
condition de la constituer d’abord comme objet. Il ne suffit pas d’observer la
société réelle pour faire de la sociologie. Car cette observation, faisant partie de
la société, risque fort d’en reproduire les préjugés, les illusions, les évidences.
C’est à peu près ce qui distingue la sociologie du journalisme.

SOCIOLOGISME C’est vouloir tout expliquer par la sociologie. Cela ne
semble pas sans quelque pertinence. Le neurologue, le
physicien ou le philosophe vivent dans une société, dira-t-on : le sociologue a
donc vocation à les étudier et à rendre compte de leur travail. Et après tout
pourquoi pas ? Une sociologie des sciences ou de la philosophie est assurément
possible. Mais la sociologie ne saurait pour autant nous apprendre ce que sont
le cerveau ou l'univers, ni quelle philosophie est la meilleure ou la plus vraie.
D'ailleurs l'argument vaudrait aussi bien, ou aussi mal, dans l’autre sens. Le
sociologue a un cerveau et fait partie de l’univers : c’est donc aux neurobiologistes
et aux physiciens, pourrait suggérer tel ou tel, de nous dire ce qu’il en est
de la société et de la sociologie. Tous ces {\it ismes} sont ridicules, et se détruisent
mutuellement.

Au reste, le sociologisme, même à le considérer isolément, s’enferme dans
une aporie comparable à celles du biologisme ou du psychologisme. Si tout est
déterminé socialement, la sociologie l’est aussi: ce n’est qu’un fait social
comme un autre, qui n’a pas plus de valeur que n’importe quelle idéologie.
Que reste-t-il du sociologisme ? Pour que la sociologie puisse prétendre à la
vérité, il faut que la raison, qu’elle met en œuvre, lui échappe — que la vérité ne
soit pas un fait social. À la gloire du rationalisme, qui est la seule doctrine dont
Durkheim ait accepté de se réclamer ({\it Les règles de la méthode sociologique}, Préface
de la 1$^{\text{re}}$ éd.)

SOI Le sujet, considéré dans son objectivité. Cette contradiction le rend
insaisissable, voire impossible. Le bouddhisme enseigne qu’il n’y a pas
de soi, ni en moi (pas d’{\it atman}) ni en tout (pas de {\it Brahman}). Il n’y a que des

%— 543 —
%{\footnotesize XIX$^\text{e}$} siècle — {\it }
agrégats et des processus, qui sont tous conditionnés et impermanents. C’est
dire que le sujet n’est pas une substance, mais une histoire. Pas une essence,
mais un accident. Pas un principe mais un résultat, toujours éphémère. Tel est
aussi l’esprit, me semble-t-il, de nos sciences humaines. Cela met l'amour de soi
à sa place, qui n’est pas la première.

SOLIDARITÉ L'abus du mot, depuis des années, tend à lui faire perdre
toute signification rigoureuse. Ce n’est le plus souvent qu’une
générosité qui n’ose pas dire son nom (par exemple quand on donne de l'argent
à une organisation humanitaire) ou qui ne sait se manifester que sous la
contrainte (par exemple quand on crée un Impôt de Solidarité sur la Fortune).
Mais pourquoi faudrait-il avoir honte — quand on l’est, et on l’est si
rarement — d’être généreux ? Et comment la contrainte pourrait-elle suffire à la
solidarité ?

À parler de solidarité à tout bout de champ, nos politiques et nos belles
âmes la vident de tout contenu. C’est, avec la tolérance, la vertu {\it politiquement
correcte} par excellence. Cela ne la condamne pas, mais rend son usage malaisé.
Ce n’est plus un concept, c’est un slogan. Plus une idée, un idéal. Plus un outil,
une incantation. On voudrait l’abandonner aux meetings ou aux journaux. On
aurait tort. La confusion du langage, même politiquement correcte, est toujours
politiquement dangereuse.

Mieux vaut revenir au sens précis du mot, tel que le suggère l’étymologie.
{\it Solidaire} vient du latin {\it solidus}. Dans un corps solide, les différentes parties sont
solidaires en ceci qu’on ne peut agir sur l’une sans agir aussi sur les autres. Par
exemple une boule de billard : un choc sur un seul de ses points fait rouler la
boule entière. Ou dans un moteur: deux pièces, même séparées l’une de
l’autre, sont solidaires si elles ne peuvent bouger qu’ensemble. La solidarité
n’est pas d’abord un sentiment, encore moins une vertu. C’est une cohésion
interne ou une dépendance réciproque, l’une et l’autre objectives et dépourvues,
au moins en ce premier sens, de toute visée normative. Une boule de
billard ovale serait sans doute moins commode, mais n’en serait pas moins
solide pour autant.

C’est ce qui donne son sens, dans le latin juridique, à lexpression « {\it in
solido} », qui signifie en bloc ou pour le tout. Des débiteurs sont solidaires
quand chacun peut être tenu pour responsable (au cas où les autres s’avéreraient
insolvables) de la totalité de la somme empruntée. C’est bien sûr une
garantie, pour le prêteur, et un risque, pour chacun des emprunteurs. Par
exemple dans un couple marié sous le régime de la communauté des biens :
chacun des deux époux peut se trouver ruiné, sans l’avoir voulu, par les dettes
%— 544 —
%{\footnotesize XIX$^\text{e}$} siècle — {\it }
de l’autre, quand bien même elles auraient été faites à son insu ou contre sa
volonté. Les époux sont donc financièrement solidaires : responsables, ensemble
et pour le tout, de ce qui peut leur arriver, même séparément, ou de ce que peut
faire, même seul, l’un d’entre eux.

Mais le mot va bien au-delà de cette acception purement juridique. Deux
individus sont objectivement solidaires, si ce qu’on fait à l’un agit aussi, inévitablement,
sur l’autre (par exemple parce qu’ils ont les mêmes intérêts), ou si ce
que l’un fait engage également le second. C’est ce qui fonde le syndicalisme :
chacun y défend ses intérêts, mais en défendant aussi ceux des autres. C’est ce
qui fonde le mutualisme, spécialement face au danger, et donc toutes nos compagnies
d’assurance (chacune est fondée, même quand elle est purement capitaliste,
sur la mutualisation des risques). Un mauvais conducteur, dans une
mutuelle, fait perdre de l’argent à tous les sociétaires ; mais même les meilleurs
sont protégés par les cotisations de tous : qu’on vole sa voiture à l’un d’entre
eux, les autres paieront, ou plutôt ont déjà payé, pour la lui rembourser.

C’est où la différence entre la {\it générosité} et la {\it solidarité} apparaît le mieux.
Faire preuve de générosité, c’est agir en faveur de quelqu'un {\it dont on ne partage
pas les intérêts} : vous lui faites du bien sans que cela vous en fasse à vous, voire
à vos dépens ; vous le servez sans que cela vous serve. Par exemple quand vous
donnez dix francs au SDF qui fait la manche (dix francs ? tant que ça?) : il a
dix francs de plus, vous avez dix francs de moins. Ce n’est pas solidarité, c’est
générosité. Vous auriez tort d’en avoir honte, mais tort aussi de vous en
contenter. Car enfin le SDF n’en reste pas moins SDF pour autant. Et qui
serait assez généreux pour l’héberger ou lui payer un loyer ?

Faire preuve de solidarité, à l'inverse, c’est agir en faveur de quelqu'un {\it dont
on partage les intérêts} : en défendant les siens, vous défendez aussi les vôtres ; en
défendant les vôtres, vous défendez les siens. Par exemple quand des salariés
font grève pour réclamer une augmentation de salaire : ils la demandent pour
tous, mais chacun sait bien qu’il se bat aussi pour lui-même. Il en va de même
quand vous adhérez à un syndicat, quand vous souscrivez une police d’assurance
ou quand vous payez votre tiers provisionnel. Vous savez bien que vous
y trouvez votre compte (même si, s'agissant des impôts, il faut qu’un système
de contrôles et de sanctions possibles vous aide à vous persuader que c’est bien
votre intérêt de les payer...). Ce n’est pas générosité ; c’est solidarité. Vous
auriez tort, [à encore, d’en avoir honte, mais tout autant de vous en contenter.
Car enfin vous n'êtes pas encore sorti de l’égoïsme. Combien de salauds syndiqués,
assurés, et qui payent leurs impôts strictement ?

La générosité, dans son principe, est désintéressée. Aucune solidarité ne
l'est. Être généreux, c’est renoncer, au moins en partie, à ses intérêts. Être solidaire,
c’est les défendre avec d’autres. Être généreux, c’est se libérer, au moins
%— 545 —
%{\footnotesize XIX$^\text{e}$} siècle — {\it }
partiellement, de l’égoïsme. Être solidaires, c’est être égoïstes ensemble et intelligemment
(plutôt que bêtement, chacun pour soi ou les uns contre les autres).
La générosité est le contraire de l’égoïsme. La solidarité serait plutôt sa socialisation
efficace. C’est pourquoi la générosité, moralement, vaut mieux. Et c’est
pourquoi la solidarité, socialement, politiquement, économiquement, est beaucoup
plus importante. Qui ne met l'abbé Pierre plus haut que la moyenne des
syndiqués, des assurés, des contribuables ? Et qui ne compte, pour défendre ses
propres intérêts, sur l’État, les syndicats et les assurances, davantage que sur la
sainteté ou la générosité de son prochain ? Cela n’empêche pas l’abbé Pierre,
faut-il le préciser, de s’assurer, de se syndiquer ni de payer des impôts, ne serait-ce
que la TVA, à laquelle nul n'échappe — pas plus que cela ne dispense un
contribuable assuré et syndiqué de faire preuve, parfois, de générosité. Mais
les deux concepts, sans être en rien incompatibles, n’en restent pas moins différents.

S’il fallait compter sur la générosité des uns et des autres pour que tous les
malades puissent se soigner, des millions d’entre eux mourraient sans soins. Au
lieu de quoi on a inventé une chose très simple, au moins moralement, bien
plus modeste que la générosité, et bien plus efficace : la Sécurité sociale, et spécialement
l'assurance maladie. Nous n’en sommes pas moins égoïstes pour
autant. Mais nous en sommes mieux soignés.

Nul ne cotise à la Sécu par générosité. II le fait par intérêt, fut-il contraint
(merci l'URSSAF), mais ne peut les défendre efficacement, dans une société
solidaire, qu’en défendant aussi et par là même ceux des autres.

Nul ne souscrit une assurance par générosité. Nul ne paye ses impôts par
générosité. Et quel étrange syndicaliste ce serait, celui qui ne se syndiquerait
que par générosité ! Pourtant la Sécu, les assurances, les syndicats et la fiscalité
ont fait beaucoup plus, pour la justice et la protection des plus faibles, que le
peu de générosité dont parfois nous sommes capables.

Primauté de la générosité ; primat de la solidarité. La générosité, pour
l'individu, est une vertu morale ; la solidarité, pour le groupe, une nécessité
économique, sociale, politique. La première, subjectivement, vaut mieux. Mais
elle est objectivement à peu près sans portée. La seconde, moralement, ne vaut
guère ; mais elle est, objectivement, beaucoup plus efficace.

C’est où morale et politique divergent. La morale nous dit à peu près :
puisque nous sommes tous égoïstes, essayons de l’être un peu moins. La politique
dirait plutôt : puisque nous sommes tous égoïstes, essayons de l’être
ensemble et intelligemment ; essayons de développer entre nous des convergences
objectives d'intérêts, qui puissent aussi, subjectivement, nous rassembler
(par quoi la solidarité, de nécessité qu’elle est d’abord, peut devenir aussi une
vertu civique ou politique). La morale prône la générosité. La politique impose
%— 546 —
%{\footnotesize XIX$^\text{e}$} siècle — {\it }
et justifie la solidarité. C’est pourquoi on a besoin des deux, bien sûr, mais {\it plus
encore de politique}. Qu'est-ce qui vaut mieux ? Vivre dans une société où tous
les individus sont égoïstes, quoique inégalement, ou vivre dans une société sans
État, sans assurances, sans syndicats, sans Sécurité sociale ? Autant se demander
si la civilisation vaut mieux que l’état de nature, le progrès mieux que la barbarie,
ou la solidarité mieux que la guerre civile.

Je reviens à nos SDF. Beaucoup nous vendent des journaux dans le métro.
Quand vous en achetez un, est-ce générosité ou solidarité ? Cela dépend de vos
motivations, qui peuvent être mêlées. Mais, pour simplifier, on peut dire que
c’est surtout solidarité si vous y trouvez votre intérêt. Pourquoi l'y trouvez-vous ?
Parce que vous vous mettez à la place du SDF ? Ce serait compassion
plutôt que solidarité. Parce que vous vous dites que l’existence de ces journaux
vous permettra, si vous venez à perdre votre emploi, d’en vendre à votre tour ?
C’est douteux, puisque cette possibilité ne dépend presque aucunement du fait
que vous achetiez ou non ce journal aujourd’hui. Le plus vraisemblable, c’est
que vous n’y trouverez votre intérêt que si le journal en question est... intéressant.
Si vous pensez que c’est le cas, par exemple parce que vous avez l’habitude
de le lire et trouvez qu’il est bien fait, vous l’achetez par intérêt : c’est solidarité,
non générosité. Si ce n’est pas le cas, si vous savez d’avance que ce journal vous
tombera des mains, si vous ne l’achetez que pour faire plaisir au SDF ou lui
rendre service, c’est générosité et non plus solidarité. Qu'est-ce qui vaut
mieux ? Moralement, c’est bien sûr la générosité. Mais celle-ci ne règle en rien
la question de l’exclusion ou de la précarité. Le SDF a quelques francs de plus,
vous quelques francs de moins ; il n’en est pas moins exclu, ni la société moins
injuste. Mieux vaudrait que ces journaux deviennent de bons journaux, que des
millions de lecteurs les achètent par intérêt, comme quand nous achetons nos
quotidiens ou nos magazines habituels, autrement dit par égoïsme. Moralement,
ce serait moins méritoire. Mais socialement, beaucoup plus efficace : le
vendeur ne serait plus un exclu, mais un marchand de journaux.

C’est le plus étonnant à penser. Quand j'achète mon journal chez le
libraire, ni lui ni moi n’agissons par générosité — pas plus d’ailleurs que les journalistes
ou les propriétaires du journal. Nous y cherchons tous notre intérêt,
mais nous ne pouvons le trouver que dans la mesure où ces intérêts convergent,
au moins pour une part, et c’est ce qu’ils font en effet (le journal, autrement,
aurait déjà disparu). C’est en quoi la Presse est aussi un marché (voir ce mot),
et cela, loin de la condamner, est ce qui la sauve. Tout marché fonctionne à
l’égoïsme. Mais il ne peut fonctionner efficacement et durablement qu’à la
condition de créer ou de maintenir des convergences objectives (qui peuvent
éventuellement devenir aussi subjectives) d’intérêts. L'égoïsme est le moteur, en
chacun. La solidarité la régulation, pour tous.

%— 547 —
%{\footnotesize XIX$^\text{e}$} siècle — {\it }
Libéralisme ? Pourquoi faudrait-il avoir peur du mot ? Dans une société de
marché, les journaux sont toujours plus intéressants que dans une société collectiviste.
Cela vaut, l'expérience le prouve, pour toute marchandise. Les vêtements
sont toujours de meilleure qualité si marchands et fabricants y trouvent
leur intérêt. Il faudra donc qu’ils le trouvent, et le marché, dans ce domaine,
s’est avéré autrement efficace que la planification et les contrôles (qui débouchent,
presque inévitablement, sur le marché noir). Mais on aurait tort, évidemment,
de croire que le marché puisse suffire à tout : d’abord parce qu’il ne
vaut que pour les marchandises (or la liberté n’en est pas une, ni la justice, ni
la santé, ni la dignité.) ; ensuite parce qu’il ne saurait suffire à sa propre régulation.
Que deviendrait le commerce, sans un droit du commerce ? Et comment
ce droit lui-même serait-il une marchandise ? Comment serait-il à
vendre ? Et comment pourrait-il suffire, s’agissant de ce qui ne s’achète pas ?
On le voit, là encore, dans la Presse. L’abandonner purement et simplement
aux lois du marché, c’est mettre en cause son indépendance (face au pouvoir de
l'argent), sa qualité, sa diversité, son pluralisme. On a donc inventé un certain
nombre de garde-fous, de même que des aides à la Presse, qui ne suppriment
pas les phénomènes de marché (un journal sans lecteurs aura toujours du mal à
survivre, et c’est très bien ainsi), mais qui les modèrent ou en limitent les effets.
L'information est aussi une marchandise. Mais la liberté d’information n’en est
pas une. Un journal s’achète. La liberté des journalistes et des lecteurs, non.

Cela vaut aussi pour la santé, pour la justice, pour l’éducation, et même, au
moins pour une part, pour la nourriture ou le logement. Aucun de ces phénomènes
n'échappe totalement au marché. Aucun, sauf à renoncer à la protection
des plus faibles, ne peut lui être totalement abandonné. Le marché crée de la
solidarité, mais il crée aussi de l’inégalité, de la précarité, de l'exclusion. C’est
pourquoi nous avons besoin d’un État, d’un droit du commerce, d’un droit
social, d’un droit de la Presse, etc., mais aussi de syndicats, d’associations,
d'organismes paritaires de contrôle et de gestion. Le marché est plus efficace
qu’une économie administrée. Mais la loi de tous (la démocratie) vaut mieux
que la loi de la jungle. La Sécu est plus efficace, socialement, que la générosité.
Mais aussi plus juste, politiquement, que de simples assurances privées. C’est
où les ultra-libéraux se trompent. Que le marché crée de la solidarité, cela
n'implique pas qu’il y suffise. C’est où le collectivisme se trompe. Que le
marché ne suffise pas à la solidarité, cela ne veut pas dire que celle-ci puisse se
passer de celui-là. À la gloire de la politique, des syndicats et de la Sécu.

SOLIPSISME C’est ne croire qu’en sa propre existence : considérer qu’on
est soi-même ({\it ipse}) le seul ({\it solus}) existant. Le fait est que
%— 548 —
%{\footnotesize XIX$^\text{e}$} siècle — {\it }
tout le reste, n’étant connu que par les sensations, reste en quelque chose douteux,
alors que le {\it moi} semble pouvoir se recommander de la certitude immédiate
du {\it cogito}. La doctrine, pour irréfutable qu’elle soit, n’en reste pas moins
invraisemblable : comment expliquer l'existence du moi, si rien d’autre
n'existe ? Aussi le solipsisme n’a-t-il pas de partisans. C’est moins une doctrine
qu’un problème, pour les philosophes idéalistes. Si on part du sujet, comment
en sortir ?

SOLITUDE Ce n’est pas la même chose que l’isolement. Être isolé, c’est être
coupé des autres : sans relations, sans amis, sans amours. État
anormal, pour l’homme, et presque toujours douloureux ou mortifère. Alors
que la solitude est notre condition ordinaire : non parce que nous n'avons pas
de relations avec autrui, mais parce que ces relations ne sauraient abolir notre
solitude essentielle, qui tient au fait que nous sommes seuls à être ce que nous
sommes et à vivre ce que nous vivons. « Dans la mesure où nous sommes seuls,
écrit Rilke, l'amour et la mort se rapprochent. » Non qu’il n’y ait pas d'amour,
ou qu’on soit seul à mourir ; mais parce que personne ne peut mourir ou aimer
à notre place. C’est pourquoi «on mourra seul », disait Pascal : non parce
qu’on devrait mourir isolé (du temps de Pascal, ce n’était presque jamais le cas :
il y avait ordinairement un prêtre, la famille, des amis...), mais parce que personne
ne peut mourir à notre place. C’est pourquoi on vit seul, toujours : parce
que personne ne peut vivre à notre place. Ainsi l'isolement est l'exception ; la
solitude, la règle. C’est le prix à payer d’être soi.

SOMMEIL C’est comme une suspension périodique de la vigilance et de
l’activité : le corps se fait presque immobile, l'esprit se ralentit, la
conscience s’oublie ou se regarde rêver. La vie fatigue et tue ; seul le sommeil
nous permet de survivre, au moins un temps. Aussi est-il « le premier besoin de
l’homme, disait Alain, plus pressant même que la faim, et qui suppose société
et veilleurs à tour de rôle, d’où toutes les institutions de police ». Il faut que les
uns montent la garde, pendant que les autres dorment : la société est fille de la
peur, disait encore Alain, davantage que de la faim ; et du sommeil, ajouterais-je,
davantage que de l’ambition.

SOMNOLENCE Un état intermédiaire entre la veille et le sommeil, qui peut
indifféremment préparer celui-ci ou celle-là, mais aussi,
parfois, nous en séparer. Elle peut être délicieuse ou insupportable, selon les
%— 549 —
%{\footnotesize XIX$^\text{e}$} siècle — {\it }
exigences du moment : c’est ce qui distingue l’insomnie de la grasse matinée, et
l'extrême fatigue du repos. « Il arrive que la somnolence soit volontaire, remarquait
Alain ; c’est alors un moyen de se reposer tout en restant ouvert aux
signes. » Mais aussi qu’elle s’impose à nous, et nous enferme.

SONGES Un autre mot, plus littéraire, pour dire les rêves. L'âme y est en
liberté, remarque Voltaire, et elle est folle. C’est qu’elle a perdu les
rails du réel. Heureusement que le réveil nous rend la raison en nous rendant
au monde.

{\it SOPHIA} La sagesse théorique ou contemplative. Se distingue par là de la {\it phronèsis}
(la prudence, la sagesse pratique). C’est une différence que le
français ne fait pas, et il a raison : la vraie sagesse serait la conjonction des deux.

SOPHISME C'était à Montpellier, il y a une quinzaine d’années, dans la
cour, transformée en amphithéâtre, d’un bel hôtel particulier
du {\footnotesize XVIII$^\text{e}$} siècle. Je participais cet été-là, dans le cadre d’un festival organisé par
France-Culture, à un débat sur la religion, retransmis en direct par la radio.
Mon propos où mon athéisme agacent l’un des participants : « Je me demande
de quel droit vous pouvez parler de religion, me lance-t-il, puisque vous ne
croyez pas en Dieu ! »

L’argument, qui me paraît sophistique, m'’agace à mon tour. Je lui
réponds : « Autant dire qu’il faut, pour pouvoir parler légitimement de la
musique de Beethoven, être sourd, sous prétexte que Beethoven, lui, l’était ! »
Rires sur les gradins : j’avais marqué un point. Pourtant je voyais bien que ma
riposte était sophistique, au moins autant que l'attaque de l’autre, ou plutôt
davantage (ce n’était dans sa bouche, selon toute vraisemblance, qu’un paralogisme).
Mon excuse, outre l’éventuelle drôlerie de la chose, était de n’en être
pas dupe, ni d’essayer vraiment de duper : c'était polémique plutôt qu’argumentation,
défense plutôt qu’attaque, ironie plutôt que pensée. Enfin il faisait
très beau et nous étions en vacances : un sophisme, entre intellectuels, peut
faire parfois, dans certaines circonstances choisies, un divertissement acceptable.
Toutefois il convient de ne pas en abuser, sans quoi ce n’est plus débat
mais combat, plus pensée mais cirque. Je suis revenu bien vite à des choses plus
sérieuses. Si seuls les croyants pouvaient parler de religion, comment un athée
pourrait-il justifier son athéisme, et pourquoi organiser, entre croyants et
incroyants, un débat ?

%— 550 —
%{\footnotesize XIX$^\text{e}$} siècle — {\it }
Qu'est-ce qu’un sophisme ? C’est une faute volontaire, dans un raisonnement
(par différence avec le paralogisme, faute involontaire), qui vise à
tromper ou à embarrasser. Il est sans valeur quant au fond, sinon parfois par
les difficultés qu’il fait paraître. C’est une arme plus qu’une pensée, et à
double tranchant : c’est jouer avec la vérité, ou plutôt avec son apparence, au
lieu de la servir. Mes excuses, donc, cher collègue, pour ce sophisme d'il y a
quinze ans.

SOPHISTE Celui qui fait profession de sagesse ({\it sophia}) ou de sophismes.
Double faute : la sagesse n’est pas un métier ; un sophisme n’est
ni une preuve ni une excuse. Le mot, en ce sens, est toujours péjoratif : le
sophiste, c’est celui qui cherche moins la vérité que le pouvoir, le succès ou
l'argent. C’est contre eux que Socrate inventa, ou réinventa, la philosophie.

En un autre sens, plus large et plus neutre, on peut appeler {\it sophiste} toute
personne qui se réclame de la sophistique (voir ce mot) ou en relève, à commencer
par ses fondateurs (Protagoras, Gorgias, Prodicos, Antiphon.….).
C’est alors une catégorie historique, non polémique. Elle ne saurait pourtant
nous dispenser, philosophiquement, de nous situer par rapport à eux. Qu'ils
fassent partie de l’histoire de la philosophie, c’est entendu. Mais cette histoire
n’a jamais dispensé personne de philosopher. La mode, depuis plusieurs
décennies, est à la réhabilitation des sophistes. Réhabilitation sans doute
légitime. Mais faut-il pour autant, comme faisait Nietzsche, donner tort à
Socrate ?

SOPHISTIQUE Toute pensée qui se soumet à autre chose qu’à la vérité, ou
qui soumet la vérité à autre chose qu’à elle-même. C'est
considérer que la vérité n’est qu’une valeur comme une autre, réductible
comme telle au point de vue, à l'évaluation ou aux désirs qui la gouvernent. Le
mot, en ce sens, relève du langage technique : il est sans visée péjorative. Que
Protagoras et Nietzsche soient des sophistes, cela n’empêche pas qu'ils soient
aussi des génies. Toutefois cela m’empêche de les suivre absolument. « Qu'un
jugement soit faux, écrit Nietzsche, ce n’est pas, à notre avis, une objection
contre ce jugement. [..] Le tout est de savoir dans quelle mesure ce jugement
est propre à promouvoir la vie, à l’entretenir, à conserver l'espèce, voire à
l'améliorer » ({\it Par-delà le bien et le mal}, I, 4). Sophistique vitaliste. Qu’un jugement
soit faux, c’est au contraire pour moi une très forte objection contre ce
jugement ; et qu’il soit favorable à l’espèce n’y change rien.

%— 551 —
%{\footnotesize XIX$^\text{e}$} siècle — {\it }
SORCELLERIE Une magie mauvaise ou rituelle. On remarquera que le sorcier
est du côté du rite ; la sorcière, du côté du mal. Même
la superstition est machiste.

SOTTISE  « Le mal le plus contraire à la sagesse, disait Alain, c’est exactement
la sottise, j'entends l’erreur par précipitation ou prévention » ({\it Éléments
de philosophie}, VI, 8). La sottise se distingue par là de la bêtise, qui serait
plutôt le contraire de l'intelligence et se trompe par lenteur ou incapacité.

SOUCI C’est la mémoire de l’avenir, mais inquiète plutôt que confiante.
Disposition essentielle à l’homme, en tant qu’il est esprit (donc
mémoire) et fragilité (donc inquiétude). Heidegger avait raison de faire du
souci une structure originaire du {\it Dasein}, toujours projeté en avant de soi, toujours
préoccupé, toujours tourné vers l’avenir ou la mort. Mais les Grecs
n'avaient pas tort d’y voir le contraire de la sagesse. Il est essentiel aux humains,
et à eux seuls, de n’être pas des sages. C’est pourquoi ils ont à philosopher : pour
se rapprocher de la sagesse, au moins un peu, en s’éloignant d’eux-mêmes. En
devenant moins humains ? Non pas. Mais en devenant moins égoïstes, moins
inquiets, moins soucieux — en s’ouvrant davantage au présent, à l’action et à tout.

SOUHAIT La formulation d’une espérance. Aussi ne souhaite-t-on que ce
qui ne dépend pas de nous. C’est comme une prière, mais sans
Dieu. Superstition, ou politesse. On peut se passer de la première, non de la
seconde.

SOUVERAIN Le plus grand : celui qui l'emporte, ou doit l'emporter, sur
tous les autres. Par exemple le souverain bien : ce serait le bien
le plus grand, ou le bien ultime (vis-à-vis duquel les autres biens ne seraient que
des moyens). Ainsi le bonheur, selon Aristote, le plaisir, selon Épicure, ou la
vertu, selon les stoïciens. Le vrai souverain bien, s’il était possible, serait plutôt
la conjonction des trois.

Quand il est utilisé seul et comme substantif, le mot fait presque toujours
référence à la politique : il désigne le plus grand de tous les pouvoirs, dans un
territoire donné, le pouvoir premier (d’où les autres procèdent) ou ultime (celui
qui a les moyens, au moins en droit, de s’imposer à tous les autres). Concrètement,
le souverain, c’est celui qui {\it fait la loi}, comme on dit, ou qui désigne ceux
%— 552 —
%{\footnotesize XIX$^\text{e}$} siècle — {\it }
qui la font. Tel est le sens du mot chez Hobbes : le souverain est le dépositaire
de l’autorité publique, à laquelle tous, par le pacte social, ont convenu de se
soumettre ; c’est en lui que réside « l'essence de la République » ({\it Léviathan},
XVII et XVIII). Tel est le sens du mot chez Rousseau : le souverain est la République
elle-même, en tant qu’elle est active ({\it Contrat social}, I, 7), et cette souveraineté
« consiste essentiellement dans la volonté générale », qui s'exprime par
la loi ({\it ibid.}, III, 15).

Le souverain peut prendre des formes différentes : ce peut être Dieu ou le
clergé, dans une théocratie, un roi, dans une monarchie absolue, un groupe,
dans une aristocratie, ou bien, et c’est évidemment préférable, le peuple
entier, dans une démocratie — quand bien même il n’exercerait cette souveraineté,
comme presque toujours et malgré Rousseau, que par la médiation
de ses représentants. C’est bien clairement l'esprit de nos institutions, tel
qu’il est énoncé par l’article 3 de la Constitution de 1958 : « La souveraineté
nationale appartient au peuple, qui l’exerce par ses représentants et par la voie
du référendum. » Mais Hobbes, partisan de la monarchie absolue, a bien
montré qu'il ny a de monarchie ou d’aristocratie (et même, ajouterais-je, de
théocratie) qu’à la condition que le peuple d’abord y consente ; si bien que
« c’est le peuple qui règne, précise-t-il, en quelque sorte d’État que ce soit »
({\it De Cive}, XII, 8 ; voir aussi VII, 11). C’est ce qui donne raison aux démocrates
(«la démocratie, dira Marx, est l'essence de toute constitution
politique » : {\it Critique de la philosophie politique de Hegel}, I, a), mais qui ne
suffit pas à assurer leur triomphe. Combien de peuples ont préféré se soumettre
ou se démettre ?

La souveraineté, en droit, ne peut être qu’absolue. Elle cesserait autrement
d’être souveraine. Non, certes, que le peuple, dans une démocratie, ait tous les
droits. Mais parce qu’il est seul à même de les délimiter (par la constitution, par
la loi) et reste maître de modifier cette délimitation (toute constitution démocratique
prévoit les modalités démocratiques de changement de la
constitution ; sans quoi ce ne serait plus le peuple qui serait souverain mais la
constitution : on ne serait plus en démocratie mais en {\it nomocratie}). C'est pourquoi
la démocratie n’est jamais une garantie, fût-ce contre le pire ; l’histoire,
hélas, le montre assez.

Mais toute souveraineté, en fait, reste relative : c’est où l’on sort du droit
pour rentrer dans la politique, d’où le droit sort. C'est ce qu'ont compris
Machiavel et Spinoza. La souveraineté n’est qu’une abstraction, certes nécessaire,
mais qui n’en reste pas moins abstraite pour autant. La vérité, c'est qu’il
n’y a que des pouvoirs, toujours finis, toujours multiples, qui se contrecarrent
mutuellement — que des forces et des rapports de forces.

%— 553 —
%{\footnotesize XIX$^\text{e}$} siècle — {\it }
De là l’idée, chez Montesquieu et les libéraux, de la séparation des pouvoirs.
Idée légitime. Mais qui ne saurait annuler ni l’unicité de la souveraineté
(la République une et indivisible) ni la multiplicité mouvante des rapports de
forces. Entre les deux, le suffrage universel, qui est la souveraineté en acte. C’est
la mesure, en même temps que l’effectuation toujours recommencée, d’un rapport
de forces. Aucune souveraineté ne saurait dispenser les démocrates de
gagner les élections. À la gloire des partis et des militants.

SPÉCISME Ce serait l'équivalent du racisme, mais appliqué aux rapports
entre les espèces. Serait {\it spéciste} toute personne considérant que
les animaux, hommes compris, ne sont pas tous égaux en droits et en dignité.
La notion, qui part de bons sentiments (respecter les droits des animaux), est
pourtant dangereuse : c’est trop gommer la différence entre les humains et les
bêtes. Différence de degré, à ce que je crois, plutôt que de nature, mais qui justifie
qu’on les traite en effet différemment. « Si l’on n’avait pas mis les animaux
dans des wagons à bestiaux, me dit un jour une collègue, Hitler n’y aurait pas
mis les Juifs. » Sans doute. Mais cela ne fait pas du commerce de la viande
l'équivalent du nazisme, ni du nazisme un commerce.

SPIRITISME C'est la croyance aux esprits, qui seraient les Âmes des morts,
et la prétention de les faire parler. Spiritualisme superstitieux,
ou superstition spiritualiste. L’étonnant n’est pas tant qu’une table puisse
tourner sans raison apparente (il y a bien d’autres mystères autrement étonnants),
mais que de bons esprits puissent prendre au sérieux ce qu’elle aurait à
nous dire, qui est en général d’une platitude désolante. On remarquera avec
intérêt que la table, quand c’est Victor Hugo qui la faisait tourner, parlait en
alexandrins. Mais aussi qu’elle n’a rien produit qui puisse se comparer sérieusement
aux {\it Contemplations} où à {\it La légende des siècles}.

SPIRITUALISME Toute doctrine qui affirme l’existence de substances pensantes
immatérielles, autrement dit d’esprits irréductibles
à quelque corps que ce soit. Le spiritualisme est le plus souvent dualiste. II
admet deux types de substances, l'esprit et la matière, ou l'âme et le corps, qui
seraient en l’homme à la fois ontologiquement distinctes et intimement unies :
ainsi chez Descartes ou Maine de Biran. Mais il peut arriver qu’il soit moniste,
s’il ne reconnaît l’existence que de substances spirituelles : c’est le cas, spécialement,
chez Leibniz ou Berkeley.

%— 554 —
%{\footnotesize XIX$^\text{e}$} siècle — {\it }
SPIRITUALISTES Ce sont ces gens qui prennent leur esprit ou leur âme au
sérieux : ils croient porter l’absolu au chaud de leur
conscience ! Cela ne les dispense pas d’avoir un corps. Mais justement : ils l’{\it ont},
ce qui suppose qu'ils soient autres que lui. Il est vrai qu’ils diraient aussi bien :
« J'ai une âme ». Mais quel est alors ce je qui posséderait et leur âme et leur
corps ? La somme des deux ? Mais alors il ne serait plus {\it un}. Autre chose ? Mais
alors ils ne seraient plus spiritualistes. Ce qu’ils pensent, me semble-t-il, c’est
plutôt que leur âme a un corps, que leur corps a une âme, et qu’ils sont, eux,
cette Âme ou cet esprit qu’ils connaissent de l’intérieur, dont leur corps serait
l’objet le plus proche, le plus immédiat, le plus indissociable, mais qui n’en
serait pas moins {\it objet} pour autant. Philosophes du sens intime, et du sens tout
court.

SPIRITUALITÉ La vie de l'esprit. On se trompe si on la confond avec la religion,
qui n’est qu’une des façons de la vivre. Ou avec le spi-
ritualisme, qui n’est qu’une des façons de la penser. Pourquoi les croyants
auraient-ils seuls un esprit ? Pourquoi sauraient-ils seuls s’en servir ? La spiritualité
est une dimension de la condition humaine, non le bien exclusif des
Églises ou d’une école.

Une spiritualité laïque ? Cela vaut mieux qu’une spiritualité cléricale, ou
qu'une laïcité sans esprit.

Une spiritualité sans Dieu ? Pourquoi non ? C’est ce qu’on appelle traditionnellement
la sagesse, du moins l’une de ses formes. Pourquoi faudrait-il
croire en Dieu pour que l'esprit en nous soit vivant ?

Le mot vient du latin {\it spiritus}, que les Grecs pouvaient traduire par {\it psukhê}
(l’étymologie, dans les deux langues, fait référence au souffle vital, à la respiration)
aussi bien que par {\it pneuma}. C’est dire que la frontière, entre le spirituel et
le psychique, est poreuse. L'amour, par exemple, peut appartenir aux deux. La
religion peut appartenir aux deux. La foi est un objet psychique comme un
autre. Mais c’est aussi une expérience spirituelle. Disons que tout ce qui est spirituel
est psychique, mais que tout ce qui est psychique n’est pas spirituel. Le
psychisme est l’ensemble, dont la spiritualité serait le sommet ou la pointe. En
pratique, en effet, on parle de spiritualité pour la partie de la vie psychique qui
semble la plus élevée : celle qui nous confronte à Dieu ou à l’absolu, à l'infini
ou au tout, au sens ou au non-sens de la vie, au temps ou à l'éternité, à la prière
ou au silence, au mystère ou au mysticisme, au salut ou à la contemplation.
C’est pourquoi les croyants y sont tellement à l’aise. C’est pourquoi les athées
en ont tellement besoin.

%555
%{\footnotesize XIX$^\text{e}$} siècle — {\it }
La spiritualité, pour les croyants, a un objet clairement défini (quoiqu'il
soit inconnaissable), qui serait un sujet, qui serait Dieu. La spiritualité est alors
une rencontre, un dialogue, une histoire d’amour ou de famille. « Mon Père »,
disent-ils. Est-ce spiritualité ou psychologie ? Mystique ou affectivité ? Religion,
ou infantilisme ?

L’athée est plus démuni ou moins puéril. Ce n’est pas un Père qu’il
cherche, ni qu’il trouve. Ce n’est pas un dialogue qu’il instaure. Pas un amour
qu'il rencontre. Pas une famille qu’il habite. Mais l’univers. Mais Pinfini. Mais
le silence. Mais la présence de tout à tout. Non une transcendance, donc, mais
limmanence. Non un Dieu mais l’universel devenir, qui le contient et l’emporte.
Non un sujet, mais l’universelle présence. Non un Verbe ou un sens,
mais l’universelle vérité. Qu'il n’en connaisse qu’une infime partie, cela
n'empêche pas qu’elle le contienne tout entier.

Une spiritualité sans Dieu ? Ce serait une spiritualité de l’immanence
plutôt que de la transcendance, de la méditation plutôt que de la prière, de
l'unité plutôt que de la rencontre, de la fidélité plutôt que de la foi, de l’enstase
plutôt que de l’extase, de la contemplation plutôt que de l'interprétation, de
l'amour plutôt que de l’espérance, et qui n’en déboucherait pas moins sur une
mystique, telle que je l’ai définie, c’est-à-dire sur une expérience de l'éternité,
de la plénitude, de la simplicité, de l’unité, du silence. Je ne l’ai vécu, s’agissant
de ces derniers états, que très exceptionnellement. Mais assez, toutefois,
pour que ma vie en soit définitivement transformée. Et pour que le mot de {\it spiritualité}
cesse de me faire peur.

SPONTANÉITÉ Ce qui vient de soi-même ({\it sponte sua}, de sa propre initiative),
non d’une force extérieure ou d’une contrainte. Un
synonyme de volontaire ? Pas tout à fait, puisqu’une conduite instinctive ou
passionnelle peuvent être spontanées, sans que la volonté y soit pour rien et
parfois contre elle. Et puisqu’un acte volontaire peut n'être pas spontané
(quand je cède à la pression ou à la contrainte, j’agis bien volontairement, non
spontanément : par exemple quand je donne mon portefeuille à l'individu
armé qui me menace). La spontanéité est plutôt du côté de l’action ou de la
réaction immédiates, du désir ou de limpulsion irréfléchis, sans autre
contrainte que soi, sans autre source que soi, et qu’il y ait ou non contrôle conscient
et délibéré. La volonté peut en naître, point s’en contenter.

STOÏCIEN Qui se réclame du stoïcisme. Cela n’a jamais suffi pour être stoïque.

%— 556 —
%{\footnotesize XIX$^\text{e}$} siècle — {\it }
STOÏCISME École philosophique de l'Antiquité, fondée par Zénon de Citium,
renouvelée par Chrysippe, prolongée par Sénèque, Épictète et
Marc Aurèle. Son nom vient du lieu (un portique : {\it stoa}) où Zénon enseignait.
C’est dire que le fondateur n’a pas donné son nom à l’école : les stoïciens se
voulaient avant tout disciples de Socrate et des cyniques, dont ils systématisent
l’enseignement. Platon voyait en Diogène un « Socrate devenu fou » ; Zénon
serait plutôt un Diogène devenu raisonnable.

Le stoïcisme est un matérialisme volontaire et volontariste : il ne reconnaît
l'existence que des corps et n’attache de valeur qu'aux volontés. Tout ce qui ne
dépend pas de nous est moralement indifférent ; seul ce qui en dépend peut
être bien ou mal. Seule la vertu vaut donc absolument, et c’est elle, non le
plaisir, qui fait le bonheur. Le moralisme des stoïciens est ainsi à opposé de
l’hédonisme épicurien, comme leur physique continuiste est à l’opposé de l’atomisme.
Ce sont pourtant deux rationalismes. Mais la raison stoïcienne ne se
contente pas d’expliquer, comme fait l’épicurienne : elle juge, elle commande,
elle gouverne le sage comme le monde. C’est qu’elle est Dieu, ou ce qu’il y a de
divin en tout. De là cette piété stoïcienne, qui est un fatalisme mais libérateur,
et un panthéisme, mais à visée humaniste. Tout est rationnel ; à nous de
devenir raisonnables. Tout est juste ; à nous d’agir avec justice. C’est aussi un
cosmopolitisme. « La raison, qui fait de nous des êtres raisonnables, nous est
commune, écrit Marc Aurèle : c’est elle qui ordonne ce qui est à faire ou non ;
par conséquent, la loi aussi est commune ; s’il en est ainsi, nous sommes
concitoyens : nous vivons ensemble sous un même gouvernement, le monde
est comme une cité ; car à quel autre gouvernement commun pourrait-on dire
que tout le genre humain est soumis ? » ({\it Pensées pour moi-même}, IV, 4). C’est
enfin un actualisme : « Seul le présent existe », disait Chrysippe, et il suffit seul
au salut. Il n’y a donc rien à espérer : il s’agit de vouloir, pour tout ce qui
dépend de nous, et de supporter, pour tout ce qui n’en dépend pas. École de
courage, de lucidité, de sérénité. Aussi appelle-t-on {\it stoïcisme}, en un sens élargi,
tout ce qui semble relever d’une telle attitude. C’est qu’on peut être stoïque
sans être stoïcien. Pas besoin, pour faire ce qu’on doit ou supporter ce qui
advient, de croire en une quelconque providence, ni même en quelque système
que ce soit. Marc Aurèle l’a reconnu : « Si Dieu existe, tout est bien ; si les
choses vont au hasard, ne te laisse pas aller, toi aussi, au hasard » (IX, 28).

STOÏQUE Qui relève du stoïcisme, ou qui en serait digne. Le mot désigne
moins une pensée qu’une attitude. Se dit ordinairement d’un très
grand courage, spécialement contre la douleur. Nul besoin pour cela d’être
%— 557 —
%{\footnotesize XIX$^\text{e}$} siècle — {\it }
stoïcien. Épicure, face à la maladie, se montra stoïque ; mais n’en restait pas
moins épicurien pour autant.

STRUCTURALISME Un courant de pensée, importé de la linguistique et
des sciences humaines, dont certains voulurent faire
une philosophie. Il s’agit, dans les objets qu’on étudie, de privilégier la structure
ou le système plutôt que les éléments ou leur somme, et spécialement de
penser les phénomènes humains, et l’homme lui-même, comme effets de structures
plutôt que comme création ou subjectivité. S’oppose à ce titre à l’existentialisme,
auquel il succéda, dans les années 1960, comme mode parisienne. Les
grands noms, en France, étaient Lévi-Strauss, Foucault, Lacan, Althusser…
Beaucoup de talent et de travail. L'ensemble, une fois débarrassé des oripeaux
de la mode, reste stimulant par l’ambition intellectuelle et la radicalité anti-humaniste
(au sens de l’anti-humanisme théorique). On cite souvent, pour
illustrer, le thème de {\it la mort de l'homme}, tel qu’il apparaissait, en 1966, à la
dernière page du plus fameux des livres de Foucault : « L'homme est une invention
dont l’archéologie de notre pensée montre aisément la date récente. Et
peut-être la fin prochaine » ({\it Les mots et les choses}, X, 6). Mais le texte, malgré sa
beauté, ou à cause d’elle, reste quelque peu équivoque. La maxime du structuralisme,
à mon sens, serait plutôt une formule, que je crois vraie, de Claude
Lévi-Strauss : « Le but dernier des sciences humaines n’est pas de constituer
l'homme, mais de le dissoudre » ({\it La pensée sauvage}, IX). Faut-il alors renoncer
à l’humanisme ? Pas nécessairement. Mais l’humanisme qui reste disponible est
un humanisme pratique, non théorique : il porte non sur ce que nous {\it savons} de
l’homme (qu’il fait partie de la nature : anti-humanisme théorique), mais sur ce
que nous {\it voulons} pour lui (qu’il reste humain, au sens normatif du terme). Les
sciences humaines ne sont pas humanistes. C’est donc à nous de l’être.

Le structuralisme semble d’abord s'éloigner du matérialisme : la position
d’un élément, dans une structure donnée, importe davantage que la matière
dont il est fait. Mais c’est pour déboucher sur ce que Gilles Deleuze appellera
un « nouveau matérialisme ». Ce n’est pas assez, soulignait Lévi-Strauss, que
« d’avoir résorbé des humanités particulières dans une humanité générale ; cette
première entreprise en amorce d’autres, qui incombent aux sciences exactes et
naturelles : réintégrer la culture dans la nature, et finalement la vie dans
l’ensemble de ses conditions physico-chimiques » ({\it op. cit.}, IX). L'homme n’est
pas un empire dans un empire : c’est toujours Spinoza qui renaît. S’il n’est de
sens que « de position », comme l’a montré Lévi-Strauss, si « le sens résulte toujours
de la combinaison d’éléments qui ne sont pas eux-mêmes signifiants »,
comme il dit encore, il n’y a plus de {\it sujet} du sens — ni Dieu ni homme -, ni de
%— 558 —
%{\footnotesize XIX$^\text{e}$} siècle — {\it }
{\it sens du sens} : tout sens est toujours réductible à autre chose, qui n’en a pas
(Lévi-Strauss, « Dialogue avec Ricœur », {\it Esprit}, 1963, p.637). Comme le
remarque Deleuze, qui cite ce texte, « le sens, pour le structuralisme, est toujours
un résultat, un effet: non seulement un effet comme produit, mais un effet
d'optique, un effet de langage, un effet de position » (« À quoi reconnaît-on le
structuralisme ? », in F. Châtelet, {\it Histoire de la philosophie}, t. 8). Le structuralisme
est « un kantisme sans sujet transcendantal », disait Ricœur, et Lévi-Strauss
accepte la formule. C’est qu’il n’y a plus de sujet du tout, sinon comme effet de
structures ou d'illusions : Dieu est mort, et l’homme aussi peut-être (M. Foucault,
{\it op. cit.}). Cela ne dispense pas d’être humain, ni ne suffit à le devenir.

STRUCTURE Du latin {\it structura}, l’arrangement, la disposition, l’assemblage.
Le mot désigne un ensemble complexe et ordonné, mais dont
l’organisation importe davantage que le contenu : c’est un système de relations
plutôt qu’une somme d’éléments, et les éléments eux-mêmes y sont moins
définis par ce qu’ils sont que par leur place dans l’ensemble et la fonction que
cette place leur assigne. Aussi y a-t-il davantage, dans un tout structuré, que la
somme de ses parties. Soit par exemple une maison. Si on ne la considère que
du point de vue de ses matériaux, elle n’est rien de plus que leur somme :
quelques milliers de briques, quelques sacs de ciment, quelques centaines de
tuiles, quelques poutres et chevrons, quelques clous, quelques tuyaux, quelques
vitres, un peu de plâtre et de peinture... Mais tout cela, entassé sur le chantier,
ne fait pas encore une maison. C’est qu’il y manque la structure, telle qu’elle est
dessinée, par exemple, sur le plan de l’architecte. On remarquera que la structure,
sans ses éléments, ne fait pas non plus une maison : on n’habite pas
davantage un plan qu’un tas de briques. Il faut donc les deux. Parler de la structure
de la maison, c’est insister sur les rapports entre les éléments, sur leur place
et leur fonction respectives, davantage que sur ces éléments eux-mêmes, lesquels
n’ont de sens ou d'utilité qu’en fonction de leur position. C’est reconnaître
qu’une maison n’est pas réductible aux matériaux qui la composent ; et
que le fait qu’elle soit construite en briques ou en pierres, sans être forcément
indifférent, importe moins que la disposition des unes ou des autres, dont
dépend aussi leur fonction. C’est pourquoi la notion de structure est spécialement
importante en linguistique : parce que les unités phoniques qu’elle rencontre
sont arbitraires, qui ne deviennent signifiantes que par leurs relations
avec d’autres, autrement dit que par leur place et leur fonction dans un
ensemble structuré (une langue). Mais c’est pourquoi aussi elle joue un rôle
majeur dans la plupart des sciences humaines : parce que rien de ce qui est proprement
humain (le langage, la culture, la politique, Part, la religion...) n’est
%— 559 —
%{\footnotesize XIX$^\text{e}$} siècle — {\it }
compréhensible indépendamment des systèmes de relations qui le rendent possible
et le constituent.

STYLE Ce n’est pas l’homme, malgré Buffon, puisqu’un homme remarquable
peut en manquer : j'en connais quelques-uns qui écrivent
platement, et plusieurs stylistes talentueux qui m'ont semblé, à les lire ou à
les rencontrer, humainement bien médiocres. Le style est une certaine
manière d’agir, d'écrire ou de créer, qui manifeste certes une subjectivité,
mais pour autant seulement qu’elle dispose d’un certain talent, bien spécifique,
et d’un certain métier. Ce n’est pas l’homme, mais la capacité qu’il a,
et ils ne l’ont pas tous, d’inventer une expression qui lui ressemble ou le distingue
des autres. C’est ce qu’il y a de singulier dans le talent, ou de talentueux
dans la singularité. Le style est donc une force, mais aussi une limite.
Les plus grands artistes n’en ont pas, ou en ont plusieurs, ou ne cessent de
s’en libérer. Par quoi « {\it styliste} » peut devenir péjoratif. C’est attacher trop
d'importance à la forme et à la singularité. S’il avait quelque chose à dire,
attacherait-il — ou attacherions-nous — autant d'importance à la façon dont il
le dit ? Cet artiste à ce point prisonnier de sa manière ou de sa singularité,
comment serait-il universel ? Comparez par exemple Cioran, qui a un style, et
Montaigne, qui a du génie.

Que le style vaille pourtant mieux que la platitude ou la banalité, nul ne le
contestera. Mais comment le style pourrait-il suffire? Comment serait-il
l'essentiel ? Je vois bien que le Greco a un style, que Renoir a un style, que Picasso
en a plusieurs. Je ne suis pas sûr que Velézquez, qui les dépasse, en ait un.

SUBLIMATION Un changement d’état (du plus lourd au plus léger), ou
d'orientation (du plus bas vers le plus haut). Le mot, qui
désigne d’abord une élévation morale, est très vite utilisé par les alchimistes,
puis par les chimistes, pour désigner le passage d’un corps de l’état solide à
l’état gazeux. Dans la philosophie contemporaine, en revanche, c’est l’acception
freudienne qui domine. La sublimation est le processus par lequel la pulsion
sexuelle change d’objet et de niveau, trouvant ainsi à s'exprimer,
quoique indirectement, de façon socialement valorisée et en dehors de toute
satisfaction proprement érotique. C’est le cas spécialement dans l’art, dans la
pensée, dans la spiritualité, sans doute aussi dans tout amour, dès lors qu’il
ne se réduit pas à l’attirance sexuelle. Dans la sublimation, écrit Freud, « les
émotions sont détournées de leur but sexuel et orientées vers des buts socialement
supérieurs, qui n’ont plus rien de sexuel » ({\it Introduction à la psychanalyse},
%— 560 —
%{\footnotesize XIX$^\text{e}$} siècle — {\it }
1). C’est mettre les énergies du Ça au service d’autre chose, qui vaut
mieux. Au service de quoi ? De la civilisation : « C’est à l'enrichissement psychique
résultant de ce processus de sublimation, écrit Freud, que sont dues
les plus nobles acquisitions de l’esprit humain » ({\it Cinq leçons...}, V). Les désirs
infantiles peuvent ainsi « manifester toute leur énergie et substituer au penchant
irréalisable de l'individu un but supérieur, [...] un objectif plus élevé
et de plus grande valeur sociale » ({\it ibid.}), tout en procurant à l'individu des
satisfactions « plus délicates et plus élevées » ({\it Malaise dans la civilisation}, II).
Cela vaut mieux que la névrose (qui reste prisonnière des désirs infantiles
qu’elle refoule). Cela vaut mieux que la perversion (qui les satisfait). Cela
vaut mieux qu’une sexualité simplement animale (qui les ignore). C’est où
l'humanité s’invente, peut-être, en inventant des dieux. Ce n’est pas le sentiment
du sublime ; c’est le devenir sublime du sentiment.

SUBLIME Ce qu'il y a de plus haut ({\it sublimis}), de plus impressionnant, de
plus admirable. Se dit surtout d’un point de vue esthétique : c’est
une beauté qui emporte ou écrase, comme si un peu d’effroi se mêlait au plaisir.
C’est qu’on se sent trop petit, face à tant de grandeur. C’est qu'on ne comprend
pas qu’une telle chose soit possible, ou comment elle Pest. C’est que
l'admiration bouscule le jeu ordinaire de nos facultés ou de nos catégories.
C’est que tant de hauteur nous élève, au moins en partie, jusqu’à nous faire
sentir douloureusement ce qui en nous reste bas ou médiocre.

« Nous nommons {\it sublime} ce qui est absolument grand », écrit Kant, « ce en
comparaison de quoi tout le reste est petit ». Aussi voulait-il que le sentiment
du sublime, même face à la nature, n’exprime que la grandeur de l'esprit
(CE, \S 23 à 29). Je dirais plutôt que le sublime est le sentiment, dans l'esprit
humain, de ce qui le dépasse, nature ou génie, et l'emporte. C’est pourquoi il
est souvent associé au beau, sans l’être toutefois nécessairement. Une tempête
est-elle belle ? Cela peut dépendre des goûts (Kant la jugeait hideuse). Mais elle
. n’en donnera pas moins le sentiment du sublime, par la démesure, par l’excès
de grandeur ou de force, par l'évidence, face à elle, de notre peritesse, de notre
impuissance, de notre fragilité. Est sublime ce qui semble absolument grand :
ce en comparaison de quoi je ne suis rien, ou presque rien. Et qui fait en moi
comme une mort heureuse.

Lors de son premier voyage en Grèce, qu’il fit tard, Marcel Conche
m'envoya une carte postale d'Athènes, représentant le Parthénon. Au dos, cette
phrase : « Si Kant avait connu le Parthénon, il n'aurait pas opposé le beau et le
sublime. » C’est que la même admiration, qui m'écrase, me réjouit.

%— 561 —
%{\footnotesize XIX$^\text{e}$} siècle — {\it }
SUBSOMPTION  Le fait de subsumer, autrement dit de penser le particulier
sous le général, par exemple un objet sous un concept ou
une action sous une règle.

SUBSTANCE  Étymologiquement, c’est ce qui est {\it sous}. Sous quoi ? Sous l’apparence,
sous le changement, sous les prédicats : la substance
est un autre mot pour dire l'essence, la permanence, le sujet, ou la conjonction
des trois.

La {\it substance} c’est l'{\it essence}, c’est-à-dire l’{\it être} : {\it ousia}, en grec, peut se traduire,
selon l’auteur ou le contexte, par l’un ou l’autre de ces trois mots. En ce sens,
seul l'être individuel est véritablement substance : lui seul est un être, à proprement
parler. Ainsi Socrate, ce caillou ou Dieu. L’humanité, la minéralité ou la
divinité ne sont que des abstractions.

La substance, c’est ce qui demeure identique à soi sous la multiplicité des
accidents ou des changements. Et il faut que quelque chose demeure, sans quoi
tout changement et tout accident deviendraient inintelligibles (puisqu'il n’y
aurait rien qui puisse changer, ni à quoi quelque chose puisse arriver). Par
exemple quand je dis que Socrate a vieilli: cela suppose qu’il est toujours
Socrate. La substance, en ce sens, est le sujet du changement, en tant que ce
sujet subsiste ou persiste.

C’est aussi le sujet d’une proposition : « ce dont tout le reste est affirmé,
comme dit Aristote, mais qui n’est pas lui-même affirmé d’autre chose » ({\it Méta-physique},
Z, 3). Le sujet de tous les prédicats, donc, qui n’est prédicat d’aucun
sujet. Par exemple quand je dis que Socrate est juste ou se promène. Ni la justice
ni la promenade ne sont des substances : ce ne sont que des prédicats, attribués
à une substance (en l’occurrence Socrate), laquelle ne saurait être le prédicat
d'aucune substance. C’est cette acception logique du mot qui explique
qu’Aristote, à propos des termes généraux, parle parfois de «substances
secondes » : l’homme ou l’humanité peuvent être le sujet d’une proposition,
qui leur attribue tel ou tel prédicat. Mais ce ne sont des substances que par
analogies : seuls les individus sont des « substances premières », c’est-à-dire des
substances proprement. C’est où Aristote s’écarte de Platon, et tel est peut-être
le principal enjeu de cette notion, aujourd’hui vieillie, de substance.

Chez Kant, la substance est l’une des trois catégories de la relation. Elle est
ce qui ne change pas dans ce qui change. Son schème est « la permanence du
réel dans le temps ». Son principe, la « première analogie de l'expérience » :
« Tous les phénomènes contiennent quelque chose de permanent ({\it substance})
considéré comme l’objet lui-même, et quelque chose de changeant, considéré
comme une simple détermination de cet objet, c’est-à-dire un mode de son
%— 562 —
%{\footnotesize XIX$^\text{e}$} siècle — {\it }
existence » ({\it C. R. Pure}, Analytique des principes). Ou bien, dans la deuxième
édition : « La substance persiste dans tout le changement des phénomènes, et sa
quantité n’augmente ni ne diminue dans la nature. » Il y a bien longtemps que
cette permanence a cessé, pour nos physiciens, d’être une évidence. Si quelque
chose se conserve, c’est l’énergie. Mais ce n’est pas une chose, ni un être individuel,
ni un sujet (si ce n’est au sens purement logique du terme). Quel sens y
aurait-il à y voir une substance ?

SUBSUMER Mettre sous ou dans. C’est inscrire un être ou une catégorie
dans un ensemble plus général. Par exemple Socrate peut être
subsumé sous le concept d'homme, qui peut à son tour être subsumé sous celui
de mammifère, qui peut l’être à son tour sous celui d’animal... On n’y gagne
ordinairement pas grand-chose : « On échange un mot pour un autre mot,
comme dit Montaigne, et souvent plus inconnu. Je sais mieux ce que c’est
qu’homme que je ne sais ce que c’est qu’animal, ou mortel, ou raisonnable.
Pour satisfaire à un doute, ils m'en donnent trois : c’est la tête de l’hydre »
(III, 13, p. 1069). C’est dire qu'aucune subsomption ne saurait suffire à
définir : l’emboîtement des généralités importe moins que l’enchaînement des
causes, des idées ou des expériences.

SUGGESTION C'est agir sur quelqu'un par des signes, sans avoir besoin
pour cela de le convaincre. C’est une espèce de magie, ou
plutôt la magie, presque toujours, n’est qu’une espèce de suggestion.

La suggestion culmine dans l’hystérie et l'hypnose, mais se manifeste, à des
degrés divers, dans tout groupe humain. Cet homme qui bâille me fait bäiller.
Cet autre qui me trouve mauvaise mine ou m’annonce ma mort me rend
presque malade, voire malade tout à fait. C’est que je suis sous influence, sans
le vouloir, parfois sans le savoir. C’est la suggestion même : une influence que
lon subit involontairement, et qui passe moins par la raison ou la volonté que
par limitation ou la soumission. Les individus y sont plus ou moins sensibles.
C’est pourquoi «il n’est permis de parler librement, disait Alain, qu’à celui
dont on prévoit qu’il résistera librement ». C’est refuser la magie ou la manipulation.

SUICIDE  L’homicide de soi. C’est pourquoi certains y voient un crime,
C’est pourquoi j'y vois un droit. « Comme je n’offense les lois qui
sont faites contre les larrons quand j’emporte le mien et que je me coupe ma
%— 563 —
%{\footnotesize XIX$^\text{e}$} siècle — {\it }
bourse, écrit Montaigne, ni des boutefeux [les incendiaires] quand je brûle
mon bois, aussi ne suis-je tenu aux lois faites contre les meurtriers pour m'avoir
ôté la vie » ({\it Essais}, II, 3). Attention toutefois de ne pas faire du suicide plus de
cas qu’il ne convient. Ce n’est ni un sacre ni un sacrement. Ni une morale ni
une métaphysique. Se suicider, c’est choisir non la mort (c’est un choix que
l’on n’a pas : il faudra mourir de toute façon) mais {\it le moment} de sa mort. C’est
un acte tout d'opportunité, et point l’absolu parfois qu’on veut y voir. Il s’agit,
ni plus ni moins, de gagner du temps sur l’inévitable, de devancer le néant, de
prendre le destin, si l’on veut, de vitesse. C’est le raccourci définitif.

C’est aussi un droit, pour chacun, d’autant plus absolu qu’il se moque du
droit. « Le présent que nature nous ait fait le plus favorable, écrit encore Montaigne,
c'est de nous avoir laissé la clef des champs » ({\it ibid.}). C’est la liberté
minimale et maximale.

SUJET Ce qui est {\it jeté sous} ou {\it sous-jacent}. C’est un équivalent, au moins pour
l'étymologie, de « substance » ou d’« hypostase ». Mais l’étymologie
n'a jamais suffi à faire une définition.

Qu'est-ce qu’un sujet ? Pour la logique, c’est un être quelconque, dès lors
qu'on lui attribue un prédicat. Par exemple quand je dis « La Terre est ronde » :
{\it Terre} est le sujet, ronde le {\it prédicat}. On voit que le mot, en cette acception, ne
dit rien sur la nature du sujet, sinon qu’il est un être (une substance) ou considéré
comme tel (une entité).

Pour la philosophie politique, le {\it sujet} s'oppose au {\it souverain}, comme ce qui
obéit à ce qui commande, et au {\it citoyen}, comme ce qui n’est pas libre à ce qui
l'est. Cela n'exclut pas que les mêmes individus soient à la fois sujets (membres
du peuple) et citoyens (membres du souverain) ; mais ne le garantit pas non
plus.

Dans la philosophie moderne, le mot touche davantage à la théorie de la
connaissance et à la morale, voire à la métaphysique : le sujet s’oppose à l’objet
comme ce qui connaît à ce qui est connu, ou comme ce qui veut et agit à ce qui
est fait. Ce serait l’être humain, ou plutôt une certaine façon de le penser :
comme le sujet (au sens à la fois grammatical et ontologique) de sa pensée et de
sa vie. C’est en ce sens qu’on parle — à propos de Descartes, de Kant, de
Sartre... — d’une « {\it philosophie du sujet} ». Le sujet, c’est celui qui dit {\it je}, pour
autant qu'il se désigne légitimement par là : c’est celui qui pense ou agit, mais
en tant qu'il serait le principe de ses pensées ou de ses actes, plutôt que leur
somme, leur flux ou leur résultat. Par exemple chez Descartes : « {\it Je pense, donc
Je suis} ». C’est que {\it penser} est un verbe, qui suppose un sujet. Métaphysique de
grammairien, qui prend le langage pour une preuve ou la grammaire pour une
%— 564 —
%{\footnotesize XIX$^\text{e}$} siècle — {\it }
métaphysique. Autant dire « {\it Il pleut, donc il est} ». Cela ne fera pas de la pluie
un sujet.

Hume, avant Nietzsche, a montré, dans un chapitre génial de son {\it Traité},
que nous n’avons aucune expérience, ni donc aucune connaissance, d’un tel
{\it sujet} : que nous ne connaissons de nous-mêmes « qu’un faisceau ou une collection
de perceptions différentes qui se succèdent les unes aux autres avec une
rapidité inconcevable », sans que rien nous autorise à penser que nous sommes
autre chose que leur flux ni, donc, que nous en sommes la cause, la substance
ou le principe sous-jacents ({\it Traité de la nature humaine}, I, IV, 6, « L'identité
personnelle »). C’est un des rares textes où la philosophie occidentale, sans le
savoir, s'approche du bouddhisme. Pas de sujet, pas de moi, sinon illusoire, pas
de soi ({\it anatta}) : tout n’est que flux et agrégats, qu’impermanence et processus
({\it paticca-samuppada} : production conditionnée). « Seule la souffrance existe,
mais on ne trouve aucun souffrant ; les actes sont; mais on ne trouve pas
d’acteur. Il n’y a pas de moteur immobile derrière le mouvement. Il n’y a pas
de penseur derrière la pensée » (W. Rahula, {\it L'enseignement du Bouddha}, 2 ; voir
aussi le chap. 6). Le sujet n’en demeure pas moins, comme croyance, comme
illusion, comme {\it mot} ; mais il n’explique rien. Il n’est pas ce que nous sommes,
mais ce que nous croyons être. Non une substance, mais une hypostase. Non
notre vérité, mais notre méconnaissance (l’ensemble des illusions que nous
nous faisons sur nous-mêmes). Non le principe de nos actes ou de nos pensées,
mais leur enchaînement, qui nous enchaîne. Non un principe, mais une histoire.
Non notre liberté de sujet, mais notre {\it assujettissement}.

On n’en conclura pas qu’il faudrait pour cela renoncer à la liberté. Mais
que la subjectivité ne saurait y suffire : seule la vérité libère, qui n'est pas un
sujet.

Philosophie non plus du sujet mais de la connaissance. Non plus de la
liberté, mais de la libération.

SUPERSTITION C’est donner du sens à ce qui n’en a pas. Par exemple un
chat noir, un rêve, une éclipse. Notion polémique, donc
relative : on est toujours le superstitieux de quelqu'un, qui se prétend le seul
herméneute légitime. C’est en quoi la superstition se distingue de la religion,
du moins pour les croyants (parce qu’elle invente de faux signes ou de faux
dieux), et tend à l’absorber, pour les athées (puisque aucun Dieu n’est le vrai,
ni aucun sens).
On dira que la psychanalyse donne bien un sens aux rêves ou aux symptômes,
sans relever pour autant de la superstition. Sans doute. Mais c’est que
ce sens n’est que l’envers d’un processus causal : rêves et symptômes ne sont des
%— 565 —
%{\footnotesize XIX$^\text{e}$} siècle — {\it }
{\it symboles} que parce qu’ils sont d’abord des {\it indices} ou des {\it effets}. Aussi n’ont-ils
rien de surnaturel : leur interprétation finit par déboucher sur quelque chose —
la sexualité, l'inconscient — qui est dépourvu de toute signification transcendante,
et même immanente. La sémiologie renvoie à une étiologie, qui
l'explique et la borne. Il n’y a pas de sens du sens, ni de sens absolu, ni de sens
ultime : il n’y a que le réel et la pulsion, qui ne signifient rien. C’est où Freud
est le contraire d’un superstitieux, et la psychanalyse, le contraire d’une religion.
Toute superstition soumet le réel au sens : elle explique ce qui est (un
rêve, une éclipse, un chat noir) par ce que cela veut dire (par exemple un malheur
à venir). L'analyse fait l'inverse. Elle soumet le sens au réel : elle explique
ce que cela veut dire (le sens d’un rêve, d’un acte manqué, d’un symptôme) par
ce qui est (un désir refoulé, un traumatisme, une névrose). La superstition
donne du sens à ce qui n’en a pas ; la psychanalyse ramène le sens à autre chose,
qui le dissout. C’est pourquoi l’on se trompe quand on demande à la psychanalyse
le sens de sa vie. Elle ne peut donner que le sens de nos symptômes ou
de nos rêves. Ou bien ce n’est plus analyse mais superstition. Plus connaissance
(de mon histoire) mais religion (de mon inconscient). Pauvres petits analysants,
qui cherchent un sens ! Freud, lui, ne cherchait que la vérité. On dira que les
deux sont liés, que telle est la {\it voie royale} de la psychanalyse. Reste, toutefois,
à ne pas la prendre à contresens. Freud est le contraire d’un prophète. Il
n'annonce pas, il explique. Il ne parle pas, il écoute. Le sens, chez lui, n’est
qu'un chemin vers la vérité. C’est toujours superstition, à l'inverse, que de ne
voir dans la vérité qu’un chemin vers le sens. L’inconscient parle, sans doute ;
mais il n’a rien à dire. La cure est de paroles (c’est une « {\it talking cure} ») ; mais la
santé, de silence.

Notons pour finir que toute superstition tend à se vérifier. Celui qui casse
un miroir et s’en effraie, sa crainte confirme déjà le présage qui l’inspire. La
superstition porte malheur.

SURHUMAIN Ce qui excède la mesure humaine. Nietzsche y voyait un but
et un sens (« le sens de la terre ») : l’homme n’existerait que
pour être dépassé ; le surhomme serait à l’homme ce que l’homme est au singe
({\it Zarathoustra}, T, Prologue). Montaigne, contre les stoïciens et plus raisonnablement,
n’y voyait qu’une sottise : « À la vile chose, dit Sénèque, et abjecte que
l’homme, s’il ne s'élève au-dessus de l'humanité ! Voilà un bon mot et un utile
désir, mais pareillement absurde. Car de faire la poignée plus grande que le
poing, la brassée plus grande que le bras, et d’espérer enjamber plus que
l'étendue de nos jambes, cela est impossible et monstrueux. Ni que l’homme se
%— 566 —
%{\footnotesize XIX$^\text{e}$} siècle — {\it }
monte au-dessus de soi et de l'humanité » (II, 12, p. 604). Être pleinement
humain fait une tâche suffisante.

SURMOI L'une des trois instances (avec le {\it moi} et le {\it ça}) de la seconde
topique de Freud : c’est l'instance de la moralité, des idéaux, de la
Loi. Elle résulte de l’intériorisation des interdictions et valorisations parentales.
Ce qu’ils nous ont interdit, voilà que nous nous l’interdisons à nous-mêmes ;
ce qu’ils nous ont imposé, voilà que nous nous l’imposons ; ce qu’ils aimaient,
voilà que nous le jugeons aimable. Ce n’est pas toujours le cas ? Certes, puisque
cette intériorisation ne se fait ni toujours ni complètement. C’est pourquoi
nous n'avons pas tout à fait la même morale que nos parents. Il n’en reste pas
moins que chaque génération éduque la suivante, et que toute morale pour cela
vient du passé. Il n’y a pas de morale de l’avenir : il n’y a de morale que présente —
que fidèle et critique. On aurait tort d’en faire un absolu, voire d’y
croire tout à fait. Mais tort aussi de prétendre s’en exempter. Il n’est pas
interdit d'interdire, et même il est interdit de ne s’interdire rien.

Le {\it surmoi} représente le passé de la société, explique Freud, de même que
le {\it ça} représente le passé de l'espèce. Ce n’est pas une raison pour les juger l’un
et l’autre réactionnaires. Sans le ça, pas d’avenir. Sans le surmoi, pas de progrès.

SURNATUREL Qui excéderait la puissance de la nature. Ce ne peut être
que de la magie, de la superstition ou de la religion, et c’est
pourquoi, pour un matérialiste, ce n’est rien.

SYLLOGISME Un type de raisonnement déductif, formalisé par Aristote,
qui unit trois termes, reliés deux à deux et dont chacun apparaît
deux fois, en trois propositions. Il est susceptible de plusieurs formes ou
figures différentes. Mais l'exemple canonique, qu’on ne trouve pas chez Aristote,
est le suivant :
\begin{center}
{\it Tout homme est mortel ;

Socrate est un homme ;

Donc Socrate est mortel.}
\end{center}
Les deux premières propositions sont les prémisses (majeure et mineure) ;
la troisième, la conclusion. Les trois termes (mortel, homme, Socrate) sont
appelés respectivement grand terme, moyen terme et petit terme. On remarquera
que l’ordre des prémisses n’est pas ce qui importe, ni même toujours leur
%— 567 —
%{\footnotesize XIX$^\text{e}$} siècle — {\it }
extension. Le grand terme est celui qui sert de prédicat dans la conclusion ; la
majeure, celle des deux prémisses qui contient le grand terme. Le petit terme
est celui qui sert de sujet dans la conclusion ; la mineure, celle des deux prémisses
qui contient le petit terme. Enfin le moyen terme est le seul à apparaître
dans les deux prémisses : c’est lui qui les met en rapport et permet la conclusion,
où il ne figure pas.

Le syllogisme est-il valide ? Cela dépend bien sûr des prémisses. De leur
vérité ? Non pas. Certes, la conclusion n’est nécessairement vraie que si les prémisses
le sont ; mais cela touche moins à la validité du raisonnement qu’au
contenu des trois propositions. Soit par exemple ce syllogisme, qu’on trouve
chez Lewis Carroll :

\begin{center}
{\it 
Tous les chats comprennent le français ;

Quelques poulets sont des chats ;

Donc quelques poulets comprennent le français.}
\end{center}

Que l’inférence soit formellement valide, cela ne garantit nullement la
vérité de la conclusion. Mais l’inverse est vrai aussi : que la conclusion puisse
être fausse (puisque les prémisses le sont) n’annule pas la validité, au moins formelle,
du raisonnement. On s’en rend compte si l’on donne au syllogisme, à la
façon d’Aristote, la forme d’une implication. La proposition « {\it Si tous les chats
comprennent le français et si quelques poulets sont des chats, alors quelques poulets
comprennent le français} » est une proposition vraie. L'essentiel, d’un point de
vue logique, n’est pas dans le contenu des propositions, mais dans la légitimité
de l’inférence. Deux prémisses fausses, comme dans l'exemple de Lewis Carroll,
peuvent justifier cette inférence, alors que deux prémisses, même vraies, n’autorisent
pas forcément à conclure. D’abord parce qu’il faut qu’elles aient un
moyen terme commun, dans un même genre : Que tous les hommes soient
mortels et que Milou soit un chien, cela peut faire l’objet de deux propositions,
mais pas constituer les deux prémisses d’un syllogisme. Ensuite parce que nos
deux prémisses doivent encore respecter un certain nombre de règles, qu’on
trouve dans les manuels. Par exemple : « De deux propositions particulières, on
ne peut rien conclure » (que certains hommes soient mortels et que certains
philosophes soient des hommes, c’est assurément vrai, mais ne permet pas de
savoir si les philosophes sont mortels, ni lesquels). Ou encore : « De deux propositions
négatives, on ne peut rien conclure » (qu'aucun homme ne soit

. immortel et que Socrate ne soit pas un chien, c’est très vraisemblable, mais ne
nous dit rien sur la mortalité de Socrate). Ces règles, qui sont nombreuses, sont
aussi passablement oubliées. Je ne connais guère de philosophes qui s’en servent.
Mais j’en connais encore moins, en tout cas parmi les bons, qui les transgressent.

%— 568 —
%{\footnotesize XIX$^\text{e}$} siècle — {\it }
SYMBOLE Parfois synonyme de signe, voire (par influence de l’anglo-américain,
et surtout depuis Peirce) de signe conventionnel : c’est en
ce sens, par exemple, qu’on parle de symboles mathématiques.

Mais la langue résiste. Un feu rouge n’est pas un symbole. Un mot n’est pas
un symbole. La colombe en est un, pour la paix, comme la balance, pour la justice.
Les signes mathématiques ? Cela dépend lesquels. Les signes +, — ou $\surd$ ne
sont pas des symboles ; les signes > ou < en sont, même pauvres et au moins
partiellement.

Qu'est-ce qu’un symbole ? C’est un signe non arbitraire et non exclusivement
conventionnel, dans lequel le signifiant (par exemple l’image d’une
colombe, ou l’image d’une balance) et le signifié (par exemple l’idée de paix, ou
l’idée de justice) sont unis par un rapport de ressemblance ou d’analogie. Un
faucon ne ferait pas aussi bien l'affaire, ou symboliserait autre chose. C’est qu'il
y a en effet quelque chose de paisible, ou que nous jugeons tel, dans l'aspect ou
le comportement des colombes. Et que la balance doit être {\it juste}, c’est-à-dire
respecter une forme d’égalité ou de proportion entre ses deux plateaux. C'est
pourquoi les symboles sont souvent suggestifs : parce qu’ils unissent le sensible
et l’intelligible, l'imaginaire et la pensée. Aussi convient-il, en philosophie, de
s’en méfier. Le meilleur symbole ne remplacera jamais un argument.

SYMPATHIE C’est sentir avec, ensemble ou de la même façon. Le mot dit
la même chose, en grec, que {\it compassion} en latin. Mais cela ne
fait pas, en français, deux synonymes. C’est que la sympathie est affectivement
neutre : on peut sympathiser dans la joie comme dans la tristesse. Alors que la
compassion, en français, ne se dit que négativement : on compatit avec la souffrance
ou le malheur d’autrui, non avec sa joie ou son bonheur. C’est ce qui
rend la sympathie plus sympathique, plus plaisante, et plus équivoque. Qui
voudrait partager la joie du méchant ou le plaisir du tortionnaire ? Toute souffrance
mérite compassion. Toute joie ne mérite pas sympathie.

SYMPTÔME C’est un effet qui désigne sa cause. D’où l'illusion d’un sens,
quand il faudrait n’y voir que causalité. Une fièvre ne veut rien
dire. Mais elle a une cause, qu’on peut connaître et combattre.

Le mot sert surtout pour désigner les signes — c’est-à-dire les effets observables
et reconnaissables — des maladies. Mais il est susceptible, spécialement
depuis Freud, d’un emploi plus vaste. Vous bâillez ? C’est un symptôme de
fatigue, ou d’ennui, ou de dépression. Vous avez oublié votre parapluie ?
Symptôme. Vous êtes en retard ? Symptôme. En avance ? Symptôme. Juste à
%— 569 —
%{\footnotesize XIX$^\text{e}$} siècle — {\it }
l'heure ? Symptôme. C’est la forme moderne de l’herméneutisme ou de la
superstition. Tout est symptôme en nous, ou peut l'être, qu’on interprète, faute
d'en connaître les causes, pour en comprendre le sens. Et en effet c’est toujours
possible. Mais à quoi bon ? Si tout est symptôme, et l'interprétation elle-même,
c'est que tout a une cause, qu’on ignore le plus souvent et qu’on ne peut en
conséquence qu'imaginer.. Labyrinthe de l'imaginaire, dont la vérité seule,
dirait Spinoza, guérit. La santé est le silence des organes. La sagesse, le silence
de l'esprit. Tout peut avoir un sens, mais le sens n’en a pas. Tout peut s’interpréter,
mais à proportion seulement de l’ignorance qu’on en a.

SYNCRÉTISME C’est comme un éclectisme sans choix ou sans rigueur : la
juxtaposition de plusieurs thèses mal coordonnées, empruntées
à des doctrines incompatibles ou disparates.

Piaget et Wallon ont appelé {\it syncrétisme} une tendance de la perception et de
la pensée du jeune enfant, qui perçoit l’ensemble plutôt que les détails : il relie
tout à tout de façon globale et confuse. La précision et la rigueur ne viendront
que plus tard.

SYNDROME Ensemble de symptômes. Une maladie ? Pas forcément. Pas toujours.
Ce pourrait être plusieurs maladies différentes. Ce sera
au diagnostic d’en décider. Le syndrome est son point de départ, qui relève de
l'observation. La maladie, comme concept, serait plutôt son point d’arrivée,
qui relève de l'explication. La guérison ? Ce n’est plus diagnostic, mais pronostic
ou thérapie.

SYNTHÈSE {\it Sunthesis}, en grec, c’est la réunion, la composition, l’assemblage :
synthétiser, c’est poser ({\it tithenai}) ensemble ({\it sun}). La synthèse
s’oppose par là à l'analyse, qui sépare ou décompose.

La synthèse va du simple au composé, disait Leibniz. Elle constitue ou
reconstitue un tout à partir d'éléments déjà donnés. L'analyse va du composé
au simple. Elle décompose un tout en ses éléments. Cela peut s'entendre
chimiquement : on peut produire une eau de synthèse, en assemblant des
atomes d’oxygène et d'hydrogène. Et obtenir par analyse, à partir d’une eau
quelconque, des atomes d'oxygène et d'hydrogène.

Mais le mot « synthèse », en philosophie, sert surtout à désigner un processus
intellectuel, qui ne compose guère que des idées. C’est le cas, spécialement,
chez Descartes. La synthèse constitue, après celles de l'évidence et de
%— 570 —
%{\footnotesize XIX$^\text{e}$} siècle — {\it }
l'analyse, la troisième règle de sa méthode : « Conduire par ordre mes pensées,
en commençant par les objets les plus simples et les plus aisés à connaître, pour
monter peu à peu, comme par degrés, jusqu'à la connaissance des plus
composés » ({\it Discours de la méthode}, II). Il en résulte que l’analyse est première
(le simple n’est pas donné d’abord : il faut le conquérir), et plus proche de « la
vraie voie par laquelle une chose a été méthodiquement inventée » (Réponses
aux IF objections, AT, 121). La synthèse sert surtout à démontrer ce qu’on
connaît déjà : c’est une méthode d’exposition plutôt que de découverte. Comparez
par exemple, chez Descartes, les {\it Méditations}, qui sont écrites selon l’ordre
analytique, et les {\it Principes de la philosophie}, qui suivent l’ordre synthétique.

Mais la synthèse n’est pas seulement un ordre ou une méthode. C’est
aussi un moment, chez Hegel ou Marx, de la dialectique : celui où les deux
contraires sont réunis en un troisième terme, qui les dépasse (c’est-à-dire les
supprime tout en les conservant). Ainsi le devenir, après l’être et le néant
(Hegel, {\it Logique}, I, 1). Le fruit, après la graine et la plante (Engels,
{\it Anti-Dühring}, XI). Ou le communisme, après la lutte des classes (le prolétariat « ne
l'emporte qu’en s’abolissant lui-même et en abolissant son contraire », qui est
la propriété privée : Marx et Engels, {\it La sainte famille}, IV ; voir aussi {\it Le Capital},
I, 8$^\text{\,e}$ section, chap. 32). Négation de la négation : nouvelle affirmation. C’est
un beau moment, dans le travail du négatif : celui du repos, mais en mouvement,
et de la fécondité. Trop beau pour être vrai ? C’est ce qu’on peut penser.
On ne lui est pas fidèle, en tout cas, en le réduisant, comme souvent nos étudiants
dans leurs dissertations, à une apologie du juste milieu, de la mollesse ou
de l’indécision. Je pense, on l’a compris, au fameux plan « Thèse, antithèse,
synthèse ». C’est un plan comme un autre, qui n’est ni obligatoire ni interdit.
Encore faut-il le comprendre dialectiquement. {\it Thèse}, {\it antithèse}, {\it synthèse}, cela ne
saurait vouloir dire : {\it Blanc}, {\it Noir}, {\it Gris}. Et pas davantage : {\it Oui}, {\it Non}, {\it Peut-être}.
Ou bien ce n’est plus une synthèse, mais une échappatoire ou un compromis.
Plus de la dialectique, mais du chewing-gum.

SYNTHÉTIQUES (JUGEMENTS -) Ce sont les jugements qui « ajoutent
au concept du sujet un prédicat qui
n'était pas du tout pensé dans le sujet, et qu'aucune analyse de celui-ci n'aurait
pu en tirer » (C. R Pure, introd., IV). S’opposent aux jugements analytiques.
Par exemple, explique Kant, « {\it Tous les corps sont étendus} » est un jugement analytique
(la notion d’étendue est incluse dans celle de corps: un corps sans
étendue serait contradictoire) ; alors que « {\it Tous les corps sont pesants} » est un
jugement synthétique (la notion de poids n’est pas comprise dans celle de
corps : l’idée d’un corps sans poids n’est pas intrinsèquement contradictoire ;
%— 571 —
%{\footnotesize XIX$^\text{e}$} siècle — {\it }
seule l’expérience nous apprend qu’ils en ont tous un). Les jugements d’expérience
sont tous synthétiques, souligne Kant, mais tous les jugements synthétiques
ne sont pas d'expérience. Il en est certes qui sont {\it a posteriori} (c’est le cas
de « Tous les corps sont pesants »), mais d’autres qui sont {\it a priori} (« {\it Tout ce qui
arrive a une cause} »). Expliquer la possibilité de ces derniers — donc la possibilité
des sciences — est l’un des enjeux principaux de la {\it Critique de la raison pure}.

SYSTÈME Un assemblage ordonné, où chaque élément est nécessaire à la
cohésion de l’ensemble et en dépend. Ainsi parle-t-on du système
nerveux, du système solaire, d’un système informatique. En philosophie, se
dit le plus souvent d’un ensemble d’idées, « mais en tant qu’on les considère
dans leur cohérence, comme dit Lalande, plutôt que dans leur vérité ». C’est
que la pluralité même des systèmes, qui sont incompatibles (puisque chacun
prétend dire la vérité sur le tout), interdit de les accepter tous comme de se
satisfaire de l’un d’entre eux. Les systèmes sont tous faux, disait Alain, et le système
des systèmes, qui serait l’hégélianisme (parce qu’il fait de la contradiction
entre les systèmes le moteur de son propre développement) l’est tout autant. La
cohérence n’est pas une preuve. Elle n’est même pas un argument. Combien de
délires cohérents ? La paranoïa, disait Freud, est « un système philosophique
déformé » ; un système philosophique, ajouterais-je volontiers, est une paranoïa
réussie. Jaime mieux les contradictions de la vie et les aspérités du réel.

Que beaucoup de grandes philosophies soient des systèmes n’est pourtant
pas niable. Presque toutes y tendent, et pour des raisons nécessaires. Il faut bien
faire tenir ensemble ce qu’on croit vrai. Il faut bien penser tout, ou le tout. Le
système est l'horizon de la philosophie : c’est une pensée où tout se tient,
comme une synthèse supérieure, comme un tout organique, et cela vaut mieux
qu'une pensée qui se délite ou se contredit. Reste à n’en être pas dupe, à ne pas
prendre cette cohérence pour une preuve, à ne pas s’y enfermer. Pourquoi faudrait-il
te soumettre à ce que tu as déjà pensé ? C’est la vérité qui importe, non
la cohérence. La pensée d’aujourd’hui, non celle d’hier. Le réel, non le système.
Quelle tristesse que de ne penser que pour se donner raison ! Quelle folie que
de prétendre posséder l’horizon ! Les sciences donnent un meilleur exemple,
qui font tout pour être contredites, et qui avancent par là. Platon donne un
meilleur exemple, et Montaigne, et Pascal. L’horizon est devant nous ; il faut
donc avancer. Le système serait la philosophie de Dieu. Mais Dieu n’est pas
philosophe.

Il y a quelque chose de pathétique chez les auteurs de système. Ils croient
penser le tout ; ils ne font que bricoler leurs petites idées. Comment pourraient-ils
contenir l’univers, puisqu'ils en font partie ? Le monde continue sans

%— 572 —
%{\footnotesize XIX$^\text{e}$} siècle — {\it }
eux. La philosophie continue sans eux, et c’est tant mieux. Si un système réussissait,
c’en serait fini de la philosophie. Mais ils ont tous échoué, même les plus
grands. Le cartésianisme est mort. Le leibnizianisme est mort. Le spinozisme est
mort. Raison de plus pour lire Descartes, Leibniz ou Spinoza, qui valent mieux
que leurs systèmes. Battez les cartes et les idées. Le jeu n’est pas fait ; il est à
faire.

Un système philosophique est comme un château de cartes : vous en retirez
une, tout le reste s’écroule. C’est que chaque carte ne tient que par l’ensemble,
et le tient. J'aime mieux le jeu ouvert et vivant, les cartes qui volent, les joueurs
qui s'affrontent, les plis qui se font ou se défont, jusqu’à la victoire, jusqu’à la
défaite, jusqu’à la prochaine partie. J’aime qu’on batte les cartes à chaque fois.
Qu'on invente ses coups à chaque fois, en fonction du jeu et des adversaires.
Que chaque partie soit neuve et incertaine. L’infini est là, non dans le château
de cartes dérisoire et figé.


%
%TU {\it }
\chapter{TU}

\section{Tabou}
%TABOU
C’est comme un interdit sacré. De là l’envie de le violer, par curiosité,
par défi, par bravade. Mieux vaut une loi claire, librement
acceptée et discutée.

\section{Talent}
%TALENT
Plus qu’un don, moins que du génie. Un enfant qui est doué pour
les mathématiques ou le dessin, on ne dira pas forcément qu’il a du
talent. Et un artiste talentueux ou génial, comme Cézanne, peut n'être que
moyennement doué. Le don est une facilité à apprendre. Le talent, une puissance
de créer. Le don est donné à la naissance : il touche à la génétique. Le
talent se conquiert davantage pendant l’enfance et l’adolescence : il touche à
l’histoire, à la psychologie, à l'aventure d’être soi ou de le devenir. Le don est
impersonnel. Le talent serait plutôt la personne même, quand elle parvient à
s'exprimer de façon créatrice et singulière.

On sait que le mot vient d’une métaphore. Dans la fameuse {\it parabole des
talents}, Jésus compare implicitement les capacités que chacun a reçues à des
pièces de monnaie (des « talents »), qu’il doit faire fructifier. L'important, c’est
moins le talent qu’on a que ce qu’on en fait : ce n’est plus talent, mais œuvre
ou gâchis.

\section{Tautologie}
%TAUTOLOGIE
Une proposition qui est toujours vraie, soit parce que le prédicat
ne fait que répéter le sujet (« Dieu est Dieu »), soit
parce qu’elle reste valide indépendamment de son contenu et même de la
valeur de vérité des éléments qu’elle met en œuvre. La logique formelle est faite
de tautologies : « Si {\it p} implique {\it q}, et si {\it non-q}, alors {\it non-p} » (c'est ce qu’on
% 574
appelle le {\it modus tollens}) est toujours vrai, quels que soient le contenu et la
valeur de vérité de {\it p} et de {\it q}.

On remarquera que le mot de tautologie, pris en ce sens, n’a rien de péjoratif,
Mais même pris au sens de répétition, il ne l’est pas forcément. Quand
Parménide nous dit que l’être est, il fait une tautologie. Mais cela, loin de le
réfuter, le rend irréfutable.

\section{Technique}
%TECHNIQUE
Un ensemble d'instruments (outils, machines, logiciels.) et
de savoir-faire, permettant d’obtenir un certain résultat.

Le mot vient du grec {\it tekhnê}, qui est l’équivalent du latin {\it ars}. Mais les deux
vocables, en français, ont évolué en des sens opposés. La technique se distingue
de l’art, et même de l’artisanat, par son efficacité impersonnelle : un objet technique
peut être fabriqué identiquement par tous les individus compétents et
convenablement outillés ; un produit artisanal ou une œuvre d’art, non. C’est
pourquoi l’art est plus singulier, plus contrasté, plus précieux. Et la technique,
plus efficace.

Les techniques constituent un élément essentiel du progrès, aussi bien pour
l'individu (« sans technique, chantait Brassens, un don n’est rien qu’une sale
manie ») que pour la société (c’est à elles qu’on doit le développement des
forces productives, comme disait Marx, dont tout le reste découle). Elles n’en
ont pas moins leurs dangers ; mais elles valent mieux, presque toutes, presque
toujours, que leur contraire, qui est soumission aveugle à la nature. Tailler un
silex, fût-ce pour s’en faire une arme, cela vaut mieux que se laisser dévorer ou
massacrer.

On a beaucoup reproché à Descartes d’avoir voulu, selon l'expression
fameuse du {\it Discours de la méthode}, nous rendre « comme maîtres et possesseurs
de la nature ». De là viendrait tout le mal, la mise en coupe réglée de la nature,
son pillage, son saccage (son {\it arraisonnement}, dit Heidegger), le travail à la
chaîne, la bombe atomique, la dégradation irréversible de l’environnement.
L’oubli de l’être déboucherait sur le culte de l'utile et du rendement, l’humanisme
sur le machinisme, le rationalisme sur la barbarie et la déraison... C’est
oublier le « {\it comme} », qui maintenait une certaine distance (s’il s’agit de nous
rendre {\it comme} maîtres et possesseurs de la nature, c’est que nous ne le sommes
pas et ne le serons jamais). C’est oublier, surtout, d’où nous venons. Un de mes
amis, écologiste radical, ne jure que par le paléolithique : la révolution néolithique
serait la faute première, dont toutes les autres découlent. Les hommes,
m’explique-t-il, cessent alors de vivre en harmonie avec la nature : ils commencent
à la transformer, à la pressurer, à la défigurer.. Pour d’autres ce serait la
révolution industrielle, les technosciences, la révolution informatique. Ils ne
% 575
me feront pas regretter la préhistoire, ni le Moyen-Âge, ni même le {\footnotesize XIX$^\text{e}$} siècle.
Une imprimerie vaut mieux qu’un stylet. Un ordinateur vaut mieux qu’un
boulier. Une machine à laver, mieux qu’un lavoir. Un vaccin, mieux qu’un
grigri.

Le danger n’en demeure pas moins, ou plutôt il ne peut que s’aggraver,
d’une civilisation technicienne, qui prendrait les moyens pour des fins. Nos
techniques, aujourd’hui, sont bien davantage que des outils. Ce sont des pensées,
mais objectivées, mais instrumentalisées. Ce sont des sciences, mais appliquées.
Or quoi de plus normal que de se soumettre au vrai ? Et quoi de plus
vrai que les sciences ? Double contresens : double idolâtrie. La vérité sans la
charité n’est pas Dieu, disait Pascal. Les sciences sans l'humanité sont inhumaines.

Les techniques, historiquement, sont antérieures aux sciences, mais depuis
longtemps transformées par elles : les nôtres en dépendent de plus en plus, au
point d’en être à peu près indissociables (c’est ce qu’on appelle les techno-sciences).
De là une puissance démultipliée, qui devient en effet inquiétante, et
d'autant plus qu’elle s’autonomise davantage : les moyens tendent à nous
imposer leurs fins, ou plutôt à valoir comme telles (l’efficacité devient une
valeur en soi). Nos techniques nous gouvernent, au moins autant que nous les
gouvernons. Avec un marteau, dit-on souvent, on peut faire ce qu’on veut,
planter un clou ou fracasser un crâne. Sans doute. Mais on peut aussi ne rien
faire, et tel est le cas le plus fréquent. Ce n’est plus vrai de nos machines, dont
beaucoup fonctionnent jour et nuit, qu’il faut amortir, qu’il faut rentabiliser,
qui fabriquent d’autres machines, qui créent jusqu'aux besoins qu’elles viennent
satisfaire, qui nous font vivre, et qu’on ne peut plus arrêter, bien souvent,
sans mettre en cause l’existence même de nos sociétés. Nos voitures menacent
l’environnement, ou plutôt elles font plus que le menacer. Mais on ne
reviendra pas à la traction hippomobile. Nos télévisions menacent l’intelligence.
Mais on ne reviendra pas au règne presque exclusif de l'écrit. Il faut
avancer toujours, comme en vélo, mais en essayant de rester maître au moins
de sa vitesse et de sa direction. Personne n’a décidé de faire la révolution industrielle,
ni la révolution informatique et communicationnelle. Comment
quelqu'un pourrait-il les arrêter ou les abolir ? L'histoire des techniques est irréversible,
comme l’histoire des sciences, et pour la même raison. Pas question de
revenir en arrière, et c’est tant mieux. Mais pas question non plus de laisser le
marché ou les machines décider à notre place. Que les techniques créent des
besoins, c’est entendu. Mais comment pourraient-elles remplacer des volontés ?
L'issue, malgré Heidegger, n’est ni dans la technophobie ni dans la contemplation
fascinée de l’être ou des origines, mais dans la soumission résolue des
moyens que nous nous sommes donnés aux fins que nous nous fixons {\bf --} ce qui
% 576
est morale, pour l'individu, et politique, pour les citoyens. Si le peuple est souverain,
il est exclu que les machines ou les technocrates le soient.

\section{Technocratie}
%TECHNOCRATIE
Le pouvoir de la technique, ou plutôt des techniciens.
C’est une forme de barbarie, qui voudrait soumettre la
politique et le droit à l’ordre techno-scientifique : tyrannie des experts. On y
parvient insensiblement, dès lors qu’on veut que les plus compétents gouvernent
ou décident. Contre quoi il faut rappeler que la démocratie non seulement
n’en a pas besoin mais l’exclut : ce n’est pas parce que le peuple est compétent
qu’il est souverain, c’est parce qu’il est souverain qu'aucune compétence
ne saurait, politiquement, valoir sans lui ou contre lui. Les experts sont là pour
l'éclairer, non pour décider à sa place.

\section{Téléologie}
%TÉLÉOLOGIE
L'étude des finalités ({\it telos}, en grec, c’est la fin). Cette étude
peut être utile et légitime, montre Kant, mais à condition de
ne considérer le concept de finalité que comme un concept régulateur, qui ne
vaut que pour la faculté de juger réfléchissante : il s’agit de faire {\it comme} si la
nature poursuivait un but, tout en sachant qu’on ne pourra jamais montrer que
c’est en effet le cas ({\it Critique de la faculté de juger}, II, 2, \S 75 et 76), voire en
pensant qu’il n’en est rien. La téléologie ne débouche sur une théologie que
subjectivement, dit Kant, et pour ceux-là seuls, ajouterai-je, qui en sont dupes.

\section{Téléonomie}
%TÉLÉONOMIE
Une finalité sans finalisme, donc sans causes finales : une
finalité pensée comme effet de causes efficientes (par
exemple, dans le darwinisme, comme effet de l’évolution des espèces et de la
sélection naturelle).

\section{Témérité}
%TÉMÉRITE
Un courage disproportionné face au danger : le téméraire prend
des risque exagérés, pour un enjeu qui ne les justifie pas. C’est
moins un excès de courage qu’un manque de prudence.

\section{Témoignage}
%TÉMOIGNAGE
C’est dire ce qu’on sait ou ce qu’on croit, quand on n’a
aucun moyen de le prouver. Tient lieu de preuve, dans certaines
questions de fait, quand les témoins sont multiples, et quand les preuves
font défaut.

% 577
\section{Tempérament}
%TEMPÉRAMENT
Étymologiquement, c’est un mélange, un équilibre, une
proportion. Se dit en musique du système identifiant
deux notes très voisines (par exemple’un do dièse et un ré bémol) afin de
diviser l’octave, pour l’adapter aux instruments à sons fixes, en douze demi-tons
égaux : c’est en ce sens que Bach nomme l’un de ses chefs-d’œuvre {\it Le
clavier bien tempéré}. Mais le mot désigne surtout un certain type de constitution
individuelle, à la fois physique et psychologique, dont on crut longtemps,
depuis Hippocrate et Galien, qu’il dépendait du mélange, plus ou
moins équilibré ou déséquilibré, de quatre humeurs : la lymphe, le sang, la
bile, l’atrabile (ou bile noire). De là quatre tempéraments traditionnels,
fondé sur la prédominance de l’une des quatre : le lymphatique, le sanguin,
le bilieux, enfin l’atrabilaire ou le mélancolique. Cette classification est
aujourd’hui abandonnée. Mais l’idée de tempérament suit son cours, qui
suppose une typologie des individualités : c’est « un ensemble de traits généraux,
comme dit Lalande, qui caractérisent la constitution physiologique
individuelle d’un être », mais en tant, ajouterai-je, qu’elle a des retombées
psychologiques et qu’elle est susceptible d’entrer dans une certaine classification.
C’est une façon particulière, mais commune à des millions de gens,
d’être son corps. Se distingue par là du caractère, plus individuel, moins physiologique.
Et contribue à le façonner.

\section{Tempérance}
%TEMPÉRANCE
La modération dans les plaisirs sensuels. C’est une exigence
de la prudence, mais aussi de la dignité. L’intempérant est
esclave, et c’est la liberté qui est bonne.

Il ne s’agit pas de ne pas jouir (tempérance n’est pas ascétisme), mais de
jouir mieux {\bf --} ce qui suppose qu’on reste maître de ses désirs. Ainsi le gourmet,
contre le goinfre qu’il porte en lui. L’amateur de vins, contre l’ivrogne.
L'amant, contre le violeur ou le goujat.

La tradition y voit une vertu cardinale. C’est qu’il n’y a pas de vertu sans
maîtrise de soi, ni de maîtrise de soi sans tempérance.

\section{Temporalité}
%TEMPORALITÉ
C’est une dimension de la conscience : sa façon d’habiter
le présent en retenant le passé et en anticipant l'avenir.
Elle n’est pas la vérité du temps, montre Marcel Conche, mais sa négation (elle
fait exister ensemble, comme «unité ek-statique du passé, du présent et du
futur », ce qui ne saurait en vérité coexister). Ce n’est pas le temps réel, mais
notre façon de le vivre ou de l’imaginer.

% 578
\section{Temps}
%TEMPS 
« Le temps, disait Chrysippe, se prend en deux acceptions. » Il est
d’usage de les confondre, et c’est cette confusion, presque toujours,
qu’on appelle le temps.

Le temps, c’est d’abord la durée, mais considérée indépendamment de ce
qui dure, autrement dit abstraitement. Non un être, donc, mais une pensée.
C’est comme la continuation indéfinie et indéterminée d’une inexistence : ce
qui continuerait encore, c’est du moins le sentiment que nous avons, si plus
rien n'existait.

Ce temps abstrait {\bf --} l’{\it aiôn} des stoïciens {\bf --} peut se concevoir, et se conçoit
ordinairement, comme la somme du passé, du présent et de l’avenir. Mais ce
présent n’est alors qu’un instant sans épaisseur, sans durée, sans temps (s’il
durait, il faudrait le diviser en passé et en avenir), et c’est en quoi il n’est rien,
ou presque rien. En ce sens, et comme disait encore Chrysippe, « aucun temps,
n’est rigoureusement présent ». C’est ce qui le distingue de la durée. À le considérer
abstraitement, le temps est constitué essentiellement de passé et d’avenir
(alors qu’on ne peut durer qu’au présent), et pour cela indéfiniment divisible
(ce que le présent n’est jamais) et mesurable (ce que le présent n’est pas davantage).
C’est le temps des savants et des horloges. « Pour déterminer la durée,
écrit Spinoza, nous la comparons à la durée des choses qui ont un mouvement
invariable et déterminé, et cette comparaison s’appelle le temps. » Comparaison
n’est pas raison : le présent, incomparable et indivisible, n’en continue pas
moins.

Quant au temps concret ou réel {\bf --} le {\it chronos} des stoïciens {\bf --}, ce n’est que la
durée de tout, autrement dit la continuation indéfinie de l’univers, qui
demeure toujours le même, comme disait à peu près Spinoza, bien qu’il ne
cesse de changer en une infinité de manières. C’est la seconde acception du
mot : non plus une pensée, mais l'être même de ce qui dure et passe. Non la
somme d’un passé et d’un avenir, mais la perduration du présent. C’est le
temps de la nature ou de l’être : le devenir en train de devenir, le changement
continué des étants. Le passé ? Ce n’est rien de réel, puisque ce n’est plus.
L'avenir ? Ce n’est rien de réel, puisque ce n’est pas encore. Dans la nature, il
n’y a que du présent. C’est ce qu'avait vu Chrysippe (« seul le présent existe »),
et c’est ce que Hegel, à sa façon, confirmera : « La nature, où le temps est le
{\it maintenant}, ne parvient pas à différencier d’une façon durable ces dimensions
du passé et du futur : elles ne sont nécessaires que pour la représentation subjective,
le souvenir, la crainte ou l’espérance » ({\it Précis de l'Encyclopédie}, \S 259).
Comment mieux dire qu’elles ne sont nécessaires que pour l’esprit, point pour
le monde ? Le temps de l’âme n’est qu’une {\it distension}, comme disait saint
Augustin, entre le passé et l’avenir (c’est ce qu’on appelle la temporalité). Le
temps de la nature, qu’une tension ({\it tonos}), qu’un effort ({\it conatus}) ou un acte
% 579
({\it energeia}), dans le présent. Ces deux temps, toutefois, ne sont pas sur le même
plan : l’âme fait partie du monde, comme la mémoire et l'attente font partie du
présent. Le temps, dans sa vérité, est donc celui de la nature : ce n’est qu’un
perpétuel, quoique multiple et changeant, {\it maintenant}. C’est en quoi il ne fait
qu’un avec l'éternité.

Deux sens, donc : une abstraction ou un acte. La durée, abstraction faite de
ce qui dure, ou l’être même, en tant qu’il continue. Une pensée, ou un devenir.
La somme du passé et de l'avenir, qui ne sont rien, ou la continuation du présent,
qui est tout. Un non-être, ou l’être-temps. Ce qui nous sépare de l’éternité,
ou l'éternité même.

\section{Temps perdu}
%TEMPS PERDU
C'est le passé, en tant qu’il n’en reste rien, ou le présent, en
tant qu’il n’est que l'attente de l’avenir. Aussi est-ce le
contraire de l'éternité. Misère de l’homme. Le temps perdu, c’est le temps
même.

\section{Temps retrouvé}
%TEMPS RETROUVÉ
C’est une espèce d’éternité de la mémoire, où le temps
soudain se révèle (« un peu de temps à l’état pur », dit
Proust), dans sa vérité, et par là (en cet instant « affranchi de l’ordre du
temps ») s’abolit. Voilà que le présent et le passé ne font qu’un, ou plutôt, pour
différents qu’ils demeurent (la madeleine dans le thé et la madeleine dans la
tisane sont deux), voilà qu’ils se rencontrent dans un même présent, qui est
celui de l'esprit, qui est celui de l’art, voilà qu’ils libèrent « l’essence permanente
et habituellement cachée des choses », qui est simplement leur vérité, toujours
présente, ou leur éternité. Car la vérité ne passe pas, tout est là, car le temps ne
passe pas (c’est nous, dirait Proust comme Ronsard, qui passons en lui), et cette
contemplation, quoique fugitive, est d’éternité. Le temps retrouvé est ainsi la
même chose que le temps perdu («la vraie vie, la vie enfin découverte et
éclaircie, la seule vie par conséquent réellement vécue. »), et pourtant son
contraire.

\section{Tendance}
%TENDANCE
La direction d’un être ou d’un processus quelconque. Se dit
spécialement de l'orientation du désir, mais en tant qu’elle est
plutôt naturelle (à la différence de l’inclination) et collective (à la différence du
penchant). Disons que c’est la pente naturelle de l’espèce, comme le penchant
est celle de l’individu. L’équivalent à peu près de l’{\it hormè} des stoïciens (la tendance,
la pulsion), qui est comme un {\it conatus} biologique (« l’{\it hormè} fondamentale
% 580
de tout être vivant est de se conserver soi-même » : Chrysippe, cité par
Diogène Laërce, VII, 85).

\section{Tendresse}
%TENDRESSE
La douceur pour ceux qu’on aime, et l’amour de cette douceur.

\section{Terme (grand, moyen ou petit {\bf --})}
%TERME (GRAND, MOYEN OU PETIT -)
On appelle ainsi les trois éléments,
unis deux à deux et
intervenant chacun deux fois, d’un syllogisme. Dans l’exemple canonique,
« mortels » est le grand terme, « Socrate » le petit terme, et « hommes » le moyen
terme. Contrairement à ce que pourrait laisser croire cet exemple, ce n’est pas leur
extension qui les définit (ce ne serait pertinent que dans certains modes du syllogisme).
Le grand terme, qui est le prédicat de la conclusion, figure dans la
majeure ; le petit terme, qui est le sujet de la conclusion, dans la mineure ; enfin
le moyen terme figure dans les deux prémisses, mais pas dans la conclusion.

\section{Terrorisme}
%TERRORISME
Ce n’est pas régner par la terreur, comme fait le despotisme,
mais combattre, par la terreur qu’on suscite, le règne d’un
autre. C’est utiliser la violence à des fins politiques, contre un pouvoir qu’on ne
peut vaincre démocratiquement ou militairement.

Le terrorisme est l’arme des faibles ; c’est ce qui peut parfois le justifier, mais
seulement au service d’une cause juste et contre un adversaire qu'on ne saurait
affronter autrement. Les nazis appelaient « terroristes » nos Résistants, et après
tout pourquoi pas ? Ceux-ci combattaient sans uniforme, ils faisaient exploser des
bombes, qui pouvaient tuer des civils, et plusieurs n'auraient pas hésité, s’ils
l'avaient pu, à semer la terreur à Berlin ou à Vienne. Mais ce qui peut être légitime
contre Hitler et en temps de guerre ne l’est pas contre un État démocratique
et en temps de paix. Cela laisse une marge d’appréciation (où commence la
démocratie ? où finit la paix ?), et c’est pourquoi la dénonciation du terrorisme ne
saurait tenir lieu d’analyse politique. Mais la politique, contre le terrorisme, ne.
suffit pas davantage. Les terroristes sont des combattants de l'ombre, qui ne se
soucient pas des lois de la guerre et qui n’hésitent pas à frapper, le cas échéant, des
civils ou des innocents. C’est une raison suffisante, en tout État démocratique,
pour le refuser et pour le combattre, y compris militairement. Contre le fanatisme,
la raison. Contre la violence aveugle, la force lucide.

\section{{\it Tetrapharmakon}}
%{\it TETRAPHARMAKON}
Mot grec, signifiant littéralement « quadruple remède »,
ou « remède à quatre ingrédients ». En philosophie,
il s’agit en l'occurrence de quatre maximes qu’un épicurien du {\footnotesize II$^\text{e}$} siècle après
% 581
Jésus-Christ {\bf --} Diogène d’'Œnoanda {\bf --} avait fait graver, pour l'édification des passants
et de la postérité, sur un mur, où elles furent en effet redécouvertes au {\footnotesize XIX$^\text{e}$}
siècle. Elles sont souvent citées depuis, à juste titre, comme l’un des meilleurs
résumés de la pensée d’Épicure, telle qu’elle apparaît aussi bien dans la {\it Lettre à
Ménécée} (dont je suis ici l’ordre) que dans les quatre premières {\it Maximes capitales} :

\hspace{4.1cm} « Il n’y a rien à craindre des dieux ;

\hspace{4.1cm} Il n’y a rien à craindre de la mort ;

\hspace{4.1cm} On peut supporter la douleur ;

\hspace{4.1cm} On peut atteindre le bonheur. »

On n’oubliera pas qu’il s’agit d’un remède philosophique, ou de la philosophie
comme remède : l'important n’est pas de le répéter (ce n’est pas un
mantra) mais de le {\it méditer}, comme dit Épicure ({\it Lettre à Ménécée}, \S 135),
autrement dit d’essayer de le comprendre et de le vivre.

\section{Théisme}
%THÉISME
Toute doctrine qui affirme l'existence d’un Dieu personnel,
transcendant et créateur. C’est le contraire de l’athéisme. Le mot
ne faisant référence à aucune religion particulière, on peut dire aussi bien qu’il les
contient toutes (le christianisme ou l’Islam sont deux théismes en ce sens) ou
qu'il ne saurait se réduire à aucune. C’est ce qui explique que le mot, en pratique,
prenne souvent un sens plus déterminé ou plus polémique, qui est une croyance
en Dieu indépendante de quelque religion positive que ce soit, voire les récusant
toutes. C’est la position de Voltaire ou du Vicaire savoyard de Rousseau.

Par différence avec le déisme, le théisme suppose qu’on peut connaître {\bf --}
que ce soit par analogie, par raisonnement ou par révélation {\bf --} au moins
quelques-uns des attributs de Dieu (par exemple qu’il est tout-puissant, omniscient,
créateur, parfaitement bon et juste, aimant et miséricordieux...). Le
déisme n’affirme qu’une existence ; le théisme croit connaître aussi, ou reconnaître,
bien sûr partiellement, une essence. La différence entre les deux reste
pourtant fluctuante et n’interdit pas les degrés intermédiaires. Le déisme est un
théisme vague. Le théisme, un déisme déterminé.

\section{Théiste}
%THÉISTE
Celui qui croit en Dieu, spécialement s’il ne se reconnaît dans
aucune religion établie. Le théiste, explique Voltaire, « n’embrasse
aucune des sectes qui toutes se contredisent ». C’est un croyant sans rites, sans
Église, sans théologie. « Faire le bien, voilà son culte ; être soumis à Dieu, voilà
sa doctrine. Le mahométan lui crie : “Prends garde à toi si tu ne fais pas le pèlerinage
de La Mecque !” “Malheur à toi, lui dit un récollet, si tu ne fais pas un
voyage à Notre-Dame-de-Lorette !” Il rit de Lorette et de La Mecque ; mais il
% 582
secourt l’indigent et il défend l'opprimé » ({\it Dictionnaire}, article « Théiste »).
Sa foi serait donc une morale ? Pas seulement, puisqu’un athée peut se passer
de celle-là sans renoncer à celle-ci. Le théiste ne se contente pas de faire le bien ;
il croit que c’est le Bien qui l’a fait et qui doit au bout du compte le juger. C’est
pourquoi il se soumet à Dieu, comme dit Voltaire. Mais pourquoi faudrait-il se
soumettre à ce qu’on ne comprend pas ?

\section{Théodicée}
%THÉODICÉE
C’est un mot forgé par Leibniz, qui en fit le titre d’un de ses
livres ({\it Essais de théodicée : Sur la bonté de Dieu, la liberté de
l’homme et l'origine du mal}). Il exprime moins la justice de Dieu, malgré l’étymologie
({\it dikè}, en grec, c’est la justice), que sa justification par nous. C’est une
espèce de plaidoirie. Il s’agit de montrer que Dieu est innocent, comme disait
Platon, et que l'existence du mal n’est pas un argument insurmontable contre
son existence et sa bonté. Le livre, quoique moins éblouissant que le {\it Discours de
métaphysique}, est un chef-d'œuvre. Mais c’est un chef-d'œuvre agaçant, par la
volonté de justifier l’injustifiable. À ne pas lire quand on souffre trop. Cela rendrait
injuste avec Leibniz.

\section{Théologales (vertus {\bf --})}
%THÉOLOGALES (VERTUS -)
Ce sont les trois vertus principales de la tradition
chrétienne, qui touchent moins à la
morale qu’à la religion : la foi, l'espérance, la charité. On les dit {\it théologales}
parce qu’elles auraient Dieu même pour objet. On remarquera avec saint Paul,
qui ne les appelle pas ainsi, que la « plus grande des trois », et la seule qui « ne
passera pas », est la charité (I, Co, 13). C’est suggérer ce que saint Augustin et
saint Thomas diront expressément : que la foi et l'espérance n’ont de sens que
provisoire et pour autant seulement que nous sommes séparés du Royaume. Au
paradis, elles seront obsolètes. Il n’y aura plus lieu de croire en Dieu, puisque
nous Le verrons face à face, et plus rien à espérer. D'ailleurs, remarque saint.
Thomas, « il y eut dans le Christ une charité parfaite, et il n’eut cependant ni
la foi ni l'espérance » ({\it Somme théologique}, Ia IIæ, quest. 65, 5). C’est qu’il était
Dieu, bien sûr, et que Dieu n’a pas à croire ni à espérer quoi que ce soit
(puisqu'il est à la fois omniscient et omnipotent). Cela n’en donne pas moins
un sens singulier, et singulièrement fort, à ce qu’un livre fameux appelait, c’est
son titre, « l'imitation de Jésus-Christ ». Comment imiter en lui ce qu’il n’avait
pas ? Cela laisse une chance aux athées. C’est l’amour qui sauve, point la foi,
point l'espérance. C’est ce qu’on peut appeler le Royaume, où rien n’est à
croire, puisque tout est à connaître, où rien n’est à espérer, puisque tout est à
% 583
faire ou à aimer. Nous y sommes : le Royaume est en nous, comme dit Jésus,
ou nous en lui, et il n’y en a pas d’autre.

\section{Théologie}
%THÉOLOGIE
La « science » de Dieu ? Même avec des guillemets, l’expression
serait contradictoire. La théologie est moins une science
qu'une étude : c’est un discours rationnel (un {\it logos}), tenu par des hommes,
mais portant sur le divin. Elle s’appuie le plus souvent, du moins dans les religions
révélées, sur ce que Dieu est censé avoir dit de lui-même. S'il avait été
plus clair, la théologie n’existerait pas ou serait inutile.

On parle de théologie apophatique, ou négative, quand elle ne procède
que par négations : non en disant ce que Dieu est (ce qui serait le ramener à
nos catégories humaines), mais en disant ce qu’il n’est pas. C’est un antidote
contre l’anthropomorphisme. L’athéisme en est un autre, plus simple et plus
efficace.

\section{Théologien}
%THÉOLOGIEN
Celui qui consacre sa vie à l’étude de Dieu, c’est-à-dire, en
pratique, à l'étude de ce que les hommes en ont dit ou, s’il
croit à la Révélation, de ce que Dieu aurait dit de lui-même. Les résultats sont
impressionnants par la masse, par l'intelligence, par l’érudition, parfois par la
profondeur. Toutefois Dieu, après ces dizaines de milliers de pages, n’en reste
pas moins incertain et incompréhensible. « J’ai connu un vrai théologien, rapporte
Voltaire : plus il fut véritablement savant, plus il se défia de tout ce qu'il
savait. [...] À sa mort, il avoua qu’il avait consumé inutilement sa vie. »
Anticléricalisme ? Peut-être. Mais saint Thomas lui-même, tout à la fin de sa
vie, écrivit à son ami Réginald, qui s’enquérait de ses travaux, ceci : « Je ne puis
plus écrire. J'ai vu des choses auprès desquelles mes écrits ne sont que de la
paille. » Ce qu’il avait vu, nul ne le sait. Reste la paille.

\section{Théorème}
%THÉORÈME
Une proposition démontrée, à l’intérieur d’un système hypothético-déductif.
En philosophie, cela n’existe donc pas : on
parlera plutôt de {\it thèses} (voir ce mot).

\section{Théorétique}
%THÉORÉTIQUE
Le mot, calqué du grec, fait presque toujours référence à
son usage aristotélicien. Est {\it théorétique} ce qui relève de la
{\it théôria}, c’est-à-dire de la connaissance pure ou désintéressée. Les {\it sciences théorétiques}
{\bf --} mathématique, physique, théologie {\bf --} sont celles qui se contentent de
% 584
connaître (par différence avec les sciences pratiques ou poiètiques, qui servent
à l’action ou à la production : {\it Métaphysique}, E, 1). L’{\it intellect théorétique} est
celui qui connaît ou contemple, indépendamment de quelque action que ce
soit ; il « ne pense rien qui ait rapport à la pratique, et n’énonce rien sur ce qu’il
faut éviter ou poursuivre » ({\it De anima}, III, 9). La {\it vie théorétique}, qui est le
sommet de la sagesse et du bonheur ({\it Éthique à Nicomaque}, X, 7-8), est la vie
contemplative : c’est à la fois une activité (« l’activité de l’intellect ») et une joie
(« la joie de connaître », X, 7).

\section{{\it Théôria}}
%{\it THÉÔRIA}
Mot grec, signifiant {\it vision} ou {\it contemplation}. Contrairement à ce
qu’on croit parfois, l’étymologie ne renvoie pas à la vision qu’on
aurait de Dieu ({\it théos}), mais simplement à la contemplation ({\it oros}, celui qui
observe) d’un spectacle ({\it théa}, qui donnera notre théâtre). Il n’en reste pas
moins que le spectacle en question est d’abord un oracle ou une fête religieuse,
et que le mot, à son origine, est associé par là, ou peut l'être, à la religion.
Platon nommera ainsi la contemplation des Idées. Mais la fortune du mot doit
surtout à Aristote, qui y verra l’activité propre de l’intellect, donc le sommet du
bonheur et de la vertu: l’activité contemplative, c’est-à-dire la « joie de
connaître », est « le parfait bonheur de l’homme » ({\it Éthique à Nicomaque}, X, 7)
et la seule activité de Dieu ({\it ibid.}, X, 8).

\section{Théoriscisme}
%THÉORICISME
C'est accorder trop de crédit à la pensée théorique ou
abstraite : croire par exemple qu’elle suffira à changer le
monde, la vie, les hommes. C’est l’inverse de l’activisme, et une autre faute.

\section{Théorie}
%THÉORIE
Le mot, en français, s’est éloigné de son étymologie grecque (voir
l'article « {\it Théôria} »). La théorie, pour nous, relève moins de la
contemplation que du travail, moins de la joie de connaître que de l'effort de
penser. Qu’est-ce qu’une théorie ? Un ensemble, en principe cohérent, de concepts
et de propositions, qui vise à produire un effet de connaissance ou à
rendre compte d’au moins une partie du réel. Si les propositions qui la composent
sont des axiomes et des théorèmes, il s’agit d’une théorie hypothético-déductive.
Si ce sont des hypothèses vérifiées ou falsifiables, d’une théorie
inductive ou expérimentale. Elle n’en est pas moins abstraite dans les deux cas.
Cela ne veut bien sûr pas dire qu’elle soit déconnectée de toute pratique. Il faut
au contraire y voir, soulignait Althusser, « une forme spécifique de la pratique,
appartenant elle aussi à l’unité complexe de la “pratique sociale” d’une société
% 585
humaine déterminée ». C’est pourquoi on peut parler de {\it pratique théorique} : « La
pratique théorique rentre sous la définition générale de la pratique [comme processus
de transformation]. Elle travaille sur une matière première (des représentations,
concepts, faits) qui lui est donnée par d’autres pratiques » (qu’elles soient
empiriques, techniques ou idéologiques) et les transforme ({\it Pour Marx}, VI, 1).

\section{Thèse}
%THÈSE
Une proposition indémontrable, mais qui peut faire l’objet d’une
argumentation. Se dit spécialement, dans une démarche dialectique,
d’une proposition considérée comme première, par rapport et par opposition
à une deuxième (l'antithèse), en attendant qu’une troisième (la synthèse)
vienne dépasser leur contradiction.

Le mot, dans son usage philosophique contemporain, doit beaucoup à
Louis Althusser. Une thèse, explique-t-il, n’est pas une proposition scientifique :
elle ne relève pas de la connaissance, mais de la pratique ; elle a moins
un {\it objet} qu’un {\it enjeu} ; elle n’est ni vraie ni fausse (je dirais plutôt : ni démontrable
ni falsifiable), mais elle peut être {\it juste} ou non. Parce qu’elle correspondrait
à la justice ? Non pas ; mais parce qu’elle fait preuve de {\it justesse} (c’est-à-dire
d’un rapport opératoire à la pratique, fût-ce à la pratique théorique) et
peut faire l’objet de « justifications rationnelles » ({\it Philosophie et philosophie
spontanée des savants}, I et II). C’est l'élément de base d’une philosophie : une
position pratique (une thèse doit transformer quelque chose ou produire
quelque effet) dans la théorie.

\section{Tiers exclu (principe du {\bf --})}
%TIERS EXCLU (PRINCIPE DU -)
Il stipule que, de deux propositions contradictoires,
l’une est vraie et l’autre fausse,
nécessairement, ce qui exclut toute autre possibilité. {\it P} ou {\it non-P}. Dieu est un
tuyau d’arrosage ou Dieu n’est pas un tuyau d’arrosage. Cela n’exclut pas qu’il
soit autre chose, mais que cet {\it autre chose} puisse offrir une troisième issue à la
question de savoir s’il est ou non un tuyau d’arrosage. Le principe du tiers-exclu,
contrairement à ce qu’on croit parfois, n’interdit ni la finesse, ni la subtilité,
ni les compromis, ni la complexité. Il interdit la confusion et la bêtise.
Le principe du tiers exclu entraîne que deux propositions contradictoires ne
peuvent être fausses toutes les deux (alors que le principe de non-contradiction
entraîne qu'elles ne peuvent être vraies toutes les deux), de telle sorte que la
fausseté de l’une suffit à prouver la vérité de l’autre. C’est ce qui fonde les raisonnements
par l'absurde, où le vrai brille encore, jusque dans son absence.
{\it Verum index sui}, disait Spinoza, {\it et falsi}. Le faux, miroir du vrai.

% 586
\section{Timidité}
%TIMIDITÉ
C’est une sensibilité exagérée au regard de l’autre, comme une
peur d’être jugé, comme une honte d’être soi, mais sans culpabilité,
et sans autre raison de rougir ou de bafouiller que cette rougeur même ou
cet embarras de la parole. Les uns le ressentent surtout devant une foule ;
d’autres, dont je suis, dans le tête-à-tête. C’est peut-être que les premiers craignent
surtout d’être vus ; les seconds, d’être devinés.

\section{Tolérance}
%TOLÉRANCE
Tolérer, c’est laisser faire ce qu’on pourrait empêcher ou
punir. Cela ne vaut pas comme approbation, ni même
comme neutralité. Ce comportement que je tolère (le sectarisme, la superstition,
la bêtise...), je peux aussi le combattre, en moi ou en autrui. Mais je
m'interdis de l’interdire : je ne le combats que par les idées, point par la loi ou
la force. C’est aimer la liberté plus que son propre camp, le débat plus que la
contrainte, la paix plus que la victoire.

Doit-on tout tolérer ? Bien sûr que non, puisqu'il faudrait pour cela tolérer
l'intolérance, y compris quand elle menace la liberté, et laisser les plus faibles
sans défense : ce serait abandonner le terrain aux fanatiques et aux assassins !

Il y a de l’intolérable : c’est tout ce qui rendrait la tolérance suicidaire ou
coupable.

Tolérance n’est ni laxisme ni faiblesse. Il n’est pas interdit d’interdire, mais
seulement d'interdire ce qui doit être protégé (la liberté de conscience et
d'expression, le libre affrontement des arguments et des idées...) ou ce qu'on
pourrait combattre, sans danger pour la liberté, autrement qu’en l’interdisant.
On dira que cela laisse, en pratique, une marge importante d’appréciation. Cela
même doit être toléré. Dans quelles limites ? Celles de l’État de droit. Pourquoi
interdire un groupuscule totalitaire, tant qu’il n’agite que des idées ou des
imbéciles ? Mais qu’il viole la loi ou verse dans le terrorisme, voilà qui appelle
une sanction immédiate. Le tolérer, ce serait s’en rendre complice.

\section{Topique}
%TOPIQUE
Qui concerne les lieux ({\it topoi}), et notamment les lieux communs
(au sens non péjoratif du terme : c’est en ce sens qu’on parle des
{\it Topiques} d’Aristote, pour désigner l’un des six traités de sa logique ou de son
{\it Organon}, en l'occurrence celui consacré aux lieux communs de l’argumentation
dialectique). Cette acception n’a plus d’usage qu’historique. Au sens moderne
du terme, une {\it topique} est une espèce de schéma, ou plutôt de modèle schématisable,
qui sert à représenter dans l’espace, et comme différents lieux, ce qui
n'existe ou n’est connu que de façon non spatiale. Il est arrivé à Althusser d’utiliser
le mot à propos de Marx (concernant la distinction entre l’infrastructure
% 587
et la superstructure, ainsi que les différents niveaux de l’une et de l’autre),
comme à moi à propos des quatre ordres (voir l’article « Distinction des
ordres ») qui me paraissent structurer toute vie sociale. Mais le mot, dans son
usage ordinaire, est presque toujours d'inspiration psychanalytique. Freud proposa
en effet, à quelque vingt ans d’intervalle, deux modèles différents pour
penser les différents « lieux » de l'appareil psychique. La première topique distingue
le {\it conscient}, le {\it préconscient} et l'{\it inconscient} ; la seconde, qui n’est ni superposable
à la première ni incompatible avec elle, distingue le {\it ça}, le {\it moi} et le
{\it surmoi} (voir ces mots). La première est surtout descriptive ; la seconde, davantage
explicative.

\section{Torture}
%TORTURE
C'est imposer à quelqu'un, volontairement, une souffrance
extrême, parfois par pure cruauté, plus souvent pour en obtenir
quelque aveu ou dénonciation. Comportement spécifiquement humain, qui en
dit long sur notre espèce. On évitera pourtant d’en faire un argument en faveur
de la misanthropie. Ce serait donner raison aux tortionnaires, contre leurs victimes,
et tenir pour rien l’héroïsme de celles, même rares, qui sont mortes sans
parler.

\section{Totalitarisme}
%TOTALITARISME
Pouvoir total (d’un parti ou de l’État) sur le tout (d’une
société) : c’est la forme moderne et bureaucratique de la
tyrannie. Le totalitarisme constitue un système politique où tous les pouvoirs
appartiennent en fait à un même clan, qui impose partout son idéologie, son
organisation, ses hommes. Règne ordinairement au nom du bien et du vrai ;
gouverne par le mensonge et la terreur.

Le mot, qui est apparu dès les années 20, sert surtout à désigner ce que les
dictatures fascistes et communistes pouvaient avoir en commun : un parti de
masse, une idéologie d’État, un contrôle absolu des moyens d’information et de
propagande, la suppression des libertés individuelles, l’absence de toute vraie
séparation entre les pouvoirs, un régime inquisitorial et policier, qui débouche
sur la terreur et culmine dans les camps. Ces traits communs, qui sont incontestables,
ne signifient pas que ces deux régimes soient foncièrement identiques,
pas plus que leurs différences, qui sont tout aussi incontestables, ne sauraient
annuler ces convergences objectives. Nazisme et stalinisme sont-ils
comparables ? Bien sûr, puisqu'ils ont des traits communs, et puisqu’on ne
pourrait répondre non qu’à la condition de les comparer d’abord ! Sont-ils
identiques ? Bien sûr que non, puisqu'ils cesseraient alors d’être deux et de pouvoir
% 588
être comparés ! Moyennant quoi le débat, sur leurs ressemblances et différences,
peut durer toujours.

Les convergences sont surtout objectives et organisationnelles ; les différences,
surtout subjectives et idéologiques. Les deux systèmes s’imposent à peu
près de la même façon, mais au nom d’idéologies opposées. Non, certes, parce
que l’un aurait fait le mal pour le mal, comme le croient les naïfs, quand l’autre
ne l’aurait fait qu’au nom du bien, voire par accident ou par erreur. Le nazisme,
pour un nazi, est un bien, et l’enfer totalitaire, qu’il soit de droite ou de
gauche, n’est pavé, comme dit Todorov, que de bonnes intentions. Si ces deux
systèmes s'opposent, ce n’est pas comme le bien et le mal, mais comme deux
conceptions opposées du bien, qui débouchent sur deux maux parallèles. L’un
veut imposer le pouvoir d’une race ou d’un peuple, sur d’autres races ou sur
d’autres peuples. L'autre, le pouvoir d’une classe sur d’autres classes, mais
pour les abolir toutes : pour qu’il n’y ait plus que humanité heureuse et libre.
La pensée du premier est essentiellement biologique, hiérarchique, guerrière ;
celle du second, essentiellement historique, égalitaire, universaliste. C’est ce
qui rend le communisme plus sympathique, plus trompeur (l'écart entre son
discours et ses actes est plus grand), et plus dangereux peut-être. Du moins
c’est ce qui peut sembler, une fois que le nazisme a été vaincu par la force. Le
communisme ne le sera que par la fatigue, l'impuissance et le ridicule {\bf --} que
par lui-même.

\section{Totalité}
%TOTALITÉ
Tous les éléments d’un ensemble, mais en tant que celui-ci constitue
une unité. La totalité, qui est chez Kant l’une des trois catégories
de la quantité, est ainsi définie par lui à partir des deux autres, dont elle
opère la conjonction : une {\it totalité}, c’est l'{\it unité} d’une {\it pluralité}. C'est ce qui
autorise à parler de {\it la} totalité comme d’un absolu : ce serait l’unité de toutes les
pluralités (l’ensemble de tous les ensembles : la {\it summa summarum} de Lucrèce).
Mais nous n’en avons d’expérience, par définition, que partielle.

\section{Tout}
%TOUT
Le sens du substantif change en fonction de l’article. {\it Un} tout, c’est
un ensemble unifié ou ordonné. {\it Le} Tout, surtout quand on l'écrit
avec une majuscule, c’est l’ensemble de tous les ensembles. On pourra dire par
exemple qu’un monde est un tout, qu’il peut en exister une infinité, dont
l’ensemble serait le Tout. En ce dernier sens, qui correspond au {\it to pan} d’Épicure
ou à la {\it summa summarum} de Lucrèce (la somme des sommes), c’est un
synonyme d’univers, au sens philosophique du terme. L'idée qu’il en existe plusieurs
serait contradictoire.

% 589
\section{Tragique}
%TRAGIQUE
Ce n’est pas le malheur ou le drame. Ce n’est pas la catastrophe
{\bf --} ou bien c’est la catastrophe humaine, celle d’être soi, celle de
se savoir mortel. Le tragique, c’est tout ce qui résiste à la réconciliation, aux
bons sentiments, à l’optimisme béat ou bélant. C’est la contradiction insoluble,
mais existentielle plutôt que logique (par exemple entre notre finitude et notre
désir d’infini). C’est une espèce de dialectique, si l’on veut, mais sans synthèse
ni dépassement {\bf --} une dialectique sans pardon. C’est le divorce, mais sans
réconciliation. C’est le conflit sans issue, en tout cas sans issue satisfaisante,
entre deux points de vue l’un et l’autre légitimes, du moins dans leur ordre, et
qui n’en sont que davantage opposés. Par exemple le conflit entre les lois de
l’État et celles de la conscience (Antigone), entre le destin et la volonté
(Œdipe), entre les dieux et les hommes (Prométhée), entre la passion et le
devoir (spécialement chez Corneille), ou entre deux passions (spécialement
chez Racine). Je ne prends ces exemples littéraires que par commodité. La
tragédie, comme genre littéraire, se nourrit du tragique ; elle ne l’épuise pas.
C’est qu’il est une dimension de la condition humaine et de l’histoire : l’incapacité
où nous sommes de trouver, même intellectuellement, une solution pleinement
satisfaisante au problème que constitue, au moins à nos yeux, notre
existence. C’est pourquoi la mort est tragique. C’est pourquoi la vie est tragique.
Parce qu'elles nous confrontent toutes deux à l'impossible ou à
l'absurde, à l’inacceptable, à l’inconsolable. Parce que toute vie est l’histoire
d’un échec, comme dit Sartre, et parce que la mort en est un autre. Parce que
toute vie est un combat, mais sans victoire ni repos.

Cela vaut aussi pour les peuples. « Le tragique, disait un homme politique,
c’est quand tout le monde a raison : la situation au Proche-Orient est
tragique. » Les Palestiniens ont raison de vouloir vivre chez eux, de vouloir
leur indépendance, leur souveraineté, leur État. Les Israéliens ont raison de
vouloir leur sécurité. Mais on ne voit pas comment ces deux légitimités pourraient
n’en faire qu’une, ni même cohabiter sans renoncer l’une et l’autre à
une part {\bf --} au moins une part {\bf --} de leur bon droit. Il faudra donc des
compromis : il faudra accepter une solution qui ne sera pleinement satisfaisante
pour personne, qui ne sera pas juste, mais qui vaudra mieux pourtant
que la guerre et le terrorisme. Ce sera sortir de la tragédie, non du tragique.
De la guerre, non du conflit. De la haine, non de l’amertume. Le tragique est
le goût même du réel {\bf --} parce qu’il ne nous obéit pas, parce qu’il n’est jamais
tout à fait à notre goût.

On remarquera que la catastrophe nazie n’était pas {\it tragique}, en ce sens.
Elle ne pourrait sembler l’être qu’à ceux qui jugeraient le nazisme légitime,
ou nieraient qu'une issue parfaitement satisfaisante était envisageable, au
moins en théorie, au moins en droit, qui était la défaite, le plus tôt possible,
% 590
du nazisme. Mais que la raison et le droit n’aient pu y suffire, cela confirme
ce qu’il y a d’irréductiblement tragique dans l’histoire humaine : qu’il ne
suffit pas d’avoir raison pour vaincre, que le droit ne peut rien sans la force,
qu’on ne peut combattre le mal, presque toujours, que par un autre mal (la
violence, la guerre, la répression), certes moindre, mais qui ne saurait non
plus tout à fait nous satisfaire. Hitler n’est pas un personnage tragique. Ni
ceux, s’il en existe, qui n’opposèrent au nazisme que leurs bons sentiments.
Mais Churchill, si. Mais Cavaillès, si. Le tragique, ce n’est pas le conflit entre
le bien et le mal ; c’est le conflit entre deux biens, ou entre deux maux.

On parle de {\it philosophie tragique} lorsqu’une pensée, loin de vouloir nous
satisfaire totalement, nous confronte à de l’inacceptable, à de l’injustifiable, à ce
que Clément Rosset appelle la {\it logique du pire} (contre la logique du meilleur de
Leibniz) et nous voue par là à l’insatisfaction ou au combat. Ainsi Pascal ou
Nietzsche. La frontière peut traverser une même école, voire un même individu.
Par exemple Lucrèce est un penseur tragique ; Épicure, non. Marc Aurèle
est un penseur tragique ; Épictète, non. Spinoza et Kant le sont parfois, point
toujours. À chacun de trouver ses maîtres, en fonction du tragique qu’il accepte
ou requiert.

Le grand théoricien du tragique, et l’un de ses plus illustres représentants,
est évidemment Nietzsche. Qu'est-ce que le tragique ? C’est la vie telle qu’elle
est, sans justification, sans providence, sans pardon, c’est la volonté de
l’affirmer toute, de l’accepter toute, avec la souffrance dedans, avec la joie
dedans, sans ressentiment, sans mauvaise conscience, sans nihilisme, c’est
l'amour du destin ou du hasard, du devenir et de la destruction, c’est « {\it le oui
par excellence} », sans religion, sans « moraline », c’est le sentiment que le réel est
à prendre ou à laisser, joint à la volonté joyeuse de le prendre. « L'artiste tragique
n’est pas un pessimiste, il dit {\it oui} à tout ce qui est problématique et terrible,
il est {\it dionysien} » ({\it Le crépuscule des idoles}, III, 6). C’est qu’il aime la vie
comme elle est, comme elle vient, comme elle passe ({\it amor fati}). C’est qu’il n’a
pas besoin d’autre chose, ni espérance ni consolation. C’est qu’il n’a même pas
besoin d’y croire tout à fait. Le contraire du tragique, ce n’est pas le comique,
c’est l’esprit de sérieux.

\section{Transcendance}
%TRANSCENDANCE
C'est l’extériorité et la supériorité absolues : l’ailleurs
de tous les ici (et même de tous les ailleurs), et leur
dépassement. L'absence suprême, donc, qui serait aussi le comble de la présence
{\bf --} le point de fuite du sens. Car « le sens du monde, écrit Wittgenstein,
doit se trouver en dehors du monde ». La transcendance est ce {\it dehors} ou le suppose.
C’est le Royaume absent, qui nous voue à l’exil.

% 591
Ce sens premier ou général est susceptible de variations. Est transcendant
tout ce qui se trouve {\it au-delà de}. Mais au-delà de quoi ? Au-delà de la conscience
(c’est le sens phénoménologique : l'arbre que je perçois n’est pas {\it dans} la
conscience ; il est un objet transcendant {\it pour} la conscience) ; au-delà de l’expérience
possible (c’est le sens kantien) ; au-delà du monde ou de tout (c’est le
sens classique).

Peut désigner aussi le mouvement qui y mène, comme un dépassement,
mais réservé au {\it Dasein}, de tout donné ou de toute limite. La liberté, spécialement,
serait ce pouvoir de {\it transcender} toute situation, tout conditionnement,
tout déterminisme. Ce serait une façon d’être extérieur à sa propre histoire, à
son propre corps, à sa propre situation, ou de pouvoir en sortir. Il y a du
miracle dans la transcendance, dès lors qu’elle prétend s’expérimenter de l’intérieur
ou ici-bas.

\section{Transcendant}
%TRANSCENDANT
Au sens classique : ce qui est extérieur et supérieur au
monde. Dieu, en ce sens, serait transcendant, et lui seul
peut-être.

Chez Kant, ce qui est extérieur à l'expérience, et hors de sa portée.

Chez Husserl et les phénoménologues, ce qui est extérieur à la conscience,
vers quoi elle se projette ou « s'éclate ». C’est en ce sens que Sartre parle d’une
{\it transcendance de l'ego} : le Moi ne fait pas partie de la conscience, il n’est que
l’un de ses objets ; il n’est pas {\it de} la conscience, mais {\it pour} la conscience ; il n’est
pas « dans la conscience », mais « dehors, dans le monde ».

Dans la philosophie contemporaine, on parle aussi de {\it transcendance} pour
désigner tout ce qui est irréductible à la matière, à la nature ou à l’histoire. Pour
Luc Ferry, par exemple, l'homme est un être transcendant, non parce qu’il
serait extérieur au monde ou à la société, mais parce qu’il ne saurait totalement
s’y réduire : il y a en lui une capacité d’excès, d’arrachement, de liberté absolue,
et cette « transcendance dans l’immanence » (l'expression est de Husserl) en fait
une espèce de Dieu, dont l’humanisme serait la religion.

Pour le matérialiste, au contraire, rien n’est transcendant : il n’y a que la
nature, il n’y a que l’histoire, il n’y a que tout, dont l’homme fait partie et ne
saurait se libérer totalement.

\section{Transcendatal}
%TRANSCENDANTAL
Ce n’est pas un synonyme de transcendant.

Le mot est d’origine scolastique : il désigne les attributs
qui {\it transcendent} les genres de l’être ou les catégories d’Aristote, c’est-à-dire
% 592
qui les dépassent et peuvent pour cela convenir à tout être : l’Un, le Vrai, le
Bien et l’Être lui-même sont des transcendantaux.

Mais aujourd’hui, et depuis deux siècles, le mot est presque toujours pris en
un sens kantien : est {\it transcendantal} tout ce qui concerne les conditions a priori de
l'expérience, ainsi que les connaissances qui, prétendument, en découlent. C’est
l’inempirique de l'empiricité. «J'appelle {\it transcendantale}, écrit Kant, toute
connaissance qui, en général, s’occupe moins des objets que de nos concepts {\it a
priori} des objets », ou, précise la seconde édition, que « de notre manière de les
connaître, en tant que ce mode de connaissance doit être possible {\it a priori} »
({\it C. R. Pure}, Introd., VII). Rien à voir donc avec le transcendant, auquel le transcendantal
s’opposerait plutôt. Est transcendant, en effet, tout ce qui est {\it au-delà}
de l’expérience ; transcendantal, tout ce qui est ex deçà, et qui la permet. « Le mot
{\it transcendantal}, insiste Kant, ne signifie pas quelque chose qui s’élève au-dessus de
toute expérience, mais ce qui certes la précède ({\it a priori}) sans être destiné cependant
à autre chose qu’à rendre possible uniquement une connaissance empirique.
Si ces concepts dépassent l’expérience, leur usage se nomme transcendant, et il est
distingué de l’usage immanent, c’est-à-dire borné à l’expérience » ({\it Prolégomènes...},
Appendice). Transcendant s'oppose à immanent, mais de l’extérieur :
ce qui est transcendant n’est pas immanent, ce qui est immanent n’est pas transcendant.
Transcendantal s'oppose à empirique, mais de l’intérieur. Dans une
connaissance empirique, dès lors qu’elle est nécessaire et universelle, il y a forcément
du transcendantal, c’est-à-dire de l’inempirique. « Que toute connaissance
commence {\it avec} l'expérience, cela n’entraîne pas qu’elle vienne toute {\it de}
l'expérience » ({\it C. R. Pure}, Introd., I). Par exemple quand je dis que 7 + 5 = 12 ou
que tout fait a une cause. En tant que ces connaissances sont universelles et nécessaires,
elles ne sauraient dériver tout entières de l’expérience : il faut qu’elles aient
une source {\it a priori} (l'entendement, ses catégories, ses principes). Même chose
pour l’espace et le temps : ce sont des formes {\it a priori} de la sensibilité, qui rendent
l'expérience possible et ne sauraient pour cela en résulter. Leur {\it idéalité transcendantale}
(le fait qu’ils n’existent qu’à titre de conditions subjectives de l'intuition.
sensible) est le gage de leur {\it réalité empirique} (pour tout objet d’une expérience
possible), mais n’en relève pas : les conditions {\it a priori} de l'expérience ne sauraient
par définition être objets d’expérience. C’est en quoi le transcendantal fait
comme une transcendance paradoxale, à l’intérieur même de l’immanence {\bf --} d’où
certains voudront tirer une religion de l’homme ou de l'esprit (c’est ce que Luc
Ferry, reprenant une expression de Husserl, appelle «la transcendance dans
l’immanence »). Mais cela Kant, à ma connaissance, ne l’a jamais fait. Le vrai
Dieu pour lui est transcendant, ce qui exclut qu’on puisse le connaître, certes,
mais aussi que le transcendantal soit Dieu.

% 593
\section{Transfert}
%TRANSFERT
C'est un déplacement : un changement de lieu ou d’objet. Se
dit spécialement, en psychanalyse, du report d’un certain
nombre d’affects inconscients (désir ou rejet, amour ou haine) sur une autre
personne que celle qui les suscita à l’origine, spécialement pendant la petite
enfance. Ce processus, qui est omniprésent (« il s’établit spontanément, écrit
Freud, dans toutes les relations humaines »), agit pourtant de façon plus spectaculaire
durant la cure analytique : le patient déverse sur son analyste « un trop
plein d’excitations affectueuses, souvent mélées d’hostilité, qui n’ont leur
source ou leur raison d’être dans aucune expérience réelle ; la façon dont elles
apparaissent, et leurs particularités, montrent qu’elles dérivent d’anciens désirs
du malade devenus inconscients » ({\it Cinq leçons...}, V). De là une espèce d’accélération
du travail psychique, dans lequel le psychanalyste, grâce au transfert,
joue le rôle d’un « ferment catalytique, qui attire temporairement sur lui les
affects qui viennent d’être libérés » ({\it ibid.}). C’est dire que les patients accordent
presque toujours trop d'intérêt à leur analyste, jusqu’à en être, quand ils en parlent,
quelque peu ridicules ou fatigants. Mais cette survalorisation fait partie de
la cure (ils lui accordent trop d'importance, mais cela même lui en donne),
comme le retour à la lucidité, quand il advient, fait partie de la guérison.

\section{Transsubstantiation}
%TRANSSUBSTANTIATION
C'est la première fois, sauf oubli de ma part,
que j'écris ce mot. Que Voltaire ait jugé bon
de le mettre dans son Dictionnaire en dit long sur les enjeux de son époque, et
sur ce qui nous en sépare. {\it Sic transit gloria Dei...}

Qu'est-ce que la transsubstantiation ? La transformation d’une substance
en une autre, et spécialement la transformation du pain et du vin, lors de
l'eucharistie, en corps et sang du Christ. Croyance absurde ? C’est ce que pensait
Voltaire, qui s’étonnait de « ce vin changé en sang, et qui a le goût du vin,
de ce pain changé en chair, et qui a le goût du pain », enfin de ces croyants qui
« mangent et boivent leur dieu, qui chient et pissent leur dieu »... Étonnement
compréhensible, devant l’incompréhensible. Mais que serait une religion,
répondrait Pascal, qui n’étonnerait pas et qu’on pourrait comprendre ? C’est
qu'il s’agit moins d’une absurdité, pour les catholiques, que d’un mystère ou
d’un miracle. Et après tout, pourquoi pas ? Si Dieu a pu créer le monde, on
aurait tort de le chipoter sur les détails. On s’étonne que Voltaire, qui trouve la
création si plausible, soit tellement choqué par la présence réelle du Christ dans
le pain et le vin. Créer quelque chose à partir de rien, cela fait, me semble-t-il,
une transsubstantiation autrement étonnante.

% 594
\section{Travail}
%TRAVAIL
C’est une activité fatigante ou ennuyeuse, qu’on fait en vue d’autre
chose. Qu’on puisse l’aimer ou y trouver du plaisir, c’est entendu.
Mais ce n’est un travail, non un jeu, que parce qu’il ne vaut pas par lui-même,
ni pour le seul plaisir qu’on y trouve, mais en fonction d’un résultat qu’on en
attend (un salaire, une œuvre, un progrès...) et qui justifie les efforts qu’on lui
consacre. Ce n’est pas une fin en soi : ce n’est qu’un moyen, qui ne vaut qu’au
service d’autre chose. C’est ce que prouvent les vacances et le salaire.
Travailler ? Il le faut bien ? Mais qui le ferait gratuitement ? Qui ne désire le
repos, les loisirs, la liberté ? Le travail, pris en lui-même, ne vaut rien. C’est
pourquoi on le paie. Ce n’est pas une valeur. C’est pourquoi il a un prix.

Une valeur, c’est ce qui vaut par soi. Ainsi l’amour, la générosité, la justice,
la liberté... Pour aimer, vous demandez combien ? Ce ne serait plus amour
mais prostitution. Pour être généreux, juste, libre, il faut qu’on vous paie ? Ce
ne serait plus générosité mais égoïsme, plus justice mais commerce, plus liberté
mais esclavage. Pour travailler ? Vous demandez quelque chose, vous avez évidemment
raison, et vous pestez, bien souvent, de ne pas obtenir davantage.

Une valeur, c’est ce qui n’est pas à vendre. Comment aurait-elle un prix ?
C’est une fin, pas un moyen. À quoi bon aimer ? À quoi bon être généreux,
juste, libre ? Il n’y a pas de réponse. Il ne peut y en avoir. À quoi bon travailler ?
Il y a une réponse, ou plutôt il y en a plusieurs excellentes : pour gagner sa vie,
pour être utile, pour s'occuper, pour s’épanouir, pour s'intégrer dans la société,
pour montrer de quoi on est capable... Même le bénévole n’y échappe pas. S’il
travaille, c’est pour autre chose que le travail (pour le plaisir, pour le groupe,
pour une certaine idée de l'humanité ou de soi...). Cela met le travail à sa
place, qui n’est pas la première.

On m'objectera le chômage de longue durée, et certes je ne conteste pas
qu'il y ait là une tragédie. Mais point du tout, comme on le dit parfois, parce
que le chômeur y perdrait sa dignité. Où avez-vous vu que la dignité d’un
homme dépende de son travail ? Et pourquoi, si tel était Le cas, ne pas plaindre
aussi le milliardaire, qui n’a plus besoin, le pauvre, de travailler ? Mais il n’en
est rien. Si tous les hommes sont égaux en droits et en dignité, comme nous le
voulons, comme nous avons raison de le vouloir, il est exclu que la dignité des
uns et des autres soit proportionnée à la quantité de travail, bien sûr inégale,
qu’ils fournissent. Si le chômage est un malheur, et c’en est évidemment un, ce
n'est pas par l'absence de travail. C’est par l’absence d’argent, c’est par la
misère, c’est par l'isolement ou l'exclusion. Mieux vaut être rentier que smicard,
et cela en dit long sur le travail.

Aristote, dans son génial bon sens, a dit l’essentiel : « Le travail tend au
repos, et non pas le repos au travail. » Il n’est pas vrai qu’on se repose le week-end
pour pouvoir travailler toute la semaine, ni qu’on prenne des vacances,
% 595
comme le voudraient les patrons, pour mieux travailler toute l’année. C’est
l'inverse. On travaille pour gagner sa vie et son repos, pour pouvoir profiter de
ses soirées, de ses week-ends, de ses vacances, bref on travaille pour vivre, alors
qu’il serait fou de vivre pour travailler !

« Je n’ai vu personne, me disait une infirmière, qui regrette, sur son lit de
mort, de ne pas avoir travaillé une heure de plus. » Mais de ne pas avoir assez
vu ses enfants, mais de ne pas avoir assez vécu, réfléchi, aimé, combien sont-ils,
sur leur lit de mort, à le regretter amèrement ?

On parle de « salle de travail », dans nos maternités. C’est que le mot a
d’abord désigné une souffrance, un tourment, une peine.

Dans la Bible, le travail est un châtiment, et telle est aussi l’étymologie du
mot (le {\it trepalium}, d’où vient {\it travail}, était un instrument de torture) comme
encore son sens chez Montaigne. C’est moins vrai aujourd’hui. C’est l’un des
progrès que nous devons au machinisme et aux luttes syndicales. Ce n’est pas
une raison, tordant le bâton dans l’autre sens, pour faire du travail une récompense
ou une valeur. Ce n’est qu’un moyen, j'y insiste, qui ne vaut qu’à proportion
du résultat qu’il obtient ou vise. De l’argent ? Pas toujours. Pas seulement.
Le travail est apparu bien avant la monnaie. Et combien de travaux non
rémunérés ? L’humanité doit d’abord produire les moyens de sa propre existence,
comme disait Marx, ce qui ne va pas sans transformation de la nature et
de soi {\bf --} sans travail. « En même temps qu'il agit par ce travail sur la nature extérieure
et la modifie, soulignait Marx, l’homme modifie sa propre nature et les
facultés qui y sommeillent » ({\it Le Capital}, I, chap. 7). C’est humaniser l’homme en
humanisant le monde. Mais c’est l'humanité qui vaut, non le travail. Aussi le travail
devient-il inhumain, ou déshumanisant, quand le moyen qu’il est tend à
l'emporter sur la fin qu’il vise, ou doit viser. C’est ce que Marx appelle
l'aliénation : quand le travailleur se nie dans son travail, au lieu de s’y réaliser.

\section{Tristesse}
%TRISTESSE
L'un des affects fondamentaux : le contraire de la joie, aussi difficile
qu’elle à définir. C’est comme une souffrance, mais qui
serait de l’âme. Comme une déperdition d’être, de puissance, de vitalité. Comme
une fatigue, mais qu'aucun repos ne suffirait à abolir. « La tristesse est le passage
de l’homme d’une plus grande à une moindre perfection », écrit Spinoza, autrement
dit une diminution de sa puissance d’agir ({\it Éthique}, III, déf. 3 des affects).
C'est exister moins, et le sentir, et en souffrir. Se distingue toutefois du malheur
par l’inconstance ou la mobilité : la tristesse est moins un état qu’un {\it passage}, en
effet ; le malheur, moins un passage qu’un état. Les tristesses vont et viennent,
comme les joies ; le malheur est ce qui reste, quand toute joie semble impossible.
Le malheur est une tristesse qui s’installe. La tristesse, un malheur qui passe.

% 596
\section{Troisième homme (argument du {\bf --})}
%TROISIÈME HOMME (ARGUMENT DU -)
C’est un argument d’Aristote,
contre Platon, ou de
Platon, déja, contre lui-même. Il apparaît dès le {\it Parménide} (132 a-b). Une idée
est ce qu’il y a de commun entre plusieurs individus (par exemple la grandeur,
entre plusieurs objets grands). Mais si elle existe en elle-même (le Grand en soi),
elle est à son tour un être individuel ; il faut dès lors, pour penser le rapport entre
les objets grands et le Grand en soi, quelque chose de commun, qui ferait une
troisième entité. Puis, pour assurer l’unité entre cette troisième et les deux autres,
une quatrième, et ainsi à l'infini : l'unité ne cesse de fuir et les Idées vont se multiplier
à l'infini. Comment Platon se sortait-il de l’objection ? Sans doute par
l’unicité de chaque Idée, mais postulée plutôt que démontrée (voir {\it République}, X,
597 c). Il en fallait plus pour impressionner le Stagirite, qui reprend l’argument
dans sa {\it Métaphysique} (A 9, Z 7, M 4). Si l’on se donne un Homme en soi pour
penser ce qui est commun aux différents hommes, on aura besoin d’un {\it troisième
homme} pour penser l’unité entre les hommes et l'Homme en soi, puis d’un quatrième
pour penser l’unité entre ces trois types d’êtres, et ainsi à l'infini : prêter à
l’idée d’homme une existence séparée (Homme intelligible, l'Homme en soi), ce
n’est pas se donner les moyens de penser l’unité des hommes sensibles (celle du
genre humain) ; c’est au contraire la perdre dans une multiplication indéfinie
d’abstractions hypostasiées. Il faut donc renoncer à substantialiser l’universel, ce
qui revient à rompre avec le platonisme. « Je suis l'ami de Platon, disait Aristote,
mais plus encore de la vérité » (voir {\it Éthique à Nicomaque}, L, 4, 1096 a).

\section{Trope}
%TROPE
Une figure de style ou de logique : c’est jouer avec les mots ou les
idées.

Toute figure de style est-elle un trope ? Non pas. Le trope joue avec le sens
des mots plutôt qu'avec leur place ou leur arrangement: c’est une figure
sémantique. Par exemple la métaphore et la métonymie sont des tropes ; le
chiasme et l’accumulation, non.

En philosophie, le mot est pris en un sens logique plutôt que rhétorique. Le
trope est un type d’argument (il joue avec les idées, non avec les mots), comme
une figure de la pensée, comme un raisonnement prédécoupé ou en kit. Le mot
peut désigner, par exemple, tel ou tel mode ou figure du syllogisme. Mais les
tropes les plus célèbres sont ceux d’Énésidème et d’Agrippa, qui relèvent de la
tradition sceptique et tendent à imposer la suspension du jugement. C’est une
espèce de machine de guerre contre tout dogmatisme. On en trouve la liste
chez Diogène Laërce ({\it Vies et doctrines}, IX) et Sextus Empiricus ({\it Hypotyposes
pyrrhoniennes}, I). Ceux d’Énésidème sont au nombre de dix, qui visent à montrer
que toutes nos représentations sont relatives : elles varient en fonction du
% 597
sujet qui perçoit (homme ou animal, tel homme ou tel autre, avec tel organe
sensoriel ou tel autre, etc.), mais aussi en fonction des circonstances (position,
distance, mélange, quantité...), des relations, des fréquences et des modes de
vie. Ceux d’Agrippa, plus ramassés, plus frappants, sont au nombre de cinq : il
y a le trope de la discordance (les opinions s'opposent, chez les philosophes
comme chez les profanes : pourquoi privilégier l’une d’entre elles ?), celui de la
régression à l'infini (toute preuve devant être prouvée par une autre, et ainsi à
l'infini, on n’en aura jamais fini de prouver quoi que ce soit : tout reste donc
douteux), celui de la relation (rien ne peut être appréhendé en soi : toute représentation
est relative au sujet et aux circonstances), celui des principes (qui ne
sont que des hypothèses indémontrables : puisqu'il faut en poser pour démontrer
quoi que ce soit, toute démonstration reste ainsi incertaine), enfin celui du
diallèle ou cercle vicieux (qui prétend démontrer une proposition à partir d’une
autre qui en dépend : les deux sont donc sans valeur). Ces cinq tropes, comme
les dix d’Énésidème, aboutissent à la suspension du jugement : le mieux,
puisqu'on ne peut trancher, est de s’interdire toute assertion dogmatique.

\section{Truisme}
%TRUISME
Une vérité évidente et sans portée. À ne pas confondre avec la
tautologie, qui n’est pas toujours évidente et rarement sans portée.

\section{Tyrannie}
%TYRANNIE
C'est exercer le pouvoir au-delà de son domaine légitime
(Locke), et spécialement dans un ordre où l’on n’a aucun titre
légitime à le faire (Pascal) : ainsi le roi qui veut être aimé ou cru, quand il ne
mérite, en tant que roi, que d’être obéi ({\it Pensées}, 58-332). C’est vouloir régner
sur les esprits par la force (barbarie), ou sur la force par l'esprit (angélisme), et
c’est en quoi toute tyrannie, même effrayante, est ridicule (voir ce mot) : c’est
le ridicule au pouvoir, ou la confusion des ordres érigée en système de gouvernement.
La faute du tyran est de « vouloir régner partout », ce que nul ne peut,
et « hors de son ordre », ce que nul ne doit ({\it ibid.}).

En un sens plus général, on désigne par tyrannie le pouvoir absolu d’un
seul, quand il est illégitime, violent ou arbitraire. Le mot, dans son usage
moderne, vaut toujours comme condamnation. C’est que nous y voyons, à
juste titre, le contraire de l’État de droit: « Là où le droit finit, la tyrannie
commence » (Locke, {\it Deuxième traité du gouvernement civil}, chap. XVII).

% 598
\section{Uniquité}
%UBIQUITÉ La faculté d’être présent partout à la fois. Ce serait le propre de
Dieu, s’il existe. Mais alors, pourquoi dit-on qu’il est aux
cieux ? C’est qu’il n’est présent ici-bas, répondrait Simone Weil, que sur le
mode de l'absence ou du retrait (il n’est là qu’en tant qu’il n’y est pas). C’est un
Dieu caché : son ubiquité nous en apprend moins sur lui que sa transcendance.

\section{Un}
%UN
Le premier élément d’une énumération (le zéro, qui fut inventé beaucoup
plus tard, sert moins à nombrer qu’à calculer). Peut désigner à ce
titre aussi bien l’{\it unité} (l’un des éléments d’une pluralité possible : un parmi
d’autres) que l’{\it unicité} (quand il n’y a pas de pluralité : un seul). Ces deux
sens sont d’autant moins incompatibles que le second suppose le premier.
Par exemple si j'entends, me réveillant la nuit, une horloge sonner un coup :
rien, lorsque ce coup retentit, ne me permet de savoir s’il est une heure ou
plus : seuls me l’apprendront d’autres coups ou le silence. L’unicité n’est
qu’une unité sans suite ; la pluralité, qu’un ensemble d’unités. C’est donc
l'unité qui est première, dont l’unicité et la pluralité ne sont que des occurrences.
Cela semble donner raison à Parménide ou à Plotin. Mais non, pourtant,
puisque rien ne prouve que cette unité première soit unique (cela peut
donner raison tout autant à Démocrite : les atomes sont des unités en
nombre infini), ni même qu’elle {\it soit} première : elle ne l’est que pour la
pensée ; pourquoi la matière, qui ne pense pas, devrait-elle s’y soumettre ? Il
n’est pas impossible qu’il n’y ait d’abord qu’une multiplicité indéfinie, sans
unités, sans êtres, sans substances : qu’il n’y ait que des flux et des processus.
L'Un, alors, serait moins leur principe que leur ensemble : c’est ce que nous
appelons l’univers.

% 599
\section{Unicité}
%UNICITÉ
Le fait d’être unique. On peut accorder à Leibniz que c’est le
propre de tout être (principe des indiscernables), mais inégalement :
deux feuilles d’un même arbre, quoique différentes l’une de l’autre, sont
pourtant moins uniques qu'un être qui ne ressemble à aucun autre ni n’entre,
comme élément, dans aucune multiplicité. Dieu ou le Tout sont plus uniques,
en ce sens, que ce qu'ils créent ou contiennent, et eux seuls, peut-être, le sont
absolument.

\section{Union}
%UNION
Le devenir-un d’une multiplicité, qui reste pourtant hétérogène ou
« plurielle », comme on dit aujourd’hui. C’est ce qui distingue
l'{\it union} (le fait d’être unis) de l’{\it unité} (le fait d’être un). Les militants progressistes,
dans ma jeunesse, distinguaient traditionnellement l’union de la gauche
(qui supposait une alliance de classes et de partis) et l’unité de la classe ouvrière
(qui supposait, ou aurait supposé, qu’on revienne en amont de la scission de
1920). Je ne sais si les militants d’aujourd’hui font encore ces distinctions. Philosophiquement,
on retiendra surtout que l’union est un processus ou un
combat ; l’unité, un état ou un idéal.

\section{Unité}
%UNITÉ
Le fait d’être un. À ne pas confondre avec l’unicité (le fait d’être un
seul), ni avec l’union (le fait d’être unis), qui la supposent l’une et
l’autre. Il n’y aurait pas {\it un seul}, ou on ne pourrait le penser, s’il n’y avait
d’abord {\it un}. Mais il n’y aurait pas davantage pluralité, ni donc union : il n’y
aurait pas {\it plusieurs} s’il n’y avait pas d’abord {\it un}, et encore {\it un}, et encore {\it un}.
C’est pourquoi on peut compter sur ses doigts : parce que chaque doigt est un.
« Le pluriel suppose le singulier », écrivait Leibniz : il n’y aurait pas plusieurs
êtres s’il n’y avait plusieurs fois un être. Et « ce qui n’est pas véritablement {\it un}
être, continuait-il, n’est pas non plus véritablement un {\it être} ». Ainsi l’unité est
première, au moins pour la pensée. De là, sans doute, le privilège métaphysique
de l’Un. Mais si la nature ne pense pas ?

\section{Univers}
%UNIVERS
Pour la plupart des philosophes, c’est l’ensemble de tout ce qui
existe ou arrive. Il est donc exclu qu’il y en ait plusieurs : si c’était
le cas, l’univers serait leur somme.

Que penser alors de hypothèse, parfois évoquée par les physiciens contemporains,
d’une pluralité d’univers ? Qu’elle correspond à l’idée philosophique
d’une pluralité des {\it mondes}, et rend les deux mots à peu près synonymes. Quand
% 600
on veut éviter l'ambiguïté, mieux vaut, philosophiquement, distinguer le
monde et le Tout (voir ces mots), et laisser l’univers aux physiciens.

\section{Universaux (querelle des {\bf --})}
%UNIVERSAUX (QUERELLE DES {\bf --})
C’est un débat qui traverse et structure
toute la pensée du Moyen-Âge. Il
s’agit de savoir quel type de réalité accorder aux idées générales ou universelles.
Sont-ce des êtres réels, comme le voulait Platon (réalisme), ou bien de simples
conceptions de notre esprit (conceptualisme), voire de purs mots (nominalisme) ?
Le matérialisme n’a bien sûr le choix qu’entre ces deux dernières solutions,
qui s'opposent peut-être moins qu’elles ne se complètent.

\section{Universel}
%UNIVERSEL
Qui vaut pour l’univers entier, ou pour la totalité d’un
ensemble donné. C’est en ce dernier sens que les droits de
l’homme sont universels : non parce que l’univers les reconnaîtrait (pourquoi
l'univers serait-il humaniste ?), mais parce qu’ils valent, en droit, pour tout être
humain. On voit que si {\it universel} s'oppose à {\it particulier}, ce n’est pas de façon
simple. Les droits de l’homme sont une particularité humaine (ils ne valent que
pour l'humanité), mais n’en sont pas moins universels pour autant (ils doivent
s'appliquer à tout être humain, y compris s’il ne les respecte pas).

L’universel, remarque Alain, est le lieu des pensées. Une vérité qui ne
serait pas vraie, en droit, pour tous, ne serait pas une vérité du tout. Cela,
remarquons-le, ne dépend pas du degré de généralité de la pensée considérée.
Que tu sois en train de lire cet article, c’est un fait très singulier. Mais
il n’y a pas un seul point de l’univers où l’on puisse nier cette vérité sans
faire preuve d’ignorance ou de mauvaise foi. Et comme tout est vrai, toujours,
tout est universel : le plus petit de nos mensonges est universellement
mensonger.

« La pensée, disait encore Alain, ne doit pas avoir d’autre chez soi que tout
l'univers ; c’est là seulement qu’elle est libre et vraie. Hors de soi ! Au-dehors ! »
L’universel, pour l’esprit, est la seule intériorité vraie.

\section{Univoque}
%UNIVOQUE
Qui ne se prend qu’en un seul sens, quel que soit l'emploi ou
le contexte et y compris quand on l’applique à des objets différents.
Sorti du langage scientifique, et encore, c’est beaucoup moins la règle
que l’exception. S’oppose à équivoque (qui a au moins deux significations différentes)
et parfois à plurivoque (qui en a plusieurs).

% 601
\section{Urbanité}
%URBANITÉ
La politesse des villes. C’est supposer qu’on trouve une politesse
aussi à la campagne, et qu’elle n’est pas la même. Quand
on croise chaque jour des milliers d’inconnus, la politesse inévitablement
devient plus nécessaire, plus superficielle, plus systématique. La foule impose sa
loi, qui est d’anonymat et de prudence.

\section{Usage/usure}
%USAGE/USURE
User, c’est d’abord se servir de. Puis détériorer peu à peu,
affaiblir, amoindrir. L'usage est antérieur à l’usure, et l’entraîne.
Ainsi une paire de souliers : on ne peut en user sans l’user. Le lien, entre
ces deux notions, n’est pourtant pas sans quelques notables exceptions. On
peut se servir de son corps et de son cerveau sans les user d’abord, et ils s’usent
d'autant plus, semble-t-il, qu’on s’en sert moins. C’est qu’on est dans l’ordre du
vivant, qui résiste à l’usure par l’exercice et la régénération (comme une
machine, disait Leibniz, qui réparerait d’elle-même ses rouages). Toutefois cela
ne dure qu’un temps. La matière ou l’entropie imposent leur loi, peu à peu, qui
est de dégradation et de mort. C’est ce qu’on appelle le vieillissement, usure
biologique. « Tout faiblit peu à peu, écrit Lucrèce, tout marche vers la mort,
usé par la longueur du chemin de la vie » ({\it De rerum natura}, II, 1173-1174).
Cela donne tort aux optimistes, non aux vivants.

\section{Utile}
%UTILE
Ce qui sert à quelque chose d’autre, à condition que ce quelque
chose soit bon ou jugé tel. Notion relative, donc : ce qui est utile
aux uns peut être nuisible aux autres, voire à la fois utile et nuisible pour les
mêmes (par exemple la voiture, utile pour les voyages, nuisible pour l’environnement).
Il n’y a pas d’utilité absolue : l’utile n’est pas une fin en soi ; ce n’est
qu'un moyen efficace, en vue d’une fin désirée. C’est pourquoi l’utilitarisme
aura besoin de se donner une fin ultime, qui sera presque toujours le bonheur
du plus grand nombre : est utile ce qui le favorise, nuisible ce qui lui fait obstacle.
Mais cette fin est elle-même sujette à caution. Si vous mettez la vérité ou
la justice plus haut que le bonheur, les frontières de l’utile et de l’inutile vont
se déplacer. Mais n’en existeront pas moins.

\section{Utilitarisme}
%UTILITARISME
Toute doctrine qui fonde ses jugements de valeur sur l’utilité.
Un égoïsme ? Non pas, puisque l’utilité est définie,
chez la plupart des utilitaristes (spécialement chez Bentham et John Stuart
Mill), comme ce qui contribue au bonheur du plus grand nombre. Rien
n'exclut donc qu’un utilitariste se sacrifie pour les autres, s’il considère que la
% 602
quantité globale de bonheur en est augmentée (s’il juge, cela revient au même,
que son sacrifice est utile). Et il est bien difficile, qu’on soit utilitariste ou pas,
de se sacrifier inutilement (serait-ce encore un sacrifice ?) ou même, sauf rigorisme
particulier, sans avoir le sentiment que le bonheur de l'humanité en est
au moins possiblement augmenté. Par quoi l’utilitarisme est moins une morale
particulière qu’une philosophie particulière de la morale : les comportements,
en pratique, seront souvent les mêmes, mais pensés ou justifiés différemment.

Jean-Marie Guyau a bien montré qu’Épicure était une espèce d’utilitariste
avant la lettre, comme Spinoza (s'agissant de ce dernier, voir par exemple
{\it Éthique}, IV, prop. 20 et 24), ou plutôt que « c’est l’épicurisme, uni au naturalisme
de Spinoza, qui renaît chez Helvétius et d’Holbach », avant de « susciter
dans la patrie de Hobbes des partisans plus nombreux encore » et de prendre
«sa forme définitive » chez Bentham et Mill ({\it La morale d'Épicure et ses rapports
avec les doctrines contemporaines}, 1878, Introduction). Qu’on en juge :

« La doctrine qui donne comme fondement à la morale l'utilité ou le principe du
plus grand bonheur, affirme que les actions sont bonnes ou sont mauvaises dans la
mesure où elles tendent à accroître le bonheur, ou à produire le contraire du bonheur.
Par “bonheur” on entend le plaisir et l’absence de douleur ; par “malheur”, la douleur
et la privation de plaisir. [...] Cette théorie de la moralité est fondée sur une conception
de la vie selon laquelle le plaisir et l’absence de douleur sont les seules choses désirables
comme fins, et que toutes les choses désirables ne le sont que pour le plaisir qu’elles
donnent elles-mêmes ou comme moyens de procurer le plaisir et d’éviter la douleur »
(John Stuart Mill, {\it L'utilitarisme}, II).

Faut-il alors renoncer à toute élévation, à toute spiritualité ? Aucunement,
puisque celles-ci peuvent être (comme le dit Mill de la vertu) un moyen ou une
partie du bonheur. C’est que le bonheur ({\it happiness}) est bien autre chose que la
satisfaction ({\it content}) des instincts ou des appétits. Chacun a les plaisirs qu’il
mérite, qui fondent aussi le bonheur qu’il ambitionne. Pour un individu aux
aspirations élevées, note Mill, toutes les satisfactions ne se valent pas. De là
cette vigoureuse formule, qui m’a toujours rendu son auteur extrêmement
sympathique : « Il vaut mieux être un homme insatisfait qu’un porc satisfait ; il
vaut mieux être Socrate insatisfait qu’un imbécile satisfait. Et si l’imbécile ou le
porc sont d’un avis différent, c’est qu’ils ne connaissent qu’un côté de la
question : le leur. L’autre partie, pour faire la comparaison, connaît les deux
côtés » ({\it ibid.}).

On remarquera pourtant que l'utilité peut être un critère de valeur, mais
non de vérité. C’est ce qui distingue, ou qui peut distinguer, l’utilitarisme du
pragmatisme, voire de la sophistique. Une vérité inutile, et même nuisible (une
vérité qui ne serait pas « avantageuse pour la pensée », comme disait William
% 603
James), n’en serait pas moins vraie pour cela. Et un mensonge utile ou une
erreur avantageuse, pas moins faux. C’est en quoi l’utilitarisme, même moralement
justifié, ne saurait tenir lieu de philosophie : ce n’est pas parce qu’une
idée est favorable au bonheur du plus grand nombre qu’il faut la penser (ce ne
serait plus philosophie mais sophistique, plus utilitarisme mais méthode
Coué) ; c’est parce qu’elle semble vraie. Un utilitariste pourrait objecter que la
vérité, même désagréable, est plus utile, au bout du compte, qu’une illusion,
même confortable, et que l’utilitarisme est sauvé par là de la sophistique. Dont
acte. Mais il n’échappe au cercle qu’en acceptant que la vérité de cette dernière
idée ne dépend pas de son utilité (puisque son utilité, si elle est vraie, en
dépend). La vérité ne peut être utile que si ce n’est pas son utilité qui fait sa
vérité. Elle ne peut être une valeur que si elle n’a pas besoin de valoir pour être
vraie. Par quoi l’utilitarisme n’est acceptable qu’à l’intérieur du rationalisme,
non contre lui.

Il reste que l’utilitarisme, même s'agissant des actions, pèche peut-être par
optimisme. « Si les hommes n’avaient en vue que l’utile, tout s’arrangerait, écrit
Alain. Mais il n’en est pas ainsi. » C’est qu’ils agissent par passion, beaucoup
plus que par intérêt. De là les guerres, nuisibles pour presque tous. L’amour-propre
est un moteur plus puissant que l’égoïsme, et plus dangereux.

\section{Utopie}
%UTOPIE
Ce qui n’existe nulle part (en aucun lieu : {\it u-topos}). Un idéal ? Si
l’on veut, mais programmé, mais organisé, mais planifié, souvent
avec un soin maniaque des détails : c’est un idéal qui ne se résigne pas à en être
un, qui se prend pour une prophétie ou un mode d’emploi. Se dit spécialement
des sociétés idéales ; l’utopie est alors une fiction politique, qui sert moins à
condamner la société existante (pas besoin d’utopie pour cela) qu’à en proposer
une autre, déjà conçue dans ses détails, qui n’aurait plus qu’à être réalisée. Ainsi
chez Platon, Thomas More (qui inventa le mot) ou Fourier.

Le mot peut se prendre positivement ou négativement : il peut désigner ce
qui n’existe pas encore mais existera un jour ; ou bien ce qui n’existe pas et
n’existera jamais. Dans le premier cas, c’est un but, vers lequel il faut tendre ;
dans le second, une illusion, à laquelle il vaut mieux renoncer. Dans l’usage
courant, c’est ce second sens qui tend à l'emporter : une utopie, c’est un but ou
un programme qu'on juge irréalisable. Parce qu’on manque d’imagination,
d’audace, de confiance ? C’est ce que certains suggèrent : l'utopie d’aujourd’hui
serait la réalité de demain. Et d’évoquer les congés payés, la sécurité sociale, la
télévision, Internet, qu’on aurait jugé utopiques il y a quelques siècles. Mais
c’est confondre utopie et science-fiction, Thomas More et Jules Verne. Les
grandes utopies du passé (depuis la République de Platon jusqu’aux socialismes
% 604
utopistes du {\footnotesize XIX$^\text{e}$} siècle), nous semblent aussi irréalisables aujourd’hui qu’hier,
et plus dangereuses. C’est qu’on voit trop ce qu’elles supposeraient de
contraintes et de bourrage de crâne (de totalitarisme). Une utopie, ce n’est pas
seulement un projet de société qui semble présentement impossible ; c’est une
société parfaite, qui ne laisserait rien à transformer. Ce serait la fin de l’histoire,
la fin des conflits, comme une espèce de paradis collectif : cela ressemble à un
Club Méditerranée définitif, autrement dit à la mort.


%
%VWXYZ {\it }

VACANCE (S) Au singulier : vide, absence, oisiveté… Il faut que la vie quotidienne
soit bien dure, ou bien vaine, pour que le même
terme, au pluriel et à l'envers de son vide initial, en vienne à suggérer le plein
d’une vie, pour une fois, à peu près intense et joyeuse. Et bien triste qu’il
faille pour cela partir. « Partir, c’est vivre beaucoup », semblent-ils croire. C’est
qu’ils n’habitent que la mort ou le travail.

Les vacances, considérées comme congés payés, restent ainsi prisonnières
des formes modernes d’aliénation, dont elles sont l'expression autant que l’antidote.
Est aliéné celui dont la vie est ailleurs — celui qui doit partir pour rentrer
chez soi. Une société injuste mais riche devait sécréter ces ersatz d’utopie. Ce
sont les bacchanales de notre temps — Dionysos au Club Méditerranée. On
aurait tort de les mépriser, comme de s’en contenter.

VALEUR Ce qui vaut, et le fait de valoir. Un prix ? Seulement pour ce qui
en a un, pour ce qui est à vendre. Par exemple la valeur d’une
marchandise : son prix indique sa valeur d’échange, dans un marché donné,
telle qu’elle résulte du temps de travail moyen socialement nécessaire à sa production
(selon Marx) ou de la loi de l'offre et de la demande (selon la plupart
des économistes libéraux). Mais la justice ? Mais la liberté ? Mais la vérité ?
Elles peuvent avoir un coût, dans telle ou telle circonstance. Mais elles n’ont
pas de prix : elles ne sont pas à vendre. Aussi faut-il distinguer ce qui a une
valeur (qu’un prix, dans une logique d’échange, peut mesurer à peu près), et ce
qui est une valeur, qui n’a pas de prix et ne saurait être échangé valablement
contre de l'argent, ni même contre une autre valeur. Échanger la justice contre
la liberté ? Ce serait manquer à la justice. Échanger la vérité contre la justice ?
%— 606 —
Ce serait manquer à la vérité. Ainsi les valeurs n’ont pas de prix : elles ont une
{\it dignité}, comme disait Kant, qui n’admet pas d’équivalent et ne saurait être
échangée contre autre chose. Faut-il y voir pour autant un absolu ? Non pas,
me semble-t-il, puisqu'il reste à comprendre d’où vient cette valeur qu’on lui
prête ou qu’elle est. Une valeur, c’est ce qui vaut, disais-je ; mais qu'est-ce que
valoir ? C’est être désirable ou désiré. C’est vrai pour les marchandises : elles
n’ont de valeur d'échange, soulignait Marx, qu’à la condition d’avoir d’abord
une valeur d'usage. Or celle-ci n’est pas un absolu. « La marchandise est
d’abord un objet extérieur, qui, par ses propriétés, satisfait des besoins humains
de n’importe quelle espèce. Que ces besoins aient pour origine l’estomac ou la
fantaisie, leur nature ne change rien à l’affaire » ({\it Le Capital}, I, 1). Comment
mieux dire qu’il s’agit moins de {\it besoin} que de {\it désir}, moins d’utilité que d’{\it usage}
en effet ? Un objet, même apparemment inutile, peut avoir une grande valeur,
s’il est fortement désiré par beaucoup : ainsi une pierre précieuse ou une œuvre
d’art (ce n’est pas parce qu’elles sont utiles qu’elles sont désirées, c’est parce
qu’elles sont désirées qu’elles semblent utiles, et le sont en effet). Un objet
manifestement utile, à l'inverse, n’a de valeur qu’à proportion du désir qu’on
en a. Les commerçants le savent bien. Ce n’est pas l'utilité qui fait la valeur
d'usage, c’est la valeur d’usage, telle qu’elle résulte elle-même du désir, qui fait
l'utilité. Si une marchandise n’a de valeur d'échange qu’à la condition d’avoir
une valeur d'usage, il faut donc en conclure qu’elle n’a de valeur qu’à proportion
non de je ne sais quelle utilité objective ou absolue, si tant est que la
notion puisse avoir un sens, mais du désir, historiquement déterminé, qui la
vise. Cela donne raison à la loi de l'offre et de la demande, me semble-t-il,
davantage qu’à la théorie marxiste de la valeur (un vêtement à la mode peut
demander moins de temps de travail qu’un autre, et valoir pourtant beaucoup
plus cher), mais ce n’est pas ici ce qui m'importe. Philosophiquement, sinon
économiquement, les deux théories se rejoignent en ceci qu’il n’y a de valeurs,
dans un marché donné, que relatives : que pour et par le désir. C’est où l’on
rencontre les valeurs morales ou spirituelles. Qu’elles soient en dehors de tout
marché — sans équivalent, sans prix, sans échange possible —, je l’indiquais en
commençant. Mais cela ne prouve pas qu’elles soient en dehors de tout désir !
Comment la justice serait-elle une valeur, si personne ne désirait la justice ?
Comment la vérité pourrait-elle valoir, si personne ne l’aimait ou ne la
désirait ? C’est ce que Spinoza, dans un scolie abyssal de l’{\it Éthique}, nous invite
à penser : « Nous ne nous efforçons à rien, ne voulons, n’appétons ni ne désirons
aucune chose, parce que nous la jugeons bonne ; mais, au contraire, nous
jugeons qu’une chose est bonne parce que nous nous efforçons vers elle, la voulons,
appétons et désirons » (III, 9, scolie). Chacun a le sentiment du contraire.
Si j'aime la richesse ou la justice, n’est-ce pas parce qu’elles sont bonnes ? Si je
%— 607 —
désire cette femme, n’est-ce pas parce qu’elle est belle ? Non pas, répondrait
Spinoza : c’est parce que tu aimes la richesse et la justice qu’elles te paraissent
bonnes ; c’est parce que tu désires cette femme qu’elle te paraît belle. Le fait est
que richesse ou justice sont indifférentes à quelques-uns. Et qu’un singe, à ce
que nous jugeons la plus belle des femmes, préférerait une guenon. Parce qu’il
a mauvais goût ? Il faudrait être bien naïf pour le penser. Mais parce qu’il n’a
pas le même désir. Relativisme sans appel : une valeur c’est ce qui est désirable,
et elle n’est désirable que parce qu’elle est désirée. Ce qui vaut, c’est ce qui plaît
ou réjouit, pour un individu et dans une société donnée. C’est pourquoi
l'argent, pour certains, vaut plus que la justice. Et c’est pourquoi la justice,
pour d’autres, vaut davantage que l’argent. Il n’y a pas de valeurs absolues. Il
n'y a que des désirs et des conflits de désirs, que des affects et des hiérarchies
entre affects. C’est ce que Spinoza, à propos du bien et du mal mais aussi de
l'argent et de la gloire, explique fort nettement :

« Par {\it bien}, j'entends ici tout genre de joie et tout ce qui y mène, et principalement
ce qui satisfait un désir, quel qu’il soit. Par {\it mal}, j'entends tout genre de tristesse, et
principalement ce qui frustre un désir. Nous avons en effet montré plus haut (dans le
scolie de la prop. 9) que nous ne désirons aucune chose parce que nous la jugeons
bonne, mais qu’au contraire nous appelons bonne la chose que nous désirons ; conséquemment,
nous appelons mauvaise la chose que nous avons en aversion. Chacun juge
ainsi ou estime, selon son affect, quelle chose est bonne, quelle mauvaise, quelle
meilleure, quelle pire, quelle enfin la meilleure ou la pire. Ainsi l’avare juge que l’abondance
d’argent est ce qu’il y a de meilleur, la pauvreté ce qu’il y a de pire. L’ambitieux
ne désire rien tant que la gloire et ne redoute rien tant que la honte. À l’envieux, rien
n'est plus agréable que le malheur d'autrui, et rien plus insupportable que le bonheur
d’un autre ; et ainsi chacun juge, d’après ses propres affects, qu’une chose est bonne ou
mauvaise, utile ou inutile » ({\it Éthique} III, 39, scolie).

Tel est aussi l'esprit de Nietzsche : « Évaluer, c’est créer. C’est leur évaluation
qui fait des trésors et des joyaux de toutes choses évaluées » ({\it Zarathoustra},
I, « Des mille et un buts»). Évaluer, ce n’est pas mesurer une valeur qui
préexisterait à l'évaluation ; c’est mesurer la valeur qu’on donne à ce qu’on
évalue, ou créer de la valeur en la mesurant. Spinoza-Nietzsche, même
combat ? Oui, assurément, s'agissant du relativisme. S'agissant des valeurs ellesmêmes ?
Cela dépend desquelles (Spinoza, lui, n’a jamais prétendu les renverser
toutes). Mais la vraie question, sur quoi ils s’opposent, est celle de la vérité.
Nietzsche, surtout dans ses dernières œuvres, tend à la considérer comme une
valeur parmi d’autres, ce que Spinoza ne saurait accepter. Que telle chose me
semble bonne ou mauvaise, cela dépend du désir que j’en ai. Mais qu’elle soit
vraie, non. Ils se rejoignent dans le relativisme (s'agissant des valeurs), s’opposent
%— 608 —
sur le rationalisme (s’agissant de la vérité ou de la raison). C’est où j'ai
choisi Spinoza contre Nietzsche, et le cynisme contre la sophistique. La vérité
est-elle une valeur ? Oui, si nous la jugeons bonne ou utile, ou simplement si
nous l’aimons ou la désirons. Elle en est donc une, pour presque tous : tous les
hommes aiment la vérité, disait saint Augustin, puisque aucun, même parmi les
menteurs, n’aime être trompé. Mais ce n’est pas parce qu’elle est vraie qu’elle
vaut, encore moins parce qu'elle vaut qu’elle est vraie. Disjonction des ordres :
la valeur de la vérité dépend du désir qu’on en a, mais sa vérité, non. Toute
valeur est subjective (y compris la vérité comme valeur) ; aucune vérité ne l’est.
Toute valeur est relative ; toute vérité (en tant qu’elle est vraie, en tant qu’elle
est la même en nous et en Dieu, comme dit Spinoza) est absolue. Toute valeur
est de l’homme. Toute vérité, de Dieu. C’est pourquoi nous sommes toujours
dans le vrai, et hors d’état pourtant de le posséder absolument. C’est pourquoi
nous le cherchons. C’est pourquoi nous le désirons. C’est pourquoi il vaut, au
moins pour nous, au moins par nous. Le jour où plus personne n’aimera la
vérité, elle aura cessé par là même d’être une valeur. Mais n’en sera pas moins
vraie pour autant.

Si vous n’aimez pas la vérité, n’en dégoûtez pas les autres. Et ne tenez pas,
si vous l’aimez, cet amour pour une preuve.

VALIDITÉ C’est le nom logique de la vérité, ou plutôt son équivalent formel.
Une inférence est valide lorsqu’elle permet de passer du vrai
au vrai (de la vérité des prémisses à la vérité de la conclusion) ou lorsqu'elle est
vraie quelle que soit l'interprétation qu’on en peut donner. On remarquera que
la validité d’un raisonnement ne dépend pas de la vérité de ses conclusions, pas
plus d’ailleurs que celle-ci ne dépend forcément de celle-à. Un raisonnement
valide peut aboutir à une conclusion fausse (si l’une au moins de ses prémisses
est fausse). C’est le cas, par exemple, du fameux sophisme du cornu : « Tu as
tout ce que tu n’as pas perdu ; tu n’as pas perdu de cornes ; donc tu as des
cornes » ; le raisonnement est valide, la conclusion est fausse (c’est que la.
majeure, quoique on puisse ne pas s’en rendre compte immédiatement, l’est
aussi). Et un raisonnement non valide, à l’inverse, peut aboutir à une conclusion
vraie : « Tous les hommes sont mortels ; Socrate est mortel ; donc Socrate
est un homme » est un raisonnement non valide.

VANITÉ On pense d’abord à l’Ecclésiaste : {\it « Vanité des vanités, tout est
vanité... »} C’est dire que tout est vide ou vain (venus : vide, creux,
sans substance), que rien n’a de valeur ou d’importance, sinon illusoire, sinon
%— 609 —
fugace, que le néant vaudrait mieux ou tout autant, enfin que rien ne vaut la
peine d’être vécu ni désiré. Est-ce vrai ? Il n’y a pas de réponse absolue ; il n’y
a que le désir qu’on éprouve ou pas de ces presque riens qui font notre vie, bonheur
et malheur, qui vont disparaître, certes, qui disparaissent déjà, mais qui
n’en sont pas moins vrais ni délectables, pour qui s’en délecte, ou douloureux,
pour qui en souffre. Montaigne, qui est notre Ecclésiaste, a consacré à cette
notion le plus beau de ses Essais (III, 9, « De la vanité »). Mais lui n’en tirait
aucune leçon nihiliste. C’est qu’il aimait la vie, quand l’Ecclésiaste, de son
propre aveu, la détestait. Par exemple Montaigne se plaît aux voyages. « Il y a
de la vanité, dites-vous, en cet amusement. — Mais où non ? Et ces beaux préceptes
sont vanité, et vanité toute la sagesse » (III, 9, 988). Le maître de Montaigne
est le vent, qui ne va nulle part, qui n’a rien à prouver, mais qui « s’aime
à bruire et à s’agiter ». Vent : vanité. « Le vent part au midi, tourne au nord,
disait l’Ecclésiaste, il tourne, tourne et va, et sur son parcours retourne le
vent... » Ainsi fait Montaigne : « S'il fait laid à droite, je prends à gauche. Ai-je
laissé quelque chose derrière moi? J'y retourne; c’est toujours mon
chemin. » On lui objecte son âge : « Vous ne reviendrez jamais d’un si long
chemin. — Que m'en chaut-il ? Je ne l’entreprends ni pour en revenir, ni pour
le parfaire ; j’entreprends seulement de me branler [de me mouvoir] pendant
que le branle me plaît. Et me promène pour me promener...» Vanité de la
sagesse constatait l’Ecclésiaste, et Montaigne en est d’accord. Mais il ajoute,
avec le vent : sagesse de la vanité.

En un autre sens, la vanité est une forme, particulièrement ridicule, de
lamour-propre. C’est être plein du vide de soi : c’est se glorifier de ce qu’on
croit être, c’est admirer en soi ce qu’on imagine que les autres y admirent (la
vanité, écrit Bergson, est « une admiration de soi fondée sur l’admiration qu’on
croit inspirer aux autres »), ou vouloir qu’ils admirent ce qu’on y admire soi-même.
Nul n’y échappe tout à fait. C’est ce qu’a vu Pascal : « La vanité est si
ancrée dans le cœur de l’homme qu’un soldat, un goujat, un cuisinier, un crocheteur
se vante et veut avoir ses admirateurs, et les philosophes mêmes en veulent,
et ceux qui écrivent contre veulent avoir la gloire d’avoir bien écrit, et ceux
qui les lisent veulent avoir la gloire de les avoir lus, et moi qui écris ceci ai peut-être
cette envie, et peut-être que ceux qui le liront.. » ({\it Pensées}, 627-150). Se
savoir vaniteux, toutefois, c’est déjà l’être moins. « La seule cure contre la
vanité, disait encore Bergson, c’est le rire » ; mais à condition de savoir rire de
soi.

Enfin, on appelle {\it vanités}, dans l’histoire de la peinture, certaines natures
mortes évoquant — par une fleur fanée, un crâne, une bougie consumée… — le
peu que nous sommes et que nous durons. C’est revenir au sens de l’Ecclésiaste,
qui est le sens existentiel, tout en essayant de nous guérir de la vanité au sens
%— 610 —
psychologique ou moral. On dirait que ces peintres veulent nous dégoûter de
la vie, pour que nous ne nous intéressions plus qu’à la mort ou à la religion.
Mais les plus doués n’arrivent même pas à nous dégoûter de la peinture. Est-ce
nous qui sommes trop vains, ou la peinture qui est trop belle ?

VÉCU La vie elle-même, mais au passé (fût-ce d’un dixième de seconde) et
dans sa singularité individuelle et immédiate, ou prétendument
immédiate (la conscience fait une médiation suffisante). C’est la vie à la première
personne, telle qu’elle a été éprouvée, telle qu’on s’en souvient, telle
qu’on en porte témoignage. Souvent opposé à la pensée, à la théorie, à l’abstraction.
Mais ce n’est guère qu’une abstraction de plus. Un vécu qui ne serait
pas pensé, on ne pourrait rien en dire. Vaudrait-il même la peine d’être vécu ?
Le vécu n’est pas la vie : il n’est que la conscience que nous en prenons et
en gardons — que son souvenir ou sa trace. « Notre vie est si vaine, disait Chateaubriand,
qu’elle n’est qu’un reflet de notre mémoire. » Le vécu est ce reflet,
et cette vanité.

VELLÉITÉ Une volition sans force, sans continuité, sans constance. C’est
vouloir sans agir, ou sans agir vraiment (dans la durée). Ce n’est
donc pas vouloir : c’est désirer vouloir, ou imaginer qu’on veut.

VÉRACITÉ Ce n’est pas la même chose que la vérité. La véracité est la qualité
de celui qui dit vrai, qui ne trompe ni ne se trompe. Elle est
donc une disposition, mais objective (on peut être sincère sans être vérace), du
sujet. La vérité serait plutôt l’objectivité même. La vérité, pour le dire autrement,
est le propre de ce qui est vrai ; la véracité, de ce qui est véridique. Ainsi
le vrai Dieu, chez Descartes, est un Dieu vérace. Mais ce n’est pas parce qu’il
est vérace qu’il est le vrai Dieu ; c’est parce qu’il est le vrai Dieu (ou parce qu’il
est vraiment Dieu) qu’il est vérace.

VERBE Parfois synonyme, surtout avec une majuscule, de Parole ou de Logos :
ce serait l’acte de Dieu, ou Dieu en acte, en tant qu’il fait sens. On
l’identifie traditionnellement à la deuxième personne de la Trinité : « le Verbe
s’est fait chair, lit-on dans le prologue de Jean, et il a habité parmi nous... »

Au sens ordinaire, un verbe est un mot, mais qui désigne le plus souvent un
mouvement, un événement ou un acte (par différence avec les {\it noms}, qui
%— 611 —
désignent plutôt des choses, des entités ou des individus). De là ce {\it langage-monde} 
qu'a imaginé Francis Wolff ({\it Dire le monde}, PUF, 1997), qui ne serait
fait que de verbes: monde du devenir pur, sans rien qui demeure ni qui
change, monde d’accidents sans substances, d’actions sans sujets, d'événements
sans essences et sans choses... C’est le monde à peu près d'Héraclite ou du
Bouddha, et c’est le vrai peut-être. Mais notre langage — « cet étrange entrelacs
de noms et de verbes » qu’évoquait Platon — est incapable de le dire comme
notre esprit de le penser. C’est que tout verbe, pour nous, a besoin d’un sujet.
De Rà le fameux {\it « Je pense donc je suis »} de Descartes, dont Nietzsche a bien vu
qu'il ne tirait son évidence que de la croyance. en la grammaire. Il m'arrive
de penser que le « réel voilé » de la mécanique quantique doit ressembler à ce
monde de verbes sans sujets et d’événements sans choses. C’est pourquoi nos
physiciens ne peuvent l’énoncer exactement, ni nous tout à fait le comprendre.
Nous n'avons pas la langue qu’il faudrait pour cela.

VÉRIFICATION C’est tester la vérité d’un énoncé, en vue de l’attester. On
peut ainsi vérifier un calcul, en le refaisant ou en en faisant
un autre, ou une hypothèse, en la soumettant à l'expérience. Reste à savoir ce
que valent ce nouveau calcul ou cette expérience. S'agissant du calcul, on considère
que la probabilité d’une erreur décroît rapidement, en proportion du
nombre de vérifications effectuées, si possible par plusieurs individus et selon
des procédures différentes. Toutefois cela reste soumis à la fiabilité de notre
raison, qui est sans vérification possible (puisque toute vérification la suppose).
S'agissant de l'expérience ou de l’expérimentation, on bute sur le problème de
l'induction (voir ce mot). Comment vérifier un énoncé universel (« tous les
cygnes sont blancs ») en additionnant des constats singuliers (ce cygne est
blanc, et cet autre, et cet autre..), étant entendu qu’on ne peut en dresser la
liste exhaustive et qu’une seule exception suffirait à invalider l'énoncé ? Il n’y a
donc pas de vérification en toute rigueur. En revanche, montre Popper, il y a
des falsifications suffisantes : un seul cygne noir ou coloré suffit à prouver qu’ils
ne sont pas tous blancs. Cette asymétrie entre vérifiabilité et falsifiabilité est au
cœur de la démarche expérimentale. Vérifier une théorie ou une hypothèse, ce
n’est jamais prouver en toute rigueur qu’elle est vraie, c’est essayer de montrer
qu’elle est fausse. On la tient pour vraie — au moins relativement et provisoirement
— tant qu'elle a résisté à toutes les tentatives de falsification. On remarquera
que cette solution popperienne ou darwinienne (les théories les plus
faibles sont éliminées, seules les meilleures subsistent) du problème de l’induction
est davantage épistémologique que métaphysique. Elle explique le fonctionnement
des sciences ; elle ne dit rien de leur vérité globale : non seulement
%— 612 —
parce qu’il se pourrait que toute pensée ne soit qu’un rêve, comme le reconnaît
Popper, mais parce que le test expérimental devrait lui-même être vérifié, et ne
peut l’être absolument. Il suffit d’un seul cygne noir pour prouver qu’ils ne
sont pas tous blancs. Mais comment prouver qu’un cygne est vraiment noir ?
Toute vérification et toute falsification supposent une vérité antécédente — celle
du monde, celle de l'expérience, celle de la raison — qui est invérifiable et infalsifiable.
Si nous n’étions d’abord dans le vrai, nous n’aurions aucune chance de
rencontrer le faux. Si la vérité n’était antérieure à toute vérification, il n’y aurait
rien à vérifier. C’est où l’on retrouve Spinoza : {\it « Habemus enim ideam veram »}
(car nous avons une idée vraie, {\it T.R.E.}, 27). Cela est sans preuve, mais il n’y
aurait pas de preuve autrement, ni rien à prouver.

VERITAS Le nom latin et scolastique de la vérité, qu’on distingue à ce titre
de l’{\it alèthéia}, qui est son nom grec. Il est usuel, depuis Heidegger,
de se servir de ces deux mots pour désigner deux conceptions, ou deux acceptions,
différentes de la vérité : la vérité comme adéquation entre la pensée et le
réel ({\it l'ad{\ae}quatio rei et intellectus} des scolastiques : {\it veritas}), d’une part, et, d’autre
part, la vérité comme non-voilement de l'être même (la vérité intrinsèque de la
chose, qu’elle soit connue ou pas, ce que j’appellerais volontiers sa présentation
silencieuse : {\it alèthéia}). L'{\it alèthéia} est de l'être ou du silence ; la {\it veritas}, de la
pensée ou du discours. Certains en concluront que l’{\it alèthéia} est moins une
{\it vérité} qu’une {\it réalité}. Mais si le réel n’était vrai aussi et d’abord, quelle pensée
pourrait l’être ?

VÉRITÉ Ce qui est vrai, ou le fait de l’être, ou le caractère de ce qui l’est.
C’est donc une abstraction (la vérité n’existe pas : il n’y a que des
faits ou des énoncés vrais). Mais cette abstraction nous permet seule de penser.
S’il n’y avait rien de commun, au moins pour la pensée, entre deux propositions
vraies, il n’y aurait aucun sens à dire qu’elles le sont, ni donc, intellectuellement,
à dire quoi que ce soit : tous les discours se vaudraient et ne vaudraient
rien (puisqu'on pourrait dire aussi bien, ou aussi mal, le contraire de ce qu’on
dit). Il n’y aurait aucune différence entre un délire et une démonstration, entre
une hallucination et une perception, entre une connaissance et une ignorance,
entre un faux témoignage et un témoignage véridique, entre un savant et un
ignorant, entre un historien et un mythomane. Ce serait la fin de la raison, et
de la déraison. {\it Veritas norma sui et falsi}, disait Spinoza : la vérité est norme
d’elle-même et du faux ({\it Éthique}, II, 43, scolie). Il n’y aurait, sans cette normativité
immanente, aucun moyen de se tromper, ni de ne se tromper pas, aucun
%— 613 —
moyen de mentir ni de ne mentir pas. Par quoi une seule erreur reconnue, et
ce n'est pas cela qui manque, un seul mensonge démasqué, et ils sont légion,
suffit à attester au moins l’idée de vérité. Abstraction, donc, mais nécessaire.
Même le silence, pour l'esprit, en relève. S’il est vraiment silencieux, c’est une
vérité. S’il ne l’est pas, c’en est une autre.

Où en sommes-nous avec la vérité ? La question est aussi ancienne que la
philosophie (plus ancienne ? non, puisque cette question est déjà philosophique,
puisqu'elle est, peut-être bien, la philosophie même), mais se pose,
aujourd’hui, à nouveaux frais. Tout se passe comme si les progrès mêmes de
la connaissance rendaient la notion de vérité plus problématique. Il y a là un
paradoxe, qui en dit long sur notre modernité. Aucune époque n’a disposé
d'autant de connaissances, ni d’aussi précises, ni d’aussi fiables. Un bon élève
de nos lycées en sait beaucoup plus — sur le monde, sur l’histoire, sur presque
tout — que n'en savaient Aristote ou Descartes. Nos savants, qui sont sans
doute la principale gloire de notre triste époque, multiplient comme jamais
découvertes et expérimentations. La biologie ou la physique de notre temps
eussent sidéré — si tant est qu’ils aient pu les comprendre — un Buffon ou un
Laplace. Il n’est pas jusqu’à nos médias, dans leur médiocrité essentielle, qui
ne contiennent une multitude d’informations sans commune mesure avec
celles dont disposaient les esprits les plus avancés des siècles passés. Bref, on
en sait beaucoup plus qu’on n’en a jamais su, dans presque tous les domaines,
et l’on pourrait croire que la notion de vérité en serait sortie renforcée
d'autant. On sait qu’il n’en est rien, et c’est peut-être, philosophiquement, la
marque principale du siècle qui vient de s'achever. La vérité ? Quel scientifique,
aujourd’hui, prétendrait la connaître ? Quel artiste s’en préoccupe
encore ? Et combien de philosophes vont jusqu’à dire qu’elle n’existe pas,
qu’elle n’a jamais existé, qu’elle n’est que la dernière illusion dont il importe
de s’affranchir ?

Il y a à cela plusieurs raisons, tant théoriques que pratiques. Les raisons
théoriques peuvent être rattachées, par commodité, à la révolution kantienne
ou à ce qu'elle manifeste. Dès lors que nous sommes séparés du réel par les
moyens mêmes qui nous servent à le connaître, il n’est que trop clair que nous
ne pourrons jamais le connaître tel qu’il est en lui-même ou absolument. Nous
ne connaissons pas l’être : nous ne connaissons que des phénomènes, que le
monde tel qu’il apparaît à travers les formes de notre sensibilité et de notre
entendement, que des objets que nous construisons (par notre perception, par
notre langage, par nos sciences), qui sont sans rapport assignable avec les choses
en soi. On dira que cela n’annule pas nos connaissances, que cela permet au
contraire de les penser comme possibles et nécessaires... Certes. Mais une
connaissance qui ne porte plus sur l’être, est-ce encore une vérité ? « Le même
%— 614 —
est à la fois penser et être », disait Parménide, et c’est ce qui nous devient de
moins en moins concevable. « La vérité consiste en l’être, disait Descartes, la
vérité est une même chose avec l’être », et c’est cela même — l’être, la vérité,
l’heureuse indistinction des deux — que nous avons perdu et qui nous sépare,
philosophiquement, du bonheur. Nous voilà chassés du pays de la vérité,
chassés du pays de l'être, puisque c’est le même, et c’est cet exil que nous appelons
le monde.

L’oubli de l’être se fit parfois, nous rappelle Heidegger, au nom de la vérité
— parce qu’elle n’était vérité que du sujet. Mais combien plus redoutable
l’oubli conjoint des deux, comme une lente enfoncée dans le phénoménisme
ou la sophistique ! Si rien n’est vrai, comme le voulait Nietzsche, que reste-t-il
à vivre et à penser ? Nos rêves, nos désirs, nos interprétations, nos fantasmes,
nos illusions ? Soit, mais alors ils se valent tous — puisque aucune vérité entre
eux ne vient trancher — et ne valent rien. C’est où la sophistique mène au nihilisme,
et Nietzsche à notre modernité. S’il n’y a pas de faits, s’il n’y a que des
interprétations, selon la fameuse formule de {\it La volonté de puissance}, le monde
même se dérobe : il n’y a plus que des discours sur le monde. C’est comme un
monde virtuel, qui aurait absorbé le vrai jusqu’à le dissoudre. Qu’on puisse y
vivre, peut-être. Mais à quoi bon, alors, vouloir le vivre et le penser {\it en vérité} ?
Pourquoi ne pas se contenter d’un beau mensonge, d’un discours habile ou
d’une illusion confortable ? Philosophie de bavards et de sophistes, où la philosophie
se meurt. S’il n’y a pas de vérité, on peut penser n’importe quoi, mais
aussi on ne peut plus penser du tout. Si rien n’est vrai, il n’est pas vrai que rien
ne soit vrai. Si tout est faux, il est faux que tout le soit. Cette autocontradiction,
loin de la réfuter, rend la sophistique irréfutable : puisqu'on ne pourrait la
réfuter qu’au nom d’une vérité au moins possible, qu’elle récuse. Alors ? Alors
il n’y a plus que les rapports de forces et le conflit — aussi inépuisable qu’épuisant
— des interprétations. C’est le monde de la guerre, du marché et des
médias. C’est notre monde. Plutôt c’est ce que certains voudraient qu’il soit, un
monde sans être, sans réalité, sans vérité, un monde sans consistance, un
monde virtuel, répétons-le, où il n’y aurait plus que des signes et des échanges,
que des simulacres et des marchands, un monde pour rire, comme un jeu de
l'esprit, et cela m’a donné bien souvent, quand j'étais étudiant, comme une
envie de pleurer.

Il faut en sortir. Comment ? Par un retour décidé à l’idée de vérité. Qu’on
ne puisse jamais la connaître tout entière ni absolument, c’est aujourd’hui une
évidence, avec laquelle je ne prétends aucunement rompre. Au reste, c’en était
une déjà pour Montaigne, Pascal ou Hume. Mais ceux-là n’ont jamais prétendu
pour autant qu’elle n’existait pas, ni qu’on ne pouvait aucunement y
accéder ! Ils ont simplement contesté, c’est tout autre chose, qu’on puisse le
%— 615 —
faire avec certitude. C’est ce qui distingue le sceptique (pour qui rien n’est certain)
du sophiste (pour qui rien n’est vrai). Les deux positions ne sont ni identiques
ni convergentes. Que rien ne soit certain, cela ne prouve pas que tout
soit faux. Que tout soit douteux, cela ne prouve pas que rien ne soit vrai. Au
contraire, même, puisqu’une proposition quelconque, fût-elle sceptique, n’est
pensable que sous l’idée de vérité (c’est ce que j'appelle, corrigeant Spinoza par
Montaigne, {\it la norme de l'idée vraie donnée ou possible}), ce qui interdit, ou
devrait interdire, qu’elle prétende s’en libérer. Formidable culot de Nietzsche :
« Qu'un jugement soit faux, ce n’est pas, à notre avis, une objection contre ce
jugement » ({\it Par-delà le bien et le mal}, X, 4). Je pense très exactement le
contraire, comme la quasi-totalité des scientifiques d’aujourd’hui, et c’est ce
qui nous rattache, tous ensemble et malgré Nietzsche, aux Lumières. C’est où
Popper, disons-le en passant, nous permet d'échapper au relativisme intégral.
Qu’aucune théorie ne puisse jamais, en toute rigueur, être vérifiée expérimentalement,
cela ne veut pas dire qu’elles se valent toutes : puisqu'elles peuvent au
moins être réfutées ou falsifiées, puisqu'elles le sont en effet, voyez l’histoire des
sciences, et se succèdent ainsi dans un ordre à la fois irréversible et normatif qui
est celui du progrès même de nos connaissances. Cela rejoint une des fulgurances
de Pascal: nous ne connaissons jamais la vérité directement, mais
devons « prendre pour véritables les choses dont le contraire nous paraît faux »
({\it De l'esprit géométrique}, p. 352 b ; voir aussi la pensée 905-385). C’est le véritable
ordre, qui se fait « par approfondissement et ratures », comme disait
Cavaillès, et qui n’en instaure pas moins — ou plutôt qui n’en instaure que
mieux — « des résultats dont la validité est hors du temps » (Cavaillès, {\it Sur la
logique}, WI ; {\it Lettre à P. Labérenne}, 1938). C’est parce qu’il y a une histoire des
sciences (et non malgré cette histoire) que les sciences ne se réduisent pas à leur
historicité, contrairement à ce que croyait Montaigne, et nous ouvrent — dans
l’histoire, par l’histoire — à quelque chose qui la dépasse. Quoi ? L’éternité. Il y
eut d’abord Ptolémée, certes, puis Newton, puis Einstein... Mais cette succession,
parce qu’elle n’est ni contingente ni réversible, nous ouvre à un univers où
l’idée même de succession perd de sa pertinence ou, en tout cas, de sa portée.
Entre Ptolémée et Einstein, ce n’est pas la vérité qui a changé ; c’est la connaissance
que nous en avons. La vérité, elle, ne change pas, même quand elle est la
vérité d’un univers où tout change. C’est ce qu'avait vu Spinoza : toute vérité
est éternelle, et elle seule. C’est ce qu'avait vu Pascal : quelque respect qu’on ait
pour les Anciens, expliquait-il, « la vérité doit toujours avoir l’avantage,
quoique nouvellement découverte, puisqu’elle est toujours plus ancienne que
toutes les opinions qu’on en a eues, et que ce serait ignorer sa nature que de
s’imaginer qu’elle ait commencé d’être au temps qu’elle a commencé d’être
connue » ({\it Sur le traité du vide}, p. 232 b). C’est aussi ce qu'avait vu Frege, par
%— 616 —
d’autres voies. La vérité n’a pas besoin d’être connue pour être vraie (« elle n’a
besoin d’aucun porteur »), et c’est en quoi « l’être vrai d’une pensée est indépendant
du temps » ({\it Écrits logiques}, p.184 et 191). Soit un fait éphémère
quelconque : j'écris l’article « Vérité » de mon {\it Dictionnaire philosophique} ; ou
bien, c'était l'exemple de Frege : il y a devant ma fenêtre un arbre couvert de
feuillage vert. Rien de cela ne durera bien longtemps. Mais jamais la vérité qui
s’y dévoile ne deviendra fausse ou mensongère. Dès lors qu’il est vrai que cet
arbre, ici et maintenant, est vert, cette vérité est éternelle : que cet arbre soit
vert, en ce moment que je dis, cela sera vrai encore quand il aura perdu ses
feuilles ou sera mort. C’est en quoi le présent, dans {\it « est vrai »}, n’indique pas
« le présent de celui qui parle, remarque Frege, mais, si l’on permet l’expression,
un {\it tempus} de l’intemporalité » ({\it ibid.}, p. 193). Bref, toute vérité est éternelle,
quoique aucune connaissance ne le soit, et c’est ce qui interdit de
confondre les connaissances (toujours historiques) et les vérités (toujours éternelles)
sur quoi elles portent.

Où veux-je en venir ? À ceci que renoncer à la vérité c’est renoncer à
l'éternité en même temps qu’à l’être, ce qui nous sépare du monde même où
nous sommes et du seul lieu possible du salut. Il n’y a que la connaissance qui
libère et qui sauve. À la gloire d’Épicure et de Spinoza : l'éternité c’est maintenant,
le salut c’est le monde, mais pour autant seulement que nous l’habitons
en vérité.

Quant aux raisons pratiques du discrédit actuel de l’idée de vérité, elles
tiennent à mon sens à l'impossibilité où nous sommes, au moins depuis Hume,
de combler l'écart qui sépare l’être du devoir-être, le vrai du bien, disons les
vérités des valeurs. Sur ce point, fort débattu, je ne transige guère. Si la vérité
est l’être ({\it alèthéia}) ou l'adéquation à l’être ({\it veritas}), je ne vois pas comment elle
pourrait le juger ou porter sur ce qui doit être. C’est où Hume et Spinoza se
rejoignent, malgré tout ce qui les oppose, et je n’ai jamais pu, sur ce point
majeur, m’éloigner d’eux. Une vérité est l’objet au moins possible d’une
connaissance ; une valeur, l’objet au moins imaginaire d’un désir. Cela nous
introduit dans deux ordres différents — l’ordre théorique, l’ordre pratique —, qui
ne pourraient être conjoints qu’en Dieu ou dans un sujet transcendantal. Mais
je ne crois ni à l’un ni à l’autre. Est-ce à dire que nous sommes voués à la
schizophrénie ? Nullement, puisque nous pouvons désirer le vrai et connaître
nos désirs, au moins partiellement, puisque nous ne cessons d’y tendre — fût-ce
pour constater, comme presque toujours, l’abîme qui les sépare. C’est ce qui
nous fait hommes et qui nous voue à la philosophie. Le contraire de cette schizophrénie,
qui serait autrement le lot de notre époque, c’est l'amour de la
vérité, qui est à la fois une vertu morale et une exigence intellectuelle.

%— 617 —
VÉRITÉ ÉTERNELLE Elles le sont toutes. C’est donc un pléonasme, mais
utile par ce qu’il souligne. Ce qui est vrai aujour-
d’hui le sera encore demain, ou bien ne l'était pas aujourd’hui. Il y a trois arbres
dans le champ : vérité éternelle. Dans dix mille ans, ces arbres n’y seront plus,
ni le champ sans doute ; mais il sera vrai toujours qu’ils y furent. L’éternité est
ainsi ce qui distingue le {\it vrai} du {\it réel} (ou le temps, ce qui distingue le {\it réel} du
{\it vrai}). Car le réel change, dans le temps : trois arbres, un champ, puis plus
d'arbres, puis plus de champ... On ne se baigne jamais deux fois dans le même
fleuve réel. Mais qui une fois s’y est baigné, éternellement cela restera vrai. Les
hommes passent, et les fleuves, et le réel. La vérité ne passera pas. Le vrai est
ainsi l'éternité du réel (ce pourquoi ils coïncident dans le présent) : c’est le réel
{\it sub specie aeternitatis}. On serait tenté de dire : et le réel, {\it l'image mobile} du vrai.
Mais ce serait s’enfermer déjà dans le platonisme. Ce qu’il faut comprendre ici,
et qui est la grande difficulté, c’est que le vrai et le réel sont en réalité (en vérité)
la même chose : parce que le temps n’est rien que le présent, qui est l'éternité
même.

VERTU La vertu est l'effort pour se bien conduire, qui définit le bien par cet
effort même. Ce n’est pas l'application d’une règle qui lui préexiste-
rait, encore moins le respect d’un interdit transcendant : c’est la réalisation, à la
fois normée et normative, d’un individu, qui devient à lui-même sa propre
règle en ne s’interdisant que ce qu’il juge indigne de ce qu’il est ou veut être.

Le mot {\it arétè}, que les latins traduisaient par {\it virtus}, signifie d’abord une puissance
ou une excellence. Par exemple la vertu d’un couteau est de couper, la
vertu d’un médicament de soigner, et la vertu d’un être humain de vivre et
d’agir humainement. C’est où l’on rencontre la vertu morale ou éthique. C’est
une puissance, mais normative. C’est une excellence, mais en acte. C’est une
disposition acquise (on ne naît pas vertueux, on le devient) à faire le bien, disait
Aristote, c’est-à-dire à faire ce qu’on doit, quand on le doit, comme on le doit,
mais sans autre guide, et presque sans autre règle, que sa vertu même. Cela ne
va pas sans raison, mais pas non plus sans volonté. Cela ne va pas sans effort,
mais pas non plus sans plaisir ou sans joie. Celui qui donne sans plaisir n’est pas
généreux : ce n’est qu’un avare qui se force. Celui qui résiste à la débauche sans
plaisir n’est pas tempérant : il n’est que continent et frustré.

On sait qu’Aristote définissait la vertu comme juste milieu (ou médiété) entre
deux extrêmes opposés mais tous les deux vicieux (quoiqu’ils puissent l'être inégalement),
« l'un par excès et l’autre par défaut » ({\it Éthique à Nicomaque}, II, 5-6,
1106 b — 1107 a). Ainsi le courage, entre la témérité et la couardise : le téméraire
prend des risques inconsidérés (il pèche par excès), le couard n’en prend pas assez
%— 618 —
(il pèche par défaut) ; l'homme courageux prend les risques qu’il faut, comme il
faut, quand il faut.-On se trompe évidemment si l’on y voit l’apologie de la tiédeur,
du centrisme ou de la médiocrité. Le juste milieu est un extrême, lui aussi,
mais vers le haut : c’est un sommet, c’est une perfection ({\it ibid.}), comme une ligne
de crête entre deux abîmes, ou entre deux marais.

« Par vertu et puissance j'entends la même chose, écrit Spinoza : la-vertu, en
tant qu’elle se rapporte à l’homme, est l’essence même ou la nature de l’homme
en tant qu’il a le pouvoir de faire certaines choses se pouvant connaître par les
seules lois de sa nature » ({\it Éthique}, IV, déf. 8 ; voir aussi la démonstration de la
prop. 20). C’est une occurrence du {\it conatus}, et sa forme spécifiquement
humaine. La vertu, c’est la puissance de vivre et d’agir humainement, au sens
normatif du terme, c’est-à-dire « sous la conduite de la raison » (IV, 37, scolie)
et conformément au « modèle de la nature humaine » (IV, Préface) que nous
nous sommes fixé. La raison n’y suffit pas : ce n’est pas elle qui fait agir, mais
le désir. Le désir n°y suffit pas : encore faut-il désirer la raison ou (cela revient
au même) la liberté, et en être capable. Ainsi le désir de vertu (comme puissance,
non comme manque) est la vertu même, mais en tant seulement qu’il
agit. Le {\it conatus} est « la première et unique origine de la vertu» (IV, 22,
corollaire) : c’est tendre vers son bien (IV, 18, scolie), qui est aussi celui de
l'humanité (IV, 36-37), et le réaliser par là (IV, 73, scolie). La vertu est un
effort réussi : c’est la puissance en acte, en vérité et en joie.

VEULERIE Complaisance de soi à soi. C’est se résigner trop vite à sa propre
médiocrité, au point de ne plus la voir, au point de la prendre
pour une espèce de vertu. Cela vaut mieux que la honte ? Je ne sais, ou plutôt
je n’en crois rien. La honte est une souffrance, mais qui peut faire avancer (voir
Spinoza, {\it Éthique}, IV, 58, scolie). La veulerie serait plutôt un confort, qui freine
et enferme. Le veule est incapable de se dominer, de se commander, de se surmonter.
Il s'aime comme il est, mais en oubliant cette puissance en lui — la
volonté — qu’il est aussi, et qu’il doit être. « Là où ça était, disait Freud, je dois
advenir. » Le veule se croit déjà advenu, comme d’autres se croient déjà arrivés.
Il prend son {\it moi} pour un destin, au lieu d’y voir un enjeu, un combat, une
tâche. Narcissisme mou, ou mollesse narcissique. J’y vois l’un des péchés capitaux,
et le contraire de Pexigence.

VICE Le contraire de la vertu, qu’elle surmonte et contre quoi elle se définit :
c’est une disposition au mal, comme la vertu est une disposition
au bien.
%— 619 —
Aristote nous a habitués à penser que les vices allaient par deux, qui s’opposent
l'un à l’autre et tous les deux — l’un par excès, l’autre par défaut — à une
même vertu. Ainsi la témérité et la couardise, la vanité et la bassesse, la prodigalité
et l’avarice : c’est contre l’un et l’autre de ces deux vices opposés que le
courage, la grandeur d’âme et la libéralité instaurent un juste milieu (une
médiété : {\it mésotès}) qui est aussi un sommet.

Ce n’est que par un triste contresens, qui a moins à voir avec la morale
qu'avec la religion ou la pudibonderie, qu’on a pu faire du vice une espèce de
synonyme péjoratif de la sexualité, ou considérer la sexualité, cela revient au
même, comme intrinsèquement vicieuse. Les Grecs, qui aimaient le corps et
le plaisir, n’avaient aucune raison de diaboliser le sexe. Simplement ils ne
pouvaient envisager qu’on lui soumette le tout d’une existence ou qu’on en
fasse, comme certains aujourd’hui et tout aussi absurdement, une exigence
suprême. Ni l’obsédé sexuel ni le pudibond, ni le débauché ni le peine-à-jouir
ne sont enviables ou admirables. Les uns pèchent par excès, les autres par
défaut de sensualité. Entre ces deux abîimes ou ces deux marécages, reste à
inventer la ligne de crête du plaisir cultivé, maîtrisé, partagé — la vertu des
amants.

VIE La plus belle définition que j'en connaisse est celle de Bichat : « La vie
est l’ensemble des fonctions qui résistent à la mort » ({\it Recherches physiologiques},
I, 1). C’est une occurrence du {\it conatus}, mais spécifique : une certaine
manière, pour un être donné, de persévérer dans son être en le développant
(croissance), en le reconstituant (par des échanges avec son milieu : nutrition,
respiration, photosynthèse.….), en s’adaptant, enfin en tendant à se reproduire
(génération). Vivre, c’est faire l’effort de vivre : {\it le dur désir de durer} est le vrai
goût en nous de la vie, et le principe, montre Spinoza, de toute vertu ({\it Éthique},
IV, prop. 21, 22 et corollaire).

Le mot désigne aussi la durée de cet effort — ce qui sépare la conception de
la mort. Une vie vaut moins par cette durée, pourtant, que par ce qu’on en fait.
Du moins en va-t-il ainsi pour la plupart des humains : le bonheur, non la longévité,
est le but ; l'humanité, non la santé, est la norme. C’est où l’on s'éloigne
de Bichat ou de la biologie pour retrouver Montaigne et la philosophie. « La
mort est le bout, non le but de la vie ; c’est sa fin, son extrémité, non pourtant
son objet. Elle doit être elle-même à soi sa visée : son dessein, sa droite étude
est se régler, se conduire, se souffrir » ({\it Essais}, III, 12). Apprendre à mourir ? À
quoi bon, puisqu'on y parviendra de toute façon ? Mais apprendre à vivre :
c’est la philosophie même.

%— 620 —
VIEILLESSE Le vieillissement est l’usure d’un vivant, laquelle diminue ses
performances (sa puissance d’exister, de penser, d’agir..) et le
rapproche de la mort. C’est donc un processus, dont on remarquera qu'il est
moins une évolution qu’une involution, moins un progrès qu’une dégradation,
moins une avancée qu'un recul. La vieillesse est l’état qui résulte de ce
processus : état par définition peu enviable (qui ne préférerait rester jeune ?), et
pourtant, pour presque tous, préférable à la mort. C’est que la mort n'est rien,
quand la vieillesse est encore quelque chose.

Je ne crois guère aux avantages de la vieillesse, encore moins à sa valeur ou
grandeur (malgré Hugo) intrinsèques. Qu’on puisse progresser en vieillissant,
chacun peut le constater autour de soi, comme il peut constater que cela reste
toutefois l’exception. Même alors, d’ailleurs, ce n’est pas grâce à la vieillesse
qu’on progresse ; c’est malgré elle, et contre elle bien souvent. Un gain d’expérience,
de maturité, de culture ? On le doit moins à la vieillesse qu’à la vie, qui
continue malgré tout, moins à l’usure qu’à la résistance, moins à l’âge qu'on a
qu'aux années qui y menèrent, où on ne l’avait pas. La vie est une richesse. Le
temps est une richesse. Être vieux, non : ce n’est que le temps qui manque et la
vie qui s’en va. Qu'on ait davantage d’expérience à soixante-dix ans qu'à vingt,
c'est une donnée de fait, que l’arithmétique suffit à expliquer, mais qui n’est
pas liée directement à la vieillesse : si nous étions programmés autrement par
nos gènes, nous pourrions avoir soixante-dix ans sans être vieux pour autant, de
même que nous pourrions être vieux, comme dans beaucoup d'espèces animales,
dès quinze ou vingt ans. On se trompe quand on réduit la vieillesse à un
âge : que les deux, en fait, aillent presque toujours de pair, cela n’empêche pas
qu’il s’agisse, en droit, de deux réalités différentes. On sait qu’il existe quelques
pathologies très rares qui entraînent une accélération du vieillissement, jusqu’à
faire un vieillard d’un homme de trente ans. Et nul n’ignore qu’il y a des octogénaires
presque intacts, plus verts et ouverts que bien des jeunes gens. C'est
qu’ils ne sont pas encore vieux, ou moins que leur âge ne le laisserait supposer.
Toutefois ces exceptions heureuses ne doivent pas cacher la règle qu’elles
confirment : chez presque tous, le temps, à partir d’un certain âge, entraîne une
dégradation irréversible, qu’on peut parfois ralentir mais qu’on ne saurait
empêcher. Que ce soit physiquement ou intellectuellement, la plupart sont
moins performants à quarante ans qu’à vingt, à soixante qu'à quarante, à
quatre-vingts qu'à soixante... C’est une espèce d’entropie à la première
personne : dans un organisme vivant, passé le cap de la maturité, le désordre et
la fatigue tendent vers un maximum. Le vieillissement est cette tendance ; la
vieillesse, son résultat. Cela n’empêcha pas Kant d’écrire la {\it Critique de la faculté
de juger} à soixante ans passés, ni Hugo, à quatre-vingts ans, de garder le génie
et la vitalité que l’on sait. Mais ils le doivent plus à leur santé qu’à leur vieillesse.
%— 621 —
Et plus à la chance qu’au génie. Montaigne, qui vécut plutôt vieux pour son
époque, qui ne se mit à écrire qu'assez tard, ne s’est jamais illusionné sur les
bénéfices de la vieillesse :

« Je hais cet accidentel repentir que l’âge apporte. Celui qui disait anciennement
être obligé aux années de quoi elles l'avaient défait de la volupté, avait autre opinion
que la mienne ; je ne saurai jamais bon gré à l’impuissance de bien qu’elle me fasse.
[...] Je serais honteux et envieux que la misère et défortune de ma décrépitude eût à se
préférer à mes bonnes années saines, éveillées, vigoureuses ; et qu’on eût à m’estimer
non par où j'ai été, mais par où j'ai cessé d’être. [...] La vieillesse nous attache plus de
rides en l'esprit qu’au visage ; et ne se voit point d’âmes, ou fort rares, qui en vieillissant
ne sentent à l’aigre et au moisi. L'homme marche entier vers son croît et son décroît »
({\it Essais}, III, 2).

Celui-là aimait trop la vie et la vérité pour dire du bien de la vieillesse. Il se
contentait de l’accepter sereinement. Cela me paraît, concernant la vieillesse,
une ambition suffisante. La mort ramasse les copies, mais ne les note pas.

VIOLENCE C'est l’usage immodéré de la force. Elle est parfois nécessaire (la
modération n’est pas toujours possible), jamais bonne. Toujours
regrettable, pas toujours condamnable. Son contraire est la douceur
(qu’on ne confondra pas avec la faiblesse, contraire de la force). La douceur est
une vertu ; la faiblesse, une faiblesse ; la violence, une faute — sauf quand elle
est indispensable et légitime. Contre les faibles ou les doux, la violence est
impardonnable : ce n’est que lâcheté, cruauté, bestialité. Contre les violents, en
revanche, on ne peut se l’interdire absolument : ce serait laisser libre cours aux
barbares ou aux voyous. La non-violence ? Elle n’est bonne, souligne Simone
Weil, que si elle est efficace. Cela indique assez le but et le chemin : « S’efforcer
de substituer de plus en plus dans le monde la non-violence {\it efficace} à la
violence » ({\it La pesanteur et la grâce}, « Violence »). Cela suppose beaucoup de
maîtrise, de courage, d'intelligence, mais « dépend aussi de l’adversaire » ({\it ibid.}).
Gandhi, contre les Anglais, est admirable. Mais cela ne donne pas tort aux
Résistants, contre les nazis, ni aux Alliés, contre la Wehrmacht. La violence
n’est acceptable que lorsque son absence serait pire. Elle l’est donc parfois.
Reste à la limiter, à la contrôler, à l’encadrer. C’est pourquoi on a besoin d’un
État, pour exercer, comme disait Max Weber, « le monopole de la violence
légitime » : on a besoin d’une armée (pour se défendre contre la violence extérieure),
d’une police (contre la violence intérieure), de lois, de tribunaux, de
prisons. Et besoin aussi, entre les individus, d’une paix au moins minimale.
Le contraire de la violence, c’est la douceur ; mais son antidote, à l'échelle de la
%— 622 —
Cité, c’est l’art de gérer les conflits avec le moins de violence possible : police,
politesse, politique.

VIRTUEL Ce qui n’existe qu’en puissance (mais mieux vaut dire alors {\it potentiel})
où qu’en simulation. Le mot, dont on nous rebat les oreilles,
vient de {\it virtus} : c’est comme un doublet de notre {\it vertu}, et point par hasard. Il
y a puissance dans les deux cas. Mais la vertu est une puissance en acte. La
virtualité, une puissance qui reste en puissance. La vertu est puissance incarnée,
quand le virtuel, trop souvent, se contente des images. La vertu est de l’homme,
quand le virtuel, de plus en plus, appartient aux machines. La vertu est courage,
quand le virtuel ne sait, au mieux, qu'être sans danger. La vertu est justice, quand
le virtuel ne connaît, au mieux, que la justesse. La vertu est aimante, quand le virtuel
ne sait, au mieux, qu'être aimable.
C’est donc la vertu qui est bonne, et qui importe. Comme il serait triste et
coupable de ne vivre que virtuellement !

VITALISME C'est expliquer la vie par elle-même (ou par un « principe
vital »), donc renoncer à l'expliquer. S’oppose en cela au matérialisme,
qui explique la vie par la matière inanimée, et se distingue de l’animisme,
qui l’explique par une âme immatérielle.

VOLITION L'acte de vouloir. Cela suppose un désir, mais ne s’y réduit pas
(toute volition est désir, tout désir n’est pas volition) : vouloir,
c'est désirer en acte. C’est pourquoi nous ne pouvons vouloir que ce qui
dépend de nous, et à condition seulement de le faire. Essayez un peu de vouloir
vous lever sans vous lever en effet. Il faudrait être paralytique ou ligoté ; mais
alors vous lever, pour vous, ne serait plus une volition, mais un simple désir,
voire un regret ou une espérance. Vouloir, c’est faire. Une volonté qui n’agit
pas n’est plus une volition, ni même tout à fait une volonté : c’est un projet, un
vœu, ou une lâcheté.

VOLONTÉ La faculté de vouloir : l’acte en puissance ou la puissance en
acte.
On ne la confondra pas avec le désir, qui est son genre prochain. On peut
désirer simultanément plusieurs choses contradictoires (par exemple fumer et
ne pas fumer), mais point les vouloir : parce qu’on ne veut vraiment que ce
%— 623 —
qu’on fait, et que nul ne peut, au même moment, faire et ne pas faire la même
chose. La volonté est une certaine espèce de désir : c’est un désir dont la satisfaction
dépend de nous. « Mais si j’échoue ? » Cela n’y change rien : la volonté
portait sur l’action, non sur la réussite (qui n’était l’objet que d’une espérance).
Toute volonté est puissance de choix : c’est le pouvoir déterminé de se déterminer
soi-même. Cela distingue assez la volonté du libre arbitre (qui serait le
pouvoir {\it indéterminé} de se déterminer soi), de l’espérance, qui désire plus qu’elle
ne peut, enfin de la veulerie, qui renonce à choisir. Par quoi la volonté n’est pas
seulement une faculté ; c’est aussi une vertu.

VRAI Ce qui est, ou ce qui est conforme à ce qui est. On peut distinguer ces
deux sens, en parlant respectivement de {\it veritas rei} et de {\it veritas intellectus},
comme faisaient les scolastiques (la vérité de la chose, la vérité de l’entendement),
ou bien en distinguant avec Heidegger l’{\it althéia} (la vérité comme
dévoilement de l’étant, ce que j’appellerais plutôt la pure {\it présentation} du réel)
et la {\it veritas} (la vérité comme accord ou correspondance entre la pensée et le
réel : l’{\it ad{\ae}quatio rei} et {\it intellectus} des scolastiques, qui n’est vérité que d’une
{\it représentation}). L'{\it alèthéia} serait la vérité originaire, telle qu’on la trouve chez les
présocratiques. La {\it veritas} n’apparaîtrait, même en grec, qu'avec Platon : ce
serait la forme, à la fois logique et métaphysique, de l'oubli de l’être, au bénéfice
de l’humanisme (le vrai n'étant plus vérité de l’étant mais vérité de
l’homme). Ces deux conceptions n’en restent pas moins solidaires, et même
indissociables. Soit par exemple la table sur laquelle j'écris. Je ne peux rien
penser de son {\it alèthéia} sans passer par la {\it veritas} d’un discours. Mais pas davantage
dire sa {\it veritas} sans supposer son {\it alèthéia}. Supposons que j'énonce à son
propos un certain nombre de propositions vraies, fussent-elles approximatives,
portant par exemple sur sa forme, sa superficie ou son poids. Ces propositions
sont vraies (au sens de la {\it veritas}) si et seulement si elles correspondent à la réalité
(à l'{\it alèthéia}). C’est ce que Tarski appelle la conception sémantique de la
vérité : la proposition « La neige est blanche » est vraie si et seulement si la
neige est blanche ; la proposition « Cette table est rectangulaire » est vraie si et
seulement si cette table est rectangulaire. Mais si la neige et cette table n’étaient
{\it vraiment} ce qu'elles sont, cette adéquation elle-même n’aurait ni sens ni vérité.
Une pensée ne peut être conforme à ce qui est (c’est-à-dire vraie au sens de la
{\it veritas}) que si ce qui est est vraiment ce qu'il est (c’est-à-dire vrai au sens de
l’{\it alèthéia}). Les deux notions, ou les deux faces de la notion, restent pourtant
toutes les deux problématiques, mais pour des raisons différentes. L’{\it alèthéia},
parce qu’on ne peut rien en dire qui ne relève de la {\it veritas}. La {\it veritas}, parce que
toute conformité entre la pensée et le réel est par définition indémontrable,
%— 624 —
puisqu’on ne connaît du réel que ce qu’on en pense. Cela ne prouve pas que
tout ce que nous pensons soit faux, mais nous interdit de prouver absolument
que telle ou telle de nos pensées soit vraie. À la gloire du pyrrhonisme.

VULGARITÉ Une bassesse commune et de mauvais goût.
{\it Vulgus}, en latin, c’est la foule, le commun des hommes, les
hommes du commun. {\it Vulgaire} serait donc synonyme à peu près de {\it populaire},
et le fut en effet longtemps. Mais le peuple est souverain ; la foule, non. De là
peut-être, dans un univers démocratique, l’évolution de plus en plus divergente
des deux mots. Être vulgaire, ce n’est pas être du peuple, ni apprécié par lui ;
c’est manquer d’élévation, d'élégance, de distinction, de noblesse. La popularité
est une chance ou un risque. La vulgarité, un confort et une tare. On se trompe
quand on la confond avec la grossièreté (on peut dire des gros mots sans être
vulgaire, être vulgaire sans en dire). Mais plus encore si on y voit une audace ou
une force. Ce n’est que suivre la pente, en allant toujours vers Le plus facile, le
plus bas, le plus racoleur : c’est ne plaire qu’à la partie déplaisante de soi, et de
tous. La notion, toutefois, relève davantage de l’esthétique que de la morale.
Un brave homme peut être vulgaire ; un salaud ne l’être pas. C’est que la vulgarité
touche moins aux actes qu’aux manières, moins aux sentiments qu'à la
sensibilité. Être vulgaire, c’est presque toujours ignorer qu’on l’est. C’est être
prisonnier de sa propre bassesse, au point qu’on ne la perçoit plus. C’est être
une foule à soi tout seul. Ce serait un péché capital, si c'était un péché. Mais ce
n’est qu’une faute de goût.

WAGNÉRIEN Disciple ou zélateur de Wagner. C’est une forme redoutable
de mélomane, qui prend la musique pour une conception du
monde, l’opéra pour une religion, et Wagner pour un Dieu. Ces trois erreurs
font une espèce de système, qui les rend sourds. À moins que ce ne soit
l'inverse.

Nietzsche a écrit contre cette {\it maladie}, comme il dit, et contre le génie
subtil et dangereux de Wagner, quelques-unes de ses plus belles pages, qui rendent
à Mozart (« le génie gai, enthousiaste, tendre et amoureux de Mozart »:
l'hommage qu’il mérite — et à Bizet, bien davantage qu’il ne mériterait.

{\it WELTANSCHAUUNG} Vision du monde, en allemand. C’est une espèce de
philosophie spontanée ou implicite : un ensemble
d’intuitions, de croyances, d’idées vagues, avec, dans la bouche d’un Français,
%— 625 —
je ne sais quoi de prétentieux et d’obscur, qui tient à la langue utilisée, comme
s’il suffisait de parler allemand pour être plus intelligent. La philosophie du
pauvre ? Plutôt l’idéologie du riche, ou du snob.

XÉNOPHOBIE La haine de l'étranger. C’est une forme de bêtise qui consiste
à se croire chez soi. La chose semble instinctive aux bêtes, et
naturelle aux hommes. La philosophie, qui nous fait tous étrangers, combat
cette illusion. Et la sagesse, qui suppose le dépassement du {\it soi}, la dissout. La
haine s’éteint alors, en même temps que la peur.

YOGA Travail du corps, d’origine indienne, qui tend au repos de l'esprit.
Ascèse, qui tend à la délivrance. Méditation, qui tend au silence.
C’est faire passer la bête sous le joug (les deux mots ont la même racine indoeuropéenne),
mais pour la libérer, ou se libérer soi. C’est n’être qu’un avec son
corps, pour n'être qu’un avec tout. Le yoga est ainsi l’homologue fonctionnel de
la philosophie, qui tend au même résultat mais travaille plutôt sur l'esprit ou les
concepts. Le yogi croit davantage aux postures, aux mouvements, à la respiration,
à la concentration, à l'attention absolument pure. Le but est de s’affranchir du
mental pour atteindre à la conscience absolue ou inconditionnée. L'efficacité sur
l’âme d’une telle pratique corporelle et spirituelle n’est plus à démontrer, ni très
mystérieuse. Si le corps et l’âme sont une seule et même chose, comme dit Spinoza,
le yoga n’est jamais qu’une façon, historiquement située, de penser juste. Il
semble, grande leçon, que le penseur s’y perde, et s’y sauve.

ZÈLE C’est un soin jaloux ou fervent, pour la cause d’un autre, comme si
l’on craignait de n’en faire jamais assez — au point parfois (par exemple
dans les grèves du zèle) d’en faire trop. « Le zèle consiste à faire plus qu’on ne
doit strictement », comme disait Alain, voire plus qu’on ne devrait. C’est que
la frontière, entre le zèle et l’excès de zèle, reste floue. En faire tant, n’est-ce pas
déjà en faire trop ? Et pour quelle raison ? Par générosité ? Par dévouement ?
Par conscience professionnelle ? C’est plus souvent une peur d’être blâmé ou
un désir de plaire, qui rendent le zèle, même efficace, un peu suspect. Les
patrons n’en sont pas dupes. Les collègues, encore moins.

ZEN C'est une forme de bouddhisme, relevant du Grand Véhicule et dérivée
du Tch’an chinois, qui s’est développée au Japon. On y cherche
%— 626 —
l’illumination (Le {\it satori}) par la méditation assise et sans objet ({\it zazen}), laquelle
peut elle-même être préparée, ou accompagnée, par un certain nombre d’exercices
(les {\it kôans}, le tir à l'arc, l’art floral, les arts martiaux...). Le but est
d'atteindre une attention absolument pure, qui crée, ou plutôt qui laisse se
déployer, un état de paix et de vide intérieur. Ceux qui l’ont vécu en parlent
comme d’une expérience de plénitude. Il s’agit d’observer, de façon neutre et
tranquille, son propre fonctionnement, aussi troublé soit-il, jusqu’à expérimenter
qu’il n’y a rien là de substantiel à observer (que tous ces processus sont
impermanents et vides). Le réel n’en continue pas moins, ou plutôt il n’en
continue que mieux — parce qu’on n’en est plus séparé par l’ego. C’est se vider
de soi, pour qu’il n’y ait plus que tout.

ZÉTÉTIQUE Du grec {\it zétêtikos}, qui cherche ou qui aime chercher. C’est un
autre nom pour désigner le scepticisme, ou plutôt sa méthode,
qui consiste à chercher toujours la vérité, sans rien affirmer, même pas l’impossibilité
de l’atteindre (voir Sextus Empiricus, {\it Hypotyposes pyrrhoniennes}, I et
III). Se distingue par là du dogmatisme, qui croit avoir trouvé, mais aussi de la
sophistique, qui renonce à chercher.

À quoi bon chercher, demandera-t-on, si on ne peut trouver ? C’est qu’on
ne peut savoir autrement si on le peut, ni ce qu’on cherche.

Et comment le dire, si on ne l’a point trouvé ? En disant au moins le mouvement
de sa quête, sans l'arrêter et sans y croire tout à fait. Pyrrhon, qui est le
maître de l’école zététique, était amené pour cela à ne parler qu’ironiquement,
ou plutôt (puisqu'il s’agit, note Marcel Conche, d’une « ironie à l’égard de lui-même »)
qu'avec humour. Les mots ne sont qu’un moment de l'apparence pure
et universelle, comme dit Marcel Conche, qu’un moment du devenir, comme
je préférerais dire, qu’ils ne sauraient ni contenir tout entier ni transformer — si
ce n’est illusoirement — en essences fixes ou immuables. Aussi l’{\it ataraxia} (la
sagesse, la paix de l’âme) n’allait-elle pas, pour Pyrrhon, sans {\it aphasia} (le non-discours,
le silence). Non qu’on ne puisse parler ({\it aphasia} n’est pas aphasie), ni
qu’on ne le doive, mais parce que « les mots jamais ne sauraient annuler le
silence » (M. Conche, {\it Pyrrhon ou l'apparence}, X, 1).

Il est juste de terminer par là un recueil de définitions. Car cela seul mérite
d’être dit, qui n’en a pas besoin. C’est où la philosophie, qui est un certain type
de discours, conduit à la sagesse, qui est une certaine qualité de silence. Ce dont
on peut parler, et cela seul, on peut aussi le taire.


%
%%
\begin{appendix}
%

%%%%%%%%%%%%%%%%%%%%%
\chapter{La vie}
%%%%%%%%%%%%%%%%%%%%%
%{\it }{\oe}

David Hume est né le 26 août 1711 à Édimbourg
où son pére exerçait la profession d’avocat. Ce
dernier étant mort en 1714, Mrs Hume se retira
avec ses trois enfants John, Katherine et David
dans la propriété familiale de Ninewells, le domaine
des « Neuf Sources » situé dans la pittoresque campagne
(avec ses falaises, ses ruisseaux et ses bois)
du Berwickshire. L’oncle de David, pasteur du
village voisin de Chirnside, dirigea sa toute première
éducation. L’enseignement religieux que reçut le
jeune David semble avoir été particulièrement
austère et maladroit (le Révérend George Hume
se plaisait, dans ses sermons, à humilier publiquement
les jeunes filles dont la grossesse révélait les
péchés charnels). L’antipathie précoce de Hume
pour le christianisme vient en partie de là.

Cependant le petit David échappa assez rapidement
à cette atmosphère déprimante. Élève dès
l'âge de onze ans du collège d’Édimbourg (renommé
à juste titre et qui deviendra plus tard université),
il se trouve dans une ambiance intellectuelle beaucoup 
plus stimulante. Il y écoute les cours de « philosophie
naturelle », c’est-a-dire de physique, de
Robert Stewart (disciple de Newton aprés avoir été
cartésien) et se souviendra certainement de ses
%6
%{\it }{\oe}
leçons lorsqu’il rêvera d’appliquer la méthode expérimentale
à la morale et à la métaphysique. Mais
la formation de Hume au collège fut essentiellement
littéraire. C’est de cette époque que date son goût
de Virgile et de Cicéron. Ce sont les textes (notamment
le {\it De Natura Deorum}) où Cicéron résume les
débats philosophiques des Stoïciens et des Epicuriens
qui découvrent à Hume le monde des discussions
métaphysiques. Revenu à Ninewells dès sa quinzième
année, le jeune David se livre avec passion à la
lecture des anciens et des modernes. Il dévore
Montaigne, Bacon, Malebranche, Bayle, mais aussi
Milton, Pope, Swift, Shaftesbury. À l’âge de vingt
ans, il a déja rempli un gros cahier de réflexions sur
le problème religieux, sur la psychologie, sur l’histoire.
Cette activité intellectuelle bouillonnante, les
leçons de scepticisme qu’il tire de lectures aussi
diverses, un conflit avec sa famille qui voudrait,
malgré lui, l'orienter vers des études juridiques
provoquent une crise de dépression passagère dont
nous trouvons le témoignage dans un curieux brouillon
de lettre à un médecin célèbre (qui n’est pas
comme on l’a cru George Cheyne, mais le D$^\text{r}$ Arbuthnot).
David qui n’a hérité de son pére que d’une
toute petite rente doit de toute urgence prendre
un état. Après un bref essai dans le commerce (au
service d’un marchand de Bristol), David Hume
décide de ne plus résister à sa vocation : Il sera
philosophe et homme de lettres, et il entend conquérir
la gloire. Pour pouvoir subsister, il se rend en France
(où la vie est à l’époque beaucoup moins chère),
s’installe à Reims en 1734 à l'hôtel du {\it Perroquet
vert}, puis à La Flèche en Anjou, tout près du collège
%7
%{\it }{\oe}
de Jésuites ou Descartes fut élève. Il y rédige, à
peine âgé de vingt-trois ans, son chef-d’{\oe}uvre :
le {\it Traité de la nature humaine}.

Revenu à Londres en 1737, il a la chance de trouver
un éditeur, et la prudence (ou si l’on veut la faiblesse)
de supprimer les chapitres sur la religion
(il espère la protection de l’évèque Butler). Les deux
premiers livres du traité ainsi publiés « tombérent
mort-nés de la presse », raconta plus tard Hume dans
sa courte {\it Autobiographie}. Ce n’est pas tout à fait
vrai. En fait l’{\oe}uvre intéressa quelques critiques,
mais n’atteignit pas le grand public (qui seul donne
la notoriété). A cette époque, Hume entre en relations
avec Hutcheson, professeur à Glasgow, qui
lui présente son jeune étudiant Adam Smith (qui
restera toujours l’ami de Hume) et lui trouve un
éditeur pour les deux livres suivants du {\it Traité de
la nature humaine} dont le succès n’est pas plus
grand. Hume cependant ne doute pas de sa valeur.
Son échec vient de la présentation trop lourde et
trop savante de sa philosophie et non pas du fond
({\it more from the manner than the matter} dira l'{\it Autobiographie}).
Hume décide alors d’écrire des essais
courts et brillants et publie en 1741 a Édimbourg
ses {\it Essais moraux et politiques} (humilié par ses
insuccés il présente d’ailleurs cet ouvrage comme son
premier livre !). Cette fois les lecteurs sont nombreux,
et Hume croit pouvoir présenter sa candidature à
la chaire de philosophie morale de l’Université de
Glasgow. L’opposition des chrétiens empêche sa
nomination. En 1746, Hume devient le secrétaire
particulier du général Saint-Clair, un Écossais qui
est son parent éloigné, et l’accompagne dans une
%8
%{\it }{\oe}
mission diplomatique à Vienne et à Turin. Pendant
son voyage paraissent ses {\it Essais philosophiques sur
l'entendement humain} (plus tard {\it Enquête sur l’entendement
humain}), qui reprennent dans un style nouveau
les deux premiers livres du {\it Traité de la nature
humaine}. Cette fois les chapitres sur le miracle et
sur la providence particulière paraissent avec le
reste (1748). A son retour c’est le troisième livre du
Traité que Hume reprend avec l’{\it Enquête sur les
principes de la morale} (1751). Dès lors la notoriété
de Hume s’affirme. Il entre en relations épistolaires
avec Montesquieu à propos de l’{\it Esprit des lois}.
Et s'il échoue une nouvelle fois (1751) dans sa candidature
à l’Université de Glasgow, il devient conservateur
de la bibliothéque de la « Faculté des Avocats »
à Edimbourg. Il trouve là tous les documents
nécessaires pour écrire de 1754 à 1761 sa monumentale
{\it Histoire d’ Angleterre}, de Jules César à Jacques II.
Le premier volume qui traite des régnes de Jacques
I$^\text{er}$ et de Charles I$^\text{er}$ déclenche un petit scandale
dans les milieux religieux, et Hume n’améliore pas
son cas auprès des chrétiens en publiant ses {\it Quatre
dissertations} (dont son {\it Essai sur l'histoire naturelle
de la religion} et son {\it Essai sur le suicide}). Il est vrai
qu'il s’empresse de retirer l’{\it Essai sur le suicide}
ainsi qu’un {\it Essai sur l’immortalité de l’âme} pour les
remplacer par un {\it Essai sur la règle du goût}. Hume
subit aussi avec patience les tracasseries des avocats
d’Édimbourg qui lui reprochent d’avoir acheté
pour la bibliothéque les {\it Contes} de La Fontaine, et
l'{\it Histoire amoureuse des Gaules} de Bussy-Rabutin !

C’est en France que Hume devait connaître la
gloire. A l’appel de lord Hertford, ambassadeur
%9
%{\it }{\oe}
d’Angleterre à Paris, il va exercer de 1763 à 1766
les fonctions de secrétaire d’ambassade. Il sera
même quelque temps — lorsque lord Hertford
est rappelé et en attendant son successeur — « chargé
d’affaires », c’est-à-dire en fait ambassadeur, grâce
à la puissante protection de la comtesse de Boufflers.

A Edimbourg Hume inquiète, à Londres il n’est
qu’un Écossais, qu’un intellectuel provincial. A
Paris, le petit monde des philosophes qui a lu ses
Essais, qui connaît l’opposition de Hume à la
« superstition » et au « fanatisme » le tient pour un
philosophe de premier plan. Il deviendra très vite
l'ami de d’Alembert, de Diderot, d’Helvétius, du
baron d’Holbach. Certes, officiellement, Hume est
déiste, à un diner du baron d’Holbach il confesse
même n’avoir jamais rencontré d’athée (« Regardez
autour de vous, répond le baron : Il y en a quinze
autour de cette table ! »). En fait ses positions anti-religieuses,
son Essai sur les miracles le rendent
sympathique aux encyclopédistes qui le tiendront
désormais pour un « frére ». A Paris, c’est un véritable
triomphe. Les plus grandes dames s’arrachent ce
quinquagénaire bedonnant. C’est la duchesse de
La Vallière qui tient à le voir dès son arrivée à Paris,
avant même qu’il ait pu changer de costume !
C’est Mme Du Deffand, Mme Geoffrin, Mlle de Lespinasse
qui le fêtent dans leurs « salons ». même
l'apparence physique de Hume, assez ingrate (il
est obèse, son visage empâté est peu expressif)
qui lui avait valu naguére, tandis qu’il voyageait
en Italie avec le général Saint-Clair, les railleries
du jeune James Caulfield, futur lord Charlemont,
et quelques déboires amoureux, est maintenant
%10
%{\it }{\oe}
trouvée sympathique. Son visage un peu lourd, son
fort accent écossais donnent au grand philosophe
un air débonnaire, une simplicité de bon aloi
(Mme Du Deffand l'appelle « mon cher paysan »,
Mme Geoffrin « mon gros drôle, mon gros coquin »).
Hume vole de succès en succès, écrit à son ami le
D$^\text{r}$ Robertson : « Je ne mange que de l’ambroisie,
je ne bois que du nectar, je ne respire que de l’encens,
je ne marche que sur des fleurs. »

Marie-Charlotte Hippolyte de Campet de Saujeon,
comtesse de Boufflers et maîtresse en titre du prince
de Conti, avait dès 1761 écrit à Hume pour lui dire
toute son admiration pour sa philosophie « sublime ».
Elle avait tenté de le rencontrer au cours d’un voyage
à Londres, sans succés car Hume n’avait pas voulu
quitter Édimbourg. A Paris elle l'invite tout de
suite à ses lundis, puis à ses vendredis plus intimes.
Il semble que cette jolie femme de trente-cinq ans
ait été quelque peu amoureuse du gros philosophe
vieillissant. Hume la considéra toujours comme une
amie très chère (le philosophe, cinq jours avant sa
mort, lui écrit encore le 20 août 1776 pour lui
annoncer qu’il se sent perdu et lui dire une dernière
fois son « affection » et son « respect »), mais il ne
paraît pas que leurs relations aient jamais été plus
intimes. Nous connaissons mal la vie intime de Hume,
mais il semble que le philosophe, resté célibataire
(à Edimbourg, sa s{\oe}ur Katherine tenait son ménage)
se soit, à cause d’anciens déboires, ou par souci de
préserver son indépendance et son travail, toujours
méfié des passions amoureuses.

En 1766, le nouvel ambassadeur d’Angleterre, le
duc de Richmond, arrive à Paris et Hume repart
% 11
%{\it }{\oe}
en Angleterre. C’est ici que se situe un des épisodes
les plus mal éclaircis de la vie de Hume. Dès ses
lettres de 1761 la comtesse de Boufflers avait intéressé
Hume à Jean-Jacques Rousseau. Pendant le
séjour à Paris, la marquise de Verdelin demande au
philosophe écossais de chercher pour Jean-Jacques
un refuge en Angleterre. Celui-ci, proscrit de Genéve,
sa ville natale, interdit de séjour en France, persécuté
par les habitants de Môtiers-Travers, en butte à la
haine des encyclopédistes qui le tiennent pour un
dévot, est dans une des périodes les plus critiques de
son existence. Hume, ému par les malheurs de
Jean-Jacques, et tout d’abord enthousiasmé par sa
simplicité et sa franchise (Rousseau est un Socrate
moderne, dira-t-il), part avec lui le 4 janvier 1766.
Ils arrivent à Londres le 13 ou Jean-Jacques est
fété et reçoit de Hume mille témoignages d’amitié.
Mais Rousseau qui déteste le monde et cherche la
solitude n’entend rester à Londres que jusqu’à
l'arrivée de sa servante-maîtresse Thérèse Le Vasseur.
En fait c’est seulement le 19 mars 1766 qu’il part
pour Wooton, maison de campagne dans les bois du
Derbyshire qu’un ami de Hume, Davenport, avait
mise à sa disposition.

Moins de trois mois ont suffi pour altérer la belle
amitié de Hume et de Rousseau. Désormais Rousseau
considére Hume comme un traître et comme un
malhonnête homme. Que s’est-il passé ? Il est certain
que Rousseau a toujours eu un caractère soupçonneux,
une tendance paranoiaque au délire d’interprétation
que les réelles persécutions dont il fut
victime n’ont fait qu’aggraver. Avant même d’atteindre Calais,
le premier soir du voyage, un curieux
% 12
%{\it }{\oe}
incident se produit. Hume, Rousseau et Luze, un
ami suisse qui va à Londres pour ses affaires, descendent
dans un hotel de Senlis et couchent dans une
chambre à trois lits. Jean-Jacques qui cherche en
vain le sommeil entend soudain Hume dire à plusieurs
reprises « à pleine voix » et avec « une véhémence extrême » :
« Je tiens Jean-Jacques Rousseau. »
Hume n’est-il pas l’ami intime des ennemis les plus
acharnés de Rousseau, les encyclopédistes athées,
Diderot, d’Alembert et d’Holbach ? Ne s’est-il
pas mis d’accord avec cette clique pour faire de
Rousseau en quelque sorte son prisonnier ? Dès le
départ ce rêve ou cette hallucination (car Luze
n’a pas été éveillé par ces voix véhémentes) met
Rousseau sur ses gardes.

Cela suffit-il pour que nous n’accordions aucun
crédit aux accusations formulées ultérieurement par
Jean-Jacques ? Certaines assurément sont délirantes.
A Londres, Rousseau est exaspéré par les
compliments de Hume qui a toujours à son chevet
un tome de la {\it Nouvelle Héloïse}. Pure hypocrisie,
juge Rousseau, car Hume ne peut aimer ce roman !
En réalité Hume, qui fait effectivement des réserves
sur les idées de Jean-Jacques, apprécie son roman,
le tient pour le chef-d’{\oe}uvre du philosophe français
(lettre à Blair du 25 mars 1766). Mais il y a plus
grave. Hume n’ignore pas qu’à Paris son ami Walpole
a fait à Rousseau une plaisanterie trés méchante
(écrivant et faisant publier une lettre d’invitation a
Rousseau, au nom du roi de Prusse qui lui promet à
son choix faveurs ou persécutions, puisqu’il paraît
aimer les persécutions !). Or tandis que Rousseau se
préoccupe de faire venir à Londres des papiers pour
% 13
%{\it }{\oe}
la rédaction de ses {\it Confessions}, papiers restés en
France, Hume lui propose Walpole comme commissionnaire !
Hume fait faire une enquête à la banque
Rougemont sur les vraies ressources de Jean-Jacques
qui crie toujours misère. D’autre part Hume veut
toujours se charger d’expédier et de retirer de la
poste le courrier de Jean-Jacques, et celui-ci se
plaindra que son courrier lui parvienne toujours
décacheté et grossièrement refermé. Hume répondra
plus tard que s’il tenait à se charger du courrier de
Rousseau, c’est parce que ce dernier, alléguant sa
pauvreté, ne voulait payer le port d’aucune lettre
(a l’époque, c’est le destinataire qui payait) et que
lui, David Hume, ne tenait pas a laisser trop longtemps
le courrier à la poste, mais désirait le soustraire
promptement {\it from the curiosily and indiscretion
of the clerks of Post-office} !

Quoi qu’il en soit de toutes ces accusations, il
reste que Hume porte la responsabilité d’avoir mis
la querelle sous les yeux du public en laissant éditer
à ses amis français l’{\it Exposé succinct} de son différend
avec Rousseau, livrant ainsi Jean-Jacques à de
nouvelles railleries. Ce fut peut-être en cette affaire
sa seule faute, une entreprise, écrira-t-il à Adam
Smith le 17 octobre 1767, que « l’un et l’autre nous
avons été enclins a blâmer parfois, à regretter
toujours ». Si Hume ne fut pas le traître qu’a imaginé
Rousseau, peut-être manqua-t-il ici de patience et
de générosité. Une des faiblesses de Hume est d’avoir
été toujours trés préoccupé de sa réputation. Et
il craignait que Rousseau, dans ses {\it Confessions}, ne
raconte l'histoire à sa manière (en réalité le récit de
Rousseau s’arrétera avant les événements de 1766).
% 14
%{\it }{\oe}

Hume, grace à son fidèle protecteur, lord Hertford,
devient sous-secrétaire d’État en 1767. Par un
étrange retour des choses c’est lui qui, entre autres
affaires, est chargé de régler les conflits et de décider
de l’avancement des pasteurs de cette Église
d’Écosse qui naguére avait tenté d’entraver sa
carriére !

En 1769 Hume rentre à Édimbourg désormais
riche et considéré. Il choisit de finir ses jours dans
sa ville natale, intense foyer de culture, qui est bien
plutôt que Londres la capitale intellectuelle de
l’Angleterre de ce temps. A Édimbourg, cette
« Athènes du Nord », Hume retrouve en effet des
amis éminents, Adam Smith, le juriste lord Kames,
Ferguson, et aussi des adversaires loyaux et courtois
comme le théologien George Campbell qui avait
rédigé une critique de l’{\it Essai sur les miracles}, critique
fort appréciée par Hume lui-même. Hume
s’emploie à corriger ses {\oe}uvres pour de prochaines
éditions, à répondre à son courrier, et il rejoint
fréquemment ses amis écossais au {\it Poker Club} et a
la {\it Select Society}, club philosophique et littéraire
qu’il avait fondé lui-même en 1754.

Très rapidement sa santé décline. Il souffre d’une
tumeur de l'intestin, et dès le début de 1776 il se
sait perdu. En avril il rédige son testament. Il a
dans ses papiers un ouvrage inédit commencé
dès 1751 et dont il a déja soumis à cette époque les
premiers chapitres à son ami Gilbert Elliot of Minto :
{\it Les dialogues sur la religion naturelle}. Adam Smith
n'est pas très favorable à la publication de cet
ouvrage. C’est donc le neveu de Hume qui sera
chargé de cette édition posthume. {\it Les Dialogues}
% 15
%{\it }{\oe}
paraîtront en 1779, plus de deux ans après la mort
de Hume.

Hume est mort avec la plus grande sérénité.
Dans sa brève autobiographie, rédigée le 18 avril 1776
il déclare : « Il est difficile d’être plus détaché de la
vie que je ne le suis à présent. » Le 13 août, il dit qu'il
se console d’abandonner des amis, car « {\it hélas, on
ne laisse que des mourants}, comme Ninon de Lenclos
le dit sur son lit de mort. La mort m'apparaît si
peu terrible maintenant qu’elle approche, que je
dédaigne de citer des héros et des philosophes comme
exemples de courage. Le témoignage d’une femme de
plaisir qui néanmoins était également philosophe
est suffisant ». Hume s’éteignit sans angoisse dans
l'après-midi du dimanche 25 août 1776. C’était la
veille de son soixante-cinquième anniversaire.
%%%%%%%%%%%%%%%%%%%%%%%%%%%%%%%%%%%%%%%%%%%%%%%%%%%%%%%%%%%%%%%%%%%%%%%%

%
\newpage
%

%%%%%%%%%%%%%%%%%%%%%
\chapter{L'{\oe}uvre}
%%%%%%%%%%%%%%%%%%%%% \textsc{}

%%%%%%%%%%%%%%%%%%%%%%%%%
\section{Ouvrages publiés du vivant de Hume}
%%%%%%%%%%%%%%%%%%%%%%%%%
% 
%{\it }
{\it A treatise of Human Nature} (1739-1740, 3 vol.).

{\it Essays moral and political} (3 vol., 1741-1742, 1748).

{\it Philosophical essays concerning Human Understanding}
(1748 ; à partir de 1758 le mot {\it Inquiry} remplace {\it Philosophical essays}).

{\it An Inquiry concerning the principles of Morals} (1751).

{\it Political discourses} (1752).

{\it The History of Great Britain} (1754-1757).

{\it Four Dissertations : 1. The natural history of Religion ;
II. Of the passions ; III. Of tragedy ; IV. Of the standard
of Taste (1757).} En 1755, Hume avait remis à l’éditeur
les trois premiéres de ces dissertations et une quatriéme
consacrée aux mathématiques et à la physique. Hume
la retire sur le conseil d’un ami mathématicien. Il la
remplace par deux dissertations : {\it On suicide} et {\it On the
immortality of soul.} Tandis que ces deux essais sont
déja en vente, Hume les retire et les remplace par une
seule dissertation : {\it Of the Standard of Taste. }

{\it The history of England} (1759 a 1767).

{\it Exposé succinct de la contestation qui s’est élevée entre
M. Hume et M. Rousseau} (1766).

%%%%%%%%%%%%%%%%%%%%%%%%%
\section{Ouvrages posthume}
%%%%%%%%%%%%%%%%%%%%%%%%%

{\it The life of David Hume written by himself} (1777).
{\it Two essays (On Suicide et The Immortality of the Soul)}
(1777).

%50 HUME
%{\it }
 

{\it Dialogues concerning Natural Religion} (1779).

J. H. \textsc{Burton}, {\it Life and Correspondance of David Hume,}
Edinburgh, 1846, 2 vol.

J. Y. T. \textsc{Greig}, {\it The letters of David Hume}, Oxford, 2 vol.,
1932.

R. \textsc{Klibansky} et E. C. \textsc{Mossner}, {\it }New letters of David
Hume, Oxford, 1954.

Rappelons qu’une édition anglaise classique comprend
toute l’{\oe}uvre philosophique de Hume : {\it The philosophical
works of David Hume}, éd. T. H. Green and T. H. Grose,
London, 1874-1875, 4 vol.

%%%%%%%%%%%%%%%%%%%%%%%%%
\section{Traductions françaises}
%%%%%%%%%%%%%%%%%%%%%%%%%{\it }

{\it {\oe}vres philosophiques choisies (Enquéte sur l'entendement,
Traité de la nature humaine, Dialogues de la religion
naturelle)}  traduites par Maxime \textsc{David} avec préface
de \textsc{Lévy-Bruhl}, Paris, Alcan, 1912. La traduction
Maxime \textsc{David} des {\it Dialogues sur la religion naturelle}
a été rééditée en 1964 chez J.-J. Pauvert dans la collection 
« Libertés » avec une présentation et des notes
de Clément \textsc{Rosset}.

{\it Traité de la nature humaine}, préfacé et traduit par André
\textsc{Leroy}, Editions Aubier (1$^{\text re}$ éd., 1946), 2 vol.

{\it Enquête sur l'entendement humain} (trad. \textsc{Leroy}, Aubier,
1947).

{\it Enquête sur les principes de la morale. Les quatre philosophes} 
(trad. \textsc{Leroy}, Aubier, 1947).

Il existe une traduction frangaise de 1788 des quatre
dissertations {\it (L’histoire naturelle de la. religion, Les passions, 
La tragédie, La règle du goût)}. Une nouvelle traduction 
des {\oe}uvres de Hume est en cours d’édition.

 

 
%%%%%%%%%%%%%%%%%%%%%%%%%%%%%%%%%%%%%%%%%%%%%%%%%%%%%%%%%%%%%%%%%%%%%%%%

%
\newpage
%

%%%%%%%%%%%%%%%%%%%%%
\chapter{Bibliographie}
%%%%%%%%%%%%%%%%%%%%%

%%%%%%%%%%%%%%%%%%%%%%%%%
\section{En anglais}
%%%%%%%%%%%%%%%%%%%%%%%%%{\it }

Hendel, {\it Studies on the philosophy of D. Hume}, Princeton,
1925.

A. E. Taylor, {\it David Hume and the miraculous}, Gambridge, 1927.

J. Laird, {\it Hume’s Philosophy of Human Nature}, London,
1932.

{\it Hume and present day problems}, Aristotelian Society,
suppl., vol. XVIII, London, 1939 (4 symposia sur
l'identité du moi, sur les concepts {\it a priori}, sur l’éthique,
sur la religion naturelle avec des articles de Taylor,
de Lairp, de Jessop).

Norman Kemp Smith, {\it Philosophy of David Hume}, London,
1941.

D. G. C. Mac Nabb, {\it David Hume, His theory of Knowledge
and Morality}, London, 1954.

E. G. Mossner, {\it The life of David Hume}, London, 1954.

%%%%%%%%%%%%%%%%%%%%%%%%%
\section{En français}
%%%%%%%%%%%%%%%%%%%%%%%%%

G. Compayré, {\it }La philosophie de D. Hume, Paris, 1873.

G. Lechartier, {\it David Hume sociologue et moraliste},
Paris, 1900.

L. Levy-Bruhl, Orientation de la pensée de D. Hume,
{\it Revue de métaphysique et de morale}, 1909.

A. Leroy, {\it Critique et religion chez D. Hume}, Paris, 1931.

%92
Laporte, Le septicisme du Hume, {\it Revue philosophique,}
1933-1934.

G. Brercer, Husserl et Hume, {\it Revue internationale de
philosophie}, 1939.

{\it Mélange David Hume}, divers articles, {\it Revue internationale
de philosophie}, Bruxelles, 1952.

DELEUZE, Empirisme et subjectivité, Paris, Presses universitaires 
de France, 1953.

A. Leroy, {\it David Hume}, Paris, Presses Universitaires de
France, 1953.

O. Brunet, Philosophie et Esthétique chez D. Hume,
Paris, Librairie A.-G. Nizet, 1965.

%%%%%%%%%%%%%%%%%%%%%%%%%%%%%%%%%%%%%%%%%%%%%%%%%%%%%%%%%%%%%%%%%%%%%%%%

%
%
%%%%%%%%%%%%%%%%%%%%%
\section{Encyclopédie de la philosophie}
%%%%%%%%%%%%%%%%%%%%%
%
\subsection{Nécessité}
Est dit nécessaire « ce qui ne peut être autrement qu'il n'est »,
telle est du moins la définition aristotélicienne de la notion (grec {\it anankaion}).
Elle s'applique
notamment pour caractériser la nature de
la relation qui relie entre elles les propositions (prémisses et protases) du syllogisme scientifique ou démonstratif. Est
nécessaire ce qui appartient à un sujet
partout et toujours : ainsi la propriété
d’avoir la somme de ses angles intérieurs
égale à deux droits est une propriété
nécessaire pour tout triangle en tant que
tel. Mais, outre le sens logique et ontologique, il y a un sens psychologique, de
contrainte inévitable. Ce qui a conduit certains à émettre l'hypothèse que le concept
est peut-être né d’une projection anthropomorphique de l’idée originelle de coercition,
ce qui aurait pour conséquence que la
notion de nécessité déontologique (c’est-à-dire d’obligation) précéderait historiquement toutes les autres formes de nécessité.
La notion de nécessité physique et causale comme subordination aux lois de la
nature et celle de nécessité logique
comme propriété de ce qui est « forcément » vrai en vertu des lois logiques
seraient apparues par la suite, selon un
ordre inversé par rapport à celui qui est
parfois considéré logiquement rationnel.

\subsection{Nécessité et modalité}

La notion leibnizienne de nécessité
comme vérité dans tous les mondes possibles à été reprise dans l’analyse sémantique contemporaine des modalités.
D'après ces analyses, il apparaît que la
notion modale fondamentale est la notion
de possibilité plutôt que celle de nécessité. À côté de cette notion leibnizienne
de nécessité absolue, la logique. contemporaine a caractérisé différentes notions
de nécessité relative, en-isolant des sous-classes pertinentes à l’intérieur de la
classe de tous les mondes possibles,
comme celle des mondes où sont valides
les lois de la nature ou les lois de code
pénal. Toute logique de la nécessité,
quelle qu’elle soit, a comme condition
minimum d'éviter ce que l’on appelle
« l’effondrement des modalités », à savoir
la démonstration de l’équivalence entre
une proposition nécessaire et une proposition dépourvue d’opérateurs modaux
(cette équivalence est présente dans toute
philosophie fataliste ou strictement déterministe : il suffit de penser à l’assimilation
opérée par Hegel entre ce qui est rationnel et ce qui est réel). Dans une acception
élargie, la nécessité logique est parfois
assimilée à l’analycité. Les énoncés analytiques résultent nécessaires dans la
mesure où les règles linguistiques
« créent » la signification des termes que
l’on y trouve, que ces termes soient des
constantes logiques (« et », « ou »,
« tous », etc.) ou des mots du langage
ordinaire comme « célibataire » ou « non-
marié ». Du point de vue pragmatiste
d'auteurs comme W.V.O. Quine, la nécessité d’un énoncé quelconque consiste dans
son immunité à l’intérieur du système des
connaissances prouvées, c’est-à-dire dans
le fait que le renoncement à l’énoncé en
question est trop coûteux pour le système
dans son ensemble. Tout en excluant qu’il
existe une distinction historiquement définitive entre analytique et synthétique et
entre nécessaire et contingent, Quine.
pense toutefois que la nécessité logique
peut être justement attribuée. à des
énoncés dont la vérité dépend uniquement des constantes logiques qui sont
présentes en lui. Il s’agit d’un pas en avant
par rapport à la conception du Tractatus
de L. Wittgenstein, qui assimilait les
nécessités logiques aux tautologies du calcul propositionnel, excluant de cette
manière les vérités logiques qui dépendent de la présence de quantificateurs. De
toute façon, cette conviction est partagée
par presque toute  l'épistémologie
contemporaine, surtout par l’épistémologie empiriste : que la nécessité des lois
scientifiques et des inférences garanties
par elles ne dépend pas de l’existence de
connexions nécessaires dans la nature,
mais doit être indirectement référée à la
nécessité logique ou au concept d’implication logiquement nécessaire.

%\vspace{0.35cm}
%$\to$ existence ; mondes possibles ; possibilité ; quantificateurs ; Quine

%%%%%%%%%%%%%%%%%%%%%%%%%%%%%%%%%%%%%%%%%%%%%%%%%%%%%%%%%%%%%%%%%%%%%%%%

%%%%%%%%%%%%%%%%%%%%%%%%%%%%%%%%%%%%%
\section{Pratique de la philosophie}
%%%%%%%%%%%%%%%%%%%%%%%%%%%%%%%%%%%%

\subsection{Opinion}

{\footnotesize
\begin{itemize}[leftmargin=1cm, label=\ding{32}, itemsep=1pt]
\item {\bf \textsc{Étymologie} :} latin {\it opinari},
« émettre une opinion ».
\item {\bf \textsc{Sens ordinaire} :} avis,
jugement porté sur
un sujet, qui ne relève pas d'une
connaissance rationnelle vérifiable,
et dépend donc du système de
valeurs en fonction duquel on se
prononce.
\item {\bf \textsc{Philosophie} :} jugement
sans fondement rigoureux,
souvent dénoncé dans la mesure où
il se donne de façon abusive les
apparences d’un savoir.
\end{itemize}
}

L'interrogation sur la nature de la vérité
et les moyens de l’atteindre a conduit
nombre de philosophes à distinguer,
entre les différents types de connaissance
possibles, ceux qui conduisent effectivement
à la vérité, et ceux qui en éloignent.
En un premier sens, l’opinion est ainsi
traditionnellement considérée comme un
genre de connaissance peu fiable, fondée
sur des impressions, des sentiments, des
croyances où des jugements de valeur
subjectifs. Pour Spinoza, par exemple,
elle est forcément « sujette à l'erreur et n’a
jamais lieu à l'égard de quelque chose
dont nous sommes certains mais à l'égard
de ce que l’on dit conjecturer ou supposer »
({\it Court traité}, chap. II). Depuis Platon,
et jusque chez de nombreux penseurs
contemporains, l'opinion est
dénoncée comme a priori douteuse, illusoire
ou fausse, voire dangereuse, lorsqu’elle
cherche à s'imposer en dissimulant
la faiblesse de ses fondements sous
les apparences de la plus claire certitude.
Selon Adorno ({\it Modèles critiques}, 1963),
« l'opinion s’approprie ce que la connaissance
ne peut atteindre pour s’y substituer »,
elle rassure à bon compte, parce
qu’« elle offre des explications grâce auxquelles
on peut organiser sans contradiction
la réalité contradictoire ». Tel est bien
le « fonctionnement psychique » qui soustend,
par exemple, les opinions racistes :
pour être plus crédible, la peur de l’autre
prend le masque de l'affirmation de son
infériorité ou de la mise en garde contre
le danger qu'il est censé représenter. La
justesse de ces analyses ne doit pas faire
oublier qu'en un autre sens, l'opinion
constitue une forme de connaissance
utile, voire un type de jugements éminemment
respectables. Dans le {\it Ménon},
Platon reconnaît aux opinions droites la
faculté, sur les sujets qui ne relèvent ni de
la science ni de la simple conjecture,
d'éclairer l’action humaine. Dans le
domaine moral par exemple, à défaut de
vérités certaines, des intuitions justes
relatives au bien peuvent guider efficacement
l'éducation ou l’action, en leur
fixant pour but la satisfaction d'intérêts
conformes aux exigences de la réflexion,
et non à la soumission aux apparences ou
au plaisir immédiat. Enfin, sur toutes les
questions qui engagent des choix individuels
qu'aucune autorité ne peut légitimement
contraindre {\bf --} la religion, la
préférence politique, l'adhésion à une
conception du monde {\bf --} la liberté d’opinion
est un droit fondamental, dans les
sociétés démocratiques en tout cas, dès
l'instant où ceux auxquels elle est garantie
n'en usent pas au détriment de la
liberté d'autrui.

Analysée dans le {\it Traité
théologico-politique}, où Spinoza insiste
sur la nécessité d'une indépendance
absolue des opinions religieuses et de
leur expression par rapport à l'État, la
liberté d'opinion est proclamée dans la
Déclaration des droits de l'homme et du
citoyen de 1789. Et depuis près d'un
siècle, elle est au cœur du principe de la
laïcité qui garantit (en particulier en
France) la séparation entre l'Église et
l'État.

{\footnotesize
\begin{itemize}[leftmargin=1cm, label=\ding{32}, itemsep=1pt]
\item {\bf \textsc{Termes voisins} :} avis ; croyance.
\item {\bf \textsc{Termes opposés} :} science.
\end{itemize}
}

\subsubsection{Opinion publique}

Ensemble fluctuant de prises de positions
portant sur des questions politiques,
 morales, économiques... Les
« sondages d'opinion » prétendent en
constituer une sorte de baromètre.

{\footnotesize
\begin{itemize}[leftmargin=1cm, label=\ding{32}, itemsep=1pt]
\item {\bf \textsc{Corrélats} :} connaissance ;
conviction ; croyance ; doute ; foi ;
jugement ; préjugé.
\end{itemize}
}

%%%%%%%%%%%%%%%%%%%%%%%%%%%%%%%%%%%%%%%%%%%%%%%%%%%%
\subsection{Préjugé}

{\footnotesize
\begin{itemize}[leftmargin=1cm, label=\ding{32}, itemsep=1pt]
\item {\bf \textsc{Étymologie} :} latin {\it praejudicare},
« juger préalablement ».
\item {\bf \textsc{Sens ordinaire} :} Opinion admise sans
jugement ni raisonnement.
\end{itemize}
}

Le terme préjugé est souvent employé
dans un sens péjoratif, pour dénoncer
l'erreur ou au moins l'absence de
réflexion qui conduit un individu à
adhérer à une idée fausse {\bf --} dont il n’a
pas pris la peine de contrôler le bien-fondé {\bf --}
voire à la défendre contre des
idées justes, ou à condamner des individus
au nom de cette idée (par
exemple, les opinions racistes sont des
préjugés).

{\footnotesize
\begin{itemize}[leftmargin=1cm, label=\ding{32}, itemsep=1pt]
\item {\bf \textsc{Termes voisins} :} opinion.
\item {\bf \textsc{Termes opposés} :} savoir ; science.
\item {\bf \textsc{Corrélats} :} certitude ; croyance ;
dogme ; doute ; foi.
\end{itemize}
}

%%%%%%%%%%%%%%%%%%%%%%%%%%%%%%%%%%%%%%%%%%%%%%%%%%%%

\subsection{Erreur}

{\footnotesize
\begin{itemize}[leftmargin=1cm, label=\ding{32}, itemsep=1pt]
\item {\bf \textsc{Étymologie} :} latin {\it error}, « course
à l'aventure », de {\it errare}, «errer».
\item {\bf \textsc{Logique et sciences} :} affirmation
fausse, c'est-à-dire non conforme
aux règles de la logique, et/ou en
contradiction avec les données
expérimentales.
\item {\bf \textsc{Psychologie} :} état
de l'esprit qui tient pour vrai ce
qui est faux, et réciproquement
(ex. : « être dans l’erreur »).
\end{itemize}
}

L'erreur doit être soigneusement distinguée
aussi bien de la faute (qui engage
plus nettement notre responsabilité) que
de l’illusion (qui n’est pas vaincue par
le savoir). L'erreur procède toujours de
notre jugement : elle résulte, selon Descartes,
d’un décalage permanent entre
notre volonté, qui est infinie, et notre
entendement, qui ne l'est pas. Nous
nous trompons parce que nous outrepassons
nos possibilités intellectuelles,
par étourderie ou vanité : l'erreur n’est
donc qu'une privation de connaissance.
L'épistémologie contemporaine, au
contraire, donne à l’erreur un tout autre
statut, plus « positif ». Bachelard, notamment,
montre que les « vérités » scientifiques
ne sont jamais que provisoires,
qu'elles doivent constamment être remaniées
et corrigées. La connaissance
scientifique ne peut pas faire l'économie
de l’erreur.

{\footnotesize
\begin{itemize}[leftmargin=1cm, label=\ding{32}, itemsep=1pt]
\item {\bf \textsc{Termes voisins} :} fausseté ; illusion ;
incorrection.
\item {\bf \textsc{Termes opposés} :} vérité.
\item {\bf \textsc{Corrélats} :} connaissance ; Évidence ;
faute ; illusion ; jugement.
\end{itemize}
}

%%%%%%%%%%%%%%%%%%%%%%%%%%%%%%%%%%%%%%%%%%%%%%%%%%%%

%
%\newpage
%
\end{appendix}
%

%
\newpage
%
%====================== INCLUSION DE LA BIBLIOGRAPHIE ======================
%
%récupérer les citation avec "/footnotemark" : 
\nocite{*}
%
% choix du style de la biblio
\bibliographystyle{plain}
%
% inclusion de la biblio
%\cleardoublepage
%\addcontentsline{toc}{chapter}{Bibliographie}
%\bibliography{bibliographie.bib}
%
\end{document}
%%%%%%%%%%%%%%%%%%%%%%%%%%%%%%%%%%%%%%%%%%%%%%%%%%%%%%%%%%%%%%%%%%%%%%%%%%%%%%%%%
