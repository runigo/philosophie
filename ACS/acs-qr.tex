%{\it }
\chapter{QR}
%{\footnotesize XIX$^\text{e}$} siècle — {\it }
\section{Qualité}
%QUALITÉ
Ce qui répond à la question {\it qualis ?} (quel est-il ? comment est-il ?).
Par exemple : « Il est grand et fort ; il est très gentil et un peu
bête. » Ce sont des {\it qualités}, et l’on voit que le mot, en philosophie, n’est
pas forcément élogieux (la qualité s’oppose à l’essence ou à la quantité
beaucoup plus qu’au défaut, qui n’est qu’une qualité négative). La qualité,
c’est ce qui fait qu’un être est comme il est (par différence avec l’essence ou
quiddité, qui fait qu’il est {\it ce} qu’il est), ou le fait qu’il le soit : c’est une
manière d’être ou une propriété, qui vient s'ajouter à la substance ou la
modifier. C’est la troisième catégorie d’Aristote : « J’appelle {\it qualité} ce en
vertu de quoi on est dit être tel » ({\it Catégories}, 8 ; voir aussi {\it Métaphysique}, $\Delta$,
14).

Chez Kant, les catégories de la qualité sont la réalité, la négation et la
limitation. Elles correspondent aux trois qualités du jugement, qui est soit
affirmatif, soit négatif, soit indéfini ({\it C. R. Pure}, Analytique, I).

On distingue parfois, spécialement depuis Locke, les {\it qualités premières},
qui sont indissociables de la matière et comme objectives (la solidité,
‘étendue, la forme, la vitesse...), des {\it qualités secondes}, qui n'existent que
pour et par le sujet qui les perçoit (la couleur, l'odeur, la saveur, le son...).
Celles-ci s’expliquent par celles-là (par exemple la couleur, par l’action
d’une certaine longueur d’onde de la lumière sur les cellules nerveuses de la
rétine), mais n’en sont pas moins réelles à leur façon. Quand je dis que cet
arbre est actuellement couvert de feuillage vert, j’énonce bien une proposition
vraie. Je ne me trompe que si je crois que sa couleur est indépendante
de la lumière et du regard. Mais je me tromperais bien davantage si je le
disais rouge ou bleu. Le sujet, c’est du réel aussi.

%— 483 —
%{\footnotesize XIX$^\text{e}$} siècle — {\it }
\section{Quantité}
%QUANTITÉ
Ce qui répond à la question « Combien ? ». C’est une certaine
grandeur, qui renvoie à une échelle donnée de numération ou
de mesure. Par exemple : combien sont-ils ? combien ça pèse ? combien ça
coûte ? Quelle longueur, quelle hauteur, quelle superficie ? Chez Kant, et en
général d’un point de vue logique, la quantité des jugements mesure leur
extension : ils peuvent être universels, particuliers ou singuliers (voir ces mots).
De à les catégories de la quantité, qui sont l'unité, la pluralité et la totalité
({\it C. R. Pure}, Analytique, I).

\section{Question}
%QUESTION
Le réel ne parle pas de lui-même. Mais il répond parfois, par
politesse, quand l’homme l’interroge. On appelle {\it question} ce
type de discours qui en sollicite un autre, dont il attend une information.
Questionner, c’est parler pour faire parler : par quoi le sens rebondit, pour ainsi
dire, sur son appel. Attitude proprement humaine, que les animaux ignorent (il
en est qui sont doués de langage, mais point qui soient aptes au dialogue, au jeu
libre des questions et des réponses) et que les dieux nous envient. À force de
connaître toutes les réponses, cela fait une telle torpeur en soi, une telle incuriosité
de tout, une telle lassitude. L’Olympe n’est pas ce qu’on croit. Le sens
ne rebondit plus, voilà, et les dieux s’ennuient. C’est pourquoi ils ont créé les
hommes : pour se distraire en les regardant se poser des questions.

\section{Quiddité}
%QUIDDITÉ
Ce qui répond à la question {\it « Quid? »} (Quoi?) ou {\it « Quid
sit ? »} (Qu'est-ce que c’est ?). On y répond ordinairement par
une définition. Ce livre aurait donc pu s'appeler {\it Quiddités}, si le titre n’était déjà
pris par Quine et s’il n’était à ce point ésotérique ou archaïsant.

La question {\it Quid sit ?} (Qu'est-ce que c’est?) est traditionnellement
opposée à la question {\it An sit ?} (Est-ce que c’est ?) : la {\it quiddité} est un synonyme
scolastique d’{\it essence}, qu’elle soit générique ou individuelle, et s'oppose à ce
titre, comme elle, à {\it l'existence}. Une fois qu’on a défini le bonheur, Socrate ou
Dieu, on n'a pas encore répondu à la question de leur existence. Mais pour
pouvoir répondre à la question de leur existence, il faut d’abord savoir ce qu'ils
sont, ou seraient : telle est la quiddité, toujours nécessaire, jamais suffisante.

\section{Quiétisme}
%QUIÉTISME
L'Église catholique y voit une hérésie, qui serait celle, en
France, de Mme Guyon et de Fénelon. J'y verrais plutôt une
tentation et un danger : la tentation du repos, le danger de l’inaction. Faire
preuve de quiétisme, c’est croire que la quiétude suffit à tout, quand elle ne
%— 484 —
%{\footnotesize XIX$^\text{e}$} siècle — {\it }
suffit qu’à elle-même. Autant prétendre que le mysticisme, même le plus pur et
le plus élevé, pourrait tenir lieu de morale, de politique ou de philosophie, ce
qui serait une évidente sottise. N'oublions pas pourtant que la réciproque est
vraie aussi : morale, politique et philosophie ne sauraient tenir lieu de quiétude,
ni même vraiment y mener. Tous les mots du monde, même les mieux choisis,
ne feront jamais un silence. Tous les combats du monde, même les plus nécessaires,
jamais une paix. Ainsi le quiétisme est la vérité du mysticisme, et sa
limite.

\section{Quiétude}
%QUIÉTUDE
Un repos sans trouble, sans crainte, sans espérance, sans fatigue :
c’est le nom chrétien de l’ataraxie.

L’oraison de quiétude est celle qui ne demande rien, qui n’espère rien,
même pas le salut : elle s’abandonne passivement à Dieu ou au silence, jusqu’à
s’y perdre. C’est la pointe extrême de l’acquiescement mystique, de la contemplation,
enfin du {\it pur amour}, celui qui aime Dieu, comme disait Fénelon, « sans
aucun motif d'intérêt personnel ». À la fin, il n’y a plus que Dieu, et la question
du salut ne se pose plus.

\section{Race}
%RACE
Un groupe, biologiquement défini, au sein d’une même espèce animale
(s'agissant des végétaux, on parlera plutôt de différentes
{\it variétés}). Par exemple les différentes races d’ânes ou de chevaux : un baudet du
Poitou ou un pur-sang arabe se distinguent respectivement d’un âne de Nubie
ou d’un percheron par un certain nombre de caractères héréditaires, qui sont à
la fois communs (à la race) et distinctifs (dans l’espèce). Il en va de même, au
moins à première vue, des êtres humains : ce n’est pas être raciste que de constater
que les Scandinaves présentent souvent un certain nombre de caractères
héréditaires communs, qui les distinguent, au moins superficiellement, des
Japonais ou des Pygmées.

Les races se distinguent des espèces, outre leur extension moindre, par
l’interfécondité : un mâle et une femelle de races différentes mais appartenant
à la même espèce sont normalement interféconds (ils peuvent.se reproduire
entre eux et engendrer des descendants eux-mêmes féconds), alors que deux
individus d’espèces différentes ne le sont pas. Il est vrai que, si les deux espèces
sont voisines, deux de leurs représentants respectifs peuvent éventuellement
s’accoupler (quoi que cela ne se produise pratiquement jamais à l’état sauvage),
mais leur descendant, sauf exception, sera stérile : ainsi un baudet et une
jument engendrant un mulet, ou une ânesse et un cheval engendrant un
bardot ; ni le mulet ni le bardot n’auront de descendance.

%— 485 —
%{\footnotesize XIX$^\text{e}$} siècle — {\it }
La notion de race est surtout descriptive ; mais il arrive souvent, s'agissant
des espèces domestiques et dans un milieu donné, qu’elle prenne un sens normatif,
voire prescriptif. Les notions de pur-sang ou de pedigree l’indiquent
assez. Ce sont catégories d’éleveurs et de marchands. C’est dire que la notion,
appliquée à l'humanité, est toujours suspecte. Au reste, on sait aujourd’hui
qu’elle n’a guère de sens : non seulement parce qu’il n’y a pas de races pures (les
différents groupes humains, qui dérivent sans doute d’une souche unique,
n'ont cessé depuis de se mêler), mais aussi parce que les caractéristiques différentielles
(la couleur de la peau, la taille, la forme du nez...) sont en l’occurrence
trop superficielles pour être prises — sauf par les racistes ou les imbéciles
— tout à fait au sérieux. Les progrès de la génétique confirment d’ailleurs et
l'unité de l’espèce humaine (99,99 \% du génome est commun) et la non-pertinence
biologique du concept de race (deux individus de la même race peuvent
être plus éloignés l’un de l’autre, d’un point de vue génétique, que deux individus
de races différentes). Toutefois ce n’est qu'un point de fait, certes heureux,
mais dont on aurait tort d’exagérer l'importance en matière d’antiracisme.
Quand bien même les races humaines se distingueraient davantage
qu'elles ne font, jusqu’à avoir (comme on voit dans certaines espèces animales)
des aptitudes différentes, cela ne changerait rien au respect indifférencié qu’on
doit à tout être humain. Le racisme n’est pas seulement une erreur
intellectuelle ; c’est aussi, et d’abord, une faute morale.

\section{Racisme}
%RACISME
« Je ne suis pas raciste », me dit un jour ma grand-mère. Puis elle
ajouta, en guise d’explication : « C’est vrai, quoi : c’est pas de leur
faute s’ils sont noirs ! » Elle avait plus de quatre-vingts ans, elle s'était occupée
de nous, mieux que nos parents, elle nous aimait plus que tout au monde...
J'avoue n’avoir pas eu le courage de lui expliquer, comme il l'aurait fallu, que
sa raison de n’être pas raciste. était raciste.

La même, un autre jour : « Je n’aime pas les Allemands ; ils sont tous
racistes. » Même remarque : c'était du racisme anti-allemand.

Qu'est-ce que le racisme ? Toute doctrine qui fait dépendre la valeur des
individus du groupe biologique, ou prétendu tel, auquel ils appartiennent.
C’est une pensée prisonnière du corps, comme un matérialisme barbare. Sa
logique est de frapper.

Matérialisme ? Voire. Car le corps y est beaucoup moins pensé comme la
{\it cause} de telle ou telle valeur psychique ou spirituelle que comme son {\it signe}.
Ainsi la blancheur ou la noirceur du corps révéleraient celles de l’âme.. Spiritualisme,
bien plutôt, à fleur de peau. C’est une herméneutique de l’épiderme.

%— 486 —
%{\footnotesize XIX$^\text{e}$} siècle — {\it }
C’est aussi, presque toujours, un narcissisme collectif et haineux. Deux raisons
de plus (le narcissisme, la haine) de le combattre.

\section{Raison}
%RAISON
C'est le rapport vrai au vrai, ou du vrai à lui-même. Mais qu'est-ce
que le vrai ? Nous n’y avons guère accès, sinon par la mise en évidence
du faux. De là un sens plus étroit ou plus spécifique : la raison est le pouvoir
de penser, en l’homme, conformément aux lois immanentes de la pensée,
en tout. Elle est pour cela toujours nécessaire (toujours soumise à des lois) et
toujours libre (elle n’a d’autres lois que les siennes propres). Un raisonnement
mathématique, dans sa perfection, en donne à peu près l’idée, qui est d’être une
liberté sans sujet : une liberté sans libre arbitre. C’est la liberté de Dieu, dirait
Spinoza (la nécessité de la nature ou du vrai), et la {\it libération} du sage, qui
devient Dieu dans l’exacte mesure où il cesse d’être soi. C’est là le bon usage,
non narcissique, du {\it connais-toi toi-même} : connaître le moi, c’est le dissoudre.
La raison, parce qu’elle est universelle, est comme une catharsis de l’égoïsme.
C’est ce qui explique que les sages, sans se piquer de morale, fassent ordinairement
preuve de la plus grande générosité. Là où le {\it moi} était, {\it tout} (la vérité) doit
advenir.

La raison est impersonnelle, universelle, objective. Aucun atome jamais n’a
violé la moindre de ses lois, ni aucun homme : le réel est rationnel ; le rationnel
est réel. Du moins c’est ce que pensent les rationalistes. Qu'ils soient incapables
de le prouver ne les réfute pas, puisque toute preuve et toute réfutation le supposent.

On distingue parfois, surtout depuis Kant, la {\it raison pratique}, celle qui commande,
et la {\it raison théorique}, celle qui connaît. Je n’ai jamais expérimenté la
première, ni réussi à la penser. Qu’une action puisse être raisonnable ou non,
c’est une affaire entendue. Mais pourquoi ? Parce qu’elle serait conforme ou
non à la raison ? Non pas (la folie l’est, puisqu’elle est rationnelle). Mais parce
qu’elle est conforme ou non à notre {\it désir} de raison (c’est-à-dire ici de cohérence,
de lucidité, d’efficacité.....). Aristote, Spinoza, Hume, plus éclairants que
Kant. Ce n’est pas la raison qui commande ou qui fait agir. C’est le désir : « Il
n’y a qu’un seul principe moteur, la faculté désirante » (Aristote, {\it De anima}, III,
10). La raison ne peut à elle seule réduire aucun affect (Spinoza, {\it Éthique}, IV,
prop. 7 et 14), ni produire aucune action (Hume, {\it Traité}..., livre II, III, 3 : «Il
n’est pas contraire à la raison de préférer la destruction du monde entier à une
égratignure de mon doigt »). Ainsi il n’y a pas de raison pratique ; il n’y a que
des actions raisonnables. Elles n’en sont pas pour cela plus rationnelles ; mais
plus efficaces, plus libres, plus heureuses.

%— 487 —
%{\footnotesize XIX$^\text{e}$} siècle  {\it }
\section{Raison suffisante (principe de —)}
%RAISON SUFFISANTE (PRINCIPE DE -)
« Nos raisonnements sont fondés sur deux grands principes,
écrit Leibniz: {\it celui de la contradiction} [qu’on appelle plus volontiers
aujourd’hui principe de non-contradiction], en vertu duquel nous jugeons
faux ce qui en enveloppe, et vrai ce qui est opposé ou contradictoire au faux ;
et {\it celui de la raison suffisante}, en vertu duquel nous considérons qu'aucun fait
ne saurait se trouver vrai ou existant, aucune énonciation véritable, sans qu'il
y ait une raison suffisante, pourquoi il en soit ainsi et non pas autrement.
Quoique ces raisons le plus souvent ne puissent point nous être connues »
({\it Monadologie}, \S 31 et 32 ; voir aussi Théodicée, I, \S 44). C’est parier sur la
rationalité du réel, ou plutôt ce n’est pas un pari mais la condition, pour
nous, de toute rationalité possible. La raison suffisante, c’est ce qui répond
suffisamment à la question {\it « Pourquoi ? »}. Le principe stipule qu’il est toujours
possible, au moins en droit, d'y répondre: que rien n’est {\it « sans
pourquoi »}, malgré Angelus Silesius, sinon peut-être la série entière des raisons
ou la raison ultime (il n’est pas impossible, me semble-t-il malgré Leibniz,
que ce soit sans raison qu'il y ait quelque chose, par exemple tout ou
Dieu, plutôt que rien). Une chose ne peut s’expliquer que par une autre : par
exemple la rose par sa graine ou le monde par Dieu. Mais comment expliquer
qu'il y ait quelque chose, puisque toute explication le suppose ? Appelons
{\it Tout}, selon le sens ordinaire du mot, l’ensemble de tout ce qui est ou arrive,
par exemple la somme de Dieu et du monde. Que tout ce qui existe ou arrive
dans le Tout puisse et doive s'expliquer, cela n’entraîne pas nécessairement
que le Tout lui-même soit explicable, et même cela rend cette explicabilité
par avance impensable : puisque la raison qu’on pourrait en donner devrait
en faire partie et ne pourrait dès lors en rendre raison. Certains m’objecteront
que la sommation de Dieu et du monde est un concept impossible, qui
mélange des ordres différents. Admettons-le. Mais le même raisonnement,
appliqué à Dieu seul (quelle est sa raison suffisante ? Aucune ou lui-même :
on ne peut donc l’expliquer, puisque toute explication qu’on pourrait en
donner le suppose), interdit pareillement de le soumettre au principe. C’est
dire que le principe de raison suffisante, qui affirme que tout peut s’expliquer,
ne saurait s’appliquer valablement au Tout lui-même (sauf à supposer
autre chose que tout, ce qui est contradictoire) ni à la raison suffisante de
tout. Il ne peut pas davantage s’appliquer à soi (quelle est la raison suffisante
du principe de raison suffisante ? On ne peut répondre, et c’est en quoi justement
c’est un principe : voir le grand livre de Francis Wolff, {\it Dire le monde},
PUF, 1997, p. 85 à 87). Par quoi le principe de raison suffisante reste métaphysiquement
insuffisant. Tout s'explique, sauf le Tout lui-même et que tout
s'explique.

%— 488 —
%{\footnotesize XIX$^\text{e}$} siècle — {\it }
\section{Raisonnable}
%RAISONNABLE
Ce qui est conforme à la raison pratique, comme dirait
Kant, ou plutôt, comme je préférerais dire, à notre désir de
vivre conformément à la raison {\it (homologoumenôs)}. On remarquera que ce désir
suppose toujours autre chose que la raison, qui ne désire pas. C’est ce qui
interdit de confondre le raisonnable et le rationnel. Est rationnel ce que la
raison peut connaître ou expliquer. Est raisonnable ce qu’elle peut justifier, eu
égard à un certain nombre de désirs ou d’idéaux donnés par ailleurs. Cet
{\it ailleurs} est l’histoire ; cet {\it égard} est la pensée.

\section{Raisonnement}
%RAISONNEMENT
Une inférence, ou, plus souvent, une suite continue d’inférences.
C’est établir une vérité (si le raisonnement est
valide) par l’enchaînement ordonné de plusieurs autres. À quoi l’on objectera
qu’un raisonnement, même valide, peut n’aboutir qu’à une probabilité, voire à
une impossibilité de trancher. Mais il n’est valide, répondrai-je, que s’il montre
que cette probabilité ou cette impossibilité sont vraies. C’est important, spécialement,
en philosophie. Les propositions « Il est probable que nous puissions
savoir quelque chose » et « Il est impossible que nous sachions tout » sont des
propositions vraies, qu’un raisonnement peut établir.

\section{Ranc{\oe}ur}
%RANCŒUR
C'est une rancune envieuse, pour un bien qu’on ne nous a pas
fait.

\section{Rancune}
%RANCUNE
Une vengeance rentrée, qui s’est rancie. Nous en voulons à
quelqu'un du mal qu’il nous a fait, dont nous gardons le souvenir
et, faute de pouvoir le lui rendre, le goût. C’est une haine présente, pour
une souffrance passée. Le mal qu’on nous a fait, longtemps nous fait du mal.

\section{Rasoir d'Ockham}
%RASOIR D'OCKHAM
Guillaume d’Ockham, qui mourut au milieu du
{\footnotesize XIV$^\text{e}$} siècle, est l’un des plus grands penseurs du
Moyen-Âge. C’est un nominaliste : il ne reconnaît l'existence que d’êtres singuliers,
considère la distinction entre l’essence et l'existence comme vide de sens,
et ne voit dans le genre et l’espèce, comme dans toute idée générale, qu’une
conception de l’âme ({\it intentio animae}, ce qui peut le rapprocher du conceptualisme)
ou « un signe attribuable à plusieurs sujets » (ce qui en fait un nominaliste).
Hors du cercle des spécialistes, il n’est pourtant guère connu ni cité,
sinon à propos de ce {\it rasoir} auquel son nom est traditionnellement attaché. De
%— 489 —
%{\footnotesize XIX$^\text{e}$} siècle — {\it }
quoi s'agit-il ? D’un principe d’économie : ne pas multiplier inutilement les
entités, couper tout ce qui dépasse du réel ou de l'expérience, c’est-à-dire en
l'occurrence toute idée qui n’est pas indispensable ou qui prétendrait exister en
soi ou de façon séparée. C’est un instrument d’hygiène intellectuelle, en même
temps qu’une arme contre le platonisme.

\section{Rationalisme}
%RATIONALISME
J'avais cité, dans l’un de mes livres, la formule fameuse de
Hegel : « Ce qui est rationnel est réel ; ce qui est réel est
rationnel. » Michel Polac, dans un article critique, s’en était énervé : « On ne
peut plus dire ça, protestait-il, depuis Freud et la mécanique quantique ! »
C'était bien sûr se méprendre. La mécanique quantique est l’une des plus étonnantes
victoires de la raison humaine, et je ne connais pas d’auteur plus rationaliste
que Freud. Que nous n’ayons pas accès absolument à l'absolu (que le
réel soit « voilé ») ni ne puissions vivre de façon tout à fait raisonnable (que la
raison en nous ne soit pas tout, ni même l'essentiel), cela même, cher Michel
Polac, est pleinement rationnel. Il serait contradictoire, étant ce que nous
sommes, qu’il en aille autrement. Par exemple l'inconscient ne se soucie pas du
principe de non-contradiction ; c’est sa façon de lui rester soumis (un inconscient
logicien, ce serait contradictoire), et la psychanalyse n’est possible,
comme science où comme discipline rationnelle, qu’à la condition de le respecter.
Une contradiction, découverte dans un texte de Freud, serait une objection
autrement plus forte, contre la psychanalyse, que le plus déraisonnable des
délires, qui n’en est pas une. La folie est aussi rationnelle que la santé mentale
(la psychiatrie serait autrement impossible), le rêve aussi rationnel, quoique de
façon différente, que la conscience éveillée : Freud n’a inventé la psychanalyse
que pour comprendre rationnellement ce qui, {\it avant lui}, semblait irrationnel, ce
qui devrait nous aider, pensait-il, à vivre de façon un peu plus raisonnable.

Mais qu'est-ce que le rationalisme ? Le mot se prend principalement en
deux sens.

Au sens large et courant, qui est celui que je viens d'évoquer, le rationalisme
exprime d’abord une certaine confiance en la raison : c’est penser qu’elle
peut et doit tout comprendre, au moins en droit, autrement dit que le réel est
rationnel, en effet, et que l’irrationnel n’existe pas. Le rationalisme s’oppose
alors à l’irrationalisme, à l’obscurantisme, à la superstition. C’est l'esprit des
Lumières, et la lumière de l'esprit.

Au sens étroit et technique, le mot relève de la théorie de la connaissance :
on appelle {\it rationalisme} toute doctrine pour laquelle la raison en nous est indépendante
de l'expérience (parce qu’elle serait innée ou {\it a priori}) et la rend
possible ; c’est le contraire de l’empirisme.

%— 490 —
%{\footnotesize XIX$^\text{e}$} siècle — {\it }
On remarquera qu’un même philosophe peut donc être rationaliste aux
deux sens du terme (Descartes, Leibniz, Kant), mais aussi au sens large sans
l’être au sens strict (Épicure, Diderot, Marx, Cavaillès : c’est bien sûr le courant
dans lequel je me reconnais), voire au sens strict sans l’être au sens large (Heidegger ?).

\section{Rationnel}
%RATIONNEL
Ce qui est conforme à la raison théorique, autrement dit ce
que la raison peut penser, calculer ({\it ratio}, en latin, c’est
d’abord le calcul), connaître et, au moins en droit, expliquer. La folie n’est pas
moins rationnelle que la santé mentale (la psychiatrie serait autrement impossible).
Mais elle est moins raisonnable (la psychiatrie serait autrement inutile).

\section{Réalisme}
%RÉALISME
Au sens courant : voir les choses comme elles sont, et s’y adapter
efficacement. Le contraire, non de l’idéalisme, mais de la niaiserie,
de l’utopie ou de l’angélisme.

Au sens esthétique : tout courant artistique qui soumet l’art à l'observation
et à l’imitation de la réalité, davantage qu’à l’imagination ou à la morale. Historiquement,
cela peut désigner une époque en particulier: en gros, la
deuxième moitié du {\footnotesize XIX$^\text{e}$} siècle, qui s'oppose au romantisme comme le symbolisme,
au tournant du siècle, s’opposera à son tour au réalisme. Mais le mot
peut bien sûr être utilisé plus largement : Molière et Philippe de Champaigne
sont davantage réalistes que Corneille ou Poussin ; Rembrandt, le Caravage et
Zurbaran, davantage que Botticelli ou Boucher. Ces exemples disent assez que
le mot, pris en ce sens large, n’a rien de péjoratif. Il peut pourtant garder, dans
certains de ses emplois, quelque chose de restrictif. Que Courbet ou Flaubert
soient des artistes réalistes, cela ne retire rien à leur génie. Mais que le mot
s'applique mal à Chardin ou Stendhal, Rembrandt ou Proust (qui en relèvent
pourtant, mais l’excèdent de toutes parts), il me semble que cela ajoute quelque
chose au leur. C’est peut-être que tout {\it isme} enferme, ou que la réalité est plus
riche que le réalisme. C’est aussi que le réalisme, en ce sens restrictif, serait
plutôt un prosaïsme : il souffre moins d’un excès d’observation que d’un
manque de poésie.

Au sens proprement philosophique, enfin, le réalisme est une doctrine qui
affirme l'existence d’une réalité indépendante de l’esprit humain, que celui-ci
peut connaître au moins en partie. On parle par exemple de {\it réalisme moral},
pour désigner une doctrine qui affirme l’objectivité de la morale ou la réalité de
faits moraux irréductibles à quelque illusion ou croyance que ce soit (voir
Ruwen Ogien, {\it Le réalisme moral}, PUF, 1999). Mais le réalisme est surtout
%— 491 —
%{\footnotesize XIX$^\text{e}$} siècle  {\it }
évoqué en un sens plus général ou plus métaphysique. Il n’affirme pas l’existence
de telle ou telle réalité (par exemple morale), mais d’{\it une} réalité, quelle
qu’elle soit, voire de {\it la} réalité. Le mot, en ce sens technique et quoiqu'il soit un,
peut alors désigner deux courants très différents, selon la nature du réel
considéré : réalisme des Idées, des universaux ou de l’intelligible (par exemple
chez Platon, saint Anselme ou Frege), ou bien réalisme du monde sensible ou
matériel (par exemple chez Épicure, Descartes ou Popper). Le premier s'oppose
au nominalisme ou au conceptualisme ; le second, à une forme d’idéalisme ou
d’immatérialisme. On notera que ces deux réalismes s’opposent souvent l’un à
l’autre (Épicure contre Platon), mais pas toujours : Russel, au moins un temps,
et Popper ont soutenu l’un et l’autre.

\section{Réalité (principe de —)}
%RÉALITÉ (PRINCIPE DE)
Ce n’est pas le contraire du principe de plaisir,
ni même tout à fait son complément. C’est
plutôt sa forme lucide et intelligente, qui s’y soumet en tenant compte — dans
sa recherche du plaisir et pour l’atteindre — de la réalité. Il s’agit toujours de
jouir le plus possible, de souffrir le moins possible (principe de plaisir), mais {\it en
tenant compte des contraintes et des dangers du réel} (principe de réalité). Cela
nous conduit à différer souvent le plaisir, voire à y renoncer ponctuellement ou
à accepter un déplaisir, pour jouir, plus tard, davantage ou plus longtemps.
Ainsi arrête-t-on de fumer, quand on y parvient, pour la même raison qui nous
a fait fumer si longtemps (pour le plaisir), mais appliquée autrement : parce
qu'on juge (le principe de réalité est un principe intellectuel) que le tabac
apporte, au bout du compte et au moins statistiquement, davantage de désagréments
que de jouissances. Ce n’est pas échapper au principe de plaisir ; c’est s’y
soumettre autrement. Ainsi va-t-on chez le dentiste pour le plaisir, comme on
va travailler pour le plaisir, quand bien même ce plaisir, le plus souvent, ne
nous attend ni chez le dentiste ni au travail. Il serait plus agréable de rester au
lit ? Sans doute, à court terme. Mais le principe de réalité est justement ce qui
nous libère de la dictature du court terme : principe de prudence (c’est l’équivalent
à peu près de la {\it phronèsis} des Anciens) et d'imagination.

\section{Réel}
%RÉEL
L'ensemble des choses ({\it res}) et des événements, connus ou inconnus,
durables ou éphémères, en tant qu’ils sont présents : c’est l’ensemble
de tout ce qui arrive ou continue. Se distingue par là de la vérité, qui n’arrive ni
ne dure, mais demeure. Par exemple que je sois assis actuellement devant mon
ordinateur, que je m’interroge sur la définition du réel, qu’il y ait un bouquet de
fleurs sur la table, des voitures dans la rue, que la Terre tourne autour du Soleil,
%— 492 —
%{\footnotesize XIX$^\text{e}$} siècle — {\it }
que d’autres étoiles naissent ou meurent, etc., c'est du réel. Quand le bouquet
sera fané, quand je serai mort, quand les voitures, la Terre et le Soleil auront disparu,
quand d’autres étoiles ou rien auront remplacé celles qui naissent ou se consument
actuellement, ce ne sera plus du réel. Mais il restera vrai que tout cela s’est
produit, comme il était vrai, avant que cela n'arrive, que cela se produirait. Éternité
du vrai, impermanence du réel. Les deux ne coïncident qu’au présent ; ils
coïncident donc toujours, pour tout réel donné (non, certes, pour toute vérité),
et c’est le présent même : le point de tangence du réel et du vrai. Non, pourtant,
que l’un et l’autre soient exactement sur le même plan. Ce n’est pas parce que
quelque chose était vrai de toute éternité que cela arrive ; c’est parce que cela
arrive que c'était, que c’est et que ce sera vrai de toute éternité. La vérité est sans
puissance propre, sans force, sans réalité : ce n’est que l’ombre portée, dans la
pensée, du réel, ou plutôt que sa lumière, pour nous, antécédente et rémanente.
Les matérialistes ne peuvent lui accorder tout à fait l’être ; ni les rationalistes lui
refuser pourtant toute consistance (puisque la vérité, même considérée indépendamment
du réel, a ses contraintes propres, qui sont celles de la logique). De là,
pour tout matérialisme rationaliste, une espèce de tension, qui fait sa difficulté et
sa limite. Un Dieu serait plus simple. Un monde d’Idées serait plus simple. Ou
bien la sophistique et la bêtise. Mais pourquoi faudrait-il se tenir au plus simple ?
Que le matérialisme soit difficile et limité, cela ne le réfute pas. Si le réel n’est pas
une pensée, comment la pensée pourrait-elle le saisir sans peine et sans limites ?
Ainsi le réel a toujours le dernier mot, mais ce n’est pas un mot: c’est ce qui
interdit qu'aucun discours, jamais, le dise adéquatement. Ce petit mot de réel, si
commode, si pauvre, n’est qu’une étiquette que nous collons, parce que cela nous
est utile, sur l'infini silence de ce qui dure et passe. Nos discours en font partie,
comme nos rêves et nos erreurs. C’est l’ensemble le plus vaste, le plus complet, le
plus concret : le divers du donné et de ce qui pourrait l’être — l’objet d’une expérience
possible ou impossible. Spinoza l’appelait la nature ({\it infinita infinitis modis,
hoc est omnia...}) ou Dieu, qui est tout. La définition du mot exclut que quoi que
ce soit lui échappe.

\section{Référent}
%RÉFÉRENT
L'objet, réel ou imaginaire mais non linguistique (sauf dans le
métalangage), d’un signe linguistique. Soit par exemple le mot
«chien » : ni le signifiant ni le signifié n’aboient ou ne mordent ; leur référent
peut faire l’un et l’autre.

\section{Réflexe}
%RÉFLEXE
Un mouvement involontaire, qui répond à un stimulus extérieur.
Par exemple l’œil qui se ferme, devant le coup, ou la main qui se
%— 493 —
%{\footnotesize XIX$^\text{e}$} siècle — {\it }
retire, devant la douleur. On parle de {\it réflexes conditionnés} pour ceux qui associent
artificiellement deux stimuli (par exemple une sonnerie et un aliment),
jusqu’à ce que l’un des deux suffise à produire un réflexe qui ne lui est pas normalement
associé (par exemple la sonnerie entraînant la salivation). On peut
aussi parler de réflexe, en un sens plus large, pour toute réaction quasi automatique,
résultant d’un apprentissage ou d’une habitude : ainsi chez le marin ou
l’automobiliste. Cela se fait sans que j'aie besoin d’y penser, ce qui est bien
commode et souvent efficace, mais ne saurait suffire : le réflexe ne dispense ni
de volonté ni de réflexion.

\section{Réflexion}
%RÉFLEXION
Au sens large : un effort particulier de pensée. Au sens étroit :
un retour de la pensée sur elle-même, qui se prend alors pour
objet. Ce dernier mouvement serait, avec la sensation, l’un des deux constituants
de l’expérience, donc l’une des deux sources, comme dit Locke, de nos
idées : nous n’aurions sans elle aucune idée de « ce qu’on appelle apercevoir,
penser, douter, croire, raisonner, connaître, vouloir et toutes les différentes
actions de notre âme » ({\it Essai}, II, 1, \S 4). La réflexion est donc une espèce de
sens intérieur, mais intellectuel et délibéré : c’est « la connaissance que l’âme
prend de ses différentes opérations, par où l’entendement vient à s’en former
des idées » ({\it ibid.}). Mouvement nécessaire, mais qui ne saurait épuiser à lui seul
le champ de la pensée. Mieux vaut philosopher à la façon des Grecs, conseille
Marcel Conche ({\it Présence de la nature}, II), plutôt que s’enfermer, comme Descartes
ou Husserl, dans la réflexivité ou le sujet : mieux vaut penser le réel
(réfléchir, au sens large) plutôt que se regarder penser (s’enfermer dans la
réflexivité, au sens étroit). Le moi, certes, fait partie du réel : penser le monde,
c’est donc aussi se penser. Mais il n’en est qu’une partie infime : se penser n’a
jamais suffi à penser ce qui est, ni même ce qu’on est (un vivant). La logique et
la neurobiologie nous en apprennent plus, sur la pensée, que la réflexion (au
sens étroit). Mieux vaut penser la pensée, comme dit Alain, que se penser soi
({\it Cahiers de Lorient}, I, p. 72). Mieux vaut connaître et réfléchir (au sens large) à
ce qu'on sait ou croit savoir, plutôt que s’enfermer dans la réflexion (au sens
étroit). « La pensée, disait encore Alain, ne doit pas avoir d’autre chez soi que
tout l’univers ; c’est là seulement qu’elle est libre et vraie. Hors de soi! Au
dehors ! » ({\it ibid.}). La réflexion mène à tout, à condition d’en sortir.

\section{Refoulement}
%REFOULEMENT
L'un des grands concepts de la psychanalyse. C’est le rejet
d’une représentation dans l’inconscient, où elle reste bloquée.
Il s’agit de protéger le moi, spécialement quand il est écartelé entre les
%— 494 —
%{\footnotesize XIX$^\text{e}$} siècle — {\it }
désirs du ça et les exigences du sur-moi. Mais le remède, parfois, est pire que le
mal: ce qui a été refoulé peut, sous les déformations que la résistance lui
impose, venir perturber la vie consciente (retour du refoulé : actes manqués,
rêves, symptômes). Par quoi le refoulement, sans être en lui-même pathologique,
peut devenir pathogène. Son remède n’est pas le défoulement, mais la
cure analytique. Son contraire, pas le retour du refoulé (qui lui reste soumis),
mais l’acceptation. On n’oubliera pas pourtant qu’accepter une représentation
n’est pas forcément satisfaire le désir qui s’y exprime. C’est la vérité, non la
jouissance, qui libère et guérit.

\section{Réfutation}
%RÉFUTATION
C'est démontrer la fausseté d’une proposition ou d’une
théorie. On y parvient ordinairement en montrant qu’elle
est incohérente (réfutation logique) ou démentie par l'expérience (falsification).
Ces deux voies, en philosophie, restent incertaines. La philosophie n’est pas
une science : les objections qu’on y fait, même rationnellement argumentées,
peuvent toujours être intégrées (« dépassées ») dans le système qu’elles attaquent,
ou faire l’objet elles-mêmes d’un certain nombre d’objections. Personne,
À ma connaissance, n’a jamais réfuté valablement Malebranche ou Berkeley.
C’est sans importance : leur philosophie n’en est pas moins morte pour autant.

\section{Règle}
%RÈGLE
Un énoncé normatif, qui sert moins à comprendre qu’à agir. Un
esprit fini ne peut s’en passer (voyez par exemple Spinoza, {\it Ethique},
V, 10, scolie), pas plus qu’un esprit libre s’en contenter.

\section{Regret}
%REGRET
Un désir présent qui porte sur le passé, mais comme doublement
en creux : c’est le manque en nous de ce qui ne fut pas. Se distingue
par là de la nostalgie (le manque en nous de ce qui fut) et s'oppose à la
gratitude (la joie présente de ce qui fut). « {\it Le regret}, disait Camus, {\it cette autre
forme de l'espoir...} » Mais le contexte — il s’agit de Don Juan, dans le {\it Mythe de
Sisyphe} — indique qu’il pense plutôt à ce que j'appelle la nostalgie. De fait, ce
sont les deux symétriques de l’espérance : le manque du passé (en tant qu’il fut
ou ne fut pas), comme l’espérance est le manque de l'avenir (qui sera peut-être).
Cette asymétrie rend le regret plus douloureux, et l’espérance plus forte.

\section{Régulateur}
%RÉGULATEUR
Ce qui fournit une règle, un horizon ou un fil conducteur,
sans permettre de dire ce qui est. S’oppose, spécialement
%— 495 —
%{\footnotesize XIX$^\text{e}$} siècle — {\it }
chez Kant, à {\it constitutif}. Par exemple l’idée de finalité dans la nature n’est qu’un
principe régulateur pour la faculté de juger réfléchissante : il est utile de la chercher,
impossible de la prouver ({\it C.F.J.}, \S 67 et 75). Comment expliquer l'œil,
sans supposer qu'il est là {\it pour voir} ? Mais comment prouver que c’est en effet
le cas ? On ne le peut : il faut faire {\it comme si}, disait un de mes professeurs, sans
jamais pouvoir prouver que c’est {\it comme ça}. Un principe régulateur indique une
direction, vers quoi il faut tendre ; il ne saurait déterminer ce qui est (il n’est
pas constitutif). Il aide à penser ; il ne suffit pas à connaître.

\section{Relatif}
%RELATIF
Je ne sais plus qui soulignait plaisamment la grandeur du peuple
juif, ou l'importance pour nous de ses apports, en cinq noms

propres :

Moïse, qui enseigne que la Loi est tout ;

Jésus, qui enseigne que l’amour est tout ;

Marx, qui enseigne que l’argent est tout ;

Freud, qui enseigne que le sexe est tout ;

Einstein, qui enseigne... que tout est relatif.

La formule est jolie. On évitera pourtant de la prendre trop au sérieux.
Quant au fond, ce n’est qu’une série de contresens. Si la Loi était tout, il n’y
aurait plus besoin de Loi. Si l'amour était tout, il n’y aurait pas de monde (nous
serions déjà au paradis) et Jésus serait venu pour rien. Si l’argent était tout, il
n'y aurait pas de marxisme. Si le sexe était tout, pas de psychanalyse. Enfin, si
tout était relatif, quel sens y aurait-il à affirmer la supériorité de la Théorie de
la Relativité sur l’astronomie ptolémaïque ou la mécanique céleste de Newton ?
Mais essayons d’abord de définir.

Qu'est-ce que le relatif ? Le contraire de l'absolu : est relatif ce qui est non
séparé et non séparable (sinon par abstraction), autrement dit ce qui existe en
autre chose (relativité des modes ou des accidents, absoluité de la substance) ou
en dépend (relativité des effets, absoluité d’une cause qui ne serait causée ou
influencée par rien). Dans les doctrines religieuses, on considère ordinairement
que Dieu seul est absolu : toutes les créatures émanent de lui ou en dépendent
et sont donc relatives ; lui seul ne dépend de rien. Les athées ou les matérialistes
diront plutôt que tout est relatif (toute cause est elle-même effet d’une autre
cause, et ainsi à l'infini, tout événement est influencé par d’autres événements),
à la seule exception peut-être du Tout lui-même, dont on ne voit guère comment
il pourrait résulter ou dépendre d’autre chose — sinon de lui-même ou de
son état antérieur, C’est ainsi que nous sommes au cœur de l'absolu, tout en
étant voués au relatif.

%— 496 —
%{\footnotesize XIX$^\text{e}$} siècle — {\it }
Il faudrait donc donner raison à Einstein ? Sans doute, mais en évitant le
contresens habituel sur la Théorie de la Relativité. Celle-ci n’indique aucunement
que tout est relatif, au sens trivial du terme, c’est-à-dire subjectif ou
variable (relatif à un certain sujet ou à un certain point de vue). La théorie
d’Einstein, posant la relativité respective de l’espace et du temps, débouche au
contraire sur un certain nombre d’invariants, à commencer par la vitesse de la
lumière ou l’équivalence de la masse et de l'énergie, qui ne dépendent, précisément,
ni du sujet ni du point de vue. En ce sens, elle n’est pas {\it plus relative} mais
{\it plus absolue} que celle de Newton, qu’elle explique (comme un cas particulier :
pour des vitesses et des distances point trop grandes) et qui ne l'explique pas.
Que tout soit relatif (dans le tout, seul absolu), cela n'autorise pas à penser
n'importe quoi, ni n'importe comment.

\section{Relativisme}
%RELATIVISME
Toute doctrine qui affirme l'impossibilité d’une doctrine
absolue. En ce sens large, ce n’est qu’un truisme. Comment
un esprit fini pourrait-il avoir accès absolument à l’absolu, si l'absolu est un
esprit infini ou n’est pas {\it esprit} du tout ? Le relativisme ne prend son véritable
sens qu’en se particularisant, sous deux formes principales. Il faut en effet distinguer
un relativisme épistémique ou gnoséologique, d’une part, et un relativisme
éthique ou normatif, d’autre part. Les deux peuvent aller de pair (par
exemple chez Montaigne), mais aussi séparément (par exemple chez Spinoza,
qui ne relève que du second, ou chez Kant, qui ne relève que du premier).

Le relativisme épistémique ou gnoséologique affirme la relativité de toute
connaissance : nous n’avons accès à aucune vérité absolue. C’est le contraire du
dogmatisme théorique. Un scepticisme ? Pas forcément, puisqu’une connaissance
relative n’en est pas moins connaissance pour autant, et peut même, au
moins dans son ordre, être considérée comme certaine. Montaigne ou Hume
sont assurément relativistes, en ce sens ; mais Kant, qui n'était pas sceptique,
l’est également, comme d’ailleurs, aujourd’hui, la plupart de nos savants. C’est
l’un des résultats paradoxaux de la physique quantique. Mieux ils connaissent
le monde, moins ils ont le sentiment de le connaître absolument.

Quant au relativisme éthique ou normatif, il porte sur les valeurs, dont
il affirme la relativité. Nous n’avons accès à aucune valeur absolue ; tout
jugement de valeur est relatif à un certain sujet (subjectivisme), à certains
gènes (biologisme), à une certaine époque (historicisme), à une certaine
société ou culture (sociologisme, culturalisme) — voire, c’est d’ailleurs ce que
je crois, à tout cela à la fois. C’est le contraire du dogmatisme pratique. Un
nihilisme ? Pas forcément. Une valeur relative n’en est pas moins réelle pour
autant, ni ne cesse pour cela de valoir. Que la valeur d’une marchandise, par
%— 497 —
%{\footnotesize XIX$^\text{e}$} siècle — {\it }
exemple, ne soit pas absolue (elle dépend des conditions de sa production, du
marché, de la monnaie......), cela ne signifie pas que cette marchandise ne vaille
rien, ni que son prix soit arbitraire. Que la compassion, de même, soit diversement
appréciée (en fonction des cultures, des époques, des individus...), cela
n'entraîne pas qu’elle soit sans valeur, ni qu’elle ne vaille pas mieux, par
exemple, que l'indifférence ou la cruauté. Je dirais même plus : c’est uniquement
à la condition d’exister comme valeur, pour tel ou tel groupe humain,
qu'une valeur peut être relative, ce qu’un pur néant ne saurait être. Le relativisme,
loin de déboucher nécessairement sur le nihilisme (qui n’est que sa
forme outrancière, un peu comme le scepticisme outré que dénonçait Hume
l'est au relativisme gnoséologique, ou scepticisme modéré, qu’il professe), est
plutôt une raison forte de le refuser (intellectuellement) et de lui résister (moralement),
en même temps, et au fond pour la même raison, qu’au dogmatisme
pratique. Ces deux dernière positions ont en effet en commun de ne vouloir
reconnaître de valeurs qu’absolues : les uns affirment qu’il en existe de telles
(dogmatisme pratique), d’autres le nient (nihilisme), mais ils ne s’opposent en
cela que sur la base d’un accord premier, qu’on peut appeler leur absolutisme
commun. Les relativistes sont moins exigeants et plus lucides. Ce n’est pas
l'absolu qu’ils cherchent, ni le néant qu’ils trouvent. Ils s'intéressent aux conditions
réelles du marché (pour les valeurs économiques), de l’histoire, de la
société et de la vie (pour les valeurs morales, politiques ou spirituelles), et en
trouvent plus qu’assez pour en vivre et même, le cas échéant, pour en mourir.
J'attends qu’on m'explique quelles raisons un nihiliste pourrait avoir de combattre
la barbarie au péril de sa propre vie, et pourquoi il faudrait, pour la combattre,
se réclamer de valeurs absolues. « Parce que sinon, m’a-t-on souvent
répondu, le barbare vous opposera ses valeurs à lui : par exemple, s’il s’agit d’un
nazi, la pureté de la race, le culte du chef, de la nation et de la force, qu’il opposera
à votre respect efféminé ou judaïsé des droits de l’homme... » Je réponds
que c’est en effet ce qui se passe, et que je trouve curieux qu’on m’objecte ainsi
le réel même qui me donne raison. Qu'un nazi soit nazi au nom de certaines
valeurs, et qu’un démocrate le combatte au nom d’autres valeurs, c’est une
donnée de fait, qui prouve que ces valeurs existent, au moins pour nous, au
moins par nous, et suffisent. Si vous avez besoin, pour être antinazi, que
l’absolu le soit aussi, libre à vous. Mais imaginez que Dieu soit nazi et nous le
fasse savoir : deviendriez-vous nazi pour autant ? Ou qu’il n’y ait pas d’absolu
du tout: renonceriez-vous pour cela à respecter les droits de l’homme ?
Curieuse morale, qui dépend d’une métaphysique douteuse, comme elles sont
toutes !

Que toute valeur soit relative, cela ne prouve aucunement que rien ne
vaille. Cela le rend même improbable : comment un néant serait-il relatif ? Le
%— 498 —
%{\footnotesize XIX$^\text{e}$} siècle — {\it }
nihilisme n’est qu’un relativisme outré, ou vautré. Le relativisme, à l'inverse, est
un nihilisme ontologique (s'agissant des valeurs : elles ne sont pas des êtres ni
des Idées en soi) mais doublé d’un réalisme pratique (les valeurs existent réellement,
au moins pour nous, puisqu'elles nous font agir, ou puisque nous agis-
sons pour elles). Une valeur n’est pas une vérité : elle est l’objet d’un désir, non
d’une connaissance ; elle relève de l’action, non de la contemplation. Mais elle
n’est pas non plus un pur néant, ni une simple illusion : elle vaut vraiment, au
moins pour nous, au moins par nous, puisqu'il est vrai que nous la désirons. Ce
qu’il y a d’illusoire, dans nos valeurs, ce n’est pas leur valeur, mais le sentiment
que nous avons, presque inévitablement, de leur absoluité. Ou pour le dire
autrement : il n’y a d’absolu moral que pour et par la volonté. C’est ce que
j'appelle un absolu pratique : ce que je veux absolument, c’est-à-dire de façon
inconditionnelle ou non négociable. Parce que cela existerait en soi ? Nullement.
Mais parce que cela est indissociable de mon désir de vivre et d’agir
humainement. L'essentiel est exprimé par Spinoza, dans le décisif scolie de la
proposition 9 du livre III de l'{\it Éthique} : « Nous ne faisons effort vers rien, ne
voulons, n’appétons ni ne désirons aucune chose parce que nous jugeons
qu’elle est bonne ; mais, au contraire, nous jugeons qu’une chose est bonne
parce que nous faisons effort vers elle, parce que nous la voulons, appétons et
désirons. » Que toute valeur soit relative au désir qui la vise (donc à la vie, à
l’histoire, à l'individu : {\it au désir biologiquement, historiquement et biographiquement
déterminé}), ce n’est pas une raison pour cesser de la désirer, ni pour prétendre
que ce désir (qui peut lui-même être désiré) est sans valeur. Quand tu
bandes, as-tu besoin que Dieu ou la vérité bande aussi ? Pourquoi faudrait-il,
pour aimer la justice, qu’elle existe absolument ? C’est plutôt l’inverse : si elle
existait, elle n’aurait pas besoin de nous et nous serions dès lors moins tenus de
l'aimer. Mais cela n’est point. Ce n’est pas parce que la justice est bonne qu’il
faut l'aimer, ni parce qu’elle existe qu’il faut s’y soumettre. C’est parce que
nous l’aimons qu’elle est bonne (raison de plus pour l'aimer : elle ne vaut qu’à
cette condition !), et parce qu’elle n’existe pas, comme disait Alain, qu’il faut la
faire. Nihilisme, philosophie de la paresse ou du néant. Relativisme, philosophie
du désir et de l’action.

\section{Religion}
%RELIGION
Un ensemble de croyances et de pratiques qui ont Dieu, ou des
dieux, pour objet. Cela fait lien (selon une étymologie possible
et douteuse, qui rattache le mot à {\it religare} : la religion relie les croyants entre
eux, en les reliant tous à Dieu) et sens (puisqu'il existe autre chose que ce
monde, qui peut être son but ou sa signification). Qui n’en rêverait ? Toutefois
rien ne prouve que ce soit autre chose qu’un rêve.

%— 499 —
%{\footnotesize XIX$^\text{e}$} siècle — {\it }
« Croire en un Dieu, disait Wittgenstein, c’est voir que la vie a un sens. »
Disons que c’est le croire, et prendre ce sens au sérieux. Par quoi la religion est
le contraire de l'humour et de la connaissance : c’est le sens du sens enfin saisi,
fût-ce obscurément, recueilli (selon une autre étymologie tout aussi possible et
douteuse : {\it religere}, recueillir), perpétuellement relu (troisième étymologie :
{\it relegere}, relire), à la fois hypostasié et adoré. Comme ce sens est toujours absent,
la religion se fait espérance et foi. Cela, qui nous manque (le sens), ne manque
de rien — et nous sera donné, un jour. Reste, d’ici là, à prier, à croire, à obéir.
Toute religion débouche sur une morale dogmatique ou en procède : le bien
érigé en vérité, le devoir en Loi, la vertu en soumission. Bossuet a résumé
l'essentiel en une phrase : « Tout le bien vient de Dieu, tout le mal de nous
seuls. » La religion est la honte de l'esprit. {\it Mea culpa, mea maxima culpa...}
C’est aussi ce qui la sauve, parfois. Mieux vaut la honte que limpudence (Spinoza,
{\it Éthique} IV, 58, scolie). Mieux vaut une vertu soumise que pas de vertu
du tout. Mieux vaut aimer Dieu que n’aimer rien ou que n’aimer que soi. Au
reste cet amour, comme tout amour, est une joie, et source de joies (or « tout
ce qui donne de la joie est bon » : {\it Éthique}, IV, appendice, 30), donc source
d'amour... C’est ce qu’il y a de fort dans la sainteté, et de vrai dans la religion.
J'ai connu quelques vrais croyants dont l’évidente supériorité, au moins par
rapport à moi, devait trop à leur foi pour que je m’autorise à la condamner. La
religion n’est haïssable que lorsqu'elle débouche sur la haine ou la violence : ce
n'est plus religion mais fanatisme.

\section{Réminiscence}
%RÉMINISCENCE
Dans le langage courant, c’est un souvenir involontaire,
voire partiellement inconscient ou méconnu comme
souvenir. Se dit surtout d’expériences sensorielles ou affectives (la madeleine de
Proust) ; ce sont des souvenirs qui s'imposent à nous, mais comme venant de
très loin, au point souvent d’en rester mystérieux ou méconnaissables. Par
exemple en art : on parle de {\it réminiscence} quand on croit reconnaître, dans une
œuvre donnée, la trace, mais involontaire et souvent inconsciente, d’un artiste
antérieur. Ce n'est pas un plagiat. Ce n’est pas une citation. Ce n’est même pas
une allusion ou un clin d’œil. C’est un écho, mais inaperçu du créateur, d’une
création autre. Cela fera le bonheur, plus tard, des érudits. Il faut bien que tout
le monde vive. Cela fait, surtout, un peu de l'épaisseur impersonnelle, ou transpersonnelle,
de l’art vrai. Le génie est le contraire d’une table rase.

Dans le langage philosophique, le mot sert surtout à traduire l’{\it anamnèsis}
des Grecs. Il peut alors désigner, à l'opposé du sens précédent, la recherche ou
la mobilisation volontaire d’un souvenir (Aristote, {\it De la mémoire}, 2 ; mieux
vaudrait, me semble-t-il, traduire par {\it remémoration} ou {\it anamnèse}). Mais il fait
%— 500 —
%{\footnotesize XIX$^\text{e}$} siècle — {\it }
plus souvent référence à Platon : la réminiscence est la trace en nous des Idées
éternelles, que notre âme aurait perçues, entre deux incarnations, face à face.
C’est ce qui nous permet, comme le petit esclave du {\it Ménon}, de découvrir, sans
sortir de nous-mêmes, des vérités que nous ignorions. Connaître ne serait que
reconnaître ; penser, que se ressouvenir. Si c'était vrai, l’histoire des sciences
n’avancerait qu’à reculons, vers une vérité qui la précède. C’est en effet le cas,
dira-t-on, puisque toute vérité est éternelle. Mais non. Qu'elle soit éternelle
est au contraire ce qui interdit de l’enfermer dans le passé : elle est tout autant
dans l'avenir, qui n’est rien, et davantage dans le présent, qui la contient toute.
La réminiscence platonicienne n’est qu’une métaphore, comme l’éternel retour
de Nietzsche, pour penser l'éternité du vrai.

\section{Remontrance}
%REMONTRANCE
C’est montrer à quelqu’un le mal qu’il a fait. Cela n’est
guère utile qu'avec les enfants, à condition de le faire
avec délicatesse, et avec les puissants, à condition de le faire sans démagogie.

\section{Remords}
%REMORDS
Une tristesse présente, pour une faute passée, comme une honte
de soi à soi. C’est un sentiment, qui peut être déchirant, non
une vertu. Le remords touche pourtant à la morale, par le jugement (la conscience
douloureuse d’avoir mal agi). C’est comme une nostalgie du bien. Se
mue en repentir, quand s’y ajoute la volonté de se reprendre.

\section{Renaissance}
%RENAISSANCE
Le fait de renaître. Mais le mot, en philosophie, est plus
souvent utilisé avec une majuscule. Il désigne alors une
époque, un mouvement ou un concept.L'époque couvre les
{\footnotesize XV$^\text{e}$} et {\footnotesize XVI$^\text{e}$} siècles.
Le mouvement, qui part de l'Italie du Nord, se déploie progressivement dans
toute l’Europe : c’est celui de la redécouverte conjointe de l'Antiquité et de
l'individu. Il en naît un {\it ars nova}, qui est comme la pointe extrême et exquise
de ce mouvement. Mais ce n’est que sa pointe. La Renaissance touche aussi à
l’économie, ou est touchée par elle (c’est l’époque où émerge ce que nous appelons
aujourd’hui le capitalisme), à la politique (par le renforcement des Cités
ou des États), à la pensée (par le progrès des sciences et de l’humanisme), à la
spiritualité (par la Réforme puis la Contre-Réforme), enfin et en général à la
conception du monde (aussi bien par la découverte de l'Amérique que par le
passage, comme dira Koyré, du monde clos, celui des Anciens et du Moyen
Âge, à l'univers infini, celui des Modernes). C’est l’époque de Brunelleschi et
de Gutenberg, de Donatello et de Van Eyck, d’Érasme et de Rabelais, de
%— 501 —
%{\footnotesize XIX$^\text{e}$} siècle — {\it }
Machiavel et de Montaigne, de Copernic et de Christophe Colomb, mais
aussi de Luther et Giordano Bruno, de Van der Weyden et Dürer, de Josquin
Des Prés et Palestrina, de Léonard de Vinci et Michel-Ange, de Raphaël et du
Titien.... Époque admirable, plus qu'aucune autre peut-être, au moins pour
les arts plastiques, et les contemporains ne s’y trompèrent pas. Voici par
exemple ce qu'Alberti écrivait dans la dédicace adressée à Brunelleschi,
excusez du peu, de son traité {\it De la peinture} : « Pour les Anciens, qui avaient
des exemples à imiter et des préceptes à suivre, atteindre dans les arts
suprêmes ces connaissances qui exigent de nous tant d’efforts aujourd’hui
était sans doute moins difficile. Et notre gloire, je l'avoue, ne peut être que
plus grande, nous qui, sans précepteurs et sans exemples, avons créé des arts
et des sciences jamais vus ou entendus. » On voit que la {\it Rinascita}, comme on
disait dès le Quattrocento, ne s’enferme nullement dans la nostalgie de
l'Antiquité : admiration pour les Anciens n'exclut pas une admiration
redoublée pour les contemporains, quand ils se montrent, et dans des conditions
peut-être plus difficiles, à la hauteur de leurs glorieux et antiques prédécesseurs.
Il reste que la Renaissance n’est pas seulement un progrès ou une
éclosion ; elle ne fut elle-même qu’en retrouvant quelques-uns des secrets ou
des idéaux de l’Antiquité. C’est ce qui nous mène au concept : on peut parler
de Renaissance, en un sens plus général mais qui reste analogique, pour tout
mouvement de renouveau qui se fonde sur un retour — au moins partiel, au
moins provisoire — à une époque plus ancienne. Le mot, qui continue de
valoir positivement, peut alors prendre un sens prospectif. Travailler à une
Renaissance, c’est reconnaître une décadence préalable, à quoi l’on essaie
d'échapper. C’est remonter vers la source, mais pour ne point renoncer à
l’océan. Reculer, au moins en apparence, mais pour avancer. C’est donc le
contraire d’une position réactionnaire ou conservatrice : un progressisme
cultivé et fidèle, qui veut éclairer l’avenir par l'étude patiente du passé, et qui
préfère rivaliser avec les maîtres d’autrefois, comme disait Fromentin, plutôt
qu'avec les contemporains ou les journaux.

\section{Renommée}
%RENOMMÉE
Moins que la gloire, plus que la réputation (qui peut être
mauvaise) ou la notoriété (qui est neutre). On peut parler
d’un criminel notoire. On hésiterait à parler d’un criminel renommé. Non
pourtant que la renommée tienne lieu de jugement de valeur — un écrivain
renommé peut être un médiocre écrivain —, mais en ceci plutôt qu’elle véhicule
les jugements de valeur {\it des autres}. C’est pourquoi ses trompettes, comme disait
Brassens, sont mal embouchées. C’est qu’on n’y souffle pas soi-même.

%— 502 —
%{\footnotesize XIX$^\text{e}$} siècle — {\it }
\section{Repentir}
%REPENTIR
« C’est une espèce de tristesse, disait Descartes, qui vient de ce
qu’on croit avoir fait quelque mauvaise action ; et elle est très
amère, parce que sa cause ne vient que de nous. Ce qui n'empêche pas néanmoins
qu’elle soit fort utile » ({\it Passions}, III, 191). Cette définition pourrait
valoir aussi bien, et peut-être mieux, pour le remords. Descartes distingue ces
deux affections par le doute, qui serait présent dans le remords, absent dans
le repentir. Mais cet usage ne s’est pas imposé. Janet, ici, est plus éclairant :
« {\it Remords} se distingue de {\it repentir}, qui désigne un état d’âme plus volontaire,
moins purement passif. Le repentir est déjà presque une vertu ; le remords
est un châtiment » ({\it Traité de philosophie}, p. 655, cité par Lalande). Disons
que le remords n’est qu’un sentiment, quand le repentir est déjà une volonté :
c’est la conscience douloureuse d’une faute passée, jointe à la volonté de
l’éviter désormais et de la réparer si possible. Une vertu ? C’est ce que contestait
Spinoza. D'abord parce que tout repentir suppose la croyance au libre
arbitre ({\it Éthique}, III, déf. 27 des affects), et est illusoire par là. La connaissance
des causes et de soi vaudrait mieux. Ensuite, parce que c’est une tristesse,
quand il n’est de vertu vraie que joyeuse. Enfin parce que le repentir
n’est que le sentiment d’une impuissance, non la connaissance d’une puissance
même limitée : « Le repentir n’est pas une vertu, c’est-à-dire qu'il ne
tire pas son origine de la raison ; celui qui se repent de ce qu’il a fait est deux
fois misérable ou impuissant » ({\it Éthique}, IV, prop. 54 ; voir aussi la démonstration,
qui renvoie à celle de la prop. 53, sur l'humilité). La première misère
est d’avoir mal agi ; la seconde, de mal penser. Pourtant le repentir, comme
la honte, vaut mieux que la bonne conscience du salaud satisfait : « La honte,
quoiqu'’elle ne soit pas une vertu, est bonne cependant, en tant qu’elle dénote
dans l’homme envahi par la honte un désir de vivre honnêtement, tout
comme la douleur, qu’on dit bonne en tant qu’elle montre que la partie
blessée n’est pas encore pourrie. Bien qu’il soit triste donc, en réalité,
l’homme qui a honte de ce qu’il a fait est cependant plus parfait que l'impudent
qui n’a aucun désir de vivre honnêtement » (IV, 58, scolie). On retrouve
l'idée de Janet, selon laquelle le repentir est {\it presque} une vertu. Ce n’en est pas
une, puisque ce n’est pas un acte. Mais cela peut y mener, et même ce n'est
repentir (et non simplement remords) qu’à la condition d’y mener au moins
en partie.


\section{Représentation}
%REPRÉSENTATION
Tout ce qui se présente à l'esprit, ou que l'esprit se
représente : une image, un souvenir, une idée, un fan-
tasme... sont des représentations.

%— 503 —
%{\footnotesize XIX$^\text{e}$} siècle — {\it }
C’est ce qui faisait dire à Schopenhauer que «le monde est ma repré-
sentation » — car je ne sais rien de lui, hormis ce que j’en perçois ou en pense.
Mais s’il n’y avait que des représentations, que représenteraient-elles ?

\section{Réprobation}
%RÉPROBATION
Un jugement de valeur négatif, sur l’acte d’un autre. Cela
ne touche à la morale que par les conséquences qu’on en
tire pour soi-même. Sans quoi ce n’est que médisance ou bonne conscience. La
miséricorde et le silence, presque toujours, valent mieux.

\section{République}
%RÉPUBLIQUE
Étymologiquement, c’est la chose publique ({\it res publica}). Mais
le mot désigne bien davantage : une forme d’organisation de
la société et de l’État, dans laquelle le pouvoir appartient à tous, au moins en
droit, et s'exerce, au moins en principe, au bénéfice de tous. Selon une formule
traditionnelle, c’est le pouvoir du peuple, par le peuple, pour le peuple — même
si ce pouvoir s'exerce le plus souvent par l’intermédiaire de représentants élus.
C’est donc une démocratie, mais radicale. Il peut se faire qu’une démocratie ait
un roi, si le peuple le juge bon ou laccepte (lAngleterre et l’Espagne,
aujourd’hui, sont assurément des démocraties : c’est le peuple, non le roi, qui y
décide de la politique suivie, et même du maintien ou non de la monarchie) ;
mais alors ce n’est pas une république (puisqu’une partie du pouvoir, en
l'occurrence le choix du monarque, échappe au peuple). En ce premier sens,
qui est constitutionnel, la république est donc une démocratie où tout le pouvoir
appartient au peuple et ne s’exerce que par ses élus : la France, les USA ou
l’Allemagne sont des républiques ; l'Angleterre et l'Espagne, non.

Il peut se faire aussi, et il se fait souvent, que le pouvoir, dans une démocratie,
ne se mette au service que des plus influents ou des plus nombreux ;
mais alors, même sans roi, ce n’est plus tout à fait une république, qui veut que
le pouvoir vise l'intérêt commun et non la simple sommation ou moyenne des
intérêts particuliers. On voit que le mot, en ce dernier sens, est moins constitutionnel
que normatif : il suppose un jugement de valeur, et comme une volonté
obstinée de résister aux égoïsmes, aux privilèges, aux corporatismes, aux Églises,
et même aux individus. La liberté ? Assurément. Mais pas au prix de l'égalité,
de la laïcité, de la justice. La république, en ce sens, est moins un type de gouvernement
qu’un idéal ou un principe régulateur : être républicain, c’est vouloir
que la démocratie se mette au service du peuple, non à celui, comme c’est
sa pente naturelle, de la majorité ou de l'idéologie dominante. On comprend
que cela ne dispense pas de respecter la démocratie, ni n’autorise à violer, fût-ce
dans l’intérêt du peuple, les libertés individuelles. Qui peut le plus peut le
%— 504 —
%{\footnotesize XIX$^\text{e}$} siècle — {\it }
moins. La démocratie, pour un républicain, est le minimum obligé ; la république,
le maximum souhaitable.

\section{Résignation}
%RÉSIGNATION
C’est renoncer à la satisfaction d’un désir, qui subsiste
pourtant. Ce n’est plus la révolte, qui dit non, ni tout à fait
Pacceptation, qui dit oui. La résignation dirait plutôt « {\it oui mais} », ou « {\it oui
malgré tout} », ou « {\it tant pis} », mais sans y croire tout à fait. C’est comme un travail
du deuil inachevé, peut-être inachevable, qui s’accepte tel. Ce n’est pas la
sagesse, faute de joie. Ce n’est pas — ou plus — le malheur. C’est une espèce
d’entre-deux morne et confortable. Double piège. Double échec. Trop confortable
pour qu’on veuille en sortir. Trop morne pour qu’on se plaise à y rester.
Cest l’état souvent des vieilles gens, ou de ceux qui ont vieilli avant l'heure.
C’est ce qui la rend peu attirante. « Ce mot de résignation m'irrite, disait
George Sand ; dans l’idée que je m’en fais, à tort ou à raison, c’est une sorte de
paresse qui veut se soustraire à l’inexorable logique du malheur » ({\it Histoire de
ma vie}, X). Mais elle ne le peut, et encore, que par l'habitude et le renoncement.
Ce n’est pas une victoire ; c’est un abandon. C’est ce qui la rend nécessaire,
parfois, et insuffisante toujours. C’est comme une sagesse minimale, pour
ceux qui seraient incapables de la vraie. Sa formule semble se trouver, étonnamment,
dans les {\it Nourritures terrestres} de Gide : « Où tu ne peux pas dire {\it tant
mieux}, dis {\it tant pis}. Il y a là de grandes promesses de bonheur. » C’est trop dire
sans doute, ou cela suppose un bonheur préalable, ou une sagesse ultime, qui
n’est plus résignation mais acceptation pleine et entière. À côté de quoi la résignation
n’est qu’un moment. Elle ne vaut vraiment que pour ceux qui ne s’y
résignent pas, où qui la dépassent. Ce n’est un chemin qu’à la condition d’en
sortir.

\section{Résistance}
%RÉSISTANCE
Une force, en tant qu’elle s'oppose à une autre. C’est l’état
ordinaire du {\it conatus} : tout être s'efforce de persévérer dans
son être, et s’oppose par là, autant qu’il le peut, à ceux qui le pressent, l’agressent
ou le menacent. Ainsi la résistance d’un corps, contre un autre qui l’écrase.
D'un organisme, contre les microbes. De la vie, contre la mort. D’un homme
libre, contre les tyrans.

Destutt de Tracy et Maine de Biran voyaient dans la résistance des corps
extérieurs à notre action sur eux l’une des sources (avec l'effort, qui en est indissociable)
de notre conscience et de nous-mêmes et de quelque chose (le monde)
qui n’est pas nous. C’est une récusation en acte du solipsisme. Mais c’est sans
doute Spinoza qui, pour penser la résistance, est le plus précieux. La résistance
%— 505 —
%{\footnotesize XIX$^\text{e}$} siècle — {\it }
n’est pas un accident, ni la marque de je ne sais quelle pensée réactive. Elle est
la vérité de l'être, en tant qu'il est puissance d’exister et d’agir, dès lors que cette
puissance est une (dans la substance) et multiple (par les modes). Seul l'infini
est pure affirmation ({\it Éthique}, I, 8, scolie). Toute chose finie peut être limitée
({\it Éth.} I, déf. 2) ou détruite ({\it Éth.} IV, axiome) par une autre de même nature.
C’est ce qui la voue à la résistance. Exister, c’est insister (s’efforcer d’être et de
durer) ; mais c’est aussi, par là même, résister : le {\it conatus} est cette « puissance
singulière d’affirmation et de résistance » (Laurent Bove, {\it La stratégie du conatus,
Affirmation et résistance chez Spinoza}, Vrin, 1996, p. 14) par quoi chaque être
fini tend à persévérer dans son être en résistant à l’écrasement ou à l'oppression.
Cela vaut en particulier pour l'être humain ({\it Éthique}, IV, prop. 3), qui résiste à
la tristesse et à la mort. L’éthique spinoziste est une éthique de la puissance et
de la joie. Mais c’est aussi, par à même, « une éthique de la résistance et de
l'amour » (L. Bove, {\it op. cit.}, p. 139 sq.). La politique de Spinoza est une politique
de la puissance et de la liberté. C’est pourquoi elle débouche sur une stratégie
de la résistance et de la souveraineté : si « c’est l’obéissance qui fait les
sujets » ({\it T. Th.P.}, XVII), « c’est la résistance qui fait les citoyens » (L. Bove, {\it op.
cit.}, p. 301). Pas étonnant qu’Alain ait rêvé de fonder «le parti Spinoza » :
« Obéir en résistant, disait-il, c’est tout le secret. Ce qui détruit l’obéissance est
anarchie ; ce qui détruit la résistance est tyrannie » (Propos du 24 avril 1911 ;
voir aussi mon article sur la philosophie politique d’Alain, « Le philosophe
contre les pouvoirs », {\it Revue internationale de philosophie}, n° 215, 2001, spécialement
aux p. 150 à 160). Pas étonnant que Cavaillès, l’un de nos plus grands
résistants au nazisme, se soit toujours dit spinoziste : il trouvait dans cette
pensée de quoi éclairer son combat. La Résistance, telle que Cavaillès et
d’autres la menaient, était la seule façon, face à la barbarie nazie, de persévérer
— quitte à en mourir — dans leur être de citoyens et d'hommes libres.

On parle aussi de {\it résistance} en psychanalyse. C’est une force, explique
Freud, qui s'oppose à la conscience et à l’analyse : elle empêche les représentations
inconscientes de remonter à la surface ou en déforme les manifestations.
Cette résistance résulte du refoulement, ou plutôt c’est comme un refoulement
continué : elle en maintient l’efficace et par là le confirme. C’est un obstacle,
pendant la cure, en même temps qu’un matériau.

\section{Résolution}
%RÉSOLUTION
« Rien de plus facile que d’arrêter de fumer : je l’ai fait au
moins cent fois ! » Cette boutade marque à peu près la dis-
tance qu’il y a entre la {\it décision}, qui est une volonté instantanée, et la {\it résolution},
qui serait une volonté continuée. C’est vouloir vouloir, mais dans la durée :
vouloir (aujourd’hui) vouloir encore (demain ou dans dix ans). Par exemple
%— 506 —
%{\footnotesize XIX$^\text{e}$} siècle — {\it }
celui qui arrête de fumer, en effet, ou qui commence des études difficiles : une
décision n’y suffit pas ; encore faudra-t-il maintenir cette décision dans le
temps. Il s’agit moins de vouloir, en l’occurrence, que de vouloir continuer à
vouloir. Mais comment, puisqu'on ne peut vouloir qu’au présent ? La résolution
voudrait s’armer d’avance contre la lassitude, le renoncement, la versatilité.
Ce n’est qu’un leurre. Il faudra vouloir à nouveau chaque jour, et à chaque instant
de chaque jour. La résolution est simplement l’état d’une volonté qui le
sait et s’y prépare. Ce n’est pas un gage suffisant de réussite. Mais son absence,
presque toujours, est gage d’insuccès.

\section{Respect}
%RESPECT
Le sentiment en nous de la dignité de quelque chose (spécialement
de la loi morale, chez Kant) ou de quelqu'un (une personne).
On s’est parfois étonné que je n’en fasse pas l’une des grandes vertus de
mon {\it Petit traité}. C’est qu’elle m’a semblé équivoque. Dire de quelqu'un qu’il
est {\it respectueux}, ce n’est pas toujours ni souvent souligner l’une de ses vertus.
On imagine déjà des courbettes, des complaisances, des hiérarchies, toute la
gymnastique de l’intérêt et des grandeurs d’établissement — moins le sentiment
de la dignité de l’autre qu’un oubli de l’égale dignité de tous. Bien des fois, c’est
l’irrespect, spécialement face aux puissants, qui serait nécessaire et méritoire.
Voyez Diogène ou Brassens. Quant au respect qu’on doit aux plus faibles ou à
tous, la politesse, la compassion et la justice en disent l’essentiel. « Le devoir de
respecter mon prochain, écrit Kant, est compris dans la maxime de ne ravaler
aucun autre homme au rang de pur moyen au service de mes fins » ({\it Doctrine de
la vertu}, \S 25). C’est l'antidote de l’égoïsme, et comme le contrepoids de
l'amour (qui incite les humains à se rapprocher les uns des autres, alors que le
respect les conduit à maintenir entre eux une certaine distance : {\it ibid.}, \S 24).
C’est moins une vertu de plus que la conjonction de plusieurs. Le respect n’en
est pas moins nécessaire, ou plutôt il l’est d’autant plus, sans être pourtant
suffisant : il ne dispense ni d’amour ni de générosité. Il est vrai que la réciproque
est vraie aussi. L'amour et la générosité, sans respect, ne sauraient nous
satisfaire : ce ne serait que concupiscence ou condescendance.

\section{Responsabilité}
%RESPONSABILITÉ
« Responsable, mais non coupable. » La formule, dans
la bouche d’un ministre, avait choqué. Prise en elle-
même, elle n’était pourtant ni absurde ni contradictoire. Je suis responsable de
tout ce que j'ai fait volontairement, ou de tout ce que j’ai laissé faire et que
j'aurais pu empêcher. Ainsi suis-je responsable de mes erreurs. Aucun élève ne
demandera qu’on relève sa note, ou qu’on l’attribue à quelqu'un d’autre, sous
%— 507 —
%{\footnotesize XIX$^\text{e}$} siècle — {\it }
prétexte qu’il n’a pas fait exprès de se tromper. Aucun homme politique sérieux
ne demandera qu’on tienne ses échecs pour rien. Cela ne signifie pas qu’ils se
sentent coupables, ni qu’ils le soient. Je suis responsable de mes erreurs et de
mes échecs. Je ne suis coupable que des fautes que j’ai accomplies délibérément,
en sachant qu’elles étaient des fautes. C’est la différence qu’il y a, en voiture,
entre griller un stop qu’on n’a pas vu, et foncer délibérément sur quelqu'un.
S’il y a mort d'homme, on se sentira sans doute responsable dans les deux cas.
On ne sera coupable, au moins de cette mort, que dans le second (ce qui
n'exclut pas qu’on soit coupable, dans le premier, d’inattention, d’excès de
vitesse ou d’imprudence). Les tribunaux en tiennent compte, qui ne punissent
pas les ivrognes homicides, sur nos routes, aussi sévèrement que certains le
voudraient : c’est qu’ils sont coupables d’avoir conduit en état d'ivresse, mais
pas plus que tous ceux qui le font en ayant la chance de ne tuer personne. Que
l’on condamne ces derniers, quand ils sont pris, plus sévèrement qu’on ne le
fait, cela me paraît urgent. Mais faut-il pour autant traiter les premiers — qui
n'ont pas bu davantage mais qui ont eu moins de chance — comme des
assassins ? Ce ne serait plus justice mais vengeance. Ce sont des chauffards ?
sans doute, et cela mérite d’être sanctionné. Mais ce ne sont pas des assassins :
ils sont coupables de conduite en état d’ivresse ; ils sont responsables, mais
non coupables, de la mort d’un individu. Je ne dis pas cela, mes exemples
l’indiquent assez, pour exempter nos ministres. La responsabilité, en politique
comme ailleurs, suffit à justifier une sanction politique (la démission, le
renvoi, la non-réélection...). Seule la culpabilité mérite une sanction pénale.
Le ministre en question était-il coupable ? Ce n’est pas à moi d’en décider : je
n'ai pour cela ni compétence ni goût. Mais qu’il ait été responsable, avec
d’autres, de la mort de plusieurs centaines d’hémophiles et de transfusés fait
une charge assez lourde, dont il était légitime de tenir compte. Notre ministre
l’a d’ailleurs fait, au moins en partie. Il y avait quelque injustice, me semble-t-il,
à lui reprocher sa formule comme si elle était intrinsèquement absurde ou
lâche.

Être responsable, c’est pouvoir et devoir répondre de ses actes. C’est donc
assumer le pouvoir qui est le sien, jusque dans ses échecs, et accepter d’en supporter
les conséquences. Seul le très jeune enfant et le dément y échappent, et
cela dit peut-être l'essentiel : la responsabilité est Le prix à payer d’être libre.

\section{Ressentiment}
%RESSENTIMENT
La rancune des faibles. Le mot, dans la langue philosophique,
est définitivement marqué par Nietzsche : le ressentiment
est une « vengeance imaginaire », par laquelle les esclaves, incapables
d’agir, essaient de compenser leur infériorité réelle en condamnant fantasmatiquement
%— 508 —
%{\footnotesize XIX$^\text{e}$} siècle — {\it }
— par la morale et la religion — les barbares ou les aristocrates qui les
oppriment, dont ils ne peuvent autrement triompher. Ce mouvement marque
« la révolte des esclaves dans la morale » ({\it Généalogie...}, 1, 10), et c’est en quoi
les Juifs furent « le peuple sacerdotal du ressentiment par excellence » (I, 16).
Le ressentiment opère un renversement des valeurs (le « bon » des maîtres,
c’est-à-dire l’aristocrate, devient le « méchant » des esclaves), que Nietzsche
veut renverser à son tour. Le paradoxe de l’histoire, explique Nietzsche, c’est
que les faibles ont gagné. Parce qu’ils étaient beaucoup plus nombreux, beaucoup
plus sournois, beaucoup plus patients, beaucoup plus prudents... Le
temps, le nombre et la fatigue travaillent pour eux. En Europe, spécialement,
les Juifs n’ont cessé de gagner : en Grèce, avec « ce Juif de Socrate » (et avec
Platon, peut-être formé « chez les Juifs d'Égypte »), dans la « Rome judaïsée »,
celle de l'Église, dans la Réforme (« la Judée triompha de nouveau »), enfin,
« dans un sens plus décisif, plus radical encore, la Judée remporta une nouvelle
victoire sur l’idéal classique avec la Révolution française : c’est alors que la
dernière noblesse politique qui subsistait encore en Europe, celle des {\footnotesize XVII$^\text{e}$} et
{\footnotesize XVII$^\text{e}$} siècles français, s’effondra sous le coup des instincts populaires du ressentiment »
(I, 16). Ces lignes, qui sont désagréables (mais je pourrais citer bien
pire), ne sauraient toutefois suffire à invalider le concept de ressentiment, qui
reste éclairant. Elles doivent pourtant pousser à une certaine vigilance. Le
contraire du ressentiment, ou plutôt son symétrique, c’est le mépris, qui n’est
pas moins désagréable. Le ressentiment est la force des faibles ; le mépris, la faiblesse
des forts. Concepts utiles, affects dangereux. La miséricorde, dans les
deux cas, vaut mieux.

\section{Résurrection}
%RÉSURRECTION
Le fait de ressusciter, autrement dit de vivre à nouveau
alors qu’on était mort (se distingue par là de l’immorta-
lité), tout en restant le même individu, autrement dit le même composé âme-corps
(se distingue par là de la réincarnation). Ainsi Lazare ou Jésus. L’Ancien
Testament, sur le sujet, reste flou. La croyance en la résurrection n’apparaît
dans le judaïsme qu’assez tardivement, et plutôt comme un sujet de discorde :
les saducéens, si l’on en croit saint Paul, refusaient là-dessus de suivre les pharisiens
({\it Acte des Apôtres}, 23). Elle est en revanche, comme chacun sait, l’une des
pierres angulaires du christianisme. Le Christ est mort, il est ressuscité, et tel est
aussi le sort qui nous attend. Sous quelle forme ? On ne sait trop. Le {\it Credo}
annonce « la résurrection des corps », ce qui est bien embarrassant. Un corps,
même spirituel, doit avoir un âge, une forme, un certain aspect... Mais lequel ?
Est-ce le corps du vieillard, qui ressuscite, ou celui de l’adolescent ? Aura-t-il un
sexe et un ventre ? Aura-t-il les désirs qui vont avec ? les plaisirs qui vont avec ?

%— 509 —
%{\footnotesize XIX$^\text{e}$} siècle — {\it }
Sera-t-il beau ou laid, gros ou maigre, grand ou petit ? Comment serait-ce un
{\it corps} autrement ? La plupart des chrétiens jugent ces questions bien niaises : ils
préfèrent croire en l’immortalité de l’âme, comme Platon, et c’est en effet plus
commode. Mais qu’on ne parle plus, alors, de résurrection.

\section{Rêve}
%RÊVE
C’est comme une hallucination, mais qu’on n’aurait au sens propre
que pendant le sommeil: Cela suffit pourtant à faire peser un doute
sur notre état de veille. C’est l’argument de Descartes, dans la première
Méditation : « Je vois si manifestement qu’il n’y a point d’indices concluants,
ni de marques assez certaines par où l’on puisse distinguer nettement la veille
d’avec le sommeil, que j’en suis tout étonné ; et mon étonnement est tel, qu’il
est presque capable de me persuader que je dors. » C’est l'interrogation de
Pascal ({\it Pensées}, 131-434) : « Qui sait si cette autre moitié de la vie, où nous
pensons veiller, n’est pas un autre sommeil, un peu différent du premier ? »
Pourquoi différent ? Par la continuité, à quoi nous reconnaissons le réel, ou ce
que nous prenons pour tel. « Si nous rêvions toutes les nuits la même chose,
écrit ailleurs Pascal, elle nous affecterait autant que les objets que nous voyons
tous les jours. Et si un artisan était sûr de rêver douze heures durant qu’il est
roi, je crois qu'il serait presque aussi heureux qu’un roi qui rêverait toutes les
nuits douze heures durant qu’il est artisan » (802-122). J'aime beaucoup ce
{\it presque}, comme le {\it un peu} du fragment précédent, qu’on retrouve d’ailleurs
dans l’alexandrin parfait qui clôt celui-ci : « {\it Car la vie est un songe un peu moins
inconstant.} »

\section{Réversibilité}
%RÉVERSIBILITÉ
Ce qui peut se retourner sans perdre ses propriétés : par
exemple un manteau, si l’intérieur peut devenir l'extérieur,
ou un film, s’il peut se projeter indifféremment dans les deux sens. La notion sert
surtout en physique : les équations microscopiques sont réversibles ; les macroscopiques
ne le sont pas. C’est que le hasard et l’entropie, pour tout phénomène
complexe, interdisent qu’on revienne au point de départ. La tasse de café ne se
réchauffe pas toute seule ; les fleuves ne remontent pas vers leur source ; le
désordre, dans un système isolé, ne peut que s’accroître. C’est ce qu’on appelle la
flèche du temps, par quoi c’est l’irréversibilité qui est le vrai. « Ni temps passé /
Ni les amours reviennent / Sous le pont Mirabeau coule la Seine... »

\section{Révolte}
%RÉVOLTE
Une opposition résolue et violente : c’est le refus d’obéir, de se
soumettre, et même d’accepter. Le mot sert surtout, et de plus
%— 510 —
%{\footnotesize XIX$^\text{e}$} siècle — {\it }
en plus, pour désigner une attitude individuelle (pour les révoltes collectives,
on parlera plus volontiers d’émeute, de soulèvement, de révolution......). C’est
que Camus est passé par là. « Qu'est-ce qu’un homme révolté ? Un homme qui
dit non. Mais s’il refuse, il ne renonce pas : c’est aussi un homme qui dit oui,
dès son premier mouvement » ({\it L'homme révolté}, 1). Oui à quoi ? À sa révolte, à
son combat, aux valeurs qui le fondent ou en naissent. « Le révolté, au sens étymologique,
fait volte-face. Il marchait sous le fouet du maître. Le voilà qui fait
face. Il oppose ce qui est préférable à ce qui ne l’est pas. Toute valeur n’entraîne
pas la révolte, mais tout mouvement de révolte invoque tacitement une valeur »
({\it ibid.}). Cette «affirmation passionnée » est ce qui distingue la révolte du
ressentiment : « Apparemment négative, puisqu'elle ne crée rien, la révolte est
profondément positive, puisqu'elle révèle ce qui, en l’homme, est toujours à
défendre » ({\it ibid.}). C’est où l’on sort de la solitude : la révolte est « un lieu
commun qui fonde sur tous les hommes la première valeur. Je me révolte, donc
nous sommes » (ibid.). C’est où l’on sort du nihilisme. C’est où l’on sort,
même, de la révolte, ou plutôt c’est où, sans en sortir, on l’inclut dans un
ensemble plus vaste, qui est la vie, dans une valeur plus haute, qui est l’humanité.
La révolte est « le mouvement même de la vie » ; elle est « amour et fécondité,
ou elle n’est rien » ({\it op. cit.}, V). « L'homme est la seule créature qui refuse
d’être ce qu’elle est» ({\it op. cit.}, Introduction). Mais cela, au moins, il faut
l’accepter. Ainsi la révolte n’est qu’un passage, entre l’absurde et l’amour, entre
le {\it non} et le {\it oui}. C’est pourquoi il faut y entrer, puisque « l'absurde n’est qu’un
point de départ » ({\it ibid.}). C’est pourquoi on n’a pas le droit d’en sortir tout à
fait. Cela fait comme un rythme ternaire. D’abord le {\it non} du monde à l’homme
(l'absurde) ; puis le {\it non} de l’homme au monde (la révolte) ; enfin le grand {\it oui}
de la sagesse ou de l’amour (le «tout est bien » de Sisyphe). Mais ce {\it oui}
n’annule aucun des deux non qui le précèdent et le préparent. Il les prolonge.
Il les accepte. Cela vaut pour l’absurde, qui n’est qu’un point de départ mais
qui demeure inentamé (la sagesse n’est ni une justification ni une herméneutique).
Cela vaut plus encore pour la révolte. Dire {\it oui} à tout, ce qui est l’unique
sagesse, c’est dire {\it oui} aussi à ce {\it non} de la révolte et de l’homme.

\section{Révolution}
%RÉVOLUTION
Une révolte collective et triomphante : c’est une émeute qui
a réussi, au moins un temps, jusqu’à bouleverser les struc-
tures de la société ou de l’État. Les exemples archétypiques sont la Révolution
française de 1789 et la Révolution soviétique de 1917. Beaucoup de bonnes
raisons dans les deux cas. Beaucoup d’horreurs dans les deux cas. Mais une très
grande différence : on n’est jamais revenu tout à fait sur la première (Napoléon
l'installe au moins autant qu’il la clôt), alors que la seconde n’aura abouti, au

%— 511 —
%{\footnotesize XIX$^\text{e}$} siècle — {\it }
bout du compte, qu’à un capitalisme sous-développé, plus sauvage et plus
mafieux que le nôtre... C’est sans doute qu’il est moins difficile de transformer
l’État que la société (le féodalisme, pour l'essentiel, était déjà mort {\it avant} 1789),
plus facile de faire de nouvelles lois qu’une humanité nouvelle. Les fonctionnaires
finissent toujours par obéir. L'économie et l'humanité, non.

\section{Rhétorique}
%RHÉTORIQUE
L’art du discours (par différence avec l’éloquence, qui est
l’art de la parole), en tant qu ‘il vise à la persuasion. C’est
mettre la forme, avec son efficace propre, au service de la pensée. Par exemple
un chiasme, une antithèse ou une métaphore : cela ne prouve rien, cela n’est
même pas un argument, mais peut aider à convaincre. Il convient donc de ne
pas en abuser. Une rhétorique qui se voudrait suffisante ne serait plus rhétorique
mais sophistique. Elle n’en reste pas moins nécessaire, ou il serait bien
prétentieux de prétendre absolument s’en passer. Les meilleurs s’en servent.
Voyez Pascal et Rousseau : qu’ils aient été d’éblouissants rhéteurs ne les a pas
empêchés d’être des écrivains et des philosophes de génie. Il est vrai que Montaigne
finit par séduire davantage, par plus de liberté, d’inventivité, de spontanéité...
C’est qu’il se soucie moins de convaincre. La vérité lui suffit. La liberté
lui suffit. Cela ne signifie pourtant pas qu’il se soit passé entièrement de rhétorique,
mais simplement qu’il sut, mieux que d’autres, s’en libérer. Apprends
d’abord ton métier. Puis oublie-le.

\section{Ridicule}
%RIDICULE
« On ne prouve pas qu’on doit être aimé, écrit Pascal, en exposant
d’ordre les causes de l’amour ; cela serait ridicule » ({\it Pensées},
298-283). Pascal n’explique jamais. Cela fait une partie de son charme.
Essayons donc de comprendre. Ce qui est ridicule, c’est de confondre des
ordres différents, en l’occurrence l’ordre du cœur et celui de l'esprit ou de la
raison. C'était le début du fragment : « Le cœur a son ordre, l'esprit a le sien,
qui est par principe et démonstration. Le cœur en a un autre. » Essayez un peu
de démontrer rationnellement à quelqu'un qu’il doit vous aimer : son rire ou
son mépris donneront raison à Pascal, et il le citera peut-être : « Le cœur a ses
raisons que la raison ne connaît point » ({\it Pensées}, 423-277 ; voir aussi le fragment
110-282). Même chose pour le roi qui dit : « Je suis fort, donc on doit
m'aimer. » Son discours est faux et tyrannique, note Pascal ({\it Pensées}, 58-332) :
il confond l’ordre de la chair, où le roi règne et où la force l'emporte, avec les
ordres du cœur et de l’esprit, où la royauté ni la force ne sont rien. Même chose
enfin, mais on pourrait multiplier les exemples, pour celui qui s’étonnerait de
la basse extraction de Jésus-Christ : « Il est bien ridicule de se scandaliser de la

%— 512 —
%{\footnotesize XIX$^\text{e}$} siècle — {\it }
bassesse de Jésus-Christ, comme si cette bassesse était du même ordre duquel
est la grandeur qu’il venait faire paraître » ({\it Pensées}, 308-793). C'est toujours
confondre les ordres. Autant s'étonner que nos puissants ne soient pas des
saints.

Le ridicule, ce n’est donc pas seulement ce qui prête à rire (tout comique
n’est pas ridicule) : c’est ce qui prête à rire en confondant des ordres différents,
ou parce qu’on les confond. Cela rejoint le sens ordinaire du mot. Quelqu'un
fait un faux pas et tombe : si je le juge ridicule par là, ou s’il craint de lavoir
été, c’est que lui ou moi confondons l’ordre de la chair, où la pesanteur règne,
avec celui de l’esprit, où elle n’est rien. Par quoi toute tyrannie est ridicule, qui
veut faire adorer la force ou forcer la pensée à obéir ; et tout rire, contre les
tyrans, libérateur.

\section{Rire}
%RIRE
Mouvement involontaire et joyeux du visage et du thorax, face au
comique ou au ridicule. C’est une espèce de réflexe, mais qui ne va
guère sans un minimum de réflexion : on ne rit, presque toujours, que pour
autant qu’on a compris quelque chose — füt-ce, dans le comique de l'absurde,
qu'il n’y a rien à comprendre. Bergson voulait y voir « du mécanique plaqué sur
du vivant » : nous rions, disait-il, toutes les fois qu’une personne nous donne
l'impression d’une machine ou d’une chose ({\it Le rire}, I). J'aurais tendance, avec
Clément Rosset, à inverser la formule : à voir plutôt, dans ce qui nous fait rire,
«du vivant plaqué sur du mécanique (...) et se volatilisant à son contact »
({\it Logique du pire}, IV, 4). Un automate imitant un homme n’est presque jamais
drôle ; un homme ressemblant à une machine est presque toujours ridicule ou
comique, et d'autant plus qu’il s’en rend moins compte ou l’a moins décidé.
Ainsi, c'est un exemple qu’on trouvait chez Bergson, de cet orateur « qui
éternue au moment le plus pathétique de son discours ». Le corps se venge,
contre les simagrées de l'esprit. Le réel, contre les prétentions du sens. Par quoi
l'esprit se déprend de lui-même et de tout. C’est l'esprit vrai : du sens plaqué
sur du non-sens, et partant en éclats. C’est pourquoi, comme le notait Bergson,
«il n’y a pas de comique en dehors de ce qui est proprement humain » ou doué
d'intelligence. Parce qu’il n’y a de comique que par le sens, et de sens que pour
l'esprit — mais qui ne s’en amuse que dans la mesure où il cesse d’y croire. Nous
rions lorsque le sens se heurte plaisamment au réel, jusqu’à se pulvériser à son
contact. Rire : explosion de sens. Aussi peut-on rire de tout (tout sens est fragile,
tout sérieux, ridicule) ; c’est ce que prouve l'humour, et qui le rend nécessaire.
Deux formules me reviennent en mémoire, qui ont guidé ma jeunesse,
comme deux balises de l'esprit. Celle d’Épicure, balise souriante : {\it « Il faut rire
%— 513 —
%{\footnotesize XIX$^\text{e}$} siècle — {\it }
tout en philosophant »} (S.V., 41). Et celle de Spinoza, qui semble dire le
contraire : {\it « Non ridere, non lugere, neque detestari, sed intelligere »} (T. P., X, 4 :
« Ne pas rire, ne pas pleurer, ne pas détester, mais comprendre »). L’opposition,
entre les deux, n’est pourtant que de surface. Épicure n’a jamais cru que
le rire puisse suffire (la philosophie, avant comme après, reste nécessaire) ; ni
Spinoza, qu'on doive y renoncer. La célèbre formule du {\it Traité politique} n’est
en rien une condamnation du rire; ce n’est une condamnation que de la
raillerie, de la dérision, du rire haineux ou méprisant. Spinoza s’en explique
dans l’{\it Éthique} : ce n’est pas le rire qui est mauvais dans la raillerie, c’est la haine
(IV, 45, corollaire 1). Quant au rire considéré en lui-même, c’est au contraire
«une pure joie », dont on aurait bien tort de se priver. Pourquoi serait-il plus
légitime d’apaiser la faim et la soif que de combattre la mélancolie ({\it ibid.},
scolie ; voir aussi C. T., II, 11) ? On rit parfois de bonheur, plus souvent pour
surmonter l'angoisse ou la tristesse. Ce n’est pas un hasard si tant de nos humoristes
s’avouent, dans le privé, d'humeur sombre. Rions avant d’être heureux,
conseillait La Bruyère, de peur de mourir sans avoir ri.

\section{Roman}
%ROMAN
Un genre littéraire, sans autres contraintes que la narrativité et la
fiction. C’est raconter une histoire inventée comme si elle était
vraie, ou une histoire vraie comme si elle était inventée. Le mensonge est son
principe, mais aussi, bien souvent, son ressort principal. Il s’agit de rendre intéressant
ce qui ne l’est pas, de donner sens à ce qui en est dépourvu, de faire
rêver plutôt que penser, d’émouvoir plutôt que d'éclairer, de passionner plutôt
que d’éduquer. Le roman, presque inévitablement, exagère la vie, comme dirait
mon ami Marc, et ne vaut, sauf chez les plus grands, que par cette exagération.
Cela explique son succès, chez presque tous, et la terrible tentation qu’il exerce
parfois, même chez les meilleurs, quand l'esprit se relâche. C’est le mensonge
érigé en esthétique ou en divertissement.

Il arrive pourtant que la vérité affleure (le {\it mentir-vrai} d'Aragon), et se serve
de ce masque pour s’avouer. Le roman est un détour par la fiction, qui peut
ramener au vrai : nous lui devons quelques-uns des plus grands chefs-d’œuvre de
la littérature universelle. Force m'est pourtant de reconnaître que j'en lis de
moins en moins, et avec de moins en moins de plaisir. Il m’arrive de penser que
le roman est un genre mineur, dont chaque réussite confirme, en les dépassant,
les limites : Proust, Céline ou Joyce sont grands {\it malgré} ou {\it contre} le roman, me
semble-t-il, plutôt que grâce à à lui. Cela pourrait sembler justifier à l’avance le
Nouveau-roman, qui n’est qu’un antiroman (pas de personnages, pas di intrigue
pas d'aventure, pas de romanesque). Mais l’ennui ne justifie rien. C’est ce qui
sauve le roman, ou qui le sauvera. On n’a pas fini d’inventer des histoires.

%— 514 —
%{\footnotesize XIX$^\text{e}$} siècle — {\it }
\section{Romantisme}
%ROMANTISME
L’opposé du classicisme, mais en aval et quant au fond (par
différence avec le baroque, qui le serait plutôt en amont,
du moins en France, et quant à la forme). Les romantiques ont besoin de règles
pour les violer, de traditions pour s’y opposer, de maîtres pour s’en libérer, de
raison, enfin, pour vouloir s’en affranchir ou lui préférer les sentiments. En
quoi c’est un mouvement second, et secondaire le plus souvent. Mais il exprime
aussi quelque chose d’essentiel à l’âme humaine, qui est le malheur éperdu de
durer, de n’être pas Dieu, de devoir finir... Le temps est son mal et sa raison
être. D’où la nostalgie toujours, qui est le sentiment romantique par
excellence : cendre, pour le poète, du feu qui le consume! La vraie vie est
ailleurs — et nous sommes ici. Le romantisme est pour cela condamné au
double langage ou aux arrière-mondes, à l’idéalisme ou à la déception. Il aspire
à l'infini, ne trouve que le fini. Il cherche l'absolu, ne trouve que le moi. Il voudrait
se fondre dans l’unité, se heurte partout au multiple ou à la dualité. Il
voudrait s’abandonner à l'inspiration, ne réussit que par le travail. Il exalte la
passion, l’imaginaire, la sensibilité. Il ne débouche que sur la lassitude ou
l'ennui. La fuite est sa tentation ; le rêve, son excuse. C’est un art passionnel,
qui n’a guère le choix qu’entre l’onirisme et la religion : esthétique de l'exil ou
de l'évasion, du mystère ou du déchirement. Cela ne retire rien aux plus grands
(Novalis et Hülderlin, Byron et Keats, Delacroix, Berlioz, Nerval...), mais
m’empêche de les suivre tout à fait. Seul Hugo, pour mon goût, fait exception.
Mais c’est qu’il est l’exception absolue : il excède le romantisme autant que
Bach le baroque. Il récuse à lui seul la formule injuste et profonde de Goethe :
{\it « J'appelle classique ce qui est sain, romantique ce qui est malade. »} Ou plutôt il la
récuserait, si une exception, aussi glorieuse fût-elle, pouvait invalider... une
définition.

\section{Rumeur}
%RUMEUR
Un bruit anonyme, mais chargé de sens. C’est donc que quelqu’un
parle. Qui? Personne, tout le monde: le sujet de la
rumeur est le {\it on}, qui est moins un sujet qu’une foule impersonnelle et insaisissable.
La rumeur fait comme un discours sans sujet, dont personne n’a à
répondre. C’est ce qui la rend particulièrement propice aux fausses nouvelles,
aux sottises, aux calomnies. Seul celui qui la lance, s’il y en a un, en est vraiment
responsable. Seuls ceux qui se taisent ou la combattent en sont vraiment
innocents. L'idéal serait de n’y prêter aucune attention. La moindre des choses,
de ne pas en rajouter. Ce ne sont que des on-dit, qu’on ne peut toujours
ignorer mais qu’il faudrait s’interdire de propager.
%
%{\footnotesize XIX$^\text{e}$} siècle — {\it }

