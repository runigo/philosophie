%{\footnotesize XIX$^\text{e}$} siècle — {\it }

\chapter{E}
%
%{\footnotesize XIX$^\text{e}$} siècle — {\it }

%ÉCHANGE
\section{Échange}
Changement simultané de propriétaire et de propriété, le plus
souvent sous la forme d’une cession mutuelle : ce qui appartenait
à l’un appartient désormais à l’autre, et réciproquement.

Les ethnologues, prenant le mot en un sens élargi, distinguent trois
échanges fondamentaux : l’échange des biens, qui est troc ou commerce ;
l'échange des signes, qui est langage ; l’échange des femmes, qui est alliance. Ce
dernier échange, qui nous choque (parce qu’on y échange des sujets, non des
objets), tend à disparaître : on n’échange plus les femmes ; hommes et femmes
se donnent ou se prêtent, sans que nul pourtant puisse jamais les posséder.
C'est libérer le couple des rapports marchands, et la seule forme réalisable,
peut-être bien, du communisme — non par la propriété commune, qui ne serait
que collectivisme, mais par l'abolition de la propriété. Par quoi l'amour et le
respect, tant qu'ils durent, sont la seule utopie vraie.

%ÉCHEC
\section{Échec}
L'écart entre le résultat qu’on visait et celui qu’on obtient. C’est
pourquoi l’histoire de toute vie, comme disait Sartre, est l’histoire
d’un échec : le réel résiste et nous emporte.

On n'échappe à l'échec qu’en cessant de viser un résultat. Non parce qu’on
cesserait d'agir, ce qui ne serait qu’un échec de plus, mais parce qu’on ne vise plus
que l’action même. C’est ce qu’on appelle la sagesse, qui serait la seule vie réussie.
On n’a une chance de l’atteindre qu’à condition de cesser de la poursuivre.

%ÉCLECTISME
\section{Éclectisme}
Une pensée qui se constituerait par emprunts et collages : un
système de bric et de broc, une doctrine qui serait un

%— 191 —
%{\footnotesize XIX$^\text{e}$} siècle — {\it }
mélange de doctrines, comme un patchwork théorique. L’éclectisme se doit
pourtant d’être cohérent : c’est ce qui le distingue du syncrétisme, et qui en fait
une philosophie possible (par exemple chez Leibniz, qui n’utilise pas le mot, ou
chez Victor Cousin, qui baptise ainsi son école). Dans la philosophie contemporaine,
le mot tend toutefois à valoir comme condamnation. C’est que les
grands philosophes n’empruntent pas : ils absorbent et recréent, transforment
et dépassent. « Les abeilles pillotent [pillent, butinent] deçà delà les fleurs,
remarquait Montaigne après tant d’autres, mais elles en font après le miel, qui
est tout leur ; ce n’est plus thym, ni marjolaine ; ainsi les pièces empruntées
d’autrui, il les transformera et confondra pour en faire un ouvrage tout sien, à
savoir son jugement » (I, 26, 152). Cette métaphore, quoique rebattue, dit à
peu près l'essentiel. Le miel n’est pas un mélange de fleurs. Mais cela n’a jamais
dispensé les abeilles de butiner.

%ÉCOLE
\section{École}
Le lieu où l’on enseigne et où l’on apprend. Cela suppose un maître,
celui qui sait et enseigne, et des écoliers, qui ne savent pas et sont là
pour apprendre. La définition même de l’école semble rétrograde et antidémocratique.
C’est bien ainsi. Toute école représente le passé, qu’elle doit transmettre
à ceux, plus tard, qui inventeront l'avenir. Et aucune ne saurait se
soumettre, sans y perdre son âme, à l’exigence démocratique, qui est celle du
nombre et de l'égalité. On ne va pas voter, dans les classes, pour savoir comment
s'écrit un mot, combien font trois fois huit, ou quelles sont les causes de
la première guerre mondiale. Ni pour savoir s’il faut étudier l'orthographe,
l’arithmétique et l’histoire. Le maître ne peut transmettre son savoir que si son
pouvoir est à peu près reconnu par tous. Par quoi il ny a pas d’école sans discipline,
ni de discipline sans sanctions. L'école démocratique ? C’est celle qui
est soumise à la démocratie, c’est-à-dire au peuple souverain, qui décide des
budgets, des programmes, des objectifs. Non celle qui serait soumise, absurdement,
aux suffrages des élèves ou des parents. Ouvrir l’école sur la vie ? Ce
serait l'ouvrir au marché, à la violence, aux fanatismes de toutes obédiences.
Mieux vaut la fermer sur elle-même — lieu d’accueil et de recueillement — pour
l'ouvrir aux savoirs et à tous.

%ÉCOLOGIE
\section{Écologie}
L'étude des milieux ou des habitats — {\it oikos}, en grec, c’est la maison —,
spécialement des biotopes (les milieux vivants) et en
général de la biosphère (Pensemble des biotopes). On ne la confondra pas avec
l’écologisme, qui est l’idéologie qui s’en réclame, ni même avec la protection de
l’environnement, qui l'utilise. Rien n’interdirait, en théorie, d’utiliser l'écologie

%— 192 —
%{\footnotesize XIX$^\text{e}$} siècle — {\it }
pour polluer le plus efficacement possible. En pratique, toutefois, personne
ne le souhaite. C’est ce qui fait que le mot, presque toujours, prend un sens
normatif, voire prescriptif. Reste à savoir si on veut mettre l’écologie au service
de l'humanité (écologie humaniste), ou mettre l’humanité à son service (écologie
radicale). Les deux positions sont respectables. Cela ne nous dispensera
pas, entre les deux, de choisir ou de chercher un compromis.

%ÉCONOMIE
\section{Économie}
Étymologiquement, c’est la loi ou l’administration ({\it nomos}) de
la maison ({\it oikos}). La première économie est domestique : c’est
la gestion des biens d’une famille, de ses ressources, de ses dépenses, ce que
Montaigne appelait le {\it mesnage}, où il voyait une tâche « plus empêchante [plus
absorbante] que difficile ». L'essentiel est ici de ne pas dépenser plus qu’on ne
possède ou qu’on ne gagne : de là le sens courant du mot, spécialement au pluriel
(« faire des économies »), qui vise surtout une restriction des dépenses. Au
sens moderne, le mot désigne à la fois une science humaine et l’objet qu’elle
étudie : l’économie, c’est tout ce qui concerne la production, la consommation
et l’échange des biens matériels, marchandises ou services, aussi bien à l’échelle
des individus et des entreprises (micro-économie) qu’à l'échelle de la société ou
de la planète (macro-économie). C’est moins l’art de réduire les dépenses que
celui d’augmenter les richesses. Son lieu de prédilection est le marché, où règne
la loi de l'offre et de la demande. Tout ce qui est rare est cher, mais à condition
seulement d’être désiré par plusieurs (par quoi toute valeur, même économique,
reste subjective) et de pouvoir être échangé (par quoi la subjectivité de
la valeur, dans un marché donné, fonctionne objectivement). Si tout était à
vendre, l’économie régnerait seule. On n’y échappe que par ce qui ne vaut rien
(la gratuité) et par ce qui n’a pas de prix (la dignité, la justice). On n’y échappe
donc, et jamais totalement, que par exception ou par devoir. C’est ce que nous
rappellent la misère, chez les pauvres, et l’avidité, chez les riches.

%ÉCRITURE
\section{Écriture}
C’est une technique, inventée il y a quelque cinq mille ans, qui
permet de fixer la parole ou la pensée sur un support durable,
au moyen de signes plus ou moins symboliques (pictogrammes, idéogrammes)
ou conventionnels (les lettres d’un alphabet). C’est inscrire la pensée dans
l’espace, où elle se fige et se conserve, et la libérer par là, au moins partiellement,
du temps. Les livres remplacent la mémoire, ou plutôt la soutiennent, la
démultiplient, la sauvent. Dans une société sans écriture, dit-on souvent, «un
vieillard qui meurt, c’est une bibliothèque qui brûle ». Or les vieillards meurent
tous, toujours. Les bibliothèques ne brûlent que par exception. L’accumulation

%— 193 —
%{\footnotesize XIX$^\text{e}$} siècle — {\it }
des souvenirs, des idées, des savoirs, devient ainsi, grâce à l’écriture, indéfinie.
On ne se contente plus de résister individuellement à l'oubli ; on ajoute des
traces aux traces, des œuvres aux œuvres. C’est passer d’une logique de la répétition,
qui est celle du mythe, à une logique de l’accumulation et du progrès,
qui est celle de l’histoire. Ce n’est donc pas un hasard, ni pure convention, si
l'invention de l'écriture marque la fin de la préhistoire. La conservation du
passé bouleverse notre rapport à l’avenir : le présent, qui les sépare et les relie,
est entré dans l’histoire.

%ÉDUCATION
\section{Éducation}
C’est transformer un petit d'homme — le même à la naissance,
à très peu près, que son ancêtre d’il y a dix mille ans — en être
humain civilisé. Cela suppose qu’on lui transmette, dans la mesure du possible,
ce que l’humanité à fait de meilleur ou de plus utile, ou qu’elle juge être tel :
certains savoirs et savoir-faire (à commencer par la parole), certaines règles, certaines
valeurs, certains idéaux, enfin l’accès à certaines œuvres et la capacité
d’en jouir. C’est reconnaître qu’il n’y a pas de transmission héréditaire des
caractères acquis, et que l’humanité, en chacun, est aussi une acquisition : on
naît homme, ou femme, on {\it devient} humain. C’est reconnaître que la liberté
n'est pas donnée d’abord, qu’elle ne va pas sans raison ni la raison sans
apprentissage : on ne naît pas libre, on le devient. Cela ne va pas sans amour,
dans la famille, mais pas non plus sans contraintes. Et encore moins, à l’école,
sans travail, sans efforts, sans discipline. Le plaisir ? On n’en a jamais trop. Mais
telle n’est pas la principale fonction de l’école, ni même de la famille. L’éducation
est presque toute du côté du principe de réalité. Il s’agit non de remplacer
l'effort par le plaisir, mais d’aider l'enfant à trouver du plaisir, peu à peu, dans
l'effort accepté et maîtrisé. Jouer ? On ne le fait jamais trop. Mais c’est le travail
qui est grand, et qui fait grandir. D'ailleurs les enfants jouent à travailler, très
vite, et cela indique assez la direction. L'éducation n’est pas au service des
enfants, comme on le dit toujours, mais au service des adultes qu’ils veulent et
doivent devenir.
On se trompe pourtant si l’on croit que l’éducation doit inventer l’avenir.
De quel droit parents et pédagogues, qui sont en charge de l'éducation, choisiraient-ils
l'avenir des enfants à leur place ? La vraie fonction de l'éducation, et
spécialement de l’école, ce n’est pas d’inventer l’avenir, c’est de transmettre le
passé. C’est ce qu'avait vu Hannah Arendt, dès les années 50 : « Le conservatisme,
pris au sens de conservation [je préférerais dire la transmission], est
l'essence même de l’éducation », disait-elle. Non, certes, parce qu’il faudrait
renoncer à transformer le monde, mais au contraire pour permettre aux enfants
de le faire, s’ils le veulent, quand ils seront grands : « C’est justement pour préserver
% 194 —
%{\footnotesize XIX$^\text{e}$} siècle — {\it }
ce qui est neuf et révolutionnaire dans chaque enfant que l’éducation
doit être conservatrice » ({\it La crise de la culture}, V). C’est ce qu'avait vu Alain,
dès les années 20 : « L'enseignement doit être résolument retardataire. Non pas
rétrograde, tout au contraire. C’est pour marcher dans le sens direct qu’il prend
du recul ; car, si l’on ne se place point dans le moment dépassé, comment le
dépasser ? ({\it Propos sur l'éducation}, XVII). Vous pouvez bien mettre des ordinateurs
et des journaux dans toutes les classes. Cela ne remplacera jamais les
chefs-d’œuvre — littéraires, artistiques, scientifiques — qui ont fait de l’humanité
ce qu’elle est. D’ailleurs ordinateurs et journaux sont du passé aussi (dès qu’ils
se répandent, ils sont obsolètes) et vieilliront plus vite que Pascal ou Newton,
Hugo ou Rembrandt. Le progrès ? Il suppose la transmission, et ne saurait par
conséquent autoriser qu’on y renonce. L'avenir ? Ce n’est pas une valeur en soi
(sans quoi la mort, pour chacun, en serait une). Il ne vaut, ou plutôt il ne
vaudra, que par fidélité d’abord à ce que nous avons reçu, que nous avons à
charge de transmettre. Du passé, ne faisons pas table rase.

%EFFET
\section{Effet}
Un fait quelconque, en tant qu’il résulte d’une ou plusieurs causes.
Ainsi tout fait est effet, et c’est ce que signifie le principe de causalité.

%EFFICIENTE (CAUSE -)
\section{Efficiente (cause —)}
L’une des quatre sortes de causes selon Aristote, et
la seule que la modernité ait véritablement
retenue. Une cause efficiente, c’est une cause qui n’est ni finale ni formelle
(voir ces mots), et qui ne se réduit pas non plus à la simple matière dont l'effet
est constitué (ce qu’Aristote appelait la cause matérielle). C’est dire qu’elle produit
ses effets par son action propre : c’est le « moteur prochain », comme disait
Aristoté, autrement dit ce qui meut ou transforme une matière première. Par
exemple le sculpteur est la cause efficiente de la statue, comme les intempéries,
la pollution et les touristes sont les causes efficientes de son inexorable dégradation.

On dit parfois que la cause efficiente est celle qui précède son effet (par
opposition à la cause finale, qui le suivrait). Mais elle ne le produirait pas si elle
n'avait avec lui au moins un point de tangence dans le temps, que ce soit directement
ou par telle ou telle de ses suites. Par exemple mes parents décédés ne
sont la cause efficiente de mon existence présente que parce qu’ils ont été celle
de ma conception : preuve en est que je peux désormais exister sans eux comme
eux sans moi. Ce n’est pas le passé qui agit sur le présent ou qui le cause ; c’est
le présent qui continue le passé en agissant sur lui-même. La cause efficiente est
le nom qu’on donne traditionnellement à cette action et à cette continuation.

%— 195 —
%{\footnotesize XIX$^\text{e}$} siècle — {\it }
%EFFORT
\section{Effort}
Une force volontaire ou instinctive, quand elle s'oppose à une
résistance. Maine de Biran voyait dans l’effort « le fait primitif du
sens intime » : celui par lequel le moi se découvre ou se constitue « par le seul
fait de la distinction qui s'établit entre le sujet de cet effort libre et le terme qui
résiste immédiatement par son inertie propre ». Mais comment savoir si le moi
est la cause (comme le veut Biran) ou l'effet (comme je préférerais dire) de cet
effort ? C’est où il faut choisir entre le {\it conatus} spinoziste et l'{\it effort} biranien, qui
est sa version française et spiritualiste.

%ÉGALITÉ
\section{Égalité}
Deux êtres sont égaux lorsqu'ils sont de même grandeur ou possèdent
la même quantité de quelque chose. La notion n’a donc de
sens que relatif: elle suppose une grandeur de référence. On parlera par
exemple de l'égalité de deux distances, de deux poids, de deux fortunes, de
deux intelligences... Mais une distance n’est pas égale à un poids, ni une fortune
à une intelligence. Une égalité absolue ? Ce serait une identité, et nul ne
serait légal, en ce sens, que de soi. Prise absolument, la notion perd son sens ou
devient autre. On ne peut lui être absolument fidèle qu’à la condition
d’accepter qu’elle soit relative.

Les hommes sont-ils égaux ? Tout dépend de quoi l’on parle : question de
fait, ou question de droit ? En fait, c’est bien sûr l'inégalité qui est la règle : les
hommes ne sont ni aussi forts, ni aussi intelligents, ni aussi généreux les uns
que les autres. Ces différences parfois s’équilibrent ou se compensent : tel sera
plus fort que tel autre, qui sera plus intelligent ou moins égoïste... Mais il
arrive aussi, et peut-être plus souvent, qu’elles s'ajoutent : certains semblent
avoir toutes les chances, tous les talents, toutes les vertus, quand d’autres n’ont
que des faiblesses, que des tares ou du malheur. Si toutes les très belles femmes
étaient stupides et méchantes, cela serait moins dur pour les laides. Si tous les
champions étaient idiots et impuissants, ce serait moins blessant pour les
autres. Mais ce n’est pas le cas, et les voilà qui deviennent, cela fait une inégalité
de plus, riches à millions. En fait, donc, il n’y a pas de doute : quelle que soit
la grandeur considérée, et même en essayant d’unifier ces grandeurs en une
moyenne, les êtres humains, à les considérer comme individus, sont manifestement
inégaux. Un Dieu juste, peut-être, aurait pu ou dû l’éviter. Mais s’il n’y
a pas de Dieu ? Ou s’il en a jugé autrement ?

La démocratie ? Les droits de l’homme ? J'y suis attaché autant qu’un autre.
Mais pourquoi faudrait-il, pour être démocrate, prétendre qu’Eichmann est
l’égal, en fait, d’Einstein ou de Cavaillès ?

On a insuffisamment répondu à Le Pen, quand on lui objecte, sans autre
précision, que tous les hommes sont égaux. Tous les individus ? Ce serait
%— 196 —
%{\footnotesize XIX$^\text{e}$} siècle — {\it }
opposer un pieux mensonge à un mensonge impie. Toutes les races ? Si le
concept est sans pertinence, comme semblent le penser nos généticiens, ce problème
n’en a pas davantage. Si l’on continue pourtant à se servir de cette
notion équivoque et désagréable, comme font beaucoup d’antiracistes (« toutes
les races sont égales »), attention de ne pas tomber dans la même confusion que
font ou qu’entretiennent ceux que nous combattons. Que les hommes soient
égaux {\it en droit et en dignité}, cela seul dépend de nous, cela seul, moralement,
politiquement, importe. Qu'ils soient égaux en fait, cela dépend de la nature, et
rien ne garantit qu’elle soit démocrate, progressiste et humaniste. Elle ignore
nos lois. Pourquoi faudrait-il accepter aveuglément la sienne ? « La biologie a
depuis longtemps réfuté le racisme », disent de braves gens. Je m’en réjouis fort.
Mais fallait-il être raciste, tant que ce n’était pas le cas ? Faudrait-il le devenir, si
la biologie changeait d’orientation ? Faut-il soumettre nos principes au diktat des
laboratoires ? N’être antiraciste que sous réserve d’inventaire génétique ? Ce serait
confondre le fait et le droit, et c’est en quoi il est essentiel de les distinguer.
Quand bien même les généticiens nous expliqueraient demain que les Noirs sont
en effet plus doués pour la course à pied que les Blancs, quand bien même ils
découvriraient que ce n’est pas absolument par hasard (ni pour des raisons seulement
culturelles) que les Juifs ont proportionnellement davantage de prix Nobel
que les soi-disant aryens, cela ne donnerait évidemment aucun droit ni aucune
dignité supplémentaires aux uns ou aux autres. La question n’est pas de savoir si
les hommes sont égaux en fait — cela ne dépend pas de nous, et les individus ne
le sont pas —, mais si nous voulons qu’ils Le soient (et il suffit que nous le voulions
pour qu’ils le soient en effet) en droit et en dignité.

La réponse n’est pas dans les laboratoires des généticiens, ni dans les tests
des psychologues. Elle est dans nos cœurs et dans nos parlements — dans nos
principes et dans nos lois. Ne comptons pas sur la biologie pour être égalitaire
à notre place. Ne renonçons pas à l’être sous prétexte que la nature, s'agissant
des individus, ne l’est pas.

Ce n’est pas parce que les hommes sont égaux qu’ils ont les mêmes droits.
C’est parce qu’ils ont les mêmes droits qu’ils sont égaux.

%ÉGLISE
\section{Église}
La communauté des croyants, pour une religion donnée et lorsqu’elle
a atteint une certaine dimension : c’est une secte trop nombreuse
pour accepter d’en être une.

Quand il est utilisé sans autre précision, le mot désigne presque toujours
l’une des Églises chrétiennes, et spécialement l'Église catholique. C’est qu’elles
ont réussi, en tant qu’institutions, plus solidement que les autres. Cela ne suffit
pas à leur donner raison, ni tort.

%— 197 —
%{\footnotesize XIX$^\text{e}$} siècle — {\it }
« Jésus annonçait le Royaume, écrit Alfred Loisy, et c’est l'Église qui est
venue. » Cela pourrait presque valoir comme définition. Une Église, c’est ce
qui vient à la place du Royaume qu’elle annonce. Seuls les vrais mystiques et les
vrais athées peuvent s’en passer.

%EGO
\section{Ego}
Le moi, le plus souvent considéré comme objet de la conscience. C’est
moins ce que je suis que ce que je crois être, moins le {\it je} que le {\it me} (par
exemple quand on dit « je {\it me} connais », « je {\it me} sens triste »..), moins le sujet
transcendantal, malgré Husserl, que l’objet transcendant, comme dit Sartre,
d’une conscience impersonnelle ({\it La transcendance de l'Ego}, Conclusion :
« L’Ego n’est pas propriétaire de la conscience, il en est l’objet » ; voir aussi
{\it L'être et le néant}, p. 147). Ce n’est le sujet de la pensée qu’autant qu’elle croit
en avoir un, et c’est pourquoi ce n’est rien.

%ÉGOCENTRISME
\section{Égocentrisme}
C'est se mettre au centre de tout. Se distingue de
l’égoïsme par une dimension plus intellectuelle.
L’égoïsme est une faute ; l’égocentrisme serait plutôt une illusion ou une erreur
de perspective. C’est le point de vue spontané de l’enfant et de l’imbécile. Son
remède est le décentrement ; son contraire, l’universel.

%ÉGOÏSME
\section{Égoïsme}
Ce n’est pas l'amour de soi ; c’est l’incapacité à aimer quelqu'un
d'autre, ou à l’aimer autrement que pour son bien à soi. C’est
pourquoi j'y vois un péché capital (l’amour de soi serait plutôt une vertu) et le
principe de tous.
C’est aussi une tendance constitutive de la nature humaine. On ne la surmonte
que par effort ou par amour — par vertu ou grâce.

%ÉGOTISME
\section{Égotisme}
Ce n’est pas le culte du moi. Chez Stendhal, qui popularisa le
mot, c'est bien plutôt son étude bienveillante et lucide, son
analyse, son approfondissement, son perfectionnement — non son culte, mais sa
culture.

%ÉIDÉTIQUE
\section{Éidétique}
Qui porte sur l’essence (l’{\it eîdos}) ou les essences. Le mot n’est
guère utilisé que par les phénoménologues.

%— 198 —
%{\footnotesize XIX$^\text{e}$} siècle — { }
%EÎDOS
\section{\it Eîdos}
Ce qu’on voit, par les yeux ou par l'esprit : la forme, l’idée, l'essence.
Mais pourquoi le dire en grec ?

%ÉLÉATES
\section{Éléates}
Élée était une colonie grecque, au sud de l'Italie. C’est là que sont
nés Parménide et Zénon d’Élée (qu’on ne confondra pas avec
Zénon de Cittium, le fondateur du stoïcisme), qui dominent l’école dite des
éléates. Si l’on se fie à ce qu’en a retenu la tradition, ils niaient l'existence du
mouvement, du devenir, de la multiplicité, pour ne plus célébrer que
l'unicité et l’immuabilité de l’être. C’était donner tort aux apparences et à
l'opinion : de là les paradoxes de Zénon (Achille incapable de rattraper une
tortue, la flèche immobile en plein vol.) et l’Être éternel et absolument plein
de Parménide. L’être est, le non-être n’est pas. Comment le devenir (qui suppose
le passage de l'être au non-être, du non-être à l’être) pourrait-il exister ?
L’éléatisme, en ce sens, serait le contraire de l’héraclitéisme.

Pour ma part, guidé par Marcel Conche (Parménide, {\it Le poème : fragments},
PUF, 1996), j'y vois plutôt une pensée de l’éternelle présence : non le contraire
de l’héraclitéisme, mais sa saisie {\it sub specie aeternitatis}. Au présent, l'être et le
devenir sont un. Tout change (c’est ce qui donne raison à Héraclite), mais rien
ne change qu’au présent, qui reste toujours identique à soi (c’est ce qui donne
raison à Parménide : « Ni il n’était, ni il ne sera, puisqu'il est maintenant »).
Ainsi l'éternité et le temps sont une seule et même chose. Quelle chose ? Le
présent, qui passe et demeure — comme une flèche, en effet, à la fois mobile
(elle n’est plus où elle était, pas encore où elle sera) et immobile (elle est ici et
maintenant) en plein vol.

%ÉLITE
\section{Élite}
Étymologiquement, c’est l’ensemble de ceux qui ont été élus ; mais
ils l'ont été par le hasard ou par eux-mêmes plutôt que par Dieu ou
par le peuple. C’est une espèce d’aristocratie laïque et méritocratique : les
meilleurs, quand ils ne le doivent qu’à leur talent ou à leur travail. L'erreur
serait d'y croire tout à fait ou en bloc. Il y a des élites différentes (celle du sport
n’est pas celle du savoir, qui n’est pas celle des affaires ou de la politique), dont
la médiocrité, quand on y pénètre, ne cesse de surprendre. Une salle de profs,
dans la plus prestigieuse des universités, ne donne pas une très haute idée de
l'humanité. J'imagine qu’il en va de même dans d’autres milieux, que j'ai
moins fréquentés. Nul n’est le meilleur en tout et absolument. Un sujet d'élite,
c'est un individu dont la médiocrité générale, qui est celle de l’espèce, comporte
au moins une exception, qui est celle du talent ou du savoir-faire. C'est
mieux que rien. Ce n’est pas grand-chose.

%— 199 —
%{\footnotesize XIX$^\text{e}$} siècle — {\it }
On s'étonne qu’un grand philosophe ait été nazi, que tel autre ait été mesquin
ou lâche... C’est qu’ils appartenaient à une élite, non à toutes. Autant
s'étonner qu’un ministre fasse des fautes de français.

%ÉLITISME
\section{Élitisme}
Toute pensée qui reconnaît l’existence d’une élite ou favorise son
émergence. L’élitisme républicain, qui vise à produire les élites
dont la République a besoin, s’oppose donc à l’aristocratisme (qui voudrait que
les élites se reproduisent par le sang) comme à l’égalitarisme (qui voudrait qu’il
n'y ait pas d’élites du tout). C’est ce qui le rend doublement nécessaire. Pourquoi
l'égalité des chances devrait-elle aboutir à un nivellement par le bas ?

%ÉLOQUENCE
\section{Éloquence}
L’art de la parole (c’est ce qui distingue l’éloquence de la rhétorique,
qui serait plutôt l’art du discours) ou le talent qui
permet d’y exceller. Art mineur, talent dangereux.

%ÉMOTION
\section{Émotion}
C'est un affect momentané, qui nous meut plus qu’il ne nous
structure (comme ferait un sentiment) ou qu’il ne nous
emporte (comme ferait une passion). Par exemple la colère, la peur ou le coup
de foudre sont des émotions. Mais qui peuvent déboucher sur des passions ou
des sentiments, comme sont par exemple la haine, l’anxiété ou l’amour. « Une
suite d'émotions vives et liées au même objet produit la passion, écrivait Alain ;
et l’état de passion surmonté se nomme sentiment. » Mais l’enchaînement peut
se prendre aussi dans l’autre sens : toute passion est source d'émotions, et l’on
ne surmonte guère que les passions fatiguées. Les frontières, entre ces différents
affects ou concepts, sont floues, et ce flou est essentiel à l’émotion : si l’on y
voyait absolument clair, on ne serait pas ému. Par exemple ce trouble, entre
deux regards qui se croisent, cette attirance, cette sensualité légère et joyeuse,
cette inquiétude aussi, cette accélération du cœur et de la pensée, est-ce un
amour qui commence ou un désir qui passe ? On ne le saura vraiment que plus
tard, quand l'émotion sera retombée ou installée.

%EMPIRIQUE
\section{Empirique}
Qui vient de l’expérience ou en dépend. Le mot, dans la philosophie
continentale et depuis Kant, se prend plutôt en
mauvaise part. Toute expérience étant particulière et contingente, une
connaissance empirique ne saurait être ni universelle ni nécessaire. Que les
choses, telles que nous avons pu les observer, se soient jusqu’à présent passées

%— 200 —
%{\footnotesize XIX$^\text{e}$} siècle — {\it }
de telle et telle façons, cela ne prouve pas qu’elles se passent et se passeront
{\it toujours} comme cela. Toute connaissance empirique reste donc approxima-
tive et provisoire. Mais une connaissance qui n'aurait absolument rien
d’empirique ne serait plus une connaissance du tout. Si nous n’avions l’expé-
rience des signes et des figures, de l’évidence et de l’absurdité, que resterait-il
des mathématiques ?

%EMPIRISME
\section{Empirisme}
Toute théorie de la connaissance qui accorde la première place
à l’expérience. C’est refuser les idées innées d’un Descartes
autant que les formes {\it a priori} d’un Kant. La raison, pour l’empiriste, n'est pas
une donnée première ou absolue : elle est elle-même issue de l'expérience, aussi
bien extérieure (sensations) qu’intérieure (réflexion), et en dépend autant
qu’elle la met en forme (spécialement par l’usage des signes). L'empirisme
s'oppose en cela au rationalisme en son sens étroit et gnoséologique. Cela ne
l'empêche évidemment pas d’être rationaliste au sens large ou normatif : la plupart
des grands empiristes (Épicure, Bacon, Hobbes, Locke, Hume...) se sont
battus pour que la raison l’emporte, non certes contre l’expérience, ce qu’elle ne
peut ni ne doit, mais contre l’obscurantisme et la barbarie.

C’est ce qui amènera l’empirisme logique, au {\footnotesize XX$^\text{e}$} siècle, à s'intéresser surtout
aux sciences, au point de rejeter toute métaphysique : « L'analyse logique
rend un verdict de non-sens, écrit Carnap, contre toute prétendue connaissance
qui veut avoir prise par-delà ou par-derrière l’expérience. » La logique ne
connaît que soi (elle est analytique, non synthétique). Seule l'expérience — et
spécialement l’expérimentation scientifique — nous permet de connaître le
monde.

Reste toutefois à penser cela même que l’on connaît, ce qui suppose toujours
autre chose, que l’on ne connaît pas. C’est où la philosophie revient, y
compris dans sa dimension spéculative, et qui lui interdit de se prendre pour
une science. L’empirisme n’est pas nécessairement antimétaphysique. Mais il
est antidogmatique, ou doit l'être. Si toute connaissance vient de l’expérience,
comment pourrions-nous tout connaître ou connaître quoi que ce soit
absolument ? Comment être certain que l'expérience dit vrai, puisqu'on ne
pourrait en décider que. par une autre expérience ! ? Comment être certain que
P inexpérimentable n'existe pas, puisqu'il serait par nature hors de notre
portée ? L’empirisme, qui fut d’abord dogmatique (spécialement chez Épi-
cure et Lucrèce), a surtout à voir, dans les temps modernes, avec le scepti-
cisme. Plus l’expérience progresse, mieux on en comprend les limites, qui
sont les nôtres.

%— 201 —
%{\footnotesize XIX$^\text{e}$} siècle — {\it }
%EMPORTEMENT
\section{Emportement}
Une colère qui passe à l’acte. C’est suivre son corps, au
lieu de le gouverner.

%ÉNERGIE
\section{Énergie}
Une force en puissance ({\it dunamis}) ou en acte ({\it energeia}) : c’est la
capacité de produire un effort ou un travail.

Nos physiciens nous apprennent que l'énergie peut prendre des formes différentes
(cinétique, thermique, électrique), qu’elle ne se perd ni ne se crée
(elle se conserve toujours, même si elle se dégrade en chaleur), enfin qu’elle est
équivalente à la masse ({\it E = mc$^2$}). Philosophiquement, le mot ferait une assez
bonne traduction pour le {\it conatus} spinoziste. Tout être tend à persévérer dans
son être, et toute énergie se conserve : les deux idées, pour différentes qu’elles
soient évidemment, sont au moins compatibles. Cela n’empêche ni la mort ni
la fatigue.

%ENFANCE
\section{Enfance}
Le premier âge de la vie : les années qui séparent la naissance de
l’adolescence ou de la puberté. C’est l’âge de la plus grande fragilité —
l'enfant est à peu près sans défense contre le mal et le malheur — et des
plus grandes promesses. C’est ce qui nous impose, vis-à-vis des enfants, les plus
grands devoirs (devoir de protection, de respect, d’éducation...), sans aucun
droit jamais sur eux. « Cette faiblesse est Dieu », disait Alain. C’est qu’elle commande
absolument, par l'incapacité où elle est de punir comme de récompenser.
L’enfant-roi est sa caricature : s’il gouverne, il ne règne plus.

Les enfants veulent grandir. Notre devoir est de les y aider, et pour cela de
grandir nous-mêmes. C’est la seule façon d’être fidèle à l'enfant que nous fûmes
et que nous sommes. « Nous poussons notre enfance devant nous, écrit encore
Alain, et tel est notre avenir réel. »

%ENFER
\section{Enfer}
Le lieu du plus grand malheur. Les religions y voient souvent un
châtiment, qui viendrait, après leur mort, punir les méchants. Les
matérialistes, pour qui la mort n’est rien, y voient plutôt une métaphore :
« C’est ici-bas, écrit Lucrèce, que la vie des sots devient un véritable enfer. »
Hélas ! pas des sots seulement. La mort en délivre, plus sûrement que l’intelligence.

%ENGAGEMENT
\section{Engagement}
C'est mettre son action ou sa personne au service d’un
combat que l’on croit juste. Le mot sert surtout pour les
%— 202 —
%{\footnotesize XIX$^\text{e}$} siècle — {\it }
intellectuels, jusqu’à en désigner un certain type (« l’intellectuel engagé »). Le
risque, pour eux, est de soumettre aussi leur pensée aux nécessités du combat —
quand elle devrait ne se soumettre qu’au vrai, ou à ce qui semble tel. Mieux
vaudraient, me semble-t-il, des intellectuels citoyens. Participer au débat
public, dans la limite de ses compétences, cela fait assurément partie de la responsabilité
d’un intellectuel. Mais cela ne l’oblige pas à soumettre sa pensée,
comme firent certains, à une cause déjà constituée par ailleurs. La bonne foi est
plus importante que la foi. La liberté de l'esprit plus importante que l’engagement.

On pense aux Romains d’Astérix: {\it « Engagez-vous, engagez-vous, qu'ils
disaient ! »} Le mot est d’abord militaire, et l’idée, dans son usage intellectuel, en
a gardé quelque chose. Tout engagement suppose l’obéissance. Toute pensée la
récuse. C’est assez, chers collègues, d’agir avec les autres. Cela ne saurait nous
autoriser à penser pour leur faire plaisir ou pour leur donner raison.

%ÉNIGME
\section{Énigme}
Un problème qu’on ne peut résoudre, non parce qu’il excède nos
moyens de connaissance (ce n’est pas un mystère), ni pour des raisons
seulement logiques (ce n’est pas une aporie), mais parce qu’il est mal posé.
C’est pourquoi «l’énigme n'existe pas», comme disait Wittgenstein, ou
n'existe que pour ceux qui en sont dupes : ce n’est qu’un jeu ou une illusion.

%ENNUI
\section{Ennui}
«Le temps, c’est ce qui passe quand rien ne se passe. » Cette for-

mule, dont j'ignore l’auteur, dit la vérité de l'ennui : c’est une expé-
rience du temps, mais réduit absurdement à lui-même, comme s'il était
quelque chose en dehors de ce qui dure et change. On s’ennuie quand le temps
semble vide : parce que rien n'arrive, parce qu’on n’a rien à faire, ou parce
qu’on échoue à s’y intéresser. Souvent c’est parce qu’on attend un avenir qui ne
vient pas, ou qui vient trop lentement pour notre goût, bref qui nous empêche
de désirer le présent : on s'ennuie quand on est séparé du bonheur par son
attente, sans pouvoir agir pour accélérer sa venue. Mais on s’ennuie aussi, bien
souvent, quand on n'attend plus rien, quand on n’est plus séparé du bonheur
par aucun manque, sans pouvoir pour autant être heureux. C’est l’ennui selon
Schopenhauer : l'absence du bonheur au lieu même de sa présence attendue.
On n’y échappe que par la souffrance, comme on n’échappe à la souffrance que
par l'ennui :

\vspace{0.5cm}
{\footnotesize
« Vouloir, s’efforcer, voilà tout leur être ; c’est comme une soif inextinguible. Or tout
vouloir a pour principe un besoin, un manque, donc une douleur... Mais que la
%— 203 —
%{\footnotesize XIX$^\text{e}$} siècle — {\it }
volonté vienne à manquer d’objet, qu’une prompte satisfaction vienne à lui enlever
tout motif de désirer, et les voilà tombés dans un vide épouvantable, dans l'ennui ; leur
nature, leur existence leur pèse d’un poids intolérable. La vie donc oscille, comme un
pendule, de droite à gauche, de la souffrance à l'ennui ; ce sont là les deux éléments
dont elle est faite, en somme. De là ce fait bien significatif par son étrangeté même : les
hommes ayant placé toutes les douleurs, toutes les souffrances dans l’enfer, pour remplir
le ciel n’ont plus trouvé que l'ennui. » ({\it Le monde...}, IV, 57.)
}

\vspace{0.5cm}
L’ennui a pourtant son utilité, qui est de désillusion. Quelqu'un qui ne
s’ennuierait jamais, que saurait-il de soi et de sa vie ? Souvenez-vous de Pascal :

\vspace{0.5cm}
{\footnotesize
« Ennui.

Rien n’est si insupportable à l’homme que d’être dans un plein repos, sans passions,
sans affaires, sans divertissement, sans application.

Il sent alors son néant, son abandon, son insuffisance, sa dépendance, son impuissance,
son vide.

Incontinent il sortira du fond de son âme l'ennui, la noirceur, la tristesse, le chagrin,
le dépit, le désespoir. »
}

\vspace{0.5cm}
C’est toucher le fond, ou constater qu’il n’y en a pas. Bonne occasion pour
s'occuper enfin d’autre chose que de soi.

%EN SOI
\section{En soi}
Ce qui existe en soi, c’est ce qui existe indépendamment d’autre
chose (la substance) et de nous (la chose en soi chez Kant). Mais
c'est aussi ce qui existe sans se penser, indépendamment de toute réflexion et
de toute conscience : c’est l’être qui est ce qu’il est, explique Sartre, de façon
opaque ou massive, sans autre rapport à soi que d'identité ({\it L'être et le néant},
p. 33-34). Se distingue par là du pour soi, comme la matière se distingue de la
conscience ou de l’esprit.

%ENSTASE
\section{Enstase}
Néologisme forgé sur le modèle d’{\it extase}, et qui lui sert de contraire.
C’est entrer en soi, pour se fondre en tout — comme un plongeon
dans l’immanence (dans l’absolu où nous sommes).

Le mot sert surtout pour décrire certaines expériences mystiques, spécialement
orientales. Si l’atman et le Brahman sont un, comme dans l’hindouisme,
ou s’ils n'existent pas, comme dans le bouddhisme, comment pourrait-on
passer de l’un à l’autre? Mystiques non de la rencontre, mais de
l'unité ou de l’immersion.

%— 204 —
%{\footnotesize XIX$^\text{e}$} siècle — {\it }
%ENTÉLÉCHIE
\section{Entéléchie}
L'être en acte, par opposition avec l'être en puissance. Synonyme
en ce sens d’{\it energeia}. Mais le mot, qui appartient au
vocabulaire aristotélicien, désigne l’acte accompli (celui qui a sa fin, {\it telos}, en
lui-même), plutôt que celui en train de se faire (qui tend vers sa fin, et ne l’a
donc pas atteinte). C’est l’acte parfaitement achevé, ou la perfection en acte.

Chez Leibniz, le mot désigne les monades, en tant qu’elles ont en elles
«une certaine perfection, qui les rend sources de leurs actions internes »
({\it Monadologie}, 18).

%ENTENDEMENT
\section{Entendement}
La raison modeste et laborieuse, celle qui refuse les facilités
de l’intuition et de la dialectique autant que les tentations
de l’absolu, et qui se donne par là les moyens de connaître. C’est la puissance
de comprendre, en tant qu’elle est toujours finie et déterminée — notre
accès particulier (puisque humain) à l’universel. Son défaut est d'imaginer partout
un ordre, pour ne se perdre pas.

%ENTHOUSIASME
\section{Enthousiasme}
« Le mot grec, écrit Voltaire, signifie {\it émotion d'entrailles,
agitation intérieure}. » Ce n’est pas ce qu’on lit
dans nos dictionnaires, qui rattachent le mot au verbe {\it enthousiazein}, « être inspiré
par la divinité », lui-même dérivé du substantif {\it theos}. L’enthousiasme est
un transport divin, ou se croit tel, ou y ressemble. Il se pourrait pourtant que
Voltaire n’ait pas tout à fait tort : que les entrailles jouent un plus grand rôle,
dans l’enthousiasme, que la divinité.

En un sens plus large, le mot désigne une exaltation joyeuse ou admirative. C’est
une espèce d'ivresse, mais tonique, mais généreuse, que la raison doit pourtant
apprendre à contrôler. Le danger est d’y perdre tout sens critique, toute indépendance,
toute lucidité, tout recul. C’est vrai surtout des enthousiasmes collectifs, qui
font la mode et les fanatismes. « L'esprit de parti dispose merveilleusement à
l'enthousiasme, notait Voltaire ; il n’est point de faction qui n’ait ses énergumènes. »

%ENTITÉ
\section{Entité}
Un étant ({\it ens, entis}), qu’on peut penser, mais qu’on ne peut prendre
tout à fait pour une chose ou un individu : c’est un être abstrait, ou
l’abstraction d’un être.

%ENTROPIE
\section{Entropie}
Qualifie l’état d’un système physique isolé (ou considéré comme
tel) par la quantité de transformation spontanée dont il est
%— 205 —
%{\footnotesize XIX$^\text{e}$} siècle — {\it }
capable : l’entropie est à son maximum quand le système est devenu incapable
de se modifier lui-même — parce qu’il a atteint son état d'équilibre, qui est
aussi, d’un point de vue statistique et au niveau des particules qui le composent,
son état le moins ordonné ou le plus probable. Ainsi, c’est l’exemple traditionnel,
dans une tasse de café : il est exclu que le café se réchauffe ou se
sépare de lui-même du sucre qu’on y a mis (il ne peut que refroidir et rester
sucré). Le second principe de la thermodynamique stipule que l’entropie, dans
un système clos, ne peut que croître, ce qui suppose que le désordre y tend vers
un maximum : c’est ce que confirment l’histoire de l'univers (à l’exception de
la vie) et la chambre de nos enfants (à l'exception du ménage). Le soleil ou les
parents paient la facture.

%ENVIE
\section{Envie}
Le désir de ce qu’on n’a pas et qu’un autre possède, joint au désir
d’être cet autre ou de prendre sa place. Il y a de la haine dans l’envie,
presque toujours. Et de l’envie dans la haine, bien souvent. Comme on nous
pardonne mieux nos échecs que nos réussites ! Cela vaut spécialement dans la
vie intellectuelle. La haine s’accroît en proportion du succès.

%ÉPICURISME
\section{Épicurisme}
La doctrine d’Épicure et de ses disciples. C’est un matérialisme
radical, qui prolonge l’atomisme de Démocrite : rien
n'existe que les atomes en nombre infini dans le vide infini ; rien n’advient que
leurs mouvements ou leurs rencontres. C’est aussi un sensualisme paradoxal,
puisque les atomes et le vide, qui font toute la réalité, sont insensibles. C’est
enfin, et surtout, un hédonisme exigeant : le plaisir, qui est le seul bien, culmine
dans ces plaisirs de l’âme que sont la philosophie, la sagesse et l'amitié.
Cette âme n’est bien sûr qu’une partie du corps, composée d’atomes simplement
plus mobiles que les autres : elle mourra avec lui. Pas d’autre vie que
celle-ci. Pas d’autre récompense que le plaisir de bien la vivre. Nulle providence.
Nul destin. Nulle finalité. Notre monde? Ce n’est qu’un agrégat
d’atomes, qui est né par hasard, qui aura nécessairement une fin. Les dieux ? Ils
sont aussi matériels que le reste, et incapables d’ordonner une nature qu’ils
n'ont pas créée et qui les contient. Au reste, ils ne s’occupent pas des humains :
leur propre bonheur leur suffit. À nous, qui ne sommes pas des dieux, de
prendre modèle sur eux. Cela passe par un «quadruple remède» (le
{\it tetrapharmakon}) : comprendre que la mort n’est rien, qu’il n’y a rien à craindre
des dieux, qu’on peut supporter la douleur, qu’on peut atteindre le bonheur.
Le remède est simple. Le chemin, toutefois, ne va pas de soi : il suppose que
nous renoncions aux désirs vains, ceux qui ne peuvent être rassasiés (désirs de
%— 206 —
%{\footnotesize XIX$^\text{e}$} siècle — {\it }
gloire, de pouvoir, de richesse), pour nous consacrer aux désirs naturels et
nécessaires (manger, boire, dormir, philosopher...), qui sont bornés et faciles à
satisfaire. Beaucoup ont cru que l’hédonisme épicurien débouchait ici sur une
espèce d’ascétisme. C’est se méprendre. S’il faut renoncer à jouir toujours plus,
ce n’est pas par dédain du plaisir ou fascination pour l'effort : c’est pour jouir au
mieux. « Épicure, nous dit Lucrèce, fixa des bornes au désir comme à la
crainte. » Les deux vont ensemble. Si tu désires n’importe quoi, tu auras peur de
tout. Si tu ne désires que ce qui est à portée de main ou d'âme, tu n’auras peur
de rien. L’épicurisme n’est pas un ascétisme ; c’est un hédonisme {\it a minima}.
Encore ne l’est-il que relativement aux objets de la jouissance. Car la jouissance
elle-même, libérée du manque et de la peur, est une jouissance maximale : « Du
pain d’orge et de l’eau donnent le plaisir extrême, écrivait Épicure, lorsqu'on les
porte à sa bouche en ayant faim ou soif. » Ainsi le plaisir est « le commencement
et la fin de la vie heureuse », mais pour celui seulement qui sait choisir entre ses
désirs. C’est la sagesse la plus simple et la plus difficile : l’art de jouir (plaisirs du
corps) et de se réjouir (plaisirs de l’âme) sereinement — « comme un dieu parmi
les hommes ». C’est où l’hédonisme mène à l’eudémonisme.

%ÉPISTÉMOLOGIE
\section{Épistémologie}
C’est la partie de la philosophie qui porte non sur le
savoir en général (théorie de la connaissance, gnoséologie),
mais sur une ou plusieurs sciences en particulier. Une théorie de la
connaissance se situe plutôt en amont du savoir : elle se demande à quelles
conditions les sciences sont possibles. Une épistémologie, en aval : elle s’interroge
moins sur les conditions des sciences que sur leur histoire, leurs méthodes,
leurs concepts, leurs paradigmes. Elle sera le plus souvent régionale ou plurielle
(l’épistémologie des mathématiques n’est pas celle de la physique, qui n’est pas
celle de la biologie...). C’est philosophie appliquée, comme elle est presque
toujours, mais sur le terrain des sciences plutôt que sur le sien propre. L’épistémologue
campe en terre étrangère (c’est ordinairement un scientifique qui s’est
lancé dans la philosophie, ou un philosophe qui s'intéresse aux sciences). Il
méprise un peu les autochtones, ceux qui ne parlent qu’une seule langue, qui
ne connaissent qu’un seul continent, où ils se croient chez eux. Il voudrait leur
faire la leçon. Eux l’écoutent, quand ils en ont le temps, avec la politesse
quelque peu condescendante qu’on réserve aux touristes étrangers.

%{\it ÉPOCHÉ}
\section{Époché}
Le mot, en grec, signifie arrêt ou interruption. Dans le langage
philosophique, où l’on omet souvent de le traduire, il désigne la
suspension du jugement (spécialement chez les sceptiques) ou la mise entre
%— 207 —
%{\footnotesize XIX$^\text{e}$} siècle — {\it }
parenthèses du monde objectif (spécialement chez les phénoménologues) : c’est
alors s'abstenir de toute position portant sur le monde, pour ne plus laisser
paraître que l’absolu de la conscience ou de la vie. Synonyme, en ce dernier
sens, de réduction phénoménologique (Husserl, {\it Méditations cartésiennes}, I, 8).

%ÉQUIPE
\section{Équipe}
Un petit groupe organisé, en vue d’une fin commune. Le contraire
d’une foule, et le moyen parfois de la contrôler ou de lui plaire.

%ÉQUITÉ
\section{Équité}
La vertu qui permet d’appliquer la généralité de la loi à la singularité
des situations concrètes : c’est « un correctif de la loi », écrit
Aristote ({\it Éthique à Nicomaque}, V, 14), qui permet d’en sauver l'esprit quand la
lettre n’y suffit pas. C’est justice appliquée, justice en situation, justice vivante,
et la seule qui soit vraiment juste.

En un second sens, plus vague, le mot finit par désigner la justice elle-même,
en tant qu'elle ne peut se réduire ni à l'{\it égalité} (donner ou demander à
tous les mêmes choses, ce ne serait pas juste : ils n’ont ni les mêmes besoins ni
les mêmes capacités) ni à la {\it légalité} (puisqu’une loi peut être injuste). Disons
que c’est la justice au sens moral du terme, qui permet seule de juger l’autre.
On remarquera qu’elle porte pourtant l'égalité dans son nom ({\it aequus}, égal).
Cela dit sans doute l'essentiel. Elle est la vertu qui restaure légalité de droit,
non seulement contre les inégalités de fait, qui demeurent, mais en en tenant
compte. Par exemple en matière de fiscalité : un impôt progressif, qui taxe plus
lourdement les plus riches, est plus équitable qu’un impôt simplement proportionnel,
qui demanderait à chacun la même portion de son revenu. C’est considérer
que les hommes sont égaux en droits et en dignité, même quand ils sont
inégaux, comme presque toujours, en talents, pouvoirs, richesses.

%ÉQUIVOQUE
\section{Équivoque}
Synonyme à peu près d’ambiguïté, avec quelque chose de
plus inquiétant : l’équivoque est comme une ambiguïté
volontaire, néfaste ou menaçante. Une situation ambiguë, par exemple entre
un homme et une femme, peut être charmante. Une situation équivoque sera
plutôt gênante ou suspecte. C’est une ambiguïté mal intentionnée ou mal
interprétée.

%ÉRISTIQUE
\section{Éristique}
L’art de la controverse ({\it erizein}, disputer), ou ce qui en relève.
Mais c'est moins un art qu’un artifice. « Est éristique, écrit
%— 208 —
%{\footnotesize XIX$^\text{e}$} siècle — {\it }
Aristote, le syllogisme qui part d'opinions qui, tout en paraissant probables, en
réalité ne le sont pas » ({\it Topiques}, I, 1). Le substantif peut devenir un synonyme
de sophistique. Mais il désigne plus spécifiquement un type d’argumentation
critique, qui tend moins à établir une vérité qu’à réfuter la position d’un adversaire.
Les mégariques s’en étaient fait une espèce de spécialité : de là le nom
d’école éristique qu’on leur donne parfois.

%ÉROTISME
\section{Érotisme}
L'art de jouir ? Plutôt l’art de désirer, et de faire désirer, jusqu’à
jouir du désir même (le sien, celui de l’autre) pour en obtenir
une satisfaction plus raffinée ou plus durable. L’orgasme est à la portée de
n'importe qui: chacun, pour soi-même, y suffit. Mais qui voudrait s’en
contenter ?

%ERREUR
\section{Erreur}
Le propre de l’erreur est qu’on la prend pour une vérité. C’est ce
qui distingue l'erreur de la simple fausseté (on peut savoir qu’on
dit faux, non savoir qu’on se trompe) et qui lui interdit d’être volontaire. Ainsi
l'erreur n’est pas seulement une idée fausse : c’est une idée fausse qu’on croit
vraie. En tant qu’elle est fausse, elle n’a d’être que négatif (voir {\it fausseté}). Mais
en tant qu’elle est idée, elle fait partie du réel ou du vrai (on se trompe
réellement : l'erreur est vraiment fausse). Par exemple, explique Spinoza, « les
hommes se trompent en ce qu’ils se croient libres ; et cette opinion consiste en
cela seul qu’ils ont conscience de leurs actions et sont ignorants des causes par
où ils sont déterminés » ({\it Éthique}, II, 35, scolie). Ce n’est pas parce que nous
sommes libres que nous nous trompons, comme le voulait Descartes ; c’est
parce que nous nous trompons que nous nous croyons libres, et cette erreur
n’est elle-même qu’une vérité incomplète (puisqu'il est vrai que nous agissons).
On ne se trompe que par ignorance ou impuissance. L'erreur n’est rien de
positif : il n’y a que des connaissances partielles ou inachevées. Par quoi la
pensée est un travail, et l'erreur, un moment nécessaire.

%ESCHATOLOGIE
\section{Eschatologie}
La doctrine des fins dernières de l’humanité ou du
monde. Le mot {\it fin} se prenant en deux sens (comme
{\it finitude} ou comme {\it finalité}), il semble que l’eschatologie doive aussi être
double : la mort et la fin du monde en relèveraient, autant que le jugement dernier
ou la résurrection. En pratique, toutefois, on ne parle guère d’eschatologie
que pour des pensées finalistes ou religieuses. C’est que le néant ne fait pas sens.
Une eschatologie matérialiste, comme on voit chez Lucrèce, serait plutôt une
%— 209 —
%{\footnotesize XIX$^\text{e}$} siècle — {\it }
anti-eschatologie : d’abord parce que l’univers est sans fin ({\it De rerum}, I, 1001),
ensuite parce que aucun monde, dans l’univers, n’échappe à la mort (II, 1144-1174,
V, 91-125 et 235-415), ni aucune vie dans le monde (III, 417-1094). Le
salut n’est pas à espérer, il est à faire.

%ÉSOTÉRIQUE
\section{Ésotérique}
Ce qui est réservé aux initiés ou aux spécialistes. Le mot,
pris en lui-même, n’est pas péjoratif. Mais il le devient, légitimement,
si l'initiation est elle-même réservée à certains, et spécialement si elle
suppose une foi préalable : c’est soumettre l’universel au particulier, l’école à la
secte, et l'esprit au gourou.

%ÉSOTERISME
\section{Ésotérisme}
Toute doctrine qui réserve la vérité aux initiés. C’est le
contraire des Lumières, et un proche parent, presque toujours,
de l’occultisme. C’est un obscurantisme savant, ou qui voudrait l’être.

%ESPACE
\section{Espace}
Ce qui reste quand on a tout ôté : le vide, mais à trois dimensions.
On voit que ce n’est qu’une abstraction (si l’on ôtait vraiment tout,
il n’y aurait plus rien : ce ne serait pas l’espace mais le néant), qu’on conçoit
davantage qu’on ne l’expérimente. C’est l'étendue, mais considérée indépendamment
des corps qui l’occupent ou la délimitent. C’est l’univers, mais considéré
indépendamment de son contenu (indépendamment de lui-même !).
L'espace est à l’étendue ce que le temps est à la durée : son abstraction,
qu'on finit par prendre pour son lieu ou sa condition. S'il n’y avait pas le
temps, se demande-t-on, comment les corps pourraient-ils durer ? S’il n’y avait
pas l’espace, comment pourraient-ils s'étendre ? C’est soumettre le réel à la
pensée, quand c’est le contraire qu’il faut faire. Ce n’est pas parce qu’il y a du
temps que l'être dure ; c’est parce qu’il dure qu’il y a du temps. Ce n’est pas
parce qu'il y a de l’espace que l’être est étendu ; c’est parce qu’il est étendu — ou
parce qu'il s’étend — qu’il y a de l’espace. L'espace n’est pas un être ; c’est le lieu
de tous, considéré en faisant abstraction de leur existence ou de leur localisation.
Non un être, donc, mais une pensée : c’est le lieu universel et vide. Comment
ne serait-il pas infini, continu, homogène, isotrope et indéfiniment divisible,
puisque aucun corps, par définition, ne peut le limiter, le rompre ou
l’orienter ? Mais cela nous en apprend plus sur notre pensée que sur l'étendue
du réel ou de l’univers.
L'espace est-il seulement une forme de la sensibilité, comme le voulait
Kant ? Ce n’est guère vraisemblable : s’il n’était que cela, {\it où} la sensibilité aurait-elle
%— 210 —
%{\footnotesize XIX$^\text{e}$} siècle — {\it }
pu apparaître ? Et comment penser l’extension de l’univers, des milliards
d’années avant l'existence de toute vie et de toute sensibilité ? On dira que cela
ne prouve rien, puisque cette extension et ces milliards d’années n’existent pour
nous qu’à partir de notre sensibilité. J’en suis d’accord : l’idéalité transcendantale
de l’espace n’est pas réfutable. Mais son objectivité ne l’est pas davantage
(que l’espace soit une forme de la sensibilité, cela n’empêche pas qu’il soit aussi
une forme de l’être), tout en étant plus économe : elle ne suppose pas l'existence
d’un être non spatial, comme Kant est obligé de faire, pour rendre
l’espace pensable.

L'espace, comme le temps, contient tout, mais dans la simultanéité d’un
même présent : c’est « l’ordre des coexistences, écrivait Leibniz, comme le
temps est l’ordre des successions ». On remarquera qu’au présent les deux ne
font qu’un, qu’on peut appeler (comme font les physiciens, mais pour d’autres
raisons) l’espace-temps. C’est le lieu des présences, ou le présent comme lieu.

%ESPÈCE
\section{Espèce}
Un ensemble, le plus souvent défini, à l’intérieur d’un ensemble
plus vaste (le genre prochain), par une ou plusieurs caractéristiques
communes (les différences spécifiques). Par exemple l’espérance et la volonté
sont deux espèces de désir, comme les tigres et les chats deux espèces de félins.
Pour le biologiste, une espèce se reconnaît ordinairement à l’interfécondité :
deux individus de sexes différents appartiennent à une même espèce s’ils peuvent
se reproduire et engendrer un être lui-même fécond (par quoi l’âne et le
cheval sont deux espèces différentes : mulets et bardots sont stériles). C’est
pourquoi il vaut mieux parler de l'{\it espèce humaine} que du {\it genre humain}. Que
l'unité de l'humanité soit une valeur morale, cela n'empêche pas qu’elle soit
aussi et d’abord un fait biologique.

%ESPÉRANCE
\section{Espérance}
Une certaine espèce de désir : c’est un désir qui porte sur ce
qu'on n’a pas, ou qui n’est pas (espérer c’est désirer sans
jouir), dont on ignore s’il est ou s’il sera satisfait (espérer c’est désirer sans
savoir), enfin dont la satisfaction ne dépend pas de nous (espérer, c’est désirer
sans pouvoir). S’oppose pour cela à la volonté (un désir dont la satisfaction
dépend de nous), à la prévision rationnelle (lorsque l’avenir peut faire l’objet
d’un savoir ou d’un calcul de probabilités), enfin à l’amour (lorsqu’on désire ce
qui est ou ce dont on jouit). Cela indique assez le chemin : ne t’interdis pas
d’espérer ; apprends plutôt à vouloir, à connaître, à aimer.
L’espérance porte le plus souvent sur l’avenir : c’est qu’il est, de tous nos
objets de désirs, celui qui est le plus souvent soustrait à toute jouissance, à toute
%— 211 —
%{\footnotesize XIX$^\text{e}$} siècle — {\it }
connaissance et à toute action possibles. Le passé est mieux connu. Le présent,
plus disponible. Cela n'empêche pas d’espérer ce qui fut (« j’espère que je ne
l'ai pas blessé ») ou ce qui est (« j'espère qu’il est guéri ») : il suffit pour cela
qu'on le désire, que cela ne dépende pas de nous et qu’on ignore ce qu’il en est.
L'orientation temporelle est moins essentielle à l’espérance que l'impuissance et
l'ignorance : nul n’espère ce qu’il sait ni ce qu’il peut. L’espérance, marque de
notre faiblesse. Comment serait-ce une vertu ? C’est le désir le plus facile et le
plus faible.

%ESPOIR
\section{Espoir}
Souvent synonyme d’espérance. Lorsqu'on veut les distinguer, c’est
presque toujours au bénéfice de cette dernière : l’espérance serait
une vertu, l'espoir ne serait qu’une passion. C’est le cas spécialement dans la
théologie chrétienne, où l’espérance est l’une des trois vertus théologales, parce
qu'elle a Dieu même pour objet. Qu’en conclure ? Qu’à chaque fois que
j'espère autre chose que Dieu, ou autrement qu’en Dieu, ce n’est pas une espérance,
ce n’est qu’un espoir, passionnel et vain comme ils sont tous. Et que
cette distinction n’a guère de sens pour un philosophe non religieux : les Grecs
ne la faisaient pas, et je ne vois nulle raison de la faire.

Spinoza ne la faisait pas davantage. Qu'est-ce que l'espoir ? « Une joie
inconstante, répon\-dait-il, née de l’idée d’une chose future ou passée, de l’issue
de laquelle nous doutons en quelque mesure. » C’est pourquoi, selon une formule
fameuse de l’{\it Éthique}, « il n’y a pas d’espoir sans crainte ni de crainte sans
espoir » (III, 50, scolie, et déf. 13 des affects, explication). Le même doute,
nécessaire à l’un et l’autre, fait qu’ils ne peuvent exister qu’ensemble. Espérer,
c'est craindre d’être déçu ; craindre, c’est espérer d’être rassuré. La sérénité, si
elle exclut la crainte, exclut donc aussi tout espoir : ce que j’ai appelé le {\it gai
désespoir}, et que Spinoza, plus sage que moi, appelle la sagesse ou la béatitude.
Le thème est stoïcien avant d’être spinoziste : « Tu cesseras de craindre, disait
Hécaton, si tu as cessé d’espérer. » Et cynique avant d’être stoïcien : « Seul est
libre, disait Démonax, celui qui n’a ni espoir ni crainte. » Le sage n’espère rien :
il a cessé d’avoir peur. Il ne craint rien : il a cessé d’espérer quoi que ce soit.
Parce qu’il serait sans désirs ? Au contraire : parce qu’il ne désire que ce qui est
(ce n’est plus espérance mais amour) ou que ce qu’il peut (ce n’est plus espérance
mais volonté).

On dira que cette sagesse est pour nous hors d’atteinte : l'espoir est là, toujours,
puisque la faiblesse est là, puisque l'ignorance est là, puisque l’angoisse
est là. Sans doute. Aussi ne sert-il à rien, je l’ai dit bien souvent, de vouloir
s’'amputer vivant de toute espérance : ce serait faire de la sagesse un nouvel
espoir, qui nous en séparerait aussitôt. Développons plutôt notre part de puissance,
%— 212 —
%{\footnotesize XIX$^\text{e}$} siècle — {\it }
de liberté, de joie : apprenons à connaître, à agir, à aimer. La sagesse
n’est pas un idéal ; c’est un processus. « Plus nous nous efforçons de vivre sous
la conduite de la raison, écrit Spinoza, plus nous faisons effort pour nous
rendre moins dépendants de l'espoir, nous affranchir de la crainte, commander
à la fortune autant que nous le pouvons, et diriger nos actions suivant le sûr
conseil de la raison » ({\it Éthique}, IV, scolie de la prop. 47).

%ESPRIT
\section{Esprit}
La puissance de penser, en tant qu’elle a accès au vrai, à l’universel
ou au rire.

Le mot, en ce sens, ne s’utilise guère qu’au singulier (parler {\it des esprits}, c'est
superstition). C’est que la vérité, pour autant qu’on y accède, est la même en
tous. C’est en quoi elle est libre (elle n’obéit à personne, pas même au cerveau
qui la pense), et libère. Cette liberté en nous, qui n’est pas celle d’un sujet mais
de la raison, c’est l’esprit même.

On se trompe si l’on y voit une substance, mais pas moins si l’on n’y voit
qu’un pur néant. L'esprit n’est pas une hypothèse, disait Alain, puisqu'il est
incontestable que nous pensons. Ni une substance, puisqu'il ne peut exister
seul. Disons que c’est le corps en acte, en tant qu'il a la vérité en puissance.

En puissance, non en acte. Aucun esprit n’est la vérité ; aucune vérité n'est
esprit (ce serait Dieu). C’est pourquoi l'esprit doute de lui-même et de tout. Il
sait qu’il ne sait pas, ou qu’il ne sait que peu. Il s’en inquiète ou s’en amuse.
Deux façons (par la réflexion, par le rire) d’être fidèle à soi, sans se croire.
L'esprit, sous ces deux formes, semble le propre de l’homme. C’est aussi une
vertu : celle qui surmonte le fanatisme et la bêtise.

%ESPRIT FAUX
\section{Esprit faux}
Un don particulier pour l’erreur. C’est l'incapacité à raisonner
juste, au moins sur certaines questions, et quelque intelligent
qu’on puisse être par ailleurs. C’est comme un manque de bon sens, qui
autoriserait À penser n'importe quoi. Qui disait de Sartre qu’il était « un grand
esprit faux » ? Cela ne l’empêchait pas d’avoir du talent, et sans doute davantage
que l'inventeur de la formule. «Les plus grands génies peuvent avoir
l'esprit faux sur un principe qu’ils ont reçu sans examen, notait Voltaire.
Newton avait l'esprit faux quand il commentait l'Apocalypse. » Et Voltaire,
pour d’autres raisons, quand il parle de l'Ancien Testament.

%ESSENCE
\section{Essence}
Le mot, qui semble mystérieux, a pourtant une étymologie transparente :
il est forgé à partir de l’infinitif du verbe {\it être} en latin
%— 213 —
%{\footnotesize XIX$^\text{e}$} siècle — {\it }
{\it (esse)}. En l'occurrence, l’étymologie trompe moins que le mystère : l’essence
d’une chose, c’est son être vrai ou profond (par opposition aux apparences, qui
peuvent être superficielles ou trompeuses), autrement dit ce qu’elle est (par
opposition au simple fait qu’elle soit : son existence ; mais aussi à ce qui lui
arrive : ses accidents). Synonyme à peu près de nature (l'essence d’une chose,
c'est sa nature véritable), mais préférable : parce que le mot peut s’appliquer
aussi à des objets culturels. L’essence d’un homme, par exemple, c’est ce qu’il
est. Qui peut croire que la nature suffise à l'expliquer ?

L’essence, c’est donc ce qui répond aux questions « Quoi ? » ou « Qu'est-ce
que c’est ? » ({\it Quid ?}, en latin, ce pourquoi les scolastiques parlaient plutôt de
{\it quiddité}). Reste à savoir si ce qui répond à cette question c’est une définition
ou un être, et si cet être est individuel ou générique. Soit cette table sur laquelle
j'écris. Quelle est son essence ? D’être une table, ou d’être cette table-ci ? Des
mots, ou du réel ? Une idée, ou un processus ? Il se pourrait qu’il n’y ait pas
d’essences du tout, mais seulement des accidents, des rencontres, de l’histoire —
non des êtres, mais des événements. On dira que pour que quelque chose
arrive, il faut déjà que quelque chose soit. Sans doute. Mais pourquoi serait-ce
autre chose que cela même qui arrive ?

C'est où l’on retrouve Spinoza. Qu'est-ce que l’essence d’une chose
singulière ? Non, du tout, une abstraction ou une virtualité, mais son être
même, considéré dans sa dimension affirmative, autrement dit dans sa puissance
d'exister : ce qui la fait être (ce qui la « pose », écrit Spinoza), mais de
l’intérieur (à la différence des causes qui la font être de l'extérieur). Par exemple
cette table : qu’elle ait des causes externes, c’est une évidence ; mais elle n’existerait
pas si elle n’avait en elle-même une essence affirmative — une puissance
d’être — qui ne peut pas plus exister ou être conçue sans la table que la table ne
peut exister ou être conçue sans elle ({\it Éthique}, II, déf. 2). Il en résulte que
« l'effort par lequel chaque chose s’efforce de persévérer dans son être n’est rien
en dehors de l’essence actuelle de cette chose » ({\it Éth.}, III, prop. 7) : l'essence
d’un être, c’est sa puissance d’exister ; son existence, c’est son essence en acte.

Toute la difficulté est de penser les deux ensemble, dans leur simultanéité
nécessaire — ce qui récuse l’essentialisme aussi bien que l’existentialisme. « L’essence,
écrit Sartre après Hegel, c’est ce qui a été » ({\it L'être et le néant}, p.72
et 577). Mais comment, si le passé n’est plus ? L’être et l'événement au présent
ne font qu’un : ainsi l'essence et l’existence.

%ESSENTIALISME
\section{Essentialisme}
Le contraire de l’existentialisme et du nominalisme : c’est
croire que l'essence précède l'existence, ou qu’un être est
contenu dans sa définition. Confiance exagérée dans le langage ou la pensée.
%— 214 —
%{\footnotesize XIX$^\text{e}$} siècle — {\it }
C’est le péché mignon des philosophes. La critique de l’essentialisme, bien
avant Popper ou Sartre, a été énoncée au plus court par Shakespeare : «Il y a
plus de choses dans le ciel et sur la terre, Horatio, que n’en rêve ta
philosophie. » Ou encore : « Qu’y a-t-il donc en un nom ? Ce que nous appelons
une rose, sous un autre nom, sentirait aussi bon. » Non qu’on doive se
passer de définitions, ni même qu’on le puisse ; mais parce que aucune définition
ne tient lieu d’expérience, ni d’existence.

%ESTHÈTE
\section{Esthète}
Celui qui aime le beau — et spécialement le beau artistique — plus
que tout, au point de lui sacrifier ou de lui soumettre tout le
reste. Le vrai ? Le bien ? Ils ne valent, pour l’esthète, que s’ils sont beaux :
mieux vaut un beau mensonge qu’une vérité laide ; mieux vaut un beau crime
qu’une faute de goût. L’esthète n’est pas toujours un artiste, ni souvent (la plupart
des grands créateurs mettent le vrai ou le bien plus haut que le beau). C’est
un croyant. Il a fait de l’art sa religion : l'esthétique lui tient lieu de logique, de
morale, de métaphysique. Philosophiquement, cela culmine chez Nietzsche :
« Pour nous, seul le jugement esthétique fait loi », écrit-il. Et d’ajouter : « L'art
et rien que l’art ! C’est lui qui nous permet de vivre, qui nous persuade de vivre,
qui nous stimule à vivre. L'art a {\it plus de valeur} que la vérité. L'art au service
de l'illusion — voilà notre culte » ({\it Volonté de puissance}, IV, 8, et III, 582). Ce
qui résume au fond l'essentiel, par quoi Nietzsche est un esthète, et qui
m’empêche d’être nietzschéen.

%ESTHÉTIQUE
\section{Esthétique}
L’étude ou la théorie du beau. On considère habituellement
que c’est une partie de la philosophie, plutôt qu'un des
Beaux-Arts. C’est justice. Le concept de beau n’est pas beau. Le concept
d'œuvre d’art n’en est pas une. C’est pourquoi les artistes se méfient des esthéticiens,
qui prennent le beau pour une pensée.

%ESTHÉTIQUE TRANSCENDANTALE
\section{Esthétique transcendentale}
La première partie de la {\it Critique
de la raison pure} de Kant. Elle ne
porte pas sur le beau, qui sera étudié dans la {\it Critique de la faculté de juger}, mais
sur la sensation ({\it aisthèsis}) ou la sensibilité. Elle est transcendantale en tant
qu’elle fait ressortir les conditions de possibilité de toute expérience (l’espace et
le temps comme formes {\it a priori} de la sensibilité). Se distingue de la Logique
transcendantale, comme les formes de la sensibilité se distinguent des formes de
la pensée (les catégories et principes de l’entendement).

%— 215 —
%{\footnotesize XIX$^\text{e}$} siècle — {\it }
%ESTIME
\section{Estime}
C’est un respect particulier : non celui qu’on doit à tout être
humain, mais celui qu’on réserve à ceux qu’on juge les meilleurs,
tant que leur valeur ne passe pas la norme commune ou la nôtre (auquel cas ce
n'est plus estime mais admiration). L’estime manifeste une sorte d'égalité positive,
qui en fait le prix. Ce n’est pas encore l’amitié, mais presque toujours une
de ses conditions. Je peux estimer sans aimer. Mais comment aimer celui ou
celle que je méprise ?

%ÉTANT
\section{Étant}
L'être en train d’être : l'être au présent, et le seul. L’usage substantivé
de ce participe présent, quoique heurtant nos oreilles, permet
d'éviter l'ambiguïté du mot {\it être}, qui en français désigne à la fois {\it ce qui est} (le {\it to
on} des Grecs, l’{\it ens} des Latins ou des scolastiques : l’étant) et le fait que ce qui
est {\it soit} (acte d’être : {\it to einai} ou {\it esse}). Dans son grand livre sur saint Thomas,
Étienne Gilson notait que ce dernier disposait de « deux vocables distincts,
pour désigner un étant, {\it ens}, et pour désigner l’acte même d’être, {\it esse} », ce qui
n'est pas le cas du français et rend un certain nombre de traductions, faute de
faire cette différence, inintelligibles. « La seule solution satisfaisante du problème,
ajoutait Gilson en note, serait d’avoir le courage de reprendre la terminologie
essayée au {\footnotesize XVII$^\text{e}$} siècle par quelques scolastiques français, qui traduisaient
{\it ens} par {\it étant}, et {\it esse} par {\it être} » ({\it Le thomisme}, Vrin, 1979, p. 170). Cet
usage tend aujourd’hui à se répandre, mais davantage par l'influence du {\it Seiend}
allemand que par celle de saint Thomas ou de Gilson. Chez Heidegger et les
heideggériens, l’étant (ce qui est : cette table, cette chaise, vous, moi...) se distingue
en effet de l’être — c’est la fameuse « différence ontologique » — sans être
pourtant autre chose. Que l'arbre soit un arbre, et cet arbre-ci, c’est sa banalité
d’étant. Mais qu’il {\it soit}, c’est l'événement de l'être. Ainsi l’étant est l’être même,
quand on s'interroge sur ce qu’il est au lieu de s'étonner de ce qu’il soit. Et
réciproquement : l’être est l’étant, quand on s’étonne qu’il {\it soit} au lieu de chercher
seulement {\it ce} qu’il est ou à quoi il peut servir.

%ÉTAT
\section{État}
Une façon d’être. Pris absolument, et avec une majuscule, c’est un
corps politique, qui rassemble un certain nombre d'individus (le
peuple) sous un même pouvoir (le souverain). Quand le peuple et le souverain
sont un, l’État est une république.

%ÉTAT CIVIL
\section{État civil}
Le contraire de l’état de nature : c’est la vie en société, en tant
qu’elle suppose un pouvoir et des lois.

%— 216 —
%{\footnotesize XIX$^\text{e}$} siècle — {\it }
%ÉTAT DE NATURE
\section{État de nature}
L'état sans État : situation des êtres humains avant
l'instauration d’un pouvoir commun, de règles communes,
voire avant toute vie en société. État purement hypothétique, vraisemblablement
insatisfaisant. « La vie de l’homme, disait Hobbes, est alors solitaire,
besogneuse, pénible, quasi animale, et brève » ({\it Léviathan}, I, 13).

%ÉTENDUE
\section{Étendue}
L‘étendue est à l’espace ce que la durée est au temps: son
contenu, sa condition, sa réalité. L’étendue d’un corps, c’est la
portion d’espace que ce corps occupe ; l’espace n’est que l’abstraction d’une
étendue qui existerait indépendamment des corps qui l’occupent ou la traversent.
C’est donc l’étendue qui est première : ce n’est pas parce que les corps
sont dans l’espace qu’ils sont étendus ; c’est parce qu’ils sont étendus, ou parce
qu’ils s'étendent, qu’il y a de l’espace. Cette étendue, on pourrait l’appeler aussi
bien, et peut-être mieux, l'{\it extension} (comme on pourrait appeler la durée
{\it duration}) : c’est le fait de s’étendre et d’occuper ainsi un certain espace.

%ÉTERNITÉ
\section{Éternité}
Si c'était un temps infini, quel ennui ! Cela donnerait raison à
Woody Allen : « L’éternité c’est long, surtout vers la fin... »
C’est qu’il n’y aurait pas de fin : on n’en aurait jamais fini d’attendre, et aucune
raison pour commencer quoi que ce soit. Cela ferait comme un dimanche
infini. Quelle plus belle image de l’enfer ?

Mais l'éternité, au sens où la prennent la plupart des philosophes, c’est tout
autre chose. Ce n’est pas un temps infini (car alors il ne serait composé que de
passé et d’avenir, qui ne sont pas), ni pourtant l’absence de temps (car alors ce
ne serait rien): c’est un présent qui reste présent, comme {\it un perpétuel
aujourd'hui}, disait saint Augustin, et c’est le présent même. Qui a jamais vécu
un seul {\it hier} ? un seul {\it demain} ? Qui a jamais vu le présent cesser ou disparaître ?
C’est toujours aujourd’hui, c’est toujours maintenant : c’est toujours l'éternité,
et c’est en quoi, en effet, elle est éternelle.

On ne confondra pas l'éternité avec l’immuabilité. Que tout change, c’est
une vérité éternelle. Mais rien ne change qu’au présent, et c’est l’éternité vraie.
On ne se baigne jamais deux fois dans le même fleuve ? Sans doute. Mais
encore moins dans un fleuve passé ou futur. Ainsi il n’y a que le présent : il n’y
a que l'éternité du {\it il y a}. Parménide et Héraclite, même combat !

L’éternité peut se penser de deux façons, qu’on peut formuler, par commodité,
selon les deux attributs de Spinoza : selon l’étendue ou selon la pensée.
Selon l'étendue, l'éternité ne fait qu’un avec le devenir : c’est le toujours-présent
du réel (être, c’est être maintenant). Selon la pensée, elle ne fait qu’un avec
%— 217 —
%{\footnotesize XIX$^\text{e}$} siècle — {\it }
la vérité : c’est le toujours-présent du vrai (une vérité n’est jamais future ou
passée : ce qui était vrai l’est encore, ce qui le sera l’est déjà). C’est où le réel et
le vrai, pour la pensée, se séparent : ce qui était réel ne l’est plus, ce qui était vrai
l’est toujours. Par exemple la promenade que je fis hier : ce n’est plus réel, c’est
toujours vrai. Ou celle que je ferai demain, si j’en fais une : ce n’est pas encore
réel, c’est déjà vrai. On évitera pourtant d’absolutiser cette différence. Le réel et
le vrai ne coïncident qu’au présent, certes ; mais ils coïncident donc {\it toujours},
pour tout réel donné, et {\it nécessairement}. Ainsi ces deux éternités n’en font
qu'une (le présent est le lieu de leur conjonction : le point de tangence du réel
et du vrai). C’est en quoi je suis libre de me promener ou non aujourd’hui : ce
n'est pas parce que c'était déjà vrai de toute éternité que je le ferai au présent ;
c'est parce que je le fais au présent, si je le fais, que c’est vrai de toute éternité.
Le réel commande : le présent commande, puisqu'il n’y a rien d’autre, et c’est
en quoi les deux attributs, au présent, ne font qu’un. Pluralité des attributs,
dirait Spinoza, unité de la substance ou de la nature. L’éternité n’est pas un
autre monde ; c’est la vérité de celui-ci.

%ÉTHIQUE
\section{Éthique}
C’est souvent un synonyme de morale, en plus chic. Mieux vaut
donc, quand on ne les distingue pas, parler plutôt de morale.
Mais si on veut les distinguer ? L’étymologie ne nous aide guère. « Morale » et
« éthique » viennent de deux mots — {\it ethos} en grec, {\it mos} ou {\it mores} en latin — qui
signifiaient à peu près la même chose (les mœurs, les caractères, les façons de
vivre et d’agir) et que les Anciens considéraient comme la traduction l’un de
l’autre. Aussi est-ce une distinction qu’ils ne faisaient pas : {\it morale} et {\it éthique} ne
seraient pour eux, si nous les interrogions en français, que deux façons différentes —
l’une d’origine grecque, l’autre d’origine latine — de dire la même
chose. Si l’on veut pourtant se servir de ces deux mots pour penser deux réalités
différentes, comme un usage récent nous y pousse, le plus opératoire est sans
doute de prendre au sérieux ce que l’histoire de la philosophie nous propose de
plus clair : Kant, parmi les Modernes, est le grand philosophe de la morale ; et
Spinoza, de l'éthique. Sans reprendre en détail ce que j'ai montré ailleurs
({\it Valeur et vérité}, chap. 8), cela amène à opposer la morale et l’éthique comme
l'absolu (ou prétendu tel) et le relatif, comme l’universel (ou prétendu tel) et le
particulier, enfin comme l’inconditionnel (impératif catégorique de Kant) et
le conditionné (qui n’admet d’impératifs qu'hypothétiques). En deux mots : la
morale commande, l'éthique recommande. Ces oppositions débouchent sur
deux définitions différentes, que je ne fais ici que rappeler :

Par {\it morale}, j'entends le discours normatif et impératif qui résulte de
l’opposition du Bien et du Mal, considérés comme valeurs absolues ou transcendantes.
%— 218 —
%{\footnotesize XIX$^\text{e}$} siècle — {\it }
Elle est faite de commandements et d’interdits : c’est l’ensemble de
nos devoirs. La morale répond à la question « Que dois-je faire ? ». Elle se veut
une et universelle. Elle tend vers la vertu et culmine dans la sainteté (au sens de
Kant : au sens où une volonté sainte est une volonté conforme en tout à la loi
morale).

Et j'entends par {\it éthique} un discours normatif mais non impératif (ou sans
autres impératifs qu'hypothétiques), qui résulte de l'opposition du {\it bon} et du
{\it mauvais}, considérés comme valeurs simplement relatives. Elle est faite de
connaissances et de choix : c’est l’ensemble réfléchi et hiérarchisé de nos désirs.
Une éthique répond à la question « Comment vivre ? ». Elle est toujours particulière
à un individu ou à un groupe. C’est un art de vivre : elle tend le plus
souvent vers le bonheur et culmine dans la sagesse.

L'erreur, entre l’une et l’autre, serait de vouloir choisir. Nul ne peut se
passer d'éthique, puisque la morale ne répond que très incomplètement à la
question « Comment vivre ? », puisqu'elle ne suffit ni au bonheur ni à la
sagesse. Et seul un sage pourrait se passer de morale : parce que la connaissance
et l’amour lui suffiraient. Nous en sommes loin, et c’est pourquoi nous avons
besoin de morale (voir ce mot).

L’éthique est pourtant la notion la plus vaste. Elle inclut la morale, alors
que la réciproque n’est pas vraie (répondre à la question « Comment vivre ? »,
c’est entre autres choses déterminer la place de ses devoirs ; répondre à la question
« Que dois-je faire ? », cela ne suffit pas à dire comment vivre). Elle est
aussi la plus fondamentale : elle dit la vérité de la morale (qu’elle n’est qu’un
désir qui se prend pour un absolu), et la sienne propre (qu’elle est comme une
morale désillusionnée et libre). Ce serait la morale de Dieu, s’il existait. Nous
ne pouvons ni tout à fait l’atteindre, ni tout à fait y renoncer.

Ainsi l'éthique est un travail, un processus, un cheminement : c’est le
chemin réfléchi de vivre, en tant qu’il tend vers la vie bonne, comme disaient
les Grecs, ou la moins mauvaise possible, et c’est la seule sagesse en vérité.

%ETHNIE
\section{Ethnie}
Un peuple, mais considéré d’un point de vue culturel plutôt que
biologique (ce n’est pas une race) ou politique (ce n’est ni une
nation ni un État).

%ETHNOCENTRISME
\section{Ethnocentrisme}
C'est juger les cultures des autres à partir de la sienne
propre, érigée (le plus souvent inconsciemment) en
absolu. Tendance spontanée de tout être humain, dont on ne sort, toujours
incomplètement, que par l’étude patiente et généreuse des autres cultures, ce
%— 219 —
%{\footnotesize XIX$^\text{e}$} siècle — {\it }
qui amène à relativiser celle dans laquelle on a été élevé. La difficulté est alors
de ne pas renoncer pour autant à toute exigence d’universalité, ni à toute normativité.
Si tous les points de vue se valaient, au nom de quoi combattre
l’ethnocentrisme ?

%ETHNOCIDE
\section{Ethnocide}
La destruction délibérée d’une culture. À ne pas confondre
avec le génocide, qui veut supprimer une race ou un peuple.
Par exemple au Tibet, occupé par la Chine : c’est l’ethnocide qui menace, non,
semble-t-il, le génocide.

%ETHNOGRAPHIE
\section{Ethnographie}
L’étude descriptive d’une ethnie ou en général d’un
groupe humain, considéré dans ses spécificités culturelles
ou comportementales. Se distingue de l’ethnologie par son aspect surtout
empirique : l’ethnographe observe et décrit ; l’ethnologue compare, classe,
interprète, théorise. L’ethnographie requiert une présence sur le terrain, souvent
pendant de très longues périodes ; l’ethnologie, s'appuyant sur les travaux
ethnographiques disponibles, peut se faire en chambre ou en amphithéâtre, ce
qui est tout de même plus confortable. En pratique, un ethnologue est souvent
un ethnographe qui a réussi ou qui s’est lassé des voyages.

On parle surtout d’ethnographie à propos de populations dites primitives.
Mais rien n'empêche, et cela se fait de plus en plus, d’appliquer les mêmes exigences
à l’étude d’une cité ouvrière, d’une entreprise ou d’un parti politique.
C’est alors une partie de la sociologie, davantage que de l’anthropologie : elle
nous en apprend moins sur l’homme que sur la société.

%ETHNOLOGIE
\section{Ethnologie}
L'étude comparative des ethnies et en général des groupes
humains. Correspond à peu près à ce que les Anglo-Saxons,
qui ne parlent plus guère d’ethnologie, appellent plutôt l’anthropologie sociale
et culturelle. L’ethnologie fait partie des sciences humaines : elle contribue à
nous faire mieux connaître l'humanité, en faisant ressortir un certain nombre
de différences mais aussi d’invariants structurels ou comportementaux.
Comme le remarque Lévi-Strauss, l’ethnologie prolonge une remarque de
Rousseau : « Quand on veut étudier les hommes, écrivait ce dernier, il faut
regarder près de soi ; mais pour étudier l’homme, il faut apprendre à porter la
vue au loin; il faut d’abord observer les différences pour découvrir les
propriétés » ({\it Essai sur l'origine des langues}, VIII, cité dans {\it La pensée sauvage}, IX).

%— 220 —
%{\footnotesize XIX$^\text{e}$} siècle — {\it }
L’un des apports décisifs de l’ethnologie est d’opérer un décentrement, qui
met l’ethnocentrisme à distance (quitte à en faire un objet d’étude). «Il faut
beaucoup d’égocentrisme et de naïveté, souligne Lévi-Strauss, pour croire que
l’homme est tout entier réfugié dans un seul des modes historiques ou géographiques
de son être, alors que la vérité de l’homme réside dans le système de
leurs différences et de leurs communes propriétés » ({\it La pensée sauvage}, XX ; voir
aussi {\it Anthropologie structurale deux}, XVIII, 3). L’ethnologie tend à l’universel,
comme toute science, mais par l'étude du particulier.

%ÉTHOLOGIE
\section{Éthologie}
Étude objective des mœurs ou des comportements, chez les
hommes comme chez les bêtes, sans aucune visée normative.
Ce dernier point est ce qui distingue l’éthologie de l'éthique, à peu près comme
l’objectivité de la biologie (pour laquelle la vie est un fait, non une valeur) la
distingue de la médecine (qui suppose la vie et la santé comme normes). Disons
que l’éthologie est une science, ou tend à en être une ; l'éthique serait plutôt un
art : c’est l’art de vivre le mieux qu’on peut.

C’est en quoi l'éthique de Spinoza, contrairement à ce qu’on en a dit, ne se
réduit nullement à une éthologie. Qu'il faille connaître et comprendre avant de
juger, c’est une évidence. Que connaissance et valeur soient irréductibles l’une
à l’autre, c’est un point essentiel du spinozisme (toute vérité est objective : elle
n’a que faire de nos désirs ; toute valeur est subjective : elle n’existe que pour
autant que nous la désirons). Mais dès lors que nous sommes essentiellement
des êtres de désirs, non de purs sujets connaissants, nous ne pouvons ni ne
devons renoncer à juger : ce serait nous prendre pour Dieu, et nous vouer par
là à l'illusion, ou bien renoncer à l'humanité, et nous vouer par là au malheur
ou à la barbarie. Cela n'empêche pas qu’il y ait dans l’{\it Éthique} un moment éthologique,
ou un point de vue éthologique, qui est fortement marqué : « Je considérerai
les actions et les appétits humains, écrit Spinoza au début du livre III,
comme s’il était question de lignes, de surfaces et de solides. » Mais les livres IV
et V montrent clairement que cela ne suffit pas : que nous avons besoin aussi
de « former une idée de l’homme qui soit comme un modèle de la nature
humaine placé devant nos yeux», en référence auquel nous jugerons les
hommes « plus ou moins parfaits », et les actions plus ou moins bonnes ou
mauvaises (IV, Préface). L’éthologie est nécessaire, non suffisante. Le but n’est
pas seulement de connaître les hommes, mais d’en devenir un point trop
imparfait, autrement dit de nous rapprocher le plus que nous pouvons de « la
liberté de l’âme ou béatitude » (V, Préface). À quoi l’éthologie peut et doit
contribuer, mais au service d’une visée normative (« un bien véritable », dit Spinoza)
qu’elle peut connaître, comme fait, mais qu’elle ne saurait à elle seule justifier
%— 221 —
%{\footnotesize XIX$^\text{e}$} siècle — {\it }
comme valeur. La sagesse n’est ni un absolu ni une science : elle ne vaut
que pour qui la désire ou s’efforce vers elle (III, 9, scolie). Ne compte pas sur la
vérité pour être sage à ta place.

%ÉTIOLOGIE
\section{Étiologie}
L'étude des causes. Se dit surtout en médecine, par différence
avec la sémiologie ou symptomatologie (l'étude des symptômes).

%ÉTONNEMENT
\section{Étonnement}
Au sens fort et classique : une surprise qui foudroie ou
frappe de stupeur. Au sens moderne : toute surprise qui
ne s'explique pas seulement par la soudaineté, mais bien par l'aspect étrange ou
mystérieux du phénomène considéré, C’est en ce sens que l’étonnement est
essentiel à la philosophie, qui s'étonne moins de ce qui est nouveau ou inattendu
que de ce qui résiste à l’évidence ou à la familiarité. Le philosophe
s'étonne de ce qui n’étonne pas, ou plus, la plupart de ses contemporains.
« C’est l’étonnement qui poussa, comme aujourd’hui, les premiers penseurs
aux spéculations philosophiques », remarquait déjà Aristote ({\it Métaphysique}, A,
2), et c’est ce que Jeanne Hersch, revenant sur vingt-cinq siècles de philosophie,
a brillamment confirmé ({\it L'étonnement philosophique, Une histoire de la
philosophie}, Gallimard, Folio-Essais, rééd. 1993). Par exemple l’existence du
monde est étonnante : non qu’elle soit soudaine ou imprévue, mais en ceci
qu'elle plonge l'esprit, pour peu qu’il s'interroge, dans une perplexité qui peut
aller, en effet, jusqu’à la stupeur. Pourquoi y a-t-il quelque chose plutôt que
rien ? Il en va de même de notre propre existence dans le monde. « Quand je
considère, écrit Pascal, la petite durée de ma vie, absorbée dans l'éternité précédente
et suivante, le petit espace que je remplis et même que je vois, abimé dans
l'infinie immensité des espaces que j'ignore et qui m’ignorent, je m’effraye et
m'étonne de me voir ici plutôt que là, car il n’y a point de raison pourquoi ici
plutôt que là, pourquoi à présent plutôt que lors. Qui m’y a mis ? Par l’ordre et la
conduite de qui ce lieu et ce temps ont-ils été destinés à moi ? » De cet étonnement,
on ne sort que par l'explication rationnelle, lorsqu’elle est possible, ou par
l'habitude. C’est pourquoi la philosophie n’en sort guère, et y ramène.

%ÊTRE
\section{Être}
«On ne peut pas entreprendre de définir l’être, observait Pascal, sans
tomber dans cette absurdité [d’expliquer un mot par ce mot même] :
car on ne peut définir un mot sans commencer par celui-ci, {\it c'est}, soit qu’on
l’exprime ou qu’on le sous-entende. Donc pour définir l’être, il faudrait dire
%— 222 —
%{\footnotesize XIX$^\text{e}$} siècle — {\it }
{\it c'est}, et ainsi employer le mot défini dans la définition » ({\it De l'esprit géométrique},
I). Ce que le {\it Vocabulaire} de Lalande, sans citer Pascal, confirmera : {\it être} est « un
terme simple, impossible à définir ». Non que nous ne sachions ce que signifie
le mot, mais en ceci plutôt que nous ne pouvons le définir sans présupposer ce
savoir, même vague, que nous en avons. Si « l’être se dit en plusieurs sens »,
comme le remarquait Aristote (chez lequel chacun de ces sens débouche sur
une {\it catégorie} : l'être se dit comme substance, comme quantité, comme qualité,
comme relation...), cela ne nous dit pas encore ce que c’est qu'être, ni ce que
ces différents sens peuvent avoir en commun.

Pour essayer d’y voir plus clair, on remarquera d’abord qu’{\it être}, en français,
est à la fois un verbe et un substantif, et que c’est le verbe qui est premier (le
substantif, une fois le verbe supposé défini, poserait moins de problème : l'{\it être},
mais on pourrait dire aussi bien {\it étant}, c’est {\it ce qui est}). S'agissant du verbe, on
distingue traditionnellement deux usages principaux : un usage absolu (« cette
table {\it est} »), et un usage relatif, logique ou copulatif (qui lie un sujet et un
prédicat : « cette table {\it est} rectangulaire »). « Le verbe {\it être} est employé en deux
sens, observe par exemple saint Thomas : d’une part il désigne l’acte d’exister,
d’autre part il marque la structure d’une proposition que l'esprit forme en joignant
un prédicat à un sujet. » Ces deux sens sont-ils vraiment différents ? Ne
pourrait-on pas, par exemple, donner à la proposition {\it « Cette table est»}, la
forme copulative {\it « Cette table est un être »} ? Sans doute, mais cela ne nous
apprendrait rien de plus sur la table. Au premier sens, remarque Kant, « {\it Être}
n'est évidemment pas un prédicat réel, c’est-à-dire un concept de quelque
chose qui puisse s’ajouter au concept d’une chose. C’est simplement la position
d’une chose ou de certaines déterminations en soi ». Au second sens, autrement
dit « dans l’usage logique, ce n’est que la copule d’un jugement » ({\it Critique de
la raison pure}, « L'idéal de la raison pure », 4). Métaphysiquement, c’est bien
sûr le premier sens surtout qui fait problème. Pourquoi l'être n’est-il pas un
prédicat réel ? Parce qu’il n’ajoute rien au sujet supposé. Par exemple, explique
Kant, que Dieu soit ou ne soit pas, le concept de Dieu n’en est pas pour cela
changé : c’est pourquoi on ne peut pas passer du concept à l’existence, ni donc
démontrer (comme le voudrait la preuve ontologique) l'existence de Dieu à
partir de sa simple définition.

On remarquera que dans cet usage absolu, et sauf distinction particulière à
tel ou tel philosophe, {\it être} signifie à peu près {\it exister} : c’est le contraire de n’être
pas, comme l'être est le contraire du néant. C’est où l’on retrouve Parménide.
« L’être est » : il y a de l’être, et non pas rien. Voilà ce que toute expérience et
toute pensée nous apprennent ou supposent. Être, c’est faire partie de cet {\it il y
a} : c'est être présent dans l’espace et le temps (ce que j'appelle {\it exister}), c’est persévérer
dans la présence (ce que j’appelle {\it insister}), ou simplement être présent
%— 223 —
%{\footnotesize XIX$^\text{e}$} siècle — {\it }
(ce que j'appelle {\it être}, proprement). Que cela ne vaille pas comme définition,
c'est une évidence — puisque toutes ces expressions supposent l'être —, mais qui
nous renvoie à nouveau à Pascal. On ne peut définir que ce qui est (les étants),
point l'être même, que tout discours suppose. Spinoza, qui dans les {\it Pensées
métaphysiques} risquait pourtant une définition (il entend par {\it être} « tout ce que,
quand nous en avons une perception claire et distincte, nous trouvons qui
existe nécessairement où au moins peut exister»), se gardera bien, dans
l’{\it Éthique}, de la reprendre ou d’en proposer une autre. Bel exemple, qu’on peut
suivre. L’être n’est pas d’abord un concept, qu’on pourrait définir ; il est une
expérience, une présence, un acte, que toute définition suppose et qu'aucune
ne saurait contenir. Par quoi l'être est silence, et condition du discours.

%ÊTRE-LÀ
\section{Être-là}
Voir {\it Dasein}.

%EUDÉMONISME
\section{Eudémonisme}
Toute éthique qui fait du bonheur ({\it eudaimonia}) le souverain
bien. C’est le cas, depuis Socrate, de la quasi-totalité
des écoles antiques, qui s’accordaient à penser que tout homme veut être
heureux et que tel est le but aussi de la philosophie. Cela n’empêchait pas les
philosophes de s'opposer résolument les uns aux autres — non sur ce but, qui
leur est commun, mais sur son contenu ou ses conditions. Qu'est-ce qui fait le
bonheur ? Le savoir (Socrate), la justice (Platon, dans la {\it République}), un mixte
de plaisir et de sagesse (Platon, dans le {\it Philèbe}), la raison ou la contemplation
(Aristote), l'indifférence (Pyrrhon), le plaisir (Épicure), la vertu (les stoïciens) ?
Ces différents eudémonismes s'opposent davantage qu’ils ne se complètent. Ils
cherchent la même chose — le bonheur —, mais ce n’est pas le même bonheur
qu'ils trouvent. L’eudémonisme est un lieu commun de la sagesse grecque.
Mais ce lieu est une arène, où les philosophes s’affrontaient. Les Modernes préfèrent
parler d’autre chose. Non qu’ils aient forcément renoncé au bonheur.
Mais parce qu’ils ont renoncé au souverain bien (voir ce mot).

%EUGÉNISME
\section{Eugénisme}
C’est vouloir améliorer l'espèce humaine, non par l’éducation
des individus mais par la sélection ou la manipulation
des gènes — en transformant le patrimoine héréditaire de l’humanité plutôt
qu'en développant son patrimoine culturel. L'idée, aujourd’hui disqualifiée par
l'usage qu’en firent les nazis, pouvait paraître belle. Agir sur les gènes ? On le
fait bien pour différentes espèces animales, ou pour tel ou tel être humain (les
thérapies géniques). Pourquoi ne pas améliorer l'humanité elle-même ? La
%— 224 —
%{\footnotesize XIX$^\text{e}$} siècle — {\it }
réponse, très difficile à argumenter dans le détail, me paraît tenir pour l’essentiel
en une phrase, qui n’a rien à voir avec la biologie : {\it Parce que tous les êtres
humains sont égaux en droits et en dignité}. Cela, qui vaut spécialement pour le
droit de vivre et de faire des enfants, rend toute idée d’un {\it tri}, au sein de
l'humanité, inacceptable : parce qu’elle est attentatoire à l’égale dignité de tous.
On a le droit de faire ou pas des enfants, mais pas celui de choisir les enfants
que l’on fait. On objectera qu’un tel choix existe pourtant, dans les avortements
thérapeutiques. Sans doute. Mais pour combattre une souffrance,
point pour fabriquer un surhomme. Pour épargner un individu, point pour
améliorer l'espèce. Par compassion, point par eugénisme. Cela indique à peu
près la voie, qui requiert d’autant plus de vigilance qu’elle est étroite et tortueuse.

%EUROPE
\section{Europe}
L'Europe n’est pas vraiment un continent : ce n’est qu’un cap de
l'Asie. Ce n’est pas un État: ce n’est qu’une communauté, et
encore, d’États indépendants. Combien de guerres entre eux dans le passé !
Combien, encore aujourd’hui, de conflits d’intérêts ou de sensibilités ! Ni la
géographie ni l’histoire ne suffisent à faire de l’Europe autre chose qu’une abstraction
ou qu’un idéal. Il faut donc qu’elle soit un idéal ou qu’elle ne soit rien,
en tout cas rien qui vaille, rien qui mérite d’être défendu. L'Europe n'existe
pas ; elle est à faire. Autant dire qu’elle n’existe que par les défis qu’elle affronte,
dont le premier sans doute est celui de sa propre existence. L'Europe ne vaut
qu’autant que nous le voulons, qu’autant que nous {\it la} voulons. Ce n’est ni un
continent ni un État: c’est un effort, c’est un combat, c’est une exigence.
L'Europe est devant nous, au moins autant que derrière. Mais elle ne vaut — et
elle ne vaudra — que par fidélité à ce qu’elle fut. Fidélité critique, cela va sans
dire, et d’ailleurs la critique (y compris réflexive) fait partie de son passé. Fidélité
à Socrate, à Montaigne, à Hume, à Kant — et à nous-mêmes. L'Europe est
notre origine et notre but, notre lieu et notre destin : l’Europe est notre tâche.
La vraie question reste celle de Rousseau : Qu'est-ce qui fait qu’un peuple
est un peuple ? Ou pour l’Europe en construction : qu'est-ce qui fait que plusieurs
peuples, tout en restant différents, peuvent tendre, et dans quelles
limites, à n’en faire qu’un ? Cela suppose des institutions, et qu’on choisisse
entre les deux modèles, fédéral ou confédéral, qui s’offrent à nous. Rassemblement
de Républiques (confédération), ou République rassemblée (fédération) ?
Souveraineté nationale, pour chaque pays, ou supranationale, pour l’ensemble ?
Aucune de ces deux voies n’est indigne, et aucune n’est facile. Mais refuser de
choisir entre l’une et l’autre serait une façon sûre de les fermer toutes deux.

%— 225 —
%{\footnotesize XIX$^\text{e}$} siècle — {\it }
Toutes les institutions resteront vaines, pourtant, si l’Europe ne sait
affronter le principal défi qui s'offre à elle, qui est celui de son esprit ou, cela
revient au même, de sa civilisation. L'Europe n’est pas une race ; c’est un espace
économique, politique et culturel. Mais il faut ajouter : culturel d’abord et surtout.
L'économie n’est qu’un moyen. La politique n’est qu’un moyen. Au service
de quoi ? De certaines valeurs, de certaines traditions, de certains idéaux —
au service d’une civilisation. Celle-ci est un fait de l’histoire. L'Europe, c’est
d’abord lempire romain : le mariage obligé d’Athènes et de Jérusalem, sur
l'autel de leur vainqueur et le civilisant peu à peu... C’est d’où nous sommes
issus, que nous ne pourrons continuer qu'à la condition de ne pas le trahir.
C’est ce que Rémy Brague appelle {\it la voie romaine} : être européen, c’est n’exister
que par cette tension en soi «entre un classicisme à assimiler et une barbarie
intérieure à dominer ». L'Europe, la vérité de l’Europe, c’est la Renaissance, ou
plutôt c’est « cette série ininterrompue de “Renaissances” qui constitue l’histoire
de la culture européenne », comme dit encore Rémy Brague ({\it Europe, la
voie romaine}, p. 165), ou, mieux encore, c’est cette hésitation toujours, cette
oscillation toujours, cette tension toujours entre la Renaissance et la décadence,
entre les Lumières et l’obscurantisme, entre la fidélité et la barbarie. Fidélité
critique, là encore : être européen, en ce sens, c’est être fidèle à la meilleure part
de l’Europe, telle qu’elle se donne dans les sommets indépassés de son histoire.
« {\it Notre patrie sacrée, l'Europe}... », disait Stefan Zweig. Mais à la condition seulement
de choisir ce qui mérite, dans cette patrie, d’être défendu.

On dira que la civilisation européenne est devenue mondiale, en tout cas
occidentale, et qu’elle ne se distingue plus guère, ou de moins en moins, de sa
filleule américaine. Sans doute, et c’est un danger encore qui la menace que
cette dissolution dans ce qu’elle croit son triomphe, qui pourrait être sa défaite
ultime. Le développement sans précédent des moyens de communication et des
échanges ne peut qu’entraîner, à l’échelle de la planète, une réduction des différences.
Sommes-nous pour autant condamnés à l’uniformité ? À l’expansion
irrésistible d’une sous-culture {\it « made in USA »}, avec son esthétique de {\it fast-food}
et de {\it sit-com} ? Le {\it show-biz} est-il l'avenir de l’homme ? L’américanisation, celui
de l'Europe ? Ce n’est pas sûr, mais ce n’est pas impossible. C’est ce qui donne
aux Européens des raisons de s'inquiéter, et de se battre. Contre quoi ? Contre
la barbarie qu’ils portent en eux, ou qu’ils importent, qui risque de les
emporter. Pour quoi ? Pour une Renaissance de l’Europe, et c’est l’Europe
même.

%EUTHANASIE
\section{Euthanasie}
Étymologiquement : une bonne mort. En pratique, le mot
ne sert guère que pour désigner une mort délibérément
%— 226 —
%{\footnotesize XIX$^\text{e}$} siècle — {\it }
acceptée ou provoquée, avec l’aide de la médecine, pour abréger les souffrances
d’un malade incurable : c’est une mort médicalement assistée. Le mot, qui fut
lui aussi compromis dans l’abjection nazie, s’en sort pourtant mieux que celui
d’eugénisme. C’est sans doute que l'euthanasie, à condition qu’elle soit strictement
contrôlée, pose moins de problèmes : d’abord parce qu’elle ne concerne
que des individus, point l'espèce elle-même ; ensuite, et surtout, parce qu’elle
ne vaut que pour des malades incurables, qui l’ont expressément demandée
(euthanasie volontaire) ou dont les proches, si les malades ne peuvent
s'exprimer, l’ont demandé à leur place (euthanasie non volontaire). J'y vois un
progrès plus qu’un danger. Quand la médecine ne peut nous guérir, pourquoi
ne nous aiderait-elle pas à mourir ? Le danger n’en existe pas moins, qui est
celui d’une élimination systématique des malades les plus lourds. Raison de
plus pour qu’ une loi, comme il convient dans un État de droit, vienne fixer des
limites et imposer des contrôles.

%ÉVANGILE
\section{Évangile}
Du grec {\it euangelos}, le bon messager, celui qui apporte une
bonne nouvelle. Avec une majuscule, et souvent au pluriel, c’est
le nom donné aux quatre livres qui retracent la vie et l’enseignement de Jésus-Christ.
Voltaire rappelle qu’ils ont été « fabriqués environ un siècle après Jésus-Christ »,
et qu’il en existe plusieurs autres, dits apocryphes, qui mériteraient
autant d'intérêt. Cela confirme la singularité de cette histoire. Quand bien
même ce ne serait qu’un roman, ce que je ne crois pas, et quoiqu'il soit parfois
ennuyeux, ce {\it roman}-là resterait, de tous les livres de l'humanité, lun des plus
éclairants. Pour ce qu’il nous dit sur Dieu ? Guère. Mais pour ce qu’il nous dit
sur nous-mêmes. Pour la résurrection de son personnage principal ? Non plus.
Mais pour sa vie.

Si on laisse de côté l’invraisemblable exploitation théologique qui en sera
faite, les Évangiles sont le récit d’une existence, et le portrait, même approximatif,
d’un individu. On aurait bien tort de les abandonner aux Églises. Jésus,
pour moi, n’est pas un prophète — je ne crois pas aux prophètes —, encore
moins le Messie ou Dieu. C'était un homme, et d’ailleurs il n’a jamais prétendu
être autre chose. C’est pourquoi il m'intéresse. C’est pourquoi il me touche.
Par la simplicité. Par la fragilité. Par l'humanité nue. Qui peut imaginer, lisant
les Évangiles, que cet homme ait pu se prendre pour Dieu ? Pour son fils ?
Nous le sommes tous, puisque ce Dieu-là, selon la prière même que Jésus nous
laissa, serait Notre Père. Bref, Jésus, tel que je le vois, tel que je crois le comprendre
en lisant les Évangiles, n’a jamais été chrétien. Pourquoi le serions-nous ?
C'était un Juif pieux. C’était un homme plein de sagesse et d’amour. La
seule façon de lui être vraiment fidèle, pour ceux qui ne sont ni juifs ni
%— 227 —
%{\footnotesize XIX$^\text{e}$} siècle — {\it }
croyants, c’est d’être un peu plus sages, un peu plus aimants, un peu plus
humains, et pour cela d’abord de respecter la justice et la charité, qui sont toute
la loi. C’est ce que Spinoza appelait « l'esprit du Christ », qui est l'esprit tout
court et le principal message des Évangiles.

%ÉVÉNEMENT
\section{Événement}
Ce qui advient, plutôt que ce qui est ou dure : « non ce qui
subsiste, mais ce qui survient» (Francis Wolff, {\it Dire Le
monde}, 1). L'événement s'oppose en cela à la substance, à l'être, à la chose — à
tout ce qui demeure. Au monde ? Seulement si l’on suppose un monde qui
serait fait de choses, d’essences ou de substances. Mais il se pourrait qu’il soit
fait plutôt d’événements : que le monde soit l’ensemble de tout ce qui arrive,
comme disait Wittgenstein, plutôt que de tout ce qui est (la totalité des événements,
non des choses), ou du moins que cette distinction n’ait de sens que
pour nous, qui pensons et vivons dans le temps, non pour le réel, qui n’existe
qu’au présent. Tout événement occupe une certaine durée, fût-elle infiniment
brève : rien ne se passe que dans un présent qui passe. Toute durée est faite
d'événements, qu’ils soient très lents ou très rapides : l'expansion de l’univers,
la dérive des continents, un enfant qui grandit ou qui tombe, un oiseau qui
s'envole. Advenir et durer au présent sont un : ainsi l'être et l'événement.
(On notera que la notion d’événement, en philosophie et contrairement à
l'usage historique ou journalistique du mot, est le plus souvent dépourvue de
toute visée normative. Là où le langage courant ne parle d'événement que pour
un fait d’une certaine importance — un train qui arrive à l'heure n’est pas un
événement, un train qui déraille en est un —, les philosophes prennent ordinairement
le mot dans son sens neutre et son extension maximale : tout ce qui
arrive ou a lieu est un événement, quand bien même cela n’aurait d’importance
pour personne. Le seul {\it non-événement}, pour le philosophe, c’est celui qui
n’advient pas.)

%ÉVIDENCE
\section{Évidence}
Ce qui s’impose à la pensée, ce qui ne peut être contesté ou nié,
ce dont la vérité paraît immédiatement et ne peut être mise en
doute. Il n’y aurait pas autrement de certitude, et c’est pourquoi il n’y en a
jamais d’absolue. Par exemple le {\it cogito}, le postulat d’Euclide ou l’immobilité de
la Terre ont longtemps été tenus pour des évidences, ce qu’ils ne sont plus.
C’est dire que l’évidence dépend de l’état des connaissances. Comment pourrait-elle
les fonder ou les garantir ?
Si l’on se fie à l’étymologie ({\it evidens}, en latin, vient de {\it videre}), le modèle de
l'évidence est visuel : {\it « Je l'ai vu, te dis-je, vu de mes yeux vu, ce qui s'appelle vu »},
%— 228 —
%{\footnotesize XIX$^\text{e}$} siècle — {\it }
tel est, dans les mots de Molière, le type même de l’évidence. Pour la vie courante,
c’est un critère très fiable, et d’autant plus qu’il est attesté par un plus
grand nombre d'individus : si plusieurs témoins vous ont vu assassiner
quelqu'un, vous aurez quelque peine à faire croire que vous n'êtes pour rien
dans sa mort. Sous réserve toutefois de la vraisemblance, de la confrontation
et de la critique des témoignages. Les milliers d'individus qui ont vu fort distinctement
la sainte Vierge, que ce soit ensemble ou séparément, ne convainquent
que ceux, sauf exception, qui y croyaient déjà. Quoi de plus évident, tant
qu’on est dedans, qu’un rêve ou qu’un délire ?

« Les dieux existent, disait Épicure : la connaissance que nous en avons est
évidente » ({\it Lettre à Ménécée}, 123). Je ne connais pas de phrase qui m’ait davantage
poussé vers l’athéisme.

%ÉVOLUTION
\section{Évolution}
La transformation, souvent lente et en tout cas progressive,
d’un être ou d’un système : s’oppose à {\it permanence} (l'absence
de changement) et à {\it révolution} (un changement brusque et global).

Le vocable doit beaucoup de son succès, à partir du {\footnotesize XIX$^\text{e}$} siècle, aux différentes
{\it théories de l'évolution} (spécialement celle de Darwin, même si ce dernier
n’utilisa le mot qu'avec réticence) visant à expliquer l’origine et le développement
des espèces vivantes. Cet exemple privilégié montre qu’une évolution
peut se faire de façon discontinue et hasardeuse (les mutations) ; elle suppose
toutefois la continuité, au moins relative et fût-elle reconstruite après coup,
d’un processus. « Nul n’appellera stades évolutifs, remarque le {\it Lalande}, les
transformations qu’on observe dans un kaléidoscope. » Non, pourtant, que
chacun de ces mouvements soit irrationnel ou sans cause ; mais parce que leur
série paraît sans logique, sans continuité, sans orientation. C’est dire que les
mutations, à elles seules, ne suffraient pas à parler d'évolution des espèces : il y
faut encore la sélection naturelle et l’apparente finalité qu’elle entraîne. Par
quoi l’{\it évolution}, qui avance vers des stades de plus en plus complexes ou différenciés,
s'oppose à l’{\it involution}, qui régresse vers le plus simple, le plus homogène,
ou le plus pauvre. La croissance, pour l'individu, est une évolution ; le
vieillissement, une involution.

%EXACTITUDE
\section{Exactitude}
Une vérité modeste, qui tiendrait tout entière dans la précision
des mesures, des descriptions, des constats, sans prétendre
pour cela atteindre l’être ou l'absolu : c’est une adéquation de surface, la
seule peut-être qui permette d’avancer en profondeur.

%— 229 —
%{\footnotesize XIX$^\text{e}$} siècle — {\it }
L’exactitude dépend bien sûr de l’échelle considérée. Une erreur d’un
micron, en biologie ou en physique des particules, peut être plus inexacte
qu’une erreur de plusieurs kilomètres, en astronomie. Il n’est d’exactitude que
relative : c’est une erreur minimale.

%EXCEPTION
\section{Exception}
Un cas singulier, qui semble violer une loi et par là la suppose.
On dit que l’exception confirme la règle ; le vrai est
qu’elle l’enfreint sans l’abolir. Par exemple quand le Comité national consultatif
d'éthique suggère une {\it « exception d'euthanasie »} : c’est reconnaître que la
règle, pour les médecins comme pour tous, reste le respect de la vie humaine ;
mais que ce respect peut justifier parfois qu’on l’interrompe, quand elle ne
pourrait continuer que dans l'horreur. Respecter la vie humaine, c’est aussi lui
permettre de rester humaine jusqu’au bout.

%EXEMPLE
\section{Exemple}
Un cas particulier, qui sert à illustrer une loi ou une vérité générale.
Un exemple ne prouve jamais rien (alors qu’un contre-exemple
peut être une réfutation suffisante), mais il aide à comprendre, et à
faire comprendre. C’est comme une expérience de pensée, à visée surtout pédagogique
ou persuasive. En philosophie on ne peut guère s’en passer, ni s’en
contenter.

%EXERCICE
\section{Exercice}
Une action, le plus souvent répétitive, qui ne se justifie que par
d’autres, qu’elle prépare ou facilite. C’est s’habituer au difficile,
pour qu’il le soit moins.

Les Anciens parlaient d’exercices ({\it askèsis}) de sagesse : parce qu’il est difficile
d’être simple ou libre, et qu’on n’y parvient qu’à la condition de s’y entraîner.
Diogène, en hiver, étreignant une statue gelée : c’est s’exercer à vouloir, pour
apprendre à agir.

Il reste qu'aucun exercice ne vaut par soi. C’est ce que les ascètes
oublient, parfois, et que les sages leur rappellent. Vas-tu passer ta vie à faire
des gammes ?

%EXHIBITIONNISME
\section{Exhibitionnisme}
C'est jouir du spectacle que l’on donne ou que l’on
est, et d'autant plus qu’il est plus intime ou plus obscène.
L’exhibitionnisme attente à la pudeur, ou (entre amants) s’en libère.

%— 230 —
%{\footnotesize XIX$^\text{e}$} siècle — {\it }
%EXIGENCE
\section{Exigence}
Un désir confiant et résolu, qui ne se résigne pas au médiocre
ou au pire. C’est le contraire de la veulerie (s'agissant de soi) ou
de la complaisance (s’agissant d’autrui).

%EXISTENCE
\section{Existence}
Souvent synonyme d’être. L’étymologie suggère pourtant une
différence. Exister, c’est naître ou se trouver {\it (sistere) dehors
(ex)}, autrement dit — il n’y a pas de {\it dehors} absolu — dans autre chose : c’est être
dans le monde, dans l’univers, dans l’espace et le temps. Par exemple, on hésitera
à dire que les êtres mathématiques {\it existent}. Et quand bien même Dieu
{\it serait}, il ne pourrait pas pour autant, remarquait Lagneau, être dit {\it exister} en ce
sens. « Exister, c’est dépendre, écrira Alain, c’est être battu du flot extérieur. »
Si Dieu existait, il ne serait pas Dieu, puisqu'il serait une partie de l’univers,
puisqu'il serait dans un dehors, puisqu'il en dépendrait, et c’est pourquoi il
n'existe pas. « L'existence toujours suppose l'existence, hors d’elle toujours et
autre ; et voilà l'existence. » Son essence est de n’en pas avoir : « La nature,
même intérieure, de toute chose est hors delle. La relation est la loi de
l'existence » ({\it Entretiens au bord de la mer}, VI). Alain, précurseur de l’existentialisme ?
C’est ce que Jean Hyppolite s’amusait à suggérer, qui n’est pas
tout à fait faux ({\it Figures de la pensée philosophique}, XX et X). Mais à ceci près que
l'existence, pour Alain, ne saurait se dire de l’homme seul. L'existence est la loi
du monde, ou le monde même comme loi. L'homme ne s’en distingue que par
la conscience qu’il en prend, qui le sépare de ce {\it dehors} qui le fait exister, et de
lui-même. C’est en quoi il {\it ex-siste}, au sens cette fois heideggérien ou existentialiste
du terme, toujours hors de soi, toujours en avant de soi et de tout, toujours
jeté (dans le monde) et se projetant (dans l'avenir), toujours autre qu’il
n’est, toujours libre, toujours voué au souci ou à l'angoisse, toujours tourné
vers la mort ou le néant. On voit que ces deux sens, pour différents qu’ils
demeurent, peuvent être pris ensemble. Exister, c’est être dehors: c'est
dépendre ou se séparer. S’oppose par là à un Être absolu, qui ne serait qu’indépendance
et intériorité. Exister, c’est n’être pas Dieu : c’est être au monde, toujours
pris dans un dehors (toujours dedans, donc, mais {\it un dedans qui n'est pas
soi}), toujours dépendant, toujours luttant ou résistant. « Ce qui n'existe pas
c’est l’inhérence, écrit Alain, c’est l'indépendance, c’est le changement [seulement]
interne, c’est le dieu ; et ce n’est rien. »

%EXISTENTIALISME
\section{Existencialisme}
Toute philosophie qui part de l’existence individuelle
plutôt que de l’être ou du concept (c’est en ce sens que
Pascal et Kierkegaard sont souvent considérés comme les précurseurs de l’existentialisme),
%— 231 —
%{\footnotesize XIX$^\text{e}$} siècle — {\it }
et spécialement, selon une formule fameuse de Jean-Paul Sartre,
toute doctrine pour laquelle {\it « l'existence précède l'essence »}. Qu'est-ce à dire ?
Que l’homme n’a pas d’abord une essence, qui lui préexisterait et dont il resterait
prisonnier, mais qu'il existe « avant de pouvoir être défini par aucun
concept » et ne {\it sera} (quand on pourra parler de son essence au passé) que ce
qu’il aura {\it choisi} d’être. C’est dire qu’il est libre absolument : « Qu'est-ce que
signifie ici que l’existence précède l’essence ? Cela signifie que l’homme existe
d’abord, se rencontre, surgit dans le monde, et qu’il se définit après. L'homme,
tel que le conçoit l’existentialisme, s’il n’est pas définissable, c’est qu’il n’est
d’abord rien. Il ne sera qu’ensuite, et il sera tel qu’il se sera fait. Ainsi il n’y a
pas de nature humaine, puisqu'il n’y a pas de Dieu pour la concevoir. [...]
L'homme n’est rien d’autre que ce qu’il se fait » {\it (L'existentialisme est un humanisme)}.
Par quoi l’existentialisme est une philosophie de la liberté, au sens
métaphysique du terme, et l’une des plus radicales qui fut jamais.

Reste à savoir si on peut la faire sienne. Comment {\it exister}, comment faire ou
choisir quoi que ce soit, comment s’inventer ou se projeter, avant d’{\it être} d’abord
quelque chose ou quelqu'un ? Qui dirait d’un nouveau-né qu’il n’est {\it rien} ? Et
comment considérer qu'être ce qu’on est soit seulement une {\it situation}, qu’il
nous appartiendrait de transcender, et point, au moins en partie, une {\it détermination},
dont il est exclu que nous sortions jamais (puisque changer, c’est toujours
{\it se} changer) ? « Chaque personne est un choix absolu de soi », écrit Sartre
dans {\it L'Être et le Néant} ; c'est ce que je n’ai jamais pu croire ni penser. Comment
choisir sans être d’abord ? Ou plutôt quel sens y a-t-il, au présent, à distinguer
ce que je {\it fais} ou {\it veux} de ce que je {\it suis} ? Exister, c’est être en acte et en
situation : l'essence et l’existence, au présent, sont une seule et même chose.
Certes ce n’est pas ainsi que nous le vivons ou l’imaginons ; nous avons le sentiment
d’être ce que le passé a fait de nous, et de choisir ce que nous ferons de
l'avenir. « L’essence, c’est {\it ce qui a été} », écrit Sartre dans {\it L'Être et le Néant};
l'existence, à l’inverse, c’est ce qui n’est pas encore, ce qui se jette vers l’avenir,
ce qui {\it sera}, si je le veux ou le fais. « La liberté s'échappe vers le futur, elle se
définit par la fin qu’elle pro-jette, c’est-à-dire par le futur qu’elle a à être »
({\it L'Étre et le Néant}, p. 577). Mais cette distinction, entre l’essence et l’existence,
n’a dès lors de sens que pour la conscience, qui se donne un passé et un avenir,
point pour le réel lui-même, qui n’existe qu’au présent — et dont la conscience,
qu’elle le veuille ou pas, fait partie. Un souvenir n’existe qu’au présent. Un
projet n'existe qu’au présent. Et comment le présent pourrait-il n’être pas ce
qu’il est ou être autre ? Sartre, en toute cohérence, explique que la liberté n’est
possible que comme néant, point comme être, ce qui m’a toujours paru une
réfutation suffisante : la liberté, en ce sens absolu, n’est possible qu’à la condition
de n’être pas. L’existentialisme n’est qu’un humanisme imaginaire.

%— 232 —
%{\footnotesize XIX$^\text{e}$} siècle — {\it }
Faut-il alors retomber dans un essentialisme qui nous enfermerait à jamais
dans ce que nous sommes, pour lequel l’existence ne serait qu’un {\it effet} de
l'essence ? Nullement. Au présent, l’essence et l'existence ne font qu’un, et ne
sauraient se {\it précéder} mutuellement. Ni existentialisme, donc, ni essentialisme :
l'existence ne précède pas plus l’essence que l’essence ne précède l’existence.
Elles n'existent qu’ensemble, dans un même monde, dans un même présent, et
c’est ce que signifie exister.

%EXOTÉRIQUE
\section{Exotérique}
Qui s'adresse à tous, y compris à ceux qui sont en dehors
{\it (exô)} de l’école ou du groupe. S’oppose à l’enseignement
ésotérique ou acroamatique, qui ne s'adresse qu’aux initiés ou aux spécialistes,
On remarquera que l’école publique, dès lors qu’elle est laïque, gratuite et obligatoire,
tend à relativiser cette opposition : c’est qu’elle forme des élèves, non
des disciples ; des citoyens, non des initiés.

EXPÉRIENCE
\section{Expérience}
Notre voie d’accès au réel: tout ce qui vient en nous du
dehors (expérience externe), et même du dedans (expérience
interne), en tant que cela nous apprend quelque chose. S’oppose à la raison,
mais aussi la suppose et l’inclut. Pour un être tout à fait dépourvu d’intelligence,
aucun fait ne ferait expérience, puisqu'il ne lui apprendrait rien. Et un
raisonnement, pour nous, n’est qu’un fait comme un autre. Ainsi on ne sort
pas de l'expérience ; c’est ce qui donne raison à l’empirisme et qui lui interdit
d’être dogmatique.

%EXPÉRIMENTATION
\section{Expérimentation}
Une expérience active et délibérée : c’est interroger
le réel, au lieu de se contenter de l'entendre (expérience)
et même de l’écouter (observation). Se dit spécialement de l’expérimentation
scientifique, qui vise ordinairement à tester une hypothèse en la soumettant
à des conditions inédites, artificiellement obtenues (le plus souvent en
laboratoire) et reproductibles. Cela suppose qu’on cherche quelque chose, et
même, presque toujours, qu’on sache ce qu’on cherche : il n’y a pas d’expérimentation
sans une hypothèse préalable et une théorie — fût-elle fausse ou provisoire —
de référence. « Pour un esprit scientifique, écrit Bachelard, toute
connaissance est une réponse à une question. S’il n’y a pas eu de question, il ne
peut y avoir connaissance scientifique. Rien ne va de soi. Rien n’est donné.
Tout est construit. » ({\it La formation de l'esprit scientifique}, X). L’expérimentation
est une expérience qui ne va pas de soi — une expérience {\it construite}.

%— 233 —
%{\footnotesize XIX$^\text{e}$} siècle — {\it }
Aucune expérimentation ne suffit jamais à prouver la vérité d’une hypothèse,
encore moins d’une théorie. On cherche une preuve ; on ne trouve
qu'un exemple, ou un contre-exemple : ce dernier seul est probant. Vous
pouvez vérifier dix mille fois que la nature a horreur du vide ou que les corps
les plus lourds tombent plus vite que les autres, vous n’aurez pas prouvé par là
que c’est vrai ; une seule expérimentation, si elle est bien conduite et reproductible,
peut suffire à montrer que c’est faux. Ainsi l’expérimentation n’est décisive
ou cruciale que par les théories qu’elle permet d’éliminer. Les sciences
expérimentales avancent par conjectures et réfutations, montre Popper, non
par induction et vérification : « c’est la falsifiabilité d’un système, et non sa vérifiabilité,
qu’il faut prendre comme critère de démarcation » d’une démarche
expérimentale ({\it La logique de la découverte scientifique}, I). C’est ce qui permet
aux sciences d’avancer, sans jamais les autoriser à s’arrêter.

%EXPLICATION
\section{Explication}
Le fait d'expliquer, c’est-à-dire de donner la cause, le sens
ou la raison. Le principe de raison et le principe de causalité
entraînent que tout fait, quel qu’il soit, a une explication : l’inexplicable
n'existe pas. On remarquera que cette explication n’a en elle-même aucune
visée normative, et ne saurait donc valoir comme approbation ou comme justification.
Qu’une maladie puisse s'expliquer, cela ne la rend ni moins pathologique
ni moins grave, Que le nazisme puisse s’expliquer, cela ne le rend ni
moins ignoble ni moins lourd de conséquences. J'ai lu plusieurs fois que la
Shoah était par nature inexplicable, qu’il fallait la laisser telle, qu’on ne pourrait
entreprendre de l'expliquer, d’ailleurs en pure perte, qu’à la condition de nier
d’abord son irréductible et atroce singularité. C’est donner raison à l’irrationalisme
nazi, et à la nuit contre les Lumières. Pourquoi le racisme serait-il
inexplicable ? Et quoi de plus explicable, quand le racisme atteint ce degré de
fanatisme et de haine, qu’il devienne assassin ? Racisme de masse : crime de
masse. Mieux vaut essayer de le comprendre, pour le combattre. Mais si l’on
comprend, dira-t-on, on ne peut plus juger ! C’est se méprendre. Ce n’est pas
la cancérologie qui nous dit que le cancer est mauvais ; mais elle nous aide à le
combattre. L’explication ne tient jamais lieu de jugement de valeur, ni le jugement,
d'explication.

%EXTASE
\section{Extase}
C’est sortir de soi et de tout, pour se fondre en autre chose (spécialement
en Dieu) — comme un saut dans la transcendance ou
dans l’absolu. S’oppose par là à l’{\it enstase} (voir ce mot).

%— 234 —
%{\footnotesize XIX$^\text{e}$} siècle — {\it }
%EXTENSION
\section{Extension}
L'ensemble des objets désignés par un même signe ou compris
dans un même concept. Définir ce concept {\it en extension},
ce sera dresser la liste, quand c’est possible, de tous les objets auquel il
s'applique. S’oppose à {\it compréhension} (voir ce mot). L'extension du concept
« homme » est l’ensemble de tous les hommes. Les femmes en font-elles partie ?
Cela dépend de la {\it compréhension} du concept.

%EXTRÉMISME
\section{Extrémisme}
Propension à aller jusqu’au bout, dans une direction donnée,
en oubliant ce que les autres directions peuvent avoir
aussi de légitime ou de sensé. Si la droite n’est qu’une erreur, l'extrême gauche
a raison, contre la gauche, comme l'extrême droite contre la droite si la gauche
est le mal. Et à quoi bon autrement être de droite ou de gauche ? Ainsi l’extrémisme
est la tentation des plus convaincus ou des plus haineux : double
danger, double force.

« On ne pense bien qu'aux extrêmes », disait Louis Althusser, et cela sans
doute n’est pas tout à fait faux. Un marxiste ou un ultra-libéral, parlant d’économie,
seront presque toujours plus intéressants, intellectuellement, qu’un centriste
ou un social-démocrate. Mais le réel résiste, qui n’est pas une pensée. « Le
peuple se trompe, observait Montaigne : on va bien plus facilement par les
bouts, où l’extrémité sert de borne d’arrêt et de guide, que par la voie du
milieu, large et ouverte, et selon l’art que selon la nature, mais bien moins
noblement aussi, et moins recommandablement » ({\it Essais}, III, 13). Quel penseur
plus radical pourtant que celui-ci ? Et quel vivant plus modéré ? On ne
pense bien qu'aux extrêmes. On ne vit bien que dans l’entre-deux.

« La sagesse est l’extrême de vivre », ai-je pourtant écrit quelque part. Mais
c'est qu’elle n’est qu’une idée de philosophe. « La sagesse a ses excès, écrit
encore Montaigne, et n’a pas moins besoin de modération que la folie » (III, 5).
Celui-ci était sage véritablement, qui ne crut jamais à la sagesse.
%{\footnotesize XIX$^\text{e}$} siècle — {\it }

