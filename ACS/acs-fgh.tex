%{\footnotesize XIX$^\text{e}$} siècle — {\it }

\chapter{FGH}
\section{Fable}
%FABLE
Une histoire inventée, qu’on ne cherche pas à faire passer pour vraie,
ou dont on ne peut envisager qu’elle le soit : c’est un mythe qui
donne à penser ou à rire, plutôt qu’à croire.

\section{Factice}
%FACTICE
Ce qui est faux ou fabriqué.

\section{Facticité}
%FACTICITÉ
Dans la langue philosophique contemporaine, ici très influencée
par l’allemand, le substantif a rarement le sens que l’adjectif
semble annoncer : la {\it facticité} caractérise non ce qui est faux ou fabriqué, mais ce
qui est un fait, à la fois nécessaire (puisqu'il est là) et contingent (il aurait pu ne
pas y être), comme ils sont tous. Mieux vaudrait, en ce sens, parler de factualité.

\section{Faculté}
%FACULTÉ
Une puissance innée ou {\it a priori} : par exemple la puissance de
sentir (la sensibilité), de penser (l'intelligence, l’entendement), de
désirer, de vouloir, d'imaginer, de se souvenir. La difficulté est de rattacher la pluralité
de ces facultés, qui semble un fait d'expérience, à l’unité de l'esprit ou du cerveau,
sans laquelle il n’y aurait pas d’expérience du tout. La neurobiologie, dans ce
domaine, a sans doute davantage à nous apprendre que la philosophie. La « doctrine
des facultés », comme on disait jadis, a laissé place aux sciences cognitives.

\section{Fait}
%FAIT
Un événement quelconque, dès lors qu’il est constaté ou établi — ce
qui ne peut se faire que par expérience. On parle de « fait scientifique »
%— 236 —
quand il a été l’objet d’une expérimentation, ou à tout le moins d’une observation
rigoureuse, ce qui suppose presque toujours une théorie préalable et une
technologie adaptée : c’est un fait « bien fait », comme dit Bachelard, plutôt
que tout fait.

En philosophie, on oppose traditionnellement la question de fait {\it (quid
facti)} à la question de droit {\it (quid juris)}, comme ce qui est à ce qui doit ou
devrait être. Par exemple, qu’il y ait des riches et des pauvres : c’est là un fait
incontestable ; mais qui ne dit rien sur sa légitimité. « L'égalité des biens serait
juste, écrit Pascal, mais. » Mais quoi ? Mais cela n’est pas en fait, et même le
droit en a décidé autrement. Car le droit lui-même, pris au sens juridique, n’est
qu’un fait comme un autre.

Cela vaut aussi d’un point de vue gnoséologique ou pratique. Que nous
ayons, en fait, des sciences et une morale, cela ne nous dit pas ce qu’elles valent,
ni à quelles conditions. En faire la critique ? C’est toujours légitime, mais cela
ne fera qu’un fait de plus, qui viendra s'ajouter aux autres sans pouvoir les
fonder.

Ainsi il n’y a que des faits, et c’est ce qu’on appelle le monde.

\section{Falsifiabilité}
%FALSIFIABILITÉ
Néologisme proposé par Karl Popper, qui y voit la ligne
de démarcation entre les sciences empiriques, d’un côté,
et de l’autre les énoncés métaphysiques, pseudo-scientifiques, ou encore relevant
de la seule logique formelle. Un énoncé n’est {\it falsifiable} que s’il peut être
contredit, au moins en principe, par l’expérience, autrement dit que si l’on peut
concevoir au moins un fait susceptible, le cas échéant, de le réfuter. Par
exemple les énoncés « Il pleuvra ici demain » ou « Tous les cygnes sont blancs »
sont falsifiables : on peut imaginer un fait qui les réfute (qu’il ne pleuve pas ici
demain, ou qu’on voie un cygne qui ne soit pas blanc). Ce sont des énoncés
empiriques. En revanche, les énoncés «IL pleuvra ou il ne pleuvra pas ici
demain », « Dieu existe » ou « le communisme est l’avenir de l’humanité » ne
sont pas falsifiables : on ne peut concevoir aucun fait qui suffirait à prouver
qu'ils sont faux. Ce ne sont donc pas des énoncés empiriques. On remarquera
que ces trois énoncés ont pourtant un sens (puisqu’on peut les comprendre, les
approuver ou les critiquer), et que le premier est assurément vrai (c'est une
tautologie). La falsifiabilité n’est ni un critère de signification ni un critère de
vérité, mais seulement un critère d’empiricité et donc, s'agissant des sciences
expérimentales, de scientificité possible. « Un système n’est empirique ou scientifique
que s’il est susceptible d’être soumis à des tests expérimentaux », reconnaît
classiquement Popper ; mais aucun test, ajoute-t-il, ne suffit jamais à
prouver la vérité d’une théorie : quand bien même j'aurais vu cent mille cygnes
%— 257 —
blancs ou comparé cent mille fois la vitesse de corps en chute libre, cela ne suffira
pas à prouver que {\it tous} les cygnes sont blancs ni que {\it tous} les corps, dans le
vide, tombent à la même vitesse. « Les théories ne sont donc {\it jamais} vérifiables
empiriquement », conclut Popper ; elles ne peuvent être testées que négativement :
« c’est la falsifiabilité et non la vérifiabilité d’un système qu’il faut
prendre comme critère de démarcation » ({\it La logique de la découverte scientifique},
I, 6 ; voir aussi le chap. IV).

J'ai longtemps regretté que les traducteurs français se soient résignés au
néologisme {\it falsifiabilité} (alors que « falsifier », en français, a un tout autre sens,
et que le mot {\it réfutabilité} était disponible). Mais, outre que les deux mots existent
chez Popper, et qu’il était légitime de les traduire en français par deux
mots différents, on remarquera avec Alain Boyer qu’ils ne sont pas tout à fait
synonymes : une théorie peut être {\it réfutée} par un argument simplement logique
(par exemple en mathématiques) ; elle ne peut être {\it falsifiée} que par un fait
empirique. Ainsi {\it réfutabilité} est le genre prochain ; la {\it falsifiabilité} est une espèce
particulière de réfutabilité, dont l’empiricité est la différence spécifique. Ce qui
autorise une définition simplifiée : est falsifiable tout énoncé possiblement
réfutable par l’expérience.

\section{Famille}
%FAMILLE
Ensemble d'individus, liés par le sang, le mariage ou l'amour.
Où finit la famille ? Cela dépend des époques, des régions, des
contextes. De nos jours, et dans nos pays, on peut distinguer la famille au sens
restreint (le père, la mère, leurs enfants), et la famille au sens large (il faut alors
ajouter les grands-parents, les oncles et tantes, cousins et neveux, sans parler de
la belle-famille ni des familles recomposées.....). Où commence-t-elle ? Cela
peut se discuter, notamment d’un point de vue juridique. Pour ma part, et
d’un point de vue philosophique, je répondrai simplement : la famille commence
à l'enfant. Un ménage sans enfant, ce n’est pas une famille : c’est un
couple. Alors qu’une mère célibataire, qui élève seule son ou ses enfants, c’est
évidemment une famille. On m'objectera qu’un orphelinat, où il y a tant
d’enfants, n’est pas une famille pour autant. Certes ; mais c’est que les enfants
y sont considérés simplement comme enfants, non comme fils ou filles. Cela
fait toute la différence, qui distingue aussi la famille de l’école. La famille, c’est
la filiation acceptée, assumée, {\it cultivée}. Car la famille est un fait de culture,
autant ou davantage qu’un fait biologique. Deux adultes qui adoptent un
enfant, c’est une famille ; un couple qui abandonne le sien, ce n’en est pas une.
La famille est la filiation selon l'esprit, ou le devenir-esprit de la filiation.
La famille, sous des formes bien sûr différentes, semble avoir existé à toutes
les époques, et en tous lieux : « le fait de famille, reconnaît Claude Lévi-Strauss,
%— 238 —
est universel » ({\it Le regard éloigné}, p. 80). Ce n’est pas sans poser un problème.
Si l’on admet avec Lévi-Strauss que l’universel est le critère de la nature, et la
règle particulière le critère de la culture, comment expliquer l'existence {\it universelle}
d’une institution évidemment {\it réglée}, et donc culturelle ? On reconnaît ici
la problématique qui est celle de Lévi-Strauss quand il s'interroge sur la prohibition
de l'inceste, et ce n’est pas un hasard : si la famille, comme la prohibition
de l'inceste, présente les deux caractères en principe opposés de deux ordres
exclusifs (luniversalité de la nature, la particularité réglée de la culture), c’est
que la famille réalise concrètement — non une fois pour toutes mais à chaque
génération, et pour chaque individu de chaque génération — ce que la prohibition
de l’inceste ne fait qu’instituer formellement : le {\it passage} de la nature à la
culture, de humanité biologique à l’humanité culturelle — de la filiation selon
la chair à la filiation selon l'esprit, de l'humanité comme espèce à l’humanité
comme valeur.

On sait que la prohibition de l'inceste, pour les ethnologues, vaut moins
par ce qu’elle interdit que par ce qu’elle impose : l'échange sexuel avec un
membre d’une autre famille, d’où résulte l'alliance entre les familles et donc la
société. Ce que je ne peux trouver chez les miens — la jouissance sexuelle du
corps de l’autre —, il faut que je le cherche à l'extérieur, dans une autre famille,
et c’est ce qui permet, ou impose, d’en fonder une troisième... « Dans tous les
cas, remarque encore Lévi-Strauss, la parole de l’Écriture : {\it “Tu quitteras ton père
et ta mère”}, fournit sa règle d’or (ou, si l’on préfère, sa loi d’airain) à l’état de
société. » La famille n’est pas seulement l’élément premier de la société, comme
le voulait Auguste Comte (plusieurs familles dispersées ne font pas encore une
société), mais bien sa condition : elle représente la nature dans la culture, par la
filiation, et la culture dans la nature, par la prohibition de l'inceste. Elle est le
creuset où animalité et humanité ne cessent de se fondre : elle réalise le passage
de la nature à la culture, en imposant le passage de la famille à la société.

La famille, qui donne tout à l’enfant, finit ainsi par donner son enfant
même. À qui ? À un autre homme, à une autre femme, certes, mais aussi — et
d’abord, et surtout — à lui-même. C’est ce dernier don, le plus beau, le plus difficile
qu’on appelle {\it liberté}. La famille donne et perd : elle donne {\it pour perdre},
même, pour que l'enfant s’en aille, pour qu’il puisse quitter sa famille, et c’est
ce qu’on appelle élever un enfant.

\section{Fanatisme}
%FANATISME
«Le fanatisme, disait Alain, ce redoutable amour de la
vérité. » Mais il n’aime que la sienne. C’est un dogmatisme
haineux ou violent, trop sûr de sa bonne foi pour tolérer celle des autres. Le terrorisme
est au bout.

%— 239 —
On remarquera qu’il n’y a pas de fanatisme dans les domaines où une
preuve est possible (pas de fanatisme en mathématiques, en physique, presque
pas en histoire, quand les faits sont un peu anciens et bien établis), et c’est ce
qui énerve les fanatiques : parce qu’ils ne peuvent ni faire partager leur certitude,
ni accepter qu’elle n’en soit pas une. Le fanatisme touche à la foi, mais
l’exacerbe. À l'enthousiasme, mais le pervertit. Il est à la superstition, disait
Voltaire, « ce que le transport est à la fièvre, ce que la rage est à la colère. Celui
qui a des extases, des visions, qui prend des songes pour des réalités, et ses imaginations
pour des prophéties, est un enthousiaste ; celui qui soutient sa folie
par le meurtre est un fanatique. » C’est se laisser emporter par sa faiblesse, au
point de la prendre pour une force.

\section{Fantaisie}
%FANTAISIE
L’imagination, mais joueuse plutôt que visionnaire, plaisante
plutôt que fascinante, enfin sans trop d'illusions sur elle-même.
C’est l'imagination la plus libre et la plus sympathique : celle qui n’est pas dupe
de ses rêves, ni de soi.

\section{Fantasme}
%FANTASME
Une image ou un scénario suscités par le désir, mais dont on
voit clairement qu’ils sont imaginaires (c’est ce qui distingue le
fantasme de l'illusion), voire, parfois, qu’ils doivent le rester.

\section{Fatalisme}
%FATALISME
Croyance en la fatalité de tout. Cela revient à décourager
l’action : tout fatalisme est paresseux ou devrait l'être.

\section{Fatalité}
%FATALITÉ
Le nom superstitieux du destin : tout serait écrit à l'avance, de
sorte que l'avenir serait aussi impossible à changer que le passé.
Et certes il était vrai, il y a cent mille ans, que tu lirais ces lignes aujourd’hui.
Mais ce n’est pas parce que c'était vrai que tu les lis ; c’est parce que tu les lis
que c'était vrai. La fatalité n’est qu’un contresens sur l’éternité : c’est soumettre
le réel au vrai, quand toute action fait l'inverse.

\section{Fatigue}
%FATIGUE
C’est un affaiblissement, durable ou passager, de la puissance
d’exister et d’agir, suite à un effort trop intense ou trop long. On
dirait une usure ou un épuisement du {\it conatus}, mais qui affecterait le corps ou
le cerveau plutôt que l’âme. C’est ce qui distingue la fatigue de la tristesse, et
%— 240 —
qui explique qu’elles aillent si souvent ensemble. Toute tristesse fatigue, et il
n'est guère de fatigue, si elle dure, qui ne rende un peu triste. L'usage et l’expérience
interdisent pourtant de les confondre absolument : aucun repos ne suffit
à la joie, aucune joie au repos.

Vivre fatigue, et la fatigue ne se dit au sens propre que des êtres vivants. Elle
est l’entropie de vivre. Cela fait comme une lourdeur de tout l’être : les jambes,
la tête, les paupières, la pensée... Le corps n’est plus qu’un poids, et l'esprit
n’est plus rien. C’est le triomphe des imbéciles et des physiciens. Une force obscure
— la vie même — nous pousse vers la mort ou le repos. Vivre fatigue et tue.
Le sommeil est une homéopathie de la mort.

\section{Fausseté}
%FAUSSETÉ
C’est une pensée qui ne correspond pas au réel ou au vrai. Elle
en fait pourtant partie (elle existe réellement : elle est vraiment
fausse), et c’est en quoi sa fausseté reste une détermination extrinsèque ou négative.
« Il n’y a dans les idées rien de positif à cause de quoi elles sont dites
fausses », écrit Spinoza, ce qui veut dire qu'aucune idée n’est fausse en elle-même
ou par ce qu’elle est, mais seulement par ce qu’elle n’est pas ou qui — si
on la compare à une idée vraie — lui fait défaut : « La fausseté consiste dans une
privation de connaissance qu’enveloppent les idées inadéquates, c’est-à-dire
mutilées et confuses » ({\it Éthique}, II, prop. 33 et 35). Ainsi tout est vrai en Dieu
ou en soi, sans que cela nous empêche de mentir ou de nous tromper. C’est que
nous ne sommes pas Dieu. La fausseté est la marque en nous de la finitude : ce
n’est qu’un premier pas vers le vrai. L'erreur est de vouloir s’arrêter.

\section{Fausseté des vertus humaines}
%FAUSSETÉ DES VERTUS HUMAINES
C’est le titre d’un livre de Jacques
Esprit (1611-1678), comparable,
avec moins de talent, aux {\it Maximes} de La Rochefoucauld. Toutes nos vertus ne
seraient que des vices déguisés, que des ruses de l'intérêt ou des mensonges de
l'amour-propre. Voltaire lui consacre un des articles de son {\it Dictionnaire}. Il lui
reproche de ne critiquer la morale que pour faire le lit de la religion, en l’occurrence
catholique, et surtout de mettre Marc Aurèle ou Épictète sur le même
plan que le premier coquin venu. Car enfin si toute vertu est fausse, pourquoi
admirer ces deux-là ou s’interdire de ressembler à celui-ci ? Non, pourtant, que
l'amour-propre n’ait en effet ses pièges, ses leurres, ses illusions. Comment la
vertu serait-elle transparente, puisqu’elle est humaine ? Mais elle n’en continue
pas moins, même opaque, de valoir mieux que son absence. J'imagine, chers
immoralistes, que vous faites une différence entre Cavaillès et ses bourreaux.
Que vous mettez un homme courageux, généreux et droit plus haut qu’un
%— 241 —
salaud égoïste et lâche. Mais alors pourquoi ce mot de {\it vertu} vous agace-t-il si
fort ? Parce que vous doutez qu’elle soit complètement désintéressée ? La belle
affaire ! Qu’un héros puisse trouver du plaisir à en être un, cela en fait-il un
méchant homme ? Parce que vous ne savez pas ce que c’est ? Je vous renvoie à
la définition que j’en donne, qui reprend celles d’Aristote et de Spinoza. Mais
pour aller au plus court, je vous répondrais volontiers ce que Voltaire lançaïit à
Jacques Esprit : « Qu'est-ce que la vertu, mon ami ? C’est de faire du bien :
fais-nous-en, et cela suffit. Alors nous te ferons grâce du motif. »

\section{Faute}
%FAUTE
Une erreur pratique, qui s’écarte moins du vrai que du bien ou du
juste. Par exemple une faute d'orthographe : ce n’est pas qu’elle soit
{\it moins vraie} que l'écriture correcte, mais qu’elle n’est pas {\it la bonne}. Toute faute
suppose une norme de référence, qu’elle reconnaît (dans son principe) et
méconnaît ou transgresse (dans son détail).

Une faute est souvent une erreur, mais pas toujours. Il arrive que je fasse ce
que je crois, à tort, être bien, mais aussi que je fasse ce que je sais pertinemment
être mal. C’est à peu près la différence qu’il y a entre une faute intellectuelle
(une erreur de jugement) et une faute morale. Je suis responsable de la
première ; coupable de la seconde.

\section{Favoritisme}
%FAVORITISME
C'est manquer à la justice par amour ou par solidarité.
Comment est-ce possible ? C’est qu’il s’agit d’un amour
toujours particulier, d’une solidarité toujours partielle, contre la justice universelle.
L'amour n’excuse pas tout ; la solidarité non plus. C’est ce que le mot de
favoritisme, qui vaut universellement comme condamnation, nous rappelle.

\section{Félicité}
%FÉLICITÉ
Ce serait un bonheur absolu : une joie permanente, et qui persévérerait
dans son intensité. Mais la notion en est contradictoire :
ce serait un {\it passage} (Spinoza, {\it Éthique}, III, déf. 2 des affects, et explic. de la
déf. 3) qui ne {\it passerait} pas. Son impossibilité nous distingue des dieux ; son
rêve, des animaux. Elle est au paradis terrestre ce que la béatitude serait à
l'autre. Double mensonge.

\section{Féminité}
%FÉMINITÉ
C'était à la rue d’Ulm, dans les soixante-dix. Je bavardais avec
un ami, dans un couloir de l’École normale supérieure. Soudain,
je vois arriver trois jeunes femmes, bottées, casquées, la cigarette au bec,
%— 241 —
qui me demandent d’un ton rogue : {\it « C'est où, les chiottes ? »} Elles devaient descendre
de moto, et je n’ai évidemment rien contre. Elles avaient bien sûr le
droit de fumer et de dire des gros mots. Mais elles étaient étonnamment masculines,
au pire sens du terme : sans douceur, sans finesse, sans poésie. Cela
suggère, au moins par différence, ce qu'est la féminité. Non une essence ou un
absolu, cela va de soi (les trois motardes, aussi peu féminines qu’elles m’aient
paru, n’en étaient pas moins femmes pour autant), mais un certain nombre de
traits ou de caractères qu’on trouve plus souvent chez les femmes, sans lesquels
l'humanité se réduirait à la masculinité, avec tout ce qu’elle comporte de violence
et de lourdeur, de prosaïsme et d’ambition — ce que Rilke appelait « le
mâle prétentieux et impatient ».

Les deux notions de féminité et de masculinité ne peuvent se définir que
lune par l’autre. C’est ce qui les rend toujours relatives et insatisfaisantes, mais
aussi nécessaires. Freud, à la fin de sa vie, s’interrogeait encore sur ce que veulent
les femmes. Peut-être savait-il mieux, parce qu’il en était un, ce que veulent
les hommes : le pouvoir, le sexe, l’argent, l'efficacité, la gloire. On m’objectera
que les femmes n’y sont pas non plus indifférentes. Je le sais bien. Mais il
me semble qu’elles auront tendance, plus souvent que les hommes, à privilégier
un certain nombre d’enjeux qui relèvent davantage de la vie privée et affective :
la parole, l'amour, les enfants, le bonheur, la durée, la paix, la vie... Il faut
certes se méfier de ces catégories, toujours trop massives, toujours trop vagues,
et qui risquent d’enfermer chaque être humain dans un rôle convenu, qu’il
n'aurait pas choisi. Mais comment — sauf à refuser à la différence sexuelle toute
autre réalité que physiologique — s’en passer tout à fait ? Il m’est arrivé de dire,
par boutade, que l'amour était une invention des femmes, qu’une humanité
exclusivement masculine n’en aurait jamais eu l’idée — que le sexe et la guerre
lui auraient suffi, toujours. Disons, plus sérieusement, qu’hommes et femmes
ont tendance à vivre différemment l'articulation de l’amour et de la sexualité.
La plupart des hommes mettent l'amour au service du sexe, quand les femmes,
du moins la plupart d’entre elles, mettraient plutôt le sexe au service de
l'amour. Ce n’est qu’une tendance, dont il se peut qu’elle soit davantage culturelle
que naturelle. Elle n’en fait pas moins partie, me semble-t-il, de notre
expérience. Cela laisse une chance à la séduction et au couple (puisque nous
désirons les uns et les autres et l’amour et le sexe), mais n’est pas non plus sans
entraîner parfois, dans nos relations, un certain nombre de difficultés ou de
malentendus.

On pourrait faire des remarques du même genre sur la violence et la douceur,
sur le rapport au temps ou à l’action. Quelques semaines de guerre suffisent
à tout détruire : les hommes s’en chargent fort bien. Après quoi il faut des
%— 243 —
années de patience et d’efforts pour que la vie reprenne ses droits : je ne suis pas
sûr que nous en serions capables sans les femmes.

Mais revenons à nos trois motardes. C’était la première fois, je crois bien,
que la féminité, alors peu en vogue, m’apparaissait comme une valeur. J'avais
vingt ans. Je n'avais pas encore lu Rilke (« la femme est sans doute plus mûre,
plus près de l’humain que l’homme... »), ni Colette, ni Simone Weil, ni Etty
Hillesum... Certes, j'avais fait toutes mes études secondaires dans un lycée
mixte : sortant de l’école communale, alors exclusivement masculine, cela
m'avait paru une espèce de paradis, comme quand on quitte un wagon de
bidasses pour entrer dans un lieu civilisé.. Mais je disais « les filles » alors,
plutôt que les femmes ou la féminité, et n’aurais guère osé, surtout, en tirer
quelque idée générale que ce soit. Le féminisme, ces années-là, était une espèce
d’évidence, pour tout intellectuel progressiste. Mais la féminité, non : beaucoup,
chez les hommes comme chez les femmes, n’y voyaient qu’un dernier
piège ou une dernière illusion, dont il était urgent — au nom de l’universel ou
de la Révolution, voire au nom du féminisme lui-même — de se libérer. Regar-
dant les trois motardes s'éloigner, je réalisais que ce n’était pas si simple, et que
nous risquions, sur cette pente-là, de perdre quelque chose d’important. Elles
ne m'ont pas rendu le féminisme moins sympathique. Mais la féminité, plus
précieuse.

\section{Femme}
%FEMME
Un être humain, de sexe féminin. On dira que c’est l’humanité qui
importe, non le sexe. Peut-être. Mais que l’humanité soit sexuée
n’est pas non plus anecdotique.

La différence sexuelle est sans doute l’une des plus fortes, des plus constantes,
des plus structurantes qui soient. Chacun d’entre nous ne cesse de s’y
confronter. Et pourtant toute tentative de caractériser positivement cette
moitié-là de l’humanité (donc aussi de caractériser l’autre) ne débouche que sur
des approximations ou des platitudes. Que les femmes soient ordinairement
moins violentes que les hommes, qu’elles aient davantage le sens du concret, de
la durée, du quotidien (une certaine façon, chez les meilleures, d’être de plain-pied
avec la vie ou le réel), qu’elles soient plus douées pour l’amour et l’intimité,
moins portées vers la pornographie et le pouvoir, c’est ce qui semble souvent
vrai, mais qui ne va pas, chez les hommes comme chez les femmes, sans de
nombreux contre-exemples, qui interdisent d’en faire une loi ou une essence.
La différence, entre les deux sexes, reste floue, et doit autant ou davantage à la
culture, selon toute vraisemblance, qu’à la nature. Toutefois cela ne prouve pas
que les femmes n’existent pas, ni qu’elles ne soient tendanciellement différentes
des hommes. Pourquoi le flou existerait-il moins que le net, ou le culturel
%— 244 —
moins que le naturel ? « On ne naît pas femme, disait Simone de Beauvoir, on
le devient. » C’était mettre curieusement le corps entre parenthèses. La biologie
m'éclaire davantage (on naît femme, ou homme, puis on devient ce que l’on
est, de façon plus ou moins féminine ou masculine), mais peu importe : ce
{\it devenir}-là, quand bien même il devrait tout à la culture, est l’un des plus beaux
cadeaux que l'humanité se soit faits à elle-même.

\section{Fête}
%FÊTE
Un moment privilégié, souvent inscrit d’avance dans le calendrier en
souvenir d’un autre, qu’il commémore: occasion d’abord de
recueillement, puis de réjouissances. De nos jours, les réjouissances tendent de
plus en plus à l’emporter. C’est pourquoi elles sont souvent un peu tristes ou
contraintes — ou le seraient, sans l'alcool. « Une fête est un excès permis, écrit
Freud, voire ordonné » ({\it Totem et tabou}, IV, 5). Mais quoi de plus oppressant
qu’un excès obligatoire ? Quoi de plus démoralisant qu’une joie programmée ?
Heureusement que la fête nous le fait oublier ! Puis c’est aussi l’occasion de
revoir ses amis, à quoi le hasard et l’amitié ne suffisent pas toujours.

\section{Fétichisme}
%FÉTICHISME
C’est aimer un objet plutôt qu’un sujet, un symbole plutôt
que ce qu’il symbolise, ou la partie plutôt que le tout : par
exemple une statuette plutôt que le dieu, la valeur d'échange ou l'argent plutôt
que la valeur d’usage ou le travail (Marx), une partie du corps ou un vêtement
plutôt que le corps entier et sexué (Freud). C’est avouer et dénier à la fois une
absence : celle du dieu, celle du travail vivant, celle (selon Freud) du phallus de
la mère.

En un sens plus large, on pourrait parler de fétichisme pour tout amour qui
reste prisonnier de son objet, dont il croit dépendre. Si je l'aime parce qu'il est
seul aimable, comment pourrais-je jouir ou me réjouir sans lui ? À quoi Spinoza
objecte que le désir est premier, qui donne à l’objet sa valeur : ce n’est pas
parce qu’il est aimable que je l’aime, c’est parce que je l’aime qu’il est aimable
(pour moi), et c’est en quoi tout l’est ou peut l’être. Ainsi tout amour est fétichiste,
tant qu’il n’est pas universel. C’est avouer et dénier à la fois l’absence en
nous de l’amour pour tout le reste. Seule la charité y échappe. Mais en sommes-nous
capables ?

\section{Fidéisme}
%FIDÉISME
Toute doctrine, spécialement religieuse, qui ne se fonde que sur
la foi {\it (fides)}, à l'exclusion de toute connaissance rationnelle.
C’est le contraire du rationalisme en matière de religion. L'Église y voit traditionnellement
%— 245 —
deux hérésies : ni foi seule ni raison seule ne suffisent ; la vraie
religion a besoin des deux. Sans doute. Mais quelle est la vraie religion ? Seule
la foi répond, et c’est ce qui donne raison, malgré tout, au fidéisme — ou à
l’athéisme.

\section{Fidélité}
%FIDÉLITÉ
On ne la confondra pas avec l'exclusivité. Être fidèle à ses amis,
ce n’est pas n’en avoir qu’un. Être fidèle à ses idées, ce n’est pas
se contenter d’une seule. Même en matière amoureuse ou sexuelle, et malgré
l'usage ordinaire du mot, la fidélité ne se réduit pas plus à l'exclusivité qu’elle
ne la suppose nécessairement. Rien n'empêche, au moins en théorie, deux
amants de se rester fidèles, quand bien même ils pratiqueraient l’échangisme ou
s’autoriseraient mutuellement des aventures avec d’autres. Combien d’époux, à
l'inverse, sans jamais se tromper mutuellement, au sens sexuel du terme, ne cessent
pourtant de se mentir, de se mépriser, de se haïr parfois, qui sont pour cela
plus infidèles que les plus libérés des amants ?

La fidélité n’est pas l’exclusivité ; c’est la constance, c’est la loyauté, c’est la
gratitude, mais tournées toutes les trois vers l’avenir au moins autant que vers
le passé. Vertu de mémoire, certes, mais aussi d’engagement : c’est le souvenir
reconnaissant de ce qui a eu lieu, joint à la volonté de l’entretenir, de le protéger,
de le faire durer, tant que c’est possible, bref de résister à l'oubli, à la trahison,
à l’inconstance, à la frivolité, et même à la lassitude. C’est pourquoi la
fidélité amoureuse a souvent à voir, en pratique, avec l'exclusivité sexuelle : dès
lors qu'on s’est promis celle-ci, elle fait partie de celle-à. Faut-il se la
promettre ? Affaire de goût ou de convenance, à ce que je crois, davantage que
de morale, La fidélité, sans cette exclusivité, n’en vaut pas moins. Mais elle me
semble plus inconfortable, plus exposée, plus aléatoire, enfin plus difficile et
trop pour moi.

On parle aussi de fidélité en matière de religion : c’est la vertu des croyants,
qui restent fidèles à leur foi ou à leur Église. On aurait tort de croire que les
athées en seraient pour cela dispensés. C’est le contraire qui me paraît vrai.
Tant que la foi est là, elle pousse à la fidélité. C’est dire que la fidélité n’est
jamais aussi nécessaire que lorsque la foi fait défaut. Que les deux mots aient la
même origine (le latin {\it fides}), cela ne signifie pas qu’ils soient synonymes. La foi
est une croyance ; la fidélité, une volonté. La foi est une grâce ou une illusion ;
la fidélité, un effort. La foi est une espérance ; la fidélité, un engagement.
Allons-nous, sous prétexte que nous ne croyons plus en Dieu, oublier toutes ces
valeurs que nous avons reçues, dont la plupart sont d’origine religieuse, certes,
mais dont rien ne prouve qu’elles aient besoin d’un Dieu pour subsister, et
dont tout prouve au contraire que nous avons besoin d’elles pour survivre
%— 246 —
d’une façon qui nous paraisse humainement acceptable ? Allons-nous, sous
prétexte que Dieu est mort, laisser son héritage en déshérence ? Que Dieu
existe ou pas, qu'est-ce que cela change à la valeur de la sincérité, de la générosité,
de la justice, de la miséricorde, de la compassion, de l'amour ? Vertu de
mémoire, disais-je, qui vaut aussi pour la mémoire des civilisations. S'agissant
de la nôtre, puisqu'il faut bien en dire un mot, la vraie question me paraît être
la suivante : que reste-t-il de l'Occident chrétien, quand il n’est plus chrétien ?
Je ne vois guère que deux réponses possibles. Ou bien vous pensez qu’il n’en
reste rien, et alors nous sommes une civilisation morte, ou mourante : nous
n'avons plus rien à opposer ni au fanatisme, à l’extérieur, ni au nihilisme, à
l'intérieur (et le nihilisme est de très loin le danger principal). Autant aller se
coucher et attendre la fin, qui ne tardera pas... Ou bien, deuxième possibilité,
vous pensez qu’il en reste quelque chose ; et si ce qu’il en reste ce n’est plus une
{\it foi} commune (puisqu'elle a cessé, de fait, d’être commune : un Français sur
deux se dit athée ou agnostique, un sur quatorze est musulman...), ce ne peut
être qu’une {\it fidélité} commune : le refus d’oublier ce qui nous a faits, de trahir ce
que nous avons reçu, de le laisser finir ou dépérir avec nous. La fidélité, c’est ce
qui reste de la foi quand on l’a perdue : un attachement partagé à ces valeurs
que nous avons reçues et que nous avons à charge de transmettre (la seule façon
d’être vraiment fidèle à ce qu’on a reçu, c’est d’assurer sa pérennité, pour autant
qu’elle dépend de nous, autrement dit sa transmission). Du passé ne faisons pas
table rase : ce serait vouer l’avenir à la barbarie.

« Quand on ne sait où l’on va, dit un proverbe africain, il faut se souvenir
d’où l’on vient. » C’est la seule façon de savoir où l’on {\it veut} aller. Cela vaut pour
notre civilisation comme cela vaut pour les autres, et pour leur ensemble qui est
la civilisation même. Fidélité à l'humanité, je veux dire (car la fidélité n’est ni
la complaisance ni l’aveuglement) à sa meilleure part: celle qui la rend
humaine, et nous avec.

« La fidélité, disait Alain, est la principale vertu de l'esprit. » C’est qu’il n’y
a pas d’esprit sans mémoire, et que la mémoire pourtant n’y suffit pas : encore
faut-il vouloir ne pas oublier, ne pas trahir, ne pas abandonner, ne pas
renoncer, et c’est la fidélité même.

\section{Fierté}
%FIERTÉ
Contentement de soi (fût-ce dans son rapport à autrui : « Je suis
fier de toi »), mais qui ne va guère sans un peu de mépris pour les
autres. C’est le sentiment, qui se croit légitime, de sa propre supériorité, ou en
tout cas de sa propre valeur, en tant qu’elle excède la moyenne ou qu’elle serait
(mais par qui ?) {\it méritée}. Si la fierté se situe quelque part entre la dignité et
%— 247 —
l’orgueil, elle est plus proche de celui-ci : c’est un défaut qui se prend pour une
vertu, une petitesse qui se croit grande. Manque d’humilité, donc de lucidité.

La fierté ne vaut que comme défense, contre le mépris dont on est l’objet.
Qu’on manifeste pour la « fierté homosexuelle » (la {\it gay pride}), soit. Une manifestation
pour la fierté hétéro ne serait qu’un rassemblement de beaufs ou
d’homophobes.

\section{Fin}
%FIN
Le mot a deux sens très différents, qu’il importe de ne pas confondre :
il peut désigner la limite ou le but, le terme ou la destination, la {\it finitude}
ou la {\it finalité}. Par exemple que la mort soit la {\it fin} de la vie, cela fait partie
de sa définition ; mais ne nous dit pas si elle est son {\it but}, comme le voulait
Platon, ou simplement, et comme disait Montaigne, son {\it bout} ({\it Essais}, III, 12,
1051). On dira que les deux peuvent ne faire qu’un : tel serait le cas de la ligne
d’arrivée dans une course. Je ne suis pas sûr que l’exemple soit pertinent (le but,
dans une course, est moins d’arriver que de vaincre), ni, encore moins, qu’on
puisse le généraliser. Ce n’est pas pour son dernier mot qu’on écrit un livre, pas
pour son dernier jour qu’on vit tous les autres, pas pour sa clôture qu’on cultive
un jardin.

Ce qui reste vrai, en revanche, et que les Grecs savaient mieux que nous,
c’est que ces deux sens sont à la fois liés et asymétriques : la finalité suppose la
finitude, non la finitude la finalité. L’infini, par définition, ne peut aller nulle
part, ni tendre vers quoi que ce soit. Imaginez une autoroute infinie : {\it où} pourrait-elle
aller ? Un univers infini : {\it vers quoi} pourrait-il tendre ? Un Dieu infini :
{\it à quoi} pourrait-il servir ? Alors que le moindre de nos actes, pour fini qu’il soit,
n’en a pas moins une finalité (le but que nous visons à travers lui). L’infini n’est
pas un but plausible, ni n’en peut avoir. Seul le fini vaut la peine qu’il se donne.

\section{Finale (cause —)}
%FINALE (CAUSE —)
Une cause, c’est ce qui répond à la question {\it « Pourquoi ? »} ;
une cause finale, ce qui y répond par l’énoncé
d’un {\it but}. Par exemple, explique Aristote, la santé est la {\it cause finale} de la promenade,
si, à la question : « Pourquoi se promène-t-il ? », on peut répondre
légitimement : « Pour sa santé » ({\it Physique}, II, 3). Ainsi la cause finale est {\it ce en
vue de quoi} quelque chose existe, qui n’existerait pas sans cela.

Mais alors pourquoi ne se promènent-ils pas tous, ni toujours ? La santé
n'est-elle une fin que pour quelques-uns, et à quelques moments ? Non pas ;
mais elle n’agit, comme fin, qu’à condition qu’un désir actuel et actif la vise.
C’est où l’on échappe au finalisme — et à Aristote — par Spinoza : la santé n’est
%— 248 
une cause finale que pour autant que le désir de santé est une cause efficiente
({\it Éthique}, IV, Préface).

\section{Finalisme}
%FINALISME
Toute doctrine qui accorde aux causes finales un rôle effectif.
On l’illustre souvent par les exemples délicieusement outrés de
Bernardin de Saint-Pierre. Pourquoi les melons sont-ils divisés en côtes ? Pour
qu’on puisse plus aisément les manger en famille. Pourquoi Dieu nous a-t-il
donné des fesses ? Pour que nous puissions plus confortablement nous asseoir
et méditer sur les merveilles de sa création. Mais cela ne doit pas faire oublier
que la plupart des grands philosophes — depuis Platon et Aristote jusqu’à
Bergson et Teilhard de Chardin — ont été finalistes. Au reste, comment y
échapper, si l’on croit en un Dieu créateur ou ordonnateur ? Même Descartes,
qui fit tant pour que les scientifiques cessent de chercher des causes finales, ne
contestait pas qu’elles puissent exister en Dieu, mais seulement que nous puissions
les connaître ({\it Principes}, III, 1-3). Quant à Leibniz, il y voyait « le principe
de toutes les existences et des lois de la nature, parce que Dieu se propose toujours
le meilleur et le plus parfait » ({\it Discours de métaphysique}, 19). Pourquoi
aurions-nous des yeux, si ce n’était {\it pour voir} ?

Ce dernier exemple, repris par Leibniz et tant d’autres, suggère bien l’essentiel.
Pourquoi voyons-nous ? Parce que nous avons des yeux. Pourquoi avons-nous
des yeux ? Pour voir. Ainsi les yeux sont la cause efficiente de la vue ; la
vue, la cause finale des yeux. Mais laquelle de ces deux causes existe d’abord ?
Est-ce la fonction qui crée l’organe (ce qui est une forme de finalisme), ou
l'organe qui crée la fonction ? Les matérialistes, presque tous, presque seuls,
choisissent résolument le deuxième terme de alternative. Penser que nous
avons des yeux pour voir, explique Lucrèce, « c’est faire un raisonnement qui
renverse le rapport des choses, c’est mettre partout la cause après l’effet » (IV,
823 sq.). Mais pourquoi, alors, avons-nous des yeux ? Par hasard ? Non pas ;
mais seulement par des causes efficientes, qui renvoient au passé de l’espèce
(par l’hérédité et la sélection naturelle), non à l’avenir de l’individu.

Spinoza, sur cette question comme sur tant d’autres, est du côté des matérialistes.
Le finalisme, pour lui aussi, « renverse totalement la nature : il considère
comme effet ce qui en réalité est cause, et met après ce qui de nature est
avant » ({\it Éthique}, I, Appendice). C’est là le préjugé fondamental, dont tous les
autres dérivent : « Les hommes supposent communément que toutes les choses
de la nature agissent, comme eux-mêmes, en vue d’une fin » ({\it ibid.}). S'agissant
de la nature, c’est clairement une illusion. Mais s'agissant d’eux-mêmes ? C'en
est une aussi, du moins si l’on y voit la cause de leurs actes : cette illusion
s'appelle alors le libre arbitre. Soit par exemple cette maison que je construis.
%— 249 —
Pourquoi le fais-je ? Si on répond {\it « pour l'habiter »}, comme beaucoup le feront
spontanément, cela signifie que ce qui n’est pas encore (l’habitation) explique
et produit ce qui est (le travail de construction), que c’est la fin, comme dira
Sartre, qui « éclaire ce qui est » : tel est le paradoxe de la liberté ({\it L'Étre et le
Néant}, p. 519-520, 530, 577-578...). Mais comment ce qui n’est pas encore
pourrait-il produire ou expliquer quoi que ce soit ? Que les hommes agissent
toujours « en vue d’une fin », comme le reconnaît Spinoza, cela ne prouve pas
que cette fin soit la cause de l’action : ce qu’on prend pour une cause finale
n’est rien d’autre qu’un désir déterminé (ici le désir d’habiter cette maison),
et ce désir «est en réalité une cause efficiente » ({\it Éthique}, IV, Préface). Les
hommes se trompent en ce qu’ils se figurent être libres ; cette illusion n’est
pas autre chose qu’un finalisme à la première personne : ils ont conscience de
viser une fin, point des causes efficientes qui les déterminent à le faire
(I, Appendice).

Ainsi il n’y a pas de finalité du tout : il n’y a que la puissance aveugle de la
nature, et celle, qui peut être éclairée, du désir.

\section{Finalité}
%FINALITÉ
Le fait de tendre vers une fin ou un but. C’est le cas par exemple
de la plupart de nos actions, et même, selon Freud, de nos actes
manqués. Cela ne prouve pas que cette fin soit la {\it cause} de l’acte. C’est où la
finalité, qui est une donnée de la conscience ou un fait, se distingue du finalisme,
qui est une doctrine et un contresens. On agit {\it pour} une fin, mais {\it par} un
désir : la finalité n’est elle-même qu’un effet de l'efficience du désir.

On pourrait en dire autant de la finalité du vivant, qui est un fait d’expérience,
mais dont rien ne prouve qu’elle soit une {\it cause}. C’est pourquoi nos biologistes
parlent plutôt, pour la désigner, de téléonomie, qui est une finalité sans
finalisme (une finalité pensée comme effet, point comme cause).

\section{Finesse (esprit de —)}
%FINESSE (ESPRIT DE —)
Le sens des nuances, du flou, de la complexité
inépuisable du réel, bref de tout ce qui se sent
davantage que cela ne se voit ou ne se démontre. S’oppose traditionnellement,
surtout depuis Pascal, à l'esprit de géométrie ({\it Pensées}, 512-1 et 513-4).

\section{Fini}
%FINI
En philosophie, c’est moins ce qui est terminé que ce qui peut l'être :
ce qui n’est pas infini. La {\it Symphonie inachevée} est aussi {\it finie} que toutes
les autres. Et nous sommes voués à la finitude bien avant d’être morts.

%— 250 
\section{Finitude}
%FINITUDE
Le fait d’avoir une limite, un terme, une borne {\it (finis)} — de n’être
pas infini. Les Anciens y voyaient plutôt un bonheur, pour qui
savait s’en contenter. « Épicure fixa des bornes {\it (finem statuit)} au désir comme à
la crainte », se réjouit Lucrèce, et il n’y aurait pas de sagesse autrement. Le faux
infini des désirs nous voue à l’insatisfaction, au malheur, à la démesure. Seul
celui qui s’accepte fini peut échapper à l'angoisse : la sagesse est une finitude
heureuse, dans l'infini qui nous contient. Quel bonheur d’être un homme,
quand on ne prétend pas à autre chose!

Chez les Modernes, et spécialement chez les existentialistes, la finitude
prend des couleurs plus sombres : c’est comme une amputation de l'infini, qui
resterait, telle un membre fantôme, à jamais douloureuse. C’est le malheur de
n'être pas Dieu. La finitude, en ce sens, est le propre de l’homme, en tant qu’il
est voué à la mort. Non qu’il soit seul fini, ni qu’il le soit par la mort seule.
Mais parce qu’il est seul à savoir clairement qu’il l’est (les animaux, selon toute
vraisemblance, n’ont aucune notion de l'infini, ni donc de la finitude), et qu’il
va mourir. Attention, toutefois, de ne pas accorder trop à la mort. Sur la finitude,
il me semble que le sexe et la fatigue nous en apprennent davantage.

\section{Flatterie}
%FLATTERIE
C'est dire à quelqu'un, pour attirer ses bonnes grâces, plus de
bien qu’on n’en pense. Quand c’est fait un peu habilement,
cela ne rate presque jamais, ce qui en dit long sur l'humanité. L’amour-propre,
presque inévitablement, l'emporte sur l'amour de la vérité. « Ainsi, écrit Pascal,
la vie humaine n’est qu’une illusion perpétuelle ; on ne fait que s’entre-tromper
et s’entre-flatter. Personne ne parle de nous en notre présence comme il en
parle en notre absence. L’union qui est entre les hommes n’est fondée que sur
cette mutuelle tromperie ; et peu d’amitiés subsisteraient, si chacun savait ce
que son ami dit de lui lorsqu'il n’y est pas, quoiqu'il en parle alors sincèrement
et sans passion » ({\it Pensées}, 978-100).

\section{Foi}
%FOI
Croyance sans preuve, comme toute croyance, mais qui s’en passe avantageusement,
par volonté, confiance ou grâce. Avantage équivoque,
voire suspect. C’est se croire, se fier, ou se soumettre. Toute foi pêche par suffisance,
ou par insuffisance.

« La foi, écrit Kant, est une croyance qui n’est suffisante que subjectivement »
({\it C. R. Pure}, théorie de la méthode, II, 3). Elle ne l’est donc que
pour les sujets qui se suffisent de leur subjectivité. Pour les autres, le doute
l'accompagne et la sauve.

%— 251 —
Au sens le plus ordinaire, le mot désigne une croyance religieuse, et tout ce
qui lui ressemble. C’est croire en une vérité qui serait une valeur, en une valeur
qui serait une vérité. Avoir foi en la justice, par exemple, c’est non seulement
aimer la justice mais croire qu’elle existe. Avoir foi en l’amour, c’est non seulement
l’aimer mais en faire un absolu, qui existerait indépendamment de nos
amours très relatives. C’est pourquoi la foi porte spécialement sur Dieu : parce
qu’il serait la conjonction absolue de la valeur (qu’on doit aimer) et de la vérité
(qu’on peut connaître ou reconnaître). C’est aussi sa limite : si l’on connaissait
Dieu, on n’aurait plus besoin d’y croire.

La foi porte également sur l'avenir. C’est comme une utopie métaphysique :
l’espérance s’invente un objet, qui la transforme en vérité. Il s’agit de
croire, comme disait Kant, que « quelque chose est. puisque quelque chose
doit arriver ». Ce mensonge, dans sa sincérité, est la religion même.

La foi ne se nourrit que de l’ignorance de son objet. « Je dus donc mettre
de côté le {\it savoir}, reconnaît encore Kant, afin d’obtenir une place pour la {\it foi} »
({\it C. R. Pure}, Préface de la 2$^\text{e}$ éd.). Les savants, depuis vingt-cinq siècles, font
l'inverse.

\section{Folie}
%FOLIE
« Le fou a tout perdu, disait un psychiatre, sauf la raison. » Mais elle
tourne à vide : elle a perdu les {\it rails} du réel. Le délire paranoïaque,
par exemple, peut être d’une formidable cohérence (ce n’est pas un hasard si
Freud compare les systèmes philosophiques à des paranoïas réussies) ; mais il
est fermé sur lui-même au lieu d’ouvrir sur le monde. C’est une leçon pour le
philosophe : la pensée n’échappe à la folie que par son dehors, qui est le réel ou
la pensée des autres. Ce qui n’est vrai que pour toi ne l’est pas.

\section{Fondement}
%FONDEMENT
Disons d’abord, avec Marcel Conche, ce que ce n’est pas :
un fondement n’est ni un principe, ni une cause, ni une origine.
La cause explique un fait ; le fondement établit un droit ou un devoir.
L'origine rend raison d’un devenir ; le fondement, d’une valeur. Enfin un principe
n’est que le point de départ — qui peut être hypothétique ou douteux —
d’un raisonnement ; le fondement serait « la justification radicale du principe
lui-même» ({\it Le fondement de la morale}, Introduction). Qu'est-ce qu’un
fondement ? La justification nécessaire et suffisante d’un droit, d’un devoir,
d’une valeur ou d’un principe, de telle sorte que l'esprit puisse {\it et doive} leur
donner son assentiment. Un fondement, c’est donc ce qui garantit la valeur ou
la vérité de ce qu’il fonde : ce qui nous permettrait d’être certain (non seulement
en fait mais en droit) d’avoir raison.

%— 252 —
C’est pourquoi il n’y a pas de fondement, me semble-t-il, ni ne peut y en
avoir : parce qu’il faudrait qu’il soit d’abord lui-même rationnellement
démontré ou établi, ce qui n’est possible qu’à la condition de fonder d’abord la
valeur de notre raison, laquelle ne peut l'être ni sur elle-même (car il y aurait là
un cercle) ni sur autre chose (car il y aurait là une régression à l’infini, cette
autre chose devant à son tour être fondée, et ne pouvant l’être que sur la raison
ou sur autre chose). Non, certes, que la raison ne vaille rien, ce qui n’est ni
démontrable ni vraisemblable, mais parce que sa valeur, qui permet nos
démonstrations, ne peut elle-même être rationnellement démontrée. La proposition
«Il y a de vraies démonstrations » est indémontrable — puisque toute
démonstration le suppose.

Un fondement des mathématiques ? Non seulement on a fait des mathématiques
bien avant de disposer de quelque fondement que ce soit, mais les
mathématiciens d’aujourd’hui, si brillants, si performants, ont à peu près
renoncé à en chercher un. Au reste, dès lors que le théorème de Güdel a établi
que, dans un système formel contenant au moins l’arithmétique, on ne peut ni
tout démontrer (il y a des énoncés indécidables) ni démontrer que le système
n’est pas contradictoire (la cohérence du système est elle-même indécidable à
l'intérieur de ce système), on ne voit guère, d’un point de vue philosophique,
quel sens il y aurait à prétendre {\it fonder} les mathématiques : comment garantir
une cohérence que l’on ne peut démontrer ? La proposition « Les mathématiques
sont vraies » (ou « Les mathématiques sont cohérentes ») n’est pas susceptible
d’une démonstration mathématique, ni de quelque démonstration que
ce soit. Cela, qui interdit de fonder les mathématiques, n’empêche pas d’en
faire, et ne leur retire rien d’autre que l'illusion de l’absoluité.

Un fondement de la morale ? Ce ne pourrait être que la conjonction nécessaire
et absolue (ni contingente ni dépendante) du vrai et du bien, de la valeur
et de la vérité : ce ne pourrait être que Dieu, et c’est pourquoi ce n’est pas. Que
vaudrait une morale qui aurait besoin d’un Dieu pour valoir ? Ce serait une
morale dépendante d’une religion, qu’il faudrait à son tour fonder : démontrez-moi
quelle est la vraie religion, je vous dirai quelle est la vraie morale ! Si on
laisse de côté ce fondement théologique, qui n’en serait pas un, tout fondement
de la morale doit lui-même être démontré (ce qui nous renvoie aux apories précédentes),
et peut d’autant moins l’être que sa vérité même, à supposer qu’on
puisse l’établir, n’y suffirait pas : car pourquoi devrais-je me soumettre au vrai ?
Pourquoi ne pas préférer le faux, l'erreur, l'illusion ? Cet individu, par exemple,
qui n’hésite pas à assassiner, à violer, à torturer, pourquoi devrait-il se soumettre
au principe de non-contradiction ? Et pourquoi aurions-nous besoin
d’un fondement pour le combattre ou pour lui résister ? L’horreur suffit. La
compassion suffit, et vaut mieux.

%— 253 —
\section{Force}
%FORCE
Une puissance en acte ({\it energeia}, en grec, plutôt que {\it dunamis}). Se
dit spécialement en mécanique : on appelle {\it force} ce qui modifie le
mouvement (ou le repos) d’un corps, qui resterait autrement — par le principe
d’inertie — rectiligne et uniforme.

On oppose souvent la force au droit, comme les lois de la nature aux lois
des hommes. On a raison. Le droit du plus fort n’est pas un droit ; le droit du
plus faible, pas une force. C’est pourquoi on a besoin d’un État, pour que la
force et le droit aillent ensemble.

On parle parfois de {\it force d'âme} pour désigner le courage. C’est encore,
quoiqu'en un sens métaphorique, le contraire de l’inertie : la puissance de
modifier son propre mouvement, ou son propre repos. Le corps voudrait fuir,
et l'on ne fuit pas. Céder, et l’on ne cède pas. Frapper, et l’on ne frappe pas.
C'est ce qui fait croire à l'âme, et l’on a à nouveau bien raison. Mais elle
n'existe que par courage et volonté : « Ce beau mot ne désigne nullement un
être, disait Alain, mais toujours une action. » Ainsi toute Âme est force d'âme,
mais non toute force.

\section{Forclore}
%FORCLORE
C’est enfermer dehors. Se dit par exemple d’un droit, lorsque
son délai d'application a été dépassé, ou d’une représentation,
quand on refuse de la prendre en compte. Le mot, qui n’appartenait guère qu’au
vocabulaire des juristes, a trouvé une nouvelle jeunesse et un nouvel emploi dans
la psychanalyse, spécialement lacanienne, pour traduire le {\it Verwerfung} de Freud :
c'est rejeter une représentation ou un signifiant hors du sujet ou de son univers
symbolique (et non pas, comme dans le refoulement, à l’intérieur de l’inconscient),
de telle sorte qu’ils feront retour du dehors, spécialement sous forme
d’hallucinations : « Ce qui a été aboli à l’intérieur revient de l'extérieur », écrit
Freud, ce qui veut dire, précise Lacan, que « ce qui a été forclos du symbolique
réapparaît dans le réel ». Ce mécanisme serait à l’origine des psychoses, et les distinguerait
des névroses (qui doivent davantage au refoulement et au retour du
refoulé). Le névrosé est prisonnier du passé, qu’il a refoulé ; le psychotique, du
présent, qu'il a forclos : il s’enferme hors du réel (délire, hallucination) en voulant
l’enfermer hors de lui (forclusion). Cela me fait penser à cette histoire d’un fou
aveugle, qui se heurte à une colonne, qui en fait le tour plusieurs fois, qui la
palpe, comme à la recherche d’une issue, puis qui finit par s’écrier : « Je suis
enfermé ! » Ainsi le psychotique, enfermé dans le réel même qu’il rejette.

\section{Formalisme}
%FORMALISME
Juger sur la forme, plutôt que sur le contenu matériel ou
affectif. Ainsi en logique formelle ou en mathématiques :
%— 25 —
on raisonne sur des x et des y, à l’intérieur d’un système de signes lui-même
réglé par une axiomatique et sans se soucier de ce que ces signes peuvent signifier.
C’est remplacer la représentation par le calcul, et il n’y aurait pas de
science autrement. Cela, toutefois, ne prouve pas que le monde soit fait d’x et
d'y.

En philosophie morale, on parle de formalisme, spécialement chez Kant,
pour désigner une doctrine morale qui fait de « la pure forme d’une loi » (donc
de l’exigence d’universalisation possible) l'essentiel de la moralité, indépendamment
des affects mis en jeu comme des effets de l’action. C’est mettre le devoir
plus haut que les sentiments, et l’intention plus haut que la réussite.

\section{Forme}
%FORME
Soit, par exemple, cette statue d’Apollon. On peut, avec Aristote,
distinguer sa {\it matière} (le marbre dont elle est faite) et sa {\it forme} (celle
que le sculpteur lui a donnée). On comprend que la forme est la fin, vers
laquelle tend le travail du sculpteur (la matière n’est qu’un matériau : un point
de départ), et où il s’arrête. Mais la forme est aussi l’essence ou la quiddité.
Qu'est-ce que c’est ? Une statue d’Apollon. On aurait pu faire, avec le même
bloc de marbre, une tout autre statue, comme on aurait pu faire la même, pour
l'essentiel, avec un autre bloc, voire en bois ou en bronze. Ainsi la forme est à
la fois la définition et le définitif: ce qu’est cette statue, une fois qu’elle est
achevée. Non, bien sûr, qu’on ne puisse la briser (le définitif n’est pas éternel),
mais parce qu’elle a atteint sa perfection ou son entéléchie : elle n’était qu’en
puissance dans la matière ; elle est en acte dans sa forme.

Une forme sans matière ? Ce ne serait plus une forme mais une {\it idée} (en grec
c’est le même mot : {\it eîdos}) ou Dieu. C’est où l’on remonte d’Aristote à Platon —
ou du moins où l’on remonterait, si une forme sans matière était autre chose
qu’une abstraction.

Chez Kant, la forme est ce qui met en forme, autrement dit ce qui, venant
du sujet, organise la matière de la sensation. Ainsi les formes de la sensibilité
(l'espace et le temps) ou de l’entendement (les catégories). Reste à savoir si ces
formes existent elles-mêmes indépendamment de la matière, comme le veut
Kant, ou si elles ne sont que l’effet en nous de sa puissance auto-organisatrice.
C’est où il faut choisir entre l’idéalisme et le matérialisme : entre la transcendance
de la forme et l’immanence de la structure.

\section{Formelle (cause —)}
%FORMELLE (CAUSE)
L’une des quatre causes chez Aristote: celle qui
répond à la question « Pourquoi ? » par l’énoncé
d’une {\it forme}. Par exemple : Pourquoi cette maison ? À cause des briques (cause
%— 255 
matérielle), du plaisir de l’habiter (cause finale), du maçon, de l’architecte, du
maître d'œuvre (causes efficientes) ? Sans doute. Mais aucune de ces causes
n'aurait fait une maison sans le plan qui la rend possible et qu’elle actualise —
sans la forme ou idée ({\it eîdos}) de la maison, non de façon séparée, comme le voudrait
Platon, mais telle qu’elle existe d’abord dans l'esprit des bâtisseurs puis
dans la maison elle-même (sa forme immanente). On dira qu’à ce compte, la
cause de la maison... c’est la maison. Pourquoi non ? En ce sens, écrit Aristote,
«nous entendons par cause l’essence ({\it ousia}) de la chose, ce qui fait qu’elle est
ce qu'elle est » ({\it Métaphysique}, A, 3 ; Tricot traduit : « la substance formelle ou
quiddité »). Aucune des trois autres causes, sans celle-ci, ne rendra jamais
compte du réel. Mais celle-ci fait-elle autre chose que le présupposer ?

\section{Fortune}
%FORTUNE
Le hasard ou la richesse. Comme le remarque Alain, la rencontre
de ces deux sens est bien instructive : « c’est ramener
l’origine des richesses au pur hasard ; ce qui va au fond ; car le travail n’enrichit
pas sans quelque rencontre de fortune. Ainsi demander si la fortune est juste,
c'est demander si la loterie est juste » ({\it Définitions}, art. « Fortune »). Heureusement
qu'il y a les impôts et la sécurité sociale.

\section{Foule}
%FOULE
Une multitude d’individus rassemblés, considérée d’un point de
vue seulement quantitatif. Il y manque la qualité : les corps
s’additionnent ; les esprits, non. De là cette force collective des passions, des
émotions, des pulsions. Le plus bas, presque inévitablement, prend le dessus.
C’est parfois plaisant, parfois exaltant, parfois effrayant, jamais admirable.
Toute foule est dérisoire ou dangereuse.

\section{Franchise}
%FRANCHISE
Une sincérité simple et directe. C’est s’interdire non seulement
le mensonge, mais la dissimulation et le calcul. La franchise
va souvent contre la politesse, parfois contre la compassion. À réserver à
ses amis et aux puissants.

\section{Fraude}
%FRAUDE
Une tromperie intéressée. Voltaire se demande « s’il faut user de
fraudes pieuses avec le peuple », pour le maintenir dans le droit
chemin. Platon répondait que oui. Voltaire s’y refuse : la vraie religion, celle
qui est « sans superstition », n’a pas besoin de ces mensonges ; et « la vertu doit
%— 256 —
être embrassée par amour, non par crainte ». Mais en quoi, alors, a-t-elle besoin
de religion ?

\section{Frivolité}
%FRIVOLITÉ
Le contraire de la gravité, À ne pas confondre avec la légèreté.
La légèreté est sans lourdeur ; la frivolité, sans profondeur.
C’est moins un goût pour les petites choses qu’une incapacité à s'intéresser aux
grandes. On n’est pas frivole parce qu’on aime la bonne chère, les jeux de mots
ou les bals. Mais on l’est, assurément, si l’on est incapable d’apprécier autre
chose. Célimène est frivole, non parce qu’elle est coquette, mais parce qu’elle
ne sait pas aimer.

\section{Frustration}
%FRUSTRATION
Un manque, quand on est incapable de le satisfaire ou d’y
renoncer. Se distingue par là de l’espérance (qui peut être
satisfaite), du deuil (qui renonce à l'être) et du plaisir (qui est la satisfaction
même).
La frustration aboutit presque toujours à exagérer les plaisirs qu’on n’a pas
(obsession), et d’autant plus que d’autres en jouissent (envie). Contre quoi le
plaisir est un premier pas vers la sagesse.

\section{Futur}
%FUTUR
Un autre mot pour l'avenir. Si on veut les distinguer, on peut dire
que le futur désigne davantage une dimension du temps que son
contenu. L'avenir, c’est ce qui viendra ; le futur, le {\it temps} à venir. L'avenir est
fait d'événements, dont nous ignorons la plupart. Le futur n’est fait que de lui-même :
c’est un temps vide, bien sûr imaginaire, que l’avenir viendra remplir.
Mais alors ce ne sera plus du futur, ni de l’avenir : ce sera du présent, qui est le
seul temps réel.
% 257

\section{Gaieté}
%GAIETÉ
J'aime cet {\it e} central, facultatif et muet, comme un rayon de
lumière ou de silence. J’y reconnais quelque chose de la gaieté : sa
transparence, sa fragilité, sa fraîcheur, sa légèreté, sa délicieuse inutilité...
Qu’est-elle ? Une disposition à la joie, qui la rend facile, naturelle, spontanée,
comme déjà là avant même qu’on ait quelque raison de se réjouir. Vertu
d’insouciance, qui serait d’humeur plus que de volonté. Sa force est dans sa
superficialité : les grand malheurs comme les grandes joies sont trop profonds
pour elle ; ils la traversent davantage qu’ils ne l’atteignent. Être gai, c’est avoir
la joie facile ou à fleur de peau. Quel talent plus enviable ? Quel charme plus
séduisant ?

\section{Généalogie}
%GÉNÉALOGIE
L'étude des origines, de la filiation, de la genèse. Se dit surtout
des familles et, depuis Nietzsche, des valeurs : c’est rattacher
un individu à ses ancêtres ou une valeur à un type de vie, pour les faire
valoir ou au contraire pour les dévaluer. Dans {\it La généalogie de la morale}, spécialement,
Nietzsche s'interroge sur « l’origine de nos préjugés moraux» (y
compris sur la valeur de la vérité), afin de se diriger « vers une véritable histoire
de la morale ». Travail d’historien ? Si l’on veut. Mais c’est une histoire normative
et critique, qui met la santé plus haut que le vrai, et qui doit déboucher, en
retour, sur une nouvelle évaluation. « Nous avons besoin d’une {\it critique} des
valeurs morales, écrit Nietzsche, et {\it la valeur de ces valeurs} doit tout d’abord être
mise en question. » Comment ? Par l’étude des « conditions et des milieux qui
leur ont donné naissance, au sein desquels elles se sont développées et déformées
(la morale en tant que conséquence, symptôme, masque, tartuferie,
maladie ou malentendu ; mais aussi la morale en tant que cause, remède, stimulant,
%— 258 —
entrave ou poison). » Philosophie à coups de marteaux. Mais c’est le
marteau d’un archéologue, voire d’un médecin (pour tester les réflexes), avant
d’être celui d’un iconoclaste.

\section{Général}
%GÉNÉRAL
Qui concerne un vaste ensemble (un genre) ou la plupart de ses
éléments. S'oppose à {\it spécifique} (qui concerne un ensemble moins
vaste: une espèce), à {\it particulier} (qui ne vaut que pour une partie d’un
ensemble), enfin et surtout à {\it singulier} (qui ne vaut que pour un seul individu
ou un seul groupe). À ne pas confondre avec {\it universel}, qui concerne tous les
genres ou tous les individus d’un même genre. Par exemple la parole est un
attribut {\it général} de l'humanité (il y a des individus qui ne parlent pas), mais un
trait {\it universel} des peuples. Et la prohibition de l'inceste, qui est une règle universelle,
est généralement respectée.

\section{Génération}
%GÉNÉRATION
Le fait d’engendrer {\it (generare)}, le temps qu’il y faut (de la
naissance d’un individu à celle de ses enfants : environ un
quart de siècle), ou bien l’ensemble des individus qui ont été engendrés à peu
près à la même époque, qui ont donc à peu près le même âge et souvent un certain
nombre d’expériences ou de préoccupations communes ou proches. Ce
dernier sens pose bien sûr la question des limites, ici toujours floues et qui doivent
plus à l’histoire qu’à la génétique. La génération à laquelle on appartient,
c’est moins celle de sa naissance que de sa jeunesse : la génération des soixante-
huitards est née dans les années 40 ou 50, mais ce ne sont pas ces années-là qui
la définissent. Et la « génération Mitterrand », si elle existe, est l’ensemble de
ceux qui ont été jeunes — et non qui sont nés ou qui ont vécu — sous sa
présidence : ni mes amis, s’ils ont mon âge, ni mes enfants (qui sont nés dans
les années 80) n’en font partie. Une génération se forge pendant l’adolescence :
c’est l’ensemble de ceux qui ont été jeunes dans la même période historique. Ils
trimbaleront ce poids ou cette légèreté toute leur vie, comme une patrie com-
mune, comme un accent, comme un terroir, jusqu'à ne plus pouvoir com-
prendre tout à fait ceux qui les suivent, qui ne les comprennent pas davantage.
Ils sont {\it « pays »}, comme on disait autrefois, mais dans le temps plutôt que dans
l’espace : ils sont contemporains, depuis leur jeunesse et par elle, de la même
histoire ou de la même éternité. C’est parfois une chance, parfois un handicap,
plus souvent un mixte des deux. C’est où le hasard se mue en destin ; c’est où
le destin, qu’on le fasse ou qu’on le subisse, reste essentiellement hasardeux. Si
nous étions nés vingt ans plus tôt ou plus tard, que serions-nous ? Nous serions
quelqu’un d’autre, et c’est pourquoi nous ne serions pas.

%— 259 —
\section{Générosité}
%GÉNÉROSITÉ
C’est la vertu du don. On dira qu’on donne aussi par
amour. Sans doute, et c’est pourquoi l’amour est généreux.
Mais toute générosité n’est pas aimante, et même elle n’est une vertu spécifique
que vis-à-vis de ceux que l’on n'aime pas. Qui se jugerait généreux parce qu’il
couvre ses enfants de cadeaux ? Il sait bien que c’est amour, non générosité. La
générosité est la vertu du don, en tant qu’elle excède l'amour dont on est
capable. Vertu classique plutôt que chrétienne. Morale, plutôt qu’éthique. Cela
fixe sa limite en même temps que sa grandeur. L'amour ne se commande pas ;
la générosité, si. L'amour ne dépend pas de nous (c’est nous qui en
dépendons) ; la générosité, si. Pour être généreux, il suffit de le vouloir ; pour
aimer, non. La générosité touche à la liberté, comme l’a vu Descartes : c’est la
conscience d’être libre, expliquait-il, jointe à la résolution d’en bien user ({\it Traité
des passions}, III, 153). Cela suppose qu’on vainque en soi tout ce qui n’est pas
libre : sa propre petitesse, sa propre avidité, sa propre peur, enfin la plupart de
ses passions, jusqu’à mépriser son propre intérêt, pour ne plus s’occuper que du
bien qu’on peut faire à autrui (III, 156). Pas étonnant que la générosité soit si
rare ! Donner c’est perdre, et l’on voudrait garder. C’est prendre un risque, et
l’on a peur. Nul n’est libre qu’à la condition d’abord de se surmonter. Nul n’est
généreux par naissance, malgré l’étymologie, mais par éducation, mais par
choix, mais par volonté. La générosité est vertu du don, mais n’est pas elle-même
un don ; c’est une conquête, c’est une victoire, qui ne va pas sans courage
(les deux mots, chez Corneille, sont à peu près synonymes) et qui peut
aller jusqu’à l’héroïsme. C’est donner ce qu’on possède, plutôt qu’en être possédé.
C’est la liberté à l'égard de soi et de sa peur : le contraire de l’égoïsme et
de la lâcheté.

\section{Génèse}
%GENÈSE
Un devenir primordial, comme en amont de la naissance ou du
réel. C’est moins une origine que ce qui en résulte. Moins un commencement,
que le processus qui y mène ou le constitue. Toute genèse prend
du temps : elle ne peut être qu’historique ou mythique.

\section{Génétique}
%GÉNÉTIQUE
Comme substantif, c’est désormais une science, celle de
l’hérédité. Comme adjectif, c’est plutôt un point de vue, qui
peut certes concerner les gènes ou ce qui en dépend (une maladie génétique),
mais qui porte plus souvent, en philosophie, sur la genèse ou le devenir d’un
être quelconque. Une {\it définition génétique} est celle qui comporte en elle, comme
le voulait Spinoza, l’origine ou la cause prochaine de ce qu’elle définit (par
exemple cette définition du cercle, dans le {\it Traité de la réforme de l'entendement} :
%— 260 —
« Une figure qui est décrite par une ligne quelconque dont une extrémité est
fixe et l’autre mobile »). Et l’{\it épistémologie génétique} est celle qui étudie la
connaissance scientifique, comme le voulait Piaget, dans son développement,
aussi bien individuel (chez l’enfant) que collectif (dans l’histoire des sciences) :
c’est s'interroger sur le processus de la connaissance plutôt que sur son origine
ou son fondement.

\section{Génie}
%GÉNIE
L'abbé Dubos, au début du {\footnotesize XVIII$^\text{e}$} siècle, en donnait la définition suivante :
« On appelle génie l'aptitude qu’un homme a reçue de la
nature pour faire bien et facilement certaines choses que les autres ne sauraient
faire que très mal, même en prenant beaucoup de peine. » En ce sens général,
c’est un synonyme de {\it talent}. L'usage s’est pourtant généralisé de distinguer les
deux notions. D’abord par une différence de degré : le génie est comme un
talent extrême ; le talent, comme un génie limité. Mais aussi par une différence
plus mystérieuse, qui semble de statut ou d’essence. « Le talent fait ce qu'il
veut ; le génie, ce qu’il peut. » Cette formule, dont je ne sais plus l’auteur,
indique au moins une direction. Le génie est une puissance créatrice, qui
excède non seulement la puissance commune (ce que fait déjà le talent) mais
celle même du créateur, au point d'échapper, au moins en partie, à son contrôle
ou à sa volonté. On ne choisit pas d’avoir du génie, ni lequel, ni même toujours
ce qu’on en fait. Le génie est un « don naturel », écrit Kant, autrement dit « une
disposition innée de l'esprit, par laquelle la nature donne à l’art ses règles »
({\it Critique de la faculté de juger}, X, \S 46). Cela ne signifie pas que le génie n’ait
pas besoin d’être cultivé, mais qu'aucune culture ne saurait en donner ou le
remplacer. Mozart, si son père n'avait pas été le pédagogue que l’on sait,
n’aurait peut-être jamais été musicien. Mais aucun pédagogue ne fera un
Mozart d’un enfant sans génie. Le génie est comme un dieu personnel (el était
le sens, en latin, de {\it genius}), qu’on ne choisit pas, mais qui nous choisit. C’est
marquer assez ce qu’il doit au hasard ou à l'injustice. Comment se consoler de
n'être pas Mozart ?

La différence avec le talent ne doit pas être exagérée, ni pourtant, me
semble-t-il, tout à fait abolie. Si on laisse de côté l’exaltation romantique, j'y
vois plus une différence de degré que de nature, de point de vue que d’orientation.
Toutefois quelque chose résiste, dans certaines œuvres, qui interdit de n’y
voir que talent et travail. Voyez Bach ou Michel-Ange, Rembrandt ou Shakespeare,
Newton ou Einstein, Spinoza ou Leibniz. Illusion rétrospective ? Sans
doute, pour une part. Si l’on fait du génie une exception absolue, il est clair
qu’il est toujours mythique, et qu'il convient pour cela de n’en parler qu'à
propos des morts. Tout vivant est médiocre par tel ou tel côté. Seuls le temps
%— 261 
et l’absence donneront à certains cette stature démesurée. Mais enfin l’œuvre
reste, qui maintient ou rétablit les proportions. « Un livre n’est jamais un chef-d'œuvre,
remarquaient finement les Goncourt, il le devient : le génie est le
talent d’un homme mort. » Ces deux-là toutefois sont morts, qui n’ont toujours
que du talent.

\section{Génie (malin —)}
%GÉNIE (MALIN —)
Chez Descartes, c’est un petit dieu ou démon, bien sûr
imaginaire, qui nous tromperait toujours ({\it Méditations}, 1).
Le but de cette fiction est d’exagérer le doute (puisqu’on va considérer comme
faux ce qui n’est qu’incertain), afin de nous désaccoutumer de nos préjugés, de
nos anciennes opinions, enfin de toute croyance. C’est tordre le bâton dans
l’autre sens, pour le redresser. Le but est d’atteindre une certitude absolue —
celle qui résisterait à l'hypothèse du malin génie. Ce sera le {\it cogito}, qui n’est
peut-être qu’un génie un peu plus {\it malin} que les autres.

\section{Génocide}
%GÉNOCIDE
L’extermination d’un peuple. Ce n’est pas seulement un crime
de masse : c’est un crime contre l'humanité, en tant qu’elle est
une et plurielle.

\section{Genre}
%GENRE
Un vaste ensemble, qui ne se définit pourtant que par rapport à
d’autres : plus vaste que l’espèce (un genre en comporte plusieurs),
plus réduit que l’ordre (au sens biologique du terme : le genre {\it Homo}, dont
{\it Homo sapiens} est la seule espèce survivante, fait partie de l’ordre des primates).
Notion par nature relative. Ce qui est {\it genre} pour ses espèces peut être
{\it espèce} pour un autre genre, qui l’inclut. Par exemple le quadrilatère, qui est
un genre pour ses différentes espèces (trapèze, losange, rectangle...) est lui-même
une espèce du genre polygone, qui peut à son tour être considéré
comme une espèce du genre figure géométrique. Tout dépend de l'échelle et
du point de vue adoptés. C’est pourquoi on parle de {\it genre prochain}, depuis
Aristote, pour désigner l’ensemble immédiatement supérieur (en extension) à
celui qu’on veut définir (en compréhension) : il suffira pour cela d’indiquer
la ou les différences spécifiques de ce dernier. Par exemple un quadrilatère est
un polygone (genre prochain) à quatre côtés (différence spécifique), comme
un trapèze est un quadrilatère (genre prochain) dont deux des côtés sont
parallèles (différence spécifique). C’est une façon d’ordonner le réel, pour
pouvoir le dire.

%— 262 
\section{Géométrie (esprit de —)}
%GÉOMÉTRIE (ESPRIT DE —)
L’art de raisonner juste, explique Pascal, sur
des principes « palpables mais éloignés de
l'usage commun » : une fois qu’on les voit, « il faudrait avoir l'esprit tout à fait
faux pour mal raisonner sur des principes si gros qu’il est presque impossible
qu’ils échappent » ({\it Pensées}, 512-1). S’oppose à l'esprit de finesse (voir ce mot).

\section{Gloire}
%GLOIRE 
Ce serait la confirmation de notre valeur par le grand nombre de
ceux qui en ont moins et le reconnaissent : comble de lélitisme (la
gloire ne va qu’à quelques-uns) en même temps que de la démagogie (elle
dépend de tous), et qui trouve ses limites dans cette contradiction. Qu'elle
nous tente, c’est ce que Descartes, Pascal et Spinoza ont assez expliqué : c’est
l'amour de soi, mais comblé par les louanges dont on imagine être l’objet. C’est
mettre l'humanité plus haut que tout, comme il convient, et soi plus haut que
tout autre. C’est un humanisme narcissique.

Grandeur de l’homme : de pouvoir admirer. Misère de l’homme : d’avoir
besoin de l'être. « La plus grande bassesse de l’homme est la recherche de la
gloire, écrit Pascal, mais c’est cela même qui est la plus grande marque de son
excellence ; car, quelque possession qu’il ait sur la terre, quelque santé et commodité
essentielle qu’il ait, il n’est pas satisfait s’il n’est dans l'estime des
hommes » ({\it Pensées}, 470-404). La gloire, tant qu’on y rêve, fait comme un
simulacre de salut. Son seul avantage, pour qui l’atteindrait de son vivant, serait
qu’elle nous guérirait de la désirer, peut-être. On dit aussi qu’elle sauve de la
mort ; mais c’est l’inverse qui est vrai.

\section{Gnose}
%GNOSE
Doctrine religieuse des premiers siècles de l’ère chrétienne, peut-être
d'inspiration platonicienne (voire manichéenne : ce monde est
le mal, tout le bien vient d’ailleurs), qui voudrait assurer le salut par la connaissance
{\it (gnosis)} de Dieu, telle qu’elle est transmise aux initiés par une tradition
primordiale et secrète. Le gnosticisme est un ésotérisme. C’est une superstition
de la connaissance.

L'Église considère le gnosticisme comme une hérésie. De fait, les commentateurs
y voient souvent une contamination du christianisme par lhellénisme.
Ce peut être aussi l'inverse, et c’est en quoi Simone Weil, à bien des égards,
relève de ce courant. Le gnosticisme, qu’on retrouve dans d’autres religions et
à d’autres époques, se reconnaît presque toujours à la haine du monde, du
corps ou de soi. Le gnostique ne veut sauver que son esprit, et par l'esprit seul.
D'où le paradoxe de la gnose, qui est d’être une sotériologie pessimiste. Le
monde est une prison ; le gnostique ne trouve de salut que dans la fuite.

%— 263 —
\section{Gnoséologie}
%GNOSÉOLOGIE
L'étude ou la philosophie de la connaissance {\it (gnôsis)}. Plus
abstraite que l’épistémologie (qui porte moins sur la
connaissance en général que sur les sciences en particulier). Le mot vaut surtout
par l'adjectif {\it gnoséologique}, qui est commode et n’a guère de synonyme. Le
substantif, en français, reste rare : les philosophes parleront plus volontiers de
théorie de la connaissance.

\section{Goût}
%GOÛT
C'est la faculté de juger du beau et du laid, du bon et du mauvais,
comme un plaisir qui serait critère de vérité. Le goût touche au
corps, par la sensation, et à l'esprit, par la culture. Il s’'éduque ; il ne se crée pas.
Le goût prétend à l’universel (jai le sentiment que tout le monde devrait trouver
beau, en droit, ce que je juge être tel), tout en restant subjectif (je n’ai aucun moyen
d'obtenir, en fait, l'accord de tous). C’est ce qui le voue presque inévitablement au
conflit et à la polémique. Il ne s’agit pas de tout aimer, de tout admirer, encore
moins de faire semblant. « Le vrai goût, disait Auguste Comte, suppose toujours un
vif dégoût. » Et Kant, plus profondément : « Une obligation de jouir est une évidente
absurdité. » Le goût ne se commande pas, puisque c’est lui qui commande.

Ainsi le plaisir a toujours raison, mais ne prouve rien. On peut {\it discuter} du
goût (prétendre à l’assentiment nécessaire d’autrui), observe Kant, non {\it disputer}
à son sujet (décider par des preuves). C’est ce qu’on oublie presque toujours, et
qui nous voue, en un autre sens, aux disputes.

\section{Gouvernment}
%GOUVERNEMENT
Ce n'est pas le souverain (qui fait la loi), mais un
« corps intermédiaire », comme disait Rousseau, « chargé
de l'exécution des lois et du maintien de la liberté » ({\it Contrat social}, III, 1). C’est
donc le pouvoir exécutif, ou plutôt son sommet. Il ne gouverne légitimement
qu’à la condition d’obéir. « Le gouvernement reçoit du souverain, écrit excellemment
Rousseau, les ordres qu’il donne au peuple. » Dans une démocratie,
cela signifie qu’il doit être soumis, d’une manière ou d’une autre, au suffrage
universel et au contrôle du parlement.

\section{Grâce}
%GRÂCE
Un don sans raison, sans condition, sans mérite. Le réel en est une,
tant qu’il nous épargne ou nous comble, et la seule.

\section{Grandeur}
%GRANDEUR
Une quantité quelconque, mais perçue positivement, voire
emphatiquement (c’est le con\-traire de la petitesse). C’est
%— 264 —
pourquoi on se sert de l’expression {\it grandeur d'âme} pour traduire la {\it megalopsuchia}
où {\it magnanimitas} des Anciens. C’est que la quantité semble ici valoir
comme qualité. « Il n’y a point d'âme vile, disait Alain, mais seulement un
manque d'âme. »

\section{Gratitude}
%GRATITUDE
C'est le souvenir reconnaissant de ce qui a eu lieu : souvenir
d’un bonheur ou d’une grâce, et bonheur lui-même, et grâce
renouvelée. C’est par quoi c’est une vertu : parce qu’elle se réjouit de ce qu’elle
doit, quand l’amour-propre préférerait l'oublier.

La gratitude porte sur ce qui fut, en tant que ce qui fut demeure. C’est la
joie de la mémoire, et le contraire de la nostalgie : il s’agit d’aimer le passé, non
en tant qu’il manque (nostalgie), mais dans sa vérité toujours présente, qui ne
manque jamais. C’est Le temps retrouvé, et Proust a bien montré ce qui s’y joue
de joie ou d’éternité. La mémoire fait comme un port, dans la tempête de vivre.
Au contraire, disait Épicure, « la vie de l’insensé est ingrate et inquiète : elle se
porte tout entière vers l'avenir ». La gratitude est le contraire de la nostalgie, et
l'inverse de l’espérance.

\section{Gratuit}
%GRATUIT
Ce n’est pas ce qui n’a pas de prix (gratuité n’est pas dignité),
mais ce pour quoi on n’exige pas d’être payé ou récompensé — ce
qui est disponible sans échange ou offert sans contrepartie. On parle aussi
d’{\it acte gratuit}, depuis Gide, pour désigner un acte sans motif. Ce serait bien sûr
se tromper que d’y voir un acte libre : à supposer qu’il soit en effet sans motif,
il n’est pas pour autant sans causes (car alors il n’existerait pas). Et il n’est pas
besoin non plus de travailler {\it gratuitement} pour travailler {\it librement}... Ainsi le
gratuit n’est pas le libre, mais le désintéressé : marque, selon les cas, d’indifférence,
de surabondance, de générosité, ou de folie.

\section{Gravité}
%GRAVITÉ
Je n’aime pas trop qu’on en dise du mal. Tout le monde ne peut
pas être Mozart, et la légèreté n’y a jamais suffi. Au reste, la gravité,
malgré l’étymologie, n’est pas la lourdeur. Elle est plutôt une sensibilité à
ce qui pèse, au poignant des choses et de la vie. « Le vice, la mort, la pauvreté,
les maladies, sont sujets graves, écrit Montaigne, et qui grèvent » ({\it Essais}, III, 5,
p. 841). C’est comme un tragique quotidien, mais sans emphase, sans grandiloquence,
sans {\it affolement}, comme dit encore Montaigne — un tragique qui ne
se prend pas au tragique, mais qui ne parvient pas non plus à en rire. On ne la
%— 265 —
confondra pas avec l’esprit de sérieux : son contraire n’est pas l'humour, mais
la frivolité.

\section{Groupe}
%GROUPE
Un ensemble d'individus en interaction, de telle sorte qu'il y a
plus dans le groupe que la simple addition des comportements
individuels. Notion par nature relative, et même floue, qu’on ne peut guère
préciser que par différence. C’est à la fois moins (d’un point de vue quantitatif)
et plus (d’un point de vue qualitatif) qu’une foule : c’est comme une foule
limitée, unifiée, presque toujours hiérarchisée. Par exemple dans un stade, lors
d’un match de football : parmi la {\it foule} des spectateurs, il y a des {\it groupes} de supporteurs,
et deux groupes de joueurs (les équipes).

\section{Guerre}
%GUERRE
«La guerre, écrivait Hobbes, ne consiste pas dans un combat
effectif, mais dans une disposition avérée, allant dans ce sens,
aussi longtemps qu’il n’y a pas d’assurance du contraire. Tout autre temps se
nomme paix » ({\it Léviathan}, I, 13). Cela, qui distingue la guerre de la bataille,
suggère assez que la guerre, entre les États, est la disposition première : la guerre
est donnée ; la paix, il faut la {\it faire}. C’est ce qui donne raison aux pacifiques,
sans donner tort aux militaires.

On remarquera que le but d’une guerre est ordinairement la victoire, qui
est une paix avantageuse. Que le droit y trouve aussi son compte n’est jamais
garanti, mais peut seul la justifier. Une guerre juste ? Elle peut l’être par ses
buts, jamais totalement par ses moyens. Le mieux, presque toujours, est de
“éviter : le rapport violent des forces (la guerre) n’est légitime que lorsque leur
rapport non violent (la politique) serait suicidaire ou indigne.
% 266

\section{Habitude}
%HABITUDE
La facilité qui naît de la répétition. Un acte qu’on accomplit
souvent devient ainsi presque instinctif, remarquait déjà Aristote
({\it Rhétorique}, I, 11), et c’est pourquoi on dit souvent que l’habitude est une
seconde nature : c’est comme une nature acquise, qui viendrait corriger la première
ou prendre sa place. Reste à savoir, observe Pascal, si cette nature elle-même
n’est pas une première habitude ({\it Pensées}, 126-93).

L'habitude, en diminuant la difficulté, rend la conscience moins nécessaire.
Elle peut même s’en passer tout à fait, abandonnant le corps, pour ainsi dire, à
lui-même. C’est une « spontanéité irréfléchie » (Ravaisson, {\it De l'habitude}), qui
permet de penser à autre chose. Ainsi le virtuose, libéré des notes, n’a plus à
s’occuper que de la musique.

Cabanis et Destutt de Tracy ont souligné que les effets de l’habitude sont
contrastés, qui peuvent aussi bien développer une faculté que l’engourdir : une
oreille exercée entendra ce qu’une autre n’entendrait pas, comme un nez exercé
percevra des arômes pour d’autres imperceptibles ; mais on finit par ne plus
entendre un bruit trop habituel, par ne plus sentir une odeur trop constante.
« L’accoutumance accroît toutes les habitudes actives, remarquait déjà Hume,
et affaiblit les habitudes passives » ({\it Traité...}, II, 3, 5 ; il semble que l’idée
vienne de Joseph Butler). C’est ce qui justifiera la distinction, chez Maine de
Biran, entre les impressions passives (les sensations : voir, entendre, sentir, toucher...)
et les impressions actives (les perceptions : regarder, écouter, humer,
palper...). L'habitude affaiblit ou obscurcit les premières, avive ou précise les
secondes. Elle fait qu’on entend moins (par exemple le tic-tac du réveil ou le
bruit d’une autoroute) et qu’on écoute mieux (par exemple une musique ou un
souffle). Aussi peut-elle servir de réactif pour dissocier en nous ce qui est passif
(que Maine de Biran rattachera au corps) de ce qui est actif (qu’il rattachera au
%— 267 —
moi ou à la volonté: {\it De l'influence de l'habitude}, chap. I et II). C’est où le
thème de l’habitude, d’abord empiriste, chez Condillac et Hume, devient spiritualiste.
Cela continuera jusqu’à Ravaisson : « La réceptivité diminue, la
spontanéité augmente ; telle est la loi générale de l'habitude » (Ravaisson,
{\it op. cit.}, II). C’est l’esprit qui redescend dans la nature, ou le devenir esprit de la
spontanéité naturelle. Dualisme ? Ce n’est pas si sûr. Spiritualisme ? Ce n’est
qu’une possibilité parmi d’autres. Il se pourrait aussi bien, et peut-être mieux,
que l’âme ne soit qu’une habitude du corps.

\section{{\it Habitus}}
%{\it HABITUS}
Une manière d’être et d’agir (une disposition), mais acquise et
durable. Le mot, réactualisé par Bourdieu, sert surtout aux
sociologues. Un {\it habitus}, en ce sens, c’est comme une idéologie incarnée et
génératrice de pratiques : c’est notre façon d’être nous-mêmes et d’agir comme
nous agissons, mais en tant qu’elle résulte de notre insertion dans une société
donnée, dont nous incorporons inconsciemment les structures, les clivages, les
valeurs, les hiérarchies.. Par quoi chacun fait librement, ou en tout cas volontairement,
ce qu’il est socialement déterminé à vouloir.

\section{Haine}
%HAINE
« La seule chose universelle, me dit un jour Bernard Kouchner,
c’est la haine ! » Il revenait d’une de ses expéditions humanitaires,
au fin fond de l’horreur et du monde. La seule ? Je n’irais pas jusque-là. Mais
que la haine soit universelle, en effet, partout présente, partout agissante, c’est
ce que trop de massacres ne cessent de nous confirmer. Reste à la penser, pour
essayer d’en sortir ou de s’en protéger. Qu'est-ce que la haine ? « Une tristesse,
répondait Spinoza, qu’accompagne l’idée d’une cause extérieure » ({\it Éthique}, III,
13, scolie, et déf. 7 des Affects). Haïr, c’est {\it s'attrister de}. Or c’est la joie qui est
bonne : toute haine, par définition, est mauvaise. C’est aussi ce qui la rend
mortifère. Celui qui hait, ajoute Spinoza, « s'efforce d’écarter et de détruire la
chose qu’il a en haine » — parce qu’il préfère la joie, comme tout le monde,
autrement dit par amour. Mais c’est un amour malheureux, qui en veut à
l’autre de son propre échec. Ainsi toute haine, même justifiée, est injuste.

\section{Hallucination}
%HALLUCINATION
C'est percevoir ce qui n’est pas. Mais comme nous
n'avons aucun moyen de savoir ce qui est qu’à la
condition de le percevoir, directement ou indirectement, nous n’avons non
plus aucun moyen de distinguer absolument la perception de l’hallucination,
sinon en confrontant nos perceptions à celles d’autrui ou au souvenir de nos
%— 268 —
perceptions passées. Encore cela ne nous dit-il pas si c’est l’hallucination qui est
une perception pathologique, ou si c’est la perception qui est une hallucination
collective et durable... On ne peut en décider par des preuves, et cela n’a pas
tant d'importance. Ce que tout le monde perçoit fait partie du réel commun,
quand bien même il n’aurait d’autre réalité que cette perception (Berkeley). Ce
que je perçois seul, quand les autres devraient le percevoir, est réputé
hallucinatoire : c’est un réel privé, mais qu’on ne sait pas tel, comme un monde
intérieur qu’on prendrait abusivement pour l’autre. « Il y a pour les éveillés un
monde unique et commun, disait Héraclite, mais chacun des endormis se
détourne dans un monde particulier. » L’hallucination est comme un rêve
éveillé ; le rêve, comme une hallucination endormie.

\section{Harmonie}
%HARMONIE
C'est un accord heureux ou agréable, entre plusieurs éléments
simultanés mais indépendants les uns des autres : par exemple
entre plusieurs sons (l'harmonie s’oppose alors à la mélodie, qui unit des sons
successifs), entre plusieurs couleurs, entre plusieurs individus. Leibniz parlait
d'{\it harmonie préétablie} entre l'âme et le corps ; c’est qu’il refusait que ces deux
substances puissent agir l’une sur l’autre et constatait pourtant, comme tout le
monde, leur singulier accord (je veux lever le bras : mon bras se lève). L'âme
et le corps seraient donc comme deux horloges, l’image est de Leibniz, tellement
bien fabriquées et réglées au départ qu’elles seront toujours d’accord par
la suite, sans qu’on ait besoin pour cela de supposer entre elles quelque relation
causale que ce soit. Cette théorie, aussi difficile à accepter qu’à réfuter, explique
que l'expression ait souvent pris, par la suite, un sens péjoratif : une harmonie
préétablie serait une espèce de miracle originel, trop étonnant pour qu’on
puisse y croire. C’est reconnaître que l’harmonie n’est jamais le plus probable,
ce qui explique qu’elle soit rarement donnée au départ. Elle résulte d’un travail
ou d’une adaptation, plus souvent que de la chance.

\section{Hasard}
%HASARD
Ce n’est ni l’indétermination ni l'absence de cause. Quoi de plus
déterminé qu’un dé qui roule sur une table ? Le six sort ? C’est là
un effet, qui résulte de causes très nombreuses (le geste de la main, l'attraction
terrestre, la résistance de l'air, la forme du dé, sa masse, son angle de contact
avec la nappe, ses frottements contre elle, ses rebonds, son inertie...). Si l'on
juge pourtant légitimement que le {\it six} est sorti {\it par hasard}, c’est que ces causes
sont trop nombreuses et trop indépendantes de notre volonté pour qu’on
puisse, lorsqu'on jette le dé, choisir ou prévoir le résultat qu’on obtiendra. Ainsi
le hasard est une détermination imprévisible et involontaire, qui résulte de la
%— 269 —
rencontre de plusieurs séries causales indépendantes les unes des autres, comme
disait Cournot, rencontre qui échappe pour cela à tout contrôle comme à toute
intention. Ce n’est pas le contraire du déterminisme : c’est le contraire de la
liberté, de la finalité ou de la providence.

Un autre exemple ? On peut reprendre celui de Spinoza, dans l’Appendice
de la première partie de l'{\it Éthique}. Une tuile tombe d’un toit. Il y a à cela des
causes (le poids de la tuile, la pente du toit, le vent qui soufflait, un clou rongé
par la rouille, qui finit par céder), dont chacune s'explique à son tour par une
ou plusieurs autres, et ainsi à l'infini. Vous étiez, à ce moment précis, sur le
trottoir, juste à la verticale du toit. Cela s'explique aussi, ou peut s'expliquer,
par un certain nombre de causes : vous alliez à un rendez-vous, vous aviez
choisi l'itinéraire le plus simple, vous pensiez que la marche à pied vous ferait
du bien. Ni la chute de la tuile ni votre présence sur le trottoir ne sont donc
sans causes. Mais les deux séries causales (celle qui fait tomber la tuile, celle qui
vous amène où vous êtes), outre leur complexité propre, qui suffirait à les
rendre hasardeuses, sont indépendantes l’une de l’autre : ce n’est pas parce que
la tuile tombe que vous êtes là, ni parce que vous êtes là qu’elle tombe. Si elle
vous brise le crâne, vous serez donc bien mort par hasard : non parce qu’il y
aurait (à une exception au principe de causalité, mais parce que celui-ci s’est
exercé de façon irréductiblement multiple, imprévisible et aveugle. Ou bien il
faut imaginer un Dieu qui aurait voulu ou prévu la rencontre de la tuile et de
votre crâne. La providence est un anti-hasard, et le hasard une anti-religion.

Le hasard se calcule, mais dans sa masse plutôt que dans son détail. C’est ce
qui permet aux assureurs de mesurer les risques que nous prenons, et qu’ils
prennent : un accident de voiture, aussi imprévisible qu’il puisse être, fait partie
d’une série (le nombre d’accidents dans une période donnée) qui peut se prévoir
à peu près. C’est vrai aussi pour les jeux de hasard. S’il est impossible, sauf
trucage, de prévoir le résultat d’un seul coup de dé, il est facile de calculer la
répartition statistique de coups très nombreux : chacune des six possibilités se
vérifiera, si vous jouez assez longtemps, environ une fois sur six, et s’approchera
d’autant plus de cette moyenne que la série sera plus longue. C’est pourquoi la
chance ne dure pas toujours, ni la malchance, du moins dans les phénomènes
qui ne dépendent que du hasard. Simplement la vie ne dure pas assez longtemps,
et est soumise à des causes trop lourdes et trop constantes, pour que le
hasard vienne toujours corriger l'injustice.

Toute vie n’en est pas moins hasardeuse, dans son détail comme dans son
principe. La naissance de chacun d’entre nous, quelques années avant notre
conception, était d’une probabilité extrêmement faible ; comme c’est vrai aussi
des naissances de nos parents, de nos grands-parents, etc., qui conditionnent la
nôtre, il en résulte que notre existence, il y a quelques siècles, était d’une probabilité
%— 270 —
quasi nulle, comme celle, si l’on prend assez de recul, de tout événement
contingent. C’est en quoi tout réel, aussi banal qu’il soit, a quelque chose
de rétrospectivement surprenant, qui tient au fait qu’il était non seulement
imprévisible à l’avance mais hautement improbable : c’est l’exception du possible.
L'univers fait une espèce de loterie, dont le présent serait le gros lot.
D’aucuns s’étonnent que ce soit justement ce numéro-là qui soit sorti, alors
que c'était tellement improbable... Mais qu'aucun numéro ne sorte, une fois la
loterie lancée, l'était bien davantage.

\section{Hédonisme}
%HÉDONISME
Toute doctrine qui fait du plaisir ({\it hèdonè}) le souverain bien
ou le principe de la morale : ainsi chez Aristippe, Épicure
(quoique son hédonisme se double d’un eudémonisme), ou aujourd’hui chez
Michel Onfray. Ce n’est pas forcément un égoïsme, puisqu'on peut prendre en
compte le plaisir des autres ; ni un matérialisme, puisqu'il peut exister des plaisirs
spirituels. C’est même le point faible de l’hédonisme : la doctrine n’est
acceptable qu’à la condition de donner au mot {\it plaisir} une extension tellement
vaste qu’il ne veut plus dire grand-chose. Je veux bien que celui qui meurt sous
la torture, plutôt que de dénoncer ses camarades, agisse {\it pour le plaisir} (ou pour
éviter une souffrance plus grande : celle d’avoir trahi, celle de ses camarades qui
seraient autrement torturés à leur tour, celle de la défaite....). Mais alors l’hédonisme
n’est qu’une espèce de théorie passe-partout, qui perd sa vertu discriminante.
Si tout le monde en relève, à quoi bon s’en réclamer ?

La maxime de l’hédonisme est bien énoncée par Chamfort : « Jouis et fais
jouir, sans faire de mal ni à toi ni à personne, voilà, je crois, toute la morale »
({\it Maximes}..., 319). Formule sympathique, et même vraie pour une bonne part,
mais courte. C’est ériger le principe de plaisir (qui ne se veut que descriptif) en
éthique (qui serait normative). Mais comment ce principe, dans son universalité
simple, pourrait-il suffire ? Reste à savoir quels plaisirs, et pour qui, peuvent
justifier quelles souffrances. Il faut donc choisir {\it entre les plaisirs}, comme disait
Épicure, et il est douteux que le plaisir, moralement, y suffise. Combien de
salauds jouisseurs ? Combien de souffrances admirables ? Et un mensonge qui
ne fait de mal à personne, fût-il même agréable pour tout le monde (vous vous
vantez d’un exploit que vous n’avez jamais accompli : vos auditeurs sont
presque aussi contents que vous), en quoi en est-il moins méprisable ? On me
répondra que le mépris est une espèce de déplaisir, et que mon exemple
confirme par là l’hédonisme qu’il prétend réfuter.. J'y consens, mais cela
redouble plutôt mes réticences. L’hédonisme est aussi irréfutable qu’insatisfaisant :
il n'échappe au paradoxe que pour tomber dans la tautologie.

%— 271 —
\section{Héraclitéisme}
%HÉRACLITÉISME
La pensée d'Héraclite, et toute pensée qui en reprend la
thèse centrale : qu’il n’y a pas d'êtres immuables, que
tout change, que tout coule {\it (« Panta rhei »)}, que tout est devenir. Ainsi peut-on
parler de l’héraclitéisme de Montaigne. « Je ne peins pas l'être, disait-il, je
peins le passage. » C’est le seul être qui nous soit accessible. « Le monde n’est
qu'une branloire pérenne. Toutes choses y branlent sans cesse : la terre, les
rochers du Caucase, les pyramides d'Égypte, et du branle public et du leur. La
constance même n'est autre chose qu’un branle plus languissant » ({\it Essais}, III,
2). C’est le contraire de l’éléatisme, ou sa vérité {\it sub specie temporis}.

\section{Herméneute}
%HERMÉNEUTE
Au sens général : celui qui interprète, c’est-à-dire qui
cherche le sens de quelque chose (d’un signe, d’un discours,
d’un événement). En un sens plus restreint, j’appelle {\it herméneute} toute
personne qui prend le sens absolument au sérieux : qui veut en rendre raison
par lui-même ou par un autre sens, au lieu de l’expliquer par ses causes, qui ne
signifient rien. C’est supposer un sens ultime ou infini (un sens du sens), qui
serait la vérité même. Mais si la vérité voulait dire quelque chose, elle serait
Dieu. Toute herméneutique, en ce sens restreint, est religieuse, ou tend à le
devenir : ce n’est qu’une superstition du sens.

\section{Héroïsme}
%HÉROÏSME
C’est un courage extrême et désintéressé, face à tous les maux
réels ou possibles.
Ce courage-là ne résiste pas seulement à la peur, mais aussi à la souffrance,
à la fatigue, à l'abattement, au dégoût, à la tentation. Vertu d’exception, pour
des individus d’exception. Nul n’est tenu d’être un héros, et c’est ce qui rend
les héros admirables.

\section{Hétéronomie}
%HÉTÉRONOMIE
C’est être soumis à une autre loi que la sienne propre, et
le contraire par là de l’autonomie (voir ce mot). S’applique
spécialement, chez Kant, à toute détermination de la volonté par autre
chose que la loi qu’elle se fixe à elle-même (la loi morale), par exemple par tel
ou tel objet de la faculté de désirer (le plaisir, le bonheur...) ou par tel ou tel
commandement extérieur, fût-il par ailleurs légitime. Obéir à ses penchants,
c’est être esclave. Obéir à l’État ou à Dieu ? On ne le peut de façon autonome
qu’à la condition de leur obéir par devoir, non par crainte ou par espérance. On
n'a le droit d’obéir (hétéronomie) qu’à la condition de se gouverner (autonomie).

%— 272 —
\section{Heureux}
%HEUREUX
Être heureux, étymologiquement, c’est avoir de la chance. Cela
en dit long sur le bonheur : non que la chance y suffise, mais
parce que aucun bonheur, sans elle, n’est possible. Tu es heureux ? C’est
d’abord que rien ne t’en empêche, qui serait plus fort que toi. Et comment
serais-tu plus fort que tout ?

Être heureux, c’est aussi avoir le sentiment de l’être. C’est ce que Marcel
Conche appelle le {\it cogito eudémonique} de Montaigne : « Je pense être heureux,
donc je le suis. » Cela laisse une marge à l’évaluation, au travail sur soi, à la philosophie.
Plutôt que de regretter toujours ce qui n’est pas (« Quel malheur de
n'être pas heureux ! »), apprends plutôt, quand c’est possible, à jouir de ce qui
est (« Quel bonheur de n être pas malheureux ! »).

C’est aussi une question de tempérament. Heureux ceux qui sont doués
pour le bonheur !

\section{Heuristique}
%HEURISTIQUE
Qui concerne la recherche ou la découverte ({\it heuriskein} :
trouver, de {\it heuris}, le flair). Par exemple une hypothèse
heuristique est une hypothèse qui prétend moins résoudre un problème que
permettre de le poser autrement, et mieux : elle ne propose pas une solution,
elle aide à la chercher.

\section{{\it Hic et nunc}}
%{\it HIC ET NUNC}
Ici et maintenant, en latin. C’est la situation de tout être, de
tout sujet, de tout événement : son ancrage singulier dans
l’'universel devenir. Même la mémoire et l'imagination n’y échappent pas (se
souvenir d’un passé, imaginer un ailleurs ou un avenir, c’est toujours s’en souvenir
ou les imaginer {\it ici et maintenant}). Nos utopies sont datées, autant que
nos émotions, et vieillissent plus mal.

\section{Hiérarchie}
%HIÉRARCHIE
C’est un classement normatif, qui instaure des liens de
domination, de dépendance ou de subordination. La norme
est le plus souvent de puissance ou d’argent. C’est ce qu’on appelle la hiérarchie
sociale, qui rend l’idée même de hiérarchie suspecte. Si tous les hommes sont
égaux en droit et en dignité, comment pourrait-on les classer par ordre
d'importance ou de valeur ? Mais c’est que la hiérarchie, quand elle est légitime,
porte moins sur les êtres que sur les fonctions ou les œuvres. Ainsi dans
l'État ou dans un parti, dans une entreprise ou une Église, en art ou en sport.
Que tous les hommes soient égaux en droit et en dignité, cela ne signifie pas
que tous les pouvoirs le soient, ni toutes les responsabilités, ni tous les talents.

%— 273 —
L'erreur, à quoi pousse l’étymologie, est d’y voir du sacré {\it (hiéros)}, quand il ne
s’agit que d'organisation ou d'efficacité. Le protocole, qui met en scène cette
hiérarchie, le montre bien : ce n’est qu’une apparence réglée, qui ne dit rien sur
la valeur des individus mais qui manifeste quelque chose de l'institution ou des
pouvoirs. Pascal, comme toujours, va au plus court : « M. N... est un plus
grand géomètre que moi ; en cette qualité il veut passer devant moi. Je lui dirai
qu'il n’y entend rien. La géométrie est une grandeur naturelle ; elle demande
une préférence d’estime ; mais les hommes n’y ont attaché aucune préférence
extérieure. Je passerai donc devant lui, et l’estimerai plus que moi en qualité de
géomètre » ({\it Trois discours...}, 2). La hiérarchie des pouvoirs n’est pas celle des
talents, comme celle de la naissance n’est pas celle des vertus. Pascal encore :
« Il n’est pas nécessaire, parce que vous êtes duc, que je vous estime ; mais il est
nécessaire que je vous salue. » Il n’y a pas de hiérarchie absolue. Cela, loin de
les annuler toutes, justifie leur pluralité. C’est vrai spécialement en démocratie :
que tous les citoyens soient égaux, cela ne dit rien sur leurs fonctions ou leurs
mérites respectifs, ni ne dispense d’admirer les uns et d’obéir à d’autres. Ainsi
l’idée de hiérarchie revient toujours, mais au pluriel et sans qu’on puisse jamais
en absolutiser une.

\section{Histoire}
%HISTOIRE
L'ensemble non seulement de tout ce qui arrive (le monde),
mais aussi de tout ce qui est arrivé et arrivera : la totalité diachronique
des événements. C’est en ce sens qu’on parle d’une histoire de l’univers,
qui serait la seule histoire universelle. En pratique, le mot a pourtant rarement
une extension aussi large : sauf précision particulière, il ne désigne que le
passé humain, et la connaissance de ce passé. De là deux sens différents, que le
français ne distingue pas : il y a l’histoire réelle ({\it Geschichte} en allemand : ce
qu'on appelait en latin les {\it res gestas}, les faits accomplis, le passé tel qu’il fut,
l’histoire des hommes historiques) et l’histoire comme discipline (qu’on appelle
parfois {\it Historie} en allemand : {\it l’historia rerum gestarum}, la connaissance du
passé, l’histoire des historiens). On ne connaît la première que par la seconde ;
mais la seconde n’existe que par et dans la première.

L’histoire a-t-elle un sens ? La science historique a celui qu’on lui prête ou
qu'on y trouve : faire de l’histoire, cela peut viser un certain but ou signifier
quelque chose, qui variera en fonction des historiens. Mais l’histoire réelle ?
Quel sens pourrait-elle avoir ? Qu’on le prenne comme but ou comme signification,
ce sens ne pourrait être qu'autre chose que l’histoire (comment aller
vers soi ? comment se signifier soi ?), qui serait son message ou sa fin. Mais
l’idée d’une fin de l’histoire est contradictoire ou absurde : elle ne peut exister
ni si l’histoire continue (car sa fin alors n’en serait pas une) ni si elle s’arrête (car
%— 274 —
sa fin alors ne serait pas sienne, et ne ferait pas sens). Quant à penser une signification
de l’histoire, c’est lui supposer un sujet, qui veuille dire, à travers elle,
quelque chose. Mais si ce sujet est {\it dans} l'histoire, comment ce qu’il veut dire
serait-il le sens de l’histoire, puisqu'il en fait partie ? Et comment, s’il est au-dehors,
pourrait-il s'y exprimer ? On dira qu’il en est ainsi de tout sens. Mais
non, puisque le sens d’une phrase, par exemple, n’est ni son point final ni une
partie de cette phrase : son sens est un dehors, qui est visé de l’extérieur par
celui qui l’énonce (lequel ne fait pas partie de la phrase). Mais hors de l’histoire,
quoi, et pour quel locuteur ? « Le sens du monde, disait Wittgenstein, doit se
trouver hors du monde. » Le sens de l’histoire, pareillement, ne peut exister
qu’en dehors d’elle. C’est ce qu’on appelle Dieu, lorsqu'on y croit : ce n’est plus
histoire mais théodicée. Pour les autres, ceux qui n’adorent aucun Dieu, on
peut dire de l’histoire ce que Shakespeare disait de toute vie : « C’est une histoire
pleine de bruit et de fureur, racontée par un idiot, et qui ne signifie rien. »
Cela ne nous empêche pas, en elle, de viser tel ou tel but, ni de tenir tel ou tel
discours. Mais nous interdit de croire que c’est elle qui parle ou vise à travers
nous. Que signifie la Guerre de 1914 ? Quel but visait-elle ? Aucun, bien sûr,
puisque les individus qui la firent poursuivaient eux-mêmes des buts différents,
qui donnaient à leur existence, bien souvent, des significations opposées. Il en
va de même pour tout événement. La Révolution française ? La Révolution
russe ? Ceux qui les firent visaient sans doute un but ou un sens, mais pas
davantage que ceux qui les combattirent. C’est ce qu'avait vu Engels :
« L'histoire se fait de telle façon que le résultat final se dégage toujours des
conflits d’un grand nombre de volontés individuelles, dont chacune à son tour
est faite telle qu’elle est par une foule de conditions particulières d’existence ; il
y a donc là d'innombrables forces qui se contrecarrent mutuellement, un
groupe infini de parallélogrammes de forces, d’où ressort une résultante — l’événement
historique — qui peut être regardée elle-même, à son tour, comme le
produit d’une force agissant comme un tout, de façon inconsciente et aveugle.
Car ce que veut chaque individu est empêché par chaque autre, et ce qui s’en
dégage est quelque chose que personne n’a voulu » ({\it Lettre à Joseph Bloch}, du
21 septembre 1890). C’est précisément parce que tout individu, dans l’histoire,
poursuit un but, ou plusieurs, qu’il est exclu que l’histoire elle-même veuille
aller quelque part ou signifier quelque chose. Si tout sujet est historique, comment
l’histoire serait-elle sujet ? Si tout sens est dans l’histoire, comme l’histoire
elle-même pourrait-elle en avoir un ? Cela ne nous empêche pas d’y poursuivre
des buts, répétons-le, ni même de les atteindre parfois. Mais le sens alors qui
s’en dégage n’est pas celui de l’histoire ; c’est celui de notre action. Mieux vaut
un militant qu’un prophète.

%— 275 —
\section{Historicisme}
%HISTORICISME
C'est vouloir tout expliquer par l’histoire. Mais si l’histoire
produit tout, y compris les explications qu’on en donne,
que valent ces explications, et que vaut l’historicisme ?

Il faut que l’histoire soit rationnelle, ou bien que la raison soit historique :
rationalisme ou historicisme. Les deux ne peuvent être vrais ensemble et totalement.
Mais ils peuvent l’être l’un et l’autre dans des ordres différents : rationalisme
dans l’ordre théorique (contre la sophistique), historicisme dans l’ordre
pratique (contre le dogmatisme pratique). Toute valeur est historique ; toute
vérité, éternelle. Qu’on ne puisse connaître totalement celle-ci, ce n’est pas une
objection contre elle — puisque toute objection la suppose. Qu’on ne puisse
totalement se défaire de celle-là, et qu’on ne le doive, ce n’est pas davantage une
objection contre l’historicisme : l’histoire, en effet, peut l’expliquer, par
l'impossibilité où nous sommes d’en sortir. L’historicisme et le rationalisme
peuvent ainsi s’articuler l’un à l’autre : l’histoire explique tout, sauf ce qu’il y a
de vrai dans nos explications.

\section{Holisme}
%HOLISME
Une pensée qui accorde davantage d’importance au tout ({\it holos},
entier) qu’à ses parties, ou qui s’interdit de réduire un ensemble
aux éléments qui le composent. Appliqué à la société, le holisme s’oppose à
l’individualisme.

\section{Hominisation}
%HOMINISATION
L’humanité n’est pas une essence, c’est une histoire, et
cette histoire est d’abord naturelle : l’hominisation est ce
processus biologique par lequel {\it homo sapiens} se distingue progressivement — par
mutations et sélection naturelle — des espèces dont il descend. Reste, ensuite, à
devenir humain, au sens normatif du terme : ce n’est plus {\it hominisation}, mais
{\it humanisation}. La seconde, sans la première, serait impossible. Mais la première,
sans la seconde, serait vaine : cela ne ferait qu’un grand singe de plus.

\section{Homme}
%HOMME
Au sens étroit : un membre de l'espèce humaine, de sexe masculin
et d’âge adulte. Au sens large ou générique : tout être humain,
quel que soit son âge ou son sexe. C’est en ce sens que tous les hommes sont
égaux en droits et en dignité. Cela ne supprime bien sûr pas la différence
sexuelle, pas plus que celle-ci ne porte atteinte à l’unité de l’espèce. Que
l'humanité soit sexuée, c’est au contraire ce qui lui permet d’exister et d’être
humaine. Qu'est-ce, en effet, qu’un être humain ? Un animal qui parle, qui raisonne,
qui vit en société, qui travaille, qui rit, qui crée ? Rien de tout cela,
%— 276 —
puisque la découverte, sur Terre ou sur une autre planète, d’une espèce vivante
douée de langage, de raison, etc., ne changerait pas les limites de l’espèce
humaine ; et puisqu’un débile profond, même incapable de parler, de raisonner,
de travailler, de créer, de rire, et fût-il, tel l'Enfant sauvage d’Itard,
dépourvu de toute socialité, n’en est pas moins {\it homme} pour autant. On peut
faire le même reproche à la fameuse définition de Linné : {\it Animal rationale,
loquens, erectum, bimane}. Elle ne vaut pas pour tous les hommes, et rien ne
prouve qu’elle vaille pour eux seuls. Aucune définition fonctionnelle ou normative
n’est ici acceptable, puisqu’elle reviendrait à exclure de l’espèce humaine
ceux qui sont hors d’état de {\it fonctionner} normalement. Il faut donc un autre critère,
non plus fonctionnel mais générique, non plus normatif mais spécifique.
La biologie en propose un, pour toute espèce animale, qui est l’interfécondité :
un individu appartient à une espèce s’il peut se reproduire par croisement avec
un autre membre de cette espèce et engendrer un être lui-même fécond. Cela
toutefois ne vaut que pour l'espèce, point pour l’individu : un homme stérile
ne cesse pas pour cela d’être un homme. Nous devons donc, si nous voulons un
critère qui soit individuellement opératoire, prendre le problème par l’autre
côté : celui non de l’engendrement, mais de la filiation. On aboutit alors à la
définition suivante, qui me paraît seule valoir pour tout le défini et pour aucun
autre : {\it Est un être humain tout être né de deux êtres humains}. On m’objectera
que cette définition est circulaire, puisqu’elle suppose humanité. Mais c’est
moins une faiblesse définitionnelle qu’un fait de l'espèce, que tout individu
suppose en effet pour pouvoir exister et être défini. Quant à définir l'espèce
elle-même, c’est aux naturalistes de le faire, qui nous apprennent qu’{\it homo
sapiens} fait partie de la classe des mammifères et de l’ordre des primates, dont il
se distingue spécifiquement (quoiqu'il puisse y avoir des exceptions individuelles)
par un certain nombre de caractères génétiques bien connus : un cerveau
plus gros, un pouce opposable aux autres doigts, un larynx apte à la
parole... L’humanité, sans ces caractères biologiques, ne serait pas ce qu’elle
est. Mais ce n’est pas parce qu’il a ces caractères qu’un individu est humain :
c’est parce qu’il est né de deux êtres humains qu’il peut les avoir — et qu’il restera
humain, d’ailleurs, quand bien même tel ou tel de ces caractères lui ferait
défaut. Si c’est la filiation qui fait l’homme, l’humanité ne peut être définie, en
chacun, par ses performances. Comment un handicap, aussi grave soit-il, pourrait-il
annuler ce qu’il suppose ?

On m'objectera aussi le clonage reproductif, qui permettrait d’engendrer
un être humain à partir d’un seul individu. J’y vois moins une objection qu’une
raison forte de refuser le clonage : non parce qu’il invaliderait ma définition
(laquelle pourrait s’en trouver simplement modifiée : Est un être humain,
pourrait-on dire sans difficulté, tout être né d’au moins un être humain), mais
%— 277 —
parce qu’il mettrait en cause l’un des traits les plus précieux de l'humanité, qui
est l’engendrement, par deux individus différents, et parce qu’ils sont différents,
d’un troisième individu, qui ne saurait pour cela être identique à aucun
des deux premiers. Se reproduire, pour un homme ou une femme, ce n’est
jamais se reproduire à l'identique, et c’est très bien ainsi. Un être humain
engendré par clonage, à partir d’un seul individu, appartiendrait assurément à
l'espèce (d’où la définition modifiée que je propose par anticipation) ; mais
l’humanité, si la chose se généralisait, en serait moins riche de différences, et
par là moins humaine : ce n’est pas une définition, qu’il faut sauver, mais ce
qu'il y a d’infiniment diversifié et imprévisible dans l'humanité. Procréer, c’est
créer, non répéter. Le droit d’être différent de ses parents fait partie des droits
de l’homme.

\section{Honnêteté}
%HONNÊTETÉ
C’est la justice à la première personne, telle qu’elle s'impose
surtout dans les rapports de propriété, les échanges, les
contrats : le respect non seulement de la {\it légalité}, dans un pays donné, mais de
l'{\it égalité}, au moins en droit, entre tous les individus concernés. Par exemple si je
vends un appartement sans indiquer à l’acheteur tel vice caché qui s’y trouve,
ou même tel inconvénient du voisinage. Il se peut que je n’aie violé aucune loi
(cela dépend de l’état de la législation : nul vendeur n’est légalement obligé, que
je sache, d’indiquer que son voisin est bruyant ou grossier) ; mais je n’aurai pas
été tout à fait {\it honnête} avec lui, puisque j'aurai profité d’une inégalité (de
connaissance, d’information) entre nous pour en tirer parti à son désavantage.
La règle de l'honnêteté est donc une règle de justice, en tant qu’elle s'impose à
chacun dans ses rapports, spécialement marchands ou contractuels, avec autrui.
Elle est bien énoncée par Alain : « Dans tout contrat et dans tout échange,
mets-toi à la place de l’autre, mais avec tout ce que tu sais, et, te supposant aussi
libre des nécessités qu’un homme peut l'être, vois si, à sa place, tu approuverais
cet échange ou ce contrat » ({\it 81 chapitres...}, VI, 4). L’échange ou le contrat sont
honnêtes quand je peux répondre oui; malhonnête dans le cas contraire.
L’honnêteté est moins fréquente qu’on ne le croit.

En pratique, la notion sert surtout par rapport à la propriété : être honnête,
ce serait respecter la propriété d’autrui ; être malhonnête, s’en emparer indûment.
C’est bien commode pour les propriétaires. Et certes le voleur, même le
plus pauvre, manque à l'honnêteté. Toute personne qui a été cambriolée le sait
bien : qu’on prenne ce qui lui appartient, sans son accord et sans contrepartie,
cela n’est pas juste. Mais combien de riches, même n’ayant jamais volé personne,
même honnêtes en ce sens, attentent — par trop de richesse, par trop
d’égoïsme et de bonne conscience — à la justice ? C’est que l'égalité des hommes
%— 278 —
entre eux ne saurait se cantonner au monde du droit, des échanges, des
contrats. L’honnêteté est une justice de propriétaires, ou vis-à-vis d'eux. Mais
c’est la justice, moralement, qui vaut, non la propriété.

\section{Honneur}
%HONNEUR
La dignité, quand elle passe par le regard des autres. Ou
l’amour-propre, quand il se prend au sérieux. Peut pousser à
l’héroïsme autant qu’à la guerre ou à l'assassinat (les « crimes d’honneur »).
C’est un sentiment foncièrement équivoque, qu’on ne saurait ni admirer ni
mépriser tout à fait. C’est une passion noble ; mais ce n’est qu’une passion,
point une vertu. Qu’on ne puisse socialement s’en passer, jen suis d’accord.
Raison de plus, individuellement, pour s’en méfier. « L’honneur national,
disait Alain, est comme un fusil chargé. » Et que dire de l’honneur de ces adolescents
qui s’entre-tuent, à la porte de nos collèges, pour un regard ou une
injure ? L’honneur a fait plus de morts que la honte, et plus d’assassins que de
héros.

\section{Honte}
%HONTE
Ce n’est pas le sentiment de culpabilité, puisqu’on peut avoir honte
en se sachant innocent. Par exemple parce qu’on souffre, sous le
regard d’autrui, de se sentir ridicule ou pitoyable. Un faux pas, une disgrâce
physique, une tache sur le visage ou un vêtement, peuvent y suffire. Et combien
de femmes, après avoir été violées, disent la honte qu’elles ont ressentie,
non assurément qu’elles se jugeassent coupables, mais parce qu’elles s'étaient
senties humiliées, méprisées, avilies, quoiqu’elles n’y soient pour rien, enfin en
quelque chose atteintes dans leur honneur ou leur dignité. Le jugement compte
moins ici que la sensibilité, la morale moins que la souffrance, la culpabilité
moins que l’amour-propre. On ne peut donc accepter tout à fait, du moins en
français, la définition que proposait Spinoza : « La honte est une tristesse
qu’accompagne l’idée d’une action, dont nous imaginons qu’elle est blâmée par
d’autres » ({\it Éthique}, III, déf. 31 des Affects). Que ce puisse être le cas, dans telle
ou telle honte particulière, n'autorise pas à penser que ce soit le cas de toutes.
On peut avoir honte là où l’on n’a pas agi, là où aucun blâme n’est envisa-
geable, simplement parce que l’image qu’on donne de soi, ou que les autres en
ont, ne correspond pas à celle que l’on voudrait offrir. Certains peuvent avoir
honte de leur corps, de leur misère, de leur inculture, de leurs parents parfois,
non parce qu'ils s’en sentent responsables, mais parce qu’ils se sentent amoin-
dris ou humiliés, sous le regard des autres, d’être dans ce corps, cet état ou cette
famille... Descartes a bien vu que la honte avait à voir avec l’amour de soi, et
que cela, loin de la condamner, pouvait faire, parfois, son utilité ({\it Passions}, III,
%— 279 —
205 et 206). Reste à n’en être pas prisonnier. C’est un amour malheureux ou
blessé, qu’il importe de guérir.

On remarquera qu’on n’a pas honte devant les animaux, ni tout à fait dans
la solitude (ce ne serait plus honte mais remords ou repentir). La honte est un
sentiment de soi à soi, mais par la médiation d’un ou de plusieurs autres. La
honte, souligne Jean-Paul Sartre, est «une appréhension unitaire de trois
dimensions ». J'ai honte lorsque le sujet que je suis se sent objet, pour un autre
sujet : « J'ai honte de {\it moi} devant {\it autrui} » ({\it L'être et le néant}, p. 350). On n’y
échappe qu’en échappant aux regards, par la solitude, ou au statut d’objet, par
l’amour ou le respect. Nietzsche, en trois aphorismes, a peut-être dit l'essentiel :

« {\it Qui appelles-tu mauvais ?} — Celui qui veut toujours faire honte.

{\it Que considères-tu comme ce qu'il y a de plus humain ?} — Épargner la honte à
quelqu'un.

{\it Quel est le sceau de la liberté conquise ?} — Ne plus avoir honte de soi-même »
({\it Le gai savoir}, III, 273-275).

\section{Humanisation}
%HUMANISATION
On naît homme, ou femme ; on devient humain. Ce
processus, qui vaut pour l’espèce autant que pour l’individu,
c’est ce qu’on peut appeler l’humanisation : c’est le devenir humain de
l’homme — le prolongement culturel de l’hominisation.

\section{Humanisme}
%HUMANISME
Historiquement, c’est d’abord un courant intellectuel de la
Renaissance (ceux qu’on appelle les humanistes : Pétrarque,
Pic de La Mirandole, Érasme, Budé.....), fondé sur l'étude des humanités
grecques et latines et débouchant sur une certaine valorisation de l’individu.
Mais le mot, en philosophie, a un sens beaucoup plus large : être humaniste,
c'est considérer l’humanité comme une valeur, voire comme la valeur suprême.
Reste à savoir si cette valeur est elle-même un absolu, qui se donne à connaître,
à reconnaître, à contempler, ou bien si elle reste relative à notre histoire, à nos
désirs, à une certaine société ou civilisation. On parlera dans le premier cas
d’{\it humanisme théorique}, lequel peut être métaphysique ou transcendantal, mais
tend toujours à devenir une religion de l’homme (voyez {\it L'homme-Dieu}, de Luc
Ferry) ; dans le second, d’{\it humanisme pratique}, qui ne prétend à aucun absolu,
à aucune religion, à aucune transcendance : ce n’est qu’une morale ou un guide
pour l’action. Le premier est une foi ; le second, une fidélité. Le premier fait de
l’humanité un principe, une essence ou un absolu ; le second n’y voit qu’un
résultat, qu’une histoire, qu’une exigence. La vraie question est de savoir s’il
faut croire en l’homme (humanisme théorique) pour vouloir le bien des individus,
%— 280 —
ou si l’on peut vouloir leur bien (humanisme pratique) quand bien
même on aurait toutes les raisons de ne pas s’illusionner sur ce qu’ils sont. Tel
était l’humanisme de Montaigne, dont j’ai parlé ailleurs ({\it Valeur et vérité}, p. 94-95
et 238-240). Tel aussi celui de La Mettrie. « Je déplore le sort de l'humanité,
écrivait-il, d’être, pour ainsi dire, en d’aussi mauvaises mains que les siennes. »
Ce n’est pas une raison pour l’abandonner à son sort, puisque ces mains, précisément,
sont les nôtres. En bon matérialiste, La Mettrie ne voit dans les êtres
humains que de purs produits de la matière et de l’histoire (c’est la thèse
fameuse de l’{\it homme machine}, qui fonde un anti-humanisme théorique). Mais
le médecin qu’il était aussi n’a pas renoncé pour cela à les soigner, pas plus que
le philosophe à les comprendre et à leur pardonner. « Savez-vous pourquoi je
fais encore quelque cas des hommes ?, demandait-il. C’est que je les crois
sérieusement des machines. Dans l’hypothèse contraire, j’en connais peu dont
la société fût estimable. Le matérialisme est l’antidote de la misanthropie. »
Humanisme sans illusions, et de sauvegarde. Ce n’est pas la valeur des hommes
qui fonde le respect que nous leur devons ; c’est ce respect qui leur donne de la
valeur. Ce n’est pas parce que les hommes sont bons qu’il faut les aimer ; c'est
parce qu’il n’y a pas de bonté sans amour. Enfin, ce n’est pas parce qu’ils sont
libres qu’il faut les éduquer ; c’est pour qu’ils aient une chance, peut-être, de le
devenir. C’est ce que j'appelle l’humanisme pratique, qui ne vaut que par les
actions qu’il suscite. Ce n’est pas une croyance ; c’est une volonté. Pas une
religion ; üne morale. Croire en l’homme ? Je ne vois pas ce que cela signifie,
puisque son existence est avérée, ni pourquoi ce serait nécessaire. Pas besoin de
croire en l’homme pour vouloir le bien des individus et le progrès de l’humanité.
Au reste, nous partons de si bas qu’il doit bien être possible de nous élever
quelque peu. C’est où l’on retrouve le premier sens du mot {\it humanisme}, qui
renvoie aux études, à la culture, à l’étude attentive et fidèle du passé humain.
C’est la seule voie pour l'avenir, si l’on veut qu’il soit acceptable. L'homme
n’est pas Dieu. Faisons au moins en sorte, et l’on n’en a jamais fini, qu’il soit à
peu près humain.

\section{Humanité}
%HUMANITÉ
Le mot se prend en deux sens, l’un descriptif l’autre normatif :
humanité, le cas est assez singulier, est à la fois une espèce
animale et une vertu. Et non par métaphore (comme on dit d’un homme très
courageux : « C’est un lion»), mais par métonymie: c’est passer du tout
(l'espèce humaine) à la partie (puisque certains membres de l’espèce seront dits
inhumains), du donné au résultat (humanité, c’est ce que l'espèce humaine a
fait de soi, qu’elle doit préserver), de la nature à la culture, du fait à la valeur,
de l’existence à l’exigence, de l’appartenance à la fidélité. C’est pourquoi on
%— 281 —
parle aussi des {\it Humanités}, pour désigner la culture, spécialement littéraire :
parce que le passé de l'humanité s’y reflète, qui nous apprend ce que nous
sommes par ce que nous pouvons et devons devenir. « Il n’y a point d’Humanités
modernes, disait Alain. Il faut que le passé éclaire le présent, sans quoi nos
contemporains ne sont à nos yeux que des animaux énigmatiques. » L’humanité
n’est devant nous, comme idéal, que parce qu’elle est d’abord derrière
nous, comme histoire, comme mémoire, comme fidélité, Que nous devions
nous soucier de nos descendants, c’est une évidence. Mais qui ne vient pas
d’eux.

\section{Humilité}
%HUMILITÉ
Être humble, c’est avoir le sentiment de sa propre insuffisance.
C’est pourquoi c’est une vertu : être {\it suffisant}, c'est manquer à
la fois de lucidité et d’exigence. Voyez par exemple la prétention de Greuze,
Boucher, Fragonard, et l'humilité de Chardin. On ne confondra donc pas
l’humilité avec la haine de soi, encore moins avec la servilité ou la bassesse.
L’homme humble ne se croit pas inférieur aux autres : il a cessé de se croire
supérieur. Il n’ignore pas ce qu’il vaut, ou peut valoir : il refuse de s’en
contenter. Vertu humble (qui se vanterait de la sienne prouverait par à qu’il en
manque), mais nécessaire. C’est le contraire de l’orgueil, de la vanité, de
l’amour aveugle de soi-même, de la complaisance, de la suffisance, j'y reviens,
et c'est pourquoi elle est si précieuse : il ne lui manque qu’un peu de simplicité
et d’amour pour être bonne absolument.

\section{Humour}
%HUMOUR
C'est une forme de comique, mais qui fait rire surtout de ce qui
n'est pas drôle. Par exemple ce condamné à mort qu’évoque
Freud, qu’on mène un lundi à l’échafaud : « Voilà une semaine qui commence
bien ! », murmure-t-il. Ou Woody Allen : « Non seulement Dieu n'existe pas,
mais essayez de trouver un plombier pendant le week-end ! » Ou encore Pierre
Desproges annonçant sa maladie au public : « Plus cancéreux que moi, tu
meurs ! » Cela suppose un travail, une élaboration, une création. Ce n’est pas le
réel qui est drôle, mais ce qu’on en dit. Non son sens, mais son interprétation
— ou son non-sens. Non le plaisir qu’il nous offre, mais celui que nous prenons
à constater qu’il n’en propose aucun qui puisse nous satisfaire. Conduite de
deuil : nous cherchons un sens ; nous constatons qu’il fait défaut ou se détruit ;
nous rions de notre propre déconfiture. Et cela fait comme un triomphe pourtant
de l'esprit.

L'humour se distingue de l'ironie par la réflexivité ou l’universalité. L’ironiste
rit des autres. L’humoriste, de soi ou de tout. Il s’inclut dans le rire qu’il
%— 282 —
suscite. C’est pourquoi il nous fait du bien, en mettant l’{\it ego} à distance. L’ironie
méprise, exclut, condamne ; l’humour pardonne ou comprend. L’ironie blesse ;
humour soigne ou apaise.

Il y a du tragique dans l’humour ; mais c’est un tragique qui refuse de se
prendre au sérieux. Il travaille sur nos espérances, pour en marquer la limite,
sur nos déceptions, pour en rire, sur nos angoisses, pour les surmonter. « Ce
n’est pas que j'aie peur de la mort, explique par exemple Woody Allen, mais je
préférerais être ailleurs quand cela se produira. » Défense dérisoire ? Sans doute.
Mais qui s’avoue telle, et qui indique assez, contre la mort, qu’elles le sont
toutes. Si les fidèles avaient le sens de l'humour, que resterait-il de la religion ?

\section{Hypocrisie}
%HYPOCRISIE
C’est vouloir passer pour ce qu’on n’est pas, afin d’en tirer
avantage — non par vanité, comme dans le snobisme, mais
par calcul ou intérêt ; non pour imiter ceux qu’on admire ou qu’on envie, mais
pour duper ceux qu’on méprise ou qu’on veut utiliser. Le snob est un simulateur
sincère, qui se dupe lui-même ; l’hypocrite, un simulateur mensonger, qui
dupe les autres. Ainsi, Tartuffe: sil admirait les dévots, il ne serait pas
hypocrite ; il serait snob. Mais il n’a que faire d’admirer : il ne veut paraître ce
qu’il n’est pas que pour tromper et utiliser. L’hypocrisie est une mauvaise foi
lucide et intéressée. C’est pourquoi elle est rare (la lucidité l’est toujours) et
habituellement efficace (elle peut compter sur le snobisme des autres : si Orgon
n'avait voulu passer pour pieux, à ses propres yeux autant qu’à ceux des autres,
il ne se serait pas entiché à ce point de Tartuffe). Contre elle, la lucidité et
l'humour : se défier des autres, et de soi.

\section{Hypostase}
%HYPOSTASE
Ce qui se trouve sous (c’est l'équivalent grec du latin {\it substantia}),
autrement dit ce qui porte ou fonde. Le mot désigne
une réalité existant en soi, mais considérée, surtout depuis le néoplatonisme,
dans son rapport à d’autres hypostases, dont elle procède ou qu’elle engendre :
ainsi, chez Plotin, l’Un (première hypostase) engendre l’Intellect (deuxième
hypostase, qui émane de l’Un), lequel engendre à son tour l'Âme du monde
(troisième hypostase, qui émane de l’Intellect). Le mot sera repris par la tradi-
tion chrétienne, pour désigner les trois Personnes de la Trinité — le Père, le Fils,
le Saint Esprit —, considérées comme trois hypostases d’un seul et même être
({\it ousia}), qui est Dieu. Ces deux usages, néoplatonicien puis chrétien, fortement
marqués de mysticisme, expliquent peut-être que le mot ait fini par se distin-
guer assez fortement de celui de substance. On peut dire d’une pierre qu’elle est
une substance (si l’on considère qu’elle existe en soi) ; on évitera de dire qu’elle
%— 283 —
est une hypostase. C’est qu’il y a du mystère dans l’hypostase : c’est une substance
qu'on ne comprend pas, qui nous dépasse, qu’on ne peut expérimenter,
si on le peut, que de façon surnaturelle ou mystique. De là le sens péjoratif que
le mot, dans la période moderne, finit par prendre : une hypostase serait une
substance supposée ou fictive, une entité à laquelle on accorderait, à tort, une
réalité indépendante. Ainsi Platon hypostasie ses idées, comme Descartes son
{\it cogito}. Un matérialiste leur objectera que les Idées ou l’âme ne sont que des
fictions : une façon d’ériger la pensée, qui n’est qu’un acte du corps, au rang de
réalité indépendante ou substantielle. L’hypostase, en ce dernier sens, n’est
qu'une abstraction hypostasiée : on sépare d’abord la pensée de ce qui la produit
(le corps, le cerveau), puis on en fait une réalité existant en soi. Reste à
savoir, toutefois, si la matière n’est pas, elle aussi, une hypostase.

\section{Hypostasier}
%HYPOSTASIER
Considérer comme hypostase ou comme substance. Le
mot a le plus souvent un sens péjoratif : c’est prêter une
réalité absolue ou indépendante à ce qui n’est qu’un processus, qu’un accident
ou qu’une abstraction.

\section{Hypothèse}
%HYPOTHÈSE
C’est une supposition, qui prend ordinairement place dans
une démarche démonstrative ou expérimentale : une idée
qu’on admet provisoirement comme vraie, afin d’en déduire les conséquences
et, le cas échéant, d’en confirmer ou d’en infirmer la vérité. Dans les sciences
expérimentales, c’est une «explication anticipée», disait Claude Bernard,
qu’on soumet à l'expérience afin d’en tester la validité. Ces sciences, qu’on a dit
longtemps inductives (parce qu’elles iraient du fait à la loi), sont plutôt
hypothético-expérimentales : leurs hypothèses ne sont scientifiques, montre Popper,
que dans la mesure où elles peuvent être soumises à l’expérience et, le cas
échéant, réfutées par elle (voir l’article « Falsifiabilité »). En mathématiques, les
hypothèses sont plutôt des conventions, qui valent moins par elles-mêmes que
par le système des conséquences nécessaires qu'on en peut déduire (les
théorèmes) : elles forment une axiomatique, qui sert de base à un système
hypothético-déductif.

\section{Hypothético-déductive (méthode —)}
%HYPOTHÉTICO-DÉDUCTIVE (MÉTHODE —)
Toute méthode qui part
d’hypothèses pour en
déduire des conséquences, que celles-ci soient falsifiables (dans les sciences
expérimentales) ou non. Se dit spécialement des mathématiques, qui visent
% 284 
moins à vérifier leurs hypothèses (comment une convention pourrait-elle être
démontrée ?) qu’à produire, à partir d’elles, un système cohérent : la vérité est
moins dans les théorèmes que dans le lien nécessaire qui les unit aux hypothèses
de départ (principes, axiomes, postulats....) ou à d’autres théorèmes. C’est l’une
des conséquences épistémologiques de l'invention des géométries non euclidiennes :
le postulat d’Euclide n’est plus une évidence ni une proposition à
démontrer, mais une simple convention, qu’on peut poser ou pas, et qui ne
débouche que sur un système géométrique particulier (la géométrie euclidienne),
parmi d’autres possibles. Il en va de même des autres axiomes ou postulats,
de telle sorte que le statut des théorèmes eux-mêmes s’en trouve transformé.
Il n’y a plus pour eux de vérité séparée : « leur vérité, c’est seulement
leur intégration au système » (R. Blanché, {\it L'axiomatique}, PUF, p. 7). Ce système
est-il vrai ? Ce n’est plus la question : il lui suffit d’être cohérent. C’est
pourquoi les mathématiques ne suffisent pas.

\section{Hypothétique (jugement —)}
%HYPOTHÉTIQUE (JUGEMENT —)
Ce n’est pas un jugement qui ne prétendrait
qu’au statut d’hypothèse ou
de simple possibilité (on parle alors de jugement {\it problématique}). Un jugement
est {\it hypothétique} lorsqu'il énonce une relation entre une hypothèse et l’une au
moins de ses conséquences. Par exemple : « Si Médor est un homme, il est
mortel. » Ou bien : « Si Médor est un triangle, ses trois angles sont égaux à
deux droits. » On voit que ce jugement, même valide, ne prouve rien sur la
nature de mon chien, ni sur sa forme ou sa mortalité. Le jugement, considéré
en tant que tel, n’en est pas moins assertorique : la relation entre l’hypothèse et
sa conséquence est énoncée comme un fait, non comme une hypothèse. Rien
n'empêche pour autant qu’un jugement hypothétique, du point de vue de la
relation, soit par ailleurs problématique, quant à sa modalité (voir ce mot). Par
exemple : « Si Dieu existe, il est possible que l’âme soit immortelle. » Cela ne
prouve pas davantage que Dieu existe, ni que l’âme soit immortelle. Mais
reconnaît que l'existence de celui-là ne saurait garantir l’immortalité de celle-ci.

\section{Hystérie}
%HYSTÉRIE
C’est une névrose, qui enferme l’hystérique dans l'apparence
qu’il veut prendre : le voilà comme prisonnier à la surface de
soi. L’utérus, malgré l’étymologie, n’y est pour rien. Il y a des hommes hystériques,
et certains psychiatres me disent qu’il y en a de plus en plus. C’est que
la notion relève de la psychopathologie, non de la physiologie. Et dépend de
l’évolution de la société, selon toute vraisemblance, presque autant que des histoires
individuelles. Les grandes crises d’hystérie, telles que les décrivait
%— 285 —
Charcot, se font rares (il m’est pourtant arrivé d’en voir une). C’est peut-être
que notre société, où tout est spectacle, n’a plus besoin de tels débordements —
que l’hystérie s’est banalisée en se répandant. Société du spectacle : hystérisation
de la société. Plus besoin de courir les hôpitaux. La rue et la télévision suffisent.

Freud voyait dans l’hystérie un effet du refoulement. D’un point de vue
philosophique, j’y verrais plutôt, mais ce n’est pas contradictoire, une incapacité
à supporter la vérité, une fuite dans l’apparence, un enfermement dans le
simulacre. C’est une maladie du mensonge, mais d’abord à soi-même. L’hystérique
est un simulateur sincère, comme un comédien qui se prendrait pour son
personnage. Il veut faire illusion, et y parvient en effet, jusqu’à y croire lui-même.
Il veut séduire, et y parvient souvent. Mais cela ne fait que le couper un
peu plus du réel, que l’enfermer davantage dans le semblant, dans le factice,
dans la superficialité. Hyperexpressivité, mais à vide ; émotivité à fleur de peau,
mais sans chair ou sans cœur. Volubilité, suggestibilité, mythomanie. Beaucoup
de charme au-dehors, beaucoup de vide au-dedans. L’hystérique en fait
trop ; mais c’est pour masquer (et {\it se} masquer) un manque d’être. Chatoiement
de surface, absence de profondeur. Multiplication des signes, fuite du sens.
Somatisation, théâtralisation, donjuanisme. Besoin éperdu de séduire, incapacité
d’aimer et de jouir. C’est comme un narcissisme extraverti, qui ne saurait
s'aimer que dans le regard de l’autre. Avec l’âge, cela devient de plus en plus
difficile : la dépression ou l’hypocondrie menacent. Tristesse du comédien,
quand le public se détourne.
%{\footnotesize XIX$^\text{e}$} siècle — {\it }

