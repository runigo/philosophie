%VWXYZ {\it }

VACANCE (S) Au singulier : vide, absence, oisiveté… Il faut que la vie quotidienne
soit bien dure, ou bien vaine, pour que le même
terme, au pluriel et à l'envers de son vide initial, en vienne à suggérer le plein
d’une vie, pour une fois, à peu près intense et joyeuse. Et bien triste qu’il
faille pour cela partir. « Partir, c’est vivre beaucoup », semblent-ils croire. C’est
qu’ils n’habitent que la mort ou le travail.

Les vacances, considérées comme congés payés, restent ainsi prisonnières
des formes modernes d’aliénation, dont elles sont l'expression autant que l’antidote.
Est aliéné celui dont la vie est ailleurs — celui qui doit partir pour rentrer
chez soi. Une société injuste mais riche devait sécréter ces ersatz d’utopie. Ce
sont les bacchanales de notre temps — Dionysos au Club Méditerranée. On
aurait tort de les mépriser, comme de s’en contenter.

VALEUR Ce qui vaut, et le fait de valoir. Un prix ? Seulement pour ce qui
en a un, pour ce qui est à vendre. Par exemple la valeur d’une
marchandise : son prix indique sa valeur d’échange, dans un marché donné,
telle qu’elle résulte du temps de travail moyen socialement nécessaire à sa production
(selon Marx) ou de la loi de l'offre et de la demande (selon la plupart
des économistes libéraux). Mais la justice ? Mais la liberté ? Mais la vérité ?
Elles peuvent avoir un coût, dans telle ou telle circonstance. Mais elles n’ont
pas de prix : elles ne sont pas à vendre. Aussi faut-il distinguer ce qui a une
valeur (qu’un prix, dans une logique d’échange, peut mesurer à peu près), et ce
qui est une valeur, qui n’a pas de prix et ne saurait être échangé valablement
contre de l'argent, ni même contre une autre valeur. Échanger la justice contre
la liberté ? Ce serait manquer à la justice. Échanger la vérité contre la justice ?
%— 606 —
Ce serait manquer à la vérité. Ainsi les valeurs n’ont pas de prix : elles ont une
{\it dignité}, comme disait Kant, qui n’admet pas d’équivalent et ne saurait être
échangée contre autre chose. Faut-il y voir pour autant un absolu ? Non pas,
me semble-t-il, puisqu'il reste à comprendre d’où vient cette valeur qu’on lui
prête ou qu’elle est. Une valeur, c’est ce qui vaut, disais-je ; mais qu'est-ce que
valoir ? C’est être désirable ou désiré. C’est vrai pour les marchandises : elles
n’ont de valeur d'échange, soulignait Marx, qu’à la condition d’avoir d’abord
une valeur d'usage. Or celle-ci n’est pas un absolu. « La marchandise est
d’abord un objet extérieur, qui, par ses propriétés, satisfait des besoins humains
de n’importe quelle espèce. Que ces besoins aient pour origine l’estomac ou la
fantaisie, leur nature ne change rien à l’affaire » ({\it Le Capital}, I, 1). Comment
mieux dire qu’il s’agit moins de {\it besoin} que de {\it désir}, moins d’utilité que d’{\it usage}
en effet ? Un objet, même apparemment inutile, peut avoir une grande valeur,
s’il est fortement désiré par beaucoup : ainsi une pierre précieuse ou une œuvre
d’art (ce n’est pas parce qu’elles sont utiles qu’elles sont désirées, c’est parce
qu’elles sont désirées qu’elles semblent utiles, et le sont en effet). Un objet
manifestement utile, à l'inverse, n’a de valeur qu’à proportion du désir qu’on
en a. Les commerçants le savent bien. Ce n’est pas l'utilité qui fait la valeur
d'usage, c’est la valeur d’usage, telle qu’elle résulte elle-même du désir, qui fait
l'utilité. Si une marchandise n’a de valeur d'échange qu’à la condition d’avoir
une valeur d'usage, il faut donc en conclure qu’elle n’a de valeur qu’à proportion
non de je ne sais quelle utilité objective ou absolue, si tant est que la
notion puisse avoir un sens, mais du désir, historiquement déterminé, qui la
vise. Cela donne raison à la loi de l'offre et de la demande, me semble-t-il,
davantage qu’à la théorie marxiste de la valeur (un vêtement à la mode peut
demander moins de temps de travail qu’un autre, et valoir pourtant beaucoup
plus cher), mais ce n’est pas ici ce qui m'importe. Philosophiquement, sinon
économiquement, les deux théories se rejoignent en ceci qu’il n’y a de valeurs,
dans un marché donné, que relatives : que pour et par le désir. C’est où l’on
rencontre les valeurs morales ou spirituelles. Qu’elles soient en dehors de tout
marché — sans équivalent, sans prix, sans échange possible —, je l’indiquais en
commençant. Mais cela ne prouve pas qu’elles soient en dehors de tout désir !
Comment la justice serait-elle une valeur, si personne ne désirait la justice ?
Comment la vérité pourrait-elle valoir, si personne ne l’aimait ou ne la
désirait ? C’est ce que Spinoza, dans un scolie abyssal de l’{\it Éthique}, nous invite
à penser : « Nous ne nous efforçons à rien, ne voulons, n’appétons ni ne désirons
aucune chose, parce que nous la jugeons bonne ; mais, au contraire, nous
jugeons qu’une chose est bonne parce que nous nous efforçons vers elle, la voulons,
appétons et désirons » (III, 9, scolie). Chacun a le sentiment du contraire.
Si j'aime la richesse ou la justice, n’est-ce pas parce qu’elles sont bonnes ? Si je
%— 607 —
désire cette femme, n’est-ce pas parce qu’elle est belle ? Non pas, répondrait
Spinoza : c’est parce que tu aimes la richesse et la justice qu’elles te paraissent
bonnes ; c’est parce que tu désires cette femme qu’elle te paraît belle. Le fait est
que richesse ou justice sont indifférentes à quelques-uns. Et qu’un singe, à ce
que nous jugeons la plus belle des femmes, préférerait une guenon. Parce qu’il
a mauvais goût ? Il faudrait être bien naïf pour le penser. Mais parce qu’il n’a
pas le même désir. Relativisme sans appel : une valeur c’est ce qui est désirable,
et elle n’est désirable que parce qu’elle est désirée. Ce qui vaut, c’est ce qui plaît
ou réjouit, pour un individu et dans une société donnée. C’est pourquoi
l'argent, pour certains, vaut plus que la justice. Et c’est pourquoi la justice,
pour d’autres, vaut davantage que l’argent. Il n’y a pas de valeurs absolues. Il
n'y a que des désirs et des conflits de désirs, que des affects et des hiérarchies
entre affects. C’est ce que Spinoza, à propos du bien et du mal mais aussi de
l'argent et de la gloire, explique fort nettement :

« Par {\it bien}, j'entends ici tout genre de joie et tout ce qui y mène, et principalement
ce qui satisfait un désir, quel qu’il soit. Par {\it mal}, j'entends tout genre de tristesse, et
principalement ce qui frustre un désir. Nous avons en effet montré plus haut (dans le
scolie de la prop. 9) que nous ne désirons aucune chose parce que nous la jugeons
bonne, mais qu’au contraire nous appelons bonne la chose que nous désirons ; conséquemment,
nous appelons mauvaise la chose que nous avons en aversion. Chacun juge
ainsi ou estime, selon son affect, quelle chose est bonne, quelle mauvaise, quelle
meilleure, quelle pire, quelle enfin la meilleure ou la pire. Ainsi l’avare juge que l’abondance
d’argent est ce qu’il y a de meilleur, la pauvreté ce qu’il y a de pire. L’ambitieux
ne désire rien tant que la gloire et ne redoute rien tant que la honte. À l’envieux, rien
n'est plus agréable que le malheur d'autrui, et rien plus insupportable que le bonheur
d’un autre ; et ainsi chacun juge, d’après ses propres affects, qu’une chose est bonne ou
mauvaise, utile ou inutile » ({\it Éthique} III, 39, scolie).

Tel est aussi l'esprit de Nietzsche : « Évaluer, c’est créer. C’est leur évaluation
qui fait des trésors et des joyaux de toutes choses évaluées » ({\it Zarathoustra},
I, « Des mille et un buts»). Évaluer, ce n’est pas mesurer une valeur qui
préexisterait à l'évaluation ; c’est mesurer la valeur qu’on donne à ce qu’on
évalue, ou créer de la valeur en la mesurant. Spinoza-Nietzsche, même
combat ? Oui, assurément, s'agissant du relativisme. S'agissant des valeurs ellesmêmes ?
Cela dépend desquelles (Spinoza, lui, n’a jamais prétendu les renverser
toutes). Mais la vraie question, sur quoi ils s’opposent, est celle de la vérité.
Nietzsche, surtout dans ses dernières œuvres, tend à la considérer comme une
valeur parmi d’autres, ce que Spinoza ne saurait accepter. Que telle chose me
semble bonne ou mauvaise, cela dépend du désir que j’en ai. Mais qu’elle soit
vraie, non. Ils se rejoignent dans le relativisme (s'agissant des valeurs), s’opposent
%— 608 —
sur le rationalisme (s’agissant de la vérité ou de la raison). C’est où j'ai
choisi Spinoza contre Nietzsche, et le cynisme contre la sophistique. La vérité
est-elle une valeur ? Oui, si nous la jugeons bonne ou utile, ou simplement si
nous l’aimons ou la désirons. Elle en est donc une, pour presque tous : tous les
hommes aiment la vérité, disait saint Augustin, puisque aucun, même parmi les
menteurs, n’aime être trompé. Mais ce n’est pas parce qu’elle est vraie qu’elle
vaut, encore moins parce qu'elle vaut qu’elle est vraie. Disjonction des ordres :
la valeur de la vérité dépend du désir qu’on en a, mais sa vérité, non. Toute
valeur est subjective (y compris la vérité comme valeur) ; aucune vérité ne l’est.
Toute valeur est relative ; toute vérité (en tant qu’elle est vraie, en tant qu’elle
est la même en nous et en Dieu, comme dit Spinoza) est absolue. Toute valeur
est de l’homme. Toute vérité, de Dieu. C’est pourquoi nous sommes toujours
dans le vrai, et hors d’état pourtant de le posséder absolument. C’est pourquoi
nous le cherchons. C’est pourquoi nous le désirons. C’est pourquoi il vaut, au
moins pour nous, au moins par nous. Le jour où plus personne n’aimera la
vérité, elle aura cessé par là même d’être une valeur. Mais n’en sera pas moins
vraie pour autant.

Si vous n’aimez pas la vérité, n’en dégoûtez pas les autres. Et ne tenez pas,
si vous l’aimez, cet amour pour une preuve.

VALIDITÉ C’est le nom logique de la vérité, ou plutôt son équivalent formel.
Une inférence est valide lorsqu’elle permet de passer du vrai
au vrai (de la vérité des prémisses à la vérité de la conclusion) ou lorsqu'elle est
vraie quelle que soit l'interprétation qu’on en peut donner. On remarquera que
la validité d’un raisonnement ne dépend pas de la vérité de ses conclusions, pas
plus d’ailleurs que celle-ci ne dépend forcément de celle-à. Un raisonnement
valide peut aboutir à une conclusion fausse (si l’une au moins de ses prémisses
est fausse). C’est le cas, par exemple, du fameux sophisme du cornu : « Tu as
tout ce que tu n’as pas perdu ; tu n’as pas perdu de cornes ; donc tu as des
cornes » ; le raisonnement est valide, la conclusion est fausse (c’est que la.
majeure, quoique on puisse ne pas s’en rendre compte immédiatement, l’est
aussi). Et un raisonnement non valide, à l’inverse, peut aboutir à une conclusion
vraie : « Tous les hommes sont mortels ; Socrate est mortel ; donc Socrate
est un homme » est un raisonnement non valide.

VANITÉ On pense d’abord à l’Ecclésiaste : {\it « Vanité des vanités, tout est
vanité... »} C’est dire que tout est vide ou vain (venus : vide, creux,
sans substance), que rien n’a de valeur ou d’importance, sinon illusoire, sinon
%— 609 —
fugace, que le néant vaudrait mieux ou tout autant, enfin que rien ne vaut la
peine d’être vécu ni désiré. Est-ce vrai ? Il n’y a pas de réponse absolue ; il n’y
a que le désir qu’on éprouve ou pas de ces presque riens qui font notre vie, bonheur
et malheur, qui vont disparaître, certes, qui disparaissent déjà, mais qui
n’en sont pas moins vrais ni délectables, pour qui s’en délecte, ou douloureux,
pour qui en souffre. Montaigne, qui est notre Ecclésiaste, a consacré à cette
notion le plus beau de ses Essais (III, 9, « De la vanité »). Mais lui n’en tirait
aucune leçon nihiliste. C’est qu’il aimait la vie, quand l’Ecclésiaste, de son
propre aveu, la détestait. Par exemple Montaigne se plaît aux voyages. « Il y a
de la vanité, dites-vous, en cet amusement. — Mais où non ? Et ces beaux préceptes
sont vanité, et vanité toute la sagesse » (III, 9, 988). Le maître de Montaigne
est le vent, qui ne va nulle part, qui n’a rien à prouver, mais qui « s’aime
à bruire et à s’agiter ». Vent : vanité. « Le vent part au midi, tourne au nord,
disait l’Ecclésiaste, il tourne, tourne et va, et sur son parcours retourne le
vent... » Ainsi fait Montaigne : « S'il fait laid à droite, je prends à gauche. Ai-je
laissé quelque chose derrière moi? J'y retourne; c’est toujours mon
chemin. » On lui objecte son âge : « Vous ne reviendrez jamais d’un si long
chemin. — Que m'en chaut-il ? Je ne l’entreprends ni pour en revenir, ni pour
le parfaire ; j’entreprends seulement de me branler [de me mouvoir] pendant
que le branle me plaît. Et me promène pour me promener...» Vanité de la
sagesse constatait l’Ecclésiaste, et Montaigne en est d’accord. Mais il ajoute,
avec le vent : sagesse de la vanité.

En un autre sens, la vanité est une forme, particulièrement ridicule, de
lamour-propre. C’est être plein du vide de soi : c’est se glorifier de ce qu’on
croit être, c’est admirer en soi ce qu’on imagine que les autres y admirent (la
vanité, écrit Bergson, est « une admiration de soi fondée sur l’admiration qu’on
croit inspirer aux autres »), ou vouloir qu’ils admirent ce qu’on y admire soi-même.
Nul n’y échappe tout à fait. C’est ce qu’a vu Pascal : « La vanité est si
ancrée dans le cœur de l’homme qu’un soldat, un goujat, un cuisinier, un crocheteur
se vante et veut avoir ses admirateurs, et les philosophes mêmes en veulent,
et ceux qui écrivent contre veulent avoir la gloire d’avoir bien écrit, et ceux
qui les lisent veulent avoir la gloire de les avoir lus, et moi qui écris ceci ai peut-être
cette envie, et peut-être que ceux qui le liront.. » ({\it Pensées}, 627-150). Se
savoir vaniteux, toutefois, c’est déjà l’être moins. « La seule cure contre la
vanité, disait encore Bergson, c’est le rire » ; mais à condition de savoir rire de
soi.

Enfin, on appelle {\it vanités}, dans l’histoire de la peinture, certaines natures
mortes évoquant — par une fleur fanée, un crâne, une bougie consumée… — le
peu que nous sommes et que nous durons. C’est revenir au sens de l’Ecclésiaste,
qui est le sens existentiel, tout en essayant de nous guérir de la vanité au sens
%— 610 —
psychologique ou moral. On dirait que ces peintres veulent nous dégoûter de
la vie, pour que nous ne nous intéressions plus qu’à la mort ou à la religion.
Mais les plus doués n’arrivent même pas à nous dégoûter de la peinture. Est-ce
nous qui sommes trop vains, ou la peinture qui est trop belle ?

VÉCU La vie elle-même, mais au passé (fût-ce d’un dixième de seconde) et
dans sa singularité individuelle et immédiate, ou prétendument
immédiate (la conscience fait une médiation suffisante). C’est la vie à la première
personne, telle qu’elle a été éprouvée, telle qu’on s’en souvient, telle
qu’on en porte témoignage. Souvent opposé à la pensée, à la théorie, à l’abstraction.
Mais ce n’est guère qu’une abstraction de plus. Un vécu qui ne serait
pas pensé, on ne pourrait rien en dire. Vaudrait-il même la peine d’être vécu ?
Le vécu n’est pas la vie : il n’est que la conscience que nous en prenons et
en gardons — que son souvenir ou sa trace. « Notre vie est si vaine, disait Chateaubriand,
qu’elle n’est qu’un reflet de notre mémoire. » Le vécu est ce reflet,
et cette vanité.

VELLÉITÉ Une volition sans force, sans continuité, sans constance. C’est
vouloir sans agir, ou sans agir vraiment (dans la durée). Ce n’est
donc pas vouloir : c’est désirer vouloir, ou imaginer qu’on veut.

VÉRACITÉ Ce n’est pas la même chose que la vérité. La véracité est la qualité
de celui qui dit vrai, qui ne trompe ni ne se trompe. Elle est
donc une disposition, mais objective (on peut être sincère sans être vérace), du
sujet. La vérité serait plutôt l’objectivité même. La vérité, pour le dire autrement,
est le propre de ce qui est vrai ; la véracité, de ce qui est véridique. Ainsi
le vrai Dieu, chez Descartes, est un Dieu vérace. Mais ce n’est pas parce qu’il
est vérace qu’il est le vrai Dieu ; c’est parce qu’il est le vrai Dieu (ou parce qu’il
est vraiment Dieu) qu’il est vérace.

VERBE Parfois synonyme, surtout avec une majuscule, de Parole ou de Logos :
ce serait l’acte de Dieu, ou Dieu en acte, en tant qu’il fait sens. On
l’identifie traditionnellement à la deuxième personne de la Trinité : « le Verbe
s’est fait chair, lit-on dans le prologue de Jean, et il a habité parmi nous... »

Au sens ordinaire, un verbe est un mot, mais qui désigne le plus souvent un
mouvement, un événement ou un acte (par différence avec les {\it noms}, qui
%— 611 —
désignent plutôt des choses, des entités ou des individus). De là ce {\it langage-monde} 
qu'a imaginé Francis Wolff ({\it Dire le monde}, PUF, 1997), qui ne serait
fait que de verbes: monde du devenir pur, sans rien qui demeure ni qui
change, monde d’accidents sans substances, d’actions sans sujets, d'événements
sans essences et sans choses... C’est le monde à peu près d'Héraclite ou du
Bouddha, et c’est le vrai peut-être. Mais notre langage — « cet étrange entrelacs
de noms et de verbes » qu’évoquait Platon — est incapable de le dire comme
notre esprit de le penser. C’est que tout verbe, pour nous, a besoin d’un sujet.
De Rà le fameux {\it « Je pense donc je suis »} de Descartes, dont Nietzsche a bien vu
qu'il ne tirait son évidence que de la croyance. en la grammaire. Il m'arrive
de penser que le « réel voilé » de la mécanique quantique doit ressembler à ce
monde de verbes sans sujets et d’événements sans choses. C’est pourquoi nos
physiciens ne peuvent l’énoncer exactement, ni nous tout à fait le comprendre.
Nous n'avons pas la langue qu’il faudrait pour cela.

VÉRIFICATION C’est tester la vérité d’un énoncé, en vue de l’attester. On
peut ainsi vérifier un calcul, en le refaisant ou en en faisant
un autre, ou une hypothèse, en la soumettant à l'expérience. Reste à savoir ce
que valent ce nouveau calcul ou cette expérience. S'agissant du calcul, on considère
que la probabilité d’une erreur décroît rapidement, en proportion du
nombre de vérifications effectuées, si possible par plusieurs individus et selon
des procédures différentes. Toutefois cela reste soumis à la fiabilité de notre
raison, qui est sans vérification possible (puisque toute vérification la suppose).
S'agissant de l'expérience ou de l’expérimentation, on bute sur le problème de
l'induction (voir ce mot). Comment vérifier un énoncé universel (« tous les
cygnes sont blancs ») en additionnant des constats singuliers (ce cygne est
blanc, et cet autre, et cet autre..), étant entendu qu’on ne peut en dresser la
liste exhaustive et qu’une seule exception suffirait à invalider l'énoncé ? Il n’y a
donc pas de vérification en toute rigueur. En revanche, montre Popper, il y a
des falsifications suffisantes : un seul cygne noir ou coloré suffit à prouver qu’ils
ne sont pas tous blancs. Cette asymétrie entre vérifiabilité et falsifiabilité est au
cœur de la démarche expérimentale. Vérifier une théorie ou une hypothèse, ce
n’est jamais prouver en toute rigueur qu’elle est vraie, c’est essayer de montrer
qu’elle est fausse. On la tient pour vraie — au moins relativement et provisoirement
— tant qu'elle a résisté à toutes les tentatives de falsification. On remarquera
que cette solution popperienne ou darwinienne (les théories les plus
faibles sont éliminées, seules les meilleures subsistent) du problème de l’induction
est davantage épistémologique que métaphysique. Elle explique le fonctionnement
des sciences ; elle ne dit rien de leur vérité globale : non seulement
%— 612 —
parce qu’il se pourrait que toute pensée ne soit qu’un rêve, comme le reconnaît
Popper, mais parce que le test expérimental devrait lui-même être vérifié, et ne
peut l’être absolument. Il suffit d’un seul cygne noir pour prouver qu’ils ne
sont pas tous blancs. Mais comment prouver qu’un cygne est vraiment noir ?
Toute vérification et toute falsification supposent une vérité antécédente — celle
du monde, celle de l'expérience, celle de la raison — qui est invérifiable et infalsifiable.
Si nous n’étions d’abord dans le vrai, nous n’aurions aucune chance de
rencontrer le faux. Si la vérité n’était antérieure à toute vérification, il n’y aurait
rien à vérifier. C’est où l’on retrouve Spinoza : {\it « Habemus enim ideam veram »}
(car nous avons une idée vraie, {\it T.R.E.}, 27). Cela est sans preuve, mais il n’y
aurait pas de preuve autrement, ni rien à prouver.

VERITAS Le nom latin et scolastique de la vérité, qu’on distingue à ce titre
de l’{\it alèthéia}, qui est son nom grec. Il est usuel, depuis Heidegger,
de se servir de ces deux mots pour désigner deux conceptions, ou deux acceptions,
différentes de la vérité : la vérité comme adéquation entre la pensée et le
réel ({\it l'ad{\ae}quatio rei et intellectus} des scolastiques : {\it veritas}), d’une part, et, d’autre
part, la vérité comme non-voilement de l'être même (la vérité intrinsèque de la
chose, qu’elle soit connue ou pas, ce que j’appellerais volontiers sa présentation
silencieuse : {\it alèthéia}). L'{\it alèthéia} est de l'être ou du silence ; la {\it veritas}, de la
pensée ou du discours. Certains en concluront que l’{\it alèthéia} est moins une
{\it vérité} qu’une {\it réalité}. Mais si le réel n’était vrai aussi et d’abord, quelle pensée
pourrait l’être ?

VÉRITÉ Ce qui est vrai, ou le fait de l’être, ou le caractère de ce qui l’est.
C’est donc une abstraction (la vérité n’existe pas : il n’y a que des
faits ou des énoncés vrais). Mais cette abstraction nous permet seule de penser.
S’il n’y avait rien de commun, au moins pour la pensée, entre deux propositions
vraies, il n’y aurait aucun sens à dire qu’elles le sont, ni donc, intellectuellement,
à dire quoi que ce soit : tous les discours se vaudraient et ne vaudraient
rien (puisqu'on pourrait dire aussi bien, ou aussi mal, le contraire de ce qu’on
dit). Il n’y aurait aucune différence entre un délire et une démonstration, entre
une hallucination et une perception, entre une connaissance et une ignorance,
entre un faux témoignage et un témoignage véridique, entre un savant et un
ignorant, entre un historien et un mythomane. Ce serait la fin de la raison, et
de la déraison. {\it Veritas norma sui et falsi}, disait Spinoza : la vérité est norme
d’elle-même et du faux ({\it Éthique}, II, 43, scolie). Il n’y aurait, sans cette normativité
immanente, aucun moyen de se tromper, ni de ne se tromper pas, aucun
%— 613 —
moyen de mentir ni de ne mentir pas. Par quoi une seule erreur reconnue, et
ce n'est pas cela qui manque, un seul mensonge démasqué, et ils sont légion,
suffit à attester au moins l’idée de vérité. Abstraction, donc, mais nécessaire.
Même le silence, pour l'esprit, en relève. S’il est vraiment silencieux, c’est une
vérité. S’il ne l’est pas, c’en est une autre.

Où en sommes-nous avec la vérité ? La question est aussi ancienne que la
philosophie (plus ancienne ? non, puisque cette question est déjà philosophique,
puisqu'elle est, peut-être bien, la philosophie même), mais se pose,
aujourd’hui, à nouveaux frais. Tout se passe comme si les progrès mêmes de
la connaissance rendaient la notion de vérité plus problématique. Il y a là un
paradoxe, qui en dit long sur notre modernité. Aucune époque n’a disposé
d'autant de connaissances, ni d’aussi précises, ni d’aussi fiables. Un bon élève
de nos lycées en sait beaucoup plus — sur le monde, sur l’histoire, sur presque
tout — que n'en savaient Aristote ou Descartes. Nos savants, qui sont sans
doute la principale gloire de notre triste époque, multiplient comme jamais
découvertes et expérimentations. La biologie ou la physique de notre temps
eussent sidéré — si tant est qu’ils aient pu les comprendre — un Buffon ou un
Laplace. Il n’est pas jusqu’à nos médias, dans leur médiocrité essentielle, qui
ne contiennent une multitude d’informations sans commune mesure avec
celles dont disposaient les esprits les plus avancés des siècles passés. Bref, on
en sait beaucoup plus qu’on n’en a jamais su, dans presque tous les domaines,
et l’on pourrait croire que la notion de vérité en serait sortie renforcée
d'autant. On sait qu’il n’en est rien, et c’est peut-être, philosophiquement, la
marque principale du siècle qui vient de s'achever. La vérité ? Quel scientifique,
aujourd’hui, prétendrait la connaître ? Quel artiste s’en préoccupe
encore ? Et combien de philosophes vont jusqu’à dire qu’elle n’existe pas,
qu’elle n’a jamais existé, qu’elle n’est que la dernière illusion dont il importe
de s’affranchir ?

Il y a à cela plusieurs raisons, tant théoriques que pratiques. Les raisons
théoriques peuvent être rattachées, par commodité, à la révolution kantienne
ou à ce qu'elle manifeste. Dès lors que nous sommes séparés du réel par les
moyens mêmes qui nous servent à le connaître, il n’est que trop clair que nous
ne pourrons jamais le connaître tel qu’il est en lui-même ou absolument. Nous
ne connaissons pas l’être : nous ne connaissons que des phénomènes, que le
monde tel qu’il apparaît à travers les formes de notre sensibilité et de notre
entendement, que des objets que nous construisons (par notre perception, par
notre langage, par nos sciences), qui sont sans rapport assignable avec les choses
en soi. On dira que cela n’annule pas nos connaissances, que cela permet au
contraire de les penser comme possibles et nécessaires... Certes. Mais une
connaissance qui ne porte plus sur l’être, est-ce encore une vérité ? « Le même
%— 614 —
est à la fois penser et être », disait Parménide, et c’est ce qui nous devient de
moins en moins concevable. « La vérité consiste en l’être, disait Descartes, la
vérité est une même chose avec l’être », et c’est cela même — l’être, la vérité,
l’heureuse indistinction des deux — que nous avons perdu et qui nous sépare,
philosophiquement, du bonheur. Nous voilà chassés du pays de la vérité,
chassés du pays de l'être, puisque c’est le même, et c’est cet exil que nous appelons
le monde.

L’oubli de l’être se fit parfois, nous rappelle Heidegger, au nom de la vérité
— parce qu’elle n’était vérité que du sujet. Mais combien plus redoutable
l’oubli conjoint des deux, comme une lente enfoncée dans le phénoménisme
ou la sophistique ! Si rien n’est vrai, comme le voulait Nietzsche, que reste-t-il
à vivre et à penser ? Nos rêves, nos désirs, nos interprétations, nos fantasmes,
nos illusions ? Soit, mais alors ils se valent tous — puisque aucune vérité entre
eux ne vient trancher — et ne valent rien. C’est où la sophistique mène au nihilisme,
et Nietzsche à notre modernité. S’il n’y a pas de faits, s’il n’y a que des
interprétations, selon la fameuse formule de {\it La volonté de puissance}, le monde
même se dérobe : il n’y a plus que des discours sur le monde. C’est comme un
monde virtuel, qui aurait absorbé le vrai jusqu’à le dissoudre. Qu’on puisse y
vivre, peut-être. Mais à quoi bon, alors, vouloir le vivre et le penser {\it en vérité} ?
Pourquoi ne pas se contenter d’un beau mensonge, d’un discours habile ou
d’une illusion confortable ? Philosophie de bavards et de sophistes, où la philosophie
se meurt. S’il n’y a pas de vérité, on peut penser n’importe quoi, mais
aussi on ne peut plus penser du tout. Si rien n’est vrai, il n’est pas vrai que rien
ne soit vrai. Si tout est faux, il est faux que tout le soit. Cette autocontradiction,
loin de la réfuter, rend la sophistique irréfutable : puisqu'on ne pourrait la
réfuter qu’au nom d’une vérité au moins possible, qu’elle récuse. Alors ? Alors
il n’y a plus que les rapports de forces et le conflit — aussi inépuisable qu’épuisant
— des interprétations. C’est le monde de la guerre, du marché et des
médias. C’est notre monde. Plutôt c’est ce que certains voudraient qu’il soit, un
monde sans être, sans réalité, sans vérité, un monde sans consistance, un
monde virtuel, répétons-le, où il n’y aurait plus que des signes et des échanges,
que des simulacres et des marchands, un monde pour rire, comme un jeu de
l'esprit, et cela m’a donné bien souvent, quand j'étais étudiant, comme une
envie de pleurer.

Il faut en sortir. Comment ? Par un retour décidé à l’idée de vérité. Qu’on
ne puisse jamais la connaître tout entière ni absolument, c’est aujourd’hui une
évidence, avec laquelle je ne prétends aucunement rompre. Au reste, c’en était
une déjà pour Montaigne, Pascal ou Hume. Mais ceux-là n’ont jamais prétendu
pour autant qu’elle n’existait pas, ni qu’on ne pouvait aucunement y
accéder ! Ils ont simplement contesté, c’est tout autre chose, qu’on puisse le
%— 615 —
faire avec certitude. C’est ce qui distingue le sceptique (pour qui rien n’est certain)
du sophiste (pour qui rien n’est vrai). Les deux positions ne sont ni identiques
ni convergentes. Que rien ne soit certain, cela ne prouve pas que tout
soit faux. Que tout soit douteux, cela ne prouve pas que rien ne soit vrai. Au
contraire, même, puisqu’une proposition quelconque, fût-elle sceptique, n’est
pensable que sous l’idée de vérité (c’est ce que j'appelle, corrigeant Spinoza par
Montaigne, {\it la norme de l'idée vraie donnée ou possible}), ce qui interdit, ou
devrait interdire, qu’elle prétende s’en libérer. Formidable culot de Nietzsche :
« Qu'un jugement soit faux, ce n’est pas, à notre avis, une objection contre ce
jugement » ({\it Par-delà le bien et le mal}, X, 4). Je pense très exactement le
contraire, comme la quasi-totalité des scientifiques d’aujourd’hui, et c’est ce
qui nous rattache, tous ensemble et malgré Nietzsche, aux Lumières. C’est où
Popper, disons-le en passant, nous permet d'échapper au relativisme intégral.
Qu’aucune théorie ne puisse jamais, en toute rigueur, être vérifiée expérimentalement,
cela ne veut pas dire qu’elles se valent toutes : puisqu'elles peuvent au
moins être réfutées ou falsifiées, puisqu'elles le sont en effet, voyez l’histoire des
sciences, et se succèdent ainsi dans un ordre à la fois irréversible et normatif qui
est celui du progrès même de nos connaissances. Cela rejoint une des fulgurances
de Pascal: nous ne connaissons jamais la vérité directement, mais
devons « prendre pour véritables les choses dont le contraire nous paraît faux »
({\it De l'esprit géométrique}, p. 352 b ; voir aussi la pensée 905-385). C’est le véritable
ordre, qui se fait « par approfondissement et ratures », comme disait
Cavaillès, et qui n’en instaure pas moins — ou plutôt qui n’en instaure que
mieux — « des résultats dont la validité est hors du temps » (Cavaillès, {\it Sur la
logique}, WI ; {\it Lettre à P. Labérenne}, 1938). C’est parce qu’il y a une histoire des
sciences (et non malgré cette histoire) que les sciences ne se réduisent pas à leur
historicité, contrairement à ce que croyait Montaigne, et nous ouvrent — dans
l’histoire, par l’histoire — à quelque chose qui la dépasse. Quoi ? L’éternité. Il y
eut d’abord Ptolémée, certes, puis Newton, puis Einstein... Mais cette succession,
parce qu’elle n’est ni contingente ni réversible, nous ouvre à un univers où
l’idée même de succession perd de sa pertinence ou, en tout cas, de sa portée.
Entre Ptolémée et Einstein, ce n’est pas la vérité qui a changé ; c’est la connaissance
que nous en avons. La vérité, elle, ne change pas, même quand elle est la
vérité d’un univers où tout change. C’est ce qu'avait vu Spinoza : toute vérité
est éternelle, et elle seule. C’est ce qu'avait vu Pascal : quelque respect qu’on ait
pour les Anciens, expliquait-il, « la vérité doit toujours avoir l’avantage,
quoique nouvellement découverte, puisqu’elle est toujours plus ancienne que
toutes les opinions qu’on en a eues, et que ce serait ignorer sa nature que de
s’imaginer qu’elle ait commencé d’être au temps qu’elle a commencé d’être
connue » ({\it Sur le traité du vide}, p. 232 b). C’est aussi ce qu'avait vu Frege, par
%— 616 —
d’autres voies. La vérité n’a pas besoin d’être connue pour être vraie (« elle n’a
besoin d’aucun porteur »), et c’est en quoi « l’être vrai d’une pensée est indépendant
du temps » ({\it Écrits logiques}, p.184 et 191). Soit un fait éphémère
quelconque : j'écris l’article « Vérité » de mon {\it Dictionnaire philosophique} ; ou
bien, c'était l'exemple de Frege : il y a devant ma fenêtre un arbre couvert de
feuillage vert. Rien de cela ne durera bien longtemps. Mais jamais la vérité qui
s’y dévoile ne deviendra fausse ou mensongère. Dès lors qu’il est vrai que cet
arbre, ici et maintenant, est vert, cette vérité est éternelle : que cet arbre soit
vert, en ce moment que je dis, cela sera vrai encore quand il aura perdu ses
feuilles ou sera mort. C’est en quoi le présent, dans {\it « est vrai »}, n’indique pas
« le présent de celui qui parle, remarque Frege, mais, si l’on permet l’expression,
un {\it tempus} de l’intemporalité » ({\it ibid.}, p. 193). Bref, toute vérité est éternelle,
quoique aucune connaissance ne le soit, et c’est ce qui interdit de
confondre les connaissances (toujours historiques) et les vérités (toujours éternelles)
sur quoi elles portent.

Où veux-je en venir ? À ceci que renoncer à la vérité c’est renoncer à
l'éternité en même temps qu’à l’être, ce qui nous sépare du monde même où
nous sommes et du seul lieu possible du salut. Il n’y a que la connaissance qui
libère et qui sauve. À la gloire d’Épicure et de Spinoza : l'éternité c’est maintenant,
le salut c’est le monde, mais pour autant seulement que nous l’habitons
en vérité.

Quant aux raisons pratiques du discrédit actuel de l’idée de vérité, elles
tiennent à mon sens à l'impossibilité où nous sommes, au moins depuis Hume,
de combler l'écart qui sépare l’être du devoir-être, le vrai du bien, disons les
vérités des valeurs. Sur ce point, fort débattu, je ne transige guère. Si la vérité
est l’être ({\it alèthéia}) ou l'adéquation à l’être ({\it veritas}), je ne vois pas comment elle
pourrait le juger ou porter sur ce qui doit être. C’est où Hume et Spinoza se
rejoignent, malgré tout ce qui les oppose, et je n’ai jamais pu, sur ce point
majeur, m’éloigner d’eux. Une vérité est l’objet au moins possible d’une
connaissance ; une valeur, l’objet au moins imaginaire d’un désir. Cela nous
introduit dans deux ordres différents — l’ordre théorique, l’ordre pratique —, qui
ne pourraient être conjoints qu’en Dieu ou dans un sujet transcendantal. Mais
je ne crois ni à l’un ni à l’autre. Est-ce à dire que nous sommes voués à la
schizophrénie ? Nullement, puisque nous pouvons désirer le vrai et connaître
nos désirs, au moins partiellement, puisque nous ne cessons d’y tendre — fût-ce
pour constater, comme presque toujours, l’abîme qui les sépare. C’est ce qui
nous fait hommes et qui nous voue à la philosophie. Le contraire de cette schizophrénie,
qui serait autrement le lot de notre époque, c’est l'amour de la
vérité, qui est à la fois une vertu morale et une exigence intellectuelle.

%— 617 —
VÉRITÉ ÉTERNELLE Elles le sont toutes. C’est donc un pléonasme, mais
utile par ce qu’il souligne. Ce qui est vrai aujour-
d’hui le sera encore demain, ou bien ne l'était pas aujourd’hui. Il y a trois arbres
dans le champ : vérité éternelle. Dans dix mille ans, ces arbres n’y seront plus,
ni le champ sans doute ; mais il sera vrai toujours qu’ils y furent. L’éternité est
ainsi ce qui distingue le {\it vrai} du {\it réel} (ou le temps, ce qui distingue le {\it réel} du
{\it vrai}). Car le réel change, dans le temps : trois arbres, un champ, puis plus
d'arbres, puis plus de champ... On ne se baigne jamais deux fois dans le même
fleuve réel. Mais qui une fois s’y est baigné, éternellement cela restera vrai. Les
hommes passent, et les fleuves, et le réel. La vérité ne passera pas. Le vrai est
ainsi l'éternité du réel (ce pourquoi ils coïncident dans le présent) : c’est le réel
{\it sub specie aeternitatis}. On serait tenté de dire : et le réel, {\it l'image mobile} du vrai.
Mais ce serait s’enfermer déjà dans le platonisme. Ce qu’il faut comprendre ici,
et qui est la grande difficulté, c’est que le vrai et le réel sont en réalité (en vérité)
la même chose : parce que le temps n’est rien que le présent, qui est l'éternité
même.

VERTU La vertu est l'effort pour se bien conduire, qui définit le bien par cet
effort même. Ce n’est pas l'application d’une règle qui lui préexiste-
rait, encore moins le respect d’un interdit transcendant : c’est la réalisation, à la
fois normée et normative, d’un individu, qui devient à lui-même sa propre
règle en ne s’interdisant que ce qu’il juge indigne de ce qu’il est ou veut être.

Le mot {\it arétè}, que les latins traduisaient par {\it virtus}, signifie d’abord une puissance
ou une excellence. Par exemple la vertu d’un couteau est de couper, la
vertu d’un médicament de soigner, et la vertu d’un être humain de vivre et
d’agir humainement. C’est où l’on rencontre la vertu morale ou éthique. C’est
une puissance, mais normative. C’est une excellence, mais en acte. C’est une
disposition acquise (on ne naît pas vertueux, on le devient) à faire le bien, disait
Aristote, c’est-à-dire à faire ce qu’on doit, quand on le doit, comme on le doit,
mais sans autre guide, et presque sans autre règle, que sa vertu même. Cela ne
va pas sans raison, mais pas non plus sans volonté. Cela ne va pas sans effort,
mais pas non plus sans plaisir ou sans joie. Celui qui donne sans plaisir n’est pas
généreux : ce n’est qu’un avare qui se force. Celui qui résiste à la débauche sans
plaisir n’est pas tempérant : il n’est que continent et frustré.

On sait qu’Aristote définissait la vertu comme juste milieu (ou médiété) entre
deux extrêmes opposés mais tous les deux vicieux (quoiqu’ils puissent l'être inégalement),
« l'un par excès et l’autre par défaut » ({\it Éthique à Nicomaque}, II, 5-6,
1106 b — 1107 a). Ainsi le courage, entre la témérité et la couardise : le téméraire
prend des risques inconsidérés (il pèche par excès), le couard n’en prend pas assez
%— 618 —
(il pèche par défaut) ; l'homme courageux prend les risques qu’il faut, comme il
faut, quand il faut.-On se trompe évidemment si l’on y voit l’apologie de la tiédeur,
du centrisme ou de la médiocrité. Le juste milieu est un extrême, lui aussi,
mais vers le haut : c’est un sommet, c’est une perfection ({\it ibid.}), comme une ligne
de crête entre deux abîmes, ou entre deux marais.

« Par vertu et puissance j'entends la même chose, écrit Spinoza : la-vertu, en
tant qu’elle se rapporte à l’homme, est l’essence même ou la nature de l’homme
en tant qu’il a le pouvoir de faire certaines choses se pouvant connaître par les
seules lois de sa nature » ({\it Éthique}, IV, déf. 8 ; voir aussi la démonstration de la
prop. 20). C’est une occurrence du {\it conatus}, et sa forme spécifiquement
humaine. La vertu, c’est la puissance de vivre et d’agir humainement, au sens
normatif du terme, c’est-à-dire « sous la conduite de la raison » (IV, 37, scolie)
et conformément au « modèle de la nature humaine » (IV, Préface) que nous
nous sommes fixé. La raison n’y suffit pas : ce n’est pas elle qui fait agir, mais
le désir. Le désir n°y suffit pas : encore faut-il désirer la raison ou (cela revient
au même) la liberté, et en être capable. Ainsi le désir de vertu (comme puissance,
non comme manque) est la vertu même, mais en tant seulement qu’il
agit. Le {\it conatus} est « la première et unique origine de la vertu» (IV, 22,
corollaire) : c’est tendre vers son bien (IV, 18, scolie), qui est aussi celui de
l'humanité (IV, 36-37), et le réaliser par là (IV, 73, scolie). La vertu est un
effort réussi : c’est la puissance en acte, en vérité et en joie.

VEULERIE Complaisance de soi à soi. C’est se résigner trop vite à sa propre
médiocrité, au point de ne plus la voir, au point de la prendre
pour une espèce de vertu. Cela vaut mieux que la honte ? Je ne sais, ou plutôt
je n’en crois rien. La honte est une souffrance, mais qui peut faire avancer (voir
Spinoza, {\it Éthique}, IV, 58, scolie). La veulerie serait plutôt un confort, qui freine
et enferme. Le veule est incapable de se dominer, de se commander, de se surmonter.
Il s'aime comme il est, mais en oubliant cette puissance en lui — la
volonté — qu’il est aussi, et qu’il doit être. « Là où ça était, disait Freud, je dois
advenir. » Le veule se croit déjà advenu, comme d’autres se croient déjà arrivés.
Il prend son {\it moi} pour un destin, au lieu d’y voir un enjeu, un combat, une
tâche. Narcissisme mou, ou mollesse narcissique. J’y vois l’un des péchés capitaux,
et le contraire de Pexigence.

VICE Le contraire de la vertu, qu’elle surmonte et contre quoi elle se définit :
c’est une disposition au mal, comme la vertu est une disposition
au bien.
%— 619 —
Aristote nous a habitués à penser que les vices allaient par deux, qui s’opposent
l'un à l’autre et tous les deux — l’un par excès, l’autre par défaut — à une
même vertu. Ainsi la témérité et la couardise, la vanité et la bassesse, la prodigalité
et l’avarice : c’est contre l’un et l’autre de ces deux vices opposés que le
courage, la grandeur d’âme et la libéralité instaurent un juste milieu (une
médiété : {\it mésotès}) qui est aussi un sommet.

Ce n’est que par un triste contresens, qui a moins à voir avec la morale
qu'avec la religion ou la pudibonderie, qu’on a pu faire du vice une espèce de
synonyme péjoratif de la sexualité, ou considérer la sexualité, cela revient au
même, comme intrinsèquement vicieuse. Les Grecs, qui aimaient le corps et
le plaisir, n’avaient aucune raison de diaboliser le sexe. Simplement ils ne
pouvaient envisager qu’on lui soumette le tout d’une existence ou qu’on en
fasse, comme certains aujourd’hui et tout aussi absurdement, une exigence
suprême. Ni l’obsédé sexuel ni le pudibond, ni le débauché ni le peine-à-jouir
ne sont enviables ou admirables. Les uns pèchent par excès, les autres par
défaut de sensualité. Entre ces deux abîimes ou ces deux marécages, reste à
inventer la ligne de crête du plaisir cultivé, maîtrisé, partagé — la vertu des
amants.

VIE La plus belle définition que j'en connaisse est celle de Bichat : « La vie
est l’ensemble des fonctions qui résistent à la mort » ({\it Recherches physiologiques},
I, 1). C’est une occurrence du {\it conatus}, mais spécifique : une certaine
manière, pour un être donné, de persévérer dans son être en le développant
(croissance), en le reconstituant (par des échanges avec son milieu : nutrition,
respiration, photosynthèse.….), en s’adaptant, enfin en tendant à se reproduire
(génération). Vivre, c’est faire l’effort de vivre : {\it le dur désir de durer} est le vrai
goût en nous de la vie, et le principe, montre Spinoza, de toute vertu ({\it Éthique},
IV, prop. 21, 22 et corollaire).

Le mot désigne aussi la durée de cet effort — ce qui sépare la conception de
la mort. Une vie vaut moins par cette durée, pourtant, que par ce qu’on en fait.
Du moins en va-t-il ainsi pour la plupart des humains : le bonheur, non la longévité,
est le but ; l'humanité, non la santé, est la norme. C’est où l’on s'éloigne
de Bichat ou de la biologie pour retrouver Montaigne et la philosophie. « La
mort est le bout, non le but de la vie ; c’est sa fin, son extrémité, non pourtant
son objet. Elle doit être elle-même à soi sa visée : son dessein, sa droite étude
est se régler, se conduire, se souffrir » ({\it Essais}, III, 12). Apprendre à mourir ? À
quoi bon, puisqu'on y parviendra de toute façon ? Mais apprendre à vivre :
c’est la philosophie même.

%— 620 —
VIEILLESSE Le vieillissement est l’usure d’un vivant, laquelle diminue ses
performances (sa puissance d’exister, de penser, d’agir..) et le
rapproche de la mort. C’est donc un processus, dont on remarquera qu'il est
moins une évolution qu’une involution, moins un progrès qu’une dégradation,
moins une avancée qu'un recul. La vieillesse est l’état qui résulte de ce
processus : état par définition peu enviable (qui ne préférerait rester jeune ?), et
pourtant, pour presque tous, préférable à la mort. C’est que la mort n'est rien,
quand la vieillesse est encore quelque chose.

Je ne crois guère aux avantages de la vieillesse, encore moins à sa valeur ou
grandeur (malgré Hugo) intrinsèques. Qu’on puisse progresser en vieillissant,
chacun peut le constater autour de soi, comme il peut constater que cela reste
toutefois l’exception. Même alors, d’ailleurs, ce n’est pas grâce à la vieillesse
qu’on progresse ; c’est malgré elle, et contre elle bien souvent. Un gain d’expérience,
de maturité, de culture ? On le doit moins à la vieillesse qu’à la vie, qui
continue malgré tout, moins à l’usure qu’à la résistance, moins à l’âge qu'on a
qu'aux années qui y menèrent, où on ne l’avait pas. La vie est une richesse. Le
temps est une richesse. Être vieux, non : ce n’est que le temps qui manque et la
vie qui s’en va. Qu'on ait davantage d’expérience à soixante-dix ans qu'à vingt,
c'est une donnée de fait, que l’arithmétique suffit à expliquer, mais qui n’est
pas liée directement à la vieillesse : si nous étions programmés autrement par
nos gènes, nous pourrions avoir soixante-dix ans sans être vieux pour autant, de
même que nous pourrions être vieux, comme dans beaucoup d'espèces animales,
dès quinze ou vingt ans. On se trompe quand on réduit la vieillesse à un
âge : que les deux, en fait, aillent presque toujours de pair, cela n’empêche pas
qu’il s’agisse, en droit, de deux réalités différentes. On sait qu’il existe quelques
pathologies très rares qui entraînent une accélération du vieillissement, jusqu’à
faire un vieillard d’un homme de trente ans. Et nul n’ignore qu’il y a des octogénaires
presque intacts, plus verts et ouverts que bien des jeunes gens. C'est
qu’ils ne sont pas encore vieux, ou moins que leur âge ne le laisserait supposer.
Toutefois ces exceptions heureuses ne doivent pas cacher la règle qu’elles
confirment : chez presque tous, le temps, à partir d’un certain âge, entraîne une
dégradation irréversible, qu’on peut parfois ralentir mais qu’on ne saurait
empêcher. Que ce soit physiquement ou intellectuellement, la plupart sont
moins performants à quarante ans qu’à vingt, à soixante qu'à quarante, à
quatre-vingts qu'à soixante... C’est une espèce d’entropie à la première
personne : dans un organisme vivant, passé le cap de la maturité, le désordre et
la fatigue tendent vers un maximum. Le vieillissement est cette tendance ; la
vieillesse, son résultat. Cela n’empêcha pas Kant d’écrire la {\it Critique de la faculté
de juger} à soixante ans passés, ni Hugo, à quatre-vingts ans, de garder le génie
et la vitalité que l’on sait. Mais ils le doivent plus à leur santé qu’à leur vieillesse.
%— 621 —
Et plus à la chance qu’au génie. Montaigne, qui vécut plutôt vieux pour son
époque, qui ne se mit à écrire qu'assez tard, ne s’est jamais illusionné sur les
bénéfices de la vieillesse :

« Je hais cet accidentel repentir que l’âge apporte. Celui qui disait anciennement
être obligé aux années de quoi elles l'avaient défait de la volupté, avait autre opinion
que la mienne ; je ne saurai jamais bon gré à l’impuissance de bien qu’elle me fasse.
[...] Je serais honteux et envieux que la misère et défortune de ma décrépitude eût à se
préférer à mes bonnes années saines, éveillées, vigoureuses ; et qu’on eût à m’estimer
non par où j'ai été, mais par où j'ai cessé d’être. [...] La vieillesse nous attache plus de
rides en l'esprit qu’au visage ; et ne se voit point d’âmes, ou fort rares, qui en vieillissant
ne sentent à l’aigre et au moisi. L'homme marche entier vers son croît et son décroît »
({\it Essais}, III, 2).

Celui-là aimait trop la vie et la vérité pour dire du bien de la vieillesse. Il se
contentait de l’accepter sereinement. Cela me paraît, concernant la vieillesse,
une ambition suffisante. La mort ramasse les copies, mais ne les note pas.

VIOLENCE C'est l’usage immodéré de la force. Elle est parfois nécessaire (la
modération n’est pas toujours possible), jamais bonne. Toujours
regrettable, pas toujours condamnable. Son contraire est la douceur
(qu’on ne confondra pas avec la faiblesse, contraire de la force). La douceur est
une vertu ; la faiblesse, une faiblesse ; la violence, une faute — sauf quand elle
est indispensable et légitime. Contre les faibles ou les doux, la violence est
impardonnable : ce n’est que lâcheté, cruauté, bestialité. Contre les violents, en
revanche, on ne peut se l’interdire absolument : ce serait laisser libre cours aux
barbares ou aux voyous. La non-violence ? Elle n’est bonne, souligne Simone
Weil, que si elle est efficace. Cela indique assez le but et le chemin : « S’efforcer
de substituer de plus en plus dans le monde la non-violence {\it efficace} à la
violence » ({\it La pesanteur et la grâce}, « Violence »). Cela suppose beaucoup de
maîtrise, de courage, d'intelligence, mais « dépend aussi de l’adversaire » ({\it ibid.}).
Gandhi, contre les Anglais, est admirable. Mais cela ne donne pas tort aux
Résistants, contre les nazis, ni aux Alliés, contre la Wehrmacht. La violence
n’est acceptable que lorsque son absence serait pire. Elle l’est donc parfois.
Reste à la limiter, à la contrôler, à l’encadrer. C’est pourquoi on a besoin d’un
État, pour exercer, comme disait Max Weber, « le monopole de la violence
légitime » : on a besoin d’une armée (pour se défendre contre la violence extérieure),
d’une police (contre la violence intérieure), de lois, de tribunaux, de
prisons. Et besoin aussi, entre les individus, d’une paix au moins minimale.
Le contraire de la violence, c’est la douceur ; mais son antidote, à l'échelle de la
%— 622 —
Cité, c’est l’art de gérer les conflits avec le moins de violence possible : police,
politesse, politique.

VIRTUEL Ce qui n’existe qu’en puissance (mais mieux vaut dire alors {\it potentiel})
où qu’en simulation. Le mot, dont on nous rebat les oreilles,
vient de {\it virtus} : c’est comme un doublet de notre {\it vertu}, et point par hasard. Il
y a puissance dans les deux cas. Mais la vertu est une puissance en acte. La
virtualité, une puissance qui reste en puissance. La vertu est puissance incarnée,
quand le virtuel, trop souvent, se contente des images. La vertu est de l’homme,
quand le virtuel, de plus en plus, appartient aux machines. La vertu est courage,
quand le virtuel ne sait, au mieux, qu'être sans danger. La vertu est justice, quand
le virtuel ne connaît, au mieux, que la justesse. La vertu est aimante, quand le virtuel
ne sait, au mieux, qu'être aimable.
C’est donc la vertu qui est bonne, et qui importe. Comme il serait triste et
coupable de ne vivre que virtuellement !

VITALISME C'est expliquer la vie par elle-même (ou par un « principe
vital »), donc renoncer à l'expliquer. S’oppose en cela au matérialisme,
qui explique la vie par la matière inanimée, et se distingue de l’animisme,
qui l’explique par une âme immatérielle.

VOLITION L'acte de vouloir. Cela suppose un désir, mais ne s’y réduit pas
(toute volition est désir, tout désir n’est pas volition) : vouloir,
c'est désirer en acte. C’est pourquoi nous ne pouvons vouloir que ce qui
dépend de nous, et à condition seulement de le faire. Essayez un peu de vouloir
vous lever sans vous lever en effet. Il faudrait être paralytique ou ligoté ; mais
alors vous lever, pour vous, ne serait plus une volition, mais un simple désir,
voire un regret ou une espérance. Vouloir, c’est faire. Une volonté qui n’agit
pas n’est plus une volition, ni même tout à fait une volonté : c’est un projet, un
vœu, ou une lâcheté.

VOLONTÉ La faculté de vouloir : l’acte en puissance ou la puissance en
acte.
On ne la confondra pas avec le désir, qui est son genre prochain. On peut
désirer simultanément plusieurs choses contradictoires (par exemple fumer et
ne pas fumer), mais point les vouloir : parce qu’on ne veut vraiment que ce
%— 623 —
qu’on fait, et que nul ne peut, au même moment, faire et ne pas faire la même
chose. La volonté est une certaine espèce de désir : c’est un désir dont la satisfaction
dépend de nous. « Mais si j’échoue ? » Cela n’y change rien : la volonté
portait sur l’action, non sur la réussite (qui n’était l’objet que d’une espérance).
Toute volonté est puissance de choix : c’est le pouvoir déterminé de se déterminer
soi-même. Cela distingue assez la volonté du libre arbitre (qui serait le
pouvoir {\it indéterminé} de se déterminer soi), de l’espérance, qui désire plus qu’elle
ne peut, enfin de la veulerie, qui renonce à choisir. Par quoi la volonté n’est pas
seulement une faculté ; c’est aussi une vertu.

VRAI Ce qui est, ou ce qui est conforme à ce qui est. On peut distinguer ces
deux sens, en parlant respectivement de {\it veritas rei} et de {\it veritas intellectus},
comme faisaient les scolastiques (la vérité de la chose, la vérité de l’entendement),
ou bien en distinguant avec Heidegger l’{\it althéia} (la vérité comme
dévoilement de l’étant, ce que j’appellerais plutôt la pure {\it présentation} du réel)
et la {\it veritas} (la vérité comme accord ou correspondance entre la pensée et le
réel : l’{\it ad{\ae}quatio rei} et {\it intellectus} des scolastiques, qui n’est vérité que d’une
{\it représentation}). L'{\it alèthéia} serait la vérité originaire, telle qu’on la trouve chez les
présocratiques. La {\it veritas} n’apparaîtrait, même en grec, qu'avec Platon : ce
serait la forme, à la fois logique et métaphysique, de l'oubli de l’être, au bénéfice
de l’humanisme (le vrai n'étant plus vérité de l’étant mais vérité de
l’homme). Ces deux conceptions n’en restent pas moins solidaires, et même
indissociables. Soit par exemple la table sur laquelle j'écris. Je ne peux rien
penser de son {\it alèthéia} sans passer par la {\it veritas} d’un discours. Mais pas davantage
dire sa {\it veritas} sans supposer son {\it alèthéia}. Supposons que j'énonce à son
propos un certain nombre de propositions vraies, fussent-elles approximatives,
portant par exemple sur sa forme, sa superficie ou son poids. Ces propositions
sont vraies (au sens de la {\it veritas}) si et seulement si elles correspondent à la réalité
(à l'{\it alèthéia}). C’est ce que Tarski appelle la conception sémantique de la
vérité : la proposition « La neige est blanche » est vraie si et seulement si la
neige est blanche ; la proposition « Cette table est rectangulaire » est vraie si et
seulement si cette table est rectangulaire. Mais si la neige et cette table n’étaient
{\it vraiment} ce qu'elles sont, cette adéquation elle-même n’aurait ni sens ni vérité.
Une pensée ne peut être conforme à ce qui est (c’est-à-dire vraie au sens de la
{\it veritas}) que si ce qui est est vraiment ce qu'il est (c’est-à-dire vrai au sens de
l’{\it alèthéia}). Les deux notions, ou les deux faces de la notion, restent pourtant
toutes les deux problématiques, mais pour des raisons différentes. L’{\it alèthéia},
parce qu’on ne peut rien en dire qui ne relève de la {\it veritas}. La {\it veritas}, parce que
toute conformité entre la pensée et le réel est par définition indémontrable,
%— 624 —
puisqu’on ne connaît du réel que ce qu’on en pense. Cela ne prouve pas que
tout ce que nous pensons soit faux, mais nous interdit de prouver absolument
que telle ou telle de nos pensées soit vraie. À la gloire du pyrrhonisme.

VULGARITÉ Une bassesse commune et de mauvais goût.
{\it Vulgus}, en latin, c’est la foule, le commun des hommes, les
hommes du commun. {\it Vulgaire} serait donc synonyme à peu près de {\it populaire},
et le fut en effet longtemps. Mais le peuple est souverain ; la foule, non. De là
peut-être, dans un univers démocratique, l’évolution de plus en plus divergente
des deux mots. Être vulgaire, ce n’est pas être du peuple, ni apprécié par lui ;
c’est manquer d’élévation, d'élégance, de distinction, de noblesse. La popularité
est une chance ou un risque. La vulgarité, un confort et une tare. On se trompe
quand on la confond avec la grossièreté (on peut dire des gros mots sans être
vulgaire, être vulgaire sans en dire). Mais plus encore si on y voit une audace ou
une force. Ce n’est que suivre la pente, en allant toujours vers Le plus facile, le
plus bas, le plus racoleur : c’est ne plaire qu’à la partie déplaisante de soi, et de
tous. La notion, toutefois, relève davantage de l’esthétique que de la morale.
Un brave homme peut être vulgaire ; un salaud ne l’être pas. C’est que la vulgarité
touche moins aux actes qu’aux manières, moins aux sentiments qu'à la
sensibilité. Être vulgaire, c’est presque toujours ignorer qu’on l’est. C’est être
prisonnier de sa propre bassesse, au point qu’on ne la perçoit plus. C’est être
une foule à soi tout seul. Ce serait un péché capital, si c'était un péché. Mais ce
n’est qu’une faute de goût.

WAGNÉRIEN Disciple ou zélateur de Wagner. C’est une forme redoutable
de mélomane, qui prend la musique pour une conception du
monde, l’opéra pour une religion, et Wagner pour un Dieu. Ces trois erreurs
font une espèce de système, qui les rend sourds. À moins que ce ne soit
l'inverse.

Nietzsche a écrit contre cette {\it maladie}, comme il dit, et contre le génie
subtil et dangereux de Wagner, quelques-unes de ses plus belles pages, qui rendent
à Mozart (« le génie gai, enthousiaste, tendre et amoureux de Mozart »:
l'hommage qu’il mérite — et à Bizet, bien davantage qu’il ne mériterait.

{\it WELTANSCHAUUNG} Vision du monde, en allemand. C’est une espèce de
philosophie spontanée ou implicite : un ensemble
d’intuitions, de croyances, d’idées vagues, avec, dans la bouche d’un Français,
%— 625 —
je ne sais quoi de prétentieux et d’obscur, qui tient à la langue utilisée, comme
s’il suffisait de parler allemand pour être plus intelligent. La philosophie du
pauvre ? Plutôt l’idéologie du riche, ou du snob.

XÉNOPHOBIE La haine de l'étranger. C’est une forme de bêtise qui consiste
à se croire chez soi. La chose semble instinctive aux bêtes, et
naturelle aux hommes. La philosophie, qui nous fait tous étrangers, combat
cette illusion. Et la sagesse, qui suppose le dépassement du {\it soi}, la dissout. La
haine s’éteint alors, en même temps que la peur.

YOGA Travail du corps, d’origine indienne, qui tend au repos de l'esprit.
Ascèse, qui tend à la délivrance. Méditation, qui tend au silence.
C’est faire passer la bête sous le joug (les deux mots ont la même racine indoeuropéenne),
mais pour la libérer, ou se libérer soi. C’est n’être qu’un avec son
corps, pour n'être qu’un avec tout. Le yoga est ainsi l’homologue fonctionnel de
la philosophie, qui tend au même résultat mais travaille plutôt sur l'esprit ou les
concepts. Le yogi croit davantage aux postures, aux mouvements, à la respiration,
à la concentration, à l'attention absolument pure. Le but est de s’affranchir du
mental pour atteindre à la conscience absolue ou inconditionnée. L'efficacité sur
l’âme d’une telle pratique corporelle et spirituelle n’est plus à démontrer, ni très
mystérieuse. Si le corps et l’âme sont une seule et même chose, comme dit Spinoza,
le yoga n’est jamais qu’une façon, historiquement située, de penser juste. Il
semble, grande leçon, que le penseur s’y perde, et s’y sauve.

ZÈLE C’est un soin jaloux ou fervent, pour la cause d’un autre, comme si
l’on craignait de n’en faire jamais assez — au point parfois (par exemple
dans les grèves du zèle) d’en faire trop. « Le zèle consiste à faire plus qu’on ne
doit strictement », comme disait Alain, voire plus qu’on ne devrait. C’est que
la frontière, entre le zèle et l’excès de zèle, reste floue. En faire tant, n’est-ce pas
déjà en faire trop ? Et pour quelle raison ? Par générosité ? Par dévouement ?
Par conscience professionnelle ? C’est plus souvent une peur d’être blâmé ou
un désir de plaire, qui rendent le zèle, même efficace, un peu suspect. Les
patrons n’en sont pas dupes. Les collègues, encore moins.

ZEN C'est une forme de bouddhisme, relevant du Grand Véhicule et dérivée
du Tch’an chinois, qui s’est développée au Japon. On y cherche
%— 626 —
l’illumination (Le {\it satori}) par la méditation assise et sans objet ({\it zazen}), laquelle
peut elle-même être préparée, ou accompagnée, par un certain nombre d’exercices
(les {\it kôans}, le tir à l'arc, l’art floral, les arts martiaux...). Le but est
d'atteindre une attention absolument pure, qui crée, ou plutôt qui laisse se
déployer, un état de paix et de vide intérieur. Ceux qui l’ont vécu en parlent
comme d’une expérience de plénitude. Il s’agit d’observer, de façon neutre et
tranquille, son propre fonctionnement, aussi troublé soit-il, jusqu’à expérimenter
qu’il n’y a rien là de substantiel à observer (que tous ces processus sont
impermanents et vides). Le réel n’en continue pas moins, ou plutôt il n’en
continue que mieux — parce qu’on n’en est plus séparé par l’ego. C’est se vider
de soi, pour qu’il n’y ait plus que tout.

ZÉTÉTIQUE Du grec {\it zétêtikos}, qui cherche ou qui aime chercher. C’est un
autre nom pour désigner le scepticisme, ou plutôt sa méthode,
qui consiste à chercher toujours la vérité, sans rien affirmer, même pas l’impossibilité
de l’atteindre (voir Sextus Empiricus, {\it Hypotyposes pyrrhoniennes}, I et
III). Se distingue par là du dogmatisme, qui croit avoir trouvé, mais aussi de la
sophistique, qui renonce à chercher.

À quoi bon chercher, demandera-t-on, si on ne peut trouver ? C’est qu’on
ne peut savoir autrement si on le peut, ni ce qu’on cherche.

Et comment le dire, si on ne l’a point trouvé ? En disant au moins le mouvement
de sa quête, sans l'arrêter et sans y croire tout à fait. Pyrrhon, qui est le
maître de l’école zététique, était amené pour cela à ne parler qu’ironiquement,
ou plutôt (puisqu'il s’agit, note Marcel Conche, d’une « ironie à l’égard de lui-même »)
qu'avec humour. Les mots ne sont qu’un moment de l'apparence pure
et universelle, comme dit Marcel Conche, qu’un moment du devenir, comme
je préférerais dire, qu’ils ne sauraient ni contenir tout entier ni transformer — si
ce n’est illusoirement — en essences fixes ou immuables. Aussi l’{\it ataraxia} (la
sagesse, la paix de l’âme) n’allait-elle pas, pour Pyrrhon, sans {\it aphasia} (le non-discours,
le silence). Non qu’on ne puisse parler ({\it aphasia} n’est pas aphasie), ni
qu’on ne le doive, mais parce que « les mots jamais ne sauraient annuler le
silence » (M. Conche, {\it Pyrrhon ou l'apparence}, X, 1).

Il est juste de terminer par là un recueil de définitions. Car cela seul mérite
d’être dit, qui n’en a pas besoin. C’est où la philosophie, qui est un certain type
de discours, conduit à la sagesse, qui est une certaine qualité de silence. Ce dont
on peut parler, et cela seul, on peut aussi le taire.

