
%%%%%%%%%%%%%%%%%%%%%
\chapter{John Rawls, encyclopedie de la philosophie}
%%%%%%%%%%%%%%%%%%%%%
Rawls John (Baltimore, Maryland, 1921)
philosophe américain. Professeur à Baltimore, il a proposé dans de nombreux
écrits ({\it Justice as Fairness, The Journal of
Philosophy}, n° 22, {\it }La Justice comme
équité, 1957; {\it }Distributive Justice in
P. Lasen et W.G. Runcinan (ed.) {\it Philosophy, Politics and Society, La Justice distributive}, 1967 ;
 {\it À Theory of Justice, Théorie
de la justice}, 1971) une théorie de la justice distributive qui a eu une large
influence. L'objectif de Rawls est de
mettre en évidence les critères qui règlent
l'attribution des coûts et des bénéfices
parmi les participants à l’entreprise
sociale. Les critères proposés doivent être
justifiés rationnellement. A cette fin,
Rawls présente deux arguments qu’il
développe : parallèlement dans son
exposé : l'argument du contrat et l’argument intuitif. Le premier reprend de
manière originale le modèle classique du
contrat social. Il fonde les principes de la
justice sur la démonstration selon laquelle
ces principes seraient préférés s'ils faisaient l'objet d’un choix rationnel de la
part de citoyens hypothétiques. Mais le
choix des principes ne doit pas être seulement rationnel, il doit être également
moral : c’est pourquoi les termes du choix
doivent être précisés de manière à garantir l'équité des conditions qui, à leur tour,
garantiront l'équité du choix en tant que
tel. L’argument intuitif veut, quant à lui,
montrer le caractère plausible, d’un point
de vue intuitif, aussi bien de la manière
dont on décrit cette situation de choix
(position originaire) que des principes de
la justice. La position originaire prévoit
que les sujets du choix soient dotés d’une
égale liberté, qu'ils soient rationnels (dans
%
le sens classique de la théorie économique), qu'ils soient réciproquement
désintéressés et qu'ils ignorent tout de
leurs conditions sociales respectives : les
seules indications dont ils disposent ne
sont que générales. Ce dispositif dit du
« voile d’ignorance » transforme des individus modérément égoïstes en ce que
Kant appelait des personnes morales,
c'est-à-dire animées seulement par des
considérations universelles, mais non pas
par des intérêts privés. Nul ne sait ce que
sera son niveau de vie mais, quel qu’il
soit, il exigera certaines ressources fondamentales, les biens principaux, comme la
santé, la liberté, un revenu, certaines
compétences et une certaine dose d’estime de soi. Pour chacun des sujets du
choix, il est donc rationnel de maximiser
les biens principaux. Par ailleurs, comme
on ne sait pas quelle position sociale on
pourra occuper, il est rationnel de choisir,
comme critère de distribution des biens,
le principe de maximum, qui maximise les
avantages pour les positions désavantagées. Conformément à cette caractérisation de la situation du choix, les principes
de justice mis en jeu sont alors au nombre
de deux. Selon le premier (principe d’égalité), tous ont un droit égal à la plus large
liberté compatible avec celle d'autrui. Ce
principe est prioritaire, car c’est de la
liberté que dépendent la possibilité de
choisir son niveau de vie et le respect de
soi. Le second (principe de différence) est
le véritable principe distributif : les seules
inégalités admissibles sont celles qui peuvent produire des avantages pour les
désavantagés, et qui sont liées à l’ouverture universelle des charges et offices.
L'œuvre de Rawls (en même temps que
celle de Thomas Nagel par exemple) a
représenté un retour assumé de la
recherche éthique analytique vers des
thèmes et des problèmes propres à
l'éthique normative, après une longue
période de concentration presque exclusive sur des questions de caractère métaéthique (R. Hare, Ch. L. Stevenson).
%%%%%%%%%%%%%%%%%%%%%%%%%%%%%%%%%%%%%%%%%%%%%%%%%%%%%%%%%%%%%%%%%%%%%%%%%%%%%%%%%%%%%
