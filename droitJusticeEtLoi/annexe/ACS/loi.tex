
%%%%%%%%%%%%%%%%%%%%% 345
\section{Loi}
%%%%%%%%%%%%%%%%%%%%%
%{\it }
Un énoncé universel et impératif. En ce sens, il est clair qu’il n’y a pas
de « lois de la nature » : on n’en parle que par analogie, pour évoquer
ou expliquer certaines régularités observables. La rationalité de l’univers est au
contraire toute silencieuse (pas d’énoncés) et simple (pas d’impératifs). L’identité
est ordre qui lui suffit. Et sa nécessité, s’il fallait la formuler, ne se dirait
qu’à l’indicatif. Les lois humaines tendent vers ce modèle (« tout condamné à
mort aura la tête tranchée. »), sans pouvoir jamais complètement y atteindre.
Faute de pouvoir, justement, ou abus de volonté. D’où l’idée de Dieu, qui
serait une volonté toute-puissante : un indicatif-impératif. L’idée de silence,
poussée à sa limite, nous débarrasse de ces deux anthropomorphismes.

Une loi, c’est ce qui s’impose (la nécessité), ou devrait s'imposer (la règle,
l'obligation). On parle dans le premier cas de lois de la nature ; dans le second,
de lois morales ou juridiques. Les premières, qui ne sont voulues par personne,
s'imposent à tous. Les secondes, qui sont voulues par la plupart, ne s'imposent
à personne : elles n’existent, comme loi, que par la puissance que nous gardons,
malgré elles, de les violer. Si le meurtre ne restait possible, aucune loi n’aurait
besoin de l’interdire. Si la gravitation universelle pouvait être violée, ce ne serait
plus une loi.

C’est le sens juridique qui est premier : la loi est d’abord une obligation
imposée par le souverain. On ne parle de lois de la nature que secondairement,
parce qu’on imagine que la nature obéit elle aussi à quelqu'un, qui serait Dieu.
Toutefois ce n’est qu’une métaphore. La nature n’est pas assez libre pour pouvoir
obéir. Dieu le serait trop pour pouvoir commander.

On s’en voudrait de ne pas citer la définition fameuse que donnait
Montesquieu : « Les lois, dans la signification la plus étendue, sont les rapports
nécessaires qui dérivent de la nature des choses » ({\it L'esprit des lois}, I, 1). On ne
peut pourtant s’en contenter : car comment, si cette définition était bonne, y
aurait-il de {\it mauvaises lois} ? C’est que Montesquieu, comme Auguste Comte
après lui, pense d’abord aux lois de la nature, qui ne sont ni bonnes ni mauvaises,
qui ne sont que « des relations constantes entre les phénomènes
observés » ({\it Discours sur l'esprit positif}, \S 12). Les lois humaines sont d’une autre
%346
sorte : elles ne s’imposent que pour autant qu’on les impose, ce qui ne garantit
ni qu’elles soient justes ni qu’il faille leur obéir. Ainsi cette notion de loi reste
irréductiblement hétérogène : elle sert surtout à masquer cette dualité même
qui la constitue, afin que le fait semble juste ou que la justice semble faite. C’est
une façon d’échapper au désordre et au désespoir. La vérité est qu’il n’y a que
des faits. Mais qui pourrait la supporter ?

%%%%%%%%%%%%%%%%%%%%%%%%%%%%%%%%%%%%%%%%%%%%%%%%%%%%%%%%%%%%%%%%%%%%%%%%%%%%%%%%%%%%%
