
%%%%%%%%%%%%%%%%%%%%% 328
\section{Justice}
%%%%%%%%%%%%%%%%%%%%%
%{\it }
L'une des quatre vertus cardinales : celle qui respecte l’égalité et la
légalité, les droits (des individus) et le droit (de la Cité). Cela suppose
%329
que la loi soit la même pour tous, que le droit respecte les droits, enfin que
la justice (au sens juridique) soit juste (au sens moral). Comment le garantir ?
On ne le peut absolument ; c’est pourquoi la politique, même lorsqu'elle y
tend, reste conflictuelle et faillible. C’est pourtant la seule voie. Aucun pouvoir
n’est la justice ; mais il n’y a pas de justice sans pouvoir.

« Sans doute légalité des biens est juste, écrit Pascal, mais... » Mais quoi ?
Mais le droit en a décidé autrement, qui protège la propriété privée et par là
l’inégalité des biens. Faut-il le regretter ? Ce n’est pas sûr (il se peut qu’une
société inégalitaire soit plus prospère, même pour les plus pauvres, qu’une
société égalitaire). Ce n’est pas impossible (notamment si on met la justice plus
haut que la prospérité). Qui en décidera ? Le droit positif ({\it jus}, en latin, d’où
vient justice), ou plutôt ceux qui le font. Les plus justes ? Non pas. Les plus
forts — donc presque toujours, dans nos sociétés démocratiques, les plus nombreux.
La propriété privée fait-elle partie du droit naturel ? Fait-elle partie des
droits de l’homme ? Ce sont deux questions différentes, mais toutes deux insolubles
par le droit seul. Questions philosophiques plutôt que juridiques, et politiques
plutôt que morales. « Ne pouvant faire qu’il soit force d’obéir à la justice,
continue Pascal, on a fait qu’il soit juste d’obéir à la force. Ne pouvant fortifier
la justice, on à justifié la force, afin que le juste et le fort fussent ensemble, et
que la paix fût, qui est le souverain bien » ({\it Pensées}, 81-299 ; voir aussi le
fr. 103-298).

C’est où l’on rencontre la fiction, mais éclairante, du contrat social. « La
justice, écrivait Épicure, n’est pas un quelque chose en soi ; elle est seulement,
dans les rassemblements des hommes entre eux, quels qu’en soient le volume et
le lieu, un certain contrat en vue de ne pas faire de tort et de ne pas en subir »
({\it Maximes capitales}, 33 ; voir aussi les maximes 31 à 38). Peu importe que ce
contrat ait existé en fait ; il suffit à la justice qu’il puisse exister en droit : il est
« la règle, souligne Kant, et non pas l’origine de la constitution de l’État, non
le principe de sa fondation mais celui de son administration » ({\it Réfl.}, Ak. XVIII,
n° 7734 ; voir aussi {\it Théorie et pratique}, I, corollaire). Une décision est juste
quand elle pourrait être approuvée, en droit, par tous (par tout un peuple, dit
Kant) et par chacun (s’il fait abstraction de ses intérêts égoïstes ou contingents :
c'est ce que Rawls appelle la « position originelle » ou le « voile d’ignorance »).
Cela vaut pour l’État, mais tout autant pour les individus. « Le moi est injuste
en soi, écrivait Pascal, en ce qu’il se fait le centre de tout » ({\it Pensées}, 597-455),
Contre quoi toute justice est universelle, au moins dans son principe, et n’agit,
en chacun, que contre l’égoïsme ou par décentrement. Cela donne à peu près
la règle, telle qu’Alain la formule, et qui ne vaut pour tous que parce qu’elle
vaut d’abord pour chacun : « Dans tout contrat et dans tout échange, mets-toi
à la place de l’autre, mais avec tout ce que tu sais, et, te supposant aussi libre
%330
des nécessités qu’un homme peut l'être, vois si, à sa place, tu approuverais cet
échange ou ce contrat » ({\it 81 chapitres}, VI, 4, « De la justice »). Cela vaut pour
les individus, mais donc aussi pour les citoyens. Pour la morale, mais donc aussi
— si les citoyens font leur devoir — pour la politique. « Est juste, écrivait Kant,
toute action ou toute maxime qui permet à la libre volonté de chacun de
coexister avec la liberté de tout autre suivant une loi universelle » ({\it Doctrine du
droit}, Introd., \S C). Cette coexistence des libertés sous une même loi suppose
leur égalité, au moins en droit, ou plutôt elle la réalise (malgré les inégalités de
fait, qui sont innombrables), et elle seule : c’est la justice même, toujours à faire
ou à refaire, toujours à défendre ou à conquérir.

%%%%%%%%%%%%%%%%%%%%%%%%%%%%%%%%%%%%%%%%%%%%%%%%%%%%%%%%%%%%%%%%%%%%%%%%%%%%%%%%%%%%%
