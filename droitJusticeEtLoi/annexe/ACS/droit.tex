
%%%%%%%%%%%%%%%%%%%%% 182
\section{Droit}
%%%%%%%%%%%%%%%%%%%%%
%{\it }
Une possibilité garantie par la loi (le droit à la propriété, à la sûreté,
à l'information...) ou exigée par la conscience (les droits de
l’homme). Pris absolument, le droit est l’ensemble des lois qui limitent et
garantissent — l’un ne va pas sans l’autre — ce qu’un individu peut faire, à l’intérieur
d’une société donnée, sans encourir de sanction et sans que quiconque
puisse l’en empêcher sans en encourir. Cela suppose un système de contraintes,
donc une répression au moins possible : il n’y a de droit, en ce sens, que par la
force, et telle est la fonction de l’État. Le droit naturel n’est qu’une abstraction ;
les droits de l’homme, qu’un idéal. Seul le droit positif permet de passer, grâce
à l'État, du {\it droit} au {\it fait}. C’est une raison forte de préférer l’état civil, même
injuste, à l’état de nature : mieux vaut un droit imparfait que pas de droit du
tout.
%183
Chacun sait qu’il n’y a pas de droits sans devoirs. Mais point, comme on le
croit parfois, parce qu’on ne pourrait bénéficier de ceux-là qu’à la condition de
respecter d’abord ceux-ci. Un tortionnaire, on n’a pas le droit pour autant de le
torturer. Un voleur, pas le droit de le voler. Qu'ils n’aient pas fait leur devoir
ne nous dispense aucunement, fût-ce vis-à-vis d’eux, des nôtres. Mais il y a
plus. Un nouveau-né ou un débile profond n’ont aucun devoir, et pourtant
une multitude de droits. C’est dire assez que mon droit n’est pas défini par mes
devoirs, mais par les devoirs des autres à mon égard. Si j’ai aussi des devoirs, ce
qui est bien clair, ce n’est pas parce que j’ai des droits, mais parce que les autres
en ont. Ainsi le droit, pris absolument, fixe les droits et les devoirs de chacun,
les uns vis-à-vis des autres et de tous. C’est ce qui nous permet de vivre librement
ensemble, par la limitation, rationnellement instituée, de notre liberté.
Ma liberté s’arrête, selon la formule fameuse, où commence celle des autres.
« Le droit, écrit Kant, est la limitation de la liberté de chacun à la condition de
son accord avec la liberté de tous, en tant que celle-ci est possible selon une loi
universelle. » Mais il n’existe, c’est ce qu’on appelle le droit positif, que par des
lois particulières.

%%%%%%%%%%%%%%%%%%%%%
\section{Droit naturel}
%%%%%%%%%%%%%%%%%%%%%
Ce serait un droit inscrit dans la nature ou la raison,
indépendamment de toute législation positive: un
droit d’avant le droit, qui serait universel et servirait de fondement ou de
norme aux différents droits positifs. En pratique, chacun y met un peu ce qu’il
veut (par exemple, chez Locke, la liberté, l'égalité, la propriété privée, la peine
de mort), ce qui est bien commode mais ne permet guère de résoudre
quelque problème effectif que ce soit. Que disent la nature ou la raison sur
l'avortement, l’euthanasie, la peine de mort ? Sur le droit du travail et des
affaires ? Sur le PACS et la libéralisation des drogues douces ? Sur le meilleur
type de régime ou de gouvernement ? On a pu fonder sur le droit naturel, selon
la conception qu’on s’en faisait, aussi bien la supériorité de la monarchie
absolue (chez Hobbes) que celle de la démocratie (chez Spinoza). C’est dire
assez la malléabilité de la notion. Les droits de l’homme ? Ce n’est pas la nature
qui les définit mais l’humanité, non la raison mais la volonté. Pour ma part, si
l’on tient à parler d’un droit naturel, j’aurais tendance à dire que ce n’est pas un
droit du tout, mais le simple règne des faits ou des forces. « Le droit naturel de
la nature entière et conséquemment de chaque individu s’étend jusqu'où va sa
puissance, écrit Spinoza, et donc tout ce que fait un homme suivant les lois de
sa nature propre, il le fait en vertu d’un droit de nature souverain, et il a sur la
nature autant de droit qu’il a de puissance » ({\it Traité politique}, II, 4). C’est ainsi,
précise-t-il ailleurs, que « les grands poissons mangent les petits en vertu d’un
%184
droit naturel souverain » ({\it }Traité théologico-politique, XNT). C’est la loi de la
jungle, dont seul le droit positif nous sépare.

%%%%%%%%%%%%%%%%%%%%%
\section{Droit positif}
%%%%%%%%%%%%%%%%%%%%%
L'ensemble des lois effectivement instituées, dans une
société donnée, quel que soit par ailleurs le mode (coutumier
ou écrit, démocratique ou monarchique..) de cette institution. C’est un
{\it droit} qui existe en {\it fait}.

%%%%%%%%%%%%%%%%%%%%%%%%%%%%%%%%%%%%%%%%%%%%%%%%%%%%%%%%%%%%%%%%%%%%%%%%%%%%%%%%%%%%%
