
%%%%%%%%%%%%%%%%%%%%%
\chapter{John Rawls, pratique de la philosophie}
%%%%%%%%%%%%%%%%%%%%%

%%%%%%%%%%%%%%%%%%%%%%%%%
\section{Pratique de la philosophie}
%%%%%%%%%%%%%%%%%%%%%%%%%

\subsection{Repères biographiques}

RAWLS JOHN\ \ \ (NÉ EN 1921)

\vspace{0.25cm}
Philosophe américain, né en 1921.
Son ouvrage Théorie de la justice
(1971) a connu un retentissement
%
considérable aux États-Unis et continue de susciter actuellement de
nombreux débats et polémiques. Le
succès de l'ouvrage est à la mesure
des critiques — aussi nombreuses
que véhémentes — qu'il a pu susciter.

{\it }

%%%%%%%%%%%%%%%%%%%%%%%%%%%%%%%%%%%%%%%%%%%%%%%%%%%%%%%%

\subsection{Un premier paradoxe}

Contemporaine du déclin de la doctrine
marxiste, la philosophie de John Rawls
renoue avec la tradition idéaliste de la
philosophie politique. Écartant tous les
faits (comme Rousseau) dans un premier temps au moins, John Rawls pose
le problème de la justice en termes
d'instauration : comment pourrait-on
instituer, se demande-t-il, une forme
juste d'organisation sociale, abstraction
faite de toute considération particulière
(traditions et mœurs propres à telles et
telles sociétés) ? Partant de là, il imagine
une situation parfaitement hypothétique
(comparable à l’« état de nature » des
anciens théoriciens du contrat social),
dans laquelle un ensemble de personnes
doivent choisir les principes de répartition des biens fondamentaux qu'ils
souhaitent adopter pour une société à
venir. Le point important est le suivant :
ces personnes ignorent quelle sera leur
position dans cette future société (« voile
d’ignorance »). Ils ne peuvent donc vouloir favoriser qui que ce soit :
 par hypothèse donc, ils opteront pour l’organisation la meilleure pour tous, c'est-à-dire
pour la solution qui serait la plus avantageuse globalement, et qui ne sacrifierait à priori aucune catégorie sociale (ni
les plus favorisés par la naissance, ni les
plus démunis, ni qui que ce soit...). La
décision générale — pour finir — est
très paradoxale : bien qu'aucun individu
raisonnable placé dans une telle situation ne puisse désirer une société injuste
(qui puisse sacrifier les intérêts ou les
droits de quelques-uns au profit de la
communauté ou de l’une de ses parties),
tous pourtant doivent s’accorder sur la
reconnaissance du bien-fondé des inégalités sociales et économiques.

\subsection{Deux principes de base}


Les principes retenus seront, selon John
Rawls, les suivants :

« 1. Chaque personne doit avoir un droit
égal au système le plus étendu de
libertés de base égales pour tous, qui
soit compatible avec le même système
pour les autres.

« 2, Les inégalités sociales et économiques
doivent être organisées de façon à ce que,
%
à la fois : a. on puisse raisonnablement
s'attendre à ce qu'elles soient à l'avantage
de chacun ; b. qu’elles soient attachées à
des positions et des fonctions ouvertes à
tous. » ({\it Théorie de la justice}.)

Le premier principe, qui exprime l’engagement de John Rawls en faveur du libéralisme,
 signifie que la liberté est le premier des biens et que la justice — conçue
comme équité — est d'abord et essentiellement la répartition égale entre tous
les hommes de cela même qui constitue
leur valeur et leur dignité. Ce principe ne
peut souffrir aucune exception, et il est
absolument prioritaire : la liberté de quiconque ne saurait être sacrifiée, en aucun
cas, et pour quelque raison que ce soit.
Le deuxième principe en revanche, est
beaucoup plus original et ambigu. Que
signifie-t-il exactement ? Que les inégalités sociales et économiques peuvent
être tolérées, en ce sens qu’elles constituent globalement une situation plus
 fructueuse pour tous (les inégalités, en effet,
servent de stimulant à l’activité, elles
augmentent les réserves totales de biens
et de produits disponibles). Mais cette
tolérance rencontre des limites très
strictes : les positions les plus favorisées
doivent être accessibles à tous (principe
démocratique de l'égalité des chances) et
les inégalités ne sont tolérables que si
elles profitent à tout le monde, où aux
plus défavorisés. Admettons par exemple
qu'une disposition soit apparemment très
inégalitaire : l'institution d'une école privée de très haut niveau, où même d’une
filière très élitiste dans l’école publique.
Faut-il la rejeter d'emblée ? Certainement
pas, répondrait John Rawls, car l’une ou
l'autre, à certaines conditions (bourses,
encouragement et soutien des plus
motivés, en particulier lorsqu'ils sont
défavorisés), peuvent profiter à tous.
Ainsi le maintien d'une première classe,
dans un métro, peut profiter à tous, si l'on
décide d'en faciliter l'accès aux plus han-
dicapés aux heures de pointe...

La théorie de John Rawls a été contestée
par les milieux intellectuels de droite et
de gauche, et sur plusieurs plans. À
droite, on lui reproche de célébrer l’État-
providence (État-assistance) parce qu'il
insiste sur la nécessité de prendre
d'abord en compte l'intérêt des plus
démunis ; à gauche, de légitimer la
logique des institutions économiques
dominantes (le «marché »). D'autres
enfin s'indignent de sa prétention à tirer
d'une conception individualiste, occidentale, et pour tout dire, « kantienne »
de l'homme, une conception de la justice
%
 intemporelle et universelle. John
Rawls a répondu à ces critiques dans ses
travaux ultérieurs, et le débat est loin
d’être clos. {\it Cf}. Michaël Walzer.


\vspace{0.25cm}
 PRINCIPAUX ÉCRITS : {\it Théorie de la justice} (1971);
 {\it Anarchie, État et Utopie} (1988) ;
 {\it Justice et démocratie} (1993).

%%%%%%%%%%%%%%%%%%%%%%%%%%%%%%%%%%%%%%%%%%%%%%%%%%%%%%%
