
%%%%%%%%%%%%%%%%%%%%%
\section{Encyclopédie de la philosophie}
%%%%%%%%%%%%%%%%%%%%%
%{\it }

{\bf loi} 

\vspace{.7cm}
Notion appliquée à l’origine aux
comportements humains, dans le sens
d’une règle de conduite ou d’une norme
de justice qui, si elle est violée, comporte
la punition du coupable. D'après une
interprétation moderne, cette notion
aurait ensuite été étendue au cosmos par
les philosophes  présocratiques, en
commençant par Anaximandre, avec une
projection anthropomorphique. Mais chez
les autres présocratiques, comme chez
Héraclite, se trouve déjà esquissé le rapport inverse : l’homme est à son tour soumis, à travers la raison, à la loi divine qui
régit tout le cosmos. C’est surtout le stoïcisme qui a élaboré cette conception, que
Cicéron a rendue populaire. À partir de
là, l’idée d’une loi éternelle se transmet à
la pensée chrétienne : une loi immuable,
universelle, qui coïncide avec la « raison
droite conformément à la nature », qui
spécifie en quoi consiste la justice dans les
rapports entre les hommes, indépendamment de la multiplicité et de la variabilité
des lois qui sont de fait en vigueur chez
les différents peuples. Les sophistes ont
particulièrement insisté sur la variété des
coutumes et des lois civiles — on pense
notamment aux « Raisonnements doubles » (gr. {\it dissoi logoi}), un texte anonyme
de la fin du {\footnotesize V}$^\text{e}$ siècle avant Jesus-Christ où l’on
démontre comment des concepts opposés,
tels que juste-injuste, vrai-faux, peuvent
s'appliquer, tour à tour, aux mêmes
choses, en les considérant comme de
pures « conventions », et en les opposant
parfois à la « nature ». Certains sophistes
plus tardifs, tels Calliclès ou Thrasymaque, en ont tiré une doctrine selon
laquelle seul le droit du plus fort est
conforme à la nature.

Loi naturelle et loi positive

Pour réagir à ces positions, et en reprenant à son compte la critique de Platon,
Aristote affirme l’existence, au-dessus des
%
lois des différents peuples, de normes
«non écrites », universellement valables,
donc en mesure de prescrire ce qui est
juste « par nature ». Les sceptiques infirmeront, quant à eux, l’idée de lois naturelles, comprises dans ce sens. En
revanche, le succès connu par la formulation stoïcienne de cette idée est surtout
dû aux juristes impériaux de l’époque
romaine : c’est avec eux qu'apparaît la
notion d’un « droit naturel » (dans le sens
objectif du mot «droit», précisément
comme synonyme de loi juste) qui se
transmettra aussi bien dans la patristique
que dans le-Code justinien. Comprise en
général comme un « instinct » que Dieu a
imprimé à l'âme humaine, la loi de nature
coïncidait avec la morale elle-même, en
fonctionnant comme un critère idéal pour
l'évaluation des usages et des normes juridiques existants. D’après saint Thomas, il
s’agit là de lois « humaines », appelées par
la suite « positives », différentes des lois
« divines » qui concernent la fin surnaturelle de l’homme. Les premières ne sont
valables que si elles n’enfreignent pas,
mais précisent et particularisent la loi
« naturelle », qui est la façon dont la loi
objective « éternelle » est présente dans
les créatures rationnelles, loi qui a. son
siège dans l'esprit de Dieu et qui régit
l'univers tout entier, s’identifiant ainsi
avec la Providence. La conception thomiste de la loi naturelle est nettement
rationaliste. Le volontarisme d’Ockham
s’opposera à cette conception : une loi
requiert un législateur, et dans le cas de
la loi naturelle ce législateur est Dieu, qui
a décidé tout à fait librement, et non selon
une rationalité éthique supposée objective, ce qu'est la vertu et ce qu'est le vice,
de sorte que, s’il l’avait voulu, il aurait pu
en décider tout à fait différemment.
Luther et Calvin adopteront eux aussi
cette position ; tandis que Grotius optera
pour une position rationaliste. Ce dernier
représente, dans son ouvrage Du droit de
la guerre et de la paix ({\it De jure belli ac
pacis}, 1625), la transition entre la tradition antique et médiévale du « droit naturel» et la tradition moderne, qui se
poursuivra jusqu’à Kant, avec néanmoins
beaucoup d’orientations différentes (principaux représentants : Hobbes, Spinoza,
Pufendorf, Locke, Rousseau). Cependant,
à la même époque, la négation sceptique
de l’idée de loi naturelle trouvait un nouvel écho, en commençant par Montaigne,
%
frappé par les contrastes entre les normes
suivies par les différents peuples de la
Terre, suite aux découvertes ethnographiques dans le Nouveau Monde et en
Extrême-Orient. Au {\footnotesize IX}$^\text{e}$  et au {\footnotesize XX}$^\text{e}$ siècles, le
« positivisme juridique » s’est affirmé
contre la doctrine du droit naturel, avec
la thèse selon laquelle le seul droit qui
existe est celui qui est « valable » dans la
réalité, c’est-à-dire le droit qu’on appelle
positif, indépendamment de sa valeur
morale ou de sa conformité à l’idée de
« justice ». Hans Kelsen est le principal
représentant contemporain de cette tendance.

La loi scientifique

Durant la première moitié du xvir s.,
parallèlement à la naissance de la doctrine moderne du droit naturel, l'usage de
l'expression « lois de la nature » s’affirme
dans le domaine des sciences physiques,
dans l’acception qui est ensuite devenue
courante et qui a été progressivement
étendue aux autres sciences, à mesure
qu'elles se constituaient. On peut même
dire qu’à partir de ce moment la découverte de «lois» allant dans ce sens est
considérée comme la fonction propre à
toute science. Francis Bacon avait déjà
pressenti cet usage, qui s’est imposé avec
Descartes, pour désigner aussi bien des
principes (par exemple celui d'inertie),
que des relations fonctionnelles entre des
grandeurs variables, quand le rapport
entre elles est constant, et qu'on peut
l’'énoncer dans une formule mathématique, à la manière de Galilée. C’est à la
moitié du xvur s. que remonte la proposition de Montesquieu d'étendre la notion
de «loi» au domaine des phénomènes
sociaux et politiques, dans le but de
déterminer les rapports entre les institutions des différents peuples et leurs conditions naturelles ou historiques. C'est
l’origine de la science qui allait ensuite
prendre le nom de « sociologie ». La psychologie allait se présenter comme une
science expérimentale à l’enseigne de la
recherche des «lois» des phénomènes
mentaux. Au contraire, l'expression « lois
de la pensée » pour indiquer les principes
du raisonnement, propre au {\footnotesize XIX}$^\text{e}$ siècle,
dépend d’une conception de la logique de
type psychologique, qui a été complètement abandonnée par la suite. Les lois de
la physique classique, qui ont représenté
historiquement un modèle idéal, et ceci
%
même pour les sciences qui se sont constituées par la suite, étaient toutes des lois
déterministes, c’est-à-dire rigoureusement
universelles (au sens de : privées d’exception). Ce modèle a disparu au {\footnotesize XX}$^\text{e}$ siècle, avec
le développement de la mécanique quantique, dont les lois sont probabilistes, ou
statistiques, même si elles demeurent de
type causal : d’où la présentation des lois
déterministes, propres au macrocosme,
comme des cas limites, par rapport aux
lois probabilistes propres au niveau subatomique de la matière. La notion de loi
scientifique est généralement associée à
celle de prévision (mais, pour qu’elle remplisse cette fonction, il faut que la loi soit
d’une «simplicité » suffisante). À l’origine, aussi bien Charles Sanders Peirce
que Ernst Mach ont insisté sur ce point,
dans le cadre d’une conception pragmatiste de la science (que partageaient aussi,
du reste, des penseurs opposés à l’empirisme, tels Henri Bergson et Benedetto
Croce).

 

— droit naturel e droit (philosophie
du) e Kelsen

%%%%%%%%%%%%%%%%%%%%%%%%%%%%%%%%%%%%%%%%%%%%%%%%%%%%%%%%%%%%%%%%%%%%%%%%%%%
