
%%%%%%%%%%%%%%%%%%%%%
\section{Pratique de la philosophie}
%%%%%%%%%%%%%%%%%%%%%

{\bf LOI}

\vspace{0.35cm}
%
\begin{itemize}[leftmargin=1cm, label=\ding{32}, itemsep=1pt]
\item {\footnotesize ÉTYMOLOGIE} : latin {\it lex}, «loi».
\item {\footnotesize DROIT ET POLITIQUE} : règle obligatoire établie par une autorité souveraine et régissant les rapports des
hommes au sein d'une société.
\item {\footnotesize PHILOSOPHIE} : règle d'action à laquelle il est obligatoire de se
conformer pour réaliser le bien moral ({\it cf.} Kant).
\item {\footnotesize SCIENCE ET ÉPISTÉMOLOGIE} : rapport mesurable, universel et constant établi entre les phénomènes naturels (ex. : « la loi de la chute des corps »).
\end{itemize}


La loi, au sens juridique ou au sens moral,
pose une obligation ; elle est de l’ordre de
la règle. La loi, au sens scientifique, décrit
une relation qui ne comporte jamais d’exception ; elle est de l’ordre de la nécessité.
Toutefois, la loi juridique, dans la mesure
où elle se distingue du simple décret, présente un caractère de généralité et
 d’abstraction tout comme la loi au sens scientifique. Comme le souligne Rousseau au
chapitre {\footnotesize VI} du livre 1 {\it Du Contrat social}, la
loi est générale par son objet et par sa
source : par son objet, car la loi ne statue
pas sur un individu, mais sur des règles
générales de vie sociale qui s'imposent à
tous ; par sa source, car elle ne résulte pas
de la volonté particulière d’un individu, ni
même de celle d'une majorité d'individus,
mais de la volonté générale de tous, abstraction faite de leurs intérêts privés. De
même, pour Kant, la loi morale revêt par
définition un caractère d'universalité. Elle
ne prescrit en effet aucun devoir particulier, mais est la raison pratique elle-même
en tant qu'elle s'impose à l’homme par sa
forme, qui est l’universalité. Sa formule
est : « Agis de telle sorte que la maxime de
ta volonté puisse toujours valoir en même
temps comme principe d'une législation
universelle » ({\it Critique de la raison pratique}, Première partie, livre I). Mais tandis
que l'universalité qui caractérise la loi au
sens scientifique est une universalité donnée, l'universalité qui caractérise la loi
morale est seulement exigible mais
comporte en elle-même la possibilité de sa
transgression. C'est que la loi scientifique
appartient au domaine de la nature, la loi
humaine au domaine de la liberté.

\begin{itemize}[leftmargin=1cm, label=\ding{32}, itemsep=1pt]
\item {\footnotesize TERME VOISIN} : impératif ; norme ; obligation ; règle.
\item {\footnotesize TERME OPPOSÉ} : arbitraire.
\item {\footnotesize CORRÉLATS} : institution ; légalité; obligation ; ordre ; pouvoir ; règle ; république ; science ; volonté générale.
\end{itemize}

%%%%%%%%%%%%%%%%%%%%%%%%%%%%%%%%%%%%%%%%%%%%%%%%%%%%%%%%%%%%%%%%%%%%%%%%%%%
