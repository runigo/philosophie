
%%%%%%%%%%%%%%%%%%%%%
\section{Pratique de la philosophie}
%%%%%%%%%%%%%%%%%%%%%

{\bf JUSTICE}

Aussi ancienne que la philosophie elle-même, la réflexion sur la justice occupe
une place prépondérante dans la pensée
contemporaine depuis une vingtaine
d'années. Ce regain d'intérêt s'explique
par la renaissance et le renouvellement
des interrogations concernant les droits
de l’homme en général et l’État de droit
en particulier. La question de la justice
tendrait à se confondre aujourd’hui avec
%
celle du bien commun, conçu comme
respect mutuel des personnes, équilibre
des libertés et solidarité sociale. Mais
l'élucidation de l’idée de justice conçue
dans ses termes actuels et la formulation
des principes qui la constituent,
impliquent un détour par les théories
classiques et les débats que celles-ci
n'ont cessé de susciter.

\subsection{La justice, vertu globale}

La justice est-elle une vertu, ou bien
une organisation générale harmonieuse
de la vie sociale ? Elle est l’une et l’autre
à la fois, selon Platon qui, dans la République, espérait pouvoir concilier les
deux acceptions possibles du terme. Elle
est en nous, comme dans la cité, le principe qui maintient chaque instance à sa
place tout en présidant à l'harmonie de
l’ensemble. De même que dans l'État,
les magistrats commandent aux guerriers
et aux artisans, de même, dans l’âme,
l'esprit ou la raison commande au cœur
({\it thumos}) et au ventre ({\it epithumia}). La
justice, vertu globale, est ainsi ce qui
donne à chaque partie d’un ensemble la
place qui lui revient, celle qui lui est due
compte tenu de son essence. La tradition
biblique, reprise par saint Augustin,
confirmera cette approche très globale
de la justice et l’élargira même aux rap-
ports entre l’homme et Dieu : la justice
n'est pas seulement le souci et le respect
du bon droit, elle peut même aller au-
delà de ce qui est dû (Matthieu, XX) ; en
outre, elle ne réside pas tant dans les
actes et les œuvres que dans la pureté
intérieure de l’homme sanctifié par la
grâce. Mais une telle conception de la
justice comme vertu purement intérieure, indissociable de l'amour et de la
charité, pourrait constituer un détournement du sens usuel du terme.

\subsection{La justice comme norme du droit}

La justice dans son sens habituel diffère
autant de la vertu platonicienne que de
la vertu chrétienne, et ce, à trois points
de vue : elle n’est pas une qualité purement intérieure mais concerne exclusivement les relations avec autrui; elle
n'inclut pas le rapport de l’homme au
divin ; elle ne constitue pas nécessairement un idéal de perfection : un citoyen
juste n'est pas pour autant un saint !
Vertu civique, la justice devrait être définie précisément, selon Aristote comme
une « disposition à accomplir des actions
qui produisent et conservent le bonheur, et les éléments de celui-ci, pour
une communauté politique » ({\it Éthique à
Nicomaque}, V,1). Ainsi conçue, cette
%
disposition se décompose ensuite en
justice générale ou justice légale qui a
pour objet l'utilité commune de la cité,
et en justice particulière ou justice au
sens strict du terme, qui est orientée vers
le bien des particuliers. Celle-ci
comporte à nouveau deux aspects : la
justice corrective, qui concerne les transactions entre les individus et qui se
conforme au principe d'égalité; et la
justice distributive qui applique le principe de proportionnalité dans la répartition des avantages et des honneurs en
fonction des mérites de chacun. Dans
tous les cas, l’objet de la justice est toujours l'établissement d'un juste milieu
reposant en dernière instance sur le
principe de l'égalité. Une telle conception de la justice se retrouvera dans
toute la tradition occidentale chrétienne
puis laïque, mais son objet, à partir du
stoïcisme ne se limitera plus à la cité :
il s’étendra au bien commun de l'humanité.

Les théories modernes de la justice
On retiendra des analyses précédentes
que la justice, conformément aux théories d'inspiration aristotélicienne, repose
en règle générale sur un double principe : celui de l'égalité (« La loi doit être
la même pour tous ») et celui de l’équité
(« On doit offrir à chacun ce qui lui est
dû »), Résolument fidèle à Aristote sur ce
point, John Rawls, contrairement aux
philosophes utilitaristes, accorde à la
justice une prééminence sur tous les
autres impératifs tels que l'efficacité, la
stabilité, l’organisation, etc. Se situant
par hypothèse dans un état préconstitu-
tionnel dans lequel les individus ration-
nels construisent librement une société
juste, sans connaître quelle sera la posi-
tion de chacun dans cette société, John
Rawls postule que les contractants
devront se déterminer en fonction de
deux principes. Selon le premier prin-
cipe, « chaque personne doit avoir un
droit égal au système le plus étendu des
libertés de base égales pour tous »;
selon le second principe (« principe de
différence »), les inégalités sociales sont
acceptables si, et seulement si: 1. on
peut raisonnablement s'attendre à ce
qu'elles soient raisonnablement avantageuses pour chacun ; 2. elles sont attachées à des positions et des fonctions
ouvertes à tous. En d'autres termes, la
justice — conçue comme équité — si
elle implique l'égalité sur un certain plan
(celui de la liberté), n'exclut pourtant
pas l'inégalité, c'est-à-dire les différences
%
de statuts économiques et sociaux :
seules les inégalités qui ne profitent pas
à tous doivent être tenues pour injustes
({\it cf.} John Rawls {\it en annexe}).

C'est ce second principe qui a suscité les
plus grandes critiques car il implique
forcément une intervention de l'État,
toujours problématique, pour corriger
ou tout au moins équilibrer les mécanismes ou les inégalités naturelles par le
biais, notamment, des impôts. Quoiqu'il
en soit, l'ouvrage de John Rawls, s’il a
alimenté des polémiques parfois excessives, a également suscité un débat
approfondi. Signalons également les travaux de Luc Ferry et Alain Renaut qui
renouent avec une longue tradition
républicaine du droit et de la philosophie politique français. Pour ces deux
auteurs, la justice doit être conçue
comme un équilibre des libertés individuelles, tempéré par des institutions
garantissant une solidarité sociale effective,
%
 et réalisé dans le cadre de ce que
l'on appelle l'État de droit.

\subsection{JUSTICE DISTRIBUTIVE}

Caractère que prend la justice lorsqu'elle
s'efforce de déterminer ce qui est dû à
chacun en fonction de ses mérites.

\subsection{JUSTICE COMMUTATIVE}

Caractère que prend la justice lorsqu'elle
conçoit ce qui est dû à chacun comme
devant être strictement équivalent (interchangeable).


\begin{itemize}[leftmargin=1cm, label=\ding{32}, itemsep=1pt]
\item {\footnotesize TEXTES CLÉS} : Platon, {\it Gorgias}, {\it La République} ; Aristote, {\it Éthique à
Nicomaque}, livre V ; J. Rawls, {\it théorie de la justice}.
\item {\footnotesize TERMES VOISINS} : droit ; équité ; légalité ; légitimité ; vertu.
\item {\footnotesize TERMES OPPOSÉS} : illégalité ; inégalité ; iniquité ; injustice ; violence.
\item {\footnotesize CORRÉLATS} : devoir ; droit ; État.
\end{itemize}

%%%%%%%%%%%%%%%%%%%%%%%%%%%%%%%%%%%%%%%%%%%%%%%%%%%%%%%%%%%%%%%%%%%%%%%%%%%
