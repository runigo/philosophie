
%%%%%%%%%%%%%%%%%%%%%
\section{Encyclopédie de la philosophie}
%%%%%%%%%%%%%%%%%%%%%
%
%1
%%%%%%%%%%%%%%%%%%%%%%%%%%%%%%%%%%%%%%%%%%%%%%%%%%%
{\bf justice}

\vspace{.7cm}
Terme qui, dans l’histoire de la
pensée, indique, selon les cas, la conformité à une norme (naturelle, divine, positive) ou à l'idéal auquel la norme doit se
référer pour être valable, et l’ambition
d'égalité ou d’ordre à établir ou à restaurer. Le concept de justice a donc été l’objet de définitions diverses en fonction des
différents présupposés idéologiques ou
éthiques qui sont à la base de son interprétation.

\subsection{La {\it dikè} et la {\it dikaiosunè} des Grecs}


Dans la pensée grecque, la justice est
avant tout un attribut qui concerne l’univers en général et pas uniquement
l’homme ou la communauté humaine. La
justice est la soumission à un ordre universel en raison duquel toutes les choses:
occupent une place et ont une fonction
donnée, ce dont témoigne le mieux la tragédie grecque classique où sont étroitement imbriqués l’exercice de la justice
humaine et la nécessité de réparer tout
acte de démesure, en gr. l’{\it hybris}. La justice de la polis (la ville) et de l’homme,
Ulysse, Œdipe ou Antigone, n’est qu’une
partie de la justice universelle. Le
contraire de la justice est la transgression
%
de l’ordre des choses, la volonté de sortir
de sa propre place et d'accéder à des
degrés supérieurs. L’{\it hybris} suscite la
colère des dieux et la punition. La loi
({\it nomos}) est une norme qui dicte la
conduite des astres et la conduite du
citoyen dans la cité : astres et citoyen se
fondent sur un ordre naturel qui assigne
à chacun sa place et son comportement.
Cependant, avec les grands sophistes,
l’union de la nature et de la loi est rompue : les lois politiques ne sont considérées que relativement à leur aspect positif,
en tant que normes dictées par l'intérêt
de la survivance d’une communauté, voire
celle du tyran qui l’institue et la dirige
selon son intérêt. La justice devient
« l'avantage du plus fort » (selon Thrasymaque chez Platon, {\it La République}, I,
338a-339b). Platon tente de faire rendre
à la justice une valeur indépendante des
intérêts temporels des individus et d’en
faire un absolu : dans le {\it Gorgias}, il réfute
la doctrine des sophistes en la comparant
à la gastronomie qui donne du plaisir au
corps mais le corrompt, tandis que l’enseignement de la justice est semblable à la
médecine qui entretient le corps en état
de santé. Dans ce dialogue, il aboutit à
une caractérisation de la justice en termes
d'ordre et d’harmonie. {\it La République}
renferme en fait l'exposé le plus complet
du concept platonicien de justice, tant
pour l’homme que pour le meilleur
régime politique (gr. {\it politeia}). La {\it polis}
idéale est constituée de trois classes, chacune ayant une fonction précise qui correspond à la nature de ses éléments : la
classe des philosophes aura pour mission
de diriger, celle des guerriers de défendre
la {\it polis}, celle des artisans de procurer le
bien-être matériel. Ainsi, chez l’homme,
où l’on retrouve cette tripartition (intelligence ou {\it noûs}, cœur ou {\it thumos}, appétits
ou {\it épithumia}), la justice consistera en
l'équilibre parfait des parties et dans le
fait que chacune accomplisse son propre
devoir. Aristote ({\it Éthique à Nicomaque},
V) distingue la justice commutative de la
justice distributive. La première consiste
en la simple réciprocité qui fait que chacun doit percevoir l’équivalent de ce qu'il
a donné (rétablissement de l'égalité). La
seconde s'exprime, au contraire, par un
rapport proportionnel où le premier
terme est le mérite des individus et le
second leur dû. Cette justice, qui se
conforme particulièrement à l’art de gouverner,

%
 exclut l'égalité qui apparaît
comme une injustice dans la mesure où
elle ne respecte pas la proportionnalité
avec le premier terme du rapport.

\subsection{La {\it justitia} et le {\it jus} des Romains}

Dans le droit romain, la justice est définie par Ulpien comme le fait de « donner
à chacun sa part » ({\it suum cuique tribuere}) :
mais comme cette « part » n’a pas de définition, cette conception de la justice reste
tautologique (Hans Kelsen, {\it Théorie générale du droit et de l’État}, 1945). Cependant, l’effort pour penser l’exercice de la
justice comme relevant nécessairement de
l'application réglée d’un code de lois et
de règles qui intègre la jurisprudence fait
partie des réalisations effectives de la
romanité. Ces tentatives de codifications
sont à l’origine d’une pratique et d’une
science nouvelles, celle du droit ({\it jus})
comme science du juste et de l’égal ({\it ars bona et æqui}). Le symbole du droit
romain est le {\it corpus juris civilis}, constitué
sur ordre de l’empereur Justinien (527-565), et qui se présente sous la forme
d’une synthèse en quatre parties (les {\it Institutes}, le {\it Codex justinianum}, les {\it Novell{\oe}},
le {\it Digeste}). Cet ensemble théorique et
pratique fait entrer la justice dans l’horizon du droit qui sert à régler les rapports
humains, et contribue à mettre de côté les
discussions philosophiques sur la nature
de la justice au profit de sa pratique.
Parmi les nombreux exemples de cette
neutralisation du discours sur les fondements de la justice, on trouve l’adage du
droit romain qui veut que la {\it res judicata pro veritate accipitur} («la chose jugée soit
acceptée comme vérité », {\it Digeste} 50, 17,
207). En d’autres termes, la nécessité
d'accepter comme juste et vrai ce qui a
été tranché suppose une idéalisation de
l'institution, une fiction qui clôt dogmatiquement la discussion.

\subsection{Les conceptions médiévales}

Les réflexions philosophiques médiévales,
au moins dans leur version chrétienne, ne
modifient pas fondamentalement l’articulation entre discours théorique sur la justice et pratique juridique au nom de la
justice. Dans la tradition chrétienne, la
justice fait partie — avec la tempérance, la
force et la prudence — des quatre vertus
cardinales (du lat. {\it cardo}, « gond », « pivot ») qui, avec les vertus théologales (foi,
espérance et charité), forment l’armature
des discours et des traités portant sur les
%
qualités morales. L'aspect juridique de la
notion romaine de la {\it justitia} est réintégré
notamment dans le cadre de la grande
réforme grégorienne de la papauté, sous
Grégoire VII (1073-1085), qui réinscrit la
construction du droit romain au cœur du
droit canon et des procédures justifiant le
pouvoir et l'autorité pontificaux sur les
souverains séculiers.

\subsection{La notion moderne de justice}

Avec la doctrine du droit naturel, le
problème de la justice est proposé comme
problème de l'existence de lois naturelles
(ou rationnelles) antérieures à toutes lois
positives. La justice tend à être comprise,
soit comme une vertu d'ordre privé (se
comporter avec les autres comme on voudrait qu’ils se comportent avec soi, restituer aux autres ce qui leur est dû,
réciprocité), soit comme une norme générale susceptible d'assurer une vie pacifique en communauté, c’est-à-dire qui
impose à chacun le respect des libertés
d'autrui, dans un système de limitation
réciproque de la liberté naturelle originelle. Parmi les théoriciens du droit naturel, Hobbes se distingue de ce schéma en
optant pour une démarche strictement
nominaliste, y compris pour ce qui
concerne la définition de la justice : antérieurement à l’existence d’un État, et
donc d’un souverain qui édicte les lois, il
n’y a ni justice ni injustice, mais seulement
un droit qui, étant illimité (le droit de tous
sur tout), s’annule lui-même et investit le
champ de l'arbitraire et du hasard. La
constitution d’un État sanctionne aussi les
normes de comportement en fonction
desquelles s’instaure une communauté
civile ({\it societas civilis}) : c’est en fonction
de ces normes que l’on définit ce qui est
juste et ce qui est injuste. Dans la pensée
moderne, la valeur de la justice a ainsi été
rapprochée d’autres valeurs qui semblaient la contenir, lorsqu'elle ne leur
était pas subordonnée : l’utilité sociale (en
particulier dans la pensée anglaise des
{\footnotesize XVIII}$^{\text e}$ et {\footnotesize XIX}$^{\text e}$ siècle) et le bonheur du plus
grand nombre, ou bien la paix et la convivialité comme respect et limite réciproques. Mais surtout, à partir de la fin du
{\footnotesize XVIII}$^{\text e}$ siècle, la justice fut identifiée à l'égalité
politique et sociale : toute la littérature
socialiste du {\footnotesize XIX}$^{\text e}$ siècle. utilise le terme « justice » selon cette acception. La théorie de
la justice (distributive) de John Rawls, la
plus influente à la fin du {\footnotesize XX}$^{\text e}$ siècle, se rapporte au contraire à la tradition classique
%
de la justice. Pour Rawls, la justice est la
réalisation des principes de justice ; ceux-ci étant les principes de distribution des
coûts et des bénéfices aux participants de
l’entreprise sociale choisis par un sujet en
état de garantir le maximum d'équité
({\it fairness}) dans son choix.

%Antigone   constitution   droit naturel   loi   Rawls   tragédie

 
%%%%%%%%%%%%%%%%%%%%%%%%%%%%%%%%%%%%%%%%%%%%%%%%%%%%%%%%%%%%%%%%%%%%%%%%
