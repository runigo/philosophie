
%%%%%%%%%%%%%%%%%%%%%
\section{Encyclopédie de la philosophie}
%%%%%%%%%%%%%%%%%%%%%
%
{\bf philosophie du droit}


\vspace{.7cm}
Discipline qui étudie les fondements du droit, sa nature et
sa valeur par rapport aux valeurs éthiques
et politiques, tandis que l’étude du droit
positif comme donnée objective relève
spécifiquement des sciences juridiques.
%

\subsection{La tradition antique et médiévale}

La discussion autour de la signification,
des limites et du fondement des lois
accompagne, depuis les origines, le déve-
loppement de la philosophie occidentale.
Dans {\it La République} platonicienne, se
trouve posé le problème qui fonde la philosophie du droit : celui du rapport entre
droit et pouvoir. La solution de Platon
fonde l’autonomie du droit par rapport au
pouvoir sur l’essence universelle et transcendante de la justice. D’où la distinction,
fondamentale dans l’histoire de la philosophie du droit occidentale jusqu’à l’âge
moderne, entre droit naturel et droit positif, entre le modèle d’un ordre parfait,
fondé sur l’essence immuable de la nature
humaine, et l’ordre changeant et impar-
fait émanant d’un pouvoir historiquement
déterminé. Si, pour Platon et Aristote
({\it Éthique à Nicomaque}), le droit ou plutôt
le « juste » naturel se présente comme
l’expression de la rationalité de la nature
humaine, au Moyen Âge cette autonomie
de la raison humaine est rapportée à la
Révélation divine. Les œuvres principales
de la philosophie chrétienne médiévale,
depuis La {\it Cité de Dieu} ({\it De civitate Dei})
de saint Augustin jusqu’à la {\it Somme théologique} ({\it Summa theologiae}) de saint
Thomas, définissent la loi naturelle ({\it lex naturalis}) comme un reflet dans l’homme
de la loi divine ({\it lex divina}). Toutes deux
sont des émanations de l’essence divine,
raison pour laquelle le droit naturel est
ce qui permet à l’homme de participer à
l’ordre de la providence divine. Un droit
naturel est donc reconnu comme fondement du droit positif, mais 1l est subordonné à la loi divine.

\subsection{Le jusnaturalisme moderne}

On peut considérer que la crise des fondements théologiques du droit coïncide
avec la période des guerres de Religion,
qui marquent le processus de constitution
de l’État moderne et le processus de laïcisation du savoir. À cette époque, l’idée du
droit naturel est reprise systématiquement par la doctrine du {\it jus naturale}
(« droit naturel » qui a donné « jusnaturalisme »). L'ensemble des positions théoriques relevant de cette dénomination,
même très différentes entre elles, trouve
son unité dans le choix de la méthode
rationnelle et dans la tentative de donner
à la science du droit et de la morale un
statut épistémologique analogue à celui
%
des sciences mathématiques, en la libérant des conflits insolubles provoqués par
la crise de l’universalisme religieux. Ce
principe méthodologique apparaît pour la
première fois dans le {\it Droit de la guerre et de la paix} ({\it De jure belli ac pacis}, 1625)
de Grotius, et il est précisé de manière
rigoureuse et déductive dans l’œuvre de
Hobbes. Chez Grotius. l'affirmation du
caractère spontanément normatif de la
raison humaine revient à s’opposer, d’un
côté, à la dogmatique théologique de la
scolastique et du calvinisme, et, de l’autre,
à la théorie moderne de l’État qui, avec
{\it Le Prince de Machiavel} et {\it La République} (1576) de Jean Bodin, plaçait le pouvoir
du souverain en dehors de toute limitation morale et juridique. Pour Grotius
donc, dans la mesure où il est fondé sur
la sphère de la raison pure, le droit naturel est à la fois antérieur et autonome par
rapport à tout pouvoir divin et humain.
Mais la construction de Grotius reposait
encore sur la conception aristotélicienne,
stoïcienne et scolastique de l’homme
comme « animal naturellement social »,
parce qu'il voyait dans l’{\it appetitus societais} la donnée première de la nature
humaine. Dans le {\it De cive} (1641), Hobbes
réfute cette idée de manière radicale, et
il instaure une dichotomie entre l’état de
nature où, entre les individus isolés, ne
peut que régner la guerre de tous contre
tous, et l’état civil, dans lequel l’unité
politique, loin d’être donnée {\it naturaliter},
est le produit du transfert des droits naturels de chaque individu à la volonté du
souverain, qui monopolise la production
du droit et ne connaît aucune limite qui
résiderait dans le droit naturel des gouvernés. Le jusnaturalisme moderne en son
entier adopte le couple dichotomique état
de nature-état civil : il n’est plus question
de développement linéaire entre société
naturelle et société politique, entre individu et Etat. Le droit naturel exprime la
dimension de la personne privée, dimension séparée et opposée à l’unité rationnelle du droit positif. De sorte que
l'essence de l’État moderne est appréhendée dans le processus de centralisation et
d’unification du politique, réponse à la
crise du pluralisme de l'Etat féodal. Les
termes qui entrent en jeu dans ce processus sont, d’un côté, les individus libres et
égaux de l’état de nature, de l’autre, l'Etat
qui naît de l’accord rationnel entre les
individus, c’est-à-dire {\it in fine} du contrat ou
%
du pacte social. L'État n’est donc plus le
reflet d’un ordre cosmique à la fois divin
et naturel, mais une forme technique et
rationnelle de réglementation des rapports humains, légitimée par le consensus
des individus. A l’intérieur de la philosophie moderne du droit, le rapport entre
droit naturel des individus et pouvoir central de l'État détermine la différence
entre État absolu et État limité ou libéral.
De même, dans le {\it Tractatus theologico-politicus} (1670) de Spinoza, la constituon de l’État implique le renoncement
presque total de l'individu aux droits
naturels, même si, de façon significative,
le droit à la liberté d’opinion et de jugement reste inaliénable. Dans {\it les Deux
Traités sur le gouvernement civil} (1690) de
Locke, au contraire, le droit public
exprimé par le pouvoir d’État n’a d’autre
fonction que de garantir tel un juge
impartial la perfection du droit privé.
Avec Rousseau et Kant, le schéma théorique du jus naturale subit d’importantes
transformations. Dans le {\it Contrat social}
(1762) de Rousseau, le transfert des droits
naturels à l'Etat est total ; mais ce, dans
la mesure où il n’identifie pas la liberté
avec le droit naturel, mais avec l’obéissance à une loi établie par la raison
humaine elle-même, et qui ne peut être
établie que par un contrat social, qui, en
aliénant tous les droits naturels, trans-
forme l’homme naturel en citoyen. La
{\it Métaphysique des mœurs} (1797) de Kant
représente la forme théorique la plus
accomplie de l’État libéral, comme Etat
de droit et tout à la fois « personne mora-
le », autorité universelle qui résorbe en
elle-même la multiplicité des vouloirs.
Pour Kant, le contrat sur lequel se fonde
le passage à l’état civil n’est pas un acte de
renoncement, total ou partiel, aux droits
individuels, mais la reconnaissance d’un
devoir et la condition de la liberté, dans
la mesure où le sujet du contrat n’est pas
l'homme empirique mais la personne
morale en tant que sujet de la loi morale ;
et donc l'Etat n’est pas, comme chez
Locke, le simple garant des intérêts particuliers, mais l’unité morale de l’humanité
associée.

\subsection{La crise du droit naturel}

L'œuvre de Hegel, depuis les écrits de
la période de Iéna jusqu'aux {\it Principes de
la philosophie du droit} de 1821, marque
la crise irréversible des théories du droit
%
naturel. Pour Hegel, l'État ne peut être
fondé sur un contrat, car il ne peut être
déduit de la somme des volontés individuelles. Il n’est pas une unité formelle et
mécanique mais un tout substantiel et
organique. Il n’est pas une volonté
commune, somme des volontés particulières, mais une volonté universelle, une
totalité éthique irréductible à chacun de
ses composants individuels. Après Hegel,
on ne peut plus opposer au droit positif
un droit naturel entendu comme structure
méta-historique de la personne privée.
Chez deux des plus grands théoriciens
contemporains de l’État moderne, Max
Weber et Hans Kelsen, toute légitimation
de l'État à caractère métaphysique, religieux ou encore jusnaturaliste se dissout
dans l'identification complète entre droit
et Etat. Dans {\it Économie et société} (1920),
Weber situe la légitimité du pouvoir dans
l’État moderne dans sa légalité, dans le
fait qu'il soit exercé en conformité avec
des règles établies rationnellement. La
Théorie pure du droit (1933) de Kelsen
représente l'Etat comme la personnification de l’ordre juridique dans sa totalité ;
le pouvoir étatique en vient ainsi à coïncider avec la validité de cet ordre juridique.
Le nœud qui maintient la différence entre
ces deux systèmes théoriques, et qui en
rend problématique aussi la cohérence
interne, est encore une fois le rapport
entre droit et pouvoir : la théorie weberienne du pouvoir légal analyse le droit en
fonction du pouvoir, Kelsen appréhende
le pouvoir en fonction du système normatif.

\subsection{La crise du positivisme et du formalisme}

On peut dire que la réflexion philosophique contemporaine sur la question
juridique est une réflexion « post-positiviste », expression par laquelle on a coutume d'indiquer des attitudes qui,
quoique différentes entre elles, ont en
commun une position critique voisine à
l'égard de l’un des traits caractéristiques
du positivisme juridique, à savoir l’idée
que les normes sont telles, non en raison
de leur contenu, mais de la façon dont
elles ont vu le jour et de la forme sous
laquelle elles ont vu le jour. Le post-positivisme, entendu comme dépassement du
formalisme  kelsénien, présente deux
visages, selon qu’il se détermine en un
refus du positivisme en tant que tel, ou
%
en une tentative de dépassement du seul
formalisme. La première direction peut
être qualifiée de phénoménologique (du
point de vue de la méthode) et de « néo-
jusnaturaliste » (du point de vue axiologique et ontologique). La seconde direction peut être qualifiée d’analytique,
post-utilitariste et tendanciellement rationaliste. Dans le premier cas, le droit est
conçu, non comme le fruit d’une capacité
instrumentale ou d’une disposition pratique de l’homme, mais comme une réalité en soi, comme une façon d’être propre
à l’homme, et que celui-ci met en œuvre
en tant que condition de sa propre existence. L’autonomie du droit se fonde sur
l'expérience de l’autonomie de l’être et
non pas (comme dans le normativisme
kelsénien) sur la différenciation entre
jugement juridique et jugement éthique,
entendu comme produit de la pratique
humaine. Cette façon de penser l’autonomie du juridique à conduit à considérer
que le droit qui présente des aspects
concernant aussi bien la structure psychique que la structure relationnelle dirigée vers l'extérieur devait être étudié sous
l'angle de l’analyse de la structure d’être
du droit, soit dans la perspective d’une
psychologie ou d’une anthropologie philosophique, qui réinterpréterait les résultats de la psychanalyse contemporaine
(sous forme d’une reprise tant de l’œuvre
de Freud que des suggestions de l'analyse
existentielle de Ludwig Binswanger) ; soit
encore en se plaçant du côté d’une
conception critique de la société et du
droit (Jürgen Habermas) ; soit enfin dans
une perspective plus proprement ontologique. La seconde direction, l'orientation
analytique, ne vise pas tant l’abandon que
l’approfondissement et le dépassement de
certains éléments qui étaient déjà présents dans la version normativiste du positivisme. Ce qui apparaît important pour
juger de la validité des normes n’est plus
la forme sous laquelle elles sont posées,
mais leur façon de se poser. L’attention
est portée non pas tant sur les procédures
de production normative que sur leur
structure linguistique, sémiotique et
logique. Il s’agit d’une attitude encore
essentiellement descriptive, qui maintient
le primat du droit positif.

Une fois écarté le modèle de justification de la validité du droit qui faisait référence de manière plus ou moins explicite
à une autorité, on a fait ressortir les traits
%
spécifiques émergeant de l’analyse même
du langage et des énoncés juridiques. La
limite de la perspective analytique dépend
du fait que, si elle s’est démontrée capable
de décrire le langage juridique, elle n’a
pas pu proposer, sans aller à l’encontre de
graves difficultés, un principe distinctif du
droit. Du point de vue strictement analytique, 1l s’est avéré difficile de démontrer
que le caractère prescriptif du langage
juridique était quelque chose de spécifique par rapport à celui d’autres langages. De manière analogue, la logique
déontique (Georg Henrik von Wright),
finissant par se concevoir elle-même non
comme une logique spécifique, différente
de la logique générale, mais comme un
moment particulier de celle-ci, s’est trouvée confrontée aux mêmes problèmes. Si
les énoncés déontiques peuvent être évalués seulement sur la base de critères de
vérité / fausseté, alors que le droit peut
l’être en termes de validité / non-validité,
il paraît difficile de parcourir exclusivement la voie de l’analyse de la structure
logique des énoncés juridiques pour pouvoir décrire un critère quelconque qui
permette d'identifier un droit pourvu de
validité. Et c’est face à des problèmes de
cette nature que se sont développées des
recherches fondées sur la théorie des
actes de langage (John Langshaw Austin,
John Searle), capables de saisir, dans une
optique restant au demeurant descriptive,
la dimension éminemment pratique du
droit. L’attitude analytique traditionnelle
s’est heurtée à d’autres difficultés, dérivant de son empirisme rigide, et qui expliquent certains de ses développements
plus récents. Si la tendance rigoureusement empiriste a offert des réponses à des
exigences de garantie des libertés, dans le
domaine de la définition des tenants et
aboutissants juridiques, elle a fini par ne
pas rendre compte du fait que les signifiés
juridiques sont aussi le fruit d’un processus articulé d'interprétation. La saisie du
signifié qui peut se poser comme normatif
est le résultat d’un processus particulier,
de type analytique, mais aussi et surtout
d’un processus synthétique, constitué par
le « renvoi des sens » propre aux entités
sémiotiques (entités sémiotiques internes
au droit), et par le rapport entre celles-ci
et les entités extérieures. L’attention à ce
renvoi a conduit à concevoir le droit (dans
une perspective descriptive et objective)
comme un système de rapports sémiotiques
%
 et plus particulièrement narratifs
(B. S. Jackson), et (dans une perspective
de type subjectif) comme un processus
interprétatif. Dans le premier cas se sont
développées des recherches de sémiotique juridique : dans le second a été proposé un nouveau type d’herméneutique,
analytique en l'occurrence (R. Alexy).
L’herméneutique juridique présente trois
orientations prédominantes : celle, précisément, d'inspiration analytique ; une
autre de type phénoménologique et heideggerienne ; et celle enfin qui est liée à
la relève de la rhétorique. Bien que se
référant à des conceptions du droit éloignées les unes des autres, ces tendances
ont en commun la conscience qu'il est
nécessaire d’asseoir l’autonomie du droit
face à l’arbitraire d’une raison totalement
libre ou d’une décision totalement irrationnelle. La crise du positivisme et du
formalisme normativiste a contribué au
développement d’autres orientations, qui
peuvent être définies, dans un sens large,
comme sociologiques et réalistes. Il
semble que de telles orientations tendent
toutefois à proposer à nouveau certains
traits de type prescriptiviste, qui dénoncent la persistance de liens avec le normativisme. La dimension juridique a été
interprétée aussi bien dans la théorie du
droit (O. Weinberger, N. Maccormick)
qu’en sociologie (G. Teubner), dans une
perspective où la réalité du droit, sa forme
sociale se présentent non comme un
simple produit de l’agir, mais comme un
« système » de rapports ou une organisation qui prétend avoir une validité autonome par rapport aux pratiques
qu’elle-même ordonne. Le réalisme juridique lui
aussi s’est mis à proposer des éléments
prescriptivistes analogues, par exemple
lorsque, dans la mise en évidence d’un
fondement ultime de la normativité, on se
réfère au consensus, et en particulier à un
consensus entendu non comme un fait pur
(un événement), mais comme quelque
chose relevant d’un devoir-être, en
d’autres termes, au consensus des opinions qualifiées. Ces éléments prescriptivistes soulignent un des aspects les plus
évidents des récents développements de
la philosophie du droit. Le dépassement
du positivisme et du formalisme normativiste a conduit, d’un côté. à réévaluer la
« fonction » (au sens précis d’exercice
d’une charge bien circonscrite) du droit,
ainsi que sa dimension plus spécifiquement
%
 sociologique (voir Niklas Luhmann) ; mais, d’un autre côté, il a conduit
à déplacer la recherche sur le fondement
normatif du droit au niveau de l'éthique.
La tentative de dépassement de l’utilitarisme individualiste a conduit à la réévaluation de certaines instances kantiennes
universalistes en éthique (John Rawls),
ainsi que du rôle des « principes juridiques » en droit (Ronald Dworkin). Ces
instances ont rendu possible un nouvel
usage de la théorie de la justice en fonction de la fondation du droit, ouvrant
alors un processus qui a donné lieu sur
le plan axiologique à certaines convergences significatives entre positions philosophiques différentes : jusnaturalisme
d'inspiration phénoménologique, « jusrationalisme », réalisme. Quoique par le
biais d’approches différentes, ces orientations ont en commun le fait qu’elles soulignent toutes le caractère central de
l'éthique. Enfin, les réflexions de Pierre
Legendre sur la manière dont les discours
normatifs s'efforcent de trouver un fondement dans les catégories des sciences
humaines dominées par un scientisme
comportementaliste ouvrent un champ
nouveau à la philosophie du droit : il
s’agit de comprendre comment se nouent
le discours dogmatique, la production des
normes dans les sociétés postindustrielles
et la pérennisation des institutions.

% justice   Legendre

%%%%%%%%%%%%%%%%%%%%%%%%%%%%%%%%%%%%%%%%%%%%%%%%%%%%%%%%%%%%%%%%%%%%%%%%%%%
