
%%%%%%%%%%%%%%%%%%%%%
\section{Pratique de la philosophie}
%%%%%%%%%%%%%%%%%%%%%

{\bf DROIT}

\vspace{0.35cm}
%

Le droit (du latin {\it directus}, « sans courbure ») est né de la nécessité de régler,
voire de rectifier les relations entre les
hommes. Comme le dit Kant, les
hommes sont insociablement sociables :
ils veulent vivre en société, car ils savent
que c'est nécessaire (sociabilité), mais
chacun n'est pas disposé à s'imposer à
lui-même les exigences entraînées par
cette existence collective (insociabilité).
Livrés à eux-mêmes, les rapports
humains seraient donc passionnels,
engendreraient conflits et insécurité,
débouchant sur une situation contraire à
l'objectif poursuivi par l'association.
D'où la nécessité d’instituer un arbitrage
impartial s'appliquant équitablement à
tous. Il faut alors comprendre à quelles
conditions un droit est véritablement un
droit, garantissant réellement à chacun la
faculté d'user de ses droits (« droits subjectifs »), mais lui imposant conjointement des devoirs légitimes.

\subsection{L’impossibilité de fonder le droit
sur le fait}

Le droit est nécessairement institué : si
sa fonction est de rectifier, il serait
contradictoire qu'il aille chercher ses
fondements dans ce qui est déjà là. Les
faits ne justifient pas le droit. Dans un
passage célèbre du {\it Gorgias} de Platon
(483{\it b sq.}), l'un des personnages, Calliclès, affirme que le droit, qui met les
hommes à égalité devant la loi, est
injuste. La véritable loi, c’est le fait de la
nature — l'inégalité: « Si le plus fort
domine le moins fort et s'il est supérieur
à lui, c'est là le signe que c’est juste [...]
conformément à la nature du droit, c'est-à-dire conformément à la loi, oui, par
Zeus, à la loi de nature...» Une telle
volonté de rabattre le droit sur des rapports
%
 de force naturels est en réalité un
déni du principe même du droit : réduisant le droit au fait, elle refuse le droit
au profit de la violence, Socrate a du
reste beau jeu de répondre que si on se
ralliait à cette thèse, il faudrait se soumettre à la foule des « faibles » qui,
 toujours plus forte que Calliclès, imposerait
sa loi... Mais en réalité, aucune force,
fût-ce celle de la foule, ne fondera
jamais aucun droit. Comme le montre
magistralement Rousseau ({\it Du contrat social}, I, 3), l'idée de « droit du plus fort »
est contradictoire dans les termes. Parce
que le « plus fort », en effet, n'existe pas :
s'il suffisait d’être « plus fort» pour être
toujours le maître, on ne ferait pas appel
au droit. Parce que se réclamer du droit,
c'est instituer des obligations durables,
irréductibles aux faits réels, qui peuvent
bien les violer, mais ne sauraient les
annuler : il ne suffit pas qu'un voleur ait
la force ou l’habileté de me prendre
mon portefeuille pour en faire sa propriété légale, même s’il le possède en
fait. Un droit digne de ce nom ne saurait
être « un droit qui périt quand la force
cesse ». Pour comprendre que le « droit
du plus fort » est une absurdité, il suffit
de voir qu'il suffirait alors d’avoir la
force de désobéir pour en avoir le
droit.

\subsection{Droit naturel et droit positif}

Si le droit ne peut se fonder par le fait,
il faut cependant admettre que les faits
nous imposent le droit. C'est ce que
démontrent les théoriciens du « droit
naturel ». Il ne s'agit pas pour eux de
voir dans la nature un modèle du droit,
mais d'établir que, imaginés sans société
ni loi, les hommes seraient obligés
d'instaurer le droit. Pour Hobbes, par
exemple, c'est en vertu de la « loi de
nature » qui « interdit aux gens de faire
ce qui mène à la destruction de leur vie »
({\it Léviathan}, chap. XIV), qu'il serait obligatoire de sortir de cet état d'insécurité
en instaurant l'association, le droit et le
pouvoir qui l'institue. Même chez
Hobbes donc, pourtant théoricien de la
souveraineté absolue, c’est pour corriger
la nature et empêcher les rapports de
force inter-individuels que les hommes
ont institué le droit. Le droit naturel n’est
pas un droit existant naturellement, mais
la mise en évidence de la vraie nature
du droit. Ce n'est pas la nature, mais la
raison qui institue le droit, précisément
pour corriger la nature. Si bien qu'il
devient possible de se réclamer du droit
naturel pour combattre les excès des différents
%
 droits positifs (les systèmes juridiques tels qu'ils sont réellement institués dans les diverses sociétés). Ici
encore, le droit (naturel) rectifie le fait
(le droit positif). De ce point de vue, les
critiques adressées à l’idée de droit naturel
 ({\it cf.} par exemple Hans Kelsen, {\it Théorie pure du droit}, 1934) ont tout à fait
raison de refuser qu'il se réduise aux
simples commandements de la nature
ou de Dieu, mais n'éliminent pas la
question de savoir ce qui fonde le droit,
sauf à courir le risque de réduire la
norme au fait (les systèmes de droit
positif et leur logique interne).

\subsection{Les conditions de légitimité du droit}

S'interroger sur ce qui fonde le droit,
c'est se demander à quelles conditions
une loi est juste — et cela seul nous
autorisera à parler de « lois injustes », à
distinguer légitimité et simple légalité,
et à penser les conditions d’un droit de
résistance à l'oppression. En dépit de ses
limites (parce qu'il fait délibérément abstraction des conditions historiques
d'existence du droit positif, toujours déjà
là), le modèle théorique du contrat
nous fait clairement comprendre que la
logique interne du droit, c'est la réciprocité : il faut que le droit soit institué de
telle sorte que chacun, pour peu qu'il
soit suffisamment éclairé, y reconnaisse
les conditions de satisfaction équitable
de ses intérêts. Comme le montre très
bien Rousseau, la condition fondamentale de légitimité du droit — et du pouvoir qui l’institue — c'est sa conformité
à la volonté générale, qui n'est jamais
addition et soustraction de volontés particulières aveuglées par des intérêts
privés, mais recherche de l'intérêt général. S'il remplit ces conditions, le droit
pourra user de la force (droit pénal),
non plus comme d'un fondement abusif,
mais comme d’un instrument de respect
des lois, c'est-à-dire de liberté. Car,
puisque le droit est rendu nécessaire par
l'incapacité des individus à régler spontanément leurs relations, il serait vain
d'imaginer qu'il suffise de promulguer la
loi pour la faire respecter. Contrairement
à la morale, qui repose sur la seule
autorité de la conscience de chacun, le
droit est nécessairement contraignant,
Peut-être faut-il ajouter que, né de l’imperfection de l’homme,
 le droit est lui-même toujours imparfait. Comme le dit
Kant : « Dans un bois aussi courbe que
celui dont est fait l'homme, on ne peut
rien tailler de tout à fait droit. La nature
ne nous impose que de nous rapprocher
%
de cette idée » ({\it Idée d'une histoire universelle [...]}, 6$^e$ proposition). De là les
ambiguïtés de la notion de justice : si la
justice au sens légal peut être déclarée
injuste, c'est précisément parce que
l'humanité, incapable de se conformer
entièrement à son essence morale, fait
du droit, selon la belle expression de
Kant, « une idée à réaliser dans un horizon infini », c'est-à-dire à laquelle on n’a
jamais le droit de renoncer, vers laquelle
on doit toujours avancer, mais qu'on ne
doit jamais croire totalement accomplie.

\begin{itemize}[leftmargin=1cm, label=\ding{32}, itemsep=1pt]
\item {\footnotesize TEXTES CLÉS} : Platon, {\it Criton} ; Th. Hobbes, {\it Du citoyen} ;
 J.-J. Rousseau, {\it Du contrat social} ;
 E. Kant, {\it Idée d'une histoire universelle au point de vue cosmopolitique}.
\item {\footnotesize TERME VOISIN} : légalité.
\item {\footnotesize TERME OPPOSÉ} : état de nature; fait; violence.
\item {\footnotesize CORRÉLATS} : devoir ; institution ; justice ; légalité ; loi ; personne ; pouvoir.
\end{itemize}

%%%%%%%%%%%%%%%%%%%%%%%%%%%%%%%%%%%%%%%%%%%%%%%%%%%%%%%%%%%%%%%%%%%%%%%%%%%
