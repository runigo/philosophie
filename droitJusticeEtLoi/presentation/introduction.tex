
\thispagestyle{empty}

\begin{center}
\Large
%Introduction
Préambule
\normalsize
\end{center}
\vspace{3cm}

Ce document est une compilation d'articles provenant de quatre ouvrages : un dictionnaire encyclopédique de poche, {\it La pratique de la philosophie} destiné aux lycéens, une encyclopédie de la philosophie destinée aux néophytes, et le dictionnaire philosophique d'André Comte-Sponville.
les chapitres contiennent les articles des trois premiers ouvrages, les articles du dictionnaire de philosophie d'André Comte-Sponville sont reproduits en annexe.

\vspace{1.3cm}


Chaque chapitre contient les articles correspondant à une notion particulière. Ces notions ont été choisies en raison de leurs liens avec la question du droit. Ces choix ont été guidés : 1. Par les renvois vers d'autres articles présent dans les ouvrages, 2. Mes propres choix, liés à ma subjectivité, 3. La volonté d'obtenir une quantité raisonnable d'information.

\vspace{1.3cm}

Dans l'ouvrage {\it La pratique de la philosophie}, l'article concernant le droit contient un paragraphe concernant le droit naturel. Le chapitre concernant le droit contient ce paragraphe. Le chapitre concernant le droit naturel ne reproduisant que l'article de l'ouvrage l'{\it Encyclopédie de la philosophie}.
J'ai reproduit en annexe les articles concernant John Rawls.

Les articles compilés dans ce document comportent donc les choix "discutables" réalisés dans les trois ouvrages utilisés. Il s'agit donc d'un document de travail destiné à apporter quelques éléments de réflexion et une synthèse relativement élémentaire des points de vues philosophiques.

\vspace{1.3cm}

Les trois premiers chapitres abordent les thèmes 
.
 Les chapitres suivants élargissent le champ de vision philosophique en abordant les thèmes .

\vspace{2.3cm}

\hfill Stephan Runigo

%%%%%%%%%%%%%%%%%%%%%%%%%%%%%%%%%%%%%%%%%%%%%%%%
