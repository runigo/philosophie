
%%%%%%%%%%%%%%%%%%%%%
\section{Pratique de la philosophie}
%%%%%%%%%%%%%%%%%%%%%
%1

\subsection{Déterminisme}

\begin{itemize}[leftmargin=1cm, label=\ding{32}, itemsep=1pt]
\item {\footnotesize ÉTYMOLOGIE}: latin {\it terminus}, « borne », « limite ».
\item {\footnotesize ÉPISTÉMOLOGIE}: {\bf 1.} Relation nécessaire entre une cause et son effet. {\bf 2.} Système de causes et d'effets entretenant entre eux des relations nécessaires.
\item {\footnotesize MÉTAPHYSIQUE}: doctrine selon laquelle l'ensemble du réel est un système de causes et d'effets nécessaires, y compris les faits qui paraissent de façon illusoire relever de la liberté ou de la volonté.
\end{itemize}


La notion de déterminisme est au fondement de celle de loi physique. Elle pose
qu'il est possible de formuler un lien tel
qu'une ou plusieurs causes étant données, tel(s) effet(s) s'ensuit(vent) nécessairement. Le déterminisme ne doit pas
être confondu avec la simple causalité,
qui établit aussi un lien entre deux événements, le premier produisant le
second, sans pour autant que cette relation soit présentée comme nécessaire (la
même cause aurait pu produire un autre
effet). Le déterminisme s'oppose donc
aux relations de causalité dues au hasard
où à la liberté. Contrairement au fatalisme, le déterminisme ne suppose pas
une providence ; il est « aveugle ». Il faut
se méfier du langage courant, qui
emploie souvent l'expression « être déterminé » pour désigner une décision de la
volonté, c'est-à-dire une autodétermination ({\it cf.} Autonomie en annexe), qui est en réalité le
contraire d’un déterminisme. L'idée du
déterminisme ne s'oppose pas nécessairement à l'existence de la liberté : selon
une image d’Alain, on pourrait dire que
le déterminisme est à la liberté ce que
l'eau est au nageur.
%2

\begin{itemize}[leftmargin=1cm, label=\ding{32}, itemsep=1pt]
\item {\footnotesize TERME VOISIN} : causalité.
\item {\footnotesize TERME OPPOSÉ} : indéterminisme.
\end{itemize}

\subsection{Principe du déterminisme}

Au sens épistémologique, principe selon
lequel, dans un domaine donné, à tout
événement peuvent être assignées une ou
plusieurs causes, les mêmes causes produisant rigoureusement les mêmes effets.
Le principe du déterminisme permet de
distinguer les sciences exactes, formulant
des lois strictes et universelles (en particulier les sciences de la nature), des sciences
humaines, où l'intervention de la liberté,
sans interdire l'explication par la causalité,
exclut que celle-ci soit soumise à une
stricte nécessité (par exemple en histoire), Dans le cadre même de la nature,
certains domaines ont été considérés
comme échappant au déterminisme, parce
qu'ils ne paraissaient pas soumis à des
régularités autorisant des prévisions rigoureuses (par exemple, les « météores », les
séismes, les tremblements de terre, etc.
qui, tout en respectant les lois de la nature,
ne présentent pas une telle régularité), Les
philosophes empiristes (en particulier
Hume) se sont interrogés sur la pertinence
de l’idée du déterminisme, en faisant valoir
que les relations de cause à effet étant des
relations de faits, elles ne peuvent être établies que par une expérience inductive et
ne sauraient par conséquent être considérées comme des relations nécessaires et
universelles (la nécessité ne pouvant être
établie que par des moyens logiques).
Cette difficulté a fait l’objet d’une réflexion
centrale dans la philosophie de Kant, qui
soutiendra que le déterminisme de la
nature ne relève pas des « choses en soi »,
mais de l’ordre phénoménal selon lequel
notre esprit appréhende la nature.

La physique du {\footnotesize XX}$^\text{\,e}$ siècle parle
d'« indéterminisme physique » (en microphysique) à propos de phénomènes dont
l'observation directe est impossible ou ne
permet pas de prédire les effets rigoureux
d'un ensemble de causes. Ces phénomènes sont alors étudiés au moyen de
méthodes de nature statistique où probabiliste (relations d'incertitude d’Heisenberg, par exemple). Cet indéterminisme
théorique a conduit les physiciens à
débattre de la question de savoir si l'on
devait maintenir, pour l'ensemble des phénomènes naturels, le principe du déterminisme physique : l'indétermination de certains phénomènes doit-elle être mise sur le
compte d'une limite provisoire de la
connaissance physique ou bien relève-t-elle de la nature même de ces phénomènes ? C'est, par exemple, le propos de
%3
Louis de Broglie dans {\it Physique et micro-
physique.}

\subsection{Principe du déterminisme psychique}

L'explication freudienne de la « maladie » de l'esprit a pour effet inquiétant
d'effacer la frontière entre le normal et
le pathologique : les mécanismes psychiques étant les mêmes, la maladie
n'est décrétée que lorsque le sujet
éprouve de douloureuses difficultés à
s'adapter au monde extérieur, voire à
lui-même. Dans cet esprit, un certain
nombre d'événements inexpliqués de la
vie normale (rêves, actes manqués,
lapsus) deviennent, pour la psychanalyse, doublement précieux : {\bf 1.} Sur le
plan théorique, ils permettent d’unifier
l'explication des mécanismes  psychiques, normaux ou pathologiques :
toutes ces manifestations sont, au même
titre que la « maladie », des modes d’expression déguisée des pulsions refoulées ; l'étude de leurs
mécanismes aide donc à comprendre les
fonctionnements symboliques permettant l'expression indirecte de l’inconscient. Rapidement, Freud étendra ces
investigations aux domaines de la
culture, de la religion et de l’art ({\it cf.}
Sublimation en annexe). {\bf 2.} Sur le plan pratique,
rêves, actes manqués, etc., doivent, dans
le cadre de la cure, être considérés
comme de précieux indicateurs des
sources traumatiques de la maladie,
puisqu'ils en sont l'expression déguisée.
Aussi Freud énonce-t-il très tôt le principe du déterminisme psychique, selon
lequel ces manifestations doivent être
considérées comme inscrites dans une
chaîne de causes et d'effets dont la
cause initiale est une source pathogène
refoulée. Dans le cadre de la cure, l'application de ce principe conduit à la pratique des libres associations : « Je ne
pouvais pas me figurer qu'une idée surgissant spontanément dans la
conscience d’un malade, surtout une
idée éveillée par la concentration de son
attention, pût être tout à fait arbitraire et
sans rapport avec la représentation
oubliée que nous voulions retrouver »
({\it Cinq Leçons sur la psychanalyse}). La
névrose est elle-même une fuite hors de
la réalité, dont les manifestations sont
des compromis entre pulsions refoulées
et exigences du surmoi,

Aussi Freud est-il amené à parler de
« refuge dans la maladie », puisqu'en
celle-ci toutes les instances psychiques
trouvent leur bénéfice dans ce qui apparaît comme un mode dérivé de leur
satisfaction. La « maladie », lorsque la
nature du conflit psychique qui l’a causée est devenue consciente pour le sujet,
peut trouver à se résoudre par diverses
voies : la satisfaction des pulsions refoulées, devenues conscientes et désormais
assumées ; le refus conscient de cette
satisfaction ; le détournement (sublimation) de leur énergie vers des activités
socialement valorisées, telle que l’art, la
religion, les activités intellectuelles, etc.

\begin{itemize}[leftmargin=1cm, label=\ding{32}, itemsep=1pt]
\item {\footnotesize CORRÉLATS} : cause ; causalité ; destin ;  empirisme ;  fatalisme ; hasard ; induction ; liberté ; loi ; nature ; nécessité ; science ; volonté.
\end{itemize}

%%%%%%%%%%%%%%%%%%%%%%%%%%%%%%%%%%%%%%%%%%%%%%%%%%%%%%%%%%%%%%%%%%%%%%%%%%%
