
%%%%%%%%%%%%%%%%%%%%%
\section{Encyclopédie de la philosophie}
%%%%%%%%%%%%%%%%%%%%%
%1
%%%%%%%%%%%%%
{\bf droit naturel}

\vspace{.7cm}
Doctrine philosophico-juridique qui soutient l’existence de normes
de droit déduites de la nature. On peut se
le représenter comme une sorte de code
moral à implications juridiques censé
s'imposer à tout être humain. Les théoriciens du droit naturel classique (qu’on
appelle les jusnaturalistes) avaient pour
ambition de fonder le droit sur la nature
qu'ils considéraient comme immuable.
Ces normes naturelles étaient censées
constituer la base et le cadre du droit
positif ({\it jus positivum}), c'est-à-dire des lois
posées, des lois en vigueur. Le droit naturel prend en fait sa pleine dimension à
l’âge classique dans le sillage du mouvement qui vit la naissance d’une nouvelle
physique et d’une nouvelle conception de
la science. Les théoriciens du droit naturel
s'efforcent de traduire dans le domaine
%
du droit ce mouvement général que l’on
peut décrire comme un moment où la raison humaine tend à prendre son autonomie à l’égard des autorités traditionnelles.
Certes, dès l’Antiquité, 1l était question
de droit naturel, en particulier chez Aristote et chez les stoïciens qui influencèrent
le droit romain, mais la doctrine du droit
naturel proprement dite remonte pour les
historiens du droit à Hugo Grotius qui
l’inaugure dans son traité Du droit de la
guerre et de la paix ({\it De jure belli ac pacis},
1625). Elle demeure un concept central de
la philosophie politique et de la philosophie du droit en gros jusqu’à Rousseau
(Du contrat social, 1762), même si on peut
considérer la pensée de Kant ({\it Métaphysique des mœurs}, 1794) et celle de Fichte
({\it Fondements du droit naturel}, 1796)
comme autant de développements du
concept de droit naturel.

\subsection{Les racines du droit naturel
dans l'Antiquité}


Les Anciens n’ignoraient pas la notion
de droit naturel, même si ce dernier
n'avait pas la fonction de systématisation
rationnelle du droit qu'il tint par la suite
chez les jusnaturalistes du {\footnotesize XVII}$^\text{\,e}$ siècle. Il faut
rappeler d’abord que les Grecs et les
Romains usaient d’un même mot pour
désigner le « juste » et le « légal », en gr.
{\it dikaion} et en lat. {\it jus}. Leur langue respective rendait donc
 moins perceptible la distinction entre l’idée du juste en soi, qui
procède de l’idée de justice au sens philosophique et moral, et un juste au sens de
ce qui est légal, conforme au droit. Néanmoins, cela n’a pas empêché que naisse
l’idée d’un droit déduit de la nature.
L'idée de droit naturel s'oppose à ceux
qui soutiennent que seuls la convention,
le contrat, l’accord entre les hommes peuvent être source d'obligation juridique.
C’est sans doute dans la pensée politique
des sophistes, qui opposaient la nature et
la loi, que s’exprime le plus nettement ce
conflit. D’après certains d’entre eux, chercher le principe d’une obligation quelconque dans la nature n’est pas possible.
Le monde de la nécessité naturelle ne
saurait être fondateur d’aucun droit
contraignant, la force de la nécessité naturelle n’ayant pas besoin d’être légitimée
pour s'imposer à tous. De cette hypothèse
de départ, les sophistes ont tiré les conclusions les plus diverses : depuis celle d’un
naturalisme politique cynique et violent,
%
comme l’exprime Calliclès dans le {\it Gorgias} de Platon, à l’idée que le droit ne
peut reposer que sur des conventions
comme le soutient Antiphon. Platon, de
son côté, tend à rejeter l’idée que le droit
puisse relever de la coutume et se constituer sur la tradition des lois en vigueur.
Pour lui, une loi injuste moralement ne
mérite pas le nom de loi ({\it Lois}, IV, 715)
et, à la base et comme fin de tout droit et
de toute norme, il y a l’idée de justice,
la forme du Juste en soi. L’art du juriste
consiste à distinguer le juste et l’injuste,
ce qui implique davantage une formation
de philosophe qu’une compétence pro-
prement juridique. On ne peut donc
compter Platon parmi les précurseurs du
droit naturel. Généralement, les historiens du droit tendent à faire d’Aristote
le père de la doctrine du droit naturel à
condition de préciser aussitôt qu'il ne fait
pas de coupure aussi nette que les théoriciens modernes entre droit positif et droit
naturel, entre la justice des lois existantes
et la justice dite naturelle. À ses yeux,
l'une et l’autre sont autant de voies
complémentaires permettant d'éclairer la
décision juste. Ainsi, Aristote, en posant
que l’homme est par nature un animal
politique ({\it Politique} 1, 2), c’est-à-dire destiné à vivre dans la communauté de la
cité, ne constatait pas tant un fait de
nature qu'une nécessité qui lui paraissait
pouvoir être tirée de la nature même de
l’homme. Autrement dit, l’homme ne
peut se réaliser, s’accomplir en tant
qu'homme que dans le cadre de la cité
entendue comme une communauté politique, et celle-ci exige l’institution de lois
positives conformes à la nature spécifique
de chaque peuple. Cette approche du problème du droit naturel se trouve sensiblement modifiée dans le cadre du monde
romain dont chacun reconnaît l’importance dans l’histoire du droit. Ainsi, dans
sa {\it République}, Cicéron ne parle pas
expressément de droit de nature ({\it jus
naturæ}). Il est plus enclin à développer
l'idée voisine mais différente de loi naturelle ({\it lex naturalis}). Celle-ci est une loi
non écrite et innée : « Une loi vraie, qui
est la droite raison, conforme à la nature
({\it natura congruens}), toujours d’accord
avec elle-même, non sujette à périr ({\it sempiterna}), qui nous appelle à remplir notre
fonction, interdit la fraude et nous en
détourne » ({\it De republica} III, 22). Dans le
Traité des lois, Cicéron précise que « la loi
%
est la raison suprême ({\it summa ratio}), gravée dans notre nature, qui ordonne ce
qu'on doit faire et interdit ce qu’on doit
éviter ». Il en conclut que « le droit est
constitué non par l’opinion mais par la
nature » ({\it ibid.}, I, 10). Pour Cicéron, le
principe ultime d’où les lois tirent leur
justification est la raison et l’on ne peut
parler de droit naturel que dans la mesure
où la nature s'accorde avec la raison. A
cet effort, sinon de systématisation et
de rationalisation, du moins de mise en
ordre de l’art juridique, correspond dans
l’ordre proprement juridique le travail des
juristes latins, qui éprouvèrent le besoin
de consigner dans des compilations leur
pratique juridique. Ces compilations
n'avaient donc pas de prétention à la
cohérence et à l’exhaustivité, comme nos
codes modernes ; les juristes affirmaient
même que toute définition en matière de
droit civil est « dangereuse et prête à être
réfutée » ({\it Digeste}, 50-16-203). C’étaient
donc davantage des recueils d’opinions
discordantes permettant de mettre en
perspective les décisions juridiques. Il fau-
dra ainsi attendre Ulpien, un jurisconsulte
romain du {\footnotesize III}$^\text{\,e}$ siècle apr. J.-C., pour qu'en cli-
mat stoïcien on donne une première for-
mulation juridique du droit naturel, qui
s'étend d’ailleurs à l’ensemble des vivants
et envisage pour la première fois un droit
des animaux : « Le droit naturel, c’est ce
que la nature a appris à tous les êtres ani-
més » ({\it Digeste}, 1-1-1-3). Paul, un autre
jurisconsulte, dans une explication des
sens du mot {\it jus}, définit comme suit l’idée
de droit naturel : « {\it Jus} peut désigner cela
qui est toujours juste et équitable, comme
le droit naturel ({\it jus naturale}) » (Commentaire de Sabinus). La tradition jurisprudentielle domine donc le droit romain,
même si ce dernier n’est pas indifférent à
la notion philosophique de droit naturel,
qu'elle est la première à formuler expressément.

\subsection{L’inflexion donnée au concept
de droit naturel par le christianisme}

Il est possible de compter Augustin
parmi les précurseurs des doctrines du
droit naturel. Certes, il n’était pas théoricien du droit et n’a pas vraiment écrit de
traité concernant spécifiquement les questions d'ordre juridique. Cependant son
œuvre, qui influença toute la tradition
Juridique médiévale, témoigne bien des
difficultés d’une pensée qui doit concilier
%
la révélation chrétienne et le droit sacré
qui en résulte avec les problèmes posés
par les relations de droit concrètes. Pour
Augustin, un peu comme pour Platon
mais dans un sens différent, la vraie jus-
tice est « toute intérieure », et « elle ne
juge point des choses par les coutumes et
les pratiques extérieures, mais par la rectitude immuable de la loi éternelle de ce
Dieu tout-puissant » ({\it Les Confessions} III,
7). La justice se confond avec la volonté
divine, toute justice et tout droit reposent
sur cette loi éternelle divine, dépassant
l’esprit humain. Toutes les lois profanes,
celles de la cité terrestre, sont donc essen-
tiellement injustes, et l’ensemble des
traités connus sous le titre {\it La Cité de Dieu}
met l’accent sur l'injustice des institutions
temporelles. Pris à la rigueur, le {\it jus humanum}, le droit fait par les hommes et pour
les hommes, n’est pas un droit. Cepen-
dant, malgré cet aspect critique, Augustin
n’en professe pas moins la nécessité
d’obéir aux lois imparfaites de la cité terrestre. La recherche de la paix civile,
même fondée sur l'injustice, et l’ordre
extérieur de la cité lui paraissent préférables à une remise en cause des pouvoirs
établis, des coutumes et des usages profanes, qui serait sans doute créatrice de
troubles et, par là, de plus grandes injustices encore : on trouvera l’écho de cette
pensée politique d’Augustin dans les doctrines politiques d’un Pascal. La loi divine
et éternelle, fondement de toute loi naturelle et donc de tout droit naturel, ne
repose pas sur l’idée d’une répartition
juste et équitable mais sur l’accord avec
la volonté divine, qui exige des vertus
morales proprement chrétiennes, comme
la charité, le don gratuit, la générosité.
Celles-ci trouvent dans le prêt gratuit, la
remise des dettes et l’aumône des expressions adéquates. Ainsi, ce qu'on peut
appeler le droit naturel chrétien, ou plus
exactement la loi naturelle, reflet de la loi
divine, notion vaste et peu précise du
point de vue juridique, n’est guère accessible à l’homme, tant celui-ci est enfoncé
dans le péché. En gros, la pensée d’Augustin laisse-t-elle ouvertes les deux perspectives : tenter de rendre chrétien le
contenu du droit profane existant, et créer
une sorte de droit naturel sacral, ou
compléter les lacunes du droit sacral en
ayant recours au droit naturel classique
encadré et limité par la loi naturelle. La
postérité d’Augustin hésitera donc entre
%
la séparation des ordres, où l’on rend à
César ce qui est à César d’après la formule de l'Évangile de Matthieu, et l’exigence de faire de la cité terrestre
 l’avant-goût de la cité de Dieu. La pensée médiévale chrétienne se trouve prise en étau
entre un droit fondé sur la morale évangélique et un droit donnant une certaine
place au droit naturel. La volonté des
papes, qui étaient à la fois des monarques
temporels et spirituels, d'établir ou d’affermir leur autorité sur les rois et les
empereurs les a conduits, à partir du
{\footnotesize XII}$^\text{\,e}$ siècle, à recourir aux ressources du droit
romain et à réactualiser, dans l’œuvre des
canonistes, l’idée du droit naturel. Pour
certains membres du clergé, le droit
canon, très influencé par les règles du
droit monastique, devait valoir pour tous
les fidèles. De là, un effort pour rationaliser et rendre universel le droit sacral chrétien.
 En témoigne le {\it Decretum Gratiani}
({\it Décret de Gratien}, compilation juridique
du {\footnotesize XI}$^\text{\,e}$ siècle) qui défend l’idée d’un droit
naturel sacral « contenu dans les Ecritures » et en même temps accorde une autorité subsidiaire au droit naturel profane
dans la mesure où il n’entre pas en opposition avec les règles édictées par la
morale chrétienne. L’effort des canonistes
a donc été de concilier, voire de juxtaposer, l’ancien {\it jus naturale} du droit romain
et la définition sacrale du droit naturel.
L’essor de la scolastique (à partir du {\footnotesize XIII}$^\text{\,e}$ siècle), en contribuant à libérer les
sciences profanes de la tutelle de la théologie, a rendu possible la renaissance de
l’idée de droit naturel classique. Cependant, il faut se garder d’entendre sous le
terme de nature ce que nous entendons
aujourd’hui. Pour la pensée médiévale, la
nature est un ensemble très riche de
formes spécifiques dans lesquelles 1l est
possible de lire des fins naturelles, donc
des indications utiles à l'élaboration de la
règle de droit.

\subsection{La naissance du droit naturel moderne}

Les temps modernes se caractérisent
par une nouvelle conception de la nature,
qui n’est pas sans répercussions sur l’idée
même du droit naturel. À une conception
qualitative et finalisée d’une nature hiérarchisée, riche de ces formes substantielles dont l’observation autorisait sinon
de dégager des règles de droit du moins
de guider le raisonnement du juriste, succède progressivement une conception
%
quantitative et mécaniste d’une nature
homogène, où règnent, en physique, le
principe de la causalité efficiente et les
modèles mathématiques, donc rien qui
puisse aider le juriste dans sa recherche
du fondement du droit. Bref, la conception moderne de la nature tend à creuser
écart entre la sphère des normes juridiques et morales et la sphère du monde
des faits, entre le devoir-être et l’être. La
primauté progressivement accordée à la
notion d’individu est un deuxième trait de
la modernité. Elle est préparée à la fin du
Moyen Age notamment par les écoles
nominalistes (Guillaume d’Ockham et
l’école franciscaine) qui affirmaient que la
réalité individuelle était la seule réalité de
plein droit, et qu’il ne convenait pas d’ac-
corder d’autre réalité que nominale aux
universaux (genre, espèce, etc.). Ainsi
s’est formée plus nettement la notion
d’autonomie de l'individu, condition première de l’idée que la cause de l’institution de la société puisse être la volonté
des individus. Toutes ces évolutions se
sont traduites dans le domaine de la pensée juridique par l'apparition du concept
de droit subjectif, c’est-à-dire de l’idée
qu’un droit puisse appartenir en propre à
un sujet individuel et procéder de sa
nature de personne. La primauté du sujet
individuel a ainsi eu pour conséquence la
remise en question de la notion traditionnelle de droit. Il faut désormais entendre
le mot droit en deux sens distincts : le
droit comme loi ou règle, simple licence
de faire ou de ne pas faire quelque chose,
et le droit comme pouvoir ({\it potestas,
dominium}) de l'individu, comme « qualité
morale, attachée à la personne, en vertu
de quoi on peut légitimement avoir ou
faire certaines choses » (Grotius, {\it De jure
belli ac pacis} I, 1). Le droit ne résulte dès
lors plus d’une enquête sur la nature mais
repose pour partie sur la volonté des individus. À cette émergence de la notion
d’individu, il faut également rattacher le
développement des théories philosophiques ou juridiques de la notion d’état
de nature et de contrat social, lieux
communs de la pensée politique classique.
On y voit opposés l’état de société à un
état de nature, historique ou fictif, originaire ou imaginaire, dans lequel l’individu, réduit à lui-même. exerce son droit
de nature. La plupart du temps, la sortie
de cet état de nature se fait par une sorte
%
de pacte ou de contrat social, qui institue
une nouvelle source de légitimité.

\subsection{Les théoriciens du droit naturel moderne}

Grotius et Pufendorf étaient d’abord
soucieux de rationaliser le droit, ils insistent sur la nécessité de séparer le droit
du fait, « comme les mathématiciens en
examinant les figures font abstraction
des corps qu'elles modifient » (Grotius,
{\it op. cit.}, Préface). L'un comme l’autre
envisagent la nature humaine comme
essentiellement sociable, mais il ne s’agit
pas tant d’affirmer le caractère fondamental de la nature politique de l’homme, à la
manière d’Aristote, que de souligner, à la
manière des stoïciens, l’existence d’une
tendance, d’une inclination à la vie en
société. Ils n’envisagent donc pas le droit
naturel de la même manière que les classiques. Grotius pense que le droit naturel
procède de la nature humaine, qu'il considère comme sociable et raisonnable.
Aussi, les principes du droit naturel doivent être dégagés non de l’observation de
la nature mais de la raison elle-même
dans laquelle ils sont inscrits. Ce droit
naturel, dicté en nous par la droite raison
({\it dictamen rectae rationis}), serait encore
valide « quand bien même Dieu n’existe-
rait pas ». Grotius a donc pour ambition
déclarée de conférer à la doctrine du droit
naturel le caractère d’une science a priori,
indépendante de l’expérience, fondée sur
des évidences que la raison ne saurait
remettre en cause et auxquelles elle doit
donner une forme systématique. Il s’en
faut que ce modèle de rationalité déductive inspiré de la géométrie euclidienne
soit réalisé dans les faits. La propriété des
biens fait partie, aux yeux de Grotius, de
ces évidences du droit de nature, car « le
larcin est défendu par le droit naturel ».
Pufendorf, lui, considère que le droit
naturel procède moins de la raison qu'il
n'est connu par elle. Le principe général
du droit naturel, selon Pufendorf, découle
de la constitution naturelle de l’homme,
à savoir de sa sociabilité. Aussi, « la loi
fondamentale du droit naturel : c’est que
chacun doit être porté à former et à entretenir, en tant qu'il dépend de lui, une
société paisible avec les autres [...]» ({\it Le
Droit de la nature et des gens} III, XV).

\subsection{L'usage du concept de droit naturel
par la philosophie politique classique}

Les philosophes se sont aussi emparés
de la notion de droit naturel. Hobbes,
%
Locke et Spinoza ont envisagé, chacun à
leur manière, le droit de nature ({\it jus naturae}). Pour Hobbes qui, contrairement aux
juristes de l’école du droit naturel, considère que l’homme n’est pas naturellement
sociable, le droit de nature se confond
avec la liberté naturelle de chaque individu. Chaque homme possède en propre
le droit de faire tout ce qu'il juge utile à
sa conservation. Le jus naturale de
Hobbes est donc propre à l'individu, s’enracine dans sa volonté et, en fin de
compte, se présente comme un pouvoir
({\it power}) ordonné à la fin qui est la conservation de la vie. C’est donc la recherche
de la sécurité qui porte la problématique
du droit naturel pour Hobbes. Ce droit à
la sécurité individuelle ne disparaît d’ailleurs pas avec l’état de nature et persiste
à l’état social : il est inaliénable. En outre,
Hobbes s'efforce de distinguer le droit
naturel de la loi naturelle, cette dernière
étant dictée par la droite raison et concernant les choses que l’on doit faire ou éviter de faire pour la conservation de sa vie.
Bien que Locke ne développe pas une
théorie du droit naturel proprement dite,
il est possible de la déduire de l’idée qu’il
se fait de la loi naturelle et de la liberté.
L'homme à l'état de nature est, pour
Locke comme pour Hobbes, isolé et indépendant vis-à-vis de ses semblables. Il est
essentiellement un état caractérisé par la
liberté, où l’homme a « la parfaite liberté
d’agir et de disposer de sa personne et de
ses propriétés dans les limites de la loi de
nature » ({\it Second Traité du gouvernement
civil}). Le point de vue de Spinoza, sur le
droit de nature, semble à première vue
pouvoir être rapproché de celui de
Hobbes. Mais si la lettre des formulations
spinozistes du droit naturel, qu’il assimile
à la loi naturelle, semble autoriser ce rapprochement avec Hobbes, l’esprit en est
tout différent. Pour Spinoza, le droit naturel reflète et dépend des rapports de
puissance réels entre les individus, et
« les plus gros poissons mangent les plus
petits d’un droit naturel souverain ». Le
droit naturel spinoziste est donc en continuité avec une conception de la nature
qui enveloppe le devenir humain, et
comprend aussi le droit naturel des animaux et des choses. Hobbes, qui réserve
l’expression de droit naturel à l'être
humain, a été, quant à lui, conduit à souligner, davantage que Spinoza, la rupture
entre l’état de nature et l’état de société,
%
et donc à envisager l’état de nature et le
droit naturel comme autant de moments
d’arrachement de l’homme à la nature.
Dans tous les cas donc, l’état de nature
est un état (réel ou hypothétique) dans
lequel la définition des droits est faite
pour des individus entre lesquels il
n'existe pas d’autorité acceptée par tous
susceptible de sanctionner les manquements au droit, de façon impartiale. L’institution de cette autorité est le problème
fondamental du passage à la société civile.
On voit alors comment la question du
droit naturel dans la philosophie politique
classique tend à se fondre avec la question
de l’origine des sociétés, et donc avec la
description d’un état de nature que l’on
envisage suivant les cas comme ayant
réellement existé ou comme une hypothèse opératoire permettant de mieux
comprendre la constitution de l’ordre
social, voire d’en instituer un meilleur. Ce
droit naturel appartient en propre à la
personne, à l’individu, et est le plus souvent envisagé comme le droit d’exercer sa
puissance pour faire respecter son dû.

\subsection{Les critiques du droit naturel}

On a souvent présenté Rousseau
comme l’un des critiques les plus radicaux
de l’idée même de droit naturel. Toute
légitimité, tout droit politique ne procéderaient à ses yeux que du contrat social, et
il n’y aurait plus de place pour le droit
naturel. Toutefois, si Rousseau critique
les doctrines qui s’efforcent de faire reposer les principes du droit politique sur le
droit naturel, n'ayant pas de mots assez
durs pour ceux qui tentent de transformer
par leurs arguties le fait en droit, il reconnaît parallèlement que le contrat social
n’est possible que dans la mesure où
l’homme est doué d’une nature morale et
libre. La possibilité même de contracter
une obligation, c’est-à-dire de s’obliger à
la respecter, repose au fond sur ce principe du libre engagement. C’est bien
parce qué l’homme est libre qu'il peut
librement s'engager. Or cette condition
de liberté ne saurait être étrangère à la loi
naturelle, ainsi que le laisse entendre la
réponse polémique que Rousseau fait aux
critiques du procureur Tronchin : « Il n’y
a point de liberté sans lois, ni où quelqu'un est au-dessus des lois : dans l’état
même de nature l’homme n'est libre qu’à
la faveur de la loi naturelle, qui commande à tous.» Cette loi naturelle, qui
%
commande à tous, se limite, dans la description de l’état de nature qu’en donne
le {\it Second Discours}, à l'amour de soi (instinct de conservation)
 et à la pitié naturelle (répugnance naturelle à voir souffrir
son semblable). La loi naturelle qui
consiste dans la conscience de l’obligation
morale de respecter la parole donnée
« qu’il faudrait plutôt appeler loi de raison » demeure sur cette base une autorité
supérieure à l'Etat, principalement en
matière de morale. Aussi « [n'est-il] pas
plus permis d’enfreindre les lois naturelles
par le contrat social, qu'il n’est permis
d’enfreindre les lois positives par des
contrats en particulier, et ce n’est que par
ces lois mêmes qu'existe la liberté qui
donne force à l’engagement ». Il est donc
pour le moins rapide de ne voir dans
l’œuvre de Rousseau que la critique de la
loi naturelle. Et cela d’autant plus qu'avec
le pacte social la loi naturelle prend une
forme rationnelle et n’est plus ce « sentiment vrai, mais très vague et souvent
étouffé par l’amour de nous-même ».
Selon une formule audacieuse d’un brouillon du {\it Contrat social}, grâce au développement des facultés (langage, raison,
sociabilité) corrélatif à la formation des
sociétés, le droit naturel propre à l’état de
nature doit laisser la place à un droit naturel raisonné, dans lequel se forment « nos
premières notions du juste et de l’injuste »
et dont les règles découlent toutes de
cette loi fondamentale, qui ne « prend sa
pleine force » qu'avec le pacte social. En
fait, la loi naturelle acquiert alors avec le
pacte la force d’une loi civile, et en ce sens
on peut dire qu’elle se maintient et avec
elle le droit naturel épuré. On comprend
mieux alors l’injonction du chapitre IV du
{\it Contrat — Des bornes du pouvoir souverain} : « Il s’agit de bien distinguer les
droits respectifs du citoyen et du souverain, et les devoirs qu'ont à remplir les
premiers en qualité de sujets, du droit
naturel dont ils doivent jouir en qualité
d'hommes.» L'Etat ne saurait donc
contrevenir à la loi naturelle, au « droit
naturel raisonné », bien au contraire il
doit en être le garant, donner aux devoirs
de la loi naturelle (réciprocité, justice) la
force de la loi civile.

\subsection{Droit naturel et droits de l’homme}

La notion de droit naturel a eu tendance à subir un net déclin depuis la fin
du {\footnotesize XVIII}$^\text{\,e}$ siècle notamment du fait du développement
%
 du positivisme juridique, qui fait
de l'Etat souverain la source unique du
droit. Toutefois ce déclin du jusnaturalisme doit être nuancé par l’affirmation et
la revendication progressive de « droits de
l’homme » qui tendent de plus en plus,
dans le contexte mondial de l’économie
libérale, à s'identifier aux droits de l’individu. Déjà, la Déclaration des droits de
l’homme et du citoyen de la Révolution,
en son article 2, se référait aux droits
naturels (au pluriel) de l’homme : «Le
but de toute association politique est la
conservation des droits naturels et imprescriptibles de l’homme. Ces droits sont la
liberté, la propriété, la sûreté et la résistance à l’oppression.» Au {\footnotesize XIX}$^\text{\,e}$ siècle, une
nouvelle interprétation des droits de
homme s’est traduite par la revendication de droits économiques et sociaux,
l'épanouissement de l’individu devenant
l’objet d’une revendication politique.
Chacun peut ainsi revendiquer individuellement son droit au bonheur, au travail,
au sexe, aux loisirs, à la santé, à la culture,
etc. Ces droits relèvent-ils encore du droit
en général, du droit naturel en particulier ? On peut en douter. En tout cas, la
revendication constante de ces nouveaux
droits accentue l’opposition si caractéristique du monde contemporain, où l’on
voit l'individu, isolé et autosuffisant,
ayant, malgré une forte tendance à promouvoir la valeur générale et abstraite de
solidarité, le plus souvent perdu le sens
des nécessités de la vie en commun,
n'avoir de cesse qu'il ne se dresse contre
l’État tout en lui demandant d’être davantage État-Providence à son service.
%




%\vspace{0.35cm}
%$\to$\ \ cause, hasard

%%%%%%%%%%%%%%%%%%%%%%%%%%%%%%%%%%%%%%%%%%%%%%%%%%%%%%%%%%%%%%%%%%%%%%%%%%
