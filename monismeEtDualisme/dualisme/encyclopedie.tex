
%%%%%%%%%%%%%%%%%%%%%
\section{Encyclopédie de la philosophie}
%%%%%%%%%%%%%%%%%%%%%
%{\bf }  {\footnotesize V}$^\text{e}$ siècle  {\it }

{\bf dualisme}  terme par lequel on définit, en
général, toute doctrine qui recourt à deux
principes explicatifs, quel que soit son
%
champ d'investigation (on parle en effet
de dualisme religieux, métaphysique, psychologique, scientifique, méthodologique,
etc.). Il est utilisé pour la première fois
par. Th. Hyde dans l’{\it Historia religionis
veterum persarum} (1700), en référence à
l'antique religion perse de Zoroastre, puis
de Mani, qui était fondée sur l’existence
de deux divinités co-éternelles en perpétuel conflit : le Bien et le Mal (ou la
Lumière et les Ténèbres). Dans son
acception première, le mot se réfère donc
essentiellement aux croyances religieuses
qui admettent deux divinités ou deux
principes cosmogoniques : c’est ainsi que
Pierre Bayle l'utilise dans son {\it Dictionnaire}, et Leibniz dans sa {\it Théodicée}. On
doit à Christian Wolff une première
extension du concept, lorsqu'il révèle un
dualisme métaphysique typique de la philosophie à l’époque moderne : de dérivation cartésienne, ce dualisme établit une
ligne de partage entre « substances matérielles et substances spirituelles » ({\it Psychologia rationalis}, § 39). Kant oppose au
dualisme métaphysique un dualisme critique (entre phénomènes et choses en
soi), tandis que la réflexion philosophique
du {\footnotesize XIX}$^\text{e}$, à partir des idéalistes, récuse
toute forme de dualisme, en privilégiant
une explication de la réalité de type
moniste. Diverses formes de dualisme
gnoséologique et métaphysique réapparaissent par la suite : entre apparence et
réalité chez Francis Herbert Bradley,
entre intuition et concept chez Henri
Bergson, entre religion et science chez
William James. Aujourd’hui on a coutume de parler sur un mode moins spécifique d’un dualisme méthodologique dans
différents domaines, tels que celui qui établit le partage entre explication et
compréhension dans les sciences de la
nature et dans celles de l’esprit, entre
structure et superstructure dans la sociologie marxiste, entre inconscient et
conscience en psychanalyse, tautologie et
vérification empirique en logistique,
synchronie et diachronie en linguistique
ainsi que dans le structuralisme anthropologique et épistémologique.

 

  %  -> monisme

 
% \label{labelSite1}
%Référence : \cite{nomSite1}
%%%%%%%%%%%%%%%%%%%%%%%%%%%%%%%%%%%%%%%%%%%%%%%%%%%%%%%%%%%%%%%%%%%%%%%%
