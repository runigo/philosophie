
%%%%%%%%%%%%%%%%%%%%%
\section{Pratique de la philosophie}
%%%%%%%%%%%%%%%%%%%%%
%{\bf }  {\footnotesize V}$^\text{e}$ siècle  {\it }
{\bf D{\footnotesize UALISME}} (n.m.)

\begin{itemize}[leftmargin=1cm, label=\ding{32}, itemsep=1pt]
\item {\footnotesize ÉTYMOLOGIE} :  latin {\it dualis}, « double ».
\item {\footnotesize SENS ORDINAIRE} : dualité ; opposition de deux doctrines ou de deux opinions.
%\item {\footnotesize LOGIQUE} : 
\item {\footnotesize MÉTAPHYSIQUE} : théorie selon laquelle la réalité
est formée de deux substances indépendantes l’une de l’autre et de
nature absolument différente : par exemple, l'esprit et la matière ou,
comme chez Descartes, l'âme et le corps. La théorie contraire est le
monisme.
\end{itemize}

\begin{itemize}[leftmargin=1cm, label=\ding{32}, itemsep=1pt]
%\item {\footnotesize TERME VOISIN} : 
\item {\footnotesize TERME OPPOSÉ} : monisme
\item {\footnotesize CORRÉLATS} : âme; corps ; esprit ; matière.
\end{itemize}

% \label{labelSite1}
%Référence : \cite{nomSite1}
%%%%%%%%%%%%%%%%%%%%%%%%%%%%%%%%%%%%%%%%%%%%%%%%%%%%%%%%%%%%%%%%%%%%%%%%
