\documentclass[11pt, a4paper]{report}
%\documentclass[11pt, a4paper]{article}

%====================== PACKAGES ======================
\usepackage[french]{babel}

\frenchbsetup{StandardLists=true}
\usepackage{enumitem}
\usepackage{pifont}

\usepackage[utf8x]{inputenc}
%\usepackage[latin1]{inputenc}

%pour gérer les positionnement d'images
\usepackage{float}
\usepackage{amsmath}
\DeclareMathOperator{\dt}{dt}
\usepackage{graphicx}
%\usepackage{tabularx}
\usepackage[colorinlistoftodos]{todonotes}
\usepackage{url}

%pour les informations sur un document compilé en PDF et les liens externes / internes
\usepackage[pdfborder=0]{hyperref}
\hypersetup{
	colorlinks = true
	}

%pour la mise en page des tableaux
\usepackage{array}
\usepackage{tabularx}
\usepackage{multirow}
\usepackage{multicol}
\setlength{\columnsep}{50pt}

%pour utiliser \floatbarrier
%\usepackage{placeins}
%\usepackage{floatrow}

%espacement entre les lignes
\usepackage{setspace}

%modifier la mise en page de l'abstract
\usepackage{abstract}

%police et mise en page (marges) du document
\usepackage[T1]{fontenc}
\usepackage[top=2cm, bottom=2cm, left=2cm, right=2cm]{geometry}

%Pour les galerie d'images
\usepackage{subfig}

\usepackage{pdfpages}

\usepackage{tikz}
\usetikzlibrary{trees}
\usetikzlibrary{decorations.pathmorphing}
\usetikzlibrary{decorations.markings}
\usetikzlibrary{decorations.pathreplacing,calligraphy}
%\usetikzlibrary{decorations}
\usetikzlibrary{angles, quotes}
\usepackage{verbatim}

\usepackage{appendix}

\usepackage{comment}

\usepackage{xcolor}

%\PreviewEnvironment{tikzpicture}
%\setlength\PreviewBorder{0pt}%

%====================== INFORMATION ET REGLES ======================

%rajouter les numérotation pour les \paragraphe et \subparagraphe
\setcounter{secnumdepth}{4}
\setcounter{tocdepth}{4}

\hypersetup{							% Information sur le document
pdfauthor = {Stephan Runigo},			% Auteurs
pdftitle = {Documentation},			% Titre du document
pdfsubject = {Documentation},		% Sujet
pdfkeywords = {Document},	% Mots-clefs
pdfstartview={FitH}}	% ajuste la page à la largeur de l'écran
%pdfcreator = {MikTeX},% Logiciel qui a crée le document
%pdfproducer = {} % Société avec produit le logiciel

%======================== DEBUT DU DOCUMENT ========================
%
\begin{document}
%
%régler l'espacement entre les lignes
\newcommand{\HRule}{\rule{\linewidth}{0.5mm}}
%
% Titre, résumé, ... %
%
%
\begin{titlepage}
%
~\\[1cm]

\begin{center}
%\includegraphics[scale=0.5]{./presentation/chambreABulle}
\end{center}

\textsc{\Large }\\[0.5cm]

% Title \\[0.4cm]
\HRule

\begin{center}
{\huge \bfseries Spiritualisme et\\
matérialisme\\[0.4cm] }
\end{center}

\HRule \\[1.5cm]


\vfill

\hfill
\begin{minipage}{0.4\textwidth}
\begin{flushright} \large
%\emph{Auteur:}\\
%Stephan \textsc{Runigo}
Extraits de dictionnaires et d'encyclopédies
\end{flushright}
\end{minipage}

\vfill
{\sf \footnotesize
\begin{itemize}[leftmargin=1cm, label=\ding{32}, itemsep=1pt]
\item {\bf Objet : } Étudier les concepts liés au spiritualisme et au matérialisme.
\item {\bf Contenu : } Définitions et philosophie encyclopédique.
\item {\bf Public concerné : } Néophyte.
\end{itemize}
}

\vfill

% Bottom of the page
{\large \today}

\end{titlepage}

\newpage
%\begin{center}
\Large
Résumé
\normalsize
\end{center}
\vspace{3cm}
\begin{itemize}[leftmargin=1cm, label=\ding{32}, itemsep=21pt]
\item {\bf Objet : éclairer } .
\item {\bf Contenu : définitions} .
\item {\bf Public concerné : néophyte} .
\end{itemize}

\vspace{3cm}

résumé.

\vspace{3cm}

Ce document contient des définitions provenant de divers dictionnaires.




\thispagestyle{empty}

\begin{center}
\Large
%Introduction
Préambule
\normalsize
\end{center}
\vspace{3cm}

Ce document est une compilation d'articles provenant de quatre ouvrages : un dictionnaire encyclopédique de poche, {\it La pratique de la philosophie} destiné aux lycéens, une encyclopédie de la philosophie destinée aux néophytes, et le dictionnaire philosophique d'André Comte-Sponville.
les chapitres contiennent les articles des trois premiers ouvrages, les articles du dictionnaire de philosophie d'André Comte-Sponville sont reproduits en annexe.

\vspace{1.3cm}

Chaque chapitre contient les articles correspondant à une notion particulière. Ces notions ont été choisies en raison de leurs liens avec la question du hasard. Ces choix ont été guidés : 1. Par les renvois vers d'autres articles présent dans les ouvrages, 2. Mes propres choix, liés à mes connaissances, 3. La volonté d'obtenir une quantité raisonnable d'information.

\vspace{1.3cm}

Dans {\it La pratique de la philosophie}, l'article concernant la nécessité renvoit à des textes de Spinoza dont le choix reste subjectif à l'ouvrage. J'ai néanmoins reproduit en annexe ces textes ainsi que l'article concernant Spinoza. (les autres annexes sont d'autres renvois de cet ouvrage)

Les articles compilés dans ce document comportent donc les choix "discutables" réalisés dans les trois ouvrages utilisés. Il s'agit donc d'un document de "travail" destiné à apporter quelques points de vues philosophiques de manière relativement élémentaire.

\vspace{1.3cm}

Les trois premiers chapitres abordent les thèmes du hasard, de la nécessité puis du vitalisme. Les chapitres suivants élargissent le champ de vision philosophique en abordant les thèmes du déterminisme, de la contingence et de la providence.

\vspace{2.3cm}

\hfill Stephan Runigo

%%%%%%%%%%%%%%%%%%%%%%%%%%%%%%%%%%%%%%%%%%%%%%%%

%
\newpage
%
%

\thispagestyle{empty}
\begin{center}
\Large
%Introduction
Les ouvrages utilisés
\normalsize
\end{center}

\vspace{1.3cm}

Premières de couverture

\begin{center}
\includegraphics[scale=0.43]{./presentation/dictionnaire-1}
\hfill
\includegraphics[scale=0.43]{./presentation/pratique-1}
\hfill
\includegraphics[scale=0.43]{./presentation/encyclopedie-1}
\end{center}

\vspace{1.3cm}

Quatrièmes de couverture

\begin{center}
\includegraphics[scale=0.43]{./presentation/dictionnaire-2}
\hfill
\includegraphics[scale=0.43]{./presentation/pratique-2}
\hfill
\includegraphics[scale=0.43]{./presentation/encyclopedie-2}
\end{center}



%\newpage
%

%
% Table des matières
\tableofcontents
\thispagestyle{empty}
\setcounter{page}{0}
%
%espacement entre les lignes des tableaux
\renewcommand{\arraystretch}{1.5}
%
%====================== INCLUSION DES CHAPITRES ======================
%
~
\thispagestyle{empty}
%recommencer la numérotation des pages à "1"
\setcounter{page}{0}
\newpage
%
%\input{./chapitre1/chapitre1.tex}
\input{./monade/.tex}
\input{./monisme/.tex}
\input{./dualisme/.tex}
\input{./spiritualisme/.tex}
%\input{.//.tex}
%
%
%====================== INCLUSION DE LA BIBLIOGRAPHIE ======================
%
%récupérer les citation avec "/footnotemark" : 
\nocite{*}
%
% choix du style de la biblio
\bibliographystyle{plain}
%
% inclusion de la biblio
\cleardoublepage
\addcontentsline{toc}{chapter}{Bibliographie}
\bibliography{bibliographie.bib}
%
%====================== FIN DU DOCUMENT ======================
%
\end{document}
%%%%%%%%%%%%%%%%%%%%%%%%%%%%%%%%%%%%%%%%%%%%%%%%%%%%%%%%%%%%%%%%%%%%%%%%%%%%%%%%%
