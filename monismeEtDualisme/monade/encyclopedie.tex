
%%%%%%%%%%%%%%%%%%%%%
\section{Encyclopédie de la philosophie}
%%%%%%%%%%%%%%%%%%%%%
%{\bf }  {\footnotesize V}$^\text{e}$ siècle  {\it }
{\bf monade} terme utilisé dans l’école pythagoricienne pour désigner l’unité originelle
de laquelle dérive la série des nombres
(en grec, {\it monâs}, « unité »). Archytas de
Tarente puis Proclus distinguèrent la
monade de l’Un absolu, dont elle constitue le principe de limitation intelligible.
Platon à son tour appelle « monades » les
Idées, pour en bien manifester le caractère essentiel et l'éloignement par rapport
à la réalité empirique. Le terme, tout en
préservant la connotation d’absolument
« simple » et d’« irréductible » à la division, se charge de significations transcendantes chez les néoplatoniciens chrétiens
(Théodoric de Chartres, Dominique Gundisalvi), pour lesquels il désigne Dieu
comme unité ultime et essentielle. Dans
la pensée de la Renaissance, le concept de
monade est utilisé par Nicolas de Cues,
pour qui il désigne un microcosme, une
unité en miniature, un « miroir du tout »
(dans ce sens, il est déjà présent dans
l’œuvre de al-Kindï). Mais c’est surtout
Giordano Bruno qui développe ce
concept, en en faisant le fondement de sa
mathématique magique ({\it De minimo, De
la monade}). Pour Bruno, les monades
sont les plus petites parties qui composent
les corps et ce qui en définit la structure.
La notion de monade est reprise dans un
sens analogue par Henry More ({\it Enchiridion metaphysicum}) ainsi que par Leibniz, qui en fera le fondement d’une
véritable conception de l’univers ou « monadologie ». Les monades leibniziennes
sont des substances ou des principes actifs
qui, chacun à leur façon, réfléchissent
l'univers entier en un harmonieux enchaînement de perceptions. Par l’intermédiaire de l’enseignement de Martin
Knutzen, le concept est repris par Kant
dans la phase pré-critique de sa pensée.
Dans la {\it Monadologie physique}, Kant
tente en effet de concilier la métaphysique de Leibniz et la physique de Newton au moyen du concept de monade
physique : impénétrables et élastiques, les
monades physiques sont les éléments premiers de l’espace universel newtonien.
Le terme réapparaît à l’époque du
romantisme, par exemple chez Goethe et
surtout chez Johann Friedrich Herbart,
dont les « réels » sont une reprise de la
monadologie leibnizienne. À la fin du
{\footnotesize XIX}$^\text{e}$ siècle, Charles Renouvier utilise le
concept de monade pour indiquer la
« substance simple » comme élément qui
entre dans la composition des substances
complexes, c’est-à-dire des données de
l'expérience ({\it Nouvelle  Monadologie}).
Mais à la même période le terme se
retrouve également dans l’œuvre de
Rudolf Hermann Lotze, qui conçoit les
monades à la manière de Leibniz, comme
unités de conscience ou singularités spirituelles, constitutives de la réalité ultime
de l’univers ({\it Microcosmus}). Enfin, Alfred
North Whitehead et Edmund Husserl
parlent également de monade : chez le
premier, le terme indique les événements
%
temporels de l’esprit, chez le second, il
caractérise le rapport intersubjectif des
{\it ego} transcendantalement « réduits » (corrélation inter-monadique).

 

% -> Husserl ; Leibniz

%%%%%%%%%%%%%%%%%%%%%%%%%%%%%%%%%%%%%%%%%%%%%%%%%%%%%%%%%%%%%%%%%%%%%%%%
