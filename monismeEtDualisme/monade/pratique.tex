
%%%%%%%%%%%%%%%%%%%%%
\section{Pratique de la philosophie}
%%%%%%%%%%%%%%%%%%%%%
%{\bf }  {\footnotesize V}$^\text{e}$ siècle  {\it }
{\bf M{\footnotesize ONADE}} (n.f.)

\begin{itemize}[leftmargin=1cm, label=\ding{32}, itemsep=1pt]
\item {\footnotesize ÉTYMOLOGIE} : grec {\it monados}, « unité ».
%\item {\footnotesize SENS ORDINAIRE} : 
%\item {\footnotesize LOGIQUE} : 
\item {\footnotesize MÉTAPHYSIQUE} : terme utilisé d’abord par Platon pour désigner l’Idée en tant que réalité une, toujours identique à elle-même et incorruptible ; ce terme fut ensuite repris par Leibniz qui l'a rendu célèbre.
\end{itemize}

Chez Leibniz, la monade désigne une
« substance simple, sans parties, qui
entre dans les composés ». Si les
monades sont, selon l'expression de
Leibniz lui-même, de véritables
« atomes » ou éléments des choses, elles
ne sont pourtant pas matérielles, puisqu'elles sont indivisibles et incorruptibles. Du point de vue extérieur, elles
ne peuvent donc commencer ni finir, si
ce n'est par création ou annihilation, ni
même être modifiées. Les monades sont
« sans portes, ni fenêtres ». Du point de
vue intérieur, elles contiennent par
contre non seulement leurs attributs,
mais aussi l’univers tout entier, qu'elles
« expriment » de leur point de vue. Il
existe cependant entre les monades des
degrés de perfection. Leibniz distingue
les monades douées de perception et
d’appétition, mais non de mémoire,
comme sont les plantes ; les monades
douées de mémoire, comme chez les
animaux ; et enfin les monades douées
de raison et d’aperception, c’est-à-dire
de conscience réfléchie, comme chez
%
les humains. Seul l’homme, capable de
saisir l'harmonie du monde, qu'il réfléchit comme un miroir vivant, peut s'élever jusqu’à l'idée de son créateur.

%\vspace{.3cm}

\begin{itemize}[leftmargin=1cm, label=\ding{32}, itemsep=1pt]
%\item {\footnotesize TERME VOISIN} : 
%\item {\footnotesize TERME OPPOSÉ} : 
\item {\footnotesize CORRÉLATS} : âme ; identité ; inter-subjectivité ; personne ; substance.
\end{itemize}

{\bf « M{\footnotesize ONADOLOGIE}} »

Titre donné de façon posthume à un
ouvrage de Leibniz, écrit en 1714, par
l'éditeur allemand Erdmann, qui le
publia en 1840.

%%%%%%%%%%%%%%%%%%%%%%%%%%%%%%%%%%%%%%%%%%%%%%%%%%%%%%%%%%%%%%%%%%%%%%%%
