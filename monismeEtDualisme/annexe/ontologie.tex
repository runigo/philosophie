
%%%%%%%%%%%%%%%%%%%%%
\chapter{Ontologie}
%%%%%%%%%%%%%%%%%%%%%

%%%%%%%%%%%%%%%%%%%%%
%\section{Pratique de la philosophie}
%%%%%%%%%%%%%%%%%%%%%
%{\bf }  {\footnotesize V}$^\text{e}$ siècle  {\it }
{\bf O{\footnotesize NTOLOGIE}} (n.f.)

\begin{itemize}[leftmargin=1cm, label=\ding{32}, itemsep=1pt]
\item {\footnotesize ÉTYMOLOGIE} : grec {\it on}, {\it ontos}, participe présent de {\it einai}, « être », et
{\it logos}, « discours ».
\item {\footnotesize PHILOSOPHIE} : discours qui prend pour objet non
pas telle catégorie particulière d’être
mais « l'être en tant qu'être ».
\end{itemize}

Bien que le mot ontologie ne date que
du {\footnotesize XVII}$^\text{e}$ siècle, le principe peut en être
trouvé chez Aristote, selon lequel « il y
a une science qui étudie l'être en tant
qu'être », opposée aux sciences particulières « qui découpent quelque partie de
l'être et en étudient les propriétés »
({\it Métaphysique}, $\Gamma$, 1).

L'ontologie serait ainsi l'étude de l’essence de l’être de ce qui fait qu'un être
est ; elle serait l'étude du fondement de
l'ordre des choses. C’est pourquoi l’ontologie a fini par devenir synonyme de
métaphysique et s'est posée comme
science suprême ou universelle.
Heidegger reproche à la métaphysique
traditionnelle d’avoir donné à la question ontologique une réponse positive
en cherchant l'essence de l'être dans un
être ultime et inconditionné (Dieu par
exemple, considéré comme un « être
suprême »). Car celui-ci est encore,
même s'il est condition de tous les
autres, un être, un étant — fut-il
suprême — et non pas l'être. Ainsi Heidegger distingue l’ontologie des questions ontiques, qui ont pour objet les
étants.

La philosophie contemporaine fait donc
de l’ontologie la compréhension du sens
de l’être, plus qu'une « science » déterminée. Cette question du sens de l'être,
si elle est abordée à partir d'une
réflexion sur le langage, devient une
interrogation sur le statut des entités à
l'œuvre dans le discours (Bertrand Russell, W.O. Quine). Par exemple, pour
Quine, la question : « Y at-il des dinosaures ? » est une question qui relève des
sciences naturelles ; mais la question de
savoir ce qu'on signifie par « il y a » est
une question ontologique.

\begin{itemize}[leftmargin=1cm, label=\ding{32}, itemsep=1pt]
\item {\footnotesize TERME VOISIN} : métaphysique.
\end{itemize}

{\bf O{\footnotesize NTIQUE}} (adj.)

\begin{itemize}[leftmargin=1cm, label=\ding{32}, itemsep=1pt]
\item {\footnotesize ÉTYMOLOGIE} : grec {\it on}, {\it ontos}, participe présent de {\it einai}, « être ».
\item {\footnotesize CHEZ HEIDEGGER} : désigne ce qui se rapporte à l'étant, c'est-à-dire aux êtres du monde, par opposition à l'ontologique qui se rapporte à « l'être en tant qu'être ».
\end{itemize}


{\bf P{\footnotesize REUVE ONTOLOGIQUE}} (n.f.)

{\it Cf.} Preuve de l'existence de Dieu.

\begin{itemize}[leftmargin=1cm, label=\ding{32}, itemsep=1pt]
\item {\footnotesize CORRÉLATS} : être ; théologie.
\end{itemize}

{\bf P{\footnotesize REUVES DE L’EXISTENCE DE DIEU}}

Kant distingue, pour en montrer l'im-
possibilité, trois grands types de « preu-
ves » de l'existence de Dieu :

1. la preuve {\it physico-théologique} : Dieu
est conclu à partir de l'ordre régnant
dans le monde ;

2. la preuve {\it cosmologique}: Dieu, être
nécessaire, est conclu à partir de la
contingence du monde — ces deux
preuves sont a {\it posteriori} : elles
remontent de l'effet à la cause ;

3. la preuve {\it }ontologique: preuve {\it a priori}, elle veut démontrer l'existence
de Dieu par l'analyse de son essence ou
concept. On trouve cette preuve chez
saint Anselme ({\footnotesize XI}$^\text{e}$ siècle) et chez Descartes. L'existence de Dieu est nécessairement contenue dans le concept
d'un être parfait, parce que s’il manquait
à cet être un attribut comme l'existence,
il ne serait pas parfait.

La critique de Kant est la suivante : parler d’existence équivaut à poser un
concept en regard de l’expérience, mais
l'existence n’est pas un attribut de ce
concept ; « Dieu existe » n’est donc pas
un jugement de même forme que « Dieu
est bon ». On peut conclure de l'essence
de Dieu à sa bonté, simple attribut, mais
pas nécessairement à son existence.

%%%%%%%%%%%%%%%%%%%%%%%%%%%%%%%%%%%%%%%%%%%%%%%%%%%%%%%%%%%%%%%%%%%%%%%%%%%%%%%%%%%%%
