
%%%%%%%%%%%%%%%%%%%%%
\chapter{Attribut}
%%%%%%%%%%%%%%%%%%%%%

%%%%%%%%%%%%%%%%%%%%%
%\section{Pratique de la philosophie}
%%%%%%%%%%%%%%%%%%%%%
%{\bf }  {\footnotesize V}$^\text{e}$ siècle  {\it }
{\bf A{\footnotesize TTRIBUT}} (n.m.)

\begin{itemize}[leftmargin=1cm, label=\ding{32}, itemsep=1pt]
\item {\footnotesize ÉTYMOLOGIE} : latin {\it attribuere},
« attribuer ».
\item {\footnotesize SENS ORDINAIRE} : ce qui
est propre à un être ou à une chose
et permet de la distinguer de toute
autre.
\item {\footnotesize LOGIQUE} : (synonyme de prédicat) : ce qui, dans une proposition,
est affirmé ou nié d’un sujet (ex. :
« Socrate [sujet] est mortel [attribut] »).
%\item {\footnotesize MÉTAPHYSIQUE} : 
\end{itemize}

Quelle est la nature des choses ? Suivant en
cela la tradition scolastique, la philosophie
a longtemps formulé cette question en
termes de substance, d’attribut, et d’accident. La substance est la réalité permanente, l’accident est un caractère qui peut
être modifié ou supprimé, tandis que l’attribut est une propriété importante ou
essentielle, ce qui permet de définir
quelque chose. Ainsi, selon Descartes, ni
la saveur ni la couleur, ni l'odeur ne constituent des propriétés essentielles de la cire :
seule l’étendue en est un véritable attribut,
c'est-à-dire nous permet de la connaître ({\it cf. Deuxième Méditation}
, analyse du morceau
de cire). Dans une perspective voisine, Spinoza définit l’attribut comme « ce que
l'entendement perçoit de la substance
comme constituant son essence »
({\it Éthique}, 1, 1) : ce qui signifie que toutes
les caractéristiques essentielles de la substance doivent se retrouver dans ses attributs.

\begin{itemize}[leftmargin=1cm, label=\ding{32}, itemsep=1pt]
\item {\footnotesize TERME VOISIN} : prédicat; propriété ; qualité.
\item {\footnotesize TERME OPPOSÉ} : substance.
\item {\footnotesize CORRÉLATS} : accident ;
essence ; expression ; mode.
\end{itemize}

%%%%%%%%%%%%%%%%%%%%%%%%%%%%%%%%%%%%%%%%%%%%%%%%%%%%%%%%%%%%%%%%%%%%%%%%%%%%%%%%%%%%%
