
%%%%%%%%%%%%%%%%%%%%%
\chapter{Substance}
%%%%%%%%%%%%%%%%%%%%%

%%%%%%%%%%%%%%%%%%%%%
%\section{Pratique de la philosophie}
%%%%%%%%%%%%%%%%%%%%%
%{\bf } «  » {\footnotesize V}$^\text{e}$ siècle  {\it }
{\bf S{\footnotesize UBSTANCE}} (n.f.)

\begin{itemize}[leftmargin=1cm, label=\ding{32}, itemsep=1pt]
\item {\footnotesize ÉTYMOLOGIE} : latin {\it substantia},
« substance », de {\it substare}, « se tenir
dessous ».
\item {\footnotesize CHIMIE} : matière scientifiquement définie, c'est-à-dire
considérée du point de vue des propriétés par lesquelles elle se distingue des autres corps.
\item {\footnotesize CHEZ ARISTOTE} : sujet et support permettant tel ou tel changement ; la substance première est l'être individuel
qui demeure le même tout en subissant des modifications ; la substance
seconde est le sujet (au sens grammatical du terme cette fois) d’une
proposition, dont on peut affirmer
ou nier des prédicats: ainsi
l'homme, ou l’animal, qui est ceci
ou cela, mais qui n'existe pas véritablement, c’est-à-dire concrètement.
\item {\footnotesize CHEZ DESCARTES} : 1. Support permanent des attributs,
qualités ou accidents ; la « substance
pensante » a pour attribut principal
la pensée, tandis que la substance
matérielle a pour attribut essentiel
l'étendue. 2. Ce qui n’a besoin que
de soi-même pour exister (« à proprement parler, il n’y a que Dieu qui
soit tel »).
\item {\footnotesize CHEZ SPINOZA} : synonyme de Dieu.
\item {\footnotesize CHEZ KANT} : concept {\it a priori}, relevant du jugement catégorique, et constituant la
première des catégories de la relation.
\end{itemize}

Il est assez difficile de dissocier, chez
Aristote, la notion de substance de la
notion d'essence. En un certain sens, la
forme (ou quiddité) est plus indiscutablement substance que l'individu qui
n'est qu'un composé de forme et de
matière. Cependant, l'individu est bien
la seule véritable « substance », dans la
mesure où une forme ne peut exister
que réalisée dans une matière, c'est-à-dire dans un être individuel : l'essence
et l'être individuel se rejoignent alors, et
constituent ce qu'Aristote appelle la
« forme spécifique » de l'individu. Un
problème du même ordre se pose à
nouveau chez Descartes qui établit une
distinction entre substance et attribut
essentiel, tout en reconnaissant que la
substance seule, indépendamment de
ses attributs, est pratiquement inconnaissable. Le problème sera, d’une certaine
manière, résolu par Kant pour qui la
substance, c'est-à-dire l’idée de permanence du réel dans le temps, est un
concept {\it a priori} (ou catégorie du jugement), c'est-à-dire une condition, ou
forme a priori de pensée, et non pas
une chose réelle qui serait indépendante du sujet et néanmoins connaissable.

\begin{itemize}[leftmargin=1cm, label=\ding{32}, itemsep=1pt]
\item {\footnotesize TERME VOISIN} : essence ; réalité.
\item {\footnotesize TERME OPPOSÉ} : accident ; attribut ; prédicat ; qualité.
\item {\footnotesize CORRÉLATS} : Dieu: eccéité ; forme ; matière.
\end{itemize}

%%%%%%%%%%%%%%%%%%%%%%%%%%%%%%%%%%%%%%%%%%%%%%%%%%%%%%%%%%%%%%%%%%%%%%%%%%%%%%%%%%%%%
