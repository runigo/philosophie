
%%%%%%%%%%%%%%%%%%%%%
\chapter{Essence}
%%%%%%%%%%%%%%%%%%%%%

%%%%%%%%%%%%%%%%%%%%%
%\section{Pratique de la philosophie}
%%%%%%%%%%%%%%%%%%%%%
%{\bf }  {\footnotesize V}$^\text{e}$ siècle  {\it }
{\bf E{\footnotesize SSENCE}} (n.f.)

\begin{itemize}[leftmargin=1cm, label=\ding{32}, itemsep=1pt]
\item {\footnotesize ÉTYMOLOGIE} : latin {\it essentia}, de
{\it esse}, « être », traduction du grec {\it ousia}.
\item {\footnotesize SENS ORDINAIRE} : ce qui fait
la nature d’une chose ou d’un être.
\item {\footnotesize PHILOSOPHIE} : 1. Par opposition à
accident, ce qui constitue la nature
permanente d’un être, indépendamment de ce qui lui arrive; en ce
sens, proche de « substance ». 2. Par
opposition à existence, ce qu'est un
être, ce qui le définit, indépendamment du fait qu'il existe ; en ce sens,
proche de « concept ».
\end{itemize}

Chercher l'essence d'un être, c’est chercher ce qui en constitue la nature. Cette
quête de l'essence est celle que mène
Platon, à travers ses {\it Dialogues}. Il
cherche à y définir, par exemple, ce que
sont la justice, la beauté, le courage en
soi, indépendamment des objets sensibles où ces essences — qu'il appelle
Idées — s'incarnent imparfaitement.
C’est pourquoi il oppose le monde intelligible, monde des essences pures ou
des Idées, au monde sensible. Pour Platon, l'essence, parce qu'elle présente les
caractères du vrai — universalité et
nécessité — a plus d'existence et de
dignité que les objets sensibles. C'est
cette priorité de l'essence que contestera
le nominalisme, à travers la fameuse
« querelle des Universaux ». Pour le
nominalisme, l'essence n’est qu’une idée
générale, construite par abstraction, et le
nom qui lui correspond n'est qu'un
signe commode pour représenter les
objets toujours singuliers qui, seuls,
existent, Mais on peut aussi admettre
qu'il existe des essences singulières.
L'essence ici s'opposera non plus à
l'accident, puisqu'une essence singulière contient en elle tous ses accidents,
ou attributs, mais s'opposera à l'existence, comme le possible au réel. Pour
Leibniz, par exemple, et dans une perspective chrétienne, la Création est ce
par quoi Dieu fait passer les essences à
l'existence. C'est cette antériorité de
l'essence sur l'existence que contestera
l’existentialisme athée. À travers la formule « L'existence précède l'essence »,
l'existentialisme entend affirmer que
l'homme se crée en quelque sorte lui-même, à travers ses actes et ses Choix.
En d’autres termes, l'homme n'a d'autre
définition que celle qu'il se donne.

\begin{itemize}[leftmargin=1cm, label=\ding{32}, itemsep=1pt]
\item {\footnotesize TERME VOISIN} : concept ; idée ;
substance.
\item {\footnotesize TERME OPPOSÉ} : accident ; existence.
\item {\footnotesize CORRÉLATS} : existence ; existentialisme ; idée ;
jugement ; nominalisme ; réalisme ;
universel (querelle des Universaux).
\end{itemize}

%%%%%%%%%%%%%%%%%%%%%%%%%%%%%%%%%%%%%%%%%%%%%%%%%%%%%%%%%
