
%%%%%%%%%%%%%%%%%%%%%
\section{Encyclopédie de la philosophie}
%%%%%%%%%%%%%%%%%%%%%
%{\bf }  {\footnotesize V}$^\text{e}$ siècle  {\it }
{\bf monisme} terme utilisé pour la première
fois par Christian Wolff en 1734 dans la
{\it Psychologia  rationalis} ({\it Psychologie
rationnelle}), pour désigner tout système
philosophique qui admet une seule sorte
de substance (en grec {\it monos} signifie
%
« seul », « unique »). Le terme s’appliquait aussi bien aux conceptions idéalistes
qu'aux conceptions matérialistes, dans la
mesure où elles s’accordaient pour reconduire l’ensemble de la réalité à un principe unitaire sous-jacent à l’apparente
multiplicité et discontinuité des phénomènes, et pour nier (pour des raisons qui
peuvent être différentes) toute dualité
entre la matière et l’esprit, entre le monde
et Dieu. C’est pourquoi, conformément à
cette orientation, ont été définies comme
« monistes » la philosophie de Spinoza,
dans laquelle la pensée et l'étendue sont
les attributs d’une substance unique ; la
conception leibnizienne de la substance,
qui réduit la substante corporelle à la substance spirituelle ; la vision matérialiste
du monde liée au mécanisme rationnel ;
et les philosophies idéalistes de Schelling,
de Hegel et de Francis Herbert Bradley.
Entre la fin du {\footnotesize XIX}$^\text{e}$ et le début du {\footnotesize XX}$^\text{e}$ s
furent appelées « monistes », d’un côté,
les conceptions du biologiste allemand
Ernst Heinrich Haeckel (qui fonda en
1906, avec W. Ostwald, la « {\it Deutscher
Monistenbund} », Association moniste
allemande), et, d’un autre côté, les positions philosophico-scientifiques  exprimées par la revue américaine {\it The Monist},
fondée par P. Carus en 1888. Depuis sa
fondation en 1936, cette revue publia des
contributions qui convergeaient dans la
volonté de libérer la philosophie du subjectivisme pour en faire une science du
réel conçu comme un système rigoureusement unitaire.

%%%%%%%%%%%%%%%%%%%%%%%%%%%%%%%%%%%%%%%%%%%%%%%%%%%%%%%%%%%%%%%%%%%%%%%%
