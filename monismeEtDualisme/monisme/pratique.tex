
%%%%%%%%%%%%%%%%%%%%%
\section{Pratique de la philosophie}
%%%%%%%%%%%%%%%%%%%%%
%{\bf }  {\footnotesize V}$^\text{e}$ siècle  {\it }
{\bf M{\footnotesize ONISME}} (n.m.)

\begin{itemize}[leftmargin=1cm, label=\ding{32}, itemsep=1pt]
\item {\footnotesize ÉTYMOLOGIE} : dérivé du grec {\it monos}, « unique ».
%\item {\footnotesize SENS ORDINAIRE} : 
%\item {\footnotesize LOGIQUE} : 
\item {\footnotesize PHILOSOPHIE, MÉTAPHYSIQUE} : doctrine selon
laquelle toute la réalité renvoie à
une substance fondamentale posée
comme principe unique d’explication.
\end{itemize}
 
Depuis le philosophe Wolff
({\footnotesize XVIII}$^\text{e}$ siècle), inventeur du terme
{\it monisme}, on distingue traditionnellement un monisme de type matérialiste
(tout est matière) d'un monisme de type
idéaliste (tout est esprit). Si l'un et
l'autre s'opposent évidemment entre
eux, le monisme en général s'oppose au
dualisme qui affirme, avec Descartes,
l'existence de deux substances distinctes, la matière et l'esprit. Toutefois il
existe aussi, dans la philosophie
contemporaine, un « monisme neutre »
(W. James, E. Mach, B. Russell). Celui-ci pose que le monde physique et le
monde psychique ne sont ni deux substances différentes, ni réductibles l'un à
l'autre, mais qu'ils constituent deux
registres de phénomènes renvoyant à
une même substance sur la nature de
laquelle on n’a pas à se prononcer. Ce
monisme neutre, d'inspiration positiviste, vise surtout à fonder
ontologiquement la possibilité de l'unité
du savoir sans recourir à des hypothèses
métaphysiques arbitraires sur la « matérialité » ou l’« idéalité » foncière du réel.

\begin{itemize}[leftmargin=1cm, label=\ding{32}, itemsep=1pt]
%\item {\footnotesize TERME VOISIN} : 
\item {\footnotesize TERME OPPOSÉ} : dualisme.
\item {\footnotesize CORRÉLATS} : âme (union de l'âme et du
corps) ; esprit ; matière ; ontologie ;
substance.
\end{itemize}

%%%%%%%%%%%%%%%%%%%%%%%%%%%%%%%%%%%%%%%%%%%%%%%%%%%%%%%%%%%%%%%%%%%%%%%%
