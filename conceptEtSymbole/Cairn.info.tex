\section{Quand dire c’est montrer : le langage symbolique}

\vspace{0.24cm}
{\footnotesize Enjeux théologique, Elbatrina Clauteaux, Revue Lumen Vitae 2017

https://www.cairn.info/revue-lumen-vitae-2017-1-page-19.htm}
\vspace{0.31cm}


\hfill
\begin{minipage}[c]{.65\linewidth}
La nature est un temple où des vivants piliers

Laissent parfois sortir des confuses paroles ;

L’homme y passe à travers des forêts de symboles

Qui l’observent avec des regards familiers.

Comme de longs échos qui de loin se confondent

Dans une ténébreuse et profonde unité,

Vaste comme une nuit et comme la clarté,

Les parfums, les couleurs et les sons se répondent...

{\scriptsize (Charles Baudelaire, « Correspondances », dans Les Fleurs du mal.)}
\end{minipage}
\hfill
\vspace{0.31cm}

Si parler c’est dire quelque chose sur quelque chose à quelqu’un, le langage symbolique montre en disant et dit en montrant. Un flou imaginant, schématisme métaphorique pré-conceptuel prend vie dans notre esprit grâce à l’imagination et la machine de notre raison se met en marche, toute tendue vers une compréhension, vers une réception qui re-décrit la réalité plate d’une nature où des piliers devenus vivants font un concert en clair-obscur où les échos des parfums, des couleurs, des sons, de textures, se répondent.

Ces vers des Fleurs du Mal de Baudelaire peuvent donner à penser au théologien puisqu’ils interprètent en montrant, en faisant entendre, sentir et presque toucher la beauté sonore et symphonique de la création. Mais c’est un langage verbal et le questionnement qui nous habite voudrait englober les arts « silencieux », non verbaux, les arts plastiques : la peinture, la sculpture, l’architecture, etc. Ces arts qui parlent en montrant. Il s’agit d’une tentative de penser le rapport entre la théologie, œuvre de pensée spéculative et conceptuelle, et l’expérience artistique, expérience de l’artiste et du récepteur de l’œuvre. Pour ce faire, après une problématisation préliminaire et une double présupposition qui ouvre la réflexion, nous ferons quelques remarques sur le rapport entre symbole et concept avant de conclure sur les puissances du symbolique pour la théologie des arts dans une conclusion ouverte.

\subsection{De la connaissance artistique au savoir théologique}

Quelque chose d’unique, d’original, de sensiblement touchant et stimulant pour la pensée est exprimé à travers l’expérience artistique, une expérience hautement symbolique. À la manière de la liturgie, la catéchèse et tous les lieux théologiques pratiques, elle provoque la réflexion de la foi. L’art est une pratique bien spécifique de l’expression humaine, une pratique dans laquelle l’homme exerce sa transcendance transcendante et immanente, et cela non seulement à l’intérieur du christianisme ou du religieux. L’art appartient à la manière d’être de l’homme. Le traitement de l’art en théologie requiert de ce fait une compétence en conversation avec l’anthropologie, la sociologie, la philosophie, l’histoire. L’art est cet interstice qui ouvre le réel à son possible, il fait de l’inédit, de l’original, de l’extraordinaire, du nouveau à partir de l’ancien. Il se situe dans une logique de « surplus », d’excédence…

\vspace{0.24cm}
{\footnotesize « C’est parce qu’au sein de sa productivité plastique la nature nous réserve encore quelque chose à figurer, c’est parce qu’elle laisse un espace vide à l’esprit créateur de l’homme que l’art est “possible”. »}
\vspace{0.31cm}

Ainsi autant l’expression de l’artiste que la réception de l’observateur de son œuvre sont des manières de dire et de vivre le réel et l’humain autrement, avec un coefficient « plus ». Car, en tant qu’expérience, l’art est cet intervalle parcouru de soi à soi à l’intérieur duquel une œuvre naît pour l’artiste et « quelque chose » touche le récepteur. Tous deux sont affectés. Il s’agit d’un advenir pour les deux au cœur d’une expérience.

\vspace{0.24cm}
{\footnotesize « Créer quelque chose qui puisse servir de modèle sans que ce quelque chose ne soit lui-même déjà conforme à des règles : voilà ce qu’est l’art. Il y est manifestement impossible de séparer la détermination de l’art, comme création du génie, de la congénialité du récepteur de cet art. Toutes deux reposent sur un libre jeu. »}
\vspace{0.31cm}

Dès lors, le geste théologique propre à une reprise spéculative de l’expérience artistique implique un rapport sui generis entre le subjectif et l’objectif, entre l’artiste et le récepteur de l’œuvre, entre le symbolique qui crée une co-appartenance et le conceptuel qui permet l’écart du jugement, la distanciation intellectuelle. Car le geste théologique doit être une reprise conceptuelle des intentions signifiantes du geste artistique qui est déjà une herméneutique en acte. Dans ce sens, on peut dire paradoxalement que s’il y a bien discontinuité entre les deux herméneutiques, artistique et théologique, il n’y a pas rupture dans la trajectoire du sens. Deux gestes herméneutiques au service d’une réserve de sens qui n’est pas complètement manifeste et qui est toujours ouverte à plus de compréhension. L’herméneutique théologique, dans une prise de distance réflexive salutaire, ne cessera pas de se référer à ce surplus de sens et offrira à l’une ou à l’autre intention signifiante annoncée symboliquement dans la création et dans l’observation artistique, un espace conceptuel où elle pourra se déployer. L’interprétation théologique de l’expérience artistique orientera toujours le sens dans l’horizon de la foi et de cette manière sera plus déterminée qu’une interprétation purement philosophique ou conceptuelle de cette expérience, car tout en étant redevable d’un milieu spéculatif, elle a une visée de foi.

Si l’expérience artistique objet de la reprise théologique est étrangère à la foi, il est alors important de se demander si elle est légitime… Au nom de quel principe je peux concevoir une réflexion théologique à partir d’une expression artistique non croyante… Quelle serait sa condition de possibilité ?

\subsection{L’art c’est du langage symbolique et sa condition de possibilité se trouve dans la condition symbolique de la réalité, le Symbolique transcendantal}

Afin de justifier la reprise théologique d’une œuvre d’art dont l’expression est étrangère à la foi, nous proposons une double présupposition. D’une part, nous prenons la notion de langage dans son acception la plus large. Nous parlons de langage symbolique autant pour nous référer au langage verbal de la littérature, de la poésie, de la conversation et communication, comme au langage des arts plastiques en général. Car l’art est un langage au sens large. Certes, nous connaissons la distinction faite par Hegel entre langage et art dans son Esthétique : le langage se sert des signes arbitraires, extérieurs à l’idée, alors que l’art donne aux idées une existence sensible. Mais dans cette conception, le langage est considéré comme simple moyen de communication et l’idée, la pensée, précède le langage ; alors que le tournant linguistique du milieu du {\footnotesize XX}$^\text{e}$ siècle montre que pensée et langage sont contemporains et que le langage n’est pas seulement un moyen-par, mais qu’il est également un milieu-en, une médiation.

D’autre part, le langage symbolique religieux, littéraire, artistique a comme condition de possibilité la condition symbolique de la réalité que nous sommes et en face de laquelle nous nous situons. Cette condition de possibilité nous l’appelons le « Symbolique transcendantal », car si nous parlons en métaphores, si nous créons des symboles, c’est parce que la réalité est symbolisable, métaphorisable.

Afin de donner à entendre cette conceptualisation, nous ferons référence à la pensée philosophique de Stanislas Breton qui parle de « la fonction méta » de la métaphysique devant la réalité. Il exprime la condition symbolique de la réalité avec le concept gracile et floral d’« ordre métaphoral ». Sa pensée rigoureuse frise la poésie, joue avec elle, nous offrant par là une consistance et une épaisseur ontologique inouïe. L’emploi de la métaphore qui transporte le sens avec les sens lui était familier. Il appelle « métaphoral » l’ordre du monde qui est en demande du métaphorique. Il s’agit de l’être pré-métaphorique des choses se comprenant dans une ronde, une danse de correspondances multiples qui manifeste et exige la parole. Cette parole, réfléchissant la danse du métaphoral premier, ne peut être que métaphorique à l’origine, puisqu’elle livre « les essences des choses », l’extrait volatil du parfum de l’être des choses. C’est une manière de rendre compte de la métaphoricité fondamentale du langage.

Dans ce sens, Breton cite un texte de Maître Eckhart extrêmement suggestif :

\vspace{0.24cm}
{\footnotesize « L’irradiation (du Principe) qui engendre (ses) images dans un miroir (n’est autre) que l’avènement d’une parole visible. Quant à la réflexion du miroir lui-même, elle est sa propre réponse ou locution (qui répercute le don originel). Cette locution, du reste, est aussi bien une collocution, par laquelle les essences des choses se parlent mutuellement, s’embrassent et s’unissent en l’intimité de leurs principes, pour raconter la gloire de Dieu. »}
\vspace{0.31cm}

Selon les anciens, la sensibilité comme telle déborde les limites d’un traité sur la « sensation », car les puissances de l’âme sensitive aristotélicienne tiennent leur actualisation d’une part de l’inconscient vital qu’elles servent et d’autre part du connaissant intelligible qui les utilise. Nous avons là schématiquement la double polarité qui définit la condition humaine comme un horizon où la symbolique de la terre se croise avec celle du ciel. Ainsi, les fonctions sensorielles humaines permettent d’aller plus loin que la perception animale, mais, inversement, cette direction humaine ne s’abstrait pas du terreau originel biologique et affectif dans une salutaire tension. La sensibilité humaine est conditionnée par la nature, mais aussi par la résonance des sens échangée entre toutes ces perceptions colorées, sonores, tactiles, gustatives qui retentissent spirituellement en nous.

Stanislas Breton, avec son concept d’« ordre métaphoral », nous permet d’inscrire plus clairement la nécessité métaphysique du langage symbolique comme réponse humaine pour exprimer les relations mutuelles que soutiennent les divers ordres de la réalité sensible, humain, animal, végétal.

Néanmoins, en montrant une telle danse de correspondances dans la réalité, ne court-on pas le risque de tout mélanger, de tout confondre ? L’éternel penchant totalisant prendrait-il le dessus sous la conception de cet « ordre métaphoral » ?

Nous évoquons cet « ordre métaphoral » à cause de ce pressentiment qui nous habite et nous hante, celui que Stanislas Breton désigne comme une « parenté et peut-être plus entre l’ordre des choses, l’ordre des idées et l’ordre du langage », le pressentiment d’un Outremont dont les frontières sont des passages. Ce pressentiment est présent de manière éminente chez l’artiste. Mais comment trouver les mots pour dire ce pressentiment, les mots qui nous permettent d’échapper à toute pensée totalisante, globalisante, mais également à toute pensée réductrice qui nierait la communication des ordres ? Il se peut que ce dire soit le dire métaphorique, symbolique, avec sa vérité en tension, c’est et ce n’est pas ceci ou cela ; il se peut que ce dire ne soit pas seulement verbal, il se peut que le dire de l’œuvre d’art montre lui aussi ce Symbolique-là qui le fonde.

L’étymologie du suvmbolon, du verbe sum-ballw, indique l’idée de refaire une unité et une harmonie présupposées. Si, en effet, le sens primitif de suvmbolon désignait un objet coupé en deux dont deux personnes conservaient chacune une moitié, c’est parce que chaque fragment renvoyait symboliquement à l’unité et à la brisure, à une présence et à une absence. C’est donc à partir de l’expérience primaire dont l’étymologie garde la trace que nous envisageons le Symbolique comme cette structure de l’esprit humain qui le rendant capable d’entrer en relation avec le monde, exprime sa constitutive ouverture, mais aussi ce Symbolique est comme la trame de relations constitutives du monde. Le Symbolique est en tant qu’accord à, accord de, et dans ce sens, on pourrait parler de valeur transcendantale. Alors que le symbole concret est la cristallisation de cet accord réalisé par l’homme, le Symbolique est sa condition de possibilité.

\vspace{0.24cm}
{\footnotesize « Qu’est-ce à dire, sinon que la physis des anciens physiologues est le langage symbolique ; elle n’est pas encore conçue comme un poème ; elle est vécue comme poème ; l’homme est donc le lieu où vient se dire la physis ; il n’est pas celui qui impose des sens arbitraires à une nature vide de sens ; bref, en son premier matin, le symbole est une manifestation du sens de l’être. »}
\vspace{0.31cm}

Partant, si un symbole tel, particulier donc, ne fonctionne que dans un ensemble culturel, rituel, contextuel, cela ne veut pas dire que son fondement soit seulement particulier. Un symbole particulier renvoie bien sûr à un monde particulier qu’il cristallise, mais sa vertu symbolique, son pouvoir symbolisant, ne lui vient pas seulement de ce monde-là, mais d’un principe d’interdépendance universelle, ontologiquement constitutif de la réalité. Nous parlerons alors d’un principe symbolique et d’une prétention à l’universel de la logique par lui générée. Cela fait penser à l’universalité de l’œuvre d’art...

\subsection{Le langage symbolique et le langage conceptuel}

Si une théologie de l’expérience artistique est possible, il est important de se pencher sur le fonctionnement du rapport entre le langage symbolique et le langage conceptuel. Car comme nous dit Paul Ricœur :

\vspace{0.24cm}
{\footnotesize « Si aucun concept n’épuise l’exigence de « penser plus » portée par le symbole, cela signifie seulement qu’aucune catégorisation donnée ne rend compte des potentialités sémantiques tenues en suspens dans le symbole ; mais c’est le travail du concept qui seul peut témoigner de cet excès de sens. »}
\vspace{0.31cm}

« Le travail du concept » nous occupe et nous préoccupe en théologie de l’expérience artistique parce que non seulement c’est la seule manière de montrer l’excès de sens porté par le symbole artistique, mais encore son excédent d’être. Certes, le symbolique est ancré dans le réel, c’est du réel présenté symboliquement et sans présence réelle, il n’y a pas de présence symbolique. Cependant si le saut conceptuel qui est fait par le discours théologique prend son élan dans le socle symbolique artistique, il évolue dans un autre espace de sens moyennant une translation. Il s’agit d’un discours second qu’il faut produire dans un registre autre.

L’expérience vive à laquelle est rattaché le symbole est offerte à une première interprétation, dans et par le symbole même, puisqu’il appartient à la manière d’être de l’homme d’avoir conscience du réel symboliquement, comme nous l’avons montré. Mais dans cette interprétation première et fondatrice, l’interprété symbolisé et l’interprétant symbolisant se recouvrent et cela rend le symbolique difficile à cerner oscillant entre logos et bios.

C’est alors que le travail de conceptualisation sauve les « intentions signifiantes originaires » du symbole au détriment bien sûr de l’ouverture essentielle donnée par celui-ci dont la signifiance est dynamique, polysémique et toujours capable de nouveaux espaces de compréhension. Le travail du concept est de stabiliser la signification le temps de la conceptualisation, d’enfermer dans des contours précis, de faciliter la saisie de l’esprit. Si le symbole délie le sens en arômes multicolores, en essences métaphorales, le concept lie et relie les sens dans un bouquet ordonné et unifié.

Le philosophe Jean Ladrière a très bien analysé la proximité et la différence entre symbole et concept. En étudiant le discours théologique qui se prête mieux que tout autre à cette analyse puisque ce discours « essaie de faire apparaître, dans sa discursivité même, tout le contenu d’intelligibilité du langage religieux (un langage symbolique) dont il est la systématisation réfléchissante », Ladrière montre une « correspondance formelle » entre symbole et concept. En quoi consiste cette correspondance, alors qu’apparemment bien au contraire, dans le règne de la signification, tout semble les opposer ? Ladrière trouve une ressemblance dans leur manière d’opérer. Tous deux opèrent une translation, ils font littéralement hJ metaforav, un transport de sens. Qu’est-ce à dire ?

Le symbole assimile deux significations appartenant à des champs sémantiques spécifiques. La signification première est comparée par Ladrière à « une rampe de lancement à partir de laquelle la signification seconde peut prendre son vol » ; elle indique et introduit dans la signification seconde par son propre sens, moyennant une transvaluation. La signification seconde donne un horizon de sens qui préexiste au symbole en tant que possibilité non encore assimilée. Car le symbole assimile les deux et établit un nouveau support signifiant dont l’essence est une liaison. Le symbole est un ajointement du symbolisant et du symbolisé.

C’est dans la perception du réel que s’enracine le concept, dans un langage déjà existant le disant. Il y a d’abord des significations originaires toutes proches du vécu, sentant encore la fraîcheur de la réalité, sollicitant la conceptualisation. Celle-ci consiste à dégager ces significations de leur gangue originaire moyennant un processus qui est une transformation. Ce processus ne peut se faire que parce que, comme dans le cas du symbole, il y a un horizon préexistant, constitué et constituant, celui du logos spéculatif. Le concept quant à lui est le seul support de l’actualisation de cet horizon, mais ne vient à jour qu’en lui et par lui. Ainsi la sortie de la gangue perceptive est contemporaine de l’arrivée à la lumière conceptuelle.

Symbole et concept opèrent un transport de sens afin de rendre compte du réel dans un équivalent effort de mise à distance vis-à-vis de celui-ci, mais alors que le symbole, suspendu au réel concret, ne s’en détache pas tout en distinguant le symbolisant du symbolisé puisqu’il en tient lieu, le concept, lui, se dégage du réel concret avec aisance.

30
Cette analogie entre la symbolisation et la conceptualisation a ses limites, car « si le spéculatif s’origine dans le métaphorique, il a néanmoins sa nécessité en lui-même ». Le symbole et le concept sont différents. Le symbole retient indéfectiblement les deux significations articulées, alors que le concept se défait du langage perceptif premier et fonctionne à l’intérieur de sa nouvelle sphère en système avec les autres concepts, en parfaite résonance. La signification esquissée symboliquement requiert un “saut” vers un autre espace de discours qui a son propre fonctionnement et ses propres termes logiques : « Signifier est toujours autre chose que représenter […]. L’imaginatio est un niveau et un régime de discours. L’intellectio est un autre niveau et un autre régime. »

Dans l’opération de la conceptualisation, il y a un risque sévère, celui de laisser de côté l’expérience originaire et de fonctionner dans une aire idéale, manquant ainsi à sa vocation de concept, c’est-à-dire de garder une référence au réel pour mieux cerner la pluralité d’éléments dans l’unité d’un sens :

\vspace{0.24cm}
{\footnotesize « Ce que l’on gagne en intelligibilité discursive, on le perd en substance compréhensive et l’opération, si elle a quelque valeur, ne peut garder sa légitimité qu’au prix d’un perpétuel recommencement, qui ramène l’effort spéculatif vers ses sources, c’est-à-dire vers le jaillissement de sens du langage originaire. »}
\vspace{0.31cm}

Ainsi, si « consubstantiel » est plus précis que « lumière née de la lumière », cette dernière expression symbolique dit plus que le concept lui-même. Il y a entre symbole et concept une sorte de jeu car l’un n’abolit pas l’autre :

\vspace{0.24cm}
{\footnotesize « Mais cette discontinuité des modalités sémantiques implique-t-elle que l’ordre conceptuel abolisse ou détruise l’ordre métaphorique ? Pour ma part j’incline à voir l’ordre du discours comme un univers dynamisé par un jeu d’attractions et de répulsions qui ne cessent de mettre en position d’interaction et d’intersection des mouvances dont les foyers organisateurs sont décentrés les uns par rapport aux autres, sans que jamais ce jeu trouve son repos dans un savoir absolu qui en résorberait les tensions. »}
\vspace{0.31cm}

L’enjeu de ce processus d’interprétation sera un jeu incessant entre symbole-concept-symbole-concept… Le discours théologique est un jeu incessant, un va-et-vient où l’un (la conceptualisation) se reçoit de l’autre (la symbolisation artistique, liturgique etc.) incessamment ; ou l’un (la symbolisation) est en demande ou en attente de l’autre (la conceptualisation) incessamment, comme intelligence de la foi. Une féconde distance entre les deux est à garder, une distanciation qui n’oublie pas le lien vital et la proximité salutaire des deux.

\subsection{Ouverture conclusive : les puissances du symbolique artistique pour la théologie}

L’œuvre d’art, comme toute écriture ou inscription symbolique, est déjà une distanciation interprétante de la réalité perçue par l’artiste et cette interprétation est sensible. Dès lors l’ouverture ou le passage vers la théologie est une interprétation conceptuelle d’une interprétation sensible.

L’interprétation symbolique de l’art a une valeur cognitive de l’ordre du sensible, une connaissance qui implique le sentir et qui peut même devenir jouissance esthétique, car l’art s’adresse à l’affection et à l’entendement tout ensemble. C’est cette expérience-là que nous trouvons dans l’art et c’est cette expérience qui fait l’objet de l’esthétique philosophique. Certes, parler de connaissance sensible c’est devoir s’expliquer et se justifier face à la grande tradition épistémologique qui considère cela comme un paradoxe, un oxymore… Mais justement, l’expérience artistique et l’expérience du beau sont cette paradoxale connaissance… Et, lorsqu’on y est impliqué, l’intelligence connaît quelque chose d’unique qui touche la sensibilité en même temps que l’intelligence. Il y a dans l’expérience artistique (artiste et observateur) quelque chose qui « retient », qui touche, tout l’être de soi. C’est l’expérience même qui est connaissance et cognoscible tout un et c’est de cela qu’il faut rendre compte dans une réflexion à partir de l’art. En d’autres termes, penser l’expérience artistique c’est nommer le « surplus » artistique qui consiste à dire l’unicité et l’originalité du geste créateur, connu et reconnu dans l’œuvre d’art.

\vspace{0.24cm}
{\footnotesize « L’attitude de jouissance dont l’art implique la possibilité qu’il provoque est le fondement même de l’expérience esthétique ; il est impossible d’en faire abstraction, il faut au contraire la reprendre comme objet de réflexion théorique, si nous voulons aujourd’hui défendre contre ses détracteurs – lettrés ou non lettrés – la fonction sociale de l’art et des disciplines scientifiques qui sont à son service. »}
\vspace{0.31cm}

Ces orientations philosophiques et sociologiques nous entraînent dans les possibles enjeux théologiques de l’expérience artistique. Notre rapport au réel n’est pas simple. Il est pertinent de distinguer, avec Jean-Yves Lacoste, une différence entre connaissance et savoir, même s’il peut arriver qu’il y ait parfois recoupement des deux. Nous pouvons savoir beaucoup de choses sur Dieu sans pour cela le connaître ; la fonction du savoir est théorique et non existentielle. Mais, inversement, nous pouvons connaître Dieu en sachant très peu de choses sur lui, comme par exemple savoir que nous le connaissons comme inconnu maintenant. Car connaître c’est du sentir, du vivre, et rien ne se ressent autant que l’absence de celui que l’on voudrait qu’il soit présent ! Sentir qu’on ne sent pas.

\vspace{0.24cm}
{\footnotesize « Dieu connu comme inconnu est Dieu connu comme absent […] Nous ne 
sentons une absence que sur le fond d’une présence qui ne nous a peut-être jamais 
été concédée mais dont nous savons plus ou moins (que ce soit en nous fondant sur 
des témoignages ou, plus radicalement, en postulant que le Dieu dont nous parlons 
ne peut demeurer inconnu) qu’elle serait […] Mais nous pouvons aussi espérer le 
lendemain où nous (le) sentirons et le lendemain absolu où nous (le) sentirons en 
plénitude... »}
\vspace{0.31cm}

Car nous croyons à la résurrection de la chair et cette résurrection, résurrection déjà anticipée (le déjà là et pas encore eschatologique de notre foi et espérance), se fait sentir, donc connaître, dans des rares expériences liturgiques, mystiques, artistiques, amoureuses, communautaires…

La liturgie, comme l’art et tout ce qui est symbolique, ne s’adresse pas seulement à l’entendement, mais aussi au sentiment. Il peut arriver alors que, dans l’expérience liturgique et dans l’expérience artistique, la présence de Dieu signifiée par l’action liturgique ou artistique se fasse sentir. Ces expériences sont alors d’ordre eschatologique et l’homme qui les éprouve vit une anticipation de la résurrection de la chair, un avant-goût de l’eschaton. Il est fondé en raison de le croire possible… Ainsi, en montrant, l’œuvre d’art peut donner à connaître Dieu et savoir cela ce n’est pas moins faire de la théologie. L’art lui-même n’est pas de la théologie, mais il n’est pas hors théologie.

Notes

    [1]
    Charles Baudelaire, « Correspondances », dans Les Fleurs du mal, Aubry, Paris, 1942, p. 75 et suiv.
    [2]
    « Ainsi, le symbolisme, pris dans son niveau de manifestation dans des textes, marque l’éclatement du langage vers l’autre que lui-même : ce que j’appelle son ouverture ; cet éclatement, c’est dire ; et dire, c’est montrer ; les herméneutiques rivales se déchirent non sur la structure du double-sens, mais sur le mode de son ouverture, sur la finalité du montrer » (Paul Ricœur, Le conflit des interprétations, Seuil, Paris, 1969, p. 68).
    [3]
    Cf. Olivier Abel, « Une poétique de l’action », dans L’homme capable, autour de Paul Ricœur, Paris, Rue Descartes, n°53 bis Hors série, novembre 2006.
    [4]
    « L’art comme tout langage analogique dit beaucoup plus que ce qu’il montre, mais ne le dit qu’en le montrant » (Jérôme Cottin, « Aniconisme, corporéité, fragilité : l’art et la question du sens en post-modernité », dans Transversalités Supplément 3, Institut catholique de Paris, Paris, 2015, p. 145). C’est nous qui soulignons.
    [5]
    Hans-Georg Gadamer, L’actualité du beau, Éd. Alinea, Paris, 1992, p. 31 et suiv.
    [6]
    Claude Romano définit l’expérience comme « une traversée, ce qui suppose une distance intervallaire et un franchissement de soi à soi par lequel seulement nous pouvons accueillir ce qui nous advient, en nous advenant à nous-mêmes comme autre » (L’événement du monde, PUF, Paris, 1998, p. 195).
    [7]
    H.-G. Gadamer, op. cit., p. 42.
    [8]
    Cf. Être, monde, imaginaire, Seuil, Paris, 1976 ; « Mythe et imaginaire en théologie chrétienne », dans Le mythe et le symbole. De la connaissance figurative de Dieu, Beauchesne, coll. Philosophie n° 2, Paris, 1977 ; Poétique du sensible, Cerf, Paris, 1988 ; Philosophie buissonnière, Éd. Jérôme Millon, Grenoble, 1989 ; Libres commentaires, Cerf, Paris, 1990 ; « Sur l’ordre métaphoral », dans Jean Greisch et Richard Kearney (Dir.), Paul Ricœur. Les métamorphoses de la raison herméneutique, Cerf, Paris, 1991.
    [9]
    Nous citons la traduction faite par Stanislas Breton du texte latin de Maître Eckhart dans « Sur l’ordre métaphoral », dans Paul Ricœur, Les métamorphoses de la raison herméneutique, op. cit., p. 373 : « Irradatio sive gignitio imaginis in speculo est visibilis locutio, speculi vero repercussio est ipsius responsio sive locutio… Haec autem locutio et collocutio qua… rerum quidditates sibi loquuntur, se osculantur et uniuntur suis intimis et intime, enarrant gloriam Dei » (in Genesim II, Lateinische Werke, ed. Konrad Weiss, Stuttgart, 1966, p. 617 et 620). C’est nous qui soulignons.
    [10]
    S. Breton fait allusion aux commentaires de saint Thomas sur le De sensu et Sensato et De Memoria et Reminiscentia d’Aristote, dans « Mythe et imaginaire en théologie chrétienne », Le mythe et le symbole, op. cit., p. 169-181.
    [11]
    Nous ne pouvons pas échapper à la dualité féconde de cet entre-deux, de ce va-et-vient qui nous permet quotidiennement « la joie de voir autrement le même », sans pour autant tomber dans un dualisme manichéen, car l’essential, nous dit S. Breton, n’est pas la dualité, mais la mobilité instauratrice de sens rappelée avec l’image de l’échelle de Jacob sur laquelle les anges montent et descendent (Cf. Id., Poétique du sensible, op. cit., p. 10-18).
    [12]
    Id., « Sur l’ordre métaphoral », dans P. Ricœur, Les métamorphoses de la raison herméneutique, op. cit., p. 378.
    [13]
    La vérité métaphorique du langage symbolique récapitule toutes les formes de tension sémantiques selon P. Ricœur : « tension entre sujet et prédicat, entre interprétation littérale et interprétation métaphorique, entre identité et différence » (La métaphore vive, Seuil, coll. Points Essais n° 347, Paris, 1975, p. 398).
    [14]
    Du verbe sum-ballw : jeter ensemble, d’où réunir, rapprocher, échanger, rassembler, mettre en commun, communiquer, comparer (A. tr.) ; se rencontrer avec, s’entretenir avec (A. intr.) ; mêler (M. tr.).
    [15]
    Jacques-Raoul Marello, « Symbole et réalité. Réflexion sur une distinction ambiguë », dans Le mythe et le symbole. De la connaissance figurative de Dieu, Beauchesne Paris, 1977, p. 156 et suiv.
    [16]
    Cf. Dominique Dubarle, « Pratique du symbole et connaissance de Dieu », dans Ibid., p. 214 et suiv.
    [17]
    P. Ricœur, « Parole et symbole », dans RSR 49/1-2, 1975, Le symbole, p. 151.
    [18]
    Id., La métaphore vive, Seuil, coll. L’ordre philosophique, Paris, 1975, p. 374-375.
    [19]
    L’articulation du sens. II Les langages de la foi, Cerf, Paris, 1984, p. 169 et 173.
    [20]
    Cf. Ibid., p. 173 et suiv.
    [21]
    Cf. Ibid., p. 174 et suiv.
    [22]
    Jean-Louis Souletie (Dir.), Nommer Dieu. L’analogie revisitée, Lessius, coll. Donner raison, Namur, 2016, p. 212.
    [23]
    P. Ricœur, La métaphore vive, op. cit., p. 381 et suiv.
    [24]
    L’articulation du sens. II Les langages de la foi, op. cit., p. 175 et suiv.
    [25]
    Id., La métaphore vive, op. cit., p. 382.
    [26]
    Cf. Ibid., p. 323-399.
    [27]
    De ai[sqhsiı, faculté de percevoir par les sens, dont le fruit est la sensation. Mais aussi, chez Platon, faculté de percevoir par l’intelligence, avoir la sensation ou la perception d’un comprendre. Au sens actif, faire voir ou comprendre à quelqu’un quelque chose en lui faisant « tomber sur le sens ». (Anatole Bailly, Abrégé du dictionnaire grec français, Hachette, Paris, 1901). Comprendre dans cette perspective est plus que connaître à la manière kantienne ; il s’agit de « sentir » avec l’intelligence, plus près du connaître biblique. Déjà, Aristote, dans la Métaphysique, L. Q, ch. 10, 1051 b 22-25, disait « toucher et dire de façon simple, c’est cela le vrai ; et ne pas avoir de contact, c’est ignorer tout court » (c’est nous qui soulignons).
    [28]
    Cf. H.-G. Gadamer, L’actualité du beau, op. cit., p. 35-43.
    [29]
    Hans-Robert Jauss, « Petite apologie de l’expérience esthétique », dans Pour une esthétique de la réception, Gallimard, Paris, 1978, p. 137 (entièrement souligné par l’auteur).
    [30]
    Jean-Yves Lacoste, « “Resurrectio carnis”. Du savoir théologique à la connaissance liturgique », dans La phénoménalité de Dieu, Cerf, Paris, 2008, p. 205–227. Son souci est liturgique, mais cette argumentation peut aisément s’appliquer à l’expérience artistique.
    [31]
    Ibid., p. 209, 219.

    Mis en ligne sur Cairn.info le 08/12/2019
    https://doi.org/10.2143/LV.00.0.0000000 
