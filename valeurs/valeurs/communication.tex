
%%%%%%%%%%%%%%%%%%%%%
\section{Communication}
%%%%%%%%%%%%%%%%%%%%%
%commun entre lui et moi? Celui qui n’a jamais réfléchi ne peut être
%ni clément, ni juste, ni pitoyable. »
% {\footnotesize X}$^\text{e}$ « » “” ° {\it } {\bf } \textsc{} \textbf{\textit {}}
% 88
\subsection{La communication médiate}
Outre la communication
directe, 1l existe d’ailleurs une communication médiate, indirecte,
{\it par l'intermédiaire des signes et du langage}. Le « langage » émotionnel
dont il a été question au \S 84, ne constitue d’abord, chez l’enfant,
qu’un ensemble de réflexes. Mais ces réflexes, par exemple le cri,
deviennent des {\bf signes} lorsqu'ils sont employés {\it intentionnellement}
comme moyens de communication avec autrui (t. I, \S 196). C’est
donc {\bf la relation avec autrui} qui, des réactions du «langage » émotionnel,
fait des signes proprement dits. Chez l’adulte, les réactions
qui accompagnent les émotions peuvent être employées volontairement
et deviennent alors la {\bf mimique} : un homme victime d’une injustice
peut mimer la colère, alors qu’il conserve son sang-froid, et eeci en
vue de signifier à autrui son mdignation. Il y a eu d’aillewrs, avant le
langage parlé, un {\bf langage par gestes} qui existe encore chez certarnes
peuplades et qui, au hieu d’être comme ehez nous un simple auxiliaire
de la parole, constituait le moyen d'expression principal et se
suffisant à lui-même. Il reste quelque chose de ce « langage d’action »,
non seulement dans nos gestes, mais dams ce que \textsc{G. Dumas} a appelé
la {\bf mimique vocale} (t. I, \S 192). Quant au {\bf langage parlé}, il constitue
évidemment le moyen de communication le plus courant. Mais
on a souvent signalé son perfection dès qu’il s’agit de traduire ce
qu’il y a d’original, de proprement individuel dans nos sentiments.
Que de malentendus qui rendent les âmes opaques les unes aux autres
et qui résultent souvent d’un mot mal compris, d’une expression à
laquelte les deux interlocuteurs ne donnent pas le même sens !

% 89
\subsection{La communication sur le plan des valeurs}
Il existe
d’ailleurs d’autres modes d'expression qui permettent aux consciences
e communiquer et même de communier sur un plan supérieur, celui
des valeurs. Tel est, par exemple, {\it l'Art}, qui institue parfois une union
qu’on peut appeler, avec \textsc{R. Le Senne}, « le concert des esprits » :
« Du chant populaire à l’art de la {\footnotesize IX}$^\text{e}$ Symphonie, du {\it Messie} ou des
{\it Béatitudes}, le chœur qui, comme dans la tragédie antique, permet
toutes les combinaisons entre la liberté des solistes et la sympathie
du peuple, manifeste, en les ennoblissant, les mouvements de la douleur
et de la joie dans une communauté fraternelle. » L'Art réalise
done bien une {\it communion} puisque, sans effacer les originalités individuelles,
il ramène « vers l’universalité émotionnelle de la condition
humaine ». La Religion établit entre les âmes une communion plus
% 166
intime encore puisqu'elle englobe tous les aspects de la spiritualité :
accord des intelligences dans la foi commune, fusion des cœurs dans
l'amour, collaboration des volontés dans les actes. Le corps lui-même
y est intéressé grâce aux rites et aux cérémonies. célébrées en commun.
— Enfin on peut observer qu’il existe une communion des esprits, non
pas seulement sur le plan instinctif ou affectif, mais sur le plan
{\bf rationnel}. Les modes de communication des consciences tels qu’ils
sont décrits par Scheler, c’est-à-dire sous la forme surtout affective,
«préparent sans doute l’assimilation des esprits et des volontés ;
ils ne peuvent l’achever et la consolider s'il ne s’y joint la réflexion,
l'entente délibérée, le consentement explicite ». Autrement, ils nous
laissent « encore en partie étrangers et obscurs les uns aux autres ».
En ce sens, la vraie communion des esprits « ne s’achève que par la
rationalité » (\textsc{G. Belot}). \textsc{A. Lalande} lui aussi remarque que « la
raison est {\it communauté} », caractère d’ailleurs lié à son caractère
normatif (t. I, \S 217) : « elle s'exprime par une législation spontanée,
théorique et pratique, lien des hommes en tant que “ semblables ”,
non en tant qu’échangistes, exploitants et exploités, en cellules différenciées
d’un organisme collectif ». C’est en ce sens aussi que Malebranche,
dans son {\it Traité de Morale}, distingue deux sortes de sociétés
entre les hommes : une société « de commerce », toute temporelle,
qui ne subsiste que « dans une communion de biens particuliers et
périssables », et une société « de religion », qui est une société des
esprits, « réglée par la Raison » et qui unit les hommes dans la communion
des biens éternels.

%90. 
\subsection{L’amitié et l’amour}
Il y a ainsi toute une hiérarchie des
formes de la communion entre les consciences depuis la sympathie
qui peut être parfois toute passagère et superficielle, jusqu’à l’amour.
Dans l'intervalle, se trouve l’{\bf amitié} qui est durable et qui réalise cette
{\it fusion des consciences} qu’a illustrée le mot célèbre de Montaigne à
propos de son amitié avec La Boëtie : « Si l’on me presse de dire pourquoi
je l’aimais, je sens que cela ne se peut exprimer qu’en répondant :
“ Parce que c’était lui, parce que c'était moi. ” » L'amitié a été célébrée
par la plupart des moralistes anciens (Aristote, Épicure, Cicéron).
Elle est en effet, comme a dit \textsc{Lacordaire}, « le don complet de soi-même » : elle est confiance, {\it ouverture} totale d’une conscience à une
conscience et, en ce sens, elle est déjà en germe dans la perception
même d’autrui : « Si elle est, la perception du {\it toi} ne peut qu'être
bilatérale ; car, afin que je perçoive une personnalité, il faut que
cette personnalité s’ouvre à moi, il faut au moins qu’elle ne se refuse
pas et que je puisse entrer dans son jaillissement intérieur »
(\textsc{Nédoncelle}. — Quant à l’{\bf amour}, nous avons déjà remarqué (\S 85)
avec Scheler qu’il se caractérise par sa fonction {\it valorisante}. Sans
doute, il y a lieu de distinguer, dans l’usage courant du mot {\it amour},
bien des sens différents. Mais ces divers sens peuvent, nous semble-t-il,
se ramener à trois fondamentaux :

1° ou bien l’amour s’adresse directement à {\it la valeur elle-même} :
c’est en ce sens qu’on parle de l’amour de Dieu, de l’amour de la
Patrie, etc. La {\it charité}, au sens chrétien du terme, est cet amour
sublimé sur le plan de la transcendance (\S 175-176) ;

2° ou bien l’amour s’adresse, en tant que sentiment vivant et
agissant, à d’{\it autres individus} : tel est l'amour paternel, maternel,
fraternel, conjugal, et même l’amour-amitié. Il a alors pour objet la
valeur de la personne et il tend toujours à «élever la personne aimée
au plus haut degré de valeur positive possible », à considérer ce qu’il
y a de meilleur en elle (\S 122); à

3° ou bien, enfin, il s’agit de l’amour, au sens le plus ordinaire du
terme, de l'{\it amour sexuel}. Même en ce sens, nous avons remarqué
(\S 19 {\it A}) que l’amour ne se confond pas avec l'instinct, et d’ailleurs
certains philosophes contemporains, même très spiritualistes, comme
Scheler lui-même (cf. aussi Jean \textsc{Guitton}, {\it Essai sur l'amour humain}),
ont admis que, jusque dans les illusions de l’amour physique, il y a
« l'expérience d’une réalité supérieure » et que l’âme peut « sublimer
ce qui vient des sens ». Quoi qu’il en soit, il est certain que, dans tout
amour pour une personne donnée empiriquement, «nous nous formons
une image idéale de sa valeur » (Scheler).

\subsection{La solitude}
A l’opposé de ces sentiments de communion,
il semble qu’il y ait lieu de placer la solitude. Et, de fait, il y a une
solitude qui est {\bf échec de la communion} (\textsc{Gusdorf}) entre les consciences.
La haine, la jalousie, l'envie et aussi le désespoir (\S 37) isolent
celui qui en est animé. « L'enfer », ce n’est pas « les autres », comme le
dit \textsc{Sartre} dans {\it Huis-clos} ; c’est au contraire le {\it moi} replié dans son
orgueil et son ressentiment et refusant les valeurs de communauté. On
a vu, en effet, que la personne est faite, en grande partie, de ces
valeurs qu’elle puise dans la vie en société (\S 76) et que, loin de diminuer
le {\it moi}, l'expérience d’autrui, en tant que préalable à la prise de
conscience de celui-ci par lui-même (\S 75), l’étend et l’enrichit. La vie

1. C'est l'instinct qui aboutit à ce que D. Lacacxe, dans son étude sur la {\it Jalousie
amoureuse}, appelle «l'amour captatif», celui qui prétend {\it posséder} autrui comme un
{\it objet}. L'amour captatif s'oppose à « l'amour oblatif », où l'amant s'offre à l'être aimé
pour être en quelque sorte possédé par lui et surtout à « l’amour-communion », celui
dont il est question ci-dessus et où les deux partenaires se considèrent comme parties
d’un tout qui les dépasse l’un et l’autre.
% 168
sociale n’est pas une mutilation ni un asservissement du moi et, comme
le dit \textsc{Spinoza}, « l’homme qui est conduit par la raison, est plus libre
dans la Cité où il vit selon la loi commune, que dans la solitude où il
n’obéit qu’à lui-même ».

Mais la solitude peut être aussi tout autre chose. C’est un fait que
toutes les âmes soucieuses de spiritualité ont toujours recherché la
solitude. Ce n’est point parce que, comme le dit, non sans quelque
amertume, l’auteur de l’{\it imitation} rappelant un passage de Sénèque,
« chaque fois que l’on va dans la compagnie des hommes, on en revient
moins homme qu’on n’était ». C’est parce qu’au delà des valeurs de
communauté, toujours plus ou moins déterminées par les contingences,
il y a les {\it valeurs universelles} (cf. \S 128 {\it fin}), il y a la {\it valeur en
soi}. Être seul avec soi-même, c’est se retrouver en présence de ces
valeurs fondamentales, en présence de « l’unique nécessaire ». Une
telle solitude avec soi-même est aux antipodes de la solitude desséchée
qui s’est coupée de toute communion spirituelle. En ce sens, il y a
bien, comme le dit L. \textsc{Lavelle}, une « séparation qui unit ». Car ce
serait une erreur de croire que la communion doive détruire en nous
ce qui fait « l’inimitable singularité de toute existence individuelle »,
La solitude, accompagnée de la méditation silencieuse, nous permet
de nous retremper dans cette singularité « qui exprime la part d’absolu
dont chaque être est, pour ainsi dire, porteur », et ainsi, « c’est l’être
le plus personnel et le plus solitaire qui est capable d'accomplir l'acte
de communion le plus désintéressé et le plus pur ». D’autre part, si la
solitude peut être parfois « une tentation », le sage, lui, « ne cherche
en elle qu’une sorte d’exercice spirituel qui doit prouver sa valeur et
sa fécondité dans ces relations avec le dehors qu’elle avait paru abolir.
Alors seulement nous apprenons à vivre comme nous imaginons qu’il
fallait vivre quand nous étions seul. Si dans la solitude nous formons
l’idée d’une société parfaite avec nous-même, avec l’univers et avec
tous les êtres, c’est le retour dans le monde qui, par une sorte de paradoxe,
en interrompant cette solitude, la réalise et l’oblige à porter
son fruit » (\textsc{Lavelle}). En ce sens, {\bf solitude et communauté sont
complémentaires l’une de l’autre}.

\subsection{De l’Interpsychologie à la Psychologie sociale}
%92
Jusqu’ici
nous avons étudié les rapports avec autrui indépendamment
des groupes à l’intérieur desquels ils s’établissent, Contrairement en
effet à ce qu’affirment certaines doctrines contemporaines, l’homme
n’est pas {\it tout} entier social (t. I, \S 35). Non seulement il peut se libérer
de l’emprise du social {\it par en haut}, par la conquête de l'autonomie
personnelle (\S 78), mais il échappe aussi au social par {\it en-bas}, par ce [...]
%169
%%%%%%%%%%%%%%%%%%%%%%%%%%%%%%%%%%%%%%%%%%%%%%%%%%%%%%%%%%%%%%%%%%%%%%%%
