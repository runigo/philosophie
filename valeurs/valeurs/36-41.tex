
%%%%%%%%%%%%%%%%%%%%%
\section{36 41}
%%%%%%%%%%%%%%%%%%%%%
%
% 30 {\footnotesize X}$^\text{e}$ « » “” ° {\it } {\bf } \textsc{} \textbf{\textit {}}
A-t-on le droit d’assimiler ainsi les états affectifs d’{\it origine physique} et
les états affectifs d’{\it origine morale} ? Les objections se présentent d'elles-mêmes :
1° dans un cas, la cause directe est d'ordre physiologique ;
dans l’autre, les éléments physiologiques, s’ils existent, passent.de
loin au second plan derrière les éléments d'ordre intellectuel, spirituel,
social ; 
2° il y a une telle différence de {\it complexité} entre les deux sortes
d'états qu’ils ne sont même pas exactement comparables : dans un
cas, on a affaire à une simple sensation ; dans le second, à un sentiment
extrêmement riche et nuancé. Mais, ainsi envisagé, le problème
est mal posé. Il ne s’agit nullement de mettre plaisirs et douleurs
d’ordre physique et plaisirs et douleurs d’ordre moral sur le même
plan, ni surtout de leur attribuer même valeur\footnote{
Ce qui donne une apparence paradoxale au jugement de Ribot, c'est que nous
avons tendance à nous placer au point de vue moral plutôt qu'au Pr de vue psychologique :
« On nous a si bien appris, dit \textsc{Titchener}, à distinguer les plaisirs élevés des
basses jouissances, que le seu] fait de nier qu'il existe dans le domaine du plaisir et du
déplaisir différentes espèces, apparaît une opini dal, »
}. Il s’agit d'examiner
si, {\it en tant qu'agréables ou désagréables} et abstraction faite des complications,
déterminations ct structurations que peuvent y ajouter le
plaisir et la douleur en tant que phénomènes spécifiques, les émotions,
les sentiments, etc., ces divers états n’ont pas, quelle que soit leur
cause, une tonalité commune. Or, comme le dit G. \textsc{Dumas}, « ce qui
nous donne l'illusion d’une différence de nature, c’est que, dans les
deux cas, nous considérons les causes du désagréable au lieu de
considérer {\it le désagréable lui-même} ».

\subsection{Les réactions affectives}% 26. 
De fait, force nous est bien de
constater que les réactions de l’agréable d’une part, celles du désagréable
d’autre part se présentent toujours sous la même forme, que
leur cause soit physique ou morale.

{\it A}. D’une façon générale, on
peut dire que les états {\it agréables} s’accompagnent d’une \textbf{\textit {suractivité}}
générale des fonctions physiologiques : la circulation et la respiration
sont accélérées ; les échanges organiques sont activés ; le temps de
réaction est diminué. Les états {\it désagréables} présentent le tableau
exactement contraire d’une \textbf{\textit {dépression}} : ralentissement de la circulation
et de tous les échanges organiques ; lenteur des réactions, etc.

{\it B}. Quant aux réactions {\it externes}, il semble bien qu’on puisse maintenir
l'indication déjà donnée par \textsc{Cabanis} : « Dans la douleur [le désagréable],
l’animal se retire tout entier sur lui-même, comme pour
présenter le moins de surface possible », tandis que, « dans le plaisir
[l’agréable], tous ses organes semblent aller au-devant des impressions;
ils s’épanouissent pour les recevoir par plusieurs-points ». H. \textsc{Piéron}
% 36{\footnotesize X}$^\text{e}$ « » “” ° {\it } {\bf } \textsc{} \textbf{\textit {}}
signale de même les réactions d'\textbf{\textit {expansion}} et de
\textbf{\textit {recherche}} comme
caractéristiques de l’{\it agréable}, celles de \textbf{\textit {retrait}} ct de
\textbf{\textit {fuite}} comme caractéristiques
du {\it désagréable}. Chez l’homme, les réactions de la face,
les jeux de {\it physionomie} s'associent à cette attitude générale : nous
disons que le visage « s’épanouit » dans la joie ou le plaisir (en tant
qu’agréables), qu'il « se contracte », « se crispe » ou bien qu’il « s’allonge »
dans les états désagréables.

{\it C}. On pourrait se poser la question de savoir si ces réactions sont
les {\it effets} des états affectifs ou si au contraire ces derniers ne sont pas
la manifestation subjective d’un comportement global constitué par
l’ensemble de ces réactions. Si l’on se réfère à la conception {\it unitaire}
de l’être humain qui a été présentée au tome I, 30 (et qui n’est pas
incompatible avec un dualisme métaphysique : {\it cf}. tome I, \S 351 {\it F}),
cette question paraîtra bien vaine. Il y a en nous {\it solidarité intime
entre le physique et le moral}, avec prédominance tantôt de l’un, tantôt
de l’autre. S'il est des états affectifs qui ne sont guère que le réflet
d’un état physique(\S 33), il n’y a pas de raison pour rejeter l’influence
réciproque de l’affectivité sur notre comportement, Dès 1902, le
Dr J.-P. \textsc{Morat}, s’élevant contre une conception trop étroite du
déterminisme biologique inspirée du déterminisme physique, affirmait :
« {\it Dans l'être vivant, le mouvement dépend de la sensibilité,
comme la sensibilité dépend du mouvement.} » Même un sentiment tout
contemplatif comme l'admiration se traduit par des actes, et
\textsc{Paulhan} rappelle l’histoire de Soulouque, empereur d'Haïti, qui,
au récit de la clémence d’Auguste envers Cinna, fit grâce, tout
vindicatif qu’il était, à l’un de ses ennemis condamné à mort.

\subsection{Théorie de l’affectivité}% 27
Ces réactions affectives s'expliquent
en fonction d’une théorie générale de l’affectivité.

{\it A}. Conditions générales. — Nous avons remarqué qu’en principe
l’état agréable correspond à la tendance satisfaite, l’état désagréable à
la tendance contrariée. Il est bien vrai en effet que les états affectifs
sont étroitement liés à l’activité sous. sa forme spontanée. Les personnes
dites apathiques, auxquelles tout est indifférent, sont aussi
celles qui sont, comme on dit, « sans ressort», c’est-à-dire chez qui
les tendances, les impulsions naturelles sommcillent. — Toutefois,
sous cette forme, la théorie demeure vague et même, comme on va
le voir, partiellement inexacte.

\textsc{Aristote} avait dit que « le plaisir [disons : l’état agréable] parachève l’acte ».
Affirmation discutable \footnote{Surtout si l’on se souvient que par « acte » Aristote entend, non l'activité en général
mais la {\it pleine réalisation de l'être} par opposition à l'être virtuel ou « en puissance ".} ;
car, ainsi que l’observe
%37
\textsc{Paulhan}, «la tendance, le besoin, l'instinct qui aboutissent à peu
près parfaitement à l’acte, deviennent inconscients et sont à peine
aperçus par nous ». Tout au plus se produit-il en pareil cas un état
d’{\it euphorie}, à peine senti et qui est plutôt {\it état neutre} qu’état agréable.
« Légèrement contrariés, continue Paulhan, et tendant vers leur satisfaction
sans obstacles trop redoutables, [la tendance et le besoin] provoquent
une impression agréable. Plus fortement contrariés et tendant
vers un fonctionnement moins harmonieux, c’est le désagréable
qu’ils amènent.» \textsc{Descartes} avait fort bien vu cela :

{\footnotesize
« Tous les sens, écrit-il, dans son {\it Compendium musicæ}, sont susceptibles
de quelque délectation. Pour cette délectation, ...l’objet doit être tel qu'il
tombe sous le sens sans trop de difficulté ni de façon trop confuse. Parmi les
objets d’un sens, celui qui est le plus agréable à l'esprit n’est pas celui qui
est le plus facilement perçu par le sens ni celui qui l’est le plus difficilement,
mais bien celui qui ne l’est pas si facilement qu'il ne laisse que chose à
désirer à la tendance par laquelle les sens se portent vers leurs objets, ct
qui ne l’est pas non plus si difficilement que le sens en soit fatigué. »
}

On peut donc conclure de là, avec H. \textsc{Spencer}, que « le plaisir [état
agréable] accompagne généralement \textbf{\textit {les activités moyennes}} » ou avec
G. \textsc{Dumas}, que « \textbf{\textit {l'agréable correspond à des excitations légères
et le désagréable à des excitations plus fortes}} ».

{\footnotesize
Les exemples concrets ne manquent pas à l'appui de cette loi. Des expériences
d'H. \textsc{Beaunis} sur la sensation du sucré ont montré qu’à très faible
dose, cette sensation est indifférente ; qu’à dose moyenne, elle est agréable ;
et qu’à très forte dose, elle devient écœurante et désagréable. \textsc{Hamilton}
interprétait ainsi la tonalité affective des couleurs : les couleurs vives qui
stimulent modérément l'organe de la vue sont agréables ; mais le noir, surtout
le noir sans reflets ou, à l'opposé, les couleurs trop éclatantes sont
désagréables parce qu’elles « le stimulent en deçà ou au delà de cette mesure ».
L'activité musculaire suit la même loi : une immobilité prolongée peut
devenir très désagréable, voire pénible ; un exercice physique est agréable
s’il est modéré, mais il redevient désagréable s’il dépasse nos forces ou s’il
se prolonge jusqu’à la fatigue. — On pourrait même étendre la loi aux
impressions d'{\it ordre moral}. L'isolement, entravant l'exercice des facultés
qui s’exercent dans le commerce de nos semblables, peut devenir extrémement
désagréable ; à l’opposé, le commerçant qui ne sait plus « où donner
de la tête » pour répondre à une clientèle trop nombreuse, n'est guère mieux
partagé ; nous trouvons au contraire très agréable la société de quelques
amis avec qui nous conversons familièrement et sans contrainte.
}

Bien entendu, il ne faut pas tenir compte seulement de l'intensité
objective de l’excitant, mais de son intensité {\it par rapport au sujet} ct
à l’activité dont il dispose, et ceci explique déjà en grande partie les
variations individuelles. \textsc{Ribot} cite le cas d’une malade dont le
système nerveux élail devenu si irritable que, bien qu’elle vécut
% 38
constamment dans une demi-obscurité, les personnes qui venaient
lui rendre visite devaient s'abstenir de porter des pièces de vêtement
blanches, sans quoi elle poussait des cris de douleur. Les enfants se
plaisent au bruit, que les personnes âgées ne peuvent supporter.
Telle excitation, tel exercice qui nous sera agréable quand nous
sommes dispos, deviendra pénible si nous sommes déprimés ou fatigués.
Une conversation qui nous charme ordinairement, nous paraitra fastidieuse
si nous sommes surmenés.

{\footnotesize
On a objecté cependant qu'il existe des excitants qui sont agréables ou
désagréables par \textsc{eux-mêmes}, à tous les degrés. On a cité notamment les
grincements (de la craie sur l’ardoise, d'un corps dur sur une vitre, d’une
scie sur la pierre, etc.), certaines odeurs ou saveurs, les couleurs sales, etc.
Quelques-uns de ces cas peuvent s'expliquer, ainsi que le remarque Dumas,
par le fait qu'il s’agit, par exemple, de « tonalités auditives très élevées »
qui, comme le bruit strident dont nous avons déjà parlé, deviennent désagréables
ou même douloureuses du fait des tensions tympaniques excessives
qu’elles provoquent. D’autres, comme les « couleurs sales», sont peut-être
surtout désagréables à cause des associations d'idées qu'elles évoquent.
Mais tous les cas ne peuvent s'expliquer ainsi : par exemple les odeurs
putrides et excrémentielles sont toujours désagréables par elles-mêmes.
Faut-il dire simplement, comme on le fait parfois, que de tels excitants
sont désagréables par leur qualité, et non leur intensité, parce que \textsc{contraires
à nos tendances} ? Ce serait s’enfermer dans un cercle vicieux puisque la
tendance elle-même ne nous est connue que par l'état affectif qui la manifeste
à la conscience, et l'explication serait bien verbale.
}

%\normalsize
{\it B. Conditions biologiques.} — Aussi bien ne suffit-il pas de se placer
à un point de vue purement individuel et subjectif, L’individu fait
partie de l'{\it espèce} et il fait partie d’un groupe social. — Il faut donc
faire intervenir d’abord des conditions d’ordre \textbf{\textit {biologique}}.

1° Nos tendances elles-mêmes sont le résultat de toute une {\it histoire
de l'espèce}
\footnote{Voir au t. Ile \S 279 : la Biologie comme « histoire naturelle ».}
, et l'exemple suivant, du biologiste Fr. \textsc{Houssay}, montre
comment cette histoire pourrait nous aider à expliquer, par exemple,
pourquoi certaines odeurs sont agréables, d’autres désagréables :

« Notre gamme de sensations olfactives se sectionne en deux zones, celle
de l’aliment, celle de l’excrément. Les odeurs analogues à celles de l’excrément
sont dites mauvaises, celles analogues à l’aliment sont dites bonnes.
Pour l’excrément, tout le monde sera d'accord ; pour l’aliment, c'est moins
net. [C'est que] notre sens olfactif est resté au stade de notre dernière
étape animale. Les bonnes odeurs sont pour nous, et sans conteste, celles
qui sont analogues aux essences de fleurs et de fruits. Cette fois, on saisit
le rapport avec l'aliment de l’arboricole frugivore que nous étions avant
l'humanité. »

% 39
2° Le point de vue biologique peut, d'autre part et plus géneralement,
nous offrir une explication du fait, constaté ci-dessus, que
les états {\it agréables} accompagnent le plus souvent les excitations
{\it moyennes}. C’est que l’activité d’un être vivant est faite d’un ensemble
d'échanges avec le milieu extérieur, tel que l’organisme répare {\it par
son fonctionnement même} les pertes subies du fait de ces échanges. On
comprend ainsi qu’une stimulation modérée joue le rôle d’une mise
en train qui, loin de provoquer une diminution de l’énergie organique,
se solde en définitive par un gain : ne le constatons-nous pas très
simplement lorsqu’après avoir fait une petite promenade au grand
air, nous nous sentons plus dispos qu'auparavant ? Il existe donc une
dépense optimum d’énergie qui correspond, comme a dit Georges
\textsc{Dwelshauvers}, à un « tonus affectif » qui est précisément celui de
l’agréable.

{\it C. Conditions sociales.} — Mais l’homme est aussi, et peut-être surtout,
un être \textbf{\textit {social}}. Or nous avons montré (\S 19 et suiv.) combien
l'influence du milieu social modèle et transforme nos tendances. Nous
avions déjà vu (t. I, \S 34) à quel point nos impulsions, même les plus
fondamentales, subissent ainsi « l’action irrésistible du modèle culturel ».
Les états affectifs qui en résultent, sont donc nécessairement
eux aussi, selon l’expression de Ch. \textsc{Blondel}:, « de toute part
socialisés ».

1° Certes, la vie affective nous paraît être ce qu’il y a de plus
intime en nous : c’est pourquoi nos sentiments nous paraissent uniques
et, en eux-mêmes, incommunicables, ineffables (t. I, 6 200) ; qu’on se
rappelle le mot de Montaigne à propos de son amitié pour La Boëtie
(\S 90). Toutefois, le seul fait que nous pouvons nommer ces états
affectifs, les classer d’après des caractères communs, suffit à montrer
qu’ici non plus l'influence sociale n’est pas absente (exercice 7) :

{\footnotesize
« Le langage n'exprime pas la stricte intimité des consciences proprement
individuelles, mais il ne souligne pas davantage des traits
immuables de l'espèce, puisque l’image qu’il nous offre n’est pas toujours
et partout semblable à elle-même. En revanche, il consacre expressément
l'expérience que le groupe qui le parle, a prise de la vie affective»
(Ch. \textsc{Blondel}).
}

2° Ajoutons que le milieu dans lequel se développent nos états
affectifs est nécessairement un milieu humain, donc social. Comme le
dit encore \textsc{Blondel}, « un état affectif qui se retrancherait de touté
communion humaine, qui échapperait dans sa magnifique intimité
à toute influence exogène », est un mythe. Des états affectifs, « vécus
et répandus parmi des hommes », qui se communiquent, se propagent,
s’exaltent par es phénomènes de « sympathie » dont il a été question
% 40
au \S 20, et qui subissent la répercussion de cette communication
voilà au contraire la réalité.

3° Répercussion aussi des {\it jugements de valeur} que l’on porte sur
eux. « Tout fait humain est jugé et apprécié par le groupe. Les états
affectifs n’échappent pas à cette règle... Ainsi s'établit entre eux une
échelle de valeur, une hiérarchie sociale et morale.» Il suffit de se
référer à la distinction courante des plaisirs « nobles » et des « basses »
Jouissances, pour comprendre l’importance de cette évaluation.

{\it D. Conditions intellectuelles.} — Enfin il faut tenir compte de la
répercussion de la vie \textbf{\textit {intellectuelle}} sur la vie affective, ({\it exercice 7}).
L'imagination va parfois jusqu’à nous {\it faire illusion} sur nos propres
états : « L'analyse psychologique, dit André \textsc{Gide}, a perdu pour moi
tout intérêt du jour où je me suis avisé que l’homme éprouve ce qu’il
imagine éprouver, De là à penser qu’il s’imagine éprouver ce qu’il
éprouve... »; entre aimer, par exemple, et imaginer qu’on aime,
« quek dieu verrait la différence? Dans le domaine des sentiments, le
réel ne se distingue pas de l'imaginaire » (cité par \textsc{Blondel}). Ceci
est vrai même de nos états affectifs élémentaires : \textsc{Bergson} avait
déjà remarqué que le nom d’un mets réputé exquis nous fait croire
que nous y prenons goût, alors qu’en réalité il n’est pas tellement
agréable, — Toutefois le rôle de l'intelligence n’est pas ici tout entier
d'illusion. Nos états affectifs sont {\it par eux-mêmes} quelque chose d’assez
simple puisqu'ils se réduisent à l’agréable et au désagréable, S'ils
sont au contraire, en fait, si riches en variétés et en nuances de
toute espèce, c’est que l’élément intellectuel vient, en s’y incorporant
{\it déterminer, diversifier, spécifier} de mille et mille manières les tonalités
fondamentales de l’agréable et du désagréable : qu’on songe
seulement à la différence qu’il y a entre le plaisir de boire chez l’ivrogne
et le plaisir de déguster un vin de grand crû chez un gourmet qui en
savoure le bouquet et le velouté, en apprécie l'ancienneté, en discerne
l’origine et même parfois la date précise de récolte! C’est alors le
plaisir d’un « connaisseur ».

\subsection{La douleur physique} % 28
Abordons maintenant l'étude de
ces phénomènes plus déterminés que sont {\it le plaisir et la douleur
proprement dits}. — On a distingué {\it trois variétés} de la douleur physique :
1° la sensation de {\it piqûre}, douleur aiguë, mais passagère, superficielle
et bien localisée ; 2° la sensation de {\it pincement}, plus profonde et plus
persistante ; 3° la sensation de {\it brûlure}, également profonde et persistante,
mais plus diffuse, qui peut être causée soit par les excitants
thermiques extrêmes (chauds ou froids), soit par des excitants chimiques
irritants, soit enfin par des perturbations chimiques internes (production [...]
% 41
%%%%%%%%%%%%%%%%%%%%%%%%%%%%%%%%%%%%%%%%%%%%%%%%%%%%%%%%%%%%%%%%%%%%%%%%%%%
