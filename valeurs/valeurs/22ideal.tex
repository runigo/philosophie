
%%%%%%%%%%%%%%%%%%%%%
\section{Les tendances idéales et les sentiments correspondants}
%%%%%%%%%%%%%%%%%%%%%
% {\footnotesize X}$^\text{e}$ « » “” ° {\it } {\bf } \textsc{} \textbf{\textit {}}
%\subsection{} 22
Les tendances idéales sont celles qui correspondent aux grandes
\textbf{\textit {valeurs}} humaines. D’abord attachées à des groupes (par exemple les
religions sont d’abord des religions nationales), elles s’en sont détachées
grâce à ce processus d’{\it universalisation} dont il vient d’être question.
Mais, par un phénomène corrélatif, elles se sont en même temps
{\it individualisées} en ce sens que, détachées du groupe, elles se sont
{\it intériorisées}, elles font appel de plus en plus à ce qu’il y a de plus
{\it personnel} et de plus {\it intime} en nous. Ce sont des tendances extrémement
complexes dont nous ne pouvons indiquer ici que les éléments
essentiels.

{\it A}. \textbf{\textit {Le sentiment religieux}} semble être d’abord le sentiment du
\textbf{\textit {sacré}} avant d’être celui du {\it divin}, au sens du moins où la Divinité
% 29
est personnalisée. Ce sentiment du sacré a été interprété comme le
sentiment de la \textbf{\textit {peur}}, de la \textbf{\textit {frayeur}}
que l’homme éprouve devant
l'inconnu, le sentiment d’un {\it mysterium tremendum} (R. \textsc{Otto} ; cf. tome I,
\S 219 {\it fin}). Tout au moins il est le sentiment d’une {\it transcendance
absolue}, d’un ordre de choses ou d’êtres qui est sans commune mesure
avec celui des choses et des êtres profanes. Mais.à ce sentiment
s’ajoutent, dans les religions supérieures, celui de l’\textbf{\textit {amour}} et celui
d’une \textbf{\textit {communauté spirituelle}} : ce sont ces derniers qui dominent
dans le Christianisme.

{\it B}. \textbf{\textit {Le sentiment moral}}, d’abord confondu, semble-t-il, avec le sentiment
religieux \footnote{Telle fut du moins la thèse de Durkugim. Selon M. Prapinss au
contraire, le
sentiment moral et le sentiment religieux n'auraient ni même erigine ni même fin (voir
ci-dessous \S 230).}
s’en distingue peu à peu. Mais il comporte comme
lui deux éléments : le sentiment d’une \textbf{\textit {obligation}} (devoir) et celui
d’une \textbf{\textit {aspiration}} (bien). Nous reviendrons plus longuement sur ce
point aux chapitres XIII et suivants.

{\it C}. \textbf{\textit {Le sentiment esthétique}} plonge ses racines jusque dans les tendances
organiques. On a vu (\S14) que le besoin de mouvement
donne naissance à une «activité de jeu» qui, socialisée et réglementée,
devient le jeu proprement dit. Or, {\it en un sens, l'art est un jeu,}
c'est-à-dire une activité désintéressée, sans but utilitaire direct. La
{\it danse}, si importante dans les sociétés archaïques, souvent liée aux
cérémonies religieuses (danses sacrées), forme ici le lien entre le jeu
et l’art : elle s’accompagne de chant et de musique, exige ornements
et parure, etc. L'art est donc une « socialisation du jeu » (\textsc{Lalo}). Mais
voici une différence essentielle-: l’art est \textbf{\textit {créateur}} (d’où l'importance
qu’y tient la technique), tandis que le jeu ne laisse aucune œuvre
derrière lui. De là aussi unc différence au point de vue psychologique :
le jeu, se situant dans l'imaginaire, s’accompagne souvent
(on le voit très nettement chez l’enfant) d’une sorte de « rêve éveillé » :
il est à lui-même sa propre fin, pour la satisfaction du sujet ; l’art, au
contraire; tout en constituant parfois une évasion, {\it est aux prises
avec le réel} et par là il se rapproche du travail (\S 15). — Le besoin
d'{\it exercice des sens} peut prendre, lui aussi, une valeur esthétique :
Paul \textsc{Valéry} a écrit qu’une simple boîte de couleurs, par «l’étal
délicieux des laques, des terres, des oxydes et des alumines » chante
déjà de tous ses tons l’œuvre future. — Mais il va de soi qu'il y a
dans le sentiment esthétique bien d’autres éléments. Ils paraissent
cependant se ramener, dans l’ensemble, au sentiment d’une \textbf{\textit {harmonie}}
s'adressant à la fois à l’intelligence {\it et aux sens}. Ici plus que partout
% 30
ailleurs, il y a lieu de tenir compte des variations sociales de ce sen+
timent selon le goût des différentes civilisations et des différentes
époques (voir le chap. XXIV).

{\it D}. \textbf{\textit {Le sentiment intellectuel}} s'attache au vrai, comme le sentiment
esthétique au beau. \textbf{\textit {Curiosité}} d’abord purement sensible (\S 19 B),
puis besoin de connaître pour des fins pratiques, il devient ensuite
culte du vrai pour le vrai, besoin désintéressé de connaître et surtout
de comprendre ; besoin de cohérence logique et d’harmonie
intellectuelle (\S 238 ; cf. t. I, \S 159 et 167).
 [...]
%%%%%%%%%%%%%%%%%%%%%%%%%%%%%%%%%%%%%%%%%%%%%%%%%%%%%%%%%%%%%%%%%%%%%%%%%%%
