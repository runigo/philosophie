
%%%%%%%%%%%%%%%%%%%%%
\section{Esthétique}
%%%%%%%%%%%%%%%%%%%%%
%chapitre XI
%L'ACTION ESTHÉTIQUE
%L'ART : CRÉATION ET CONTEMPLATION
% {\footnotesize X}$^\text{e}$ « » “” ° {\it } {\bf } \textsc{} \textbf{\textit {}}
\subsection{Qu'est-ce que l’Esthétique : métaphysique du Beau
ou philosophie de l’Art?}
Le mot {\it esthétique} (du grec {\it aisthèsis},
sensation) signifie étymologiquement : théorie de la sensibilité. C’est
en ce sens que l’a pris \textsc{Kant} dans sa {\it Critique de la Raison pure} où il
intitule « Esthétique transcendantale » la théorie des principes a priori
de la connaissance sensible : l’espace et le temps (\S 278 À). Toutefois,
dans sa {\it Critique du Jugement} (1790), Kant applique aussi ce terme au
jugement de valeur relatif au Beau : le « jugement de goût ». Dès 1750,
un disciple de Wolff, A.-G. \textsc{Baumgarten} (1714-1762) avait, dans son
{\it Æsthetica}, défini l’Esthétique comme une théorie du Beau, la beauté
étant, disait-il, la perfection saisie par la voie des sens. C’est cette
dernière acception qui a prévalu.

L’Esthétique a été longtemps rangée parmi les « sciences normatives »
à côté de la {\it Logique} et de la {\it Morale} (t. I, 6 16): on verra en quel
sens elle peut l’être encore. Ses jugements, comme les jugements
logiques ou moraux, sont en effet des {\it jugements de valeur}. Elle aurait
ainsi pour objet de faire la théorie de ces jugements, d’énoncer en
conséquence les {\bf normes générales du beau} (la critique d’art ou la
%197
critique littéraire jugeant au contraire les œuvres prises individuellement).
Mais cette définition enveloppe quelque ambiguïté. De telles
normes ne peuvent, semble-t-il, être établies que si l’on s’appuie sur
une certaine conception du Beau. L’Esthétique a donc été d’abord
une {\bf Métaphysique du Beau} : elle cherchait à définir l’essence même du
Beau et, en cela, comme on va le voir, elle s’inspirait de \textsc{Platon}.

{\it A.} On sait (t. I, \S 190) que, selon \textsc{Platon}, l’Idée est une réalité substantielle,
une essence intelligible, dont tout ce qui existe n’est que la copie.
De même, la beauté sensible n’est, à ses yeux, qu’{\it un reflet de la Beauté intelligible,
de l’Idée du Beau}. Dans le {\it Banquet}, il décrit comment, par la dialectique
de l'Amour, l'âme s'élève de la beauté des corps à la beauté spirituelle,
de celle-ci à la beauté des sciences et de là enfin à « la Beauté éternelle,
incréée et impérissable, qui n’a rien de sensible, mais qui existe
absolument par elle-même et en elle-même et de laquelle participent toutes
les autres beautés ». PLorin définira de même le Beau comme la {\it Forme}
qui domine la matière et lui impose sa propre unité, et saint Augustin
affirmera que « les belles conceptions qui de l’âme vont aux mains de
l'artiste, ont pour origine la Beauté qui est au-dessus de toutes les âmes ».
Cette inspiration platonicienne se retrouvera au XVIIe siècle dans l'{\it Essai
sur le beau} (1741) du P. \textsc{André} qui essaie de concilier saint Augustin et
Descartes. Mais, à côté du {\it Beau essentiel}, il fera une place au {\it beau naturel}
qui dépend d’un libre décret divin, et au {\it beau arbitraire} qui ne dépend
que de l’homme. — Après Kant, on reviendra au dogmatisme avec \textsc{Hegel}
({\it Esthétique}, 1842) qui, tout en définissant autrement l’Idée absolue (tome I,
\S 115 B et 190 D), verra lui aussi dans le beau {\it la manifestation sensible de
cette Idée}. \textsc{Schopenhaur} au contraire s’éloignera profondément du platonisme en
faisant consister la substance même du monde dans une 4 poussée
aveugle et irrésistible », une sorte d'instinct qu'il appelle {\it Volonté}. Mais il
voit dans les Idées les divers degrés d’objectivation de la Volonté universelle.
Or le rôle de l'art est, selon lui, de 4 reproduire les Idées éternelles
qu'il a conçues par le moyen de la contemplation pure, c'est-à-dire l'essentiel
et le permanent de tous les phénomènes du monde » et ainsi, de {\it nous
libérer de l'impulsion du désir}, d'{\it affranchir la connaissance du vouloir-vivre},
et il cite comme exemple « ces merveilleux peintres hollandais qui
ont contemplé d'une intuition si objective les objets les plus insigniflants »
et dont les œuvres, leurs tableaux d’intérieurs notamment, sont « une preuve
impérissable de sérénité d'esprit ».

B. \textsc{Kant} avait cependant ouvert une tout autre voie. Dans sa
{\it Critique du Jugement}, il s'était efforcé de déterminer, non l'{\it essence}
du Beau, mais les principes a priori du {\bf jugement esthétique} et il les
avait ramenés à quatre. Le beau se définissait alors : 1° du point de vue de
la qualité (t. I, \S 211), comme {\it ce qui est l’objet d’une satisfaction désintéressée} ;
2° du point de vue de la quantité, comme {\it ce qui plaît universellement
sans concept} ; 3° du point de vue de la relation, comme la
{\it forme de la finalité d’un objet perçue sans représentation de fin} (puisqu’il
est désintéressé) ; 4° du point de vue de la modalité, comme
%198
{\it l’objet d’un jugement nécessaire, mais d’une nécessité purement subjective}
(puisque sans concept). L’Esthétique de Kant était donc, comme
l'était aussi sa Morale, {\bf formaliste}, c’est-à-dire qu’elle cherchait à
définir le beau par sa forme, non par son contenu ou son essence,
et ce formalisme était un formalisme {\bf critique} qui cherchait dans la
nature même de nos facultés le fondement de ces rapports formels.

C. Toutefois, si le beau n’est ni un donné supra-sensible ni non plus
quelque chose de donné dans la nature, on verra qu’il n’est pas davantage
une pure forme {\it a priori} de nos jugements ni une simple création
de l'esprit. Certes, {\bf le beau est une valeur} et nous savons que les
valeurs sont choses d'esprit (t. I, \S 18); mais elles se {\it déterminent}
dans leur application au réel. Le beau est le résultat d’un rapport
{\it dialectique} entre l’homme et la nature, d’{\it une action créatrice} de l’homme
à partir du donné naturel, à l’aide de procédés techniques qu’il élabore
lui-même, et c’est cette action qui s’appelle l’{\it Art}. L’esthétique apparaît
ainsi comme une {\bf philosophie de l'Art}, non comme une Métaphysique
du Beau. Elle ne prétend pas construire {\it a priori} les normes
du Beau, ni les imposer à l’Art du dehors. Participant au caractère
{\it réflexif} de toute discipline philosophique (t. I, \S 19), elle cherchera à
tirer ces normes de l’activité esthétique elle-même, celle des artistes
et des écrivains, tout comme la Morale cherche à dégager les normes
de la conduite de notre vie morale effectivement vécue.

\subsection{Classification des arts}
Comment classer les différents
arts ? On pourrait être tenté de le faire d’après les {\it sens} auxquels ils
s’adressent. Mais un tel principe de classification serait bien insuffisant :
où classerait-on par exemple la danse?

A. Kant a distingué trois espèces de Beaux-arts : 1° les Arts {\it parlants}
qui sont l’{\it Éloquence} et la {\it Poésie} ; — 2° les Arts {\it figuratifs} qui expriment
les idées de façon sensible et qui sont : a. la {\it Plastique} qui comprend elle-même
la {\it Sculpture} (laquelle figure des objets pouvant exister dans la nature)
et l'{\it Architecture} (laquelle présente des objets possibles seulement par l’art),
et b. la {\it Peinture} ou art de l'{\it apparence sensible} qui comprend la
{\it Peinture proprement dite}
et l'{\it Art des jardins} ; — 3° les Arts du « {\it jeu des sensations} »
qui sont la {\it Musique} ou art du « jeu artificiel des sensations auditives » et
le {\it Coloris} ou art du « jeu artificiel des sensations visuelles ». Kant ajoute à
cette liste les arts composites tels que le {\it Théâtre}, le {\it Chant},
l'{\it Opéra}, la {\it Danse}, etc.

B. Schopenhauer, sur la base des conceptions indiquées ci-dessus,
hiérarchise les arts comme les {\it Idées} elles-mêmes, Au plus bas degré, vient
l'{\it Architecture} qui nous donne l'intuition d’Idées qui ne sont que les plus
bas degrés de l’objectivation de la Volonté, celles des qualités de la matière
telles que la pesanteur, la cohésion, la résistance : déjà cependant l'intérêt
esthétique s’y manifeste par la lutte entre la pesanteur (qui manifeste
l'effort de la matière pour adhérer au sol) et la résistance (qui l'élève au--
%199
%%%%%%%%%%%%%%%%%%%%%%%%%%%%%%%%%%%%%%%%%%%%%%%%%%%%%%%%%%%%%%%%%%%%%%%%%%%
