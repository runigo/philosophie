
%%%%%%%%%%%%%%%%%%%%%
\section{Les valeurs et l'expérience morale}
%%%%%%%%%%%%%%%%%%%%%
%chapitre XIV
% {\footnotesize X}$^\text{e}$ « » “” ° {\it } {\bf } \textsc{} \textbf{\textit {}}
\subsection{La notion de Valeur}
L'analyse de la conscience morale
nous a permis de prendre conscience d’un caractère fort important des
réalités morales : c’est qu’elles ne se situent pas sur le plan empirique
ou naturel, mais qu’elles constituent un autre ordre. Cet ordre
c’est celui des \textbf{\textit {valeurs}}.

{\it A}. Le premier caractère des valeurs est précisément, en effet
leur \textbf{\textit {transcendance}} par rapport à l’ordre des simples faits. « Une réalité,
écrivait G. BeLor dans le {\it Vocabulaire philosophique} de Lalande
est transcendante par rapport à une autre quand elle réunit les deux
%243
caractères : 1° de lui être supérieure ; 2° de ne pouvoir être atteinte à
partir de la première par un mouvement continu. » Or nous avons
constaté notamment : 1° que tous les éléments de la conscience
impliquent un {\it dépassement} : soit des tendances naturelles, soit de la
sensibilité spontanée, soit des simples jugements de réalité ; — 2° que,
si l’ordre de la nature prépare ou annonce l’ordre des valeurs morales,
il y a cependant {\it discontinuité} de l’un à l’autre, de même que, si une
méthode {\it positive} comme celle de la « science des mœurs » peut
apporter une contribution utile à la Morale, la méthode {\it réflexive} qui
lui convient est d’un tout autre ordre. Les valeurs transcendent donc
l’ordre des faits comme l'{\it idéal} transcende le {\it réel}, comme {\it ce qui doit
être} transcende {\it ce qui est}. Parler, comme certains auteurs contemporains,
de la pure « facticité » des valeurs (c’est-à-dire de leur existence
à titre de simples faits), c’est proprement les nier.

{\it B}. Il s’en faut en effet que les valeurs s'imposent à la conscience
de l'extérieur, comme une réalité foncièrement étrangère ou une chose
matérielle. Les valeurs, nous l'avons déjà remarqué, sont essentiellement
choses {\it d'esprit}. C’est pourquoi, en même temps que transcendantes,
les valeurs peuvent être dites \textbf{\textit {immanentes}}, en ce sens du
moins qu’elles répondent à certaines exigences de la conscience, à
certaines {\it aspirations} de notre esprit et qu’ainsi elles peuvent être
objet d'amour, d'enthousiasme et de dévouement. — Certains sont
même allés, sous l'influence de la philosophie allemande, jusqu’à
réduire les valeurs à de simples objets de {\it désir} ou jusqu’à poser la {\it sensibilité}
comme fondement de leur appréciation. « La valeur d’une chose,
écrit \textsc{Ehrenfels} (1859-1932) est sa désirabilité... C'est le rapport qui
existe entre un objet et un sujet et qui nous fait comprendre que
le sujet désire effectivement l’objet. » — D'autres, comme Max
Scheler, au lieu de réduire les valeurs à de simples rapports, en
font des {\it essences idéales}, des qualités irréductibles (\S 131), qui ne
peuvent être saisies que par une intuition de l’ordre du sentiment,
une intuition « émotionnelle » : tels seraient le sacré, le juste, le
beau, le noble, l’agréable, etc. — D’autres enfin, s'inspirant d'un
point de vue {\it existentialiste}, reprochent à Scheler d’avoir admis que
les valeurs « sont originairement distinctes de l'organisme » (\textsc{Gusdorf}).
Selon eux, les valeurs ont au contraire « leur fondement
dans la sphère des instincts » : il y a des « valeurs biologiques », comme
par exemple de conserver son être, de manger, de s’abriter, de se
reproduire. Aussi les valeurs s’expriment-elles d’abord dans la vie
personnelle « sous la forme de désirs, de pulsions injustifiables ».
Tout au plus y a-t-il « une seconde naissance des valeurs » lorsqu'elles
%244
% {\it }{\bf }\textsc{}{\footnotesize X}$^\text{e}$
passent « de l’ordre biologique à l’ordre sensori-moteur », mais uniquement
« pour acquérir par là leur indispensable expression » par
le langage. La valeur devient ainsi un simple {\it fait} de l'existence
humaine : elle n’est plus « ce qui est digne d’être recherché », elle est
« ce qui est recherché, en fait, par l’individu d’une manière originaire,
ce qui se trouve en nous à l’état d'intention ». Elle est le « principe
d'orientation immanent de {\it toute} activité ». Elle se trouve « partout
présente dans la condition humaine » en tant que « structure de réalité
immanente à notre action », et les valeurs morales sont « des promotions
de l'existence humaine dans son intégralité » (\textsc{Gusdorf}).

Mais, quelle que soit la forme qu’on lui donne, cette thèse de la
{\it pure immanence} de la valeur est gravement exagérée. Elle méconnaît,
dans le premier cas (Ehrenfels), que la {\it désirabilité} des fins morales
présente des caractères tout à fait spéciaux (\S 122) ; dans le second
(Scheler), que l’{\it affectivité}, étant sans règle, ne peut être le fondement
de l'appréciation des valeurs (\S 131); dans le troisième (Gusdorf),
que, la valeur étant {\it ce qui doit être} par différence avec {\it ce qui est}, c’est,
en vérité, la {\it nier} que de la ramener à « ce qui est recherché en fait »
et que ce n’est que par une véritable adultération de cette notion
qu’on peut parler de «valeurs biologiques ». Comme l’a remarqué
E. \textsc{Bréhier}, dans le cas des valeurs authentiques (le Beau comme
le Bien), le désir ne part pas du sujet qui apprécie, encore moins
de l’être organique, mais {\it de la valeur elle-même} qui nous adresse
un appel, exerce sur nous une attraction, nous invite à nous hausser
jusqu’à elle, sur un plan supérieur. Du côté du sujet, au contraire,
on ne rencontre parfois qu’absence de désir et de sentiment : « Combien
de fois arrive-t-il que nous ne sentions pas comme belle une
chose que nous savons et jugeons belle ! Combien de fois la consigne du
devoir et le mécanisme de l’action remplacent-ils le désir du bien
moral et de la vertu ! » Pour la même raison, il est impossible de
considérer l’{\it agréable}, ainsi que le fait \textsc{Scheler}, comme une {\it valeur},
même inférieure. L’agréable, remarque avec raison G. \textsc{Gusdorf} ne
coïncide même pas « avec l'utilité organique ». A plus forte raison
n'est-ce pas une valeur. C’est la loi de toute tendance et, par conséquent,
du désir, même coupable, ou de la passion, même pervertie,
d’engendrer, lorsqu'ils se satisfont, un état agréable, un plaisir ou
une joie. — L'aspect d’{\it immanence} de la valeur doit donc toujours être
conjugué avec sa {\it transcendance}. Il y a là une « réciprocité de perspectives »
que l’analyse de la joie morale (\S 122) nous avait déjà
laissé entrevoir et que nous retrouverons dans les deux idées, complémentaires
et non opposées, du Devoir et du Bien.
%245

{\it C}. Troisième caractère : les valeurs sont \textbf{\textit {hiérarchiques}} : elles
s’étagent les unes par rapport aux autres en un système ordonné et
cohérent, en une « table des valeurs ». Un des progrès de la conscience
morale a consisté précisément dans la reconnaissance de cette
hiérarchie, tandis que les codes moraux archaïques (Livre des morts des
Égyptiens, par exemple) nous montrent des valeurs d'importance
très inégale placées sur le même plan. Les valeurs ont aussi un caractère
de \textbf{\textit {polarité}}. Leur hiérarchie est doublement orientée, en un sens
positif et en un sens négatif. À chaque valeur s'oppose une
contre-valeur : de même qu’au {\bf vrai} s'oppose le {\bf faux} et au {\bf beau} le {\bf laid}, au {\bf bien}
s’oppose le {\bf mal}. Entre ces deux pôles, les valeurs s’étagent en
gradation ascendante ou descendante ; un acte moral peut être plus
ou moins méritoire, une faute plus ou moins grave.

{\it D}. Enfin les jugements de valeur ont un caractère \textbf{\textit {synnomique}}
(du grec {\it syn}, ensemble, et {\it nomos}, règle), c’est-à-dire
qu'ils sont « conçus par ceux qui les énoncent comme valables en droit
{\it pour tout les autres esprits}
avec lesquels ils peuvent entrer en société » (\textsc{Lalande}).
— Ce caractère est souvent méconnu de nos jours. \textsc{Scheler} soutient
qu’il existe des « essences strictement individuelles » et que telles
sont le plus souvent les valeurs morales : « Selon ces valeurs
absolument singulières, chaque personne doit, pour être morale, agir d'une
façon différente de celle des autres. » Plus récemment, sous l’influence
de l’existentialisme sartrien selon lequel « il n’y a pas de morale
générale », on a été jusqu’à soutenir que chacun se crée ses propres
valeurs de façon arbitraire et valable pour lui seul. C'est ainsi que
Raymond \textsc{Polin} conclut sa {\it Création des valeurs} en disant : « Il
apartient à chaque homme de choisir l'orientation et la portée de sa
propre transcendance et de décider, dans une incertitude essentielle,
au delà de toute détermination, de ses valeurs et de ses actions. Or la
structure de l’acte de transcendance est essentiellement subjective. »
De même, G. \textsc{Gusdorf} (voir \S 130) affirme « le pouvoir discrétionnaire
de la personne sur sa propre destinée — Reconnaissons que les
valeurs, étant d’ordre spirituel, {\it doivent toujours être assumées librement
par l'individu}. En ce sens, il existe une {\it précarité} essentielle des
valeurs (\textsc{Dupréel}) ; car, objet d’un libre choix, elles sont toujours
menacées par des choix qui peuvent leur être contraires. Mais cette
précarité n’enlève rien à leur consistance qui en fait des objets d'adhésion
commune. A. \textsc{Lalande} remarque avec raison que même l’individualiste
qui ne reconnaît comme valeur que « l’unique » cherche
à faire accepter son opinion : « Plaider contre le conformisme, c’est
vouloir convaincre les hommes de s’y soustraire et, par conséquent,
%246
tendre à les réunir sur de nouvelles normes. » Ainsi Nietzsche qui
veut détruire les « anciennes tables » de valeurs, y oppose dans son
{\it Zarathoustra} des « tables nouvelles ». Si l'on disait, comme certains :
« Chacun sa vérité », il n’y aurait plus de vérité. De même, si je dis :
« Chacun sa morale », il n’y a plus de morale. Quand je dis qu’une
chose est {\it bonne}, j'entends bien en effet qu’elle l’est, non seulement
pour moi, mais pour quiconque, se trouvant {\it dans des conditions analogues},
juge sainement. Toute valeur authentique est d’abord valeur
de communauté et il est de l’essence des valeurs de tendre, au moins
à la limite, à l’{\it universalité}. J.-P. \textsc{Sartre} lui-même, tout en soutenant
que le {\it choix}, sans être « gratuit » au sens gidien (puisque l’homme est
toujours « engagé » dans une situation) est cependant une {\it création}
entièrement « libre », est amené à reconnaître qu’« il n’est pas un de
nos actes qui, en créant l’homme que nous voulons être, ne crée en
même temps une image de l’homme tel que nous estimons qu’il doit
être » et ainsi qu’il y a bien « une universalité de l’homme », mais une
universalité « perpétuellement construite ».

\subsection{L’ « expérience morale »}%129
En étudiant les grandes conceptions
de la vie morale (chap. XVII), nous verrons que les {\it systèmes
moraux} mutilent, en général, la réalité morale dont ils n’expriment
le plus souvent qu’un aspect : ils ont donc quelque chose d’unilatéral,
D'autre part, ces systèmes apparaissent comme des constructions
{\it conceptuelles}, un peu artificielles, en marge de la moralité vécue, de
l'expérience même des valeurs. On comprend dès lors que, par opposition
à ce qu’il pouvait y avoir d'étroit et d’arbitraire dans les
« théories », d’autres penseurs aient fait appel à
l’\textbf{\textit {expérience effective de la vie morale concrète}}.
Mais ces tentatives de retour à l'expérience
immédiate ne pouvaient elles-mêmes faire autrement, à moins de
cesser d’être philosophiques, que de s’exprimer, se définir, se systématiser
et, en définitive, de se constituer, elles aussi, en théories,
parfois tout aussi arbitraires et unilatérales que celles auxquelles
elles s’opposent. Aussi est-il nécessaire d’analyser avec soin les caractères
qu’elles attribuent à l’expérience des valeurs et les facteurs
qu’elles y discernent. Nous ne suivrons pas ici l’ordre chronologique.
Nous commencerons par les théories qui nous paraissent les plus
éloignées de l'expérience morale authentique, pour terminer par
celles qui s’en rapprochent le plus.

% {\it }{\bf }\textsc{}{\footnotesize X}$^\text{e}$
\subsection{« L'existence morale » et l’homme total}%130
{\it A}. Peut-on
d’abord identifier l'expérience morale avec l'existence humaine prise
dans sa totalité? C’est ce qu’a soutenu G. \textsc{Gusdorf} dans sa théorie
%247
de \textbf{\textit {l'existence morale}}. L'homme moderne, dit cet auteur, vit dans un
état de crise perpétuelle ; il est tiraillé en des sens divers, dépossédé de
lui-même et a ainsi perdu « le sens de son unité ». La Morale doit
s’efforcer de le lui rendre en procédant à « un regroupement totalitaire
de l’homme en situation dans l’espace et dans le temps ». Or les
Morales intellectualistes, se présentant comme des Morales de l’universalité
et de l'éternité, n’ont fait que « dépersonnaliser » l’homme
et l’exiler de sa situation concrète. Les Morales métaphysiques placent
le Bien « dans un arrière-monde » et, chez Kant (cf. ci-dessous \S 160),
« l'impératif catégorique ne peut sauver son universalité que par
son absence à ce monde difficile et urgent où tous les hommes doivent
prendre parti dans l’immédiat au mieux des circonstances et d’eux-mêmes ».
La Morale sociologiste elle-même, tout en se prétendant
indépendante des systèmes, n’était « qu’un nouveau nom pour la
prétention métaphysique de dépossession de l’homme ». À ces « Morales
abstraites », il convient de substituer une \textbf{\textit {Morale concrète}} qui tienne
compte du « caractère historique, dramatique de l'expérience vécue »,
qui soit « une morale du temps présent » et qui pose en principe
« le pouvoir discrétionnaire de la personne sur sa propre destinée ».
Le moraliste se bornera à « aider les hommes à prendre conscience
des conditions concrètes de leur existence » de façon à permettre
à chacun de « devenir ce qu'il est ». Il définira « certaines constantes
de l’expérience et certains styles de vie ». Mais il « n’imposera pas
une solution » : il « emploiera l'indicatif plutôt que l’optatif » (il
n’est plus question, bien sûr, de l’impératif !) et, respectant le caractère
fondamental d’« intériorité » de l’existence morale, il laissera
le sujet libre de faire son choix à l’aide des « matériaux » qu’il lui aura
ainsi fournis, ce choix restant « l’apanage et le devoir de l’homme
concret » (Gusdorf).

{\it B}. Cette conception de « l’existence morale » repose sur tout un
système de postulats qui nous paraissent fort discutables :

1° un {\it anti-intellectualisme} qui transfère à « nos capacités d’amour,
d’intuition, de sympathie » les fonctions de connaissance qui appartiennent
en propre à l'intelligence et qui refuse à celle-ci la « compréhension »,
non seulement d'autrui (voir tome I, \S 292), mais
même de l’univers; d’où une {\it méconnaissance du rôle de l’intelligence
dans la vie morale} et de l'attitude {\it réflexive} (t. I, \S 19) de la
pensée morale : s’il est un faux intellectualisme qui prétend déduire
la moralité ou la construire {\it a priori}, le rôle de l’intelligence est bien,
en revanche, de {\it penser} la moralité, de la {\it systématiser} et de la {\it fonder} ;

2° une psychologie qui, sous le prétexte d’une conception unitaire
de l’être humain, méconnaît la {\it complexité} de sa nature et les contradictions
%248
internes qu’elle recèle et qui ne voit, par suite, dans la
{\it personne}, qu’une efflorescence des instincts, et non un dépassement,
une conquête sur les instincts ; d’où l’insuffisance de la formule
nietzschéenne : « devenir ce que l’on est » dont nous avons déjà
dénoncé l’équivoque;

3° une conception purement {\it intérieure} de la moralité dont nous
avons aussi signalé les dangers (\S 126) ;

4° plus généralement, un {\it subjectivisme} qui dénie aux valeurs toute
objectivité autre qu’une objectivité par {\it en bas}, si l’on peut ainsi parler,
puisqu’elle situe « le foyer commun » de ces valeurs « au niveau
de la réalité biologique, principe de nos instincts et de notre affectivité »
(Gusdorf), la « conscience discursive » étant responsable de
la diversification et de l’intellectualisation de ces « intentions primitives » ;
or nous avons vu que, si toute valeur doit certes être {\it assumée}
par un choix personnel, les valeurs n’en sont pas moins « synnomiques »
(\S 128 D), à tel point qu’on peut dire avec A. \textsc{Lalande} :
« Une valeur devient d’autant moins valeur qu’elle est plus parfaitement
individuelle» ;

5° enfin et surtout, une conception des {\it valeurs} sur laquelle nous
avons déjà formulé nos réserves (\S 128 B) et qui, de fait, enlève au
{\it choix} personnel tout critère et tout fondement. Si les valeurs ne sont
qu’une « promotion de l’existence humaine », si par suite « les conditions
de la vie morale se confondent avec les conditions mêmes de la
vie psychologique », comme ces conditions sont multiples et comme la
pure existence est divisée contre elle-même, au nom de quoi la conscience
personnelle pourra-t-elle {\it choisir} entre les différentes orientations
possibles de cette existence ? Pourquoi, par exemple, si les valeurs
ont « leur fondement dans la sphère des instincts », nous dire que « la
valeur morale s'affirme dans le dépassement de l’égoïsme biologique,
dans le renoncement et le sacrifice » ? Il ne suffit pas d’ajouter que
l'existence humaine doit être prise « dans son intégralité » et que la
moralité implique « la recherche de l’unité personnelle » (on ne voit
d’ailleurs pas pourquoi : du point de vue de l'{\it existence} pure, une vie
éruptive et incohérente vaut bien une vie unifiée) : une telle intégration
suppose un principe de {\it hiérarchisation}; or nous avons déjà
remarqué (\S 121) qu’il n’y a pas de commune mesure entre la tendance
et la volonté proprement morale et que l’unité du {\it moi} peut parfaitement
se réaliser sur un tout autre plan que celui de la moralité, par
exemple sur celui de l’égoïsme conscient, voire de la volonté délibérée
du mal. Comme l’écrivait autrefois D. \textsc{Roustan}, « vaut-il la peine de
réclamer pour notre volonté le pouvoir de faire triompher ses préférences
si la valeur se confond avec l’existence même ? » Il n’y a plus
%249
alors qu’à se fier à la pure spontanéité instinctive \footnote{Beaucoup plus cohérente, du point de vue « existentiel», nous paraît être la tentative
esquissée dès 1894 par Jean \textsc{Weser} en fonction de la conception bergsonienne de
la liberté. Cette conception nous montre en effet que chacun de nos actes est « nouveauté
absolue » et que notre liberté est totale : « La spontanéité, est libre, plus libre même
que nous ne le voudrions, libre malgré nous ; mais elle est folle, comme le génie. » Dès
lors, « nous n'avons pas de donnée plus haute que la réalité : {\it il n'y a pas de raison
d'être supérieure à l'existence} ». Et ainsi, « la réalité ne connaît d'autre valeur que celle
du fait », et, par suite, « le succès, pourvu qu'il soit implacable et farouche, pourvu que
le vaincu soit bien vaincu, détruit, aboli sans espoir, le succès justifie tout », En un mot,
« chacun est à soi-même la loi la plus haute, ...le fait est tout ». L'auteur concluait à « un
amoralisme supérieur » et dénonçait la morale comme « le plus insolent empiètement de
l'intelligence sur la spontanéité ». — S'il était conséquent avec lui-même, l'existentialisme
aboutirait en effet à un pur immoralisme. De fait, SARTRE nous dit que, puisque
« choisir d’être ceci ou cela, c'est affirmer en même temps la valeur de ce que nous
choisissons », « nous ne pouvons jamais choisir le mal : ce que nous choisissons, c'est
toujours le bien ». N'est-ce pas effacer toute distinction entre le bien et le mal ?},
et c’est, à notre
avis, par une pure inconséquence que l’on en vient à reconnaître que
la moralité implique le « désir d’un ordre, d’une cohérence, d’une
intelligibilité intrinsèque du comportement humain » et qu’en dépit
du caractère spécifiquement personnel de l'existence, « il ne saurait
y avoir en définitive qu’une seule vérité morale » qui assure, « par
delà les limitations de l'expérience, l’unité des intentions humaines ».
N'est-ce pas admettre un principe de rationalité supérieur à l’existence
elle-même et en revenir à l’universalisme d’abstraction (une
« nature humaine ») qu’on avait d’abord banni?

\subsection{L'expérience morale d’après Max Scheler : rôle du sentiment}%131
{\it A}. Le philosophe allemand Max \textsc{Scheler} (1873-1928)
avait proposé une théorie qui donne aux valeurs plus de consistance,
mais qui en fait des objets d'expérience purement affective (\S 128 B).
D’après la Phénoménologie de Husserl, nous pouvons avoir l’expérience
immédiate, l’{\it intuition} de certaines \textbf{\textit {essences}} idéales et extratemporelles.
Selon la théorie de \textsc{Scheler}, les \textbf{\textit {valeurs}} en général,
et les valeurs morales notamment, sont de ces essences. Ce sont
des qualités irréductibles qui sont l’objet d’expérience affective,
« émotive ». Elles sont {\it irrationnelles}, {\it alogiques}, en ce sens qu’elles
sont sans caractère significatif : ainsi un nouveau-né peut parfaitement
éprouver, ressentir la bonté de sa mère sans être capable d’en
pénétrer la signification. Ces {\it a priori} affectifs se disposent par degrés
en un ordre hiérarchique, à la fois selon leurs « supports », des {\it choses}
aux {\it personnes} et, selon leurs qualités, des « valeurs sensibles » telles
que l’{\it agréable}, à la valeur supérieur du {\it sacré} ou du {\it divin}, en passant
par les « valeurs vitales » et les « valeurs spirituelles ». Il y a autant
d’espèces de valeurs morales qu’il y a de rangs dans cette hiérarchie :
il y a ainsi des morales de l’{\it agréable}, du {\it vital}, du {\it spirituel}, du {\it divin}.

%250
L'expérience morale s’étage en une pluralité de couches superposées
dans les actes de \textbf{\textit {sentiment pur}} qui saisissent les valeurs à l’état
isolé, les actes de \textbf{\textit {préférence}} qui saisissent leur hiérarchie, les actes de
\textbf{\textit {sentiment éprouvé en commun}} ou de \textbf{\textit {sympathie}} qui saisissent le
moi affectif d'autrui, et enfin les intuitions de l’\textbf{\textit {amour}} qui actualisent
les valeurs les plus élevées (\S 122). L'amour est, selon Scheler, le véritable
fondement de l’expérience morale : « L'acte de l’amour ne suit
pas le sentiment pur des valeurs et l’acte de préférence : il les précède
comme un pionnier et un guide. » L’amour nous permet, en outre,
de saisir la personnalité totale des autres individus et aussi les personnalités
collectives des groupes. Car l’expérience morale présente, chez
Scheler, cette particularité qu’elle peut être, non seulement individuelle,
mais aussi \textbf{\textit {collective}} : ce peut être, par exemple, celle par
laquelle se crée, dans la « sympathie », la {\it communauté} morale de tout
un peuple.

{\it B}. Nous pourrions reprendre ici toutes les objections que nous
avons déjà élevées (\S 120 A) contre les thèses qui font de la conscience
morale une {\it intuition} pure. Mais la thèse se trouve ici aggravée du fait
que l'intuition morale est présentée comme purement \textbf{\textit {affective}}. De
là, l'erreur déjà signalée (\S 128 B) qui consiste à ériger l’{\it agréable} en
valeur. De là aussi, la tentative faite pour constituer une table unique
des valeurs {\it sans réserver une place spéciale aux valeurs morales} et la
méconnaissance de ce qui fait la spécificité propre de la moralité.
De là encore les erreurs que dénonce G. \textsc{Gurvitch} : « Pour sauver
la vie d’un enfant, je me vois dans la nécessité de détruire un tableau
précieux. Selon l’analyse de Scheler, il y a là un acte amoral, puisque
les valeurs du spirituel sont plus hautes que celles du vital » C'est
qu’en effet le sentiment est absolument incapable de servir de critère
et de règle : « La sensibilité, dit \textsc{Bréhier}, n’est pas une norme, mais
elle a une norme »; et cette norme, elle la reçoit des éléments rationnels,
par lesquels d’ailleurs elle cherche toujours à se « justifier ».
Nous reviendrons à propos de la morale bergsonienne (\S 162 C)
sur cette indétermination de l’élément affectif. Or, chez \textsc{Scheler}
comme chez \textsc{Bergson}, l’intuition affective ne se détermine qu'après
coup intellectuellement, de sorte que l’élément intellectuel apparaît
comme parasitaire.

\textsc{Scheler} lui-même reconnaît que le sentiment pur, le sentiment
éprouvé en commun par exemple, est aveugle pour les valeurs. Seul
l'{\it amour} tendrait vers des valeurs positives. Mais d’où tiendrait-il
ce privilège si ce n’est précisément des éléments intellectuels qui le
déterminent ? Chez un \textsc{Malebranche} au contraire, comme l’a bien
observé L. \textsc{Lavelle} dans son {\it Traité des valeurs}, la préoccupation
%251
de la valeur ne se dissociait pas de la préoccupation de la vérité.
« Celui qui voit les rapports de perfection, écrit Malebranche, voit
les vérités qui doivent régler son estime et par conséquent cette espèce
d'amour que l'estime détermine. » C’est dire, ajoute Lavelle, « que la
valeur se définit {\it non point par ce qui est aimé, mais par ce qui est
digne de l'être}, quel que soit le rôle que le sentiment soit appelé à
jouer pour le reconnaître ».

Scheler prétend aussi que l’amour rend clairvoyant. Nous avons
déjà critiqué cette idée. Mais elle est peut-être plus dangereuse encore
au point de vue moral qu’au point de vue intellectuel. L'amour
humain, en effet, est préférentiel : isolé du souci de la vérité, il nous rend
injustes, ou trop indulgents, ou trop sévères ; il nous éloigne de cette
attitude du {\it spectateur impartial} (\S 120) que les sentimentalistes eux-mêmes
ont reconnue nécessaire à la saine appréciation morale, C’est
pourquoi le sentiment, sous toutes ses formes, peut faire commettre
bien des erreurs : il peut inspirer de très belles actions et de magnifiques
dévouements, mais il peut aussi être la source de maints entraînements
irréfléchis. Et ceci est encore plus vrai dans le cas, prévu par
Scheler, où il se présente sous forme d’\textbf{\textit {expérience collective}} : « Quand
je demande à l’histoire, écrivait \textsc{Daunou}, quels ont été les effets de cet
enthousiasme qui précède les méthodes exactes, et qui les exclut,
elle l’accuse de la plupart des erreurs et des malheurs du monde. C’est
par lui que les prestiges passent pour des réalités, les déviations pour
des découvertes et les pas rétrogrades pour des marches triomphales. »
— Concluons donc que le sentiment ne peut servir de critère à l’expérience
morale. Il peut et doit être {\it inspirateur} : c’est un mobile, une
force d’impulsion nécessaire. Mais il ne saurait être {\it régulateur} : « Quoi-qu’on
puisse se laisser animer par le sentiment, disait Malebranche,
il ne faut jamais s’y laisser conduire. »

\subsection{L'expérience morale d’après Frédéric Rauh : rôle de
la raison}%132
{\it A}. Le philosophe français Frédéric \textsc{Rauh} (fig. 29)
[voir aussi Roustan] avait donné, avant Scheler, une interprétation
de l’expérience morale qui, bien qu’encore un peu vague, était sur certains
points plus satisfaisante. Une première différence est que, tandis
que, pour Scheler comme d’ailleurs pour Bergson, l’expérience morale
se prolonge directement en expérience religieuse, \textsc{Rauh} affirme « l’indépendance
de la morale ». Mais, si sa conception de l'expérience morale
s’oppose à toute transcendance {\it religieuse}, elle s’oppose aussi, et non
moins nettement, à toute Morale purement {\it théorique}, à toute déduction
{\it abstraite} : « Si l’on entend par {\it théorie}, écrit-il, une doctrine telle qu’on
puisse déduire logiquement ou plutôt idéologiquement telle croyance
%252
morale ou encore la position d’un dogme moral immuable, la conscience
doit avant tout s’en affranchir », ce qui n’empêche d’ailleurs que
les théories puissent être utiles « comme moyen de suggestion et surtout
d’épreuve ». Mais « un principe n’est valable pour une
conscience morale que du jour où il se dégage pour elle de
l’action ou au contact de ceux qui agissent ». C’est cette \textbf{\textit {action}}
qui constitue l’expérience morale : « Une doctrine morale
ne vaut que si elle a été vérifiée dans et par l’action, au
contact du milieu qu’elle concerne. » En ce sens, une croyance
morale « ne se prouve pas, elle s’éprouve». Ainsi, ce qu’il convient
d’opposer aux concepts métaphysiques ou aux théories abstraites, « ce n’est pas
l'intuition pure, c’est l’idée expérimentale ». Sans doute,
\textsc{Rauh} parle-t-il souvent d’un « sentiment {\it immédiat de l'idéal} »,
Mais il précise aussi que l’expérience morale est une véritable
%Fig. 29. — Frépéric RAux.
\textbf{\textit {expérimentation}}, analogue à l'expérience scientifique : « De
même que la science expérimentale ne s’apprend qu’au laboratoire,
l’analyse de la croyance morale, du mode d’action de
l’honnête homme nous révélera les règles pratiques de l’action
morale.» Il ne s’agit donc pas de
consulter simplement la conscience : « Comme il y a une
vérité scientifique objective, il y a une vérité morale objective. Cette
vérité est celle que, dans des conditions déterminées d’expérience,
tout homme raisonnable reconnaîtrait comme accessible sinon à tous,
au moins à celui qui vit dans ces conditions » ; et ainsi, « le critère
définitif en matière d’idéal, c’est la conscience intérieure \textbf{\textit {rationnelle}} ».
%253
Le désintéressement, l’impartialité ne sont rien d’autre, au fond, que
la rationalité: « Ce qui caractérise l’honnête homme, c’est de se placer,
pour savoir ce qu’en somme il veut faire, dans une attitude impartiale,
impersonnelle. Il juge en sa propre cause comme en celle d’autrui. Cela
s’appelle {\it être raisonnable} », et \textsc{Rauh} va même jusqu’à écrire: « Le sentiment
de la rationalité est le même quand j’affirme un devoir d’humanité
et quand j’affirme la loi d'attraction. Ce devoir s'impose comme une
vérité aussi bien que les lois mêmes de la nature. » Enfin \textsc{Rauh}, comme
Scheler, admet que l'expérience morale peut être \textbf{\textit {collective}} aussi bien
qu’individuelle : « On peut caractériser la conscience d’une société comme
celle d’un individu. » Il importe surtout que l’individu conserve le
contact avec les grands idéaux collectifs de son époque : « Le véritable
honnête homme a présents à l’esprit, non seulement tout le contenu
de sa propre conscience, mais toute la conscience contemporaine,
tous les types moraux actuellement vivants. » La morale est donc
essentiellement \textbf{\textit {actuelle}} : « {\it Celui-là seul est homme qui vit la vie de
son temps... La matière de la réflexion morale, c’est le journal, la rue,
la vie, la bataille au jour le jour.} » Toutefois, il ne s’agit nullement ici,
comme pour Scheler, d’une expérience immédiate, ni principalement
émotive. Les méthodes {\it objectives} préconisées par les sociologues ont
ici leur place : « Une bonne statistique de la population, des maladies
produites par l'alcoolisme instruit davantage que des analyses littéraires
de l’âme française. » Ce sont ces méthodes qui permettront
d'atteindre la moralité, non dans les doctrines, mais dans l'expérience
collective vécue : « En ce sens, les sociologues ont raison de chercher
à connaître la conscience morale spontanée d’une société, non par les
systèmes philosophiques ou sociaux qui peuvent la méconnaître,
mais par les institutions, les coutumes, l’expression inconsciente d’un
idéal. »

{\it B}. On a sans doute remarqué, dans ces conceptions de \textsc{Rauh}, des
directions assez divergentes. Nous croyons que \textsc{Rauh} n’a réussi ni
à donner à sa pensée une parfaite unité, ni à découvrir un critère de
l'expérience morale vraiment décisif (cf. \S 133). Ce que nous retiendrons
surtout de ses idées, c’est le rôle qu’il attribue à la Raison dans
l'expérience morale. Peut-être même va-t-il trop loin en ce sens : il est
excessif d’assimiler la rationalité du Devoir à celle d’une loi naturelle,
comme la loi de l'attraction. Cette dernière est rationnelle en ce sens
qu’elle est l’interprétation correcte, valable pour la \textbf{\textit {raison}}, de {\it faits}
expérimentaux ; mais elle s’impose elle-même comme une sorte de
grand {\it fait}. Tout autre est le caractère de l'expérience morale qui —
\textsc{Rauh} lui-même l’a dit expressément — « n’est pas une expérience d’un
fait, mais de véritables idéaux ». C’est toute la différence, sur laquelle
%254
nous avons déjà insisté, entre le fait et la valeur. Si \textbf{\textit {la Raison a un rôle
dans la vie morale}}, il est donc autre, bien qu’analogue, que dans la
science. Quel est-il {\it exactement} ?

1° La raison, avons-nous affirmé, est essentiellement la \textbf{\textit {faculté
de l’ordre}}. Comme telle, elle transcende l’ordre des faits ou de la
nature : elle introduit dans le chaos de nos instincts et de nos tendances
une unité, une {\it cohérence} qui est la condition de la maîtrise de soi et
la base de cette harmonie intérieure qui était l'idéal du « sage »
antique. « Vivre d’accord avec soi-même », telle était la maxime des
premiers Stoïciens. Il y a là un aspect de la moralité qui lui est tellement
essentiel que nous avons vu l’existentialisme moral lui-même
(\S 130 {\it fin}) obligé de faire une place, contrairement à ses prémisses,
à ce désir d’ordre, de cohérence, d’intelligibilité. Et, de fait, si nous
comparons la « vie droite » de l’honnête homme avec l’inconséquence
de celui qui est le jouet de ses caprices et de ses passions, nous voyons
se manifester chez le premier un « besoin de rester constamment
d’accord avec soi » qui, au fond, est « l'essence même de la Raison »
(\textsc{Parodi}). Que cet idéal de cohérence, d'harmonie intérieure n’exprime
qu’un aspect, trop exclusivement {\it statique}, de la vie morale, c'est ce
que nous verrons bientôt (\S 163 {\it fin}). Mais il suffit d'introduire ici
la distinction qu’a faite André \textsc{Lalande} entre la « raison constituée »
et la « raison constituante » pour apercevoir qu’il n’y a pas
là une objection irréfragable. Cette faculté est, en même temps,
\textbf{\textit {hiérarchisante}} ; et nous montrerons, à propos des conflits de devoirs,
combien ce caractère est important : le rôle de la Raison est ici de
mettre chaque chose (et chaque valeur) à sa place. C’est la même
fonction enfin qu’accomplit la Raison quand elle constitue ces {\it concepts
idéaux} qui ont amené \textsc{Renouvier} (\S 123) à mettre la Morale en
parallèle avec une science rationnelle comme les Mathématiques.

2° Un autre caractère de la Raison que \textsc{Rauh} a bien mis en lumière,
est de tendre à l’\textbf{\textit {universalité}}. De là cette tendance à l’{\it universalisation}
des valeurs, cette attitude d’{\it impartialité}, qui caractérisent les
jugements moraux, à tel point que \textsc{Kant} a pu donner (\S 160) cette
volonté d’universalisation comme un des critères de la moralité.

3° Enfin la Raison peut être caractérisée comme un \textbf{\textit {effort vers la
pensée claire}} par opposition à la pensée confuse. La psychologie
de la volonté nous a montré que l’acte volontaire se caractérise par
la conception claire du but et des moyens requis pour l’atteindre.
C’est pourquoi, sans aller jusqu’à dire, avec \textsc{Socrate}, que toute faute
est une erreur ou une ignorance ou, avec \textsc{Descartes}, qu’ « il suffit de
bien juger pour bien faire », on doit affirmer l’intérêt qu’il y a, même
du point de vue moral, à se faire sur les problèmes humains des
%255
notions aussi claires et distinctes que possible : « Travaillons à bien
penser, écrit \textsc{Pascal} : voilà le principe de la morale.»

\subsection{Critique de l'expérience immédiate}%133
Nous pouvons
maintenant nous faire une idée plus juste de ce qu’on doit appeler
« l’expérience morale ». Le tort de beaucoup de doctrines — c'était
déjà celui des doctrines du « sens moral » (\S 120 A) et de la « morale
ouverte » de Bergson (\S 162 B) — a été d’y voir une expérience \textbf{\textit {immédiate}},
intuitive. \textsc{Rauh} lui-même oppose l'expérience morale à la
réflexion qui, dit-il, « ne crée rien », qui « n’est qu’un instrument
de connaissance ». Or nous avons déjà dénoncé l’équivoque de
cette notion d’{\it immédiat} : « Il arrive, écrit G. \textsc{Davy}, que l’on prenne
pour simple et immédiat ce qui est, en réalité, complexe... N'en
est-il pas fréquemment, sinon toujours, ainsi des sentiments
moraux? » André \textsc{Darbon} (1874-1943) exprime la même opinion.
Des philosophies inspirées de la Phénoménologie allemande nous
invitent, dit-il, à « nous replonger dans le vécu », à « ressaisir toutes
les valeurs par intuition immédiate ». Théories séduisantes, mais qui
renferment une grande part d’illusion ! Partout en effet, dans la
Science, dans l’Art, l'expérience est « une expérience mûrie et pensée ».
La science dépasse de beaucoup nos simples intuitions sensibles ; l’art
musical n’est pas « la première musique que nous entendons chanter
en nous ». Ainsi, « {\it on ne peut ramener l'expérience à l’immédiateté ; on ne
le peut dans aucun domaine} » et, au fond, « l’immédiat que nous pouvons
atteindre n’est jamais un véritable immédiat ». Pourquoi en
serait-il autrement de l’expérience morale? « L'homme d’expérience,
dans l’ordre moral, ce n’est pas l’innocent qui obéit sans discernement
aux premiers mouvements de la nature, mais celui qui, ayant beaucoup
vécu avec les hommes et ayant pensé sa vie, s’est arrêté, après des
détours et des repentirs, mais avec une ferme décision, dans la conscience
de son véritable bien. L'expérience morale n’est pas un commencement ;
elle est plutôt une fin » (\textsc{Darbon}).

D’autre part, les théoriciens de l'existence ou de l'expérience
morale ont eu le tort, en général, d’opposer cette expérience immédiate
à toute \textbf{\textit {justification rationnelle}}, en particulier par des {\it principes
abstraits}. Or, ainsi que le remarquait dès 1919 D. \textsc{Parodi}
(et ceci est encore plus vrai aujourd’hui) : « L’abstraction est fort
mal vue par nos philosophes les plus illustres! Mais il se pourrait
bien que l’abstraction fût une condition inévitable de l'exercice de
la pensée dans tous les ordres... Qu'est-ce qu’un débat d’ordre moral,
%4; (Allusion, alors, à Bergson.]
%256
sinon l'effort pour définir aussi clairement que possible les diverses
raisons qui peuvent recommander telle action ou en détourner, et
cela en les considérant d’abord chacune en elle-même et abstraction
faite de toutes les autres, quitte à rechercher ensuite comment elles
se modifient, se neutralisent ou s’accordent une fois rapprochées à
nouveau... Se rire des principes abstraits et des raisons d’ordre général,
prétendre saisir d'emblée ce qu’il y a d’unique dans chaque conjoncture,
d’inassimilable à la raison dans chaque décision, c’est favoriser
toutes les équivoques, tous les sophismes du cœur, de l’égoïsme,
de l’amour-propre, de la passion. L’intuition est sans critère pour se
justifier, s’imposer à autrui, se défendre contre soi-même ; elle ne se
mesure qu’à son intensité seule qui ne lui donne de valeur que pour
qui la ressent et au moment où il la ressent » (\textsc{Parodi}).

Sur ce point, il faut reconnaître que F. \textsc{Rauh}, en rapprochant
l’expérience morale de l’expérimentation scientifique, en lui attribuant
un caractère rationnel, en maintenant l'existence d’une « vérité
morale objective », a su éviter l’illusion de « l’expérience » purement
intuitive. Est-il parvenu cependant à fournir de la vérité de l’expérience
morale un {\it critère suffisamment décisif} ? Sa conception semble
encore trop proche d’un empirisme qui frise parfois le pragmatisme,
bien qu'il ait fait lui-même des réserves sur cette dernière doctrine,
pour qu’on puisse répondre affirmativement. Dire que l’expérience
morale est liée à « l’action », qu’on ne peut éprouver les principes
moraux « que dans la conduite, qu’en agissant », c’est fort bien.
Mais {\it quel sera le critère de l’action réussie ?} Nous retrouvons ici la
même difficulté que nous avons déjà rencontrée à propos de la théorie
de « l’existence morale » (\S 130 B). Dans le domaine scientifique,
l'expérience possède un critère objectif. Faut-il voir dans ce « {\it sentiment}
de rationalité » qu’invoque Rauh, celui de l’expérience morale ?
Ce serait un critère encore bien subjectif.

\subsection{L'expérience morale collective}%134
Aussi bien, l’expérience
morale ne saurait-elle, ainsi que l’ont reconnu \textsc{Rauh} et même, en un
sens, \textsc{Scheler}, rester purement individuelle. La confrontation de
l’expérience intime de la conscience avec celle de la {\it conscience d'autrui}
et surtout avec les grands {\it idéaux collectifs} peut déjà nous préserver
des illusions d’une expérience trop subjective. Comme l’écrit \textsc{Darbon},
« l'expérience morale est une expérience collective des générations » et,
pour se targuer d’expérience morale, il faut avoir « beaucoup vécu avec
les hommes ». Mais, ainsi que le précise \textsc{Rauh}, cette expérience doit
être une expérience « critique », c’est-à-dire dans laquelle l'esprit,
au lieu de se laisser emporter par le torrent des passions collectives
%257
(\S 131 {\it fin}), demeure capable de « juger ». Peut-être même faut-il
aller plus loin : \textsc{Rauh} lui-même, tout en reconnaissant le bien-fondé
d’une certaine information sociologique, semble en avoir restreint
exagérément la portée. Il est très légitime d’être surtout préoccupé,
comme lui, par le {\it présent}, par l’{\it actuel}. Mais le présent peut-il se
comprendre (et même se vivre) pleinement si on le détache du passé?
N’est-il pas nécessaire de l’intégrer dans \textbf{\textit {cette immense expérience
collective de l'humanité en quête de ses propres valeurs}} dont la
science sociologique des mœurs (\S 127) nous présente le tableau ?
Et n’y a-t-il pas là un complément et un élargissement indispensables
de cette expérience toujours limitée, toujours fragmentaire, toujours
partielle et partiale qui s’effectue dans l'intimité de la conscience
individuelle ?

\subsection{La détermination des valeurs}%135
On aperçoit ainsi la solution
d’un problème trop négligé par les théoriciens de l’expérience
morale : celui de la {\it détermination des valeurs}, et l’on voit en même
temps comment répondre à l’argument des sceptiques fondé sur les
variations de la conscience (\S 124). Nous avons déjà remarqué que
ces variations s’effectuent, non au hasard, mais en liaison avec l’évolution
de la structure sociale. C’est en effet à l’intérieur de celle-ci
(c’est-à-dire en fonction des conditions réelles de l’existence humaine)
que s’effectue la {\it détermination} des exigences de la conscience. Ces
dernières se résument, au fond, en une exigence de {\it spiritualisation},
sur le plan purement humain de la conduite de l’homme. De cette
exigence fondamentale, la conscience s’efforce d’incarner ce qu’elle
peut dans chaque \textbf{\textit {situation}} sociale particulière, tandis que la raison
s'applique à {\it coordonner} le tout, à {\it hiérarchiser} les valeurs et à
{\it universaliser}
celles qui n’avaient d’abord été reconnues qu'avec une extension
restreinte. Soit par exemple le principe de la {\it dignité de la personne
humaine}. Il n’est rien d’autre, en dernière analyse, que celui
de la valeur de l’esprit. Mais, dans les sociétés fortement stratifiées,
comme les sociétés à base de castes, il n’est guère appliqué qu’aux
individus des classes supérieures : les autres sont des « parias », sans
valeur personnelle et sans droits. Dans l’antiquité classique, il vaut
pour les hommes libres, mais ne s’oppose pas à une institution comme
l’esclavage qui nous paraît aujourd’hui sa négation même. Son universalisation
est fonction de l'effacement des distinctions sociales et
du développement de la division du travail. Autre exemple : le {\it respect
de la femme} n’interdit pas, dans certaines sociétés, sa demi-claustration
dans le gynécée, ni même chez nous, à une époque récente, son
entière subordination à l’homme à qui, dans la famille, elle doit
%258
% {\it }{\bf }\textsc{}{\footnotesize X}$^\text{e}$ « » “”°
« obéissance ». Pour nous, il semble requérir l’égalité des droits, non
seulement dans la famille, mais même dans la vie publique. On pourrait
multiplier les exemples. — {\it La véritable expérience morale est celle
de cet incessant effort pour incarner les valeurs dans l'existence réelle
(c'est-à-dire sociale) de l’homme, pour définir aussi clairement que
possible leurs exigences dans chaque situation concrète, pour les coordonner
enfin en un système hiérarchique et cohérent}.

\subsection{Les conflits de devoirs (cas de conscience)}%136
Les mêmes
considérations nous permettent de résoudre, {\it dans la mesure où il
peut l'être par la Morale théorique}, le célèbre problème des « cas de
conscience ».

Précisons d’abord qu’il n’y a proprement {\it cas de conscience} que
lorsqu'il y a \textbf{\textit {conflit de devoirs}}. Si au contraire c’est un simple intérêt
qui se trouve en conflit avec un devoir, la solution peut certes être
pénible, elle peut créer des drames intérieurs sur le plan {\it pratique} ;
mais, sur le plan {\it théorique}, elle ne saurait faire de doute : car un intérêt,
comme tel, n’est pas une valeur ; il doit donc obligatoirement
céder devant le devoir. — Mais {\it comment peut-il exister des conflits de
devoirs ?} Si les devoirs étaient la valeur à l’état pur, on ne s’expliquerait
pas qu’elle fût ainsi divisée contre elle-même. Dans une conception
comme celle de \textsc{Kant} où la moralité est identique à la rationalité
(\S 160), les conflits de devoirs ne s’expliquent guère. Si au contraire
on tient compte de la détermination des valeurs en fonction de
leur {\it incarnation sociale} et de leur {\it codification par la raison}, de tels
conflits sont concevables. C’est d’ailleurs aussi ce qui fournit le principe
de leur solution. Nous disons : {\it le principe} ; car il est bien évident
que chaque cas de conscience est, par définition, un cas d’espèce
et qu’il appartient à la volonté libre de chacun d’assumer la responsabilité
de la décision. La Morale théorique ne peut qu’indiquer ici des
{\it directives générales} en fonction desquelles cette décision sera prise.
Ces directives sont de deux espèces :

1° Eu égard à leur détermination sociale, les valeurs morales
s’échelonnent en fonction de l’\textbf{\textit {extension des cercles sociaux}} correspondants.
On peut donc poser en principe que, toutes choses égales
par ailleurs, les devoirs envers la famille priment les devoirs de conservation
individuelle, les devoirs envers la patrie priment les devoirs
envers la famille, etc.

2° Eu égard à leur codification rationnelle, les valeurs s’étagent en
une \textbf{\textit {hiérarchie}} (\S 128 C) qui établit entre elles un ordre de prévalence
selon qu’elles se rapprochent plus ou moins de la valeur fondamentale :
celle de la vie spirituelle. C’est ainsi que le devoir de conservation
%259 {\it }{\bf }\textsc{}{\footnotesize X}$^\text{e}$
individuelle passe au second plan quand il s’agit de sauver la dignité
de la personne : car, en tant qu'être spirituel, celle-ci représente une
valeur d’universalité qui dépasse de beaucoup celle de l'individu.
Dans certains cas l’individu peut se trouver délié de ses devoirs envers
l’État si celui-ci impose un régime qui viole les droits essentiels de la
personne. Inversement, il y a des circonstances où la Patrie incarne
des valeurs si hautes que, si ces valeurs sont menacées, elle devient
presque un absolu. Dans {\it La Vallée heureuse} de Jules \textsc{Roy}, on voit un
Français, pendant l'occupation allemande, engagé dans la R.A.F.
et qui a reçu l’ordre de venir bombarder le territoire de son pays :
« Chevrier risquait de tuer sa mère et de raser son village natal si
l’ennemi prenait la fantaisie d’y enterrer un poste de commandement
ou un parc d'artillerie. » Et pourtant, ajoute l’auteur, « Chevrier ne
regrettait pas de venir attaquer son pays ». Il s’agit en effet d’un
« combat contre la tyrannie » et il faut « détruire l’ennemi à tout prix ».
D'ailleurs, l’aviateur se sent en communion avec les combattants de
la Résistance qui, en France, « minent les ponts, incendient les wagons
d’essence et impriment des journaux clandestins ». Il est à remarquer
en effet qu’en pareil cas l'individu, tout en prenant ses responsabilités,
ne juge pas d’un point de vue purement individuel et subjectif,
mais au nom de valeurs qu’il sait ressenties également par d’autres,
de valeurs de communauté.

SUJETS DE TRAVAUX

Exercices. — 1. {\it Étudier le tableau des valeurs proposé par} G. \textsc{Gusdorf},
p. 98-99. {\it Le comparer avec les différents types de valeurs distingués par}
C. \textsc{Bouglé}, chap. I. — 2. {\it Commenter ce texte de} L. \textsc{Lavelle} : « Le terme
même de valeur accuse déjà une subordination du corps à l’esprit qui seul
est capable de donner une valeur même au corps, ce qui suffit à montrer
que l'expression « valeur de fait » n’a proprement aucun sens, si le fait ne
reçoit jamais sa valeur que de l'esprit qui, au moment où il s’en empare,
lui donne une consécration qui le transfigure. » — 3. {\it Faire de même avec cet
autre texte de} R.-E. \textsc{Lacombe} : « Les valeurs qui fondent ma vie présente,
je les considère non pas [ainsi que le veut J.-P. \textsc{Sartre}] comme posées
arbitrairement par un acte de ma liberté, qu’un autre acte de liberté pourrait
annuler, mais comme ayant un fondement qui, en un certain sens,
s'impose à ma volonté » (R. Ph., 1963, p. 43). — 4. {\it Déterminer les différents
sens du mot} expérience {\it dans les phrases suivantes :} « La haute vaillance Dont
je ne fais ici que trop d'expérience » (\textsc{Corneille}), « Celui qui a eu l’expérience
d’un grand amour néglige l'amitié » (\textsc{La Bruyère}), « Par expériences
incontestables, j'entends principalement les faits que la foi nous enseigne et
ceux dont nous sommes convaincus par le sentiment intérieur que nous
avons de ce qui se passe en nous » (\textsc{Malebranche}), « Le mystique fait des
expériences : cela signifie qu’il a conscience de passer par certains états d'âme
subjectifs » (\textsc{Boutroux}), « Une expérience religieuse ayant ses caractères
%260
distincts est une chose qui se constate » (W. \textsc{James}), « {\it Expérience pure}, tel
est le nom que je donnais au flux immédiat de la vie qui nous fournit les
matériaux plus tard mis en œuvre par notre réflexion. C’est seulement pour
les enfants nouveau-nés ou les adultes dans un état à demi comateux...
qu’on peut parler d'expérience pure » (1.), « Il y aurait une dernière entreprise
à tenter : ce serait d'aller chercher l'expérience à sa source ou plutôt
au-dessus de ce {\it tournant} décisif où, s’infléchissant dans le sens de notre
utilité, elle devient proprement l'expérience {\it humaine} » (\textsc{Bergson}), « La
notion générale de l'expérience [dans les civilisations d'Occident] est surtout
{\it cognitive}. On ne saurait l’appliquer telle quelle à l'expérience des primitifs
qui est surtout {\it affective} » (\textsc{Lévy-Bruhl}), « Le plus souvent, dans la philosophie
contemporaine, ce qu’on attend d'elle [l'expérience intérieure],
c’est qu’elle produise une sympathie de la conscience avec une réalité profonde
dont nous ne percevons, dit-on, que l'écorce » (J. \textsc{Nabert}), « Aux
expériences fondamentales de la solitude, de l’échec, de la faute, se rattachent
la plupart des sentiments qu’engendre l’expansion du moi» (1o.),
« L'expérience est un mode de connaissance où l’objet est donné en original ;
en ayant l'expérience d'autrui, nous disons qu'il est lui-même devant nous »
(HusserL), « Un esprit entre dans la philosophie quand il s'aperçoit que
tout ce qui est à connaître se donne es:entiellement à lui dans son expérience,
au moment où il est comme à tout moment » (R. \textsc{Le Senne}), « Une
expérience peut être saturée de préjugés » (G. \textsc{Marcel}). — 5. {\it Chercher des
exemples de conflits de devoirs (par exemple, si un de vos camarades était justement
accusé d'un délit, quelle serait votre attitude en pareil cas?)}.

Exposés oraux. — 1. {\it L'expérience morale et ses conditions collectives}
d’après \textsc{Darbon}, chap. V et VI. — 2. {\it Valeur et détermination} d’après
\textsc{Le Senne}, {\it Obstacle et Valeur}, Aubier, 1934, chap. IV, et {\it Traité}, passim.

Discussion. — 1. {\it Discuter la conception de la valeur impliquée dans ce texte
de} G. \textsc{Gusdorf} : « La notion de valeur doit être comprise dans son sens le
plus général. La personne sollicite du milieu la satisfaction d’exigences
extrêmement diverses, des plus primitives aux plus raffinées. Il faut que
soient satisfaits les besoins organiques de nourriture, de boisson. Il faut
que soit assuré un confort minimum dans l’existence. Mais à ces réclamations
frustes s’en ajoutent de plus élevées. » — 2. {\it Sentiment et raison dans
la vie morale. Que penser de ce jugement de Montesquieu :} « Chose singulière :
ce n’est presque jamais la raison qui fait les choses raisonnables et on ne va
presque jamais à elles par elle »?

Lectures. — {\it a.} Jean \textsc{Weber}, {\it Une étude réaliste de l'acte et ses conséquences
morales}, dans la R. M. M., sept. 1894, p. 531. — {\it b.} *Fr. \textsc{Rauh}, {\it L'Expérience
morale}, Alcan, 1903, rééd. 1951. — {\it c.} *D. \textsc{Parodi}, {\it Le Problème moral et la
pensée contemporaine}, Alcan, 3$^\text{e}$ éd., 1930, 1$^\text{re}$ partie, \S 3, et 2$^\text{e}$ partie,
\S 3, 5 et app. : {\it le Rationalisme moral}. — {\it d.} A. \textsc{Darbon}, {\it Une Philosophie
de l'expérience}, P. U. F., 1946, 2$^\text{e}$ partie, chap. III. — {\it e.} *D. \textsc{Roustan}, {\it La
Morale de Rauh}, dans {\it La Raison et la Vie}, P. U. F., 1946, p. 121. — {\it f.} G. \textsc{Gurvitch},
{\it Morale théorique et Science des mœurs}, P. U. F., 2$^\text{e}$ éd., 1948, chap. II
à IV. — {\it g.} *A. \textsc{Lalande}, La Raison et les Normes, Hachette, 1948, chap. IV
et VI. — {\it h.} G. \textsc{Gusdorf}, {\it Traité de l'existence morale}, A. Colin, 1950. — {Spéc.
sur le problème des {\it valeurs}) :{\it i.} C. \textsc{Bouglé}, {\it Leçons de Sociologie sur l'évol.
des valeurs}, A. Colin, 1922. — {\it j.} E. \textsc{Bréhier}, {\it Doutes sur la philos. des valeurs},
dans la R. M. M., 1939, p. 399. — {\it k.} D. \textsc{Parodi}, {\it La conduite humaine et les
valeurs idéales}, P. U. F., 1939. — {\it l.} J.-P. \textsc{Sartre}, {\it L'existentialisme est un
humanisme}, Nagel, 1946. — {\it m.} R. \textsc{Polin}, {\it La création des valeurs}, P. U. F.
1944. — {\it n.} R. \textsc{Daval}, {\it La valeur morale}, P. U. F., 1950. — {\it o.} —L. \textsc{Lavelle},
{\it Traité des Valeurs}, P. U. F, 1951, 2 vol. — {\it p.} *R. \textsc{Ruyer}, {\it Philosophie de
la valeur}, A. Colin, 1952. — {\it q.} P. \textsc{}Césari, {\it La valeur}, P. U. F., 1957. —
{\it r.} R. \textsc{Mehl}, {\it De l'autorité des valeurs}, P. U. F., 1957. — {\it s.} * Paul \textsc{Combès},
{\it Valeur et liberté}, P. U. F., 1960.
%261

%%%%%%%%%%%%%%%%%%%%%%%%%%%%%%%%%%%%%%%%%%%%%%%%%%%%%%%%%%%%%%%%%%%%%%%%%%%
