
%%%%%%%%%%%%%%%%%%%%%
\chapter{Philosophes}
%%%%%%%%%%%%%%%%%%%%%

%%%%%%%%%%%%%%%%%%%%%%%%%
\section{Fréderic Rauh}
%%%%%%%%%%%%%%%%%%%%%%%%%
%
Fig. 29. — Frépéric RAux.
1861-1909.

Voici ce qu'écrivait, deux jours après la mort
du maître, un des élèves de Rauh : « Les élèves
de M. Rauh ne reverront plus le visage cris
et la voix émouvante de celui qui s'est épuisé
à leur communiquer son ardeur... Ah ! qu'il
était vivant, courageux et sincère ! Il ne suffit
pas de dire qu'il vivait sa pensée. Il la
nourrissail vraiment de sa substance. Il faisait
passer en elle tout ce qu’il avait en
lui de chaleur et de passion. Il l'arrachait
du plus profond de lui-même. Que c'était
beau de le voir qui cherchait, tâtonnait, peinait,
jusqu'à ce qu'il eût rencontré le réel...,
c'est trop peu dire : jusqu'à re qu'il se fût
violemment cogné au réel. Il se donnait ainsi
et s'épuisait chaque jour » (Henri FrANcx).

%%%%%%%%%%%%%%%%%%%%%%%%%%%%%%%%%%%%%%%%%%%%%%%%%%%%%%%%%%%%%%%%%%%%%%%%%%%%%%%%%%%%%
