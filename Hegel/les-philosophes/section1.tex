
%%%%%%%%%%%%%%%%%%%%%
\section{Résumé de : la Phénoménologie de l'esprit}
%%%%%%%%%%%%%%%%%%%%%


Hegel

Cet ouvrage présente les figures successives que prend l'esprit dans son auto-déploiement vers le savoir absolu : certitude sensible, perception, entendement... et le processus dialectique qui mène d'une figure à l'autre.
Il s'agit du premier ouvrage majeur de Hegel, qui a eu un retentissement très important dans l'histoire de la philosophie.


%1/12
Il faut se rendre compte de notre chance : cela ne fait que quelques dizaines d’années que le texte de {\it la Phénoménologie de l’esprit} peut être lu en français. Il a fallu attendre plus de cent trente ans pour que la première traduction voie le jour : c’est celle de Jean {\bf Hyppolite}, en 1939-1941, qui a eu le courage de s’attaquer à ce monument. Jusqu’alors, en France on ne connaissait surtout Hegel que par {\it l’Encyclopédie des sciences philosophiques}, et un ouvrage plus accessible, plus « grand public », {\it l’Esthétique}.

Si J. Hyppolite s’est lancé dans un tel projet, c’est parce qu’il a eu la chance de suivre les cours d’Alexandre {\bf Kojève}, à l’Ecole pratique des Hautes études pendant les années 30. On peut dire que ce dernier est l’un des principaux artisans de l’introduction de Hegel en France, ayant beaucoup œuvré pour la réception de sa pensée par le public francophone.

Hegel lui-même n’a pas aidé la postérité à retenir cet ouvrage. Le titre originel de l’œuvre « Système de la science », sous-titré « Première partie, la Phénoménologie de l’Esprit », donnait un rôle honorifique à cet ouvrage, dans l’économie du système hégélien : c’est le livre par lequel il fallait commencer pour découvrir celui-ci.

Mais aucune seconde partie ne verra le jour. Au lieu de cela, Hegel rédige (après une {\it Logique et les Principes de la philosophie du droit}) une {\it Encyclopédie des sciences philosophiques}, qui repense à nouveaux frais l’économie du système hégélien. En effet, dans cet ouvrage, la phénoménologie n’est plus considérée comme la première partie, mais comme une simple {\bf sous-partie} d’une section de la troisième partie du système, intitulée la « philosophie de l’esprit ». Elle est donc reléguée à une place mineure, dans le système ainsi repensé par Hegel. Cela, ainsi que le caractère tardif de sa traduction, pourrait faire penser que {\it la Phénoménologie de l’Esprit} est une œuvre mineure.

Naturellement, ce n’est pas le cas : ce chef d’œuvre, reconnu par Levinas comme l’un des {\it « cinq plus beaux livres de l’histoire de la philosophie »}, révolutionne son époque, et exerce une grande influence sur les penseurs ultérieurs. Au-delà de la profondeur conceptuelle dont il fait preuve, de nombreuses pages sont remarquables du point de vue purement littéraire, bien loin de l’aridité abstraite à laquelle le nom de Hegel est souvent associé.

{\it La Phénoménologie de l’esprit} est probablement le premier livre dans lequel vous devriez vous plonger si vous souhaitez comprendre la philosophie hégélienne. Elle mérite donc pour nous de garder son statut originel : celle d’{\bf œuvre première}, par laquelle on rentre dans le système hégélien.

Mais quelle traduction privilégier ? Quatre traductions sont en effet disponibles, dans les principales maisons d’édition : celle d’Hyppolite, la plus ancienne, celle de Lefebvre, de Bourgeois ou de Labarrière. Les débats font rage, pour déterminer laquelle respecte au mieux la lettre et l’esprit du texte hégélien, mais aussi la langue française. Certaines sont jugées trop tarabiscotées, ampoulées, ou inutilement jargonneuses, rendant le texte de {\it la Phénoménologie}, déjà complexe, encore plus obscur.

De notre côté, après de nombreuses comparaisons, nous avons choisi celle de {\bf Lefebvre} (éditions GF), mais il peut être utile de consulter les autres traductions sur une citation donnée : parfois les autres traductions sont meilleures et nous n’hésiterons pas à utiliser celles-ci lorsque cela est nécessaire, en prenant soin de préciser la version utilisée.

On connaît les circonstances dans lesquelles l’ouvrage a été rédigé : Hegel travaille à l’université d’Iéna, en Prusse, dans l’actuelle Allemagne, comme Privatdozent. Il s’agit d’un statut universitaire peu gratifiant puisque non rémunéré, désignant les professeurs sans chaire. Assistant de Schelling, il défend celui-ci dans un opuscule intitulé {\it La différence entre les systèmes philosophiques de Fichte et de Schelling}. Mais il finit par prendre ses distances avec lui, pour élaborer sa propre pensée, sa propre doctrine, et commence à rédiger {\it la Phénoménologie de l’esprit}.

Mais les troubles de l’époque le rattrapent : l’Europe est ravagée par les guerres napoléoniennes, et l’Empereur en personne vient diriger – et remporter- la bataille d’Iéna. L’histoire raconte qu’Hegel est obligé de fuir la ville, dissimulant précieusement sous son manteau le manuscrit de {\it la Phénoménologie} qu’il vient d’achever, et apercevant au loin {\bf Napoléon} sur son cheval blanc, relate ainsi l’événement : {\it « J’ai vu l’Empereur – cette âme du monde – sortir de la ville pour aller en reconnaissance ; c’est effectivement une sensation merveilleuse de voir un pareil individu qui, concentré en un point de l’espace, assis sur son cheval, s’étend sur le monde et le domine »} \footnote{Les références des citations sont disponibles dans l'ouvrage Hegel : lecture suivie}.

Cette rencontre fugace de deux brillants génies, deux géants ayant chacun exercé une profonde influence dans leur domaine respectif, l’un en politique, l’autre en philosophie, ne peut que frapper l’imagination.

En 1807, l’ouvrage est publié ; Hegel a alors 37 ans, et vient de rentrer dans l’Histoire.

Le livre ne connaîtra aucun remaniement ultérieur, à l’exception de la préface. En 1831, il corrige la première moitié de celle-ci avant d’être emporté par une épidémie de choléra qui ravage l’Europe entière.

A présent que nous en savons plus sur le {\bf contexte} dans lequel l’œuvre a été rédigée, tournons-nous vers le {\bf contenu} même de celle-ci. Quelle théorie Hegel développe-t-il dans cet ouvrage, et en quoi est-ce que cela marque une révolution dans l’Histoire de la philosophie ?

%2/12
Dans la célèbre préface de {\it la Phénoménologie}, Hegel propose une {\bf nouvelle définition} de la {\bf vérité}, qui rompt avec celle qui était traditionnellement acceptée : {\it « le vrai est le Tout »}. Qu’est-ce que cela veut dire ?

On a tendance, lorsqu’on entend deux théories opposées, à chercher laquelle des deux est vraie, en excluant qu’elles puissent l’être toutes les deux. On respecte naturellement le principe de {\bf non-contradiction}, tel qu’il a été défini par Aristote dans {\it la Métaphysique} : {\it « il est impossible qu’une seule et même chose soit, et tout à la fois ne soit pas, à une même autre chose, sous le même rapport ».}

Ainsi la mise en présence de doctrines opposées est vécue comme une contradiction intolérable qu’il faut résoudre au plus vite, en déterminant laquelle est vraie.

D’autre part, l’opinion commune a une conception intemporelle de la vérité : si une idée est vraie, c’est qu’elle l’est de toute éternité. Le modèle mathématique 2+2=4 en est un exemple privilégié. Voici une vérité qui le sera de tout temps.

Il n’y a pas de temporalité de la vérité, c’est-à-dire qu’il n’y a pas de développement progressif, qui ferait qu’à tel moment une idée serait inadéquate, non applicable, fausse, puis qu’à un certain stade de l’Histoire celle-ci deviendrait vraie, avant de devenir, à nouveau, obsolète.

Récapitulons : l’opinion commune {\it « conçoit moins la diversité de systèmes philosophiques comme le développement progressif de la vérité qu’elle ne voit dans cette diversité la seule contradiction »}.

C’est précisément ce modèle que Hegel rejette, en un geste inaugural et révolutionnaire. Il introduit le devenir, (ou encore : la temporalité, l’histoire) et la contradiction au cœur même de la notion de vérité, en la définissant comme un tout.

La vérité n’est pas donnée dès le départ, mais connaît un « développement progressif » au cours duquel elle va prendre une certaine forme ; alors on peut dire qu’à ce moment de son développement, certaines théories seront vraies, d’autres fausses. Puis la vérité prendra d’autres formes, et les théories deviendront obsolètes, inadéquates à ce nouveau stade du développement, donc fausses. Une autre théorie deviendra la vérité de ce moment de l’Histoire, avant de laisser place à d’autres, etc.

De ce fait, s’il y a une vérité universelle, cela ne peut être que le {\bf Tout} lui-même, c’est-à-dire l’ensemble des différentes formes successives que prend la vérité en son devenir, et non l’une d’entre elles en particulier.

On comprend donc mieux pourquoi Hegel soutient que {\it « le vrai est le Tout »}.

Il utilise une {\bf métaphore} particulièrement parlante pour illustrer cette idée : la {\bf plante}, elle aussi, connaît un développement progressif. Au départ simple bourgeon, elle s’épanouit : on voit apparaître un bouton, puis la fleur éclot, et donne un fruit. Or ce serait absurde de voir une contradiction entre le bouton et la fleur, ou entre la fleur et le fruit, de penser qu’ils se contredisent, et qu’il faut déterminer lequel des trois représente la vérité de la plante. Ce sont trois formes successives que prend la plante au cours de son développement, et l’un n’a pas plus de vérité que les autres, ou encore : chacun représente la vérité de la plante à un moment différent :

\begin{center}
\setlength{\fboxsep}{3mm}
\fbox{
{\it Le bouton disparaît dans l’éclatement de la floraison, et l’on pourrait dire que le bouton est réfuté par la fleur. A l’apparition du fruit également, la fleur est dénoncée comme un faux être-là de la plante, et le fruit s’introduit à la place de la fleur comme sa vérité. Ces formes ne sont pas seulement distinctes, mais encore chacune refoule l'autre, parce qu'elles sont mutuellement incompatibles. Mais en même temps leur nature fluide en fait des moments de l'unité organique dans laquelle elles ne se repoussent pas seulement, mais dans laquelle l'une est aussi nécessaire que l'autre, et cette égale nécessité constitue seule la vie du tout.}
}
\end{center}

De même, les différents systèmes philosophiques (scepticisme, stoïcisme, criticisme kantien, etc.) ne représentent pas la vérité, mais un moment du développement de celle-ci, se sont épanouis à un moment donné (par exemple la Grèce puis Rome pour le stoïcisme) avant de laisser place à d’autres systèmes philosophiques, plus adaptés au nouveau monde qui venait d’apparaitre (le monde chrétien, etc).

Ainsi ce que l’on saisit sans difficulté pour la fleur (il n’y a pas de contradiction entre ses différentes formes), on ne le conçoit pas naturellement dans le champ de la philosophie : le fait qu’il y ait plusieurs théories philosophiques pose problème, le fait qu’elles aboutissent à des conclusions contraires semble inacceptable, et on essaie de résoudre ce problème, cette contradiction, en prouvant la théorie que l’on préfère, et en réfutant les autres :

{\it « Tandis que, d’une part, la contradiction portée à un système philosophique a coutume de ne pas se concevoir elle-même de cette manière, et que d’autre part, la conscience qui appréhende celui-ci ne sait généralement pas affranchir cette contradiction de son unilatéralité ou la conserver affranchie de celle-ci, ni reconnaître dans la figure de ce qui semble conflictuel et en contrariété avec soi autant de moments mutuellement nécessaires. »}

Il ne faut donc pas considérer la contradiction comme un scandale, mais {\bf l’accueillir} comme une caractéristique essentielle de la vérité, conçue comme un Tout en devenir, passant d’une figure successive à une autre, qui ont chacune leur légitimité à une époque donnée.

En proposant une telle redéfinition, Hegel accomplit donc ce geste révolutionnaire : {\bf l’Histoire} entre dans la définition même de la vérité. Le modèle de la vérité mathématique, intemporelle parce que valable de tout temps, n’est plus adéquat.

Quelles sont les conséquences de cette irruption du Temps au cœur de la vérité ? C’est ce que nous allons à présent examiner.

%3/12
Cette « {\bf temporalisation} » de la vérité a en effet plusieurs conséquences fondamentales.

Tout d’abord, l’Histoire ne va pas au hasard, mais est le lieu de déploiement progressif de la vérité. Il y a donc une fin de l’Histoire, une finalité, un but, un objectif (telos en grec) : {\it « le vrai est le devenir de lui-même, le cercle qui présuppose comme sa finalité et qui a pour commencement sa fin et qui n’est effectif que par sa réalisation et par sa fin »}.

Ou encore : {\it « le vrai est le Tout. Mais le Tout n’est que l’essence s’accomplissant définitivement par son développement. Il faut dire de l’Absolu qu’il est essentiellement résultat, qu’il n’est qu’à la fin ce qu’il est en vérité »}.

La question surgit alors : quelle est cette {\bf fin de l’Histoire}, vers laquelle on s’achemine progressivement ?

Ce n’est pas ici le lieu de répondre à cette question. Il ne convient pas de donner « le fin mot de l’Histoire », si l’on peut se permettre ce jeu de mot, dans une préface. Cela ne serait d’aucune utilité. Découvre-t-on le nom du coupable au début d’un roman policier ?

Puisque {\it « la chose n’est pas épuisée dans la fin qu’elle vise, mais dans le développement progressif de sa réalisation »}, cette fin ne peut être pleinement appréciée que lorsqu’on a parcouru les stades successifs des formes que va prendre la vérité, dans son déploiement progressif.

Ainsi {\it « la fin est l’universel non vivant, de même que la tendance n’est que la pure poussée encore privée de son effectivité, et que le résultat nu est le cadavre qui a laissé cette tendance derrière lui »}.

Au stade où nous en sommes, celui {\bf d’une préface}, il faut plutôt être attentif au principe même du développement de la vérité, plutôt que chercher à découvrir les étapes concrètes de celui-ci (cela fera l’objet du corps de l’ouvrage), et ce à quoi il mène ultimement (ce qu’on découvrira à la fin du livre).

Une préface est de toute façon inadaptée à l’ouvrage même que Hegel est en train de rédiger. La Phénoménologie de l’Esprit est probablement le seul livre dont la préface commence par... la critique du principe même d’une préface, en philosophie : {\it « les explications qu’on a coutume de donner dans une préface, en tête d’un ouvrage, pour éclairer les fins que l’auteur s’y est proposées, les motivations qui sont les siennes [...] semblent non seulement superflues s’agissant d’un ouvrage de philosophie, mais même, compte tenu de la nature de la chose, inadéquates et contraires au but recherché »}.

Si en effet le vrai est le Tout, englobant une vaste diversité de formes qu’il prend successivement, il ne peut être exprimé dans une préface qui par nature ne peut avoir qu’une longueur limitée. On peut seulement dans une préface présenter quelques remarques inessentielles et contingentes, puisque tout ce qui est essentiel ne peut faire partie que d’un {\bf système} dont l’exposition va occuper plusieurs centaines de pages :

{\it « Quoi qu’il puisse convenir de dire dans une préface en matière de philosophie et de quelque façon qu’on le fasse, par exemple en donnant un aperçu historique de l’intention ou et du point de vue global adopté, du contenu général et des résultats obtenus", tout cela ne saurait être qu’une "liaison d’affirmations et d’assertions sur le vrai s’énonçant à tort et à travers – cela ne peut valoir pour la façon de procéder en laquelle la vérité philosophique serait à présenter »}.

En réalité, si le vrai est un tout, on ne peut l’exposer que dans un système : {\it « la vraie figure dans laquelle la vérité existe ne peut être que le système scientifique de celle-ci »}.

Il ne s’agit pas d’exposer quelques vérités de manière rhapsodique, c’est-à-dire « au petit bonheur », mais en fournir un exposé complet, totalisant et systématique car les différentes formes successives que prend la vérité se déduisent les unes des autres : elles s’articulent selon un processus nécessaire qu’il s’agit de mettre au jour.

C’est ainsi seulement que la philosophie deviendra science ; elle ne sera alors plus amour (philo) de la sagesse, ou du savoir (logos), selon ce que suggère son étymologie, mais savoir. Cet idéal de scientificité, qui rompt avec l’humilité originelle véhiculée par son étymologie, est revendiqué par Hegel : {\it « contribuer à ce que la philosophie approche de la forme de la science – du but [qui consiste] à pouvoir renoncer à son nom d’amour du savoir et à être savoir effectif – c’est là ce que je me suis proposé »}.

La philosophie est scientifique parce qu’elle peut être présentée dans un système, et inversement, elle est systématique parce qu’elle est une science. Cet idéal de scientificité, qui n’est plus le nôtre au
%{\footnotesize XXI}$^\text{eme}$ siècle,
{\footnotesize X}$^\text{e}$
se trouve déjà chez Descartes, et sera repris ultérieurement par Husserl, ainsi qu’en témoigne le titre de son ouvrage {\it La philosophie comme science rigoureuse}.

De nos jours, un tel idéal est discrédité, et l’on ne cherche plus à faire de la philosophie une science, ou la science absolue. Mais pour Hegel au contraire, ainsi qu’il l’affirme résolument, {\it « le savoir n’est effectif et ne peut être exposé que comme science ou système »}. Ou encore {\it « la nécessité intérieure pour le savoir d’être science est dans la nature de celui-ci »}.

Et si cela est possible, c’est encore une fois parce que le vrai est un Tout. C’est d’une telle définition de la vérité que découle cette conception de la science, et nulle autre. D’une autre définition aurait découlé un autre modèle du savoir : par exemple, celui du fragment présocratique.

%4/12
Si le vrai est un Tout dont les formes se déploient l’une après l’autre dans l’Histoire, on est éloigné ici de toute forme de mysticisme, qui prétendrait saisir immédiatement la vérité absolue, par une sorte d’intuition. Ce qui n’est pas donné immédiatement, mais se développe progressivement ne peut se saisir par cette saisie immédiate de la vérité que serait {\bf l’intuition}. A celle-ci, il faut préférer le {\bf concept}, qui est le vrai élément dans lequel la vérité peut se présenter, le seul qui peut assurer la scientificité du savoir qui en est issu : {\it « la vérité [...] n’a que dans le concept l’élément de son existence »}.

Hegel consacre plusieurs paragraphes à dénoncer cette illusion, l’intuition mystique, qui serait {\it « savoir immédiat de l’absolu, [de la] religion, [de] l’être »}. Il la résume ainsi : {\it « on n’est pas censé concevoir l’absolu, mais le sentir et le contempler. Ce n’est pas le concept, mais le sentiment qu’on en a et ce qu’on en contemple qui sont censés à la fois mener les débats et être énoncés »}.

Si Hegel développe ici une critique de la doctrine de l’intuition, c’est que celle-ci est à la « mode » au moment où il rédige ces lignes. Il explique l’origine de cet engouement ainsi : avec la philosophie des {\bf Lumières}, l’avènement de la raison et la critique de la religion qui en est issue, l’homme a perdu son lien naturel avec le monde.

Cette critique des Lumières a dissous les rapports que l’homme entretenait avec son milieu, avec l’être. L’époque qui suit, celle de Hegel, celle du {\bf romantisme} allemand triomphant, constitue une sorte de réaction : par contrecoup, l’homme essaie de retrouver son lien naturel - donc immédiat - avec l’être, ou Dieu, en rejetant le concept (assimilé aux Lumières) et en privilégiant l’intuition, ce rapport immédiat.

Avec la raison, la critique des Lumières, la réflexion qu’il opère dorénavant sur le monde et sur lui-même, l’homme ne vit plus de manière naturelle, ne coïncide plus avec lui-même de manière naïve. C’est une douleur dont il est conscient et à laquelle il essaie de trouver un remède, en retrouvant son lien originel avec l’être, par l’intuition :

{\it « Non seulement sa vie essentielle est perdue pour lui, mais il est également conscient de cette perte et de la finitude qui est son contenu. Maudissant le méchant état qui est le sien, l’esprit exige maintenant de la philosophie non pas tant le savoir de ce qu’il est, que de parvenir de nouveau grâce à elle, et seulement alors, à l’instauration de cette substantialité et de cette consistance pure et solide de l’être". L’intuition serait ce qui permettrait de "réprimer le concept différenciateur, et instaurer le sentiment de l’essence »}.

A l’intérieur même de la philosophie, un courant philosophique se développe en ce sens : il s’agit de l’« {\bf enthousiasme} », dont Jacobi est le meilleur représentant, qui soutient, contre Kant, qu’on peut avoir une connaissance directe et absolue de la chose en soi, par intuition. Une doctrine ciblée par Hegel dans les lignes suivantes : {\it « ce n’est pas le concept mais l’extase, pas la froide progression de la nécessité de la chose mais la fermentation de l’enthousiasme qui sont censés être la tenue et l’expansion et avancée continue de la richesse de la substance »}.

Du point de vue hégélien, cette doctrine n’a pas vraiment de sens. Elle consiste à voir dans l’{\bf immédiat}, le naturel, la vérité la plus profonde. Mais si la vérité est, comme le pense Hegel, un développement progressif, l’immédiat n’est que le premier stade de ce développement, et donc celui dans lequel il y a le moins de vérité, ou encore celui dans lequel la vérité est la plus pauvre. Une vérité au stade embryonnaire, non encore développée. Le commencement, l’origine, n’est pas ce qu’il y a de plus profond, mais ce qu’il y a au contraire de plus superficiel.

Ainsi, ces deux termes – pauvreté et superficialité- peuvent venir caractériser cette conception : de nos jours, {\it « l’esprit montre tant de pauvreté, qu’il semble [...] n’aspirer tout simplement pour son réconfort à l’indigent sentiment du divin »}, et {\it « de même qu’il existe une largeur vide, il y a une profondeur vide, [...] de même ce discours est une intensité sans aucune teneur, qui se comporte comme une pure et simple force sans expansion, et dès lors est la même chose que la superficialité »}.

Néanmoins, cela ne suffit pas à invalider définitivement cette doctrine. Si la vérité est un développement progressif, l’intuition de l’être ou de Dieu n’est pas la vérité ultime mais uniquement son {\bf premier stade}. Mais à l’inverse, si la vérité ne se développe pas mais est donnée d’emblée, comme le soutiennent les partisans de l’intuition, c’est la doctrine hégélienne qui nous amène à perdre cette vérité première, originelle, et nous égare dans un labyrinthe de formes successives fictives.

Comment décider entre ces deux conceptions ? Vers laquelle de ces théories devons-nous nous tourner ?

C’est ici que Hegel produit un argument décisif, qui conforte sa position, contre les partisans de la doctrine de l’intuition.

%5/12% {\footnotesize X}$^\text{e}$ “” ° {\it «  »} {\it } «  » {\bf } \textsc{} \textbf{\textit {}}
La redéfinition hégélienne de la vérité lui donne un avantage décisif : il n’a pas à réfuter telle ou telle doctrine, mais à la réduire simplement à un moment du déploiement progressif de la vérité.

En cela, il lui donne une légitimité (c’est un moment nécessaire de ce développement et doit être respecté en tant que tel), mais la réfute également (ce n’est qu’un moment, un stade de ce développement, et en cela est dépassé, ou le sera bientôt). Ou encore : il se donne ainsi le luxe de ne pas avoir à réfuter telle ou telle doctrine philosophique, c’est l’Histoire qui s’en charge.

Ainsi par exemple, il n’y a pas à produire une réfutation du stoïcisme, ce qui serait pour le moins difficile : il suffit de montrer que ce courant philosophique a correspondu à une certaine période de l’Histoire (Grèce et Rome antique), et a laissé place à une nouvelle forme du développement de la vérité, une nouvelle doctrine.

En redéfinissant la vérité, Hegel modifie également par là même ce que signifie « réfuter une théorie ». Il ne s’agit plus de trouver des erreurs factuelles, des contradictions ou un manque de preuves dans une doctrine, ce qui manque d’ailleurs de sens : quelles sont les erreurs que l’on peut trouver dans un chef d’œuvre du stoïcisme comme les Pensées pour moi-même de Marc-Aurèle ? Il s’agit de montrer en quoi cette doctrine a été, à cette époque, la forme dans laquelle la vérité s’est incarnée de manière décisive, comment elle a perdu cette légitimité, quelle est la forme qui lui a succédé dans ce rôle et pourquoi.

Ainsi, de la même manière qu’il n’y a pas à réfuter le bouton, ni la fleur, il n’y a pas à réfuter telle ou telle théorie mais simplement relativiser leur importance, et leur vérité, comme représentant simplement des moments du Tout, ou de la vérité, ce qui revient à la même chose, puisque "le vrai est le Tout"1.

C’est de cette manière que Hegel porte le coup fatal à la doctrine de l’intuition. Il remarque que si celle-ci a eu son heure de gloire (le romantisme allemand, la doctrine post-kantienne de l’enthousiasme), elle appartient déjà au passé. Aujourd’hui, on assiste à un changement d’époque : "il n’est pas difficile de voir, au demeurant, que notre époque est une époque de naissance et de passage à une nouvelle période. L’esprit a rompu avec le monde où son existence et sa représentation se tenaient jusque alors ; il est sur le point de les faire sombrer dans les profondeurs du passé, et dans le travail de sa reconfiguration".

C’est donc le Temps lui-même qui se charge d’en finir avec les doctrines de l’intuition, de l’immédiat : difficile pour ces dernières de résister à un tel adversaire ! On voit combien le procédé argumentatif de Hegel est puissant : il bénéficie de la vision d’ensemble, globale, des panthéismes, ces doctrines qui appréhendent la réalité comme un Tout.

Dans un texte magnifique, d’un point de vue littéraire, Hegel décrit ce changement d’époque, en utilisant l’image plaisante de la naissance d’un enfant : les signes avant-coureurs d’une crise (« prodromes ») s’accumulent, puis l’époque bascule vers quelque chose de nouveau, encore inconnu, par une sorte de saut qualitatif :

    Il est vrai que, de toute façon, [l’esprit] n’est jamais au repos, mais toujours en train d’accomplir un mouvement de progression continuel. Mais de la même manière que chez l’enfant, après une longue nutrition silencieuse, la première bouffée d’air interrompt cette progressivité du processus de simple accroissement – de même donc qu’il y a un saut qualitatif – et que c’est à ce moment-là que l’enfant est né, de même l’esprit en formation mûrit lentement et silencieusement en direction de sa nouvelle figure, détache morceau après morceau de l’édifice de son monde antérieur, et seuls quelques symptômes isolés signalent que ce monde est en train de vaciller ; la frivolité, ainsi que l’ennui, qui s’installent dans ce qui existe, le pressentiment vague et indéterminé de quelque chose d’inconnu, sont les prodromes de ce que quelque chose d’autre est en marche. Cet écaillement progressif, qui ne modifiait pas la physionomie du tout, est interrompu par la montée, l’éclair qui d’un seul coup met en place la conformation du monde nouveau

.


Néanmoins, on peut se demander : quels sont les signes qui indiqueraient que les doctrines de l’intuition sont dépassées, et que l’on est passé à une autre époque ?

Hegel fait référence au succès que connaît la philosophie de Schelling. Hegel a travaillé comme assistant pour celui-ci, à l’université d’Iéna, et a même rédigé un ouvrage pour le soutenir face aux attaques dont il faisait l’objet : la Différence entre les systèmes de Fichte et Schelling. Mais il a rapidement pris ses distances avec celui-ci, pour édifier sa propre pensée. On retrouve ce double mouvement ici, dans la préface de la Phénoménologie de l’esprit.

Schelling n’a pas succombé aux tentations des doctrines de l’enthousiasme, qui défendent l’idée d’une intuition de la chose en soi. Il a en revanche ressenti l’importance d’une synthèse salvatrice, qui viendrait réconcilier la Nature et l’Esprit, et a compris la nécessité de présenter cette synthèse sous la forme d’un système. Enfin, il se place du point de vue du Tout et de l’Absolu. Autant d’éléments qui ont suscité l’enthousiasme de Hegel et nourri sa propre pensée. C’est cela qui ouvre, selon lui, le passage vers une nouvelle époque, une nouvelle ère spirituelle, dans laquelle le savoir et l’esprit deviennent fondamentalement autres.

Mais les limites de la doctrine de Schelling lui sont rapidement apparues : c’est ici, dans ce passage précis de la préface, qu’il indique les raisons qui l’ont amené à s’éloigner de celui-ci.

%6/12
Tout d’abord, si Schelling formule le principe général du système « tout est un », il ne développe pas celui-ci.

Autrement dit, il fournit le concept du système, mais pas le système lui-même, dans son développement concret et détaillé.

On reste donc dans une sorte d’abstraction dont on ne peut se satisfaire :

    Cette nouveauté [l’apport de Schelling] n’a pas davantage de parfaite effectivité que l’enfant qui vient de naître ; et c’est un point qu’il est essentiel de ne pas négliger. La première entrée en scène n’est encore que son immédiateté ou son concept. Pas plus qu’un bâtiment n’est terminé quand on a posé sa fondation, le concept du tout auquel on est parvenu n’est le tout lui-même. Quand nous souhaitons voir un chêne avec toute la robustesse de son tronc, le déploiement de ses branches et les masses de son feuillage, nous ne serons pas satisfaits si, au lieu de cela, on nous fait voir un gland. De la même façon, la science, dont la frondaison couronne tout un monde de l’esprit, n’est pas achevée dans son commencement.


Oui, Schelling propose des formules vraies, sur lesquels un savoir systématique peut s’édifier et que Hegel va reprendre à son compte : « tout est un », « dans l’absolu, A = A ». Mais comme le système vers lequel elles pointent n’est pas édifié, elles restent lettre morte, abstraites, formelles, et ce formalisme ne peut convaincre : "Opposer ce savoir Un - que dans l’absolu, tout est identique – à la connaissance distinguante et accomplie [...], ou encore, donner son absolu pour la nuit où, comme on dit toutes les vaches sont noires, c’est la naïveté du vide de connaissance"1.

Cette métaphore est particulièrement parlante : une idée abstraite, qui n’est pas développée dans sa signification concrète, ne se différencie pas, dans ce vide sémantique, d’une autre. Lorsqu’on dit « tout est un » sans indiquer précisément en détail ce que désignent ce « Tout », cet « Un », cela revient à dire « rien n’est Un », ou même à ne rien dire. Dans cette abstraction formelle, toutes les idées se confondent les unes avec les autres, comme des vaches dans une nuit noire.

C’est là précisément l’apport d’Hegel : lui se propose de développer concrètement le système, exposer les différentes étapes de celui-ci, fonder la nécessité du passage d’une forme à l’autre. Tel est le rapport d’Hegel à Schelling : ce mouvement qui mène de l’abstrait au concret, et donc à la vérité, exposée sous sa forme adéquate, celle d’une science systématique.

Un nouveau trait essentiel de la pensée de Hegel nous apparaît ici. Pourquoi passe-t-on, dans le processus dialectique, d’une forme de la vérité à une autre ? Pourquoi chacune ne serait-elle qu’un simple « moment » du développement du Tout, et non sa vérité finale ? Quelle est la limitation qui la réduit à un tel statut de « moment » ? Cela dépend de la forme considérée, mais souvent, Hegel pointera son abstraction : c’est souvent parce que la forme antérieure demeure « abstraite », en reste au niveau du concept, qu’il faut passer à une nouvelle forme.

Cette progression vers le concret, l’effectif, constitue l’un des moteurs du processus dialectique, et Hegel l’utilise fréquemment lorsqu’il doit montrer pourquoi une forme donnée doit être abandonnée.

L’abstraction constitue donc une forme d’erreur chez Hegel : c’est séparer ce qui est uni en réalité. L’abstraction s’en tient, à tort, à la différence et à la séparation.

Hegel ne nie pas la différence, et la séparation, bien au contraire. Mais il ne s’agit là que d’un moment du processus dialectique, le second moment. Celui-ci est suivi par le mouvement final de réconciliation des opposés, la synthèse, qui retrouve l’identité, au cœur de la différence, afin de constituer un Tout.

Hegel résume ainsi ce processus : "seule l’identité qui se reconstitue ou la réflexion dans l’être-autre en soi-même – et non une unité originelle ou immédiate est le vrai". On ne saurait mieux dire ce qu’est la dialectique, qu’on peut décomposer ainsi en ses trois moments :

a) Moment de l’identité immédiate : A = A
b) Moment de la différence : la chose se différencie d’elle-même et va se chercher dans son opposé
c) Moment de la synthèse : la chose retrouve son identité avec elle-même, incluant son contraire, en une réconciliation des opposés

C’est par le processus dialectique, le développement concret du Tout à travers l’Histoire, lent et parfois douloureux, que l’Absolu se constitue réellement peu à peu, et non dans une formule abstraite contenue dans un livre de philosophie. C’est là la limite essentielle de la doctrine de Schelling, que Hegel vise probablement lorsqu’il soutient qu’une idée "tombe dans la fadeur, lorsqu’il y manque le sérieux, la douleur, la patience et le travail du négatif".

D’autre part, Hegel oppose son approche à celle de Schelling sur un point fondamental : le Tout, l’absolu qui se constitue peu à peu dans l’Histoire, n’est pas seulement substance, mais sujet, c’est-à-dire esprit : "dans ma façon de voir [...], tout dépend de ce qu’on appréhende le vrai non comme substance, mais tout aussi bien comme sujet".

En cela, il se distingue également de tout spinozisme. C’est important car la doctrine de Spinoza fait encore scandale à l’époque où Hegel écrit ces lignes, auprès de nombreux penseurs et théologiens. En effet, en définissant Dieu (ou le Tout) comme une substance infinie avec une infinité d’attributs, Spinoza assimile Dieu et Nature ; mais si Dieu n’est autre chose que la Nature, il n’y a pas en fait de Dieu, et ce panthéisme devient une sorte d’athéisme.

%7/12
Depuis Spinoza, on se méfie des monismes (les théories qui soutiennent que Tout est Un), car on les suspecte de mener au panthéisme et par là à l’athéisme. Hegel rassure d’éventuels détracteurs sur ce point et évite la censure, en montrant qu’on peut concevoir une forme de monisme qui préserve la notion de Dieu ; il suffit que le Tout soit défini comme esprit et non simplement comme substance :

"Comprendre Dieu comme la substance Une a révolté le siècle où cette définition [...] a été énoncée ; la raison en était [...] dans l’instinct que la conscience de soi n’y est pas conservée, mais y a tout simplement sombré"1.

Néanmoins, remarquons qu’il ne s’agit pas pour Hegel d’opposer la substance et l’esprit : on resterait dans un dualisme, incapable d’opérer une synthèse entre deux termes contraires. Au contraire Hegel définit le vrai, l’absolu, de la manière suivante : "il est la substance spirituelle".

En réalité, Hegel ne prétend pas faire œuvre originale en définissant l’absolu comme esprit. Il affirme que c’est là l’apport irremplaçable du christianisme, qui constitue la vérité de son époque : "l’absolu comme esprit : concept sublime entre tous, et qui appartient bien à l’époque moderne et à sa religion".

Le vrai est concret, comme on l’a vu ; or "le spirituel seul est l’effectif". Toute autre chose est abstraite, n’est qu’une détermination de l’esprit séparée par abstraction de celui-ci.

Nous pouvons donc recentrer notre explication autour de cette notion d’« esprit ». Au début de notre explication, nous avons dit que le projet de Hegel est d’examiner les diverses formes successives que peut prendre la vérité. Nous pouvons à présent reformuler cela en des termes plus adéquats, en disant que nous cherchons les différentes figures que revêt l’Esprit à travers l’Histoire, en son devenir dialectique.

En réalité, les deux choses sont liées : la vérité est ce qui apparaît comme telle à un esprit. Ou encore, il n’y a pas de vérité sans esprit qui la découvre. Parler des formes successives de la vérité, c’est donc parler en réalité des formes successives que prend l’esprit dans sa quête de la vérité.

Mais comme l’Esprit est le tout, l’absolu, le Un du « tout est Un », on peut proposer cette dernière formulation : on cherche ici à décrire les formes successives que prend l’Esprit dans sa quête de la vérité de ce qu’il est.

Pour cela, il doit apparaître à lui-même. Autrement dit : il doit prendre conscience de lui-même. Un lien essentiel apparaît donc entre conscience et esprit. L’esprit n’est tout d’abord qu’ « en soi » : il est ce qu’il est, sans se savoir tel.

Peu à peu, l’esprit prend conscience de lui-même : il s’apparaît à lui-même. Tout d’abord, cette conscience de soi est imparfaite ; l’esprit est donc mené de figure en figure successive, dans lequel il se précise peu à peu. Hegel nomme « pour soi » ce processus réflexif. Chez lui, conscience et « pour soi » sont synonymes.

Mais cette conscience ne s’est pas encore élevée à la dignité d’une science. On peut avoir conscience d’un phénomène sans pour autant le connaître complètement. Lorsqu’on a un savoir absolu et systématique de la marche de l’Esprit à travers l’Histoire, on accède au stade de l’en soi pour soi, qui synthétise les deux précédents, c’est-à-dire à la science :

    Le spirituel seul est l’effectif ; il est l’essence ou ce qui est en soi [...] ou encore il est en soi et pour soi. Mais cet être en soi et pour soi, il ne l’est d’abord que pour nous [...] Il doit être cela également pour lui-même – doit être le savoir du spirituel et le savoir de soi en tant qu’il est l’esprit ; c’est-à-dire qu’il doit avoir pour lui le statut d’objet, mais de manière aussi immédiate, d’objet intermédié, c’est-à-dire aboli, réfléchi en soi [...] L’esprit qui se sait ainsi [développé] comme esprit est la science. Elle est son effectivité et le royaume qu’il s’édifie dans son propre élément.


Pour résumer, il faut donc distinguer ce qu’une chose est (l’en soi), la conscience qu’on en a (le pour soi), et la science de celle-ci qui peut s’édifier (l’en soi pour soi). Celle-ci dépasse et inclue les deux moments précédents. Ainsi la science doit inclure la conscience, sinon elle n’est que "l’en soi, n’est pas en tant qu’esprit, n’est seulement encore que substance spirituelle. Il faut que cet en-soi se manifeste et devienne pour soi-même". De ce fait, la science " doit poser la conscience de soi comme ne faisant qu’un avec elle".

C’est là un long processus, qui prend du temps : pour l’accomplir, le savoir prend différentes formes, l’esprit parcourt diverses figures, et c’est pour cela qu’il y a un devenir du savoir.

On comprend alors le projet de l’ouvrage, la signification d’une « Phénoménologie de l’esprit » : "c’est ce devenir de la science en général, ou du savoir, que la présente Phénoménologie de l’esprit, [comme première partie de son système] expose".

Nous sommes à présent en mesure de mieux saisir la signification du titre du livre que nous étudions ici. Qu’est-ce que cela veut dire, chercher à dresser une Phénoménologie de l’esprit ?

Pour comprendre, il faut se reporter... à la fin de l’ouvrage. Dans le dernier chapitre, consacré au Savoir absolu, Hegel remarque que "rien n’est su qui ne soit dans l’expérience". Ou encore : "rien n’est su qui ne soit pas déjà présent comme vérité ressentie".

Ici dans la préface, cette idée est résumée ainsi : "la conscience ne sait et ne conçoit rien d’autre que ce qui est dans son expérience".

Un projet apparaît donc : il faut chercher à décrire ce dont la conscience fait l’expérience, ce qui lui apparaît dans son développement progressif.

C’est d’ailleurs précisément là le projet que reprendront à leur compte les phénoménologues (au sens orthodoxe du terme), tels que Husserl.

Or ce qui apparaît, l’apparaître, c’est en philosophie, le « phénomène », un terme qui vient du grec « phainomenon » (apparence). L’étude, la science (logos) de ces phénomènes, c’est la « phénoménologie ».

Et donc l’étude des formes successives dans lesquelles l’Esprit s’apparaît à lui-même, fait l’expérience de lui-même, prend conscience de lui-même, c’est la phénoménologie de l’esprit.

Ainsi, cet ouvrage va présenter les stades successifs du développement de l’esprit, pour devenir ce qu’il est en vérité : "l’esprit se développe et dispose ses différents moments ; ils se présentent tous comme des figures de la conscience. La science de ce chemin est science de l’expérience que fait la conscience".

On comprend mieux à présent le lien entre ces différents éléments : phénomène, expérience, conscience et esprit, et pourquoi Hegel a choisi un tel titre pour son œuvre.

Résumons : c’est parce que "rien n’est su qui ne soit dans l’expérience" , qu’il faut présenter une "science de l’expérience que fait la conscience", et qu’il faut rédiger une Phénoménologie de l’esprit.

%8/12
Dans ces conditions, on voit qu’il n'est plus question de parler comme les partisans des doctrines de l’enthousiasme, d’un « savoir immédiat » (de la chose en soi, de Dieu, etc).

En réalité, le savoir en son premier stade, "le savoir tel qu’il est d’abord", "l’esprit immédiat", n’est que la "conscience sensible"1. Cette "conscience sans esprit" n’est qu’une forme inadéquate, inférieure, dans laquelle on ne trouve que peu de vérité.

A la différence de l’enthousiasme, qui "commence comme un coup de pistolet par le savoir absolu", il faut comprendre que "pour devenir savoir proprement dit, [...] il doit se frayer un long et laborieux chemin".

Comme on l’a vu, ce sont les différentes étapes de ce chemin, dont chacune constitue l’une des figures de l’Esprit, que la Phénoménologie va présenter : "il faut supporter la longueur de ce chemin, car chaque moment est nécessaire – d’autre part, il faut s’attarder sur chacun d’eux, car chacun est lui-même une figure individuelle complète et n’est considérée absolument que dans la mesure où sa déterminité est considérée comme un tout ou comme un concret".

Cela se joue à deux niveaux qu’il faut bien distinguer, pour une meilleure compréhension du projet hégélien.

Tout d’abord, au niveau de l’Esprit universel, qui dépasse tout homme en particulier. C’est le point de vue du progrès de l’Esprit humain, considéré dans sa marche dans l’Histoire. Par exemple : la découverte des mathématiques.

C’est à cela que Hegel se réfère lorsqu’il affirme qu’"il faut examiner l’individu universel, l’esprit du monde dans le processus de sa formation".

Mais aussi au niveau de chaque individu, dont la tâche est de parcourir également pour lui-même les différentes étapes de ce progrès (par exemple en étudiant les mathématiques). C’est un chemin plus facile : "tout individu singulier parcourt [...] les différents degrés de culture de l’esprit universel, [...] mais comme autant de figures déjà déposées par l’esprit, comme des étapes d’un chemin déjà frayé et aplani".

Ou encore, "c’est parce que [...] l’esprit du monde a eu la patience de parcourir ces formes dans la longue extension du temps et de prendre sur soi d’assumer l’énorme travail de l’histoire universelle [...] que l’individu ne saurait employer moins de peine à concevoir sa substance. Mais en même temps, il a depuis lors moins de mal parce qu’en soi ceci est accompli. [...] Il ne faut plus renverser l’existence, l’être-là, en être en soi, mais seulement l’être en soi en forme de l’être pour soi".

Il faudra donc garder en tête que l’on peut lire la Phénoménologie de l’esprit d’après ces deux niveaux de lecture ; mais Hegel privilégie d’une certaine manière l’esprit universel. C’est la marche de celui-ci à travers l’Histoire qui est d’abord et avant tout décrite ; et le fait que tel ou tel esprit individuel reproduise pour lui-même cet itinéraire est certes intéressant pour lui-même, mais reste secondaire.

Hegel s’expose à présent à une objection : pourquoi exposer les différentes formes inadéquates, incomplètes, imparfaites donc toujours un peu faussées, que prend successivement l’esprit, c’est-à-dire comment il apparaît phénoménalement à lui-même, au lieu de nous livrer la vérité de ce qu’il est ? Voici, semble-t-il, ce que l’on peut attendre d’un livre de philosophie : qu’il nous donne des vérités. Et non pas une liste des erreurs successives qui nous ont mené à découvrir celle-ci :

    Comme ce système de l’expérience de l’esprit ne comprend que l’apparition phénoménale de celui-ci, la progression qui mène de ce système à la science du vrai qui est dans la figure du vrai semble être seulement négative, et l’on pourrait vouloir demeurer exempté du négatif, en ce qu’il est le faux et exiger d’être conduit sans plus attendre à la vérité ; à quoi bon s’occuper du faux ?


C’est précisément ce que l’on voit en mathématique : c’est la vérité qui nous est livrée, et non pas les erreurs successives que nous avons effectuées avant de parvenir au calcul juste, ou à la démonstration correcte, au fameux CQFD (« ce qu’il fallait démontrer ») géométrique.

Dès le 17ème siècle, ce modèle mathématique amène les philosophes à s’interroger sur la validité des résultats de leur propre discipline. Certains l’adoptent, comme Descartes ou Leibniz qui s’interrogent sur la constitution d’une « mathesis universalis », ou Spinoza qui établit son éthique « more geometrico », à la façon d’un géomètre, à partir d’axiomes et de propositions qui se déduisent les unes des autres.

Ce modèle continue à s’imposer au XIXème siècle, au moment où Hegel rédige ces lignes. Mais celui-ci se livre à une critique de ce modèle, et c’est précisément ici qu’il développe cet effort pour revaloriser la philosophie face à cette discipline reine que sont les mathématiques.

Cela l’amène à repenser la notion de vérité, pour faire plein droit aux "vérités philosophiques" :

    Les représentations sur ce point gênent tout particulièrement l’accès à la vérité. Ceci nous donnera l’occasion de parler de la connaissance mathématique, que le savoir non philosophique tient pour l’idéal que la philosophie devrait s’efforcer d’atteindre, bien que ses efforts, jusqu’à présent, soient restés vains

.

C’est cette redéfinition fondamentale, qui sous-tend cette analyse critique des mathématiques, que nous allons à présent examiner.

%9/12
Tout d’abord, il ne faut pas opposer le vrai et le faux comme deux choses absolument distinctes, sans aucun rapport. Il faut se souvenir qu’Hegel propose une nouvelle ontologie, de type dialectique, qui n’est plus fondée sur le principe d’identité ou de contradiction.

Lui soutient à l’inverse que l’identité inclue la différence, le négatif ; qu’une chose doit devenir son contraire avant de revenir en elle-même et de trouver son identité ("seule l’identité qui se reconstitue ou la réflexion dans l’être autre en soi-même – et non une unité originelle ou immédiate est le vrai"1) ; et qu’entre deux termes opposés, on trouve toujours une synthèse qui vient poser leur rapport, et trouver leur identité dans un nouveau terme qui les dépasse tout en les incluant.

Or ce mouvement dialectique concerne également le vrai, le faux, et leur opposition :

"Le vrai et le faux font partie de ces notions déterminées qu’en l’absence de mouvement, on prend pour des essences propres, chacun étant toujours de l’autre côté par rapport à l’autre, sans aucune communauté avec lui et campant sur sa position".

"A l’encontre de cela", il faut comprendre que le faux correspond à un moment essentiel, celui de la différence. Aucune identité ne peut se fonder sans différence, ce pourquoi le faux est un moment essentiel du vrai :

    Quand on dit qu’on sait quelque chose faussement, cela signifie que le savoir est en non-identité avec sa substance. Mais précisément cette non-identité est l’acte de différenciation en général, qui est un moment essentiel. Certes, de cette différenciation advient leur identité, et cette identité devenue est la vérité. Mais elle n’est pas la vérité au sens où l’on se serait débarrassé de la non-identité, comme on jette les scories séparées du métal pur, ni non plus comme on rejette l’outil du récipient terminé : la non-identité au-contraire est elle-même au titre du négatif, du Soi-même, encore immédiatement présente dans le vrai.

Ce que Hegel résume ainsi : "il ne faut pas [...] considérer le vrai comme le positif mort qui repose de son côté".

Il prend néanmoins soin de préciser que "ce n’est plus en tant que faux que le faux est un moment de la vérité". Considérés dans leur unité, vrai et faux ne "signifient [plus] ce qu’ils sont en dehors de leur unité". Il ne faut pas tomber dans les pièges dans lesquels nous mène le langage : "de même que l’expression de l’unité du sujet et de l’objet, du fini et de l’infini, de l’être et de la pensée, etc., a ceci de fâcheux qu’objet et sujet, etc., signifient ce qu’ils sont en dehors de leur unité, et qu’on ne les prend donc pas dans l’unité au sens de ce que cette expression dit, de même, ce n’est plus en tant que faux que le faux est un moment de la vérité".

Mais il reste que si le vrai contient en lui-même le faux (convenablement compris, c’est-à-dire dans leur rapport dialectique), le mode d’exposition de la vérité que propose Hegel dans la Phénoménologie de l’esprit est adéquat.

Il s’agit d’exposer ici la manière dont l’Esprit s’apparaît progressivement à lui-même. Or l’« apparaître » n’est pas l’apparence, au sens d’illusion trompeuse, mais constitue un moment essentiel de la vérité : "l’apparition est le naître ou le disparaître qui lui-même ne naît ni ne disparaît, mais est en soi et constitue l’effectivité et le mouvement de la vie de la vérité". Là encore, le projet d’une phénoménologie, science de l’apparaître (de l’Esprit à lui-même), se voit légitimé.

Après cette considération générale de la vérité, Hegel vient à en examiner un type spécifique : la vérité mathématique.

En effet, les propositions mathématiques semblent échapper à ce processus dialectique. Combien font 2+2 ? A quoi est égal le carré de l’hypoténuse d’un triangle rectangle ? Il semble qu’à ce type de question, il faille apporter une "réponse nette", c’est-à-dire reposant sur une conception classique de l’identité, du principe de contradiction, et échappant à toute dialectique. « 2 + 2 = 4 », « le carré de l’hypoténuse est égal à la somme des carrés des deux autres côtés » : le problème est résolu, sans qu’il soit nécessaire de se référer à une dialectique d’aucune sorte. C’est la même chose d’ailleurs pour les vérités historiques (date de naissance de César, etc).

On le voit, Hegel doit nécessairement répondre sur ce point : si la mathématique, cette discipline reine, échappe à la dialectique, cela ne remet-il pas en question cette dernière ? L’ontologie hégélienne, révolutionnaire dans sa nouveauté, ne survivrait pas à cette confrontation.

La dialectique, et la nouvelle ontologie qu’elle porte en elle, vient bouleverser le champ du savoir. Il lui faut proposer une nouvelle configuration du savoir, fixer la place et les limites de chaque science, montrer en quoi celles-ci sont traversées par un processus dialectique, ou si ce n’est pas le cas, expliquer pourquoi.

Une tâche d’une ampleur infinie, riche de promesses mais aussi de difficultés qui pourraient sembler insurmontables : n’est-ce pas là un projet risible que de présenter une critique des mathématiques, au regard des succès de cette discipline, et ce sans être un mathématicien professionnel ?

Il ne s’agit naturellement pas pour lui de nier la vérité des mathématiques. Il s’agit de montrer que "la nature d’une vérité de ce genre est différente de celle des vérités philosophiques". La question se pose : à quel type de vérité renvoient les propositions mathématiques ?

%10/12
Il procède en réalité en deux temps.

Il montre tout d’abord que les mathématiques sont soumises, comme la philosophie, à une dialectique. Loin de consister dans l’énoncé d’un résultat nu, le géomètre doit en présenter une démonstration, lier ce résultat aux raisons qui mènent à celui-ci. Il n’y a donc pas de "réponse nette"1 en mathématique, au sens où l’on donnerait une réponse immédiate à un problème, sans processus dialectique.

En mathématique, la démonstration est un moment essentiel du vrai, même si on n’en a pas toujours conscience : "même dans la connaissance mathématique, le caractère essentiel de la démonstration est encore loin d’avoir pour signification et nature d’être un moment du résultat proprement dit : dans ce résultat, elle est au contraire quelque chose qui est passé et qui a disparu".

Avec la démonstration apparaît la dialectique : celle qui lie un résultat aux raisons qui mènent à celui-ci, mais aussi celle qui unit le sujet (le mathématicien) à l’objet (le théorème) dans le processus de connaissance. En effet, c’est par la démonstration que le géomètre va être convaincu du résultat (le théorème), lui donner son assentiment intérieur : "on ne tiendrait [pas] pour un géomètre celui qui connaîtrait par cœur les théorèmes d’Euclide, c’est-à-dire de l’extérieur, sans connaître leurs démonstrations, sans les avoir assimilés [...] de manière intérieure".

Or on va retrouver là les rapports dialectiques qui vont jouer entre le sujet et l’objet, la conscience et son objet. Ainsi, "même les vérités nues du genre de celles qui sont citées en exemple ne sont pas exemptes du mouvement de la conscience de soi", ce mouvement dialectique que nous avons décrit plus haut. C’est la même chose pour les "vérités historiques" (date de naissance de César) : "c’est seulement dans la connaissance de celle-ci conjointement à ses raisons qui sera tenue comme quelque chose qui a une valeur vraie, quand bien même [...] seul le résultat nu est censé être ce dont il s’agit".

Concluons : le « résultat nu », la « réponse nette », ne sont, tout comme le savoir immédiat et toute autre figure prétendant échapper à la dialectique, qu’un mythe.

En un second temps, il soutient que la dialectique à l’œuvre dans les mathématiques se caractérise essentiellement par sa pauvreté. De ce fait, cette discipline ne peut prétendre qu’à la saisie d’une vérité inférieure. C’est l’un des premiers stades de la recherche de la vérité, et non son stade ultime, comme on le pense.

Les mathématiques s’appuient sur l’évidence de leurs résultats. Mais cette évidence n’est que le signe de la pauvreté de leur objet :

"L’évidence de cette connaissance défectueuse dont les mathématiques sont fières, et qu’elles arborent du reste aussi pour plastronner face à la philosophie, ne repose que sur la pauvreté de leur fin et sur le caractère défectueux de leur matière, et ressortit donc à une espèce que la philosophie ne peut que dédaigner".

C’est le nombre qui est l’objet propre des mathématiques. Il s’agit de mesurer, dénombrer, faire des opérations, réduire des figures à des grandeurs... Or, qu’en penser du point de vue dialectique ?

"– La fin qu’elles visent, ou encore leur concept est la grandeur. C’est-à-dire exactement le rapport inessentiel, sans concept. C’est pourquoi le mouvement du savoir se déroule à la surface, ne touche pas la chose même, ne touche pas l’essence ou le concept, et pour cette raison, n’est pas un concevoir".

D’autre part, l’objet propre de la géométrie est l’étude des figures dans l’espace. On pourrait soutenir que l’espace est ce qu’il y a de plus concret. Mais une réflexion rapide suffit pour prendre conscience qu’il s’agit là d’un élément abstrait, ineffectif. Or souvenons-nous que pour Hegel, la vérité réside dans le concret, l’effectif.

"- La matière à propos de laquelle les mathématiques avèrent ce réjouissant trésor de vérités est l’espace et l’Un. L’espace est l’existence dans laquelle le concept inscrit ses différences comme en un élément vide et mort, ou elles sont tout aussi bien immobiles et sans vie. L’effectif n’est pas, comme on le considère en mathématiques, une spatialité".

De ce fait, "ni la contemplation sensible concrète, ni la philosophie ne peuvent se satisfaire de cette ineffectivité qui est celle des choses des mathématiques".

Cette abstraction vient remettre en cause jusqu’au caractère dialectique des mathématiques, qu'il avait pourtant affirmé dans les paragraphes précédents :

    Dans ce genre d’élément ineffectif il n’y a aussi, au demeurant, que du vrai ineffectif, c’est-à-dire que des propositions fixées, mortes ; on peut s’arrêter à chacune d’entre elles ; la suivante recommence pour soi de nouveau sans que la première se soit elle-même transportée jusqu’à la suivante, et sans que de la nature de la chose même soit née ainsi, de cette manière, une connexion nécessaire [pour cette raison] le savoir court le long de la ligne de l’identité. Ce qui est mort en effet, ne se mouvant pas soi-même, ne parvient pas à des différences d’essence, ne parvient pas à l’opposition ou à la non-identité essentielle, ni donc non plus au passage de l’opposé dans l’opposé, au mouvement qualitatif, immanent, au mouvement autonome

Il nous semble que ce qu’il affirme ici, c’est plutôt qu’on trouve dans cette discipline un type de dialectique inférieure, appauvrie, à l’image de son objet, le nombre, et du type de vérité auquel elle accède.

En raison de leur abstraction, les mathématiques ne représentent que l’un des premiers stades du savoir, et non le stade ultime. En réalité, cette discipline ne parvient pas à s’élever à la hauteur du concept, qui est l’objet propre de la philosophie, et qui en dernier ressort détermine les relations mathématiques elles-mêmes :

"Car c’est uniquement la grandeur, la différence inessentielle, que les mathématiques considèrent. Elles font abstraction du fait que c’est le concept qui scinde l’espace en ses dimensions et qui détermine les liaisons de celles-ci et au sein de celles-ci."

Tandis que "la philosophie [...] n’examine pas de détermination inessentielle ; elle examine la détermination dans la mesure où elle est détermination essentielle. Son élément et contenu, ce n’est pas l’abstrait ou l’ineffectif, mais l’effectif, ce qui se pose soi-même et vit en soi-même, l’existence dans son concept".

On comprend alors pourquoi il affirmait que "la nature d’une vérité de ce genre est différente de celle des vérités philosophiques". La vérité mathématique ne saurait porter sur le concept, c’est la philosophie qui a ce privilège ; de ce fait, elles ne se contredisent pas, et la mathématique ne saurait constituer en rien une objection ni un modèle pour la philosophie.

%11/12
Quelle est la méthode qui va sous-tendre cette science que propose Hegel ? De la même manière que la logique sous-tend de nombreuses autres disciplines, quelle va être la logique de cette Phénoménologie de l’Esprit ?

Il n’y a pas en réalité de méthode, de logique, qui puisse être considérée comme un préalable nécessaire à la constitution et la compréhension de cette science, du système hégélien. En effet, une méthode, une logique n’est nécessaire que lorsqu’on dissocie la forme et le contenu d’une science, la discipline elle-même et son objet, la vérité, et que l’une cherche à s’organiser pour atteindre l’autre.

Or ici, il ne s’agit pas de constituer une nouvelle « discipline », mais d’exposer la vérité elle-même, de la laisser s’auto-déployer elle-même en ses différentes figures : "la vérité est le mouvement d’elle-même chez elle-même, tandis que cette méthode, c’est la connaissance extérieure à elle-même. C’est pourquoi elle est caractéristique des mathématiques"1.

Il n’y a donc pas besoin d‘une Logique comme fondement préalable au système, ou, ce qui revient au même, "la méthode [...] n’est rien d’autre que la construction de l’ensemble érigé dans son essentialité pure", c’est-à-dire l’exposé du système lui-même : "son exposition proprement dite est la Logique elle-même".

C’est pourquoi c’est le système qui est la forme la plus appropriée pour exposer la vérité, et non l’appareil mathématique : "l’Etat scientifique que les mathématiques nous ont légué – tout l’appareil d’explication, de subdivision, d’axiomes, de séries de théorèmes, avec leurs démonstrations, leurs principes, et les déductions et conclusions qu’on en tire – à déjà, au moins dans l’opinion elle-même, passablement vieilli".

Ce même reproche peut être adressé, en philosophie, à la scolastique et même à la "conversation raisonneuse" socratique ou péripatéticienne : ce ne sont pas des formes adéquates pour exposer la vérité :

"La manière qui consiste à mettre en place une proposition, et à en avancer des justifications, et pareillement à en avancer d’autres pour réfuter la proposition contraire, n’est pas la forme dans laquelle la vérité peut entrer en scène".

Pourtant, lorsqu’on pense à un système, ou à une science, on imagine des tableaux, des schémas, qui vont réduire la richesse du Tout à quelques éléments. Ainsi par exemple, la table des catégories que l’on retrouve dans la Critique de la Raison pure de Kant laisse une impression durable au lecteur, donne un caractère scientifique à l’ouvrage.

Ce n’est pas là ce qu’entend proposer Hegel : il ne s’agit pas pour lui de récapituler le devenir de l’Esprit en quelques tableaux. On perdrait là, dans cette abstraction et ce formalisme, la vie du Concept :

Certes, "la forme véritable a été mise en place dans son contenu véritable et le concept de la science a surgi", mais "on ne peut pas davantage tenir pour quelque chose de scientifique un usage de cette forme par lequel nous la voyons rabaissée au rang de schéma sans vie, de schème fantomatique stricto sensu, et l’organisation scientifique ramenée à un tableau".

On oppose parfois la science et la vie ; on nie que la richesse du Tout puisse se laisser enfermer dans un système : c’est précisément à cette opposition que Hegel s’attaque ici. Il essaie de préciser les conditions dans lesquelles celle-ci peut être levée, de décrire un système qui puisse rendre compte de la richesse de la vie.

Pour cela, il procède en deux temps :

Il commence par donner raison aux détracteurs de ce type de système ou de science, qui fonctionne par schéma, et qui n’est que le fruit d’une « pensée d’entendement » :

    L’entendement de type tabulaire conserve pour soi [non pas au sens de préserver mais de confisquer] la nécessité et le concept du contenu, ce qui fait le concret, l’effectivité et le mouvement vivant de la chose qu’il range dans son tableau, ou, plus exactement, on ne dira pas qu’il conserve tout cela pour soi mais qu’il ne le connaît pas ; car s’il avait cette intelligence des choses, il la montrerait sans doute. Il n’en connaît pas même le besoin : car alors, il mettrait sa schématisation en sourdine, ou du moins, ne pourrait grâce à elle, en savoir plus que ce que donne une table des matières ; cet entendement ne donne que la table des matières, un sommaire du contenu, mais il ne fournit pas le contenu proprement dit

.

Ces détracteurs ont bien compris que ce qui échappe à ce type de construction, c’est la vie :

    L’entendement formel laisse aux autres le soin d’ajouter cette chose capitale [qu’est la vie]. Au lieu d’entrer dans le contenu immanent de la chose, il regarde toujours le tout de très haut, et se tient au-dessus de l’existence singulière dont il parle, c’est-à-dire, ne la voit pas du tout. Tandis que la connaissance singulière exige au contraire qu’on se remette à la vie de l’objet, ou ce qui revient au même, qu’on ait devant soi et qu’on énonce la nécessité intérieure de celui-ci

.


Hegel donne un exemple frappant de cette science qui nie la vie, en même temps qu’elle appauvrit le Tout en le réduisant à quelques éléments dans un système : la physique ou « philosophie de la nature ». Ce "formalisme de la nature [...] nous enseigne par exemple que l’entendement c’est l’électricité". On sait en effet que les impulsions nerveuses du cerveau (ondes cérébrales) sont de nature électrique. On croit alors comprendre la vérité de l’esprit, alors que ce formalisme, en réduisant par abstraction celui-ci à un seul élément, en a fait disparaître l’essentiel : la vie même.

Contre "cet entendement mort et connaissance extérieure", ce "savoir sans vie", il défend, en un second temps, un autre modèle. Lequel ?

12/12
En réalité, il faut comprendre que "la science ne peut s’organiser qu’à travers la vie propre du concept ; en elle, cette déterminité [...] est l’âme du contenu accompli qui se meut elle-même"1.

Hegel soutient la possibilité de présenter un système qui présente cette vie même, qui préserve celle-ci. Pour cela, cette science doit reproduire le mouvement même du Concept, ce mouvement dialectique que nous avons déjà examiné et qu’ici Hegel décrit ainsi :

"Le mouvement de ce qui est consiste, d’une part à devenir à ses propres yeux un autre [moment négatif de la différenciation] et à devenir ainsi son propre contenu immanent [moment de la réflexion sur soi], et d’autre part ce qui est reprend en soi-même ce déploiement ou cette existence qui est la sienne [moment du retour à soi]"

Nous ne sommes alors plus ici dans une pensée d’entendement, mais nous nous élevons au stade de la Raison, et de la philosophie en sa forme la plus haute, la philosophie spéculative :

"Cette nature de la méthode scientifique, savoir d’une part qu’elle n’est pas séparée du contenu, d’autre part qu’elle se détermine par soi-même son propre rythme, a son exposition propre, comme nous l’avons déjà rappelé, dans la philosophie spéculative".

Le système que Hegel appelle de ses vœux, la science du Tout, doit donc préserver ce mouvement que l’on trouve au cœur de toute chose, que Hegel définit comme la « nécessité logique » puis « le spéculatif » :

"Par le simple fait que la substance [...] est en elle-même sujet, tout contenu est sa propre réflexion en soi-même". Ainsi "c’est dans cette nature propre à ce qui est, et qui est d’être dans son être son propre concept, que réside tout simplement la nécessité logique ; elle seule est le rationnel et le rythme du tout organique ; elle est tout autant savoir du contenu, que le contenu est concept ou essence – ou encore, elle seule est le spéculatif".

Il faut penser ce mouvement dialectique, le reproduire en soi-même afin de le saisir : "ce qui importe dans l’étude menée par la science, c’est d’assumer l’effort, la fatigue du concept".

Or il est difficile de sortir d’une pensée de l’entendement, basée sur la notion simple d’identité, pour s’élever jusqu’au spéculatif, jusqu’à une pensée du mouvement, c’est-à-dire de la contradiction et de la synthèse.

"Mais si l’on veut bien considérer que ce genre de pensée a un contenu [...] elle a également un autre côté qui lui rend le concevoir difficile".

Hegel décrit la difficulté de penser dialectiquement, c’est-à-dire de penser le mouvement inscrit au cœur de la logique, dans plusieurs paragraphes dont celui-ci :

    Dès lors que le concept est le Soi-même propre de l’objet, qui se présente comme le devenir de cet objet, ce n’est pas un sujet immobile et au repos qui porte les accidents sans sourciller, mais le concept qui se meut et reprend en soi ses déterminations. Dans ce mouvement, le sujet immobile en question est lui-même perdu [...] En sorte que la terre ferme que la pensée qui raisonne trouve chez le sujet au repos vacille, et que seul ce mouvement lui-même devient l’objet.


Hegel montre que le langage lui-même, qui est l'élément dans lequel la pensée se déploie, fait signe vers le stade spéculatif. Autrement dit, il y a une dialectique à l’œuvre au cœur même du langage.

Que voit-on en effet dans un jugement simple, tel que « le chat est noir » ? Un prédicat (noir) attribué à un sujet (le chat). Les deux termes sont séparés, bien distincts, conformément au principe d’identité, et à ce qu’exige l’entendement.

Néanmoins considérons cet autre jugement : « Dieu est l’être ». On croit qu’on a là deux termes distincts. Mais en réalité d’un certain côté, le sujet (Dieu) semble se perdre tout entier dans ce prédicat (l’être). Il semble qu’on ait ici : Dieu n’est autre chose que l’être et il n’y a donc pas de Dieu :

    Dieu semble cesser d’être ce qu’il est d’après la disposition de la proposition, savoir, le sujet ferme et consistant. La pensée, au lieu de continuer à avancer dans le passage du sujet au prédicat, et étant donné que le sujet se perd, se sent au contraire freinée et rejetée vers la pensée du sujet, puisqu’elle en déplore l’absence [...] La pensée perd tout autant le ferme terrain et le sol objectal qu'elle avait chez le sujet, qu’elle est renvoyée à eux dans le prédicat, et revient, au sein même de celui-ci, non pas en soi-même mais dans le sujet du contenu.


On voit ici le mouvement dialectique à l’œuvre, dans le langage même, entre le sujet et le prédicat. Ce que Hegel généralise ainsi :

"La nature du jugement, ou tout simplement de la proposition qui inclut en soi la différence du sujet et du prédicat est détruite par la proposition spéculative, et [...] la proposition d’identité que devient la première contient le contrecoup qui réagit à ce premier rapport".

D’où un "conflit entre la forme d’une proposition en général et l’unité du concept qui détruit cette forme".

Une synthèse vient servir d’issue dialectique à ce conflit : "dans la proposition philosophique, l’identité du sujet et du prédicat ne doit pas anéantir leur différence, qui est exprimée par la forme de la proposition, et leur unité doit surgir, au contraire, comme une harmonie".

On le voit : la forme même du langage révèle que le principe d'identité, en dépit de son évidence et simplicité apparente, doit être dépassé, vers quelque chose d’autre.

Ainsi, il faut apprendre à penser différemment, dialectiquement, « par-delà le principe d’identité ou de contradiction », et c’est difficile. C’est ce qu’il faut s’efforcer de faire, pour sortir de la simple « pensée représentative » ou du « bon sens » de l’opinion commune.

Hegel termine en ironisant sur le bon sens, et en espérant que son ouvrage recevra un accueil favorable :

"Ce qu’il y a d’excellent dans la philosophie de notre temps place sa valeur même dans la scientificité et [...] ne se fait [...] reconnaître et valoir que par elle. Et donc je peux également espérer que cette tentative de revendiquer la science pour le concept et de l’exposer dans cet élément caractéristique d’elle qui est le sien saura se frayer une entrée par la vérité intérieure de la chose. Il faut que nous soyons convaincus que le vrai a pour nature de faire irruption quand son temps est venu, et qu’il n’apparaît que lorsque ce temps est venu".


Après ce travail préparatoire, cette étude de la préface de la Phénoménologie de l’esprit, nous sommes mieux armés pour entrer dans le corps de l’ouvrage, et découvrir les différents stades de l’auto-déploiement de l’Esprit.

C’est ce que nous nous proposons de faire à présent.
%%%%%%%%%%%%%%%%%%%%%%%%%%%%%%%%%%%%%%%%%%%%%%%%%%%%%%%%%%%%%%%%%%%%%%%%%%%
