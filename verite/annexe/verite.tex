
%%%%%%%%%%%%%%%%%%%%%
\chapter{Vérités éternelles chez saint Augustin}
%%%%%%%%%%%%%%%%%%%%%

%%%%%%%%%%%%%%%%%%%%%%%%%
%\section{Encyclopédie de la philosophie}
%%%%%%%%%%%%%%%%%%%%%%%%%
%vérités éternelles chez saint Augustin,
Vérités logiques et mathématiques dans le
sens où elles sont nécessaires et donc
immuables. Tous les penseurs du Moyen
Age les ont considérées contenues dans
l'intelligence de Dieu (et d’un point de
vue chrétien, dans le « Verbe »), mais
soustraites à son choix. Des penseurs, tels
que Duns Scott et Guillaume d’Ockham,
les ont au contraire conçues comme
dépendantes du choix de Dieu et des
valeurs morales. Descartes, par ailleurs, a
étendu cette dernière thèse aux prétendues
vérités éternelles (et aux « essences »),
mais par souci de prudence, il n’a
fait que peu de publicité à cette doctrine
(il l’a exposée en appendice dans ses
{\it Méditations métaphysiques}). Pour lui, si
on ne croit pas que Dieu ait été créé ou
ait été décidé librement, alors de même
on ne croit pas aux vérités logiques et
%1649
mathématiques. On en vient donc à soumettre
Dieu à une sorte de destin et à priver
les vérités logiques et mathématiques
du caractère de l’omnipotence. Si Dieu
est vraiment omnipotent, il s’ensuit qu’il
aurait tout à fait pu décider, par exemple,
que la somme des angles internes d’un
triangle ne soit pas égale à 180°, et ainsi
de suite. Et c’est pour cela, que nous,
êtres humains, ne parvenons pas à le
comprendre. Nous ne parvenons précisément
pas à le comprendre parce que Dieu
a voulu au contraire que cet axiome (et
d’autres analogues)  vaillent comme
vérités nécessaires et premières pour
nous. Mais il l’a choisi d’une façon totalement
arbitraire, d’une façon indifférente,
dans sa totale et absolue souveraineté.
D'une manière analogue, les choses ont
été créées « bonnes », comme le dit la
Bible, mais non pas déjà parce qu’elles
correspondent à un modèle présupposé,
mais par le simple fait qu’elles ont été
voulues par Dieu telles qu’elles sont. Et,
de toute façon, elles auraient été semblables
si, par hasard, il avait décidé de
les « produire ». Cette doctrine sans précédent
fut objet de discussion tout au long
du {\footnotesize XVII}$^\text{e}$ s., de Spinoza à Nicolas Malebranche
en passant par Leibniz et Pierre
Bayle.

%%%%%%%%%%%%%%%%%%%%%%%%%%%%%%%%%%%%%%%%%%%%%%%%%%%%%%%%%%%%%%%%%%%%%%%%%%%%%%%%%%%%%
