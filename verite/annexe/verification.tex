
%%%%%%%%%%%%%%%%%%%%%
\chapter{Vérification}
%%%%%%%%%%%%%%%%%%%%%

\begin{itemize}[leftmargin=1cm, label=\ding{32}, itemsep=1pt]
\item {\bf \textsc{Étymologie} :} latin {\it verificare},
« vérifier ».
\item {\bf \textsc{Science et épistémologie} :} procédé
qui permet d'établir la vérité d'une proposition (d’une
hypothèse, etc.).
\end{itemize}

Dans le cas des sciences formelles, la
vérification est démonstrative et de
l'ordre du calcul. Dans les sciences
empiriques, il est discutable de parler de
« vérification ». Karl Kopper a montré
qu'on peut établir expérimentalement la
fausseté d’une hypothèse, alors qu'il
n'est pas possible d'en établir la vérité
({\it cf.} Falsifiabilité). Lorsque l'hypothèse a
passé avec succès un contrôle qui aurait
pu la « falsifier », il vaut mieux parler,
plutôt que de « vérification », de confirmation
ou de corroboration, lesquelles
sont toujours « jusqu'à preuve du
contraire ».

\begin{itemize}[leftmargin=1cm, label=\ding{32}, itemsep=1pt]
\item {\bf \textsc{Terme voisin} :} confirmation ;
corroboration ; preuve.
\item {\bf \textsc{Terme opposé} :} réfutation.
\item {\bf \textsc{Corrélats} :} connaissance ;
expérience ; falsifiabilité ; science.
\end{itemize}

%%%%%%%%%%%%%%%%%%%%%%%%%%%%%%%%%%%%%%%%%%%%%%%%%%%%%%%%%%%%%%%%%%%%%%%%%%%
