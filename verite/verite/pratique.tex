
\section{Pratique de la philosophie}

\subsection{Vérité}

{\footnotesize
\begin{itemize}[leftmargin=1cm, label=\ding{32}, itemsep=1pt]
\item {\bf \textsc{Étymologie} :} latin {\it veritas},
de {\it verus}, vrai.
\item {\bf \textsc{Sens ordinaire} :} {\bf 1} Caractère
conforme à la réalité. {\bf 2} Proposition dont l'énoncé
exprime la conformité d'une idée avec son objet. {\it Les
vérités mathématiques.} {\bf 3} Conformité d'un récit,
d'une relation avec un fait. {\it Trahir la vérité.}
{\bf 4} Sincérité, bonne foi {\bf Loc} {\it En vérité :}
assurément. {\it À la vérité :} en fait. {\it Dire à
quelqu'un ses quatres vérités :} lui dire sans détour ce qu'on pense de lui.{\it Vérités premières :} banalités.
%\item \hfill Sources : \textsc{Hachette} et \textsc{Robert}
\end{itemize}
}

Alors même qu'il s'efforce de le restituer
avec fidélité, le vrai n’est pourtant pas le
réel. Tandis que la réalité est par définition
indépendante de l’homme, la
vérité est toujours de l’ordre du discours
ou encore de la représentation.
Préoccupation essentielle de la
recherche philosophique, la vérité n’est
donc ni un fait, ni un donné. Au
contraire, elle doit toujours être recherchée.
Nous sommes alors renvoyés au
problème de ses conditions d'accès, et à
celui des critères du jugement vrai. La
vérité constitue également une exigence
ou encore une valeur.

% 461
\subsubsection{La recherche de la vérité en question}

Le projet de recherche de la vérité est
constitutif de la réflexion philosophique,
et c’est par lui que, dès l’origine, celle-ci
s'est définie dans la Grèce antique.

La philosophie de Platon illustre à merveille
la triple idée autour de laquelle se
formule le projet de vérité. {\bf 1.} Ce projet a
un sens : l'effort de l'esprit humain pour
parvenir à une authentique vérité peut
être couronné de succès. {\bf 2.} Une vérité
n'est telle que si celui qui l'énonce ne
répète pas comme un perroquet un savoir
étranger. Tel est le sens de la maïeutique
de Socrate : on n’enseigne pas la vérité
comme on remplirait un vase vide;
connaître la vérité, c'est, par un véritable
« accouchement de l'esprit » (maïeutique),
la retrouver comme au fond de soi, c’est-à-dire
se l’approprier. {\bf 3.} La vérité se définit
par sa permanence et son universalité,
et en cela ne doit nullement se confondre
avec la relativité et l’inconstance des opinions
humaines. Il faut donc distinguer
vérité et connaissance. Ce qui est vrai
aujourd’hui le sera demain et toujours {\bf --}
et l’est pour tous {\bf --} ou ce n’est pas, à
proprement parler, une vérité. Ce n'est
donc pas parce que la variabilité des opinions
est un fait qu’une vérité objective et
universelle est impossible. Ce qui est
impossible, au contraire, c'est d'affirmer
« à chacun sa vérité », puisqu'on l’affirme...
comme une vérité. Cela n’empêche
pas qu’on puisse légitimement dire
« à chacun ses opinions », mais il faut opérer
une distinction critique entre l’opinion,
ou vérité prétendue, et la vérité ou
opinion certifiée.

Une telle recherche de la vérité peut-elle
espérer aboutir ? C’est ce que conteste le
scepticisme, lequel veut substituer à
l'affirmation « dogmatique » de la possession
du vrai une attitude de doute et
d'examen.

Il serait évidemment contradictoire de
dire que le scepticisme est... dans le vrai,
et l’on n’a pas manqué de lui reprocher
cette apparente incohérence. Il a cependant
une valeur, qui est de nous inciter à
la modestie. En nous enseignant que nos
croyances ne sont pas ces vérités assurées
pour lesquelles nous les prenions, le
scepticisme nous empêche, à sa manière,
de nous laisser bercer par les charmes
sécurisants des « vérités toutes faites », des
fausses certitudes, c'est-à-dire de l’opinion.

\subsubsection{Définitions et critères}

{\bf 1.} Définitions de la vérité. Tout le monde
semble s'accorder depuis Thomas
% 462 a
d’Aquin, au {\footnotesize XIII}$^\text{e}$ siècle, pour définir la
vérité comme correspondance ou adéquation :
adéquation entre l'intelligence
qui conçoit, entre l'esprit et la réalité. En
d’autres termes, la proposition « il neige »,
par exemple, est vraie si et seulement si,
en fait, il neige.

Cette définition comporte une conséquence
importante : la vérité est une propriété
du langage, non du réel. « Vrai » et
« faux » sont des qualificatifs qui s’appliquent
non pas à des choses, mais à des
propositions. On parle pourtant d'or
« faux », de « vrai » ami, etc. Mais l'or
« faux » est tout aussi réel que l'or véritable.
Seulement, ce n'est pas de l'or,
mais, par exemple, du cuivre doré. Ce qui
est « faux » alors, c'est la proposition
implicite : « Ceci est de l'or ».

La définition de la vérité comme correspondance
ne fait pourtant pas l’unanimité.
On peut lui opposer d’autres définitions,
notamment celle qui caractérise la
vérité en termes de cohérence. Selon
cette conception, une théorie scientifique,
par exemple, sera dite vraie, non pas si
elle correspond aux faits, mais si les propositions
qui la constituent forment un
ensemble cohérent, c'est-à-dire si elles
sont compatibles entre elles. Cependant,
la théorie de la « vérité-cohérence »
semble difficile à soutenir : l'accord de la
pensée avec elle-même est bien une
condition nécessaire de la vérité (car on
ne peut se contredire et énoncer une
vérité), mais non une condition suffisante.
Nos pensées peuvent être entre elles
cohérentes et en contradiction avec la
réalité.

Mais la théorie de la « vérité-correspondance »
suscite elle aussi des difficultés.
D'une part, l’idée de correspondance
suppose que les faits auxquels nos propositions
ou nos croyances doivent correspondre
sont disponibles indépendamment
de notre langage. Or, rien n'est
moins sûr. Toute tentative de parler du
monde n’en est-il pas déjà une interprétation ?
D'autre part, qu'est-ce, pour une
croyance, d’être « en accord » avec les
faits? Cela signifie-t-il qu’une pensée
vraie est la copie fidèle du réel, ou,
comme le soutient le pragmatisme ({\it cf}.
W. James), qu’elle permet d'agir efficacement
sur lui ?

Il importe sans doute que nos idées
augmentent notre puissance d'agir ; mais
le pragmatisme a tort de faire du succès
une règle du vrai. Cette règle, il faut la
chercher, au contraire, dans l’art de la
preuve. Il n’y a pas de vérité sans vérification.
% 462 b
{\bf 2.} Le problème du critère de la vérité. À
quoi reconnaît-on la vérité ? À cette question,
la plupart des philosophes classiques
ont suivi Descartes pour
répondre : à l’évidence des idées vraies.
Cela signifie, comme l’affirmait Spinoza,
que la vérité est {\it index sui}, qu'elle se
montre d'elle-même, par sa seule clarté :
« Qui a une idée vraie sait en même
temps qu'elle est vraie et ne peut douter
de la vérité de sa connaissance » ({\it Éthique},
II, 43).

Mais peut-on tout reconnaître par évidence ?
Ce n’est pas nécessaire. Il suffit
de connaître par évidence les premiers
principes de la connaissance {\bf --} les vérités
premières {\bf --} et d'établir toutes les autres
par démonstration, c'est-à-dire en les
déduisant de proche en proche à partir
des premières. Ainsi, pour Descartes,
l'intuition {\bf --} c'est-à-dire l'évidence {\bf --} et
la déduction sont les deux seules voies
qui conduisent à la vérité. L'ordre du vrai
aurait donc un modèle : l’ordre géométrique,
tel qu'Euclide, dès l'Antiquité,
l'avait formalisé dans ses {\it Éléments de géométrie}.

Mais le critère de l'évidence s'est heurté à
deux objections. La première fut formulée
par Leibniz : l'évidence est un critère peu
fiable, car trop subjectif. Elle se définit par
le fait que la représentation d'une idée
s'accompagne d'un sentiment de certitude ;
mais quel crédit accorder à ce sentiment ?
Chacun de nous a fait l'expérience
d'évidences trompeuses.
Comment, alors, peut-on distinguer l'évidence
de ses faux-semblants ?

La deuxième objection résulte du développement
des sciences expérimentales :
on ne peut traiter le monde physique
comme un système mathématique et se
contenter de déduire ses lois à partir
d’axiomes « évidents ». Dans le domaine
des sciences de la nature, le critère de la
vérité doit être l'observation des faits.

Il faudrait donc distinguer deux types de
critères : les vérités purement formelles
d'une part, les vérités expérimentales ou
empiriques d’autre part.

\subsubsection{La vérité comme valeur}

Pourquoi donc vouloir la vérité ? Vaut-elle
même d'être recherchée ? Ne peut-on
lui opposer des valeurs plus hautes,
la vie par exemple? Nietzsche osa
poser ces questions radicales. Leur
mérite est au moins d’obliger à assumer
le caractère moral de l'exigence de
vérité. La vérité est un choix : nous pouvons
vouloir l’erreur, l’illusion, le mensonge,
parce que nous pouvons aimer
% 463 a
d’autres choses plus que la vérité (le
plaisir, le pouvoir, l’action...) ; et parce
que nous pouvons aussi refuser de voir
dans l'effort de la raison vers la vérité le
signe de notre dignité d'hommes. Descartes
lui-même reconnaissait qu’on
peut nier l'évidence. En ce sens, le problème
de la vérité n’est pas seulement
de la définir ou d’en énoncer les conditions,
mais relève avant tout de notre
liberté.

\begin{itemize}[leftmargin=1cm, label=\ding{32}, itemsep=1pt]
\item {\bf \textsc{Terme voisin} :} validité.
\item {\bf \textsc{Terme opposé} :} fausseté ; mensonge.
\end{itemize}

\subsection{Vérité de fait / Vérité de raison}

Quand un énoncé est vrai parce qu'il
correspond au réel qu'il décrit, il s’agit
{\it d'une vérité de fait}. Quand un énoncé
est vrai en vertu des relations logiques
entre ses termes, c’est une {\it vérité de raison}.
Ainsi a-t-on toujours besoin, pour
savoir qu’une proposition est une vérité
de fait, de la confronter à la réalité ; alors
qu’une simple analyse des termes dans
lesquels elle est énoncée suffit à établir
une vérité de raison. Il s'ensuit également
qu'une vérité de fait aurait pu ne
pas être vraie, alors qu'il est impossible
qu'une vérité de raison ne le soit pas.
Cette distinction correspond à celle que
Leibniz établit entre « vérité contingente »
et « vérité nécessaire », ou encore
à celle que Hume fait entre « relations
de fait » et « relations d'idées ».

\subsection{Vérité formelle / Vérité matérielle}

Cette distinction, due au philosophe et
logicien contemporain Robert Blanché,
invite à faire la différence, dans une
déduction, entre la vérité que renferme
chacune des propositions considérées
isolément ({\it vérité matérielle}) et la validité
({\it vérité formelle}) qui les enchaîne
ensemble jusqu’à la conclusion. Il est
ainsi possible que des propositions
vraies {\it matériellement} soient reliées par
un raisonnement {\it formellement} faux, ou,
à l'inverse, que des propositions fausses
matériellement soient reliées par un raisonnement
formellement vrai.

\begin{itemize}[leftmargin=1cm, label=\ding{32}, itemsep=1pt]
\item {\bf \textsc{Textes clés} :} Platon, {\it Ménon et
République} (VII ; E. Kant, {\it Préface
de la deuxième édition de la Critique
de la raison pure}; F.
Nietzsche, {\it Le Livre du philosophe};
M. Heidegger, {\it De l'essence de la
vérité} ({\it Questions} I).
\item {\bf \textsc{Corrélats} :} certitude ; connaissance ;
démonstration ; erreur ; évidence ;
intuition ; preuve ; réalité ;
science ; validité.
\end{itemize}

%%%%%%%%%%%%%%%%%%%%%%%%%%%%%%%%%%%%%%%%%%%%%%%%%%%%%%%%%
