\subsection{Certitude}
 — \fsb{S. abstr.} \si{Psych.} {\bf 1.} \fsb{S. subje.} (Opp.
{\it doute}$^1$ et {\it opinion}$^1$). État de l’esprit
qui « se croit en possession de la
vérité » (Goblot), qui donne son
assentiment* sans réserve aucune :
« Certitude, certitude, sentiment,
joie, paix » (Pascal, {\it mémorial}) ; « La
certitude n'existe que par l’harmonie de la nature et de l'esprit »
(Lagneau); « L’enthousiasme a toujours engendré la certitude » (Espinas), $->$ Cf. {\it Croyance} et {\it Moral}$^5$.

— \si{Épist.} {\bf 2.} \fsb{S. objec.} Caractère de ce qui
est certain au sens 2 : « C’est à la
simplicité de leur objet que les mathématiques sont redevables de leur
certitude » (D'Alembert). $->$ Terme
équivoque comme le précédent : les
confusions sont fréquentes entre le
sens 1 et le sens {\bf 2.} Cf. {\it Conviction}*.

— \fsb{S. concr.} {\bf 3.} Proposition, croyance
ou opinion certaine$^2$, ou que l’on
croit telle : « La jeunesse veut des
certitudes. »

\subsubsection{Certain}
 — \si{Psych.} {\bf 1.} \fsb{S. subje.} En parlant des
personnes : qui se croit en possession de la vérité : « Si l'homme qui se
trompe dit, au moment où il se
trompe : {\it je suis certain}, quand il a
reconnu son erreur il dit : {\it je me
croyais certain} » (Brochard).

— \si{Log.} \fsb{S. objec.} En parlant des propositions : {\bf 2.} Qui est assurément vrai:
« Il n’y a eu que les seuls mathématiciens qui ont pu trouver quelques
démonstrations, c’est-à-dire quelques
raisons certaines et évidentes »
(Descartes, {\it Méth.}, II); « Ce qui n’est
certifié que par les hommes, peut
être cru comme vraisemblable, mais
non pas comme certain » (Bossuet);
« Pour autant que les propositions
% 33 — cHi
de la mathématique se rapportent
à la réalité, elles ne sont pas certaines » (Einstein). {\it Qqfs.}, en un sens
plus fort : démontré : « S’il ne fallait
rien faire que pour le certain, on ne
devrait rien faire pour la religion:
car elle n’est pas certaine » (Pascal,
234). — {\bf 3.} Dont on est plus ou
moins assuré : « Toutes les autres
choses dont ils se pensent peut-être
plus assurés, comme d'avoir un
corps [etc.], sont moins certaines
[que l'existence de Dieu]; car, encore
qu’on ait une assurance morale de
ces choses... » (Descartes, {\it Méth.}, IV).
Cf. {\it Moral}$^5$.

