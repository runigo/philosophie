\subsection{Vérité}
 — \si{Crit.} et \si{Épist.} {\bf 1.} \fsb{S. abstr.} Caractère de
ce qui est vrai*. $->$ On peut {\it dist.} : 1° vérité {\it formelle}$^3$ : «
C’est dans l’accord avec les lois de l’entendement que consiste le formel de
la vérité » (Kant, {\it R. pure}, Dial., introd., 1), et vérité {\it
matérielle}$^1$ : « Humainement parlant, définissons la vérité : ce qui est
% 194
énoncé tel qu'il est » (Voltaire) ; « La vérité, de qq. manière qu’on la
définisse, implique l'accord du sujet$^4$ avec l’objet$^5$ » (Hamelin) ; 2°
vérité {\it mathématique} et vérité {\it expérimentale} (cf. {\it Textes
choisis}, II, p. 116) ; 3° vérité {\it absolue} et vérité {\it relative} (cf.
{\it Relativité}*). —  {\bf 2.} \fsb{S. concr.} Ce qui est vrai : « Il y a un
art pour faire voir la liaison des vérités avec leur principe » (Pascal) ; «
Toutes les vérités immuables ne sont que les rapports qui se trouvent entre
les idées, dont l'existence est nécessaire et éternelle » (Malebranche,
{\it Entr.}, X, début). {\it Vérités éternelles} : les vérités de raison,
regardées comme immuables et universelles : « Ces vérités éternelles que tout
entendement aperçoit toujours les mêmes, sont qqc. de Dieu ou plutôt sont
Dieu même » (Bossuet).

— \si{Méta.}  {\bf 3.} Qqfs., réalité : « La vérité de la chose [ la distinction de
l’âme et du corps ] » (Descartes, \si{{\it Méd.}}, préf.) ; « On peut douter
de la vérité des choses sensibles » (id. {\it Princ.}, I, 4). $->$ Tous ces
divers sens sont réunis dans cette phrase de Bossuet : « De vérité en vérité,
vous pouvez aller jusqu'à Dieu, qui est la vérité des vérités, la source de
la vérité, la vérité même, où subsistent les vérités que vous appelez
éternelles, les vérités immuables et invariables, qui ne peuvent pas ne pas
être vérités. »

\subsubsection{Vrai}
 — \si{Crit.} {\bf 1.} Qui mérite l’assentiment* : « Ne recevoir
jamais aucune chose pour vraie que je ne la connusse évidemment être telle
» (Descartes, {\it Méth.}, II) ; « Les choses que nous concevons fort
clairement et fort distinctement sont toutes vraies » ({\it ib.}, IV) ; « Les
rapports que nous pouvons affirmer sont qualifiés de {\it vrais} ou de {\it
faux} selon qu'ils s’accordent ou non avec des lois que nous constatons ou
croyons constater, {\it i. e.} selon que ces lois les impliquent ou les
excluent dans les sujets$^1$ où elles paraissent » (Renouvier) ; « Une
géométrie ne peut pas être plus vraie qu’une autre; elle peut seulement être
plus commode » (Poincaré).

\subsubsection{Sujet}
 — \fsb{S. objec.} \si{Vulg.} {\bf 1.} Ce sur quoi porte la
réflexion, le sentiment, etc. : « Le sujet d’un discours »; « Un sujet de
mécontentement ». — \si{Log. form.}  {\bf 2.} (Opp. : {\it attribut}$^1$ ou
{\it prédicat}*). Dans une proposition le terme dont on affirme ou nie
qqc. : « La logique moderne a été amenée à considérer le sujet comme une
variable dont le prédicat est une fonction » (Couturat). — \si{Méta.}
{\bf 3.} (Opp. : {\it attribut}$^2$). L'être réel qui sert de substrat* aux
attributs : « Toute qualité a son sujet d’inhérence » (Cousin). {\it Ext.}
\si{Jur.} « Sujet de droit », la personne qui en est investie.

— \fsb{S. subje.} \si{Crit.} et \si{Épist.} {\bf 4.} (Opp. :
{\it objet}$^5$). L'esprit qui connaît, par {\it opp.} à la chose connue. Ce
« sujet connaissant » peut être entendu : {\it a)} soit comme le sujet {\it
épistémologique} ou {\it transcendantal} (sujet pur), qui {\it p. e.} chez
Kant n’est autre que l’ensemble des lois {\it universelles} a priori de la
pensée : « Le sujet des catégories ne peut recevoir, du fait seul qu'il les
pense, un concept de lui-même comme objet » (Kant, {\it R. pure}, Dial, II,
1, 4, 2° éd.) ; « C’est un sujet tout réduit à sa fonction d’objectivation
» (Bréhier) ; « Fonder la nécessité des lois physiques sur leur origine
subjective qui permettrait de les déterminer a priori, cela ne se justifie
que si l'on vise le sujet pensant en général, principe d'universalité, et
non le sujet individuel » (Blanché) ; — {\it b)} soit comme le sujet {\it
empirique}, {\it i. e.} le moi individuel : « Une connaissance {\it
subjective} relative à la nature du sujet connaissant n’est pas pour cela
moins valable... C'est au contraire le caractère partial, incarné,
conditionné d’une connaissance qui fonde sa valeur » (E. Grimal, {\it R.
ph.}, 1945, p. 243). $->$ Se
% 180
tenir en garde contre la confusion des sens {\it a} et {\it b}, nettement
accusée dans le dernier texte : « L’irréalité du sujet pur n’excuse point
cette confusion : le sujet pur est pour l’épistémologue une idéalisation
aussi légitime que le point inétendu pour le géomètre » (Blanché). {\it Cf.}
Bachelard qui {\it dist.} « le sujet individuel » du « sujet quelconque » :
« Ce sujet quelconque ne saurait être le sujet empirique livré à l’empirisme
de la connaissance... C'est le sujet rationnel... le sujet de la cité
scientifique ». — \si{Méta.} {\bf 5.} L’existant individuel : « Les sujets
personnels incluent aussi les animaux » (Heidegger).

— \si{Psycho.} {\bf 6.} L'individu soumis
à une observation ou à une expérience : « Interroger un sujet. »

— \si{Pol.} {\bf 7.} L’individu, soumis à l'autorité souveraine de l'État :
« Le peuple est à l’égard des nobles ce que les sujets sont à l'égard du
monarque » (Montesquieu, {\it Lois}, III, 4).


\subsubsection{Formel}
 — {\bf 3.} au sens {\it logique} : la {\it validité} formelle
(opp. : {\it vérité matérielle}) d’une proposition est celle qui relève des
« conditions de l’usage de l’entendement en gén., sans distinction
d'objets » (Kant) : elle se ramène à la non-contradiction ou accord de la
pensée avec elle-même. {\it Logique formelle} : celle qui étudie les
conditions de la validité formelle.

\subsubsection{Matériel}
 — Qui se rapporte à la matière* : {\bf 1.} (opp. :
{\it formel}$^3$) aux sens 1, 2 ou 3 de ce mot. {\it Vérité matérielle} :
celle qui consiste dans l’accord de la pensée avec les données de
l’expérience; — {\bf 2.} (opp. : {\it spirituel}$^1$) au sens 4 : « La
substance étendue est ce qu’on nomme proprement le corps$^1$ ou la substance
des choses matérielles » (Descartes, {\it Princ.}, II, 1).

\subsubsection{Matière}
 — \si{Hist.} {\bf 1.} (Syn. : {\it cause$^4$ matérielle}. Opp. :
{\it forme}$^1$). {\it Chez Aristote et les Scolastiques} (qqfs. {\it matière
première}) : sujet$^3$ indéterminé qui est le support de la forme$^1$ et qui
en fait une réalité concrète : « Un des principes d’Aristote est que la
matière par elle-même, est informe » (Buffon). — {\bf 2.} {\it Chez Kant},
voir {\it Forme}$^2$ :
% 113
« Notre activité intellectuelle élabore la matière ({\it Stoff}) brute des
impressions sensibles en une connaissance des objets » ({\it R. pure},
introd., I).

— \si{Épist.} {\bf 3.} (Opp. : {\it forme}$^6$). Contenu (d'un jugement, d’un
raisonnement, d’une connaissance) : « L’esprit mathématique dédaigne la
matière pour ne s’attacher qu'à la forme pure » (Poincaré). D'où,
{\it ext.} : objet$^3$ ou question dont on traite.

— \si{Méta.} {\bf 4.} (Opp. : {\it esprit}$^5$), Substance qui constitue les
corps$^1$ : « La matière dont la nature consiste en cela seul qu’elle est une
chose étendue, occupe tous les espaces imaginables » (Descartes,
{\it Princ.}, II, 22) ; « L’étendue et la matière ne sont qu'une même
substance » (Malebranche, {\it Entr.}, I, 2) ; « Les éléments de la matière
peuvent se ramener à l'étendue et au mouvement » (Boutroux). {\it Chez saint
Thomas} : « matière sensible », la matière corporelle « en tant que sujet$^3$
des qualités sensibles » ; « matière intelligible », la même « en tant que
sujet de la quantité » ({\it S. th.}, I, 85, 1). {\it Chez Descartes} :
« matière subtile », sorte de fluide formé des parties les plus fines et les
plus mobiles de la matière (cf. {\it Princ.}, IV, 25).

\subsubsection{Spirituel}
 — \si{Méta.} {\bf 1.} (Opp. : {\it matériel}). Qui concerne
l'esprit$^5$, ou qui est de la nature de l'esprit : « Nos maladies
spirituelles » (Bossuet) ; « Ceux-là se trompent, qui croient que la
rébellion du corps n'est cause que des vices grossiers, et non de ceux
% 175
qu’on appelle spirituels, comme l’orgueil et l’envie » (Malebranche,
{\it Écl.}, VIII, rép. 11). — \si{Mor.} {\bf 2.} (Opp. : {\it charnel}). Qui
appartient aux fonctions supérieures de l’esprit$^6$ : « Ce qui est
proprement spirituel, c’est ce qui est intellectuel » (Lachelier, d’après
Bossuet) ; « Primauté du spirituel » (Maritain).

— (En parlant des personnes).  {\bf 3.} Qui s’adonne à la vie spirituelle : «
Il y a de faux spirituels » (Bossuet). — {\bf 4.} \si{Car.} Qui a de l'esprit
au sens 8: « Elle se croit intelligente et spirituelle » (Bourdaloue).

\subsubsection{Esprit}
 — [L. {\it spiritus}, souffle] — \si{Hist.}
{\bf 1.} {\it Autref.} fluide, gaz, matière
subtile. — {\bf 2.} {\it Esprits animaux} : sorte
de « vent très subtil », constitué
par « les plus petites parties du
sang », qui, « montant continuellement en grande abondance du
cœur dans le cerveau, va se rendre
de là par les nerfs dans les muscles
et donne le mouvement à tous les
membres » (Descartes, {\it Méth.}, V).

— \si{Méta.} {\bf 3.} Principe de la vie et de
la pensée, âme$^1$ : « Rendre l'esprit »
— {\bf 4.} (Ctr. : {\it corps}$^3$). Principe de la
pensée, âme$^2$ individuelle ; {\it d'où} :
être immatériel, âme des morts.
{\it Pur esprit}, être immatériel qui n’est
pas lié à un corps$^3$ : « Le premier de
tous les esprits, c’est Dieu » (Bossuet); « Je ne considère pas l'esprit
comme une partie de l'âme$^2$, mais
comme cette âme tout entière qui
pense » (Descartes, 5$^\text{es}$ {\it Rép.}, II, 4);
« Un seul esprit vaut tout un monde »
(Leibniz, {\it Disc. méta.}, 36). — {\bf 5.} (Ctr. :
{\it matière}). Le monde de la pensée$^1$, la
réalité spirituelle en gén. : « Avant
l’homme, l'esprit dormait dans la
nature » (Lagneau); « L'esprit est
le foyer commun qui éclaire et unit
toutes les consciences » (Lavelle).
{\it Chez Hegel} : l'Esprit est l’intériorisation
% 70
de la Nature; l'Esprit {\it subjectif} est le siège des faits psychiques
(âme, conscience, esprit$^6$); l'{\it Esprit objectif} se manifeste
dans le droit, les mœurs, la moralité; l'{\it Esprit absolu}, dans l’art,
la religion et la philosophie.

— \si{Psycho.} {\bf 6.} Conscience, ensemble des phénomènes psychiques :
« Élevons plus haut nos esprits »
(Bossuet); « Par esprit ({\it mind}) nous
entendons ce qui dans l’homme
pense, se souvient, raisonne, veut »
(Reid). — {\bf 7.} (Opp. : {\it sentiment$^4$,
cœur$^3$}). Connaissance, intelligence$^1$ :
« L'esprit comme le cœur a ses idoles»
(Renouvier)). Chez certains, {\it opp.} à
{\it Ame}$^3$ ou à {\it Vie} : « L'Esprit et la Vie,
l'Esprit et l'Ame sont en guerre par
une nécessité naturelle... L'Esprit
est le {\it Dehors} absolu comme l’Ame
est le {\it Dedans} naturel » (Klages;
voir Ame$^3$). — {\bf 8.} Tournure ou orientation d’esprit$^6$
particulière : « Un esprit élevé » ; « L'esprit scientifique » ;
« L'esprit de système ». {\it Esprit de finesse*, de géométrie*} :
voir ces mots. — {\bf 9.} Inspiration fondamentale : « L’esprit de la
monarchie est la guerre » (Montesquieu) ; « C’est dans cet esprit que... ».

— \si{Car.} {\bf 10.} Vivacité de la pensée
et faculté d'exprimer ses idées de
façon ingénieuse et piquante :
« Quand on court après l'esprit, on
n'attrape que la sottise » (Montesquieu).

\subsubsection{Corps}
 — {\bf 1.}  {\it Lato.} Tout ce qui tombe
sous nos sens : « La nature m'enseigne que plusieurs autres corps
% 46
existent autour du mien (Descartes,
Méd., VI); « Il n’y a que la foi qui
puisse nous convaincre qu'il y a
effectivement des corps » (Malebranche). En \si{Méta.}, un corps est
une substance jouissant de deux
propriétés essentielles : l'étendue et
l'impénétrabilité. En \si{Phys.}, la propriété fondamentale des corps est la
masse. — {\bf 2.}  {\it Str.} Espèce chimique :
« Les corps simples ».

— {\bf 3.} Organisme de l’homme ou
de l’animal : « Il n’y a rien que ma
nature m'enseigne plus expressément sinon que j’ai un corps » (Descartes, Méd., VI) : « Il est légitime
de dire : je suis mon corps pour
autant que je reconnais ce corps
comme n'étant pas assimilable à
un objet : c’est ainsi qu’on est amené
à faire intervenir le corps-sujet »
(G. Marcel).

— {\bf 4.} {\it Anal.} Groupe social considéré comme une unité vivante : « Le
corps social »; « Les corps constitués »; « Cet acte d'association [le
contrat$^2$ social] produit un corps
moral et collectif, composé d’autant
de membres que l’assemblée à de
voix » (Rousseau). D’où, en Théol.,
« le corps mystique de J.-C. » :
l'Église. — {\it Esprit de corps} : sentiment d'unité d’un groupe social :
« Ce n’est pas seulement dans le militaire qu'on prend l'esprit de corps »
(Rousseau).

\subsubsection{Objet}
 — [L. {\it ob-jectum}, ce qui est placé devant] — \si{Vulg.}
{\bf 1.} Ce qui s'offre à la vue, chose perçue : « Quel objet pour les yeux
d’une amante ! » (Racine) ; « Les images des objets se forment au fond de
l'œil » (Descartes). — {\bf 2.} Cause d’un sentiment : « Objet de crainte ».
{\it Spéc.}, au {\footnotesize XVII}$^\text{e}$ s., l'être aimé : « Ce cher
objet à qui j'ai pu déplaire » (Corneille). —  {\bf 3.} Matière$^1$ ou but :
« L'objet d’une science » ; « Ame qui étais née pour un objet
immortel » (Bossuet) ; « La passion a toujours un objet » (Bonnet) ;
« L'objet du mariage est d’avoir des enfants » (Buffon).

— \si{Crit.} et \si{Méta.} [Ce qui est placé devant l'esprit; d'où :]. {\bf
4.} {\it Autref.} (cf. {\it Objectif}$^1$), ce qui est pensé, représenté dans
l'esprit : « J’appelle {\it objet} ce qui dans la représentation s'offre
comme le terme immédiat du connaître : le représenté » (Renouvier). — {\bf
5.} {\it Auj.} (opp. : {\it Sujet}$^4$; cf. {\it Objectif}$^2$), la réalité
extérieure qui est pensée : « J’ai souvent remarqué en beaucoup d'exemples
qu'il y avait une grande différence entre l’objet et son idée » (Descartes,
{\it Méd.}, III) ; « Sous le titre d'objet, on peut renfermer tout ce que
l'être pensant perçoit, comme actuellement distinct du sentiment de son
existence individuelle » (Biran) ; « Il
% Cuvillier. — Vocabulaire philosophique.
% 129
y a trois conceptions possibles de l’objet : l'objet est la chose en soi
[solution que l’auteur rejette], ou c’est l’accord des idées entre elles, ou
c’est la liaison nécessaire par {\it opp.} à des liaisons
contingentes » (Hamelin).

\subsubsection{Objectif}
 — \si{Crit.} et \si{Méta.} {\bf 1.} {\it Autref.} {\it not.}
{\it chez Descartes} : qui existe dans l’esprit en tant que représenté (voir
{\it Objet}$^4$, et cf. {\it Eminent}* et {\it Idée}$^6$) : « Une chose est
objectivement ou par représentation dans l’entendement par son idée$^4$
» ({\it Méd.}, III) ; « Afin qu’une idée contienne une telle réalité
objective, elle doit avoir cela de qq. cause dans laquelle il se rencontre
pour le moins autant de réalité formelle$^1$ que cette idée contient
% 128
de réalité objective » ({\it ibid.}). Ce sens a été repris par Renouvier : «
J’appellerai {\it objectif} ce qui s’offre comme objet, {\it i. e.} qui vient
représentativement dans la connaissance » ({\it Traité de Log. gén.}, III).
—  {\bf 2.} {\it Auj.} (depuis Kant, — opp. : {\it subjectif}$^1$). Qui
existe hors de l'esprit et indépendamment de la connaissance qu’en a le
sujet$^4$ pensant (voir {\it Objet}$^5$) : « L’espace n’est pas qqc.
d'objectif ou de réel, mais de subjectif$^1$ et d’idéal$^1$ » (Kant). D'où :
extérieur à la conscience : « La pesanteur est une réalité objective » $->$
Bien {\it dist.} ces 2 sens dont la confusion entraînerait de graves erreurs.

— \si{Épist.} {\bf 3.} (Ctr. {\it subjectif}$^2$). {\it Laud.} Fondé sur une
observation impartiale, indépendante des préférences individuelles de
l’auteur : « L'objectif, c’est ce qui est impersonnel » (Hamelin) ; « Un
compte rendu très objectif ». — {\bf 4.} Fondé sur l'étude de phénomènes
objectifs$^2$ : « La méthode objective en psychologie. »

— \si{Méta.} {\bf 5.} {\it Chez Hegel} : « Esprit
objectif », voir {\it Esprit}$^5$.

\subsubsection{Relativité}
 — \si{Crit.} {\it Relativité de la connaissance} : caractère
de la connaissance d’être « relative », en ce sens : {\bf 1.} qu’elle ne peut
porter que sur des relations$^2$ (Hamilton, Comte) ; — ou bien : {\bf 2.}
qu’elle dépend du sujet$^\text{4a}$ connaissant et de la constitution de
l'esprit humain (Kant). $->$ Dans les 2 sens, {\it dist.} avec soin {\it
relatif} de {\it faux} ou {\it inadéquat}. Au sens 1, Hamelin écrit : « La
relativité de la connaissance n’est pas, comme on l’a cru qqfs., un obstacle
au savoir : elle en est le moyen ». Au sens 2, voir {\it Phénomène}$^2$.

— \si{Phys.} {\bf 3.} {\it Principe de relativité} (Einstein) : principe
selon lequel : 1° les lois des phénomènes physiques sont les mêmes pour
différents groupes d’observateurs en mouvement de translation uniforme
(relativité restreinte) ou uniformément accélérée ({\it p. e.} gravitation :
relativité généralisée) les uns par rapport aux autres; 2° par suite, la
durée des phénomènes varie suivant qu’elle est mesurée par des observateurs
en repos ou en mouvement par rapport à eux. $->$ Bien {\it dist.} ce sens des
deux précédents auxquels il est même « opposé à certains égards » (Lalande).

\subsubsection{Relation}
 — \si{Crit.} {\bf 1.} Une des catégories fondamentales de la
pensée, et même, selon Renouvier, « la catégorie des catégories » ({\it Log.
générale}, III, 27) : « Qu'est-ce que penser, sinon poser des relations ? »

— \si{Épist.} \fsb{S. abstr.} {\bf 2.} Rapport entre
deux objets, phénomènes ou quantité,
% 161
tel que toute modification de l’un entraîne une modification de l’autre : «
La science recherche des relations$^2$ constantes entre les phénomènes ». —
\fsb{S. concr.} {\bf 3.} Formule exprimant une relation$^2$, {\it spéc.} en
\si{Math.} égalité$^3$ ou inégalité.

— \si{Log.} {\bf 4.} {\it Propositions de relation} : celles qui énoncent une
relation$^2$ autre que celle d’{\it inhérence}$^2$ (v. ce mot).

\subsubsection{Attribut}
 — \si{Log.} \si{form.} {\bf 1.} (Syn. : {\it prédicat}*). Terme qui, dans
une proposition, désigne ce qu’on affirme
ou nie du sujet$^2$.

— \si{Méta.} {\bf 2.} Tout ce qui peut être
dit attribut (au sens 1) d'une substance$^1$ : « La fluidité, la dureté, la
mollesse... se pouvant séparer de la
matière, il s'ensuit que tous ces
attributs ne lui sont point essentiels » (Malebranche, {\it R. V.}, III, 2,
%25
8, 2). — {\bf 3.} Propriété essentielle et
permanente d’une substance$^1$ ({\it gén}.
opp. aux modes$^1$ qui sont des propriétés accidentelles et changeantes) :
« Nous distinguons qqfs. une substance de qqn de ses attributs [{\it p. e.}
l'âme, de la pensée ; la matière, de
l'étendue] sans lequel néanmoins il
n’est pas possible que nous en ayons
une connaissance distincte (Descartes, Prince. I, 62; {\it cf.}  Mode$^1$) :
« Par attribut, j'entends ce que
l’entendement saisit de la substance
comme constituant son essence »
(Spinoza, {\it Eth.}, I, déf. 4).


\subsubsection{Prédicat}
 — [L. {\it prædicare}, attribuer] — \si{Log.} \si{form.} Syn.
d'{\it attribut}$^1$ : « Dans une proposition vraie, la notion du prédicat
est toujours contenue dans le sujet » (Leibniz). Voir
{\it Quantification}$^2$.

\subsubsection{Quantification}
 — \si{Phys.} {\bf 1.} Attribution d’un {\it quantum}* : «
La quantification des mouvements intraatomiques ». — \si{Log.} \si{form.}
{\bf 2.} {\it Quantification du prédicat} : théorie (de Hamilton) selon
laquelle on doit,
% 154
dans toute proposition, énoncer la quantité$^3$ de l’attribut$^1$, alors que,
selon la logique classique, celui-ci est toujours pris particulièrement dans
les affirmatives, universellement dans les négatives.

\subsubsection{Quantité}
 — (Opp. : {\it qualité}*), \si{Crit.} {\bf 1.} Une des
catégories* fondamentales de la pensée : abstraction$^2$ de la grandeur,
dépouillée de toutes ses qualités$^2$ et considérée seulement comme
mesurable. Voir {\it Continu}$^2$ et {\it Discontinu}$^2$. — \si{Log.}
\si{form.} {\bf 2.} {\it Quantité d'une proposition} : le fait que le
sujet$^2$ y est pris dans toute son extension$^3$ (prop.
{\it universelles}$^3$) ou bien dans une partie seulement de son extension
(prop. {\it particulières}$^2$). — {\it Ext.} {\bf 3.} {\it Quantité d'un
terme} : le fait qu'il est pris dans la totalité ou dans une partie seulement
de son extension$^3$.

\subsubsection{Qualité}
 — (Opp. : {\it quantité}*]. \si{Crit.} {\bf 1.} Une des
catégories* fondamentales de la pensée : propriété$^1$, manière d’être : «
J’ai pensé que je ne ferais pas peu si je montrais comment il faut distinguer
les propriétés ou qualités de l'esprit des propriétés ou qualités du corps
» (Descartes, 2$^\text{e}$ {\it Rép.}) ; « Une qualité est ce qui fait qu'on
appelle une chose d’un tel nom : on ne peut le nier à
% 153
Aristote » (Malebranche, {\it R. V.}, VI, 2, 2). — {\bf 2.} Propriété
sensible et non mesurable des choses : « Nous mettons la quantité de la cause
[de la sensation] dans la qualité de l'effet » (Bergson, {\it D. I.}, I) ; «
Dès le premier coup d'œil jeté sur le monde, nous y distinguons des {\it
qualités} » (id, {\it E. C.}, IV). {\it Qualités premières} ou {\it
primaires} : celles sans lesquelles les corps ne peuvent se concevoir
(étendue, impénétrabilité). {\it Qualités secondes} ou {\it secondaires}
celles qu'on peut supprimer par abstraction sans supprimer la notion de corps
(couleur, saveur, etc.). {\it Qualités occultes} propriétés non constatables
qu’on supposait dans la nature pour expliquer les phénomènes : « Les livres
de science... sont tous pleins de raisonnements fondés sur les qualités
élémentaires et sur les qualités secondes, comme les {\it attractrices}, les
{\it rétentrices}, les {\it concoctrices}, les {\it expultrices}, et autres
semblables ; sur d’autres qu'ils appellent {\it occultes}, sur les {\it
vertus spécifiques}… » (Malebranche, R. V., III, 2, 8, 1). — \si{Log.}
\si{form.} {\bf 3.} {\it Qualité d’une proposition} : le fait qu'elle est
affirmative ou négative.

— \si{Mor.} {\bf 4.} Disposition ou valeur morale, vertu$^2$ : « Le vrai
courage est une des qualités qui supposent le plus de grandeur d'âme
» (Vauvenargues). — {\it Ext.} {\bf 5.} Valeur, en gén. : « Le monde aurait
été sauvé plus d’une fois si la qualité des âmes pouvait dispenser de la
qualité des idées » (Brunschvicg).

\subsubsection{Substrat, Substratum}
 — \si{Méta.} {\bf 1.} La
substance$^2$, en tant qu’elle sert de
% 179
support aux attributs : « Distinguer l'âme de la matière, relativement à son
{\it substrat} » (Kant, {\it R. pure}, Dial., Il, 1, 2). — {\it Ext.}
{\bf 2.} Ensemble de phénomènes$^1$ qui conditionnent d’autres phénomènes :
« [Nos habitudes] constituent, réunies, le substrat de notre activité libre
» (Bergson, {\it D. I.}, Ill). {\it Chez Durkheim} : « substrat social »,
base morphologique* de la société consistant 1° dans la répartition des
groupes sur le sol (géographie humaine) ; 2° dans les variations de volume*
et de densité* de ces groupes : « La sociologie ne peut se désintéresser de
ce qui constitue le {\it substrat} de la vie collective. »

\subsubsection{}
