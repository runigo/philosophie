
%%%%%%%%%%%%%%%%%%%%%
\section{Encyclopédie de la philosophie}
%%%%%%%%%%%%%%%%%%%%%
%549

{\bf évidence} dans la philosophie antique,
l'évidence est une caractéristique de la
« science » (par opposition à l’« opinion »),
et principalement du plus haut
degré de la science, celui que l’on atteint
par l’entendement intuitif ({\it }noûs). Pour
Platon, ce degré consiste en la connaissance
des Idées, lesquelles sont justement
caractérisées par l’évidence, contrairement
à l'expérience sensible. Cette conception de
l'évidence comme illumination de source
transcendante sera développée par le néoplatonisme
avec des accents mystiques, et
sera ensuite intégrée à la pensée chrétienne
par saint Augustin. Pour ce dernier,
l'évidence à laquelle l’âme humaine
a accès (à commencer par les mathématiques)
est un reflet de la vérité éternelle
qui est en Dieu. Pour Aristote, au contraire,
l'évidence est la caractéristique des
principes suprêmes de l’entendement
(principe de non-contradiction et du tiers
exclu) et des axiomes qui constituent les
principes de sciences spécifiques (par
exemple « le tout est plus grand que la
partie »). Cette dernière position, propre
au système de géométrie euclidienne, fut
constante jusqu’au siècle dernier. Dans
l’ensemble, la conception aristotélicienne
de l’évidence, beaucoup plus restrictive
que celle d'inspiration platonicienne, sera
adoptée dans la pensée chrétienne par les
courants scolastiques, qui justement se
réclameront d’Aristote, à commencer par
le thomisme. Pour les stoïciens au contraire,
l'évidence est sensible : c’est la
caractéristique propre à la « représentation
cataleptique » (ou « compréhensive »)
que l’on a lorsqu'un objet s'impose à l’intellect
par sa présence immédiate. Le
scepticisme antique se dresse contre cette
thèse.

Dans la philosophie moderne, l’évidence
est de nouveau considérée comme
caractéristique de l'intuition intellectuelle.
Descartes est le philosophe qui accorde le
plus d'importance à l’évidence, sur la base
de cette conception. La règle principale
de sa méthode consiste justement à ne
considérer comme vrai que ce qui est évident
(« clair et distinct »), ou qui ne laisse
aucune prise au doute. Aussi l’exercice du
doute, par accumulation des motifs qui
peuvent le justifier, est-il la meilleure
façon de vérifier ce qui parmi nos
croyances peut y résister. Toutefois, Descartes
entend aussi l'évidence comme un
phénomène simplement psychologique,
un sentiment de certitude qui entraîne
l'affirmation. Et il soutient que pour pouvoir
adopter l'évidence comme critère de
vérité objective il faut nécessairement
présupposer que les hommes sont les
créatures d’un Dieu parfait, et donc
vérace, c’est-à-dire qui ne trompe pas ses
créatures. Celles-ci seraient en effet irrémédiablement
trompées, si l’évidence qui
entraîne l’assentiment ne correspondait
effectivement à la vérité. L'emploi de
l'évidence comme marque de la vérité
n’est donc pas fondé sur l’idée que l’évidence
soit d’elle-même garantie. Le
recours à Dieu sera évincé de la pensée
moderne, pour qui toutefois le lien entre
l'évidence et l'intuition intellectuelle perdure,
tant dans le courant rationaliste que
dans le courant empirique. Pour Kant, la
seule vérité d'emblée évidente est celle du
principe de non-contradiction, et donc des
jugements qui ne dépendent que de ce
principe (c’est-à-dire « les jugements analytiques »).

Par la suite, la problématique de l’évidence
est passée de la philosophie à la
psychologie, avec en particulier l’épistémologie
conventionnaliste qui a refusé le
critère d’évidence dans sa critique du
recours à l’intuition, jugeant dogmatique
le fait de la considérer comme une source
mystérieuse de connaissance privilégiée.
Seul Edmund Husserl, pour le {\footnotesize XX}$^\text{e}$ s., a
innové au sujet de l'évidence : se ralliant
à la pensée de Descartes, il considère que
l'évidence est caractéristique des contenus
idéaux (« eidétiques »), qui se livrent
dans leur objectivité apodictique lorsqu'il
a été fait abstraction de l’existence du
monde physique et du sujet psychologique.
% --> eidétique + intuition + vérité
%%%%%%%%%%%%%%%%%%%%%%%%%%%%%%%%%%%%%%%%%%%%%%%%%%%%%%%%%%%%%%%%%%%%%%%%
